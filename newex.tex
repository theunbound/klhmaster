...additive under Lorenz boosts.

Of the three detector types, the most important to the search for photons is the electromagnetic calorimeter.

The muon tracker forms the outermost, and largest, part of the detector, since muons are one of only a small number of particles that can be expected to penetrate through the calorimeter layer regularly. Neutrinos will do so as well, however, capturing neutrinos with \textsc{atlas}' mere 90~000~t of material and 7~000 m$^3$ detector volume \cite{atlasweb} is something of a lost cause. Like the inner detector, the muon detector localises the track of a passing charged particle by picking up the ionisation trail it leaves in a sensitive material. Also like the inner detector, it ascertains the momentum and charge of passing charged particles by measuring their deflection by an applied magnetic field.

The inner detector uses two distinct detector technologies to localise the tracks of charged particles that pass through it. The innermost layer uses silicon semiconductor chips to detect the charge left in a track in a compact, high-precision system. The outer part of the inner detector uses drift straws---which pick up the ionisation charge left in an inert gas by attracting it to a high-voltage wire in the centre of the straw---to more economically, in terms of material, readout complexity and, well, economy, cover a larger volume.

Both the straw detector and part of the silicon detector, the strip detector, are very long structures, which can only report that a hit has occurred somewhere along its length. To compensate for this, successive layers of detector elements are placed at an angle to one another. The innermost layer of the silicon detector, the pixel detector, is also segmented lengthwise, and so does not require this. Still, the detector only provides a list of detector elements that have registered a hit. Particle tracks are fitted to these hits in one of the stages of the subsequent analysis of an event.

There are two calorimeter systems in \textsc{atlas}. Both are supposed to stop the particles that pass through them, and measure the energy that they deposit. \textsc{Atlas} only uses sampling calorimeters, meaning that the absorbing material has sensitive layers inserted into it at intervals. These sensitive layers measure how deep into the absorbing material the particles penetrate, and thus how much energy the carried before entering the calorimeter. In the barrel section of the hadronic calorimeter---the outermost half of the calorimeter system, which is supposed to catch the, by assumption, hadronic events that penetrate the electromagnetic calorimeter, the inner half of the calorimeter system---the sensitive layers are scintillators, a material that luminesces when exposed to ionising radiation, and the absorbing layers are iron. The remaining calorimeters, which covers both the EM calorimeters and the endcap hadronic calorimeters, have both iron and lead as absorbing material, and liquid argon as the sensitive material. As with the straw detectors, activity in the liquid argon layers are detected by an electrode, which picks up the ionisation charge left in the argon.