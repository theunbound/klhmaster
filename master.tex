%gram = xelatex
\documentclass[a4paper,11pt,openany]{memoir}
\usepackage{amsmath}
%\usepackage[sc,osf]{mathpazo}
%\usepackage[garamond]{mathdesign}
\usepackage{fontspec}
\usepackage[math-style=TeX]{unicode-math}
\usepackage{xunicode}
\usepackage{xltxtra}
\usepackage{polyglossia}
\defaultfontfeatures{Mapping=tex-text}
\setmainfont[   Path=fonts/xits/,
                BoldFont={xits-bold.otf},
                ItalicFont={xits-italic.otf},
                BoldItalicFont={xits-bolditalic.otf},
                SmallCapsFont={../MasticSC-Regular.otf}
            ]{xits-regular.otf}
\setsansfont[   Path=fonts/frutiger/,
                Scale=MatchLowercase,
                BoldFont={FrutigerLTStd-Bold.otf},
                ItalicFont={FrutigerLTStd-Italic.otf},
                BoldItalicFont={FrutigerLTStd-BoldItalic.otf}
            ]{FrutigerLTStd-Roman.otf}
            
\setmathfont{xits-math.otf}
\setmathfont[version=roman,range=\mathcal,Path=fonts/]{latinmodern-math.otf}
%\setmathfont[range=\mathcal,Scale=MatchUppercase]{Lynda Cursive}
%\setmathfont[range=\mathup]{Garamond Premier Pro}
%\setmathfont[range=\mathit]{Garamond Premier Pro}
\setdefaultlanguage{danish}
\setotherlanguage[variant=british]{english}

\usepackage[usenames,dvipsnames,svgnames,table]{xcolor}
\usepackage{graphicx,epic,eepic}
\usepackage{tkz-graph,tkz-euclide}
\usetikzlibrary{calc,intersections,shapes.geometric,decorations.pathmorphing}
\usetikzlibrary{snakes,patterns,plotmarks,decorations.text}
\renewcommand*{\VertexSmallMinSize}{4pt}
\usepackage{tabulary}
\usepackage{pdfpages}
\usepackage{wrapfig}
\usepackage{pifont}
\usepackage{multirow}
\usepackage{colortbl}

%\renewcommand{\thefootnote}{\fnsymbol{footnote}}

\renewcommand\bibname{References}
\renewcommand{\(}{\begin{equation}}
\renewcommand{\)}{\end{equation}}
\newcommand{\vet}[1]{\underline{#1}}
\newcommand{\mtx}[1]{\underline{\underline{#1}}}
\newcommand{\ono}[1]{\frac{1}{#1}}
\newcommand{\half}{\frac{1}{2}}
\newcommand{\thehead}{}
\newcommand{\head}[1]{\renewcommand{\thehead}{#1}}
\newcommand{\theohead}{}
\newcommand{\ohead}[1]{\renewcommand{\theohead}{#1}}
\newcommand{\thetnote}{}
\newcommand{\tnote}[1]{\renewcommand{\thetnote}{#1}}
\newcommand{\bra}[1]{\left\langle #1 \right|}
\newcommand{\ket}[1]{\left| #1 \right\rangle}
\newcommand{\braket}[2]{\left\langle#1\middle|#2\right\rangle}
\newcommand{\obraket}[3]{\left\langle#1\middle|#2\middle|#3\right\rangle}
\newcommand{\emf}{\mathcal{E}}
\newcommand{\di}{\text{d}}
\newcommand{\atlas}{\textsc{atlas}}
\newcommand{\Atlas}{\textsc{Atlas}}

\newenvironment{infilsf}{
    \begin{sffamily}
    \setmathfont[range=\mathup/{num},
                 Scale=MatchLowercase,
                 Path=fonts/frutiger/,
                 ]{FrutigerLTStd-Roman.otf}
    \setmathfont[range=\mathit/{latin,Latin},
                 Scale=MatchLowercase,
                 Path=fonts/frutiger/,
                    ]{FrutigerLTStd-Italic.otf}
    \setmathfont[range=\mathit/{greek,Greek},
                 Scale=MatchLowercase,
                 Path=fonts/dejavu/,
                    ]{DejaVuSans-Oblique.ttf}
    \setmathfont[range=\mathup/{greek,Greek},
                 Scale=MatchLowercase,
                 Path=fonts/dejavu/,
                    ]{DejaVuSans.ttf}
%    \setmathfont[range=\text,
%                 Scale=MatchLowercase,
%                 Path=fonts/frutiger/,
%                    ]{FrutigerLTStd-Roman.otf}
}{
    \setmathfont{xits-math.otf}
    \setmathfont[range=\mathcal,Path=fonts/]{latinmodern-math.otf}
    \end{sffamily}
}

\newenvironment{new}{\color{Blue}}{}

\definecolor{kugray}{RGB}{102,102,102}
\definecolor{kugray1}{RGB}{52,52,52}
\definecolor{natgreen}{RGB}{70,116,60}
\definecolor{natgreen1}{HTML}{63875B}
\definecolor{natgreen2}{HTML}{859B81}
\definecolor{natcomp}{HTML}{86454C}
\definecolor{natcomp1}{HTML}{9D696E}
\definecolor{natcomp2}{HTML}{B39598}
\definecolor{natlcomp}{HTML}{5d323d}
\definecolor{natrcomp}{HTML}{52325d}
\definecolor{natyellow}{HTML}{8B6448}
\definecolor{natblue}{HTML}{524F81}
\definecolor{natscatb}{HTML}{3B8178}
\definecolor{natscatg}{HTML}{418B5C}
\definecolor{natscaty}{HTML}{7C8B41}
\definecolor{natscatr}{HTML}{81783C}

\makepagestyle{fancy}

\makepsmarks{fancy}{%
\nouppercaseheads
\createmark{chapter}{left}{nonumber}{}{}
\createmark{section}{right}{shownumber}{}{ \space}
\createplainmark{toc}{both}{\contentsname}
\createplainmark{lof}{both}{\listfigurename}
\createplainmark{lot}{both}{\listtablename}
\createplainmark{bib}{both}{\bibname}
\createplainmark{index}{both}{\indexname}
\createplainmark{glossary}{both}{\glossaryname}}

\makeoddhead{fancy}
   {}{}{}
\makeoddfoot{fancy}{\makebox[0pt][r]{\raisebox{15pt}[20pt]{\textcolor{natgreen}{\rule{1.1\spinemargin}{1pt}}}\makebox[0pt][l]{\raisebox{15pt}[20pt]{\textcolor{natgreen}{\rule{\paperwidth}{1pt}}}}}\textcolor{kugray}{\textsf{\rightmark}}}{}{\textcolor{kugray}{\textsf{\thepage}}}
\makeevenfoot{fancy}{\textcolor{kugray}{\textsf{\thepage}}}{}{\textcolor{kugray}{\textsf{\leftmark}}\makebox[0pt][l]{\raisebox{15pt}{\textcolor{natgreen}{\rule{1.1\spinemargin}{1pt}}}}\makebox[0pt][r]{\raisebox{15pt}{\textcolor{natgreen}{\rule{\paperwidth}{1pt}}}}}
\setlength{\footskip}{40pt}
\pagestyle{fancy}
\aliaspagestyle{chapter}{fancy}

\captionnamefont{\sffamily\color{natgreen}\bfseries} \captiontitlefont{\footnotesize} \captionstyle{\\}
\renewcommand*{\printchaptername}{}
\renewcommand*{\chapternamenum}{}
\renewcommand*{\afterchapternum}{}
\renewcommand{\chapnumfont}{\chaptitlefont\sffamily\HUGE}
\renewcommand{\printchapternum}{\chapnumfont \colorbox{natgreen}{\textcolor{white}{\hspace{.2em}\thechapter\hspace{.2em}}}\hspace{1em}}
\setsecheadstyle{\large\bfseries}
\setsubsecheadstyle{\bfseries}
\setsubsubsecheadstyle{}
\setsecnumformat{\textsf{\color{natgreen}\csname the#1\endcsname\quad}}
\maxsecnumdepth{subsubsection}
\renewcommand{\labelenumi}{\sffamily\bfseries\color{natgreen}\theenumi.}
\renewcommand{\labelitemi}{\color{natgreen}\ding{110}}
\renewcommand{\labelitemii}{\color{natgreen}\textbullet}
\setcounter{tocdepth}{2}

\setlength{\arrayrulewidth}{2pt}

\newsubfloat{figure}

\begin{hyphenrules}{danish}
\hyphenation{be-stem-mes}
\hyphenation{rest-klas-se-sæt-ning}
\end{hyphenrules}
\begin{hyphenrules}{english}
\hyphenation{Ham-il-ton}
\hyphenation{ATLAS}
\hyphenation{Atlas}
\hyphenation{atlas}
\hyphenation{CERN}
\hyphenation{i-den-ti-cal}
\hyphenation{pro-vid-ed}
\hyphenation{Calc-HEP}
\hyphenation{par-ticles}
\hyphenation{brems-strahl-ung}
\end{hyphenrules}
\usepackage[pdfusetitle]{hyperref}
\urlstyle{sf}


\bibliography{cite}

%\includeonly{analysis}

\begin{document}
\begin{english}
\begin{titlingpage}
{
%\definecolor{kugray}{RGB}{102,102,102}
%\definecolor{natgreen}{RGB}{50,93,61}
\thispagestyle{empty}
\newlength{\topma}\setlength{\topma}{-1in}\addtolength{\topma}{-\headsep}\addtolength{\topma}{-\voffset}\addtolength{\topma}{11mm}
\newlength{\sidema}\setlength{\sidema}{-1in}\addtolength{\sidema}{-\hoffset}\addtolength{\sidema}{-\oddsidemargin}\addtolength{\sidema}{-\marginparsep}\addtolength{\sidema}{15mm}
\newlength{\textwa}\setlength{\textwa}{\paperwidth}\addtolength{\textwa}{-35mm}\addtolength{\textwa}{-\textwidth}
\newlength{\textha}\setlength{\textha}{-\textheight}\addtolength{\textha}{\paperheight}\addtolength{\textha}{-11mm}\addtolength{\textha}{-.1\paperheight}
\changepage{\textha}{\textwa}{}{\sidema}{}{-\topmargin}{-\headheight}{\topma}{}
%\setlength{\parindent}{0pt}
\noindent\begin{minipage}[t]{.8\textwidth}
\noindent\raggedright \textcolor{kugray}{\fontspec[Path=fonts/garamond/]{GaramondPremrPro.otf}\addfontfeature{LetterSpace=13.0}\fontsize{18}{17}\selectfont\textsc{university of copenhagen}}

\vspace{.2em}\textcolor{kugray}{\fontspec[Path=fonts/garamond/]{GaramondPremrPro.otf}\addfontfeature{LetterSpace=13.0}\fontsize{15}{17}\selectfont\textsc{niels bohr institute}}
\end{minipage}\hfill\begin{minipage}[t]{32mm}\raggedleft \vspace{16mm} \includegraphics[height=44mm]{figures/bionat.pdf}
\end{minipage}


\newlength{\markbump}\setlength{\markbump}{.3\paperwidth}\addtolength{\markbump}{-6mm}
\newlength{\markdown}\setlength{\markdown}{-75mm}\addtolength{\markdown}{0em}\addtolength{\markdown}{\paperheight}\addtolength{\markdown}{.15\paperwidth}
\noindent\makebox[0pt][l]{\hspace{\markbump}\raisebox{-\markdown}[0pt][0pt]{\includegraphics[height=.85\paperwidth]{figures/atlaskugrid2.pdf}}}\makebox[0pt][r]{\raisebox{3.2mm}[0pt][0pt]{\textcolor{natgreen}{\rule{.06\paperwidth}{.7pt}}}}\makebox[0pt][l]{\raisebox{3.2mm}[0pt][0pt]{\textcolor{natgreen}{\rule{.95\paperwidth}{.7pt}}}}





\bigskip

{\sffamily
\begin{hangparas}{0em}{0}
{\textbf{ }

\vspace{2em}}

{\LARGE \textbf{Master's thesis}

\vspace{.3em}}

{\Large Kristoffer Levin Hansen}

\vspace{3em}

{\Huge \raggedright
Search for new physics in diphoton production with the ATLAS detector at the LHC
}
\vfill{}

{\Large Academic advisor: Jørgen Beck Hansen}
\vspace{28em}

\today
\end{hangparas}
}
\clearpage}
\thispagestyle{empty}
  \phantom{p}
\vspace{1.16\textwidth}

\begin{center}
\includegraphics[width=.1\textwidth]{star1}

\vspace{2em}
\begin{minipage}{.6\textwidth}
\emph{``Art is never finished, merely abandoned.''}

\raggedleft\small\sffamily\bfseries\vspace{1em} -- Leonardo da Vinci
\end{minipage}
\end{center}
\clearpage
\end{titlingpage}
\frontmatter

\begin{abstract}
In this thesis, we will conduct a search for physics beyond the Standard Model by introducing into it, via the effective Lagrangian approach, a new $qq\gamma\gamma$ contract interaction. The effects of this new interaction on the distribution of invariant masses of diphotons will be assessed via the production of Monte Carlo simulations at two values of the associated mass scale $\Lambda$, and a corresponding Monte Carlo set for the Standard Model. These Monte Carlo sets will be subjected to full \atlas{} detector stimulation, before they are compared with data taken by \atlas{} during the 2012, 8 TeV run of the LHC. A fully data driven background estimation will be carried out on this data set. Finally, we will interpolate our Monte Carlo samples, so that we may compare a distribution corresponding to any $\Lambda$ with the experimental distribution. This can be used for a maximum profile likelihood fit of $\Lambda$ to data.
\end{abstract}

\end{english}
\renewcommand{\abstractname}{Resumé på dansk}
\begin{abstract}
I dette speciale vil vi søge efter ny fysik ud over Standardmodellen, ved at indføre en ny $qq\gamma\gamma$ kontaktinteraktion ved brug af en effektiv Lagrangefunktion. Vi vil undersøge effekten af en sådan ny interaktion på fordelingen af invariante masser af difotoner produceret ved protonsammenstød, ved at fremstille Monte Carlo dataset ved to værdier af den tilhørende masseskala $\Lambda$, samt et korresponderende Monte Carlo sæt for Standardmodellen. Disse Monte Carlo sæt køres igennem den fulde \atlas{} detektor simulering, før de sammenlignes med data taget af \atlas{} under 8 TeV kørslen i LHC i 2012. En fuldt datadreven baggrundsestimering vil blive udført på det eksperimentelle data sæt. Endelig vil vi interpolere vores Monte Carlo datasæt, således at vi kan sammenligne distributioner der tilsvarer enhver værdi af $\Lambda$ med den eksperimentelle distribution. På den måde kan et maksimal profil likelihood fit af $\Lambda$ til data udføres.
\end{abstract}
\begin{english}
\newpage

\tableofcontents
\mainmatter

\chapter{Introduction}

Since the late 1960s, our---at times evolving---understanding of the properties and interactions of the fundamental particles has been summarised by the Standard Model of Particle Physics. The Standard Model is formulated in the language of Quantum Field Theory\footnote{Capitalised in anticipation of the imminent use of its common abbreviation, QFT.}, and attempts to combine a theoretical model of quantum mechanical phenomena with an experimental understanding of the properties of fundamental particles and the couplings between them.

An overview of a selection of these properties and interactions is given in fig.~\ref{SMsum}.

\begin{figure}[htp]
\begin{minipage}[b]{.745\textwidth}
\begin{infilsf}
\begin{sffamily}\begin{scriptsize}
\pgfdeclarelayer{back}
\pgfsetlayers{back,main}
\begin{tikzpicture}[yscale=1.8,xscale=0.35]

\tikzstyle{quark}=[font=\footnotesize,white,circle,draw=white,fill=natgreen,inner sep=0pt,minimum size=12pt]
\tikzstyle{gauge}=[font=\footnotesize,white,circle,draw=white,fill=natcomp,inner sep=0pt,minimum size=12pt]
\tikzstyle{higgs}=[font=\footnotesize,white,circle,draw=white,fill=natyellow,inner sep=0pt,minimum size=12pt]
\tikzstyle{lepton}=[font=\footnotesize,white,circle,draw=white,fill=natblue,inner sep=0pt,minimum size=12pt]
\tikzstyle{quarko}=[fill=white,circle,draw=natgreen,inner sep=0pt,minimum size=16pt]
\tikzstyle{gaugeo}=[fill=white,circle,draw=natcomp,inner sep=0pt,minimum size=16pt]
\tikzstyle{higgso}=[fill=white,circle,draw=natyellow,inner sep=0pt,minimum size=16pt]
\tikzstyle{leptono}=[fill=white,circle,draw=natblue,inner sep=0pt,minimum size=16pt]
\tikzstyle{inter}=[kugray!50, ultra thick,cap=rect]
\tikzstyle{under}=[line width=3pt,white]

\draw (-3.5,1.8) -- (-3.5,-1.5) -- (-1.7,-1.5) [snake=zigzag] -- (-.3,-1.5) [snake=none] --  (21,-1.5) -- (21,1.8) -- cycle;
\draw (-3.5,0) node[left] {0} -- +(8pt,0);
\draw (-3.5,1) node[left] {1} -- +(8pt,0);
\draw (-3.5,-1) node[left] {-1} -- +(8pt,0);
\draw (-2,-1.5) -- +(0,2.5pt) +(0,-.9em) node[above] {0};
\foreach \x in {0,1}
\draw (\x*2.303,-1.5) -- +(0,2.5pt) 
    +(0.693,0) -- +(0.693,1.5pt) +(1.1,0) -- +(1.1,1.5pt) 
    +(1.39,0) -- +(1.39,1.5pt) +(1.61,0) -- +(1.61,1.5pt) +(1.79,0) -- +(1.79,1.5pt)
    +(1.95,0) -- +(1.95,1.5pt) +(2.08,0) -- +(2.08,1.5pt) +(2.2,0) -- +(2.2,1.5pt);
\foreach \x in {2,3,...,8}
\draw (\x*2.303,-1.5) -- +(0,2.5pt) +(0,-.9em) node[above] {$10^{\x}$}
    +(0.693,0) -- +(0.693,1.5pt) +(1.1,0) -- +(1.1,1.5pt) 
    +(1.39,0) -- +(1.39,1.5pt) +(1.61,0) -- +(1.61,1.5pt) +(1.79,0) -- +(1.79,1.5pt)
    +(1.95,0) -- +(1.95,1.5pt) +(2.08,0) -- +(2.08,1.5pt) +(2.2,0) -- +(2.2,1.5pt);
\draw (0,-1.5) -- +(0,2.5pt) +(0,-.9em) node[above] {1};
\draw (2.303,-1.5) -- +(0,2.5pt) +(0,-.9em) node[above] {10};
\draw (9*2.303,-1.5) -- +(0,2.5pt) +(0,-.9em) node[above] {$10^9$};
\node at (-4.8,1.7) [rotate=90,left] {Electromagnetic charge [e]};
\node at (21,-1.8) [left] {Mass [eV/c$^2$]};
\draw (7.74,0.667) node [quarko] (u) {} node {\tikz {\fill[natgreen] (0,0) -- ++(0,8pt) arc (90:-90:8pt) -- cycle; \draw +(180:8pt);}} node [quark] {u};
\draw (8.48,-0.333) node [quarko] (d) {} node {\tikz {\fill[natgreen] (0,0) -- ++(0,8pt) arc (90:-90:8pt) -- cycle; \draw +(180:8pt);}} node [quark] {d};
\draw (14.06,0.667) node [quarko] (s) {} node {\tikz {\fill[natgreen] (0,0) -- ++(0,8pt) arc (90:-90:8pt) -- cycle; \draw +(180:8pt);}} node [quark] {s};
\draw (11.46,-0.333) node [quarko] (c) {} node {\tikz {\fill[natgreen] (0,0) -- ++(0,8pt) arc (90:-90:8pt) -- cycle; \draw +(180:8pt);}} node [quark] {c};
\draw (18.97,0.667) node [quarko] (t) {} node {\tikz {\fill[natgreen] (0,0) -- ++(0,8pt) arc (90:-90:8pt) -- cycle; \draw +(180:8pt);}} node [quark] {t};
\draw (15.25,-0.333) node [quarko] (b) {} node {\tikz {\fill[natgreen] (0,0) -- ++(0,8pt) arc (90:-90:8pt) -- cycle; \draw +(180:8pt);}} node [quark] {b};
\draw (-2,0) ++(100:6.5pt) ++(-4.47pt,0) node (gamma) [gaugeo] {} node {\tikz {\shade[top color=white,bottom color=natcomp] (0,0)  +(-80:12pt) -- +(-45:8pt) arc (-45:-115:8pt) -- cycle; \filldraw[natcomp] (0,0) circle (8pt); \draw +(100:12pt);}} node [gauge] {$\gamma$};
\draw (-2,0) ++(-80:6.5pt) ++(4.47pt,0) node [gaugeo] (g) {} node {\tikz {\shade[top color=natcomp,bottom color=white] +(100:12pt) -- +(135:8pt) arc (135:65:8pt) -- cycle; \filldraw[natcomp] (0,0) circle (8pt); \draw +(-80:12pt);}} node [gauge] {g};
\draw (-2,0) node {\tikz \fill[natcomp] circle (1pt);};
\draw (18.33,0) ++(100:6.5pt) ++(-4.47pt,0) node [gaugeo] (Z) {} node {\tikz {\shade[top color=white,bottom color=natcomp] (0,0)  +(-80:12pt) -- +(-45:8pt) arc (-45:-115:8pt) -- cycle; \filldraw[natcomp] (0,0) circle (8pt); \draw +(100:12pt);}} node [gauge] {Z};
\draw (18.20,1) node [gaugeo] (W+) {} node {\tikz {\filldraw[natcomp] (0,0) circle (8pt);}} node [gauge] {\tiny W$^+$};
\draw (18.20,-1) node [gaugeo] (W-) {} node {\tikz {\filldraw[natcomp] (0,0) circle (8pt);}} node [gauge] {\tiny W$^-$};
\draw (18.65,0)  ++(-80:6.5pt) ++(4.47pt,0) node [higgso] (H) {} node {\tikz {\shade[top color=natyellow,bottom color=white] +(100:12pt) -- +(135:8pt) arc (135:65:8pt) -- cycle; \draw +(-80:12pt); \draw[natyellow] (0,0) circle (8pt);}} node [higgs] {H};
\draw (18.33,0) node {\tikz \fill[natcomp] circle (1pt);};
\draw (18.65,0) node {\tikz \fill[natyellow] circle (1pt);};
\draw (0.79,0) node [leptono] (ve) {} node {\tikz {\fill[natblue] (0,0) -- ++(0,8pt) arc (90:-90:8pt) -- cycle; \draw +(180:8pt);}} node [lepton] {$\nu_{\text e}$};
\draw (5.14,0) node [leptono] (vmu) {} node {\tikz {\fill[natblue] (0,0) -- ++(0,8pt) arc (90:-90:8pt) -- cycle; \draw +(180:8pt);}} node [lepton] {$\nu_\mu$};
\draw (9.65,0) node [leptono] (vtau) {} node {\tikz {\fill[natblue] (0,0) -- ++(0,8pt) arc (90:-90:8pt) -- cycle; \draw +(180:8pt);}} node [lepton] {$\nu_\tau$};
\draw (6.24,-1) node [leptono] (e) {} node {\tikz {\fill[natblue] (0,0) -- ++(0,8pt) arc (90:-90:8pt) -- cycle; \draw +(180:8pt);}} node [lepton] {e};
\draw (11.57,-1) node [leptono] (mu) {} node {\tikz {\fill[natblue] (0,0) -- ++(0,8pt) arc (90:-90:8pt) -- cycle; \draw +(180:8pt);}} node [lepton] {$\mu$};
\draw (14.39,-1) node [leptono] (tau) {} node {\tikz {\fill[natblue] (0,0) -- ++(0,8pt) arc (90:-90:8pt) -- cycle; \draw +(180:8pt);}} node [lepton] {$\tau$};

\begin{pgfonlayer}{back}
\node at (12,1) [inner sep=0] (uqs) {};
\node at (20.7,.6) [inner sep=0] (qgnode) {};
\draw [inter] (u) to[out=0,in=270,max distance=7pt]  (uqs);
\draw [inter] (s) to[out=170,in=270,max distance=4pt] (uqs);
\draw [inter] (t) to[out=190,in=270,max distance=25pt] (uqs);
\draw [inter] (d) to[out=5,in=270,max distance=30pt] (uqs);
\draw [inter] (c) to[out=60,in=270,max distance=10pt] (uqs);
\draw [inter] (b) to[out=170,in=270,max distance=20pt] (uqs)
                  to[out=90,in=40,max distance=17pt] (g);
\draw (g) node[below right] {\tikz {\draw [inter] (0,0) to[in=-40,out=-90,min distance=20pt] (0,0);}};
\draw [inter] (uqs) to[out=90,in=30,max distance=15pt] (gamma);
\draw [inter] (uqs) to[out=90,in=150,max distance=7pt] (W+);
\draw [inter] (uqs) .. controls (12,1.4) and (20.7,1.7) .. (qgnode.south)
                    to[out=270,in=10,max distance=7pt] (Z);
\draw [inter] (qgnode) to[out=270,in=20,max distance=10pt] (H);
\draw [inter] (qgnode) to[out=270,in=20,max distance=30pt] (W-);

\draw [under] (gamma) -- ++(15,0) node [inner sep=0] (gamw) {};
\draw [under] (gamw.west) to[out=0,in=90,max distance=30pt] (W-);
\draw [under] (gamw.west) to[out=0,in=250,max distance=20pt] (W+);
\draw [inter] (gamma) -- (gamw.east);
\draw [inter] (gamw.west) to[out=0,in=250,max distance=20pt] (W+);
\draw [inter] (gamw.west) to[out=0,in=90,max distance=30pt] (W-);

\node at (17,-.3) [inner sep=0] (wz-) {};
\node at (17,.6) [inner sep=0] (wz+) {};
\draw [under] (wz-) -- (wz+) to[out=270,in=160,max distance=10pt] (Z);
\draw [under] (H) to[out=30,in=-5,min distance=30pt] (W+);
\draw [inter] (W-) to[out=160,in=270,max distance=10pt] (wz-)
           to (wz+.north) to[out=90,in=200,max distance=10pt] (W+);
\node [above right] at (W+) {\tikz {
    \draw[under] (0,0) to[out=40,in=90,min distance=20pt] (0,0);
    \draw[inter] (0,0) to[out=40,in=90,min distance=20pt] (0,0);}};
\draw [inter] (wz-) to[out=90,in=200,max distance=10pt] (Z);
\draw [inter] (wz+) to[out=270,in=160,max distance=10pt] (Z);
\draw [inter] (W-) -- (H);
\draw (W-) node[below right] {\tikz {\draw [inter] (0,0) to[in=-40,out=-90,min distance=20pt] (0,0);}};
\draw [inter] (H) to[out=50,in=-10,max distance=5pt] (Z);
\draw [inter] (H) to[out=30,in=-5,min distance=30pt] (W+);

\draw (tau) ++(1.5,.25) node [inner sep=0] (lepga) {};
\draw [inter] (tau) to[out=70,in=180,max distance=3pt] (lepga)
                    to[out=0,in=270,max distance=10pt] (wz-);
\draw [inter] (mu) to[out=70,in=180,max distance=3pt] ++(1.5,.25) -- (lepga);
\draw [inter] (e) to[out=70,in=180,max distance=3pt] ++(1.5,.25)
              node [inner sep=0] (ega) {} -- (lepga);
\draw [inter] (lepga) to[out=0,in=152,max distance=10pt] (W-);

\draw (vmu) ++(-1.5,-.25) node [inner sep=0] (neul) {};
\draw [inter] (vmu) to[out=250,in=0,max distance=3pt] (neul);
\draw [inter] (vtau) to[out=185,in=0,min distance = 30pt] (neul);
\draw [inter] (ve) -- ++(0,-.5) node [inner sep=0] (neud) {};
\draw [inter] (neul.east) to[out=180,in=90,max distance=7pt] (neud);
\draw [inter] (neud.north) to [out=270,in=180,max distance=10pt] (ega);

\draw (g) ++(-1,0) node [inner sep=0] (gamlep) {};
\draw [inter] (e) to [out=250,in=0,max distance=3pt] ++(-1.5,-.25)
              node (lepneu) [inner sep=0] {};
\draw [inter] (mu) to[out=250,in=0,max distance=3pt] ++(-1.5,-.25) -- (lepneu);
\draw [inter] (tau) to[out=250,in=0,max distance=3pt] ++(-1.5,-.25) -- (lepneu);
\draw [inter] (lepneu.east) to[out=180,in=270,max distance=20pt] (neud.north);
\draw [inter] (lepneu) to[out=180,in=270,max distance=30pt] (gamlep);
\draw [inter] (gamma) to[out=230,in=90] (gamlep.south);
\end{pgfonlayer}

\draw[inter,cap=round] (-2,1) -- (0,1) node[right,kugray] {Interactions};

\draw (-1,1.4) node [quarko] (n) {} node {\tikz {\fill[natgreen] (0,0) -- ++(0,8pt) arc (90:-90:8pt) -- cycle; \draw +(180:8pt);}} node [quark] {n};
\node at (n) {\tikz {\draw [white,postaction={decorate,decoration={text along path,text align=center,text={|\tiny|Spin}}}] ++(270:10pt) arc (270:90:10pt); \draw (-1,0) (1,0);}};
\begin{scope}
\clip (n.center) -- ++(2,0) -- ++(0,.2) -- ++(-2,0) -- cycle;
\draw (n) node [gaugeo] {} node {\tikz {\filldraw[natcomp] (0,0) circle (8pt);}} node [gauge] {n};
\end{scope}
\begin{scope}
\clip (n.center) -- ++(2,0) -- ++(0,-.2) -- ++(-2,0) -- cycle;
\draw (n) node [leptono] {} node {\tikz {\fill[natblue] (0,0) -- ++(0,8pt) arc (90:-90:8pt) -- cycle; \draw +(180:8pt);}} node [lepton] {n};
\end{scope}
\begin{scope}
\clip (n.center) -- ++(-2,0) -- ++(0,-.2) -- ++(2,0) -- cycle;
\draw (n) node [higgso] {} node {\tikz {\shade[top color=natyellow,bottom color=white] +(100:12pt) -- +(135:8pt) arc (135:65:8pt) -- cycle; \draw +(-80:12pt); \draw[natyellow] (0,0) circle (8pt);}} node [higgs] {n};
\end{scope}

\draw (n) ++(160:20pt) node [right,natgreen] {\tiny Quarks};
\draw (n) ++(-160:20pt) node [right,natyellow] {\tiny Higgs boson};
\draw (n) ++(0:20pt) node [above right,natcomp] {\tiny Gauge bosons};
\draw (n) ++(0:20pt) node [below right,natblue] {\tiny Leptons};

\end{tikzpicture}
\end{scriptsize}\end{sffamily}
\end{infilsf}
\end{minipage}
\hfill\begin{minipage}[b]{.25\textwidth}
\caption{An overview of the particles of the Standard Model. The particles are arranged by mass and charge. Colour indicates particle type, the filling of the border indicates the spin of particles and lines are drawn between those particles that the Standard Model describes interactions between. The currently known maximum bounds on neutrino masses have been used to place the neutrinos in the mass direction. Table values from \cite{wikism}.\label{SMsum}}
\end{minipage}
\end{figure}

In its current form, the Standard Model makes no attempt to explain any physics beyond this.\footnote{The overview in figure~\ref{SMsum} includes massive neutrinos, which have been found experimentally, even though the Standard Model does not at present include them. There are, however, several proposed methods of extending the SM to do so.} The most obvious missing element is gravity, which finds no obvious expression in terms of Quantum Field Theory. Other seemingly missing elements come from astronomy, which bring observational evidence of a number of phenomena, such as dark matter and dark energy, which seem to require an explanation from within the realm of physics that the Standard Model describes, but for which the Standard Model offers no obvious source.

Disregarding these issues, which may be considered external its scope, the Standard Model has been remarkably successful as a model for high energy particle physics, withstanding decades of experimental tests and correctly predicting the existence and properties of several particles.\footnote{Most recently, the existence of the Higgs boson was confirmed experimentally. At the time of writing, confirmation of its predicted properties is still a work in progress.} In spite of those successes, the Standard Model is not without its issues.

As it is formulated, the SM depends on at least 19 numerical constants,\footnote{Not counting any additional constants needed to account for neutrino masses.} the value of which must be determined experimentally, since the model offers no insight into the origin of or relations between them. Worse still, as the SM is a perturbative model, higher order corrections, which tend to increase the predicted Higgs mass with no constraint save the Planck energy, must be included in the theory. To arrive at the observed Higgs mass in spite of this tendency, either some unknown physics exist between the Higgs mass scale and the Planck scale to constrain the Higgs mass, or the bare mass and couplings of the Higgs boson are very finely tuned to cancel the higher order contributions. This is one example of the hierarchy problem.

In the first case, we will obviously want to search for evidence of the postulated new, high-energy physics. In the latter case, we might expect\footnote{Or \emph{hope for}, depending on how invested we happen to be in the question of fine--tuning.} there to be some additional mechanism within the Standard Model that ensures that the bare Higgs mass and the other free parameters of the SM are given the proper value. Again, we will want to search for physics outside what is currently described by the SM, as a clue to what that mechanism might be.

There is also the possibility that neither of those mechanisms exist, since the Standard Model, strictly speaking, does not require them. In that case a search for new physics that discovers nothing is still a valuable, albeit less illuminating, result.

In this thesis, we shall approach the task of searching for physics beyond the Standard Model by introducing to it an extension via the effective Lagrangian approach. Specifically, we will introduce a $q\bar q\rightarrow\gamma\gamma$ point interaction, and then simulate collision experiments with the new interaction at various strengths, to see how the outcome is affected. We can then, finally, compare the results of those pseudoexperiments to the results of actual collision experiments performed at \textsc{cern}'s Large Hadron Collider, and look for signs of the same effects there.

\chapter{Theory}

While the detailed procedure for going from a general notion of expanding the Standard Model to creating a specific set of pseudoexperiments with which to compare experimental results are not part of the main thrust of this thesis, and will in any case be handled by various software tools in practice, what will follow is a brief overview of that process.

Since the new interaction will be introduced into the SM by the effective Lagrangian approach, the Lagrangian formulation of the Standard Model as a quantum field theory will be the starting point.

\section{The Lagrangian formulation of QFT}
In classical mechanics \cite{goldstein}, the Lagrangian formulation describes the path taken by a system between a given initial and final state---a particle with an initial and a final position, say---by finding the path between these states that minimises the action $S$, which is defined as the integral along a given path over the Lagrangian $L$:
\[S[q]=\int_\textrm{path}dtL[q,\dot{q}],\]
where $q$ is a generalised coordinate. In this picture, the Lagrangian encapsulates the dynamics of the system. It is related to the Hamiltonian $H$ by
\(L = p\dot q- H,\label{htol}\)
where $p$ is momentum.

In quantum mechanics, the picture of a system travelling along a single, well-defined path from an initial to a final configuration no longer applies. In stead, a probability of going from an initial state $\ket{q}$ to a final state $\ket{q\prime}$ can be found as the absolute square of the transition amplitude\footnote{At this point, we should note that the common notation where $\hbar = c = 1$ will be used from this chapter onwards.} \cite{sred:tramp}
\[A=\bra{q\prime}e^{-i\hat H(t\prime-t)}\ket{q\phantom\prime},\]
where $\hat H$ is the Hamiltonian of the system. Since the idea of a singular path for the system was abandoned, in stead imagine the system travelling along each possible path simultaneously, each with its own transition amplitude. The total transition amplitude, then, is the sum of all the individual transition amplitudes. This can be connected to the classical case by supposing that, for a system with a classical limit, the transition amplitudes of paths close to the classical path will tend to amplify one another, while paths far from it will tend to cancel out.

Through some notational gymnastics, which involve carving the path integral into an infinite number of time steps, each integrated over every possible configuration, and imposing some conditions on the Lagrangian, it can be shown \cite{sred:tramp} that the expression above can be written as
\(A=\int\,\mathcal Dq\,\exp\left[i\int_t^{t^\prime}dt\,[p(t)\dot q(t)-H(p(t),q(t))]\right],\label{e.Dq}\)
where the integral is over all paths with position $q$ at time $t$ and position $q\prime$ at time $t\prime$. We recognise the expression in the innermost integral from eq.~\eqref{htol}.

For a local theory, it is possible to write the Lagrangian as a spatial integral over the Lagrangian density:
\[L=\int \,d^3x\,\mathcal L.\]
Thus, the action can be written as
\(S=\int\,d^4x\,\mathcal L,\label{e.S}\)
which, unlike the previous expression for $S$, is manifestly Lorenz invariant, so long as $\mathcal L$ is a Lorenz scalar. Given the ubiquity of local quantum field theories, it is common when discussing quantum field theory to drop `density' from the name, and refer to $\mathcal L$ as the Lagrangian. Going forward, we will also follow that convention here.

Finally, to get the field theory aspect, replace the generalised coordinate $q$ with a field configuration ``coordinate'' $\phi(x)$, which depends on the Lorenz vector $x$. In short, \eqref{e.Dq} can then be written as
\(A=\int\mathcal D\phi\, e^{iS[\phi]}.\label{e.Dphi}\)

As was the case in classical mechanics, the behaviour of a theory is fully described by its Lagrangian (density), and several models can be combined by adding together their respective Lagrangians. So it is that the Standard Model is described by the SM Lagrangian $\mathcal L_{SM}$, which can be considered as a sum of several lesser Lagrangians that describe the separate sectors of the SM. However, before we venture too deeply into the Standard Model, we shall first consider an alternate way of looking at the content of the Lagrangian.

\section{Feynman diagrams}
When studying individual processes described by a theory, Feynman diagrams are a useful tool. So useful, in fact, that much of the software developed to simulate processes is formulated in terms of Feynman diagrams. To see how they work, consider the simple model described by the Lagrangian
\[\mathcal L= \half\partial^\mu\phi\partial_\mu\phi-\half m^2\phi^2-\frac{\lambda}{4!}\phi^4=\phi[\partial^2-m^2]\phi-\frac{\lambda}{4!}\phi^4\]
This is an example of a $\phi^4$ theory. The first terms in this Lagrangian, which involves two $\phi$s, describes a field propagating into another field, and the other one, which involves 4 $\phi$s, describes and interaction between four fields.

The goal will be to calculate the transition amplitude for a state $\ket{\phi_a}$ going to some other state $\ket{\phi_A}$. One procedure for doing so, which is inspired by \cite{wiki.feydiag}, is to start by writing the state in terms of an integral over all momentum states $\ket{k}$:
\[\ket{\phi}=\int\frac{d^4k}{(2\pi)^4}\tilde\phi(k)\ket{k},\]
where $k$ is a four-momentum. Using eq.~\eqref{e.Dphi}, the transition amplitude can be expressed as
\(A\propto\int\frac{d^4k_A}{(2\pi)^4}\frac{d^4k_a}{(2\pi)^4}\int\mathcal D\phi\,\phi(k_A)\phi(k_a)e^{iS}.\label{transa}\)

To express $S$ in terms of momenta, we go back to the Lagrangian and express $\phi$ in terms of its Fourier modes:
\[\phi(x)=\int\frac{d^4k}{(2\pi)^4}e^{ikx}\tilde\phi(k)\]
The Lagrangian is now
\begin{align*}\mathcal L=&-\int\frac{d^4k}{(2\pi)^4}\frac{d^4k\prime}{(2\pi)^4} e^{i(k+k\prime)x}\tilde\phi(k)[kk\prime +m^2]\tilde\phi(k\prime)\\
&-\frac{\lambda}{4!}\int\frac{d^4p_1\,d^4p_2\,d^4p_3\,d^4p_4}{(2\pi)^{16}}e^{i(p_1+p_2+p_3+p_4)x}\tilde\phi(p_1)\tilde\phi(p_2)\tilde\phi(p_3)\tilde\phi(p_4).
\end{align*}
Inserting this into eq.~\eqref{e.S}, it becomes clear that $x$ only appears as a phase factor, which means that integrating over $x$ only produces delta functions:
\begin{align*}
S=&\int\frac{d^4k}{(2\pi)^4}\tilde\phi(-k)(k^2-m^2)\tilde\phi(k)\\
&-\frac{\lambda}{4!}\int\frac{d^4p_1\,d^4p_2\,d^4p_3\,d^4p_4}{(2\pi)^{16}}\tilde\phi(p_1)\tilde\phi(p_2)\tilde\phi(p_3)\tilde\phi(p_4)\delta(p_1+p_2+p_3+p_4),
\end{align*}
where, in the first term, the delta function identified $k\prime=-k$. The first term of the action describes the free part of the theory, and the second term describes the interacting part, so we will call them $S_F$ and $S_I$, respectively. Using this expression in place of $S$, we can Taylor expand in $\lambda$:
\[e^{iS}=e^{i(S_F+S_I)}=e^{iS_F}\left(1-iS_I+\frac{(-iS_I)^2}{2}+\frac{(-iS_I)^3}{3!}+\frac{(-iS_I)^4}{4!}+\cdots\right)\]
This assumes that $\lambda$ is small enough to make the interaction merely a pertubation of the theory.


Inserting this back into eq.~\eqref{transa}, we get an expression for the transition amplitude expanded in powers of $\lambda$. If we call these terms $A_n$, so that $A\propto\sum_{n=0}^\infty A_n$, the first term of the expansion is
\[A_0=\int\frac{d^4k_A}{(2\pi)^4}\frac{d^4k_a}{(2\pi)^4}\underbrace{\int\mathcal D\phi\,\phi(k_A)\phi(k_a)e^{iS_F}}_{D_0}.\]
Looking more closely at the part labelled $D_0$ above, we can expand it to find that
\[D_0=\int\mathcal D\phi\,\phi(k_A)\phi(k_a)e^{\int\frac{d^4k}{(2\pi)^4}\phi(k)[k^2-m^2]\phi(-k)},\]
which is a Gaussian (functional) integral \cite{armbjorn}:
\[\int d^nx\,x^k\cdots x^{2N}e^{-\half x^iA_{ij}x^j}.\label{gausint}\]
As such, the integral has the following solution, provided that the participating momenta are identical:
\[D_0=\half\frac{\delta^4(k_a-k_A)}{k^2-m^2}\int\mathcal D\phi\,e^{iS_F},\]
where the delta function is introduced to ensure that the momenta are identical, as required. Introducing this delta function is equivalent to imposing momentum conservation. The remaining integral over the free action corresponds to the vacuum $0\rightarrow0$ process in free theory, and is a constant with respect to $\phi$. This constant can be interpreted as the vacuum energy content of all space, which we will nevertheless simply divide out:
\[A\propto\int\frac{d^4k_A}{(2\pi)^4}\frac{d^4k_a}{(2\pi)^4}\,\frac{\int\mathcal D\phi\,\phi(k_A)\phi(k_a)e^{iS}}{\int\mathcal D\phi\,e^{iS_F}}\]
With this, we find that 
\[D_0\Rightarrow\frac{\delta^4(k_a-k_A)}{k^2-m^2},\]
which is the propagator in momentum space.


Were we to carry out the momentum integrations over $D_0$, there would evidently be a singularity at $k^2=m^2$. This singularity can be avoided by slightly modifying the integration path. There are several ways of doing this, including Feynman's prescription, which yields the expression $1/(k^2-m^2+i\epsilon)$, the Feynman propagator in momentum space. 


The second term in the expansion is
\[A_1=\int\frac{d^4k_A}{(2\pi)^4}\frac{d^4k_a}{(2\pi)^4}\left(\prod_{n=1}^4\frac{d^4p_n}{(2\pi)^4}\right)\,D_1,\]
where\footnote{Here, $\phi^n$ is shorthand for a product of $n$ $\tilde\phi$ functions of separate momenta.}
\[D_1=-\frac{i\lambda}{4!}\delta^4(p_1+p_2+p_3+p_4)\frac{\int\mathcal D\phi\,\phi^6e^{iS_F}}{\int\mathcal D\phi\,e^{iS_F}}.\] 
Solving the Gaussian integral tell us that $D_1$ is equal to a sum of terms of the form 
\(-\ono{3!2^3}\frac{i\lambda}{4!}\delta^4(p_1+p_2+p_3+p_4)\frac{\delta^4(k_a-p_1)}{{k_a}^2-m^2}\frac{\delta^4(p_2-p_3)}{{p_2}^2-m^2}\frac{\delta^4(p_4-k_A)}{{k_A}^2-m^2},\label{1t1f}\)
where the momenta are paired in all possible combinations. There are $6!$ different combinations, but they fall into only two topologically inequivalent groups. The analogy to topology is expanded upon in figure~\ref{somethingorother}.

To see the analogy to topology, consider illustrating the flow of momentum given by terms of the form of eq.~\eqref{1t1f} as connected lines.

Indeed, consider this illustration
\(\text{
\begin{footnotesize}
\begin{tikzpicture}
\draw (1,.2) node[left]{$a$} -- ++(.2,0) ++(1.6,0) -- ++(.2,0) node[right]{$A$} 
      ++(-1,.5) -- +(45:.2) +(0,0) -- +(315:.2) +(0,0) -- +(225:.2) +(0,0) -- +(135:.2); 
     % ++(0,0) to[in=225,out=135,min distance=15mm,looseness=8] ++(0,0);
\end{tikzpicture}
\end{footnotesize}
},\label{unconn}\)
which lays out the external momenta $k_a$ and $k_A$ as line ends labelled $a$ and $A$. The line ends associated with the four $p$--momenta of the four--point interaction, which must conserve momentum internally, are connected from the outset in what we shall call a vertex. What remains is to connect these momenta in pairs as required by the last three $\delta$--functions.

Connecting the external momentum $a$ to one of the internal momenta leaves no option but to connect the other external momentum to another one of the internal momenta, which leaves only one option for connecting the remaining two internal momenta, resulting in this diagram
\(\text{
\begin{footnotesize}
\vspace{-1.5em}
\begin{tikzpicture}
\draw (1,.2) node[left]{$a$} -- ++(.2,0) node(a) {} ++(1.6,0) node(A) {} -- ++(.2,0) node[right]{$A$} 
      ++(-1,.5) -- +(45:.2) node(p4) {} +(0,0) -- +(315:.2) node(p3) {} +(0,0) -- +(225:.2) node(p2) {} +(0,0) -- +(135:.2) node(p1) {}; 
\draw[natgreen] (a) to[out=0,in=225] (p2) (A) to[out=180,in=315] (p3) (p4) to[in=135,out=45,looseness=4] (p1);
\end{tikzpicture}
\end{footnotesize}
}.\label{tadpole}\)
Note that the diagram can be drawn like this, regardless of which of the internal momenta are selected at each step, since we have neglected labelling the lines that emerge from the vertex. This gives us the topological equivalence of the diagrams that illustrate these $4!$ terms.

Making a choice distinct from the above, we can connect the external momentum $a$ to $A$, leaving the internal momenta to be connected to one another in any order, once again leaving $4!$ topologically equivalent options, and the diagram
\(\text{
\begin{footnotesize}
\vspace{-2.5em}
\begin{tikzpicture}
\draw (1,.2) node[left]{$a$} -- ++(.2,0) node(a) {} ++(1.6,0) node(A) {} -- ++(.2,0) node[right]{$A$} 
      ++(-1,.5) -- +(45:.2) node(p4) {} +(0,0) -- +(315:.2) node(p3) {} +(0,0) -- +(225:.2) node(p2) {} +(0,0) -- +(135:.2) node(p1) {}; 
\draw[natgreen] (a) -- (A)  (p3) to[in=45,out=315,looseness=4] (p4) (p1) to[out=135,in=225,looseness=4] (p2);
\end{tikzpicture}
\end{footnotesize}
}.\label{vacbub}\)

Looking for more options, we might wish to begin with the $A$ external line rather then $a$. However, if we connect the $A$ external to any of the external momenta emanating from the vertex, we will inevitably end up with a diagram topologically equivalent to diagram~\eqref{tadpole}, only with the lines laid down in a different order. Indeed, the option of laying down the lines in any order gives us an additional $3!$ terms covered by diagram~\eqref{tadpole} and \eqref{vacbub}, which both have three lines in them. In the case of diagram~\eqref{vacbub} though, we do count a factor $2$ too many, since the exchange of lines connected to the vertex were already covered by the $4!$ ways we connected those lines to the vertex.

Attempting to connect $A$ to $a$ leads us to diagrams equivalent to diagram~\eqref{vacbub}, only with the direction of the line between $a$ and $A$ reversed. In general, reversing the direction of the lines in our diagrams gives us a factor $2^3$ more diagrams for both types, although once again, we are including diagrams that were already counted in the $4!$ ways of connecting the vertex. In diagram~\eqref{tadpole} we overcount by a factor $2$ and in diagram~\eqref{vacbub} by a factor $2^2$.

This covers every possibility for connecting diagram~\eqref{unconn}, leaving us with
\[4!\,3!\,2^3\,\left(\ono2+\ono{2^3}\right)=4!\,6\,\,5=6!\]
distinct diagrams, which is also the number of distinct orderings of the momenta in eq.~\eqref{1t1f}, all of which can be drawn to look like either diagram~\eqref{tadpole} or \eqref{vacbub}. For the present purposes, we shall note that diagram~\eqref{tadpole} and \eqref{vacbub} are topologically inequivalent because they cannot be rearranged to look like one another without disconnecting a line from a vertex,\footnote{In the parlance of the topic, two topologies are inequivalent when there does not exist a continuous map that takes one into the other. However, to properly define all the terns in the previous sentence, we would need to venture somewhat beyond the scope of this thesis.}. In this context the ingoing and outgoing state labels are attached to the external lines. This becomes important when working with more in- and outgoing states and/or more vertices.

Looking back on the $D_0$ term, we note that it fits into the scheme of diagrams with this, quite straightforward, diagram
\[\text{
\begin{footnotesize}
\begin{tikzpicture}
\draw (1,.2) node[left]{$a$} -- ++(.2,0) node(a) {}
    ++(1.6,0) node(A) {} -- ++(.2,0) node[right]{$A$};
\draw[natgreen] (a) -- (A);
\end{tikzpicture}
\end{footnotesize}
}.\]
In fact, the $D_0$ term also carries the factor of $1/k^2-m^2$ that we see connected to the $\delta$--functions in eq.~\eqref{1t1f}. It seems natural, then, to associate the factor of $1/k^2-m^2$ with the lines in the diagrams above, thus having identified lines in our diagrams with the momentum space propagator. While we're at it, wa also associate the factor of $i\lambda$ with the vertex. This allows us to claim that the above diagram is completely equivalent to the $D_0$ term, and that the $D_1$ term can be expressed as
\[D_1=\ono2\left(\text{
\begin{footnotesize}
\begin{tikzpicture}[baseline=1.5em]
\draw (1,.2) node[left]{$a$} -- ++(.2,0) node(a) {} ++(1.6,0) node(A) {} -- ++(.2,0) node[right]{$A$} 
      ++(-1,.5) -- +(45:.2) node(p4) {} +(0,0) -- +(315:.2) node(p3) {} +(0,0) -- +(225:.2) node(p2) {} +(0,0) -- +(135:.2) node(p1) {}; 
\draw[natgreen] (a) to[out=0,in=225] (p2) (A) to[out=180,in=315] (p3) (p4) to[in=135,out=45,looseness=4] (p1);
\end{tikzpicture}
\end{footnotesize}
}\right)+\ono{2^3}\left(\text{
\begin{footnotesize}
\tikz[baseline=1.5em]{
\draw (1,.2) node[left]{$a$} -- ++(.2,0) node(a) {} ++(1.6,0) node(A) {} -- ++(.2,0) node[right]{$A$} 
      ++(-1,.5) -- +(45:.2) node(p4) {} +(0,0) -- +(315:.2) node(p3) {} +(0,0) -- +(225:.2) node(p2) {} +(0,0) -- +(135:.2) node(p1) {}; 
\draw[natgreen] (a) -- (A)  (p3) to[in=45,out=315,looseness=4] (p4) (p1) to[out=135,in=225,looseness=4] (p2);}
\end{footnotesize}
}\right),\]
where we have allowed the factors of $2^3$, $3!$ and $4!$ to cancel. The factors on each diagram are the symmetry factors of the diagrams, the number of ways we can reach the same diagram by two different transformations, and the factor by which each diagram counts less than $4!3!2^3$ terms, as we worked out above. These factors, it must be noted, are evident from the diagrams.

It turns out, then, that the diagrams contain the same information as the equations we started out with. These diagrams are due to Richard Feynman, and as such are referred to as Feynman diagrams.

Looking at eq.~\eqref{1t1f}, one of the internal $p$ momenta can not be fixed to the external $k$ momenta, which leaves this term proportional to a diverging integral, associated with the looping line in diagram~\eqref{tadpole}. There are established methods for renormalising these divergent terms, however for the present purposes, we note simply that for any process, there will among the lowest order terms that describe it be a tree level\footnote{In graph theory, a tree is a connected, loop-free graph.} diagram, which is then the leading order diagram for that process. Because we are working in a pertubative regime by assumption, loop-level diagrams of that process will act as higher order corrections to that leading order term.

Knowing how to construct the Feynman diagrams that describe the terms of the Lagrangian to some order, and knowing how to translate those Feynman diagrams, the possibility presents itself that we can derive the Lagrangian by constructing the proper diagrams, rather than going through the derivation above.

To do so, we set down the following Feynman rules, the rules for constructing the proper set of Feynman diagrams:

\begin{figure}[htb]
\hfill
\begin{tikzpicture}
\draw (-3,1) -- (-1,-1) (-3,-1) -- (-1,1);
\node[right] at (-1,0) {$=-i\lambda$};
\end{tikzpicture}
\hspace{5ex}
\begin{tikzpicture}
\node[right] at (-1,0) {$=\dfrac{1}{k^2-m^2}$};
\draw (-3,0) -- (-1,0);
\node at (0,-1) {};
\node at (0,1) {};
\end{tikzpicture}
\hfill \phantom{d}
\caption{The building blocks for Feynman diagrams in $\phi^4$ theory. Once constructed, find the momentum of each propagator by imposing momentum conservation at each vertex. Any momentum that cannot be related to one of the external momenta is integrated over.
\label{phi4rules}}
\end{figure}

\begin{enumerate}
\item Construct all topologically inequivalent diagrams in which the ingoing and outgoing states for the process in question and the proper number of vertices for the present order in $\lambda$ are connected by propagator lines.
\item Impose momentum conservation at all vertices and across all propagators. As was already noted once, this is equivalent to introducing delta functions over the momenta.
\item Determine the symmetry factor for each diagram.
\item Construct for each diagram its value by taking the product of the values for each element of the diagram from figure~\ref{phi4rules}. Integrate over all momenta that have not been related to external momenta with measure $d^4p/(2\pi)^4$. Divide by the symmetry factor associated with the diagram.
\end{enumerate}


For example, the $2\rightarrow2$ transition amplitude to zeroth order in $\lambda$ is given by the Feynman diagrams in figure~\ref{efeydig1}.

\begin{figure}[htp]
\begin{footnotesize}\begin{center}
\begin{tikzpicture}
\draw (-4,1) node[left]{$a$} -- (-2,1) node[right]{$A$};
\draw (-4,-1) node[left]{$b$} -- (-2,-1) node[right]{$B$};
\draw (0,0) node{{\normalsize $+$}};
\draw (2,1) node[left]{$a$} -- (4,-1) node[right]{$B$};
\draw[line width=5pt,white] (2,-1) -- (4,1) ;
\draw (2,-1) node[left]{$b$} -- (4,1) node[right]{$A$};
\end{tikzpicture}
\end{center}\end{footnotesize}
\caption{The Feynman diagrams associated with the first term in the expansion of the $2\rightarrow2$ transition amplitude. Once again, connected states are required to have the same momentum, however with more particles going into and coming out of the process, there are more than one way of connecting the ingoing and outgoing states.
\label{efeydig1}}
\end{figure}

The value of these diagrams, using the rules, is
\[\frac{\delta^4(k_1-k_A)}{{k_1}^2-m^2}\frac{\delta^4(k_2-k_B)}{{k_2}^2-m^2}+\frac{\delta^4(k_1-k_B)}{{k_1}^2-m^2}\frac{\delta^4(k_2-k_A)}{{k_2}^2-m^2},\]
which is indeed what we would get from solving the Gaussian integral. At the next order in $\lambda$, we can build the diagrams in fig.~\ref{efeydig2}.

\begin{figure}[htp]
\begin{footnotesize}
\begin{minipage}{.09\textwidth}
\normalsize \hfill
\end{minipage}
\begin{minipage}{.9\textwidth}
\begin{center}
\begin{tikzpicture}[scale=.75]
\draw (-6.5,0) to [in=225,out=315,min distance=25mm,looseness=8] (-6.5,0) to [in=135,out=45,min distance=25mm,looseness=8] (-6.5,0);
\draw (-5,0) node{\normalsize$\times$};
\draw (-4,0) node{$\left(\text{\tikz[scale=.75] \draw (0,1) (0,-1);}\right.$};
\draw (-3,1) node[left]{$a$} -- (-1,1) node[right]{$A$};
\draw (-3,-1) node[left]{$b$} -- (-1,-1) node[right]{$B$};
\draw (0,0) node{{\normalsize $+$}};
\draw (1,1) node[left]{$a$} -- (3,-1) node[right]{$B$};
\draw[line width=5pt,white] (1,-1) -- (3,1) ;
\draw (1,-1) node[left]{$b$} -- (3,1) node[right]{$A$};
\draw (4,0) node{$\left)\text{\tikz[scale=.75] \draw (0,1) (0,-1);}\right.$};
\draw (5.5,0);
\end{tikzpicture}
\end{center}
\end{minipage}

\vspace{.5em}

\begin{minipage}{.09\textwidth}
\normalsize $+$
\end{minipage}
\begin{minipage}{.9\textwidth}
\begin{center}
\begin{tikzpicture}[scale=.75]
\draw (1,1) node[left]{$a$} -- ++(2,0) node[right]{$A$}
      ++(-2,-2) node[left]{$b$} -- 
      ++(1,0) to[in=45,out=135,min distance=15mm,looseness=8] ++(0,0) --
      ++(1,0) node[right]{$B$};
\draw (4,0) node{\normalsize $+$};
\draw (5,1) node[left]{$a$} -- 
      ++(1,0) to[in=225,out=315,min distance=15mm,looseness=8] ++(0,0) --
      ++(1,0) node[right]{$A$}
      ++(-2,-2) node[left]{$b$} -- 
      ++(2,0) node[right]{$B$};
\draw (8,0) node{\normalsize $+$};
\draw (9,1) node(p1)[left]{$a$} -- ++(2,-2) node[right]{$B$};
\draw[line width=5pt,white] (p1.east) ++(0,-2) -- ++(2,2);
\draw (p1.east) ++(0,-2) node[left]{$b$} --
      +(.5,.5) to[in=180,out=90,min distance=15mm,looseness=8] +(0.5,0.5) --
      ++(1.5,1.5) node[right]{$A$};
\draw (12,0) node{\normalsize $+$};
\draw (13,1) node(p2)[left]{$a$} -- 
      +(.5,-.5) to[in=180,out=270,min distance=15mm,looseness=8] +(0.5,-0.5) --
      ++(1.5,-1.5) node[right]{$B$};
\draw[line width=5pt,white] (p2.east) ++(0,-2) -- ++(2,2);
\draw (p2.east) ++(0,-2) node[left]{$b$} --
      ++(2,2) node[right]{$A$};
\end{tikzpicture}
\end{center}
\end{minipage}

\vspace{.5em}

\begin{minipage}{.09\textwidth}
\normalsize $+$
\end{minipage}
\begin{minipage}{.9\textwidth}
\begin{center}
\begin{tikzpicture}[scale=.75]
\draw (1,1) node[left]{$a$} -- (3,-1) node[right]{$B$};
\draw (1,-1) node[left]{$b$} -- (3,1) node[right]{$A$};
\end{tikzpicture}
\end{center}
\end{minipage}

\end{footnotesize}
\caption{The Feynman diagrams associated with the second term in the expansion of the $2\rightarrow2$ transition amplitude. The joining of four momenta by the last delta function is illustrated by having four lines meet in a point.
\label{efeydig2}}
\end{figure}

Of these diagrams, those in the first two lines are simply those of fig.~\ref{efeydig1} with loops added, and can be considered the one-loop, or next-to leading order, part of those processes.
 The last diagram is the only one that we have not seen before, and it introduces a new feature. In all the diagrams we have examined so far, momentum conservation has required that one of the final states be exactly identical to one of the initial states. Not so in the final diagram, where the delta function at the vertex only requires that the sum of momenta, here defined so that positive momenta flow toward the vertex, is zero. In $S$-matrix notation, where the transition matrix $S$, which transits an initial state into a final state, can be written as
\[S=1+iT,\]
this last diagram is the first part of the non-trivial $T$-matrix.

As for the disconnected vacuum bubble seen in the top row, and in diagram~\eqref{vacbub}, note that the diagram(s) that the vacuum bubble multiplies are the diagrams for the process from the preceeding orders. At higher orders in the expansion, we will find it as a repeated pattern that the diagrams from the previous orders reoccur, multiplied with various combinations of vacuum bubble diagrams. Combining the vacuum bubble contributions on any one diagram at all orders, we find that they can be written as the exponential of the sum of all possible vacuum bubbles \cite{sred:freediagexp}. The same result can be reached by writing
\[\int\mathcal D\phi\,e^{i(S_F+S_I)},\]
the expression for the $0\rightarrow0$ process to all orders. Since this is another constant, we can divide it out like we did with the vacuum normalisation, making the expression for the transition amplitude now
\[A\propto\int\frac{d^4k_A}{(2\pi)^4}\frac{d^4k_a}{(2\pi)^4}\frac{\int\mathcal D\phi\,\phi(k_A)\phi(k_a)e^{iS}}{\int\mathcal D\phi\,e^{iS}}.\]

With that, we find that the non-trivial part of the expression, the $T$-matrix from above, can be found by taking only the connected diagrams---the diagrams in which it is possible to go from any one part to any other along connected lines---into account. Using just this process in the transition amplitude, we can calculate the probability of the system going to some final state specifically through the process described by this diagram. That probability will depend on $\lambda$, which means that if we have a way of distinguishing those events in a detector that went through this process from those that did not---looking at just the diagrams found so far, the fact that the latest diagram allows exchange of momentum between the two particles would make those events stand out from the rest---provided that contributions from even higher order terms do not muddle the picture too much, we would be able to say something about the value of $\lambda$.

The treatment so far obviously deals with a very simple model. The development of the full Standard Model is the subject of entire textbooks \cite{srednicki}, and we will not go into further detail here.

The practical upshot, though, is that the Feynman rules extend to cover the Standard Model by introducing several types of fields, which can be represented in Feynman diagrams by some new styles of lines (dashed, wavy, curled, etc.). Charge is introduced by adding a direction to the lines associated with charged particles---since reversing the charge of a particle is equivalent to reversing the time direction. These new fields interact in several new types of vertices, weighted by three coupling constants: the electromagnetic coupling $\alpha_\text{QED}$, the weak coupling constant $\alpha_W$ and the strong coupling constant $\alpha_s$. With this, we can show the Standard Model processes that produce the preponderance of two-photon final states with the two diagrams in figure~\ref{smfeyn}.


\begin{figure}[htb]
\parbox[t]{.45\textwidth}{\begin{center}\begin{footnotesize}\begin{tikzpicture} [>=triangle 45]
\draw[>-] (-1,1) -- (0,1);
\draw[->] (0,1) -- (0,0);
\draw[<-] (-1,-1) -- (0,-1) -- (0,0);
\draw (-2,1) node[left] {$q$} -- (-1,1);
\draw (-2,-1)  node[left] {$\bar q$} -- (-1,-1);
\draw[snake=coil,segment aspect=0] (0,1) -- (2,1) node[right] {$\gamma$};
\draw[snake=coil,segment aspect=0] (0,-1) -- (2,-1) node[right] {$\gamma$}; 
\end{tikzpicture}
\end{footnotesize}\end{center}
\subcaption{SM contribution at tree level. \label{lofeyn}}}\hfill
\parbox[t]{.52\textwidth}{\begin{center}\begin{footnotesize}
\begin{tikzpicture} [>=triangle 45]
\draw[>-] (1,1) node[below]{$q$} -- (2,1);
\draw[decorate, decoration={coil,amplitude=2pt, segment length=2.68pt}] 
    (-2,1) node[left]{$g$} -- (0,1) ;
\draw[decorate, decoration={coil,amplitude=2pt, segment length=2.68pt}] 
    (-2,-1) node[left]{$g$} -- (0,-1); 
\draw[<-] (1,-1) node[above]{$\bar q$} -- (2,-1);
\draw (0,1) -- (1,1);
\draw (0,-1) -- (1,-1);
\draw[-<] (0,1) -- (0,0);
\draw (0,0) -- (0,-1);
\draw[->] (2,1) -- (2,0);
\draw (2,0) -- (2,-1);
\draw[decorate, decoration={snake}] (2,1) -- (4,1) node[right]{$\gamma$};
\draw[decorate, decoration={snake}] (2,-1) -- (4,-1) node[right]{$\gamma$};
\end{tikzpicture}
\end{footnotesize}\end{center}
\subcaption{SM contribution at loop level. $g$s mark gluons.\label{boxdiag}}}\hfill
\caption{ Feynman diagrams for the two leading Standard Model processes that produce a $\gamma\gamma$ final state.\label{smfeyn}}
\end{figure}

We can get a feel for the relative strength of these two diagrams by turning to two sets of simulated collisions available from the \atlas{} collaboration.
%\footnote{The internal ATLAS names are \verbatim{mc12_8TeV.129180.Pythia8_AU2CTEQ6L1_gammagamma_2DP20.merge.NTUP_PHOTON.e1199_s1479_s1470_r3542_r3549_p1344} and \verbatim{mc12_8TeV.146800.Pythia8_AU2CTEQ6L1_GamGamBox_pT35pT20.merge.NTUP_PHOTON.e1222_s1469_s1470_r3542_r3549_p1344}.}
Plotted in figure~\ref{boxpart} are the distribution of the invariant masses of photon pairs, defined as \cite{marshaw:invmass}
\begin{align*} 
M_{\gamma\gamma}&=\sqrt{(E_1+E_2)^2+|\mathbf p_1+\mathbf p_2|^2},
\intertext{which, in the case of massless particles, can be rewritten as}
&=\sqrt{2p_1p_2(1-\cos\theta)}. \label{sinvmass}
\end{align*}

\begin{figure}[htp]
\begin{minipage}[b]{.69\textwidth}
\begin{sffamily}
\pgfdeclareplotmark{cross} {
\pgfpathmoveto{\pgfpoint{-0.3\pgfplotmarksize}{\pgfplotmarksize}}
\pgfpathlineto{\pgfpoint{+0.3\pgfplotmarksize}{\pgfplotmarksize}}
\pgfpathlineto{\pgfpoint{+0.3\pgfplotmarksize}{0.3\pgfplotmarksize}}
\pgfpathlineto{\pgfpoint{+1\pgfplotmarksize}{0.3\pgfplotmarksize}}
\pgfpathlineto{\pgfpoint{+1\pgfplotmarksize}{-0.3\pgfplotmarksize}}
\pgfpathlineto{\pgfpoint{+0.3\pgfplotmarksize}{-0.3\pgfplotmarksize}}
\pgfpathlineto{\pgfpoint{+0.3\pgfplotmarksize}{-1.\pgfplotmarksize}}
\pgfpathlineto{\pgfpoint{-0.3\pgfplotmarksize}{-1.\pgfplotmarksize}}
\pgfpathlineto{\pgfpoint{-0.3\pgfplotmarksize}{-0.3\pgfplotmarksize}}
\pgfpathlineto{\pgfpoint{-1.\pgfplotmarksize}{-0.3\pgfplotmarksize}}
\pgfpathlineto{\pgfpoint{-1.\pgfplotmarksize}{0.3\pgfplotmarksize}}
\pgfpathlineto{\pgfpoint{-0.3\pgfplotmarksize}{0.3\pgfplotmarksize}}
\pgfpathclose
\pgfusepathqstroke
}
\pgfdeclareplotmark{cross*} {
\pgfpathmoveto{\pgfpoint{-0.3\pgfplotmarksize}{\pgfplotmarksize}}
\pgfpathlineto{\pgfpoint{+0.3\pgfplotmarksize}{\pgfplotmarksize}}
\pgfpathlineto{\pgfpoint{+0.3\pgfplotmarksize}{0.3\pgfplotmarksize}}
\pgfpathlineto{\pgfpoint{+1\pgfplotmarksize}{0.3\pgfplotmarksize}}
\pgfpathlineto{\pgfpoint{+1\pgfplotmarksize}{-0.3\pgfplotmarksize}}
\pgfpathlineto{\pgfpoint{+0.3\pgfplotmarksize}{-0.3\pgfplotmarksize}}
\pgfpathlineto{\pgfpoint{+0.3\pgfplotmarksize}{-1.\pgfplotmarksize}}
\pgfpathlineto{\pgfpoint{-0.3\pgfplotmarksize}{-1.\pgfplotmarksize}}
\pgfpathlineto{\pgfpoint{-0.3\pgfplotmarksize}{-0.3\pgfplotmarksize}}
\pgfpathlineto{\pgfpoint{-1.\pgfplotmarksize}{-0.3\pgfplotmarksize}}
\pgfpathlineto{\pgfpoint{-1.\pgfplotmarksize}{0.3\pgfplotmarksize}}
\pgfpathlineto{\pgfpoint{-0.3\pgfplotmarksize}{0.3\pgfplotmarksize}}
\pgfpathclose
\pgfusepathqfillstroke
}
\pgfdeclareplotmark{newstar} {
\pgfpathmoveto{\pgfqpoint{0pt}{\pgfplotmarksize}}
\pgfpathlineto{\pgfqpointpolar{44}{0.5\pgfplotmarksize}}
\pgfpathlineto{\pgfqpointpolar{18}{\pgfplotmarksize}}
\pgfpathlineto{\pgfqpointpolar{-20}{0.5\pgfplotmarksize}}
\pgfpathlineto{\pgfqpointpolar{-54}{\pgfplotmarksize}}
\pgfpathlineto{\pgfqpointpolar{-90}{0.5\pgfplotmarksize}}
\pgfpathlineto{\pgfqpointpolar{234}{\pgfplotmarksize}}
\pgfpathlineto{\pgfqpointpolar{198}{0.5\pgfplotmarksize}}
\pgfpathlineto{\pgfqpointpolar{162}{\pgfplotmarksize}}
\pgfpathlineto{\pgfqpointpolar{134}{0.5\pgfplotmarksize}}
\pgfpathclose
\pgfusepathqstroke
}
\pgfdeclareplotmark{newstar*} {
\pgfpathmoveto{\pgfqpoint{0pt}{\pgfplotmarksize}}
\pgfpathlineto{\pgfqpointpolar{44}{0.5\pgfplotmarksize}}
\pgfpathlineto{\pgfqpointpolar{18}{\pgfplotmarksize}}
\pgfpathlineto{\pgfqpointpolar{-20}{0.5\pgfplotmarksize}}
\pgfpathlineto{\pgfqpointpolar{-54}{\pgfplotmarksize}}
\pgfpathlineto{\pgfqpointpolar{-90}{0.5\pgfplotmarksize}}
\pgfpathlineto{\pgfqpointpolar{234}{\pgfplotmarksize}}
\pgfpathlineto{\pgfqpointpolar{198}{0.5\pgfplotmarksize}}
\pgfpathlineto{\pgfqpointpolar{162}{\pgfplotmarksize}}
\pgfpathlineto{\pgfqpointpolar{134}{0.5\pgfplotmarksize}}
\pgfpathclose
\pgfusepathqfillstroke
}\begin{tiny}
\begin{tikzpicture}[x=.1\textwidth,y=.1\textwidth]
%\definecolor{c}{rgb}{1,1,1};
%\draw [color=c, fill=c] (0,0) rectangle (10,5.92133);
%\draw [color=c, fill=c] (1,0.592133) rectangle (9,5.32919);
\definecolor{c}{rgb}{0,0,0};
\draw [c] (1,0.592133) -- (1,5.32919) -- (9,5.32919) -- (9,0.592133) -- (1,0.592133);
\definecolor{c}{rgb}{1,1,1};
\draw [color=c, fill=c] (1,0.592133) rectangle (9,5.32919);
\definecolor{c}{rgb}{0,0,0};
\draw [c] (1,0.592133) -- (1,5.32919) -- (9,5.32919) -- (9,0.592133) -- (1,0.592133);
\definecolor{c}{named}{natgreen};
\draw [c] (1,0.592133) -- (1.032,0.592133) -- (1.032,0.592133) -- (1.064,0.592133) -- (1.064,0.592133) -- (1.096,0.592133) -- (1.096,0.592133) -- (1.128,0.592133) -- (1.128,0.592133) -- (1.16,0.592133) -- (1.16,0.596703) -- (1.192,0.596703) --
 (1.192,0.610416) -- (1.224,0.610416) -- (1.224,0.614987) -- (1.256,0.614987) -- (1.256,0.614987) -- (1.288,0.614987) -- (1.288,0.614987) -- (1.32,0.614987) -- (1.32,0.624129) -- (1.352,0.624129) -- (1.352,0.665267) -- (1.384,0.665267) --
 (1.384,0.624129) -- (1.416,0.624129) -- (1.416,0.6287) -- (1.448,0.6287) -- (1.448,0.646983) -- (1.48,0.646983) -- (1.48,0.656125) -- (1.512,0.656125) -- (1.512,0.642413) -- (1.544,0.642413) -- (1.544,0.710976) -- (1.576,0.710976) --
 (1.576,0.688122) -- (1.608,0.688122) -- (1.608,0.67898) -- (1.64,0.67898) -- (1.64,0.651554) -- (1.672,0.651554) -- (1.672,0.724689) -- (1.704,0.724689) -- (1.704,0.747543) -- (1.736,0.747543) -- (1.736,0.752114) -- (1.768,0.752114) --
 (1.768,0.953234) -- (1.8,0.953234) -- (1.8,1.44232) -- (1.832,1.44232) -- (1.832,2.71303) -- (1.864,2.71303) -- (1.864,3.70035) -- (1.896,3.70035) -- (1.896,4.31742) -- (1.928,4.31742) -- (1.928,4.74709) -- (1.96,4.74709) -- (1.96,4.7928) --
 (1.992,4.7928) -- (1.992,5.10362) -- (2.024,5.10362) -- (2.024,4.94364) -- (2.056,4.94364) -- (2.056,4.71966) -- (2.088,4.71966) -- (2.088,4.59625) -- (2.12,4.59625) -- (2.12,4.44541) -- (2.152,4.44541) -- (2.152,4.29457) -- (2.184,4.29457) --
 (2.184,4.00203) -- (2.216,4.00203) -- (2.216,3.93804) -- (2.248,3.93804) -- (2.248,3.77348) -- (2.28,3.77348) -- (2.28,3.27068) -- (2.312,3.27068) -- (2.312,3.42152) -- (2.344,3.42152) -- (2.344,3.18841) -- (2.376,3.18841) -- (2.376,2.94158) --
 (2.408,2.94158) -- (2.408,2.94615) -- (2.44,2.94615) -- (2.44,2.76331) -- (2.472,2.76331) -- (2.472,2.47535) -- (2.504,2.47535) -- (2.504,2.50734) -- (2.536,2.50734) -- (2.536,2.26508) -- (2.568,2.26508) -- (2.568,2.39307) -- (2.6,2.39307) --
 (2.6,2.03197) -- (2.632,2.03197) -- (2.632,2.07768) -- (2.664,2.07768) -- (2.664,1.82628) -- (2.696,1.82628) -- (2.696,1.87656) -- (2.728,1.87656) -- (2.728,1.90398) -- (2.76,1.90398) -- (2.76,1.64801) -- (2.792,1.64801) -- (2.792,1.59316) --
 (2.824,1.59316) -- (2.824,1.59773) -- (2.856,1.59773) -- (2.856,1.49717) -- (2.888,1.49717) -- (2.888,1.43775) -- (2.92,1.43775) -- (2.92,1.32348) -- (2.952,1.32348) -- (2.952,1.30977) -- (2.984,1.30977) -- (2.984,1.29148) -- (3.016,1.29148) --
 (3.016,1.25491) -- (3.048,1.25491) -- (3.048,1.16807) -- (3.08,1.16807) -- (3.08,1.14521) -- (3.112,1.14521) -- (3.112,1.17264) -- (3.144,1.17264) -- (3.144,1.1315) -- (3.176,1.1315) -- (3.176,1.18635) -- (3.208,1.18635) -- (3.208,1.08122) --
 (3.24,1.08122) -- (3.24,1.10865) -- (3.272,1.10865) -- (3.272,1.07665) -- (3.304,1.07665) -- (3.304,0.994372) -- (3.336,0.994372) -- (3.336,0.957805) -- (3.368,0.957805) -- (3.368,0.848103) -- (3.4,0.848103) -- (3.4,0.957805) -- (3.432,0.957805) --
 (3.432,0.838962) -- (3.464,0.838962) -- (3.464,0.934951) -- (3.496,0.934951) -- (3.496,0.870958) -- (3.528,0.870958) -- (3.528,0.870958) -- (3.56,0.870958) -- (3.56,0.884671) -- (3.592,0.884671) -- (3.592,0.884671) -- (3.624,0.884671) --
 (3.624,0.806965) -- (3.656,0.806965) -- (3.656,0.761256) -- (3.688,0.761256) -- (3.688,0.797823) -- (3.72,0.797823) -- (3.72,0.774969) -- (3.752,0.774969) -- (3.752,0.820678) -- (3.784,0.820678) -- (3.784,0.738402) -- (3.816,0.738402) --
 (3.816,0.793252) -- (3.848,0.793252) -- (3.848,0.765827) -- (3.88,0.765827) -- (3.88,0.765827) -- (3.912,0.765827) -- (3.912,0.733831) -- (3.944,0.733831) -- (3.944,0.770398) -- (3.976,0.770398) -- (3.976,0.710976) -- (4.008,0.710976) --
 (4.008,0.720118) -- (4.04,0.720118) -- (4.04,0.701834) -- (4.072,0.701834) -- (4.072,0.715547) -- (4.104,0.715547) -- (4.104,0.742972) -- (4.136,0.742972) -- (4.136,0.724689) -- (4.168,0.724689) -- (4.168,0.710976) -- (4.2,0.710976) --
 (4.2,0.706405) -- (4.232,0.706405) -- (4.232,0.710976) -- (4.264,0.710976) -- (4.264,0.701834) -- (4.296,0.701834) -- (4.296,0.697263) -- (4.328,0.697263) -- (4.328,0.674409) -- (4.36,0.674409) -- (4.36,0.669838) -- (4.392,0.669838) --
 (4.392,0.674409) -- (4.424,0.674409) -- (4.424,0.669838) -- (4.456,0.669838) -- (4.456,0.701834) -- (4.488,0.701834) -- (4.488,0.665267) -- (4.52,0.665267) -- (4.52,0.633271) -- (4.552,0.633271) -- (4.552,0.665267) -- (4.584,0.665267) --
 (4.584,0.633271) -- (4.616,0.633271) -- (4.616,0.688122) -- (4.648,0.688122) -- (4.648,0.669838) -- (4.68,0.669838) -- (4.68,0.637842) -- (4.712,0.637842) -- (4.712,0.633271) -- (4.744,0.633271) -- (4.744,0.656125) -- (4.776,0.656125) --
 (4.776,0.656125) -- (4.808,0.656125) -- (4.808,0.6287) -- (4.84,0.6287) -- (4.84,0.642413) -- (4.872,0.642413) -- (4.872,0.633271) -- (4.904,0.633271) -- (4.904,0.624129) -- (4.936,0.624129) -- (4.936,0.619558) -- (4.968,0.619558) --
 (4.968,0.633271) -- (5,0.633271) -- (5,0.646983) -- (5.032,0.646983) -- (5.032,0.6287) -- (5.064,0.6287) -- (5.064,0.665267) -- (5.096,0.665267) -- (5.096,0.624129) -- (5.128,0.624129) -- (5.128,0.614987) -- (5.16,0.614987) -- (5.16,0.651554) --
 (5.192,0.651554) -- (5.192,0.624129) -- (5.224,0.624129) -- (5.224,0.624129) -- (5.256,0.624129) -- (5.256,0.642413) -- (5.288,0.642413) -- (5.288,0.624129) -- (5.32,0.624129) -- (5.32,0.614987) -- (5.352,0.614987) -- (5.352,0.605845) --
 (5.384,0.605845) -- (5.384,0.614987) -- (5.416,0.614987) -- (5.416,0.601274) -- (5.448,0.601274) -- (5.448,0.610416) -- (5.48,0.610416) -- (5.48,0.605845) -- (5.512,0.605845) -- (5.512,0.633271) -- (5.544,0.633271) -- (5.544,0.610416) --
 (5.576,0.610416) -- (5.576,0.610416) -- (5.608,0.610416) -- (5.608,0.610416) -- (5.64,0.610416) -- (5.64,0.605845) -- (5.672,0.605845) -- (5.672,0.610416) -- (5.704,0.610416) -- (5.704,0.610416) -- (5.736,0.610416) -- (5.736,0.592133) --
 (5.768,0.592133) -- (5.768,0.601274) -- (5.8,0.601274) -- (5.8,0.596703) -- (5.832,0.596703) -- (5.832,0.610416) -- (5.864,0.610416) -- (5.864,0.605845) -- (5.896,0.605845) -- (5.896,0.596703) -- (5.928,0.596703) -- (5.928,0.610416) --
 (5.96,0.610416) -- (5.96,0.601274) -- (5.992,0.601274) -- (5.992,0.596703) -- (6.024,0.596703) -- (6.024,0.605845) -- (6.056,0.605845) -- (6.056,0.614987) -- (6.088,0.614987) -- (6.088,0.601274) -- (6.12,0.601274) -- (6.12,0.610416) --
 (6.152,0.610416) -- (6.152,0.601274) -- (6.184,0.601274) -- (6.184,0.596703) -- (6.216,0.596703) -- (6.216,0.605845) -- (6.248,0.605845) -- (6.248,0.605845) -- (6.28,0.605845) -- (6.28,0.596703) -- (6.312,0.596703) -- (6.312,0.601274) --
 (6.344,0.601274) -- (6.344,0.596703) -- (6.376,0.596703) -- (6.376,0.605845) -- (6.408,0.605845) -- (6.408,0.592133) -- (6.44,0.592133) -- (6.44,0.596703) -- (6.472,0.596703) -- (6.472,0.601274) -- (6.504,0.601274) -- (6.504,0.610416) --
 (6.536,0.610416) -- (6.536,0.605845) -- (6.568,0.605845) -- (6.568,0.601274) -- (6.6,0.601274) -- (6.6,0.601274) -- (6.632,0.601274) -- (6.632,0.610416) -- (6.664,0.610416) -- (6.664,0.596703) -- (6.696,0.596703) -- (6.696,0.601274) --
 (6.728,0.601274) -- (6.728,0.596703) -- (6.76,0.596703) -- (6.76,0.592133) -- (6.792,0.592133) -- (6.792,0.601274) -- (6.824,0.601274) -- (6.824,0.596703) -- (6.856,0.596703) -- (6.856,0.601274) -- (6.888,0.601274) -- (6.888,0.596703) --
 (6.92,0.596703) -- (6.92,0.605845) -- (6.952,0.605845) -- (6.952,0.596703) -- (6.984,0.596703) -- (6.984,0.596703) -- (7.016,0.596703) -- (7.016,0.592133) -- (7.048,0.592133) -- (7.048,0.592133) -- (7.08,0.592133) -- (7.08,0.596703) --
 (7.112,0.596703) -- (7.112,0.592133) -- (7.144,0.592133) -- (7.144,0.592133) -- (7.176,0.592133) -- (7.176,0.601274) -- (7.208,0.601274) -- (7.208,0.592133) -- (7.24,0.592133) -- (7.24,0.592133) -- (7.272,0.592133) -- (7.272,0.596703) --
 (7.304,0.596703) -- (7.304,0.596703) -- (7.336,0.596703) -- (7.336,0.596703) -- (7.368,0.596703) -- (7.368,0.601274) -- (7.4,0.601274) -- (7.4,0.605845) -- (7.432,0.605845) -- (7.432,0.596703) -- (7.464,0.596703) -- (7.464,0.592133) --
 (7.496,0.592133) -- (7.496,0.601274) -- (7.528,0.601274) -- (7.528,0.596703) -- (7.56,0.596703) -- (7.56,0.596703) -- (7.592,0.596703) -- (7.592,0.592133) -- (7.624,0.592133) -- (7.624,0.592133) -- (7.656,0.592133) -- (7.656,0.596703) --
 (7.688,0.596703) -- (7.688,0.592133) -- (7.72,0.592133) -- (7.72,0.592133) -- (7.752,0.592133) -- (7.752,0.592133) -- (7.784,0.592133) -- (7.784,0.592133) -- (7.816,0.592133) -- (7.816,0.592133) -- (7.848,0.592133) -- (7.848,0.596703) --
 (7.88,0.596703) -- (7.88,0.592133) -- (7.912,0.592133) -- (7.912,0.592133) -- (7.944,0.592133) -- (7.944,0.592133) -- (7.976,0.592133) -- (7.976,0.596703) -- (8.008,0.596703) -- (8.008,0.601274) -- (8.04,0.601274) -- (8.04,0.592133) --
 (8.072,0.592133) -- (8.072,0.592133) -- (8.104,0.592133) -- (8.104,0.592133) -- (8.136,0.592133) -- (8.136,0.596703) -- (8.168,0.596703) -- (8.168,0.592133) -- (8.2,0.592133) -- (8.2,0.596703) -- (8.232,0.596703) -- (8.232,0.592133) --
 (8.264,0.592133) -- (8.264,0.596703) -- (8.296,0.596703) -- (8.296,0.592133) -- (8.328,0.592133) -- (8.328,0.592133) -- (8.36,0.592133) -- (8.36,0.592133) -- (8.392,0.592133) -- (8.392,0.605845) -- (8.424,0.605845) -- (8.424,0.592133) --
 (8.456,0.592133) -- (8.456,0.592133) -- (8.488,0.592133) -- (8.488,0.601274) -- (8.52,0.601274) -- (8.52,0.592133) -- (8.552,0.592133) -- (8.552,0.592133) -- (8.584,0.592133) -- (8.584,0.596703) -- (8.616,0.596703) -- (8.616,0.592133) --
 (8.648,0.592133) -- (8.648,0.596703) -- (8.68,0.596703) -- (8.68,0.592133) -- (8.712,0.592133) -- (8.712,0.592133) -- (8.744,0.592133) -- (8.744,0.596703) -- (8.776,0.596703) -- (8.776,0.592133) -- (8.808,0.592133) -- (8.808,0.592133) --
 (8.84,0.592133) -- (8.84,0.592133) -- (8.872,0.592133) -- (8.872,0.592133) -- (8.904,0.592133) -- (8.904,0.592133) -- (8.936,0.592133) -- (8.936,0.592133) -- (8.968,0.592133) -- (8.968,0.596703) -- (9,0.596703);
\definecolor{c}{named}{natcomp};
\draw [c, fill=c!30] (1.704,0.592133) -- (1.704,0.60144) -- (1.736,0.60144) -- (1.736,0.612609) -- (1.768,0.612609) -- (1.768,0.636808) -- (1.8,0.636808) -- (1.8,0.795035) -- (1.832,0.795035) -- (1.832,1.13755) -- (1.864,1.13755) -- (1.864,1.32742) --
 (1.896,1.32742) -- (1.896,1.4205) -- (1.928,1.4205) -- (1.928,1.55266) -- (1.96,1.55266) -- (1.96,1.56383) -- (1.992,1.56383) -- (1.992,1.52474) -- (2.024,1.52474) -- (2.024,1.4782) -- (2.056,1.4782) -- (2.056,1.46517) -- (2.088,1.46517) --
 (2.088,1.43725) -- (2.12,1.43725) -- (2.12,1.34417) -- (2.152,1.34417) -- (2.152,1.31811) -- (2.184,1.31811) -- (2.184,1.22876) -- (2.216,1.22876) -- (2.216,1.2511) -- (2.248,1.2511) -- (2.248,1.17106) -- (2.28,1.17106) -- (2.28,1.11149) --
 (2.312,1.11149) -- (2.312,1.03889) -- (2.344,1.03889) -- (2.344,0.996076) -- (2.376,0.996076) -- (2.376,0.96443) -- (2.408,0.96443) -- (2.408,0.930924) -- (2.44,0.930924) -- (2.44,0.932785) -- (2.472,0.932785) -- (2.472,0.899278) -- (2.504,0.899278)
 -- (2.504,0.841572) -- (2.536,0.841572) -- (2.536,0.817373) -- (2.568,0.817373) -- (2.568,0.806204) -- (2.6,0.806204) -- (2.6,0.791312) -- (2.632,0.791312) -- (2.632,0.752221) -- (2.664,0.752221) -- (2.664,0.737329) -- (2.696,0.737329) --
 (2.696,0.759667) -- (2.728,0.759667) -- (2.728,0.70196) -- (2.76,0.70196) -- (2.76,0.728021) -- (2.792,0.728021) -- (2.792,0.698237) -- (2.824,0.698237) -- (2.824,0.705683) -- (2.856,0.705683) -- (2.856,0.694514) -- (2.888,0.694514) --
 (2.888,0.711268) -- (2.92,0.711268) -- (2.92,0.679623) -- (2.952,0.679623) -- (2.952,0.687069) -- (2.984,0.687069) -- (2.984,0.657285) -- (3.016,0.657285) -- (3.016,0.662869) -- (3.048,0.662869) -- (3.048,0.657285) -- (3.08,0.657285) --
 (3.08,0.662869) -- (3.112,0.662869) -- (3.112,0.642393) -- (3.144,0.642393) -- (3.144,0.657285) -- (3.176,0.657285) -- (3.176,0.633085) -- (3.208,0.633085) -- (3.208,0.634947) -- (3.24,0.634947) -- (3.24,0.63867) -- (3.272,0.63867) --
 (3.272,0.640531) -- (3.304,0.640531) -- (3.304,0.61447) -- (3.336,0.61447) -- (3.336,0.623778) -- (3.368,0.623778) -- (3.368,0.625639) -- (3.4,0.625639) -- (3.4,0.612609) -- (3.432,0.612609) -- (3.432,0.625639) -- (3.464,0.625639) --
 (3.464,0.618193) -- (3.496,0.618193) -- (3.496,0.608886) -- (3.528,0.608886) -- (3.528,0.621916) -- (3.56,0.621916) -- (3.56,0.620055) -- (3.592,0.620055) -- (3.592,0.61447) -- (3.624,0.61447) -- (3.624,0.605163) -- (3.656,0.605163) --
 (3.656,0.610747) -- (3.688,0.610747) -- (3.688,0.607024) -- (3.72,0.607024) -- (3.72,0.60144) -- (3.752,0.60144) -- (3.752,0.597717) -- (3.784,0.597717) -- (3.784,0.612609) -- (3.816,0.612609) -- (3.816,0.605163) -- (3.848,0.605163) --
 (3.848,0.612609) -- (3.88,0.612609) -- (3.88,0.603301) -- (3.912,0.603301) -- (3.912,0.597717) -- (3.944,0.597717) -- (3.944,0.60144) -- (3.976,0.60144) -- (3.976,0.60144) -- (4.008,0.60144) -- (4.008,0.599578) -- (4.04,0.599578) -- (4.04,0.595856)
 -- (4.072,0.595856) -- (4.072,0.599578) -- (4.104,0.599578) -- (4.104,0.595856) -- (4.136,0.595856) -- (4.136,0.603301) -- (4.168,0.603301) -- (4.168,0.599578) -- (4.2,0.599578) -- (4.2,0.599578) -- (4.232,0.599578) -- (4.232,0.599578) --
 (4.264,0.599578) -- (4.264,0.595856) -- (4.296,0.595856) -- (4.296,0.593994) -- (4.328,0.593994) -- (4.328,0.593994) -- (4.36,0.593994) -- (4.36,0.597717) -- (4.392,0.597717) -- (4.392,0.60144) -- (4.424,0.60144) -- (4.424,0.595856) --
 (4.456,0.595856) -- (4.456,0.592133) -- (4.488,0.592133) -- (4.488,0.593994) -- (4.52,0.593994) -- (4.52,0.595856) -- (4.552,0.595856) -- (4.552,0.597717) -- (4.584,0.597717) -- (4.584,0.592133) -- (4.616,0.592133) -- (4.616,0.592133) --
 (4.648,0.592133) -- (4.648,0.592133) -- (4.68,0.592133) -- (4.68,0.593994) -- (4.712,0.593994) -- (4.712,0.60144) -- (4.744,0.60144) -- (4.744,0.593994) -- (4.776,0.593994) -- (4.776,0.597717) -- (4.808,0.597717) -- (4.808,0.592133) --
 (4.84,0.592133) -- (4.84,0.593994) -- (4.872,0.593994) -- (4.872,0.593994) -- (4.904,0.593994) -- (4.904,0.593994) -- (4.936,0.593994) -- (4.936,0.595856) -- (4.968,0.595856) -- (4.968,0.595856) -- (5,0.595856) -- (5,0.592133) -- (5.032,0.592133) --
 (5.032,0.595856) -- (5.064,0.595856) -- (5.064,0.592133) -- (5.096,0.592133) -- (5.096,0.592133) -- (5.128,0.592133) -- (5.128,0.593994) -- (5.16,0.593994) -- (5.16,0.592133) -- (5.192,0.592133) -- (5.192,0.593994) -- (5.224,0.593994) --
 (5.224,0.597717) -- (5.256,0.597717) -- (5.256,0.592133) -- (5.288,0.592133) -- (5.288,0.593994) -- (5.32,0.593994) -- (5.32,0.592133) -- (5.352,0.592133) -- (5.352,0.592133) -- (5.384,0.592133) -- (5.384,0.592133) -- (5.416,0.592133) --
 (5.416,0.592133) -- (5.448,0.592133) -- (5.448,0.592133) -- (5.48,0.592133) -- (5.48,0.592133) -- (5.512,0.592133) -- (5.512,0.593994) -- (5.544,0.593994) -- (5.544,0.593994) -- (5.576,0.593994) -- (5.576,0.593994) -- (5.608,0.593994) --
 (5.608,0.592133) -- (5.64,0.592133) -- (5.64,0.592133) -- (5.672,0.592133) -- (5.672,0.593994) -- (5.704,0.593994) -- (5.704,0.595856) -- (5.736,0.595856) -- (5.736,0.592133) -- (5.768,0.592133) -- (5.768,0.592133) -- (5.8,0.592133) --
 (5.8,0.593994) -- (5.832,0.593994) -- (5.832,0.592133) -- (5.864,0.592133) -- (5.864,0.592133) -- (5.896,0.592133) -- (5.896,0.592133) -- (5.928,0.592133) -- (5.928,0.592133) -- (5.96,0.592133) -- (5.96,0.592133) -- (5.992,0.592133) --
 (5.992,0.592133) -- (6.024,0.592133) -- (6.024,0.592133) -- (6.056,0.592133) -- (6.056,0.592133) -- (6.088,0.592133) -- (6.088,0.593994) -- (6.12,0.593994) -- (6.12,0.595856) -- (6.152,0.595856) -- (6.152,0.593994) -- (6.184,0.593994) --
 (6.184,0.592133) -- (6.216,0.592133) -- (6.216,0.593994) -- (6.248,0.593994) -- (6.248,0.592133) -- (6.28,0.592133) -- (6.28,0.592133) -- (6.312,0.592133) -- (6.312,0.592133) -- (6.344,0.592133) -- (6.344,0.592133) -- (6.376,0.592133) --
 (6.376,0.592133) -- (6.408,0.592133) -- (6.408,0.593994) -- (6.44,0.593994) -- (6.44,0.592133) -- (6.472,0.592133) -- (6.472,0.593994) -- (6.504,0.593994) -- (6.504,0.592133) -- (6.536,0.592133) -- (6.536,0.592133) -- (6.568,0.592133) --
 (6.568,0.592133) -- (6.6,0.592133) -- (6.6,0.593994) -- (6.632,0.593994) -- (6.632,0.592133) -- (6.664,0.592133) -- (6.664,0.592133) -- (6.696,0.592133) -- (6.696,0.592133) -- (6.728,0.592133) -- (6.728,0.592133) -- (6.76,0.592133) --
 (6.76,0.592133) -- (6.792,0.592133) -- (6.792,0.593994) -- (6.824,0.593994) -- (6.824,0.592133) -- (6.856,0.592133) -- (6.856,0.592133) -- (6.888,0.592133) -- (6.888,0.592133) -- (6.92,0.592133) -- (6.92,0.592133) -- (6.952,0.592133) --
 (6.952,0.592133) -- (6.984,0.592133) -- (6.984,0.592133) -- (7.016,0.592133) -- (7.016,0.592133) -- (7.048,0.592133) -- (7.048,0.592133) -- (7.08,0.592133) -- (7.08,0.592133) -- (7.112,0.592133) -- (7.112,0.592133) -- (7.144,0.592133) --
 (7.144,0.592133) -- (7.176,0.592133) -- (7.176,0.592133) -- (7.208,0.592133) -- (7.208,0.592133) -- (7.24,0.592133) -- (7.24,0.592133) -- (7.272,0.592133) -- (7.272,0.592133) -- (7.304,0.592133) -- (7.304,0.592133) -- (7.336,0.592133) --
 (7.336,0.592133) -- (7.368,0.592133) -- (7.368,0.592133) -- (7.4,0.592133) -- (7.4,0.592133) -- (7.432,0.592133) -- (7.432,0.592133) -- (7.464,0.592133) -- (7.464,0.592133) -- (7.496,0.592133) -- (7.496,0.592133) -- (7.528,0.592133) --
 (7.528,0.592133) -- (7.56,0.592133) -- (7.56,0.592133) -- (7.592,0.592133) -- (7.592,0.592133) -- (7.624,0.592133) -- (7.624,0.592133) -- (7.656,0.592133) -- (7.656,0.592133) -- (7.688,0.592133) -- (7.688,0.592133) -- (7.72,0.592133) --
 (7.72,0.592133) -- (7.752,0.592133) -- (7.752,0.592133) -- (7.784,0.592133) -- (7.784,0.592133) -- (7.816,0.592133) -- (7.816,0.592133) -- (7.848,0.592133) -- (7.848,0.592133) -- (7.88,0.592133) -- (7.88,0.592133) -- (7.912,0.592133) --
 (7.912,0.592133) -- (7.944,0.592133) -- (7.944,0.592133) -- (7.976,0.592133) -- (7.976,0.592133) -- (8.008,0.592133) -- (8.008,0.592133) -- (8.04,0.592133) -- (8.04,0.592133) -- (8.072,0.592133) -- (8.072,0.592133) -- (8.104,0.592133) --
 (8.104,0.592133) -- (8.136,0.592133) -- (8.136,0.592133) -- (8.168,0.592133) -- (8.168,0.592133) -- (8.2,0.592133) -- (8.2,0.592133) -- (8.232,0.592133) -- (8.232,0.592133) -- (8.264,0.592133) -- (8.264,0.592133) -- (8.296,0.592133) --
 (8.296,0.592133) -- (8.328,0.592133) -- (8.328,0.592133) -- (8.36,0.592133) -- (8.36,0.592133) -- (8.392,0.592133) -- (8.392,0.592133) -- (8.424,0.592133) -- (8.424,0.592133) -- (8.456,0.592133) -- (8.456,0.592133) -- (8.488,0.592133) --
 (8.488,0.592133) -- (8.52,0.592133) -- (8.52,0.592133) -- (8.552,0.592133) -- (8.552,0.592133) -- (8.584,0.592133) -- (8.584,0.592133) -- (8.616,0.592133) -- (8.616,0.592133) -- (8.648,0.592133) -- (8.648,0.592133) -- (8.68,0.592133) --
 (8.68,0.592133) -- (8.712,0.592133) -- (8.712,0.592133) -- (8.744,0.592133) -- (8.744,0.592133) -- (8.776,0.592133) -- (8.776,0.592133) -- (8.808,0.592133) -- (8.808,0.592133) -- (8.84,0.592133) -- (8.84,0.592133) -- (8.872,0.592133) --
 (8.872,0.592133) -- (8.904,0.592133) -- (8.904,0.592133) -- (8.936,0.592133) -- (8.936,0.592133) -- (8.968,0.592133) -- (8.968,0.592133) -- (9,0.592133) -- (9,0.592133);
\draw [c] (1.704,0.592133) -- (1.704,0.60144) -- (1.736,0.60144) -- (1.736,0.612609) -- (1.768,0.612609) -- (1.768,0.636808) -- (1.8,0.636808) -- (1.8,0.795035) -- (1.832,0.795035) -- (1.832,1.13755) -- (1.864,1.13755) -- (1.864,1.32742) --
 (1.896,1.32742) -- (1.896,1.4205) -- (1.928,1.4205) -- (1.928,1.55266) -- (1.96,1.55266) -- (1.96,1.56383) -- (1.992,1.56383) -- (1.992,1.52474) -- (2.024,1.52474) -- (2.024,1.4782) -- (2.056,1.4782) -- (2.056,1.46517) -- (2.088,1.46517) --
 (2.088,1.43725) -- (2.12,1.43725) -- (2.12,1.34417) -- (2.152,1.34417) -- (2.152,1.31811) -- (2.184,1.31811) -- (2.184,1.22876) -- (2.216,1.22876) -- (2.216,1.2511) -- (2.248,1.2511) -- (2.248,1.17106) -- (2.28,1.17106) -- (2.28,1.11149) --
 (2.312,1.11149) -- (2.312,1.03889) -- (2.344,1.03889) -- (2.344,0.996076) -- (2.376,0.996076) -- (2.376,0.96443) -- (2.408,0.96443) -- (2.408,0.930924) -- (2.44,0.930924) -- (2.44,0.932785) -- (2.472,0.932785) -- (2.472,0.899278) -- (2.504,0.899278)
 -- (2.504,0.841572) -- (2.536,0.841572) -- (2.536,0.817373) -- (2.568,0.817373) -- (2.568,0.806204) -- (2.6,0.806204) -- (2.6,0.791312) -- (2.632,0.791312) -- (2.632,0.752221) -- (2.664,0.752221) -- (2.664,0.737329) -- (2.696,0.737329) --
 (2.696,0.759667) -- (2.728,0.759667) -- (2.728,0.70196) -- (2.76,0.70196) -- (2.76,0.728021) -- (2.792,0.728021) -- (2.792,0.698237) -- (2.824,0.698237) -- (2.824,0.705683) -- (2.856,0.705683) -- (2.856,0.694514) -- (2.888,0.694514) --
 (2.888,0.711268) -- (2.92,0.711268) -- (2.92,0.679623) -- (2.952,0.679623) -- (2.952,0.687069) -- (2.984,0.687069) -- (2.984,0.657285) -- (3.016,0.657285) -- (3.016,0.662869) -- (3.048,0.662869) -- (3.048,0.657285) -- (3.08,0.657285) --
 (3.08,0.662869) -- (3.112,0.662869) -- (3.112,0.642393) -- (3.144,0.642393) -- (3.144,0.657285) -- (3.176,0.657285) -- (3.176,0.633085) -- (3.208,0.633085) -- (3.208,0.634947) -- (3.24,0.634947) -- (3.24,0.63867) -- (3.272,0.63867) --
 (3.272,0.640531) -- (3.304,0.640531) -- (3.304,0.61447) -- (3.336,0.61447) -- (3.336,0.623778) -- (3.368,0.623778) -- (3.368,0.625639) -- (3.4,0.625639) -- (3.4,0.612609) -- (3.432,0.612609) -- (3.432,0.625639) -- (3.464,0.625639) --
 (3.464,0.618193) -- (3.496,0.618193) -- (3.496,0.608886) -- (3.528,0.608886) -- (3.528,0.621916) -- (3.56,0.621916) -- (3.56,0.620055) -- (3.592,0.620055) -- (3.592,0.61447) -- (3.624,0.61447) -- (3.624,0.605163) -- (3.656,0.605163) --
 (3.656,0.610747) -- (3.688,0.610747) -- (3.688,0.607024) -- (3.72,0.607024) -- (3.72,0.60144) -- (3.752,0.60144) -- (3.752,0.597717) -- (3.784,0.597717) -- (3.784,0.612609) -- (3.816,0.612609) -- (3.816,0.605163) -- (3.848,0.605163) --
 (3.848,0.612609) -- (3.88,0.612609) -- (3.88,0.603301) -- (3.912,0.603301) -- (3.912,0.597717) -- (3.944,0.597717) -- (3.944,0.60144) -- (3.976,0.60144) -- (3.976,0.60144) -- (4.008,0.60144) -- (4.008,0.599578) -- (4.04,0.599578) -- (4.04,0.595856)
 -- (4.072,0.595856) -- (4.072,0.599578) -- (4.104,0.599578) -- (4.104,0.595856) -- (4.136,0.595856) -- (4.136,0.603301) -- (4.168,0.603301) -- (4.168,0.599578) -- (4.2,0.599578) -- (4.2,0.599578) -- (4.232,0.599578) -- (4.232,0.599578) --
 (4.264,0.599578) -- (4.264,0.595856) -- (4.296,0.595856) -- (4.296,0.593994) -- (4.328,0.593994) -- (4.328,0.593994) -- (4.36,0.593994) -- (4.36,0.597717) -- (4.392,0.597717) -- (4.392,0.60144) -- (4.424,0.60144) -- (4.424,0.595856) --
 (4.456,0.595856) -- (4.456,0.592133) -- (4.488,0.592133) -- (4.488,0.593994) -- (4.52,0.593994) -- (4.52,0.595856) -- (4.552,0.595856) -- (4.552,0.597717) -- (4.584,0.597717) -- (4.584,0.592133) -- (4.616,0.592133) -- (4.616,0.592133) --
 (4.648,0.592133) -- (4.648,0.592133) -- (4.68,0.592133) -- (4.68,0.593994) -- (4.712,0.593994) -- (4.712,0.60144) -- (4.744,0.60144) -- (4.744,0.593994) -- (4.776,0.593994) -- (4.776,0.597717) -- (4.808,0.597717) -- (4.808,0.592133) --
 (4.84,0.592133) -- (4.84,0.593994) -- (4.872,0.593994) -- (4.872,0.593994) -- (4.904,0.593994) -- (4.904,0.593994) -- (4.936,0.593994) -- (4.936,0.595856) -- (4.968,0.595856) -- (4.968,0.595856) -- (5,0.595856) -- (5,0.592133) -- (5.032,0.592133) --
 (5.032,0.595856) -- (5.064,0.595856) -- (5.064,0.592133) -- (5.096,0.592133) -- (5.096,0.592133) -- (5.128,0.592133) -- (5.128,0.593994) -- (5.16,0.593994) -- (5.16,0.592133) -- (5.192,0.592133) -- (5.192,0.593994) -- (5.224,0.593994) --
 (5.224,0.597717) -- (5.256,0.597717) -- (5.256,0.592133) -- (5.288,0.592133) -- (5.288,0.593994) -- (5.32,0.593994) -- (5.32,0.592133) -- (5.352,0.592133) -- (5.352,0.592133) -- (5.384,0.592133) -- (5.384,0.592133) -- (5.416,0.592133) --
 (5.416,0.592133) -- (5.448,0.592133) -- (5.448,0.592133) -- (5.48,0.592133) -- (5.48,0.592133) -- (5.512,0.592133) -- (5.512,0.593994) -- (5.544,0.593994) -- (5.544,0.593994) -- (5.576,0.593994) -- (5.576,0.593994) -- (5.608,0.593994) --
 (5.608,0.592133) -- (5.64,0.592133) -- (5.64,0.592133) -- (5.672,0.592133) -- (5.672,0.593994) -- (5.704,0.593994) -- (5.704,0.595856) -- (5.736,0.595856) -- (5.736,0.592133) -- (5.768,0.592133) -- (5.768,0.592133) -- (5.8,0.592133) --
 (5.8,0.593994) -- (5.832,0.593994) -- (5.832,0.592133) -- (5.864,0.592133) -- (5.864,0.592133) -- (5.896,0.592133) -- (5.896,0.592133) -- (5.928,0.592133) -- (5.928,0.592133) -- (5.96,0.592133) -- (5.96,0.592133) -- (5.992,0.592133) --
 (5.992,0.592133) -- (6.024,0.592133) -- (6.024,0.592133) -- (6.056,0.592133) -- (6.056,0.592133) -- (6.088,0.592133) -- (6.088,0.593994) -- (6.12,0.593994) -- (6.12,0.595856) -- (6.152,0.595856) -- (6.152,0.593994) -- (6.184,0.593994) --
 (6.184,0.592133) -- (6.216,0.592133) -- (6.216,0.593994) -- (6.248,0.593994) -- (6.248,0.592133) -- (6.28,0.592133) -- (6.28,0.592133) -- (6.312,0.592133) -- (6.312,0.592133) -- (6.344,0.592133) -- (6.344,0.592133) -- (6.376,0.592133) --
 (6.376,0.592133) -- (6.408,0.592133) -- (6.408,0.593994) -- (6.44,0.593994) -- (6.44,0.592133) -- (6.472,0.592133) -- (6.472,0.593994) -- (6.504,0.593994) -- (6.504,0.592133) -- (6.536,0.592133) -- (6.536,0.592133) -- (6.568,0.592133) --
 (6.568,0.592133) -- (6.6,0.592133) -- (6.6,0.593994) -- (6.632,0.593994) -- (6.632,0.592133) -- (6.664,0.592133) -- (6.664,0.592133) -- (6.696,0.592133) -- (6.696,0.592133) -- (6.728,0.592133) -- (6.728,0.592133) -- (6.76,0.592133) --
 (6.76,0.592133) -- (6.792,0.592133) -- (6.792,0.593994) -- (6.824,0.593994) -- (6.824,0.592133) -- (6.856,0.592133) -- (6.856,0.592133) -- (6.888,0.592133) -- (6.888,0.592133) -- (6.92,0.592133) -- (6.92,0.592133) -- (6.952,0.592133) --
 (6.952,0.592133) -- (6.984,0.592133) -- (6.984,0.592133) -- (7.016,0.592133) -- (7.016,0.592133) -- (7.048,0.592133) -- (7.048,0.592133) -- (7.08,0.592133) -- (7.08,0.592133) -- (7.112,0.592133) -- (7.112,0.592133) -- (7.144,0.592133) --
 (7.144,0.592133) -- (7.176,0.592133) -- (7.176,0.592133) -- (7.208,0.592133) -- (7.208,0.592133) -- (7.24,0.592133) -- (7.24,0.592133) -- (7.272,0.592133) -- (7.272,0.592133) -- (7.304,0.592133) -- (7.304,0.592133) -- (7.336,0.592133) --
 (7.336,0.592133) -- (7.368,0.592133) -- (7.368,0.592133) -- (7.4,0.592133) -- (7.4,0.592133) -- (7.432,0.592133) -- (7.432,0.592133) -- (7.464,0.592133) -- (7.464,0.592133) -- (7.496,0.592133) -- (7.496,0.592133) -- (7.528,0.592133) --
 (7.528,0.592133) -- (7.56,0.592133) -- (7.56,0.592133) -- (7.592,0.592133) -- (7.592,0.592133) -- (7.624,0.592133) -- (7.624,0.592133) -- (7.656,0.592133) -- (7.656,0.592133) -- (7.688,0.592133) -- (7.688,0.592133) -- (7.72,0.592133) --
 (7.72,0.592133) -- (7.752,0.592133) -- (7.752,0.592133) -- (7.784,0.592133) -- (7.784,0.592133) -- (7.816,0.592133) -- (7.816,0.592133) -- (7.848,0.592133) -- (7.848,0.592133) -- (7.88,0.592133) -- (7.88,0.592133) -- (7.912,0.592133) --
 (7.912,0.592133) -- (7.944,0.592133) -- (7.944,0.592133) -- (7.976,0.592133) -- (7.976,0.592133) -- (8.008,0.592133) -- (8.008,0.592133) -- (8.04,0.592133) -- (8.04,0.592133) -- (8.072,0.592133) -- (8.072,0.592133) -- (8.104,0.592133) --
 (8.104,0.592133) -- (8.136,0.592133) -- (8.136,0.592133) -- (8.168,0.592133) -- (8.168,0.592133) -- (8.2,0.592133) -- (8.2,0.592133) -- (8.232,0.592133) -- (8.232,0.592133) -- (8.264,0.592133) -- (8.264,0.592133) -- (8.296,0.592133) --
 (8.296,0.592133) -- (8.328,0.592133) -- (8.328,0.592133) -- (8.36,0.592133) -- (8.36,0.592133) -- (8.392,0.592133) -- (8.392,0.592133) -- (8.424,0.592133) -- (8.424,0.592133) -- (8.456,0.592133) -- (8.456,0.592133) -- (8.488,0.592133) --
 (8.488,0.592133) -- (8.52,0.592133) -- (8.52,0.592133) -- (8.552,0.592133) -- (8.552,0.592133) -- (8.584,0.592133) -- (8.584,0.592133) -- (8.616,0.592133) -- (8.616,0.592133) -- (8.648,0.592133) -- (8.648,0.592133) -- (8.68,0.592133) --
 (8.68,0.592133) -- (8.712,0.592133) -- (8.712,0.592133) -- (8.744,0.592133) -- (8.744,0.592133) -- (8.776,0.592133) -- (8.776,0.592133) -- (8.808,0.592133) -- (8.808,0.592133) -- (8.84,0.592133) -- (8.84,0.592133) -- (8.872,0.592133) --
 (8.872,0.592133) -- (8.904,0.592133) -- (8.904,0.592133) -- (8.936,0.592133) -- (8.936,0.592133) -- (8.968,0.592133) -- (8.968,0.592133) -- (9,0.592133);
\definecolor{c}{rgb}{0,0,0};
\draw [c] (1,0.592133) -- (9,0.592133);
\draw [c] (1,0.734244) -- (1,0.592133);
\draw [c] (1.16,0.663188) -- (1.16,0.592133);
\draw [c] (1.32,0.663188) -- (1.32,0.592133);
\draw [c] (1.48,0.663188) -- (1.48,0.592133);
\draw [c] (1.64,0.663188) -- (1.64,0.592133);
\draw [c] (1.8,0.734244) -- (1.8,0.592133);
\draw [c] (1.96,0.663188) -- (1.96,0.592133);
\draw [c] (2.12,0.663188) -- (2.12,0.592133);
\draw [c] (2.28,0.663188) -- (2.28,0.592133);
\draw [c] (2.44,0.663188) -- (2.44,0.592133);
\draw [c] (2.6,0.734244) -- (2.6,0.592133);
\draw [c] (2.76,0.663188) -- (2.76,0.592133);
\draw [c] (2.92,0.663188) -- (2.92,0.592133);
\draw [c] (3.08,0.663188) -- (3.08,0.592133);
\draw [c] (3.24,0.663188) -- (3.24,0.592133);
\draw [c] (3.4,0.734244) -- (3.4,0.592133);
\draw [c] (3.56,0.663188) -- (3.56,0.592133);
\draw [c] (3.72,0.663188) -- (3.72,0.592133);
\draw [c] (3.88,0.663188) -- (3.88,0.592133);
\draw [c] (4.04,0.663188) -- (4.04,0.592133);
\draw [c] (4.2,0.734244) -- (4.2,0.592133);
\draw [c] (4.36,0.663188) -- (4.36,0.592133);
\draw [c] (4.52,0.663188) -- (4.52,0.592133);
\draw [c] (4.68,0.663188) -- (4.68,0.592133);
\draw [c] (4.84,0.663188) -- (4.84,0.592133);
\draw [c] (5,0.734244) -- (5,0.592133);
\draw [c] (5.16,0.663188) -- (5.16,0.592133);
\draw [c] (5.32,0.663188) -- (5.32,0.592133);
\draw [c] (5.48,0.663188) -- (5.48,0.592133);
\draw [c] (5.64,0.663188) -- (5.64,0.592133);
\draw [c] (5.8,0.734244) -- (5.8,0.592133);
\draw [c] (5.96,0.663188) -- (5.96,0.592133);
\draw [c] (6.12,0.663188) -- (6.12,0.592133);
\draw [c] (6.28,0.663188) -- (6.28,0.592133);
\draw [c] (6.44,0.663188) -- (6.44,0.592133);
\draw [c] (6.6,0.734244) -- (6.6,0.592133);
\draw [c] (6.76,0.663188) -- (6.76,0.592133);
\draw [c] (6.92,0.663188) -- (6.92,0.592133);
\draw [c] (7.08,0.663188) -- (7.08,0.592133);
\draw [c] (7.24,0.663188) -- (7.24,0.592133);
\draw [c] (7.4,0.734244) -- (7.4,0.592133);
\draw [c] (7.56,0.663188) -- (7.56,0.592133);
\draw [c] (7.72,0.663188) -- (7.72,0.592133);
\draw [c] (7.88,0.663188) -- (7.88,0.592133);
\draw [c] (8.04,0.663188) -- (8.04,0.592133);
\draw [c] (8.2,0.734244) -- (8.2,0.592133);
\draw [c] (8.36,0.663188) -- (8.36,0.592133);
\draw [c] (8.52,0.663188) -- (8.52,0.592133);
\draw [c] (8.68,0.663188) -- (8.68,0.592133);
\draw [c] (8.84,0.663188) -- (8.84,0.592133);
\draw [c] (9,0.734244) -- (9,0.592133);
\draw [anchor=base] (1,0.396729) node[]{0};
\draw [anchor=base] (1.8,0.396729) node[]{100};
\draw [anchor=base] (2.6,0.396729) node[]{200};
\draw [anchor=base] (3.4,0.396729) node[]{300};
\draw [anchor=base] (4.2,0.396729) node[]{400};
\draw [anchor=base] (5,0.396729) node[]{500};
\draw [anchor=base] (5.8,0.396729) node[]{600};
\draw [anchor=base] (6.6,0.396729) node[]{700};
\draw [anchor=base] (7.4,0.396729) node[]{800};
\draw [anchor=base] (8.2,0.396729) node[]{900};
\draw [anchor=base] (9,0.396729) node[]{1000};
\draw [anchor=base] (9,0.15) node[left] {\textit{M$_{\gamma\gamma}$} [GeV]};
\draw [c] (1,0.592133) -- (1,5.32919);
\draw [c] (1.24,0.592133) -- (1,0.592133);
\draw [c] (1.12,0.732749) -- (1,0.732749);
\draw [c] (1.12,0.873365) -- (1,0.873365);
\draw [c] (1.12,1.01398) -- (1,1.01398);
\draw [c] (1.24,1.1546) -- (1,1.1546);
\draw [c] (1.12,1.29521) -- (1,1.29521);
\draw [c] (1.12,1.43583) -- (1,1.43583);
\draw [c] (1.12,1.57644) -- (1,1.57644);
\draw [c] (1.24,1.71706) -- (1,1.71706);
\draw [c] (1.12,1.85768) -- (1,1.85768);
\draw [c] (1.12,1.99829) -- (1,1.99829);
\draw [c] (1.12,2.13891) -- (1,2.13891);
\draw [c] (1.24,2.27953) -- (1,2.27953);
\draw [c] (1.12,2.42014) -- (1,2.42014);
\draw [c] (1.12,2.56076) -- (1,2.56076);
\draw [c] (1.12,2.70137) -- (1,2.70137);
\draw [c] (1.24,2.84199) -- (1,2.84199);
\draw [c] (1.12,2.9826) -- (1,2.9826);
\draw [c] (1.12,3.12322) -- (1,3.12322);
\draw [c] (1.12,3.26384) -- (1,3.26384);
\draw [c] (1.24,3.40445) -- (1,3.40445);
\draw [c] (1.12,3.54507) -- (1,3.54507);
\draw [c] (1.12,3.68569) -- (1,3.68569);
\draw [c] (1.12,3.8263) -- (1,3.8263);
\draw [c] (1.24,3.96692) -- (1,3.96692);
\draw [c] (1.12,4.10753) -- (1,4.10753);
\draw [c] (1.12,4.24815) -- (1,4.24815);
\draw [c] (1.12,4.38877) -- (1,4.38877);
\draw [c] (1.24,4.52938) -- (1,4.52938);
\draw [c] (1.12,4.67) -- (1,4.67);
\draw [c] (1.12,4.81061) -- (1,4.81061);
\draw [c] (1.12,4.95123) -- (1,4.95123);
\draw [c] (1.24,5.09185) -- (1,5.09185);
\draw [c] (1.24,5.09185) -- (1,5.09185);
\draw [c] (1.12,5.23246) -- (1,5.23246);
\draw [anchor= east] (0.95,0.592133) node[]{0};
\draw [anchor= east] (0.95,1.1546) node[]{200};
\draw [anchor= east] (0.95,1.71706) node[]{400};
\draw [anchor= east] (0.95,2.27953) node[]{600};
\draw [anchor= east] (0.95,2.84199) node[]{800};
\draw [anchor= east] (0.95,3.40445) node[]{1000};
\draw [anchor= east] (0.95,3.96692) node[]{1200};
\draw [anchor= east] (0.95,4.52938) node[]{1400};
\draw [anchor= east] (0.95,5.09185) node[]{1600};
\draw (0.15,5.2) node[left,rotate=90] {Events / 4 GeV};
\draw [anchor=base west] (5.95,4.76001) node[]{Total contribution};
\definecolor{c}{rgb}{1,1,1};
\draw [c, fill=c] (5.1425,4.68229) -- (5.8075,4.68229) -- (5.8075,5.11751) -- (5.1425,5.11751);
\definecolor{c}{named}{natgreen};
\draw [c] (5.1425,4.8999) -- (5.8075,4.8999);
\definecolor{c}{rgb}{0,0,0};

\draw [anchor=base west] (5.95,4.13827) node[]{Box contribution};
\definecolor{c}{named}{natcomp};
\draw [c!30, fill=c!30] (5.1425,4.06055) -- (5.8075,4.06055) -- (5.8075,4.49577) -- (5.1425,4.49577);
\draw [c] (5.1425,4.27816) -- (5.8075,4.27816);
\definecolor{c}{rgb}{0,0,0};

\end{tikzpicture}
\end{tiny}
\end{sffamily}
\end{minipage}
\begin{minipage}[b]{.3\textwidth}
\caption{The distribution of invariant masses of Standard Model diphoton events as predicted by simulation. The box contribution gives just those events produced by the box diagram in fig.~\ref{boxdiag}. These are ATLAS datasets produced with pythia8 \cite{pythia}, and normalised to the luminosity of the data sample. \label{boxpart}}
\end{minipage}
\end{figure}

\section{The effective Lagrangian approach}

As was established before we ventured in to the world of Feynman diagrams, the SM Lagrangian consists of a sum of terms, each of which describes the behaviour of or interactions between the sectors of the Standard Model. It is no great stretch, then, to consider expanding the Standard Model by adding a new term to the Lagrangian, which describes some new physics. Doing so, however, is not unproblematic.

It is a property of the Standard Model\footnote{[Or it is a requirement on the Standard Model. Claiming that the SM is \emph{inherently} unitary might be something of a stretch.]} that it is unitary, meaning that the total probability of a given state to propagate into any of the possible final states evaluates to 1. Clearly, one cannot simply add any new term to the Standard Model Lagrangian without breaking this unitarity. Rather than going through the painstaking process of ensuring that the new term we will add to the SM preserves its unitarity, we will in stead build on the assumption that new physics exists at high mass scales, and think of the SM Lagrangian as simply the zeroth order term in a series expansion of some larger model. There will then be higher order corrections to the Standard Model, in some mass scale $\Lambda$ that the expansion is performed in. 
In that case, the SM is no longer assumed to be a complete model, and moreover, the expanded model is not even expected to be a complete model to an order in $\Lambda$, which allows us to sidestep the issue of unitarity.
This approach only works if the mass scale is significantly larger than the energies at which the Standard Model is being probed, since the higher order contributions would otherwise have a detectable influence in the lower energy range where the parameters of the SM are determined, meaning that the original assumption of the SM Lagrangian as a zeroth order term in an expansion in $\Lambda$ is no longer valid.

With this assumed plethora of possible higher order terms, a single possible new term can be added without worring about maintaining the integrity of the SM. The specific term considered here, which describes the $q\bar q\rightarrow\gamma\gamma$ contact interaction which is the focus of this thesis, takes the form \cite{rizzo}
\(\mathcal L_n = \frac{2ie^2}{\Lambda^4}Q_q^2F^{\mu\sigma}F^\nu_\sigma\overline{q}\gamma_\mu\partial_\nu q,\label{rizzo}\)
where $e$ is the elementary charge, $Q_q$ is the quark charge of quark $q$ and $\Lambda$, as discussed, is the associated mass scale. The power of $\Lambda$ is chosen to give the correct mass dimension of the term: in natural units,\footnote{Where $\hbar = c = 1$.} the action is unitless. The Lagrangian is integrated over 4 lengths to give the action, which means it must have a length dimension of $-4$ itself. Finally, in natural units, we can equate a unit of length with a unit of inverse mass\footnote{Because in natural units, \[[\text{length}]=[\text{length}]\frac{c}{\hbar}=[\text{length}]\frac{\left[\frac{\text{length}}{\text{time}}\right]}{\left[\frac{\text{length}^2\text{ mass}}{\text{time}}\right]}=\left[\frac{\text{length}^2\text{ time}}{\text{length}^2\text{ time}\text{ mass}}\right]=[\text{mass}]^{-1}. \]}, which means that the Lagrangian must have mass dimension 4. The factors $F^{\mu\sigma}F^\nu_\sigma$ and $\overline{q}\gamma_\mu\partial_\nu q$ each contribute a mass dimension of 4, so the mass scale, which has dimension of mass, must get a power of $-4$.

\begin{figure}[htp]\begin{center}
{\footnotesize\begin{tikzpicture} [>=triangle 45]
\draw[>-] (-1,.5) -- (0,0);
\draw[<-] (-1,-.5) -- (0,0);
\draw (-2,1) node[left] {$u_{ap,2}$} -- (-1,.5);
\draw (-2,-1)  node[left] {$\bar u_{bq,1}$} -- (-1,-.5);
\draw[snake=coil,segment aspect=0,line before snake=3mm] (0,0) -- (2,1) node[right] {$\gamma_{\mu,3}$};
\draw[snake=coil,segment aspect=0,line before snake=3mm] (0,0) -- (2,-1) node[right] {$\gamma_{\nu,4}$}; 
\node at (1,0) [right] {\footnotesize$=\dfrac{8 e{}^2}{9 \Lambda^4} \delta_{p q} \big(p_2^\rho p_3^\rho p_4^\sigma g^{\mu \nu} \gamma_{a b}^\sigma -p_2^\mu p_3^\nu p_4^\rho \gamma_{a b}^\rho -p_2^\rho p_3^\rho p_4^\mu \gamma_{a b}^\nu +p_2^\mu p_3^\rho p_4^\rho \gamma_{a b}^\nu $};
\node at (2.3,-.6) [right] {$+p_2^\rho p_3^\sigma p_4^\rho g^{\mu \nu} \gamma_{a b}^\sigma -p_2^\nu p_3^\rho p_4^\mu \gamma_{a b}^\rho -p_2^\rho p_3^\nu p_4^\rho \gamma_{a b}^\mu +p_2^\nu p_3^\rho p_4^\rho \gamma_{a b}^\mu \big)$};
\end{tikzpicture}
}\end{center}
\caption{An example of a Feynman rule as created by entering the new term from eq.~\eqref{rizzo} into LanHEP\cite{lanhep}. It gives the coupling constant of the four-point interaction between an up- and an antiup-quark, and two photons. $p_n$ represents the four-momentum of particle $n$, $\gamma_{ab}^\mu$ is the $\gamma$-matrix, $g^{\mu\nu}$ is the metric tensor, $\delta_{pq}$ is the Kronecker delta, $e$ is the elementary charge and $\Lambda$ is the associated mass scale.\label{rule}}
\end{figure}

The effect of the new term is to introduce several Feynman rules of the type shown in fig.~\ref{rule}.

These new processes may interfere constructively or destructively with the Standard Model contributions to this process. The effects of both on the distribution of invariant masses of photon pairs are illustrated in figure~\ref{interf}.

\begin{figure}[htp]
\begin{minipage}[b]{.69\textwidth}
\begin{infilsf} \tiny \makebox[0pt][l]{
\hspace{-1em}\pgfdeclareplotmark{cross} {
\pgfpathmoveto{\pgfpoint{-0.3\pgfplotmarksize}{\pgfplotmarksize}}
\pgfpathlineto{\pgfpoint{+0.3\pgfplotmarksize}{\pgfplotmarksize}}
\pgfpathlineto{\pgfpoint{+0.3\pgfplotmarksize}{0.3\pgfplotmarksize}}
\pgfpathlineto{\pgfpoint{+1\pgfplotmarksize}{0.3\pgfplotmarksize}}
\pgfpathlineto{\pgfpoint{+1\pgfplotmarksize}{-0.3\pgfplotmarksize}}
\pgfpathlineto{\pgfpoint{+0.3\pgfplotmarksize}{-0.3\pgfplotmarksize}}
\pgfpathlineto{\pgfpoint{+0.3\pgfplotmarksize}{-1.\pgfplotmarksize}}
\pgfpathlineto{\pgfpoint{-0.3\pgfplotmarksize}{-1.\pgfplotmarksize}}
\pgfpathlineto{\pgfpoint{-0.3\pgfplotmarksize}{-0.3\pgfplotmarksize}}
\pgfpathlineto{\pgfpoint{-1.\pgfplotmarksize}{-0.3\pgfplotmarksize}}
\pgfpathlineto{\pgfpoint{-1.\pgfplotmarksize}{0.3\pgfplotmarksize}}
\pgfpathlineto{\pgfpoint{-0.3\pgfplotmarksize}{0.3\pgfplotmarksize}}
\pgfpathclose
\pgfusepathqstroke
}
\pgfdeclareplotmark{cross*} {
\pgfpathmoveto{\pgfpoint{-0.3\pgfplotmarksize}{\pgfplotmarksize}}
\pgfpathlineto{\pgfpoint{+0.3\pgfplotmarksize}{\pgfplotmarksize}}
\pgfpathlineto{\pgfpoint{+0.3\pgfplotmarksize}{0.3\pgfplotmarksize}}
\pgfpathlineto{\pgfpoint{+1\pgfplotmarksize}{0.3\pgfplotmarksize}}
\pgfpathlineto{\pgfpoint{+1\pgfplotmarksize}{-0.3\pgfplotmarksize}}
\pgfpathlineto{\pgfpoint{+0.3\pgfplotmarksize}{-0.3\pgfplotmarksize}}
\pgfpathlineto{\pgfpoint{+0.3\pgfplotmarksize}{-1.\pgfplotmarksize}}
\pgfpathlineto{\pgfpoint{-0.3\pgfplotmarksize}{-1.\pgfplotmarksize}}
\pgfpathlineto{\pgfpoint{-0.3\pgfplotmarksize}{-0.3\pgfplotmarksize}}
\pgfpathlineto{\pgfpoint{-1.\pgfplotmarksize}{-0.3\pgfplotmarksize}}
\pgfpathlineto{\pgfpoint{-1.\pgfplotmarksize}{0.3\pgfplotmarksize}}
\pgfpathlineto{\pgfpoint{-0.3\pgfplotmarksize}{0.3\pgfplotmarksize}}
\pgfpathclose
\pgfusepathqfillstroke
}
\pgfdeclareplotmark{newstar} {
\pgfpathmoveto{\pgfqpoint{0pt}{\pgfplotmarksize}}
\pgfpathlineto{\pgfqpointpolar{44}{0.5\pgfplotmarksize}}
\pgfpathlineto{\pgfqpointpolar{18}{\pgfplotmarksize}}
\pgfpathlineto{\pgfqpointpolar{-20}{0.5\pgfplotmarksize}}
\pgfpathlineto{\pgfqpointpolar{-54}{\pgfplotmarksize}}
\pgfpathlineto{\pgfqpointpolar{-90}{0.5\pgfplotmarksize}}
\pgfpathlineto{\pgfqpointpolar{234}{\pgfplotmarksize}}
\pgfpathlineto{\pgfqpointpolar{198}{0.5\pgfplotmarksize}}
\pgfpathlineto{\pgfqpointpolar{162}{\pgfplotmarksize}}
\pgfpathlineto{\pgfqpointpolar{134}{0.5\pgfplotmarksize}}
\pgfpathclose
\pgfusepathqstroke
}
\pgfdeclareplotmark{newstar*} {
\pgfpathmoveto{\pgfqpoint{0pt}{\pgfplotmarksize}}
\pgfpathlineto{\pgfqpointpolar{44}{0.5\pgfplotmarksize}}
\pgfpathlineto{\pgfqpointpolar{18}{\pgfplotmarksize}}
\pgfpathlineto{\pgfqpointpolar{-20}{0.5\pgfplotmarksize}}
\pgfpathlineto{\pgfqpointpolar{-54}{\pgfplotmarksize}}
\pgfpathlineto{\pgfqpointpolar{-90}{0.5\pgfplotmarksize}}
\pgfpathlineto{\pgfqpointpolar{234}{\pgfplotmarksize}}
\pgfpathlineto{\pgfqpointpolar{198}{0.5\pgfplotmarksize}}
\pgfpathlineto{\pgfqpointpolar{162}{\pgfplotmarksize}}
\pgfpathlineto{\pgfqpointpolar{134}{0.5\pgfplotmarksize}}
\pgfpathclose
\pgfusepathqfillstroke
}
\begin{tikzpicture}[x=.045\textwidth,y=.045\textwidth]
\definecolor{c}{rgb}{1,1,1};
\draw [color=c, fill=c] (0,0) rectangle (20,13.5632);
\draw [color=c, fill=c] (2,1.35632) rectangle (19.8,13.4276);
\definecolor{c}{rgb}{0,0,0};
\draw [c] (2,1.35632) -- (2,13.4276) -- (19.8,13.4276) -- (19.8,1.35632) -- (2,1.35632);
\definecolor{c}{rgb}{1,1,1};
\draw [color=c, fill=c] (2,1.35632) rectangle (19.8,13.4276);
\definecolor{c}{rgb}{0,0,0};
\draw [c] (2,1.35632) -- (2,13.4276) -- (19.8,13.4276) -- (19.8,1.35632) -- (2,1.35632);
\colorlet{c}{kugray};
\draw [c] (2.178,13.0447) -- (2.178,13.0472);
\draw [c] (2.178,13.0472) -- (2.178,13.0497);
\draw [c] (2,13.0472) -- (2.178,13.0472);
\draw [c] (2.178,13.0472) -- (2.356,13.0472);
\draw [c] (2.534,11.7994) -- (2.534,11.8066);
\draw [c] (2.534,11.8066) -- (2.534,11.8137);
\draw [c] (2.356,11.8066) -- (2.534,11.8066);
\draw [c] (2.534,11.8066) -- (2.712,11.8066);
\draw [c] (2.89,10.9909) -- (2.89,11.0051);
\draw [c] (2.89,11.0051) -- (2.89,11.019);
\draw [c] (2.712,11.0051) -- (2.89,11.0051);
\draw [c] (2.89,11.0051) -- (3.068,11.0051);
\draw [c] (3.246,10.3749) -- (3.246,10.3989);
\draw [c] (3.246,10.3989) -- (3.246,10.4219);
\draw [c] (3.068,10.3989) -- (3.246,10.3989);
\draw [c] (3.246,10.3989) -- (3.424,10.3989);
\draw [c] (3.602,9.85739) -- (3.602,9.89444);
\draw [c] (3.602,9.89444) -- (3.602,9.92931);
\draw [c] (3.424,9.89444) -- (3.602,9.89444);
\draw [c] (3.602,9.89444) -- (3.78,9.89444);
\draw [c] (3.958,9.47339) -- (3.958,9.52465);
\draw [c] (3.958,9.52465) -- (3.958,9.57183);
\draw [c] (3.78,9.52465) -- (3.958,9.52465);
\draw [c] (3.958,9.52465) -- (4.136,9.52465);
\draw [c] (4.314,8.9621) -- (4.314,9.04106);
\draw [c] (4.314,9.04106) -- (4.314,9.1107);
\draw [c] (4.136,9.04106) -- (4.314,9.04106);
\draw [c] (4.314,9.04106) -- (4.492,9.04106);
\draw [c] (4.67,8.5558) -- (4.67,8.66704);
\draw [c] (4.67,8.66704) -- (4.67,8.76064);
\draw [c] (4.492,8.66704) -- (4.67,8.66704);
\draw [c] (4.67,8.66704) -- (4.848,8.66704);
\draw [c] (5.026,8.20363) -- (5.026,8.35328);
\draw [c] (5.026,8.35328) -- (5.026,8.47261);
\draw [c] (4.848,8.35328) -- (5.026,8.35328);
\draw [c] (5.026,8.35328) -- (5.204,8.35328);
\draw [c] (5.382,8.13448) -- (5.382,8.23104);
\draw [c] (5.382,8.23104) -- (5.382,8.31402);
\draw [c] (5.204,8.23104) -- (5.382,8.23104);
\draw [c] (5.382,8.23104) -- (5.56,8.23104);
\draw [c] (5.738,7.81536) -- (5.738,7.82);
\draw [c] (5.738,7.82) -- (5.738,7.82461);
\draw [c] (5.56,7.82) -- (5.738,7.82);
\draw [c] (5.738,7.82) -- (5.916,7.82);
\draw [c] (6.094,7.54683) -- (6.094,7.55266);
\draw [c] (6.094,7.55266) -- (6.094,7.55842);
\draw [c] (5.916,7.55266) -- (6.094,7.55266);
\draw [c] (6.094,7.55266) -- (6.272,7.55266);
\draw [c] (6.45,7.26355) -- (6.45,7.27095);
\draw [c] (6.45,7.27095) -- (6.45,7.27826);
\draw [c] (6.272,7.27095) -- (6.45,7.27095);
\draw [c] (6.45,7.27095) -- (6.628,7.27095);
\draw [c] (6.806,7.01466) -- (6.806,7.02379);
\draw [c] (6.806,7.02379) -- (6.806,7.03279);
\draw [c] (6.628,7.02379) -- (6.806,7.02379);
\draw [c] (6.806,7.02379) -- (6.984,7.02379);
\draw [c] (7.162,6.7713) -- (7.162,6.78251);
\draw [c] (7.162,6.78251) -- (7.162,6.79352);
\draw [c] (6.984,6.78251) -- (7.162,6.78251);
\draw [c] (7.162,6.78251) -- (7.34,6.78251);
\draw [c] (7.518,6.54463) -- (7.518,6.55822);
\draw [c] (7.518,6.55822) -- (7.518,6.5715);
\draw [c] (7.34,6.55822) -- (7.518,6.55822);
\draw [c] (7.518,6.55822) -- (7.696,6.55822);
\draw [c] (7.874,6.31555) -- (7.874,6.33204);
\draw [c] (7.874,6.33204) -- (7.874,6.34808);
\draw [c] (7.696,6.33204) -- (7.874,6.33204);
\draw [c] (7.874,6.33204) -- (8.052,6.33204);
\draw [c] (8.23,6.11397) -- (8.23,6.13353);
\draw [c] (8.23,6.13353) -- (8.23,6.15246);
\draw [c] (8.052,6.13353) -- (8.23,6.13353);
\draw [c] (8.23,6.13353) -- (8.408,6.13353);
\draw [c] (8.586,5.86492) -- (8.586,5.88906);
\draw [c] (8.586,5.88906) -- (8.586,5.91225);
\draw [c] (8.408,5.88906) -- (8.586,5.88906);
\draw [c] (8.586,5.88906) -- (8.764,5.88906);
\draw [c] (8.942,5.54421) -- (8.942,5.57587);
\draw [c] (8.942,5.57587) -- (8.942,5.60592);
\draw [c] (8.764,5.57587) -- (8.942,5.57587);
\draw [c] (8.942,5.57587) -- (9.12,5.57587);
\draw [c] (9.298,5.27855) -- (9.298,5.31817);
\draw [c] (9.298,5.31817) -- (9.298,5.35531);
\draw [c] (9.12,5.31817) -- (9.298,5.31817);
\draw [c] (9.298,5.31817) -- (9.476,5.31817);
\draw [c] (9.654,4.9566) -- (9.654,5.00862);
\draw [c] (9.654,5.00862) -- (9.654,5.05643);
\draw [c] (9.476,5.00862) -- (9.654,5.00862);
\draw [c] (9.654,5.00862) -- (9.832,5.00862);
\draw [c] (10.01,4.88694) -- (10.01,4.94211);
\draw [c] (10.01,4.94211) -- (10.01,4.99257);
\draw [c] (9.832,4.94211) -- (10.01,4.94211);
\draw [c] (10.01,4.94211) -- (10.188,4.94211);
\draw [c] (10.366,4.63404) -- (10.366,4.70236);
\draw [c] (10.366,4.70236) -- (10.366,4.76359);
\draw [c] (10.188,4.70236) -- (10.366,4.70236);
\draw [c] (10.366,4.70236) -- (10.544,4.70236);
\draw [c] (10.722,4.30544) -- (10.722,4.3956);
\draw [c] (10.722,4.3956) -- (10.722,4.47381);
\draw [c] (10.544,4.3956) -- (10.722,4.3956);
\draw [c] (10.722,4.3956) -- (10.9,4.3956);
\draw [c] (11.078,4.12866) -- (11.078,4.23332);
\draw [c] (11.078,4.23332) -- (11.078,4.32221);
\draw [c] (10.9,4.23332) -- (11.078,4.23332);
\draw [c] (11.078,4.23332) -- (11.256,4.23332);
\draw [c] (11.434,3.79816) -- (11.434,3.93643);
\draw [c] (11.434,3.93643) -- (11.434,4.04842);
\draw [c] (11.256,3.93643) -- (11.434,3.93643);
\draw [c] (11.434,3.93643) -- (11.612,3.93643);
\draw [c] (11.79,3.76836) -- (11.79,3.91015);
\draw [c] (11.79,3.91015) -- (11.79,4.02442);
\draw [c] (11.612,3.91015) -- (11.79,3.91015);
\draw [c] (11.79,3.91015) -- (11.968,3.91015);
\draw [c] (12.146,3.407) -- (12.146,3.59907);
\draw [c] (12.146,3.59907) -- (12.146,3.74381);
\draw [c] (11.968,3.59907) -- (12.146,3.59907);
\draw [c] (12.146,3.59907) -- (12.324,3.59907);
\draw [c] (12.502,3.35033) -- (12.502,3.55174);
\draw [c] (12.502,3.55174) -- (12.502,3.7017);
\draw [c] (12.324,3.55174) -- (12.502,3.55174);
\draw [c] (12.502,3.55174) -- (12.68,3.55174);
\draw [c] (12.858,3.28811) -- (12.858,3.50029);
\draw [c] (12.858,3.50029) -- (12.858,3.65611);
\draw [c] (12.68,3.50029) -- (12.858,3.50029);
\draw [c] (12.858,3.50029) -- (13.036,3.50029);
\draw [c] (13.214,2.49227) -- (13.214,2.90213);
\draw [c] (13.214,2.90213) -- (13.214,3.14188);
\draw [c] (13.036,2.90213) -- (13.214,2.90213);
\draw [c] (13.214,2.90213) -- (13.392,2.90213);
\draw [c] (13.57,3.28811) -- (13.57,3.50029);
\draw [c] (13.57,3.50029) -- (13.57,3.65611);
\draw [c] (13.392,3.50029) -- (13.57,3.50029);
\draw [c] (13.57,3.50029) -- (13.748,3.50029);
\draw [c] (13.926,1.35632) -- (13.926,2.08241);
\draw [c] (13.926,2.08241) -- (13.926,2.49227);
\draw [c] (13.748,2.08241) -- (13.926,2.08241);
\draw [c] (13.926,2.08241) -- (14.104,2.08241);
\draw [c] (14.282,1.35632) -- (14.282,2.08241);
\draw [c] (14.282,2.08241) -- (14.282,2.49227);
\draw [c] (14.104,2.08241) -- (14.282,2.08241);
\draw [c] (14.282,2.08241) -- (14.46,2.08241);
\draw [c] (14.638,1.35632) -- (14.638,2.08241);
\draw [c] (14.638,2.08241) -- (14.638,2.49227);
\draw [c] (14.46,2.08241) -- (14.638,2.08241);
\draw [c] (14.638,2.08241) -- (14.816,2.08241);
\draw [c] (14.994,1.76618) -- (14.994,2.49227);
\draw [c] (14.994,2.49227) -- (14.994,2.8085);
\draw [c] (14.816,2.49227) -- (14.994,2.49227);
\draw [c] (14.994,2.49227) -- (15.172,2.49227);
\draw [c] (15.35,1.35632) -- (15.35,2.08241);
\draw [c] (15.35,2.08241) -- (15.35,2.49227);
\draw [c] (15.172,2.08241) -- (15.35,2.08241);
\draw [c] (15.35,2.08241) -- (15.528,2.08241);
\draw [c] (15.706,1.35632) -- (15.706,2.08241);
\draw [c] (15.706,2.08241) -- (15.706,2.49227);
\draw [c] (15.528,2.08241) -- (15.706,2.08241);
\draw [c] (15.706,2.08241) -- (15.884,2.08241);
\draw [c] (16.062,1.35632) -- (16.062,2.08241);
\draw [c] (16.062,2.08241) -- (16.062,2.49227);
\draw [c] (15.884,2.08241) -- (16.062,2.08241);
\draw [c] (16.062,2.08241) -- (16.24,2.08241);
\draw [c] (16.774,1.35632) -- (16.774,2.08241);
\draw [c] (16.774,2.08241) -- (16.774,2.49227);
\draw [c] (16.596,2.08241) -- (16.774,2.08241);
\draw [c] (16.774,2.08241) -- (16.952,2.08241);
\definecolor{c}{rgb}{0,0,0};
\draw [c] (2,1.35632) -- (19.8,1.35632);
\draw [anchor= east] (19.8,0.596782) +(0,-1.4em) node[ rotate=0]{$M_{\gamma\gamma}\text{ [GeV]}$};
\draw [c] (3.45306,1.71846) -- (3.45306,1.35632);
\draw [c] (3.81633,1.53739) -- (3.81633,1.35632);
\draw [c] (4.17959,1.53739) -- (4.17959,1.35632);
\draw [c] (4.54286,1.53739) -- (4.54286,1.35632);
\draw [c] (4.90612,1.53739) -- (4.90612,1.35632);
\draw [c] (5.26939,1.71846) -- (5.26939,1.35632);
\draw [c] (5.63265,1.53739) -- (5.63265,1.35632);
\draw [c] (5.99592,1.53739) -- (5.99592,1.35632);
\draw [c] (6.35918,1.53739) -- (6.35918,1.35632);
\draw [c] (6.72245,1.53739) -- (6.72245,1.35632);
\draw [c] (7.08571,1.71846) -- (7.08571,1.35632);
\draw [c] (7.44898,1.53739) -- (7.44898,1.35632);
\draw [c] (7.81224,1.53739) -- (7.81224,1.35632);
\draw [c] (8.17551,1.53739) -- (8.17551,1.35632);
\draw [c] (8.53878,1.53739) -- (8.53878,1.35632);
\draw [c] (8.90204,1.71846) -- (8.90204,1.35632);
\draw [c] (9.26531,1.53739) -- (9.26531,1.35632);
\draw [c] (9.62857,1.53739) -- (9.62857,1.35632);
\draw [c] (9.99184,1.53739) -- (9.99184,1.35632);
\draw [c] (10.3551,1.53739) -- (10.3551,1.35632);
\draw [c] (10.7184,1.71846) -- (10.7184,1.35632);
\draw [c] (11.0816,1.53739) -- (11.0816,1.35632);
\draw [c] (11.4449,1.53739) -- (11.4449,1.35632);
\draw [c] (11.8082,1.53739) -- (11.8082,1.35632);
\draw [c] (12.1714,1.53739) -- (12.1714,1.35632);
\draw [c] (12.5347,1.71846) -- (12.5347,1.35632);
\draw [c] (12.898,1.53739) -- (12.898,1.35632);
\draw [c] (13.2612,1.53739) -- (13.2612,1.35632);
\draw [c] (13.6245,1.53739) -- (13.6245,1.35632);
\draw [c] (13.9878,1.53739) -- (13.9878,1.35632);
\draw [c] (14.351,1.71846) -- (14.351,1.35632);
\draw [c] (14.7143,1.53739) -- (14.7143,1.35632);
\draw [c] (15.0776,1.53739) -- (15.0776,1.35632);
\draw [c] (15.4408,1.53739) -- (15.4408,1.35632);
\draw [c] (15.8041,1.53739) -- (15.8041,1.35632);
\draw [c] (16.1673,1.71846) -- (16.1673,1.35632);
\draw [c] (16.5306,1.53739) -- (16.5306,1.35632);
\draw [c] (16.8939,1.53739) -- (16.8939,1.35632);
\draw [c] (17.2571,1.53739) -- (17.2571,1.35632);
\draw [c] (17.6204,1.53739) -- (17.6204,1.35632);
\draw [c] (17.9837,1.71846) -- (17.9837,1.35632);
\draw [c] (18.3469,1.53739) -- (18.3469,1.35632);
\draw [c] (18.7102,1.53739) -- (18.7102,1.35632);
\draw [c] (19.0735,1.53739) -- (19.0735,1.35632);
\draw [c] (19.4367,1.53739) -- (19.4367,1.35632);
\draw [c] (19.8,1.71846) -- (19.8,1.35632);
\draw [c] (3.45306,1.71846) -- (3.45306,1.35632);
\draw [c] (3.0898,1.53739) -- (3.0898,1.35632);
\draw [c] (2.72653,1.53739) -- (2.72653,1.35632);
\draw [c] (2.36327,1.53739) -- (2.36327,1.35632);
\draw [c] (2,1.53739) -- (2,1.35632);
\draw [anchor=base] (3.45306,0.908736) +(0,-.7em) node[ rotate=0]{500};
\draw [anchor=base] (5.26939,0.908736) +(0,-.7em) node[ rotate=0]{1000};
\draw [anchor=base] (7.08571,0.908736) +(0,-.7em) node[ rotate=0]{1500};
\draw [anchor=base] (8.90204,0.908736) +(0,-.7em) node[ rotate=0]{2000};
\draw [anchor=base] (10.7184,0.908736) +(0,-.7em) node[ rotate=0]{2500};
\draw [anchor=base] (12.5347,0.908736) +(0,-.7em) node[ rotate=0]{3000};
\draw [anchor=base] (14.351, 0.908736) +(0,-.7em) node[ rotate=0]{3500};
\draw [anchor=base] (16.1673,0.908736) +(0,-.7em) node[ rotate=0]{4000};
\draw [anchor=base] (17.9837,0.908736) +(0,-.7em) node[ rotate=0]{4500};
\draw [anchor=base] (19.8,   0.908736) +(0,-.7em) node[ rotate=0]{5000};
\draw [c] (2,1.35632) -- (2,13.4276);
\draw [anchor= east] (-0.18,13.4276) node[ rotate=90]{$\di\sigma/\di M_{\gamma\gamma}\text{ [pb/GeV]}$};
\draw [c] (2.267,1.37636) -- (2,1.37636);
\draw [c] (2.267,1.46751) -- (2,1.46751);
\draw [c] (2.267,1.54647) -- (2,1.54647);
\draw [c] (2.267,1.61612) -- (2,1.61612);
\draw [c] (2.534,1.67842) -- (2,1.67842);
\draw [anchor= east] (1.844,1.67842) node[ rotate=0]{$10^{-10}$};
\draw [c] (2.267,2.08827) -- (2,2.08827);
\draw [c] (2.267,2.32803) -- (2,2.32803);
\draw [c] (2.267,2.49813) -- (2,2.49813);
\draw [c] (2.267,2.63008) -- (2,2.63008);
\draw [c] (2.267,2.73789) -- (2,2.73789);
\draw [c] (2.267,2.82904) -- (2,2.82904);
\draw [c] (2.267,2.90799) -- (2,2.90799);
\draw [c] (2.267,2.97764) -- (2,2.97764);
\draw [c] (2.534,3.03994) -- (2,3.03994);
\draw [anchor= east] (1.844,3.03994) node[ rotate=0]{$10^{-9}$};
\draw [c] (2.267,3.4498) -- (2,3.4498);
\draw [c] (2.267,3.68955) -- (2,3.68955);
\draw [c] (2.267,3.85966) -- (2,3.85966);
\draw [c] (2.267,3.9916) -- (2,3.9916);
\draw [c] (2.267,4.09941) -- (2,4.09941);
\draw [c] (2.267,4.19056) -- (2,4.19056);
\draw [c] (2.267,4.26952) -- (2,4.26952);
\draw [c] (2.267,4.33916) -- (2,4.33916);
\draw [c] (2.534,4.40146) -- (2,4.40146);
\draw [anchor= east] (1.844,4.40146) node[ rotate=0]{$10^{-8}$};
\draw [c] (2.267,4.81132) -- (2,4.81132);
\draw [c] (2.267,5.05107) -- (2,5.05107);
\draw [c] (2.267,5.22118) -- (2,5.22118);
\draw [c] (2.267,5.35312) -- (2,5.35312);
\draw [c] (2.267,5.46093) -- (2,5.46093);
\draw [c] (2.267,5.55208) -- (2,5.55208);
\draw [c] (2.267,5.63104) -- (2,5.63104);
\draw [c] (2.267,5.70068) -- (2,5.70068);
\draw [c] (2.534,5.76298) -- (2,5.76298);
\draw [anchor= east] (1.844,5.76298) node[ rotate=0]{$10^{-7}$};
\draw [c] (2.267,6.17284) -- (2,6.17284);
\draw [c] (2.267,6.41259) -- (2,6.41259);
\draw [c] (2.267,6.5827) -- (2,6.5827);
\draw [c] (2.267,6.71465) -- (2,6.71465);
\draw [c] (2.267,6.82245) -- (2,6.82245);
\draw [c] (2.267,6.9136) -- (2,6.9136);
\draw [c] (2.267,6.99256) -- (2,6.99256);
\draw [c] (2.267,7.06221) -- (2,7.06221);
\draw [c] (2.534,7.12451) -- (2,7.12451);
\draw [anchor= east] (1.844,7.12451) node[ rotate=0]{$10^{-6}$};
\draw [c] (2.267,7.53436) -- (2,7.53436);
\draw [c] (2.267,7.77412) -- (2,7.77412);
\draw [c] (2.267,7.94422) -- (2,7.94422);
\draw [c] (2.267,8.07617) -- (2,8.07617);
\draw [c] (2.267,8.18398) -- (2,8.18398);
\draw [c] (2.267,8.27513) -- (2,8.27513);
\draw [c] (2.267,8.35408) -- (2,8.35408);
\draw [c] (2.267,8.42373) -- (2,8.42373);
\draw [c] (2.534,8.48603) -- (2,8.48603);
\draw [anchor= east] (1.844,8.48603) node[ rotate=0]{$10^{-5}$};
\draw [c] (2.267,8.89589) -- (2,8.89589);
\draw [c] (2.267,9.13564) -- (2,9.13564);
\draw [c] (2.267,9.30575) -- (2,9.30575);
\draw [c] (2.267,9.43769) -- (2,9.43769);
\draw [c] (2.267,9.5455) -- (2,9.5455);
\draw [c] (2.267,9.63665) -- (2,9.63665);
\draw [c] (2.267,9.71561) -- (2,9.71561);
\draw [c] (2.267,9.78525) -- (2,9.78525);
\draw [c] (2.534,9.84755) -- (2,9.84755);
\draw [anchor= east] (1.844,9.84755) node[ rotate=0]{$10^{-4}$};
\draw [c] (2.267,10.2574) -- (2,10.2574);
\draw [c] (2.267,10.4972) -- (2,10.4972);
\draw [c] (2.267,10.6673) -- (2,10.6673);
\draw [c] (2.267,10.7992) -- (2,10.7992);
\draw [c] (2.267,10.907) -- (2,10.907);
\draw [c] (2.267,10.9982) -- (2,10.9982);
\draw [c] (2.267,11.0771) -- (2,11.0771);
\draw [c] (2.267,11.1468) -- (2,11.1468);
\draw [c] (2.534,11.2091) -- (2,11.2091);
\draw [anchor= east] (1.844,11.2091) node[ rotate=0]{$10^{-3}$};
\draw [c] (2.267,11.6189) -- (2,11.6189);
\draw [c] (2.267,11.8587) -- (2,11.8587);
\draw [c] (2.267,12.0288) -- (2,12.0288);
\draw [c] (2.267,12.1607) -- (2,12.1607);
\draw [c] (2.267,12.2685) -- (2,12.2685);
\draw [c] (2.267,12.3597) -- (2,12.3597);
\draw [c] (2.267,12.4387) -- (2,12.4387);
\draw [c] (2.267,12.5083) -- (2,12.5083);
\draw [c] (2.534,12.5706) -- (2,12.5706);
\draw [anchor= east] (1.844,12.5706) node[ rotate=0]{$10^{-2}$};
\draw [c] (2.267,12.9805) -- (2,12.9805);
\draw [c] (2.267,13.2202) -- (2,13.2202);
\draw [c] (2.267,13.3903) -- (2,13.3903);
\colorlet{c}{natgreen};
\draw [c] (2.178,13.0455) -- (2.178,13.048);
\draw [c] (2.178,13.048) -- (2.178,13.0505);
\draw [c] (2,13.048) -- (2.178,13.048);
\draw [c] (2.178,13.048) -- (2.356,13.048);
\draw [c] (2.534,11.8185) -- (2.534,11.8255);
\draw [c] (2.534,11.8255) -- (2.534,11.8325);
\draw [c] (2.356,11.8255) -- (2.534,11.8255);
\draw [c] (2.534,11.8255) -- (2.712,11.8255);
\draw [c] (2.89,10.9707) -- (2.89,10.9852);
\draw [c] (2.89,10.9852) -- (2.89,10.9993);
\draw [c] (2.712,10.9852) -- (2.89,10.9852);
\draw [c] (2.89,10.9852) -- (3.068,10.9852);
\draw [c] (3.246,10.3456) -- (3.246,10.3701);
\draw [c] (3.246,10.3701) -- (3.246,10.3937);
\draw [c] (3.068,10.3701) -- (3.246,10.3701);
\draw [c] (3.246,10.3701) -- (3.424,10.3701);
\draw [c] (3.602,9.90372) -- (3.602,9.93938);
\draw [c] (3.602,9.93938) -- (3.602,9.97301);
\draw [c] (3.424,9.93938) -- (3.602,9.93938);
\draw [c] (3.602,9.93938) -- (3.78,9.93938);
\draw [c] (3.958,9.45689) -- (3.958,9.50891);
\draw [c] (3.958,9.50891) -- (3.958,9.55672);
\draw [c] (3.78,9.50891) -- (3.958,9.50891);
\draw [c] (3.958,9.50891) -- (4.136,9.50891);
\draw [c] (4.314,9.04606) -- (4.314,9.11966);
\draw [c] (4.314,9.11966) -- (4.314,9.18511);
\draw [c] (4.136,9.11966) -- (4.314,9.11966);
\draw [c] (4.314,9.11966) -- (4.492,9.11966);
\draw [c] (4.67,8.73814) -- (4.67,8.83359);
\draw [c] (4.67,8.83359) -- (4.67,8.91575);
\draw [c] (4.492,8.83359) -- (4.67,8.83359);
\draw [c] (4.67,8.83359) -- (4.848,8.83359);
\draw [c] (5.026,8.45233) -- (5.026,8.57379);
\draw [c] (5.026,8.57379) -- (5.026,8.67451);
\draw [c] (4.848,8.57379) -- (5.026,8.57379);
\draw [c] (5.026,8.57379) -- (5.204,8.57379);
\draw [c] (5.382,8.18106) -- (5.382,8.24613);
\draw [c] (5.382,8.24613) -- (5.382,8.30474);
\draw [c] (5.204,8.24613) -- (5.382,8.24613);
\draw [c] (5.382,8.24613) -- (5.56,8.24613);
\draw [c] (5.738,8.05315) -- (5.738,8.05868);
\draw [c] (5.738,8.05868) -- (5.738,8.06415);
\draw [c] (5.56,8.05868) -- (5.738,8.05868);
\draw [c] (5.738,8.05868) -- (5.916,8.05868);
\draw [c] (6.094,7.85989) -- (6.094,7.8664);
\draw [c] (6.094,7.8664) -- (6.094,7.87284);
\draw [c] (5.916,7.8664) -- (6.094,7.8664);
\draw [c] (6.094,7.8664) -- (6.272,7.8664);
\draw [c] (6.45,7.69131) -- (6.45,7.69882);
\draw [c] (6.45,7.69882) -- (6.45,7.70623);
\draw [c] (6.272,7.69882) -- (6.45,7.69882);
\draw [c] (6.45,7.69882) -- (6.628,7.69882);
\draw [c] (6.806,7.57157) -- (6.806,7.57988);
\draw [c] (6.806,7.57988) -- (6.806,7.58807);
\draw [c] (6.628,7.57988) -- (6.806,7.57988);
\draw [c] (6.806,7.57988) -- (6.984,7.57988);
\draw [c] (7.162,7.44078) -- (7.162,7.45006);
\draw [c] (7.162,7.45006) -- (7.162,7.45919);
\draw [c] (6.984,7.45006) -- (7.162,7.45006);
\draw [c] (7.162,7.45006) -- (7.34,7.45006);
\draw [c] (7.518,7.33444) -- (7.518,7.3446);
\draw [c] (7.518,7.3446) -- (7.518,7.35458);
\draw [c] (7.34,7.3446) -- (7.518,7.3446);
\draw [c] (7.518,7.3446) -- (7.696,7.3446);
\draw [c] (7.874,7.24592) -- (7.874,7.25686);
\draw [c] (7.874,7.25686) -- (7.874,7.2676);
\draw [c] (7.696,7.25686) -- (7.874,7.25686);
\draw [c] (7.874,7.25686) -- (8.052,7.25686);
\draw [c] (8.23,7.18271) -- (8.23,7.19425);
\draw [c] (8.23,7.19425) -- (8.23,7.20557);
\draw [c] (8.052,7.19425) -- (8.23,7.19425);
\draw [c] (8.23,7.19425) -- (8.408,7.19425);
\draw [c] (8.586,7.09927) -- (8.586,7.11166);
\draw [c] (8.586,7.11166) -- (8.586,7.12379);
\draw [c] (8.408,7.11166) -- (8.586,7.11166);
\draw [c] (8.586,7.11166) -- (8.764,7.11166);
\draw [c] (8.942,7.04838) -- (8.942,7.06131);
\draw [c] (8.942,7.06131) -- (8.942,7.07396);
\draw [c] (8.764,7.06131) -- (8.942,7.06131);
\draw [c] (8.942,7.06131) -- (9.12,7.06131);
\draw [c] (9.298,6.99793) -- (9.298,7.01142);
\draw [c] (9.298,7.01142) -- (9.298,7.02461);
\draw [c] (9.12,7.01142) -- (9.298,7.01142);
\draw [c] (9.298,7.01142) -- (9.476,7.01142);
\draw [c] (9.654,6.95404) -- (9.654,6.96804);
\draw [c] (9.654,6.96804) -- (9.654,6.98172);
\draw [c] (9.476,6.96804) -- (9.654,6.96804);
\draw [c] (9.654,6.96804) -- (9.832,6.96804);
\draw [c] (10.01,6.91054) -- (10.01,6.92507);
\draw [c] (10.01,6.92507) -- (10.01,6.93924);
\draw [c] (9.832,6.92507) -- (10.01,6.92507);
\draw [c] (10.01,6.92507) -- (10.188,6.92507);
\draw [c] (10.366,6.83107) -- (10.366,6.84661);
\draw [c] (10.366,6.84661) -- (10.366,6.86174);
\draw [c] (10.188,6.84661) -- (10.366,6.84661);
\draw [c] (10.366,6.84661) -- (10.544,6.84661);
\draw [c] (10.722,6.79835) -- (10.722,6.81433);
\draw [c] (10.722,6.81433) -- (10.722,6.82988);
\draw [c] (10.544,6.81433) -- (10.722,6.81433);
\draw [c] (10.722,6.81433) -- (10.9,6.81433);
\draw [c] (11.078,6.73603) -- (11.078,6.75287);
\draw [c] (11.078,6.75287) -- (11.078,6.76924);
\draw [c] (10.9,6.75287) -- (11.078,6.75287);
\draw [c] (11.078,6.75287) -- (11.256,6.75287);
\draw [c] (11.434,6.68832) -- (11.434,6.70585);
\draw [c] (11.434,6.70585) -- (11.434,6.72287);
\draw [c] (11.256,6.70585) -- (11.434,6.70585);
\draw [c] (11.434,6.70585) -- (11.612,6.70585);
\draw [c] (11.79,6.62631) -- (11.79,6.64478);
\draw [c] (11.79,6.64478) -- (11.79,6.66269);
\draw [c] (11.612,6.64478) -- (11.79,6.64478);
\draw [c] (11.79,6.64478) -- (11.968,6.64478);
\draw [c] (12.146,6.57963) -- (12.146,6.59885);
\draw [c] (12.146,6.59885) -- (12.146,6.61746);
\draw [c] (11.968,6.59885) -- (12.146,6.59885);
\draw [c] (12.146,6.59885) -- (12.324,6.59885);
\draw [c] (12.502,6.51133) -- (12.502,6.53169);
\draw [c] (12.502,6.53169) -- (12.502,6.55137);
\draw [c] (12.324,6.53169) -- (12.502,6.53169);
\draw [c] (12.502,6.53169) -- (12.68,6.53169);
\draw [c] (12.858,6.42788) -- (12.858,6.44973);
\draw [c] (12.858,6.44973) -- (12.858,6.47079);
\draw [c] (12.68,6.44973) -- (12.858,6.44973);
\draw [c] (12.858,6.44973) -- (13.036,6.44973);
\draw [c] (13.214,6.39362) -- (13.214,6.41611);
\draw [c] (13.214,6.41611) -- (13.214,6.43778);
\draw [c] (13.036,6.41611) -- (13.214,6.41611);
\draw [c] (13.214,6.41611) -- (13.392,6.41611);
\draw [c] (13.57,6.31571) -- (13.57,6.33974);
\draw [c] (13.57,6.33974) -- (13.57,6.36282);
\draw [c] (13.392,6.33974) -- (13.57,6.33974);
\draw [c] (13.57,6.33974) -- (13.748,6.33974);
\draw [c] (13.926,6.2447) -- (13.926,6.27021);
\draw [c] (13.926,6.27021) -- (13.926,6.29466);
\draw [c] (13.748,6.27021) -- (13.926,6.27021);
\draw [c] (13.926,6.27021) -- (14.104,6.27021);
\draw [c] (14.282,6.09905) -- (14.282,6.1279);
\draw [c] (14.282,6.1279) -- (14.282,6.1554);
\draw [c] (14.104,6.1279) -- (14.282,6.1279);
\draw [c] (14.282,6.1279) -- (14.46,6.1279);
\draw [c] (14.638,6.05188) -- (14.638,6.0819);
\draw [c] (14.638,6.0819) -- (14.638,6.11048);
\draw [c] (14.46,6.0819) -- (14.638,6.0819);
\draw [c] (14.638,6.0819) -- (14.816,6.0819);
\draw [c] (14.994,5.90042) -- (14.994,5.93454);
\draw [c] (14.994,5.93454) -- (14.994,5.96681);
\draw [c] (14.816,5.93454) -- (14.994,5.93454);
\draw [c] (14.994,5.93454) -- (15.172,5.93454);
\draw [c] (15.35,5.83792) -- (15.35,5.87389);
\draw [c] (15.35,5.87389) -- (15.35,5.90781);
\draw [c] (15.172,5.87389) -- (15.35,5.87389);
\draw [c] (15.35,5.87389) -- (15.528,5.87389);
\draw [c] (15.706,5.67542) -- (15.706,5.71669);
\draw [c] (15.706,5.71669) -- (15.706,5.75527);
\draw [c] (15.528,5.71669) -- (15.706,5.71669);
\draw [c] (15.706,5.71669) -- (15.884,5.71669);
\draw [c] (16.062,5.55903) -- (16.062,5.60457);
\draw [c] (16.062,5.60457) -- (16.062,5.64685);
\draw [c] (15.884,5.60457) -- (16.062,5.60457);
\draw [c] (16.062,5.60457) -- (16.24,5.60457);
\draw [c] (16.418,5.47583) -- (16.418,5.52468);
\draw [c] (16.418,5.52468) -- (16.418,5.56981);
\draw [c] (16.24,5.52468) -- (16.418,5.52468);
\draw [c] (16.418,5.52468) -- (16.596,5.52468);
\draw [c] (16.774,5.36547) -- (16.774,5.4191);
\draw [c] (16.774,5.4191) -- (16.774,5.46827);
\draw [c] (16.596,5.4191) -- (16.774,5.4191);
\draw [c] (16.774,5.4191) -- (16.952,5.4191);
\draw [c] (17.13,5.21861) -- (17.13,5.27933);
\draw [c] (17.13,5.27933) -- (17.13,5.33439);
\draw [c] (16.952,5.27933) -- (17.13,5.27933);
\draw [c] (17.13,5.27933) -- (17.308,5.27933);
\draw [c] (17.486,5.22452) -- (17.486,5.28493);
\draw [c] (17.486,5.28493) -- (17.486,5.33974);
\draw [c] (17.308,5.28493) -- (17.486,5.28493);
\draw [c] (17.486,5.28493) -- (17.664,5.28493);
\draw [c] (17.842,4.82281) -- (17.842,4.90763);
\draw [c] (17.842,4.90763) -- (17.842,4.98179);
\draw [c] (17.664,4.90763) -- (17.842,4.90763);
\draw [c] (17.842,4.90763) -- (18.02,4.90763);
\draw [c] (18.198,4.66841) -- (18.198,4.76503);
\draw [c] (18.198,4.76503) -- (18.198,4.84806);
\draw [c] (18.02,4.76503) -- (18.198,4.76503);
\draw [c] (18.198,4.76503) -- (18.376,4.76503);
\draw [c] (18.554,4.55637) -- (18.554,4.66257);
\draw [c] (18.554,4.66257) -- (18.554,4.75257);
\draw [c] (18.376,4.66257) -- (18.554,4.66257);
\draw [c] (18.554,4.66257) -- (18.732,4.66257);
\draw [c] (18.91,4.34988) -- (18.91,4.47627);
\draw [c] (18.91,4.47627) -- (18.91,4.58034);
\draw [c] (18.732,4.47627) -- (18.91,4.47627);
\draw [c] (18.91,4.47627) -- (19.088,4.47627);
\draw [c] (19.266,3.99676) -- (19.266,4.16687);
\draw [c] (19.266,4.16687) -- (19.266,4.29881);
\draw [c] (19.088,4.16687) -- (19.266,4.16687);
\draw [c] (19.266,4.16687) -- (19.444,4.16687);
\draw [c] (19.622,3.66418) -- (19.622,3.88895);
\draw [c] (19.622,3.88895) -- (19.622,4.05143);
\draw [c] (19.444,3.88895) -- (19.622,3.88895);
\draw [c] (19.622,3.88895) -- (19.8,3.88895);
\colorlet{c}{natgreen!59};
\draw [c] (2.178,13.1297) -- (2.178,13.1349);
\draw [c] (2.178,13.1349) -- (2.178,13.14);
\draw [c] (2,13.1349) -- (2.178,13.1349);
\draw [c] (2.178,13.1349) -- (2.356,13.1349);
\draw [c] (2.534,11.9028) -- (2.534,11.9174);
\draw [c] (2.534,11.9174) -- (2.534,11.9316);
\draw [c] (2.356,11.9174) -- (2.534,11.9174);
\draw [c] (2.534,11.9174) -- (2.712,11.9174);
\draw [c] (2.89,11.109) -- (2.89,11.1375);
\draw [c] (2.89,11.1375) -- (2.89,11.1646);
\draw [c] (2.712,11.1375) -- (2.89,11.1375);
\draw [c] (2.89,11.1375) -- (3.068,11.1375);
\draw [c] (3.246,10.4796) -- (3.246,10.4969);
\draw [c] (3.246,10.4969) -- (3.246,10.5137);
\draw [c] (3.068,10.4969) -- (3.246,10.4969);
\draw [c] (3.246,10.4969) -- (3.424,10.4969);
\draw [c] (3.602,9.94422) -- (3.602,9.949);
\draw [c] (3.602,9.949) -- (3.602,9.95374);
\draw [c] (3.424,9.949) -- (3.602,9.949);
\draw [c] (3.602,9.949) -- (3.78,9.949);
\draw [c] (3.958,9.49314) -- (3.958,9.50014);
\draw [c] (3.958,9.50014) -- (3.958,9.50706);
\draw [c] (3.78,9.50014) -- (3.958,9.50014);
\draw [c] (3.958,9.50014) -- (4.136,9.50014);
\draw [c] (4.314,9.06994) -- (4.314,9.07995);
\draw [c] (4.314,9.07995) -- (4.314,9.0898);
\draw [c] (4.136,9.07995) -- (4.314,9.07995);
\draw [c] (4.314,9.07995) -- (4.492,9.07995);
\draw [c] (4.67,8.69444) -- (4.67,8.70819);
\draw [c] (4.67,8.70819) -- (4.67,8.72164);
\draw [c] (4.492,8.70819) -- (4.67,8.70819);
\draw [c] (4.67,8.70819) -- (4.848,8.70819);
\draw [c] (5.026,8.26038) -- (5.026,8.28024);
\draw [c] (5.026,8.28024) -- (5.026,8.29945);
\draw [c] (4.848,8.28024) -- (5.026,8.28024);
\draw [c] (5.026,8.28024) -- (5.204,8.28024);
\draw [c] (5.382,7.92086) -- (5.382,7.93442);
\draw [c] (5.382,7.93442) -- (5.382,7.94768);
\draw [c] (5.204,7.93442) -- (5.382,7.93442);
\draw [c] (5.382,7.93442) -- (5.56,7.93442);
\draw [c] (5.738,7.53258) -- (5.738,7.53794);
\draw [c] (5.738,7.53794) -- (5.738,7.54326);
\draw [c] (5.56,7.53794) -- (5.738,7.53794);
\draw [c] (5.738,7.53794) -- (5.916,7.53794);
\draw [c] (6.094,7.13507) -- (6.094,7.14258);
\draw [c] (6.094,7.14258) -- (6.094,7.14999);
\draw [c] (5.916,7.14258) -- (6.094,7.14258);
\draw [c] (6.094,7.14258) -- (6.272,7.14258);
\draw [c] (6.45,6.70871) -- (6.45,6.71948);
\draw [c] (6.45,6.71948) -- (6.45,6.73006);
\draw [c] (6.272,6.71948) -- (6.45,6.71948);
\draw [c] (6.45,6.71948) -- (6.628,6.71948);
\draw [c] (6.806,6.24258) -- (6.806,6.25856);
\draw [c] (6.806,6.25856) -- (6.806,6.27411);
\draw [c] (6.628,6.25856) -- (6.806,6.25856);
\draw [c] (6.806,6.25856) -- (6.984,6.25856);
\draw [c] (7.162,5.82341) -- (7.162,5.84617);
\draw [c] (7.162,5.84617) -- (7.162,5.86809);
\draw [c] (6.984,5.84617) -- (7.162,5.84617);
\draw [c] (7.162,5.84617) -- (7.34,5.84617);
\draw [c] (7.518,5.67144) -- (7.518,5.69733);
\draw [c] (7.518,5.69733) -- (7.518,5.72213);
\draw [c] (7.34,5.69733) -- (7.518,5.69733);
\draw [c] (7.518,5.69733) -- (7.696,5.69733);
\draw [c] (7.874,5.65798) -- (7.874,5.68416);
\draw [c] (7.874,5.68416) -- (7.874,5.70924);
\draw [c] (7.696,5.68416) -- (7.874,5.68416);
\draw [c] (7.874,5.68416) -- (8.052,5.68416);
\draw [c] (8.23,5.79613) -- (8.23,5.81943);
\draw [c] (8.23,5.81943) -- (8.23,5.84184);
\draw [c] (8.052,5.81943) -- (8.23,5.81943);
\draw [c] (8.23,5.81943) -- (8.408,5.81943);
\draw [c] (8.586,5.94863) -- (8.586,5.96911);
\draw [c] (8.586,5.96911) -- (8.586,5.9889);
\draw [c] (8.408,5.96911) -- (8.586,5.96911);
\draw [c] (8.586,5.96911) -- (8.764,5.96911);
\draw [c] (8.942,6.04854) -- (8.942,6.06736);
\draw [c] (8.942,6.06736) -- (8.942,6.0856);
\draw [c] (8.764,6.06736) -- (8.942,6.06736);
\draw [c] (8.942,6.06736) -- (9.12,6.06736);
\draw [c] (9.298,6.14821) -- (9.298,6.16551);
\draw [c] (9.298,6.16551) -- (9.298,6.18232);
\draw [c] (9.12,6.16551) -- (9.298,6.16551);
\draw [c] (9.298,6.16551) -- (9.476,6.16551);
\draw [c] (9.654,6.22967) -- (9.654,6.24582);
\draw [c] (9.654,6.24582) -- (9.654,6.26154);
\draw [c] (9.476,6.24582) -- (9.654,6.24582);
\draw [c] (9.654,6.24582) -- (9.832,6.24582);
\draw [c] (10.01,6.26473) -- (10.01,6.28041);
\draw [c] (10.01,6.28041) -- (10.01,6.29568);
\draw [c] (9.832,6.28041) -- (10.01,6.28041);
\draw [c] (10.01,6.28041) -- (10.188,6.28041);
\draw [c] (10.366,6.28884) -- (10.366,6.3042);
\draw [c] (10.366,6.3042) -- (10.366,6.31917);
\draw [c] (10.188,6.3042) -- (10.366,6.3042);
\draw [c] (10.366,6.3042) -- (10.544,6.3042);
\draw [c] (10.722,6.30209) -- (10.722,6.31728);
\draw [c] (10.722,6.31728) -- (10.722,6.33208);
\draw [c] (10.544,6.31728) -- (10.722,6.31728);
\draw [c] (10.722,6.31728) -- (10.9,6.31728);
\draw [c] (11.078,6.2757) -- (11.078,6.29123);
\draw [c] (11.078,6.29123) -- (11.078,6.30637);
\draw [c] (10.9,6.29123) -- (11.078,6.29123);
\draw [c] (11.078,6.29123) -- (11.256,6.29123);
\draw [c] (11.434,6.28171) -- (11.434,6.29716);
\draw [c] (11.434,6.29716) -- (11.434,6.31222);
\draw [c] (11.256,6.29716) -- (11.434,6.29716);
\draw [c] (11.434,6.29716) -- (11.612,6.29716);
\draw [c] (11.79,6.2396) -- (11.79,6.25561);
\draw [c] (11.79,6.25561) -- (11.79,6.2712);
\draw [c] (11.612,6.25561) -- (11.79,6.25561);
\draw [c] (11.79,6.25561) -- (11.968,6.25561);
\draw [c] (12.146,6.1975) -- (12.146,6.21409);
\draw [c] (12.146,6.21409) -- (12.146,6.23023);
\draw [c] (11.968,6.21409) -- (12.146,6.21409);
\draw [c] (12.146,6.21409) -- (12.324,6.21409);
\draw [c] (12.502,6.19381) -- (12.502,6.21046);
\draw [c] (12.502,6.21046) -- (12.502,6.22664);
\draw [c] (12.324,6.21046) -- (12.502,6.21046);
\draw [c] (12.502,6.21046) -- (12.68,6.21046);
\draw [c] (12.858,6.13459) -- (12.858,6.15209);
\draw [c] (12.858,6.15209) -- (12.858,6.16909);
\draw [c] (12.68,6.15209) -- (12.858,6.15209);
\draw [c] (12.858,6.15209) -- (13.036,6.15209);
\draw [c] (13.214,6.05089) -- (13.214,6.06968);
\draw [c] (13.214,6.06968) -- (13.214,6.08788);
\draw [c] (13.036,6.06968) -- (13.214,6.06968);
\draw [c] (13.214,6.06968) -- (13.392,6.06968);
\draw [c] (13.57,6.02997) -- (13.57,6.04909);
\draw [c] (13.57,6.04909) -- (13.57,6.06761);
\draw [c] (13.392,6.04909) -- (13.57,6.04909);
\draw [c] (13.57,6.04909) -- (13.748,6.04909);
\draw [c] (13.926,5.94232) -- (13.926,5.96291);
\draw [c] (13.926,5.96291) -- (13.926,5.9828);
\draw [c] (13.748,5.96291) -- (13.926,5.96291);
\draw [c] (13.926,5.96291) -- (14.104,5.96291);
\draw [c] (14.282,5.81822) -- (14.282,5.84109);
\draw [c] (14.282,5.84109) -- (14.282,5.8631);
\draw [c] (14.104,5.84109) -- (14.282,5.84109);
\draw [c] (14.282,5.84109) -- (14.46,5.84109);
\draw [c] (14.638,5.80772) -- (14.638,5.83079);
\draw [c] (14.638,5.83079) -- (14.638,5.85299);
\draw [c] (14.46,5.83079) -- (14.638,5.83079);
\draw [c] (14.638,5.83079) -- (14.816,5.83079);
\draw [c] (14.994,5.68352) -- (14.994,5.70914);
\draw [c] (14.994,5.70914) -- (14.994,5.7337);
\draw [c] (14.816,5.70914) -- (14.994,5.70914);
\draw [c] (14.994,5.70914) -- (15.172,5.70914);
\draw [c] (15.35,5.55821) -- (15.35,5.58669);
\draw [c] (15.35,5.58669) -- (15.35,5.61387);
\draw [c] (15.172,5.58669) -- (15.35,5.58669);
\draw [c] (15.35,5.58669) -- (15.528,5.58669);
\draw [c] (15.706,5.43347) -- (15.706,5.46512);
\draw [c] (15.706,5.46512) -- (15.706,5.49517);
\draw [c] (15.528,5.46512) -- (15.706,5.46512);
\draw [c] (15.706,5.46512) -- (15.884,5.46512);
\draw [c] (16.062,5.27143) -- (16.062,5.30773);
\draw [c] (16.062,5.30773) -- (16.062,5.34193);
\draw [c] (15.884,5.30773) -- (16.062,5.30773);
\draw [c] (16.062,5.30773) -- (16.24,5.30773);
\draw [c] (16.418,5.17036) -- (16.418,5.2099);
\draw [c] (16.418,5.2099) -- (16.418,5.24697);
\draw [c] (16.24,5.2099) -- (16.418,5.2099);
\draw [c] (16.418,5.2099) -- (16.596,5.2099);
\draw [c] (16.774,5.11124) -- (16.774,5.1528);
\draw [c] (16.774,5.1528) -- (16.774,5.19164);
\draw [c] (16.596,5.1528) -- (16.774,5.1528);
\draw [c] (16.774,5.1528) -- (16.952,5.1528);
\draw [c] (17.13,5.01331) -- (17.13,5.05847);
\draw [c] (17.13,5.05847) -- (17.13,5.10042);
\draw [c] (16.952,5.05847) -- (17.13,5.05847);
\draw [c] (17.13,5.05847) -- (17.308,5.05847);
\draw [c] (17.486,4.8674) -- (17.486,4.91848);
\draw [c] (17.486,4.91848) -- (17.486,4.96549);
\draw [c] (17.308,4.91848) -- (17.486,4.91848);
\draw [c] (17.486,4.91848) -- (17.664,4.91848);
\draw [c] (17.842,4.72485) -- (17.842,4.78247);
\draw [c] (17.842,4.78247) -- (17.842,4.83497);
\draw [c] (17.664,4.78247) -- (17.842,4.78247);
\draw [c] (17.842,4.78247) -- (18.02,4.78247);
\draw [c] (18.198,4.53806) -- (18.198,4.60553);
\draw [c] (18.198,4.60553) -- (18.198,4.66608);
\draw [c] (18.02,4.60553) -- (18.198,4.60553);
\draw [c] (18.198,4.60553) -- (18.376,4.60553);
\draw [c] (18.554,4.43502) -- (18.554,4.50862);
\draw [c] (18.554,4.50862) -- (18.554,4.57407);
\draw [c] (18.376,4.50862) -- (18.554,4.50862);
\draw [c] (18.554,4.50862) -- (18.732,4.50862);
\draw [c] (18.91,4.21977) -- (18.91,4.30804);
\draw [c] (18.91,4.30804) -- (18.91,4.38483);
\draw [c] (18.732,4.30804) -- (18.91,4.30804);
\draw [c] (18.91,4.30804) -- (19.088,4.30804);
\draw [c] (19.266,3.7691) -- (19.266,3.89818);
\draw [c] (19.266,3.89818) -- (19.266,4.00407);
\draw [c] (19.088,3.89818) -- (19.266,3.89818);
\draw [c] (19.266,3.89818) -- (19.444,3.89818);
\draw [c] (19.622,3.79411) -- (19.622,3.9205);
\draw [c] (19.622,3.9205) -- (19.622,4.02457);
\draw [c] (19.444,3.9205) -- (19.622,3.9205);
\draw [c] (19.622,3.9205) -- (19.8,3.9205);
\colorlet{c}{natcomp};
\draw [c] (2.178,13.0438) -- (2.178,13.0463);
\draw [c] (2.178,13.0463) -- (2.178,13.0488);
\draw [c] (2,13.0463) -- (2.178,13.0463);
\draw [c] (2.178,13.0463) -- (2.356,13.0463);
\draw [c] (2.534,11.8244) -- (2.534,11.8314);
\draw [c] (2.534,11.8314) -- (2.534,11.8384);
\draw [c] (2.356,11.8314) -- (2.534,11.8314);
\draw [c] (2.534,11.8314) -- (2.712,11.8314);
\draw [c] (2.89,10.989) -- (2.89,11.0032);
\draw [c] (2.89,11.0032) -- (2.89,11.0171);
\draw [c] (2.712,11.0032) -- (2.89,11.0032);
\draw [c] (2.89,11.0032) -- (3.068,11.0032);
\draw [c] (3.246,10.3672) -- (3.246,10.3913);
\draw [c] (3.246,10.3913) -- (3.246,10.4145);
\draw [c] (3.068,10.3913) -- (3.246,10.3913);
\draw [c] (3.246,10.3913) -- (3.424,10.3913);
\draw [c] (3.602,9.90888) -- (3.602,9.94441);
\draw [c] (3.602,9.94441) -- (3.602,9.97793);
\draw [c] (3.424,9.94441) -- (3.602,9.94441);
\draw [c] (3.602,9.94441) -- (3.78,9.94441);
\draw [c] (3.958,9.49613) -- (3.958,9.5465);
\draw [c] (3.958,9.5465) -- (3.958,9.59291);
\draw [c] (3.78,9.5465) -- (3.958,9.5465);
\draw [c] (3.958,9.5465) -- (4.136,9.5465);
\draw [c] (4.314,9.05563) -- (4.314,9.1287);
\draw [c] (4.314,9.1287) -- (4.314,9.19373);
\draw [c] (4.136,9.1287) -- (4.314,9.1287);
\draw [c] (4.314,9.1287) -- (4.492,9.1287);
\draw [c] (4.67,8.83181) -- (4.67,8.92008);
\draw [c] (4.67,8.92008) -- (4.67,8.99687);
\draw [c] (4.492,8.92008) -- (4.67,8.92008);
\draw [c] (4.67,8.92008) -- (4.848,8.92008);
\draw [c] (5.026,8.62995) -- (5.026,8.73461);
\draw [c] (5.026,8.73461) -- (5.026,8.82351);
\draw [c] (4.848,8.73461) -- (5.026,8.73461);
\draw [c] (5.026,8.73461) -- (5.204,8.73461);
\draw [c] (5.382,8.60535) -- (5.382,8.66387);
\draw [c] (5.382,8.66387) -- (5.382,8.71712);
\draw [c] (5.204,8.66387) -- (5.382,8.66387);
\draw [c] (5.382,8.66387) -- (5.56,8.66387);
\draw [c] (5.738,8.46579) -- (5.738,8.47337);
\draw [c] (5.738,8.47337) -- (5.738,8.48086);
\draw [c] (5.56,8.47337) -- (5.738,8.47337);
\draw [c] (5.738,8.47337) -- (5.916,8.47337);
\draw [c] (6.094,8.37704) -- (6.094,8.38521);
\draw [c] (6.094,8.38521) -- (6.094,8.39327);
\draw [c] (5.916,8.38521) -- (6.094,8.38521);
\draw [c] (6.094,8.38521) -- (6.272,8.38521);
\draw [c] (6.45,8.29983) -- (6.45,8.30856);
\draw [c] (6.45,8.30856) -- (6.45,8.31716);
\draw [c] (6.272,8.30856) -- (6.45,8.30856);
\draw [c] (6.45,8.30856) -- (6.628,8.30856);
\draw [c] (6.806,8.27477) -- (6.806,8.28368);
\draw [c] (6.806,8.28368) -- (6.806,8.29246);
\draw [c] (6.628,8.28368) -- (6.806,8.28368);
\draw [c] (6.806,8.28368) -- (6.984,8.28368);
\draw [c] (7.162,8.24818) -- (7.162,8.25729);
\draw [c] (7.162,8.25729) -- (7.162,8.26627);
\draw [c] (6.984,8.25729) -- (7.162,8.25729);
\draw [c] (7.162,8.25729) -- (7.34,8.25729);
\draw [c] (7.518,8.20928) -- (7.518,8.2187);
\draw [c] (7.518,8.2187) -- (7.518,8.22797);
\draw [c] (7.34,8.2187) -- (7.518,8.2187);
\draw [c] (7.518,8.2187) -- (7.696,8.2187);
\draw [c] (7.874,8.19891) -- (7.874,8.20842);
\draw [c] (7.874,8.20842) -- (7.874,8.21777);
\draw [c] (7.696,8.20842) -- (7.874,8.20842);
\draw [c] (7.874,8.20842) -- (8.052,8.20842);
\draw [c] (8.23,8.18013) -- (8.23,8.18979);
\draw [c] (8.23,8.18979) -- (8.23,8.19929);
\draw [c] (8.052,8.18979) -- (8.23,8.18979);
\draw [c] (8.23,8.18979) -- (8.408,8.18979);
\draw [c] (8.586,8.15815) -- (8.586,8.16798);
\draw [c] (8.586,8.16798) -- (8.586,8.17766);
\draw [c] (8.408,8.16798) -- (8.586,8.16798);
\draw [c] (8.586,8.16798) -- (8.764,8.16798);
\draw [c] (8.942,8.15211) -- (8.942,8.162);
\draw [c] (8.942,8.162) -- (8.942,8.17172);
\draw [c] (8.764,8.162) -- (8.942,8.162);
\draw [c] (8.942,8.162) -- (9.12,8.162);
\draw [c] (9.298,8.12494) -- (9.298,8.13506);
\draw [c] (9.298,8.13506) -- (9.298,8.145);
\draw [c] (9.12,8.13506) -- (9.298,8.13506);
\draw [c] (9.298,8.13506) -- (9.476,8.13506);
\draw [c] (9.654,8.12063) -- (9.654,8.13079);
\draw [c] (9.654,8.13079) -- (9.654,8.14077);
\draw [c] (9.476,8.13079) -- (9.654,8.13079);
\draw [c] (9.654,8.13079) -- (9.832,8.13079);
\draw [c] (10.01,8.09321) -- (10.01,8.10361);
\draw [c] (10.01,8.10361) -- (10.01,8.11382);
\draw [c] (9.832,8.10361) -- (10.01,8.10361);
\draw [c] (10.01,8.10361) -- (10.188,8.10361);
\draw [c] (10.366,8.07052) -- (10.366,8.08111);
\draw [c] (10.366,8.08111) -- (10.366,8.09152);
\draw [c] (10.188,8.08111) -- (10.366,8.08111);
\draw [c] (10.366,8.08111) -- (10.544,8.08111);
\draw [c] (10.722,8.04299) -- (10.722,8.05383);
\draw [c] (10.722,8.05383) -- (10.722,8.06448);
\draw [c] (10.544,8.05383) -- (10.722,8.05383);
\draw [c] (10.722,8.05383) -- (10.9,8.05383);
\draw [c] (11.078,8.03185) -- (11.078,8.0428);
\draw [c] (11.078,8.0428) -- (11.078,8.05355);
\draw [c] (10.9,8.0428) -- (11.078,8.0428);
\draw [c] (11.078,8.0428) -- (11.256,8.0428);
\draw [c] (11.434,7.96071) -- (11.434,7.97233);
\draw [c] (11.434,7.97233) -- (11.434,7.98373);
\draw [c] (11.256,7.97233) -- (11.434,7.97233);
\draw [c] (11.434,7.97233) -- (11.612,7.97233);
\draw [c] (11.79,7.93552) -- (11.79,7.94739);
\draw [c] (11.79,7.94739) -- (11.79,7.95903);
\draw [c] (11.612,7.94739) -- (11.79,7.94739);
\draw [c] (11.79,7.94739) -- (11.968,7.94739);
\draw [c] (12.146,7.87414) -- (12.146,7.88665);
\draw [c] (12.146,7.88665) -- (12.146,7.8989);
\draw [c] (11.968,7.88665) -- (12.146,7.88665);
\draw [c] (12.146,7.88665) -- (12.324,7.88665);
\draw [c] (12.502,7.8471) -- (12.502,7.85989);
\draw [c] (12.502,7.85989) -- (12.502,7.87242);
\draw [c] (12.324,7.85989) -- (12.502,7.85989);
\draw [c] (12.502,7.85989) -- (12.68,7.85989);
\draw [c] (12.858,7.7596) -- (12.858,7.77338);
\draw [c] (12.858,7.77338) -- (12.858,7.78684);
\draw [c] (12.68,7.77338) -- (12.858,7.77338);
\draw [c] (12.858,7.77338) -- (13.036,7.77338);
\draw [c] (13.214,7.70756) -- (13.214,7.72196);
\draw [c] (13.214,7.72196) -- (13.214,7.73601);
\draw [c] (13.036,7.72196) -- (13.214,7.72196);
\draw [c] (13.214,7.72196) -- (13.392,7.72196);
\draw [c] (13.57,7.63897) -- (13.57,7.65423);
\draw [c] (13.57,7.65423) -- (13.57,7.66911);
\draw [c] (13.392,7.65423) -- (13.57,7.65423);
\draw [c] (13.57,7.65423) -- (13.748,7.65423);
\draw [c] (13.926,7.56189) -- (13.926,7.57817);
\draw [c] (13.926,7.57817) -- (13.926,7.59402);
\draw [c] (13.748,7.57817) -- (13.926,7.57817);
\draw [c] (13.926,7.57817) -- (14.104,7.57817);
\draw [c] (14.282,7.47177) -- (14.282,7.48935);
\draw [c] (14.282,7.48935) -- (14.282,7.50641);
\draw [c] (14.104,7.48935) -- (14.282,7.48935);
\draw [c] (14.282,7.48935) -- (14.46,7.48935);
\draw [c] (14.638,7.3932) -- (14.638,7.41198);
\draw [c] (14.638,7.41198) -- (14.638,7.43018);
\draw [c] (14.46,7.41198) -- (14.638,7.41198);
\draw [c] (14.638,7.41198) -- (14.816,7.41198);
\draw [c] (14.994,7.32283) -- (14.994,7.34277);
\draw [c] (14.994,7.34277) -- (14.994,7.36205);
\draw [c] (14.816,7.34277) -- (14.994,7.34277);
\draw [c] (14.994,7.34277) -- (15.172,7.34277);
\draw [c] (15.35,7.1835) -- (15.35,7.20593);
\draw [c] (15.35,7.20593) -- (15.35,7.22754);
\draw [c] (15.172,7.20593) -- (15.35,7.20593);
\draw [c] (15.35,7.20593) -- (15.528,7.20593);
\draw [c] (15.706,7.09066) -- (15.706,7.11492);
\draw [c] (15.706,7.11492) -- (15.706,7.13822);
\draw [c] (15.528,7.11492) -- (15.706,7.11492);
\draw [c] (15.706,7.11492) -- (15.884,7.11492);
\draw [c] (16.062,6.9085) -- (16.062,6.9368);
\draw [c] (16.062,6.9368) -- (16.062,6.9638);
\draw [c] (15.884,6.9368) -- (16.062,6.9368);
\draw [c] (16.062,6.9368) -- (16.24,6.9368);
\draw [c] (16.418,6.79845) -- (16.418,6.8295);
\draw [c] (16.418,6.8295) -- (16.418,6.85901);
\draw [c] (16.24,6.8295) -- (16.418,6.8295);
\draw [c] (16.418,6.8295) -- (16.596,6.8295);
\draw [c] (16.774,6.57347) -- (16.774,6.61104);
\draw [c] (16.774,6.61104) -- (16.774,6.64635);
\draw [c] (16.596,6.61104) -- (16.774,6.61104);
\draw [c] (16.774,6.61104) -- (16.952,6.61104);
\draw [c] (17.13,6.48651) -- (17.13,6.52694);
\draw [c] (17.13,6.52694) -- (17.13,6.56477);
\draw [c] (16.952,6.52694) -- (17.13,6.52694);
\draw [c] (17.13,6.52694) -- (17.308,6.52694);
\draw [c] (17.486,6.31079) -- (17.486,6.35769);
\draw [c] (17.486,6.35769) -- (17.486,6.40114);
\draw [c] (17.308,6.35769) -- (17.486,6.35769);
\draw [c] (17.486,6.35769) -- (17.664,6.35769);
\draw [c] (17.842,6.24265) -- (17.842,6.29233);
\draw [c] (17.842,6.29233) -- (17.842,6.33815);
\draw [c] (17.664,6.29233) -- (17.842,6.29233);
\draw [c] (17.842,6.29233) -- (18.02,6.29233);
\draw [c] (18.198,6.09848) -- (18.198,6.15459);
\draw [c] (18.198,6.15459) -- (18.198,6.20584);
\draw [c] (18.02,6.15459) -- (18.198,6.15459);
\draw [c] (18.198,6.15459) -- (18.376,6.15459);
\draw [c] (18.554,5.84277) -- (18.554,5.91241);
\draw [c] (18.554,5.91241) -- (18.554,5.97471);
\draw [c] (18.376,5.91241) -- (18.554,5.91241);
\draw [c] (18.554,5.91241) -- (18.732,5.91241);
\draw [c] (18.91,5.79435) -- (18.91,5.8669);
\draw [c] (18.91,5.8669) -- (18.91,5.93152);
\draw [c] (18.732,5.8669) -- (18.91,5.8669);
\draw [c] (18.91,5.8669) -- (19.088,5.8669);
\draw [c] (19.266,5.69416) -- (19.266,5.77312);
\draw [c] (19.266,5.77312) -- (19.266,5.84277);
\draw [c] (19.088,5.77312) -- (19.266,5.77312);
\draw [c] (19.266,5.77312) -- (19.444,5.77312);
\draw [c] (19.622,5.45494) -- (19.622,5.55156);
\draw [c] (19.622,5.55156) -- (19.622,5.63459);
\draw [c] (19.444,5.55156) -- (19.622,5.55156);
\draw [c] (19.622,5.55156) -- (19.8,5.55156);
\colorlet{c}{natcomp!50};
\draw [c] (2.178,13.128) -- (2.178,13.1332);
\draw [c] (2.178,13.1332) -- (2.178,13.1383);
\draw [c] (2,13.1332) -- (2.178,13.1332);
\draw [c] (2.178,13.1332) -- (2.356,13.1332);
\draw [c] (2.534,11.9) -- (2.534,11.9146);
\draw [c] (2.534,11.9146) -- (2.534,11.9288);
\draw [c] (2.356,11.9146) -- (2.534,11.9146);
\draw [c] (2.534,11.9146) -- (2.712,11.9146);
\draw [c] (2.89,10.9549) -- (2.89,10.9872);
\draw [c] (2.89,10.9872) -- (2.89,11.0179);
\draw [c] (2.712,10.9872) -- (2.89,10.9872);
\draw [c] (2.89,10.9872) -- (3.068,10.9872);
\draw [c] (3.246,10.4385) -- (3.246,10.4533);
\draw [c] (3.246,10.4533) -- (3.246,10.4678);
\draw [c] (3.068,10.4533) -- (3.246,10.4533);
\draw [c] (3.246,10.4533) -- (3.424,10.4533);
\draw [c] (3.602,9.92102) -- (3.602,9.92578);
\draw [c] (3.602,9.92578) -- (3.602,9.93051);
\draw [c] (3.424,9.92578) -- (3.602,9.92578);
\draw [c] (3.602,9.92578) -- (3.78,9.92578);
\draw [c] (3.958,9.43251) -- (3.958,9.43971);
\draw [c] (3.958,9.43971) -- (3.958,9.44682);
\draw [c] (3.78,9.43971) -- (3.958,9.43971);
\draw [c] (3.958,9.43971) -- (4.136,9.43971);
\draw [c] (4.314,8.968) -- (4.314,8.97867);
\draw [c] (4.314,8.97867) -- (4.314,8.98914);
\draw [c] (4.136,8.97867) -- (4.314,8.97867);
\draw [c] (4.314,8.97867) -- (4.492,8.97867);
\draw [c] (4.67,8.46953) -- (4.67,8.48579);
\draw [c] (4.67,8.48579) -- (4.67,8.50161);
\draw [c] (4.492,8.48579) -- (4.67,8.48579);
\draw [c] (4.67,8.48579) -- (4.848,8.48579);
\draw [c] (5.026,7.92361) -- (5.026,7.9494);
\draw [c] (5.026,7.9494) -- (5.026,7.97411);
\draw [c] (4.848,7.9494) -- (5.026,7.9494);
\draw [c] (5.026,7.9494) -- (5.204,7.9494);
\draw [c] (5.382,7.44391) -- (5.382,7.46541);
\draw [c] (5.382,7.46541) -- (5.382,7.48616);
\draw [c] (5.204,7.46541) -- (5.382,7.46541);
\draw [c] (5.382,7.46541) -- (5.56,7.46541);
\draw [c] (5.738,6.96592) -- (5.738,6.98343);
\draw [c] (5.738,6.98343) -- (5.738,7.00043);
\draw [c] (5.56,6.98343) -- (5.738,6.98343);
\draw [c] (5.738,6.98343) -- (5.916,6.98343);
\draw [c] (6.094,6.7079) -- (6.094,6.72967);
\draw [c] (6.094,6.72967) -- (6.094,6.75067);
\draw [c] (5.916,6.72967) -- (6.094,6.72967);
\draw [c] (6.094,6.72967) -- (6.272,6.72967);
\draw [c] (6.45,6.88858) -- (6.45,6.90727);
\draw [c] (6.45,6.90727) -- (6.45,6.92538);
\draw [c] (6.272,6.90727) -- (6.45,6.90727);
\draw [c] (6.45,6.90727) -- (6.628,6.90727);
\draw [c] (6.806,7.13662) -- (6.806,7.15177);
\draw [c] (6.806,7.15177) -- (6.806,7.16654);
\draw [c] (6.628,7.15177) -- (6.806,7.15177);
\draw [c] (6.806,7.15177) -- (6.984,7.15177);
\draw [c] (7.162,7.37765) -- (7.162,7.39001);
\draw [c] (7.162,7.39001) -- (7.162,7.40211);
\draw [c] (6.984,7.39001) -- (7.162,7.39001);
\draw [c] (7.162,7.39001) -- (7.34,7.39001);
\draw [c] (7.518,7.50159) -- (7.518,7.51272);
\draw [c] (7.518,7.51272) -- (7.518,7.52364);
\draw [c] (7.34,7.51272) -- (7.518,7.51272);
\draw [c] (7.518,7.51272) -- (7.696,7.51272);
\draw [c] (7.874,7.61241) -- (7.874,7.62255);
\draw [c] (7.874,7.62255) -- (7.874,7.63251);
\draw [c] (7.696,7.62255) -- (7.874,7.62255);
\draw [c] (7.874,7.62255) -- (8.052,7.62255);
\draw [c] (8.23,7.69282) -- (8.23,7.70229);
\draw [c] (8.23,7.70229) -- (8.23,7.71161);
\draw [c] (8.052,7.70229) -- (8.23,7.70229);
\draw [c] (8.23,7.70229) -- (8.408,7.70229);
\draw [c] (8.586,7.74917) -- (8.586,7.7582);
\draw [c] (8.586,7.7582) -- (8.586,7.76709);
\draw [c] (8.408,7.7582) -- (8.586,7.7582);
\draw [c] (8.586,7.7582) -- (8.764,7.7582);
\draw [c] (8.942,7.78642) -- (8.942,7.79517);
\draw [c] (8.942,7.79517) -- (8.942,7.80379);
\draw [c] (8.764,7.79517) -- (8.942,7.79517);
\draw [c] (8.942,7.79517) -- (9.12,7.79517);
\draw [c] (9.298,7.80464) -- (9.298,7.81325);
\draw [c] (9.298,7.81325) -- (9.298,7.82174);
\draw [c] (9.12,7.81325) -- (9.298,7.81325);
\draw [c] (9.298,7.81325) -- (9.476,7.81325);
\draw [c] (9.654,7.82653) -- (9.654,7.83498);
\draw [c] (9.654,7.83498) -- (9.654,7.84332);
\draw [c] (9.476,7.83498) -- (9.654,7.83498);
\draw [c] (9.654,7.83498) -- (9.832,7.83498);
\draw [c] (10.01,7.81476) -- (10.01,7.8233);
\draw [c] (10.01,7.8233) -- (10.01,7.83172);
\draw [c] (9.832,7.8233) -- (10.01,7.8233);
\draw [c] (10.01,7.8233) -- (10.188,7.8233);
\draw [c] (10.366,7.80861) -- (10.366,7.81719);
\draw [c] (10.366,7.81719) -- (10.366,7.82566);
\draw [c] (10.188,7.81719) -- (10.366,7.81719);
\draw [c] (10.366,7.81719) -- (10.544,7.81719);
\draw [c] (10.722,7.79585) -- (10.722,7.80453);
\draw [c] (10.722,7.80453) -- (10.722,7.81308);
\draw [c] (10.544,7.80453) -- (10.722,7.80453);
\draw [c] (10.722,7.80453) -- (10.9,7.80453);
\draw [c] (11.078,7.7623) -- (11.078,7.77122);
\draw [c] (11.078,7.77122) -- (11.078,7.78002);
\draw [c] (10.9,7.77122) -- (11.078,7.77122);
\draw [c] (11.078,7.77122) -- (11.256,7.77122);
\draw [c] (11.434,7.74712) -- (11.434,7.75616);
\draw [c] (11.434,7.75616) -- (11.434,7.76507);
\draw [c] (11.256,7.75616) -- (11.434,7.75616);
\draw [c] (11.434,7.75616) -- (11.612,7.75616);
\draw [c] (11.79,7.68845) -- (11.79,7.69795);
\draw [c] (11.79,7.69795) -- (11.79,7.7073);
\draw [c] (11.612,7.69795) -- (11.79,7.69795);
\draw [c] (11.79,7.69795) -- (11.968,7.69795);
\draw [c] (12.146,7.66653) -- (12.146,7.67621);
\draw [c] (12.146,7.67621) -- (12.146,7.68574);
\draw [c] (11.968,7.67621) -- (12.146,7.67621);
\draw [c] (12.146,7.67621) -- (12.324,7.67621);
\draw [c] (12.502,7.60792) -- (12.502,7.61809);
\draw [c] (12.502,7.61809) -- (12.502,7.62809);
\draw [c] (12.324,7.61809) -- (12.502,7.61809);
\draw [c] (12.502,7.61809) -- (12.68,7.61809);
\draw [c] (12.858,7.5341) -- (12.858,7.54493);
\draw [c] (12.858,7.54493) -- (12.858,7.55556);
\draw [c] (12.68,7.54493) -- (12.858,7.54493);
\draw [c] (12.858,7.54493) -- (13.036,7.54493);
\draw [c] (13.214,7.49868) -- (13.214,7.50983);
\draw [c] (13.214,7.50983) -- (13.214,7.52079);
\draw [c] (13.036,7.50983) -- (13.214,7.50983);
\draw [c] (13.214,7.50983) -- (13.392,7.50983);
\draw [c] (13.57,7.44066) -- (13.57,7.45238);
\draw [c] (13.57,7.45238) -- (13.57,7.46387);
\draw [c] (13.392,7.45238) -- (13.57,7.45238);
\draw [c] (13.57,7.45238) -- (13.748,7.45238);
\draw [c] (13.926,7.34533) -- (13.926,7.35804);
\draw [c] (13.926,7.35804) -- (13.926,7.37047);
\draw [c] (13.748,7.35804) -- (13.926,7.35804);
\draw [c] (13.926,7.35804) -- (14.104,7.35804);
\draw [c] (14.282,7.25889) -- (14.282,7.27256);
\draw [c] (14.282,7.27256) -- (14.282,7.28591);
\draw [c] (14.104,7.27256) -- (14.282,7.27256);
\draw [c] (14.282,7.27256) -- (14.46,7.27256);
\draw [c] (14.638,7.16726) -- (14.638,7.18202);
\draw [c] (14.638,7.18202) -- (14.638,7.19643);
\draw [c] (14.46,7.18202) -- (14.638,7.18202);
\draw [c] (14.638,7.18202) -- (14.816,7.18202);
\draw [c] (14.994,7.0833) -- (14.994,7.09915);
\draw [c] (14.994,7.09915) -- (14.994,7.11459);
\draw [c] (14.816,7.09915) -- (14.994,7.09915);
\draw [c] (14.994,7.09915) -- (15.172,7.09915);
\draw [c] (15.35,6.98847) -- (15.35,7.00565);
\draw [c] (15.35,7.00565) -- (15.35,7.02234);
\draw [c] (15.172,7.00565) -- (15.35,7.00565);
\draw [c] (15.35,7.00565) -- (15.528,7.00565);
\draw [c] (15.706,6.85323) -- (15.706,6.87249);
\draw [c] (15.706,6.87249) -- (15.706,6.89114);
\draw [c] (15.528,6.87249) -- (15.706,6.87249);
\draw [c] (15.706,6.87249) -- (15.884,6.87249);
\draw [c] (16.062,6.73333) -- (16.062,6.75464);
\draw [c] (16.062,6.75464) -- (16.062,6.77521);
\draw [c] (15.884,6.75464) -- (16.062,6.75464);
\draw [c] (16.062,6.75464) -- (16.24,6.75464);
\draw [c] (16.418,6.63513) -- (16.418,6.65828);
\draw [c] (16.418,6.65828) -- (16.418,6.68057);
\draw [c] (16.24,6.65828) -- (16.418,6.65828);
\draw [c] (16.418,6.65828) -- (16.596,6.65828);
\draw [c] (16.774,6.46786) -- (16.774,6.49454);
\draw [c] (16.774,6.49454) -- (16.774,6.52006);
\draw [c] (16.596,6.49454) -- (16.774,6.49454);
\draw [c] (16.774,6.49454) -- (16.952,6.49454);
\draw [c] (17.13,6.23656) -- (17.13,6.26899);
\draw [c] (17.13,6.26899) -- (17.13,6.29974);
\draw [c] (16.952,6.26899) -- (17.13,6.26899);
\draw [c] (17.13,6.26899) -- (17.308,6.26899);
\draw [c] (17.486,6.15505) -- (17.486,6.1898);
\draw [c] (17.486,6.1898) -- (17.486,6.22262);
\draw [c] (17.308,6.1898) -- (17.486,6.1898);
\draw [c] (17.486,6.1898) -- (17.664,6.1898);
\draw [c] (17.842,6.03459) -- (17.842,6.07306);
\draw [c] (17.842,6.07306) -- (17.842,6.10918);
\draw [c] (17.664,6.07306) -- (17.842,6.07306);
\draw [c] (17.842,6.07306) -- (18.02,6.07306);
\draw [c] (18.198,5.88687) -- (18.198,5.93046);
\draw [c] (18.198,5.93046) -- (18.198,5.97106);
\draw [c] (18.02,5.93046) -- (18.198,5.93046);
\draw [c] (18.198,5.93046) -- (18.376,5.93046);
\draw [c] (18.554,5.72411) -- (18.554,5.77413);
\draw [c] (18.554,5.77413) -- (18.554,5.82025);
\draw [c] (18.376,5.77413) -- (18.554,5.77413);
\draw [c] (18.554,5.77413) -- (18.732,5.77413);
\draw [c] (18.91,5.6864) -- (18.91,5.73804);
\draw [c] (18.91,5.73804) -- (18.91,5.78553);
\draw [c] (18.732,5.73804) -- (18.91,5.73804);
\draw [c] (18.91,5.73804) -- (19.088,5.73804);
\draw [c] (19.266,5.45163) -- (19.266,5.5146);
\draw [c] (19.266,5.5146) -- (19.266,5.5715);
\draw [c] (19.088,5.5146) -- (19.266,5.5146);
\draw [c] (19.266,5.5146) -- (19.444,5.5146);
\draw [c] (19.622,5.27543) -- (19.622,5.3485);
\draw [c] (19.622,5.3485) -- (19.622,5.41353);
\draw [c] (19.444,5.3485) -- (19.622,5.3485);
\draw [c] (19.622,5.3485) -- (19.8,5.3485);
\definecolor{c}{rgb}{1,1,1};
\draw [c] (6.86782,10.0862) -- (6.86782,13.046) -- (19.1954,13.046) -- (19.1954,10.0862) -- (6.86782,10.0862);
\draw [c] (6.86782,10.0862) -- (19.1954,10.0862);
\draw [c] (19.1954,10.0862) -- (19.1954,13.046);
\draw [c] (19.1954,13.046) -- (6.86782,13.046);
\draw [c] (6.86782,13.046) -- (6.86782,10.0862);
\draw [anchor=base west] (7.09869,12.5095) node[ rotate=0]{Dest. interference};
\draw [anchor=base west] (13.0929,12.5095) node[ rotate=0]{Const. interference};
\draw [anchor=base west] (8.40876,11.7696) node[ rotate=0]{};
\draw [anchor=base west] (14.4027,11.7696) node[ rotate=0]{Standard Model};
\colorlet{c}{kugray};
\draw [c] (13.0929,11.9361) -- (14.1716,11.9361);
\draw [anchor=base west] (8.40876,11.0296) node[ rotate=0]{};
\colorlet{c}{natgreen!50};
\draw [c] (7.09896,11.1961) -- (8.17762,11.1961);
\draw [anchor=base west] (14.4027,11.0296) node[ rotate=0]{$\Lambda$ = 1.00 TeV};
\colorlet{c}{natgreen};
\draw [c] (13.0929,11.1961) -- (14.1716,11.1961);
\draw [anchor=base west] (8.40876,10.2897) node[ rotate=0]{};
\colorlet{c}{natcomp!50};
\draw [c] (7.09896,10.4562) -- (8.17762,10.4562);
\draw [anchor=base west] (14.4027,10.2897) node[ rotate=0]{$\Lambda$ = 0.75 TeV};
\colorlet{c}{natcomp};
\draw [c] (13.0929,10.4562) -- (14.1716,10.4562);
\end{tikzpicture}

}\end{infilsf} \end{minipage}
\hfill\begin{minipage}[b]{.3\textwidth}
\caption{The effect on the distribution of the invariant masses of the produced photon pairs of introducing the new term into the Lagrangian at various values of the mass scale $\Lambda$, assuming constructive (non-grayed) and destructive (grayed) interference with the SM contribution. Note that the distributions that assume destructive interference produce fewer events than those that assume constructive interference at the same value of $\Lambda$. These Monte Carlo samples were produced with CalcHEP.
\label{interf}}
\end{minipage}
\end{figure}

Given that the distribution of invariant masses contain more events in the sensitive region if we assume constructive interference, a lower bound on the value of $\Lambda$ that we discover while using this assumption will lie below the lower bound that we would find, had we assumed destructive interference. Therefore, we will move forward assuming that the new term interferes constructively with the Standard Model.

One possible interpretation of this new four-point interaction is as a zero-range approximation of a process like the one shown in fig.~\ref{across}, involving some unknown mediating particle \cite{marshaw:zerorange}.

\begin{figure}[htb]
\parbox[t]{.45\textwidth}{\begin{center}\begin{footnotesize}\begin{tikzpicture} [>=triangle 45]
\draw[>-] (-1,.5) -- (0,0);
\draw[<-] (-1,-.5) -- (0,0);
\draw (-2,1) node[left] {$q$} -- (-1,.5);
\draw (-2,-1)  node[left] {$\bar q$} -- (-1,-.5);
\draw[snake=coil,segment aspect=0] (0,0) -- (2,1) node[right] {$\gamma$};
\draw[snake=coil,segment aspect=0] (0,0) -- (2,-1) node[right] {$\gamma$}; 
\end{tikzpicture}
\end{footnotesize}\end{center}
\subcaption{Point $q\bar q \gamma\gamma$ interaction.\label{cross}}}\hfill
\parbox[t]{.45\textwidth}{\begin{center}\begin{footnotesize}
\begin{tikzpicture} [>=triangle 45]
\draw[->] (-2,1) node[left] {$q$} -- (-1.5,.5);
\draw[-<] (-2,-1) node[left] {$\bar q$}  -- (-1.5,-.5);
\draw (-1.5,.5) -- (-1,0);
\draw (-1.5,-.5) -- (-1,0);
\draw[dashed] (-1,0) -- node[below] {$X$} (0,0);
\draw[snake=coil, segment aspect=0] (0,0) -- (1,1)node[right] {$\gamma$};
\draw[snake=coil, segment aspect=0] (0,0) -- (1,-1)node[right] {$\gamma$};
\end{tikzpicture}
\end{footnotesize}\end{center}
\subcaption{The $q\bar q \gamma\gamma$ interaction with mediating particle $X$.\label{across}}}\hfill
\caption{Feynman diagrams of the relevant contact interaction. (a) is the interaction described by the new term in the Lagrangian, while (b) is the type of interaction this can be considered a zero-range approximation of.\label{feyns}}
\end{figure}


\section{The cross section}
Using the methods developed so far, we can calculate the transition amplitudes, and hence the probabilities, associated with single processes. However, by long standing tradition, particle physics is interested in the cross section $\sigma$.

To understand the cross section, consider as an analogy the case of firing a single projectile at a single target. We might at this point imagine an arrow and a bulls-eye or an electron and an atomic nucleus in a piece of gold film. The probability of hitting the target is then the cross sectional area of the target divided by the cross sectional area $A$ of the space where the projectile might fly. Adding the possibility of a number, $N_P$, of projectiles being fired at a number, $N_T$, of different, non-overlapping targets, the number, $N$, of hits is calculated as
\[N=\frac{N_P N_T \sigma}{A}.\]
If we apply this picture to a quantum mechanical system, we are mixing the kinematic probability for two particles coming close enough to interact with the dynamic probability of a particular interaction occurring. We can fix this by expressing the probability $\mathcal P$ as a function of the separation between the interacting particles, given by the impact parameter $\mathbf b$, a 2-D vector, and then integrating over all $\mathbf b$. Then, we can write
\[\sigma=\int d^2b\,\mathcal P(\mathbf b)\propto\mathcal P(\text{in}\rightarrow\text{out})=\left|\braket{\phi\cdots\phi_\text{out}}{\phi\cdots\phi_\text{in}}\right|^2.\]
In this way, we can calculate the cross section of some process from its transition amplitude.

From an experimental point of view, $N_P$, $N_T$ and $A$ all depend on the immediate conditions within the accelerator. They can be combined into the luminosity $\mathscr L$, which in a proton collider contains information about how many protons are brought to an interaction point at a time, and how densely they are packed. By multiplying the luminosity with the cross section of a process, we get the frequency with which that process occurs in the detector. Integrating the luminosity over time, we get the integrated luminosity, which can be thought of as a measure of how many opportunities for interactions there have been over the period of time being integrated over, independent of the fine details of how the experiment was run\footnote{One notable example of a non--fine detail of an experiment: beam energy.}.

\section{Colliding protons \label{sec.pdfth}}
In the processes described so far, the starting point has been the interaction of a quark and an antiquark. And while being able to single out such a process experimentally would certainly be nice, single quarks sadly do not occur in nature. Because quakrs are colour charged particles, they are subject a phenomenon known as colour confinement, which requires that colour charge always occur in bundles which are colour neutral when viewed from the outside. For our purposes, protons are an abundant, stable and easy--to--handle colour--neutral bundle of quarks and gluons, which we use in collisions in place of the naked quarks that would have been optimal for the present analysis.

\begin{figure}[htp]
\begin{minipage}[b]{.69\textwidth}
\begin{infilsf} 
\tiny 
\begin{tikzpicture}
\pgfdeclareplotmark{cross} {
\pgfpathmoveto{\pgfpoint{-0.3\pgfplotmarksize}{\pgfplotmarksize}}
\pgfpathlineto{\pgfpoint{+0.3\pgfplotmarksize}{\pgfplotmarksize}}
\pgfpathlineto{\pgfpoint{+0.3\pgfplotmarksize}{0.3\pgfplotmarksize}}
\pgfpathlineto{\pgfpoint{+1\pgfplotmarksize}{0.3\pgfplotmarksize}}
\pgfpathlineto{\pgfpoint{+1\pgfplotmarksize}{-0.3\pgfplotmarksize}}
\pgfpathlineto{\pgfpoint{+0.3\pgfplotmarksize}{-0.3\pgfplotmarksize}}
\pgfpathlineto{\pgfpoint{+0.3\pgfplotmarksize}{-1.\pgfplotmarksize}}
\pgfpathlineto{\pgfpoint{-0.3\pgfplotmarksize}{-1.\pgfplotmarksize}}
\pgfpathlineto{\pgfpoint{-0.3\pgfplotmarksize}{-0.3\pgfplotmarksize}}
\pgfpathlineto{\pgfpoint{-1.\pgfplotmarksize}{-0.3\pgfplotmarksize}}
\pgfpathlineto{\pgfpoint{-1.\pgfplotmarksize}{0.3\pgfplotmarksize}}
\pgfpathlineto{\pgfpoint{-0.3\pgfplotmarksize}{0.3\pgfplotmarksize}}
\pgfpathclose
\pgfusepathqstroke
}
\pgfdeclareplotmark{cross*} {
\pgfpathmoveto{\pgfpoint{-0.3\pgfplotmarksize}{\pgfplotmarksize}}
\pgfpathlineto{\pgfpoint{+0.3\pgfplotmarksize}{\pgfplotmarksize}}
\pgfpathlineto{\pgfpoint{+0.3\pgfplotmarksize}{0.3\pgfplotmarksize}}
\pgfpathlineto{\pgfpoint{+1\pgfplotmarksize}{0.3\pgfplotmarksize}}
\pgfpathlineto{\pgfpoint{+1\pgfplotmarksize}{-0.3\pgfplotmarksize}}
\pgfpathlineto{\pgfpoint{+0.3\pgfplotmarksize}{-0.3\pgfplotmarksize}}
\pgfpathlineto{\pgfpoint{+0.3\pgfplotmarksize}{-1.\pgfplotmarksize}}
\pgfpathlineto{\pgfpoint{-0.3\pgfplotmarksize}{-1.\pgfplotmarksize}}
\pgfpathlineto{\pgfpoint{-0.3\pgfplotmarksize}{-0.3\pgfplotmarksize}}
\pgfpathlineto{\pgfpoint{-1.\pgfplotmarksize}{-0.3\pgfplotmarksize}}
\pgfpathlineto{\pgfpoint{-1.\pgfplotmarksize}{0.3\pgfplotmarksize}}
\pgfpathlineto{\pgfpoint{-0.3\pgfplotmarksize}{0.3\pgfplotmarksize}}
\pgfpathclose
\pgfusepathqfillstroke
}
\pgfdeclareplotmark{newstar} {
\pgfpathmoveto{\pgfqpoint{0pt}{\pgfplotmarksize}}
\pgfpathlineto{\pgfqpointpolar{44}{0.5\pgfplotmarksize}}
\pgfpathlineto{\pgfqpointpolar{18}{\pgfplotmarksize}}
\pgfpathlineto{\pgfqpointpolar{-20}{0.5\pgfplotmarksize}}
\pgfpathlineto{\pgfqpointpolar{-54}{\pgfplotmarksize}}
\pgfpathlineto{\pgfqpointpolar{-90}{0.5\pgfplotmarksize}}
\pgfpathlineto{\pgfqpointpolar{234}{\pgfplotmarksize}}
\pgfpathlineto{\pgfqpointpolar{198}{0.5\pgfplotmarksize}}
\pgfpathlineto{\pgfqpointpolar{162}{\pgfplotmarksize}}
\pgfpathlineto{\pgfqpointpolar{134}{0.5\pgfplotmarksize}}
\pgfpathclose
\pgfusepathqstroke
}
\pgfdeclareplotmark{newstar*} {
\pgfpathmoveto{\pgfqpoint{0pt}{\pgfplotmarksize}}
\pgfpathlineto{\pgfqpointpolar{44}{0.5\pgfplotmarksize}}
\pgfpathlineto{\pgfqpointpolar{18}{\pgfplotmarksize}}
\pgfpathlineto{\pgfqpointpolar{-20}{0.5\pgfplotmarksize}}
\pgfpathlineto{\pgfqpointpolar{-54}{\pgfplotmarksize}}
\pgfpathlineto{\pgfqpointpolar{-90}{0.5\pgfplotmarksize}}
\pgfpathlineto{\pgfqpointpolar{234}{\pgfplotmarksize}}
\pgfpathlineto{\pgfqpointpolar{198}{0.5\pgfplotmarksize}}
\pgfpathlineto{\pgfqpointpolar{162}{\pgfplotmarksize}}
\pgfpathlineto{\pgfqpointpolar{134}{0.5\pgfplotmarksize}}
\pgfpathclose
\pgfusepathqfillstroke
}
\definecolor{c}{rgb}{1,1,1};
%\draw [color=c, fill=c] (0,0) rectangle (10,6.79598);
%\draw [color=c, fill=c] (1,0.679598) rectangle (9,6.11638);
\colorlet{c}{natgreen};
\draw [c,line width=0.9] (1.09174,5.62466) -- (1.18964,5.47481) -- (1.28743,5.32496) -- (1.38533,5.18664) -- (1.48283,5.04255) -- (1.58099,4.91287) -- (1.67857,4.78031) -- (1.77628,4.65352) -- (1.87411,4.53248) -- (1.97224,4.41433) --
 (2.06982,4.29906) -- (2.16774,4.18668) -- (2.26553,4.08293) -- (2.36339,3.97631) -- (2.46118,3.87833) -- (2.55898,3.78035) -- (2.65677,3.68813) -- (2.75466,3.5988) -- (2.85239,3.51235) -- (2.95019,3.42791) -- (3.04802,3.34838) -- (3.14572,3.27115)
 -- (3.2437,3.1968) -- (3.34166,3.12619) -- (3.43917,3.05703) -- (3.5374,2.99306) -- (3.6351,2.92908) -- (3.73264,2.87058) -- (3.83052,2.81237) -- (3.92828,2.75935) -- (4.02609,2.7069) -- (4.12398,2.65878) -- (4.22182,2.61209) -- (4.31966,2.56915) --
 (4.41741,2.52795) -- (4.51524,2.48991) -- (4.61315,2.45446) -- (4.7109,2.42132) -- (4.8087,2.39135) -- (4.90658,2.3634) -- (5.00435,2.33862) -- (5.1024,2.31585) -- (5.2002,2.29654) -- (5.29791,2.27896) -- (5.39574,2.26484) -- (5.49373,2.25274) --
 (5.59128,2.24352) -- (5.68925,2.23689) -- (5.78686,2.23286) -- (5.8846,2.23141) -- (5.98266,2.23228) -- (6.08045,2.23603) -- (6.17828,2.2415) -- (6.27599,2.24957) -- (6.37379,2.25937) -- (6.47169,2.27118) -- (6.56942,2.28444) -- (6.66728,2.29856) --
 (6.7651,2.31354) -- (6.86287,2.32853) -- (6.9607,2.34294) -- (7.05855,2.35591) -- (7.15634,2.36628) -- (7.25422,2.37262) -- (7.35211,2.37377) -- (7.44954,2.36714) -- (7.54772,2.35072) -- (7.64536,2.32161) -- (7.74327,2.27695) -- (7.84106,2.21384) --
 (7.93873,2.12969) -- (8.03678,2.02249) -- (8.13457,1.89108) -- (8.23233,1.73748) -- (8.33009,1.56544) -- (8.42796,1.38188) -- (8.52571,1.19745) -- (8.62355,1.02541) -- (8.72137,0.879619) -- (8.81921,0.771641) -- (8.91702,0.707505);
\colorlet{c}{natscatg};
\draw [c,line width=0.9] (1.09174,5.59585) -- (1.18964,5.44311) -- (1.28743,5.29326) -- (1.38533,5.14918) -- (1.48283,5.00797) -- (1.58099,4.87253) -- (1.67857,4.73997) -- (1.77628,4.61317) -- (1.87411,4.48638) -- (1.97224,4.36822) --
 (2.06982,4.25007) -- (2.16774,4.13769) -- (2.26553,4.02818) -- (2.36339,3.92156) -- (2.46118,3.82069) -- (2.55898,3.71983) -- (2.65677,3.62474) -- (2.75466,3.5305) -- (2.85239,3.44203) -- (2.95019,3.35443) -- (3.04802,3.27115) -- (3.14572,3.19017)
 -- (3.2437,3.11207) -- (3.34166,3.03744) -- (3.43917,2.96395) -- (3.5374,2.89566) -- (3.6351,2.82707) -- (3.73264,2.76367) -- (3.83052,2.70027) -- (3.92828,2.64177) -- (4.02609,2.58385) -- (4.12398,2.52967) -- (4.22182,2.47694) -- (4.31966,2.42708)
 -- (4.41741,2.37925) -- (4.51524,2.334) -- (4.61315,2.29078) -- (4.7109,2.24957) -- (4.8087,2.21095) -- (4.90658,2.17378) -- (5.00435,2.13978) -- (5.1024,2.10664) -- (5.2002,2.07638) -- (5.29791,2.04756) -- (5.39574,2.02134) -- (5.49373,1.99626) --
 (5.59128,1.9735) -- (5.68925,1.95246) -- (5.78686,1.93287) -- (5.8846,1.91529) -- (5.98266,1.89886) -- (6.08045,1.88417) -- (6.17828,1.87033) -- (6.27599,1.85765) -- (6.37379,1.84584) -- (6.47169,1.83431) -- (6.56942,1.82336) -- (6.66728,1.81183) --
 (6.7651,1.79973) -- (6.86287,1.78647) -- (6.9607,1.7712) -- (7.05855,1.75305) -- (7.15634,1.73143) -- (7.25422,1.70521) -- (7.35211,1.6738) -- (7.44954,1.63605) -- (7.54772,1.59109) -- (7.64536,1.53864) -- (7.74327,1.47813) -- (7.84106,1.40925) --
 (7.93873,1.3326) -- (8.03678,1.24932) -- (8.13457,1.16085) -- (8.23233,1.07007) -- (8.33009,0.979875) -- (8.42796,0.894777) -- (8.52571,0.819506) -- (8.62355,0.758615) -- (8.72137,0.715648) -- (8.81921,0.6912) -- (8.91702,0.681552);
\colorlet{c}{natgreen!60};
\draw [c,line width=0.9] (1.09174,5.54397) -- (1.18964,5.39124) -- (1.28743,5.23563) -- (1.38533,5.09154) -- (1.48283,4.94457) -- (1.58099,4.80913) -- (1.67857,4.67081) -- (1.77628,4.54113) -- (1.87411,4.41145) -- (1.97224,4.28754) --
 (2.06982,4.16938) -- (2.16774,4.05123) -- (2.26553,3.93885) -- (2.36339,3.82646) -- (2.46118,3.71983) -- (2.55898,3.61321) -- (2.65677,3.51494) -- (2.75466,3.41495) -- (2.85239,3.32071) -- (2.95019,3.22706) -- (3.04802,3.13743) -- (3.14572,3.05012)
 -- (3.2437,2.96482) -- (3.34166,2.88298) -- (3.43917,2.802) -- (3.5374,2.72563) -- (3.6351,2.64869) -- (3.73264,2.57665) -- (3.83052,2.50403) -- (3.92828,2.43602) -- (4.02609,2.36772) -- (4.12398,2.30317) -- (4.22182,2.2392) -- (4.31966,2.17781) --
 (4.41741,2.11759) -- (4.51524,2.05966) -- (4.61315,2.00318) -- (4.7109,1.94814) -- (4.8087,1.89483) -- (4.90658,1.84267) -- (5.00435,1.79253) -- (5.1024,1.74296) -- (5.2002,1.69541) -- (5.29791,1.64844) -- (5.39574,1.6032) -- (5.49373,1.55882) --
 (5.59128,1.51588) -- (5.68925,1.47352) -- (5.78686,1.43231) -- (5.8846,1.39196) -- (5.98266,1.3522) -- (6.08045,1.31358) -- (6.17828,1.27525) -- (6.27599,1.2375) -- (6.37379,1.20062) -- (6.47169,1.16373) -- (6.56942,1.12771) -- (6.66728,1.09169) --
 (6.7651,1.05653) -- (6.86287,1.02166) -- (6.9607,0.987944) -- (7.05855,0.95489) -- (7.15634,0.923105) -- (7.25422,0.892818) -- (7.35211,0.864317) -- (7.44954,0.837834) -- (7.54772,0.813541) -- (7.64536,0.791611) -- (7.74327,0.771986) --
 (7.84106,0.754667) -- (7.93873,0.73948) -- (8.03678,0.726224) -- (8.13457,0.714669) -- (8.23233,0.704666) -- (8.33009,0.696162) -- (8.42796,0.689266) -- (8.52571,0.68426) -- (8.62355,0.681306) -- (8.72137,0.679598) -- (8.81921,0.679598) --
 (8.91702,0.679598);
\colorlet{c}{natscatg!60};
\draw [c,line width=0.9] (1.09174,5.54686) -- (1.18964,5.39124) -- (1.28743,5.23851) -- (1.38533,5.09442) -- (1.48283,4.94745) -- (1.58099,4.81201) -- (1.67857,4.67369) -- (1.77628,4.54401) -- (1.87411,4.41433) -- (1.97224,4.2933) --
 (2.06982,4.17227) -- (2.16774,4.05412) -- (2.26553,3.94173) -- (2.36339,3.82934) -- (2.46118,3.7256) -- (2.55898,3.61897) -- (2.65677,3.52071) -- (2.75466,3.42129) -- (2.85239,3.32763) -- (2.95019,3.23455) -- (3.04802,3.14579) -- (3.14572,3.05876)
 -- (3.2437,2.97433) -- (3.34166,2.89335) -- (3.43917,2.81295) -- (3.5374,2.73745) -- (3.6351,2.66137) -- (3.73264,2.59019) -- (3.83052,2.51872) -- (3.92828,2.45187) -- (4.02609,2.38472) -- (4.12398,2.32161) -- (4.22182,2.25879) -- (4.31966,2.19914)
 -- (4.41741,2.14064) -- (4.51524,2.08416) -- (4.61315,2.02969) -- (4.7109,1.97638) -- (4.8087,1.92537) -- (4.90658,1.87523) -- (5.00435,1.82739) -- (5.1024,1.78042) -- (5.2002,1.73547) -- (5.29791,1.69138) -- (5.39574,1.6493) -- (5.49373,1.60809) --
 (5.59128,1.56833) -- (5.68925,1.52971) -- (5.78686,1.49196) -- (5.8846,1.45565) -- (5.98266,1.4202) -- (6.08045,1.38591) -- (6.17828,1.35191) -- (6.27599,1.31906) -- (6.37379,1.28678) -- (6.47169,1.25508) -- (6.56942,1.22367) -- (6.66728,1.19226) --
 (6.7651,1.16114) -- (6.86287,1.12973) -- (6.9607,1.09803) -- (7.05855,1.06575) -- (7.15634,1.03261) -- (7.25422,0.998606) -- (7.35211,0.963853) -- (7.44954,0.928378) -- (7.54772,0.892616) -- (7.64536,0.857113) -- (7.74327,0.82259) --
 (7.84106,0.790055) -- (7.93873,0.760488) -- (8.03678,0.734898) -- (8.13457,0.714179) -- (8.23233,0.698773) -- (8.33009,0.688641) -- (8.42796,0.683041) -- (8.52571,0.680635) -- (8.62355,0.679598) -- (8.72137,0.679598) -- (8.81921,0.679598) --
 (8.91702,0.679598);
\colorlet{c}{natscaty};
\draw [c,line width=0.9] (1.09174,4.62758) -- (1.18964,4.50078) -- (1.28743,4.37111) -- (1.38533,4.25007) -- (1.48283,4.12904) -- (1.58099,4.01377) -- (1.67857,3.8985) -- (1.77628,3.79188) -- (1.87411,3.68237) -- (1.97224,3.58151) --
 (2.06982,3.47978) -- (2.16774,3.38152) -- (2.26553,3.28757) -- (2.36339,3.19392) -- (2.46118,3.10602) -- (2.55898,3.01698) -- (2.65677,2.93398) -- (2.75466,2.85041) -- (2.85239,2.77174) -- (2.95019,2.69365) -- (3.04802,2.61872) -- (3.14572,2.54552)
 -- (3.2437,2.47435) -- (3.34166,2.40576) -- (3.43917,2.33804) -- (3.5374,2.27406) -- (3.6351,2.2098) -- (3.73264,2.14929) -- (3.83052,2.08877) -- (3.92828,2.03142) -- (4.02609,1.97436) -- (4.12398,1.92019) -- (4.22182,1.86659) -- (4.31966,1.81529)
 -- (4.41741,1.76515) -- (4.51524,1.71645) -- (4.61315,1.6689) -- (4.7109,1.62279) -- (4.8087,1.57841) -- (4.90658,1.5349) -- (5.00435,1.49282) -- (5.1024,1.45162) -- (5.2002,1.41242) -- (5.29791,1.37352) -- (5.39574,1.33635) -- (5.49373,1.29975) --
 (5.59128,1.26488) -- (5.68925,1.23059) -- (5.78686,1.19745) -- (5.8846,1.16575) -- (5.98266,1.13462) -- (6.08045,1.10465) -- (6.17828,1.07555) -- (6.27599,1.0476) -- (6.37379,1.02051) -- (6.47169,0.994284) -- (6.56942,0.968924) -- (6.66728,0.944545)
 -- (6.7651,0.920972) -- (6.86287,0.898293) -- (6.9607,0.876507) -- (7.05855,0.855499) -- (7.15634,0.835385) -- (7.25422,0.816135) -- (7.35211,0.797836) -- (7.44954,0.780516) -- (7.54772,0.764263) -- (7.64536,0.749192) -- (7.74327,0.735388) --
 (7.84106,0.722997) -- (7.93873,0.712133) -- (8.03678,0.70292) -- (8.13457,0.69539) -- (8.23233,0.689557) -- (8.33009,0.685332) -- (8.42796,0.682528) -- (8.52571,0.680878) -- (8.62355,0.679598) -- (8.72137,0.679598) -- (8.81921,0.679598) --
 (8.91702,0.679598);
\colorlet{c}{natscaty!70};
\draw [c,line width=0.9] (1.09174,4.01953) -- (1.18964,3.90715) -- (1.28743,3.79476) -- (1.38533,3.69102) -- (1.48283,3.58439) -- (1.58099,3.48468) -- (1.67857,3.38469) -- (1.77628,3.29074) -- (1.87411,3.19709) -- (1.97224,3.10746) --
 (2.06982,3.01986) -- (2.16774,2.93427) -- (2.26553,2.85243) -- (2.36339,2.77116) -- (2.46118,2.69451) -- (2.55898,2.61728) -- (2.65677,2.54495) -- (2.75466,2.47204) -- (2.85239,2.40345) -- (2.95019,2.33545) -- (3.04802,2.27003) -- (3.14572,2.20634)
 -- (3.2437,2.1441) -- (3.34166,2.08445) -- (3.43917,2.02508) -- (3.5374,1.96918) -- (3.6351,1.91298) -- (3.73264,1.85996) -- (3.83052,1.80665) -- (3.92828,1.75679) -- (4.02609,1.70665) -- (4.12398,1.6591) -- (4.22182,1.61184) -- (4.31966,1.5666) --
 (4.41741,1.52222) -- (4.51524,1.47928) -- (4.61315,1.4375) -- (4.7109,1.39657) -- (4.8087,1.35738) -- (4.90658,1.31877) -- (5.00435,1.28159) -- (5.1024,1.24528) -- (5.2002,1.21013) -- (5.29791,1.17583) -- (5.39574,1.14298) -- (5.49373,1.11071) --
 (5.59128,1.07987) -- (5.68925,1.0499) -- (5.78686,1.02108) -- (5.8846,0.993419) -- (5.98266,0.966677) -- (6.08045,0.941317) -- (6.17828,0.916794) -- (6.27599,0.893653) -- (6.37379,0.871608) -- (6.47169,0.850744) -- (6.56942,0.831206) --
 (6.66728,0.812849) -- (6.7651,0.795761) -- (6.86287,0.779998) -- (6.9607,0.765531) -- (7.05855,0.752333) -- (7.15634,0.740403) -- (7.25422,0.729769) -- (7.35211,0.720346) -- (7.44954,0.712133) -- (7.54772,0.705061) -- (7.64536,0.699084) --
 (7.74327,0.694119) -- (7.84106,0.690099) -- (7.93873,0.686917) -- (8.03678,0.684488) -- (8.13457,0.682699) -- (8.23233,0.681446) -- (8.33009,0.680619) -- (8.42796,0.679598) -- (8.52571,0.679598) -- (8.62355,0.679598) -- (8.72137,0.679598) --
 (8.81921,0.679598) -- (8.91702,0.679598);
\colorlet{c}{natscaty!40};
\draw [c,line width=0.9] (1.09174,2.36253) -- (1.18964,2.29683) -- (1.28743,2.2317) -- (1.38533,2.17061) -- (1.48283,2.10923) -- (1.58099,2.05217) -- (1.67857,1.99482) -- (1.77628,1.94122) -- (1.87411,1.88791) -- (1.97224,1.83719) --
 (2.06982,1.78791) -- (2.16774,1.74037) -- (2.26553,1.69483) -- (2.36339,1.64988) -- (2.46118,1.60752) -- (2.55898,1.56544) -- (2.65677,1.52596) -- (2.75466,1.48677) -- (2.85239,1.45018) -- (2.95019,1.41387) -- (3.04802,1.37928) -- (3.14572,1.34586)
 -- (3.2437,1.31329) -- (3.34166,1.28246) -- (3.43917,1.2522) -- (3.5374,1.22367) -- (3.6351,1.19543) -- (3.73264,1.16892) -- (3.83052,1.14269) -- (3.92828,1.1182) -- (4.02609,1.0937) -- (4.12398,1.07094) -- (4.22182,1.04875) -- (4.31966,1.02742) --
 (4.41741,1.00696) -- (4.51524,0.987079) -- (4.61315,0.968348) -- (4.7109,0.950107) -- (4.8087,0.932759) -- (4.90658,0.915929) -- (5.00435,0.900022) -- (5.1024,0.884576) -- (5.2002,0.869994) -- (5.29791,0.855845) -- (5.39574,0.842531) --
 (5.49373,0.829679) -- (5.59128,0.817518) -- (5.68925,0.805904) -- (5.78686,0.794896) -- (5.8846,0.784493) -- (5.98266,0.77458) -- (6.08045,0.765358) -- (6.17828,0.75654) -- (6.27599,0.748356) -- (6.37379,0.740662) -- (6.47169,0.733515) --
 (6.56942,0.726916) -- (6.66728,0.720778) -- (6.7651,0.715158) -- (6.86287,0.710058) -- (6.9607,0.705444) -- (7.05855,0.701283) -- (7.15634,0.697574) -- (7.25422,0.6943) -- (7.35211,0.691445) -- (7.44954,0.688984) -- (7.54772,0.686891) --
 (7.64536,0.685142) -- (7.74327,0.683701) -- (7.84106,0.682549) -- (7.93873,0.681645) -- (8.03678,0.68096) -- (8.13457,0.680459) -- (8.23233,0.679598) -- (8.33009,0.679598) -- (8.42796,0.679598) -- (8.52571,0.679598) -- (8.62355,0.679598) --
 (8.72137,0.679598) -- (8.81921,0.679598) -- (8.91702,0.679598);
\colorlet{c}{kugray};
\draw [c,line width=0.9] (6.70163,6.11638) -- (6.7651,5.69383);
\draw [c,line width=0.9] (6.7651,5.69383) -- (6.86287,5.08866) -- (6.9607,4.52672) -- (7.05855,4.00801) -- (7.15634,3.5282) -- (7.25422,3.0919) -- (7.35211,2.69739) -- (7.44954,2.3438) -- (7.54772,2.03056) -- (7.64536,1.7565) -- (7.74327,1.51962) --
 (7.84106,1.31906) -- (7.93873,1.1522) -- (8.03678,1.01705) -- (8.13457,0.910656) -- (8.23233,0.829852) -- (8.33009,0.771324) -- (8.42796,0.731181) -- (8.52571,0.705738) -- (8.62355,0.691105) -- (8.72137,0.683759) -- (8.81921,0.680714) --
 (8.91702,0.679598);
\definecolor{c}{rgb}{0,0,0};
\draw [c,line width=0.9, opacity=0] (7.22701,3.60632) -- (7.22701,6.04885) -- (9.97126,6.04885) -- (9.97126,3.60632) -- (7.22701,3.60632);
% \draw [c,line width=0.9] (7.22701,3.60632) -- (9.97126,3.60632);
% \draw [c,line width=0.9] (9.97126,3.60632) -- (9.97126,6.04885);
% \draw [c,line width=0.9] (9.97126,6.04885) -- (7.22701,6.04885);
% \draw [c,line width=0.9] (7.22701,6.04885) -- (7.22701,3.60632);
\draw [anchor=base west] (7.91307,5.8275) node[color=c, rotate=0]{u};
\colorlet{c}{natgreen};
\draw [c,line width=0.9] (7.32992,5.89619) -- (7.81017,5.89619);
\definecolor{c}{rgb}{0,0,0};
\draw [anchor=base west] (7.91307,5.52218) node[color=c, rotate=0]{d};
\colorlet{c}{natscatg};
\draw [c,line width=0.9] (7.32992,5.59088) -- (7.81017,5.59088);
\definecolor{c}{rgb}{0,0,0};
\draw [anchor=base west] (7.91307,5.21686) node[color=c, rotate=0]{$\bar{\text{u}}$};
\colorlet{c}{natgreen!60};
\draw [c,line width=0.9] (7.32992,5.28556) -- (7.81017,5.28556);
\definecolor{c}{rgb}{0,0,0};
\draw [anchor=base west] (7.91307,4.91155) node[color=c, rotate=0]{$\bar{\text{d}}$};
\colorlet{c}{natscatg!60};
\draw [c,line width=0.9] (7.32992,4.98024) -- (7.81017,4.98024);
\definecolor{c}{rgb}{0,0,0};
\draw [anchor=base west] (7.91307,4.60623) node[color=c, rotate=0]{s};
\colorlet{c}{natscaty};
\draw [c,line width=0.9] (7.32992,4.67493) -- (7.81017,4.67493);
\definecolor{c}{rgb}{0,0,0};
\draw [anchor=base west] (7.91307,4.30092) node[color=c, rotate=0]{c};
\colorlet{c}{natscaty!70};
\draw [c,line width=0.9] (7.32992,4.36961) -- (7.81017,4.36961);
\definecolor{c}{rgb}{0,0,0};
\draw [anchor=base west] (7.91307,3.9956) node[color=c, rotate=0]{b};
\colorlet{c}{natscaty!40};
\draw [c,line width=0.9] (7.32992,4.0643) -- (7.81017,4.0643);
\definecolor{c}{rgb}{0,0,0};
\draw [anchor=base west] (7.91307,3.69028) node[color=c, rotate=0]{gluon};
\colorlet{c}{kugray};
\draw [c,line width=0.9] (7.32992,3.75898) -- (7.81017,3.75898);
\definecolor{c}{rgb}{0,0,0};
\definecolor{c}{rgb}{0,0,0};
\draw [c,line width=0.9] (1,0.679598) -- (1,6.11638) -- (9,6.11638) -- (9,0.679598) -- (1,0.679598);
% \definecolor{c}{rgb}{1,1,1};
%\draw [color=c, fill=c] (1,0.679598) rectangle (9,6.11638);
\definecolor{c}{rgb}{0,0,0};
\draw [c,line width=0.9] (1,0.679598) -- (1,6.11638) -- (9,6.11638) -- (9,0.679598) -- (1,0.679598);
\draw [c,line width=0.9] (1,0.679598) -- (9,0.679598);
\draw [anchor= east] (9,0.299023) node[color=c, rotate=0]{$x$};
\draw [c,line width=0.9] (1,0.76115) -- (1,0.679598);
\draw [c,line width=0.9] (1.09174,0.842701) -- (1.09174,0.679598);
\draw [anchor=base] (1.09174,0.37208) node[color=c, rotate=0]{$10^{-4}$};
\draw [c,line width=0.9] (1.69527,0.76115) -- (1.69527,0.679598);
\draw [c,line width=0.9] (2.04832,0.76115) -- (2.04832,0.679598);
\draw [c,line width=0.9] (2.29881,0.76115) -- (2.29881,0.679598);
\draw [c,line width=0.9] (2.4931,0.76115) -- (2.4931,0.679598);
\draw [c,line width=0.9] (2.65185,0.76115) -- (2.65185,0.679598);
\draw [c,line width=0.9] (2.78607,0.76115) -- (2.78607,0.679598);
\draw [c,line width=0.9] (2.90234,0.76115) -- (2.90234,0.679598);
\draw [c,line width=0.9] (3.0049,0.76115) -- (3.0049,0.679598);
\draw [c,line width=0.9] (3.09663,0.842701) -- (3.09663,0.679598);
\draw [anchor=base] (3.09663,0.37208) node[color=c, rotate=0]{$10^{-3}$};
\draw [c,line width=0.9] (3.70017,0.76115) -- (3.70017,0.679598);
\draw [c,line width=0.9] (4.05321,0.76115) -- (4.05321,0.679598);
\draw [c,line width=0.9] (4.3037,0.76115) -- (4.3037,0.679598);
\draw [c,line width=0.9] (4.498,0.76115) -- (4.498,0.679598);
\draw [c,line width=0.9] (4.65675,0.76115) -- (4.65675,0.679598);
\draw [c,line width=0.9] (4.79097,0.76115) -- (4.79097,0.679598);
\draw [c,line width=0.9] (4.90723,0.76115) -- (4.90723,0.679598);
\draw [c,line width=0.9] (5.00979,0.76115) -- (5.00979,0.679598);
\draw [c,line width=0.9] (5.10153,0.842701) -- (5.10153,0.679598);
\draw [anchor=base] (5.10153,0.37208) node[color=c, rotate=0]{$10^{-2}$};
\draw [c,line width=0.9] (5.70506,0.76115) -- (5.70506,0.679598);
\draw [c,line width=0.9] (6.05811,0.76115) -- (6.05811,0.679598);
\draw [c,line width=0.9] (6.30859,0.76115) -- (6.30859,0.679598);
\draw [c,line width=0.9] (6.50289,0.76115) -- (6.50289,0.679598);
\draw [c,line width=0.9] (6.66164,0.76115) -- (6.66164,0.679598);
\draw [c,line width=0.9] (6.79586,0.76115) -- (6.79586,0.679598);
\draw [c,line width=0.9] (6.91213,0.76115) -- (6.91213,0.679598);
\draw [c,line width=0.9] (7.01468,0.76115) -- (7.01468,0.679598);
\draw [c,line width=0.9] (7.10642,0.842701) -- (7.10642,0.679598);
\draw [anchor=base] (7.10642,0.37208) node[color=c, rotate=0]{$10^{-1}$};
\draw [c,line width=0.9] (7.70996,0.76115) -- (7.70996,0.679598);
\draw [c,line width=0.9] (8.063,0.76115) -- (8.063,0.679598);
\draw [c,line width=0.9] (8.31349,0.76115) -- (8.31349,0.679598);
\draw [c,line width=0.9] (8.50778,0.76115) -- (8.50778,0.679598);
\draw [c,line width=0.9] (8.66653,0.76115) -- (8.66653,0.679598);
\draw [c,line width=0.9] (8.80075,0.76115) -- (8.80075,0.679598);
\draw [c,line width=0.9] (8.91702,0.76115) -- (8.91702,0.679598);
\draw [c,line width=0.9] (1,0.679598) -- (1,6.11638);
\draw [anchor= east] (0.44,6.11638) node[color=c, rotate=90]{$xf(x,Q^2)$};
\draw [c,line width=0.9] (1.24,0.679598) -- (1,0.679598);
\draw [c,line width=0.9] (1.12,0.823685) -- (1,0.823685);
\draw [c,line width=0.9] (1.12,0.967772) -- (1,0.967772);
\draw [c,line width=0.9] (1.12,1.11186) -- (1,1.11186);
\draw [c,line width=0.9] (1.24,1.25595) -- (1,1.25595);
\draw [c,line width=0.9] (1.12,1.40003) -- (1,1.40003);
\draw [c,line width=0.9] (1.12,1.54412) -- (1,1.54412);
\draw [c,line width=0.9] (1.12,1.68821) -- (1,1.68821);
\draw [c,line width=0.9] (1.24,1.83229) -- (1,1.83229);
\draw [c,line width=0.9] (1.12,1.97638) -- (1,1.97638);
\draw [c,line width=0.9] (1.12,2.12047) -- (1,2.12047);
\draw [c,line width=0.9] (1.12,2.26455) -- (1,2.26455);
\draw [c,line width=0.9] (1.24,2.40864) -- (1,2.40864);
\draw [c,line width=0.9] (1.12,2.55273) -- (1,2.55273);
\draw [c,line width=0.9] (1.12,2.69682) -- (1,2.69682);
\draw [c,line width=0.9] (1.12,2.8409) -- (1,2.8409);
\draw [c,line width=0.9] (1.24,2.98499) -- (1,2.98499);
\draw [c,line width=0.9] (1.12,3.12908) -- (1,3.12908);
\draw [c,line width=0.9] (1.12,3.27316) -- (1,3.27316);
\draw [c,line width=0.9] (1.12,3.41725) -- (1,3.41725);
\draw [c,line width=0.9] (1.24,3.56134) -- (1,3.56134);
\draw [c,line width=0.9] (1.12,3.70542) -- (1,3.70542);
\draw [c,line width=0.9] (1.12,3.84951) -- (1,3.84951);
\draw [c,line width=0.9] (1.12,3.9936) -- (1,3.9936);
\draw [c,line width=0.9] (1.24,4.13769) -- (1,4.13769);
\draw [c,line width=0.9] (1.12,4.28177) -- (1,4.28177);
\draw [c,line width=0.9] (1.12,4.42586) -- (1,4.42586);
\draw [c,line width=0.9] (1.12,4.56995) -- (1,4.56995);
\draw [c,line width=0.9] (1.24,4.71403) -- (1,4.71403);
\draw [c,line width=0.9] (1.12,4.85812) -- (1,4.85812);
\draw [c,line width=0.9] (1.12,5.00221) -- (1,5.00221);
\draw [c,line width=0.9] (1.12,5.14629) -- (1,5.14629);
\draw [c,line width=0.9] (1.24,5.29038) -- (1,5.29038);
\draw [c,line width=0.9] (1.12,5.43447) -- (1,5.43447);
\draw [c,line width=0.9] (1.12,5.57856) -- (1,5.57856);
\draw [c,line width=0.9] (1.12,5.72264) -- (1,5.72264);
\draw [c,line width=0.9] (1.24,5.86673) -- (1,5.86673);
\draw [c,line width=0.9] (1.24,5.86673) -- (1,5.86673);
\draw [c,line width=0.9] (1.12,6.01082) -- (1,6.01082);
\draw [anchor= east] (0.95,0.679598) node[color=c, rotate=0]{0};
\draw [anchor= east] (0.95,1.25595) node[color=c, rotate=0]{0.2};
\draw [anchor= east] (0.95,1.83229) node[color=c, rotate=0]{0.4};
\draw [anchor= east] (0.95,2.40864) node[color=c, rotate=0]{0.6};
\draw [anchor= east] (0.95,2.98499) node[color=c, rotate=0]{0.8};
\draw [anchor= east] (0.95,3.56134) node[color=c, rotate=0]{1};
\draw [anchor= east] (0.95,4.13769) node[color=c, rotate=0]{1.2};
\draw [anchor= east] (0.95,4.71403) node[color=c, rotate=0]{1.4};
\draw [anchor= east] (0.95,5.29038) node[color=c, rotate=0]{1.6};
\draw [anchor= east] (0.95,5.86673) node[color=c, rotate=0]{1.8};
% \draw [c,line width=0.9, opacity=0] (7.22701,3.60632) -- (7.22701,6.04885) -- (9.97126,6.04885) -- (9.97126,3.60632) -- (7.22701,3.60632);
% \draw [c,line width=0.9] (7.22701,3.60632) -- (9.97126,3.60632);
% \draw [c,line width=0.9] (9.97126,3.60632) -- (9.97126,6.04885);
% \draw [c,line width=0.9] (9.97126,6.04885) -- (7.22701,6.04885);
% \draw [c,line width=0.9] (7.22701,6.04885) -- (7.22701,3.60632);
% \draw [anchor=base west] (7.91307,5.8275) node[color=c, rotate=0]{u};
% \colorlet{c}{natgreen};
% \draw [c,line width=0.9] (7.32992,5.89619) -- (7.81017,5.89619);
% \definecolor{c}{rgb}{0,0,0};
% \draw [anchor=base west] (7.91307,5.52218) node[color=c, rotate=0]{d};
% \colorlet{c}{natscatg};
% \draw [c,line width=0.9] (7.32992,5.59088) -- (7.81017,5.59088);
% \definecolor{c}{rgb}{0,0,0};
% \draw [anchor=base west] (7.91307,5.21686) node[color=c, rotate=0]{$\bar u$};
% \colorlet{c}{natgreen!60};
% \draw [c,line width=0.9] (7.32992,5.28556) -- (7.81017,5.28556);
% \definecolor{c}{rgb}{0,0,0};
% \draw [anchor=base west] (7.91307,4.91155) node[color=c, rotate=0]{$\bar{d}$};
% \colorlet{c}{natscatg!60};
% \draw [c,line width=0.9] (7.32992,4.98024) -- (7.81017,4.98024);
% \definecolor{c}{rgb}{0,0,0};
% \draw [anchor=base west] (7.91307,4.60623) node[color=c, rotate=0]{s};
% \colorlet{c}{natgreen!36!natscaty!40!36};
% \draw [c,line width=0.9] (7.32992,4.67493) -- (7.81017,4.67493);
% \definecolor{c}{rgb}{0,0,0};
% \draw [anchor=base west] (7.91307,4.30092) node[color=c, rotate=0]{c};
% \colorlet{c}{natscaty!70};
% \draw [c,line width=0.9] (7.32992,4.36961) -- (7.81017,4.36961);
% \definecolor{c}{rgb}{0,0,0};
% \draw [anchor=base west] (7.91307,3.9956) node[color=c, rotate=0]{b};
% \colorlet{c}{natscaty!40};
% \draw [c,line width=0.9] (7.32992,4.0643) -- (7.81017,4.0643);
% \definecolor{c}{rgb}{0,0,0};
% \draw [anchor=base west] (7.91307,3.69028) node[color=c, rotate=0]{gluon};
% \colorlet{c}{kugray};
% \draw [c,line width=0.9] (7.32992,3.75898) -- (7.81017,3.75898);
% \definecolor{c}{rgb}{0,0,0};
% %\draw [c,line width=0.9, opacity=0] (7.22701,3.60632) -- (7.22701,6.04885) -- (9.97126,6.04885) -- (9.97126,3.60632) -- (7.22701,3.60632);
% \draw [c,line width=0.9] (7.22701,3.60632) -- (9.97126,3.60632);
% \draw [c,line width=0.9] (9.97126,3.60632) -- (9.97126,6.04885);
% \draw [c,line width=0.9] (9.97126,6.04885) -- (7.22701,6.04885);
% \draw [c,line width=0.9] (7.22701,6.04885) -- (7.22701,3.60632);
% \draw [anchor=base west] (7.91307,5.8275) node[color=c, rotate=0]{u};
% \colorlet{c}{natgreen};
% \draw [c,line width=0.9] (7.32992,5.89619) -- (7.81017,5.89619);
% \definecolor{c}{rgb}{0,0,0};
% \draw [anchor=base west] (7.91307,5.52218) node[color=c, rotate=0]{d};
% \colorlet{c}{natscatg};
% \draw [c,line width=0.9] (7.32992,5.59088) -- (7.81017,5.59088);
% \definecolor{c}{rgb}{0,0,0};
% \draw [anchor=base west] (7.91307,5.21686) node[color=c, rotate=0]{$\bar u$};
% \colorlet{c}{natgreen!60};
% \draw [c,line width=0.9] (7.32992,5.28556) -- (7.81017,5.28556);
% \definecolor{c}{rgb}{0,0,0};
% \draw [anchor=base west] (7.91307,4.91155) node[color=c, rotate=0]{$\bar{d}$};
% \colorlet{c}{natscatg!60};
% \draw [c,line width=0.9] (7.32992,4.98024) -- (7.81017,4.98024);
% \definecolor{c}{rgb}{0,0,0};
% \draw [anchor=base west] (7.91307,4.60623) node[color=c, rotate=0]{s};
% \colorlet{c}{natgreen!36!natscaty!40!36};
% \draw [c,line width=0.9] (7.32992,4.67493) -- (7.81017,4.67493);
% \definecolor{c}{rgb}{0,0,0};
% \draw [anchor=base west] (7.91307,4.30092) node[color=c, rotate=0]{c};
% \colorlet{c}{natscaty!70};
% \draw [c,line width=0.9] (7.32992,4.36961) -- (7.81017,4.36961);
% \definecolor{c}{rgb}{0,0,0};
% \draw [anchor=base west] (7.91307,3.9956) node[color=c, rotate=0]{b};
% \colorlet{c}{natscaty!40};
% \draw [c,line width=0.9] (7.32992,4.0643) -- (7.81017,4.0643);
% \definecolor{c}{rgb}{0,0,0};
% \draw [anchor=base west] (7.91307,3.69028) node[color=c, rotate=0]{gluon};
% \colorlet{c}{kugray};
% \draw [c,line width=0.9] (7.32992,3.75898) -- (7.81017,3.75898);
% \definecolor{c}{rgb}{0,0,0};
% \draw [c,line width=0.9, opacity=0] (7.22701,3.60632) -- (7.22701,6.04885) -- (9.97126,6.04885) -- (9.97126,3.60632) -- (7.22701,3.60632);
% \draw [c,line width=0.9] (7.22701,3.60632) -- (9.97126,3.60632);
% \draw [c,line width=0.9] (9.97126,3.60632) -- (9.97126,6.04885);
% \draw [c,line width=0.9] (9.97126,6.04885) -- (7.22701,6.04885);
% \draw [c,line width=0.9] (7.22701,6.04885) -- (7.22701,3.60632);
% \draw [anchor=base west] (7.91307,5.8275) node[color=c, rotate=0]{u};
% \colorlet{c}{natgreen};
% \draw [c,line width=0.9] (7.32992,5.89619) -- (7.81017,5.89619);
% \definecolor{c}{rgb}{0,0,0};
% \draw [anchor=base west] (7.91307,5.52218) node[color=c, rotate=0]{d};
% \colorlet{c}{natscatg};
% \draw [c,line width=0.9] (7.32992,5.59088) -- (7.81017,5.59088);
% \definecolor{c}{rgb}{0,0,0};
% \draw [anchor=base west] (7.91307,5.21686) node[color=c, rotate=0]{$\bar u$};
% \colorlet{c}{natgreen!60};
% \draw [c,line width=0.9] (7.32992,5.28556) -- (7.81017,5.28556);
% \definecolor{c}{rgb}{0,0,0};
% \draw [anchor=base west] (7.91307,4.91155) node[color=c, rotate=0]{$\bar{d}$};
% \colorlet{c}{natscatg!60};
% \draw [c,line width=0.9] (7.32992,4.98024) -- (7.81017,4.98024);
% \definecolor{c}{rgb}{0,0,0};
% \draw [anchor=base west] (7.91307,4.60623) node[color=c, rotate=0]{s};
% \colorlet{c}{natgreen!36!natscaty!40!36};
% \draw [c,line width=0.9] (7.32992,4.67493) -- (7.81017,4.67493);
% \definecolor{c}{rgb}{0,0,0};
% \draw [anchor=base west] (7.91307,4.30092) node[color=c, rotate=0]{c};
% \colorlet{c}{natscaty!70};
% \draw [c,line width=0.9] (7.32992,4.36961) -- (7.81017,4.36961);
% \definecolor{c}{rgb}{0,0,0};
% \draw [anchor=base west] (7.91307,3.9956) node[color=c, rotate=0]{b};
% \colorlet{c}{natscaty!40};
% \draw [c,line width=0.9] (7.32992,4.0643) -- (7.81017,4.0643);
% \definecolor{c}{rgb}{0,0,0};
% \draw [anchor=base west] (7.91307,3.69028) node[color=c, rotate=0]{gluon};
% \colorlet{c}{kugray};
% \draw [c,line width=0.9] (7.32992,3.75898) -- (7.81017,3.75898);
\end{tikzpicture}

\end{infilsf}
\end{minipage}
\hfill\begin{minipage}[b]{.3\textwidth}
\caption{Parton distribution function obtained from the \textsc{cteq} collaboration. It expresses the probability of extracting a specific quark from a proton with a certain fraction, $x$, of its energy as a function of $x$ and $Q^2$. This illustration has $Q^2 = 100$ GeV$^2$. Adapted from \cite{durpdf}.\label{pdff}}
\end{minipage}
\end{figure}

While protons contain no antiquarks as valence quarks, every proton is surrounded by a `sea' of virtual particles. At a given energy scale $Q^2$, there is a certain probability for extracting one of these sea quarks in an interaction, given by the Parton Distribution Functions (PDFs), a selection of which is shown in fig.~\ref{pdff}. These functions are found experimentally by several collaborations. This thesis will deal mainly with the CTEQ set of PDFs.

Including this step in the calculation leads to the following expression, which gives the cross section for a diphoton event resulting from a proton-proton collision, $\sigma(pp\rightarrow\gamma\gamma)$ in terms of the $\sigma(q\bar q \rightarrow \gamma\gamma)$ cross section that we have determined thus far:
\[\sigma(pp\rightarrow\gamma\gamma)=\sum_q\iint dx_1\,dx_2\,f_q(x_1,Q^2)f_{\bar q}(x_2,Q^2)\sigma(q\bar q\rightarrow\gamma\gamma),\label{pdf}\]
where $f_q$ is the PDF for parton $q$.




\chapter{Simulation studies}\label{ch.mc}

Now that the theoretical description of the processes under study is in place, we might suppose that obtaining a prediction of the distributions of events that an experiment will find is a simple matter of carrying out the integrals laid out in the previous chapter. Unfortunately, these integrals defy analytical solution, which leaves only the option of integrating numerically. Specifically, we will use the Monte Carlo method for numerical integration, which prescribes inserting a random value drawn from a suitable distribution into the integrand in place of the variable of integration, and then calculating the value of the integrand. Each result estimates the value of the integral, and by averaging several such estimates, an incrementally better estimate is obtained.

Since all of the variables that are integrated over have physical significance, choosing specific values for the variables of integration can be viewed as equivalent to laying out the kinematics of a single hypothetical event. Repeatedly estimating the value of the integral a sufficient number of times to obtain a result of acceptable accuracy from the Monte Carlo integration process in effect leaves us with a large set of such simulated events. In particle physics, this process goes by Monte Carlo simulation, and the software packages that are designed to carry out these calculations are called event generators.

\section{Event generators}

The event generator used for the bulk of the work in this thesis is CalcHEP \cite{calchep}.

Given a set of initial and final states and a list of Feynman rules, CalcHEP can construct all tree level diagrams that contribute to the specified process. By specifying a composite particle, such as a proton, CalcHEP will construct Feynman diagrams with initial states composed of any combination of particles that can be extracted from the given composite, weighted by a selected PDF. For generating our event samples, we specify protons with 4 TeV momentum in the initial state and two photons in the final state. Based on the collection of Feynman diagrams thus constructed, CalcHEP can then generate events according to those Feynman diagrams, in the correct proportions.

Other choices of event generators include MadGraph \cite{madgraph5} and pythia \cite{pythia}. To illustrate the behaviour of these different generators with respect to one another, figure~\ref{evgen} plots distributions of invariant mass ($M_{\gamma\gamma}$) and transverse momentum ($p_T$) of a set of simulated events generated by each one. To accurately compare the three, all are limited to tree level processes, to conform to the limitations of CalcHEP.

\begin{figure}[htp]
\begin{minipage}[b]{.69\textwidth}
\begin{infilsf} \tiny
\hspace{-1ex}\makebox[0pt][l]{\pgfdeclareplotmark{cross} {
\pgfpathmoveto{\pgfpoint{-0.3\pgfplotmarksize}{\pgfplotmarksize}}
\pgfpathlineto{\pgfpoint{+0.3\pgfplotmarksize}{\pgfplotmarksize}}
\pgfpathlineto{\pgfpoint{+0.3\pgfplotmarksize}{0.3\pgfplotmarksize}}
\pgfpathlineto{\pgfpoint{+1\pgfplotmarksize}{0.3\pgfplotmarksize}}
\pgfpathlineto{\pgfpoint{+1\pgfplotmarksize}{-0.3\pgfplotmarksize}}
\pgfpathlineto{\pgfpoint{+0.3\pgfplotmarksize}{-0.3\pgfplotmarksize}}
\pgfpathlineto{\pgfpoint{+0.3\pgfplotmarksize}{-1.\pgfplotmarksize}}
\pgfpathlineto{\pgfpoint{-0.3\pgfplotmarksize}{-1.\pgfplotmarksize}}
\pgfpathlineto{\pgfpoint{-0.3\pgfplotmarksize}{-0.3\pgfplotmarksize}}
\pgfpathlineto{\pgfpoint{-1.\pgfplotmarksize}{-0.3\pgfplotmarksize}}
\pgfpathlineto{\pgfpoint{-1.\pgfplotmarksize}{0.3\pgfplotmarksize}}
\pgfpathlineto{\pgfpoint{-0.3\pgfplotmarksize}{0.3\pgfplotmarksize}}
\pgfpathclose
\pgfusepathqstroke
}
\pgfdeclareplotmark{cross*} {
\pgfpathmoveto{\pgfpoint{-0.3\pgfplotmarksize}{\pgfplotmarksize}}
\pgfpathlineto{\pgfpoint{+0.3\pgfplotmarksize}{\pgfplotmarksize}}
\pgfpathlineto{\pgfpoint{+0.3\pgfplotmarksize}{0.3\pgfplotmarksize}}
\pgfpathlineto{\pgfpoint{+1\pgfplotmarksize}{0.3\pgfplotmarksize}}
\pgfpathlineto{\pgfpoint{+1\pgfplotmarksize}{-0.3\pgfplotmarksize}}
\pgfpathlineto{\pgfpoint{+0.3\pgfplotmarksize}{-0.3\pgfplotmarksize}}
\pgfpathlineto{\pgfpoint{+0.3\pgfplotmarksize}{-1.\pgfplotmarksize}}
\pgfpathlineto{\pgfpoint{-0.3\pgfplotmarksize}{-1.\pgfplotmarksize}}
\pgfpathlineto{\pgfpoint{-0.3\pgfplotmarksize}{-0.3\pgfplotmarksize}}
\pgfpathlineto{\pgfpoint{-1.\pgfplotmarksize}{-0.3\pgfplotmarksize}}
\pgfpathlineto{\pgfpoint{-1.\pgfplotmarksize}{0.3\pgfplotmarksize}}
\pgfpathlineto{\pgfpoint{-0.3\pgfplotmarksize}{0.3\pgfplotmarksize}}
\pgfpathclose
\pgfusepathqfillstroke
}
\pgfdeclareplotmark{newstar} {
\pgfpathmoveto{\pgfqpoint{0pt}{\pgfplotmarksize}}
\pgfpathlineto{\pgfqpointpolar{44}{0.5\pgfplotmarksize}}
\pgfpathlineto{\pgfqpointpolar{18}{\pgfplotmarksize}}
\pgfpathlineto{\pgfqpointpolar{-20}{0.5\pgfplotmarksize}}
\pgfpathlineto{\pgfqpointpolar{-54}{\pgfplotmarksize}}
\pgfpathlineto{\pgfqpointpolar{-90}{0.5\pgfplotmarksize}}
\pgfpathlineto{\pgfqpointpolar{234}{\pgfplotmarksize}}
\pgfpathlineto{\pgfqpointpolar{198}{0.5\pgfplotmarksize}}
\pgfpathlineto{\pgfqpointpolar{162}{\pgfplotmarksize}}
\pgfpathlineto{\pgfqpointpolar{134}{0.5\pgfplotmarksize}}
\pgfpathclose
\pgfusepathqstroke
}
\pgfdeclareplotmark{newstar*} {
\pgfpathmoveto{\pgfqpoint{0pt}{\pgfplotmarksize}}
\pgfpathlineto{\pgfqpointpolar{44}{0.5\pgfplotmarksize}}
\pgfpathlineto{\pgfqpointpolar{18}{\pgfplotmarksize}}
\pgfpathlineto{\pgfqpointpolar{-20}{0.5\pgfplotmarksize}}
\pgfpathlineto{\pgfqpointpolar{-54}{\pgfplotmarksize}}
\pgfpathlineto{\pgfqpointpolar{-90}{0.5\pgfplotmarksize}}
\pgfpathlineto{\pgfqpointpolar{234}{\pgfplotmarksize}}
\pgfpathlineto{\pgfqpointpolar{198}{0.5\pgfplotmarksize}}
\pgfpathlineto{\pgfqpointpolar{162}{\pgfplotmarksize}}
\pgfpathlineto{\pgfqpointpolar{134}{0.5\pgfplotmarksize}}
\pgfpathclose
\pgfusepathqfillstroke
}
\begin{tikzpicture}[x=.045\textwidth,y=.045\textwidth]
\definecolor{c}{rgb}{1,1,1};
\draw [color=c, fill=c] (0,0) rectangle (20,13.5632);
\draw [color=c, fill=c] (0,4.06897) rectangle (20,13.5632);
\draw [color=c, fill=c] (2,4.06897) rectangle (20,13.5632);
\definecolor{c}{rgb}{0,0,0};
\draw [c] (2,4.06897) -- (2,13.5632) -- (20,13.5632) -- (20,4.06897) -- (2,4.06897);
\definecolor{c}{rgb}{1,1,1};
\draw [color=c, fill=c] (2,4.06897) rectangle (20,13.5632);
\definecolor{c}{rgb}{0,0,0};
\draw [c] (2,4.06897) -- (2,13.5632) -- (20,13.5632) -- (20,4.06897) -- (2,4.06897);
\colorlet{c}{kugray};
\draw [c] (2.18,13.2837) -- (2.18,13.2879);
\draw [c] (2.18,13.2879) -- (2.18,13.2921);
\draw [c] (2,13.2879) -- (2.18,13.2879);
\draw [c] (2.18,13.2879) -- (2.36,13.2879);
\draw [c] (2.54,11.6533) -- (2.54,11.6579);
\draw [c] (2.54,11.6579) -- (2.54,11.6625);
\draw [c] (2.36,11.6579) -- (2.54,11.6579);
\draw [c] (2.54,11.6579) -- (2.72,11.6579);
\draw [c] (2.9,10.6843) -- (2.9,10.6987);
\draw [c] (2.9,10.6987) -- (2.9,10.7127);
\draw [c] (2.72,10.6987) -- (2.9,10.6987);
\draw [c] (2.9,10.6987) -- (3.08,10.6987);
\draw [c] (3.26,10.0129) -- (3.26,10.0444);
\draw [c] (3.26,10.0444) -- (3.26,10.0737);
\draw [c] (3.08,10.0444) -- (3.26,10.0444);
\draw [c] (3.26,10.0444) -- (3.44,10.0444);
\draw [c] (3.62,9.42595) -- (3.62,9.48578);
\draw [c] (3.62,9.48578) -- (3.62,9.5382);
\draw [c] (3.44,9.48578) -- (3.62,9.48578);
\draw [c] (3.62,9.48578) -- (3.8,9.48578);
\draw [c] (3.98,8.98638) -- (3.98,9.07402);
\draw [c] (3.98,9.07402) -- (3.98,9.14661);
\draw [c] (3.8,9.07402) -- (3.98,9.07402);
\draw [c] (3.98,9.07402) -- (4.16,9.07402);
\draw [c] (4.34,8.63304) -- (4.34,8.69108);
\draw [c] (4.34,8.69108) -- (4.34,8.74213);
\draw [c] (4.16,8.69108) -- (4.34,8.69108);
\draw [c] (4.34,8.69108) -- (4.52,8.69108);
\draw [c] (4.7,8.29897) -- (4.7,8.30433);
\draw [c] (4.7,8.30433) -- (4.7,8.30962);
\draw [c] (4.52,8.30433) -- (4.7,8.30433);
\draw [c] (4.7,8.30433) -- (4.88,8.30433);
\draw [c] (5.06,7.93898) -- (5.06,7.94717);
\draw [c] (5.06,7.94717) -- (5.06,7.9552);
\draw [c] (4.88,7.94717) -- (5.06,7.94717);
\draw [c] (5.06,7.94717) -- (5.24,7.94717);
\draw [c] (5.42,7.61222) -- (5.42,7.62426);
\draw [c] (5.42,7.62426) -- (5.42,7.63596);
\draw [c] (5.24,7.62426) -- (5.42,7.62426);
\draw [c] (5.42,7.62426) -- (5.6,7.62426);
\draw [c] (5.78,7.30901) -- (5.78,7.32622);
\draw [c] (5.78,7.32622) -- (5.78,7.34275);
\draw [c] (5.6,7.32622) -- (5.78,7.32622);
\draw [c] (5.78,7.32622) -- (5.96,7.32622);
\draw [c] (6.14,6.96941) -- (6.14,6.99508);
\draw [c] (6.14,6.99508) -- (6.14,7.01929);
\draw [c] (5.96,6.99508) -- (6.14,6.99508);
\draw [c] (6.14,6.99508) -- (6.32,6.99508);
\draw [c] (6.5,6.65186) -- (6.5,6.68918);
\draw [c] (6.5,6.68918) -- (6.5,6.72348);
\draw [c] (6.32,6.68918) -- (6.5,6.68918);
\draw [c] (6.5,6.68918) -- (6.68,6.68918);
\draw [c] (6.86,6.31826) -- (6.86,6.37354);
\draw [c] (6.86,6.37354) -- (6.86,6.42243);
\draw [c] (6.68,6.37354) -- (6.86,6.37354);
\draw [c] (6.86,6.37354) -- (7.04,6.37354);
\draw [c] (7.22,6.04546) -- (7.22,6.12165);
\draw [c] (7.22,6.12165) -- (7.22,6.18622);
\draw [c] (7.04,6.12165) -- (7.22,6.12165);
\draw [c] (7.22,6.12165) -- (7.4,6.12165);
\draw [c] (7.58,5.86069) -- (7.58,5.95535);
\draw [c] (7.58,5.95535) -- (7.58,6.03269);
\draw [c] (7.4,5.95535) -- (7.58,5.95535);
\draw [c] (7.58,5.95535) -- (7.76,5.95535);
\draw [c] (7.94,5.57747) -- (7.94,5.70938);
\draw [c] (7.94,5.70938) -- (7.94,5.80986);
\draw [c] (7.76,5.70938) -- (7.94,5.70938);
\draw [c] (7.94,5.70938) -- (8.12,5.70938);
\draw [c] (8.3,4.88391) -- (8.3,5.17795);
\draw [c] (8.3,5.17795) -- (8.3,5.34995);
\draw [c] (8.12,5.17795) -- (8.3,5.17795);
\draw [c] (8.3,5.17795) -- (8.48,5.17795);
\draw [c] (8.66,4.363) -- (8.66,4.88391);
\draw [c] (8.66,4.88391) -- (8.66,5.11078);
\draw [c] (8.48,4.88391) -- (8.66,4.88391);
\draw [c] (8.66,4.88391) -- (8.84,4.88391);
\draw [c] (9.02,4.06897) -- (9.02,4.58987);
\draw [c] (9.02,4.58987) -- (9.02,4.88391);
\draw [c] (8.84,4.58987) -- (9.02,4.58987);
\draw [c] (9.02,4.58987) -- (9.2,4.58987);
\draw [c] (9.38,4.06897) -- (9.38,4.58987);
\draw [c] (9.38,4.58987) -- (9.38,4.88391);
\draw [c] (9.2,4.58987) -- (9.38,4.58987);
\draw [c] (9.38,4.58987) -- (9.56,4.58987);
\draw [c] (9.74,4.06897) -- (9.74,4.58987);
\draw [c] (9.74,4.58987) -- (9.74,4.88391);
\draw [c] (9.56,4.58987) -- (9.74,4.58987);
\draw [c] (9.74,4.58987) -- (9.92,4.58987);
\draw [c] (10.1,4.06897) -- (10.1,4.58987);
\draw [c] (10.1,4.58987) -- (10.1,4.88391);
\draw [c] (9.92,4.58987) -- (10.1,4.58987);
\draw [c] (10.1,4.58987) -- (10.28,4.58987);
\draw [c] (10.82,4.363) -- (10.82,4.88391);
\draw [c] (10.82,4.88391) -- (10.82,5.11078);
\draw [c] (10.64,4.88391) -- (10.82,4.88391);
\draw [c] (10.82,4.88391) -- (11,4.88391);
\definecolor{c}{rgb}{0,0,0};
\draw [c] (2,4.06897) -- (20,4.06897);
\draw [c] (2,4.32531) -- (2,4.06897);
\draw [c] (2.45,4.19714) -- (2.45,4.06897);
\draw [c] (2.9,4.19714) -- (2.9,4.06897);
\draw [c] (3.35,4.19714) -- (3.35,4.06897);
\draw [c] (3.8,4.19714) -- (3.8,4.06897);
\draw [c] (4.25,4.32531) -- (4.25,4.06897);
\draw [c] (4.7,4.19714) -- (4.7,4.06897);
\draw [c] (5.15,4.19714) -- (5.15,4.06897);
\draw [c] (5.6,4.19714) -- (5.6,4.06897);
\draw [c] (6.05,4.19714) -- (6.05,4.06897);
\draw [c] (6.5,4.32531) -- (6.5,4.06897);
\draw [c] (6.95,4.19714) -- (6.95,4.06897);
\draw [c] (7.4,4.19714) -- (7.4,4.06897);
\draw [c] (7.85,4.19714) -- (7.85,4.06897);
\draw [c] (8.3,4.19714) -- (8.3,4.06897);
\draw [c] (8.75,4.32531) -- (8.75,4.06897);
\draw [c] (9.2,4.19714) -- (9.2,4.06897);
\draw [c] (9.65,4.19714) -- (9.65,4.06897);
\draw [c] (10.1,4.19714) -- (10.1,4.06897);
\draw [c] (10.55,4.19714) -- (10.55,4.06897);
\draw [c] (11,4.32531) -- (11,4.06897);
\draw [c] (11.45,4.19714) -- (11.45,4.06897);
\draw [c] (11.9,4.19714) -- (11.9,4.06897);
\draw [c] (12.35,4.19714) -- (12.35,4.06897);
\draw [c] (12.8,4.19714) -- (12.8,4.06897);
\draw [c] (13.25,4.32531) -- (13.25,4.06897);
\draw [c] (13.7,4.19714) -- (13.7,4.06897);
\draw [c] (14.15,4.19714) -- (14.15,4.06897);
\draw [c] (14.6,4.19714) -- (14.6,4.06897);
\draw [c] (15.05,4.19714) -- (15.05,4.06897);
\draw [c] (15.5,4.32531) -- (15.5,4.06897);
\draw [c] (15.95,4.19714) -- (15.95,4.06897);
\draw [c] (16.4,4.19714) -- (16.4,4.06897);
\draw [c] (16.85,4.19714) -- (16.85,4.06897);
\draw [c] (17.3,4.19714) -- (17.3,4.06897);
\draw [c] (17.75,4.32531) -- (17.75,4.06897);
\draw [c] (18.2,4.19714) -- (18.2,4.06897);
\draw [c] (18.65,4.19714) -- (18.65,4.06897);
\draw [c] (19.1,4.19714) -- (19.1,4.06897);
\draw [c] (19.55,4.19714) -- (19.55,4.06897);
\draw [c] (20,4.32531) -- (20,4.06897);
\draw [c] (2,4.32531) -- (2,4.06897);
\draw [c] (20,4.32531) -- (20,4.06897);

\draw [c] (2,4.06897) -- (2,13.5632);
\draw [anchor= east] (0.1,13.5632) node[ rotate=90]{$\di\sigma/\di p_T^{\gamma_1} \text{ [pb/GeV]}$};
\draw [c] (2.3,4.11929) -- (2,4.11929);
\draw [c] (2.3,4.16926) -- (2,4.16926);
\draw [c] (2.6,4.21395) -- (2,4.21395);
\draw [anchor= east] (1.844,4.21395) node[ rotate=0]{$10^{-10}$};
\draw [c] (2.3,4.50799) -- (2,4.50799);
\draw [c] (2.3,4.67999) -- (2,4.67999);
\draw [c] (2.3,4.80203) -- (2,4.80203);
\draw [c] (2.3,4.89669) -- (2,4.89669);
\draw [c] (2.3,4.97403) -- (2,4.97403);
\draw [c] (2.3,5.03942) -- (2,5.03942);
\draw [c] (2.3,5.09607) -- (2,5.09607);
\draw [c] (2.3,5.14603) -- (2,5.14603);
\draw [c] (2.6,5.19073) -- (2,5.19073);
\draw [anchor= east] (1.844,5.19073) node[ rotate=0]{$10^{-9}$};
\draw [c] (2.3,5.48477) -- (2,5.48477);
\draw [c] (2.3,5.65677) -- (2,5.65677);
\draw [c] (2.3,5.77881) -- (2,5.77881);
\draw [c] (2.3,5.87347) -- (2,5.87347);
\draw [c] (2.3,5.95081) -- (2,5.95081);
\draw [c] (2.3,6.0162) -- (2,6.0162);
\draw [c] (2.3,6.07284) -- (2,6.07284);
\draw [c] (2.3,6.12281) -- (2,6.12281);
\draw [c] (2.6,6.1675) -- (2,6.1675);
\draw [anchor= east] (1.844,6.1675) node[ rotate=0]{$10^{-8}$};
\draw [c] (2.3,6.46154) -- (2,6.46154);
\draw [c] (2.3,6.63354) -- (2,6.63354);
\draw [c] (2.3,6.75558) -- (2,6.75558);
\draw [c] (2.3,6.85024) -- (2,6.85024);
\draw [c] (2.3,6.92758) -- (2,6.92758);
\draw [c] (2.3,6.99298) -- (2,6.99298);
\draw [c] (2.3,7.04962) -- (2,7.04962);
\draw [c] (2.3,7.09959) -- (2,7.09959);
\draw [c] (2.6,7.14428) -- (2,7.14428);
\draw [anchor= east] (1.844,7.14428) node[ rotate=0]{$10^{-7}$};
\draw [c] (2.3,7.43832) -- (2,7.43832);
\draw [c] (2.3,7.61032) -- (2,7.61032);
\draw [c] (2.3,7.73236) -- (2,7.73236);
\draw [c] (2.3,7.82702) -- (2,7.82702);
\draw [c] (2.3,7.90436) -- (2,7.90436);
\draw [c] (2.3,7.96975) -- (2,7.96975);
\draw [c] (2.3,8.0264) -- (2,8.0264);
\draw [c] (2.3,8.07636) -- (2,8.07636);
\draw [c] (2.6,8.12106) -- (2,8.12106);
\draw [anchor= east] (1.844,8.12106) node[ rotate=0]{$10^{-6}$};
\draw [c] (2.3,8.4151) -- (2,8.4151);
\draw [c] (2.3,8.5871) -- (2,8.5871);
\draw [c] (2.3,8.70914) -- (2,8.70914);
\draw [c] (2.3,8.80379) -- (2,8.80379);
\draw [c] (2.3,8.88114) -- (2,8.88114);
\draw [c] (2.3,8.94653) -- (2,8.94653);
\draw [c] (2.3,9.00317) -- (2,9.00317);
\draw [c] (2.3,9.05314) -- (2,9.05314);
\draw [c] (2.6,9.09783) -- (2,9.09783);
\draw [anchor= east] (1.844,9.09783) node[ rotate=0]{$10^{-5}$};
\draw [c] (2.3,9.39187) -- (2,9.39187);
\draw [c] (2.3,9.56387) -- (2,9.56387);
\draw [c] (2.3,9.68591) -- (2,9.68591);
\draw [c] (2.3,9.78057) -- (2,9.78057);
\draw [c] (2.3,9.85791) -- (2,9.85791);
\draw [c] (2.3,9.9233) -- (2,9.9233);
\draw [c] (2.3,9.97995) -- (2,9.97995);
\draw [c] (2.3,10.0299) -- (2,10.0299);
\draw [c] (2.6,10.0746) -- (2,10.0746);
\draw [anchor= east] (1.844,10.0746) node[ rotate=0]{$10^{-4}$};
\draw [c] (2.3,10.3686) -- (2,10.3686);
\draw [c] (2.3,10.5407) -- (2,10.5407);
\draw [c] (2.3,10.6627) -- (2,10.6627);
\draw [c] (2.3,10.7573) -- (2,10.7573);
\draw [c] (2.3,10.8347) -- (2,10.8347);
\draw [c] (2.3,10.9001) -- (2,10.9001);
\draw [c] (2.3,10.9567) -- (2,10.9567);
\draw [c] (2.3,11.0067) -- (2,11.0067);
\draw [c] (2.6,11.0514) -- (2,11.0514);
\draw [anchor= east] (1.844,11.0514) node[ rotate=0]{$10^{-3}$};
\draw [c] (2.3,11.3454) -- (2,11.3454);
\draw [c] (2.3,11.5174) -- (2,11.5174);
\draw [c] (2.3,11.6395) -- (2,11.6395);
\draw [c] (2.3,11.7341) -- (2,11.7341);
\draw [c] (2.3,11.8115) -- (2,11.8115);
\draw [c] (2.3,11.8769) -- (2,11.8769);
\draw [c] (2.3,11.9335) -- (2,11.9335);
\draw [c] (2.3,11.9835) -- (2,11.9835);
\draw [c] (2.6,12.0282) -- (2,12.0282);
\draw [anchor= east] (1.844,12.0282) node[ rotate=0]{$10^{-2}$};
\draw [c] (2.3,12.3222) -- (2,12.3222);
\draw [c] (2.3,12.4942) -- (2,12.4942);
\draw [c] (2.3,12.6162) -- (2,12.6162);
\draw [c] (2.3,12.7109) -- (2,12.7109);
\draw [c] (2.3,12.7882) -- (2,12.7882);
\draw [c] (2.3,12.8536) -- (2,12.8536);
\draw [c] (2.3,12.9103) -- (2,12.9103);
\draw [c] (2.3,12.9602) -- (2,12.9602);
\draw [c] (2.6,13.0049) -- (2,13.0049);
\draw [anchor= east] (1.844,13.0049) node[ rotate=0]{$10^{-1}$};
\draw [c] (2.3,13.299) -- (2,13.299);
\draw [c] (2.3,13.471) -- (2,13.471);
\colorlet{c}{natgreen};
\draw [c] (2.18,13.2731) -- (2.18,13.2747);
\draw [c] (2.18,13.2747) -- (2.18,13.2763);
\draw [c] (2,13.2747) -- (2.18,13.2747);
\draw [c] (2.18,13.2747) -- (2.36,13.2747);
\draw [c] (2.54,11.6659) -- (2.54,11.6765);
\draw [c] (2.54,11.6765) -- (2.54,11.6869);
\draw [c] (2.36,11.6765) -- (2.54,11.6765);
\draw [c] (2.54,11.6765) -- (2.72,11.6765);
\draw [c] (2.9,10.6691) -- (2.9,10.6978);
\draw [c] (2.9,10.6978) -- (2.9,10.7246);
\draw [c] (2.72,10.6978) -- (2.9,10.6978);
\draw [c] (2.9,10.6978) -- (3.08,10.6978);
\draw [c] (3.26,10.0576) -- (3.26,10.0601);
\draw [c] (3.26,10.0601) -- (3.26,10.0626);
\draw [c] (3.08,10.0601) -- (3.26,10.0601);
\draw [c] (3.26,10.0601) -- (3.44,10.0601);
\draw [c] (3.62,9.53583) -- (3.62,9.54042);
\draw [c] (3.62,9.54042) -- (3.62,9.54496);
\draw [c] (3.44,9.54042) -- (3.62,9.54042);
\draw [c] (3.62,9.54042) -- (3.8,9.54042);
\draw [c] (3.98,9.09436) -- (3.98,9.10208);
\draw [c] (3.98,9.10208) -- (3.98,9.10966);
\draw [c] (3.8,9.10208) -- (3.98,9.10208);
\draw [c] (3.98,9.10208) -- (4.16,9.10208);
\draw [c] (4.34,8.69455) -- (4.34,8.70691);
\draw [c] (4.34,8.70691) -- (4.34,8.71893);
\draw [c] (4.16,8.70691) -- (4.34,8.70691);
\draw [c] (4.34,8.70691) -- (4.52,8.70691);
\draw [c] (4.7,8.29362) -- (4.7,8.31346);
\draw [c] (4.7,8.31346) -- (4.7,8.33241);
\draw [c] (4.52,8.31346) -- (4.7,8.31346);
\draw [c] (4.7,8.31346) -- (4.88,8.31346);
\draw [c] (5.06,7.93123) -- (5.06,7.96164);
\draw [c] (5.06,7.96164) -- (5.06,7.99001);
\draw [c] (4.88,7.96164) -- (5.06,7.96164);
\draw [c] (5.06,7.96164) -- (5.24,7.96164);
\draw [c] (5.42,7.62606) -- (5.42,7.66962);
\draw [c] (5.42,7.66962) -- (5.42,7.70912);
\draw [c] (5.24,7.66962) -- (5.42,7.66962);
\draw [c] (5.42,7.66962) -- (5.6,7.66962);
\draw [c] (5.78,7.35019) -- (5.78,7.41047);
\draw [c] (5.78,7.41047) -- (5.78,7.46323);
\draw [c] (5.6,7.41047) -- (5.78,7.41047);
\draw [c] (5.78,7.41047) -- (5.96,7.41047);
\draw [c] (6.14,7.05267) -- (6.14,7.13819);
\draw [c] (6.14,7.13819) -- (6.14,7.20933);
\draw [c] (5.96,7.13819) -- (6.14,7.13819);
\draw [c] (6.14,7.13819) -- (6.32,7.13819);
\draw [c] (6.5,6.65721) -- (6.5,6.75996);
\draw [c] (6.5,6.75996) -- (6.5,6.84262);
\draw [c] (6.32,6.75996) -- (6.5,6.75996);
\draw [c] (6.5,6.75996) -- (6.68,6.75996);
\draw [c] (6.86,6.49404) -- (6.86,6.49634);
\draw [c] (6.86,6.49634) -- (6.86,6.49863);
\draw [c] (6.68,6.49634) -- (6.86,6.49634);
\draw [c] (6.86,6.49634) -- (7.04,6.49634);
\draw [c] (7.22,6.2213) -- (7.22,6.22448);
\draw [c] (7.22,6.22448) -- (7.22,6.22762);
\draw [c] (7.04,6.22448) -- (7.22,6.22448);
\draw [c] (7.22,6.22448) -- (7.4,6.22448);
\draw [c] (7.58,5.94687) -- (7.58,5.95125);
\draw [c] (7.58,5.95125) -- (7.58,5.95559);
\draw [c] (7.4,5.95125) -- (7.58,5.95125);
\draw [c] (7.58,5.95125) -- (7.76,5.95125);
\draw [c] (7.94,5.68108) -- (7.94,5.68708);
\draw [c] (7.94,5.68708) -- (7.94,5.69299);
\draw [c] (7.76,5.68708) -- (7.94,5.68708);
\draw [c] (7.94,5.68708) -- (8.12,5.68708);
\draw [c] (8.3,5.4057) -- (8.3,5.414);
\draw [c] (8.3,5.414) -- (8.3,5.42213);
\draw [c] (8.12,5.414) -- (8.3,5.414);
\draw [c] (8.3,5.414) -- (8.48,5.414);
\draw [c] (8.66,5.14525) -- (8.66,5.15652);
\draw [c] (8.66,5.15652) -- (8.66,5.16751);
\draw [c] (8.48,5.15652) -- (8.66,5.15652);
\draw [c] (8.66,5.15652) -- (8.84,5.15652);
\draw [c] (9.02,4.91326) -- (9.02,4.92808);
\draw [c] (9.02,4.92808) -- (9.02,4.94241);
\draw [c] (8.84,4.92808) -- (9.02,4.92808);
\draw [c] (9.02,4.92808) -- (9.2,4.92808);
\draw [c] (9.38,4.59832) -- (9.38,4.6198);
\draw [c] (9.38,4.6198) -- (9.38,4.64025);
\draw [c] (9.2,4.6198) -- (9.38,4.6198);
\draw [c] (9.38,4.6198) -- (9.56,4.6198);
\draw [c] (9.74,4.32611) -- (9.74,4.35572);
\draw [c] (9.74,4.35572) -- (9.74,4.3834);
\draw [c] (9.56,4.35572) -- (9.74,4.35572);
\draw [c] (9.74,4.35572) -- (9.92,4.35572);
\draw [c] (10.1,4.06897) -- (10.1,4.08054);
\draw [c] (10.1,4.08054) -- (10.1,4.11836);
\draw [c] (9.92,4.08054) -- (10.1,4.08054);
\draw [c] (10.1,4.08054) -- (10.28,4.08054);
\colorlet{c}{natcomp};
\draw [c] (2.18,13.3011) -- (2.18,13.302);
\draw [c] (2.18,13.302) -- (2.18,13.3029);
\draw [c] (2,13.302) -- (2.18,13.302);
\draw [c] (2.18,13.302) -- (2.36,13.302);
\draw [c] (2.54,11.6649) -- (2.54,11.6713);
\draw [c] (2.54,11.6713) -- (2.54,11.6775);
\draw [c] (2.36,11.6713) -- (2.54,11.6713);
\draw [c] (2.54,11.6713) -- (2.72,11.6713);
\draw [c] (2.9,10.6525) -- (2.9,10.6734);
\draw [c] (2.9,10.6734) -- (2.9,10.6934);
\draw [c] (2.72,10.6734) -- (2.9,10.6734);
\draw [c] (2.9,10.6734) -- (3.08,10.6734);
\draw [c] (3.26,9.90775) -- (3.26,9.95804);
\draw [c] (3.26,9.95804) -- (3.26,10.003);
\draw [c] (3.08,9.95804) -- (3.26,9.95804);
\draw [c] (3.26,9.95804) -- (3.44,9.95804);
\draw [c] (3.62,9.51424) -- (3.62,9.5169);
\draw [c] (3.62,9.5169) -- (3.62,9.51955);
\draw [c] (3.44,9.5169) -- (3.62,9.5169);
\draw [c] (3.62,9.5169) -- (3.8,9.5169);
\draw [c] (3.98,9.06341) -- (3.98,9.06795);
\draw [c] (3.98,9.06795) -- (3.98,9.07243);
\draw [c] (3.8,9.06795) -- (3.98,9.06795);
\draw [c] (3.98,9.06795) -- (4.16,9.06795);
\draw [c] (4.34,8.64794) -- (4.34,8.65534);
\draw [c] (4.34,8.65534) -- (4.34,8.6626);
\draw [c] (4.16,8.65534) -- (4.34,8.65534);
\draw [c] (4.34,8.65534) -- (4.52,8.65534);
\draw [c] (4.7,8.23698) -- (4.7,8.24899);
\draw [c] (4.7,8.24899) -- (4.7,8.26066);
\draw [c] (4.52,8.24899) -- (4.7,8.24899);
\draw [c] (4.7,8.24899) -- (4.88,8.24899);
\draw [c] (5.06,7.90895) -- (5.06,7.92662);
\draw [c] (5.06,7.92662) -- (5.06,7.94358);
\draw [c] (4.88,7.92662) -- (5.06,7.92662);
\draw [c] (5.06,7.92662) -- (5.24,7.92662);
\draw [c] (5.42,7.57803) -- (5.42,7.60412);
\draw [c] (5.42,7.60412) -- (5.42,7.6287);
\draw [c] (5.24,7.60412) -- (5.42,7.60412);
\draw [c] (5.42,7.60412) -- (5.6,7.60412);
\draw [c] (5.78,7.23476) -- (5.78,7.27386);
\draw [c] (5.78,7.27386) -- (5.78,7.30965);
\draw [c] (5.6,7.27386) -- (5.78,7.27386);
\draw [c] (5.78,7.27386) -- (5.96,7.27386);
\draw [c] (6.14,6.91271) -- (6.14,6.96983);
\draw [c] (6.14,6.96983) -- (6.14,7.02017);
\draw [c] (5.96,6.96983) -- (6.14,6.96983);
\draw [c] (6.14,6.96983) -- (6.32,6.96983);
\draw [c] (6.5,6.55358) -- (6.5,6.64072);
\draw [c] (6.5,6.64072) -- (6.5,6.71297);
\draw [c] (6.32,6.64072) -- (6.5,6.64072);
\draw [c] (6.5,6.64072) -- (6.68,6.64072);
\draw [c] (6.86,6.0278) -- (6.86,6.18906);
\draw [c] (6.86,6.18906) -- (6.86,6.30562);
\draw [c] (6.68,6.18906) -- (6.86,6.18906);
\draw [c] (6.86,6.18906) -- (7.04,6.18906);
\draw [c] (7.22,6.1219) -- (7.22,6.2664);
\draw [c] (7.22,6.2664) -- (7.22,6.37399);
\draw [c] (7.04,6.2664) -- (7.22,6.2664);
\draw [c] (7.22,6.2664) -- (7.4,6.2664);
\draw [c] (7.58,4.98541) -- (7.58,5.50632);
\draw [c] (7.58,5.50632) -- (7.58,5.73319);
\draw [c] (7.4,5.50632) -- (7.58,5.50632);
\draw [c] (7.58,5.50632) -- (7.76,5.50632);
\draw [c] (7.94,4.98541) -- (7.94,5.50632);
\draw [c] (7.94,5.50632) -- (7.94,5.73319);
\draw [c] (7.76,5.50632) -- (7.94,5.50632);
\draw [c] (7.94,5.50632) -- (8.12,5.50632);
\draw [c] (8.66,4.06897) -- (8.66,5.21228);
\draw [c] (8.66,5.21228) -- (8.66,5.50632);
\draw [c] (8.48,5.21228) -- (8.66,5.21228);
\draw [c] (8.66,5.21228) -- (8.84,5.21228);
\draw [c] (9.74,4.06897) -- (9.74,5.21228);
\draw [c] (9.74,5.21228) -- (9.74,5.50632);
\draw [c] (9.56,5.21228) -- (9.74,5.21228);
\draw [c] (9.74,5.21228) -- (9.92,5.21228);
\definecolor{c}{rgb}{1,1,1};
\draw [c] (14.1954,11.0057) -- (14.1954,12.9598) -- (19.8851,12.9598) -- (19.8851,11.0057) -- (14.1954,11.0057);
\draw [c] (14.1954,11.0057) -- (19.8851,11.0057);
\draw [c] (19.8851,11.0057) -- (19.8851,12.9598);
\draw [c] (19.8851,12.9598) -- (14.1954,12.9598);
\draw [c] (14.1954,12.9598) -- (14.1954,11.0057);
\draw [anchor=base west] (15.6178,12.4875) node[ rotate=0]{CalcHEP};
\colorlet{c}{kugray!20};
\draw [c, fill=c] (14.4088,12.4061) -- (15.4045,12.4061) -- (15.4045,12.8621) -- (14.4088,12.8621);
\colorlet{c}{kugray};
\draw [c] (14.4088,12.6341) -- (15.4045,12.6341);
\draw [anchor=base west] (15.6178,11.8362) node[ rotate=0]{pythia8};
\colorlet{c}{natgreen};
\draw [c] (14.4088,11.9828) -- (15.4045,11.9828);
\draw [anchor=base west] (15.6178,11.1849) node[ rotate=0]{MadGraph};
\colorlet{c}{natcomp};
\draw [c] (14.4088,11.3314) -- (15.4045,11.3314);
% \definecolor{c}{rgb}{1,1,1};
% \draw [color=c, fill=c] (0,0) rectangle (20,4.06897);
% \draw [color=c, fill=c] (2,0.406897) rectangle (20,4.06897);
% \definecolor{c}{rgb}{0,0,0};
% \draw [c] (2,0.406897) -- (2,4.06897) -- (20,4.06897) -- (20,0.406897) -- (2,0.406897);
% \definecolor{c}{rgb}{1,1,1};
% \draw [color=c, fill=c] (2,0.406897) rectangle (20,4.06897);
\colorlet{c}{kugray!20};
\draw [color=c, fill=c] (2,2.30251) rectangle (2.36,2.36595);
\draw [color=c, fill=c] (2.36,2.29952) rectangle (2.72,2.36893);
\draw [color=c, fill=c] (2.72,2.2269) rectangle (3.08,2.44155);
\draw [color=c, fill=c] (3.08,2.10501) rectangle (3.44,2.56344);
\draw [color=c, fill=c] (3.44,1.9124) rectangle (3.8,2.75606);
\draw [color=c, fill=c] (3.8,1.73569) rectangle (4.16,2.93277);
\draw [color=c, fill=c] (4.16,1.92417) rectangle (4.52,2.74429);
\draw [color=c, fill=c] (4.52,2.29398) rectangle (4.88,2.37447);
\draw [color=c, fill=c] (4.88,2.27292) rectangle (5.24,2.39554);
\draw [color=c, fill=c] (5.24,2.24452) rectangle (5.6,2.42393);
\draw [color=c, fill=c] (5.6,2.20677) rectangle (5.96,2.46169);
\draw [color=c, fill=c] (5.96,2.14591) rectangle (6.32,2.52254);
\draw [color=c, fill=c] (6.32,2.06416) rectangle (6.68,2.60429);
\draw [color=c, fill=c] (6.68,1.94245) rectangle (7.04,2.72601);
\draw [color=c, fill=c] (7.04,1.80702) rectangle (7.4,2.86143);
\draw [color=c, fill=c] (7.4,1.69286) rectangle (7.76,2.9756);
\draw [color=c, fill=c] (7.76,1.47716) rectangle (8.12,3.1913);
\draw [color=c, fill=c] (8.12,0.730798) rectangle (8.48,3.93766);
\draw [color=c, fill=c] (8.48,0.406897) rectangle (8.84,4.06897);
\draw [color=c, fill=c] (8.84,0.406897) rectangle (9.2,4.06897);
\draw [color=c, fill=c] (9.2,0.406897) rectangle (9.56,4.06897);
\draw [color=c, fill=c] (9.56,0.406897) rectangle (9.92,4.06897);
\draw [color=c, fill=c] (9.92,0.406897) rectangle (10.28,4.06897);
\draw [color=c, fill=c] (10.28,2.33423) rectangle (10.64,2.33423);
\draw [color=c, fill=c] (10.64,0.406897) rectangle (11,4.06897);
\draw [color=c, fill=c] (11,2.33423) rectangle (11.36,2.33423);
\draw [color=c, fill=c] (11.36,2.33423) rectangle (11.72,2.33423);
\draw [color=c, fill=c] (11.72,2.33423) rectangle (12.08,2.33423);
\draw [color=c, fill=c] (12.08,2.33423) rectangle (12.44,2.33423);
\draw [color=c, fill=c] (12.44,2.33423) rectangle (12.8,2.33423);
\draw [color=c, fill=c] (12.8,2.33423) rectangle (13.16,2.33423);
\draw [color=c, fill=c] (13.16,2.33423) rectangle (13.52,2.33423);
\draw [color=c, fill=c] (13.52,2.33423) rectangle (13.88,2.33423);
\draw [color=c, fill=c] (13.88,2.33423) rectangle (14.24,2.33423);
\draw [color=c, fill=c] (14.24,2.33423) rectangle (14.6,2.33423);
\draw [color=c, fill=c] (14.6,2.33423) rectangle (14.96,2.33423);
\draw [color=c, fill=c] (14.96,2.33423) rectangle (15.32,2.33423);
\draw [color=c, fill=c] (15.32,2.33423) rectangle (15.68,2.33423);
\draw [color=c, fill=c] (15.68,2.33423) rectangle (16.04,2.33423);
\draw [color=c, fill=c] (16.04,2.33423) rectangle (16.4,2.33423);
\draw [color=c, fill=c] (16.4,2.33423) rectangle (16.76,2.33423);
\draw [color=c, fill=c] (16.76,2.33423) rectangle (17.12,2.33423);
\draw [color=c, fill=c] (17.12,2.33423) rectangle (17.48,2.33423);
\draw [color=c, fill=c] (17.48,2.33423) rectangle (17.84,2.33423);
\draw [color=c, fill=c] (17.84,2.33423) rectangle (18.2,2.33423);
\draw [color=c, fill=c] (18.2,2.33423) rectangle (18.56,2.33423);
\draw [color=c, fill=c] (18.56,2.33423) rectangle (18.92,2.33423);
\draw [color=c, fill=c] (18.92,2.33423) rectangle (19.28,2.33423);
\draw [color=c, fill=c] (19.28,2.33423) rectangle (19.64,2.33423);
\draw [color=c, fill=c] (19.64,2.33423) rectangle (20,2.33423);
\definecolor{c}{rgb}{0,0,0};
\draw [c] (2,0.406897) -- (20,0.406897);
\draw [anchor= east] (20,0.179034) +(0,-2em) node[ rotate=0]{$p_T^{\gamma_1} \text{ [GeV]}$};
\draw [c] (2,0.516759) -- (2,0.406897);
\draw [c] (2.45,0.461828) -- (2.45,0.406897);
\draw [c] (2.9,0.461828) -- (2.9,0.406897);
\draw [c] (3.35,0.461828) -- (3.35,0.406897);
\draw [c] (3.8,0.461828) -- (3.8,0.406897);
\draw [c] (4.25,0.516759) -- (4.25,0.406897);
\draw [c] (4.7,0.461828) -- (4.7,0.406897);
\draw [c] (5.15,0.461828) -- (5.15,0.406897);
\draw [c] (5.6,0.461828) -- (5.6,0.406897);
\draw [c] (6.05,0.461828) -- (6.05,0.406897);
\draw [c] (6.5,0.516759) -- (6.5,0.406897);
\draw [c] (6.95,0.461828) -- (6.95,0.406897);
\draw [c] (7.4,0.461828) -- (7.4,0.406897);
\draw [c] (7.85,0.461828) -- (7.85,0.406897);
\draw [c] (8.3,0.461828) -- (8.3,0.406897);
\draw [c] (8.75,0.516759) -- (8.75,0.406897);
\draw [c] (9.2,0.461828) -- (9.2,0.406897);
\draw [c] (9.65,0.461828) -- (9.65,0.406897);
\draw [c] (10.1,0.461828) -- (10.1,0.406897);
\draw [c] (10.55,0.461828) -- (10.55,0.406897);
\draw [c] (11,0.516759) -- (11,0.406897);
\draw [c] (11.45,0.461828) -- (11.45,0.406897);
\draw [c] (11.9,0.461828) -- (11.9,0.406897);
\draw [c] (12.35,0.461828) -- (12.35,0.406897);
\draw [c] (12.8,0.461828) -- (12.8,0.406897);
\draw [c] (13.25,0.516759) -- (13.25,0.406897);
\draw [c] (13.7,0.461828) -- (13.7,0.406897);
\draw [c] (14.15,0.461828) -- (14.15,0.406897);
\draw [c] (14.6,0.461828) -- (14.6,0.406897);
\draw [c] (15.05,0.461828) -- (15.05,0.406897);
\draw [c] (15.5,0.516759) -- (15.5,0.406897);
\draw [c] (15.95,0.461828) -- (15.95,0.406897);
\draw [c] (16.4,0.461828) -- (16.4,0.406897);
\draw [c] (16.85,0.461828) -- (16.85,0.406897);
\draw [c] (17.3,0.461828) -- (17.3,0.406897);
\draw [c] (17.75,0.516759) -- (17.75,0.406897);
\draw [c] (18.2,0.461828) -- (18.2,0.406897);
\draw [c] (18.65,0.461828) -- (18.65,0.406897);
\draw [c] (19.1,0.461828) -- (19.1,0.406897);
\draw [c] (19.55,0.461828) -- (19.55,0.406897);
\draw [c] (20,0.516759) -- (20,0.406897);
\draw [c] (2,0.516759) -- (2,0.406897);
\draw [c] (20,0.516759) -- (20,0.406897);
\draw [anchor=base] (2,0.272621)     +(0,-.9em) node[ rotate=0]{0};
\draw [anchor=base] (4.25,0.272621)  +(0,-.9em) node[ rotate=0]{500};
\draw [anchor=base] (6.5,0.272621)   +(0,-.9em) node[ rotate=0]{1000};
\draw [anchor=base] (8.75,0.272621)  +(0,-.9em) node[ rotate=0]{1500};
\draw [anchor=base] (11,0.272621)    +(0,-.9em) node[ rotate=0]{2000};
\draw [anchor=base] (13.25,0.272621) +(0,-.9em) node[ rotate=0]{2500};
\draw [anchor=base] (15.5,0.272621)  +(0,-.9em) node[ rotate=0]{3000};
\draw [anchor=base] (17.75,0.272621) +(0,-.9em) node[ rotate=0]{3500};
\draw [anchor=base] (20,0.272621)    +(0,-.9em) node[ rotate=0]{4000};
\draw [c] (2,0.406897) -- (2,4.06897);
\draw [anchor= east] (0.1,4.06897) node[ rotate=90]{Ratio};
\draw [c] (2.54,1.20043) -- (2,1.20043);
\draw [c] (2.27,1.42719) -- (2,1.42719);
\draw [c] (2.27,1.65395) -- (2,1.65395);
\draw [c] (2.27,1.88071) -- (2,1.88071);
\draw [c] (2.27,2.10747) -- (2,2.10747);
\draw [c] (2.54,2.33423) -- (2,2.33423);
\draw [c] (2.27,2.56099) -- (2,2.56099);
\draw [c] (2.27,2.78775) -- (2,2.78775);
\draw [c] (2.27,3.0145) -- (2,3.0145);
\draw [c] (2.27,3.24126) -- (2,3.24126);
\draw [c] (2.54,3.46802) -- (2,3.46802);
\draw [c] (2.54,1.20043) -- (2,1.20043);
\draw [c] (2.27,0.973673) -- (2,0.973673);
\draw [c] (2.27,0.746914) -- (2,0.746914);
\draw [c] (2.27,0.520154) -- (2,0.520154);
\draw [c] (2.54,3.46802) -- (2,3.46802);
\draw [c] (2.27,3.69478) -- (2,3.69478);
\draw [c] (2.27,3.92154) -- (2,3.92154);
\draw [anchor= east] (1.9,1.20043) node[ rotate=0]{0.5};
\draw [anchor= east] (1.9,2.33423) node[ rotate=0]{1.0};
\draw [anchor= east] (1.9,3.46802) node[ rotate=0]{1.5};
\colorlet{c}{natgreen};
\draw [c] (2.18,2.24131) -- (2.18,2.26458);
\draw [c] (2.18,2.26458) -- (2.18,2.28784);
\draw [c] (2,2.26458) -- (2.18,2.26458);
\draw [c] (2.18,2.26458) -- (2.36,2.26458);
\draw [c] (2.54,2.3718) -- (2.54,2.43588);
\draw [c] (2.54,2.43588) -- (2.54,2.49995);
\draw [c] (2.36,2.43588) -- (2.54,2.43588);
\draw [c] (2.54,2.43588) -- (2.72,2.43588);
\draw [c] (2.9,2.16317) -- (2.9,2.32925);
\draw [c] (2.9,2.32925) -- (2.9,2.49532);
\draw [c] (2.72,2.32925) -- (2.9,2.32925);
\draw [c] (2.9,2.32925) -- (3.08,2.32925);
\draw [c] (3.26,2.25126) -- (3.26,2.42003);
\draw [c] (3.26,2.42003) -- (3.26,2.5888);
\draw [c] (3.08,2.42003) -- (3.26,2.42003);
\draw [c] (3.26,2.42003) -- (3.44,2.42003);
\draw [c] (3.62,2.30552) -- (3.62,2.64593);
\draw [c] (3.62,2.64593) -- (3.62,2.98635);
\draw [c] (3.44,2.64593) -- (3.62,2.64593);
\draw [c] (3.62,2.64593) -- (3.8,2.64593);
\draw [c] (3.98,2.03502) -- (3.98,2.4893);
\draw [c] (3.98,2.4893) -- (3.98,2.94358);
\draw [c] (3.8,2.4893) -- (3.98,2.4893);
\draw [c] (3.98,2.4893) -- (4.16,2.4893);
\draw [c] (4.34,2.11198) -- (4.34,2.42047);
\draw [c] (4.34,2.42047) -- (4.34,2.72896);
\draw [c] (4.16,2.42047) -- (4.34,2.42047);
\draw [c] (4.34,2.42047) -- (4.52,2.42047);
\draw [c] (4.7,2.27377) -- (4.7,2.38355);
\draw [c] (4.7,2.38355) -- (4.7,2.49334);
\draw [c] (4.52,2.38355) -- (4.7,2.38355);
\draw [c] (4.7,2.38355) -- (4.88,2.38355);
\draw [c] (5.06,2.24453) -- (5.06,2.41291);
\draw [c] (5.06,2.41291) -- (5.06,2.58129);
\draw [c] (4.88,2.41291) -- (5.06,2.41291);
\draw [c] (5.06,2.41291) -- (5.24,2.41291);
\draw [c] (5.42,2.33397) -- (5.42,2.59016);
\draw [c] (5.42,2.59016) -- (5.42,2.84635);
\draw [c] (5.24,2.59016) -- (5.42,2.59016);
\draw [c] (5.42,2.59016) -- (5.6,2.59016);
\draw [c] (5.78,2.44995) -- (5.78,2.83243);
\draw [c] (5.78,2.83243) -- (5.78,3.21491);
\draw [c] (5.6,2.83243) -- (5.78,2.83243);
\draw [c] (5.78,2.83243) -- (5.96,2.83243);
\draw [c] (6.14,2.63466) -- (6.14,3.24404);
\draw [c] (6.14,3.24404) -- (6.14,3.85342);
\draw [c] (5.96,3.24404) -- (6.14,3.24404);
\draw [c] (6.14,3.24404) -- (6.32,3.24404);
\draw [c] (6.5,2.12702) -- (6.5,2.746);
\draw [c] (6.5,2.746) -- (6.5,3.36498);
\draw [c] (6.32,2.746) -- (6.5,2.746);
\draw [c] (6.5,2.746) -- (6.68,2.746);
\draw [c] (6.86,2.72514) -- (6.86,3.09554);
\draw [c] (6.86,3.09554) -- (6.86,3.46594);
\draw [c] (6.68,3.09554) -- (6.86,3.09554);
\draw [c] (6.86,3.09554) -- (7.04,3.09554);
\draw [c] (7.22,2.48067) -- (7.22,2.95619);
\draw [c] (7.22,2.95619) -- (7.22,3.43172);
\draw [c] (7.04,2.95619) -- (7.22,2.95619);
\draw [c] (7.22,2.95619) -- (7.4,2.95619);
\draw [c] (7.58,1.8627) -- (7.58,2.31245);
\draw [c] (7.58,2.31245) -- (7.58,2.76221);
\draw [c] (7.4,2.31245) -- (7.58,2.31245);
\draw [c] (7.58,2.31245) -- (7.76,2.31245);
\draw [c] (7.94,1.64229) -- (7.94,2.21809);
\draw [c] (7.94,2.21809) -- (7.94,2.79388);
\draw [c] (7.76,2.21809) -- (7.94,2.21809);
\draw [c] (7.94,2.21809) -- (8.12,2.21809);
\draw [c] (8.3,2.043) -- (8.3,4.02234);
\draw [c] (8.3,4.02234) -- (8.3,4.06897);
\draw [c] (8.12,4.02234) -- (8.3,4.02234);
\draw [c] (8.3,4.02234) -- (8.48,4.02234);
\draw [c] (8.66,1.32744) -- (8.66,4.06897);
\draw [c] (9.02,0.406897) -- (9.02,4.06897);
\draw [c] (9.38,0.406897) -- (9.38,2.5);
\draw [c] (9.38,2.5) -- (9.38,4.06897);
\draw [c] (9.2,2.5) -- (9.38,2.5);
\draw [c] (9.38,2.5) -- (9.56,2.5);
\draw [c] (9.74,0.406897) -- (9.74,1.37234);
\draw [c] (9.74,1.37234) -- (9.74,2.68101);
\draw [c] (9.56,1.37234) -- (9.74,1.37234);
\draw [c] (9.74,1.37234) -- (9.92,1.37234);
\draw [c] (10.1,0.406897) -- (10.1,0.749165);
\draw [c] (10.1,0.749165) -- (10.1,1.43465);
\draw [c] (9.92,0.749165) -- (10.1,0.749165);
\draw [c] (10.1,0.749165) -- (10.28,0.749165);
\colorlet{c}{natcomp};
\draw [c] (2.18,2.38691) -- (2.18,2.41065);
\draw [c] (2.18,2.41065) -- (2.18,2.43439);
\draw [c] (2,2.41065) -- (2.18,2.41065);
\draw [c] (2.18,2.41065) -- (2.36,2.41065);
\draw [c] (2.54,2.36364) -- (2.54,2.40662);
\draw [c] (2.54,2.40662) -- (2.54,2.4496);
\draw [c] (2.36,2.40662) -- (2.54,2.40662);
\draw [c] (2.54,2.40662) -- (2.72,2.40662);
\draw [c] (2.9,2.07773) -- (2.9,2.20293);
\draw [c] (2.9,2.20293) -- (2.9,2.32813);
\draw [c] (2.72,2.20293) -- (2.9,2.20293);
\draw [c] (2.9,2.20293) -- (3.08,2.20293);
\draw [c] (3.26,1.6712) -- (3.26,1.9167);
\draw [c] (3.26,1.9167) -- (3.26,2.1622);
\draw [c] (3.08,1.9167) -- (3.26,1.9167);
\draw [c] (3.26,1.9167) -- (3.44,1.9167);
\draw [c] (3.62,2.1855) -- (3.62,2.50685);
\draw [c] (3.62,2.50685) -- (3.62,2.8282);
\draw [c] (3.44,2.50685) -- (3.62,2.50685);
\draw [c] (3.62,2.50685) -- (3.8,2.50685);
\draw [c] (3.98,1.88411) -- (3.98,2.302);
\draw [c] (3.98,2.302) -- (3.98,2.7199);
\draw [c] (3.8,2.302) -- (3.98,2.302);
\draw [c] (3.98,2.302) -- (4.16,2.302);
\draw [c] (4.34,1.88204) -- (4.34,2.15099);
\draw [c] (4.34,2.15099) -- (4.34,2.41993);
\draw [c] (4.16,2.15099) -- (4.34,2.15099);
\draw [c] (4.34,2.15099) -- (4.52,2.15099);
\draw [c] (4.7,1.99598) -- (4.7,2.05686);
\draw [c] (4.7,2.05686) -- (4.7,2.11774);
\draw [c] (4.52,2.05686) -- (4.7,2.05686);
\draw [c] (4.7,2.05686) -- (4.88,2.05686);
\draw [c] (5.06,2.1297) -- (5.06,2.22702);
\draw [c] (5.06,2.22702) -- (5.06,2.32435);
\draw [c] (4.88,2.22702) -- (5.06,2.22702);
\draw [c] (5.06,2.22702) -- (5.24,2.22702);
\draw [c] (5.42,2.08662) -- (5.42,2.2291);
\draw [c] (5.42,2.2291) -- (5.42,2.37158);
\draw [c] (5.24,2.2291) -- (5.42,2.2291);
\draw [c] (5.42,2.2291) -- (5.6,2.2291);
\draw [c] (5.78,1.8773) -- (5.78,2.07092);
\draw [c] (5.78,2.07092) -- (5.78,2.26453);
\draw [c] (5.6,2.07092) -- (5.78,2.07092);
\draw [c] (5.78,2.07092) -- (5.96,2.07092);
\draw [c] (6.14,1.90621) -- (6.14,2.2032);
\draw [c] (6.14,2.2032) -- (6.14,2.50018);
\draw [c] (5.96,2.2032) -- (6.14,2.2032);
\draw [c] (6.14,2.2032) -- (6.32,2.2032);
\draw [c] (6.5,1.67699) -- (6.5,2.08943);
\draw [c] (6.5,2.08943) -- (6.5,2.50188);
\draw [c] (6.32,2.08943) -- (6.5,2.08943);
\draw [c] (6.5,2.08943) -- (6.68,2.08943);
\draw [c] (6.86,1.03691) -- (6.86,1.53454);
\draw [c] (6.86,1.53454) -- (6.86,2.03217);
\draw [c] (6.68,1.53454) -- (6.86,1.53454);
\draw [c] (6.86,1.53454) -- (7.04,1.53454);
\draw [c] (7.22,2.19671) -- (7.22,3.25636);
\draw [c] (7.22,3.25636) -- (7.22,4.06897);
\draw [c] (7.04,3.25636) -- (7.22,3.25636);
\draw [c] (7.22,3.25636) -- (7.4,3.25636);
\draw [c] (7.58,0.406897) -- (7.58,0.853434);
\draw [c] (7.58,0.853434) -- (7.58,1.43161);
\draw [c] (7.4,0.853434) -- (7.58,0.853434);
\draw [c] (7.58,0.853434) -- (7.76,0.853434);
\draw [c] (7.94,0.409555) -- (7.94,1.47163);
\draw [c] (7.94,1.47163) -- (7.94,2.53371);
\draw [c] (7.76,1.47163) -- (7.94,1.47163);
\draw [c] (7.94,1.47163) -- (8.12,1.47163);
\draw [c] (8.66,0.406897) -- (8.66,4.06897);
\draw [c] (9.74,0.406897) -- (9.74,4.06897);
\definecolor{c}{rgb}{0,0,0};
\draw [c] (2,0.406897) -- (2,4.06897) -- (20,4.06897) -- (20,0.406897) -- (2,0.406897);

\end{tikzpicture}
}
\end{infilsf}
\end{minipage}
\begin{minipage}[b]{.3\textwidth}
\subcaption{The distribution of $p_T^{\gamma_1}$, the transverse momentum of the leading photon, in the event samples produced by three event generators.}

\phantom{p}
\end{minipage}
\begin{minipage}[b]{.69\textwidth}
\begin{infilsf} \tiny
\hspace{-1ex}\makebox[0pt][l]{\pgfdeclareplotmark{cross} {
\pgfpathmoveto{\pgfpoint{-0.3\pgfplotmarksize}{\pgfplotmarksize}}
\pgfpathlineto{\pgfpoint{+0.3\pgfplotmarksize}{\pgfplotmarksize}}
\pgfpathlineto{\pgfpoint{+0.3\pgfplotmarksize}{0.3\pgfplotmarksize}}
\pgfpathlineto{\pgfpoint{+1\pgfplotmarksize}{0.3\pgfplotmarksize}}
\pgfpathlineto{\pgfpoint{+1\pgfplotmarksize}{-0.3\pgfplotmarksize}}
\pgfpathlineto{\pgfpoint{+0.3\pgfplotmarksize}{-0.3\pgfplotmarksize}}
\pgfpathlineto{\pgfpoint{+0.3\pgfplotmarksize}{-1.\pgfplotmarksize}}
\pgfpathlineto{\pgfpoint{-0.3\pgfplotmarksize}{-1.\pgfplotmarksize}}
\pgfpathlineto{\pgfpoint{-0.3\pgfplotmarksize}{-0.3\pgfplotmarksize}}
\pgfpathlineto{\pgfpoint{-1.\pgfplotmarksize}{-0.3\pgfplotmarksize}}
\pgfpathlineto{\pgfpoint{-1.\pgfplotmarksize}{0.3\pgfplotmarksize}}
\pgfpathlineto{\pgfpoint{-0.3\pgfplotmarksize}{0.3\pgfplotmarksize}}
\pgfpathclose
\pgfusepathqstroke
}
\pgfdeclareplotmark{cross*} {
\pgfpathmoveto{\pgfpoint{-0.3\pgfplotmarksize}{\pgfplotmarksize}}
\pgfpathlineto{\pgfpoint{+0.3\pgfplotmarksize}{\pgfplotmarksize}}
\pgfpathlineto{\pgfpoint{+0.3\pgfplotmarksize}{0.3\pgfplotmarksize}}
\pgfpathlineto{\pgfpoint{+1\pgfplotmarksize}{0.3\pgfplotmarksize}}
\pgfpathlineto{\pgfpoint{+1\pgfplotmarksize}{-0.3\pgfplotmarksize}}
\pgfpathlineto{\pgfpoint{+0.3\pgfplotmarksize}{-0.3\pgfplotmarksize}}
\pgfpathlineto{\pgfpoint{+0.3\pgfplotmarksize}{-1.\pgfplotmarksize}}
\pgfpathlineto{\pgfpoint{-0.3\pgfplotmarksize}{-1.\pgfplotmarksize}}
\pgfpathlineto{\pgfpoint{-0.3\pgfplotmarksize}{-0.3\pgfplotmarksize}}
\pgfpathlineto{\pgfpoint{-1.\pgfplotmarksize}{-0.3\pgfplotmarksize}}
\pgfpathlineto{\pgfpoint{-1.\pgfplotmarksize}{0.3\pgfplotmarksize}}
\pgfpathlineto{\pgfpoint{-0.3\pgfplotmarksize}{0.3\pgfplotmarksize}}
\pgfpathclose
\pgfusepathqfillstroke
}
\pgfdeclareplotmark{newstar} {
\pgfpathmoveto{\pgfqpoint{0pt}{\pgfplotmarksize}}
\pgfpathlineto{\pgfqpointpolar{44}{0.5\pgfplotmarksize}}
\pgfpathlineto{\pgfqpointpolar{18}{\pgfplotmarksize}}
\pgfpathlineto{\pgfqpointpolar{-20}{0.5\pgfplotmarksize}}
\pgfpathlineto{\pgfqpointpolar{-54}{\pgfplotmarksize}}
\pgfpathlineto{\pgfqpointpolar{-90}{0.5\pgfplotmarksize}}
\pgfpathlineto{\pgfqpointpolar{234}{\pgfplotmarksize}}
\pgfpathlineto{\pgfqpointpolar{198}{0.5\pgfplotmarksize}}
\pgfpathlineto{\pgfqpointpolar{162}{\pgfplotmarksize}}
\pgfpathlineto{\pgfqpointpolar{134}{0.5\pgfplotmarksize}}
\pgfpathclose
\pgfusepathqstroke
}
\pgfdeclareplotmark{newstar*} {
\pgfpathmoveto{\pgfqpoint{0pt}{\pgfplotmarksize}}
\pgfpathlineto{\pgfqpointpolar{44}{0.5\pgfplotmarksize}}
\pgfpathlineto{\pgfqpointpolar{18}{\pgfplotmarksize}}
\pgfpathlineto{\pgfqpointpolar{-20}{0.5\pgfplotmarksize}}
\pgfpathlineto{\pgfqpointpolar{-54}{\pgfplotmarksize}}
\pgfpathlineto{\pgfqpointpolar{-90}{0.5\pgfplotmarksize}}
\pgfpathlineto{\pgfqpointpolar{234}{\pgfplotmarksize}}
\pgfpathlineto{\pgfqpointpolar{198}{0.5\pgfplotmarksize}}
\pgfpathlineto{\pgfqpointpolar{162}{\pgfplotmarksize}}
\pgfpathlineto{\pgfqpointpolar{134}{0.5\pgfplotmarksize}}
\pgfpathclose
\pgfusepathqfillstroke
}
\begin{tikzpicture}[x=.045\textwidth,y=.045\textwidth]
\definecolor{c}{rgb}{1,1,1};
\draw [color=c, fill=c] (0,0) rectangle (20,13.5632);
\draw [color=c, fill=c] (0,4.06897) rectangle (20,13.5632);
\draw [color=c, fill=c] (2,4.06897) rectangle (20,13.5632);
\definecolor{c}{rgb}{0,0,0};
\draw [c] (2,4.06897) -- (2,13.5632) -- (20,13.5632) -- (20,4.06897) -- (2,4.06897);
\definecolor{c}{rgb}{1,1,1};
\draw [color=c, fill=c] (2,4.06897) rectangle (20,13.5632);
\definecolor{c}{rgb}{0,0,0};
\draw [c] (2,4.06897) -- (2,13.5632) -- (20,13.5632) -- (20,4.06897) -- (2,4.06897);
\colorlet{c}{kugray};
\draw [c] (2.18,13.2621) -- (2.18,13.264);
\draw [c] (2.18,13.264) -- (2.18,13.266);
\draw [c] (2,13.264) -- (2.18,13.264);
\draw [c] (2.18,13.264) -- (2.36,13.264);
\draw [c] (2.54,12.2826) -- (2.54,12.2883);
\draw [c] (2.54,12.2883) -- (2.54,12.2939);
\draw [c] (2.36,12.2883) -- (2.54,12.2883);
\draw [c] (2.54,12.2883) -- (2.72,12.2883);
\draw [c] (2.9,11.6467) -- (2.9,11.6579);
\draw [c] (2.9,11.6579) -- (2.9,11.6688);
\draw [c] (2.72,11.6579) -- (2.9,11.6579);
\draw [c] (2.9,11.6579) -- (3.08,11.6579);
\draw [c] (3.26,11.1623) -- (3.26,11.1811);
\draw [c] (3.26,11.1811) -- (3.26,11.1992);
\draw [c] (3.08,11.1811) -- (3.26,11.1811);
\draw [c] (3.26,11.1811) -- (3.44,11.1811);
\draw [c] (3.62,10.7552) -- (3.62,10.7843);
\draw [c] (3.62,10.7843) -- (3.62,10.8118);
\draw [c] (3.44,10.7843) -- (3.62,10.7843);
\draw [c] (3.62,10.7843) -- (3.8,10.7843);
\draw [c] (3.98,10.4532) -- (3.98,10.4935);
\draw [c] (3.98,10.4935) -- (3.98,10.5306);
\draw [c] (3.8,10.4935) -- (3.98,10.4935);
\draw [c] (3.98,10.4935) -- (4.16,10.4935);
\draw [c] (4.34,10.051) -- (4.34,10.1131);
\draw [c] (4.34,10.1131) -- (4.34,10.1679);
\draw [c] (4.16,10.1131) -- (4.34,10.1131);
\draw [c] (4.34,10.1131) -- (4.52,10.1131);
\draw [c] (4.7,9.73147) -- (4.7,9.81897);
\draw [c] (4.7,9.81897) -- (4.7,9.89258);
\draw [c] (4.52,9.81897) -- (4.7,9.81897);
\draw [c] (4.7,9.81897) -- (4.88,9.81897);
\draw [c] (5.06,9.45449) -- (5.06,9.57219);
\draw [c] (5.06,9.57219) -- (5.06,9.66605);
\draw [c] (4.88,9.57219) -- (5.06,9.57219);
\draw [c] (5.06,9.57219) -- (5.24,9.57219);
\draw [c] (5.42,9.4001) -- (5.42,9.47605);
\draw [c] (5.42,9.47605) -- (5.42,9.54132);
\draw [c] (5.24,9.47605) -- (5.42,9.47605);
\draw [c] (5.42,9.47605) -- (5.6,9.47605);
\draw [c] (5.78,9.14911) -- (5.78,9.15276);
\draw [c] (5.78,9.15276) -- (5.78,9.15638);
\draw [c] (5.6,9.15276) -- (5.78,9.15276);
\draw [c] (5.78,9.15276) -- (5.96,9.15276);
\draw [c] (6.14,8.93791) -- (6.14,8.94249);
\draw [c] (6.14,8.94249) -- (6.14,8.94702);
\draw [c] (5.96,8.94249) -- (6.14,8.94249);
\draw [c] (6.14,8.94249) -- (6.32,8.94249);
\draw [c] (6.5,8.7151) -- (6.5,8.72092);
\draw [c] (6.5,8.72092) -- (6.5,8.72667);
\draw [c] (6.32,8.72092) -- (6.5,8.72092);
\draw [c] (6.5,8.72092) -- (6.68,8.72092);
\draw [c] (6.86,8.51935) -- (6.86,8.52653);
\draw [c] (6.86,8.52653) -- (6.86,8.5336);
\draw [c] (6.68,8.52653) -- (6.86,8.52653);
\draw [c] (6.86,8.52653) -- (7.04,8.52653);
\draw [c] (7.22,8.32793) -- (7.22,8.33676);
\draw [c] (7.22,8.33676) -- (7.22,8.34541);
\draw [c] (7.04,8.33676) -- (7.22,8.33676);
\draw [c] (7.22,8.33676) -- (7.4,8.33676);
\draw [c] (7.58,8.14966) -- (7.58,8.16034);
\draw [c] (7.58,8.16034) -- (7.58,8.17079);
\draw [c] (7.4,8.16034) -- (7.58,8.16034);
\draw [c] (7.58,8.16034) -- (7.76,8.16034);
\draw [c] (7.94,7.96948) -- (7.94,7.98245);
\draw [c] (7.94,7.98245) -- (7.94,7.99507);
\draw [c] (7.76,7.98245) -- (7.94,7.98245);
\draw [c] (7.94,7.98245) -- (8.12,7.98245);
\draw [c] (8.3,7.81094) -- (8.3,7.82632);
\draw [c] (8.3,7.82632) -- (8.3,7.84121);
\draw [c] (8.12,7.82632) -- (8.3,7.82632);
\draw [c] (8.3,7.82632) -- (8.48,7.82632);
\draw [c] (8.66,7.61506) -- (8.66,7.63404);
\draw [c] (8.66,7.63404) -- (8.66,7.65228);
\draw [c] (8.48,7.63404) -- (8.66,7.63404);
\draw [c] (8.66,7.63404) -- (8.84,7.63404);
\draw [c] (9.02,7.36281) -- (9.02,7.38771);
\draw [c] (9.02,7.38771) -- (9.02,7.41134);
\draw [c] (8.84,7.38771) -- (9.02,7.38771);
\draw [c] (9.02,7.38771) -- (9.2,7.38771);
\draw [c] (9.38,7.15386) -- (9.38,7.18503);
\draw [c] (9.38,7.18503) -- (9.38,7.21424);
\draw [c] (9.2,7.18503) -- (9.38,7.18503);
\draw [c] (9.38,7.18503) -- (9.56,7.18503);
\draw [c] (9.74,6.90065) -- (9.74,6.94156);
\draw [c] (9.74,6.94156) -- (9.74,6.97916);
\draw [c] (9.56,6.94156) -- (9.74,6.94156);
\draw [c] (9.74,6.94156) -- (9.92,6.94156);
\draw [c] (10.1,6.84585) -- (10.1,6.88925);
\draw [c] (10.1,6.88925) -- (10.1,6.92894);
\draw [c] (9.92,6.88925) -- (10.1,6.88925);
\draw [c] (10.1,6.88925) -- (10.28,6.88925);
\draw [c] (10.46,6.64695) -- (10.46,6.70068);
\draw [c] (10.46,6.70068) -- (10.46,6.74884);
\draw [c] (10.28,6.70068) -- (10.46,6.70068);
\draw [c] (10.46,6.70068) -- (10.64,6.70068);
\draw [c] (10.82,6.38849) -- (10.82,6.45941);
\draw [c] (10.82,6.45941) -- (10.82,6.52092);
\draw [c] (10.64,6.45941) -- (10.82,6.45941);
\draw [c] (10.82,6.45941) -- (11,6.45941);
\draw [c] (11.18,6.24946) -- (11.18,6.33177);
\draw [c] (11.18,6.33177) -- (11.18,6.40169);
\draw [c] (11,6.33177) -- (11.18,6.33177);
\draw [c] (11.18,6.33177) -- (11.36,6.33177);
\draw [c] (11.54,5.98951) -- (11.54,6.09827);
\draw [c] (11.54,6.09827) -- (11.54,6.18635);
\draw [c] (11.36,6.09827) -- (11.54,6.09827);
\draw [c] (11.54,6.09827) -- (11.72,6.09827);
\draw [c] (11.9,5.96608) -- (11.9,6.07759);
\draw [c] (11.9,6.07759) -- (11.9,6.16747);
\draw [c] (11.72,6.07759) -- (11.9,6.07759);
\draw [c] (11.9,6.07759) -- (12.08,6.07759);
\draw [c] (12.26,5.68185) -- (12.26,5.83292);
\draw [c] (12.26,5.83292) -- (12.26,5.94677);
\draw [c] (12.08,5.83292) -- (12.26,5.83292);
\draw [c] (12.26,5.83292) -- (12.44,5.83292);
\draw [c] (12.62,5.63728) -- (12.62,5.7957);
\draw [c] (12.62,5.7957) -- (12.62,5.91365);
\draw [c] (12.44,5.7957) -- (12.62,5.7957);
\draw [c] (12.62,5.7957) -- (12.8,5.7957);
\draw [c] (12.98,5.58835) -- (12.98,5.75523);
\draw [c] (12.98,5.75523) -- (12.98,5.87779);
\draw [c] (12.8,5.75523) -- (12.98,5.75523);
\draw [c] (12.98,5.75523) -- (13.16,5.75523);
\draw [c] (13.34,4.96241) -- (13.34,5.28477);
\draw [c] (13.34,5.28477) -- (13.34,5.47334);
\draw [c] (13.16,5.28477) -- (13.34,5.28477);
\draw [c] (13.34,5.28477) -- (13.52,5.28477);
\draw [c] (13.7,5.58835) -- (13.7,5.75523);
\draw [c] (13.7,5.75523) -- (13.7,5.87779);
\draw [c] (13.52,5.75523) -- (13.7,5.75523);
\draw [c] (13.7,5.75523) -- (13.88,5.75523);
\draw [c] (14.06,4.06897) -- (14.06,4.64005);
\draw [c] (14.06,4.64005) -- (14.06,4.96241);
\draw [c] (13.88,4.64005) -- (14.06,4.64005);
\draw [c] (14.06,4.64005) -- (14.24,4.64005);
\draw [c] (14.42,4.06897) -- (14.42,4.64005);
\draw [c] (14.42,4.64005) -- (14.42,4.96241);
\draw [c] (14.24,4.64005) -- (14.42,4.64005);
\draw [c] (14.42,4.64005) -- (14.6,4.64005);
\draw [c] (14.78,4.06897) -- (14.78,4.64005);
\draw [c] (14.78,4.64005) -- (14.78,4.96241);
\draw [c] (14.6,4.64005) -- (14.78,4.64005);
\draw [c] (14.78,4.64005) -- (14.96,4.64005);
\draw [c] (15.14,4.39133) -- (15.14,4.96241);
\draw [c] (15.14,4.96241) -- (15.14,5.21113);
\draw [c] (14.96,4.96241) -- (15.14,4.96241);
\draw [c] (15.14,4.96241) -- (15.32,4.96241);
\draw [c] (15.5,4.06897) -- (15.5,4.64005);
\draw [c] (15.5,4.64005) -- (15.5,4.96241);
\draw [c] (15.32,4.64005) -- (15.5,4.64005);
\draw [c] (15.5,4.64005) -- (15.68,4.64005);
\draw [c] (15.86,4.06897) -- (15.86,4.64005);
\draw [c] (15.86,4.64005) -- (15.86,4.96241);
\draw [c] (15.68,4.64005) -- (15.86,4.64005);
\draw [c] (15.86,4.64005) -- (16.04,4.64005);
\draw [c] (16.22,4.06897) -- (16.22,4.64005);
\draw [c] (16.22,4.64005) -- (16.22,4.96241);
\draw [c] (16.04,4.64005) -- (16.22,4.64005);
\draw [c] (16.22,4.64005) -- (16.4,4.64005);
\draw [c] (16.94,4.06897) -- (16.94,4.64005);
\draw [c] (16.94,4.64005) -- (16.94,4.96241);
\draw [c] (16.76,4.64005) -- (16.94,4.64005);
\draw [c] (16.94,4.64005) -- (17.12,4.64005);
\definecolor{c}{rgb}{0,0,0};
\draw [c] (2,4.06897) -- (20,4.06897);
\draw [c] (3.46939,4.32531) -- (3.46939,4.06897);
\draw [c] (3.83673,4.19714) -- (3.83673,4.06897);
\draw [c] (4.20408,4.19714) -- (4.20408,4.06897);
\draw [c] (4.57143,4.19714) -- (4.57143,4.06897);
\draw [c] (4.93878,4.19714) -- (4.93878,4.06897);
\draw [c] (5.30612,4.32531) -- (5.30612,4.06897);
\draw [c] (5.67347,4.19714) -- (5.67347,4.06897);
\draw [c] (6.04082,4.19714) -- (6.04082,4.06897);
\draw [c] (6.40816,4.19714) -- (6.40816,4.06897);
\draw [c] (6.77551,4.19714) -- (6.77551,4.06897);
\draw [c] (7.14286,4.32531) -- (7.14286,4.06897);
\draw [c] (7.5102,4.19714) -- (7.5102,4.06897);
\draw [c] (7.87755,4.19714) -- (7.87755,4.06897);
\draw [c] (8.2449,4.19714) -- (8.2449,4.06897);
\draw [c] (8.61224,4.19714) -- (8.61224,4.06897);
\draw [c] (8.97959,4.32531) -- (8.97959,4.06897);
\draw [c] (9.34694,4.19714) -- (9.34694,4.06897);
\draw [c] (9.71429,4.19714) -- (9.71429,4.06897);
\draw [c] (10.0816,4.19714) -- (10.0816,4.06897);
\draw [c] (10.449,4.19714) -- (10.449,4.06897);
\draw [c] (10.8163,4.32531) -- (10.8163,4.06897);
\draw [c] (11.1837,4.19714) -- (11.1837,4.06897);
\draw [c] (11.551,4.19714) -- (11.551,4.06897);
\draw [c] (11.9184,4.19714) -- (11.9184,4.06897);
\draw [c] (12.2857,4.19714) -- (12.2857,4.06897);
\draw [c] (12.6531,4.32531) -- (12.6531,4.06897);
\draw [c] (13.0204,4.19714) -- (13.0204,4.06897);
\draw [c] (13.3878,4.19714) -- (13.3878,4.06897);
\draw [c] (13.7551,4.19714) -- (13.7551,4.06897);
\draw [c] (14.1224,4.19714) -- (14.1224,4.06897);
\draw [c] (14.4898,4.32531) -- (14.4898,4.06897);
\draw [c] (14.8571,4.19714) -- (14.8571,4.06897);
\draw [c] (15.2245,4.19714) -- (15.2245,4.06897);
\draw [c] (15.5918,4.19714) -- (15.5918,4.06897);
\draw [c] (15.9592,4.19714) -- (15.9592,4.06897);
\draw [c] (16.3265,4.32531) -- (16.3265,4.06897);
\draw [c] (16.6939,4.19714) -- (16.6939,4.06897);
\draw [c] (17.0612,4.19714) -- (17.0612,4.06897);
\draw [c] (17.4286,4.19714) -- (17.4286,4.06897);
\draw [c] (17.7959,4.19714) -- (17.7959,4.06897);
\draw [c] (18.1633,4.32531) -- (18.1633,4.06897);
\draw [c] (18.5306,4.19714) -- (18.5306,4.06897);
\draw [c] (18.898,4.19714) -- (18.898,4.06897);
\draw [c] (19.2653,4.19714) -- (19.2653,4.06897);
\draw [c] (19.6327,4.19714) -- (19.6327,4.06897);
\draw [c] (20,4.32531) -- (20,4.06897);
\draw [c] (3.46939,4.32531) -- (3.46939,4.06897);
\draw [c] (3.10204,4.19714) -- (3.10204,4.06897);
\draw [c] (2.73469,4.19714) -- (2.73469,4.06897);
\draw [c] (2.36735,4.19714) -- (2.36735,4.06897);
\draw [c] (2,4.19714) -- (2,4.06897);

\draw [c] (2,4.06897) -- (2,13.5632);
\draw [anchor= east] (0.1,13.5632) node[ rotate=90]{$\di\sigma/\di M_{\gamma\gamma}\text{ [pb/GeV]}$};
\draw [c] (2.3,4.08473) -- (2,4.08473);
\draw [c] (2.3,4.15642) -- (2,4.15642);
\draw [c] (2.3,4.21852) -- (2,4.21852);
\draw [c] (2.3,4.2733) -- (2,4.2733);
\draw [c] (2.6,4.3223) -- (2,4.3223);
\draw [anchor= east] (1.844,4.3223) node[ rotate=0]{$10^{-10}$};
\draw [c] (2.3,4.64466) -- (2,4.64466);
\draw [c] (2.3,4.83323) -- (2,4.83323);
\draw [c] (2.3,4.96702) -- (2,4.96702);
\draw [c] (2.3,5.0708) -- (2,5.0708);
\draw [c] (2.3,5.15559) -- (2,5.15559);
\draw [c] (2.3,5.22728) -- (2,5.22728);
\draw [c] (2.3,5.28938) -- (2,5.28938);
\draw [c] (2.3,5.34416) -- (2,5.34416);
\draw [c] (2.6,5.39316) -- (2,5.39316);
\draw [anchor= east] (1.844,5.39316) node[ rotate=0]{$10^{-9}$};
\draw [c] (2.3,5.71552) -- (2,5.71552);
\draw [c] (2.3,5.90409) -- (2,5.90409);
\draw [c] (2.3,6.03788) -- (2,6.03788);
\draw [c] (2.3,6.14166) -- (2,6.14166);
\draw [c] (2.3,6.22645) -- (2,6.22645);
\draw [c] (2.3,6.29814) -- (2,6.29814);
\draw [c] (2.3,6.36024) -- (2,6.36024);
\draw [c] (2.3,6.41502) -- (2,6.41502);
\draw [c] (2.6,6.46402) -- (2,6.46402);
\draw [anchor= east] (1.844,6.46402) node[ rotate=0]{$10^{-8}$};
\draw [c] (2.3,6.78638) -- (2,6.78638);
\draw [c] (2.3,6.97495) -- (2,6.97495);
\draw [c] (2.3,7.10874) -- (2,7.10874);
\draw [c] (2.3,7.21252) -- (2,7.21252);
\draw [c] (2.3,7.29731) -- (2,7.29731);
\draw [c] (2.3,7.369) -- (2,7.369);
\draw [c] (2.3,7.4311) -- (2,7.4311);
\draw [c] (2.3,7.48588) -- (2,7.48588);
\draw [c] (2.6,7.53488) -- (2,7.53488);
\draw [anchor= east] (1.844,7.53488) node[ rotate=0]{$10^{-7}$};
\draw [c] (2.3,7.85724) -- (2,7.85724);
\draw [c] (2.3,8.04581) -- (2,8.04581);
\draw [c] (2.3,8.1796) -- (2,8.1796);
\draw [c] (2.3,8.28338) -- (2,8.28338);
\draw [c] (2.3,8.36817) -- (2,8.36817);
\draw [c] (2.3,8.43986) -- (2,8.43986);
\draw [c] (2.3,8.50196) -- (2,8.50196);
\draw [c] (2.3,8.55674) -- (2,8.55674);
\draw [c] (2.6,8.60574) -- (2,8.60574);
\draw [anchor= east] (1.844,8.60574) node[ rotate=0]{$10^{-6}$};
\draw [c] (2.3,8.9281) -- (2,8.9281);
\draw [c] (2.3,9.11667) -- (2,9.11667);
\draw [c] (2.3,9.25046) -- (2,9.25046);
\draw [c] (2.3,9.35424) -- (2,9.35424);
\draw [c] (2.3,9.43903) -- (2,9.43903);
\draw [c] (2.3,9.51072) -- (2,9.51072);
\draw [c] (2.3,9.57282) -- (2,9.57282);
\draw [c] (2.3,9.6276) -- (2,9.6276);
\draw [c] (2.6,9.6766) -- (2,9.6766);
\draw [anchor= east] (1.844,9.6766) node[ rotate=0]{$10^{-5}$};
\draw [c] (2.3,9.99896) -- (2,9.99896);
\draw [c] (2.3,10.1875) -- (2,10.1875);
\draw [c] (2.3,10.3213) -- (2,10.3213);
\draw [c] (2.3,10.4251) -- (2,10.4251);
\draw [c] (2.3,10.5099) -- (2,10.5099);
\draw [c] (2.3,10.5816) -- (2,10.5816);
\draw [c] (2.3,10.6437) -- (2,10.6437);
\draw [c] (2.3,10.6985) -- (2,10.6985);
\draw [c] (2.6,10.7475) -- (2,10.7475);
\draw [anchor= east] (1.844,10.7475) node[ rotate=0]{$10^{-4}$};
\draw [c] (2.3,11.0698) -- (2,11.0698);
\draw [c] (2.3,11.2584) -- (2,11.2584);
\draw [c] (2.3,11.3922) -- (2,11.3922);
\draw [c] (2.3,11.496) -- (2,11.496);
\draw [c] (2.3,11.5808) -- (2,11.5808);
\draw [c] (2.3,11.6524) -- (2,11.6524);
\draw [c] (2.3,11.7145) -- (2,11.7145);
\draw [c] (2.3,11.7693) -- (2,11.7693);
\draw [c] (2.6,11.8183) -- (2,11.8183);
\draw [anchor= east] (1.844,11.8183) node[ rotate=0]{$10^{-3}$};
\draw [c] (2.3,12.1407) -- (2,12.1407);
\draw [c] (2.3,12.3293) -- (2,12.3293);
\draw [c] (2.3,12.463) -- (2,12.463);
\draw [c] (2.3,12.5668) -- (2,12.5668);
\draw [c] (2.3,12.6516) -- (2,12.6516);
\draw [c] (2.3,12.7233) -- (2,12.7233);
\draw [c] (2.3,12.7854) -- (2,12.7854);
\draw [c] (2.3,12.8402) -- (2,12.8402);
\draw [c] (2.6,12.8892) -- (2,12.8892);
\draw [anchor= east] (1.844,12.8892) node[ rotate=0]{$10^{-2}$};
\draw [c] (2.3,13.2115) -- (2,13.2115);
\draw [c] (2.3,13.4001) -- (2,13.4001);
\draw [c] (2.3,13.5339) -- (2,13.5339);
\colorlet{c}{natgreen};
\draw [c] (2.18,13.2487) -- (2.18,13.2534);
\draw [c] (2.18,13.2534) -- (2.18,13.258);
\draw [c] (2,13.2534) -- (2.18,13.2534);
\draw [c] (2.18,13.2534) -- (2.36,13.2534);
\draw [c] (2.54,12.289) -- (2.54,12.3022);
\draw [c] (2.54,12.3022) -- (2.54,12.3149);
\draw [c] (2.36,12.3022) -- (2.54,12.3022);
\draw [c] (2.54,12.3022) -- (2.72,12.3022);
\draw [c] (2.9,11.6393) -- (2.9,11.6656);
\draw [c] (2.9,11.6656) -- (2.9,11.6906);
\draw [c] (2.72,11.6656) -- (2.9,11.6656);
\draw [c] (2.9,11.6656) -- (3.08,11.6656);
\draw [c] (3.26,11.2171) -- (3.26,11.2535);
\draw [c] (3.26,11.2535) -- (3.26,11.2872);
\draw [c] (3.08,11.2535) -- (3.26,11.2535);
\draw [c] (3.26,11.2535) -- (3.44,11.2535);
\draw [c] (3.62,10.6859) -- (3.62,10.7303);
\draw [c] (3.62,10.7303) -- (3.62,10.7707);
\draw [c] (3.44,10.7303) -- (3.62,10.7303);
\draw [c] (3.62,10.7303) -- (3.8,10.7303);
\draw [c] (3.98,10.3563) -- (3.98,10.4064);
\draw [c] (3.98,10.4064) -- (3.98,10.4516);
\draw [c] (3.8,10.4064) -- (3.98,10.4064);
\draw [c] (3.98,10.4064) -- (4.16,10.4064);
\draw [c] (4.34,10.2099) -- (4.34,10.2885);
\draw [c] (4.34,10.2885) -- (4.34,10.3557);
\draw [c] (4.16,10.2885) -- (4.34,10.2885);
\draw [c] (4.34,10.2885) -- (4.52,10.2885);
\draw [c] (4.7,9.79807) -- (4.7,9.80476);
\draw [c] (4.7,9.80476) -- (4.7,9.81135);
\draw [c] (4.52,9.80476) -- (4.7,9.80476);
\draw [c] (4.7,9.80476) -- (4.88,9.80476);
\draw [c] (5.06,9.56822) -- (5.06,9.57678);
\draw [c] (5.06,9.57678) -- (5.06,9.58518);
\draw [c] (4.88,9.57678) -- (5.06,9.57678);
\draw [c] (5.06,9.57678) -- (5.24,9.57678);
\draw [c] (5.42,9.35823) -- (5.42,9.36895);
\draw [c] (5.42,9.36895) -- (5.42,9.37944);
\draw [c] (5.24,9.36895) -- (5.42,9.36895);
\draw [c] (5.42,9.36895) -- (5.6,9.36895);
\draw [c] (5.78,9.16365) -- (5.78,9.17687);
\draw [c] (5.78,9.17687) -- (5.78,9.18972);
\draw [c] (5.6,9.17687) -- (5.78,9.17687);
\draw [c] (5.78,9.17687) -- (5.96,9.17687);
\draw [c] (6.14,8.92042) -- (6.14,8.93759);
\draw [c] (6.14,8.93759) -- (6.14,8.95415);
\draw [c] (5.96,8.93759) -- (6.14,8.93759);
\draw [c] (6.14,8.93759) -- (6.32,8.93759);
\draw [c] (6.5,8.71135) -- (6.5,8.73285);
\draw [c] (6.5,8.73285) -- (6.5,8.7534);
\draw [c] (6.32,8.73285) -- (6.5,8.73285);
\draw [c] (6.5,8.73285) -- (6.68,8.73285);
\draw [c] (6.86,8.55219) -- (6.86,8.5777);
\draw [c] (6.86,8.5777) -- (6.86,8.60188);
\draw [c] (6.68,8.5777) -- (6.86,8.5777);
\draw [c] (6.86,8.5777) -- (7.04,8.5777);
\draw [c] (7.22,8.3388) -- (7.22,8.37089);
\draw [c] (7.22,8.37089) -- (7.22,8.4009);
\draw [c] (7.04,8.37089) -- (7.22,8.37089);
\draw [c] (7.22,8.37089) -- (7.4,8.37089);
\draw [c] (7.58,8.11607) -- (7.58,8.15683);
\draw [c] (7.58,8.15683) -- (7.58,8.19431);
\draw [c] (7.4,8.15683) -- (7.58,8.15683);
\draw [c] (7.58,8.15683) -- (7.76,8.15683);
\draw [c] (7.94,7.91955) -- (7.94,7.96989);
\draw [c] (7.94,7.96989) -- (7.94,8.01532);
\draw [c] (7.76,7.96989) -- (7.94,7.96989);
\draw [c] (7.94,7.96989) -- (8.12,7.96989);
\draw [c] (8.3,7.72409) -- (8.3,7.78619);
\draw [c] (8.3,7.78619) -- (8.3,7.84097);
\draw [c] (8.12,7.78619) -- (8.3,7.78619);
\draw [c] (8.3,7.78619) -- (8.48,7.78619);
\draw [c] (8.66,7.54731) -- (8.66,7.62239);
\draw [c] (8.66,7.62239) -- (8.66,7.68701);
\draw [c] (8.48,7.62239) -- (8.66,7.62239);
\draw [c] (8.66,7.62239) -- (8.84,7.62239);
\draw [c] (9.02,7.36309) -- (9.02,7.45273);
\draw [c] (9.02,7.45273) -- (9.02,7.52785);
\draw [c] (8.84,7.45273) -- (9.02,7.45273);
\draw [c] (9.02,7.45273) -- (9.2,7.45273);
\draw [c] (9.38,7.0885) -- (9.38,7.20013);
\draw [c] (9.38,7.20013) -- (9.38,7.29008);
\draw [c] (9.2,7.20013) -- (9.38,7.20013);
\draw [c] (9.38,7.20013) -- (9.56,7.20013);
\draw [c] (9.74,6.91876) -- (9.74,7.04754);
\draw [c] (9.74,7.04754) -- (9.74,7.14829);
\draw [c] (9.56,7.04754) -- (9.74,7.04754);
\draw [c] (9.74,7.04754) -- (9.92,7.04754);
\draw [c] (10.1,7.12755) -- (10.1,7.23707);
\draw [c] (10.1,7.23707) -- (10.1,7.32565);
\draw [c] (9.92,7.23707) -- (10.1,7.23707);
\draw [c] (10.1,7.23707) -- (10.28,7.23707);
\draw [c] (10.46,6.66463) -- (10.46,6.82994);
\draw [c] (10.46,6.82994) -- (10.46,6.95166);
\draw [c] (10.28,6.82994) -- (10.46,6.82994);
\draw [c] (10.46,6.82994) -- (10.64,6.82994);
\draw [c] (10.82,6.36403) -- (10.82,6.5756);
\draw [c] (10.82,6.5756) -- (10.82,6.72048);
\draw [c] (10.64,6.5756) -- (10.82,6.5756);
\draw [c] (10.82,6.5756) -- (11,6.5756);
\draw [c] (11.18,6.15059) -- (11.18,6.40933);
\draw [c] (11.18,6.40933) -- (11.18,6.5746);
\draw [c] (11,6.40933) -- (11.18,6.40933);
\draw [c] (11.18,6.40933) -- (11.36,6.40933);
\draw [c] (11.54,5.85615) -- (11.54,5.86259);
\draw [c] (11.54,5.86259) -- (11.54,5.86893);
\draw [c] (11.36,5.86259) -- (11.54,5.86259);
\draw [c] (11.54,5.86259) -- (11.72,5.86259);
\draw [c] (11.9,5.9951) -- (11.9,6.32661);
\draw [c] (11.9,6.32661) -- (11.9,6.51819);
\draw [c] (11.72,6.32661) -- (11.9,6.32661);
\draw [c] (11.9,6.32661) -- (12.08,6.32661);
\draw [c] (12.26,5.59034) -- (12.26,5.59891);
\draw [c] (12.26,5.59891) -- (12.26,5.60732);
\draw [c] (12.08,5.59891) -- (12.26,5.59891);
\draw [c] (12.26,5.59891) -- (12.44,5.59891);
\draw [c] (12.62,5.45899) -- (12.62,6.01823);
\draw [c] (12.62,6.01823) -- (12.62,6.26488);
\draw [c] (12.44,6.01823) -- (12.62,6.01823);
\draw [c] (12.62,6.01823) -- (12.8,6.01823);
\draw [c] (12.98,5.29462) -- (12.98,5.30639);
\draw [c] (12.98,5.30639) -- (12.98,5.31787);
\draw [c] (12.8,5.30639) -- (12.98,5.30639);
\draw [c] (12.98,5.30639) -- (13.16,5.30639);
\draw [c] (13.34,5.18751) -- (13.34,5.95192);
\draw [c] (13.34,5.95192) -- (13.34,6.22702);
\draw [c] (13.16,5.95192) -- (13.34,5.95192);
\draw [c] (13.34,5.95192) -- (13.52,5.95192);
\draw [c] (13.7,5.01905) -- (13.7,5.03488);
\draw [c] (13.7,5.03488) -- (13.7,5.05019);
\draw [c] (13.52,5.03488) -- (13.7,5.03488);
\draw [c] (13.7,5.03488) -- (13.88,5.03488);
\draw [c] (14.06,4.87509) -- (14.06,4.89356);
\draw [c] (14.06,4.89356) -- (14.06,4.91134);
\draw [c] (13.88,4.89356) -- (14.06,4.89356);
\draw [c] (14.06,4.89356) -- (14.24,4.89356);
\draw [c] (14.42,4.70001) -- (14.42,4.72231);
\draw [c] (14.42,4.72231) -- (14.42,4.7436);
\draw [c] (14.24,4.72231) -- (14.42,4.72231);
\draw [c] (14.42,4.72231) -- (14.6,4.72231);
\draw [c] (14.78,4.5542) -- (14.78,4.58029);
\draw [c] (14.78,4.58029) -- (14.78,4.60499);
\draw [c] (14.6,4.58029) -- (14.78,4.58029);
\draw [c] (14.78,4.58029) -- (14.96,4.58029);
\draw [c] (15.14,4.36384) -- (15.14,4.39585);
\draw [c] (15.14,4.39585) -- (15.14,4.4258);
\draw [c] (14.96,4.39585) -- (15.14,4.39585);
\draw [c] (15.14,4.39585) -- (15.32,4.39585);
\draw [c] (15.5,4.20891) -- (15.5,4.24672);
\draw [c] (15.5,4.24672) -- (15.5,4.28169);
\draw [c] (15.32,4.24672) -- (15.5,4.24672);
\draw [c] (15.5,4.24672) -- (15.68,4.24672);
\draw [c] (15.86,4.06897) -- (15.86,4.07759);
\draw [c] (15.86,4.07759) -- (15.86,4.11923);
\draw [c] (15.68,4.07759) -- (15.86,4.07759);
\draw [c] (15.86,4.07759) -- (16.04,4.07759);
\colorlet{c}{natcomp};
\draw [c] (2.18,13.2779) -- (2.18,13.2806);
\draw [c] (2.18,13.2806) -- (2.18,13.2833);
\draw [c] (2,13.2806) -- (2.18,13.2806);
\draw [c] (2.18,13.2806) -- (2.36,13.2806);
\draw [c] (2.54,12.3032) -- (2.54,12.3109);
\draw [c] (2.54,12.3109) -- (2.54,12.3184);
\draw [c] (2.36,12.3109) -- (2.54,12.3109);
\draw [c] (2.54,12.3109) -- (2.72,12.3109);
\draw [c] (2.9,11.6399) -- (2.9,11.6555);
\draw [c] (2.9,11.6555) -- (2.9,11.6707);
\draw [c] (2.72,11.6555) -- (2.9,11.6555);
\draw [c] (2.9,11.6555) -- (3.08,11.6555);
\draw [c] (3.26,11.1765) -- (3.26,11.2023);
\draw [c] (3.26,11.2023) -- (3.26,11.2267);
\draw [c] (3.08,11.2023) -- (3.26,11.2023);
\draw [c] (3.26,11.2023) -- (3.44,11.2023);
\draw [c] (3.62,10.7748) -- (3.62,10.8145);
\draw [c] (3.62,10.8145) -- (3.62,10.8511);
\draw [c] (3.44,10.8145) -- (3.62,10.8145);
\draw [c] (3.62,10.8145) -- (3.8,10.8145);
\draw [c] (3.98,10.267) -- (3.98,10.2692);
\draw [c] (3.98,10.2692) -- (3.98,10.2715);
\draw [c] (3.8,10.2692) -- (3.98,10.2692);
\draw [c] (3.98,10.2692) -- (4.16,10.2692);
\draw [c] (4.34,10.0219) -- (4.34,10.0248);
\draw [c] (4.34,10.0248) -- (4.34,10.0278);
\draw [c] (4.16,10.0248) -- (4.34,10.0248);
\draw [c] (4.34,10.0248) -- (4.52,10.0248);
\draw [c] (4.7,9.77769) -- (4.7,9.78156);
\draw [c] (4.7,9.78156) -- (4.7,9.78539);
\draw [c] (4.52,9.78156) -- (4.7,9.78156);
\draw [c] (4.7,9.78156) -- (4.88,9.78156);
\draw [c] (5.06,9.54628) -- (5.06,9.55123);
\draw [c] (5.06,9.55123) -- (5.06,9.55614);
\draw [c] (4.88,9.55123) -- (5.06,9.55123);
\draw [c] (5.06,9.55123) -- (5.24,9.55123);
\draw [c] (5.42,9.31885) -- (5.42,9.32518);
\draw [c] (5.42,9.32518) -- (5.42,9.33143);
\draw [c] (5.24,9.32518) -- (5.42,9.32518);
\draw [c] (5.42,9.32518) -- (5.6,9.32518);
\draw [c] (5.78,9.10524) -- (5.78,9.1132);
\draw [c] (5.78,9.1132) -- (5.78,9.12104);
\draw [c] (5.6,9.1132) -- (5.78,9.1132);
\draw [c] (5.78,9.1132) -- (5.96,9.1132);
\draw [c] (6.14,8.89208) -- (6.14,8.9021);
\draw [c] (6.14,8.9021) -- (6.14,8.9119);
\draw [c] (5.96,8.9021) -- (6.14,8.9021);
\draw [c] (6.14,8.9021) -- (6.32,8.9021);
\draw [c] (6.5,8.67277) -- (6.5,8.68545);
\draw [c] (6.5,8.68545) -- (6.5,8.6978);
\draw [c] (6.32,8.68545) -- (6.5,8.68545);
\draw [c] (6.5,8.68545) -- (6.68,8.68545);
\draw [c] (6.86,8.48573) -- (6.86,8.50124);
\draw [c] (6.86,8.50124) -- (6.86,8.51624);
\draw [c] (6.68,8.50124) -- (6.86,8.50124);
\draw [c] (6.86,8.50124) -- (7.04,8.50124);
\draw [c] (7.22,8.32328) -- (7.22,8.34174);
\draw [c] (7.22,8.34174) -- (7.22,8.3595);
\draw [c] (7.04,8.34174) -- (7.22,8.34174);
\draw [c] (7.22,8.34174) -- (7.4,8.34174);
\draw [c] (7.58,8.1128) -- (7.58,8.13595);
\draw [c] (7.58,8.13595) -- (7.58,8.158);
\draw [c] (7.4,8.13595) -- (7.58,8.13595);
\draw [c] (7.58,8.13595) -- (7.76,8.13595);
\draw [c] (7.94,7.89863) -- (7.94,7.92778);
\draw [c] (7.94,7.92778) -- (7.94,7.9552);
\draw [c] (7.76,7.92778) -- (7.94,7.92778);
\draw [c] (7.94,7.92778) -- (8.12,7.92778);
\draw [c] (8.3,7.73014) -- (8.3,7.76508);
\draw [c] (8.3,7.76508) -- (8.3,7.79756);
\draw [c] (8.12,7.76508) -- (8.3,7.76508);
\draw [c] (8.3,7.76508) -- (8.48,7.76508);
\draw [c] (8.66,7.54719) -- (8.66,7.58971);
\draw [c] (8.66,7.58971) -- (8.66,7.62866);
\draw [c] (8.48,7.58971) -- (8.66,7.58971);
\draw [c] (8.66,7.58971) -- (8.84,7.58971);
\draw [c] (9.02,7.32931) -- (9.02,7.38304);
\draw [c] (9.02,7.38304) -- (9.02,7.4312);
\draw [c] (8.84,7.38304) -- (9.02,7.38304);
\draw [c] (9.02,7.38304) -- (9.2,7.38304);
\draw [c] (9.38,7.2022) -- (9.38,7.26378);
\draw [c] (9.38,7.26378) -- (9.38,7.31816);
\draw [c] (9.2,7.26378) -- (9.38,7.26378);
\draw [c] (9.38,7.26378) -- (9.56,7.26378);
\draw [c] (9.74,7.06068) -- (9.74,7.13237);
\draw [c] (9.74,7.13237) -- (9.74,7.19447);
\draw [c] (9.56,7.13237) -- (9.74,7.13237);
\draw [c] (9.74,7.13237) -- (9.92,7.13237);
\draw [c] (10.1,6.71563) -- (10.1,6.81941);
\draw [c] (10.1,6.81941) -- (10.1,6.9042);
\draw [c] (9.92,6.81941) -- (10.1,6.81941);
\draw [c] (10.1,6.81941) -- (10.28,6.81941);
\draw [c] (10.46,6.47806) -- (10.46,6.61185);
\draw [c] (10.46,6.61185) -- (10.46,6.71563);
\draw [c] (10.28,6.61185) -- (10.46,6.61185);
\draw [c] (10.46,6.61185) -- (10.64,6.61185);
\draw [c] (10.82,6.40513) -- (10.82,6.54975);
\draw [c] (10.82,6.54975) -- (10.82,6.6599);
\draw [c] (10.64,6.54975) -- (10.82,6.54975);
\draw [c] (10.82,6.54975) -- (11,6.54975);
\draw [c] (11.18,5.91169) -- (11.18,6.1557);
\draw [c] (11.18,6.1557) -- (11.18,6.31491);
\draw [c] (11,6.1557) -- (11.18,6.1557);
\draw [c] (11.18,6.1557) -- (11.36,6.1557);
\draw [c] (11.54,6.0866) -- (11.54,6.28949);
\draw [c] (11.54,6.28949) -- (11.54,6.43028);
\draw [c] (11.36,6.28949) -- (11.54,6.28949);
\draw [c] (11.54,6.28949) -- (11.72,6.28949);
\draw [c] (11.9,5.64477) -- (11.9,5.96713);
\draw [c] (11.9,5.96713) -- (11.9,6.1557);
\draw [c] (11.72,5.96713) -- (11.9,5.96713);
\draw [c] (11.9,5.96713) -- (12.08,5.96713);
\draw [c] (12.26,4.06897) -- (12.26,5.32241);
\draw [c] (12.26,5.32241) -- (12.26,5.64477);
\draw [c] (12.08,5.32241) -- (12.26,5.32241);
\draw [c] (12.26,5.32241) -- (12.44,5.32241);
\draw [c] (12.62,5.64477) -- (12.62,5.96713);
\draw [c] (12.62,5.96713) -- (12.62,6.1557);
\draw [c] (12.44,5.96713) -- (12.62,5.96713);
\draw [c] (12.62,5.96713) -- (12.8,5.96713);
\draw [c] (12.98,5.43281) -- (12.98,5.83334);
\draw [c] (12.98,5.83334) -- (12.98,6.04529);
\draw [c] (12.8,5.83334) -- (12.98,5.83334);
\draw [c] (12.98,5.83334) -- (13.16,5.83334);
\draw [c] (14.42,4.06897) -- (14.42,5.32241);
\draw [c] (14.42,5.32241) -- (14.42,5.64477);
\draw [c] (14.24,5.32241) -- (14.42,5.32241);
\draw [c] (14.42,5.32241) -- (14.6,5.32241);
\draw [c] (15.14,4.06897) -- (15.14,5.32241);
\draw [c] (15.14,5.32241) -- (15.14,5.64477);
\draw [c] (14.96,5.32241) -- (15.14,5.32241);
\draw [c] (15.14,5.32241) -- (15.32,5.32241);

\draw [anchor=base west] (15.5029,12.7215) node[ rotate=0]{CalcHEP};
\colorlet{c}{kugray!20};
\draw [c, fill=c] (14.245,12.6557) -- (15.2809,12.6557) -- (15.2809,13.0244) -- (14.245,13.0244);
\colorlet{c}{kugray};
\draw [c] (14.245,12.84) -- (15.2809,12.84);
\draw [anchor=base west] (15.5029,12.1947) node[ rotate=0]{pythia8};
\colorlet{c}{natgreen};
\draw [c] (14.245,12.3132) -- (15.2809,12.3132);
\draw [anchor=base west] (15.5029,11.6679) node[ rotate=0]{MadGraph};
\colorlet{c}{natcomp};
\draw [c] (14.245,11.7864) -- (15.2809,11.7864);

%\definecolor{c}{rgb}{1,1,1};
%\draw [color=c, fill=c] (0,0) rectangle (20,4.06897);
%\draw [color=c, fill=c] (2,0.406897) rectangle (20,4.06897);

\definecolor{c}{rgb}{1,1,1};
\draw [color=c, fill=c] (2,0.406897) rectangle (20,4.06897);
\definecolor{c}{rgb}{0,0,0};
\draw [c] (2,0.406897) -- (2,4.06897) -- (20,4.06897) -- (20,0.406897) -- (2,0.406897);
\colorlet{c}{kugray!20};
\draw [color=c, fill=c] (2,2.13133) rectangle (2.36,2.15724);
\draw [color=c, fill=c] (2.36,2.10729) rectangle (2.72,2.18127);
\draw [color=c, fill=c] (2.72,2.07143) rectangle (3.08,2.21713);
\draw [color=c, fill=c] (3.08,2.02265) rectangle (3.44,2.26591);
\draw [color=c, fill=c] (3.44,1.95795) rectangle (3.8,2.33062);
\draw [color=c, fill=c] (3.8,1.88955) rectangle (4.16,2.39902);
\draw [color=c, fill=c] (4.16,1.76085) rectangle (4.52,2.52771);
\draw [color=c, fill=c] (4.52,1.61822) rectangle (4.88,2.67034);
\draw [color=c, fill=c] (4.88,1.45839) rectangle (5.24,2.83018);
\draw [color=c, fill=c] (5.24,1.68213) rectangle (5.6,2.60644);
\draw [color=c, fill=c] (5.6,2.12031) rectangle (5.96,2.16826);
\draw [color=c, fill=c] (5.96,2.11423) rectangle (6.32,2.17434);
\draw [color=c, fill=c] (6.32,2.10615) rectangle (6.68,2.18242);
\draw [color=c, fill=c] (6.68,2.09728) rectangle (7.04,2.19129);
\draw [color=c, fill=c] (7.04,2.08664) rectangle (7.4,2.20192);
\draw [color=c, fill=c] (7.4,2.0746) rectangle (7.76,2.21396);
\draw [color=c, fill=c] (7.76,2.05992) rectangle (8.12,2.22865);
\draw [color=c, fill=c] (8.12,2.0445) rectangle (8.48,2.24407);
\draw [color=c, fill=c] (8.48,2.02159) rectangle (8.84,2.26698);
\draw [color=c, fill=c] (8.84,1.98438) rectangle (9.2,2.30418);
\draw [color=c, fill=c] (9.2,1.94545) rectangle (9.56,2.34312);
\draw [color=c, fill=c] (9.56,1.88596) rectangle (9.92,2.40261);
\draw [color=c, fill=c] (9.92,1.87102) rectangle (10.28,2.41755);
\draw [color=c, fill=c] (10.28,1.8096) rectangle (10.64,2.47897);
\draw [color=c, fill=c] (10.64,1.71048) rectangle (11,2.57808);
\draw [color=c, fill=c] (11,1.64668) rectangle (11.36,2.64189);
\draw [color=c, fill=c] (11.36,1.50468) rectangle (11.72,2.78389);
\draw [color=c, fill=c] (11.72,1.49031) rectangle (12.08,2.79826);
\draw [color=c, fill=c] (12.08,1.29353) rectangle (12.44,2.99504);
\draw [color=c, fill=c] (12.44,1.25879) rectangle (12.8,3.02977);
\draw [color=c, fill=c] (12.8,1.21942) rectangle (13.16,3.06915);
\draw [color=c, fill=c] (13.16,0.610569) rectangle (13.52,3.678);
\draw [color=c, fill=c] (13.52,1.21942) rectangle (13.88,3.06915);
\draw [color=c, fill=c] (13.88,0.406897) rectangle (14.24,4.06897);
\draw [color=c, fill=c] (14.24,0.406897) rectangle (14.6,4.06897);
\draw [color=c, fill=c] (14.6,0.406897) rectangle (14.96,4.06897);
\draw [color=c, fill=c] (14.96,0.406897) rectangle (15.32,4.06897);
\draw [color=c, fill=c] (15.32,0.406897) rectangle (15.68,4.06897);
\draw [color=c, fill=c] (15.68,0.406897) rectangle (16.04,4.06897);
\draw [color=c, fill=c] (16.04,0.406897) rectangle (16.4,4.06897);
\draw [color=c, fill=c] (16.4,2.14428) rectangle (16.76,2.14428);
\draw [color=c, fill=c] (16.76,0.406897) rectangle (17.12,4.06897);
\draw [color=c, fill=c] (17.12,2.14428) rectangle (17.48,2.14428);
\draw [color=c, fill=c] (17.48,2.14428) rectangle (17.84,2.14428);
\draw [color=c, fill=c] (17.84,2.14428) rectangle (18.2,2.14428);
\draw [color=c, fill=c] (18.2,2.14428) rectangle (18.56,2.14428);
\draw [color=c, fill=c] (18.56,2.14428) rectangle (18.92,2.14428);
\draw [color=c, fill=c] (18.92,2.14428) rectangle (19.28,2.14428);
\draw [color=c, fill=c] (19.28,2.14428) rectangle (19.64,2.14428);
\draw [color=c, fill=c] (19.64,2.14428) rectangle (20,2.14428);
\definecolor{c}{rgb}{0,0,0};
\draw [c] (2,0.406897) -- (20,0.406897);
\draw [anchor= east] (20,0.179034) +(0,-2em) node[ rotate=0]{$M_{\gamma\gamma}\text{ [GeV]}$};
\draw [c] (3.46939,0.516759) -- (3.46939,0.406897);
\draw [c] (3.83673,0.461828) -- (3.83673,0.406897);
\draw [c] (4.20408,0.461828) -- (4.20408,0.406897);
\draw [c] (4.57143,0.461828) -- (4.57143,0.406897);
\draw [c] (4.93878,0.461828) -- (4.93878,0.406897);
\draw [c] (5.30612,0.516759) -- (5.30612,0.406897);
\draw [c] (5.67347,0.461828) -- (5.67347,0.406897);
\draw [c] (6.04082,0.461828) -- (6.04082,0.406897);
\draw [c] (6.40816,0.461828) -- (6.40816,0.406897);
\draw [c] (6.77551,0.461828) -- (6.77551,0.406897);
\draw [c] (7.14286,0.516759) -- (7.14286,0.406897);
\draw [c] (7.5102,0.461828) -- (7.5102,0.406897);
\draw [c] (7.87755,0.461828) -- (7.87755,0.406897);
\draw [c] (8.2449,0.461828) -- (8.2449,0.406897);
\draw [c] (8.61224,0.461828) -- (8.61224,0.406897);
\draw [c] (8.97959,0.516759) -- (8.97959,0.406897);
\draw [c] (9.34694,0.461828) -- (9.34694,0.406897);
\draw [c] (9.71429,0.461828) -- (9.71429,0.406897);
\draw [c] (10.0816,0.461828) -- (10.0816,0.406897);
\draw [c] (10.449,0.461828) -- (10.449,0.406897);
\draw [c] (10.8163,0.516759) -- (10.8163,0.406897);
\draw [c] (11.1837,0.461828) -- (11.1837,0.406897);
\draw [c] (11.551,0.461828) -- (11.551,0.406897);
\draw [c] (11.9184,0.461828) -- (11.9184,0.406897);
\draw [c] (12.2857,0.461828) -- (12.2857,0.406897);
\draw [c] (12.6531,0.516759) -- (12.6531,0.406897);
\draw [c] (13.0204,0.461828) -- (13.0204,0.406897);
\draw [c] (13.3878,0.461828) -- (13.3878,0.406897);
\draw [c] (13.7551,0.461828) -- (13.7551,0.406897);
\draw [c] (14.1224,0.461828) -- (14.1224,0.406897);
\draw [c] (14.4898,0.516759) -- (14.4898,0.406897);
\draw [c] (14.8571,0.461828) -- (14.8571,0.406897);
\draw [c] (15.2245,0.461828) -- (15.2245,0.406897);
\draw [c] (15.5918,0.461828) -- (15.5918,0.406897);
\draw [c] (15.9592,0.461828) -- (15.9592,0.406897);
\draw [c] (16.3265,0.516759) -- (16.3265,0.406897);
\draw [c] (16.6939,0.461828) -- (16.6939,0.406897);
\draw [c] (17.0612,0.461828) -- (17.0612,0.406897);
\draw [c] (17.4286,0.461828) -- (17.4286,0.406897);
\draw [c] (17.7959,0.461828) -- (17.7959,0.406897);
\draw [c] (18.1633,0.516759) -- (18.1633,0.406897);
\draw [c] (18.5306,0.461828) -- (18.5306,0.406897);
\draw [c] (18.898,0.461828) -- (18.898,0.406897);
\draw [c] (19.2653,0.461828) -- (19.2653,0.406897);
\draw [c] (19.6327,0.461828) -- (19.6327,0.406897);
\draw [c] (20,0.516759) -- (20,0.406897);
\draw [c] (3.46939,0.516759) -- (3.46939,0.406897);
\draw [c] (3.10204,0.461828) -- (3.10204,0.406897);
\draw [c] (2.73469,0.461828) -- (2.73469,0.406897);
\draw [c] (2.36735,0.461828) -- (2.36735,0.406897);
\draw [c] (2,0.461828) -- (2,0.406897);
\draw [anchor=base] (3.46939,0.272621) +(0,-.9em) node[ rotate=0]{500};
\draw [anchor=base] (5.30612,0.272621) +(0,-.9em) node[ rotate=0]{1000};
\draw [anchor=base] (7.14286,0.272621) +(0,-.9em) node[ rotate=0]{1500};
\draw [anchor=base] (8.97959,0.272621) +(0,-.9em) node[ rotate=0]{2000};
\draw [anchor=base] (10.8163,0.272621) +(0,-.9em) node[ rotate=0]{2500};
\draw [anchor=base] (12.6531,0.272621) +(0,-.9em) node[ rotate=0]{3000};
\draw [anchor=base] (14.4898,0.272621) +(0,-.9em) node[ rotate=0]{3500};
\draw [anchor=base] (16.3265,0.272621) +(0,-.9em) node[ rotate=0]{4000};
\draw [anchor=base] (18.1633,0.272621) +(0,-.9em) node[ rotate=0]{4500};
\draw [anchor=base] (20,0.272621)  +(0,-.9em) node[ rotate=0]{5000};
\draw [c] (2,0.406897) -- (2,4.06897);
\draw [anchor= east] (0.1,4.06897) node[ rotate=90]{Ratio};
\draw [c] (2.54,1.05978) -- (2,1.05978);
\draw [c] (2.27,1.27668) -- (2,1.27668);
\draw [c] (2.27,1.49358) -- (2,1.49358);
\draw [c] (2.27,1.71048) -- (2,1.71048);
\draw [c] (2.27,1.92738) -- (2,1.92738);
\draw [c] (2.54,2.14428) -- (2,2.14428);
\draw [c] (2.27,2.36118) -- (2,2.36118);
\draw [c] (2.27,2.57808) -- (2,2.57808);
\draw [c] (2.27,2.79498) -- (2,2.79498);
\draw [c] (2.27,3.01188) -- (2,3.01188);
\draw [c] (2.54,3.22878) -- (2,3.22878);
\draw [c] (2.54,1.05978) -- (2,1.05978);
\draw [c] (2.27,0.842883) -- (2,0.842883);
\draw [c] (2.27,0.625983) -- (2,0.625983);
\draw [c] (2.27,0.409083) -- (2,0.409083);
\draw [c] (2.54,3.22878) -- (2,3.22878);
\draw [c] (2.27,3.44568) -- (2,3.44568);
\draw [c] (2.27,3.66258) -- (2,3.66258);
\draw [c] (2.27,3.87948) -- (2,3.87948);
\draw [anchor= east] (1.9,1.05978) node[ rotate=0]{0.5};
\draw [anchor= east] (1.9,2.14428) node[ rotate=0]{1.0};
\draw [anchor= east] (1.9,3.22878) node[ rotate=0]{1.5};
\colorlet{c}{natgreen};
\draw [c] (2.18,2.0723) -- (2.18,2.09531);
\draw [c] (2.18,2.09531) -- (2.18,2.11833);
\draw [c] (2,2.09531) -- (2.18,2.09531);
\draw [c] (2.18,2.09531) -- (2.36,2.09531);
\draw [c] (2.54,2.14221) -- (2.54,2.20994);
\draw [c] (2.54,2.20994) -- (2.54,2.27768);
\draw [c] (2.36,2.20994) -- (2.54,2.20994);
\draw [c] (2.54,2.20994) -- (2.72,2.20994);
\draw [c] (2.9,2.04835) -- (2.9,2.18074);
\draw [c] (2.9,2.18074) -- (2.9,2.31314);
\draw [c] (2.72,2.18074) -- (2.9,2.18074);
\draw [c] (2.9,2.18074) -- (3.08,2.18074);
\draw [c] (3.26,2.29428) -- (3.26,2.50976);
\draw [c] (3.26,2.50976) -- (3.26,2.72524);
\draw [c] (3.08,2.50976) -- (3.26,2.50976);
\draw [c] (3.26,2.50976) -- (3.44,2.50976);
\draw [c] (3.62,1.69498) -- (3.62,1.90617);
\draw [c] (3.62,1.90617) -- (3.62,2.11735);
\draw [c] (3.44,1.90617) -- (3.62,1.90617);
\draw [c] (3.62,1.90617) -- (3.8,1.90617);
\draw [c] (3.98,1.53704) -- (3.98,1.77377);
\draw [c] (3.98,1.77377) -- (3.98,2.0105);
\draw [c] (3.8,1.77377) -- (3.98,1.77377);
\draw [c] (3.98,1.77377) -- (4.16,1.77377);
\draw [c] (4.34,2.50693) -- (4.34,3.1379);
\draw [c] (4.34,3.1379) -- (4.34,3.76887);
\draw [c] (4.16,3.1379) -- (4.34,3.1379);
\draw [c] (4.34,3.1379) -- (4.52,3.1379);
\draw [c] (4.7,1.71696) -- (4.7,2.07899);
\draw [c] (4.7,2.07899) -- (4.7,2.44102);
\draw [c] (4.52,2.07899) -- (4.7,2.07899);
\draw [c] (4.7,2.07899) -- (4.88,2.07899);
\draw [c] (5.06,1.67434) -- (5.06,2.16577);
\draw [c] (5.06,2.16577) -- (5.06,2.65721);
\draw [c] (4.88,2.16577) -- (5.06,2.16577);
\draw [c] (5.06,2.16577) -- (5.24,2.16577);
\draw [c] (5.42,1.43562) -- (5.42,1.69815);
\draw [c] (5.42,1.69815) -- (5.42,1.96069);
\draw [c] (5.24,1.69815) -- (5.42,1.69815);
\draw [c] (5.42,1.69815) -- (5.6,1.69815);
\draw [c] (5.78,2.19321) -- (5.78,2.25967);
\draw [c] (5.78,2.25967) -- (5.78,2.32614);
\draw [c] (5.6,2.25967) -- (5.78,2.25967);
\draw [c] (5.78,2.25967) -- (5.96,2.25967);
\draw [c] (6.14,2.04097) -- (6.14,2.12156);
\draw [c] (6.14,2.12156) -- (6.14,2.20215);
\draw [c] (5.96,2.12156) -- (6.14,2.12156);
\draw [c] (6.14,2.12156) -- (6.32,2.12156);
\draw [c] (6.5,2.09639) -- (6.5,2.20066);
\draw [c] (6.5,2.20066) -- (6.5,2.30493);
\draw [c] (6.32,2.20066) -- (6.5,2.20066);
\draw [c] (6.5,2.20066) -- (6.68,2.20066);
\draw [c] (6.86,2.26209) -- (6.86,2.39655);
\draw [c] (6.86,2.39655) -- (6.86,2.53101);
\draw [c] (6.68,2.39655) -- (6.86,2.39655);
\draw [c] (6.86,2.39655) -- (7.04,2.39655);
\draw [c] (7.22,2.14778) -- (7.22,2.30946);
\draw [c] (7.22,2.30946) -- (7.22,2.47113);
\draw [c] (7.04,2.30946) -- (7.22,2.30946);
\draw [c] (7.22,2.30946) -- (7.4,2.30946);
\draw [c] (7.58,1.94081) -- (7.58,2.12796);
\draw [c] (7.58,2.12796) -- (7.58,2.31511);
\draw [c] (7.4,2.12796) -- (7.58,2.12796);
\draw [c] (7.58,2.12796) -- (7.76,2.12796);
\draw [c] (7.94,1.86226) -- (7.94,2.08651);
\draw [c] (7.94,2.08651) -- (7.94,2.31077);
\draw [c] (7.76,2.08651) -- (7.94,2.08651);
\draw [c] (7.94,2.08651) -- (8.12,2.08651);
\draw [c] (8.3,1.708) -- (8.3,1.965);
\draw [c] (8.3,1.965) -- (8.3,2.222);
\draw [c] (8.12,1.965) -- (8.3,1.965);
\draw [c] (8.3,1.965) -- (8.48,1.965);
\draw [c] (8.66,1.76412) -- (8.66,2.0906);
\draw [c] (8.66,2.0906) -- (8.66,2.41709);
\draw [c] (8.48,2.0906) -- (8.66,2.0906);
\draw [c] (8.66,2.0906) -- (8.84,2.0906);
\draw [c] (9.02,2.01354) -- (9.02,2.46974);
\draw [c] (9.02,2.46974) -- (9.02,2.92594);
\draw [c] (8.84,2.46974) -- (9.02,2.46974);
\draw [c] (9.02,2.46974) -- (9.2,2.46974);
\draw [c] (9.38,1.71618) -- (9.38,2.21585);
\draw [c] (9.38,2.21585) -- (9.38,2.71553);
\draw [c] (9.2,2.21585) -- (9.38,2.21585);
\draw [c] (9.38,2.21585) -- (9.56,2.21585);
\draw [c] (9.74,2.0017) -- (9.74,2.69943);
\draw [c] (9.74,2.69943) -- (9.74,3.39715);
\draw [c] (9.56,2.69943) -- (9.74,2.69943);
\draw [c] (9.74,2.69943) -- (9.92,2.69943);
\draw [c] (10.46,1.9273) -- (10.46,2.83926);
\draw [c] (10.46,2.83926) -- (10.46,3.75122);
\draw [c] (10.28,2.83926) -- (10.46,2.83926);
\draw [c] (10.46,2.83926) -- (10.64,2.83926);
\draw [c] (10.82,1.66859) -- (10.82,2.75991);
\draw [c] (10.82,2.75991) -- (10.82,3.85123);
\draw [c] (10.64,2.75991) -- (10.82,2.75991);
\draw [c] (10.82,2.75991) -- (11,2.75991);
\draw [c] (11.18,1.36808) -- (11.18,2.5379);
\draw [c] (11.18,2.5379) -- (11.18,3.70772);
\draw [c] (11,2.5379) -- (11.18,2.5379);
\draw [c] (11.18,2.5379) -- (11.36,2.5379);
\draw [c] (11.54,1.00892) -- (11.54,1.28198);
\draw [c] (11.54,1.28198) -- (11.54,1.55503);
\draw [c] (11.36,1.28198) -- (11.54,1.28198);
\draw [c] (11.54,1.28198) -- (11.72,1.28198);
\draw [c] (11.9,1.63319) -- (11.9,3.68039);
\draw [c] (11.9,3.68039) -- (11.9,4.06897);
\draw [c] (11.72,3.68039) -- (11.9,3.68039);
\draw [c] (11.9,3.68039) -- (12.08,3.68039);
\draw [c] (12.26,0.922168) -- (12.26,1.28667);
\draw [c] (12.26,1.28667) -- (12.26,1.65117);
\draw [c] (12.08,1.28667) -- (12.26,1.28667);
\draw [c] (12.26,1.28667) -- (12.44,1.28667);
\draw [c] (12.62,0.82657) -- (12.62,3.47523);
\draw [c] (12.62,3.47523) -- (12.62,4.06897);
\draw [c] (12.44,3.47523) -- (12.62,3.47523);
\draw [c] (12.62,3.47523) -- (12.8,3.47523);
\draw [c] (12.98,0.551562) -- (12.98,0.801543);
\draw [c] (12.98,0.801543) -- (12.98,1.05152);
\draw [c] (12.8,0.801543) -- (12.98,0.801543);
\draw [c] (12.98,0.801543) -- (13.16,0.801543);
\draw [c] (13.7,0.406897) -- (13.7,0.436151);
\draw [c] (13.7,0.436151) -- (13.7,0.575962);
\draw [c] (13.52,0.436151) -- (13.7,0.436151);
\draw [c] (13.7,0.436151) -- (13.88,0.436151);
\draw [c] (14.06,0.406897) -- (14.06,3.71642);
\draw [c] (14.06,3.71642) -- (14.06,4.06897);
\draw [c] (13.88,3.71642) -- (14.06,3.71642);
\draw [c] (14.06,3.71642) -- (14.24,3.71642);
\draw [c] (14.42,0.406897) -- (14.42,2.56399);
\draw [c] (14.42,2.56399) -- (14.42,4.06897);
\draw [c] (14.24,2.56399) -- (14.42,2.56399);
\draw [c] (14.42,2.56399) -- (14.6,2.56399);
\draw [c] (14.78,0.406897) -- (14.78,1.88275);
\draw [c] (14.78,1.88275) -- (14.78,3.79305);
\draw [c] (14.6,1.88275) -- (14.78,1.88275);
\draw [c] (14.78,1.88275) -- (14.96,1.88275);
\draw [c] (15.14,0.406897) -- (15.14,0.616783);
\draw [c] (15.14,0.616783) -- (15.14,1.07239);
\draw [c] (14.96,0.616783) -- (15.14,0.616783);
\draw [c] (15.14,0.616783) -- (15.32,0.616783);
\draw [c] (15.5,0.406897) -- (15.5,0.906309);
\draw [c] (15.5,0.906309) -- (15.5,1.84017);
\draw [c] (15.32,0.906309) -- (15.5,0.906309);
\draw [c] (15.5,0.906309) -- (15.68,0.906309);
\draw [c] (15.86,0.406897) -- (15.86,0.62246);
\draw [c] (15.86,0.62246) -- (15.86,1.27247);
\draw [c] (15.68,0.62246) -- (15.86,0.62246);
\draw [c] (15.86,0.62246) -- (16.04,0.62246);
\draw [c] (16.22,0.406897) -- (16.22,0.50892);
\draw [c] (16.22,0.50892) -- (16.22,1.04539);
\draw [c] (16.04,0.50892) -- (16.22,0.50892);
\draw [c] (16.22,0.50892) -- (16.4,0.50892);
\draw [c] (2.18,2.0723) -- (2.18,2.09531);
\draw [c] (2.18,2.09531) -- (2.18,2.11833);
\draw [c] (2,2.09531) -- (2.18,2.09531);
\draw [c] (2.18,2.09531) -- (2.36,2.09531);
\draw [c] (2.54,2.14221) -- (2.54,2.20994);
\draw [c] (2.54,2.20994) -- (2.54,2.27768);
\draw [c] (2.36,2.20994) -- (2.54,2.20994);
\draw [c] (2.54,2.20994) -- (2.72,2.20994);
\draw [c] (2.9,2.04835) -- (2.9,2.18074);
\draw [c] (2.9,2.18074) -- (2.9,2.31314);
\draw [c] (2.72,2.18074) -- (2.9,2.18074);
\draw [c] (2.9,2.18074) -- (3.08,2.18074);
\draw [c] (3.26,2.29428) -- (3.26,2.50976);
\draw [c] (3.26,2.50976) -- (3.26,2.72524);
\draw [c] (3.08,2.50976) -- (3.26,2.50976);
\draw [c] (3.26,2.50976) -- (3.44,2.50976);
\draw [c] (3.62,1.69498) -- (3.62,1.90617);
\draw [c] (3.62,1.90617) -- (3.62,2.11735);
\draw [c] (3.44,1.90617) -- (3.62,1.90617);
\draw [c] (3.62,1.90617) -- (3.8,1.90617);
\draw [c] (3.98,1.53704) -- (3.98,1.77377);
\draw [c] (3.98,1.77377) -- (3.98,2.0105);
\draw [c] (3.8,1.77377) -- (3.98,1.77377);
\draw [c] (3.98,1.77377) -- (4.16,1.77377);
\draw [c] (4.34,2.50693) -- (4.34,3.1379);
\draw [c] (4.34,3.1379) -- (4.34,3.76887);
\draw [c] (4.16,3.1379) -- (4.34,3.1379);
\draw [c] (4.34,3.1379) -- (4.52,3.1379);
\draw [c] (4.7,1.71696) -- (4.7,2.07899);
\draw [c] (4.7,2.07899) -- (4.7,2.44102);
\draw [c] (4.52,2.07899) -- (4.7,2.07899);
\draw [c] (4.7,2.07899) -- (4.88,2.07899);
\draw [c] (5.06,1.67434) -- (5.06,2.16577);
\draw [c] (5.06,2.16577) -- (5.06,2.65721);
\draw [c] (4.88,2.16577) -- (5.06,2.16577);
\draw [c] (5.06,2.16577) -- (5.24,2.16577);
\draw [c] (5.42,1.43562) -- (5.42,1.69815);
\draw [c] (5.42,1.69815) -- (5.42,1.96069);
\draw [c] (5.24,1.69815) -- (5.42,1.69815);
\draw [c] (5.42,1.69815) -- (5.6,1.69815);
\draw [c] (5.78,2.19321) -- (5.78,2.25967);
\draw [c] (5.78,2.25967) -- (5.78,2.32614);
\draw [c] (5.6,2.25967) -- (5.78,2.25967);
\draw [c] (5.78,2.25967) -- (5.96,2.25967);
\draw [c] (6.14,2.04097) -- (6.14,2.12156);
\draw [c] (6.14,2.12156) -- (6.14,2.20215);
\draw [c] (5.96,2.12156) -- (6.14,2.12156);
\draw [c] (6.14,2.12156) -- (6.32,2.12156);
\draw [c] (6.5,2.09639) -- (6.5,2.20066);
\draw [c] (6.5,2.20066) -- (6.5,2.30493);
\draw [c] (6.32,2.20066) -- (6.5,2.20066);
\draw [c] (6.5,2.20066) -- (6.68,2.20066);
\draw [c] (6.86,2.26209) -- (6.86,2.39655);
\draw [c] (6.86,2.39655) -- (6.86,2.53101);
\draw [c] (6.68,2.39655) -- (6.86,2.39655);
\draw [c] (6.86,2.39655) -- (7.04,2.39655);
\draw [c] (7.22,2.14778) -- (7.22,2.30946);
\draw [c] (7.22,2.30946) -- (7.22,2.47113);
\draw [c] (7.04,2.30946) -- (7.22,2.30946);
\draw [c] (7.22,2.30946) -- (7.4,2.30946);
\draw [c] (7.58,1.94081) -- (7.58,2.12796);
\draw [c] (7.58,2.12796) -- (7.58,2.31511);
\draw [c] (7.4,2.12796) -- (7.58,2.12796);
\draw [c] (7.58,2.12796) -- (7.76,2.12796);
\draw [c] (7.94,1.86226) -- (7.94,2.08651);
\draw [c] (7.94,2.08651) -- (7.94,2.31077);
\draw [c] (7.76,2.08651) -- (7.94,2.08651);
\draw [c] (7.94,2.08651) -- (8.12,2.08651);
\draw [c] (8.3,1.708) -- (8.3,1.965);
\draw [c] (8.3,1.965) -- (8.3,2.222);
\draw [c] (8.12,1.965) -- (8.3,1.965);
\draw [c] (8.3,1.965) -- (8.48,1.965);
\draw [c] (8.66,1.76412) -- (8.66,2.0906);
\draw [c] (8.66,2.0906) -- (8.66,2.41709);
\draw [c] (8.48,2.0906) -- (8.66,2.0906);
\draw [c] (8.66,2.0906) -- (8.84,2.0906);
\draw [c] (9.02,2.01354) -- (9.02,2.46974);
\draw [c] (9.02,2.46974) -- (9.02,2.92594);
\draw [c] (8.84,2.46974) -- (9.02,2.46974);
\draw [c] (9.02,2.46974) -- (9.2,2.46974);
\draw [c] (9.38,1.71618) -- (9.38,2.21585);
\draw [c] (9.38,2.21585) -- (9.38,2.71553);
\draw [c] (9.2,2.21585) -- (9.38,2.21585);
\draw [c] (9.38,2.21585) -- (9.56,2.21585);
\draw [c] (9.74,2.0017) -- (9.74,2.69943);
\draw [c] (9.74,2.69943) -- (9.74,3.39715);
\draw [c] (9.56,2.69943) -- (9.74,2.69943);
\draw [c] (9.74,2.69943) -- (9.92,2.69943);
\draw [c] (10.1,3.51291) -- (10.1,4.06897);
\draw [c] (10.46,1.9273) -- (10.46,2.83926);
\draw [c] (10.46,2.83926) -- (10.46,3.75122);
\draw [c] (10.28,2.83926) -- (10.46,2.83926);
\draw [c] (10.46,2.83926) -- (10.64,2.83926);
\draw [c] (10.82,1.66859) -- (10.82,2.75991);
\draw [c] (10.82,2.75991) -- (10.82,3.85123);
\draw [c] (10.64,2.75991) -- (10.82,2.75991);
\draw [c] (10.82,2.75991) -- (11,2.75991);
\draw [c] (11.18,1.36808) -- (11.18,2.5379);
\draw [c] (11.18,2.5379) -- (11.18,3.70772);
\draw [c] (11,2.5379) -- (11.18,2.5379);
\draw [c] (11.18,2.5379) -- (11.36,2.5379);
\draw [c] (11.54,1.00892) -- (11.54,1.28198);
\draw [c] (11.54,1.28198) -- (11.54,1.55503);
\draw [c] (11.36,1.28198) -- (11.54,1.28198);
\draw [c] (11.54,1.28198) -- (11.72,1.28198);
\draw [c] (11.9,1.63319) -- (11.9,3.68039);
\draw [c] (11.9,3.68039) -- (11.9,4.06897);
\draw [c] (11.72,3.68039) -- (11.9,3.68039);
\draw [c] (11.9,3.68039) -- (12.08,3.68039);
\draw [c] (12.26,0.922168) -- (12.26,1.28667);
\draw [c] (12.26,1.28667) -- (12.26,1.65117);
\draw [c] (12.08,1.28667) -- (12.26,1.28667);
\draw [c] (12.26,1.28667) -- (12.44,1.28667);
\draw [c] (12.62,0.82657) -- (12.62,3.47523);
\draw [c] (12.62,3.47523) -- (12.62,4.06897);
\draw [c] (12.44,3.47523) -- (12.62,3.47523);
\draw [c] (12.62,3.47523) -- (12.8,3.47523);
\draw [c] (12.98,0.551562) -- (12.98,0.801543);
\draw [c] (12.98,0.801543) -- (12.98,1.05152);
\draw [c] (12.8,0.801543) -- (12.98,0.801543);
\draw [c] (12.98,0.801543) -- (13.16,0.801543);
\draw [c] (13.34,0.438627) -- (13.34,4.06897);
\draw [c] (13.7,0.406897) -- (13.7,0.436151);
\draw [c] (13.7,0.436151) -- (13.7,0.575962);
\draw [c] (13.52,0.436151) -- (13.7,0.436151);
\draw [c] (13.7,0.436151) -- (13.88,0.436151);
\draw [c] (14.06,0.406897) -- (14.06,3.71642);
\draw [c] (14.06,3.71642) -- (14.06,4.06897);
\draw [c] (13.88,3.71642) -- (14.06,3.71642);
\draw [c] (14.06,3.71642) -- (14.24,3.71642);
\draw [c] (14.42,0.406897) -- (14.42,2.56399);
\draw [c] (14.42,2.56399) -- (14.42,4.06897);
\draw [c] (14.24,2.56399) -- (14.42,2.56399);
\draw [c] (14.42,2.56399) -- (14.6,2.56399);
\draw [c] (14.78,0.406897) -- (14.78,1.88275);
\draw [c] (14.78,1.88275) -- (14.78,3.79305);
\draw [c] (14.6,1.88275) -- (14.78,1.88275);
\draw [c] (14.78,1.88275) -- (14.96,1.88275);
\draw [c] (15.14,0.406897) -- (15.14,0.616783);
\draw [c] (15.14,0.616783) -- (15.14,1.07239);
\draw [c] (14.96,0.616783) -- (15.14,0.616783);
\draw [c] (15.14,0.616783) -- (15.32,0.616783);
\draw [c] (15.5,0.406897) -- (15.5,0.906309);
\draw [c] (15.5,0.906309) -- (15.5,1.84017);
\draw [c] (15.32,0.906309) -- (15.5,0.906309);
\draw [c] (15.5,0.906309) -- (15.68,0.906309);
\draw [c] (15.86,0.406897) -- (15.86,0.62246);
\draw [c] (15.86,0.62246) -- (15.86,1.27247);
\draw [c] (15.68,0.62246) -- (15.86,0.62246);
\draw [c] (15.86,0.62246) -- (16.04,0.62246);
\draw [c] (16.22,0.406897) -- (16.22,0.50892);
\draw [c] (16.22,0.50892) -- (16.22,1.04539);
\draw [c] (16.04,0.50892) -- (16.22,0.50892);
\draw [c] (16.22,0.50892) -- (16.4,0.50892);
\draw [c] (16.94,0.406897) -- (16.94,0.454972);
\colorlet{c}{natcomp};
\draw [c] (2.18,2.20697) -- (2.18,2.22306);
\draw [c] (2.18,2.22306) -- (2.18,2.23915);
\draw [c] (2,2.22306) -- (2.18,2.22306);
\draw [c] (2.18,2.22306) -- (2.36,2.22306);
\draw [c] (2.54,2.20592) -- (2.54,2.25227);
\draw [c] (2.54,2.25227) -- (2.54,2.29861);
\draw [c] (2.36,2.25227) -- (2.54,2.25227);
\draw [c] (2.54,2.25227) -- (2.72,2.25227);
\draw [c] (2.9,2.04533) -- (2.9,2.13337);
\draw [c] (2.9,2.13337) -- (2.9,2.2214);
\draw [c] (2.72,2.13337) -- (2.9,2.13337);
\draw [c] (2.9,2.13337) -- (3.08,2.13337);
\draw [c] (3.26,2.09336) -- (3.26,2.24544);
\draw [c] (3.26,2.24544) -- (3.26,2.39752);
\draw [c] (3.08,2.24544) -- (3.26,2.24544);
\draw [c] (3.26,2.24544) -- (3.44,2.24544);
\draw [c] (3.62,2.05363) -- (3.62,2.28966);
\draw [c] (3.62,2.28966) -- (3.62,2.5257);
\draw [c] (3.44,2.28966) -- (3.62,2.28966);
\draw [c] (3.62,2.28966) -- (3.8,2.28966);
\draw [c] (3.98,1.20306) -- (3.98,1.31447);
\draw [c] (3.98,1.31447) -- (3.98,1.42587);
\draw [c] (3.8,1.31447) -- (3.98,1.31447);
\draw [c] (3.98,1.31447) -- (4.16,1.31447);
\draw [c] (4.34,1.54463) -- (4.34,1.76915);
\draw [c] (4.34,1.76915) -- (4.34,1.99368);
\draw [c] (4.16,1.76915) -- (4.34,1.76915);
\draw [c] (4.34,1.76915) -- (4.52,1.76915);
\draw [c] (4.7,1.63301) -- (4.7,1.97664);
\draw [c] (4.7,1.97664) -- (4.7,2.32027);
\draw [c] (4.52,1.97664) -- (4.7,1.97664);
\draw [c] (4.7,1.97664) -- (4.88,1.97664);
\draw [c] (5.06,1.58455) -- (5.06,2.0487);
\draw [c] (5.06,2.0487) -- (5.06,2.51286);
\draw [c] (4.88,2.0487) -- (5.06,2.0487);
\draw [c] (5.06,2.0487) -- (5.24,2.0487);
\draw [c] (5.42,1.30619) -- (5.42,1.54341);
\draw [c] (5.42,1.54341) -- (5.42,1.78062);
\draw [c] (5.24,1.54341) -- (5.42,1.54341);
\draw [c] (5.42,1.54341) -- (5.6,1.54341);
\draw [c] (5.78,1.93018) -- (5.78,1.96742);
\draw [c] (5.78,1.96742) -- (5.78,2.00467);
\draw [c] (5.6,1.96742) -- (5.78,1.96742);
\draw [c] (5.78,1.96742) -- (5.96,1.96742);
\draw [c] (6.14,1.91722) -- (6.14,1.96386);
\draw [c] (6.14,1.96386) -- (6.14,2.0105);
\draw [c] (5.96,1.96386) -- (6.14,1.96386);
\draw [c] (6.14,1.96386) -- (6.32,1.96386);
\draw [c] (6.5,1.92546) -- (6.5,1.98502);
\draw [c] (6.5,1.98502) -- (6.5,2.04457);
\draw [c] (6.32,1.98502) -- (6.5,1.98502);
\draw [c] (6.5,1.98502) -- (6.68,1.98502);
\draw [c] (6.86,1.95513) -- (6.86,2.02949);
\draw [c] (6.86,2.02949) -- (6.86,2.10384);
\draw [c] (6.68,2.02949) -- (6.86,2.02949);
\draw [c] (6.86,2.02949) -- (7.04,2.02949);
\draw [c] (7.22,2.07291) -- (7.22,2.16768);
\draw [c] (7.22,2.16768) -- (7.22,2.26244);
\draw [c] (7.04,2.16768) -- (7.22,2.16768);
\draw [c] (7.22,2.16768) -- (7.4,2.16768);
\draw [c] (7.58,1.9231) -- (7.58,2.03345);
\draw [c] (7.58,2.03345) -- (7.58,2.14379);
\draw [c] (7.4,2.03345) -- (7.58,2.03345);
\draw [c] (7.58,2.03345) -- (7.76,2.03345);
\draw [c] (7.94,1.77513) -- (7.94,1.90372);
\draw [c] (7.94,1.90372) -- (7.94,2.03231);
\draw [c] (7.76,1.90372) -- (7.94,1.90372);
\draw [c] (7.94,1.90372) -- (8.12,1.90372);
\draw [c] (8.3,1.72582) -- (8.3,1.87667);
\draw [c] (8.3,1.87667) -- (8.3,2.02751);
\draw [c] (8.12,1.87667) -- (8.3,1.87667);
\draw [c] (8.3,1.87667) -- (8.48,1.87667);
\draw [c] (8.66,1.75759) -- (8.66,1.94707);
\draw [c] (8.66,1.94707) -- (8.66,2.13654);
\draw [c] (8.48,1.94707) -- (8.66,1.94707);
\draw [c] (8.66,1.94707) -- (8.84,1.94707);
\draw [c] (9.02,1.86296) -- (9.02,2.12262);
\draw [c] (9.02,2.12262) -- (9.02,2.38228);
\draw [c] (8.84,2.12262) -- (9.02,2.12262);
\draw [c] (9.02,2.12262) -- (9.2,2.12262);
\draw [c] (9.38,2.18495) -- (9.38,2.54452);
\draw [c] (9.38,2.54452) -- (9.38,2.90409);
\draw [c] (9.2,2.54452) -- (9.38,2.54452);
\draw [c] (9.38,2.54452) -- (9.56,2.54452);
\draw [c] (9.74,2.70237) -- (9.74,3.24451);
\draw [c] (9.74,3.24451) -- (9.74,3.78665);
\draw [c] (9.56,3.24451) -- (9.74,3.24451);
\draw [c] (9.74,3.24451) -- (9.92,3.24451);
\draw [c] (10.1,1.43316) -- (10.1,1.84182);
\draw [c] (10.1,1.84182) -- (10.1,2.25049);
\draw [c] (9.92,1.84182) -- (10.1,1.84182);
\draw [c] (10.1,1.84182) -- (10.28,1.84182);
\draw [c] (10.46,1.27839) -- (10.46,1.76716);
\draw [c] (10.46,1.76716) -- (10.46,2.25594);
\draw [c] (10.28,1.76716) -- (10.46,1.76716);
\draw [c] (10.46,1.76716) -- (10.64,1.76716);
\draw [c] (10.82,1.81288) -- (10.82,2.60934);
\draw [c] (10.82,2.60934) -- (10.82,3.40581);
\draw [c] (10.64,2.60934) -- (10.82,2.60934);
\draw [c] (10.82,2.60934) -- (11,2.60934);
\draw [c] (11.18,0.808135) -- (11.18,1.46066);
\draw [c] (11.18,1.46066) -- (11.18,2.11318);
\draw [c] (11,1.46066) -- (11.18,1.46066);
\draw [c] (11.18,1.46066) -- (11.36,1.46066);
\draw [c] (11.54,1.90433) -- (11.54,3.24741);
\draw [c] (11.54,3.24741) -- (11.54,4.06897);
\draw [c] (11.36,3.24741) -- (11.54,3.24741);
\draw [c] (11.54,3.24741) -- (11.72,3.24741);
\draw [c] (11.9,0.755996) -- (11.9,1.68571);
\draw [c] (11.9,1.68571) -- (11.9,2.61543);
\draw [c] (11.72,1.68571) -- (11.9,1.68571);
\draw [c] (11.9,1.68571) -- (12.08,1.68571);
\draw [c] (12.26,0.406897) -- (12.26,0.698927);
\draw [c] (12.26,0.698927) -- (12.26,1.44989);
\draw [c] (12.08,0.698927) -- (12.26,0.698927);
\draw [c] (12.26,0.698927) -- (12.44,0.698927);
\draw [c] (12.62,1.30062) -- (12.62,3.11107);
\draw [c] (12.62,3.11107) -- (12.62,4.06897);
\draw [c] (12.44,3.11107) -- (12.62,3.11107);
\draw [c] (12.62,3.11107) -- (12.8,3.11107);
\draw [c] (12.98,0.869824) -- (12.98,2.54093);
\draw [c] (12.98,2.54093) -- (12.98,4.06897);
\draw [c] (12.8,2.54093) -- (12.98,2.54093);
\draw [c] (12.98,2.54093) -- (13.16,2.54093);
\draw [c] (14.42,0.406897) -- (14.42,4.06897);
\draw [c] (15.14,0.406897) -- (15.14,4.06897);
\definecolor{c}{rgb}{0,0,0};
\draw [c] (2,0.406897) -- (2,4.06897) -- (20,4.06897) -- (20,0.406897) -- (2,0.406897);
\end{tikzpicture}
}
\end{infilsf}
\end{minipage}
\begin{minipage}[b]{.3\textwidth}
\subcaption{The distribution of $M_{\gamma\gamma},$ the invariant mass of the photon pair, in the event samples produced by three event generators.}

\phantom{p}
\end{minipage}
\caption{Plots showing the distributions of $p_T$ and $M_{\gamma\gamma}$ of events generated by the three event generators \cite{calchep,pythia,madgraph5}. Below, the ratio plots were created by dividing the content in each bin of the distributions with the content of the corresponding bin in the distribution for the CalcHEP sample. The errors on the bins were derived through standard error propagation. All these samples were produced using stratified sampling, where separate sets of events were produced for different ranges of $p_T$, as discussed in the text.
\label{evgen}}
\end{figure}

These plots, and those to follow in this chapter, have at some stage of their creation been through the analysis validation software \textsc{rivet}\footnote{\textbf{R}obust \textbf{I}ndependent \textbf{V}alidation of \textbf{E}xperiment and \textbf{T}heory.} \cite{rivet}.

The strong coupling constant, $\alpha_S$, is a so--called running coupling constant \cite{greenhpt}, meaning that its value changes depending on an energy scale, $Q$, at which it is examined. How this energy scale is determined is defined differently by default in CalcHEP than it is in pythia and MadGraph. To achieve the result in figure~\ref{evgen}, this setting in CalcHEP was changed to 
\[Q^2=\frac{p_T(\gamma_1)^2+p_T(\gamma_2)^2}{2},\]
which is the setting used in the other two event generators.

As figure~\ref{evgen} illustrates, there is a difference of 8--9 orders of magnitude between the number of events produced in the low and the high $p_T$ ranges. Covering this entire range requires more simulated events than we can feasibly analyse. To circumvent this issue, we employ stratified sampling, meaning that we create several smaller samples that cover successive ranges in a representative variable. In this case, we create four sets of events to cover the following ranges in $M_{\gamma\gamma}$:
\begin{enumerate}
\item $M_{\gamma\gamma} \in [ 0 ; 100 )$ GeV
\item $M_{\gamma\gamma} \in [ 100 ; 420 )$ GeV
\item $M_{\gamma\gamma} \in [ 420 ; 1000 )$ GeV
\item $M_{\gamma\gamma} \in [ 1000 ; \infty )$ GeV
\end{enumerate}
In fig~\ref{evgen}, one easily noticeable effect of this stratification of event samples is the jump in statistical uncertainties around 500 GeV in $p_T$ or 1\,000 GeV in $M_{\gamma\gamma}$.

In addition, the fiducial volume for the event sample is bounded by a minimum requirement of 50 GeV in $E_T$ and a maximum of 2.5 in $|\eta|$.

The event samples generated by CalcHEP and pythia are compatible with one another in both these distributions, however, the sample generated by MadGraph seems to produce fewer events with an invariant mass between 1\,000 and 2\,000 GeV compared to the other samples. This sample was not produced by the author, which makes investigation into the cause of this bias difficult. For the present analysis, the bias is included as a systematic uncertainty of 9.18\% on the overall normalisation arising from the choice of event generator, however it remains a possibility that this uncertainty could be reduced or eliminated by a more careful study.


\section{Discriminating variables}
To allow CalcHEP to generate events that involve the new contact interaction, we can simply extend the list of Feynman rules known to CalcHEP to include the new Feynman rule we found in chapter~\ref{ch.theory}, using LanHEP \cite{lanhep}. In practice, this involves giving as input a Lagrangian written in a format similar to \LaTeX's math language. LanHEP then produces a list of Feynman rules in a format appropriate for CalcHEP. This new model has two additional parameters: the size of $\Lambda$ and the sign of the interference of this new interaction with the existing processes.

With the ability to produce sets of events with varying values of $\Lambda$, the opportunity presents itself to create simulated distributions of events for several potential observables that we might examine to discriminate the effects of the contact interaction. To quantify the discriminating power of these observables, we define a figure of merit, which we will call `significance', as
\(S\equiv\ono{N}\sum_n \frac{|x_n-y_n|}{\sqrt{{\sigma_{x,n}}^2+{\sigma_{y,n}}^2}},\label{significance}\)
where the sum is over bins $n$ where both distributions have non--zero content, $N$ is the total number of bins where both distributions have non--zero content, $x_n$ and $y_n$ are the content of the two distributions in bin $n$, and $\sigma_{x,n}$ and $\sigma_{y,n}$ are the uncertainties on the bin content of each of the two distributions in bin $n$.
Results for a few selected variables are shown in figure~\ref{discr}.

\begin{figure}[htp]
\begin{minipage}[b]{.499\textwidth}
\begin{infilsf}\tiny
\hspace{-.9em}\makebox[.96\textwidth]{\begin{tikzpicture}[x=.095\textwidth,y=.095\textwidth]
\pgfdeclareplotmark{cross} {
\pgfpathmoveto{\pgfpoint{-0.3\pgfplotmarksize}{\pgfplotmarksize}}
\pgfpathlineto{\pgfpoint{+0.3\pgfplotmarksize}{\pgfplotmarksize}}
\pgfpathlineto{\pgfpoint{+0.3\pgfplotmarksize}{0.3\pgfplotmarksize}}
\pgfpathlineto{\pgfpoint{+1\pgfplotmarksize}{0.3\pgfplotmarksize}}
\pgfpathlineto{\pgfpoint{+1\pgfplotmarksize}{-0.3\pgfplotmarksize}}
\pgfpathlineto{\pgfpoint{+0.3\pgfplotmarksize}{-0.3\pgfplotmarksize}}
\pgfpathlineto{\pgfpoint{+0.3\pgfplotmarksize}{-1.\pgfplotmarksize}}
\pgfpathlineto{\pgfpoint{-0.3\pgfplotmarksize}{-1.\pgfplotmarksize}}
\pgfpathlineto{\pgfpoint{-0.3\pgfplotmarksize}{-0.3\pgfplotmarksize}}
\pgfpathlineto{\pgfpoint{-1.\pgfplotmarksize}{-0.3\pgfplotmarksize}}
\pgfpathlineto{\pgfpoint{-1.\pgfplotmarksize}{0.3\pgfplotmarksize}}
\pgfpathlineto{\pgfpoint{-0.3\pgfplotmarksize}{0.3\pgfplotmarksize}}
\pgfpathclose
\pgfusepathqstroke
}
\pgfdeclareplotmark{cross*} {
\pgfpathmoveto{\pgfpoint{-0.3\pgfplotmarksize}{\pgfplotmarksize}}
\pgfpathlineto{\pgfpoint{+0.3\pgfplotmarksize}{\pgfplotmarksize}}
\pgfpathlineto{\pgfpoint{+0.3\pgfplotmarksize}{0.3\pgfplotmarksize}}
\pgfpathlineto{\pgfpoint{+1\pgfplotmarksize}{0.3\pgfplotmarksize}}
\pgfpathlineto{\pgfpoint{+1\pgfplotmarksize}{-0.3\pgfplotmarksize}}
\pgfpathlineto{\pgfpoint{+0.3\pgfplotmarksize}{-0.3\pgfplotmarksize}}
\pgfpathlineto{\pgfpoint{+0.3\pgfplotmarksize}{-1.\pgfplotmarksize}}
\pgfpathlineto{\pgfpoint{-0.3\pgfplotmarksize}{-1.\pgfplotmarksize}}
\pgfpathlineto{\pgfpoint{-0.3\pgfplotmarksize}{-0.3\pgfplotmarksize}}
\pgfpathlineto{\pgfpoint{-1.\pgfplotmarksize}{-0.3\pgfplotmarksize}}
\pgfpathlineto{\pgfpoint{-1.\pgfplotmarksize}{0.3\pgfplotmarksize}}
\pgfpathlineto{\pgfpoint{-0.3\pgfplotmarksize}{0.3\pgfplotmarksize}}
\pgfpathclose
\pgfusepathqfillstroke
}
\pgfdeclareplotmark{newstar} {
\pgfpathmoveto{\pgfqpoint{0pt}{\pgfplotmarksize}}
\pgfpathlineto{\pgfqpointpolar{44}{0.5\pgfplotmarksize}}
\pgfpathlineto{\pgfqpointpolar{18}{\pgfplotmarksize}}
\pgfpathlineto{\pgfqpointpolar{-20}{0.5\pgfplotmarksize}}
\pgfpathlineto{\pgfqpointpolar{-54}{\pgfplotmarksize}}
\pgfpathlineto{\pgfqpointpolar{-90}{0.5\pgfplotmarksize}}
\pgfpathlineto{\pgfqpointpolar{234}{\pgfplotmarksize}}
\pgfpathlineto{\pgfqpointpolar{198}{0.5\pgfplotmarksize}}
\pgfpathlineto{\pgfqpointpolar{162}{\pgfplotmarksize}}
\pgfpathlineto{\pgfqpointpolar{134}{0.5\pgfplotmarksize}}
\pgfpathclose
\pgfusepathqstroke
}
\pgfdeclareplotmark{newstar*} {
\pgfpathmoveto{\pgfqpoint{0pt}{\pgfplotmarksize}}
\pgfpathlineto{\pgfqpointpolar{44}{0.5\pgfplotmarksize}}
\pgfpathlineto{\pgfqpointpolar{18}{\pgfplotmarksize}}
\pgfpathlineto{\pgfqpointpolar{-20}{0.5\pgfplotmarksize}}
\pgfpathlineto{\pgfqpointpolar{-54}{\pgfplotmarksize}}
\pgfpathlineto{\pgfqpointpolar{-90}{0.5\pgfplotmarksize}}
\pgfpathlineto{\pgfqpointpolar{234}{\pgfplotmarksize}}
\pgfpathlineto{\pgfqpointpolar{198}{0.5\pgfplotmarksize}}
\pgfpathlineto{\pgfqpointpolar{162}{\pgfplotmarksize}}
\pgfpathlineto{\pgfqpointpolar{134}{0.5\pgfplotmarksize}}
\pgfpathclose
\pgfusepathqfillstroke
}
\definecolor{c}{rgb}{1,1,1};
\draw [color=c, fill=c] (0,0) rectangle (10,6.80516);
\draw [color=c, fill=c] (1,0.680516) rectangle (9.95,6.73711);
\definecolor{c}{rgb}{0,0,0};
\draw [c] (1,0.680516) -- (1,6.73711) -- (9.95,6.73711) -- (9.95,0.680516) -- (1,0.680516);
\definecolor{c}{rgb}{1,1,1};
\draw [color=c, fill=c] (1,0.680516) rectangle (9.95,6.73711);
\definecolor{c}{rgb}{0,0,0};
\draw [c] (1,0.680516) -- (1,6.73711) -- (9.95,6.73711) -- (9.95,0.680516) -- (1,0.680516);
\colorlet{c}{natgreen};
\draw [c] (4.4905,1.34153) -- (4.4905,1.56231);
\draw [c] (4.4905,1.56231) -- (4.4905,1.74146);
\draw [c] (4.401,1.56231) -- (4.4905,1.56231);
\draw [c] (4.4905,1.56231) -- (4.58,1.56231);
\definecolor{c}{rgb}{0,0,0};
\colorlet{c}{natgreen};
\draw [c] (4.6695,1.98417) -- (4.6695,2.14054);
\draw [c] (4.6695,2.14054) -- (4.6695,2.27485);
\draw [c] (4.58,2.14054) -- (4.6695,2.14054);
\draw [c] (4.6695,2.14054) -- (4.759,2.14054);
\definecolor{c}{rgb}{0,0,0};
\colorlet{c}{natgreen};
\draw [c] (4.8485,2.08775) -- (4.8485,2.23283);
\draw [c] (4.8485,2.23283) -- (4.8485,2.35873);
\draw [c] (4.759,2.23283) -- (4.8485,2.23283);
\draw [c] (4.8485,2.23283) -- (4.938,2.23283);
\definecolor{c}{rgb}{0,0,0};
\colorlet{c}{natgreen};
\draw [c] (5.0275,2.31696) -- (5.0275,2.44748);
\draw [c] (5.0275,2.44748) -- (5.0275,2.56226);
\draw [c] (4.938,2.44748) -- (5.0275,2.44748);
\draw [c] (5.0275,2.44748) -- (5.117,2.44748);
\definecolor{c}{rgb}{0,0,0};
\colorlet{c}{natgreen};
\draw [c] (5.2065,2.74465) -- (5.2065,2.85048);
\draw [c] (5.2065,2.85048) -- (5.2065,2.94574);
\draw [c] (5.117,2.85048) -- (5.2065,2.85048);
\draw [c] (5.2065,2.85048) -- (5.296,2.85048);
\definecolor{c}{rgb}{0,0,0};
\colorlet{c}{natgreen};
\draw [c] (5.3855,2.58724) -- (5.3855,2.70063);
\draw [c] (5.3855,2.70063) -- (5.3855,2.80195);
\draw [c] (5.296,2.70063) -- (5.3855,2.70063);
\draw [c] (5.3855,2.70063) -- (5.475,2.70063);
\definecolor{c}{rgb}{0,0,0};
\colorlet{c}{natgreen};
\draw [c] (5.5645,2.67125) -- (5.5645,2.77937);
\draw [c] (5.5645,2.77937) -- (5.5645,2.87647);
\draw [c] (5.475,2.77937) -- (5.5645,2.77937);
\draw [c] (5.5645,2.77937) -- (5.654,2.77937);
\definecolor{c}{rgb}{0,0,0};
\colorlet{c}{natgreen};
\draw [c] (5.7435,2.48129) -- (5.7435,2.59858);
\draw [c] (5.7435,2.59858) -- (5.7435,2.703);
\draw [c] (5.654,2.59858) -- (5.7435,2.59858);
\draw [c] (5.7435,2.59858) -- (5.833,2.59858);
\definecolor{c}{rgb}{0,0,0};
\colorlet{c}{natgreen};
\draw [c] (5.9225,2.7969) -- (5.9225,2.89804);
\draw [c] (5.9225,2.89804) -- (5.9225,2.98947);
\draw [c] (5.833,2.89804) -- (5.9225,2.89804);
\draw [c] (5.9225,2.89804) -- (6.012,2.89804);
\definecolor{c}{rgb}{0,0,0};
\colorlet{c}{natgreen};
\draw [c] (6.1015,2.99456) -- (6.1015,3.08665);
\draw [c] (6.1015,3.08665) -- (6.1015,3.17063);
\draw [c] (6.012,3.08665) -- (6.1015,3.08665);
\draw [c] (6.1015,3.08665) -- (6.191,3.08665);
\definecolor{c}{rgb}{0,0,0};
\colorlet{c}{natgreen};
\draw [c] (6.2805,2.89348) -- (6.2805,2.98918);
\draw [c] (6.2805,2.98918) -- (6.2805,3.07615);
\draw [c] (6.191,2.98918) -- (6.2805,2.98918);
\draw [c] (6.2805,2.98918) -- (6.37,2.98918);
\definecolor{c}{rgb}{0,0,0};
\colorlet{c}{natgreen};
\draw [c] (6.4595,2.84908) -- (6.4595,2.94685);
\draw [c] (6.4595,2.94685) -- (6.4595,3.03553);
\draw [c] (6.37,2.94685) -- (6.4595,2.94685);
\draw [c] (6.4595,2.94685) -- (6.549,2.94685);
\definecolor{c}{rgb}{0,0,0};
\colorlet{c}{natgreen};
\draw [c] (6.6385,2.97295) -- (6.6385,3.06474);
\draw [c] (6.6385,3.06474) -- (6.6385,3.14847);
\draw [c] (6.549,3.06474) -- (6.6385,3.06474);
\draw [c] (6.6385,3.06474) -- (6.728,3.06474);
\definecolor{c}{rgb}{0,0,0};
\colorlet{c}{natgreen};
\draw [c] (6.8175,3.10011) -- (6.8175,3.18619);
\draw [c] (6.8175,3.18619) -- (6.8175,3.26514);
\draw [c] (6.728,3.18619) -- (6.8175,3.18619);
\draw [c] (6.8175,3.18619) -- (6.907,3.18619);
\definecolor{c}{rgb}{0,0,0};
\colorlet{c}{natgreen};
\draw [c] (6.9965,2.92892) -- (6.9965,3.02184);
\draw [c] (6.9965,3.02184) -- (6.9965,3.1065);
\draw [c] (6.907,3.02184) -- (6.9965,3.02184);
\draw [c] (6.9965,3.02184) -- (7.086,3.02184);
\definecolor{c}{rgb}{0,0,0};
\colorlet{c}{natgreen};
\draw [c] (7.1755,3.21763) -- (7.1755,3.29884);
\draw [c] (7.1755,3.29884) -- (7.1755,3.37368);
\draw [c] (7.086,3.29884) -- (7.1755,3.29884);
\draw [c] (7.1755,3.29884) -- (7.265,3.29884);
\definecolor{c}{rgb}{0,0,0};
\colorlet{c}{natgreen};
\draw [c] (7.3545,3.12559) -- (7.3545,3.20952);
\draw [c] (7.3545,3.20952) -- (7.3545,3.28666);
\draw [c] (7.265,3.20952) -- (7.3545,3.20952);
\draw [c] (7.3545,3.20952) -- (7.444,3.20952);
\definecolor{c}{rgb}{0,0,0};
\colorlet{c}{natgreen};
\draw [c] (7.5335,3.34731) -- (7.5335,3.42304);
\draw [c] (7.5335,3.42304) -- (7.5335,3.4932);
\draw [c] (7.444,3.42304) -- (7.5335,3.42304);
\draw [c] (7.5335,3.42304) -- (7.623,3.42304);
\definecolor{c}{rgb}{0,0,0};
\colorlet{c}{natgreen};
\draw [c] (7.7125,3.4191) -- (7.7125,3.49177);
\draw [c] (7.7125,3.49177) -- (7.7125,3.55929);
\draw [c] (7.623,3.49177) -- (7.7125,3.49177);
\draw [c] (7.7125,3.49177) -- (7.802,3.49177);
\definecolor{c}{rgb}{0,0,0};
\colorlet{c}{natgreen};
\draw [c] (7.8915,3.44293) -- (7.8915,3.51435);
\draw [c] (7.8915,3.51435) -- (7.8915,3.5808);
\draw [c] (7.802,3.51435) -- (7.8915,3.51435);
\draw [c] (7.8915,3.51435) -- (7.981,3.51435);
\definecolor{c}{rgb}{0,0,0};
\colorlet{c}{natgreen};
\draw [c] (8.0705,3.49986) -- (8.0705,3.56946);
\draw [c] (8.0705,3.56946) -- (8.0705,3.63433);
\draw [c] (7.981,3.56946) -- (8.0705,3.56946);
\draw [c] (8.0705,3.56946) -- (8.16,3.56946);
\definecolor{c}{rgb}{0,0,0};
\colorlet{c}{natgreen};
\draw [c] (8.2495,3.6706) -- (8.2495,3.73434);
\draw [c] (8.2495,3.73434) -- (8.2495,3.79409);
\draw [c] (8.16,3.73434) -- (8.2495,3.73434);
\draw [c] (8.2495,3.73434) -- (8.339,3.73434);
\definecolor{c}{rgb}{0,0,0};
\colorlet{c}{natgreen};
\draw [c] (8.4285,3.68512) -- (8.4285,3.74794);
\draw [c] (8.4285,3.74794) -- (8.4285,3.80687);
\draw [c] (8.339,3.74794) -- (8.4285,3.74794);
\draw [c] (8.4285,3.74794) -- (8.518,3.74794);
\definecolor{c}{rgb}{0,0,0};
\colorlet{c}{natgreen};
\draw [c] (8.6075,3.99767) -- (8.6075,4.05172);
\draw [c] (8.6075,4.05172) -- (8.6075,4.10288);
\draw [c] (8.518,4.05172) -- (8.6075,4.05172);
\draw [c] (8.6075,4.05172) -- (8.697,4.05172);
\definecolor{c}{rgb}{0,0,0};
\colorlet{c}{natgreen};
\draw [c] (8.7865,4.09108) -- (8.7865,4.14228);
\draw [c] (8.7865,4.14228) -- (8.7865,4.19087);
\draw [c] (8.697,4.14228) -- (8.7865,4.14228);
\draw [c] (8.7865,4.14228) -- (8.876,4.14228);
\definecolor{c}{rgb}{0,0,0};
\colorlet{c}{natgreen};
\draw [c] (8.9655,4.30501) -- (8.9655,4.35091);
\draw [c] (8.9655,4.35091) -- (8.9655,4.39469);
\draw [c] (8.876,4.35091) -- (8.9655,4.35091);
\draw [c] (8.9655,4.35091) -- (9.055,4.35091);
\definecolor{c}{rgb}{0,0,0};
\colorlet{c}{natgreen};
\draw [c] (9.1445,4.40589) -- (9.1445,4.44916);
\draw [c] (9.1445,4.44916) -- (9.1445,4.49055);
\draw [c] (9.055,4.44916) -- (9.1445,4.44916);
\draw [c] (9.1445,4.44916) -- (9.234,4.44916);
\definecolor{c}{rgb}{0,0,0};
\colorlet{c}{natgreen};
\draw [c] (9.3235,4.62319) -- (9.3235,4.66169);
\draw [c] (9.3235,4.66169) -- (9.3235,4.6987);
\draw [c] (9.234,4.66169) -- (9.3235,4.66169);
\draw [c] (9.3235,4.66169) -- (9.413,4.66169);
\definecolor{c}{rgb}{0,0,0};
\colorlet{c}{natgreen};
\draw [c] (9.5025,4.96961) -- (9.5025,5.00163);
\draw [c] (9.5025,5.00163) -- (9.5025,5.03262);
\draw [c] (9.413,5.00163) -- (9.5025,5.00163);
\draw [c] (9.5025,5.00163) -- (9.592,5.00163);
\definecolor{c}{rgb}{0,0,0};
\colorlet{c}{natgreen};
\draw [c] (9.6815,5.31335) -- (9.6815,5.33968);
\draw [c] (9.6815,5.33968) -- (9.6815,5.3653);
\draw [c] (9.592,5.33968) -- (9.6815,5.33968);
\draw [c] (9.6815,5.33968) -- (9.771,5.33968);
\definecolor{c}{rgb}{0,0,0};
\colorlet{c}{natgreen};
\draw [c] (9.8605,6.09477) -- (9.8605,6.11135);
\draw [c] (9.8605,6.11135) -- (9.8605,6.12765);
\draw [c] (9.771,6.11135) -- (9.8605,6.11135);
\draw [c] (9.8605,6.11135) -- (9.95,6.11135);
\definecolor{c}{rgb}{0,0,0};
\draw [c] (1,0.680516) -- (9.95,0.680516);
\draw [anchor= east] (9.95,-0.0816619) node[color=c, rotate=0]{$cos\theta_{\gamma_{1}}$};
\draw [c] (1,0.863234) -- (1,0.680516);
\draw [c] (1.4475,0.771875) -- (1.4475,0.680516);
\draw [c] (1.895,0.771875) -- (1.895,0.680516);
\draw [c] (2.3425,0.771875) -- (2.3425,0.680516);
\draw [c] (2.79,0.771875) -- (2.79,0.680516);
\draw [c] (3.2375,0.863234) -- (3.2375,0.680516);
\draw [c] (3.685,0.771875) -- (3.685,0.680516);
\draw [c] (4.1325,0.771875) -- (4.1325,0.680516);
\draw [c] (4.58,0.771875) -- (4.58,0.680516);
\draw [c] (5.0275,0.771875) -- (5.0275,0.680516);
\draw [c] (5.475,0.863234) -- (5.475,0.680516);
\draw [c] (5.9225,0.771875) -- (5.9225,0.680516);
\draw [c] (6.37,0.771875) -- (6.37,0.680516);
\draw [c] (6.8175,0.771875) -- (6.8175,0.680516);
\draw [c] (7.265,0.771875) -- (7.265,0.680516);
\draw [c] (7.7125,0.863234) -- (7.7125,0.680516);
\draw [c] (8.16,0.771875) -- (8.16,0.680516);
\draw [c] (8.6075,0.771875) -- (8.6075,0.680516);
\draw [c] (9.055,0.771875) -- (9.055,0.680516);
\draw [c] (9.5025,0.771875) -- (9.5025,0.680516);
\draw [c] (9.95,0.863234) -- (9.95,0.680516);
\draw [anchor=base] (1,0.285817) node[color=c, rotate=0]{-1};
\draw [anchor=base] (3.2375,0.285817) node[color=c, rotate=0]{-0.5};
\draw [anchor=base] (5.475,0.285817) node[color=c, rotate=0]{0};
\draw [anchor=base] (7.7125,0.285817) node[color=c, rotate=0]{0.5};
\draw [anchor=base] (9.95,0.285817) node[color=c, rotate=0]{1};
\draw [c] (1,0.680516) -- (1,6.73711);
\draw [anchor= east] (-0.4,6.73711) node[color=c, rotate=90]{$d\sigma/d(cos\theta_{\gamma_{1}}) [pb]$};
\draw [c] (1.1335,0.729337) -- (1,0.729337);
\draw [c] (1.1335,0.942136) -- (1,0.942136);
\draw [c] (1.1335,1.11601) -- (1,1.11601);
\draw [c] (1.1335,1.26301) -- (1,1.26301);
\draw [c] (1.1335,1.39035) -- (1,1.39035);
\draw [c] (1.1335,1.50267) -- (1,1.50267);
\draw [c] (1.267,1.60315) -- (1,1.60315);
\draw [anchor= east] (0.922,1.60315) node[color=c, rotate=0]{1};
\draw [c] (1.1335,2.26417) -- (1,2.26417);
\draw [c] (1.1335,2.65084) -- (1,2.65084);
\draw [c] (1.1335,2.92518) -- (1,2.92518);
\draw [c] (1.1335,3.13798) -- (1,3.13798);
\draw [c] (1.1335,3.31185) -- (1,3.31185);
\draw [c] (1.1335,3.45886) -- (1,3.45886);
\draw [c] (1.1335,3.5862) -- (1,3.5862);
\draw [c] (1.1335,3.69852) -- (1,3.69852);
\draw [c] (1.267,3.799) -- (1,3.799);
\draw [anchor= east] (0.922,3.799) node[color=c, rotate=0]{10};
\draw [c] (1.1335,4.46001) -- (1,4.46001);
\draw [c] (1.1335,4.84668) -- (1,4.84668);
\draw [c] (1.1335,5.12103) -- (1,5.12103);
\draw [c] (1.1335,5.33383) -- (1,5.33383);
\draw [c] (1.1335,5.5077) -- (1,5.5077);
\draw [c] (1.1335,5.6547) -- (1,5.6547);
\draw [c] (1.1335,5.78204) -- (1,5.78204);
\draw [c] (1.1335,5.89437) -- (1,5.89437);
\draw [c] (1.267,5.99484) -- (1,5.99484);
\draw [anchor= east] (0.922,5.99484) node[color=c, rotate=0]{$10^{2}$};
\draw [c] (1.1335,6.65586) -- (1,6.65586);
\colorlet{c}{natcomp};
\draw [c] (4.4905,0.920275) -- (4.4905,1.18987);
\draw [c] (4.4905,1.18987) -- (4.4905,1.39981);
\draw [c] (4.401,1.18987) -- (4.4905,1.18987);
\draw [c] (4.4905,1.18987) -- (4.58,1.18987);
\definecolor{c}{rgb}{0,0,0};
\colorlet{c}{natcomp};
\draw [c] (4.6695,1.64843) -- (4.6695,1.83114);
\draw [c] (4.6695,1.83114) -- (4.6695,1.98441);
\draw [c] (4.58,1.83114) -- (4.6695,1.83114);
\draw [c] (4.6695,1.83114) -- (4.759,1.83114);
\definecolor{c}{rgb}{0,0,0};
\colorlet{c}{natcomp};
\draw [c] (4.8485,1.95665) -- (4.8485,2.11097);
\draw [c] (4.8485,2.11097) -- (4.8485,2.24377);
\draw [c] (4.759,2.11097) -- (4.8485,2.11097);
\draw [c] (4.8485,2.11097) -- (4.938,2.11097);
\definecolor{c}{rgb}{0,0,0};
\colorlet{c}{natcomp};
\draw [c] (5.0275,2.37796) -- (5.0275,2.50432);
\draw [c] (5.0275,2.50432) -- (5.0275,2.61587);
\draw [c] (4.938,2.50432) -- (5.0275,2.50432);
\draw [c] (5.0275,2.50432) -- (5.117,2.50432);
\definecolor{c}{rgb}{0,0,0};
\colorlet{c}{natcomp};
\draw [c] (5.2065,2.50461) -- (5.2065,2.62297);
\draw [c] (5.2065,2.62297) -- (5.2065,2.72825);
\draw [c] (5.117,2.62297) -- (5.2065,2.62297);
\draw [c] (5.2065,2.62297) -- (5.296,2.62297);
\definecolor{c}{rgb}{0,0,0};
\colorlet{c}{natcomp};
\draw [c] (5.3855,2.48689) -- (5.3855,2.60542);
\draw [c] (5.3855,2.60542) -- (5.3855,2.71084);
\draw [c] (5.296,2.60542) -- (5.3855,2.60542);
\draw [c] (5.3855,2.60542) -- (5.475,2.60542);
\definecolor{c}{rgb}{0,0,0};
\colorlet{c}{natcomp};
\draw [c] (5.5645,2.23773) -- (5.5645,2.36905);
\draw [c] (5.5645,2.36905) -- (5.5645,2.48446);
\draw [c] (5.475,2.36905) -- (5.5645,2.36905);
\draw [c] (5.5645,2.36905) -- (5.654,2.36905);
\definecolor{c}{rgb}{0,0,0};
\colorlet{c}{natcomp};
\draw [c] (5.7435,2.73859) -- (5.7435,2.84308);
\draw [c] (5.7435,2.84308) -- (5.7435,2.93725);
\draw [c] (5.654,2.84308) -- (5.7435,2.84308);
\draw [c] (5.7435,2.84308) -- (5.833,2.84308);
\definecolor{c}{rgb}{0,0,0};
\colorlet{c}{natcomp};
\draw [c] (5.9225,2.77099) -- (5.9225,2.87346);
\draw [c] (5.9225,2.87346) -- (5.9225,2.96599);
\draw [c] (5.833,2.87346) -- (5.9225,2.87346);
\draw [c] (5.9225,2.87346) -- (6.012,2.87346);
\definecolor{c}{rgb}{0,0,0};
\colorlet{c}{natcomp};
\draw [c] (6.1015,2.83222) -- (6.1015,2.93102);
\draw [c] (6.1015,2.93102) -- (6.1015,3.02054);
\draw [c] (6.012,2.93102) -- (6.1015,2.93102);
\draw [c] (6.1015,2.93102) -- (6.191,2.93102);
\definecolor{c}{rgb}{0,0,0};
\colorlet{c}{natcomp};
\draw [c] (6.2805,2.8924) -- (6.2805,2.98816);
\draw [c] (6.2805,2.98816) -- (6.2805,3.07519);
\draw [c] (6.191,2.98816) -- (6.2805,2.98816);
\draw [c] (6.2805,2.98816) -- (6.37,2.98816);
\definecolor{c}{rgb}{0,0,0};
\colorlet{c}{natcomp};
\draw [c] (6.4595,2.77345) -- (6.4595,2.87442);
\draw [c] (6.4595,2.87442) -- (6.4595,2.96571);
\draw [c] (6.37,2.87442) -- (6.4595,2.87442);
\draw [c] (6.4595,2.87442) -- (6.549,2.87442);
\definecolor{c}{rgb}{0,0,0};
\colorlet{c}{natcomp};
\draw [c] (6.6385,2.95182) -- (6.6385,3.04456);
\draw [c] (6.6385,3.04456) -- (6.6385,3.12908);
\draw [c] (6.549,3.04456) -- (6.6385,3.04456);
\draw [c] (6.6385,3.04456) -- (6.728,3.04456);
\definecolor{c}{rgb}{0,0,0};
\colorlet{c}{natcomp};
\draw [c] (6.8175,2.82581) -- (6.8175,2.92349);
\draw [c] (6.8175,2.92349) -- (6.8175,3.01209);
\draw [c] (6.728,2.92349) -- (6.8175,2.92349);
\draw [c] (6.8175,2.92349) -- (6.907,2.92349);
\definecolor{c}{rgb}{0,0,0};
\colorlet{c}{natcomp};
\draw [c] (6.9965,3.1853) -- (6.9965,3.26814);
\draw [c] (6.9965,3.26814) -- (6.9965,3.34436);
\draw [c] (6.907,3.26814) -- (6.9965,3.26814);
\draw [c] (6.9965,3.26814) -- (7.086,3.26814);
\definecolor{c}{rgb}{0,0,0};
\colorlet{c}{natcomp};
\draw [c] (7.1755,3.11236) -- (7.1755,3.19737);
\draw [c] (7.1755,3.19737) -- (7.1755,3.27542);
\draw [c] (7.086,3.19737) -- (7.1755,3.19737);
\draw [c] (7.1755,3.19737) -- (7.265,3.19737);
\definecolor{c}{rgb}{0,0,0};
\colorlet{c}{natcomp};
\draw [c] (7.3545,3.29137) -- (7.3545,3.36915);
\draw [c] (7.3545,3.36915) -- (7.3545,3.44106);
\draw [c] (7.265,3.36915) -- (7.3545,3.36915);
\draw [c] (7.3545,3.36915) -- (7.444,3.36915);
\definecolor{c}{rgb}{0,0,0};
\colorlet{c}{natcomp};
\draw [c] (7.5335,3.27055) -- (7.5335,3.34877);
\draw [c] (7.5335,3.34877) -- (7.5335,3.42107);
\draw [c] (7.444,3.34877) -- (7.5335,3.34877);
\draw [c] (7.5335,3.34877) -- (7.623,3.34877);
\definecolor{c}{rgb}{0,0,0};
\colorlet{c}{natcomp};
\draw [c] (7.7125,3.49963) -- (7.7125,3.56986);
\draw [c] (7.7125,3.56986) -- (7.7125,3.63526);
\draw [c] (7.623,3.56986) -- (7.7125,3.56986);
\draw [c] (7.7125,3.56986) -- (7.802,3.56986);
\definecolor{c}{rgb}{0,0,0};
\colorlet{c}{natcomp};
\draw [c] (7.8915,3.57721) -- (7.8915,3.64464);
\draw [c] (7.8915,3.64464) -- (7.8915,3.70762);
\draw [c] (7.802,3.64464) -- (7.8915,3.64464);
\draw [c] (7.8915,3.64464) -- (7.981,3.64464);
\definecolor{c}{rgb}{0,0,0};
\colorlet{c}{natcomp};
\draw [c] (8.0705,3.53702) -- (8.0705,3.60526);
\draw [c] (8.0705,3.60526) -- (8.0705,3.66895);
\draw [c] (7.981,3.60526) -- (8.0705,3.60526);
\draw [c] (8.0705,3.60526) -- (8.16,3.60526);
\definecolor{c}{rgb}{0,0,0};
\colorlet{c}{natcomp};
\draw [c] (8.2495,3.71908) -- (8.2495,3.78159);
\draw [c] (8.2495,3.78159) -- (8.2495,3.84026);
\draw [c] (8.16,3.78159) -- (8.2495,3.78159);
\draw [c] (8.2495,3.78159) -- (8.339,3.78159);
\definecolor{c}{rgb}{0,0,0};
\colorlet{c}{natcomp};
\draw [c] (8.4285,3.81763) -- (8.4285,3.87676);
\draw [c] (8.4285,3.87676) -- (8.4285,3.93244);
\draw [c] (8.339,3.87676) -- (8.4285,3.87676);
\draw [c] (8.4285,3.87676) -- (8.518,3.87676);
\definecolor{c}{rgb}{0,0,0};
\colorlet{c}{natcomp};
\draw [c] (8.6075,3.94161) -- (8.6075,3.997);
\draw [c] (8.6075,3.997) -- (8.6075,4.04935);
\draw [c] (8.518,3.997) -- (8.6075,3.997);
\draw [c] (8.6075,3.997) -- (8.697,3.997);
\definecolor{c}{rgb}{0,0,0};
\colorlet{c}{natcomp};
\draw [c] (8.7865,4.03257) -- (8.7865,4.0852);
\draw [c] (8.7865,4.0852) -- (8.7865,4.13509);
\draw [c] (8.697,4.0852) -- (8.7865,4.0852);
\draw [c] (8.7865,4.0852) -- (8.876,4.0852);
\definecolor{c}{rgb}{0,0,0};
\colorlet{c}{natcomp};
\draw [c] (8.9655,4.19222) -- (8.9655,4.2406);
\draw [c] (8.9655,4.2406) -- (8.9655,4.28664);
\draw [c] (8.876,4.2406) -- (8.9655,4.2406);
\draw [c] (8.9655,4.2406) -- (9.055,4.2406);
\definecolor{c}{rgb}{0,0,0};
\colorlet{c}{natcomp};
\draw [c] (9.1445,4.4426) -- (9.1445,4.48513);
\draw [c] (9.1445,4.48513) -- (9.1445,4.52584);
\draw [c] (9.055,4.48513) -- (9.1445,4.48513);
\draw [c] (9.1445,4.48513) -- (9.234,4.48513);
\definecolor{c}{rgb}{0,0,0};
\colorlet{c}{natcomp};
\draw [c] (9.3235,4.57873) -- (9.3235,4.61791);
\draw [c] (9.3235,4.61791) -- (9.3235,4.65555);
\draw [c] (9.234,4.61791) -- (9.3235,4.61791);
\draw [c] (9.3235,4.61791) -- (9.413,4.61791);
\definecolor{c}{rgb}{0,0,0};
\colorlet{c}{natcomp};
\draw [c] (9.5025,4.94783) -- (9.5025,4.98022);
\draw [c] (9.5025,4.98022) -- (9.5025,5.01154);
\draw [c] (9.413,4.98022) -- (9.5025,4.98022);
\draw [c] (9.5025,4.98022) -- (9.592,4.98022);
\definecolor{c}{rgb}{0,0,0};
\colorlet{c}{natcomp};
\draw [c] (9.6815,5.34386) -- (9.6815,5.36983);
\draw [c] (9.6815,5.36983) -- (9.6815,5.39511);
\draw [c] (9.592,5.36983) -- (9.6815,5.36983);
\draw [c] (9.6815,5.36983) -- (9.771,5.36983);
\definecolor{c}{rgb}{0,0,0};
\colorlet{c}{natcomp};
\draw [c] (9.8605,6.10647) -- (9.8605,6.12295);
\draw [c] (9.8605,6.12295) -- (9.8605,6.13914);
\draw [c] (9.771,6.12295) -- (9.8605,6.12295);
\draw [c] (9.8605,6.12295) -- (9.95,6.12295);
\definecolor{c}{rgb}{0,0,0};
\definecolor{c}{rgb}{1,1,1};
\draw [color=c, fill=c] (2,5.44413) rectangle (6,6.66905);
\definecolor{c}{rgb}{0,0,0};
\draw [anchor=base west] (3,6.22502) node[color=c, rotate=0]{Standard Model};
\colorlet{c}{natgreen};
\draw [c] (2.15,6.36282) -- (2.85,6.36282);
\definecolor{c}{rgb}{0,0,0};
\draw [anchor=base west] (3,5.61255) node[color=c, rotate=0]{$\Lambda = 1.0 TeV$};
\colorlet{c}{natcomp};
\draw [c] (2.15,5.75036) -- (2.85,5.75036);
\end{tikzpicture}
}
\end{infilsf}
\vspace{-1em}
\subcaption{Significance: 0.98 \label{sigcos}}
\end{minipage}
\hfill
\begin{minipage}[b]{.499\textwidth}
\begin{infilsf} \tiny
\makebox[.96\textwidth]{\begin{tikzpicture}[x=.095\textwidth,y=.095\textwidth]
\pgfdeclareplotmark{cross} {
\pgfpathmoveto{\pgfpoint{-0.3\pgfplotmarksize}{\pgfplotmarksize}}
\pgfpathlineto{\pgfpoint{+0.3\pgfplotmarksize}{\pgfplotmarksize}}
\pgfpathlineto{\pgfpoint{+0.3\pgfplotmarksize}{0.3\pgfplotmarksize}}
\pgfpathlineto{\pgfpoint{+1\pgfplotmarksize}{0.3\pgfplotmarksize}}
\pgfpathlineto{\pgfpoint{+1\pgfplotmarksize}{-0.3\pgfplotmarksize}}
\pgfpathlineto{\pgfpoint{+0.3\pgfplotmarksize}{-0.3\pgfplotmarksize}}
\pgfpathlineto{\pgfpoint{+0.3\pgfplotmarksize}{-1.\pgfplotmarksize}}
\pgfpathlineto{\pgfpoint{-0.3\pgfplotmarksize}{-1.\pgfplotmarksize}}
\pgfpathlineto{\pgfpoint{-0.3\pgfplotmarksize}{-0.3\pgfplotmarksize}}
\pgfpathlineto{\pgfpoint{-1.\pgfplotmarksize}{-0.3\pgfplotmarksize}}
\pgfpathlineto{\pgfpoint{-1.\pgfplotmarksize}{0.3\pgfplotmarksize}}
\pgfpathlineto{\pgfpoint{-0.3\pgfplotmarksize}{0.3\pgfplotmarksize}}
\pgfpathclose
\pgfusepathqstroke
}
\pgfdeclareplotmark{cross*} {
\pgfpathmoveto{\pgfpoint{-0.3\pgfplotmarksize}{\pgfplotmarksize}}
\pgfpathlineto{\pgfpoint{+0.3\pgfplotmarksize}{\pgfplotmarksize}}
\pgfpathlineto{\pgfpoint{+0.3\pgfplotmarksize}{0.3\pgfplotmarksize}}
\pgfpathlineto{\pgfpoint{+1\pgfplotmarksize}{0.3\pgfplotmarksize}}
\pgfpathlineto{\pgfpoint{+1\pgfplotmarksize}{-0.3\pgfplotmarksize}}
\pgfpathlineto{\pgfpoint{+0.3\pgfplotmarksize}{-0.3\pgfplotmarksize}}
\pgfpathlineto{\pgfpoint{+0.3\pgfplotmarksize}{-1.\pgfplotmarksize}}
\pgfpathlineto{\pgfpoint{-0.3\pgfplotmarksize}{-1.\pgfplotmarksize}}
\pgfpathlineto{\pgfpoint{-0.3\pgfplotmarksize}{-0.3\pgfplotmarksize}}
\pgfpathlineto{\pgfpoint{-1.\pgfplotmarksize}{-0.3\pgfplotmarksize}}
\pgfpathlineto{\pgfpoint{-1.\pgfplotmarksize}{0.3\pgfplotmarksize}}
\pgfpathlineto{\pgfpoint{-0.3\pgfplotmarksize}{0.3\pgfplotmarksize}}
\pgfpathclose
\pgfusepathqfillstroke
}
\pgfdeclareplotmark{newstar} {
\pgfpathmoveto{\pgfqpoint{0pt}{\pgfplotmarksize}}
\pgfpathlineto{\pgfqpointpolar{44}{0.5\pgfplotmarksize}}
\pgfpathlineto{\pgfqpointpolar{18}{\pgfplotmarksize}}
\pgfpathlineto{\pgfqpointpolar{-20}{0.5\pgfplotmarksize}}
\pgfpathlineto{\pgfqpointpolar{-54}{\pgfplotmarksize}}
\pgfpathlineto{\pgfqpointpolar{-90}{0.5\pgfplotmarksize}}
\pgfpathlineto{\pgfqpointpolar{234}{\pgfplotmarksize}}
\pgfpathlineto{\pgfqpointpolar{198}{0.5\pgfplotmarksize}}
\pgfpathlineto{\pgfqpointpolar{162}{\pgfplotmarksize}}
\pgfpathlineto{\pgfqpointpolar{134}{0.5\pgfplotmarksize}}
\pgfpathclose
\pgfusepathqstroke
}
\pgfdeclareplotmark{newstar*} {
\pgfpathmoveto{\pgfqpoint{0pt}{\pgfplotmarksize}}
\pgfpathlineto{\pgfqpointpolar{44}{0.5\pgfplotmarksize}}
\pgfpathlineto{\pgfqpointpolar{18}{\pgfplotmarksize}}
\pgfpathlineto{\pgfqpointpolar{-20}{0.5\pgfplotmarksize}}
\pgfpathlineto{\pgfqpointpolar{-54}{\pgfplotmarksize}}
\pgfpathlineto{\pgfqpointpolar{-90}{0.5\pgfplotmarksize}}
\pgfpathlineto{\pgfqpointpolar{234}{\pgfplotmarksize}}
\pgfpathlineto{\pgfqpointpolar{198}{0.5\pgfplotmarksize}}
\pgfpathlineto{\pgfqpointpolar{162}{\pgfplotmarksize}}
\pgfpathlineto{\pgfqpointpolar{134}{0.5\pgfplotmarksize}}
\pgfpathclose
\pgfusepathqfillstroke
}
\definecolor{c}{rgb}{1,1,1};
\draw [color=c, fill=c] (0,0) rectangle (10,6.80516);
\draw [color=c, fill=c] (1,0.680516) rectangle (9.95,6.73711);
\definecolor{c}{rgb}{0,0,0};
\draw [c] (1,0.680516) -- (1,6.73711) -- (9.95,6.73711) -- (9.95,0.680516) -- (1,0.680516);
\definecolor{c}{rgb}{1,1,1};
\draw [color=c, fill=c] (1,0.680516) rectangle (9.95,6.73711);
\definecolor{c}{rgb}{0,0,0};
\draw [c] (1,0.680516) -- (1,6.73711) -- (9.95,6.73711) -- (9.95,0.680516) -- (1,0.680516);
\colorlet{c}{natgreen};
\draw [c] (1.0895,6.55881) -- (1.0895,6.5615);
\draw [c] (1.0895,6.5615) -- (1.0895,6.56416);
\draw [c] (1,6.5615) -- (1.0895,6.5615);
\draw [c] (1.0895,6.5615) -- (1.179,6.5615);
\definecolor{c}{rgb}{0,0,0};
\colorlet{c}{natgreen};
\draw [c] (1.2685,5.51874) -- (1.2685,5.52169);
\draw [c] (1.2685,5.52169) -- (1.2685,5.5246);
\draw [c] (1.179,5.52169) -- (1.2685,5.52169);
\draw [c] (1.2685,5.52169) -- (1.358,5.52169);
\definecolor{c}{rgb}{0,0,0};
\colorlet{c}{natgreen};
\draw [c] (1.4475,4.90057) -- (1.4475,4.90978);
\draw [c] (1.4475,4.90978) -- (1.4475,4.91869);
\draw [c] (1.358,4.90978) -- (1.4475,4.90978);
\draw [c] (1.4475,4.90978) -- (1.537,4.90978);
\definecolor{c}{rgb}{0,0,0};
\colorlet{c}{natgreen};
\draw [c] (1.6265,4.47229) -- (1.6265,4.49235);
\draw [c] (1.6265,4.49235) -- (1.6265,4.51104);
\draw [c] (1.537,4.49235) -- (1.6265,4.49235);
\draw [c] (1.6265,4.49235) -- (1.716,4.49235);
\definecolor{c}{rgb}{0,0,0};
\colorlet{c}{natgreen};
\draw [c] (1.8055,4.09785) -- (1.8055,4.13602);
\draw [c] (1.8055,4.13602) -- (1.8055,4.16946);
\draw [c] (1.716,4.13602) -- (1.8055,4.13602);
\draw [c] (1.8055,4.13602) -- (1.895,4.13602);
\definecolor{c}{rgb}{0,0,0};
\colorlet{c}{natgreen};
\draw [c] (1.9845,3.81744) -- (1.9845,3.87335);
\draw [c] (1.9845,3.87335) -- (1.9845,3.91966);
\draw [c] (1.895,3.87335) -- (1.9845,3.87335);
\draw [c] (1.9845,3.87335) -- (2.074,3.87335);
\definecolor{c}{rgb}{0,0,0};
\colorlet{c}{natgreen};
\draw [c] (2.1635,3.59204) -- (2.1635,3.62906);
\draw [c] (2.1635,3.62906) -- (2.1635,3.66163);
\draw [c] (2.074,3.62906) -- (2.1635,3.62906);
\draw [c] (2.1635,3.62906) -- (2.253,3.62906);
\definecolor{c}{rgb}{0,0,0};
\colorlet{c}{natgreen};
\draw [c] (2.3425,3.37893) -- (2.3425,3.38235);
\draw [c] (2.3425,3.38235) -- (2.3425,3.38572);
\draw [c] (2.253,3.38235) -- (2.3425,3.38235);
\draw [c] (2.3425,3.38235) -- (2.432,3.38235);
\definecolor{c}{rgb}{0,0,0};
\colorlet{c}{natgreen};
\draw [c] (2.5215,3.14928) -- (2.5215,3.1545);
\draw [c] (2.5215,3.1545) -- (2.5215,3.15963);
\draw [c] (2.432,3.1545) -- (2.5215,3.1545);
\draw [c] (2.5215,3.1545) -- (2.611,3.1545);
\definecolor{c}{rgb}{0,0,0};
\colorlet{c}{natgreen};
\draw [c] (2.7005,2.94084) -- (2.7005,2.94851);
\draw [c] (2.7005,2.94851) -- (2.7005,2.95598);
\draw [c] (2.611,2.94851) -- (2.7005,2.94851);
\draw [c] (2.7005,2.94851) -- (2.79,2.94851);
\definecolor{c}{rgb}{0,0,0};
\colorlet{c}{natgreen};
\draw [c] (2.8795,2.74741) -- (2.8795,2.75839);
\draw [c] (2.8795,2.75839) -- (2.8795,2.76893);
\draw [c] (2.79,2.75839) -- (2.8795,2.75839);
\draw [c] (2.8795,2.75839) -- (2.969,2.75839);
\definecolor{c}{rgb}{0,0,0};
\colorlet{c}{natgreen};
\draw [c] (3.0585,2.53077) -- (3.0585,2.54715);
\draw [c] (3.0585,2.54715) -- (3.0585,2.56259);
\draw [c] (2.969,2.54715) -- (3.0585,2.54715);
\draw [c] (3.0585,2.54715) -- (3.148,2.54715);
\definecolor{c}{rgb}{0,0,0};
\colorlet{c}{natgreen};
\draw [c] (3.2375,2.3282) -- (3.2375,2.35201);
\draw [c] (3.2375,2.35201) -- (3.2375,2.37389);
\draw [c] (3.148,2.35201) -- (3.2375,2.35201);
\draw [c] (3.2375,2.35201) -- (3.327,2.35201);
\definecolor{c}{rgb}{0,0,0};
\colorlet{c}{natgreen};
\draw [c] (3.4165,2.11539) -- (3.4165,2.15065);
\draw [c] (3.4165,2.15065) -- (3.4165,2.18184);
\draw [c] (3.327,2.15065) -- (3.4165,2.15065);
\draw [c] (3.4165,2.15065) -- (3.506,2.15065);
\definecolor{c}{rgb}{0,0,0};
\colorlet{c}{natgreen};
\draw [c] (3.5955,1.94137) -- (3.5955,1.98997);
\draw [c] (3.5955,1.98997) -- (3.5955,2.03116);
\draw [c] (3.506,1.98997) -- (3.5955,1.98997);
\draw [c] (3.5955,1.98997) -- (3.685,1.98997);
\definecolor{c}{rgb}{0,0,0};
\colorlet{c}{natgreen};
\draw [c] (3.7745,1.82349) -- (3.7745,1.88388);
\draw [c] (3.7745,1.88388) -- (3.7745,1.93322);
\draw [c] (3.685,1.88388) -- (3.7745,1.88388);
\draw [c] (3.7745,1.88388) -- (3.864,1.88388);
\definecolor{c}{rgb}{0,0,0};
\colorlet{c}{natgreen};
\draw [c] (3.9535,1.64282) -- (3.9535,1.72697);
\draw [c] (3.9535,1.72697) -- (3.9535,1.79107);
\draw [c] (3.864,1.72697) -- (3.9535,1.72697);
\draw [c] (3.9535,1.72697) -- (4.043,1.72697);
\definecolor{c}{rgb}{0,0,0};
\colorlet{c}{natgreen};
\draw [c] (4.1325,1.20039) -- (4.1325,1.38796);
\draw [c] (4.1325,1.38796) -- (4.1325,1.49768);
\draw [c] (4.043,1.38796) -- (4.1325,1.38796);
\draw [c] (4.1325,1.38796) -- (4.222,1.38796);
\definecolor{c}{rgb}{0,0,0};
\colorlet{c}{natgreen};
\draw [c] (4.3115,0.86809) -- (4.3115,1.20039);
\draw [c] (4.3115,1.20039) -- (4.3115,1.34511);
\draw [c] (4.222,1.20039) -- (4.3115,1.20039);
\draw [c] (4.3115,1.20039) -- (4.401,1.20039);
\definecolor{c}{rgb}{0,0,0};
\colorlet{c}{natgreen};
\draw [c] (4.4905,0.680516) -- (4.4905,1.01281);
\draw [c] (4.4905,1.01281) -- (4.4905,1.20039);
\draw [c] (4.401,1.01281) -- (4.4905,1.01281);
\draw [c] (4.4905,1.01281) -- (4.58,1.01281);
\definecolor{c}{rgb}{0,0,0};
\colorlet{c}{natgreen};
\draw [c] (4.6695,0.680516) -- (4.6695,1.01281);
\draw [c] (4.6695,1.01281) -- (4.6695,1.20039);
\draw [c] (4.58,1.01281) -- (4.6695,1.01281);
\draw [c] (4.6695,1.01281) -- (4.759,1.01281);
\definecolor{c}{rgb}{0,0,0};
\colorlet{c}{natgreen};
\draw [c] (4.8485,0.680516) -- (4.8485,1.01281);
\draw [c] (4.8485,1.01281) -- (4.8485,1.20039);
\draw [c] (4.759,1.01281) -- (4.8485,1.01281);
\draw [c] (4.8485,1.01281) -- (4.938,1.01281);
\definecolor{c}{rgb}{0,0,0};
\colorlet{c}{natgreen};
\draw [c] (5.0275,0.680516) -- (5.0275,1.01281);
\draw [c] (5.0275,1.01281) -- (5.0275,1.20039);
\draw [c] (4.938,1.01281) -- (5.0275,1.01281);
\draw [c] (5.0275,1.01281) -- (5.117,1.01281);
\definecolor{c}{rgb}{0,0,0};
\colorlet{c}{natgreen};
\draw [c] (5.3855,0.86809) -- (5.3855,1.20039);
\draw [c] (5.3855,1.20039) -- (5.3855,1.34511);
\draw [c] (5.296,1.20039) -- (5.3855,1.20039);
\draw [c] (5.3855,1.20039) -- (5.475,1.20039);
\definecolor{c}{rgb}{0,0,0};
\draw [c] (1,0.680516) -- (9.95,0.680516);
\draw [anchor= east] (9.95,-0.0816619) node[color=c, rotate=0]{$p_{T}^{\gamma_{1}} [GeV]$};
\draw [c] (1,0.863234) -- (1,0.680516);
\draw [c] (1.4475,0.771875) -- (1.4475,0.680516);
\draw [c] (1.895,0.771875) -- (1.895,0.680516);
\draw [c] (2.3425,0.771875) -- (2.3425,0.680516);
\draw [c] (2.79,0.771875) -- (2.79,0.680516);
\draw [c] (3.2375,0.863234) -- (3.2375,0.680516);
\draw [c] (3.685,0.771875) -- (3.685,0.680516);
\draw [c] (4.1325,0.771875) -- (4.1325,0.680516);
\draw [c] (4.58,0.771875) -- (4.58,0.680516);
\draw [c] (5.0275,0.771875) -- (5.0275,0.680516);
\draw [c] (5.475,0.863234) -- (5.475,0.680516);
\draw [c] (5.9225,0.771875) -- (5.9225,0.680516);
\draw [c] (6.37,0.771875) -- (6.37,0.680516);
\draw [c] (6.8175,0.771875) -- (6.8175,0.680516);
\draw [c] (7.265,0.771875) -- (7.265,0.680516);
\draw [c] (7.7125,0.863234) -- (7.7125,0.680516);
\draw [c] (8.16,0.771875) -- (8.16,0.680516);
\draw [c] (8.6075,0.771875) -- (8.6075,0.680516);
\draw [c] (9.055,0.771875) -- (9.055,0.680516);
\draw [c] (9.5025,0.771875) -- (9.5025,0.680516);
\draw [c] (9.95,0.863234) -- (9.95,0.680516);
\draw [c] (9.95,0.863234) -- (9.95,0.680516);
\draw [anchor=base] (1,0.285817) node[color=c, rotate=0]{0};
\draw [anchor=base] (3.2375,0.285817) node[color=c, rotate=0]{1000};
\draw [anchor=base] (5.475,0.285817) node[color=c, rotate=0]{2000};
\draw [anchor=base] (7.7125,0.285817) node[color=c, rotate=0]{3000};
\draw [anchor=base] (9.95,0.285817) node[color=c, rotate=0]{4000};
\draw [c] (1,0.680516) -- (1,6.73711);
\draw [anchor= east] (-0.4,6.73711) node[color=c, rotate=90]{$d\sigma/d(p_{T}^{\gamma_{1}}) [pb/GeV]$};
\draw [c] (1.1335,0.71262) -- (1,0.71262);
\draw [c] (1.1335,0.744493) -- (1,0.744493);
\draw [c] (1.267,0.773005) -- (1,0.773005);
\draw [anchor= east] (0.922,0.773005) node[color=c, rotate=0]{$10^{-10}$};
\draw [c] (1.1335,0.960579) -- (1,0.960579);
\draw [c] (1.1335,1.0703) -- (1,1.0703);
\draw [c] (1.1335,1.14815) -- (1,1.14815);
\draw [c] (1.1335,1.20854) -- (1,1.20854);
\draw [c] (1.1335,1.25788) -- (1,1.25788);
\draw [c] (1.1335,1.29959) -- (1,1.29959);
\draw [c] (1.1335,1.33573) -- (1,1.33573);
\draw [c] (1.1335,1.3676) -- (1,1.3676);
\draw [c] (1.267,1.39611) -- (1,1.39611);
\draw [anchor= east] (0.922,1.39611) node[color=c, rotate=0]{$10^{-9}$};
\draw [c] (1.1335,1.58369) -- (1,1.58369);
\draw [c] (1.1335,1.69341) -- (1,1.69341);
\draw [c] (1.1335,1.77126) -- (1,1.77126);
\draw [c] (1.1335,1.83165) -- (1,1.83165);
\draw [c] (1.1335,1.88098) -- (1,1.88098);
\draw [c] (1.1335,1.9227) -- (1,1.9227);
\draw [c] (1.1335,1.95883) -- (1,1.95883);
\draw [c] (1.1335,1.99071) -- (1,1.99071);
\draw [c] (1.267,2.01922) -- (1,2.01922);
\draw [anchor= east] (0.922,2.01922) node[color=c, rotate=0]{$10^{-8}$};
\draw [c] (1.1335,2.20679) -- (1,2.20679);
\draw [c] (1.1335,2.31652) -- (1,2.31652);
\draw [c] (1.1335,2.39437) -- (1,2.39437);
\draw [c] (1.1335,2.45475) -- (1,2.45475);
\draw [c] (1.1335,2.50409) -- (1,2.50409);
\draw [c] (1.1335,2.54581) -- (1,2.54581);
\draw [c] (1.1335,2.58194) -- (1,2.58194);
\draw [c] (1.1335,2.61381) -- (1,2.61381);
\draw [c] (1.267,2.64233) -- (1,2.64233);
\draw [anchor= east] (0.922,2.64233) node[color=c, rotate=0]{$10^{-7}$};
\draw [c] (1.1335,2.8299) -- (1,2.8299);
\draw [c] (1.1335,2.93962) -- (1,2.93962);
\draw [c] (1.1335,3.01747) -- (1,3.01747);
\draw [c] (1.1335,3.07786) -- (1,3.07786);
\draw [c] (1.1335,3.1272) -- (1,3.1272);
\draw [c] (1.1335,3.16891) -- (1,3.16891);
\draw [c] (1.1335,3.20505) -- (1,3.20505);
\draw [c] (1.1335,3.23692) -- (1,3.23692);
\draw [c] (1.267,3.26543) -- (1,3.26543);
\draw [anchor= east] (0.922,3.26543) node[color=c, rotate=0]{$10^{-6}$};
\draw [c] (1.1335,3.45301) -- (1,3.45301);
\draw [c] (1.1335,3.56273) -- (1,3.56273);
\draw [c] (1.1335,3.64058) -- (1,3.64058);
\draw [c] (1.1335,3.70097) -- (1,3.70097);
\draw [c] (1.1335,3.7503) -- (1,3.7503);
\draw [c] (1.1335,3.79202) -- (1,3.79202);
\draw [c] (1.1335,3.82815) -- (1,3.82815);
\draw [c] (1.1335,3.86003) -- (1,3.86003);
\draw [c] (1.267,3.88854) -- (1,3.88854);
\draw [anchor= east] (0.922,3.88854) node[color=c, rotate=0]{$10^{-5}$};
\draw [c] (1.1335,4.07611) -- (1,4.07611);
\draw [c] (1.1335,4.18584) -- (1,4.18584);
\draw [c] (1.1335,4.26369) -- (1,4.26369);
\draw [c] (1.1335,4.32407) -- (1,4.32407);
\draw [c] (1.1335,4.37341) -- (1,4.37341);
\draw [c] (1.1335,4.41513) -- (1,4.41513);
\draw [c] (1.1335,4.45126) -- (1,4.45126);
\draw [c] (1.1335,4.48314) -- (1,4.48314);
\draw [c] (1.267,4.51165) -- (1,4.51165);
\draw [anchor= east] (0.922,4.51165) node[color=c, rotate=0]{$10^{-4}$};
\draw [c] (1.1335,4.69922) -- (1,4.69922);
\draw [c] (1.1335,4.80894) -- (1,4.80894);
\draw [c] (1.1335,4.88679) -- (1,4.88679);
\draw [c] (1.1335,4.94718) -- (1,4.94718);
\draw [c] (1.1335,4.99652) -- (1,4.99652);
\draw [c] (1.1335,5.03823) -- (1,5.03823);
\draw [c] (1.1335,5.07437) -- (1,5.07437);
\draw [c] (1.1335,5.10624) -- (1,5.10624);
\draw [c] (1.267,5.13475) -- (1,5.13475);
\draw [anchor= east] (0.922,5.13475) node[color=c, rotate=0]{$10^{-3}$};
\draw [c] (1.1335,5.32233) -- (1,5.32233);
\draw [c] (1.1335,5.43205) -- (1,5.43205);
\draw [c] (1.1335,5.5099) -- (1,5.5099);
\draw [c] (1.1335,5.57029) -- (1,5.57029);
\draw [c] (1.1335,5.61963) -- (1,5.61963);
\draw [c] (1.1335,5.66134) -- (1,5.66134);
\draw [c] (1.1335,5.69748) -- (1,5.69748);
\draw [c] (1.1335,5.72935) -- (1,5.72935);
\draw [c] (1.267,5.75786) -- (1,5.75786);
\draw [anchor= east] (0.922,5.75786) node[color=c, rotate=0]{$10^{-2}$};
\draw [c] (1.1335,5.94543) -- (1,5.94543);
\draw [c] (1.1335,6.05516) -- (1,6.05516);
\draw [c] (1.1335,6.13301) -- (1,6.13301);
\draw [c] (1.1335,6.19339) -- (1,6.19339);
\draw [c] (1.1335,6.24273) -- (1,6.24273);
\draw [c] (1.1335,6.28445) -- (1,6.28445);
\draw [c] (1.1335,6.32058) -- (1,6.32058);
\draw [c] (1.1335,6.35246) -- (1,6.35246);
\draw [c] (1.267,6.38097) -- (1,6.38097);
\draw [anchor= east] (0.922,6.38097) node[color=c, rotate=0]{$10^{-1}$};
\draw [c] (1.1335,6.56854) -- (1,6.56854);
\draw [c] (1.1335,6.67827) -- (1,6.67827);
\colorlet{c}{natcomp};
\draw [c] (1.0895,6.55776) -- (1.0895,6.56046);
\draw [c] (1.0895,6.56046) -- (1.0895,6.56312);
\draw [c] (1,6.56046) -- (1.0895,6.56046);
\draw [c] (1.0895,6.56046) -- (1.179,6.56046);
\definecolor{c}{rgb}{0,0,0};
\colorlet{c}{natcomp};
\draw [c] (1.2685,5.52168) -- (1.2685,5.52461);
\draw [c] (1.2685,5.52461) -- (1.2685,5.52751);
\draw [c] (1.179,5.52461) -- (1.2685,5.52461);
\draw [c] (1.2685,5.52461) -- (1.358,5.52461);
\definecolor{c}{rgb}{0,0,0};
\colorlet{c}{natcomp};
\draw [c] (1.4475,4.88705) -- (1.4475,4.8965);
\draw [c] (1.4475,4.8965) -- (1.4475,4.90563);
\draw [c] (1.358,4.8965) -- (1.4475,4.8965);
\draw [c] (1.4475,4.8965) -- (1.537,4.8965);
\definecolor{c}{rgb}{0,0,0};
\colorlet{c}{natcomp};
\draw [c] (1.6265,4.48438) -- (1.6265,4.50397);
\draw [c] (1.6265,4.50397) -- (1.6265,4.52224);
\draw [c] (1.537,4.50397) -- (1.6265,4.50397);
\draw [c] (1.6265,4.50397) -- (1.716,4.50397);
\definecolor{c}{rgb}{0,0,0};
\colorlet{c}{natcomp};
\draw [c] (1.8055,4.17612) -- (1.8055,4.20894);
\draw [c] (1.8055,4.20894) -- (1.8055,4.23821);
\draw [c] (1.716,4.20894) -- (1.8055,4.20894);
\draw [c] (1.8055,4.20894) -- (1.895,4.20894);
\definecolor{c}{rgb}{0,0,0};
\colorlet{c}{natcomp};
\draw [c] (1.9845,3.8824) -- (1.9845,3.92737);
\draw [c] (1.9845,3.92737) -- (1.9845,3.96593);
\draw [c] (1.895,3.92737) -- (1.9845,3.92737);
\draw [c] (1.9845,3.92737) -- (2.074,3.92737);
\definecolor{c}{rgb}{0,0,0};
\colorlet{c}{natcomp};
\draw [c] (2.1635,3.78832) -- (2.1635,3.81429);
\draw [c] (2.1635,3.81429) -- (2.1635,3.83798);
\draw [c] (2.074,3.81429) -- (2.1635,3.81429);
\draw [c] (2.1635,3.81429) -- (2.253,3.81429);
\definecolor{c}{rgb}{0,0,0};
\colorlet{c}{natcomp};
\draw [c] (2.3425,3.66137) -- (2.3425,3.66432);
\draw [c] (2.3425,3.66432) -- (2.3425,3.66725);
\draw [c] (2.253,3.66432) -- (2.3425,3.66432);
\draw [c] (2.3425,3.66432) -- (2.432,3.66432);
\definecolor{c}{rgb}{0,0,0};
\colorlet{c}{natcomp};
\draw [c] (2.5215,3.56652) -- (2.5215,3.57004);
\draw [c] (2.5215,3.57004) -- (2.5215,3.57352);
\draw [c] (2.432,3.57004) -- (2.5215,3.57004);
\draw [c] (2.5215,3.57004) -- (2.611,3.57004);
\definecolor{c}{rgb}{0,0,0};
\colorlet{c}{natcomp};
\draw [c] (2.7005,3.48411) -- (2.7005,3.48821);
\draw [c] (2.7005,3.48821) -- (2.7005,3.49225);
\draw [c] (2.611,3.48821) -- (2.7005,3.48821);
\draw [c] (2.7005,3.48821) -- (2.79,3.48821);
\definecolor{c}{rgb}{0,0,0};
\colorlet{c}{natcomp};
\draw [c] (2.8795,3.43808) -- (2.8795,3.44254);
\draw [c] (2.8795,3.44254) -- (2.8795,3.44693);
\draw [c] (2.79,3.44254) -- (2.8795,3.44254);
\draw [c] (2.8795,3.44254) -- (2.969,3.44254);
\definecolor{c}{rgb}{0,0,0};
\colorlet{c}{natcomp};
\draw [c] (3.0585,3.38663) -- (3.0585,3.39153);
\draw [c] (3.0585,3.39153) -- (3.0585,3.39636);
\draw [c] (2.969,3.39153) -- (3.0585,3.39153);
\draw [c] (3.0585,3.39153) -- (3.148,3.39153);
\definecolor{c}{rgb}{0,0,0};
\colorlet{c}{natcomp};
\draw [c] (3.2375,3.35711) -- (3.2375,3.36229);
\draw [c] (3.2375,3.36229) -- (3.2375,3.36738);
\draw [c] (3.148,3.36229) -- (3.2375,3.36229);
\draw [c] (3.2375,3.36229) -- (3.327,3.36229);
\definecolor{c}{rgb}{0,0,0};
\colorlet{c}{natcomp};
\draw [c] (3.4165,3.3088) -- (3.4165,3.31447);
\draw [c] (3.4165,3.31447) -- (3.4165,3.32002);
\draw [c] (3.327,3.31447) -- (3.4165,3.31447);
\draw [c] (3.4165,3.31447) -- (3.506,3.31447);
\definecolor{c}{rgb}{0,0,0};
\colorlet{c}{natcomp};
\draw [c] (3.5955,3.25835) -- (3.5955,3.26457);
\draw [c] (3.5955,3.26457) -- (3.5955,3.27065);
\draw [c] (3.506,3.26457) -- (3.5955,3.26457);
\draw [c] (3.5955,3.26457) -- (3.685,3.26457);
\definecolor{c}{rgb}{0,0,0};
\colorlet{c}{natcomp};
\draw [c] (3.7745,3.21157) -- (3.7745,3.21835);
\draw [c] (3.7745,3.21835) -- (3.7745,3.22497);
\draw [c] (3.685,3.21835) -- (3.7745,3.21835);
\draw [c] (3.7745,3.21835) -- (3.864,3.21835);
\definecolor{c}{rgb}{0,0,0};
\colorlet{c}{natcomp};
\draw [c] (3.9535,3.16677) -- (3.9535,3.17413);
\draw [c] (3.9535,3.17413) -- (3.9535,3.18131);
\draw [c] (3.864,3.17413) -- (3.9535,3.17413);
\draw [c] (3.9535,3.17413) -- (4.043,3.17413);
\definecolor{c}{rgb}{0,0,0};
\colorlet{c}{natcomp};
\draw [c] (4.1325,3.11695) -- (4.1325,3.12503);
\draw [c] (4.1325,3.12503) -- (4.1325,3.13287);
\draw [c] (4.043,3.12503) -- (4.1325,3.12503);
\draw [c] (4.1325,3.12503) -- (4.222,3.12503);
\definecolor{c}{rgb}{0,0,0};
\colorlet{c}{natcomp};
\draw [c] (4.3115,3.06038) -- (4.3115,3.06934);
\draw [c] (4.3115,3.06934) -- (4.3115,3.07803);
\draw [c] (4.222,3.06934) -- (4.3115,3.06934);
\draw [c] (4.3115,3.06934) -- (4.401,3.06934);
\definecolor{c}{rgb}{0,0,0};
\colorlet{c}{natcomp};
\draw [c] (4.4905,2.98967) -- (4.4905,2.99989);
\draw [c] (4.4905,2.99989) -- (4.4905,3.00974);
\draw [c] (4.401,2.99989) -- (4.4905,2.99989);
\draw [c] (4.4905,2.99989) -- (4.58,2.99989);
\definecolor{c}{rgb}{0,0,0};
\colorlet{c}{natcomp};
\draw [c] (4.6695,2.91122) -- (4.6695,2.92303);
\draw [c] (4.6695,2.92303) -- (4.6695,2.93435);
\draw [c] (4.58,2.92303) -- (4.6695,2.92303);
\draw [c] (4.6695,2.92303) -- (4.759,2.92303);
\definecolor{c}{rgb}{0,0,0};
\colorlet{c}{natcomp};
\draw [c] (4.8485,2.86153) -- (4.8485,2.87448);
\draw [c] (4.8485,2.87448) -- (4.8485,2.88684);
\draw [c] (4.759,2.87448) -- (4.8485,2.87448);
\draw [c] (4.8485,2.87448) -- (4.938,2.87448);
\definecolor{c}{rgb}{0,0,0};
\colorlet{c}{natcomp};
\draw [c] (5.0275,2.74667) -- (5.0275,2.76268);
\draw [c] (5.0275,2.76268) -- (5.0275,2.7778);
\draw [c] (4.938,2.76268) -- (5.0275,2.76268);
\draw [c] (5.0275,2.76268) -- (5.117,2.76268);
\definecolor{c}{rgb}{0,0,0};
\colorlet{c}{natcomp};
\draw [c] (5.2065,2.69961) -- (5.2065,2.71707);
\draw [c] (5.2065,2.71707) -- (5.2065,2.73348);
\draw [c] (5.117,2.71707) -- (5.2065,2.71707);
\draw [c] (5.2065,2.71707) -- (5.296,2.71707);
\definecolor{c}{rgb}{0,0,0};
\colorlet{c}{natcomp};
\draw [c] (5.3855,2.57967) -- (5.3855,2.60147);
\draw [c] (5.3855,2.60147) -- (5.3855,2.62164);
\draw [c] (5.296,2.60147) -- (5.3855,2.60147);
\draw [c] (5.3855,2.60147) -- (5.475,2.60147);
\definecolor{c}{rgb}{0,0,0};
\colorlet{c}{natcomp};
\draw [c] (5.5645,2.48849) -- (5.5645,2.51428);
\draw [c] (5.5645,2.51428) -- (5.5645,2.53782);
\draw [c] (5.475,2.51428) -- (5.5645,2.51428);
\draw [c] (5.5645,2.51428) -- (5.654,2.51428);
\definecolor{c}{rgb}{0,0,0};
\colorlet{c}{natcomp};
\draw [c] (5.7435,2.391) -- (5.7435,2.42188);
\draw [c] (5.7435,2.42188) -- (5.7435,2.44959);
\draw [c] (5.654,2.42188) -- (5.7435,2.42188);
\draw [c] (5.7435,2.42188) -- (5.833,2.42188);
\definecolor{c}{rgb}{0,0,0};
\colorlet{c}{natcomp};
\draw [c] (5.9225,2.25636) -- (5.9225,2.29595);
\draw [c] (5.9225,2.29595) -- (5.9225,2.33047);
\draw [c] (5.833,2.29595) -- (5.9225,2.29595);
\draw [c] (5.9225,2.29595) -- (6.012,2.29595);
\definecolor{c}{rgb}{0,0,0};
\colorlet{c}{natcomp};
\draw [c] (6.1015,2.06153) -- (6.1015,2.11821);
\draw [c] (6.1015,2.11821) -- (6.1015,2.16506);
\draw [c] (6.012,2.11821) -- (6.1015,2.11821);
\draw [c] (6.1015,2.11821) -- (6.191,2.11821);
\definecolor{c}{rgb}{0,0,0};
\colorlet{c}{natcomp};
\draw [c] (6.2805,1.88893) -- (6.2805,1.96678);
\draw [c] (6.2805,1.96678) -- (6.2805,2.02716);
\draw [c] (6.191,1.96678) -- (6.2805,1.96678);
\draw [c] (6.2805,1.96678) -- (6.37,1.96678);
\definecolor{c}{rgb}{0,0,0};
\colorlet{c}{natcomp};
\draw [c] (6.4595,1.79675) -- (6.4595,1.88893);
\draw [c] (6.4595,1.88893) -- (6.4595,1.95756);
\draw [c] (6.37,1.88893) -- (6.4595,1.88893);
\draw [c] (6.4595,1.88893) -- (6.549,1.88893);
\definecolor{c}{rgb}{0,0,0};
\colorlet{c}{natcomp};
\draw [c] (6.6385,1.55937) -- (6.6385,1.70135);
\draw [c] (6.6385,1.70135) -- (6.6385,1.79399);
\draw [c] (6.549,1.70135) -- (6.6385,1.70135);
\draw [c] (6.6385,1.70135) -- (6.728,1.70135);
\definecolor{c}{rgb}{0,0,0};
\colorlet{c}{natcomp};
\draw [c] (6.8175,1.4916) -- (6.8175,1.65201);
\draw [c] (6.8175,1.65201) -- (6.8175,1.75204);
\draw [c] (6.728,1.65201) -- (6.8175,1.65201);
\draw [c] (6.8175,1.65201) -- (6.907,1.65201);
\definecolor{c}{rgb}{0,0,0};
\colorlet{c}{natcomp};
\draw [c] (6.9965,0.680516) -- (6.9965,1.21648);
\draw [c] (6.9965,1.21648) -- (6.9965,1.40405);
\draw [c] (6.907,1.21648) -- (6.9965,1.21648);
\draw [c] (6.9965,1.21648) -- (7.086,1.21648);
\definecolor{c}{rgb}{0,0,0};
\colorlet{c}{natcomp};
\draw [c] (8.9655,0.680516) -- (8.9655,1.21648);
\draw [c] (8.9655,1.21648) -- (8.9655,1.40405);
\draw [c] (8.876,1.21648) -- (8.9655,1.21648);
\draw [c] (8.9655,1.21648) -- (9.055,1.21648);
\definecolor{c}{rgb}{0,0,0};
\definecolor{c}{rgb}{1,1,1};
\draw [color=c, fill=c] (6,5.44413) rectangle (9.8,6.66905);
\definecolor{c}{rgb}{0,0,0};
\draw [anchor=base west] (6.95,6.22502) node[color=c, rotate=0]{Standard Model};
\colorlet{c}{natgreen};
\draw [c] (6.1425,6.36282) -- (6.8075,6.36282);
\definecolor{c}{rgb}{0,0,0};
\draw [anchor=base west] (6.95,5.61255) node[color=c, rotate=0]{$\Lambda = 1.0 TeV$};
\colorlet{c}{natcomp};
\draw [c] (6.1425,5.75036) -- (6.8075,5.75036);
\end{tikzpicture}
}
\end{infilsf}
\vspace{-1em}
\subcaption{Significance: 28.0}
\end{minipage}

\vspace{1em}

\noindent
\begin{minipage}[b]{.499\textwidth}
\begin{infilsf} \tiny
\hspace{-.2em}\makebox[.96\textwidth]{\begin{tikzpicture}[x=.095\textwidth,y=.095\textwidth]
\pgfdeclareplotmark{cross} {
\pgfpathmoveto{\pgfpoint{-0.3\pgfplotmarksize}{\pgfplotmarksize}}
\pgfpathlineto{\pgfpoint{+0.3\pgfplotmarksize}{\pgfplotmarksize}}
\pgfpathlineto{\pgfpoint{+0.3\pgfplotmarksize}{0.3\pgfplotmarksize}}
\pgfpathlineto{\pgfpoint{+1\pgfplotmarksize}{0.3\pgfplotmarksize}}
\pgfpathlineto{\pgfpoint{+1\pgfplotmarksize}{-0.3\pgfplotmarksize}}
\pgfpathlineto{\pgfpoint{+0.3\pgfplotmarksize}{-0.3\pgfplotmarksize}}
\pgfpathlineto{\pgfpoint{+0.3\pgfplotmarksize}{-1.\pgfplotmarksize}}
\pgfpathlineto{\pgfpoint{-0.3\pgfplotmarksize}{-1.\pgfplotmarksize}}
\pgfpathlineto{\pgfpoint{-0.3\pgfplotmarksize}{-0.3\pgfplotmarksize}}
\pgfpathlineto{\pgfpoint{-1.\pgfplotmarksize}{-0.3\pgfplotmarksize}}
\pgfpathlineto{\pgfpoint{-1.\pgfplotmarksize}{0.3\pgfplotmarksize}}
\pgfpathlineto{\pgfpoint{-0.3\pgfplotmarksize}{0.3\pgfplotmarksize}}
\pgfpathclose
\pgfusepathqstroke
}
\pgfdeclareplotmark{cross*} {
\pgfpathmoveto{\pgfpoint{-0.3\pgfplotmarksize}{\pgfplotmarksize}}
\pgfpathlineto{\pgfpoint{+0.3\pgfplotmarksize}{\pgfplotmarksize}}
\pgfpathlineto{\pgfpoint{+0.3\pgfplotmarksize}{0.3\pgfplotmarksize}}
\pgfpathlineto{\pgfpoint{+1\pgfplotmarksize}{0.3\pgfplotmarksize}}
\pgfpathlineto{\pgfpoint{+1\pgfplotmarksize}{-0.3\pgfplotmarksize}}
\pgfpathlineto{\pgfpoint{+0.3\pgfplotmarksize}{-0.3\pgfplotmarksize}}
\pgfpathlineto{\pgfpoint{+0.3\pgfplotmarksize}{-1.\pgfplotmarksize}}
\pgfpathlineto{\pgfpoint{-0.3\pgfplotmarksize}{-1.\pgfplotmarksize}}
\pgfpathlineto{\pgfpoint{-0.3\pgfplotmarksize}{-0.3\pgfplotmarksize}}
\pgfpathlineto{\pgfpoint{-1.\pgfplotmarksize}{-0.3\pgfplotmarksize}}
\pgfpathlineto{\pgfpoint{-1.\pgfplotmarksize}{0.3\pgfplotmarksize}}
\pgfpathlineto{\pgfpoint{-0.3\pgfplotmarksize}{0.3\pgfplotmarksize}}
\pgfpathclose
\pgfusepathqfillstroke
}
\pgfdeclareplotmark{newstar} {
\pgfpathmoveto{\pgfqpoint{0pt}{\pgfplotmarksize}}
\pgfpathlineto{\pgfqpointpolar{44}{0.5\pgfplotmarksize}}
\pgfpathlineto{\pgfqpointpolar{18}{\pgfplotmarksize}}
\pgfpathlineto{\pgfqpointpolar{-20}{0.5\pgfplotmarksize}}
\pgfpathlineto{\pgfqpointpolar{-54}{\pgfplotmarksize}}
\pgfpathlineto{\pgfqpointpolar{-90}{0.5\pgfplotmarksize}}
\pgfpathlineto{\pgfqpointpolar{234}{\pgfplotmarksize}}
\pgfpathlineto{\pgfqpointpolar{198}{0.5\pgfplotmarksize}}
\pgfpathlineto{\pgfqpointpolar{162}{\pgfplotmarksize}}
\pgfpathlineto{\pgfqpointpolar{134}{0.5\pgfplotmarksize}}
\pgfpathclose
\pgfusepathqstroke
}
\pgfdeclareplotmark{newstar*} {
\pgfpathmoveto{\pgfqpoint{0pt}{\pgfplotmarksize}}
\pgfpathlineto{\pgfqpointpolar{44}{0.5\pgfplotmarksize}}
\pgfpathlineto{\pgfqpointpolar{18}{\pgfplotmarksize}}
\pgfpathlineto{\pgfqpointpolar{-20}{0.5\pgfplotmarksize}}
\pgfpathlineto{\pgfqpointpolar{-54}{\pgfplotmarksize}}
\pgfpathlineto{\pgfqpointpolar{-90}{0.5\pgfplotmarksize}}
\pgfpathlineto{\pgfqpointpolar{234}{\pgfplotmarksize}}
\pgfpathlineto{\pgfqpointpolar{198}{0.5\pgfplotmarksize}}
\pgfpathlineto{\pgfqpointpolar{162}{\pgfplotmarksize}}
\pgfpathlineto{\pgfqpointpolar{134}{0.5\pgfplotmarksize}}
\pgfpathclose
\pgfusepathqfillstroke
}
\definecolor{c}{rgb}{1,1,1};
\draw [color=c, fill=c] (0,0) rectangle (10,6.80516);
\draw [color=c, fill=c] (1,0.680516) rectangle (9,6.12464);
\definecolor{c}{rgb}{0,0,0};
\draw [c] (1,0.680516) -- (1,6.12464) -- (9,6.12464) -- (9,0.680516) -- (1,0.680516);
\definecolor{c}{rgb}{1,1,1};
\draw [color=c, fill=c] (1,0.680516) rectangle (9,6.12464);
\definecolor{c}{rgb}{0,0,0};
\draw [c] (1,0.680516) -- (1,6.12464) -- (9,6.12464) -- (9,0.680516) -- (1,0.680516);
\colorlet{c}{natgreen};
\draw [c] (1.23529,5.61236) -- (1.23529,5.61343);
\draw [c] (1.23529,5.61343) -- (1.23529,5.61449);
\draw [c] (1.15686,5.61343) -- (1.23529,5.61343);
\draw [c] (1.23529,5.61343) -- (1.31373,5.61343);
\definecolor{c}{rgb}{0,0,0};
\colorlet{c}{natgreen};
\draw [c] (1.39216,5.0828) -- (1.39216,5.08585);
\draw [c] (1.39216,5.08585) -- (1.39216,5.08886);
\draw [c] (1.31373,5.08585) -- (1.39216,5.08585);
\draw [c] (1.39216,5.08585) -- (1.47059,5.08585);
\definecolor{c}{rgb}{0,0,0};
\colorlet{c}{natgreen};
\draw [c] (1.54902,4.73895) -- (1.54902,4.745);
\draw [c] (1.54902,4.745) -- (1.54902,4.7509);
\draw [c] (1.47059,4.745) -- (1.54902,4.745);
\draw [c] (1.54902,4.745) -- (1.62745,4.745);
\definecolor{c}{rgb}{0,0,0};
\colorlet{c}{natgreen};
\draw [c] (1.70588,4.47702) -- (1.70588,4.4872);
\draw [c] (1.70588,4.4872) -- (1.70588,4.49697);
\draw [c] (1.62745,4.4872) -- (1.70588,4.4872);
\draw [c] (1.70588,4.4872) -- (1.78431,4.4872);
\definecolor{c}{rgb}{0,0,0};
\colorlet{c}{natgreen};
\draw [c] (1.86275,4.25692) -- (1.86275,4.27268);
\draw [c] (1.86275,4.27268) -- (1.86275,4.28751);
\draw [c] (1.78431,4.27268) -- (1.86275,4.27268);
\draw [c] (1.86275,4.27268) -- (1.94118,4.27268);
\definecolor{c}{rgb}{0,0,0};
\colorlet{c}{natgreen};
\draw [c] (2.01961,4.09363) -- (2.01961,4.11543);
\draw [c] (2.01961,4.11543) -- (2.01961,4.13549);
\draw [c] (1.94118,4.11543) -- (2.01961,4.11543);
\draw [c] (2.01961,4.11543) -- (2.09804,4.11543);
\definecolor{c}{rgb}{0,0,0};
\colorlet{c}{natgreen};
\draw [c] (2.17647,3.87619) -- (2.17647,3.90977);
\draw [c] (2.17647,3.90977) -- (2.17647,3.93939);
\draw [c] (2.09804,3.90977) -- (2.17647,3.90977);
\draw [c] (2.17647,3.90977) -- (2.2549,3.90977);
\definecolor{c}{rgb}{0,0,0};
\colorlet{c}{natgreen};
\draw [c] (2.33333,3.70341) -- (2.33333,3.75072);
\draw [c] (2.33333,3.75072) -- (2.33333,3.79052);
\draw [c] (2.2549,3.75072) -- (2.33333,3.75072);
\draw [c] (2.33333,3.75072) -- (2.41176,3.75072);
\definecolor{c}{rgb}{0,0,0};
\colorlet{c}{natgreen};
\draw [c] (2.4902,3.55365) -- (2.4902,3.61729);
\draw [c] (2.4902,3.61729) -- (2.4902,3.66803);
\draw [c] (2.41176,3.61729) -- (2.4902,3.61729);
\draw [c] (2.4902,3.61729) -- (2.56863,3.61729);
\definecolor{c}{rgb}{0,0,0};
\colorlet{c}{natgreen};
\draw [c] (2.64706,3.52424) -- (2.64706,3.5653);
\draw [c] (2.64706,3.5653) -- (2.64706,3.60059);
\draw [c] (2.56863,3.5653) -- (2.64706,3.5653);
\draw [c] (2.64706,3.5653) -- (2.72549,3.5653);
\definecolor{c}{rgb}{0,0,0};
\colorlet{c}{natgreen};
\draw [c] (2.80392,3.38853) -- (2.80392,3.39051);
\draw [c] (2.80392,3.39051) -- (2.80392,3.39246);
\draw [c] (2.72549,3.39051) -- (2.80392,3.39051);
\draw [c] (2.80392,3.39051) -- (2.88235,3.39051);
\definecolor{c}{rgb}{0,0,0};
\colorlet{c}{natgreen};
\draw [c] (2.96078,3.27434) -- (2.96078,3.27681);
\draw [c] (2.96078,3.27681) -- (2.96078,3.27927);
\draw [c] (2.88235,3.27681) -- (2.96078,3.27681);
\draw [c] (2.96078,3.27681) -- (3.03922,3.27681);
\definecolor{c}{rgb}{0,0,0};
\colorlet{c}{natgreen};
\draw [c] (3.11765,3.15387) -- (3.11765,3.15702);
\draw [c] (3.11765,3.15702) -- (3.11765,3.16012);
\draw [c] (3.03922,3.15702) -- (3.11765,3.15702);
\draw [c] (3.11765,3.15702) -- (3.19608,3.15702);
\definecolor{c}{rgb}{0,0,0};
\colorlet{c}{natgreen};
\draw [c] (3.27451,3.04803) -- (3.27451,3.05191);
\draw [c] (3.27451,3.05191) -- (3.27451,3.05573);
\draw [c] (3.19608,3.05191) -- (3.27451,3.05191);
\draw [c] (3.27451,3.05191) -- (3.35294,3.05191);
\definecolor{c}{rgb}{0,0,0};
\colorlet{c}{natgreen};
\draw [c] (3.43137,2.94453) -- (3.43137,2.9493);
\draw [c] (3.43137,2.9493) -- (3.43137,2.95398);
\draw [c] (3.35294,2.9493) -- (3.43137,2.9493);
\draw [c] (3.43137,2.9493) -- (3.5098,2.9493);
\definecolor{c}{rgb}{0,0,0};
\colorlet{c}{natgreen};
\draw [c] (3.58824,2.84814) -- (3.58824,2.85392);
\draw [c] (3.58824,2.85392) -- (3.58824,2.85957);
\draw [c] (3.5098,2.85392) -- (3.58824,2.85392);
\draw [c] (3.58824,2.85392) -- (3.66667,2.85392);
\definecolor{c}{rgb}{0,0,0};
\colorlet{c}{natgreen};
\draw [c] (3.7451,2.75072) -- (3.7451,2.75773);
\draw [c] (3.7451,2.75773) -- (3.7451,2.76456);
\draw [c] (3.66667,2.75773) -- (3.7451,2.75773);
\draw [c] (3.7451,2.75773) -- (3.82353,2.75773);
\definecolor{c}{rgb}{0,0,0};
\colorlet{c}{natgreen};
\draw [c] (3.90196,2.665) -- (3.90196,2.67332);
\draw [c] (3.90196,2.67332) -- (3.90196,2.68137);
\draw [c] (3.82353,2.67332) -- (3.90196,2.67332);
\draw [c] (3.90196,2.67332) -- (3.98039,2.67332);
\definecolor{c}{rgb}{0,0,0};
\colorlet{c}{natgreen};
\draw [c] (4.05882,2.55909) -- (4.05882,2.56935);
\draw [c] (4.05882,2.56935) -- (4.05882,2.57922);
\draw [c] (3.98039,2.56935) -- (4.05882,2.56935);
\draw [c] (4.05882,2.56935) -- (4.13725,2.56935);
\definecolor{c}{rgb}{0,0,0};
\colorlet{c}{natgreen};
\draw [c] (4.21569,2.4227) -- (4.21569,2.43617);
\draw [c] (4.21569,2.43617) -- (4.21569,2.44894);
\draw [c] (4.13725,2.43617) -- (4.21569,2.43617);
\draw [c] (4.21569,2.43617) -- (4.29412,2.43617);
\definecolor{c}{rgb}{0,0,0};
\colorlet{c}{natgreen};
\draw [c] (4.37255,2.30973) -- (4.37255,2.32658);
\draw [c] (4.37255,2.32658) -- (4.37255,2.34237);
\draw [c] (4.29412,2.32658) -- (4.37255,2.32658);
\draw [c] (4.37255,2.32658) -- (4.45098,2.32658);
\definecolor{c}{rgb}{0,0,0};
\colorlet{c}{natgreen};
\draw [c] (4.52941,2.17282) -- (4.52941,2.19494);
\draw [c] (4.52941,2.19494) -- (4.52941,2.21527);
\draw [c] (4.45098,2.19494) -- (4.52941,2.19494);
\draw [c] (4.52941,2.19494) -- (4.60784,2.19494);
\definecolor{c}{rgb}{0,0,0};
\colorlet{c}{natgreen};
\draw [c] (4.68627,2.14319) -- (4.68627,2.16665);
\draw [c] (4.68627,2.16665) -- (4.68627,2.18811);
\draw [c] (4.60784,2.16665) -- (4.68627,2.16665);
\draw [c] (4.68627,2.16665) -- (4.76471,2.16665);
\definecolor{c}{rgb}{0,0,0};
\colorlet{c}{natgreen};
\draw [c] (4.84314,2.03565) -- (4.84314,2.0647);
\draw [c] (4.84314,2.0647) -- (4.84314,2.09074);
\draw [c] (4.76471,2.0647) -- (4.84314,2.0647);
\draw [c] (4.84314,2.0647) -- (4.92157,2.0647);
\definecolor{c}{rgb}{0,0,0};
\colorlet{c}{natgreen};
\draw [c] (5,1.8959) -- (5,1.93424);
\draw [c] (5,1.93424) -- (5,1.9675);
\draw [c] (4.92157,1.93424) -- (5,1.93424);
\draw [c] (5,1.93424) -- (5.07843,1.93424);
\definecolor{c}{rgb}{0,0,0};
\colorlet{c}{natgreen};
\draw [c] (5.15686,1.82072) -- (5.15686,1.86523);
\draw [c] (5.15686,1.86523) -- (5.15686,1.90304);
\draw [c] (5.07843,1.86523) -- (5.15686,1.86523);
\draw [c] (5.15686,1.86523) -- (5.23529,1.86523);
\definecolor{c}{rgb}{0,0,0};
\colorlet{c}{natgreen};
\draw [c] (5.31373,1.68018) -- (5.31373,1.73898);
\draw [c] (5.31373,1.73898) -- (5.31373,1.7866);
\draw [c] (5.23529,1.73898) -- (5.31373,1.73898);
\draw [c] (5.31373,1.73898) -- (5.39216,1.73898);
\definecolor{c}{rgb}{0,0,0};
\colorlet{c}{natgreen};
\draw [c] (5.47059,1.66751) -- (5.47059,1.7278);
\draw [c] (5.47059,1.7278) -- (5.47059,1.7764);
\draw [c] (5.39216,1.7278) -- (5.47059,1.7278);
\draw [c] (5.47059,1.7278) -- (5.54902,1.7278);
\definecolor{c}{rgb}{0,0,0};
\colorlet{c}{natgreen};
\draw [c] (5.62745,1.51383) -- (5.62745,1.59551);
\draw [c] (5.62745,1.59551) -- (5.62745,1.65706);
\draw [c] (5.54902,1.59551) -- (5.62745,1.59551);
\draw [c] (5.62745,1.59551) -- (5.70588,1.59551);
\definecolor{c}{rgb}{0,0,0};
\colorlet{c}{natgreen};
\draw [c] (5.78431,1.48973) -- (5.78431,1.57538);
\draw [c] (5.78431,1.57538) -- (5.78431,1.63916);
\draw [c] (5.70588,1.57538) -- (5.78431,1.57538);
\draw [c] (5.78431,1.57538) -- (5.86275,1.57538);
\definecolor{c}{rgb}{0,0,0};
\colorlet{c}{natgreen};
\draw [c] (5.94118,1.46327) -- (5.94118,1.5535);
\draw [c] (5.94118,1.5535) -- (5.94118,1.61977);
\draw [c] (5.86275,1.5535) -- (5.94118,1.5535);
\draw [c] (5.94118,1.5535) -- (6.01961,1.5535);
\definecolor{c}{rgb}{0,0,0};
\colorlet{c}{natgreen};
\draw [c] (6.09804,1.12483) -- (6.09804,1.29913);
\draw [c] (6.09804,1.29913) -- (6.09804,1.40109);
\draw [c] (6.01961,1.29913) -- (6.09804,1.29913);
\draw [c] (6.09804,1.29913) -- (6.17647,1.29913);
\definecolor{c}{rgb}{0,0,0};
\colorlet{c}{natgreen};
\draw [c] (6.2549,1.46327) -- (6.2549,1.5535);
\draw [c] (6.2549,1.5535) -- (6.2549,1.61977);
\draw [c] (6.17647,1.5535) -- (6.2549,1.5535);
\draw [c] (6.2549,1.5535) -- (6.33333,1.5535);
\definecolor{c}{rgb}{0,0,0};
\colorlet{c}{natgreen};
\draw [c] (6.41176,0.680516) -- (6.41176,0.950536);
\draw [c] (6.41176,0.950536) -- (6.41176,1.12483);
\draw [c] (6.33333,0.950536) -- (6.41176,0.950536);
\draw [c] (6.41176,0.950536) -- (6.4902,0.950536);
\definecolor{c}{rgb}{0,0,0};
\colorlet{c}{natgreen};
\draw [c] (6.56863,0.680516) -- (6.56863,0.950536);
\draw [c] (6.56863,0.950536) -- (6.56863,1.12483);
\draw [c] (6.4902,0.950536) -- (6.56863,0.950536);
\draw [c] (6.56863,0.950536) -- (6.64706,0.950536);
\definecolor{c}{rgb}{0,0,0};
\colorlet{c}{natgreen};
\draw [c] (6.72549,0.680516) -- (6.72549,0.950536);
\draw [c] (6.72549,0.950536) -- (6.72549,1.12483);
\draw [c] (6.64706,0.950536) -- (6.72549,0.950536);
\draw [c] (6.72549,0.950536) -- (6.80392,0.950536);
\definecolor{c}{rgb}{0,0,0};
\colorlet{c}{natgreen};
\draw [c] (6.88235,0.816057) -- (6.88235,1.12483);
\draw [c] (6.88235,1.12483) -- (6.88235,1.25931);
\draw [c] (6.80392,1.12483) -- (6.88235,1.12483);
\draw [c] (6.88235,1.12483) -- (6.96078,1.12483);
\definecolor{c}{rgb}{0,0,0};
\colorlet{c}{natgreen};
\draw [c] (7.03922,0.680516) -- (7.03922,0.950536);
\draw [c] (7.03922,0.950536) -- (7.03922,1.12483);
\draw [c] (6.96078,0.950536) -- (7.03922,0.950536);
\draw [c] (7.03922,0.950536) -- (7.11765,0.950536);
\definecolor{c}{rgb}{0,0,0};
\colorlet{c}{natgreen};
\draw [c] (7.19608,0.680516) -- (7.19608,0.950536);
\draw [c] (7.19608,0.950536) -- (7.19608,1.12483);
\draw [c] (7.11765,0.950536) -- (7.19608,0.950536);
\draw [c] (7.19608,0.950536) -- (7.27451,0.950536);
\definecolor{c}{rgb}{0,0,0};
\colorlet{c}{natgreen};
\draw [c] (7.35294,0.680516) -- (7.35294,0.950536);
\draw [c] (7.35294,0.950536) -- (7.35294,1.12483);
\draw [c] (7.27451,0.950536) -- (7.35294,0.950536);
\draw [c] (7.35294,0.950536) -- (7.43137,0.950536);
\definecolor{c}{rgb}{0,0,0};
\colorlet{c}{natgreen};
\draw [c] (7.66667,0.680516) -- (7.66667,0.950536);
\draw [c] (7.66667,0.950536) -- (7.66667,1.12483);
\draw [c] (7.58824,0.950536) -- (7.66667,0.950536);
\draw [c] (7.66667,0.950536) -- (7.7451,0.950536);
\definecolor{c}{rgb}{0,0,0};
\draw [c] (1,0.680516) -- (9,0.680516);
\draw [anchor= east] (9,-0.0816619) node[color=c, rotate=0]{$d\sigma/d(M_{\gamma\gamma}) [pb/GeV]$};
\draw [c] (2.59744,0.84384) -- (2.59744,0.680516);
\draw [c] (2.91757,0.721347) -- (2.91757,0.680516);
\draw [c] (3.2377,0.721347) -- (3.2377,0.680516);
\draw [c] (3.55782,0.721347) -- (3.55782,0.680516);
\draw [c] (3.87795,0.721347) -- (3.87795,0.680516);
\draw [c] (4.19808,0.84384) -- (4.19808,0.680516);
\draw [c] (4.51821,0.721347) -- (4.51821,0.680516);
\draw [c] (4.83834,0.721347) -- (4.83834,0.680516);
\draw [c] (5.15846,0.721347) -- (5.15846,0.680516);
\draw [c] (5.47859,0.721347) -- (5.47859,0.680516);
\draw [c] (5.79872,0.84384) -- (5.79872,0.680516);
\draw [c] (6.11885,0.721347) -- (6.11885,0.680516);
\draw [c] (6.43898,0.721347) -- (6.43898,0.680516);
\draw [c] (6.7591,0.721347) -- (6.7591,0.680516);
\draw [c] (7.07923,0.721347) -- (7.07923,0.680516);
\draw [c] (7.39936,0.84384) -- (7.39936,0.680516);
\draw [c] (7.71949,0.721347) -- (7.71949,0.680516);
\draw [c] (8.03962,0.721347) -- (8.03962,0.680516);
\draw [c] (8.35974,0.721347) -- (8.35974,0.680516);
\draw [c] (8.67987,0.721347) -- (8.67987,0.680516);
\draw [c] (9,0.84384) -- (9,0.680516);
\draw [c] (2.59744,0.84384) -- (2.59744,0.680516);
\draw [c] (2.27731,0.721347) -- (2.27731,0.680516);
\draw [c] (1.95718,0.721347) -- (1.95718,0.680516);
\draw [c] (1.63705,0.721347) -- (1.63705,0.680516);
\draw [c] (1.31693,0.721347) -- (1.31693,0.680516);
\draw [anchor=base] (2.59744,0.285817) node[color=c, rotate=0]{1000};
\draw [anchor=base] (4.19808,0.285817) node[color=c, rotate=0]{2000};
\draw [anchor=base] (5.79872,0.285817) node[color=c, rotate=0]{3000};
\draw [anchor=base] (7.39936,0.285817) node[color=c, rotate=0]{4000};
\draw [anchor=base] (9,0.285817) node[color=c, rotate=0]{5000};
\draw [c] (1,0.680516) -- (1,6.12464);
\draw [c] (1.12,0.689046) -- (1,0.689046);
\draw [c] (1.12,0.722623) -- (1,0.722623);
\draw [c] (1.12,0.75224) -- (1,0.75224);
\draw [c] (1.24,0.778734) -- (1,0.778734);
\draw [anchor= east] (0.922,0.778734) node[color=c, rotate=0]{$10^{-10}$};
\draw [c] (1.12,0.953031) -- (1,0.953031);
\draw [c] (1.12,1.05499) -- (1,1.05499);
\draw [c] (1.12,1.12733) -- (1,1.12733);
\draw [c] (1.12,1.18344) -- (1,1.18344);
\draw [c] (1.12,1.22928) -- (1,1.22928);
\draw [c] (1.12,1.26805) -- (1,1.26805);
\draw [c] (1.12,1.30162) -- (1,1.30162);
\draw [c] (1.12,1.33124) -- (1,1.33124);
\draw [c] (1.24,1.35774) -- (1,1.35774);
\draw [anchor= east] (0.922,1.35774) node[color=c, rotate=0]{$10^{-9}$};
\draw [c] (1.12,1.53203) -- (1,1.53203);
\draw [c] (1.12,1.63399) -- (1,1.63399);
\draw [c] (1.12,1.70633) -- (1,1.70633);
\draw [c] (1.12,1.76244) -- (1,1.76244);
\draw [c] (1.12,1.80829) -- (1,1.80829);
\draw [c] (1.12,1.84705) -- (1,1.84705);
\draw [c] (1.12,1.88063) -- (1,1.88063);
\draw [c] (1.12,1.91024) -- (1,1.91024);
\draw [c] (1.24,1.93674) -- (1,1.93674);
\draw [anchor= east] (0.922,1.93674) node[color=c, rotate=0]{$10^{-8}$};
\draw [c] (1.12,2.11103) -- (1,2.11103);
\draw [c] (1.12,2.21299) -- (1,2.21299);
\draw [c] (1.12,2.28533) -- (1,2.28533);
\draw [c] (1.12,2.34144) -- (1,2.34144);
\draw [c] (1.12,2.38729) -- (1,2.38729);
\draw [c] (1.12,2.42605) -- (1,2.42605);
\draw [c] (1.12,2.45963) -- (1,2.45963);
\draw [c] (1.12,2.48924) -- (1,2.48924);
\draw [c] (1.24,2.51574) -- (1,2.51574);
\draw [anchor= east] (0.922,2.51574) node[color=c, rotate=0]{$10^{-7}$};
\draw [c] (1.12,2.69003) -- (1,2.69003);
\draw [c] (1.12,2.79199) -- (1,2.79199);
\draw [c] (1.12,2.86433) -- (1,2.86433);
\draw [c] (1.12,2.92044) -- (1,2.92044);
\draw [c] (1.12,2.96629) -- (1,2.96629);
\draw [c] (1.12,3.00505) -- (1,3.00505);
\draw [c] (1.12,3.03863) -- (1,3.03863);
\draw [c] (1.12,3.06825) -- (1,3.06825);
\draw [c] (1.24,3.09474) -- (1,3.09474);
\draw [anchor= east] (0.922,3.09474) node[color=c, rotate=0]{$10^{-6}$};
\draw [c] (1.12,3.26904) -- (1,3.26904);
\draw [c] (1.12,3.37099) -- (1,3.37099);
\draw [c] (1.12,3.44333) -- (1,3.44333);
\draw [c] (1.12,3.49944) -- (1,3.49944);
\draw [c] (1.12,3.54529) -- (1,3.54529);
\draw [c] (1.12,3.58405) -- (1,3.58405);
\draw [c] (1.12,3.61763) -- (1,3.61763);
\draw [c] (1.12,3.64725) -- (1,3.64725);
\draw [c] (1.24,3.67374) -- (1,3.67374);
\draw [anchor= east] (0.922,3.67374) node[color=c, rotate=0]{$10^{-5}$};
\draw [c] (1.12,3.84804) -- (1,3.84804);
\draw [c] (1.12,3.94999) -- (1,3.94999);
\draw [c] (1.12,4.02233) -- (1,4.02233);
\draw [c] (1.12,4.07844) -- (1,4.07844);
\draw [c] (1.12,4.12429) -- (1,4.12429);
\draw [c] (1.12,4.16305) -- (1,4.16305);
\draw [c] (1.12,4.19663) -- (1,4.19663);
\draw [c] (1.12,4.22625) -- (1,4.22625);
\draw [c] (1.24,4.25274) -- (1,4.25274);
\draw [anchor= east] (0.922,4.25274) node[color=c, rotate=0]{$10^{-4}$};
\draw [c] (1.12,4.42704) -- (1,4.42704);
\draw [c] (1.12,4.529) -- (1,4.529);
\draw [c] (1.12,4.60134) -- (1,4.60134);
\draw [c] (1.12,4.65745) -- (1,4.65745);
\draw [c] (1.12,4.70329) -- (1,4.70329);
\draw [c] (1.12,4.74205) -- (1,4.74205);
\draw [c] (1.12,4.77563) -- (1,4.77563);
\draw [c] (1.12,4.80525) -- (1,4.80525);
\draw [c] (1.24,4.83174) -- (1,4.83174);
\draw [anchor= east] (0.922,4.83174) node[color=c, rotate=0]{$10^{-3}$};
\draw [c] (1.12,5.00604) -- (1,5.00604);
\draw [c] (1.12,5.108) -- (1,5.108);
\draw [c] (1.12,5.18034) -- (1,5.18034);
\draw [c] (1.12,5.23645) -- (1,5.23645);
\draw [c] (1.12,5.28229) -- (1,5.28229);
\draw [c] (1.12,5.32106) -- (1,5.32106);
\draw [c] (1.12,5.35463) -- (1,5.35463);
\draw [c] (1.12,5.38425) -- (1,5.38425);
\draw [c] (1.24,5.41074) -- (1,5.41074);
\draw [anchor= east] (0.922,5.41074) node[color=c, rotate=0]{$10^{-2}$};
\draw [c] (1.12,5.58504) -- (1,5.58504);
\draw [c] (1.12,5.687) -- (1,5.687);
\draw [c] (1.12,5.75934) -- (1,5.75934);
\draw [c] (1.12,5.81545) -- (1,5.81545);
\draw [c] (1.12,5.86129) -- (1,5.86129);
\draw [c] (1.12,5.90006) -- (1,5.90006);
\draw [c] (1.12,5.93363) -- (1,5.93363);
\draw [c] (1.12,5.96325) -- (1,5.96325);
\draw [c] (1.24,5.98975) -- (1,5.98975);
\draw [anchor= east] (0.922,5.98975) node[color=c, rotate=0]{$10^{-1}$};
\colorlet{c}{natcomp};
\draw [c] (1.23529,5.61269) -- (1.23529,5.61376);
\draw [c] (1.23529,5.61376) -- (1.23529,5.61481);
\draw [c] (1.15686,5.61376) -- (1.23529,5.61376);
\draw [c] (1.23529,5.61376) -- (1.31373,5.61376);
\definecolor{c}{rgb}{0,0,0};
\colorlet{c}{natcomp};
\draw [c] (1.39216,5.09089) -- (1.39216,5.0939);
\draw [c] (1.39216,5.0939) -- (1.39216,5.09687);
\draw [c] (1.31373,5.0939) -- (1.39216,5.0939);
\draw [c] (1.39216,5.0939) -- (1.47059,5.0939);
\definecolor{c}{rgb}{0,0,0};
\colorlet{c}{natcomp};
\draw [c] (1.54902,4.73038) -- (1.54902,4.73654);
\draw [c] (1.54902,4.73654) -- (1.54902,4.74254);
\draw [c] (1.47059,4.73654) -- (1.54902,4.73654);
\draw [c] (1.54902,4.73654) -- (1.62745,4.73654);
\definecolor{c}{rgb}{0,0,0};
\colorlet{c}{natcomp};
\draw [c] (1.70588,4.46454) -- (1.70588,4.47497);
\draw [c] (1.70588,4.47497) -- (1.70588,4.48499);
\draw [c] (1.62745,4.47497) -- (1.70588,4.47497);
\draw [c] (1.70588,4.47497) -- (1.78431,4.47497);
\definecolor{c}{rgb}{0,0,0};
\colorlet{c}{natcomp};
\draw [c] (1.86275,4.27663) -- (1.86275,4.29179);
\draw [c] (1.86275,4.29179) -- (1.86275,4.30609);
\draw [c] (1.78431,4.29179) -- (1.86275,4.29179);
\draw [c] (1.86275,4.29179) -- (1.94118,4.29179);
\definecolor{c}{rgb}{0,0,0};
\colorlet{c}{natcomp};
\draw [c] (2.01961,4.08661) -- (2.01961,4.10873);
\draw [c] (2.01961,4.10873) -- (2.01961,4.12907);
\draw [c] (1.94118,4.10873) -- (2.01961,4.10873);
\draw [c] (2.01961,4.10873) -- (2.09804,4.10873);
\definecolor{c}{rgb}{0,0,0};
\colorlet{c}{natcomp};
\draw [c] (2.17647,3.9119) -- (2.17647,3.9432);
\draw [c] (2.17647,3.9432) -- (2.17647,3.97103);
\draw [c] (2.09804,3.9432) -- (2.17647,3.9432);
\draw [c] (2.17647,3.9432) -- (2.2549,3.9432);
\definecolor{c}{rgb}{0,0,0};
\colorlet{c}{natcomp};
\draw [c] (2.33333,3.78095) -- (2.33333,3.82154);
\draw [c] (2.33333,3.82154) -- (2.33333,3.85649);
\draw [c] (2.2549,3.82154) -- (2.33333,3.82154);
\draw [c] (2.33333,3.82154) -- (2.41176,3.82154);
\definecolor{c}{rgb}{0,0,0};
\colorlet{c}{natcomp};
\draw [c] (2.4902,3.65941) -- (2.4902,3.71106);
\draw [c] (2.4902,3.71106) -- (2.4902,3.75389);
\draw [c] (2.41176,3.71106) -- (2.4902,3.71106);
\draw [c] (2.4902,3.71106) -- (2.56863,3.71106);
\definecolor{c}{rgb}{0,0,0};
\colorlet{c}{natcomp};
\draw [c] (2.64706,3.54405) -- (2.64706,3.57172);
\draw [c] (2.64706,3.57172) -- (2.64706,3.59664);
\draw [c] (2.56863,3.57172) -- (2.64706,3.57172);
\draw [c] (2.64706,3.57172) -- (2.72549,3.57172);
\definecolor{c}{rgb}{0,0,0};
\colorlet{c}{natcomp};
\draw [c] (2.80392,3.48965) -- (2.80392,3.49201);
\draw [c] (2.80392,3.49201) -- (2.80392,3.49433);
\draw [c] (2.72549,3.49201) -- (2.80392,3.49201);
\draw [c] (2.80392,3.49201) -- (2.88235,3.49201);
\definecolor{c}{rgb}{0,0,0};
\colorlet{c}{natcomp};
\draw [c] (2.96078,3.40747) -- (2.96078,3.41024);
\draw [c] (2.96078,3.41024) -- (2.96078,3.41298);
\draw [c] (2.88235,3.41024) -- (2.96078,3.41024);
\draw [c] (2.96078,3.41024) -- (3.03922,3.41024);
\definecolor{c}{rgb}{0,0,0};
\colorlet{c}{natcomp};
\draw [c] (3.11765,3.33578) -- (3.11765,3.33897);
\draw [c] (3.11765,3.33897) -- (3.11765,3.34212);
\draw [c] (3.03922,3.33897) -- (3.11765,3.33897);
\draw [c] (3.11765,3.33897) -- (3.19608,3.33897);
\definecolor{c}{rgb}{0,0,0};
\colorlet{c}{natcomp};
\draw [c] (3.27451,3.28486) -- (3.27451,3.28839);
\draw [c] (3.27451,3.28839) -- (3.27451,3.29187);
\draw [c] (3.19608,3.28839) -- (3.27451,3.28839);
\draw [c] (3.27451,3.28839) -- (3.35294,3.28839);
\definecolor{c}{rgb}{0,0,0};
\colorlet{c}{natcomp};
\draw [c] (3.43137,3.22924) -- (3.43137,3.23318);
\draw [c] (3.43137,3.23318) -- (3.43137,3.23707);
\draw [c] (3.35294,3.23318) -- (3.43137,3.23318);
\draw [c] (3.43137,3.23318) -- (3.5098,3.23318);
\definecolor{c}{rgb}{0,0,0};
\colorlet{c}{natcomp};
\draw [c] (3.58824,3.18402) -- (3.58824,3.18833);
\draw [c] (3.58824,3.18833) -- (3.58824,3.19258);
\draw [c] (3.5098,3.18833) -- (3.58824,3.18833);
\draw [c] (3.58824,3.18833) -- (3.66667,3.18833);
\definecolor{c}{rgb}{0,0,0};
\colorlet{c}{natcomp};
\draw [c] (3.7451,3.14637) -- (3.7451,3.15103);
\draw [c] (3.7451,3.15103) -- (3.7451,3.15559);
\draw [c] (3.66667,3.15103) -- (3.7451,3.15103);
\draw [c] (3.7451,3.15103) -- (3.82353,3.15103);
\definecolor{c}{rgb}{0,0,0};
\colorlet{c}{natcomp};
\draw [c] (3.90196,3.11949) -- (3.90196,3.1244);
\draw [c] (3.90196,3.1244) -- (3.90196,3.12921);
\draw [c] (3.82353,3.1244) -- (3.90196,3.1244);
\draw [c] (3.90196,3.1244) -- (3.98039,3.1244);
\definecolor{c}{rgb}{0,0,0};
\colorlet{c}{natcomp};
\draw [c] (4.05882,3.08401) -- (4.05882,3.08927);
\draw [c] (4.05882,3.08927) -- (4.05882,3.09443);
\draw [c] (3.98039,3.08927) -- (4.05882,3.08927);
\draw [c] (4.05882,3.08927) -- (4.13725,3.08927);
\definecolor{c}{rgb}{0,0,0};
\colorlet{c}{natcomp};
\draw [c] (4.21569,3.06237) -- (4.21569,3.06787);
\draw [c] (4.21569,3.06787) -- (4.21569,3.07325);
\draw [c] (4.13725,3.06787) -- (4.21569,3.06787);
\draw [c] (4.21569,3.06787) -- (4.29412,3.06787);
\definecolor{c}{rgb}{0,0,0};
\colorlet{c}{natcomp};
\draw [c] (4.37255,3.04091) -- (4.37255,3.04665);
\draw [c] (4.37255,3.04665) -- (4.37255,3.05226);
\draw [c] (4.29412,3.04665) -- (4.37255,3.04665);
\draw [c] (4.37255,3.04665) -- (4.45098,3.04665);
\definecolor{c}{rgb}{0,0,0};
\colorlet{c}{natcomp};
\draw [c] (4.52941,3.02225) -- (4.52941,3.0282);
\draw [c] (4.52941,3.0282) -- (4.52941,3.03402);
\draw [c] (4.45098,3.0282) -- (4.52941,3.0282);
\draw [c] (4.52941,3.0282) -- (4.60784,3.0282);
\definecolor{c}{rgb}{0,0,0};
\colorlet{c}{natcomp};
\draw [c] (4.68627,3.00375) -- (4.68627,3.00993);
\draw [c] (4.68627,3.00993) -- (4.68627,3.01595);
\draw [c] (4.60784,3.00993) -- (4.68627,3.00993);
\draw [c] (4.68627,3.00993) -- (4.76471,3.00993);
\definecolor{c}{rgb}{0,0,0};
\colorlet{c}{natcomp};
\draw [c] (4.84314,2.96995) -- (4.84314,2.97656);
\draw [c] (4.84314,2.97656) -- (4.84314,2.983);
\draw [c] (4.76471,2.97656) -- (4.84314,2.97656);
\draw [c] (4.84314,2.97656) -- (4.92157,2.97656);
\definecolor{c}{rgb}{0,0,0};
\colorlet{c}{natcomp};
\draw [c] (5,2.95604) -- (5,2.96283);
\draw [c] (5,2.96283) -- (5,2.96945);
\draw [c] (4.92157,2.96283) -- (5,2.96283);
\draw [c] (5,2.96283) -- (5.07843,2.96283);
\definecolor{c}{rgb}{0,0,0};
\colorlet{c}{natcomp};
\draw [c] (5.15686,2.92954) -- (5.15686,2.9367);
\draw [c] (5.15686,2.9367) -- (5.15686,2.94366);
\draw [c] (5.07843,2.9367) -- (5.15686,2.9367);
\draw [c] (5.15686,2.9367) -- (5.23529,2.9367);
\definecolor{c}{rgb}{0,0,0};
\colorlet{c}{natcomp};
\draw [c] (5.31373,2.90925) -- (5.31373,2.9167);
\draw [c] (5.31373,2.9167) -- (5.31373,2.92394);
\draw [c] (5.23529,2.9167) -- (5.31373,2.9167);
\draw [c] (5.31373,2.9167) -- (5.39216,2.9167);
\definecolor{c}{rgb}{0,0,0};
\colorlet{c}{natcomp};
\draw [c] (5.47059,2.88287) -- (5.47059,2.89073);
\draw [c] (5.47059,2.89073) -- (5.47059,2.89835);
\draw [c] (5.39216,2.89073) -- (5.47059,2.89073);
\draw [c] (5.47059,2.89073) -- (5.54902,2.89073);
\definecolor{c}{rgb}{0,0,0};
\colorlet{c}{natcomp};
\draw [c] (5.62745,2.86302) -- (5.62745,2.8712);
\draw [c] (5.62745,2.8712) -- (5.62745,2.87911);
\draw [c] (5.54902,2.8712) -- (5.62745,2.8712);
\draw [c] (5.62745,2.8712) -- (5.70588,2.8712);
\definecolor{c}{rgb}{0,0,0};
\colorlet{c}{natcomp};
\draw [c] (5.78431,2.83398) -- (5.78431,2.84264);
\draw [c] (5.78431,2.84264) -- (5.78431,2.85101);
\draw [c] (5.70588,2.84264) -- (5.78431,2.84264);
\draw [c] (5.78431,2.84264) -- (5.86275,2.84264);
\definecolor{c}{rgb}{0,0,0};
\colorlet{c}{natcomp};
\draw [c] (5.94118,2.79849) -- (5.94118,2.80778);
\draw [c] (5.94118,2.80778) -- (5.94118,2.81674);
\draw [c] (5.86275,2.80778) -- (5.94118,2.80778);
\draw [c] (5.94118,2.80778) -- (6.01961,2.80778);
\definecolor{c}{rgb}{0,0,0};
\colorlet{c}{natcomp};
\draw [c] (6.09804,2.78392) -- (6.09804,2.79349);
\draw [c] (6.09804,2.79349) -- (6.09804,2.8027);
\draw [c] (6.01961,2.79349) -- (6.09804,2.79349);
\draw [c] (6.09804,2.79349) -- (6.17647,2.79349);
\definecolor{c}{rgb}{0,0,0};
\colorlet{c}{natcomp};
\draw [c] (6.2549,2.75079) -- (6.2549,2.76101);
\draw [c] (6.2549,2.76101) -- (6.2549,2.77082);
\draw [c] (6.17647,2.76101) -- (6.2549,2.76101);
\draw [c] (6.2549,2.76101) -- (6.33333,2.76101);
\definecolor{c}{rgb}{0,0,0};
\colorlet{c}{natcomp};
\draw [c] (6.41176,2.72059) -- (6.41176,2.73144);
\draw [c] (6.41176,2.73144) -- (6.41176,2.74184);
\draw [c] (6.33333,2.73144) -- (6.41176,2.73144);
\draw [c] (6.41176,2.73144) -- (6.4902,2.73144);
\definecolor{c}{rgb}{0,0,0};
\colorlet{c}{natcomp};
\draw [c] (6.56863,2.65865) -- (6.56863,2.67092);
\draw [c] (6.56863,2.67092) -- (6.56863,2.68262);
\draw [c] (6.4902,2.67092) -- (6.56863,2.67092);
\draw [c] (6.56863,2.67092) -- (6.64706,2.67092);
\definecolor{c}{rgb}{0,0,0};
\colorlet{c}{natcomp};
\draw [c] (6.72549,2.63859) -- (6.72549,2.65136);
\draw [c] (6.72549,2.65136) -- (6.72549,2.66351);
\draw [c] (6.64706,2.65136) -- (6.72549,2.65136);
\draw [c] (6.72549,2.65136) -- (6.80392,2.65136);
\definecolor{c}{rgb}{0,0,0};
\colorlet{c}{natcomp};
\draw [c] (6.88235,2.57418) -- (6.88235,2.5887);
\draw [c] (6.88235,2.5887) -- (6.88235,2.60242);
\draw [c] (6.80392,2.5887) -- (6.88235,2.5887);
\draw [c] (6.88235,2.5887) -- (6.96078,2.5887);
\definecolor{c}{rgb}{0,0,0};
\colorlet{c}{natcomp};
\draw [c] (7.03922,2.5476) -- (7.03922,2.5629);
\draw [c] (7.03922,2.5629) -- (7.03922,2.57733);
\draw [c] (6.96078,2.5629) -- (7.03922,2.5629);
\draw [c] (7.03922,2.5629) -- (7.11765,2.5629);
\definecolor{c}{rgb}{0,0,0};
\colorlet{c}{natcomp};
\draw [c] (7.19608,2.4785) -- (7.19608,2.49605);
\draw [c] (7.19608,2.49605) -- (7.19608,2.51246);
\draw [c] (7.11765,2.49605) -- (7.19608,2.49605);
\draw [c] (7.19608,2.49605) -- (7.27451,2.49605);
\definecolor{c}{rgb}{0,0,0};
\colorlet{c}{natcomp};
\draw [c] (7.35294,2.42901) -- (7.35294,2.44837);
\draw [c] (7.35294,2.44837) -- (7.35294,2.46635);
\draw [c] (7.27451,2.44837) -- (7.35294,2.44837);
\draw [c] (7.35294,2.44837) -- (7.43137,2.44837);
\definecolor{c}{rgb}{0,0,0};
\colorlet{c}{natcomp};
\draw [c] (7.5098,2.39362) -- (7.5098,2.4144);
\draw [c] (7.5098,2.4144) -- (7.5098,2.43359);
\draw [c] (7.43137,2.4144) -- (7.5098,2.4144);
\draw [c] (7.5098,2.4144) -- (7.58824,2.4144);
\definecolor{c}{rgb}{0,0,0};
\colorlet{c}{natcomp};
\draw [c] (7.66667,2.34669) -- (7.66667,2.3695);
\draw [c] (7.66667,2.3695) -- (7.66667,2.39041);
\draw [c] (7.58824,2.3695) -- (7.66667,2.3695);
\draw [c] (7.66667,2.3695) -- (7.7451,2.3695);
\definecolor{c}{rgb}{0,0,0};
\colorlet{c}{natcomp};
\draw [c] (7.82353,2.28424) -- (7.82353,2.31006);
\draw [c] (7.82353,2.31006) -- (7.82353,2.33347);
\draw [c] (7.7451,2.31006) -- (7.82353,2.31006);
\draw [c] (7.82353,2.31006) -- (7.90196,2.31006);
\definecolor{c}{rgb}{0,0,0};
\colorlet{c}{natcomp};
\draw [c] (7.98039,2.28675) -- (7.98039,2.31244);
\draw [c] (7.98039,2.31244) -- (7.98039,2.33575);
\draw [c] (7.90196,2.31244) -- (7.98039,2.31244);
\draw [c] (7.98039,2.31244) -- (8.05882,2.31244);
\definecolor{c}{rgb}{0,0,0};
\colorlet{c}{natcomp};
\draw [c] (8.13725,2.11592) -- (8.13725,2.15199);
\draw [c] (8.13725,2.15199) -- (8.13725,2.18353);
\draw [c] (8.05882,2.15199) -- (8.13725,2.15199);
\draw [c] (8.13725,2.15199) -- (8.21569,2.15199);
\definecolor{c}{rgb}{0,0,0};
\colorlet{c}{natcomp};
\draw [c] (8.29412,2.05026) -- (8.29412,2.09135);
\draw [c] (8.29412,2.09135) -- (8.29412,2.12666);
\draw [c] (8.21569,2.09135) -- (8.29412,2.09135);
\draw [c] (8.29412,2.09135) -- (8.37255,2.09135);
\definecolor{c}{rgb}{0,0,0};
\colorlet{c}{natcomp};
\draw [c] (8.45098,2.00262) -- (8.45098,2.04778);
\draw [c] (8.45098,2.04778) -- (8.45098,2.08605);
\draw [c] (8.37255,2.04778) -- (8.45098,2.04778);
\draw [c] (8.45098,2.04778) -- (8.52941,2.04778);
\definecolor{c}{rgb}{0,0,0};
\colorlet{c}{natcomp};
\draw [c] (8.60784,1.9148) -- (8.60784,1.96855);
\draw [c] (8.60784,1.96855) -- (8.60784,2.01281);
\draw [c] (8.52941,1.96855) -- (8.60784,1.96855);
\draw [c] (8.60784,1.96855) -- (8.68627,1.96855);
\definecolor{c}{rgb}{0,0,0};
\colorlet{c}{natcomp};
\draw [c] (8.76471,1.76463) -- (8.76471,1.83697);
\draw [c] (8.76471,1.83697) -- (8.76471,1.89309);
\draw [c] (8.68627,1.83697) -- (8.76471,1.83697);
\draw [c] (8.76471,1.83697) -- (8.84314,1.83697);
\definecolor{c}{rgb}{0,0,0};
\colorlet{c}{natcomp};
\draw [c] (8.92157,1.6232) -- (8.92157,1.71879);
\draw [c] (8.92157,1.71879) -- (8.92157,1.78788);
\draw [c] (8.84314,1.71879) -- (8.92157,1.71879);
\draw [c] (8.92157,1.71879) -- (9,1.71879);
\definecolor{c}{rgb}{0,0,0};
\definecolor{c}{rgb}{1,1,1};
\draw [color=c, fill=c, fill opacity=0] (5,5.10387) rectangle (8.8,5.98854);
\draw [c] (5,5.10387) -- (8.8,5.10387);
\draw [c] (8.8,5.10387) -- (8.8,5.98854);
\draw [c] (8.8,5.98854) -- (5,5.98854);
\draw [c] (5,5.98854) -- (5,5.10387);
\definecolor{c}{rgb}{0,0,0};
\draw [anchor=base west] (5.95,5.66785) node[color=c, rotate=0]{Standard Model};
\colorlet{c}{natgreen};
\draw [c] (5.1425,5.76737) -- (5.8075,5.76737);
\definecolor{c}{rgb}{0,0,0};
\draw [anchor=base west] (5.95,5.22551) node[color=c, rotate=0]{$\Lambda = 1.0 TeV$};
\colorlet{c}{natcomp};
\draw [c] (5.1425,5.32504) -- (5.8075,5.32504);
\end{tikzpicture}
 }
\end{infilsf}
\vspace{-1em}
\subcaption{Significance: 23.7}
\end{minipage}\hfill
\begin{minipage}[b]{.499\textwidth}
\begin{infilsf} \tiny
\makebox[.96\textwidth]{\begin{tikzpicture}[x=.095\textwidth,y=.095\textwidth]
\pgfdeclareplotmark{cross} {
\pgfpathmoveto{\pgfpoint{-0.3\pgfplotmarksize}{\pgfplotmarksize}}
\pgfpathlineto{\pgfpoint{+0.3\pgfplotmarksize}{\pgfplotmarksize}}
\pgfpathlineto{\pgfpoint{+0.3\pgfplotmarksize}{0.3\pgfplotmarksize}}
\pgfpathlineto{\pgfpoint{+1\pgfplotmarksize}{0.3\pgfplotmarksize}}
\pgfpathlineto{\pgfpoint{+1\pgfplotmarksize}{-0.3\pgfplotmarksize}}
\pgfpathlineto{\pgfpoint{+0.3\pgfplotmarksize}{-0.3\pgfplotmarksize}}
\pgfpathlineto{\pgfpoint{+0.3\pgfplotmarksize}{-1.\pgfplotmarksize}}
\pgfpathlineto{\pgfpoint{-0.3\pgfplotmarksize}{-1.\pgfplotmarksize}}
\pgfpathlineto{\pgfpoint{-0.3\pgfplotmarksize}{-0.3\pgfplotmarksize}}
\pgfpathlineto{\pgfpoint{-1.\pgfplotmarksize}{-0.3\pgfplotmarksize}}
\pgfpathlineto{\pgfpoint{-1.\pgfplotmarksize}{0.3\pgfplotmarksize}}
\pgfpathlineto{\pgfpoint{-0.3\pgfplotmarksize}{0.3\pgfplotmarksize}}
\pgfpathclose
\pgfusepathqstroke
}
\pgfdeclareplotmark{cross*} {
\pgfpathmoveto{\pgfpoint{-0.3\pgfplotmarksize}{\pgfplotmarksize}}
\pgfpathlineto{\pgfpoint{+0.3\pgfplotmarksize}{\pgfplotmarksize}}
\pgfpathlineto{\pgfpoint{+0.3\pgfplotmarksize}{0.3\pgfplotmarksize}}
\pgfpathlineto{\pgfpoint{+1\pgfplotmarksize}{0.3\pgfplotmarksize}}
\pgfpathlineto{\pgfpoint{+1\pgfplotmarksize}{-0.3\pgfplotmarksize}}
\pgfpathlineto{\pgfpoint{+0.3\pgfplotmarksize}{-0.3\pgfplotmarksize}}
\pgfpathlineto{\pgfpoint{+0.3\pgfplotmarksize}{-1.\pgfplotmarksize}}
\pgfpathlineto{\pgfpoint{-0.3\pgfplotmarksize}{-1.\pgfplotmarksize}}
\pgfpathlineto{\pgfpoint{-0.3\pgfplotmarksize}{-0.3\pgfplotmarksize}}
\pgfpathlineto{\pgfpoint{-1.\pgfplotmarksize}{-0.3\pgfplotmarksize}}
\pgfpathlineto{\pgfpoint{-1.\pgfplotmarksize}{0.3\pgfplotmarksize}}
\pgfpathlineto{\pgfpoint{-0.3\pgfplotmarksize}{0.3\pgfplotmarksize}}
\pgfpathclose
\pgfusepathqfillstroke
}
\pgfdeclareplotmark{newstar} {
\pgfpathmoveto{\pgfqpoint{0pt}{\pgfplotmarksize}}
\pgfpathlineto{\pgfqpointpolar{44}{0.5\pgfplotmarksize}}
\pgfpathlineto{\pgfqpointpolar{18}{\pgfplotmarksize}}
\pgfpathlineto{\pgfqpointpolar{-20}{0.5\pgfplotmarksize}}
\pgfpathlineto{\pgfqpointpolar{-54}{\pgfplotmarksize}}
\pgfpathlineto{\pgfqpointpolar{-90}{0.5\pgfplotmarksize}}
\pgfpathlineto{\pgfqpointpolar{234}{\pgfplotmarksize}}
\pgfpathlineto{\pgfqpointpolar{198}{0.5\pgfplotmarksize}}
\pgfpathlineto{\pgfqpointpolar{162}{\pgfplotmarksize}}
\pgfpathlineto{\pgfqpointpolar{134}{0.5\pgfplotmarksize}}
\pgfpathclose
\pgfusepathqstroke
}
\pgfdeclareplotmark{newstar*} {
\pgfpathmoveto{\pgfqpoint{0pt}{\pgfplotmarksize}}
\pgfpathlineto{\pgfqpointpolar{44}{0.5\pgfplotmarksize}}
\pgfpathlineto{\pgfqpointpolar{18}{\pgfplotmarksize}}
\pgfpathlineto{\pgfqpointpolar{-20}{0.5\pgfplotmarksize}}
\pgfpathlineto{\pgfqpointpolar{-54}{\pgfplotmarksize}}
\pgfpathlineto{\pgfqpointpolar{-90}{0.5\pgfplotmarksize}}
\pgfpathlineto{\pgfqpointpolar{234}{\pgfplotmarksize}}
\pgfpathlineto{\pgfqpointpolar{198}{0.5\pgfplotmarksize}}
\pgfpathlineto{\pgfqpointpolar{162}{\pgfplotmarksize}}
\pgfpathlineto{\pgfqpointpolar{134}{0.5\pgfplotmarksize}}
\pgfpathclose
\pgfusepathqfillstroke
}
\definecolor{c}{rgb}{1,1,1};
\draw [color=c, fill=c] (0,0) rectangle (10,6.80516);
\draw [color=c, fill=c] (1,0.680516) rectangle (9.95,6.73711);
\definecolor{c}{rgb}{0,0,0};
\draw [c] (1,0.680516) -- (1,6.73711) -- (9.95,6.73711) -- (9.95,0.680516) -- (1,0.680516);
\definecolor{c}{rgb}{1,1,1};
\draw [color=c, fill=c] (1,0.680516) rectangle (9.95,6.73711);
\definecolor{c}{rgb}{0,0,0};
\draw [c] (1,0.680516) -- (1,6.73711) -- (9.95,6.73711) -- (9.95,0.680516) -- (1,0.680516);
\colorlet{c}{natgreen};
\draw [c] (1.0895,6.55786) -- (1.0895,6.56076);
\draw [c] (1.0895,6.56076) -- (1.0895,6.56362);
\draw [c] (1,6.56076) -- (1.0895,6.56076);
\draw [c] (1.0895,6.56076) -- (1.179,6.56076);
\definecolor{c}{rgb}{0,0,0};
\colorlet{c}{natgreen};
\draw [c] (1.2685,5.89365) -- (1.2685,5.89517);
\draw [c] (1.2685,5.89517) -- (1.2685,5.89669);
\draw [c] (1.179,5.89517) -- (1.2685,5.89517);
\draw [c] (1.2685,5.89517) -- (1.358,5.89517);
\definecolor{c}{rgb}{0,0,0};
\colorlet{c}{natgreen};
\draw [c] (1.4475,5.29032) -- (1.4475,5.29495);
\draw [c] (1.4475,5.29495) -- (1.4475,5.2995);
\draw [c] (1.358,5.29495) -- (1.4475,5.29495);
\draw [c] (1.4475,5.29495) -- (1.537,5.29495);
\definecolor{c}{rgb}{0,0,0};
\colorlet{c}{natgreen};
\draw [c] (1.6265,4.9148) -- (1.6265,4.92403);
\draw [c] (1.6265,4.92403) -- (1.6265,4.93296);
\draw [c] (1.537,4.92403) -- (1.6265,4.92403);
\draw [c] (1.6265,4.92403) -- (1.716,4.92403);
\definecolor{c}{rgb}{0,0,0};
\colorlet{c}{natgreen};
\draw [c] (1.8055,4.6113) -- (1.8055,4.62739);
\draw [c] (1.8055,4.62739) -- (1.8055,4.64258);
\draw [c] (1.716,4.62739) -- (1.8055,4.62739);
\draw [c] (1.8055,4.62739) -- (1.895,4.62739);
\definecolor{c}{rgb}{0,0,0};
\colorlet{c}{natgreen};
\draw [c] (1.9845,4.37914) -- (1.9845,4.4036);
\draw [c] (1.9845,4.4036) -- (1.9845,4.42604);
\draw [c] (1.895,4.4036) -- (1.9845,4.4036);
\draw [c] (1.9845,4.4036) -- (2.074,4.4036);
\definecolor{c}{rgb}{0,0,0};
\colorlet{c}{natgreen};
\draw [c] (2.1635,4.08756) -- (2.1635,4.12828);
\draw [c] (2.1635,4.12828) -- (2.1635,4.16368);
\draw [c] (2.074,4.12828) -- (2.1635,4.12828);
\draw [c] (2.1635,4.12828) -- (2.253,4.12828);
\definecolor{c}{rgb}{0,0,0};
\colorlet{c}{natgreen};
\draw [c] (2.3425,3.90711) -- (2.3425,3.96163);
\draw [c] (2.3425,3.96163) -- (2.3425,4.00702);
\draw [c] (2.253,3.96163) -- (2.3425,3.96163);
\draw [c] (2.3425,3.96163) -- (2.432,3.96163);
\definecolor{c}{rgb}{0,0,0};
\colorlet{c}{natgreen};
\draw [c] (2.5215,3.74446) -- (2.5215,3.81283);
\draw [c] (2.5215,3.81283) -- (2.5215,3.86739);
\draw [c] (2.432,3.81283) -- (2.5215,3.81283);
\draw [c] (2.5215,3.81283) -- (2.611,3.81283);
\definecolor{c}{rgb}{0,0,0};
\colorlet{c}{natgreen};
\draw [c] (2.7005,3.54259) -- (2.7005,3.61687);
\draw [c] (2.7005,3.61687) -- (2.7005,3.67512);
\draw [c] (2.611,3.61687) -- (2.7005,3.61687);
\draw [c] (2.7005,3.61687) -- (2.79,3.61687);
\definecolor{c}{rgb}{0,0,0};
\colorlet{c}{natgreen};
\draw [c] (2.8795,3.47964) -- (2.8795,3.48253);
\draw [c] (2.8795,3.48253) -- (2.8795,3.48539);
\draw [c] (2.79,3.48253) -- (2.8795,3.48253);
\draw [c] (2.8795,3.48253) -- (2.969,3.48253);
\definecolor{c}{rgb}{0,0,0};
\colorlet{c}{natgreen};
\draw [c] (3.0585,3.32213) -- (3.0585,3.326);
\draw [c] (3.0585,3.326) -- (3.0585,3.32981);
\draw [c] (2.969,3.326) -- (3.0585,3.326);
\draw [c] (3.0585,3.326) -- (3.148,3.326);
\definecolor{c}{rgb}{0,0,0};
\colorlet{c}{natgreen};
\draw [c] (3.2375,3.18373) -- (3.2375,3.18872);
\draw [c] (3.2375,3.18872) -- (3.2375,3.19362);
\draw [c] (3.148,3.18872) -- (3.2375,3.18872);
\draw [c] (3.2375,3.18872) -- (3.327,3.18872);
\definecolor{c}{rgb}{0,0,0};
\colorlet{c}{natgreen};
\draw [c] (3.4165,3.03906) -- (3.4165,3.04557);
\draw [c] (3.4165,3.04557) -- (3.4165,3.05193);
\draw [c] (3.327,3.04557) -- (3.4165,3.04557);
\draw [c] (3.4165,3.04557) -- (3.506,3.04557);
\definecolor{c}{rgb}{0,0,0};
\colorlet{c}{natgreen};
\draw [c] (3.5955,2.9128) -- (3.5955,2.92102);
\draw [c] (3.5955,2.92102) -- (3.5955,2.92899);
\draw [c] (3.506,2.92102) -- (3.5955,2.92102);
\draw [c] (3.5955,2.92102) -- (3.685,2.92102);
\definecolor{c}{rgb}{0,0,0};
\colorlet{c}{natgreen};
\draw [c] (3.7745,2.76505) -- (3.7745,2.77583);
\draw [c] (3.7745,2.77583) -- (3.7745,2.78621);
\draw [c] (3.685,2.77583) -- (3.7745,2.77583);
\draw [c] (3.7745,2.77583) -- (3.864,2.77583);
\definecolor{c}{rgb}{0,0,0};
\colorlet{c}{natgreen};
\draw [c] (3.9535,2.65979) -- (3.9535,2.67288);
\draw [c] (3.9535,2.67288) -- (3.9535,2.68537);
\draw [c] (3.864,2.67288) -- (3.9535,2.67288);
\draw [c] (3.9535,2.67288) -- (4.043,2.67288);
\definecolor{c}{rgb}{0,0,0};
\colorlet{c}{natgreen};
\draw [c] (4.1325,2.51133) -- (4.1325,2.52854);
\draw [c] (4.1325,2.52854) -- (4.1325,2.54472);
\draw [c] (4.043,2.52854) -- (4.1325,2.52854);
\draw [c] (4.1325,2.52854) -- (4.222,2.52854);
\definecolor{c}{rgb}{0,0,0};
\colorlet{c}{natgreen};
\draw [c] (4.3115,2.41328) -- (4.3115,2.4339);
\draw [c] (4.3115,2.4339) -- (4.3115,2.45306);
\draw [c] (4.222,2.4339) -- (4.3115,2.4339);
\draw [c] (4.3115,2.4339) -- (4.401,2.4339);
\definecolor{c}{rgb}{0,0,0};
\colorlet{c}{natgreen};
\draw [c] (4.4905,2.24661) -- (4.4905,2.27463);
\draw [c] (4.4905,2.27463) -- (4.4905,2.30002);
\draw [c] (4.401,2.27463) -- (4.4905,2.27463);
\draw [c] (4.4905,2.27463) -- (4.58,2.27463);
\definecolor{c}{rgb}{0,0,0};
\colorlet{c}{natgreen};
\draw [c] (4.6695,2.13671) -- (4.6695,2.17101);
\draw [c] (4.6695,2.17101) -- (4.6695,2.20145);
\draw [c] (4.58,2.17101) -- (4.6695,2.17101);
\draw [c] (4.6695,2.17101) -- (4.759,2.17101);
\definecolor{c}{rgb}{0,0,0};
\colorlet{c}{natgreen};
\draw [c] (4.8485,1.98299) -- (4.8485,2.02849);
\draw [c] (4.8485,2.02849) -- (4.8485,2.06744);
\draw [c] (4.759,2.02849) -- (4.8485,2.02849);
\draw [c] (4.8485,2.02849) -- (4.938,2.02849);
\definecolor{c}{rgb}{0,0,0};
\colorlet{c}{natgreen};
\draw [c] (5.0275,1.85052) -- (5.0275,1.90855);
\draw [c] (5.0275,1.90855) -- (5.0275,1.95633);
\draw [c] (4.938,1.90855) -- (5.0275,1.90855);
\draw [c] (5.0275,1.90855) -- (5.117,1.90855);
\definecolor{c}{rgb}{0,0,0};
\colorlet{c}{natgreen};
\draw [c] (5.2065,1.7735) -- (5.2065,1.84032);
\draw [c] (5.2065,1.84032) -- (5.2065,1.89391);
\draw [c] (5.117,1.84032) -- (5.2065,1.84032);
\draw [c] (5.2065,1.84032) -- (5.296,1.84032);
\definecolor{c}{rgb}{0,0,0};
\colorlet{c}{natgreen};
\draw [c] (5.3855,1.74241) -- (5.3855,1.81316);
\draw [c] (5.3855,1.81316) -- (5.3855,1.86922);
\draw [c] (5.296,1.81316) -- (5.3855,1.81316);
\draw [c] (5.3855,1.81316) -- (5.475,1.81316);
\definecolor{c}{rgb}{0,0,0};
\colorlet{c}{natgreen};
\draw [c] (5.5645,1.45991) -- (5.5645,1.57834);
\draw [c] (5.5645,1.57834) -- (5.5645,1.66052);
\draw [c] (5.475,1.57834) -- (5.5645,1.57834);
\draw [c] (5.5645,1.57834) -- (5.654,1.57834);
\definecolor{c}{rgb}{0,0,0};
\colorlet{c}{natgreen};
\draw [c] (5.7435,1.50025) -- (5.7435,1.61032);
\draw [c] (5.7435,1.61032) -- (5.7435,1.68841);
\draw [c] (5.654,1.61032) -- (5.7435,1.61032);
\draw [c] (5.7435,1.61032) -- (5.833,1.61032);
\definecolor{c}{rgb}{0,0,0};
\colorlet{c}{natgreen};
\draw [c] (5.9225,1.20202) -- (5.9225,1.39018);
\draw [c] (5.9225,1.39018) -- (5.9225,1.50025);
\draw [c] (5.833,1.39018) -- (5.9225,1.39018);
\draw [c] (5.9225,1.39018) -- (6.012,1.39018);
\definecolor{c}{rgb}{0,0,0};
\colorlet{c}{natgreen};
\draw [c] (6.1015,1.0783) -- (6.1015,1.31209);
\draw [c] (6.1015,1.31209) -- (6.1015,1.4358);
\draw [c] (6.012,1.31209) -- (6.1015,1.31209);
\draw [c] (6.1015,1.31209) -- (6.191,1.31209);
\definecolor{c}{rgb}{0,0,0};
\colorlet{c}{natgreen};
\draw [c] (6.2805,0.868678) -- (6.2805,1.20202);
\draw [c] (6.2805,1.20202) -- (6.2805,1.3472);
\draw [c] (6.191,1.20202) -- (6.2805,1.20202);
\draw [c] (6.2805,1.20202) -- (6.37,1.20202);
\definecolor{c}{rgb}{0,0,0};
\colorlet{c}{natgreen};
\draw [c] (6.6385,0.680516) -- (6.6385,1.01386);
\draw [c] (6.6385,1.01386) -- (6.6385,1.20202);
\draw [c] (6.549,1.01386) -- (6.6385,1.01386);
\draw [c] (6.6385,1.01386) -- (6.728,1.01386);
\definecolor{c}{rgb}{0,0,0};
\colorlet{c}{natgreen};
\draw [c] (6.8175,0.680516) -- (6.8175,1.01386);
\draw [c] (6.8175,1.01386) -- (6.8175,1.20202);
\draw [c] (6.728,1.01386) -- (6.8175,1.01386);
\draw [c] (6.8175,1.01386) -- (6.907,1.01386);
\definecolor{c}{rgb}{0,0,0};
\colorlet{c}{natgreen};
\draw [c] (6.9965,0.680516) -- (6.9965,1.01386);
\draw [c] (6.9965,1.01386) -- (6.9965,1.20202);
\draw [c] (6.907,1.01386) -- (6.9965,1.01386);
\draw [c] (6.9965,1.01386) -- (7.086,1.01386);
\definecolor{c}{rgb}{0,0,0};
\colorlet{c}{natgreen};
\draw [c] (7.5335,0.680516) -- (7.5335,1.01386);
\draw [c] (7.5335,1.01386) -- (7.5335,1.20202);
\draw [c] (7.444,1.01386) -- (7.5335,1.01386);
\draw [c] (7.5335,1.01386) -- (7.623,1.01386);
\definecolor{c}{rgb}{0,0,0};
\colorlet{c}{natgreen};
\draw [c] (7.8915,0.680516) -- (7.8915,1.01386);
\draw [c] (7.8915,1.01386) -- (7.8915,1.20202);
\draw [c] (7.802,1.01386) -- (7.8915,1.01386);
\draw [c] (7.8915,1.01386) -- (7.981,1.01386);
\definecolor{c}{rgb}{0,0,0};
\colorlet{c}{natgreen};
\draw [c] (8.0705,0.680516) -- (8.0705,1.01386);
\draw [c] (8.0705,1.01386) -- (8.0705,1.20202);
\draw [c] (7.981,1.01386) -- (8.0705,1.01386);
\draw [c] (8.0705,1.01386) -- (8.16,1.01386);
\definecolor{c}{rgb}{0,0,0};
\draw [c] (1,0.680516) -- (9.95,0.680516);
\draw [anchor= east] (9.95,-0.0816619) node[color=c, rotate=0]{$p_{T}^{\gamma\gamma}} [GeV]$};
\draw [c] (1,0.863234) -- (1,0.680516);
\draw [c] (1.358,0.771875) -- (1.358,0.680516);
\draw [c] (1.716,0.771875) -- (1.716,0.680516);
\draw [c] (2.074,0.771875) -- (2.074,0.680516);
\draw [c] (2.432,0.771875) -- (2.432,0.680516);
\draw [c] (2.79,0.863234) -- (2.79,0.680516);
\draw [c] (3.148,0.771875) -- (3.148,0.680516);
\draw [c] (3.506,0.771875) -- (3.506,0.680516);
\draw [c] (3.864,0.771875) -- (3.864,0.680516);
\draw [c] (4.222,0.771875) -- (4.222,0.680516);
\draw [c] (4.58,0.863234) -- (4.58,0.680516);
\draw [c] (4.938,0.771875) -- (4.938,0.680516);
\draw [c] (5.296,0.771875) -- (5.296,0.680516);
\draw [c] (5.654,0.771875) -- (5.654,0.680516);
\draw [c] (6.012,0.771875) -- (6.012,0.680516);
\draw [c] (6.37,0.863234) -- (6.37,0.680516);
\draw [c] (6.728,0.771875) -- (6.728,0.680516);
\draw [c] (7.086,0.771875) -- (7.086,0.680516);
\draw [c] (7.444,0.771875) -- (7.444,0.680516);
\draw [c] (7.802,0.771875) -- (7.802,0.680516);
\draw [c] (8.16,0.863234) -- (8.16,0.680516);
\draw [c] (8.518,0.771875) -- (8.518,0.680516);
\draw [c] (8.876,0.771875) -- (8.876,0.680516);
\draw [c] (9.234,0.771875) -- (9.234,0.680516);
\draw [c] (9.592,0.771875) -- (9.592,0.680516);
\draw [c] (9.95,0.863234) -- (9.95,0.680516);
\draw [anchor=base] (1,0.285817) node[color=c, rotate=0]{0};
\draw [anchor=base] (2.79,0.285817) node[color=c, rotate=0]{1000};
\draw [anchor=base] (4.58,0.285817) node[color=c, rotate=0]{2000};
\draw [anchor=base] (6.37,0.285817) node[color=c, rotate=0]{3000};
\draw [anchor=base] (8.16,0.285817) node[color=c, rotate=0]{4000};
\draw [anchor=base] (9.95,0.285817) node[color=c, rotate=0]{5000};
\draw [c] (1,0.680516) -- (1,6.73711);
\draw [anchor= east] (-0.4,6.73711) node[color=c, rotate=90]{$d\sigma/d(p_{T}^{\gamma\gamma}) [pb/GeV]$};
\draw [c] (1.1335,0.695201) -- (1,0.695201);
\draw [c] (1.1335,0.737047) -- (1,0.737047);
\draw [c] (1.1335,0.773296) -- (1,0.773296);
\draw [c] (1.1335,0.805269) -- (1,0.805269);
\draw [c] (1.267,0.83387) -- (1,0.83387);
\draw [anchor= east] (0.922,0.83387) node[color=c, rotate=0]{$10^{-10}$};
\draw [c] (1.1335,1.02203) -- (1,1.02203);
\draw [c] (1.1335,1.1321) -- (1,1.1321);
\draw [c] (1.1335,1.2102) -- (1,1.2102);
\draw [c] (1.1335,1.27077) -- (1,1.27077);
\draw [c] (1.1335,1.32026) -- (1,1.32026);
\draw [c] (1.1335,1.36211) -- (1,1.36211);
\draw [c] (1.1335,1.39836) -- (1,1.39836);
\draw [c] (1.1335,1.43033) -- (1,1.43033);
\draw [c] (1.267,1.45893) -- (1,1.45893);
\draw [anchor= east] (0.922,1.45893) node[color=c, rotate=0]{$10^{-9}$};
\draw [c] (1.1335,1.64709) -- (1,1.64709);
\draw [c] (1.1335,1.75716) -- (1,1.75716);
\draw [c] (1.1335,1.83526) -- (1,1.83526);
\draw [c] (1.1335,1.89583) -- (1,1.89583);
\draw [c] (1.1335,1.94533) -- (1,1.94533);
\draw [c] (1.1335,1.98717) -- (1,1.98717);
\draw [c] (1.1335,2.02342) -- (1,2.02342);
\draw [c] (1.1335,2.05539) -- (1,2.05539);
\draw [c] (1.267,2.08399) -- (1,2.08399);
\draw [anchor= east] (0.922,2.08399) node[color=c, rotate=0]{$10^{-8}$};
\draw [c] (1.1335,2.27216) -- (1,2.27216);
\draw [c] (1.1335,2.38223) -- (1,2.38223);
\draw [c] (1.1335,2.46032) -- (1,2.46032);
\draw [c] (1.1335,2.52089) -- (1,2.52089);
\draw [c] (1.1335,2.57039) -- (1,2.57039);
\draw [c] (1.1335,2.61223) -- (1,2.61223);
\draw [c] (1.1335,2.64848) -- (1,2.64848);
\draw [c] (1.1335,2.68046) -- (1,2.68046);
\draw [c] (1.267,2.70906) -- (1,2.70906);
\draw [anchor= east] (0.922,2.70906) node[color=c, rotate=0]{$10^{-7}$};
\draw [c] (1.1335,2.89722) -- (1,2.89722);
\draw [c] (1.1335,3.00729) -- (1,3.00729);
\draw [c] (1.1335,3.08538) -- (1,3.08538);
\draw [c] (1.1335,3.14596) -- (1,3.14596);
\draw [c] (1.1335,3.19545) -- (1,3.19545);
\draw [c] (1.1335,3.2373) -- (1,3.2373);
\draw [c] (1.1335,3.27354) -- (1,3.27354);
\draw [c] (1.1335,3.30552) -- (1,3.30552);
\draw [c] (1.267,3.33412) -- (1,3.33412);
\draw [anchor= east] (0.922,3.33412) node[color=c, rotate=0]{$10^{-6}$};
\draw [c] (1.1335,3.52228) -- (1,3.52228);
\draw [c] (1.1335,3.63235) -- (1,3.63235);
\draw [c] (1.1335,3.71044) -- (1,3.71044);
\draw [c] (1.1335,3.77102) -- (1,3.77102);
\draw [c] (1.1335,3.82051) -- (1,3.82051);
\draw [c] (1.1335,3.86236) -- (1,3.86236);
\draw [c] (1.1335,3.89861) -- (1,3.89861);
\draw [c] (1.1335,3.93058) -- (1,3.93058);
\draw [c] (1.267,3.95918) -- (1,3.95918);
\draw [anchor= east] (0.922,3.95918) node[color=c, rotate=0]{$10^{-5}$};
\draw [c] (1.1335,4.14734) -- (1,4.14734);
\draw [c] (1.1335,4.25741) -- (1,4.25741);
\draw [c] (1.1335,4.33551) -- (1,4.33551);
\draw [c] (1.1335,4.39608) -- (1,4.39608);
\draw [c] (1.1335,4.44557) -- (1,4.44557);
\draw [c] (1.1335,4.48742) -- (1,4.48742);
\draw [c] (1.1335,4.52367) -- (1,4.52367);
\draw [c] (1.1335,4.55564) -- (1,4.55564);
\draw [c] (1.267,4.58424) -- (1,4.58424);
\draw [anchor= east] (0.922,4.58424) node[color=c, rotate=0]{$10^{-4}$};
\draw [c] (1.1335,4.77241) -- (1,4.77241);
\draw [c] (1.1335,4.88247) -- (1,4.88247);
\draw [c] (1.1335,4.96057) -- (1,4.96057);
\draw [c] (1.1335,5.02114) -- (1,5.02114);
\draw [c] (1.1335,5.07064) -- (1,5.07064);
\draw [c] (1.1335,5.11248) -- (1,5.11248);
\draw [c] (1.1335,5.14873) -- (1,5.14873);
\draw [c] (1.1335,5.1807) -- (1,5.1807);
\draw [c] (1.267,5.20931) -- (1,5.20931);
\draw [anchor= east] (0.922,5.20931) node[color=c, rotate=0]{$10^{-3}$};
\draw [c] (1.1335,5.39747) -- (1,5.39747);
\draw [c] (1.1335,5.50754) -- (1,5.50754);
\draw [c] (1.1335,5.58563) -- (1,5.58563);
\draw [c] (1.1335,5.6462) -- (1,5.6462);
\draw [c] (1.1335,5.6957) -- (1,5.6957);
\draw [c] (1.1335,5.73754) -- (1,5.73754);
\draw [c] (1.1335,5.77379) -- (1,5.77379);
\draw [c] (1.1335,5.80577) -- (1,5.80577);
\draw [c] (1.267,5.83437) -- (1,5.83437);
\draw [anchor= east] (0.922,5.83437) node[color=c, rotate=0]{$10^{-2}$};
\draw [c] (1.1335,6.02253) -- (1,6.02253);
\draw [c] (1.1335,6.1326) -- (1,6.1326);
\draw [c] (1.1335,6.21069) -- (1,6.21069);
\draw [c] (1.1335,6.27127) -- (1,6.27127);
\draw [c] (1.1335,6.32076) -- (1,6.32076);
\draw [c] (1.1335,6.36261) -- (1,6.36261);
\draw [c] (1.1335,6.39885) -- (1,6.39885);
\draw [c] (1.1335,6.43083) -- (1,6.43083);
\draw [c] (1.267,6.45943) -- (1,6.45943);
\draw [anchor= east] (0.922,6.45943) node[color=c, rotate=0]{$10^{-1}$};
\draw [c] (1.1335,6.64759) -- (1,6.64759);
\colorlet{c}{natcomp};
\draw [c] (1.0895,6.55674) -- (1.0895,6.55964);
\draw [c] (1.0895,6.55964) -- (1.0895,6.56251);
\draw [c] (1,6.55964) -- (1.0895,6.55964);
\draw [c] (1.0895,6.55964) -- (1.179,6.55964);
\definecolor{c}{rgb}{0,0,0};
\colorlet{c}{natcomp};
\draw [c] (1.2685,5.89309) -- (1.2685,5.89461);
\draw [c] (1.2685,5.89461) -- (1.2685,5.89613);
\draw [c] (1.179,5.89461) -- (1.2685,5.89461);
\draw [c] (1.2685,5.89461) -- (1.358,5.89461);
\definecolor{c}{rgb}{0,0,0};
\colorlet{c}{natcomp};
\draw [c] (1.4475,5.30142) -- (1.4475,5.30596);
\draw [c] (1.4475,5.30596) -- (1.4475,5.31042);
\draw [c] (1.358,5.30596) -- (1.4475,5.30596);
\draw [c] (1.4475,5.30596) -- (1.537,5.30596);
\definecolor{c}{rgb}{0,0,0};
\colorlet{c}{natcomp};
\draw [c] (1.6265,4.90829) -- (1.6265,4.91764);
\draw [c] (1.6265,4.91764) -- (1.6265,4.92668);
\draw [c] (1.537,4.91764) -- (1.6265,4.91764);
\draw [c] (1.6265,4.91764) -- (1.716,4.91764);
\definecolor{c}{rgb}{0,0,0};
\colorlet{c}{natcomp};
\draw [c] (1.8055,4.60816) -- (1.8055,4.62435);
\draw [c] (1.8055,4.62435) -- (1.8055,4.63962);
\draw [c] (1.716,4.62435) -- (1.8055,4.62435);
\draw [c] (1.8055,4.62435) -- (1.895,4.62435);
\definecolor{c}{rgb}{0,0,0};
\colorlet{c}{natcomp};
\draw [c] (1.9845,4.35619) -- (1.9845,4.38162);
\draw [c] (1.9845,4.38162) -- (1.9845,4.40488);
\draw [c] (1.895,4.38162) -- (1.9845,4.38162);
\draw [c] (1.9845,4.38162) -- (2.074,4.38162);
\definecolor{c}{rgb}{0,0,0};
\colorlet{c}{natcomp};
\draw [c] (2.1635,4.17802) -- (2.1635,4.21254);
\draw [c] (2.1635,4.21254) -- (2.1635,4.24316);
\draw [c] (2.074,4.21254) -- (2.1635,4.21254);
\draw [c] (2.1635,4.21254) -- (2.253,4.21254);
\definecolor{c}{rgb}{0,0,0};
\colorlet{c}{natcomp};
\draw [c] (2.3425,3.97882) -- (2.3425,4.02593);
\draw [c] (2.3425,4.02593) -- (2.3425,4.06605);
\draw [c] (2.253,4.02593) -- (2.3425,4.02593);
\draw [c] (2.3425,4.02593) -- (2.432,4.02593);
\definecolor{c}{rgb}{0,0,0};
\colorlet{c}{natcomp};
\draw [c] (2.5215,3.81165) -- (2.5215,3.86871);
\draw [c] (2.5215,3.86871) -- (2.5215,3.91583);
\draw [c] (2.432,3.86871) -- (2.5215,3.86871);
\draw [c] (2.5215,3.86871) -- (2.611,3.86871);
\definecolor{c}{rgb}{0,0,0};
\colorlet{c}{natcomp};
\draw [c] (2.7005,3.67641) -- (2.7005,3.72424);
\draw [c] (2.7005,3.72424) -- (2.7005,3.76489);
\draw [c] (2.611,3.72424) -- (2.7005,3.72424);
\draw [c] (2.7005,3.72424) -- (2.79,3.72424);
\definecolor{c}{rgb}{0,0,0};
\colorlet{c}{natcomp};
\draw [c] (2.8795,3.66275) -- (2.8795,3.66576);
\draw [c] (2.8795,3.66576) -- (2.8795,3.66874);
\draw [c] (2.79,3.66576) -- (2.8795,3.66576);
\draw [c] (2.8795,3.66576) -- (2.969,3.66576);
\definecolor{c}{rgb}{0,0,0};
\colorlet{c}{natcomp};
\draw [c] (3.0585,3.57589) -- (3.0585,3.57942);
\draw [c] (3.0585,3.57942) -- (3.0585,3.58291);
\draw [c] (2.969,3.57942) -- (3.0585,3.57942);
\draw [c] (3.0585,3.57942) -- (3.148,3.57942);
\definecolor{c}{rgb}{0,0,0};
\colorlet{c}{natcomp};
\draw [c] (3.2375,3.50754) -- (3.2375,3.51154);
\draw [c] (3.2375,3.51154) -- (3.2375,3.51548);
\draw [c] (3.148,3.51154) -- (3.2375,3.51154);
\draw [c] (3.2375,3.51154) -- (3.327,3.51154);
\definecolor{c}{rgb}{0,0,0};
\colorlet{c}{natcomp};
\draw [c] (3.4165,3.44805) -- (3.4165,3.45252);
\draw [c] (3.4165,3.45252) -- (3.4165,3.45691);
\draw [c] (3.327,3.45252) -- (3.4165,3.45252);
\draw [c] (3.4165,3.45252) -- (3.506,3.45252);
\definecolor{c}{rgb}{0,0,0};
\colorlet{c}{natcomp};
\draw [c] (3.5955,3.39872) -- (3.5955,3.40361);
\draw [c] (3.5955,3.40361) -- (3.5955,3.40842);
\draw [c] (3.506,3.40361) -- (3.5955,3.40361);
\draw [c] (3.5955,3.40361) -- (3.685,3.40361);
\definecolor{c}{rgb}{0,0,0};
\colorlet{c}{natcomp};
\draw [c] (3.7745,3.35464) -- (3.7745,3.35994);
\draw [c] (3.7745,3.35994) -- (3.7745,3.36515);
\draw [c] (3.685,3.35994) -- (3.7745,3.35994);
\draw [c] (3.7745,3.35994) -- (3.864,3.35994);
\definecolor{c}{rgb}{0,0,0};
\colorlet{c}{natcomp};
\draw [c] (3.9535,3.3296) -- (3.9535,3.33516);
\draw [c] (3.9535,3.33516) -- (3.9535,3.3406);
\draw [c] (3.864,3.33516) -- (3.9535,3.33516);
\draw [c] (3.9535,3.33516) -- (4.043,3.33516);
\definecolor{c}{rgb}{0,0,0};
\colorlet{c}{natcomp};
\draw [c] (4.1325,3.29394) -- (4.1325,3.29988);
\draw [c] (4.1325,3.29988) -- (4.1325,3.30568);
\draw [c] (4.043,3.29988) -- (4.1325,3.29988);
\draw [c] (4.1325,3.29988) -- (4.222,3.29988);
\definecolor{c}{rgb}{0,0,0};
\colorlet{c}{natcomp};
\draw [c] (4.3115,3.26114) -- (4.3115,3.26744);
\draw [c] (4.3115,3.26744) -- (4.3115,3.2736);
\draw [c] (4.222,3.26744) -- (4.3115,3.26744);
\draw [c] (4.3115,3.26744) -- (4.401,3.26744);
\definecolor{c}{rgb}{0,0,0};
\colorlet{c}{natcomp};
\draw [c] (4.4905,3.24119) -- (4.4905,3.24773);
\draw [c] (4.4905,3.24773) -- (4.4905,3.25412);
\draw [c] (4.401,3.24773) -- (4.4905,3.24773);
\draw [c] (4.4905,3.24773) -- (4.58,3.24773);
\definecolor{c}{rgb}{0,0,0};
\colorlet{c}{natcomp};
\draw [c] (4.6695,3.22385) -- (4.6695,3.2306);
\draw [c] (4.6695,3.2306) -- (4.6695,3.23719);
\draw [c] (4.58,3.2306) -- (4.6695,3.2306);
\draw [c] (4.6695,3.2306) -- (4.759,3.2306);
\definecolor{c}{rgb}{0,0,0};
\colorlet{c}{natcomp};
\draw [c] (4.8485,3.19632) -- (4.8485,3.20342);
\draw [c] (4.8485,3.20342) -- (4.8485,3.21034);
\draw [c] (4.759,3.20342) -- (4.8485,3.20342);
\draw [c] (4.8485,3.20342) -- (4.938,3.20342);
\definecolor{c}{rgb}{0,0,0};
\colorlet{c}{natcomp};
\draw [c] (5.0275,3.15202) -- (5.0275,3.15972);
\draw [c] (5.0275,3.15972) -- (5.0275,3.16722);
\draw [c] (4.938,3.15972) -- (5.0275,3.15972);
\draw [c] (5.0275,3.15972) -- (5.117,3.15972);
\definecolor{c}{rgb}{0,0,0};
\colorlet{c}{natcomp};
\draw [c] (5.2065,3.1342) -- (5.2065,3.14216);
\draw [c] (5.2065,3.14216) -- (5.2065,3.1499);
\draw [c] (5.117,3.14216) -- (5.2065,3.14216);
\draw [c] (5.2065,3.14216) -- (5.296,3.14216);
\definecolor{c}{rgb}{0,0,0};
\colorlet{c}{natcomp};
\draw [c] (5.3855,3.1038) -- (5.3855,3.11222);
\draw [c] (5.3855,3.11222) -- (5.3855,3.12039);
\draw [c] (5.296,3.11222) -- (5.3855,3.11222);
\draw [c] (5.3855,3.11222) -- (5.475,3.11222);
\definecolor{c}{rgb}{0,0,0};
\colorlet{c}{natcomp};
\draw [c] (5.5645,3.06041) -- (5.5645,3.06953);
\draw [c] (5.5645,3.06953) -- (5.5645,3.07835);
\draw [c] (5.475,3.06953) -- (5.5645,3.06953);
\draw [c] (5.5645,3.06953) -- (5.654,3.06953);
\definecolor{c}{rgb}{0,0,0};
\colorlet{c}{natcomp};
\draw [c] (5.7435,3.04299) -- (5.7435,3.05241);
\draw [c] (5.7435,3.05241) -- (5.7435,3.06151);
\draw [c] (5.654,3.05241) -- (5.7435,3.05241);
\draw [c] (5.7435,3.05241) -- (5.833,3.05241);
\definecolor{c}{rgb}{0,0,0};
\colorlet{c}{natcomp};
\draw [c] (5.9225,3.01636) -- (5.9225,3.02625);
\draw [c] (5.9225,3.02625) -- (5.9225,3.03579);
\draw [c] (5.833,3.02625) -- (5.9225,3.02625);
\draw [c] (5.9225,3.02625) -- (6.012,3.02625);
\definecolor{c}{rgb}{0,0,0};
\colorlet{c}{natcomp};
\draw [c] (6.1015,2.98288) -- (6.1015,2.9934);
\draw [c] (6.1015,2.9934) -- (6.1015,3.00353);
\draw [c] (6.012,2.9934) -- (6.1015,2.9934);
\draw [c] (6.1015,2.9934) -- (6.191,2.9934);
\definecolor{c}{rgb}{0,0,0};
\colorlet{c}{natcomp};
\draw [c] (6.2805,2.93776) -- (6.2805,2.94919);
\draw [c] (6.2805,2.94919) -- (6.2805,2.96016);
\draw [c] (6.191,2.94919) -- (6.2805,2.94919);
\draw [c] (6.2805,2.94919) -- (6.37,2.94919);
\definecolor{c}{rgb}{0,0,0};
\colorlet{c}{natcomp};
\draw [c] (6.4595,2.89827) -- (6.4595,2.91056);
\draw [c] (6.4595,2.91056) -- (6.4595,2.92232);
\draw [c] (6.37,2.91056) -- (6.4595,2.91056);
\draw [c] (6.4595,2.91056) -- (6.549,2.91056);
\definecolor{c}{rgb}{0,0,0};
\colorlet{c}{natcomp};
\draw [c] (6.6385,2.84691) -- (6.6385,2.86043);
\draw [c] (6.6385,2.86043) -- (6.6385,2.8733);
\draw [c] (6.549,2.86043) -- (6.6385,2.86043);
\draw [c] (6.6385,2.86043) -- (6.728,2.86043);
\definecolor{c}{rgb}{0,0,0};
\colorlet{c}{natcomp};
\draw [c] (6.8175,2.78862) -- (6.8175,2.80367);
\draw [c] (6.8175,2.80367) -- (6.8175,2.81792);
\draw [c] (6.728,2.80367) -- (6.8175,2.80367);
\draw [c] (6.8175,2.80367) -- (6.907,2.80367);
\definecolor{c}{rgb}{0,0,0};
\colorlet{c}{natcomp};
\draw [c] (6.9965,2.78453) -- (6.9965,2.79969);
\draw [c] (6.9965,2.79969) -- (6.9965,2.81405);
\draw [c] (6.907,2.79969) -- (6.9965,2.79969);
\draw [c] (6.9965,2.79969) -- (7.086,2.79969);
\definecolor{c}{rgb}{0,0,0};
\colorlet{c}{natcomp};
\draw [c] (7.1755,2.72804) -- (7.1755,2.74486);
\draw [c] (7.1755,2.74486) -- (7.1755,2.7607);
\draw [c] (7.086,2.74486) -- (7.1755,2.74486);
\draw [c] (7.1755,2.74486) -- (7.265,2.74486);
\definecolor{c}{rgb}{0,0,0};
\colorlet{c}{natcomp};
\draw [c] (7.3545,2.61909) -- (7.3545,2.63965);
\draw [c] (7.3545,2.63965) -- (7.3545,2.65876);
\draw [c] (7.265,2.63965) -- (7.3545,2.63965);
\draw [c] (7.3545,2.63965) -- (7.444,2.63965);
\definecolor{c}{rgb}{0,0,0};
\colorlet{c}{natcomp};
\draw [c] (7.5335,2.6294) -- (7.5335,2.64958);
\draw [c] (7.5335,2.64958) -- (7.5335,2.66835);
\draw [c] (7.444,2.64958) -- (7.5335,2.64958);
\draw [c] (7.5335,2.64958) -- (7.623,2.64958);
\definecolor{c}{rgb}{0,0,0};
\colorlet{c}{natcomp};
\draw [c] (7.7125,2.60207) -- (7.7125,2.62328);
\draw [c] (7.7125,2.62328) -- (7.7125,2.64296);
\draw [c] (7.623,2.62328) -- (7.7125,2.62328);
\draw [c] (7.7125,2.62328) -- (7.802,2.62328);
\definecolor{c}{rgb}{0,0,0};
\colorlet{c}{natcomp};
\draw [c] (7.8915,2.47965) -- (7.8915,2.50623);
\draw [c] (7.8915,2.50623) -- (7.8915,2.53043);
\draw [c] (7.802,2.50623) -- (7.8915,2.50623);
\draw [c] (7.8915,2.50623) -- (7.981,2.50623);
\definecolor{c}{rgb}{0,0,0};
\colorlet{c}{natcomp};
\draw [c] (8.0705,2.43095) -- (8.0705,2.46002);
\draw [c] (8.0705,2.46002) -- (8.0705,2.48627);
\draw [c] (7.981,2.46002) -- (8.0705,2.46002);
\draw [c] (8.0705,2.46002) -- (8.16,2.46002);
\definecolor{c}{rgb}{0,0,0};
\colorlet{c}{natcomp};
\draw [c] (8.2495,2.37553) -- (8.2495,2.40771);
\draw [c] (8.2495,2.40771) -- (8.2495,2.43648);
\draw [c] (8.16,2.40771) -- (8.2495,2.40771);
\draw [c] (8.2495,2.40771) -- (8.339,2.40771);
\definecolor{c}{rgb}{0,0,0};
\colorlet{c}{natcomp};
\draw [c] (8.4285,2.31089) -- (8.4285,2.34714);
\draw [c] (8.4285,2.34714) -- (8.4285,2.37911);
\draw [c] (8.339,2.34714) -- (8.4285,2.34714);
\draw [c] (8.4285,2.34714) -- (8.518,2.34714);
\definecolor{c}{rgb}{0,0,0};
\colorlet{c}{natcomp};
\draw [c] (8.6075,2.25583) -- (8.6075,2.29594);
\draw [c] (8.6075,2.29594) -- (8.6075,2.33088);
\draw [c] (8.518,2.29594) -- (8.6075,2.29594);
\draw [c] (8.6075,2.29594) -- (8.697,2.29594);
\definecolor{c}{rgb}{0,0,0};
\colorlet{c}{natcomp};
\draw [c] (8.7865,2.20106) -- (8.7865,2.24542);
\draw [c] (8.7865,2.24542) -- (8.7865,2.28354);
\draw [c] (8.697,2.24542) -- (8.7865,2.24542);
\draw [c] (8.7865,2.24542) -- (8.876,2.24542);
\definecolor{c}{rgb}{0,0,0};
\colorlet{c}{natcomp};
\draw [c] (8.9655,2.03139) -- (8.9655,2.09196);
\draw [c] (8.9655,2.09196) -- (8.9655,2.14145);
\draw [c] (8.876,2.09196) -- (8.9655,2.09196);
\draw [c] (8.9655,2.09196) -- (9.055,2.09196);
\definecolor{c}{rgb}{0,0,0};
\colorlet{c}{natcomp};
\draw [c] (9.1445,1.87221) -- (9.1445,1.95329);
\draw [c] (9.1445,1.95329) -- (9.1445,2.01564);
\draw [c] (9.055,1.95329) -- (9.1445,1.95329);
\draw [c] (9.1445,1.95329) -- (9.234,1.95329);
\definecolor{c}{rgb}{0,0,0};
\colorlet{c}{natcomp};
\draw [c] (9.3235,1.91186) -- (9.3235,1.98727);
\draw [c] (9.3235,1.98727) -- (9.3235,2.04622);
\draw [c] (9.234,1.98727) -- (9.3235,1.98727);
\draw [c] (9.3235,1.98727) -- (9.413,1.98727);
\definecolor{c}{rgb}{0,0,0};
\colorlet{c}{natcomp};
\draw [c] (9.5025,1.61752) -- (9.5025,1.7464);
\draw [c] (9.5025,1.7464) -- (9.5025,1.83343);
\draw [c] (9.413,1.7464) -- (9.5025,1.7464);
\draw [c] (9.5025,1.7464) -- (9.592,1.7464);
\definecolor{c}{rgb}{0,0,0};
\colorlet{c}{natcomp};
\draw [c] (9.6815,1.49414) -- (9.6815,1.65506);
\draw [c] (9.6815,1.65506) -- (9.6815,1.7554);
\draw [c] (9.592,1.65506) -- (9.6815,1.65506);
\draw [c] (9.6815,1.65506) -- (9.771,1.65506);
\definecolor{c}{rgb}{0,0,0};
\colorlet{c}{natcomp};
\draw [c] (9.8605,1.80025) -- (9.8605,1.89272);
\draw [c] (9.8605,1.89272) -- (9.8605,1.96156);
\draw [c] (9.771,1.89272) -- (9.8605,1.89272);
\draw [c] (9.8605,1.89272) -- (9.95,1.89272);
\definecolor{c}{rgb}{0,0,0};
\definecolor{c}{rgb}{1,1,1};
\draw [color=c, fill=c] (6,5.44413) rectangle (9.8,6.66905);
\definecolor{c}{rgb}{0,0,0};
\draw [anchor=base west] (6.95,6.22502) node[color=c, rotate=0]{Standard Model};
\colorlet{c}{natgreen};
\draw [c] (6.1425,6.36282) -- (6.8075,6.36282);
\definecolor{c}{rgb}{0,0,0};
\draw [anchor=base west] (6.95,5.61255) node[color=c, rotate=0]{$\Lambda = 1.0 TeV$};
\colorlet{c}{natcomp};
\draw [c] (6.1425,5.75036) -- (6.8075,5.75036);
\end{tikzpicture}
}
\end{infilsf}
\vspace{-1em}
\subcaption{Significance: 23.6}
\end{minipage}
\begin{minipage}[t]{\textwidth}
\caption{The distributions of simulated events in four potential discriminating variables in event samples generated with Standard Model parameters, and with $\Lambda = 1.0$ TeV: (a) the cosine of $\theta_{\gamma_1}$, the scattering angle of the leading photon (measured, in this case, in the Collins-Soper frame \cite{collinssoper}), (b) $p_T^{\gamma_1}$, the transverse momentum of the leading (most energetic) photon, (c) $M_{\gamma\gamma}$, the invariant mass of the photon pair and (d) $|p_T|^{\gamma\gamma}$, the sum of magnitudes of $p_T$ of both photons. Both samples are generated with CalcHEP. The significance is calculated according to eq.~\eqref{significance}.
\label{discr}}
\end{minipage}
\end{figure}


Given the significances quoted in that figure, $p^{\gamma_1}_T$, the transverse momentum of the leading photon, is the obvious choice for a discriminating variable. It, along with the scattering angle, has the disadvantage when compared with the latter two methods, however, that it is dependant on identifying the leading---most energetic---photon of the pair. In a truth sample such as this, making such an identification does not present a problem, however when considering the effects on a photon of passing through the material of the detector, it becomes problematic to claim that the photon that leaves the largest energy deposit in the calorimeter is also the photon that left the hard event with the greatest amount of energy. The two remaining variables both incorporate information about both photons. The invariant mass additionally incorporates information about the relative orientation of the photons, which arguably makes it the most sophisticated measure. 


For this reason, we will use invariant mass as the discriminating variable going forward.

Meanwhile, figure~\ref{sigcos} shows no discernible difference between the SM sample and the sample generated with a 1 TeV mass scale contact interaction. Evidently, at the hard process level, the distribution of angles between photons is not affected by the new term.

\section{Parton Distribution Functions}
As described in section~\ref{sec.pdfth}, we use a set of experimentally determined functions called Parton Distribution Functions (PDFs) to describe the probability of extracting a given parton from a proton. Given that the PDFs are not exact analytical models, we must account for the uncertainty associated with the method by which  PDFs are determined. To estimate that uncertainty, we compare the distribution in invariant masses of events generated by CalcHEP using the CTEQ6 set of PDFs, which are the events that will be used moving forward, with events generated using the alternative MRST2002nlo set of PDF, which is the only alternative PDF available in CalcHEP.

\begin{figure}[htp]
\begin{minipage}[b]{.69\textwidth}
\begin{infilsf}\tiny
\begin{tikzpicture}[x=.095\textwidth,y=.095\textwidth]
\pgfdeclareplotmark{cross} {
\pgfpathmoveto{\pgfpoint{-0.3\pgfplotmarksize}{\pgfplotmarksize}}
\pgfpathlineto{\pgfpoint{+0.3\pgfplotmarksize}{\pgfplotmarksize}}
\pgfpathlineto{\pgfpoint{+0.3\pgfplotmarksize}{0.3\pgfplotmarksize}}
\pgfpathlineto{\pgfpoint{+1\pgfplotmarksize}{0.3\pgfplotmarksize}}
\pgfpathlineto{\pgfpoint{+1\pgfplotmarksize}{-0.3\pgfplotmarksize}}
\pgfpathlineto{\pgfpoint{+0.3\pgfplotmarksize}{-0.3\pgfplotmarksize}}
\pgfpathlineto{\pgfpoint{+0.3\pgfplotmarksize}{-1.\pgfplotmarksize}}
\pgfpathlineto{\pgfpoint{-0.3\pgfplotmarksize}{-1.\pgfplotmarksize}}
\pgfpathlineto{\pgfpoint{-0.3\pgfplotmarksize}{-0.3\pgfplotmarksize}}
\pgfpathlineto{\pgfpoint{-1.\pgfplotmarksize}{-0.3\pgfplotmarksize}}
\pgfpathlineto{\pgfpoint{-1.\pgfplotmarksize}{0.3\pgfplotmarksize}}
\pgfpathlineto{\pgfpoint{-0.3\pgfplotmarksize}{0.3\pgfplotmarksize}}
\pgfpathclose
\pgfusepathqstroke
}
\pgfdeclareplotmark{cross*} {
\pgfpathmoveto{\pgfpoint{-0.3\pgfplotmarksize}{\pgfplotmarksize}}
\pgfpathlineto{\pgfpoint{+0.3\pgfplotmarksize}{\pgfplotmarksize}}
\pgfpathlineto{\pgfpoint{+0.3\pgfplotmarksize}{0.3\pgfplotmarksize}}
\pgfpathlineto{\pgfpoint{+1\pgfplotmarksize}{0.3\pgfplotmarksize}}
\pgfpathlineto{\pgfpoint{+1\pgfplotmarksize}{-0.3\pgfplotmarksize}}
\pgfpathlineto{\pgfpoint{+0.3\pgfplotmarksize}{-0.3\pgfplotmarksize}}
\pgfpathlineto{\pgfpoint{+0.3\pgfplotmarksize}{-1.\pgfplotmarksize}}
\pgfpathlineto{\pgfpoint{-0.3\pgfplotmarksize}{-1.\pgfplotmarksize}}
\pgfpathlineto{\pgfpoint{-0.3\pgfplotmarksize}{-0.3\pgfplotmarksize}}
\pgfpathlineto{\pgfpoint{-1.\pgfplotmarksize}{-0.3\pgfplotmarksize}}
\pgfpathlineto{\pgfpoint{-1.\pgfplotmarksize}{0.3\pgfplotmarksize}}
\pgfpathlineto{\pgfpoint{-0.3\pgfplotmarksize}{0.3\pgfplotmarksize}}
\pgfpathclose
\pgfusepathqfillstroke
}
\pgfdeclareplotmark{newstar} {
\pgfpathmoveto{\pgfqpoint{0pt}{\pgfplotmarksize}}
\pgfpathlineto{\pgfqpointpolar{44}{0.5\pgfplotmarksize}}
\pgfpathlineto{\pgfqpointpolar{18}{\pgfplotmarksize}}
\pgfpathlineto{\pgfqpointpolar{-20}{0.5\pgfplotmarksize}}
\pgfpathlineto{\pgfqpointpolar{-54}{\pgfplotmarksize}}
\pgfpathlineto{\pgfqpointpolar{-90}{0.5\pgfplotmarksize}}
\pgfpathlineto{\pgfqpointpolar{234}{\pgfplotmarksize}}
\pgfpathlineto{\pgfqpointpolar{198}{0.5\pgfplotmarksize}}
\pgfpathlineto{\pgfqpointpolar{162}{\pgfplotmarksize}}
\pgfpathlineto{\pgfqpointpolar{134}{0.5\pgfplotmarksize}}
\pgfpathclose
\pgfusepathqstroke
}
\pgfdeclareplotmark{newstar*} {
\pgfpathmoveto{\pgfqpoint{0pt}{\pgfplotmarksize}}
\pgfpathlineto{\pgfqpointpolar{44}{0.5\pgfplotmarksize}}
\pgfpathlineto{\pgfqpointpolar{18}{\pgfplotmarksize}}
\pgfpathlineto{\pgfqpointpolar{-20}{0.5\pgfplotmarksize}}
\pgfpathlineto{\pgfqpointpolar{-54}{\pgfplotmarksize}}
\pgfpathlineto{\pgfqpointpolar{-90}{0.5\pgfplotmarksize}}
\pgfpathlineto{\pgfqpointpolar{234}{\pgfplotmarksize}}
\pgfpathlineto{\pgfqpointpolar{198}{0.5\pgfplotmarksize}}
\pgfpathlineto{\pgfqpointpolar{162}{\pgfplotmarksize}}
\pgfpathlineto{\pgfqpointpolar{134}{0.5\pgfplotmarksize}}
\pgfpathclose
\pgfusepathqfillstroke
}
\definecolor{c}{rgb}{1,1,1};
\draw [color=c, fill=c] (0,0) rectangle (9.35135,10);
\draw [color=c, fill=c] (0,4.85714) rectangle (9.35135,10);
\draw [color=c, fill=c] (0.935135,4.85714) rectangle (9.30459,9.94857);
\definecolor{c}{rgb}{0,0,0};
\draw [c] (0.935135,4.85714) -- (0.935135,9.94857) -- (9.30459,9.94857) -- (9.30459,4.85714) -- (0.935135,4.85714);
\definecolor{c}{rgb}{1,1,1};
\draw [color=c, fill=c] (0.935135,4.85714) rectangle (9.30459,9.94857);
\definecolor{c}{rgb}{0,0,0};
\draw [c] (0.935135,4.85714) -- (0.935135,9.94857) -- (9.30459,9.94857) -- (9.30459,4.85714) -- (0.935135,4.85714);
\colorlet{c}{kugray};
\draw [c] (1.01883,9.79694) -- (1.01883,9.79959);
\draw [c] (1.01883,9.79959) -- (1.01883,9.8022);
\draw [c] (0.935135,9.79959) -- (1.01883,9.79959);
\draw [c] (1.01883,9.79959) -- (1.10252,9.79959);
\definecolor{c}{rgb}{0,0,0};
\colorlet{c}{kugray};
\draw [c] (1.18622,9.38566) -- (1.18622,9.38662);
\draw [c] (1.18622,9.38662) -- (1.18622,9.38759);
\draw [c] (1.10252,9.38662) -- (1.18622,9.38662);
\draw [c] (1.18622,9.38662) -- (1.26991,9.38662);
\definecolor{c}{rgb}{0,0,0};
\colorlet{c}{kugray};
\draw [c] (1.35361,8.89544) -- (1.35361,8.89826);
\draw [c] (1.35361,8.89826) -- (1.35361,8.90105);
\draw [c] (1.26991,8.89826) -- (1.35361,8.89826);
\draw [c] (1.35361,8.89826) -- (1.4373,8.89826);
\definecolor{c}{rgb}{0,0,0};
\colorlet{c}{kugray};
\draw [c] (1.521,8.58253) -- (1.521,8.58811);
\draw [c] (1.521,8.58811) -- (1.521,8.59356);
\draw [c] (1.4373,8.58811) -- (1.521,8.58811);
\draw [c] (1.521,8.58811) -- (1.60469,8.58811);
\definecolor{c}{rgb}{0,0,0};
\colorlet{c}{kugray};
\draw [c] (1.68839,8.33709) -- (1.68839,8.34663);
\draw [c] (1.68839,8.34663) -- (1.68839,8.35578);
\draw [c] (1.60469,8.34663) -- (1.68839,8.34663);
\draw [c] (1.68839,8.34663) -- (1.77208,8.34663);
\definecolor{c}{rgb}{0,0,0};
\colorlet{c}{kugray};
\draw [c] (1.85578,8.14456) -- (1.85578,8.15908);
\draw [c] (1.85578,8.15908) -- (1.85578,8.17274);
\draw [c] (1.77208,8.15908) -- (1.85578,8.15908);
\draw [c] (1.85578,8.15908) -- (1.93947,8.15908);
\definecolor{c}{rgb}{0,0,0};
\colorlet{c}{kugray};
\draw [c] (2.02316,7.97861) -- (2.02316,7.99947);
\draw [c] (2.02316,7.99947) -- (2.02316,8.01858);
\draw [c] (1.93947,7.99947) -- (2.02316,7.99947);
\draw [c] (2.02316,7.99947) -- (2.10686,7.99947);
\definecolor{c}{rgb}{0,0,0};
\colorlet{c}{kugray};
\draw [c] (2.19055,7.77888) -- (2.19055,7.81112);
\draw [c] (2.19055,7.81112) -- (2.19055,7.83938);
\draw [c] (2.10686,7.81112) -- (2.19055,7.81112);
\draw [c] (2.19055,7.81112) -- (2.27425,7.81112);
\definecolor{c}{rgb}{0,0,0};
\colorlet{c}{kugray};
\draw [c] (2.35794,7.63037) -- (2.35794,7.67492);
\draw [c] (2.35794,7.67492) -- (2.35794,7.7122);
\draw [c] (2.27425,7.67492) -- (2.35794,7.67492);
\draw [c] (2.35794,7.67492) -- (2.44164,7.67492);
\definecolor{c}{rgb}{0,0,0};
\colorlet{c}{kugray};
\draw [c] (2.52533,7.54573) -- (2.52533,7.59928);
\draw [c] (2.52533,7.59928) -- (2.52533,7.64266);
\draw [c] (2.44164,7.59928) -- (2.52533,7.59928);
\draw [c] (2.52533,7.59928) -- (2.60903,7.59928);
\definecolor{c}{rgb}{0,0,0};
\colorlet{c}{kugray};
\draw [c] (2.69272,7.45261) -- (2.69272,7.45408);
\draw [c] (2.69272,7.45408) -- (2.69272,7.45553);
\draw [c] (2.60903,7.45408) -- (2.69272,7.45408);
\draw [c] (2.69272,7.45408) -- (2.77642,7.45408);
\definecolor{c}{rgb}{0,0,0};
\colorlet{c}{kugray};
\draw [c] (2.86011,7.34236) -- (2.86011,7.34422);
\draw [c] (2.86011,7.34422) -- (2.86011,7.34607);
\draw [c] (2.77642,7.34422) -- (2.86011,7.34422);
\draw [c] (2.86011,7.34422) -- (2.94381,7.34422);
\definecolor{c}{rgb}{0,0,0};
\colorlet{c}{kugray};
\draw [c] (3.0275,7.23271) -- (3.0275,7.23507);
\draw [c] (3.0275,7.23507) -- (3.0275,7.23742);
\draw [c] (2.94381,7.23507) -- (3.0275,7.23507);
\draw [c] (3.0275,7.23507) -- (3.11119,7.23507);
\definecolor{c}{rgb}{0,0,0};
\colorlet{c}{kugray};
\draw [c] (3.19489,7.12719) -- (3.19489,7.13017);
\draw [c] (3.19489,7.13017) -- (3.19489,7.13311);
\draw [c] (3.11119,7.13017) -- (3.19489,7.13017);
\draw [c] (3.19489,7.13017) -- (3.27858,7.13017);
\definecolor{c}{rgb}{0,0,0};
\colorlet{c}{kugray};
\draw [c] (3.36228,7.02452) -- (3.36228,7.02825);
\draw [c] (3.36228,7.02825) -- (3.36228,7.03192);
\draw [c] (3.27858,7.02825) -- (3.36228,7.02825);
\draw [c] (3.36228,7.02825) -- (3.44597,7.02825);
\definecolor{c}{rgb}{0,0,0};
\colorlet{c}{kugray};
\draw [c] (3.52967,6.93785) -- (3.52967,6.94235);
\draw [c] (3.52967,6.94235) -- (3.52967,6.94677);
\draw [c] (3.44597,6.94235) -- (3.52967,6.94235);
\draw [c] (3.52967,6.94235) -- (3.61336,6.94235);
\definecolor{c}{rgb}{0,0,0};
\colorlet{c}{kugray};
\draw [c] (3.69706,6.83785) -- (3.69706,6.84346);
\draw [c] (3.69706,6.84346) -- (3.69706,6.84893);
\draw [c] (3.61336,6.84346) -- (3.69706,6.84346);
\draw [c] (3.69706,6.84346) -- (3.78075,6.84346);
\definecolor{c}{rgb}{0,0,0};
\colorlet{c}{kugray};
\draw [c] (3.86445,6.75896) -- (3.86445,6.76562);
\draw [c] (3.86445,6.76562) -- (3.86445,6.77209);
\draw [c] (3.78075,6.76562) -- (3.86445,6.76562);
\draw [c] (3.86445,6.76562) -- (3.94814,6.76562);
\definecolor{c}{rgb}{0,0,0};
\colorlet{c}{kugray};
\draw [c] (4.03184,6.67276) -- (4.03184,6.68079);
\draw [c] (4.03184,6.68079) -- (4.03184,6.68856);
\draw [c] (3.94814,6.68079) -- (4.03184,6.68079);
\draw [c] (4.03184,6.68079) -- (4.11553,6.68079);
\definecolor{c}{rgb}{0,0,0};
\colorlet{c}{kugray};
\draw [c] (4.19922,6.56086) -- (4.19922,6.57113);
\draw [c] (4.19922,6.57113) -- (4.19922,6.58095);
\draw [c] (4.11553,6.57113) -- (4.19922,6.57113);
\draw [c] (4.19922,6.57113) -- (4.28292,6.57113);
\definecolor{c}{rgb}{0,0,0};
\colorlet{c}{kugray};
\draw [c] (4.36661,6.44185) -- (4.36661,6.45515);
\draw [c] (4.36661,6.45515) -- (4.36661,6.46773);
\draw [c] (4.28292,6.45515) -- (4.36661,6.45515);
\draw [c] (4.36661,6.45515) -- (4.45031,6.45515);
\definecolor{c}{rgb}{0,0,0};
\colorlet{c}{kugray};
\draw [c] (4.534,6.33859) -- (4.534,6.35526);
\draw [c] (4.534,6.35526) -- (4.534,6.37079);
\draw [c] (4.45031,6.35526) -- (4.534,6.35526);
\draw [c] (4.534,6.35526) -- (4.6177,6.35526);
\definecolor{c}{rgb}{0,0,0};
\colorlet{c}{kugray};
\draw [c] (4.70139,6.22081) -- (4.70139,6.24236);
\draw [c] (4.70139,6.24236) -- (4.70139,6.26206);
\draw [c] (4.6177,6.24236) -- (4.70139,6.24236);
\draw [c] (4.70139,6.24236) -- (4.78509,6.24236);
\definecolor{c}{rgb}{0,0,0};
\colorlet{c}{kugray};
\draw [c] (4.86878,6.19632) -- (4.86878,6.21905);
\draw [c] (4.86878,6.21905) -- (4.86878,6.23973);
\draw [c] (4.78509,6.21905) -- (4.86878,6.21905);
\draw [c] (4.86878,6.21905) -- (4.95248,6.21905);
\definecolor{c}{rgb}{0,0,0};
\colorlet{c}{kugray};
\draw [c] (5.03617,6.07873) -- (5.03617,6.10811);
\draw [c] (5.03617,6.10811) -- (5.03617,6.13414);
\draw [c] (4.95248,6.10811) -- (5.03617,6.10811);
\draw [c] (5.03617,6.10811) -- (5.11986,6.10811);
\definecolor{c}{rgb}{0,0,0};
\colorlet{c}{kugray};
\draw [c] (5.20356,5.99441) -- (5.20356,6.02972);
\draw [c] (5.20356,6.02972) -- (5.20356,6.0603);
\draw [c] (5.11986,6.02972) -- (5.20356,6.02972);
\draw [c] (5.20356,6.02972) -- (5.28725,6.02972);
\definecolor{c}{rgb}{0,0,0};
\colorlet{c}{kugray};
\draw [c] (5.37095,5.86254) -- (5.37095,5.90959);
\draw [c] (5.37095,5.90959) -- (5.37095,5.9486);
\draw [c] (5.28725,5.90959) -- (5.37095,5.90959);
\draw [c] (5.37095,5.90959) -- (5.45464,5.90959);
\definecolor{c}{rgb}{0,0,0};
\colorlet{c}{kugray};
\draw [c] (5.53834,5.73879) -- (5.53834,5.80036);
\draw [c] (5.53834,5.80036) -- (5.53834,5.84883);
\draw [c] (5.45464,5.80036) -- (5.53834,5.80036);
\draw [c] (5.53834,5.80036) -- (5.62203,5.80036);
\definecolor{c}{rgb}{0,0,0};
\colorlet{c}{kugray};
\draw [c] (5.70573,5.73879) -- (5.70573,5.80036);
\draw [c] (5.70573,5.80036) -- (5.70573,5.84883);
\draw [c] (5.62203,5.80036) -- (5.70573,5.80036);
\draw [c] (5.70573,5.80036) -- (5.78942,5.80036);
\definecolor{c}{rgb}{0,0,0};
\colorlet{c}{kugray};
\draw [c] (5.87312,5.62948) -- (5.87312,5.70749);
\draw [c] (5.87312,5.70749) -- (5.87312,5.76558);
\draw [c] (5.78942,5.70749) -- (5.87312,5.70749);
\draw [c] (5.87312,5.70749) -- (5.95681,5.70749);
\definecolor{c}{rgb}{0,0,0};
\colorlet{c}{kugray};
\draw [c] (6.04051,5.60538) -- (6.04051,5.68757);
\draw [c] (6.04051,5.68757) -- (6.04051,5.74792);
\draw [c] (5.95681,5.68757) -- (6.04051,5.68757);
\draw [c] (6.04051,5.68757) -- (6.1242,5.68757);
\definecolor{c}{rgb}{0,0,0};
\colorlet{c}{kugray};
\draw [c] (6.20789,5.42858) -- (6.20789,5.54874);
\draw [c] (6.20789,5.54874) -- (6.20789,5.62715);
\draw [c] (6.1242,5.54874) -- (6.20789,5.54874);
\draw [c] (6.20789,5.54874) -- (6.29159,5.54874);
\definecolor{c}{rgb}{0,0,0};
\colorlet{c}{kugray};
\draw [c] (6.37528,5.62948) -- (6.37528,5.70749);
\draw [c] (6.37528,5.70749) -- (6.37528,5.76558);
\draw [c] (6.29159,5.70749) -- (6.37528,5.70749);
\draw [c] (6.37528,5.70749) -- (6.45898,5.70749);
\definecolor{c}{rgb}{0,0,0};
\colorlet{c}{kugray};
\draw [c] (6.54267,5.19275) -- (6.54267,5.38999);
\draw [c] (6.54267,5.38999) -- (6.54267,5.49437);
\draw [c] (6.45898,5.38999) -- (6.54267,5.38999);
\draw [c] (6.54267,5.38999) -- (6.62637,5.38999);
\definecolor{c}{rgb}{0,0,0};
\colorlet{c}{kugray};
\draw [c] (6.71006,4.85714) -- (6.71006,5.13838);
\draw [c] (6.71006,5.13838) -- (6.71006,5.29713);
\draw [c] (6.62637,5.13838) -- (6.71006,5.13838);
\draw [c] (6.71006,5.13838) -- (6.79376,5.13838);
\definecolor{c}{rgb}{0,0,0};
\colorlet{c}{kugray};
\draw [c] (6.87745,4.85714) -- (6.87745,5.13838);
\draw [c] (6.87745,5.13838) -- (6.87745,5.29713);
\draw [c] (6.79376,5.13838) -- (6.87745,5.13838);
\draw [c] (6.87745,5.13838) -- (6.96115,5.13838);
\definecolor{c}{rgb}{0,0,0};
\colorlet{c}{kugray};
\draw [c] (7.04484,5.01589) -- (7.04484,5.29713);
\draw [c] (7.04484,5.29713) -- (7.04484,5.41961);
\draw [c] (6.96115,5.29713) -- (7.04484,5.29713);
\draw [c] (7.04484,5.29713) -- (7.12854,5.29713);
\definecolor{c}{rgb}{0,0,0};
\colorlet{c}{kugray};
\draw [c] (7.37962,5.01589) -- (7.37962,5.29713);
\draw [c] (7.37962,5.29713) -- (7.37962,5.41961);
\draw [c] (7.29592,5.29713) -- (7.37962,5.29713);
\draw [c] (7.37962,5.29713) -- (7.46331,5.29713);
\definecolor{c}{rgb}{0,0,0};
\colorlet{c}{kugray};
\draw [c] (7.54701,4.85714) -- (7.54701,5.13838);
\draw [c] (7.54701,5.13838) -- (7.54701,5.29713);
\draw [c] (7.46331,5.13838) -- (7.54701,5.13838);
\draw [c] (7.54701,5.13838) -- (7.6307,5.13838);
\definecolor{c}{rgb}{0,0,0};
\colorlet{c}{kugray};
\draw [c] (7.88179,4.85714) -- (7.88179,5.13838);
\draw [c] (7.88179,5.13838) -- (7.88179,5.29713);
\draw [c] (7.79809,5.13838) -- (7.88179,5.13838);
\draw [c] (7.88179,5.13838) -- (7.96548,5.13838);
\definecolor{c}{rgb}{0,0,0};
\draw [c] (0.935135,4.85714) -- (9.30459,4.85714);
\draw [anchor= east] (9.30459,4.42514) node[color=c, rotate=0]{$M_{\gamma\gamma}$ [GeV]};
\draw [c] (0.935135,4.99523) -- (0.935135,4.85714);
\draw [c] (1.10252,4.89166) -- (1.10252,4.85714);
\draw [c] (1.26991,4.89166) -- (1.26991,4.85714);
\draw [c] (1.4373,4.89166) -- (1.4373,4.85714);
\draw [c] (1.60469,4.89166) -- (1.60469,4.85714);
\draw [c] (1.77208,4.92619) -- (1.77208,4.85714);
\draw [c] (1.93947,4.89166) -- (1.93947,4.85714);
\draw [c] (2.10686,4.89166) -- (2.10686,4.85714);
\draw [c] (2.27425,4.89166) -- (2.27425,4.85714);
\draw [c] (2.44164,4.89166) -- (2.44164,4.85714);
\draw [c] (2.60903,4.92619) -- (2.60903,4.85714);
\draw [c] (2.77642,4.89166) -- (2.77642,4.85714);
\draw [c] (2.94381,4.89166) -- (2.94381,4.85714);
\draw [c] (3.11119,4.89166) -- (3.11119,4.85714);
\draw [c] (3.27858,4.89166) -- (3.27858,4.85714);
\draw [c] (3.44597,4.92619) -- (3.44597,4.85714);
\draw [c] (3.61336,4.89166) -- (3.61336,4.85714);
\draw [c] (3.78075,4.89166) -- (3.78075,4.85714);
\draw [c] (3.94814,4.89166) -- (3.94814,4.85714);
\draw [c] (4.11553,4.89166) -- (4.11553,4.85714);
\draw [c] (4.28292,4.92619) -- (4.28292,4.85714);
\draw [c] (4.45031,4.89166) -- (4.45031,4.85714);
\draw [c] (4.6177,4.89166) -- (4.6177,4.85714);
\draw [c] (4.78509,4.89166) -- (4.78509,4.85714);
\draw [c] (4.95248,4.89166) -- (4.95248,4.85714);
\draw [c] (5.11986,4.92619) -- (5.11986,4.85714);
\draw [c] (5.28725,4.89166) -- (5.28725,4.85714);
\draw [c] (5.45464,4.89166) -- (5.45464,4.85714);
\draw [c] (5.62203,4.89166) -- (5.62203,4.85714);
\draw [c] (5.78942,4.89166) -- (5.78942,4.85714);
\draw [c] (5.95681,4.92619) -- (5.95681,4.85714);
\draw [c] (6.1242,4.89166) -- (6.1242,4.85714);
\draw [c] (6.29159,4.89166) -- (6.29159,4.85714);
\draw [c] (6.45898,4.89166) -- (6.45898,4.85714);
\draw [c] (6.62637,4.89166) -- (6.62637,4.85714);
\draw [c] (6.79376,4.92619) -- (6.79376,4.85714);
\draw [c] (6.96115,4.89166) -- (6.96115,4.85714);
\draw [c] (7.12854,4.89166) -- (7.12854,4.85714);
\draw [c] (7.29592,4.89166) -- (7.29592,4.85714);
\draw [c] (7.46331,4.89166) -- (7.46331,4.85714);
\draw [c] (7.6307,4.92619) -- (7.6307,4.85714);
\draw [c] (7.79809,4.89166) -- (7.79809,4.85714);
\draw [c] (7.96548,4.89166) -- (7.96548,4.85714);
\draw [c] (8.13287,4.89166) -- (8.13287,4.85714);
\draw [c] (8.30026,4.89166) -- (8.30026,4.85714);
\draw [c] (8.46765,4.92619) -- (8.46765,4.85714);
\draw [c] (8.63504,4.89166) -- (8.63504,4.85714);
\draw [c] (8.80243,4.89166) -- (8.80243,4.85714);
\draw [c] (8.96982,4.89166) -- (8.96982,4.85714);
\draw [c] (9.13721,4.89166) -- (9.13721,4.85714);
\draw [c] (9.30459,4.99523) -- (9.30459,4.85714);
\draw [anchor=base] (0.935135,4.61029) node[color=c, rotate=0]{0};
\draw [anchor=base] (9.30459,4.61029) node[color=c, rotate=0]{5000};
\draw [c] (0.935135,4.85714) -- (0.935135,9.94857);
\draw [anchor= east] (-0.112216,9.94857) node[color=c, rotate=90]{$\text{d}\sigma/\text{d}(M_{\gamma\gamma})$ [pb/GeV]};
\draw [c] (1.074,4.86953) -- (0.935135,4.86953);
\draw [c] (1.074,4.90484) -- (0.935135,4.90484);
\draw [c] (1.074,4.93542) -- (0.935135,4.93542);
\draw [c] (1.074,4.9624) -- (0.935135,4.9624);
\draw [c] (1.21287,4.98653) -- (0.935135,4.98653);
\draw [anchor= east] (0.862195,4.98653) node[color=c, rotate=0]{$10^{-10}$};
\draw [c] (1.074,5.14528) -- (0.935135,5.14528);
\draw [c] (1.074,5.23814) -- (0.935135,5.23814);
\draw [c] (1.074,5.30403) -- (0.935135,5.30403);
\draw [c] (1.074,5.35513) -- (0.935135,5.35513);
\draw [c] (1.074,5.39689) -- (0.935135,5.39689);
\draw [c] (1.074,5.4322) -- (0.935135,5.4322);
\draw [c] (1.074,5.46278) -- (0.935135,5.46278);
\draw [c] (1.074,5.48975) -- (0.935135,5.48975);
\draw [c] (1.21287,5.51388) -- (0.935135,5.51388);
\draw [anchor= east] (0.862195,5.51388) node[color=c, rotate=0]{$10^{-9}$};
\draw [c] (1.074,5.67264) -- (0.935135,5.67264);
\draw [c] (1.074,5.7655) -- (0.935135,5.7655);
\draw [c] (1.074,5.83139) -- (0.935135,5.83139);
\draw [c] (1.074,5.88249) -- (0.935135,5.88249);
\draw [c] (1.074,5.92425) -- (0.935135,5.92425);
\draw [c] (1.074,5.95955) -- (0.935135,5.95955);
\draw [c] (1.074,5.99014) -- (0.935135,5.99014);
\draw [c] (1.074,6.01711) -- (0.935135,6.01711);
\draw [c] (1.21287,6.04124) -- (0.935135,6.04124);
\draw [anchor= east] (0.862195,6.04124) node[color=c, rotate=0]{$10^{-8}$};
\draw [c] (1.074,6.19999) -- (0.935135,6.19999);
\draw [c] (1.074,6.29286) -- (0.935135,6.29286);
\draw [c] (1.074,6.35874) -- (0.935135,6.35874);
\draw [c] (1.074,6.40985) -- (0.935135,6.40985);
\draw [c] (1.074,6.45161) -- (0.935135,6.45161);
\draw [c] (1.074,6.48691) -- (0.935135,6.48691);
\draw [c] (1.074,6.51749) -- (0.935135,6.51749);
\draw [c] (1.074,6.54447) -- (0.935135,6.54447);
\draw [c] (1.21287,6.5686) -- (0.935135,6.5686);
\draw [anchor= east] (0.862195,6.5686) node[color=c, rotate=0]{$10^{-7}$};
\draw [c] (1.074,6.72735) -- (0.935135,6.72735);
\draw [c] (1.074,6.82021) -- (0.935135,6.82021);
\draw [c] (1.074,6.8861) -- (0.935135,6.8861);
\draw [c] (1.074,6.93721) -- (0.935135,6.93721);
\draw [c] (1.074,6.97897) -- (0.935135,6.97897);
\draw [c] (1.074,7.01427) -- (0.935135,7.01427);
\draw [c] (1.074,7.04485) -- (0.935135,7.04485);
\draw [c] (1.074,7.07183) -- (0.935135,7.07183);
\draw [c] (1.21287,7.09596) -- (0.935135,7.09596);
\draw [anchor= east] (0.862195,7.09596) node[color=c, rotate=0]{$10^{-6}$};
\draw [c] (1.074,7.25471) -- (0.935135,7.25471);
\draw [c] (1.074,7.34757) -- (0.935135,7.34757);
\draw [c] (1.074,7.41346) -- (0.935135,7.41346);
\draw [c] (1.074,7.46457) -- (0.935135,7.46457);
\draw [c] (1.074,7.50632) -- (0.935135,7.50632);
\draw [c] (1.074,7.54163) -- (0.935135,7.54163);
\draw [c] (1.074,7.57221) -- (0.935135,7.57221);
\draw [c] (1.074,7.59919) -- (0.935135,7.59919);
\draw [c] (1.21287,7.62332) -- (0.935135,7.62332);
\draw [anchor= east] (0.862195,7.62332) node[color=c, rotate=0]{$10^{-5}$};
\draw [c] (1.074,7.78207) -- (0.935135,7.78207);
\draw [c] (1.074,7.87493) -- (0.935135,7.87493);
\draw [c] (1.074,7.94082) -- (0.935135,7.94082);
\draw [c] (1.074,7.99192) -- (0.935135,7.99192);
\draw [c] (1.074,8.03368) -- (0.935135,8.03368);
\draw [c] (1.074,8.06899) -- (0.935135,8.06899);
\draw [c] (1.074,8.09957) -- (0.935135,8.09957);
\draw [c] (1.074,8.12654) -- (0.935135,8.12654);
\draw [c] (1.21287,8.15068) -- (0.935135,8.15068);
\draw [anchor= east] (0.862195,8.15068) node[color=c, rotate=0]{$10^{-4}$};
\draw [c] (1.074,8.30943) -- (0.935135,8.30943);
\draw [c] (1.074,8.40229) -- (0.935135,8.40229);
\draw [c] (1.074,8.46818) -- (0.935135,8.46818);
\draw [c] (1.074,8.51928) -- (0.935135,8.51928);
\draw [c] (1.074,8.56104) -- (0.935135,8.56104);
\draw [c] (1.074,8.59634) -- (0.935135,8.59634);
\draw [c] (1.074,8.62693) -- (0.935135,8.62693);
\draw [c] (1.074,8.6539) -- (0.935135,8.6539);
\draw [c] (1.21287,8.67803) -- (0.935135,8.67803);
\draw [anchor= east] (0.862195,8.67803) node[color=c, rotate=0]{$10^{-3}$};
\draw [c] (1.074,8.83678) -- (0.935135,8.83678);
\draw [c] (1.074,8.92965) -- (0.935135,8.92965);
\draw [c] (1.074,8.99553) -- (0.935135,8.99553);
\draw [c] (1.074,9.04664) -- (0.935135,9.04664);
\draw [c] (1.074,9.0884) -- (0.935135,9.0884);
\draw [c] (1.074,9.1237) -- (0.935135,9.1237);
\draw [c] (1.074,9.15429) -- (0.935135,9.15429);
\draw [c] (1.074,9.18126) -- (0.935135,9.18126);
\draw [c] (1.21287,9.20539) -- (0.935135,9.20539);
\draw [anchor= east] (0.862195,9.20539) node[color=c, rotate=0]{$10^{-2}$};
\draw [c] (1.074,9.36414) -- (0.935135,9.36414);
\draw [c] (1.074,9.45701) -- (0.935135,9.45701);
\draw [c] (1.074,9.52289) -- (0.935135,9.52289);
\draw [c] (1.074,9.574) -- (0.935135,9.574);
\draw [c] (1.074,9.61576) -- (0.935135,9.61576);
\draw [c] (1.074,9.65106) -- (0.935135,9.65106);
\draw [c] (1.074,9.68164) -- (0.935135,9.68164);
\draw [c] (1.074,9.70862) -- (0.935135,9.70862);
\draw [c] (1.21287,9.73275) -- (0.935135,9.73275);
\draw [anchor= east] (0.862195,9.73275) node[color=c, rotate=0]{$10^{-1}$};
\draw [c] (1.074,9.8915) -- (0.935135,9.8915);
\colorlet{c}{black};
\draw [c] (1.01883,9.82315) -- (1.01883,9.82375);
\draw [c] (1.01883,9.82375) -- (1.01883,9.82435);
\draw [c] (0.935135,9.82375) -- (1.01883,9.82375);
\draw [c] (1.01883,9.82375) -- (1.10252,9.82375);
\definecolor{c}{rgb}{0,0,0};
\colorlet{c}{black};
\draw [c] (1.18622,9.41949) -- (1.18622,9.42094);
\draw [c] (1.18622,9.42094) -- (1.18622,9.42238);
\draw [c] (1.10252,9.42094) -- (1.18622,9.42094);
\draw [c] (1.18622,9.42094) -- (1.26991,9.42094);
\definecolor{c}{rgb}{0,0,0};
\colorlet{c}{black};
\draw [c] (1.35361,8.93962) -- (1.35361,8.94374);
\draw [c] (1.35361,8.94374) -- (1.35361,8.94779);
\draw [c] (1.26991,8.94374) -- (1.35361,8.94374);
\draw [c] (1.35361,8.94374) -- (1.4373,8.94374);
\definecolor{c}{rgb}{0,0,0};
\colorlet{c}{black};
\draw [c] (1.521,8.61807) -- (1.521,8.62639);
\draw [c] (1.521,8.62639) -- (1.521,8.63441);
\draw [c] (1.4373,8.62639) -- (1.521,8.62639);
\draw [c] (1.521,8.62639) -- (1.60469,8.62639);
\definecolor{c}{rgb}{0,0,0};
\colorlet{c}{black};
\draw [c] (1.68839,8.38871) -- (1.68839,8.38951);
\draw [c] (1.68839,8.38951) -- (1.68839,8.39032);
\draw [c] (1.60469,8.38951) -- (1.68839,8.38951);
\draw [c] (1.68839,8.38951) -- (1.77208,8.38951);
\definecolor{c}{rgb}{0,0,0};
\colorlet{c}{black};
\draw [c] (1.85578,8.18841) -- (1.85578,8.18965);
\draw [c] (1.85578,8.18965) -- (1.85578,8.19089);
\draw [c] (1.77208,8.18965) -- (1.85578,8.18965);
\draw [c] (1.85578,8.18965) -- (1.93947,8.18965);
\definecolor{c}{rgb}{0,0,0};
\colorlet{c}{black};
\draw [c] (2.02316,8.01568) -- (2.02316,8.01749);
\draw [c] (2.02316,8.01749) -- (2.02316,8.01929);
\draw [c] (1.93947,8.01749) -- (2.02316,8.01749);
\draw [c] (2.02316,8.01749) -- (2.10686,8.01749);
\definecolor{c}{rgb}{0,0,0};
\colorlet{c}{black};
\draw [c] (2.19055,7.85701) -- (2.19055,7.85958);
\draw [c] (2.19055,7.85958) -- (2.19055,7.86211);
\draw [c] (2.10686,7.85958) -- (2.19055,7.85958);
\draw [c] (2.19055,7.85958) -- (2.27425,7.85958);
\definecolor{c}{rgb}{0,0,0};
\colorlet{c}{black};
\draw [c] (2.35794,7.72161) -- (2.35794,7.72505);
\draw [c] (2.35794,7.72505) -- (2.35794,7.72845);
\draw [c] (2.27425,7.72505) -- (2.35794,7.72505);
\draw [c] (2.35794,7.72505) -- (2.44164,7.72505);
\definecolor{c}{rgb}{0,0,0};
\colorlet{c}{black};
\draw [c] (2.52533,7.58725) -- (2.52533,7.59187);
\draw [c] (2.52533,7.59187) -- (2.52533,7.59639);
\draw [c] (2.44164,7.59187) -- (2.52533,7.59187);
\draw [c] (2.52533,7.59187) -- (2.60903,7.59187);
\definecolor{c}{rgb}{0,0,0};
\colorlet{c}{black};
\draw [c] (2.69272,7.46642) -- (2.69272,7.46744);
\draw [c] (2.69272,7.46744) -- (2.69272,7.46846);
\draw [c] (2.60903,7.46744) -- (2.69272,7.46744);
\draw [c] (2.69272,7.46744) -- (2.77642,7.46744);
\definecolor{c}{rgb}{0,0,0};
\colorlet{c}{black};
\draw [c] (2.86011,7.35174) -- (2.86011,7.35305);
\draw [c] (2.86011,7.35305) -- (2.86011,7.35436);
\draw [c] (2.77642,7.35305) -- (2.86011,7.35305);
\draw [c] (2.86011,7.35305) -- (2.94381,7.35305);
\definecolor{c}{rgb}{0,0,0};
\colorlet{c}{black};
\draw [c] (3.0275,7.23472) -- (3.0275,7.23642);
\draw [c] (3.0275,7.23642) -- (3.0275,7.2381);
\draw [c] (2.94381,7.23642) -- (3.0275,7.23642);
\draw [c] (3.0275,7.23642) -- (3.11119,7.23642);
\definecolor{c}{rgb}{0,0,0};
\colorlet{c}{black};
\draw [c] (3.19489,7.13006) -- (3.19489,7.13219);
\draw [c] (3.19489,7.13219) -- (3.19489,7.13431);
\draw [c] (3.11119,7.13219) -- (3.19489,7.13219);
\draw [c] (3.19489,7.13219) -- (3.27858,7.13219);
\definecolor{c}{rgb}{0,0,0};
\colorlet{c}{black};
\draw [c] (3.36228,7.03107) -- (3.36228,7.03372);
\draw [c] (3.36228,7.03372) -- (3.36228,7.03634);
\draw [c] (3.27858,7.03372) -- (3.36228,7.03372);
\draw [c] (3.36228,7.03372) -- (3.44597,7.03372);
\definecolor{c}{rgb}{0,0,0};
\colorlet{c}{black};
\draw [c] (3.52967,6.9355) -- (3.52967,6.93877);
\draw [c] (3.52967,6.93877) -- (3.52967,6.94199);
\draw [c] (3.44597,6.93877) -- (3.52967,6.93877);
\draw [c] (3.52967,6.93877) -- (3.61336,6.93877);
\definecolor{c}{rgb}{0,0,0};
\colorlet{c}{black};
\draw [c] (3.69706,6.83348) -- (3.69706,6.83756);
\draw [c] (3.69706,6.83756) -- (3.69706,6.84157);
\draw [c] (3.61336,6.83756) -- (3.69706,6.83756);
\draw [c] (3.69706,6.83756) -- (3.78075,6.83756);
\definecolor{c}{rgb}{0,0,0};
\colorlet{c}{black};
\draw [c] (3.86445,6.73421) -- (3.86445,6.73928);
\draw [c] (3.86445,6.73928) -- (3.86445,6.74424);
\draw [c] (3.78075,6.73928) -- (3.86445,6.73928);
\draw [c] (3.86445,6.73928) -- (3.94814,6.73928);
\definecolor{c}{rgb}{0,0,0};
\colorlet{c}{black};
\draw [c] (4.03184,6.64983) -- (4.03184,6.65592);
\draw [c] (4.03184,6.65592) -- (4.03184,6.66186);
\draw [c] (3.94814,6.65592) -- (4.03184,6.65592);
\draw [c] (4.03184,6.65592) -- (4.11553,6.65592);
\definecolor{c}{rgb}{0,0,0};
\colorlet{c}{black};
\draw [c] (4.19922,6.51335) -- (4.19922,6.52155);
\draw [c] (4.19922,6.52155) -- (4.19922,6.52948);
\draw [c] (4.11553,6.52155) -- (4.19922,6.52155);
\draw [c] (4.19922,6.52155) -- (4.28292,6.52155);
\definecolor{c}{rgb}{0,0,0};
\colorlet{c}{black};
\draw [c] (4.36661,6.40927) -- (4.36661,6.41957);
\draw [c] (4.36661,6.41957) -- (4.36661,6.42943);
\draw [c] (4.28292,6.41957) -- (4.36661,6.41957);
\draw [c] (4.36661,6.41957) -- (4.45031,6.41957);
\definecolor{c}{rgb}{0,0,0};
\colorlet{c}{black};
\draw [c] (4.534,6.32428) -- (4.534,6.33668);
\draw [c] (4.534,6.33668) -- (4.534,6.34844);
\draw [c] (4.45031,6.33668) -- (4.534,6.33668);
\draw [c] (4.534,6.33668) -- (4.6177,6.33668);
\definecolor{c}{rgb}{0,0,0};
\colorlet{c}{black};
\draw [c] (4.70139,6.25109) -- (4.70139,6.26564);
\draw [c] (4.70139,6.26564) -- (4.70139,6.27932);
\draw [c] (4.6177,6.26564) -- (4.70139,6.26564);
\draw [c] (4.70139,6.26564) -- (4.78509,6.26564);
\definecolor{c}{rgb}{0,0,0};
\colorlet{c}{black};
\draw [c] (4.86878,6.18997) -- (4.86878,6.20659);
\draw [c] (4.86878,6.20659) -- (4.86878,6.22209);
\draw [c] (4.78509,6.20659) -- (4.86878,6.20659);
\draw [c] (4.86878,6.20659) -- (4.95248,6.20659);
\definecolor{c}{rgb}{0,0,0};
\colorlet{c}{black};
\draw [c] (5.03617,6.09495) -- (5.03617,6.11541);
\draw [c] (5.03617,6.11541) -- (5.03617,6.13418);
\draw [c] (4.95248,6.11541) -- (5.03617,6.11541);
\draw [c] (5.03617,6.11541) -- (5.11986,6.11541);
\definecolor{c}{rgb}{0,0,0};
\colorlet{c}{black};
\draw [c] (5.20356,5.99366) -- (5.20356,6.01918);
\draw [c] (5.20356,6.01918) -- (5.20356,6.04213);
\draw [c] (5.11986,6.01918) -- (5.20356,6.01918);
\draw [c] (5.20356,6.01918) -- (5.28725,6.01918);
\definecolor{c}{rgb}{0,0,0};
\colorlet{c}{black};
\draw [c] (5.37095,5.94604) -- (5.37095,5.97434);
\draw [c] (5.37095,5.97434) -- (5.37095,5.99953);
\draw [c] (5.28725,5.97434) -- (5.37095,5.97434);
\draw [c] (5.37095,5.97434) -- (5.45464,5.97434);
\definecolor{c}{rgb}{0,0,0};
\colorlet{c}{black};
\draw [c] (5.53834,5.78116) -- (5.53834,5.8217);
\draw [c] (5.53834,5.8217) -- (5.53834,5.85613);
\draw [c] (5.45464,5.8217) -- (5.53834,5.8217);
\draw [c] (5.53834,5.8217) -- (5.62203,5.8217);
\definecolor{c}{rgb}{0,0,0};
\colorlet{c}{black};
\draw [c] (5.70573,5.78768) -- (5.70573,5.82765);
\draw [c] (5.70573,5.82765) -- (5.70573,5.86167);
\draw [c] (5.62203,5.82765) -- (5.70573,5.82765);
\draw [c] (5.70573,5.82765) -- (5.78942,5.82765);
\definecolor{c}{rgb}{0,0,0};
\colorlet{c}{black};
\draw [c] (5.87312,5.70379) -- (5.87312,5.75176);
\draw [c] (5.87312,5.75176) -- (5.87312,5.79141);
\draw [c] (5.78942,5.75176) -- (5.87312,5.75176);
\draw [c] (5.87312,5.75176) -- (5.95681,5.75176);
\definecolor{c}{rgb}{0,0,0};
\colorlet{c}{black};
\draw [c] (6.04051,5.61673) -- (6.04051,5.6747);
\draw [c] (6.04051,5.6747) -- (6.04051,5.72092);
\draw [c] (5.95681,5.6747) -- (6.04051,5.6747);
\draw [c] (6.04051,5.6747) -- (6.1242,5.6747);
\definecolor{c}{rgb}{0,0,0};
\colorlet{c}{black};
\draw [c] (6.20789,5.52179) -- (6.20789,5.59301);
\draw [c] (6.20789,5.59301) -- (6.20789,5.64726);
\draw [c] (6.1242,5.59301) -- (6.20789,5.59301);
\draw [c] (6.20789,5.59301) -- (6.29159,5.59301);
\definecolor{c}{rgb}{0,0,0};
\colorlet{c}{black};
\draw [c] (6.37528,5.32553) -- (6.37528,5.43426);
\draw [c] (6.37528,5.43426) -- (6.37528,5.50769);
\draw [c] (6.29159,5.43426) -- (6.37528,5.43426);
\draw [c] (6.37528,5.43426) -- (6.45898,5.43426);
\definecolor{c}{rgb}{0,0,0};
\colorlet{c}{black};
\draw [c] (6.54267,5.14734) -- (6.54267,5.30609);
\draw [c] (6.54267,5.30609) -- (6.54267,5.39895);
\draw [c] (6.45898,5.30609) -- (6.54267,5.30609);
\draw [c] (6.54267,5.30609) -- (6.62637,5.30609);
\definecolor{c}{rgb}{0,0,0};
\colorlet{c}{black};
\draw [c] (6.71006,5.14734) -- (6.71006,5.30609);
\draw [c] (6.71006,5.30609) -- (6.71006,5.39895);
\draw [c] (6.62637,5.30609) -- (6.71006,5.30609);
\draw [c] (6.71006,5.30609) -- (6.79376,5.30609);
\definecolor{c}{rgb}{0,0,0};
\colorlet{c}{black};
\draw [c] (6.87745,4.86611) -- (6.87745,5.14734);
\draw [c] (6.87745,5.14734) -- (6.87745,5.26983);
\draw [c] (6.79376,5.14734) -- (6.87745,5.14734);
\draw [c] (6.87745,5.14734) -- (6.96115,5.14734);
\definecolor{c}{rgb}{0,0,0};
\colorlet{c}{black};
\draw [c] (7.04484,4.85714) -- (7.04484,4.98859);
\draw [c] (7.04484,4.98859) -- (7.04484,5.14734);
\draw [c] (6.96115,4.98859) -- (7.04484,4.98859);
\draw [c] (7.04484,4.98859) -- (7.12854,4.98859);
\definecolor{c}{rgb}{0,0,0};
\colorlet{c}{black};
\draw [c] (7.37962,4.85714) -- (7.37962,4.98859);
\draw [c] (7.37962,4.98859) -- (7.37962,5.14734);
\draw [c] (7.29592,4.98859) -- (7.37962,4.98859);
\draw [c] (7.37962,4.98859) -- (7.46331,4.98859);
\definecolor{c}{rgb}{0,0,0};
\colorlet{c}{black};
\draw [c] (7.54701,5.04296) -- (7.54701,5.2402);
\draw [c] (7.54701,5.2402) -- (7.54701,5.34458);
\draw [c] (7.46331,5.2402) -- (7.54701,5.2402);
\draw [c] (7.54701,5.2402) -- (7.6307,5.2402);
\definecolor{c}{rgb}{0,0,0};
\colorlet{c}{natgreen!50};
\draw [c] (1.01883,9.79578) -- (1.01883,9.79844);
\draw [c] (1.01883,9.79844) -- (1.01883,9.80106);
\draw [c] (0.935135,9.79844) -- (1.01883,9.79844);
\draw [c] (1.01883,9.79844) -- (1.10252,9.79844);
\definecolor{c}{rgb}{0,0,0};
\colorlet{c}{natgreen!50};
\draw [c] (1.18622,9.38581) -- (1.18622,9.38678);
\draw [c] (1.18622,9.38678) -- (1.18622,9.38774);
\draw [c] (1.10252,9.38678) -- (1.18622,9.38678);
\draw [c] (1.18622,9.38678) -- (1.26991,9.38678);
\definecolor{c}{rgb}{0,0,0};
\colorlet{c}{natgreen!50};
\draw [c] (1.35361,8.90402) -- (1.35361,8.90679);
\draw [c] (1.35361,8.90679) -- (1.35361,8.90952);
\draw [c] (1.26991,8.90679) -- (1.35361,8.90679);
\draw [c] (1.35361,8.90679) -- (1.4373,8.90679);
\definecolor{c}{rgb}{0,0,0};
\colorlet{c}{natgreen!50};
\draw [c] (1.521,8.57296) -- (1.521,8.57866);
\draw [c] (1.521,8.57866) -- (1.521,8.58422);
\draw [c] (1.4373,8.57866) -- (1.521,8.57866);
\draw [c] (1.521,8.57866) -- (1.60469,8.57866);
\definecolor{c}{rgb}{0,0,0};
\colorlet{c}{natgreen!50};
\draw [c] (1.68839,8.33387) -- (1.68839,8.34348);
\draw [c] (1.68839,8.34348) -- (1.68839,8.35271);
\draw [c] (1.60469,8.34348) -- (1.68839,8.34348);
\draw [c] (1.68839,8.34348) -- (1.77208,8.34348);
\definecolor{c}{rgb}{0,0,0};
\colorlet{c}{natgreen!50};
\draw [c] (1.85578,8.1562) -- (1.85578,8.17036);
\draw [c] (1.85578,8.17036) -- (1.85578,8.1837);
\draw [c] (1.77208,8.17036) -- (1.85578,8.17036);
\draw [c] (1.85578,8.17036) -- (1.93947,8.17036);
\definecolor{c}{rgb}{0,0,0};
\colorlet{c}{natgreen!50};
\draw [c] (2.02316,7.96584) -- (2.02316,7.9873);
\draw [c] (2.02316,7.9873) -- (2.02316,8.00692);
\draw [c] (1.93947,7.9873) -- (2.02316,7.9873);
\draw [c] (2.02316,7.9873) -- (2.10686,7.9873);
\definecolor{c}{rgb}{0,0,0};
\colorlet{c}{natgreen!50};
\draw [c] (2.19055,7.84862) -- (2.19055,7.87633);
\draw [c] (2.19055,7.87633) -- (2.19055,7.90105);
\draw [c] (2.10686,7.87633) -- (2.19055,7.87633);
\draw [c] (2.19055,7.87633) -- (2.27425,7.87633);
\definecolor{c}{rgb}{0,0,0};
\colorlet{c}{natgreen!50};
\draw [c] (2.35794,7.705) -- (2.35794,7.7429);
\draw [c] (2.35794,7.7429) -- (2.35794,7.77541);
\draw [c] (2.27425,7.7429) -- (2.35794,7.7429);
\draw [c] (2.35794,7.7429) -- (2.44164,7.7429);
\definecolor{c}{rgb}{0,0,0};
\colorlet{c}{natgreen!50};
\draw [c] (2.52533,7.56758) -- (2.52533,7.61869);
\draw [c] (2.52533,7.61869) -- (2.52533,7.66045);
\draw [c] (2.44164,7.61869) -- (2.52533,7.61869);
\draw [c] (2.52533,7.61869) -- (2.60903,7.61869);
\definecolor{c}{rgb}{0,0,0};
\colorlet{c}{natgreen!50};
\draw [c] (2.69272,7.5225) -- (2.69272,7.52434);
\draw [c] (2.69272,7.52434) -- (2.69272,7.52615);
\draw [c] (2.60903,7.52434) -- (2.69272,7.52434);
\draw [c] (2.69272,7.52434) -- (2.77642,7.52434);
\definecolor{c}{rgb}{0,0,0};
\colorlet{c}{natgreen!50};
\draw [c] (2.86011,7.43828) -- (2.86011,7.44049);
\draw [c] (2.86011,7.44049) -- (2.86011,7.44267);
\draw [c] (2.77642,7.44049) -- (2.86011,7.44049);
\draw [c] (2.86011,7.44049) -- (2.94381,7.44049);
\definecolor{c}{rgb}{0,0,0};
\colorlet{c}{natgreen!50};
\draw [c] (3.0275,7.36328) -- (3.0275,7.36588);
\draw [c] (3.0275,7.36588) -- (3.0275,7.36844);
\draw [c] (2.94381,7.36588) -- (3.0275,7.36588);
\draw [c] (3.0275,7.36588) -- (3.11119,7.36588);
\definecolor{c}{rgb}{0,0,0};
\colorlet{c}{natgreen!50};
\draw [c] (3.19489,7.30364) -- (3.19489,7.3066);
\draw [c] (3.19489,7.3066) -- (3.19489,7.30951);
\draw [c] (3.11119,7.3066) -- (3.19489,7.3066);
\draw [c] (3.19489,7.3066) -- (3.27858,7.3066);
\definecolor{c}{rgb}{0,0,0};
\colorlet{c}{natgreen!50};
\draw [c] (3.36228,7.25327) -- (3.36228,7.25657);
\draw [c] (3.36228,7.25657) -- (3.36228,7.25982);
\draw [c] (3.27858,7.25657) -- (3.36228,7.25657);
\draw [c] (3.36228,7.25657) -- (3.44597,7.25657);
\definecolor{c}{rgb}{0,0,0};
\colorlet{c}{natgreen!50};
\draw [c] (3.52967,7.20857) -- (3.52967,7.21221);
\draw [c] (3.52967,7.21221) -- (3.52967,7.21579);
\draw [c] (3.44597,7.21221) -- (3.52967,7.21221);
\draw [c] (3.52967,7.21221) -- (3.61336,7.21221);
\definecolor{c}{rgb}{0,0,0};
\colorlet{c}{natgreen!50};
\draw [c] (3.69706,7.16287) -- (3.69706,7.16689);
\draw [c] (3.69706,7.16689) -- (3.69706,7.17084);
\draw [c] (3.61336,7.16689) -- (3.69706,7.16689);
\draw [c] (3.69706,7.16689) -- (3.78075,7.16689);
\definecolor{c}{rgb}{0,0,0};
\colorlet{c}{natgreen!50};
\draw [c] (3.86445,7.13805) -- (3.86445,7.14229);
\draw [c] (3.86445,7.14229) -- (3.86445,7.14645);
\draw [c] (3.78075,7.14229) -- (3.86445,7.14229);
\draw [c] (3.86445,7.14229) -- (3.94814,7.14229);
\definecolor{c}{rgb}{0,0,0};
\colorlet{c}{natgreen!50};
\draw [c] (4.03184,7.09962) -- (4.03184,7.10423);
\draw [c] (4.03184,7.10423) -- (4.03184,7.10875);
\draw [c] (3.94814,7.10423) -- (4.03184,7.10423);
\draw [c] (4.03184,7.10423) -- (4.11553,7.10423);
\definecolor{c}{rgb}{0,0,0};
\colorlet{c}{natgreen!50};
\draw [c] (4.19922,7.08285) -- (4.19922,7.08763);
\draw [c] (4.19922,7.08763) -- (4.19922,7.09232);
\draw [c] (4.11553,7.08763) -- (4.19922,7.08763);
\draw [c] (4.19922,7.08763) -- (4.28292,7.08763);
\definecolor{c}{rgb}{0,0,0};
\colorlet{c}{natgreen!50};
\draw [c] (4.36661,7.05956) -- (4.36661,7.0646);
\draw [c] (4.36661,7.0646) -- (4.36661,7.06952);
\draw [c] (4.28292,7.0646) -- (4.36661,7.0646);
\draw [c] (4.36661,7.0646) -- (4.45031,7.0646);
\definecolor{c}{rgb}{0,0,0};
\colorlet{c}{natgreen!50};
\draw [c] (4.534,7.04029) -- (4.534,7.04554);
\draw [c] (4.534,7.04554) -- (4.534,7.05067);
\draw [c] (4.45031,7.04554) -- (4.534,7.04554);
\draw [c] (4.534,7.04554) -- (4.6177,7.04554);
\definecolor{c}{rgb}{0,0,0};
\colorlet{c}{natgreen!50};
\draw [c] (4.70139,7.01848) -- (4.70139,7.02398);
\draw [c] (4.70139,7.02398) -- (4.70139,7.02936);
\draw [c] (4.6177,7.02398) -- (4.70139,7.02398);
\draw [c] (4.70139,7.02398) -- (4.78509,7.02398);
\definecolor{c}{rgb}{0,0,0};
\colorlet{c}{natgreen!50};
\draw [c] (4.86878,7.00391) -- (4.86878,7.00959);
\draw [c] (4.86878,7.00959) -- (4.86878,7.01513);
\draw [c] (4.78509,7.00959) -- (4.86878,7.00959);
\draw [c] (4.86878,7.00959) -- (4.95248,7.00959);
\definecolor{c}{rgb}{0,0,0};
\colorlet{c}{natgreen!50};
\draw [c] (5.03617,6.97674) -- (5.03617,6.98277);
\draw [c] (5.03617,6.98277) -- (5.03617,6.98864);
\draw [c] (4.95248,6.98277) -- (5.03617,6.98277);
\draw [c] (5.03617,6.98277) -- (5.11986,6.98277);
\definecolor{c}{rgb}{0,0,0};
\colorlet{c}{natgreen!50};
\draw [c] (5.20356,6.95489) -- (5.20356,6.96122);
\draw [c] (5.20356,6.96122) -- (5.20356,6.96737);
\draw [c] (5.11986,6.96122) -- (5.20356,6.96122);
\draw [c] (5.20356,6.96122) -- (5.28725,6.96122);
\definecolor{c}{rgb}{0,0,0};
\colorlet{c}{natgreen!50};
\draw [c] (5.37095,6.93717) -- (5.37095,6.94375);
\draw [c] (5.37095,6.94375) -- (5.37095,6.95014);
\draw [c] (5.28725,6.94375) -- (5.37095,6.94375);
\draw [c] (5.37095,6.94375) -- (5.45464,6.94375);
\definecolor{c}{rgb}{0,0,0};
\colorlet{c}{natgreen!50};
\draw [c] (5.53834,6.91369) -- (5.53834,6.92061);
\draw [c] (5.53834,6.92061) -- (5.53834,6.92733);
\draw [c] (5.45464,6.92061) -- (5.53834,6.92061);
\draw [c] (5.53834,6.92061) -- (5.62203,6.92061);
\definecolor{c}{rgb}{0,0,0};
\colorlet{c}{natgreen!50};
\draw [c] (5.70573,6.89637) -- (5.70573,6.90356);
\draw [c] (5.70573,6.90356) -- (5.70573,6.91053);
\draw [c] (5.62203,6.90356) -- (5.70573,6.90356);
\draw [c] (5.70573,6.90356) -- (5.78942,6.90356);
\definecolor{c}{rgb}{0,0,0};
\colorlet{c}{natgreen!50};
\draw [c] (5.87312,6.86145) -- (5.87312,6.8692);
\draw [c] (5.87312,6.8692) -- (5.87312,6.8767);
\draw [c] (5.78942,6.8692) -- (5.87312,6.8692);
\draw [c] (5.87312,6.8692) -- (5.95681,6.8692);
\definecolor{c}{rgb}{0,0,0};
\colorlet{c}{natgreen!50};
\draw [c] (6.04051,6.84457) -- (6.04051,6.85262);
\draw [c] (6.04051,6.85262) -- (6.04051,6.86039);
\draw [c] (5.95681,6.85262) -- (6.04051,6.85262);
\draw [c] (6.04051,6.85262) -- (6.1242,6.85262);
\definecolor{c}{rgb}{0,0,0};
\colorlet{c}{natgreen!50};
\draw [c] (6.20789,6.81935) -- (6.20789,6.82785);
\draw [c] (6.20789,6.82785) -- (6.20789,6.83605);
\draw [c] (6.1242,6.82785) -- (6.20789,6.82785);
\draw [c] (6.20789,6.82785) -- (6.29159,6.82785);
\definecolor{c}{rgb}{0,0,0};
\colorlet{c}{natgreen!50};
\draw [c] (6.37528,6.78714) -- (6.37528,6.79627);
\draw [c] (6.37528,6.79627) -- (6.37528,6.80504);
\draw [c] (6.29159,6.79627) -- (6.37528,6.79627);
\draw [c] (6.37528,6.79627) -- (6.45898,6.79627);
\definecolor{c}{rgb}{0,0,0};
\colorlet{c}{natgreen!50};
\draw [c] (6.54267,6.7732) -- (6.54267,6.7826);
\draw [c] (6.54267,6.7826) -- (6.54267,6.79163);
\draw [c] (6.45898,6.7826) -- (6.54267,6.7826);
\draw [c] (6.54267,6.7826) -- (6.62637,6.7826);
\definecolor{c}{rgb}{0,0,0};
\colorlet{c}{natgreen!50};
\draw [c] (6.71006,6.71058) -- (6.71006,6.72136);
\draw [c] (6.71006,6.72136) -- (6.71006,6.73165);
\draw [c] (6.62637,6.72136) -- (6.71006,6.72136);
\draw [c] (6.71006,6.72136) -- (6.79376,6.72136);
\definecolor{c}{rgb}{0,0,0};
\colorlet{c}{natgreen!50};
\draw [c] (6.87745,6.68821) -- (6.87745,6.69953);
\draw [c] (6.87745,6.69953) -- (6.87745,6.71032);
\draw [c] (6.79376,6.69953) -- (6.87745,6.69953);
\draw [c] (6.87745,6.69953) -- (6.96115,6.69953);
\definecolor{c}{rgb}{0,0,0};
\colorlet{c}{natgreen!50};
\draw [c] (7.04484,6.64047) -- (7.04484,6.65304);
\draw [c] (7.04484,6.65304) -- (7.04484,6.66495);
\draw [c] (6.96115,6.65304) -- (7.04484,6.65304);
\draw [c] (7.04484,6.65304) -- (7.12854,6.65304);
\definecolor{c}{rgb}{0,0,0};
\colorlet{c}{natgreen!50};
\draw [c] (7.21223,6.60503) -- (7.21223,6.6186);
\draw [c] (7.21223,6.6186) -- (7.21223,6.63141);
\draw [c] (7.12854,6.6186) -- (7.21223,6.6186);
\draw [c] (7.21223,6.6186) -- (7.29592,6.6186);
\definecolor{c}{rgb}{0,0,0};
\colorlet{c}{natgreen!50};
\draw [c] (7.37962,6.53957) -- (7.37962,6.55523);
\draw [c] (7.37962,6.55523) -- (7.37962,6.56988);
\draw [c] (7.29592,6.55523) -- (7.37962,6.55523);
\draw [c] (7.37962,6.55523) -- (7.46331,6.55523);
\definecolor{c}{rgb}{0,0,0};
\colorlet{c}{natgreen!50};
\draw [c] (7.54701,6.49771) -- (7.54701,6.51487);
\draw [c] (7.54701,6.51487) -- (7.54701,6.53083);
\draw [c] (7.46331,6.51487) -- (7.54701,6.51487);
\draw [c] (7.54701,6.51487) -- (7.6307,6.51487);
\definecolor{c}{rgb}{0,0,0};
\colorlet{c}{natgreen!50};
\draw [c] (7.7144,6.47563) -- (7.7144,6.49363);
\draw [c] (7.7144,6.49363) -- (7.7144,6.51032);
\draw [c] (7.6307,6.49363) -- (7.7144,6.49363);
\draw [c] (7.7144,6.49363) -- (7.79809,6.49363);
\definecolor{c}{rgb}{0,0,0};
\colorlet{c}{natgreen!50};
\draw [c] (7.88179,6.41886) -- (7.88179,6.43923);
\draw [c] (7.88179,6.43923) -- (7.88179,6.45794);
\draw [c] (7.79809,6.43923) -- (7.88179,6.43923);
\draw [c] (7.88179,6.43923) -- (7.96548,6.43923);
\definecolor{c}{rgb}{0,0,0};
\colorlet{c}{natgreen!50};
\draw [c] (8.04918,6.36434) -- (8.04918,6.38729);
\draw [c] (8.04918,6.38729) -- (8.04918,6.40815);
\draw [c] (7.96548,6.38729) -- (8.04918,6.38729);
\draw [c] (8.04918,6.38729) -- (8.13287,6.38729);
\definecolor{c}{rgb}{0,0,0};
\colorlet{c}{natgreen!50};
\draw [c] (8.21657,6.36214) -- (8.21657,6.3852);
\draw [c] (8.21657,6.3852) -- (8.21657,6.40615);
\draw [c] (8.13287,6.3852) -- (8.21657,6.3852);
\draw [c] (8.21657,6.3852) -- (8.30026,6.3852);
\definecolor{c}{rgb}{0,0,0};
\colorlet{c}{natgreen!50};
\draw [c] (8.38395,6.22088) -- (8.38395,6.25226);
\draw [c] (8.38395,6.25226) -- (8.38395,6.27985);
\draw [c] (8.30026,6.25226) -- (8.38395,6.25226);
\draw [c] (8.38395,6.25226) -- (8.46765,6.25226);
\definecolor{c}{rgb}{0,0,0};
\colorlet{c}{natgreen!50};
\draw [c] (8.55134,6.16166) -- (8.55134,6.19737);
\draw [c] (8.55134,6.19737) -- (8.55134,6.22824);
\draw [c] (8.46765,6.19737) -- (8.55134,6.19737);
\draw [c] (8.55134,6.19737) -- (8.63504,6.19737);
\definecolor{c}{rgb}{0,0,0};
\colorlet{c}{natgreen!50};
\draw [c] (8.71873,6.08972) -- (8.71873,6.13148);
\draw [c] (8.71873,6.13148) -- (8.71873,6.16678);
\draw [c] (8.63504,6.13148) -- (8.71873,6.13148);
\draw [c] (8.71873,6.13148) -- (8.80243,6.13148);
\definecolor{c}{rgb}{0,0,0};
\colorlet{c}{natgreen!50};
\draw [c] (8.88612,6.03491) -- (8.88612,6.08196);
\draw [c] (8.88612,6.08196) -- (8.88612,6.12097);
\draw [c] (8.80243,6.08196) -- (8.88612,6.08196);
\draw [c] (8.88612,6.08196) -- (8.96982,6.08196);
\definecolor{c}{rgb}{0,0,0};
\colorlet{c}{natgreen!50};
\draw [c] (9.05351,5.87986) -- (9.05351,5.94575);
\draw [c] (9.05351,5.94575) -- (9.05351,5.99686);
\draw [c] (8.96982,5.94575) -- (9.05351,5.94575);
\draw [c] (9.05351,5.94575) -- (9.13721,5.94575);
\definecolor{c}{rgb}{0,0,0};
\colorlet{c}{natgreen!50};
\draw [c] (9.2209,5.75105) -- (9.2209,5.83811);
\draw [c] (9.2209,5.83811) -- (9.2209,5.90104);
\draw [c] (9.13721,5.83811) -- (9.2209,5.83811);
\draw [c] (9.2209,5.83811) -- (9.30459,5.83811);
\definecolor{c}{rgb}{0,0,0};
\colorlet{c}{natgreen};
\draw [c] (1.01883,9.82318) -- (1.01883,9.82379);
\draw [c] (1.01883,9.82379) -- (1.01883,9.82441);
\draw [c] (0.935135,9.82379) -- (1.01883,9.82379);
\draw [c] (1.01883,9.82379) -- (1.10252,9.82379);
\definecolor{c}{rgb}{0,0,0};
\colorlet{c}{natgreen};
\draw [c] (1.18622,9.41834) -- (1.18622,9.41982);
\draw [c] (1.18622,9.41982) -- (1.18622,9.4213);
\draw [c] (1.10252,9.41982) -- (1.18622,9.41982);
\draw [c] (1.18622,9.41982) -- (1.26991,9.41982);
\definecolor{c}{rgb}{0,0,0};
\colorlet{c}{natgreen};
\draw [c] (1.35361,8.94316) -- (1.35361,8.94735);
\draw [c] (1.35361,8.94735) -- (1.35361,8.95147);
\draw [c] (1.26991,8.94735) -- (1.35361,8.94735);
\draw [c] (1.35361,8.94735) -- (1.4373,8.94735);
\definecolor{c}{rgb}{0,0,0};
\colorlet{c}{natgreen};
\draw [c] (1.521,8.62602) -- (1.521,8.6344);
\draw [c] (1.521,8.6344) -- (1.521,8.64248);
\draw [c] (1.4373,8.6344) -- (1.521,8.6344);
\draw [c] (1.521,8.6344) -- (1.60469,8.6344);
\definecolor{c}{rgb}{0,0,0};
\colorlet{c}{natgreen};
\draw [c] (1.68839,8.39091) -- (1.68839,8.39175);
\draw [c] (1.68839,8.39175) -- (1.68839,8.39259);
\draw [c] (1.60469,8.39175) -- (1.68839,8.39175);
\draw [c] (1.68839,8.39175) -- (1.77208,8.39175);
\definecolor{c}{rgb}{0,0,0};
\colorlet{c}{natgreen};
\draw [c] (1.85578,8.19345) -- (1.85578,8.19475);
\draw [c] (1.85578,8.19475) -- (1.85578,8.19604);
\draw [c] (1.77208,8.19475) -- (1.85578,8.19475);
\draw [c] (1.85578,8.19475) -- (1.93947,8.19475);
\definecolor{c}{rgb}{0,0,0};
\colorlet{c}{natgreen};
\draw [c] (2.02316,8.02899) -- (2.02316,8.03085);
\draw [c] (2.02316,8.03085) -- (2.02316,8.03269);
\draw [c] (1.93947,8.03085) -- (2.02316,8.03085);
\draw [c] (2.02316,8.03085) -- (2.10686,8.03085);
\definecolor{c}{rgb}{0,0,0};
\colorlet{c}{natgreen};
\draw [c] (2.19055,7.87847) -- (2.19055,7.88105);
\draw [c] (2.19055,7.88105) -- (2.19055,7.88361);
\draw [c] (2.10686,7.88105) -- (2.19055,7.88105);
\draw [c] (2.19055,7.88105) -- (2.27425,7.88105);
\definecolor{c}{rgb}{0,0,0};
\colorlet{c}{natgreen};
\draw [c] (2.35794,7.75546) -- (2.35794,7.75884);
\draw [c] (2.35794,7.75884) -- (2.35794,7.76217);
\draw [c] (2.27425,7.75884) -- (2.35794,7.75884);
\draw [c] (2.35794,7.75884) -- (2.44164,7.75884);
\definecolor{c}{rgb}{0,0,0};
\colorlet{c}{natgreen};
\draw [c] (2.52533,7.63915) -- (2.52533,7.64351);
\draw [c] (2.52533,7.64351) -- (2.52533,7.64779);
\draw [c] (2.44164,7.64351) -- (2.52533,7.64351);
\draw [c] (2.52533,7.64351) -- (2.60903,7.64351);
\definecolor{c}{rgb}{0,0,0};
\colorlet{c}{natgreen};
\draw [c] (2.69272,7.53707) -- (2.69272,7.53837);
\draw [c] (2.69272,7.53837) -- (2.69272,7.53965);
\draw [c] (2.60903,7.53837) -- (2.69272,7.53837);
\draw [c] (2.69272,7.53837) -- (2.77642,7.53837);
\definecolor{c}{rgb}{0,0,0};
\colorlet{c}{natgreen};
\draw [c] (2.86011,7.44485) -- (2.86011,7.44643);
\draw [c] (2.86011,7.44643) -- (2.86011,7.448);
\draw [c] (2.77642,7.44643) -- (2.86011,7.44643);
\draw [c] (2.86011,7.44643) -- (2.94381,7.44643);
\definecolor{c}{rgb}{0,0,0};
\colorlet{c}{natgreen};
\draw [c] (3.0275,7.3682) -- (3.0275,7.37007);
\draw [c] (3.0275,7.37007) -- (3.0275,7.37192);
\draw [c] (2.94381,7.37007) -- (3.0275,7.37007);
\draw [c] (3.0275,7.37007) -- (3.11119,7.37007);
\definecolor{c}{rgb}{0,0,0};
\colorlet{c}{natgreen};
\draw [c] (3.19489,7.30181) -- (3.19489,7.30397);
\draw [c] (3.19489,7.30397) -- (3.19489,7.30611);
\draw [c] (3.11119,7.30397) -- (3.19489,7.30397);
\draw [c] (3.19489,7.30397) -- (3.27858,7.30397);
\definecolor{c}{rgb}{0,0,0};
\colorlet{c}{natgreen};
\draw [c] (3.36228,7.24053) -- (3.36228,7.243);
\draw [c] (3.36228,7.243) -- (3.36228,7.24544);
\draw [c] (3.27858,7.243) -- (3.36228,7.243);
\draw [c] (3.36228,7.243) -- (3.44597,7.243);
\definecolor{c}{rgb}{0,0,0};
\colorlet{c}{natgreen};
\draw [c] (3.52967,7.19473) -- (3.52967,7.19746);
\draw [c] (3.52967,7.19746) -- (3.52967,7.20016);
\draw [c] (3.44597,7.19746) -- (3.52967,7.19746);
\draw [c] (3.52967,7.19746) -- (3.61336,7.19746);
\definecolor{c}{rgb}{0,0,0};
\colorlet{c}{natgreen};
\draw [c] (3.69706,7.14897) -- (3.69706,7.15199);
\draw [c] (3.69706,7.15199) -- (3.69706,7.15496);
\draw [c] (3.61336,7.15199) -- (3.69706,7.15199);
\draw [c] (3.69706,7.15199) -- (3.78075,7.15199);
\definecolor{c}{rgb}{0,0,0};
\colorlet{c}{natgreen};
\draw [c] (3.86445,7.11784) -- (3.86445,7.12107);
\draw [c] (3.86445,7.12107) -- (3.86445,7.12426);
\draw [c] (3.78075,7.12107) -- (3.86445,7.12107);
\draw [c] (3.86445,7.12107) -- (3.94814,7.12107);
\definecolor{c}{rgb}{0,0,0};
\colorlet{c}{natgreen};
\draw [c] (4.03184,7.07928) -- (4.03184,7.08279);
\draw [c] (4.03184,7.08279) -- (4.03184,7.08625);
\draw [c] (3.94814,7.08279) -- (4.03184,7.08279);
\draw [c] (4.03184,7.08279) -- (4.11553,7.08279);
\definecolor{c}{rgb}{0,0,0};
\colorlet{c}{natgreen};
\draw [c] (4.19922,7.04963) -- (4.19922,7.05338);
\draw [c] (4.19922,7.05338) -- (4.19922,7.05707);
\draw [c] (4.11553,7.05338) -- (4.19922,7.05338);
\draw [c] (4.19922,7.05338) -- (4.28292,7.05338);
\definecolor{c}{rgb}{0,0,0};
\colorlet{c}{natgreen};
\draw [c] (4.36661,7.02975) -- (4.36661,7.03367);
\draw [c] (4.36661,7.03367) -- (4.36661,7.03752);
\draw [c] (4.28292,7.03367) -- (4.36661,7.03367);
\draw [c] (4.36661,7.03367) -- (4.45031,7.03367);
\definecolor{c}{rgb}{0,0,0};
\colorlet{c}{natgreen};
\draw [c] (4.534,7.0026) -- (4.534,7.00676);
\draw [c] (4.534,7.00676) -- (4.534,7.01084);
\draw [c] (4.45031,7.00676) -- (4.534,7.00676);
\draw [c] (4.534,7.00676) -- (4.6177,7.00676);
\definecolor{c}{rgb}{0,0,0};
\colorlet{c}{natgreen};
\draw [c] (4.70139,6.98069) -- (4.70139,6.98505);
\draw [c] (4.70139,6.98505) -- (4.70139,6.98932);
\draw [c] (4.6177,6.98505) -- (4.70139,6.98505);
\draw [c] (4.70139,6.98505) -- (4.78509,6.98505);
\definecolor{c}{rgb}{0,0,0};
\colorlet{c}{natgreen};
\draw [c] (4.86878,6.96152) -- (4.86878,6.96606);
\draw [c] (4.86878,6.96606) -- (4.86878,6.97052);
\draw [c] (4.78509,6.96606) -- (4.86878,6.96606);
\draw [c] (4.86878,6.96606) -- (4.95248,6.96606);
\definecolor{c}{rgb}{0,0,0};
\colorlet{c}{natgreen};
\draw [c] (5.03617,6.93376) -- (5.03617,6.93859);
\draw [c] (5.03617,6.93859) -- (5.03617,6.94332);
\draw [c] (4.95248,6.93859) -- (5.03617,6.93859);
\draw [c] (5.03617,6.93859) -- (5.11986,6.93859);
\definecolor{c}{rgb}{0,0,0};
\colorlet{c}{natgreen};
\draw [c] (5.20356,6.91114) -- (5.20356,6.91621);
\draw [c] (5.20356,6.91621) -- (5.20356,6.92117);
\draw [c] (5.11986,6.91621) -- (5.20356,6.91621);
\draw [c] (5.20356,6.91621) -- (5.28725,6.91621);
\definecolor{c}{rgb}{0,0,0};
\colorlet{c}{natgreen};
\draw [c] (5.37095,6.88293) -- (5.37095,6.88833);
\draw [c] (5.37095,6.88833) -- (5.37095,6.8936);
\draw [c] (5.28725,6.88833) -- (5.37095,6.88833);
\draw [c] (5.37095,6.88833) -- (5.45464,6.88833);
\definecolor{c}{rgb}{0,0,0};
\colorlet{c}{natgreen};
\draw [c] (5.53834,6.86655) -- (5.53834,6.87214);
\draw [c] (5.53834,6.87214) -- (5.53834,6.87759);
\draw [c] (5.45464,6.87214) -- (5.53834,6.87214);
\draw [c] (5.53834,6.87214) -- (5.62203,6.87214);
\definecolor{c}{rgb}{0,0,0};
\colorlet{c}{natgreen};
\draw [c] (5.70573,6.83106) -- (5.70573,6.8371);
\draw [c] (5.70573,6.8371) -- (5.70573,6.84298);
\draw [c] (5.62203,6.8371) -- (5.70573,6.8371);
\draw [c] (5.70573,6.8371) -- (5.78942,6.8371);
\definecolor{c}{rgb}{0,0,0};
\colorlet{c}{natgreen};
\draw [c] (5.87312,6.80265) -- (5.87312,6.80907);
\draw [c] (5.87312,6.80907) -- (5.87312,6.81532);
\draw [c] (5.78942,6.80907) -- (5.87312,6.80907);
\draw [c] (5.87312,6.80907) -- (5.95681,6.80907);
\definecolor{c}{rgb}{0,0,0};
\colorlet{c}{natgreen};
\draw [c] (6.04051,6.76422) -- (6.04051,6.77121);
\draw [c] (6.04051,6.77121) -- (6.04051,6.77799);
\draw [c] (5.95681,6.77121) -- (6.04051,6.77121);
\draw [c] (6.04051,6.77121) -- (6.1242,6.77121);
\definecolor{c}{rgb}{0,0,0};
\colorlet{c}{natgreen};
\draw [c] (6.20789,6.7661) -- (6.20789,6.77306);
\draw [c] (6.20789,6.77306) -- (6.20789,6.77982);
\draw [c] (6.1242,6.77306) -- (6.20789,6.77306);
\draw [c] (6.20789,6.77306) -- (6.29159,6.77306);
\definecolor{c}{rgb}{0,0,0};
\colorlet{c}{natgreen};
\draw [c] (6.37528,6.71159) -- (6.37528,6.71943);
\draw [c] (6.37528,6.71943) -- (6.37528,6.72701);
\draw [c] (6.29159,6.71943) -- (6.37528,6.71943);
\draw [c] (6.37528,6.71943) -- (6.45898,6.71943);
\definecolor{c}{rgb}{0,0,0};
\colorlet{c}{natgreen};
\draw [c] (6.54267,6.68357) -- (6.54267,6.6919);
\draw [c] (6.54267,6.6919) -- (6.54267,6.69994);
\draw [c] (6.45898,6.6919) -- (6.54267,6.6919);
\draw [c] (6.54267,6.6919) -- (6.62637,6.6919);
\definecolor{c}{rgb}{0,0,0};
\colorlet{c}{natgreen};
\draw [c] (6.71006,6.61265) -- (6.71006,6.62238);
\draw [c] (6.71006,6.62238) -- (6.71006,6.63171);
\draw [c] (6.62637,6.62238) -- (6.71006,6.62238);
\draw [c] (6.71006,6.62238) -- (6.79376,6.62238);
\definecolor{c}{rgb}{0,0,0};
\colorlet{c}{natgreen};
\draw [c] (6.87745,6.58057) -- (6.87745,6.591);
\draw [c] (6.87745,6.591) -- (6.87745,6.60098);
\draw [c] (6.79376,6.591) -- (6.87745,6.591);
\draw [c] (6.87745,6.591) -- (6.96115,6.591);
\definecolor{c}{rgb}{0,0,0};
\colorlet{c}{natgreen};
\draw [c] (7.04484,6.55294) -- (7.04484,6.56403);
\draw [c] (7.04484,6.56403) -- (7.04484,6.5746);
\draw [c] (6.96115,6.56403) -- (7.04484,6.56403);
\draw [c] (7.04484,6.56403) -- (7.12854,6.56403);
\definecolor{c}{rgb}{0,0,0};
\colorlet{c}{natgreen};
\draw [c] (7.21223,6.52157) -- (7.21223,6.53344);
\draw [c] (7.21223,6.53344) -- (7.21223,6.54473);
\draw [c] (7.12854,6.53344) -- (7.21223,6.53344);
\draw [c] (7.21223,6.53344) -- (7.29592,6.53344);
\definecolor{c}{rgb}{0,0,0};
\colorlet{c}{natgreen};
\draw [c] (7.37962,6.47599) -- (7.37962,6.4891);
\draw [c] (7.37962,6.4891) -- (7.37962,6.5015);
\draw [c] (7.29592,6.4891) -- (7.37962,6.4891);
\draw [c] (7.37962,6.4891) -- (7.46331,6.4891);
\definecolor{c}{rgb}{0,0,0};
\colorlet{c}{natgreen};
\draw [c] (7.54701,6.40874) -- (7.54701,6.42392);
\draw [c] (7.54701,6.42392) -- (7.54701,6.43816);
\draw [c] (7.46331,6.42392) -- (7.54701,6.42392);
\draw [c] (7.54701,6.42392) -- (7.6307,6.42392);
\definecolor{c}{rgb}{0,0,0};
\colorlet{c}{natgreen};
\draw [c] (7.7144,6.35528) -- (7.7144,6.37234);
\draw [c] (7.7144,6.37234) -- (7.7144,6.38823);
\draw [c] (7.6307,6.37234) -- (7.7144,6.37234);
\draw [c] (7.7144,6.37234) -- (7.79809,6.37234);
\definecolor{c}{rgb}{0,0,0};
\colorlet{c}{natgreen};
\draw [c] (7.88179,6.28909) -- (7.88179,6.30881);
\draw [c] (7.88179,6.30881) -- (7.88179,6.32696);
\draw [c] (7.79809,6.30881) -- (7.88179,6.30881);
\draw [c] (7.88179,6.30881) -- (7.96548,6.30881);
\definecolor{c}{rgb}{0,0,0};
\colorlet{c}{natgreen};
\draw [c] (8.04918,6.25022) -- (8.04918,6.27168);
\draw [c] (8.04918,6.27168) -- (8.04918,6.2913);
\draw [c] (7.96548,6.27168) -- (8.04918,6.27168);
\draw [c] (8.04918,6.27168) -- (8.13287,6.27168);
\definecolor{c}{rgb}{0,0,0};
\colorlet{c}{natgreen};
\draw [c] (8.21657,6.14231) -- (8.21657,6.16946);
\draw [c] (8.21657,6.16946) -- (8.21657,6.19374);
\draw [c] (8.13287,6.16946) -- (8.21657,6.16946);
\draw [c] (8.21657,6.16946) -- (8.30026,6.16946);
\definecolor{c}{rgb}{0,0,0};
\colorlet{c}{natgreen};
\draw [c] (8.38395,6.09901) -- (8.38395,6.12885);
\draw [c] (8.38395,6.12885) -- (8.38395,6.15525);
\draw [c] (8.30026,6.12885) -- (8.38395,6.12885);
\draw [c] (8.38395,6.12885) -- (8.46765,6.12885);
\definecolor{c}{rgb}{0,0,0};
\colorlet{c}{natgreen};
\draw [c] (8.55134,6.07598) -- (8.55134,6.10736);
\draw [c] (8.55134,6.10736) -- (8.55134,6.13496);
\draw [c] (8.46765,6.10736) -- (8.55134,6.10736);
\draw [c] (8.55134,6.10736) -- (8.63504,6.10736);
\definecolor{c}{rgb}{0,0,0};
\colorlet{c}{natgreen};
\draw [c] (8.71873,5.9713) -- (8.71873,6.01071);
\draw [c] (8.71873,6.01071) -- (8.71873,6.04433);
\draw [c] (8.63504,6.01071) -- (8.71873,6.01071);
\draw [c] (8.71873,6.01071) -- (8.80243,6.01071);
\definecolor{c}{rgb}{0,0,0};
\colorlet{c}{natgreen};
\draw [c] (8.88612,5.95172) -- (8.88612,5.99286);
\draw [c] (8.88612,5.99286) -- (8.88612,6.02772);
\draw [c] (8.80243,5.99286) -- (8.88612,5.99286);
\draw [c] (8.88612,5.99286) -- (8.96982,5.99286);
\definecolor{c}{rgb}{0,0,0};
\colorlet{c}{natgreen};
\draw [c] (9.05351,5.83042) -- (9.05351,5.88397);
\draw [c] (9.05351,5.88397) -- (9.05351,5.92735);
\draw [c] (8.96982,5.88397) -- (9.05351,5.88397);
\draw [c] (9.05351,5.88397) -- (9.13721,5.88397);
\definecolor{c}{rgb}{0,0,0};
\colorlet{c}{natgreen};
\draw [c] (9.2209,5.63286) -- (9.2209,5.71504);
\draw [c] (9.2209,5.71504) -- (9.2209,5.7754);
\draw [c] (9.13721,5.71504) -- (9.2209,5.71504);
\draw [c] (9.2209,5.71504) -- (9.30459,5.71504);
\definecolor{c}{rgb}{0,0,0};
\colorlet{c}{natcomp!50};
\draw [c] (1.01883,9.79638) -- (1.01883,9.79721);
\draw [c] (1.01883,9.79721) -- (1.01883,9.79805);
\draw [c] (0.935135,9.79721) -- (1.01883,9.79721);
\draw [c] (1.01883,9.79721) -- (1.10252,9.79721);
\definecolor{c}{rgb}{0,0,0};
\colorlet{c}{natcomp!50};
\draw [c] (1.18622,9.3852) -- (1.18622,9.38617);
\draw [c] (1.18622,9.38617) -- (1.18622,9.38714);
\draw [c] (1.10252,9.38617) -- (1.18622,9.38617);
\draw [c] (1.18622,9.38617) -- (1.26991,9.38617);
\definecolor{c}{rgb}{0,0,0};
\colorlet{c}{natcomp!50};
\draw [c] (1.35361,8.906) -- (1.35361,8.90875);
\draw [c] (1.35361,8.90875) -- (1.35361,8.91148);
\draw [c] (1.26991,8.90875) -- (1.35361,8.90875);
\draw [c] (1.35361,8.90875) -- (1.4373,8.90875);
\definecolor{c}{rgb}{0,0,0};
\colorlet{c}{natcomp!50};
\draw [c] (1.521,8.5812) -- (1.521,8.58681);
\draw [c] (1.521,8.58681) -- (1.521,8.59228);
\draw [c] (1.4373,8.58681) -- (1.521,8.58681);
\draw [c] (1.521,8.58681) -- (1.60469,8.58681);
\definecolor{c}{rgb}{0,0,0};
\colorlet{c}{natcomp!50};
\draw [c] (1.68839,8.33779) -- (1.68839,8.34733);
\draw [c] (1.68839,8.34733) -- (1.68839,8.35648);
\draw [c] (1.60469,8.34733) -- (1.68839,8.34733);
\draw [c] (1.68839,8.34733) -- (1.77208,8.34733);
\definecolor{c}{rgb}{0,0,0};
\colorlet{c}{natcomp!50};
\draw [c] (1.85578,8.15996) -- (1.85578,8.17402);
\draw [c] (1.85578,8.17402) -- (1.85578,8.18727);
\draw [c] (1.77208,8.17402) -- (1.85578,8.17402);
\draw [c] (1.85578,8.17402) -- (1.93947,8.17402);
\definecolor{c}{rgb}{0,0,0};
\colorlet{c}{natcomp!50};
\draw [c] (2.02316,8.0051) -- (2.02316,8.02482);
\draw [c] (2.02316,8.02482) -- (2.02316,8.04297);
\draw [c] (1.93947,8.02482) -- (2.02316,8.02482);
\draw [c] (2.02316,8.02482) -- (2.10686,8.02482);
\definecolor{c}{rgb}{0,0,0};
\colorlet{c}{natcomp!50};
\draw [c] (2.19055,7.82222) -- (2.19055,7.85159);
\draw [c] (2.19055,7.85159) -- (2.19055,7.87763);
\draw [c] (2.10686,7.85159) -- (2.19055,7.85159);
\draw [c] (2.19055,7.85159) -- (2.27425,7.85159);
\definecolor{c}{rgb}{0,0,0};
\colorlet{c}{natcomp!50};
\draw [c] (2.35794,7.7379) -- (2.35794,7.7732);
\draw [c] (2.35794,7.7732) -- (2.35794,7.80378);
\draw [c] (2.27425,7.7732) -- (2.35794,7.7732);
\draw [c] (2.35794,7.7732) -- (2.44164,7.7732);
\definecolor{c}{rgb}{0,0,0};
\colorlet{c}{natcomp!50};
\draw [c] (2.52533,7.69951) -- (2.52533,7.7379);
\draw [c] (2.52533,7.7379) -- (2.52533,7.77076);
\draw [c] (2.44164,7.7379) -- (2.52533,7.7379);
\draw [c] (2.52533,7.7379) -- (2.60903,7.7379);
\definecolor{c}{rgb}{0,0,0};
\colorlet{c}{natcomp!50};
\draw [c] (2.69272,7.65197) -- (2.69272,7.65466);
\draw [c] (2.69272,7.65466) -- (2.69272,7.65731);
\draw [c] (2.60903,7.65466) -- (2.69272,7.65466);
\draw [c] (2.69272,7.65466) -- (2.77642,7.65466);
\definecolor{c}{rgb}{0,0,0};
\colorlet{c}{natcomp!50};
\draw [c] (2.86011,7.60401) -- (2.86011,7.60699);
\draw [c] (2.86011,7.60699) -- (2.86011,7.60994);
\draw [c] (2.77642,7.60699) -- (2.86011,7.60699);
\draw [c] (2.86011,7.60699) -- (2.94381,7.60699);
\definecolor{c}{rgb}{0,0,0};
\colorlet{c}{natcomp!50};
\draw [c] (3.0275,7.57717) -- (3.0275,7.58033);
\draw [c] (3.0275,7.58033) -- (3.0275,7.58345);
\draw [c] (2.94381,7.58033) -- (3.0275,7.58033);
\draw [c] (3.0275,7.58033) -- (3.11119,7.58033);
\definecolor{c}{rgb}{0,0,0};
\colorlet{c}{natcomp!50};
\draw [c] (3.19489,7.54588) -- (3.19489,7.54926);
\draw [c] (3.19489,7.54926) -- (3.19489,7.5526);
\draw [c] (3.11119,7.54926) -- (3.19489,7.54926);
\draw [c] (3.19489,7.54926) -- (3.27858,7.54926);
\definecolor{c}{rgb}{0,0,0};
\colorlet{c}{natcomp!50};
\draw [c] (3.36228,7.53546) -- (3.36228,7.53893);
\draw [c] (3.36228,7.53893) -- (3.36228,7.54234);
\draw [c] (3.27858,7.53893) -- (3.36228,7.53893);
\draw [c] (3.36228,7.53893) -- (3.44597,7.53893);
\definecolor{c}{rgb}{0,0,0};
\colorlet{c}{natcomp!50};
\draw [c] (3.52967,7.52909) -- (3.52967,7.5326);
\draw [c] (3.52967,7.5326) -- (3.52967,7.53606);
\draw [c] (3.44597,7.5326) -- (3.52967,7.5326);
\draw [c] (3.52967,7.5326) -- (3.61336,7.5326);
\definecolor{c}{rgb}{0,0,0};
\colorlet{c}{natcomp!50};
\draw [c] (3.69706,7.51493) -- (3.69706,7.51855);
\draw [c] (3.69706,7.51855) -- (3.69706,7.52212);
\draw [c] (3.61336,7.51855) -- (3.69706,7.51855);
\draw [c] (3.69706,7.51855) -- (3.78075,7.51855);
\definecolor{c}{rgb}{0,0,0};
\colorlet{c}{natcomp!50};
\draw [c] (3.86445,7.50958) -- (3.86445,7.51325);
\draw [c] (3.86445,7.51325) -- (3.86445,7.51686);
\draw [c] (3.78075,7.51325) -- (3.86445,7.51325);
\draw [c] (3.86445,7.51325) -- (3.94814,7.51325);
\definecolor{c}{rgb}{0,0,0};
\colorlet{c}{natcomp!50};
\draw [c] (4.03184,7.5031) -- (4.03184,7.50681);
\draw [c] (4.03184,7.50681) -- (4.03184,7.51047);
\draw [c] (3.94814,7.50681) -- (4.03184,7.50681);
\draw [c] (4.03184,7.50681) -- (4.11553,7.50681);
\definecolor{c}{rgb}{0,0,0};
\colorlet{c}{natcomp!50};
\draw [c] (4.19922,7.49593) -- (4.19922,7.4997);
\draw [c] (4.19922,7.4997) -- (4.19922,7.50342);
\draw [c] (4.11553,7.4997) -- (4.19922,7.4997);
\draw [c] (4.19922,7.4997) -- (4.28292,7.4997);
\definecolor{c}{rgb}{0,0,0};
\colorlet{c}{natcomp!50};
\draw [c] (4.36661,7.48846) -- (4.36661,7.4923);
\draw [c] (4.36661,7.4923) -- (4.36661,7.49607);
\draw [c] (4.28292,7.4923) -- (4.36661,7.4923);
\draw [c] (4.36661,7.4923) -- (4.45031,7.4923);
\definecolor{c}{rgb}{0,0,0};
\colorlet{c}{natcomp!50};
\draw [c] (4.534,7.48285) -- (4.534,7.48673);
\draw [c] (4.534,7.48673) -- (4.534,7.49055);
\draw [c] (4.45031,7.48673) -- (4.534,7.48673);
\draw [c] (4.534,7.48673) -- (4.6177,7.48673);
\definecolor{c}{rgb}{0,0,0};
\colorlet{c}{natcomp!50};
\draw [c] (4.70139,7.47703) -- (4.70139,7.48096);
\draw [c] (4.70139,7.48096) -- (4.70139,7.48483);
\draw [c] (4.6177,7.48096) -- (4.70139,7.48096);
\draw [c] (4.70139,7.48096) -- (4.78509,7.48096);
\definecolor{c}{rgb}{0,0,0};
\colorlet{c}{natcomp!50};
\draw [c] (4.86878,7.46877) -- (4.86878,7.47278);
\draw [c] (4.86878,7.47278) -- (4.86878,7.47672);
\draw [c] (4.78509,7.47278) -- (4.86878,7.47278);
\draw [c] (4.86878,7.47278) -- (4.95248,7.47278);
\definecolor{c}{rgb}{0,0,0};
\colorlet{c}{natcomp!50};
\draw [c] (5.03617,7.4587) -- (5.03617,7.46279);
\draw [c] (5.03617,7.46279) -- (5.03617,7.46681);
\draw [c] (4.95248,7.46279) -- (5.03617,7.46279);
\draw [c] (5.03617,7.46279) -- (5.11986,7.46279);
\definecolor{c}{rgb}{0,0,0};
\colorlet{c}{natcomp!50};
\draw [c] (5.20356,7.44502) -- (5.20356,7.44924);
\draw [c] (5.20356,7.44924) -- (5.20356,7.45338);
\draw [c] (5.11986,7.44924) -- (5.20356,7.44924);
\draw [c] (5.20356,7.44924) -- (5.28725,7.44924);
\definecolor{c}{rgb}{0,0,0};
\colorlet{c}{natcomp!50};
\draw [c] (5.37095,7.4398) -- (5.37095,7.44407);
\draw [c] (5.37095,7.44407) -- (5.37095,7.44826);
\draw [c] (5.28725,7.44407) -- (5.37095,7.44407);
\draw [c] (5.37095,7.44407) -- (5.45464,7.44407);
\definecolor{c}{rgb}{0,0,0};
\colorlet{c}{natcomp!50};
\draw [c] (5.53834,7.40801) -- (5.53834,7.41258);
\draw [c] (5.53834,7.41258) -- (5.53834,7.41707);
\draw [c] (5.45464,7.41258) -- (5.53834,7.41258);
\draw [c] (5.53834,7.41258) -- (5.62203,7.41258);
\definecolor{c}{rgb}{0,0,0};
\colorlet{c}{natcomp!50};
\draw [c] (5.70573,7.39897) -- (5.70573,7.40363);
\draw [c] (5.70573,7.40363) -- (5.70573,7.40821);
\draw [c] (5.62203,7.40363) -- (5.70573,7.40363);
\draw [c] (5.70573,7.40363) -- (5.78942,7.40363);
\definecolor{c}{rgb}{0,0,0};
\colorlet{c}{natcomp!50};
\draw [c] (5.87312,7.38511) -- (5.87312,7.38992);
\draw [c] (5.87312,7.38992) -- (5.87312,7.39463);
\draw [c] (5.78942,7.38992) -- (5.87312,7.38992);
\draw [c] (5.87312,7.38992) -- (5.95681,7.38992);
\definecolor{c}{rgb}{0,0,0};
\colorlet{c}{natcomp!50};
\draw [c] (6.04051,7.35312) -- (6.04051,7.35827);
\draw [c] (6.04051,7.35827) -- (6.04051,7.36332);
\draw [c] (5.95681,7.35827) -- (6.04051,7.35827);
\draw [c] (6.04051,7.35827) -- (6.1242,7.35827);
\definecolor{c}{rgb}{0,0,0};
\colorlet{c}{natcomp!50};
\draw [c] (6.20789,7.32906) -- (6.20789,7.3345);
\draw [c] (6.20789,7.3345) -- (6.20789,7.3398);
\draw [c] (6.1242,7.3345) -- (6.20789,7.3345);
\draw [c] (6.20789,7.3345) -- (6.29159,7.3345);
\definecolor{c}{rgb}{0,0,0};
\colorlet{c}{natcomp!50};
\draw [c] (6.37528,7.30855) -- (6.37528,7.31424);
\draw [c] (6.37528,7.31424) -- (6.37528,7.31978);
\draw [c] (6.29159,7.31424) -- (6.37528,7.31424);
\draw [c] (6.37528,7.31424) -- (6.45898,7.31424);
\definecolor{c}{rgb}{0,0,0};
\colorlet{c}{natcomp!50};
\draw [c] (6.54267,7.27567) -- (6.54267,7.28178);
\draw [c] (6.54267,7.28178) -- (6.54267,7.28773);
\draw [c] (6.45898,7.28178) -- (6.54267,7.28178);
\draw [c] (6.54267,7.28178) -- (6.62637,7.28178);
\definecolor{c}{rgb}{0,0,0};
\colorlet{c}{natcomp!50};
\draw [c] (6.71006,7.23271) -- (6.71006,7.23942);
\draw [c] (6.71006,7.23942) -- (6.71006,7.24593);
\draw [c] (6.62637,7.23942) -- (6.71006,7.23942);
\draw [c] (6.71006,7.23942) -- (6.79376,7.23942);
\definecolor{c}{rgb}{0,0,0};
\colorlet{c}{natcomp!50};
\draw [c] (6.87745,7.21397) -- (6.87745,7.22096);
\draw [c] (6.87745,7.22096) -- (6.87745,7.22774);
\draw [c] (6.79376,7.22096) -- (6.87745,7.22096);
\draw [c] (6.87745,7.22096) -- (6.96115,7.22096);
\definecolor{c}{rgb}{0,0,0};
\colorlet{c}{natcomp!50};
\draw [c] (7.04484,7.18037) -- (7.04484,7.18788);
\draw [c] (7.04484,7.18788) -- (7.04484,7.19516);
\draw [c] (6.96115,7.18788) -- (7.04484,7.18788);
\draw [c] (7.04484,7.18788) -- (7.12854,7.18788);
\definecolor{c}{rgb}{0,0,0};
\colorlet{c}{natcomp!50};
\draw [c] (7.21223,7.12585) -- (7.21223,7.13432);
\draw [c] (7.21223,7.13432) -- (7.21223,7.14248);
\draw [c] (7.12854,7.13432) -- (7.21223,7.13432);
\draw [c] (7.21223,7.13432) -- (7.29592,7.13432);
\definecolor{c}{rgb}{0,0,0};
\colorlet{c}{natcomp!50};
\draw [c] (7.37962,7.10495) -- (7.37962,7.11381);
\draw [c] (7.37962,7.11381) -- (7.37962,7.12234);
\draw [c] (7.29592,7.11381) -- (7.37962,7.11381);
\draw [c] (7.37962,7.11381) -- (7.46331,7.11381);
\definecolor{c}{rgb}{0,0,0};
\colorlet{c}{natcomp!50};
\draw [c] (7.54701,7.02158) -- (7.54701,7.03222);
\draw [c] (7.54701,7.03222) -- (7.54701,7.04238);
\draw [c] (7.46331,7.03222) -- (7.54701,7.03222);
\draw [c] (7.54701,7.03222) -- (7.6307,7.03222);
\definecolor{c}{rgb}{0,0,0};
\colorlet{c}{natcomp!50};
\draw [c] (7.7144,6.98395) -- (7.7144,6.99549);
\draw [c] (7.7144,6.99549) -- (7.7144,7.00648);
\draw [c] (7.6307,6.99549) -- (7.7144,6.99549);
\draw [c] (7.7144,6.99549) -- (7.79809,6.99549);
\definecolor{c}{rgb}{0,0,0};
\colorlet{c}{natcomp!50};
\draw [c] (7.88179,6.89516) -- (7.88179,6.90917);
\draw [c] (7.88179,6.90917) -- (7.88179,6.92238);
\draw [c] (7.79809,6.90917) -- (7.88179,6.90917);
\draw [c] (7.88179,6.90917) -- (7.96548,6.90917);
\definecolor{c}{rgb}{0,0,0};
\colorlet{c}{natcomp!50};
\draw [c] (8.04918,6.85534) -- (8.04918,6.87062);
\draw [c] (8.04918,6.87062) -- (8.04918,6.88495);
\draw [c] (7.96548,6.87062) -- (8.04918,6.87062);
\draw [c] (8.04918,6.87062) -- (8.13287,6.87062);
\definecolor{c}{rgb}{0,0,0};
\colorlet{c}{natcomp!50};
\draw [c] (8.21657,6.78028) -- (8.21657,6.79828);
\draw [c] (8.21657,6.79828) -- (8.21657,6.81497);
\draw [c] (8.13287,6.79828) -- (8.21657,6.79828);
\draw [c] (8.21657,6.79828) -- (8.30026,6.79828);
\definecolor{c}{rgb}{0,0,0};
\colorlet{c}{natcomp!50};
\draw [c] (8.38395,6.75589) -- (8.38395,6.77488);
\draw [c] (8.38395,6.77488) -- (8.38395,6.79241);
\draw [c] (8.30026,6.77488) -- (8.38395,6.77488);
\draw [c] (8.38395,6.77488) -- (8.46765,6.77488);
\definecolor{c}{rgb}{0,0,0};
\colorlet{c}{natcomp!50};
\draw [c] (8.55134,6.72) -- (8.55134,6.74054);
\draw [c] (8.55134,6.74054) -- (8.55134,6.75938);
\draw [c] (8.46765,6.74054) -- (8.55134,6.74054);
\draw [c] (8.55134,6.74054) -- (8.63504,6.74054);
\definecolor{c}{rgb}{0,0,0};
\colorlet{c}{natcomp!50};
\draw [c] (8.71873,6.59488) -- (8.71873,6.62185);
\draw [c] (8.71873,6.62185) -- (8.71873,6.64598);
\draw [c] (8.63504,6.62185) -- (8.71873,6.62185);
\draw [c] (8.71873,6.62185) -- (8.80243,6.62185);
\definecolor{c}{rgb}{0,0,0};
\colorlet{c}{natcomp!50};
\draw [c] (8.88612,6.58879) -- (8.88612,6.61613);
\draw [c] (8.88612,6.61613) -- (8.88612,6.64054);
\draw [c] (8.80243,6.61613) -- (8.88612,6.61613);
\draw [c] (8.88612,6.61613) -- (8.96982,6.61613);
\definecolor{c}{rgb}{0,0,0};
\colorlet{c}{natcomp!50};
\draw [c] (9.05351,6.54112) -- (9.05351,6.57145);
\draw [c] (9.05351,6.57145) -- (9.05351,6.59823);
\draw [c] (8.96982,6.57145) -- (9.05351,6.57145);
\draw [c] (9.05351,6.57145) -- (9.13721,6.57145);
\definecolor{c}{rgb}{0,0,0};
\colorlet{c}{natcomp!50};
\draw [c] (9.2209,6.45026) -- (9.2209,6.48723);
\draw [c] (9.2209,6.48723) -- (9.2209,6.51906);
\draw [c] (9.13721,6.48723) -- (9.2209,6.48723);
\draw [c] (9.2209,6.48723) -- (9.30459,6.48723);
\definecolor{c}{rgb}{0,0,0};
\colorlet{c}{natcomp};
\draw [c] (1.01883,9.82344) -- (1.01883,9.82401);
\draw [c] (1.01883,9.82401) -- (1.01883,9.82458);
\draw [c] (0.935135,9.82401) -- (1.01883,9.82401);
\draw [c] (1.01883,9.82401) -- (1.10252,9.82401);
\definecolor{c}{rgb}{0,0,0};
\colorlet{c}{natcomp};
\draw [c] (1.18622,9.41954) -- (1.18622,9.42091);
\draw [c] (1.18622,9.42091) -- (1.18622,9.42228);
\draw [c] (1.10252,9.42091) -- (1.18622,9.42091);
\draw [c] (1.18622,9.42091) -- (1.26991,9.42091);
\definecolor{c}{rgb}{0,0,0};
\colorlet{c}{natcomp};
\draw [c] (1.35361,8.94581) -- (1.35361,8.94968);
\draw [c] (1.35361,8.94968) -- (1.35361,8.95349);
\draw [c] (1.26991,8.94968) -- (1.35361,8.94968);
\draw [c] (1.35361,8.94968) -- (1.4373,8.94968);
\definecolor{c}{rgb}{0,0,0};
\colorlet{c}{natcomp};
\draw [c] (1.521,8.6247) -- (1.521,8.6325);
\draw [c] (1.521,8.6325) -- (1.521,8.64005);
\draw [c] (1.4373,8.6325) -- (1.521,8.6325);
\draw [c] (1.521,8.6325) -- (1.60469,8.6325);
\definecolor{c}{rgb}{0,0,0};
\colorlet{c}{natcomp};
\draw [c] (1.68839,8.39738) -- (1.68839,8.3982);
\draw [c] (1.68839,8.3982) -- (1.68839,8.39901);
\draw [c] (1.60469,8.3982) -- (1.68839,8.3982);
\draw [c] (1.68839,8.3982) -- (1.77208,8.3982);
\definecolor{c}{rgb}{0,0,0};
\colorlet{c}{natcomp};
\draw [c] (1.85578,8.20566) -- (1.85578,8.2069);
\draw [c] (1.85578,8.2069) -- (1.85578,8.20814);
\draw [c] (1.77208,8.2069) -- (1.85578,8.2069);
\draw [c] (1.85578,8.2069) -- (1.93947,8.2069);
\definecolor{c}{rgb}{0,0,0};
\colorlet{c}{natcomp};
\draw [c] (2.02316,8.05261) -- (2.02316,8.05434);
\draw [c] (2.02316,8.05434) -- (2.02316,8.05606);
\draw [c] (1.93947,8.05434) -- (2.02316,8.05434);
\draw [c] (2.02316,8.05434) -- (2.10686,8.05434);
\definecolor{c}{rgb}{0,0,0};
\colorlet{c}{natcomp};
\draw [c] (2.19055,7.91731) -- (2.19055,7.91963);
\draw [c] (2.19055,7.91963) -- (2.19055,7.92194);
\draw [c] (2.10686,7.91963) -- (2.19055,7.91963);
\draw [c] (2.19055,7.91963) -- (2.27425,7.91963);
\definecolor{c}{rgb}{0,0,0};
\colorlet{c}{natcomp};
\draw [c] (2.35794,7.81727) -- (2.35794,7.82017);
\draw [c] (2.35794,7.82017) -- (2.35794,7.82302);
\draw [c] (2.27425,7.82017) -- (2.35794,7.82017);
\draw [c] (2.35794,7.82017) -- (2.44164,7.82017);
\definecolor{c}{rgb}{0,0,0};
\colorlet{c}{natcomp};
\draw [c] (2.52533,7.72854) -- (2.52533,7.73205);
\draw [c] (2.52533,7.73205) -- (2.52533,7.73551);
\draw [c] (2.44164,7.73205) -- (2.52533,7.73205);
\draw [c] (2.52533,7.73205) -- (2.60903,7.73205);
\definecolor{c}{rgb}{0,0,0};
\colorlet{c}{natcomp};
\draw [c] (2.69272,7.66258) -- (2.69272,7.66504);
\draw [c] (2.69272,7.66504) -- (2.69272,7.66747);
\draw [c] (2.60903,7.66504) -- (2.69272,7.66504);
\draw [c] (2.69272,7.66504) -- (2.77642,7.66504);
\definecolor{c}{rgb}{0,0,0};
\colorlet{c}{natcomp};
\draw [c] (2.86011,7.61022) -- (2.86011,7.61297);
\draw [c] (2.86011,7.61297) -- (2.86011,7.6157);
\draw [c] (2.77642,7.61297) -- (2.86011,7.61297);
\draw [c] (2.86011,7.61297) -- (2.94381,7.61297);
\definecolor{c}{rgb}{0,0,0};
\colorlet{c}{natcomp};
\draw [c] (3.0275,7.57547) -- (3.0275,7.57844);
\draw [c] (3.0275,7.57844) -- (3.0275,7.58137);
\draw [c] (2.94381,7.57844) -- (3.0275,7.57844);
\draw [c] (3.0275,7.57844) -- (3.11119,7.57844);
\definecolor{c}{rgb}{0,0,0};
\colorlet{c}{natcomp};
\draw [c] (3.19489,7.54304) -- (3.19489,7.54622);
\draw [c] (3.19489,7.54622) -- (3.19489,7.54937);
\draw [c] (3.11119,7.54622) -- (3.19489,7.54622);
\draw [c] (3.19489,7.54622) -- (3.27858,7.54622);
\definecolor{c}{rgb}{0,0,0};
\colorlet{c}{natcomp};
\draw [c] (3.36228,7.52349) -- (3.36228,7.52682);
\draw [c] (3.36228,7.52682) -- (3.36228,7.5301);
\draw [c] (3.27858,7.52682) -- (3.36228,7.52682);
\draw [c] (3.36228,7.52682) -- (3.44597,7.52682);
\definecolor{c}{rgb}{0,0,0};
\colorlet{c}{natcomp};
\draw [c] (3.52967,7.50984) -- (3.52967,7.51327);
\draw [c] (3.52967,7.51327) -- (3.52967,7.51665);
\draw [c] (3.44597,7.51327) -- (3.52967,7.51327);
\draw [c] (3.52967,7.51327) -- (3.61336,7.51327);
\definecolor{c}{rgb}{0,0,0};
\colorlet{c}{natcomp};
\draw [c] (3.69706,7.49964) -- (3.69706,7.50314);
\draw [c] (3.69706,7.50314) -- (3.69706,7.50659);
\draw [c] (3.61336,7.50314) -- (3.69706,7.50314);
\draw [c] (3.69706,7.50314) -- (3.78075,7.50314);
\definecolor{c}{rgb}{0,0,0};
\colorlet{c}{natcomp};
\draw [c] (3.86445,7.48929) -- (3.86445,7.49287);
\draw [c] (3.86445,7.49287) -- (3.86445,7.4964);
\draw [c] (3.78075,7.49287) -- (3.86445,7.49287);
\draw [c] (3.86445,7.49287) -- (3.94814,7.49287);
\definecolor{c}{rgb}{0,0,0};
\colorlet{c}{natcomp};
\draw [c] (4.03184,7.47804) -- (4.03184,7.48172);
\draw [c] (4.03184,7.48172) -- (4.03184,7.48534);
\draw [c] (3.94814,7.48172) -- (4.03184,7.48172);
\draw [c] (4.03184,7.48172) -- (4.11553,7.48172);
\definecolor{c}{rgb}{0,0,0};
\colorlet{c}{natcomp};
\draw [c] (4.19922,7.46861) -- (4.19922,7.47236);
\draw [c] (4.19922,7.47236) -- (4.19922,7.47605);
\draw [c] (4.11553,7.47236) -- (4.19922,7.47236);
\draw [c] (4.19922,7.47236) -- (4.28292,7.47236);
\definecolor{c}{rgb}{0,0,0};
\colorlet{c}{natcomp};
\draw [c] (4.36661,7.46499) -- (4.36661,7.46877);
\draw [c] (4.36661,7.46877) -- (4.36661,7.47249);
\draw [c] (4.28292,7.46877) -- (4.36661,7.46877);
\draw [c] (4.36661,7.46877) -- (4.45031,7.46877);
\definecolor{c}{rgb}{0,0,0};
\colorlet{c}{natcomp};
\draw [c] (4.534,7.45376) -- (4.534,7.45763);
\draw [c] (4.534,7.45763) -- (4.534,7.46144);
\draw [c] (4.45031,7.45763) -- (4.534,7.45763);
\draw [c] (4.534,7.45763) -- (4.6177,7.45763);
\definecolor{c}{rgb}{0,0,0};
\colorlet{c}{natcomp};
\draw [c] (4.70139,7.43864) -- (4.70139,7.44265);
\draw [c] (4.70139,7.44265) -- (4.70139,7.44659);
\draw [c] (4.6177,7.44265) -- (4.70139,7.44265);
\draw [c] (4.70139,7.44265) -- (4.78509,7.44265);
\definecolor{c}{rgb}{0,0,0};
\colorlet{c}{natcomp};
\draw [c] (4.86878,7.42432) -- (4.86878,7.42846);
\draw [c] (4.86878,7.42846) -- (4.86878,7.43251);
\draw [c] (4.78509,7.42846) -- (4.86878,7.42846);
\draw [c] (4.86878,7.42846) -- (4.95248,7.42846);
\definecolor{c}{rgb}{0,0,0};
\colorlet{c}{natcomp};
\draw [c] (5.03617,7.40921) -- (5.03617,7.41348);
\draw [c] (5.03617,7.41348) -- (5.03617,7.41767);
\draw [c] (4.95248,7.41348) -- (5.03617,7.41348);
\draw [c] (5.03617,7.41348) -- (5.11986,7.41348);
\definecolor{c}{rgb}{0,0,0};
\colorlet{c}{natcomp};
\draw [c] (5.20356,7.39167) -- (5.20356,7.39611);
\draw [c] (5.20356,7.39611) -- (5.20356,7.40046);
\draw [c] (5.11986,7.39611) -- (5.20356,7.39611);
\draw [c] (5.20356,7.39611) -- (5.28725,7.39611);
\definecolor{c}{rgb}{0,0,0};
\colorlet{c}{natcomp};
\draw [c] (5.37095,7.37544) -- (5.37095,7.38004);
\draw [c] (5.37095,7.38004) -- (5.37095,7.38455);
\draw [c] (5.28725,7.38004) -- (5.37095,7.38004);
\draw [c] (5.37095,7.38004) -- (5.45464,7.38004);
\definecolor{c}{rgb}{0,0,0};
\colorlet{c}{natcomp};
\draw [c] (5.53834,7.357) -- (5.53834,7.36179);
\draw [c] (5.53834,7.36179) -- (5.53834,7.36647);
\draw [c] (5.45464,7.36179) -- (5.53834,7.36179);
\draw [c] (5.53834,7.36179) -- (5.62203,7.36179);
\definecolor{c}{rgb}{0,0,0};
\colorlet{c}{natcomp};
\draw [c] (5.70573,7.32848) -- (5.70573,7.33357);
\draw [c] (5.70573,7.33357) -- (5.70573,7.33855);
\draw [c] (5.62203,7.33357) -- (5.70573,7.33357);
\draw [c] (5.70573,7.33357) -- (5.78942,7.33357);
\definecolor{c}{rgb}{0,0,0};
\colorlet{c}{natcomp};
\draw [c] (5.87312,7.3054) -- (5.87312,7.31075);
\draw [c] (5.87312,7.31075) -- (5.87312,7.31599);
\draw [c] (5.78942,7.31075) -- (5.87312,7.31075);
\draw [c] (5.87312,7.31075) -- (5.95681,7.31075);
\definecolor{c}{rgb}{0,0,0};
\colorlet{c}{natcomp};
\draw [c] (6.04051,7.27428) -- (6.04051,7.28001);
\draw [c] (6.04051,7.28001) -- (6.04051,7.28561);
\draw [c] (5.95681,7.28001) -- (6.04051,7.28001);
\draw [c] (6.04051,7.28001) -- (6.1242,7.28001);
\definecolor{c}{rgb}{0,0,0};
\colorlet{c}{natcomp};
\draw [c] (6.20789,7.24692) -- (6.20789,7.253);
\draw [c] (6.20789,7.253) -- (6.20789,7.25893);
\draw [c] (6.1242,7.253) -- (6.20789,7.253);
\draw [c] (6.20789,7.253) -- (6.29159,7.253);
\definecolor{c}{rgb}{0,0,0};
\colorlet{c}{natcomp};
\draw [c] (6.37528,7.23037) -- (6.37528,7.23668);
\draw [c] (6.37528,7.23668) -- (6.37528,7.24282);
\draw [c] (6.29159,7.23668) -- (6.37528,7.23668);
\draw [c] (6.37528,7.23668) -- (6.45898,7.23668);
\definecolor{c}{rgb}{0,0,0};
\colorlet{c}{natcomp};
\draw [c] (6.54267,7.18876) -- (6.54267,7.19567);
\draw [c] (6.54267,7.19567) -- (6.54267,7.20238);
\draw [c] (6.45898,7.19567) -- (6.54267,7.19567);
\draw [c] (6.54267,7.19567) -- (6.62637,7.19567);
\definecolor{c}{rgb}{0,0,0};
\colorlet{c}{natcomp};
\draw [c] (6.71006,7.16223) -- (6.71006,7.16955);
\draw [c] (6.71006,7.16955) -- (6.71006,7.17665);
\draw [c] (6.62637,7.16955) -- (6.71006,7.16955);
\draw [c] (6.71006,7.16955) -- (6.79376,7.16955);
\definecolor{c}{rgb}{0,0,0};
\colorlet{c}{natcomp};
\draw [c] (6.87745,7.13589) -- (6.87745,7.14365);
\draw [c] (6.87745,7.14365) -- (6.87745,7.15115);
\draw [c] (6.79376,7.14365) -- (6.87745,7.14365);
\draw [c] (6.87745,7.14365) -- (6.96115,7.14365);
\definecolor{c}{rgb}{0,0,0};
\colorlet{c}{natcomp};
\draw [c] (7.04484,7.07306) -- (7.04484,7.08196);
\draw [c] (7.04484,7.08196) -- (7.04484,7.09052);
\draw [c] (6.96115,7.08196) -- (7.04484,7.08196);
\draw [c] (7.04484,7.08196) -- (7.12854,7.08196);
\definecolor{c}{rgb}{0,0,0};
\colorlet{c}{natcomp};
\draw [c] (7.21223,7.02584) -- (7.21223,7.0357);
\draw [c] (7.21223,7.0357) -- (7.21223,7.04516);
\draw [c] (7.12854,7.0357) -- (7.21223,7.0357);
\draw [c] (7.21223,7.0357) -- (7.29592,7.0357);
\definecolor{c}{rgb}{0,0,0};
\colorlet{c}{natcomp};
\draw [c] (7.37962,6.96483) -- (7.37962,6.9761);
\draw [c] (7.37962,6.9761) -- (7.37962,6.98684);
\draw [c] (7.29592,6.9761) -- (7.37962,6.9761);
\draw [c] (7.37962,6.9761) -- (7.46331,6.9761);
\definecolor{c}{rgb}{0,0,0};
\colorlet{c}{natcomp};
\draw [c] (7.54701,6.92986) -- (7.54701,6.94202);
\draw [c] (7.54701,6.94202) -- (7.54701,6.95357);
\draw [c] (7.46331,6.94202) -- (7.54701,6.94202);
\draw [c] (7.54701,6.94202) -- (7.6307,6.94202);
\definecolor{c}{rgb}{0,0,0};
\colorlet{c}{natcomp};
\draw [c] (7.7144,6.87712) -- (7.7144,6.89076);
\draw [c] (7.7144,6.89076) -- (7.7144,6.90364);
\draw [c] (7.6307,6.89076) -- (7.7144,6.89076);
\draw [c] (7.7144,6.89076) -- (7.79809,6.89076);
\definecolor{c}{rgb}{0,0,0};
\colorlet{c}{natcomp};
\draw [c] (7.88179,6.82711) -- (7.88179,6.84232);
\draw [c] (7.88179,6.84232) -- (7.88179,6.85659);
\draw [c] (7.79809,6.84232) -- (7.88179,6.84232);
\draw [c] (7.88179,6.84232) -- (7.96548,6.84232);
\definecolor{c}{rgb}{0,0,0};
\colorlet{c}{natcomp};
\draw [c] (8.04918,6.76967) -- (8.04918,6.78692);
\draw [c] (8.04918,6.78692) -- (8.04918,6.80296);
\draw [c] (7.96548,6.78692) -- (8.04918,6.78692);
\draw [c] (8.04918,6.78692) -- (8.13287,6.78692);
\definecolor{c}{rgb}{0,0,0};
\colorlet{c}{natcomp};
\draw [c] (8.21657,6.73892) -- (8.21657,6.75737);
\draw [c] (8.21657,6.75737) -- (8.21657,6.77444);
\draw [c] (8.13287,6.75737) -- (8.21657,6.75737);
\draw [c] (8.21657,6.75737) -- (8.30026,6.75737);
\definecolor{c}{rgb}{0,0,0};
\colorlet{c}{natcomp};
\draw [c] (8.38395,6.63659) -- (8.38395,6.65965);
\draw [c] (8.38395,6.65965) -- (8.38395,6.6806);
\draw [c] (8.30026,6.65965) -- (8.38395,6.65965);
\draw [c] (8.38395,6.65965) -- (8.46765,6.65965);
\definecolor{c}{rgb}{0,0,0};
\colorlet{c}{natcomp};
\draw [c] (8.55134,6.59027) -- (8.55134,6.61578);
\draw [c] (8.55134,6.61578) -- (8.55134,6.63874);
\draw [c] (8.46765,6.61578) -- (8.55134,6.61578);
\draw [c] (8.55134,6.61578) -- (8.63504,6.61578);
\definecolor{c}{rgb}{0,0,0};
\colorlet{c}{natcomp};
\draw [c] (8.71873,6.48713) -- (8.71873,6.51907);
\draw [c] (8.71873,6.51907) -- (8.71873,6.5471);
\draw [c] (8.63504,6.51907) -- (8.71873,6.51907);
\draw [c] (8.71873,6.51907) -- (8.80243,6.51907);
\definecolor{c}{rgb}{0,0,0};
\colorlet{c}{natcomp};
\draw [c] (8.88612,6.45596) -- (8.88612,6.49015);
\draw [c] (8.88612,6.49015) -- (8.88612,6.51989);
\draw [c] (8.80243,6.49015) -- (8.88612,6.49015);
\draw [c] (8.88612,6.49015) -- (8.96982,6.49015);
\definecolor{c}{rgb}{0,0,0};
\colorlet{c}{natcomp};
\draw [c] (9.05351,6.37777) -- (9.05351,6.41831);
\draw [c] (9.05351,6.41831) -- (9.05351,6.45274);
\draw [c] (8.96982,6.41831) -- (9.05351,6.41831);
\draw [c] (9.05351,6.41831) -- (9.13721,6.41831);
\definecolor{c}{rgb}{0,0,0};
\colorlet{c}{natcomp};
\draw [c] (9.2209,6.24976) -- (9.2209,6.30332);
\draw [c] (9.2209,6.30332) -- (9.2209,6.34669);
\draw [c] (9.13721,6.30332) -- (9.2209,6.30332);
\draw [c] (9.2209,6.30332) -- (9.30459,6.30332);
\definecolor{c}{rgb}{0,0,0};
\definecolor{c}{rgb}{1,1,1};
\draw [color=c, fill=c] (3.74054,7.94286) rectangle (8.88378,9.89714);
\definecolor{c}{rgb}{0,0,0};
\draw [anchor= west] (4.06199,9.65286) node[color=c, rotate=0]{MRST2002};
\draw [anchor= west] (5.6413,9.65286) node[color=c, rotate=0]{CTEQ6};
\draw [anchor= west] (6.80133,9.65286) node[color=c, rotate=0]{};
\draw [anchor= west] (7.12278,9.65286) node[color=c, rotate=0]{};
\draw [anchor= west] (4.06199,9.16429) node[color=c, rotate=0]{};
\draw [anchor= west] (5.6413,9.16429) node[color=c, rotate=0]{};
\colorlet{c}{black};
\draw [c] (5.36807,9.16429) -- (5.59309,9.16429);
\draw [c] (5.48058,9.01771) -- (5.48058,9.31086);
\definecolor{c}{rgb}{0,0,0};
\draw [anchor= west] (6.80133,9.16429) node[color=c, rotate=0]{};
\colorlet{c}{kugray!20};
\draw [c, fill=c] (6.52809,8.99329) -- (6.75311,8.99329) -- (6.75311,9.33529) -- (6.52809,9.33529);
\colorlet{c}{kugray};
\draw [c] (6.52809,9.16429) -- (6.75311,9.16429);
\definecolor{c}{rgb}{0,0,0};
\draw [anchor= west] (7.12278,9.16429) node[color=c, rotate=0]{Standard Model};
\draw [anchor= west] (4.06199,8.67571) node[color=c, rotate=0]{};
\draw [anchor= west] (5.6413,8.67571) node[color=c, rotate=0]{};
\colorlet{c}{natgreen};
\draw [c] (5.36807,8.67571) -- (5.59309,8.67571);
\draw [c] (5.48058,8.52914) -- (5.48058,8.82229);
\definecolor{c}{rgb}{0,0,0};
\draw [anchor= west] (6.80133,8.67571) node[color=c, rotate=0]{};
\colorlet{c}{natgreen!20};
\draw [c, fill=c] (6.52809,8.50471) -- (6.75311,8.50471) -- (6.75311,8.84671) -- (6.52809,8.84671);
\colorlet{c}{natgreen!50};
\draw [c] (6.52809,8.67571) -- (6.75311,8.67571);
\definecolor{c}{rgb}{0,0,0};
\draw [anchor= west] (7.12278,8.67571) node[color=c, rotate=0]{$\Lambda = 1.00$ TeV};
\draw [anchor= west] (4.06199,8.18714) node[color=c, rotate=0]{};
\draw [anchor= west] (5.6413,8.18714) node[color=c, rotate=0]{};
\colorlet{c}{natcomp};
\draw [c] (5.36807,8.18714) -- (5.59309,8.18714);
\draw [c] (5.48058,8.04057) -- (5.48058,8.33371);
\definecolor{c}{rgb}{0,0,0};
\draw [anchor= west] (6.80133,8.18714) node[color=c, rotate=0]{};
\colorlet{c}{natcomp!50};
\draw [c, fill=c] (6.52809,8.01614) -- (6.75311,8.01614) -- (6.75311,8.35814) -- (6.52809,8.35814);
\colorlet{c}{natcomp!50};
\draw [c] (6.52809,8.18714) -- (6.75311,8.18714);
\definecolor{c}{rgb}{0,0,0};
\draw [anchor= west] (7.12278,8.18714) node[color=c, rotate=0]{$\Lambda = 0.75$ TeV};
\definecolor{c}{rgb}{1,1,1};
\draw [color=c, fill=c] (0,3.42857) rectangle (9.35135,4.85714);
\draw [color=c, fill=c] (0.935135,3.42857) rectangle (9.30459,4.85571);
\definecolor{c}{rgb}{0,0,0};
\draw [c] (0.935135,3.42857) -- (0.935135,4.85571) -- (9.30459,4.85571) -- (9.30459,3.42857) -- (0.935135,3.42857);
\definecolor{c}{rgb}{1,1,1};
\draw [color=c, fill=c] (0.935135,3.42857) rectangle (9.30459,4.85571);
\definecolor{c}{rgb}{0,0,0};
\draw [c] (0.935135,3.42857) -- (0.935135,4.85571) -- (9.30459,4.85571) -- (9.30459,3.42857) -- (0.935135,3.42857);
\colorlet{c}{kugray!20};
\draw [color=c, fill=c] (0.935135,4.13308) rectangle (1.10252,4.14797);
\definecolor{c}{rgb}{0,0,0};
\colorlet{c}{kugray!20};
\draw [color=c, fill=c] (1.10252,4.1378) rectangle (1.26991,4.14325);
\definecolor{c}{rgb}{0,0,0};
\colorlet{c}{kugray!20};
\draw [color=c, fill=c] (1.26991,4.13261) rectangle (1.4373,4.14844);
\definecolor{c}{rgb}{0,0,0};
\colorlet{c}{kugray!20};
\draw [color=c, fill=c] (1.4373,4.12494) rectangle (1.60469,4.15611);
\definecolor{c}{rgb}{0,0,0};
\colorlet{c}{kugray!20};
\draw [color=c, fill=c] (1.60469,4.11412) rectangle (1.77208,4.16693);
\definecolor{c}{rgb}{0,0,0};
\colorlet{c}{kugray!20};
\draw [color=c, fill=c] (1.77208,4.10077) rectangle (1.93947,4.18028);
\definecolor{c}{rgb}{0,0,0};
\colorlet{c}{kugray!20};
\draw [color=c, fill=c] (1.93947,4.08419) rectangle (2.10686,4.19686);
\definecolor{c}{rgb}{0,0,0};
\colorlet{c}{kugray!20};
\draw [color=c, fill=c] (2.10686,4.05554) rectangle (2.27425,4.22551);
\definecolor{c}{rgb}{0,0,0};
\colorlet{c}{kugray!20};
\draw [color=c, fill=c] (2.27425,4.02611) rectangle (2.44164,4.25494);
\definecolor{c}{rgb}{0,0,0};
\colorlet{c}{kugray!20};
\draw [color=c, fill=c] (2.44164,4.00557) rectangle (2.60903,4.27548);
\definecolor{c}{rgb}{0,0,0};
\colorlet{c}{kugray!20};
\draw [color=c, fill=c] (2.60903,4.1364) rectangle (2.77642,4.14465);
\definecolor{c}{rgb}{0,0,0};
\colorlet{c}{kugray!20};
\draw [color=c, fill=c] (2.77642,4.13528) rectangle (2.94381,4.14577);
\definecolor{c}{rgb}{0,0,0};
\colorlet{c}{kugray!20};
\draw [color=c, fill=c] (2.94381,4.13387) rectangle (3.11119,4.14718);
\definecolor{c}{rgb}{0,0,0};
\colorlet{c}{kugray!20};
\draw [color=c, fill=c] (3.11119,4.13216) rectangle (3.27858,4.14889);
\definecolor{c}{rgb}{0,0,0};
\colorlet{c}{kugray!20};
\draw [color=c, fill=c] (3.27858,4.13007) rectangle (3.44597,4.15098);
\definecolor{c}{rgb}{0,0,0};
\colorlet{c}{kugray!20};
\draw [color=c, fill=c] (3.44597,4.12792) rectangle (3.61336,4.15313);
\definecolor{c}{rgb}{0,0,0};
\colorlet{c}{kugray!20};
\draw [color=c, fill=c] (3.61336,4.12488) rectangle (3.78075,4.15617);
\definecolor{c}{rgb}{0,0,0};
\colorlet{c}{kugray!20};
\draw [color=c, fill=c] (3.78075,4.12198) rectangle (3.94814,4.15907);
\definecolor{c}{rgb}{0,0,0};
\colorlet{c}{kugray!20};
\draw [color=c, fill=c] (3.94814,4.11821) rectangle (4.11553,4.16284);
\definecolor{c}{rgb}{0,0,0};
\colorlet{c}{kugray!20};
\draw [color=c, fill=c] (4.11553,4.11217) rectangle (4.28292,4.16888);
\definecolor{c}{rgb}{0,0,0};
\colorlet{c}{kugray!20};
\draw [color=c, fill=c] (4.28292,4.104) rectangle (4.45031,4.17705);
\definecolor{c}{rgb}{0,0,0};
\colorlet{c}{kugray!20};
\draw [color=c, fill=c] (4.45031,4.0951) rectangle (4.6177,4.18595);
\definecolor{c}{rgb}{0,0,0};
\colorlet{c}{kugray!20};
\draw [color=c, fill=c] (4.6177,4.0824) rectangle (4.78509,4.19865);
\definecolor{c}{rgb}{0,0,0};
\colorlet{c}{kugray!20};
\draw [color=c, fill=c] (4.78509,4.07937) rectangle (4.95248,4.20168);
\definecolor{c}{rgb}{0,0,0};
\colorlet{c}{kugray!20};
\draw [color=c, fill=c] (4.95248,4.06261) rectangle (5.11986,4.21844);
\definecolor{c}{rgb}{0,0,0};
\colorlet{c}{kugray!20};
\draw [color=c, fill=c] (5.11986,4.04806) rectangle (5.28725,4.23299);
\definecolor{c}{rgb}{0,0,0};
\colorlet{c}{kugray!20};
\draw [color=c, fill=c] (5.28725,4.02034) rectangle (5.45464,4.26071);
\definecolor{c}{rgb}{0,0,0};
\colorlet{c}{kugray!20};
\draw [color=c, fill=c] (5.45464,3.98797) rectangle (5.62203,4.29308);
\definecolor{c}{rgb}{0,0,0};
\colorlet{c}{kugray!20};
\draw [color=c, fill=c] (5.62203,3.98797) rectangle (5.78942,4.29308);
\definecolor{c}{rgb}{0,0,0};
\colorlet{c}{kugray!20};
\draw [color=c, fill=c] (5.78942,3.95369) rectangle (5.95681,4.32736);
\definecolor{c}{rgb}{0,0,0};
\colorlet{c}{kugray!20};
\draw [color=c, fill=c] (5.95681,3.94538) rectangle (6.1242,4.33567);
\definecolor{c}{rgb}{0,0,0};
\colorlet{c}{kugray!20};
\draw [color=c, fill=c] (6.1242,3.87629) rectangle (6.29159,4.40476);
\definecolor{c}{rgb}{0,0,0};
\colorlet{c}{kugray!20};
\draw [color=c, fill=c] (6.29159,3.95369) rectangle (6.45898,4.32736);
\definecolor{c}{rgb}{0,0,0};
\colorlet{c}{kugray!20};
\draw [color=c, fill=c] (6.45898,3.76685) rectangle (6.62637,4.5142);
\definecolor{c}{rgb}{0,0,0};
\colorlet{c}{kugray!20};
\draw [color=c, fill=c] (6.62637,3.49329) rectangle (6.79376,4.78776);
\definecolor{c}{rgb}{0,0,0};
\colorlet{c}{kugray!20};
\draw [color=c, fill=c] (6.79376,3.49329) rectangle (6.96115,4.78776);
\definecolor{c}{rgb}{0,0,0};
\colorlet{c}{kugray!20};
\draw [color=c, fill=c] (6.96115,3.68286) rectangle (7.12854,4.59819);
\definecolor{c}{rgb}{0,0,0};
\colorlet{c}{kugray!20};
\draw [color=c, fill=c] (7.12854,4.14052) rectangle (7.29592,4.14052);
\definecolor{c}{rgb}{0,0,0};
\colorlet{c}{kugray!20};
\draw [color=c, fill=c] (7.29592,3.68286) rectangle (7.46331,4.59819);
\definecolor{c}{rgb}{0,0,0};
\colorlet{c}{kugray!20};
\draw [color=c, fill=c] (7.46331,3.49329) rectangle (7.6307,4.78776);
\definecolor{c}{rgb}{0,0,0};
\colorlet{c}{kugray!20};
\draw [color=c, fill=c] (7.6307,4.14052) rectangle (7.79809,4.14052);
\definecolor{c}{rgb}{0,0,0};
\colorlet{c}{kugray!20};
\draw [color=c, fill=c] (7.79809,3.49329) rectangle (7.96548,4.78776);
\definecolor{c}{rgb}{0,0,0};
\colorlet{c}{kugray!20};
\draw [color=c, fill=c] (7.96548,4.14052) rectangle (8.13287,4.14052);
\definecolor{c}{rgb}{0,0,0};
\colorlet{c}{kugray!20};
\draw [color=c, fill=c] (8.13287,4.14052) rectangle (8.30026,4.14052);
\definecolor{c}{rgb}{0,0,0};
\colorlet{c}{kugray!20};
\draw [color=c, fill=c] (8.30026,4.14052) rectangle (8.46765,4.14052);
\definecolor{c}{rgb}{0,0,0};
\colorlet{c}{kugray!20};
\draw [color=c, fill=c] (8.46765,4.14052) rectangle (8.63504,4.14052);
\definecolor{c}{rgb}{0,0,0};
\colorlet{c}{kugray!20};
\draw [color=c, fill=c] (8.63504,4.14052) rectangle (8.80243,4.14052);
\definecolor{c}{rgb}{0,0,0};
\colorlet{c}{kugray!20};
\draw [color=c, fill=c] (8.80243,4.14052) rectangle (8.96982,4.14052);
\definecolor{c}{rgb}{0,0,0};
\colorlet{c}{kugray!20};
\draw [color=c, fill=c] (8.96982,4.14052) rectangle (9.13721,4.14052);
\definecolor{c}{rgb}{0,0,0};
\colorlet{c}{kugray!20};
\draw [color=c, fill=c] (9.13721,4.14052) rectangle (9.30459,4.14052);
\definecolor{c}{rgb}{0,0,0};
\draw [c] (0.935135,3.42857) -- (9.30459,3.42857);
\draw [c] (0.935135,3.46693) -- (0.935135,3.42857);
\draw [c] (1.10252,3.43816) -- (1.10252,3.42857);
\draw [c] (1.26991,3.43816) -- (1.26991,3.42857);
\draw [c] (1.4373,3.43816) -- (1.4373,3.42857);
\draw [c] (1.60469,3.43816) -- (1.60469,3.42857);
\draw [c] (1.77208,3.44775) -- (1.77208,3.42857);
\draw [c] (1.93947,3.43816) -- (1.93947,3.42857);
\draw [c] (2.10686,3.43816) -- (2.10686,3.42857);
\draw [c] (2.27425,3.43816) -- (2.27425,3.42857);
\draw [c] (2.44164,3.43816) -- (2.44164,3.42857);
\draw [c] (2.60903,3.44775) -- (2.60903,3.42857);
\draw [c] (2.77642,3.43816) -- (2.77642,3.42857);
\draw [c] (2.94381,3.43816) -- (2.94381,3.42857);
\draw [c] (3.11119,3.43816) -- (3.11119,3.42857);
\draw [c] (3.27858,3.43816) -- (3.27858,3.42857);
\draw [c] (3.44597,3.44775) -- (3.44597,3.42857);
\draw [c] (3.61336,3.43816) -- (3.61336,3.42857);
\draw [c] (3.78075,3.43816) -- (3.78075,3.42857);
\draw [c] (3.94814,3.43816) -- (3.94814,3.42857);
\draw [c] (4.11553,3.43816) -- (4.11553,3.42857);
\draw [c] (4.28292,3.44775) -- (4.28292,3.42857);
\draw [c] (4.45031,3.43816) -- (4.45031,3.42857);
\draw [c] (4.6177,3.43816) -- (4.6177,3.42857);
\draw [c] (4.78509,3.43816) -- (4.78509,3.42857);
\draw [c] (4.95248,3.43816) -- (4.95248,3.42857);
\draw [c] (5.11986,3.44775) -- (5.11986,3.42857);
\draw [c] (5.28725,3.43816) -- (5.28725,3.42857);
\draw [c] (5.45464,3.43816) -- (5.45464,3.42857);
\draw [c] (5.62203,3.43816) -- (5.62203,3.42857);
\draw [c] (5.78942,3.43816) -- (5.78942,3.42857);
\draw [c] (5.95681,3.44775) -- (5.95681,3.42857);
\draw [c] (6.1242,3.43816) -- (6.1242,3.42857);
\draw [c] (6.29159,3.43816) -- (6.29159,3.42857);
\draw [c] (6.45898,3.43816) -- (6.45898,3.42857);
\draw [c] (6.62637,3.43816) -- (6.62637,3.42857);
\draw [c] (6.79376,3.44775) -- (6.79376,3.42857);
\draw [c] (6.96115,3.43816) -- (6.96115,3.42857);
\draw [c] (7.12854,3.43816) -- (7.12854,3.42857);
\draw [c] (7.29592,3.43816) -- (7.29592,3.42857);
\draw [c] (7.46331,3.43816) -- (7.46331,3.42857);
\draw [c] (7.6307,3.44775) -- (7.6307,3.42857);
\draw [c] (7.79809,3.43816) -- (7.79809,3.42857);
\draw [c] (7.96548,3.43816) -- (7.96548,3.42857);
\draw [c] (8.13287,3.43816) -- (8.13287,3.42857);
\draw [c] (8.30026,3.43816) -- (8.30026,3.42857);
\draw [c] (8.46765,3.44775) -- (8.46765,3.42857);
\draw [c] (8.63504,3.43816) -- (8.63504,3.42857);
\draw [c] (8.80243,3.43816) -- (8.80243,3.42857);
\draw [c] (8.96982,3.43816) -- (8.96982,3.42857);
\draw [c] (9.13721,3.43816) -- (9.13721,3.42857);
\draw [c] (9.30459,3.46693) -- (9.30459,3.42857);
\draw [anchor=base] (0.935135,3.27429) node[color=c, rotate=0]{0};
\draw [anchor=base] (9.30459,3.27429) node[color=c, rotate=0]{5000};
\draw [c] (0.935135,3.42857) -- (0.935135,4.85571);
\draw [anchor= east] (-0.112216,4.85571) node[color=c, rotate=90]{Ratio};
\draw [c] (1.2154,3.68286) -- (0.935135,3.68286);
\draw [c] (1.07527,3.91169) -- (0.935135,3.91169);
\draw [c] (1.07527,4.14052) -- (0.935135,4.14052);
\draw [c] (1.07527,4.36936) -- (0.935135,4.36936);
\draw [c] (1.2154,4.59819) -- (0.935135,4.59819);
\draw [c] (1.2154,3.68286) -- (0.935135,3.68286);
\draw [c] (1.07527,3.45403) -- (0.935135,3.45403);
\draw [c] (1.2154,4.59819) -- (0.935135,4.59819);
\draw [c] (1.07527,4.82702) -- (0.935135,4.82702);
\draw [anchor= east] (0.888378,3.68286) node[color=c, rotate=0]{0};
\draw [anchor= east] (0.888378,4.59819) node[color=c, rotate=0]{2};
\colorlet{c}{black};
\draw [c] (1.01883,4.18545) -- (1.01883,4.19145);
\draw [c] (1.01883,4.19145) -- (1.01883,4.19744);
\draw [c] (0.935135,4.19145) -- (1.01883,4.19145);
\draw [c] (1.01883,4.19145) -- (1.10252,4.19145);
\definecolor{c}{rgb}{0,0,0};
\colorlet{c}{black};
\draw [c] (1.18622,4.21047) -- (1.18622,4.2145);
\draw [c] (1.18622,4.2145) -- (1.18622,4.21852);
\draw [c] (1.10252,4.2145) -- (1.18622,4.2145);
\draw [c] (1.18622,4.2145) -- (1.26991,4.2145);
\definecolor{c}{rgb}{0,0,0};
\colorlet{c}{black};
\draw [c] (1.35361,4.22898) -- (1.35361,4.24106);
\draw [c] (1.35361,4.24106) -- (1.35361,4.25313);
\draw [c] (1.26991,4.24106) -- (1.35361,4.24106);
\draw [c] (1.35361,4.24106) -- (1.4373,4.24106);
\definecolor{c}{rgb}{0,0,0};
\colorlet{c}{black};
\draw [c] (1.521,4.20049) -- (1.521,4.22377);
\draw [c] (1.521,4.22377) -- (1.521,4.24704);
\draw [c] (1.4373,4.22377) -- (1.521,4.22377);
\draw [c] (1.521,4.22377) -- (1.60469,4.22377);
\definecolor{c}{rgb}{0,0,0};
\colorlet{c}{black};
\draw [c] (1.68839,4.21218) -- (1.68839,4.23478);
\draw [c] (1.68839,4.23478) -- (1.68839,4.25737);
\draw [c] (1.60469,4.23478) -- (1.68839,4.23478);
\draw [c] (1.68839,4.23478) -- (1.77208,4.23478);
\definecolor{c}{rgb}{0,0,0};
\colorlet{c}{black};
\draw [c] (1.85578,4.17362) -- (1.85578,4.20587);
\draw [c] (1.85578,4.20587) -- (1.85578,4.23813);
\draw [c] (1.77208,4.20587) -- (1.85578,4.20587);
\draw [c] (1.85578,4.20587) -- (1.93947,4.20587);
\definecolor{c}{rgb}{0,0,0};
\colorlet{c}{black};
\draw [c] (2.02316,4.13472) -- (2.02316,4.17799);
\draw [c] (2.02316,4.17799) -- (2.02316,4.22126);
\draw [c] (1.93947,4.17799) -- (2.02316,4.17799);
\draw [c] (2.02316,4.17799) -- (2.10686,4.17799);
\definecolor{c}{rgb}{0,0,0};
\colorlet{c}{black};
\draw [c] (2.19055,4.17383) -- (2.19055,4.24835);
\draw [c] (2.19055,4.24835) -- (2.19055,4.32287);
\draw [c] (2.10686,4.24835) -- (2.19055,4.24835);
\draw [c] (2.19055,4.24835) -- (2.27425,4.24835);
\definecolor{c}{rgb}{0,0,0};
\colorlet{c}{black};
\draw [c] (2.35794,4.15146) -- (2.35794,4.25252);
\draw [c] (2.35794,4.25252) -- (2.35794,4.35358);
\draw [c] (2.27425,4.25252) -- (2.35794,4.25252);
\draw [c] (2.35794,4.25252) -- (2.44164,4.25252);
\definecolor{c}{rgb}{0,0,0};
\colorlet{c}{black};
\draw [c] (2.52533,4.03313) -- (2.52533,4.12594);
\draw [c] (2.52533,4.12594) -- (2.52533,4.21875);
\draw [c] (2.44164,4.12594) -- (2.52533,4.12594);
\draw [c] (2.52533,4.12594) -- (2.60903,4.12594);
\definecolor{c}{rgb}{0,0,0};
\colorlet{c}{black};
\draw [c] (2.69272,4.16426) -- (2.69272,4.16803);
\draw [c] (2.69272,4.16803) -- (2.69272,4.17181);
\draw [c] (2.60903,4.16803) -- (2.69272,4.16803);
\draw [c] (2.69272,4.16803) -- (2.77642,4.16803);
\definecolor{c}{rgb}{0,0,0};
\colorlet{c}{black};
\draw [c] (2.86011,4.1538) -- (2.86011,4.15852);
\draw [c] (2.86011,4.15852) -- (2.86011,4.16324);
\draw [c] (2.77642,4.15852) -- (2.86011,4.15852);
\draw [c] (2.86011,4.15852) -- (2.94381,4.15852);
\definecolor{c}{rgb}{0,0,0};
\colorlet{c}{black};
\draw [c] (3.0275,4.13739) -- (3.0275,4.14322);
\draw [c] (3.0275,4.14322) -- (3.0275,4.14905);
\draw [c] (2.94381,4.14322) -- (3.0275,4.14322);
\draw [c] (3.0275,4.14322) -- (3.11119,4.14322);
\definecolor{c}{rgb}{0,0,0};
\colorlet{c}{black};
\draw [c] (3.19489,4.13724) -- (3.19489,4.14459);
\draw [c] (3.19489,4.14459) -- (3.19489,4.15194);
\draw [c] (3.11119,4.14459) -- (3.19489,4.14459);
\draw [c] (3.19489,4.14459) -- (3.27858,4.14459);
\definecolor{c}{rgb}{0,0,0};
\colorlet{c}{black};
\draw [c] (3.36228,4.14229) -- (3.36228,4.15159);
\draw [c] (3.36228,4.15159) -- (3.36228,4.16088);
\draw [c] (3.27858,4.15159) -- (3.36228,4.15159);
\draw [c] (3.36228,4.15159) -- (3.44597,4.15159);
\definecolor{c}{rgb}{0,0,0};
\colorlet{c}{black};
\draw [c] (3.52967,4.12256) -- (3.52967,4.13341);
\draw [c] (3.52967,4.13341) -- (3.52967,4.14426);
\draw [c] (3.44597,4.13341) -- (3.52967,4.13341);
\draw [c] (3.52967,4.13341) -- (3.61336,4.13341);
\definecolor{c}{rgb}{0,0,0};
\colorlet{c}{black};
\draw [c] (3.69706,4.11553) -- (3.69706,4.12889);
\draw [c] (3.69706,4.12889) -- (3.69706,4.14224);
\draw [c] (3.61336,4.12889) -- (3.69706,4.12889);
\draw [c] (3.69706,4.12889) -- (3.78075,4.12889);
\definecolor{c}{rgb}{0,0,0};
\colorlet{c}{black};
\draw [c] (3.86445,4.07609) -- (3.86445,4.0908);
\draw [c] (3.86445,4.0908) -- (3.86445,4.10551);
\draw [c] (3.78075,4.0908) -- (3.86445,4.0908);
\draw [c] (3.86445,4.0908) -- (3.94814,4.0908);
\definecolor{c}{rgb}{0,0,0};
\colorlet{c}{black};
\draw [c] (4.03184,4.07563) -- (4.03184,4.09343);
\draw [c] (4.03184,4.09343) -- (4.03184,4.11122);
\draw [c] (3.94814,4.09343) -- (4.03184,4.09343);
\draw [c] (4.03184,4.09343) -- (4.11553,4.09343);
\definecolor{c}{rgb}{0,0,0};
\colorlet{c}{black};
\draw [c] (4.19922,4.03074) -- (4.19922,4.05146);
\draw [c] (4.19922,4.05146) -- (4.19922,4.07217);
\draw [c] (4.11553,4.05146) -- (4.19922,4.05146);
\draw [c] (4.19922,4.05146) -- (4.28292,4.05146);
\definecolor{c}{rgb}{0,0,0};
\colorlet{c}{black};
\draw [c] (4.36661,4.04664) -- (4.36661,4.07467);
\draw [c] (4.36661,4.07467) -- (4.36661,4.1027);
\draw [c] (4.28292,4.07467) -- (4.36661,4.07467);
\draw [c] (4.36661,4.07467) -- (4.45031,4.07467);
\definecolor{c}{rgb}{0,0,0};
\colorlet{c}{black};
\draw [c] (4.534,4.06783) -- (4.534,4.10487);
\draw [c] (4.534,4.10487) -- (4.534,4.14191);
\draw [c] (4.45031,4.10487) -- (4.534,4.10487);
\draw [c] (4.534,4.10487) -- (4.6177,4.10487);
\definecolor{c}{rgb}{0,0,0};
\colorlet{c}{black};
\draw [c] (4.70139,4.13434) -- (4.70139,4.18949);
\draw [c] (4.70139,4.18949) -- (4.70139,4.24465);
\draw [c] (4.6177,4.18949) -- (4.70139,4.18949);
\draw [c] (4.70139,4.18949) -- (4.78509,4.18949);
\definecolor{c}{rgb}{0,0,0};
\colorlet{c}{black};
\draw [c] (4.86878,4.06532) -- (4.86878,4.1163);
\draw [c] (4.86878,4.1163) -- (4.86878,4.16727);
\draw [c] (4.78509,4.1163) -- (4.86878,4.1163);
\draw [c] (4.86878,4.1163) -- (4.95248,4.1163);
\definecolor{c}{rgb}{0,0,0};
\colorlet{c}{black};
\draw [c] (5.03617,4.08559) -- (5.03617,4.15534);
\draw [c] (5.03617,4.15534) -- (5.03617,4.22509);
\draw [c] (4.95248,4.15534) -- (5.03617,4.15534);
\draw [c] (5.03617,4.15534) -- (5.11986,4.15534);
\definecolor{c}{rgb}{0,0,0};
\colorlet{c}{black};
\draw [c] (5.20356,4.04234) -- (5.20356,4.11994);
\draw [c] (5.20356,4.11994) -- (5.20356,4.19754);
\draw [c] (5.11986,4.11994) -- (5.20356,4.11994);
\draw [c] (5.20356,4.11994) -- (5.28725,4.11994);
\definecolor{c}{rgb}{0,0,0};
\colorlet{c}{black};
\draw [c] (5.37095,4.15705) -- (5.37095,4.29008);
\draw [c] (5.37095,4.29008) -- (5.37095,4.42311);
\draw [c] (5.28725,4.29008) -- (5.37095,4.29008);
\draw [c] (5.37095,4.29008) -- (5.45464,4.29008);
\definecolor{c}{rgb}{0,0,0};
\colorlet{c}{black};
\draw [c] (5.53834,4.04149) -- (5.53834,4.18523);
\draw [c] (5.53834,4.18523) -- (5.53834,4.32897);
\draw [c] (5.45464,4.18523) -- (5.53834,4.18523);
\draw [c] (5.53834,4.18523) -- (5.62203,4.18523);
\definecolor{c}{rgb}{0,0,0};
\colorlet{c}{black};
\draw [c] (5.70573,4.05153) -- (5.70573,4.19845);
\draw [c] (5.70573,4.19845) -- (5.70573,4.34537);
\draw [c] (5.62203,4.19845) -- (5.70573,4.19845);
\draw [c] (5.70573,4.19845) -- (5.78942,4.19845);
\definecolor{c}{rgb}{0,0,0};
\colorlet{c}{black};
\draw [c] (5.87312,4.04653) -- (5.87312,4.23811);
\draw [c] (5.87312,4.23811) -- (5.87312,4.42969);
\draw [c] (5.78942,4.23811) -- (5.87312,4.23811);
\draw [c] (5.87312,4.23811) -- (5.95681,4.23811);
\definecolor{c}{rgb}{0,0,0};
\colorlet{c}{black};
\draw [c] (6.04051,3.95311) -- (6.04051,4.11552);
\draw [c] (6.04051,4.11552) -- (6.04051,4.27793);
\draw [c] (5.95681,4.11552) -- (6.04051,4.11552);
\draw [c] (6.04051,4.11552) -- (6.1242,4.11552);
\definecolor{c}{rgb}{0,0,0};
\colorlet{c}{black};
\draw [c] (6.20789,3.96718) -- (6.20789,4.23811);
\draw [c] (6.20789,4.23811) -- (6.20789,4.50904);
\draw [c] (6.1242,4.23811) -- (6.20789,4.23811);
\draw [c] (6.20789,4.23811) -- (6.29159,4.23811);
\definecolor{c}{rgb}{0,0,0};
\colorlet{c}{black};
\draw [c] (6.37528,3.75566) -- (6.37528,3.82168);
\draw [c] (6.37528,3.82168) -- (6.37528,3.88769);
\draw [c] (6.29159,3.82168) -- (6.37528,3.82168);
\draw [c] (6.37528,3.82168) -- (6.45898,3.82168);
\definecolor{c}{rgb}{0,0,0};
\colorlet{c}{black};
\draw [c] (6.54267,3.75782) -- (6.54267,4.00015);
\draw [c] (6.54267,4.00015) -- (6.54267,4.24248);
\draw [c] (6.45898,4.00015) -- (6.54267,4.00015);
\draw [c] (6.54267,4.00015) -- (6.62637,4.00015);
\definecolor{c}{rgb}{0,0,0};
\colorlet{c}{black};
\draw [c] (6.71006,3.57051) -- (6.71006,4.63472);
\draw [c] (6.71006,4.63472) -- (6.71006,4.85571);
\draw [c] (6.62637,4.63472) -- (6.71006,4.63472);
\draw [c] (6.71006,4.63472) -- (6.79376,4.63472);
\definecolor{c}{rgb}{0,0,0};
\colorlet{c}{black};
\draw [c] (6.87745,3.5759) -- (6.87745,4.15879);
\draw [c] (6.87745,4.15879) -- (6.87745,4.74168);
\draw [c] (6.79376,4.15879) -- (6.87745,4.15879);
\draw [c] (6.87745,4.15879) -- (6.96115,4.15879);
\definecolor{c}{rgb}{0,0,0};
\colorlet{c}{black};
\draw [c] (7.04484,3.65612) -- (7.04484,3.80185);
\draw [c] (7.04484,3.80185) -- (7.04484,3.94757);
\draw [c] (6.96115,3.80185) -- (7.04484,3.80185);
\draw [c] (7.04484,3.80185) -- (7.12854,3.80185);
\definecolor{c}{rgb}{0,0,0};
\colorlet{c}{black};
\draw [c] (7.37962,3.65612) -- (7.37962,3.80185);
\draw [c] (7.37962,3.80185) -- (7.37962,3.94757);
\draw [c] (7.29592,3.80185) -- (7.37962,3.80185);
\draw [c] (7.37962,3.80185) -- (7.46331,3.80185);
\definecolor{c}{rgb}{0,0,0};
\colorlet{c}{black};
\draw [c] (7.54701,3.57242) -- (7.54701,4.39675);
\draw [c] (7.54701,4.39675) -- (7.54701,4.85571);
\draw [c] (7.46331,4.39675) -- (7.54701,4.39675);
\draw [c] (7.54701,4.39675) -- (7.6307,4.39675);
\definecolor{c}{rgb}{0,0,0};
\definecolor{c}{rgb}{1,1,1};
\draw [color=c, fill=c] (0,2) rectangle (9.35135,3.42857);
\draw [color=c, fill=c] (0.935135,2) rectangle (9.30459,3.42714);
\definecolor{c}{rgb}{0,0,0};
\draw [c] (0.935135,2) -- (0.935135,3.42714) -- (9.30459,3.42714) -- (9.30459,2) -- (0.935135,2);
\definecolor{c}{rgb}{1,1,1};
\draw [color=c, fill=c] (0.935135,2) rectangle (9.30459,3.42714);
\definecolor{c}{rgb}{0,0,0};
\draw [c] (0.935135,2) -- (0.935135,3.42714) -- (9.30459,3.42714) -- (9.30459,2) -- (0.935135,2);
\colorlet{c}{natgreen!20};
\draw [color=c, fill=c] (0.935135,2.69243) rectangle (1.10252,2.73471);
\definecolor{c}{rgb}{0,0,0};
\colorlet{c}{natgreen!20};
\draw [color=c, fill=c] (1.10252,2.70584) rectangle (1.26991,2.7213);
\definecolor{c}{rgb}{0,0,0};
\colorlet{c}{natgreen!20};
\draw [color=c, fill=c] (1.26991,2.69152) rectangle (1.4373,2.73562);
\definecolor{c}{rgb}{0,0,0};
\colorlet{c}{natgreen!20};
\draw [color=c, fill=c] (1.4373,2.66844) rectangle (1.60469,2.7587);
\definecolor{c}{rgb}{0,0,0};
\colorlet{c}{natgreen!20};
\draw [color=c, fill=c] (1.60469,2.63816) rectangle (1.77208,2.78898);
\definecolor{c}{rgb}{0,0,0};
\colorlet{c}{natgreen!20};
\draw [color=c, fill=c] (1.77208,2.60353) rectangle (1.93947,2.82362);
\definecolor{c}{rgb}{0,0,0};
\colorlet{c}{natgreen!20};
\draw [color=c, fill=c] (1.93947,2.54946) rectangle (2.10686,2.87768);
\definecolor{c}{rgb}{0,0,0};
\colorlet{c}{natgreen!20};
\draw [color=c, fill=c] (2.10686,2.50448) rectangle (2.27425,2.92267);
\definecolor{c}{rgb}{0,0,0};
\colorlet{c}{natgreen!20};
\draw [color=c, fill=c] (2.27425,2.43377) rectangle (2.44164,2.99338);
\definecolor{c}{rgb}{0,0,0};
\colorlet{c}{natgreen!20};
\draw [color=c, fill=c] (2.44164,2.34661) rectangle (2.60903,3.08053);
\definecolor{c}{rgb}{0,0,0};
\colorlet{c}{natgreen!20};
\draw [color=c, fill=c] (2.60903,2.69896) rectangle (2.77642,2.72819);
\definecolor{c}{rgb}{0,0,0};
\colorlet{c}{natgreen!20};
\draw [color=c, fill=c] (2.77642,2.69602) rectangle (2.94381,2.73112);
\definecolor{c}{rgb}{0,0,0};
\colorlet{c}{natgreen!20};
\draw [color=c, fill=c] (2.94381,2.69291) rectangle (3.11119,2.73423);
\definecolor{c}{rgb}{0,0,0};
\colorlet{c}{natgreen!20};
\draw [color=c, fill=c] (3.11119,2.69006) rectangle (3.27858,2.73708);
\definecolor{c}{rgb}{0,0,0};
\colorlet{c}{natgreen!20};
\draw [color=c, fill=c] (3.27858,2.68735) rectangle (3.44597,2.7398);
\definecolor{c}{rgb}{0,0,0};
\colorlet{c}{natgreen!20};
\draw [color=c, fill=c] (3.44597,2.68468) rectangle (3.61336,2.74246);
\definecolor{c}{rgb}{0,0,0};
\colorlet{c}{natgreen!20};
\draw [color=c, fill=c] (3.61336,2.68167) rectangle (3.78075,2.74547);
\definecolor{c}{rgb}{0,0,0};
\colorlet{c}{natgreen!20};
\draw [color=c, fill=c] (3.78075,2.67992) rectangle (3.94814,2.74723);
\definecolor{c}{rgb}{0,0,0};
\colorlet{c}{natgreen!20};
\draw [color=c, fill=c] (3.94814,2.677) rectangle (4.11553,2.75014);
\definecolor{c}{rgb}{0,0,0};
\colorlet{c}{natgreen!20};
\draw [color=c, fill=c] (4.11553,2.67565) rectangle (4.28292,2.75149);
\definecolor{c}{rgb}{0,0,0};
\colorlet{c}{natgreen!20};
\draw [color=c, fill=c] (4.28292,2.67369) rectangle (4.45031,2.75345);
\definecolor{c}{rgb}{0,0,0};
\colorlet{c}{natgreen!20};
\draw [color=c, fill=c] (4.45031,2.672) rectangle (4.6177,2.75514);
\definecolor{c}{rgb}{0,0,0};
\colorlet{c}{natgreen!20};
\draw [color=c, fill=c] (4.6177,2.67) rectangle (4.78509,2.75715);
\definecolor{c}{rgb}{0,0,0};
\colorlet{c}{natgreen!20};
\draw [color=c, fill=c] (4.78509,2.66861) rectangle (4.95248,2.75854);
\definecolor{c}{rgb}{0,0,0};
\colorlet{c}{natgreen!20};
\draw [color=c, fill=c] (4.95248,2.66589) rectangle (5.11986,2.76125);
\definecolor{c}{rgb}{0,0,0};
\colorlet{c}{natgreen!20};
\draw [color=c, fill=c] (5.11986,2.6636) rectangle (5.28725,2.76355);
\definecolor{c}{rgb}{0,0,0};
\colorlet{c}{natgreen!20};
\draw [color=c, fill=c] (5.28725,2.66165) rectangle (5.45464,2.76549);
\definecolor{c}{rgb}{0,0,0};
\colorlet{c}{natgreen!20};
\draw [color=c, fill=c] (5.45464,2.65897) rectangle (5.62203,2.76818);
\definecolor{c}{rgb}{0,0,0};
\colorlet{c}{natgreen!20};
\draw [color=c, fill=c] (5.62203,2.65689) rectangle (5.78942,2.77025);
\definecolor{c}{rgb}{0,0,0};
\colorlet{c}{natgreen!20};
\draw [color=c, fill=c] (5.78942,2.65248) rectangle (5.95681,2.77466);
\definecolor{c}{rgb}{0,0,0};
\colorlet{c}{natgreen!20};
\draw [color=c, fill=c] (5.95681,2.65023) rectangle (6.1242,2.77692);
\definecolor{c}{rgb}{0,0,0};
\colorlet{c}{natgreen!20};
\draw [color=c, fill=c] (6.1242,2.64671) rectangle (6.29159,2.78044);
\definecolor{c}{rgb}{0,0,0};
\colorlet{c}{natgreen!20};
\draw [color=c, fill=c] (6.29159,2.64193) rectangle (6.45898,2.78521);
\definecolor{c}{rgb}{0,0,0};
\colorlet{c}{natgreen!20};
\draw [color=c, fill=c] (6.45898,2.63976) rectangle (6.62637,2.78738);
\definecolor{c}{rgb}{0,0,0};
\colorlet{c}{natgreen!20};
\draw [color=c, fill=c] (6.62637,2.62921) rectangle (6.79376,2.79794);
\definecolor{c}{rgb}{0,0,0};
\colorlet{c}{natgreen!20};
\draw [color=c, fill=c] (6.79376,2.62509) rectangle (6.96115,2.80205);
\definecolor{c}{rgb}{0,0,0};
\colorlet{c}{natgreen!20};
\draw [color=c, fill=c] (6.96115,2.61564) rectangle (7.12854,2.81151);
\definecolor{c}{rgb}{0,0,0};
\colorlet{c}{natgreen!20};
\draw [color=c, fill=c] (7.12854,2.60799) rectangle (7.29592,2.81915);
\definecolor{c}{rgb}{0,0,0};
\colorlet{c}{natgreen!20};
\draw [color=c, fill=c] (7.29592,2.59232) rectangle (7.46331,2.83482);
\definecolor{c}{rgb}{0,0,0};
\colorlet{c}{natgreen!20};
\draw [color=c, fill=c] (7.46331,2.58116) rectangle (7.6307,2.84599);
\definecolor{c}{rgb}{0,0,0};
\colorlet{c}{natgreen!20};
\draw [color=c, fill=c] (7.6307,2.57487) rectangle (7.79809,2.85227);
\definecolor{c}{rgb}{0,0,0};
\colorlet{c}{natgreen!20};
\draw [color=c, fill=c] (7.79809,2.55738) rectangle (7.96548,2.86976);
\definecolor{c}{rgb}{0,0,0};
\colorlet{c}{natgreen!20};
\draw [color=c, fill=c] (7.96548,2.53863) rectangle (8.13287,2.88851);
\definecolor{c}{rgb}{0,0,0};
\colorlet{c}{natgreen!20};
\draw [color=c, fill=c] (8.13287,2.53783) rectangle (8.30026,2.88931);
\definecolor{c}{rgb}{0,0,0};
\colorlet{c}{natgreen!20};
\draw [color=c, fill=c] (8.30026,2.47865) rectangle (8.46765,2.94849);
\definecolor{c}{rgb}{0,0,0};
\colorlet{c}{natgreen!20};
\draw [color=c, fill=c] (8.46765,2.44874) rectangle (8.63504,2.9784);
\definecolor{c}{rgb}{0,0,0};
\colorlet{c}{natgreen!20};
\draw [color=c, fill=c] (8.63504,2.40777) rectangle (8.80243,3.01937);
\definecolor{c}{rgb}{0,0,0};
\colorlet{c}{natgreen!20};
\draw [color=c, fill=c] (8.80243,2.37286) rectangle (8.96982,3.05429);
\definecolor{c}{rgb}{0,0,0};
\colorlet{c}{natgreen!20};
\draw [color=c, fill=c] (8.96982,2.25487) rectangle (9.13721,3.17227);
\definecolor{c}{rgb}{0,0,0};
\colorlet{c}{natgreen!20};
\draw [color=c, fill=c] (9.13721,2.13336) rectangle (9.30459,3.29379);
\definecolor{c}{rgb}{0,0,0};
\draw [c] (0.935135,2) -- (9.30459,2);
\draw [c] (0.935135,2.03836) -- (0.935135,2);
\draw [c] (1.10252,2.00959) -- (1.10252,2);
\draw [c] (1.26991,2.00959) -- (1.26991,2);
\draw [c] (1.4373,2.00959) -- (1.4373,2);
\draw [c] (1.60469,2.00959) -- (1.60469,2);
\draw [c] (1.77208,2.01918) -- (1.77208,2);
\draw [c] (1.93947,2.00959) -- (1.93947,2);
\draw [c] (2.10686,2.00959) -- (2.10686,2);
\draw [c] (2.27425,2.00959) -- (2.27425,2);
\draw [c] (2.44164,2.00959) -- (2.44164,2);
\draw [c] (2.60903,2.01918) -- (2.60903,2);
\draw [c] (2.77642,2.00959) -- (2.77642,2);
\draw [c] (2.94381,2.00959) -- (2.94381,2);
\draw [c] (3.11119,2.00959) -- (3.11119,2);
\draw [c] (3.27858,2.00959) -- (3.27858,2);
\draw [c] (3.44597,2.01918) -- (3.44597,2);
\draw [c] (3.61336,2.00959) -- (3.61336,2);
\draw [c] (3.78075,2.00959) -- (3.78075,2);
\draw [c] (3.94814,2.00959) -- (3.94814,2);
\draw [c] (4.11553,2.00959) -- (4.11553,2);
\draw [c] (4.28292,2.01918) -- (4.28292,2);
\draw [c] (4.45031,2.00959) -- (4.45031,2);
\draw [c] (4.6177,2.00959) -- (4.6177,2);
\draw [c] (4.78509,2.00959) -- (4.78509,2);
\draw [c] (4.95248,2.00959) -- (4.95248,2);
\draw [c] (5.11986,2.01918) -- (5.11986,2);
\draw [c] (5.28725,2.00959) -- (5.28725,2);
\draw [c] (5.45464,2.00959) -- (5.45464,2);
\draw [c] (5.62203,2.00959) -- (5.62203,2);
\draw [c] (5.78942,2.00959) -- (5.78942,2);
\draw [c] (5.95681,2.01918) -- (5.95681,2);
\draw [c] (6.1242,2.00959) -- (6.1242,2);
\draw [c] (6.29159,2.00959) -- (6.29159,2);
\draw [c] (6.45898,2.00959) -- (6.45898,2);
\draw [c] (6.62637,2.00959) -- (6.62637,2);
\draw [c] (6.79376,2.01918) -- (6.79376,2);
\draw [c] (6.96115,2.00959) -- (6.96115,2);
\draw [c] (7.12854,2.00959) -- (7.12854,2);
\draw [c] (7.29592,2.00959) -- (7.29592,2);
\draw [c] (7.46331,2.00959) -- (7.46331,2);
\draw [c] (7.6307,2.01918) -- (7.6307,2);
\draw [c] (7.79809,2.00959) -- (7.79809,2);
\draw [c] (7.96548,2.00959) -- (7.96548,2);
\draw [c] (8.13287,2.00959) -- (8.13287,2);
\draw [c] (8.30026,2.00959) -- (8.30026,2);
\draw [c] (8.46765,2.01918) -- (8.46765,2);
\draw [c] (8.63504,2.00959) -- (8.63504,2);
\draw [c] (8.80243,2.00959) -- (8.80243,2);
\draw [c] (8.96982,2.00959) -- (8.96982,2);
\draw [c] (9.13721,2.00959) -- (9.13721,2);
\draw [c] (9.30459,2.03836) -- (9.30459,2);
\draw [anchor=base] (0.935135,1.84571) node[color=c, rotate=0]{0};
\draw [anchor=base] (9.30459,1.84571) node[color=c, rotate=0]{5000};
\draw [c] (0.935135,2) -- (0.935135,3.42714);
\draw [anchor= east] (-0.112216,3.42714) node[color=c, rotate=90]{Ratio};
\draw [c] (1.2154,2.06487) -- (0.935135,2.06487);
\draw [c] (1.07527,2.19461) -- (0.935135,2.19461);
\draw [c] (1.07527,2.32435) -- (0.935135,2.32435);
\draw [c] (1.07527,2.45409) -- (0.935135,2.45409);
\draw [c] (1.07527,2.58383) -- (0.935135,2.58383);
\draw [c] (1.2154,2.71357) -- (0.935135,2.71357);
\draw [c] (1.07527,2.84331) -- (0.935135,2.84331);
\draw [c] (1.07527,2.97305) -- (0.935135,2.97305);
\draw [c] (1.07527,3.10279) -- (0.935135,3.10279);
\draw [c] (1.07527,3.23253) -- (0.935135,3.23253);
\draw [c] (1.2154,3.36227) -- (0.935135,3.36227);
\draw [c] (1.2154,2.06487) -- (0.935135,2.06487);
\draw [c] (1.2154,3.36227) -- (0.935135,3.36227);
\draw [anchor= east] (0.888378,2.06487) node[color=c, rotate=0]{0.5};
\draw [anchor= east] (0.888378,2.71357) node[color=c, rotate=0]{1};
\draw [anchor= east] (0.888378,3.36227) node[color=c, rotate=0]{1.5};
\colorlet{c}{natgreen};
\draw [c] (1.01883,2.84833) -- (1.01883,2.86547);
\draw [c] (1.01883,2.86547) -- (1.01883,2.88261);
\draw [c] (0.935135,2.86547) -- (1.01883,2.86547);
\draw [c] (1.01883,2.86547) -- (1.10252,2.86547);
\definecolor{c}{rgb}{0,0,0};
\colorlet{c}{natgreen};
\draw [c] (1.18622,2.90338) -- (1.18622,2.91495);
\draw [c] (1.18622,2.91495) -- (1.18622,2.92652);
\draw [c] (1.10252,2.91495) -- (1.18622,2.91495);
\draw [c] (1.18622,2.91495) -- (1.26991,2.91495);
\definecolor{c}{rgb}{0,0,0};
\colorlet{c}{natgreen};
\draw [c] (1.35361,2.93128) -- (1.35361,2.96498);
\draw [c] (1.35361,2.96498) -- (1.35361,2.99868);
\draw [c] (1.26991,2.96498) -- (1.35361,2.96498);
\draw [c] (1.35361,2.96498) -- (1.4373,2.96498);
\definecolor{c}{rgb}{0,0,0};
\colorlet{c}{natgreen};
\draw [c] (1.521,2.99905) -- (1.521,3.07109);
\draw [c] (1.521,3.07109) -- (1.521,3.14314);
\draw [c] (1.4373,3.07109) -- (1.521,3.07109);
\draw [c] (1.521,3.07109) -- (1.60469,3.07109);
\definecolor{c}{rgb}{0,0,0};
\colorlet{c}{natgreen};
\draw [c] (1.68839,2.95187) -- (1.68839,3.01796);
\draw [c] (1.68839,3.01796) -- (1.68839,3.08406);
\draw [c] (1.60469,3.01796) -- (1.68839,3.01796);
\draw [c] (1.68839,3.01796) -- (1.77208,3.01796);
\definecolor{c}{rgb}{0,0,0};
\colorlet{c}{natgreen};
\draw [c] (1.85578,2.77242) -- (1.85578,2.85936);
\draw [c] (1.85578,2.85936) -- (1.85578,2.9463);
\draw [c] (1.77208,2.85936) -- (1.85578,2.85936);
\draw [c] (1.85578,2.85936) -- (1.93947,2.85936);
\definecolor{c}{rgb}{0,0,0};
\colorlet{c}{natgreen};
\draw [c] (2.02316,2.84437) -- (2.02316,2.98529);
\draw [c] (2.02316,2.98529) -- (2.02316,3.12621);
\draw [c] (1.93947,2.98529) -- (2.02316,2.98529);
\draw [c] (2.02316,2.98529) -- (2.10686,2.98529);
\definecolor{c}{rgb}{0,0,0};
\colorlet{c}{natgreen};
\draw [c] (2.19055,2.58894) -- (2.19055,2.7406);
\draw [c] (2.19055,2.7406) -- (2.19055,2.89226);
\draw [c] (2.10686,2.7406) -- (2.19055,2.7406);
\draw [c] (2.19055,2.7406) -- (2.27425,2.7406);
\definecolor{c}{rgb}{0,0,0};
\colorlet{c}{natgreen};
\draw [c] (2.35794,2.59399) -- (2.35794,2.80708);
\draw [c] (2.35794,2.80708) -- (2.35794,3.02017);
\draw [c] (2.27425,2.80708) -- (2.35794,2.80708);
\draw [c] (2.35794,2.80708) -- (2.44164,2.80708);
\definecolor{c}{rgb}{0,0,0};
\colorlet{c}{natgreen};
\draw [c] (2.52533,2.57161) -- (2.52533,2.86207);
\draw [c] (2.52533,2.86207) -- (2.52533,3.15253);
\draw [c] (2.44164,2.86207) -- (2.52533,2.86207);
\draw [c] (2.52533,2.86207) -- (2.60903,2.86207);
\definecolor{c}{rgb}{0,0,0};
\colorlet{c}{natgreen};
\draw [c] (2.69272,2.78208) -- (2.69272,2.79553);
\draw [c] (2.69272,2.79553) -- (2.69272,2.80899);
\draw [c] (2.60903,2.79553) -- (2.69272,2.79553);
\draw [c] (2.69272,2.79553) -- (2.77642,2.79553);
\definecolor{c}{rgb}{0,0,0};
\colorlet{c}{natgreen};
\draw [c] (2.86011,2.732) -- (2.86011,2.74769);
\draw [c] (2.86011,2.74769) -- (2.86011,2.76338);
\draw [c] (2.77642,2.74769) -- (2.86011,2.74769);
\draw [c] (2.86011,2.74769) -- (2.94381,2.74769);
\definecolor{c}{rgb}{0,0,0};
\colorlet{c}{natgreen};
\draw [c] (3.0275,2.71918) -- (3.0275,2.73753);
\draw [c] (3.0275,2.73753) -- (3.0275,2.75588);
\draw [c] (2.94381,2.73753) -- (3.0275,2.73753);
\draw [c] (3.0275,2.73753) -- (3.11119,2.73753);
\definecolor{c}{rgb}{0,0,0};
\colorlet{c}{natgreen};
\draw [c] (3.19489,2.67837) -- (3.19489,2.69875);
\draw [c] (3.19489,2.69875) -- (3.19489,2.71913);
\draw [c] (3.11119,2.69875) -- (3.19489,2.69875);
\draw [c] (3.19489,2.69875) -- (3.27858,2.69875);
\definecolor{c}{rgb}{0,0,0};
\colorlet{c}{natgreen};
\draw [c] (3.36228,2.61706) -- (3.36228,2.63891);
\draw [c] (3.36228,2.63891) -- (3.36228,2.66077);
\draw [c] (3.27858,2.63891) -- (3.36228,2.63891);
\draw [c] (3.36228,2.63891) -- (3.44597,2.63891);
\definecolor{c}{rgb}{0,0,0};
\colorlet{c}{natgreen};
\draw [c] (3.52967,2.60869) -- (3.52967,2.63266);
\draw [c] (3.52967,2.63266) -- (3.52967,2.65664);
\draw [c] (3.44597,2.63266) -- (3.52967,2.63266);
\draw [c] (3.52967,2.63266) -- (3.61336,2.63266);
\definecolor{c}{rgb}{0,0,0};
\colorlet{c}{natgreen};
\draw [c] (3.69706,2.60537) -- (3.69706,2.63183);
\draw [c] (3.69706,2.63183) -- (3.69706,2.65828);
\draw [c] (3.61336,2.63183) -- (3.69706,2.63183);
\draw [c] (3.69706,2.63183) -- (3.78075,2.63183);
\definecolor{c}{rgb}{0,0,0};
\colorlet{c}{natgreen};
\draw [c] (3.86445,2.57148) -- (3.86445,2.59878);
\draw [c] (3.86445,2.59878) -- (3.86445,2.62607);
\draw [c] (3.78075,2.59878) -- (3.86445,2.59878);
\draw [c] (3.86445,2.59878) -- (3.94814,2.59878);
\definecolor{c}{rgb}{0,0,0};
\colorlet{c}{natgreen};
\draw [c] (4.03184,2.56799) -- (4.03184,2.59762);
\draw [c] (4.03184,2.59762) -- (4.03184,2.62725);
\draw [c] (3.94814,2.59762) -- (4.03184,2.59762);
\draw [c] (4.03184,2.59762) -- (4.11553,2.59762);
\definecolor{c}{rgb}{0,0,0};
\colorlet{c}{natgreen};
\draw [c] (4.19922,2.504) -- (4.19922,2.53336);
\draw [c] (4.19922,2.53336) -- (4.19922,2.56273);
\draw [c] (4.11553,2.53336) -- (4.19922,2.53336);
\draw [c] (4.19922,2.53336) -- (4.28292,2.53336);
\definecolor{c}{rgb}{0,0,0};
\colorlet{c}{natgreen};
\draw [c] (4.36661,2.51844) -- (4.36661,2.54968);
\draw [c] (4.36661,2.54968) -- (4.36661,2.58092);
\draw [c] (4.28292,2.54968) -- (4.36661,2.54968);
\draw [c] (4.36661,2.54968) -- (4.45031,2.54968);
\definecolor{c}{rgb}{0,0,0};
\colorlet{c}{natgreen};
\draw [c] (4.534,2.47979) -- (4.534,2.51146);
\draw [c] (4.534,2.51146) -- (4.534,2.54314);
\draw [c] (4.45031,2.51146) -- (4.534,2.51146);
\draw [c] (4.534,2.51146) -- (4.6177,2.51146);
\definecolor{c}{rgb}{0,0,0};
\colorlet{c}{natgreen};
\draw [c] (4.70139,2.47755) -- (4.70139,2.51074);
\draw [c] (4.70139,2.51074) -- (4.70139,2.54392);
\draw [c] (4.6177,2.51074) -- (4.70139,2.51074);
\draw [c] (4.70139,2.51074) -- (4.78509,2.51074);
\definecolor{c}{rgb}{0,0,0};
\colorlet{c}{natgreen};
\draw [c] (4.86878,2.45533) -- (4.86878,2.48902);
\draw [c] (4.86878,2.48902) -- (4.86878,2.52272);
\draw [c] (4.78509,2.48902) -- (4.86878,2.48902);
\draw [c] (4.86878,2.48902) -- (4.95248,2.48902);
\definecolor{c}{rgb}{0,0,0};
\colorlet{c}{natgreen};
\draw [c] (5.03617,2.45032) -- (5.03617,2.48597);
\draw [c] (5.03617,2.48597) -- (5.03617,2.52161);
\draw [c] (4.95248,2.48597) -- (5.03617,2.48597);
\draw [c] (5.03617,2.48597) -- (5.11986,2.48597);
\definecolor{c}{rgb}{0,0,0};
\colorlet{c}{natgreen};
\draw [c] (5.20356,2.44486) -- (5.20356,2.48211);
\draw [c] (5.20356,2.48211) -- (5.20356,2.51937);
\draw [c] (5.11986,2.48211) -- (5.20356,2.48211);
\draw [c] (5.20356,2.48211) -- (5.28725,2.48211);
\definecolor{c}{rgb}{0,0,0};
\colorlet{c}{natgreen};
\draw [c] (5.37095,2.39741) -- (5.37095,2.43473);
\draw [c] (5.37095,2.43473) -- (5.37095,2.47205);
\draw [c] (5.28725,2.43473) -- (5.37095,2.43473);
\draw [c] (5.37095,2.43473) -- (5.45464,2.43473);
\definecolor{c}{rgb}{0,0,0};
\colorlet{c}{natgreen};
\draw [c] (5.53834,2.42587) -- (5.53834,2.46608);
\draw [c] (5.53834,2.46608) -- (5.53834,2.5063);
\draw [c] (5.45464,2.46608) -- (5.53834,2.46608);
\draw [c] (5.53834,2.46608) -- (5.62203,2.46608);
\definecolor{c}{rgb}{0,0,0};
\colorlet{c}{natgreen};
\draw [c] (5.70573,2.34757) -- (5.70573,2.38678);
\draw [c] (5.70573,2.38678) -- (5.70573,2.42598);
\draw [c] (5.62203,2.38678) -- (5.70573,2.38678);
\draw [c] (5.70573,2.38678) -- (5.78942,2.38678);
\definecolor{c}{rgb}{0,0,0};
\colorlet{c}{natgreen};
\draw [c] (5.87312,2.3708) -- (5.87312,2.414);
\draw [c] (5.87312,2.414) -- (5.87312,2.4572);
\draw [c] (5.78942,2.414) -- (5.87312,2.414);
\draw [c] (5.87312,2.414) -- (5.95681,2.414);
\definecolor{c}{rgb}{0,0,0};
\colorlet{c}{natgreen};
\draw [c] (6.04051,2.28384) -- (6.04051,2.32546);
\draw [c] (6.04051,2.32546) -- (6.04051,2.36709);
\draw [c] (5.95681,2.32546) -- (6.04051,2.32546);
\draw [c] (6.04051,2.32546) -- (6.1242,2.32546);
\definecolor{c}{rgb}{0,0,0};
\colorlet{c}{natgreen};
\draw [c] (6.20789,2.38938) -- (6.20789,2.43755);
\draw [c] (6.20789,2.43755) -- (6.20789,2.48572);
\draw [c] (6.1242,2.43755) -- (6.20789,2.43755);
\draw [c] (6.20789,2.43755) -- (6.29159,2.43755);
\definecolor{c}{rgb}{0,0,0};
\colorlet{c}{natgreen};
\draw [c] (6.37528,2.29599) -- (6.37528,2.3438);
\draw [c] (6.37528,2.3438) -- (6.37528,2.39162);
\draw [c] (6.29159,2.3438) -- (6.37528,2.3438);
\draw [c] (6.37528,2.3438) -- (6.45898,2.3438);
\definecolor{c}{rgb}{0,0,0};
\colorlet{c}{natgreen};
\draw [c] (6.54267,2.24234) -- (6.54267,2.28933);
\draw [c] (6.54267,2.28933) -- (6.54267,2.33631);
\draw [c] (6.45898,2.28933) -- (6.54267,2.28933);
\draw [c] (6.54267,2.28933) -- (6.62637,2.28933);
\definecolor{c}{rgb}{0,0,0};
\colorlet{c}{natgreen};
\draw [c] (6.71006,2.2061) -- (6.71006,2.25831);
\draw [c] (6.71006,2.25831) -- (6.71006,2.31053);
\draw [c] (6.62637,2.25831) -- (6.71006,2.25831);
\draw [c] (6.71006,2.25831) -- (6.79376,2.25831);
\definecolor{c}{rgb}{0,0,0};
\colorlet{c}{natgreen};
\draw [c] (6.87745,2.1709) -- (6.87745,2.22393);
\draw [c] (6.87745,2.22393) -- (6.87745,2.27695);
\draw [c] (6.79376,2.22393) -- (6.87745,2.22393);
\draw [c] (6.87745,2.22393) -- (6.96115,2.22393);
\definecolor{c}{rgb}{0,0,0};
\colorlet{c}{natgreen};
\draw [c] (7.04484,2.23308) -- (7.04484,2.29578);
\draw [c] (7.04484,2.29578) -- (7.04484,2.35848);
\draw [c] (6.96115,2.29578) -- (7.04484,2.29578);
\draw [c] (7.04484,2.29578) -- (7.12854,2.29578);
\definecolor{c}{rgb}{0,0,0};
\colorlet{c}{natgreen};
\draw [c] (7.21223,2.24221) -- (7.21223,2.31071);
\draw [c] (7.21223,2.31071) -- (7.21223,2.3792);
\draw [c] (7.12854,2.31071) -- (7.21223,2.31071);
\draw [c] (7.21223,2.31071) -- (7.29592,2.31071);
\definecolor{c}{rgb}{0,0,0};
\colorlet{c}{natgreen};
\draw [c] (7.37962,2.30424) -- (7.37962,2.38821);
\draw [c] (7.37962,2.38821) -- (7.37962,2.47219);
\draw [c] (7.29592,2.38821) -- (7.37962,2.38821);
\draw [c] (7.37962,2.38821) -- (7.46331,2.38821);
\definecolor{c}{rgb}{0,0,0};
\colorlet{c}{natgreen};
\draw [c] (7.54701,2.20416) -- (7.54701,2.28838);
\draw [c] (7.54701,2.28838) -- (7.54701,2.3726);
\draw [c] (7.46331,2.28838) -- (7.54701,2.28838);
\draw [c] (7.54701,2.28838) -- (7.6307,2.28838);
\definecolor{c}{rgb}{0,0,0};
\colorlet{c}{natgreen};
\draw [c] (7.7144,2.1005) -- (7.7144,2.18015);
\draw [c] (7.7144,2.18015) -- (7.7144,2.2598);
\draw [c] (7.6307,2.18015) -- (7.7144,2.18015);
\draw [c] (7.7144,2.18015) -- (7.79809,2.18015);
\definecolor{c}{rgb}{0,0,0};
\colorlet{c}{natgreen};
\draw [c] (7.88179,2.06326) -- (7.88179,2.15027);
\draw [c] (7.88179,2.15027) -- (7.88179,2.23728);
\draw [c] (7.79809,2.15027) -- (7.88179,2.15027);
\draw [c] (7.88179,2.15027) -- (7.96548,2.15027);
\definecolor{c}{rgb}{0,0,0};
\colorlet{c}{natgreen};
\draw [c] (8.04918,2.09692) -- (8.04918,2.1993);
\draw [c] (8.04918,2.1993) -- (8.04918,2.30168);
\draw [c] (7.96548,2.1993) -- (8.04918,2.1993);
\draw [c] (8.04918,2.1993) -- (8.13287,2.1993);
\definecolor{c}{rgb}{0,0,0};
\colorlet{c}{natgreen};
\draw [c] (8.38395,2.03916) -- (8.38395,2.17311);
\draw [c] (8.38395,2.17311) -- (8.38395,2.30707);
\draw [c] (8.30026,2.17311) -- (8.38395,2.17311);
\draw [c] (8.38395,2.17311) -- (8.46765,2.17311);
\definecolor{c}{rgb}{0,0,0};
\colorlet{c}{natgreen};
\draw [c] (8.55134,2.12299) -- (8.55134,2.29197);
\draw [c] (8.55134,2.29197) -- (8.55134,2.46095);
\draw [c] (8.46765,2.29197) -- (8.55134,2.29197);
\draw [c] (8.55134,2.29197) -- (8.63504,2.29197);
\definecolor{c}{rgb}{0,0,0};
\colorlet{c}{natgreen};
\draw [c] (8.71873,2.00599) -- (8.71873,2.1819);
\draw [c] (8.71873,2.1819) -- (8.71873,2.35781);
\draw [c] (8.63504,2.1819) -- (8.71873,2.1819);
\draw [c] (8.71873,2.1819) -- (8.80243,2.1819);
\definecolor{c}{rgb}{0,0,0};
\colorlet{c}{natgreen};
\draw [c] (8.88612,2.07737) -- (8.88612,2.29544);
\draw [c] (8.88612,2.29544) -- (8.88612,2.51351);
\draw [c] (8.80243,2.29544) -- (8.88612,2.29544);
\draw [c] (8.88612,2.29544) -- (8.96982,2.29544);
\definecolor{c}{rgb}{0,0,0};
\colorlet{c}{natgreen};
\draw [c] (9.05351,2.08433) -- (9.05351,2.40683);
\draw [c] (9.05351,2.40683) -- (9.05351,2.72934);
\draw [c] (8.96982,2.40683) -- (9.05351,2.40683);
\draw [c] (9.05351,2.40683) -- (9.13721,2.40683);
\definecolor{c}{rgb}{0,0,0};
\colorlet{c}{natgreen};
\draw [c] (9.2209,2) -- (9.2209,2.17424);
\draw [c] (9.2209,2.17424) -- (9.2209,2.50547);
\draw [c] (9.13721,2.17424) -- (9.2209,2.17424);
\draw [c] (9.2209,2.17424) -- (9.30459,2.17424);
\definecolor{c}{rgb}{0,0,0};
\definecolor{c}{rgb}{1,1,1};
\draw [color=c, fill=c] (0,0) rectangle (9.35135,2);
\draw [color=c, fill=c] (0.935135,0.56) rectangle (9.30459,1.98);
\definecolor{c}{rgb}{0,0,0};
\draw [c] (0.935135,0.56) -- (0.935135,1.98) -- (9.30459,1.98) -- (9.30459,0.56) -- (0.935135,0.56);
\definecolor{c}{rgb}{1,1,1};
\draw [color=c, fill=c] (0.935135,0.56) rectangle (9.30459,1.98);
\definecolor{c}{rgb}{0,0,0};
\draw [c] (0.935135,0.56) -- (0.935135,1.98) -- (9.30459,1.98) -- (9.30459,0.56) -- (0.935135,0.56);
\colorlet{c}{natcomp!50};
\draw [color=c, fill=c] (0.935135,1.40466) rectangle (1.10252,1.41934);
\definecolor{c}{rgb}{0,0,0};
\colorlet{c}{natcomp!50};
\draw [color=c, fill=c] (1.10252,1.40352) rectangle (1.26991,1.42048);
\definecolor{c}{rgb}{0,0,0};
\colorlet{c}{natcomp!50};
\draw [color=c, fill=c] (1.26991,1.38795) rectangle (1.4373,1.43605);
\definecolor{c}{rgb}{0,0,0};
\colorlet{c}{natcomp!50};
\draw [color=c, fill=c] (1.4373,1.36344) rectangle (1.60469,1.46056);
\definecolor{c}{rgb}{0,0,0};
\colorlet{c}{natcomp!50};
\draw [color=c, fill=c] (1.60469,1.33008) rectangle (1.77208,1.49392);
\definecolor{c}{rgb}{0,0,0};
\colorlet{c}{natcomp!50};
\draw [color=c, fill=c] (1.77208,1.29241) rectangle (1.93947,1.53159);
\definecolor{c}{rgb}{0,0,0};
\colorlet{c}{natcomp!50};
\draw [color=c, fill=c] (1.93947,1.24637) rectangle (2.10686,1.57763);
\definecolor{c}{rgb}{0,0,0};
\colorlet{c}{natcomp!50};
\draw [color=c, fill=c] (2.10686,1.17024) rectangle (2.27425,1.65376);
\definecolor{c}{rgb}{0,0,0};
\colorlet{c}{natcomp!50};
\draw [color=c, fill=c] (2.27425,1.12512) rectangle (2.44164,1.69888);
\definecolor{c}{rgb}{0,0,0};
\colorlet{c}{natcomp!50};
\draw [color=c, fill=c] (2.44164,1.10213) rectangle (2.60903,1.72187);
\definecolor{c}{rgb}{0,0,0};
\colorlet{c}{natcomp!50};
\draw [color=c, fill=c] (2.60903,1.38859) rectangle (2.77642,1.43541);
\definecolor{c}{rgb}{0,0,0};
\colorlet{c}{natcomp!50};
\draw [color=c, fill=c] (2.77642,1.38603) rectangle (2.94381,1.43797);
\definecolor{c}{rgb}{0,0,0};
\colorlet{c}{natcomp!50};
\draw [color=c, fill=c] (2.94381,1.38447) rectangle (3.11119,1.43953);
\definecolor{c}{rgb}{0,0,0};
\colorlet{c}{natcomp!50};
\draw [color=c, fill=c] (3.11119,1.38254) rectangle (3.27858,1.44146);
\definecolor{c}{rgb}{0,0,0};
\colorlet{c}{natcomp!50};
\draw [color=c, fill=c] (3.27858,1.38187) rectangle (3.44597,1.44213);
\definecolor{c}{rgb}{0,0,0};
\colorlet{c}{natcomp!50};
\draw [color=c, fill=c] (3.44597,1.38145) rectangle (3.61336,1.44255);
\definecolor{c}{rgb}{0,0,0};
\colorlet{c}{natcomp!50};
\draw [color=c, fill=c] (3.61336,1.38049) rectangle (3.78075,1.44351);
\definecolor{c}{rgb}{0,0,0};
\colorlet{c}{natcomp!50};
\draw [color=c, fill=c] (3.78075,1.38013) rectangle (3.94814,1.44387);
\definecolor{c}{rgb}{0,0,0};
\colorlet{c}{natcomp!50};
\draw [color=c, fill=c] (3.94814,1.37968) rectangle (4.11553,1.44432);
\definecolor{c}{rgb}{0,0,0};
\colorlet{c}{natcomp!50};
\draw [color=c, fill=c] (4.11553,1.37917) rectangle (4.28292,1.44483);
\definecolor{c}{rgb}{0,0,0};
\colorlet{c}{natcomp!50};
\draw [color=c, fill=c] (4.28292,1.37864) rectangle (4.45031,1.44536);
\definecolor{c}{rgb}{0,0,0};
\colorlet{c}{natcomp!50};
\draw [color=c, fill=c] (4.45031,1.37823) rectangle (4.6177,1.44577);
\definecolor{c}{rgb}{0,0,0};
\colorlet{c}{natcomp!50};
\draw [color=c, fill=c] (4.6177,1.3778) rectangle (4.78509,1.4462);
\definecolor{c}{rgb}{0,0,0};
\colorlet{c}{natcomp!50};
\draw [color=c, fill=c] (4.78509,1.37718) rectangle (4.95248,1.44682);
\definecolor{c}{rgb}{0,0,0};
\colorlet{c}{natcomp!50};
\draw [color=c, fill=c] (4.95248,1.37642) rectangle (5.11986,1.44758);
\definecolor{c}{rgb}{0,0,0};
\colorlet{c}{natcomp!50};
\draw [color=c, fill=c] (5.11986,1.37535) rectangle (5.28725,1.44865);
\definecolor{c}{rgb}{0,0,0};
\colorlet{c}{natcomp!50};
\draw [color=c, fill=c] (5.28725,1.37493) rectangle (5.45464,1.44907);
\definecolor{c}{rgb}{0,0,0};
\colorlet{c}{natcomp!50};
\draw [color=c, fill=c] (5.45464,1.37229) rectangle (5.62203,1.45171);
\definecolor{c}{rgb}{0,0,0};
\colorlet{c}{natcomp!50};
\draw [color=c, fill=c] (5.62203,1.37151) rectangle (5.78942,1.45249);
\definecolor{c}{rgb}{0,0,0};
\colorlet{c}{natcomp!50};
\draw [color=c, fill=c] (5.78942,1.37028) rectangle (5.95681,1.45372);
\definecolor{c}{rgb}{0,0,0};
\colorlet{c}{natcomp!50};
\draw [color=c, fill=c] (5.95681,1.3673) rectangle (6.1242,1.4567);
\definecolor{c}{rgb}{0,0,0};
\colorlet{c}{natcomp!50};
\draw [color=c, fill=c] (6.1242,1.36491) rectangle (6.29159,1.45909);
\definecolor{c}{rgb}{0,0,0};
\colorlet{c}{natcomp!50};
\draw [color=c, fill=c] (6.29159,1.36279) rectangle (6.45898,1.46121);
\definecolor{c}{rgb}{0,0,0};
\colorlet{c}{natcomp!50};
\draw [color=c, fill=c] (6.45898,1.35917) rectangle (6.62637,1.46483);
\definecolor{c}{rgb}{0,0,0};
\colorlet{c}{natcomp!50};
\draw [color=c, fill=c] (6.62637,1.35405) rectangle (6.79376,1.46995);
\definecolor{c}{rgb}{0,0,0};
\colorlet{c}{natcomp!50};
\draw [color=c, fill=c] (6.79376,1.35167) rectangle (6.96115,1.47233);
\definecolor{c}{rgb}{0,0,0};
\colorlet{c}{natcomp!50};
\draw [color=c, fill=c] (6.96115,1.34715) rectangle (7.12854,1.47685);
\definecolor{c}{rgb}{0,0,0};
\colorlet{c}{natcomp!50};
\draw [color=c, fill=c] (7.12854,1.33911) rectangle (7.29592,1.48489);
\definecolor{c}{rgb}{0,0,0};
\colorlet{c}{natcomp!50};
\draw [color=c, fill=c] (7.29592,1.33577) rectangle (7.46331,1.48823);
\definecolor{c}{rgb}{0,0,0};
\colorlet{c}{natcomp!50};
\draw [color=c, fill=c] (7.46331,1.32091) rectangle (7.6307,1.50309);
\definecolor{c}{rgb}{0,0,0};
\colorlet{c}{natcomp!50};
\draw [color=c, fill=c] (7.6307,1.3133) rectangle (7.79809,1.5107);
\definecolor{c}{rgb}{0,0,0};
\colorlet{c}{natcomp!50};
\draw [color=c, fill=c] (7.79809,1.29284) rectangle (7.96548,1.53116);
\definecolor{c}{rgb}{0,0,0};
\colorlet{c}{natcomp!50};
\draw [color=c, fill=c] (7.96548,1.28237) rectangle (8.13287,1.54163);
\definecolor{c}{rgb}{0,0,0};
\colorlet{c}{natcomp!50};
\draw [color=c, fill=c] (8.13287,1.2602) rectangle (8.30026,1.5638);
\definecolor{c}{rgb}{0,0,0};
\colorlet{c}{natcomp!50};
\draw [color=c, fill=c] (8.30026,1.25224) rectangle (8.46765,1.57176);
\definecolor{c}{rgb}{0,0,0};
\colorlet{c}{natcomp!50};
\draw [color=c, fill=c] (8.46765,1.2398) rectangle (8.63504,1.5842);
\definecolor{c}{rgb}{0,0,0};
\colorlet{c}{natcomp!50};
\draw [color=c, fill=c] (8.63504,1.18887) rectangle (8.80243,1.63513);
\definecolor{c}{rgb}{0,0,0};
\colorlet{c}{natcomp!50};
\draw [color=c, fill=c] (8.80243,1.18606) rectangle (8.96982,1.63794);
\definecolor{c}{rgb}{0,0,0};
\colorlet{c}{natcomp!50};
\draw [color=c, fill=c] (8.96982,1.16292) rectangle (9.13721,1.66108);
\definecolor{c}{rgb}{0,0,0};
\colorlet{c}{natcomp!50};
\draw [color=c, fill=c] (9.13721,1.11264) rectangle (9.30459,1.71136);
\definecolor{c}{rgb}{0,0,0};
\draw [c] (0.935135,0.56) -- (9.30459,0.56);
\draw [anchor= east] (9.30459,0.112) node[color=c, rotate=0]{$M_{\gamma\gamma}$ [GeV]};
\draw [c] (0.935135,0.6137) -- (0.935135,0.56);
\draw [c] (1.10252,0.58685) -- (1.10252,0.56);
\draw [c] (1.26991,0.58685) -- (1.26991,0.56);
\draw [c] (1.4373,0.58685) -- (1.4373,0.56);
\draw [c] (1.60469,0.58685) -- (1.60469,0.56);
\draw [c] (1.77208,0.6137) -- (1.77208,0.56);
\draw [c] (1.93947,0.58685) -- (1.93947,0.56);
\draw [c] (2.10686,0.58685) -- (2.10686,0.56);
\draw [c] (2.27425,0.58685) -- (2.27425,0.56);
\draw [c] (2.44164,0.58685) -- (2.44164,0.56);
\draw [c] (2.60903,0.6137) -- (2.60903,0.56);
\draw [c] (2.77642,0.58685) -- (2.77642,0.56);
\draw [c] (2.94381,0.58685) -- (2.94381,0.56);
\draw [c] (3.11119,0.58685) -- (3.11119,0.56);
\draw [c] (3.27858,0.58685) -- (3.27858,0.56);
\draw [c] (3.44597,0.6137) -- (3.44597,0.56);
\draw [c] (3.61336,0.58685) -- (3.61336,0.56);
\draw [c] (3.78075,0.58685) -- (3.78075,0.56);
\draw [c] (3.94814,0.58685) -- (3.94814,0.56);
\draw [c] (4.11553,0.58685) -- (4.11553,0.56);
\draw [c] (4.28292,0.6137) -- (4.28292,0.56);
\draw [c] (4.45031,0.58685) -- (4.45031,0.56);
\draw [c] (4.6177,0.58685) -- (4.6177,0.56);
\draw [c] (4.78509,0.58685) -- (4.78509,0.56);
\draw [c] (4.95248,0.58685) -- (4.95248,0.56);
\draw [c] (5.11986,0.6137) -- (5.11986,0.56);
\draw [c] (5.28725,0.58685) -- (5.28725,0.56);
\draw [c] (5.45464,0.58685) -- (5.45464,0.56);
\draw [c] (5.62203,0.58685) -- (5.62203,0.56);
\draw [c] (5.78942,0.58685) -- (5.78942,0.56);
\draw [c] (5.95681,0.6137) -- (5.95681,0.56);
\draw [c] (6.1242,0.58685) -- (6.1242,0.56);
\draw [c] (6.29159,0.58685) -- (6.29159,0.56);
\draw [c] (6.45898,0.58685) -- (6.45898,0.56);
\draw [c] (6.62637,0.58685) -- (6.62637,0.56);
\draw [c] (6.79376,0.6137) -- (6.79376,0.56);
\draw [c] (6.96115,0.58685) -- (6.96115,0.56);
\draw [c] (7.12854,0.58685) -- (7.12854,0.56);
\draw [c] (7.29592,0.58685) -- (7.29592,0.56);
\draw [c] (7.46331,0.58685) -- (7.46331,0.56);
\draw [c] (7.6307,0.6137) -- (7.6307,0.56);
\draw [c] (7.79809,0.58685) -- (7.79809,0.56);
\draw [c] (7.96548,0.58685) -- (7.96548,0.56);
\draw [c] (8.13287,0.58685) -- (8.13287,0.56);
\draw [c] (8.30026,0.58685) -- (8.30026,0.56);
\draw [c] (8.46765,0.6137) -- (8.46765,0.56);
\draw [c] (8.63504,0.58685) -- (8.63504,0.56);
\draw [c] (8.80243,0.58685) -- (8.80243,0.56);
\draw [c] (8.96982,0.58685) -- (8.96982,0.56);
\draw [c] (9.13721,0.58685) -- (9.13721,0.56);
\draw [c] (9.30459,0.6137) -- (9.30459,0.56);
\draw [anchor=base] (0.935135,0.344) node[color=c, rotate=0]{0};
\draw [anchor=base] (1.77208,0.344) node[color=c, rotate=0]{500};
\draw [anchor=base] (2.60903,0.344) node[color=c, rotate=0]{1000};
\draw [anchor=base] (3.44597,0.344) node[color=c, rotate=0]{1500};
\draw [anchor=base] (4.28292,0.344) node[color=c, rotate=0]{2000};
\draw [anchor=base] (5.11986,0.344) node[color=c, rotate=0]{2500};
\draw [anchor=base] (5.95681,0.344) node[color=c, rotate=0]{3000};
\draw [anchor=base] (6.79376,0.344) node[color=c, rotate=0]{3500};
\draw [anchor=base] (7.6307,0.344) node[color=c, rotate=0]{4000};
\draw [anchor=base] (8.46765,0.344) node[color=c, rotate=0]{4500};
\draw [anchor=base] (9.30459,0.344) node[color=c, rotate=0]{5000};
\draw [c] (0.935135,0.56) -- (0.935135,1.98);
\draw [anchor= east] (-0.112216,1.98) node[color=c, rotate=90]{Ratio};
\draw [c] (1.13432,0.702) -- (0.935135,0.702);
\draw [c] (1.03473,0.844) -- (0.935135,0.844);
\draw [c] (1.03473,0.986) -- (0.935135,0.986);
\draw [c] (1.03473,1.128) -- (0.935135,1.128);
\draw [c] (1.03473,1.27) -- (0.935135,1.27);
\draw [c] (1.13432,1.412) -- (0.935135,1.412);
\draw [c] (1.13432,0.702) -- (0.935135,0.702);
\draw [c] (1.03473,0.56) -- (0.935135,0.56);
\draw [c] (1.13432,1.412) -- (0.935135,1.412);
\draw [c] (1.03473,1.554) -- (0.935135,1.554);
\draw [c] (1.03473,1.696) -- (0.935135,1.696);
\draw [c] (1.03473,1.838) -- (0.935135,1.838);
\draw [anchor= east] (0.888378,0.702) node[color=c, rotate=0]{0.5};
\draw [anchor= east] (0.888378,1.412) node[color=c, rotate=0]{1};
\colorlet{c}{natcomp};
\draw [c] (1.01883,1.58118) -- (1.01883,1.58823);
\draw [c] (1.01883,1.58823) -- (1.01883,1.59528);
\draw [c] (0.935135,1.58823) -- (1.01883,1.58823);
\draw [c] (1.01883,1.58823) -- (1.10252,1.58823);
\definecolor{c}{rgb}{0,0,0};
\colorlet{c}{natcomp};
\draw [c] (1.18622,1.63247) -- (1.18622,1.64459);
\draw [c] (1.18622,1.64459) -- (1.18622,1.6567);
\draw [c] (1.10252,1.64459) -- (1.18622,1.64459);
\draw [c] (1.18622,1.64459) -- (1.26991,1.64459);
\definecolor{c}{rgb}{0,0,0};
\colorlet{c}{natcomp};
\draw [c] (1.35361,1.65485) -- (1.35361,1.68982);
\draw [c] (1.35361,1.68982) -- (1.35361,1.7248);
\draw [c] (1.26991,1.68982) -- (1.35361,1.68982);
\draw [c] (1.35361,1.68982) -- (1.4373,1.68982);
\definecolor{c}{rgb}{0,0,0};
\colorlet{c}{natcomp};
\draw [c] (1.521,1.65389) -- (1.521,1.72552);
\draw [c] (1.521,1.72552) -- (1.521,1.79714);
\draw [c] (1.4373,1.72552) -- (1.521,1.72552);
\draw [c] (1.521,1.72552) -- (1.60469,1.72552);
\definecolor{c}{rgb}{0,0,0};
\colorlet{c}{natcomp};
\draw [c] (1.68839,1.69258) -- (1.68839,1.76518);
\draw [c] (1.68839,1.76518) -- (1.68839,1.83778);
\draw [c] (1.60469,1.76518) -- (1.68839,1.76518);
\draw [c] (1.68839,1.76518) -- (1.77208,1.76518);
\definecolor{c}{rgb}{0,0,0};
\colorlet{c}{natcomp};
\draw [c] (1.85578,1.53321) -- (1.85578,1.63122);
\draw [c] (1.85578,1.63122) -- (1.85578,1.72924);
\draw [c] (1.77208,1.63122) -- (1.85578,1.63122);
\draw [c] (1.85578,1.63122) -- (1.93947,1.63122);
\definecolor{c}{rgb}{0,0,0};
\colorlet{c}{natcomp};
\draw [c] (2.02316,1.4736) -- (2.02316,1.60739);
\draw [c] (2.02316,1.60739) -- (2.02316,1.74118);
\draw [c] (1.93947,1.60739) -- (2.02316,1.60739);
\draw [c] (2.02316,1.60739) -- (2.10686,1.60739);
\definecolor{c}{rgb}{0,0,0};
\colorlet{c}{natcomp};
\draw [c] (2.19055,1.67231) -- (2.19055,1.9032);
\draw [c] (2.19055,1.9032) -- (2.19055,1.98);
\draw [c] (2.10686,1.9032) -- (2.19055,1.9032);
\draw [c] (2.19055,1.9032) -- (2.27425,1.9032);
\definecolor{c}{rgb}{0,0,0};
\colorlet{c}{natcomp};
\draw [c] (2.35794,1.48521) -- (2.35794,1.73519);
\draw [c] (2.35794,1.73519) -- (2.35794,1.98);
\draw [c] (2.27425,1.73519) -- (2.35794,1.73519);
\draw [c] (2.35794,1.73519) -- (2.44164,1.73519);
\definecolor{c}{rgb}{0,0,0};
\colorlet{c}{natcomp};
\draw [c] (2.52533,1.1616) -- (2.52533,1.37623);
\draw [c] (2.52533,1.37623) -- (2.52533,1.59086);
\draw [c] (2.44164,1.37623) -- (2.52533,1.37623);
\draw [c] (2.52533,1.37623) -- (2.60903,1.37623);
\definecolor{c}{rgb}{0,0,0};
\colorlet{c}{natcomp};
\draw [c] (2.69272,1.45438) -- (2.69272,1.47785);
\draw [c] (2.69272,1.47785) -- (2.69272,1.50133);
\draw [c] (2.60903,1.47785) -- (2.69272,1.47785);
\draw [c] (2.69272,1.47785) -- (2.77642,1.47785);
\definecolor{c}{rgb}{0,0,0};
\colorlet{c}{natcomp};
\draw [c] (2.86011,1.42389) -- (2.86011,1.44956);
\draw [c] (2.86011,1.44956) -- (2.86011,1.47523);
\draw [c] (2.77642,1.44956) -- (2.86011,1.44956);
\draw [c] (2.86011,1.44956) -- (2.94381,1.44956);
\definecolor{c}{rgb}{0,0,0};
\colorlet{c}{natcomp};
\draw [c] (3.0275,1.37381) -- (3.0275,1.40031);
\draw [c] (3.0275,1.40031) -- (3.0275,1.42681);
\draw [c] (2.94381,1.40031) -- (3.0275,1.40031);
\draw [c] (3.0275,1.40031) -- (3.11119,1.40031);
\definecolor{c}{rgb}{0,0,0};
\colorlet{c}{natcomp};
\draw [c] (3.19489,1.36504) -- (3.19489,1.39329);
\draw [c] (3.19489,1.39329) -- (3.19489,1.42154);
\draw [c] (3.11119,1.39329) -- (3.19489,1.39329);
\draw [c] (3.19489,1.39329) -- (3.27858,1.39329);
\definecolor{c}{rgb}{0,0,0};
\colorlet{c}{natcomp};
\draw [c] (3.36228,1.31084) -- (3.36228,1.33887);
\draw [c] (3.36228,1.33887) -- (3.36228,1.36691);
\draw [c] (3.27858,1.33887) -- (3.36228,1.33887);
\draw [c] (3.36228,1.33887) -- (3.44597,1.33887);
\definecolor{c}{rgb}{0,0,0};
\colorlet{c}{natcomp};
\draw [c] (3.52967,1.26933) -- (3.52967,1.29708);
\draw [c] (3.52967,1.29708) -- (3.52967,1.32483);
\draw [c] (3.44597,1.29708) -- (3.52967,1.29708);
\draw [c] (3.52967,1.29708) -- (3.61336,1.29708);
\definecolor{c}{rgb}{0,0,0};
\colorlet{c}{natcomp};
\draw [c] (3.69706,1.2906) -- (3.69706,1.31959);
\draw [c] (3.69706,1.31959) -- (3.69706,1.34858);
\draw [c] (3.61336,1.31959) -- (3.69706,1.31959);
\draw [c] (3.69706,1.31959) -- (3.78075,1.31959);
\definecolor{c}{rgb}{0,0,0};
\colorlet{c}{natcomp};
\draw [c] (3.86445,1.26228) -- (3.86445,1.29113);
\draw [c] (3.86445,1.29113) -- (3.86445,1.31998);
\draw [c] (3.78075,1.29113) -- (3.86445,1.29113);
\draw [c] (3.86445,1.29113) -- (3.94814,1.29113);
\definecolor{c}{rgb}{0,0,0};
\colorlet{c}{natcomp};
\draw [c] (4.03184,1.23582) -- (4.03184,1.26463);
\draw [c] (4.03184,1.26463) -- (4.03184,1.29344);
\draw [c] (3.94814,1.26463) -- (4.03184,1.26463);
\draw [c] (4.03184,1.26463) -- (4.11553,1.26463);
\definecolor{c}{rgb}{0,0,0};
\colorlet{c}{natcomp};
\draw [c] (4.19922,1.22317) -- (4.19922,1.25221);
\draw [c] (4.19922,1.25221) -- (4.19922,1.28126);
\draw [c] (4.11553,1.25221) -- (4.19922,1.25221);
\draw [c] (4.19922,1.25221) -- (4.28292,1.25221);
\definecolor{c}{rgb}{0,0,0};
\colorlet{c}{natcomp};
\draw [c] (4.36661,1.24345) -- (4.36661,1.27334);
\draw [c] (4.36661,1.27334) -- (4.36661,1.30323);
\draw [c] (4.28292,1.27334) -- (4.36661,1.27334);
\draw [c] (4.36661,1.27334) -- (4.45031,1.27334);
\definecolor{c}{rgb}{0,0,0};
\colorlet{c}{natcomp};
\draw [c] (4.534,1.21286) -- (4.534,1.24256);
\draw [c] (4.534,1.24256) -- (4.534,1.27227);
\draw [c] (4.45031,1.24256) -- (4.534,1.24256);
\draw [c] (4.534,1.24256) -- (4.6177,1.24256);
\definecolor{c}{rgb}{0,0,0};
\colorlet{c}{natcomp};
\draw [c] (4.70139,1.16407) -- (4.70139,1.19327);
\draw [c] (4.70139,1.19327) -- (4.70139,1.22246);
\draw [c] (4.6177,1.19327) -- (4.70139,1.19327);
\draw [c] (4.70139,1.19327) -- (4.78509,1.19327);
\definecolor{c}{rgb}{0,0,0};
\colorlet{c}{natcomp};
\draw [c] (4.86878,1.133) -- (4.86878,1.16214);
\draw [c] (4.86878,1.16214) -- (4.86878,1.19128);
\draw [c] (4.78509,1.16214) -- (4.86878,1.16214);
\draw [c] (4.86878,1.16214) -- (4.95248,1.16214);
\definecolor{c}{rgb}{0,0,0};
\colorlet{c}{natcomp};
\draw [c] (5.03617,1.10763) -- (5.03617,1.13694);
\draw [c] (5.03617,1.13694) -- (5.03617,1.16624);
\draw [c] (4.95248,1.13694) -- (5.03617,1.13694);
\draw [c] (5.03617,1.13694) -- (5.11986,1.13694);
\definecolor{c}{rgb}{0,0,0};
\colorlet{c}{natcomp};
\draw [c] (5.20356,1.08819) -- (5.20356,1.11801);
\draw [c] (5.20356,1.11801) -- (5.20356,1.14783);
\draw [c] (5.11986,1.11801) -- (5.20356,1.11801);
\draw [c] (5.20356,1.11801) -- (5.28725,1.11801);
\definecolor{c}{rgb}{0,0,0};
\colorlet{c}{natcomp};
\draw [c] (5.37095,1.03657) -- (5.37095,1.06569);
\draw [c] (5.37095,1.06569) -- (5.37095,1.09481);
\draw [c] (5.28725,1.06569) -- (5.37095,1.06569);
\draw [c] (5.37095,1.06569) -- (5.45464,1.06569);
\definecolor{c}{rgb}{0,0,0};
\colorlet{c}{natcomp};
\draw [c] (5.53834,1.09699) -- (5.53834,1.12954);
\draw [c] (5.53834,1.12954) -- (5.53834,1.16209);
\draw [c] (5.45464,1.12954) -- (5.53834,1.12954);
\draw [c] (5.53834,1.12954) -- (5.62203,1.12954);
\definecolor{c}{rgb}{0,0,0};
\colorlet{c}{natcomp};
\draw [c] (5.70573,1.00655) -- (5.70573,1.03775);
\draw [c] (5.70573,1.03775) -- (5.70573,1.06895);
\draw [c] (5.62203,1.03775) -- (5.70573,1.03775);
\draw [c] (5.70573,1.03775) -- (5.78942,1.03775);
\definecolor{c}{rgb}{0,0,0};
\colorlet{c}{natcomp};
\draw [c] (5.87312,0.965773) -- (5.87312,0.997011);
\draw [c] (5.87312,0.997011) -- (5.87312,1.02825);
\draw [c] (5.78942,0.997011) -- (5.87312,0.997011);
\draw [c] (5.87312,0.997011) -- (5.95681,0.997011);
\definecolor{c}{rgb}{0,0,0};
\colorlet{c}{natcomp};
\draw [c] (6.04051,0.967418) -- (6.04051,1.00099);
\draw [c] (6.04051,1.00099) -- (6.04051,1.03455);
\draw [c] (5.95681,1.00099) -- (6.04051,1.00099);
\draw [c] (6.04051,1.00099) -- (6.1242,1.00099);
\definecolor{c}{rgb}{0,0,0};
\colorlet{c}{natcomp};
\draw [c] (6.20789,0.951846) -- (6.20789,0.986843);
\draw [c] (6.20789,0.986843) -- (6.20789,1.02184);
\draw [c] (6.1242,0.986843) -- (6.20789,0.986843);
\draw [c] (6.20789,0.986843) -- (6.29159,0.986843);
\definecolor{c}{rgb}{0,0,0};
\colorlet{c}{natcomp};
\draw [c] (6.37528,0.967072) -- (6.37528,1.00411);
\draw [c] (6.37528,1.00411) -- (6.37528,1.04115);
\draw [c] (6.29159,1.00411) -- (6.37528,1.00411);
\draw [c] (6.37528,1.00411) -- (6.45898,1.00411);
\definecolor{c}{rgb}{0,0,0};
\colorlet{c}{natcomp};
\draw [c] (6.54267,0.928294) -- (6.54267,0.966993);
\draw [c] (6.54267,0.966993) -- (6.54267,1.00569);
\draw [c] (6.45898,0.966993) -- (6.54267,0.966993);
\draw [c] (6.54267,0.966993) -- (6.62637,0.966993);
\definecolor{c}{rgb}{0,0,0};
\colorlet{c}{natcomp};
\draw [c] (6.71006,0.993964) -- (6.71006,1.03865);
\draw [c] (6.71006,1.03865) -- (6.71006,1.08333);
\draw [c] (6.62637,1.03865) -- (6.71006,1.03865);
\draw [c] (6.71006,1.03865) -- (6.79376,1.03865);
\definecolor{c}{rgb}{0,0,0};
\colorlet{c}{natcomp};
\draw [c] (6.87745,0.95975) -- (6.87745,1.00519);
\draw [c] (6.87745,1.00519) -- (6.87745,1.05062);
\draw [c] (6.79376,1.00519) -- (6.87745,1.00519);
\draw [c] (6.87745,1.00519) -- (6.96115,1.00519);
\definecolor{c}{rgb}{0,0,0};
\colorlet{c}{natcomp};
\draw [c] (7.04484,0.841521) -- (7.04484,0.886177);
\draw [c] (7.04484,0.886177) -- (7.04484,0.930833);
\draw [c] (6.96115,0.886177) -- (7.04484,0.886177);
\draw [c] (7.04484,0.886177) -- (7.12854,0.886177);
\definecolor{c}{rgb}{0,0,0};
\colorlet{c}{natcomp};
\draw [c] (7.21223,0.863838) -- (7.21223,0.915187);
\draw [c] (7.21223,0.915187) -- (7.21223,0.966535);
\draw [c] (7.12854,0.915187) -- (7.21223,0.915187);
\draw [c] (7.21223,0.915187) -- (7.29592,0.915187);
\definecolor{c}{rgb}{0,0,0};
\colorlet{c}{natcomp};
\draw [c] (7.37962,0.722681) -- (7.37962,0.770311);
\draw [c] (7.37962,0.770311) -- (7.37962,0.817941);
\draw [c] (7.29592,0.770311) -- (7.37962,0.770311);
\draw [c] (7.37962,0.770311) -- (7.46331,0.770311);
\definecolor{c}{rgb}{0,0,0};
\colorlet{c}{natcomp};
\draw [c] (7.54701,0.883883) -- (7.54701,0.949763);
\draw [c] (7.54701,0.949763) -- (7.54701,1.01564);
\draw [c] (7.46331,0.949763) -- (7.54701,0.949763);
\draw [c] (7.54701,0.949763) -- (7.6307,0.949763);
\definecolor{c}{rgb}{0,0,0};
\colorlet{c}{natcomp};
\draw [c] (7.7144,0.822645) -- (7.7144,0.890863);
\draw [c] (7.7144,0.890863) -- (7.7144,0.959082);
\draw [c] (7.6307,0.890863) -- (7.7144,0.890863);
\draw [c] (7.7144,0.890863) -- (7.79809,0.890863);
\definecolor{c}{rgb}{0,0,0};
\colorlet{c}{natcomp};
\draw [c] (7.88179,0.959744) -- (7.88179,1.05252);
\draw [c] (7.88179,1.05252) -- (7.88179,1.1453);
\draw [c] (7.79809,1.05252) -- (7.88179,1.05252);
\draw [c] (7.88179,1.05252) -- (7.96548,1.05252);
\definecolor{c}{rgb}{0,0,0};
\colorlet{c}{natcomp};
\draw [c] (8.04918,0.881614) -- (8.04918,0.977293);
\draw [c] (8.04918,0.977293) -- (8.04918,1.07297);
\draw [c] (7.96548,0.977293) -- (8.04918,0.977293);
\draw [c] (8.04918,0.977293) -- (8.13287,0.977293);
\definecolor{c}{rgb}{0,0,0};
\colorlet{c}{natcomp};
\draw [c] (8.21657,1.0512) -- (8.21657,1.17969);
\draw [c] (8.21657,1.17969) -- (8.21657,1.30817);
\draw [c] (8.13287,1.17969) -- (8.21657,1.17969);
\draw [c] (8.21657,1.17969) -- (8.30026,1.17969);
\definecolor{c}{rgb}{0,0,0};
\colorlet{c}{natcomp};
\draw [c] (8.38395,0.743696) -- (8.38395,0.850603);
\draw [c] (8.38395,0.850603) -- (8.38395,0.95751);
\draw [c] (8.30026,0.850603) -- (8.38395,0.850603);
\draw [c] (8.38395,0.850603) -- (8.46765,0.850603);
\definecolor{c}{rgb}{0,0,0};
\colorlet{c}{natcomp};
\draw [c] (8.55134,0.703704) -- (8.55134,0.815619);
\draw [c] (8.55134,0.815619) -- (8.55134,0.927535);
\draw [c] (8.46765,0.815619) -- (8.55134,0.815619);
\draw [c] (8.55134,0.815619) -- (8.63504,0.815619);
\definecolor{c}{rgb}{0,0,0};
\colorlet{c}{natcomp};
\draw [c] (8.71873,0.743384) -- (8.71873,0.898546);
\draw [c] (8.71873,0.898546) -- (8.71873,1.05371);
\draw [c] (8.63504,0.898546) -- (8.71873,0.898546);
\draw [c] (8.71873,0.898546) -- (8.80243,0.898546);
\definecolor{c}{rgb}{0,0,0};
\colorlet{c}{natcomp};
\draw [c] (8.88612,0.664926) -- (8.88612,0.811217);
\draw [c] (8.88612,0.811217) -- (8.88612,0.957509);
\draw [c] (8.80243,0.811217) -- (8.88612,0.811217);
\draw [c] (8.88612,0.811217) -- (8.96982,0.811217);
\definecolor{c}{rgb}{0,0,0};
\colorlet{c}{natcomp};
\draw [c] (9.05351,0.57102) -- (9.05351,0.719601);
\draw [c] (9.05351,0.719601) -- (9.05351,0.868182);
\draw [c] (8.96982,0.719601) -- (9.05351,0.719601);
\draw [c] (9.05351,0.719601) -- (9.13721,0.719601);
\definecolor{c}{rgb}{0,0,0};
\colorlet{c}{natcomp};
\draw [c] (9.2209,0.56) -- (9.2209,0.628119);
\draw [c] (9.2209,0.628119) -- (9.2209,0.791169);
\draw [c] (9.13721,0.628119) -- (9.2209,0.628119);
\draw [c] (9.2209,0.628119) -- (9.30459,0.628119);
\definecolor{c}{rgb}{0,0,0};
\end{tikzpicture}

\end{infilsf}
\end{minipage}
\begin{minipage}[b]{.3\textwidth}
\caption{Comparing the distribution of invariant masses of events produced by CalcHEP with the CTEQ6 (grayed) and MRST2002nlo (non--grayed) PDFs at different values of $\Lambda$, along with a ratio plot showing the differences between the distributions at each value of $\Lambda$. This is to give an idea of the systematic uncertainty on this distribution due to the choice of PDF. Table~\ref{mrsttab} summarises the fractional variation in predicted events by $\Lambda$ and invariant mass range. The invariant mass ranges used are discussed in the text. \label{mrst}}
\end{minipage}
\end{figure}
\begin{table}[htp]
\begin{minipage}[b]{\textwidth}
\begin{infilsf}{\footnotesize
\begin{center}
\arrayrulecolor{natgreen}
\begin{tabular}[b]{cr!{\color{white}|}r!{\color{white}|}r!{\color{white}|}r}\hline
&&\multicolumn{3}{c}{ \color{natgreen}{\bfseries $M_{\gamma\gamma}$ range [GeV]} } \\
&&\multicolumn{1}{c!{\color{white}\vrule}}{\bfseries [100:1000)} & \multicolumn{1}{c!{\color{white}\vrule}}{\bfseries [1000:3000)} & \multicolumn{1}{c}{\bfseries [3000:5000)} \\ %\cline{3-5}
& \textbf{0.75} & (13.1 ± 0.4) \% & (11.7 ± 0.5) \% & (31.9 ± 1.2) \% \\
&\textbf{1.00} & (12.4 ± 1.0) \% & (8.1 ± 0.5) \% & (32.1 ± 1.7) \% \\
\multirow{-3}{*}{\rotatebox[origin=c]{90}{\color{natgreen}{\bfseries $\Lambda$ [TeV]}}} &\textbf{$\infty$} & (12.0 ± 1.0) \% & (3.1 ± 0.6) \% & (56 ± 21) \%\\\hline
\end{tabular}
\end{center}}\end{infilsf}
\end{minipage}
\caption{The fractional deviation between the distribution of simulated events produced by CalcHEP with the CTEQ6 PDF versus the MSRT2002nlo PDF plotted in fig~\ref{mrst}. We use a $\Lambda$ value of $\infty$ as shorthand for the SM case, since inserting $\infty$ in $\Lambda$s place in eq.~\eqref{rizzo} causes the new term to equal zero. The errors are derived from the statistical errors in those distributions. This will form one of the systematic uncertainties on the final result. Note in regards to the lower right number that the presence of zero--value bins in this part of the distributions may skew the error too low. The lesson here is that while the two PDFs agree reasonably well on the Standard Model prediction, they produce significantly diverging results when the new interaction term is introduced. \label{mrsttab}}
\end{table}

The resulting distributions are plotted in fig~\ref{mrst}, and the difference between the sample generated using the CTEQ6 PDF set and the MRST2002 PDF set are quantified in table~\ref{mrsttab}, broken up into three invariant mass ranges: a low range with a great deal of statistics and very little separation between the three models, a middle range with good statistics and good separation between the models, and a high range where statistics start to run out, especially for the SM sample. Additionally, as we will return to in chapter~\ref{ch.data}, the data sample runs out of statistics below 2 TeV. Looking at the ratio plots in figure~\ref{mrst}, we must conclude that the predictions produced by the to PDFs differ in a non--trivial way, which depends on $\Lambda$. These deviations must be included in the analysis as a systematic uncertainty.

\begin{figure}[htb]
\includegraphics[width=\textwidth]{figures/Zep-soft}
\caption{An illustration of the processes that may surround an interesting event in a proton-proton collision, and the steps required to arrive at a final particle content of that event. In this figure, the dark gray blobs represent the incoming protons and the large red blob represents a hard quark-quark interaction. This figure reproduced from \cite{zep}. For further details on these surrounding processes and their computational representation, see e.g.~\cite{pythman}.
\label{zep}}
\end{figure}

\section{Parton level effects}
The events produced by the event generator(s) represent the hard physical process that occurs in the point where two protons interact. Given that we cannot observe such interactions at the moment they occur, and physics continues both before and after the hard event, we need to expand the scope of physical processes in the simulation, to the point where the resulting event information represents something that we might realistically observe with a detector.

Figure~\ref{zep} gives a schematic overview of the processes that might occur in a proton interaction in addition to the hard process---the red blob---along with a list of steps that a simulation of these processes typically goes through.

Initial and final state radiation are of particular importance in the present analysis, since these can directly produce photons, which might enter into our signal reconstruction.

Figure~\ref{zep} also illustrates how we might see more than one interaction between the constituents of interacting protons. In this case, several additional gluons are emitted.

Final state particles with colour charge, such as gluons, will not remain isolated due to colour confinement. These particles will develop a jet of other coloured particles about themselves, so that the colour charge is neutralised to outside observation. The simulation of the process by which colour charged particles combine into colour neutral hadrons is simulated is called hadronisation. This simulates how a single coloured particle evolves into several colour--neutral hadrons in the space between the interaction point and the detectors. Due to the kinematics involved, the hadrons produced by a single particle will remain close enough together that the detector can not distinguish the individual particles from one another. We refer to such an object as a jet. Since we are dealing with photons in the final state in the present analysis, this step is not crucial in the events generated to study this process specifically, however $\pi^0$ mesons, one of the major backgrounds to the photon signal, are produced in this way. 

In this thesis, the extension of the hard events provided by CalcHEP with these surrounding processes will be carried out in pythia8 \cite{pythia}. Figure~\ref{pythify} illustrates the effect that these surrounding processes have on the distribution of invariant masses.

\begin{figure}[htp]
\centering
\begin{minipage}[b]{.69\textwidth}\hspace{-1.5em}\makebox[0pt][l]{
\noindent\begin{infilsf}
\tiny
\pgfdeclareplotmark{cross} {
\pgfpathmoveto{\pgfpoint{-0.3\pgfplotmarksize}{\pgfplotmarksize}}
\pgfpathlineto{\pgfpoint{+0.3\pgfplotmarksize}{\pgfplotmarksize}}
\pgfpathlineto{\pgfpoint{+0.3\pgfplotmarksize}{0.3\pgfplotmarksize}}
\pgfpathlineto{\pgfpoint{+1\pgfplotmarksize}{0.3\pgfplotmarksize}}
\pgfpathlineto{\pgfpoint{+1\pgfplotmarksize}{-0.3\pgfplotmarksize}}
\pgfpathlineto{\pgfpoint{+0.3\pgfplotmarksize}{-0.3\pgfplotmarksize}}
\pgfpathlineto{\pgfpoint{+0.3\pgfplotmarksize}{-1.\pgfplotmarksize}}
\pgfpathlineto{\pgfpoint{-0.3\pgfplotmarksize}{-1.\pgfplotmarksize}}
\pgfpathlineto{\pgfpoint{-0.3\pgfplotmarksize}{-0.3\pgfplotmarksize}}
\pgfpathlineto{\pgfpoint{-1.\pgfplotmarksize}{-0.3\pgfplotmarksize}}
\pgfpathlineto{\pgfpoint{-1.\pgfplotmarksize}{0.3\pgfplotmarksize}}
\pgfpathlineto{\pgfpoint{-0.3\pgfplotmarksize}{0.3\pgfplotmarksize}}
\pgfpathclose
\pgfusepathqstroke
}
\pgfdeclareplotmark{cross*} {
\pgfpathmoveto{\pgfpoint{-0.3\pgfplotmarksize}{\pgfplotmarksize}}
\pgfpathlineto{\pgfpoint{+0.3\pgfplotmarksize}{\pgfplotmarksize}}
\pgfpathlineto{\pgfpoint{+0.3\pgfplotmarksize}{0.3\pgfplotmarksize}}
\pgfpathlineto{\pgfpoint{+1\pgfplotmarksize}{0.3\pgfplotmarksize}}
\pgfpathlineto{\pgfpoint{+1\pgfplotmarksize}{-0.3\pgfplotmarksize}}
\pgfpathlineto{\pgfpoint{+0.3\pgfplotmarksize}{-0.3\pgfplotmarksize}}
\pgfpathlineto{\pgfpoint{+0.3\pgfplotmarksize}{-1.\pgfplotmarksize}}
\pgfpathlineto{\pgfpoint{-0.3\pgfplotmarksize}{-1.\pgfplotmarksize}}
\pgfpathlineto{\pgfpoint{-0.3\pgfplotmarksize}{-0.3\pgfplotmarksize}}
\pgfpathlineto{\pgfpoint{-1.\pgfplotmarksize}{-0.3\pgfplotmarksize}}
\pgfpathlineto{\pgfpoint{-1.\pgfplotmarksize}{0.3\pgfplotmarksize}}
\pgfpathlineto{\pgfpoint{-0.3\pgfplotmarksize}{0.3\pgfplotmarksize}}
\pgfpathclose
\pgfusepathqfillstroke
}
\pgfdeclareplotmark{newstar} {
\pgfpathmoveto{\pgfqpoint{0pt}{\pgfplotmarksize}}
\pgfpathlineto{\pgfqpointpolar{44}{0.5\pgfplotmarksize}}
\pgfpathlineto{\pgfqpointpolar{18}{\pgfplotmarksize}}
\pgfpathlineto{\pgfqpointpolar{-20}{0.5\pgfplotmarksize}}
\pgfpathlineto{\pgfqpointpolar{-54}{\pgfplotmarksize}}
\pgfpathlineto{\pgfqpointpolar{-90}{0.5\pgfplotmarksize}}
\pgfpathlineto{\pgfqpointpolar{234}{\pgfplotmarksize}}
\pgfpathlineto{\pgfqpointpolar{198}{0.5\pgfplotmarksize}}
\pgfpathlineto{\pgfqpointpolar{162}{\pgfplotmarksize}}
\pgfpathlineto{\pgfqpointpolar{134}{0.5\pgfplotmarksize}}
\pgfpathclose
\pgfusepathqstroke
}
\pgfdeclareplotmark{newstar*} {
\pgfpathmoveto{\pgfqpoint{0pt}{\pgfplotmarksize}}
\pgfpathlineto{\pgfqpointpolar{44}{0.5\pgfplotmarksize}}
\pgfpathlineto{\pgfqpointpolar{18}{\pgfplotmarksize}}
\pgfpathlineto{\pgfqpointpolar{-20}{0.5\pgfplotmarksize}}
\pgfpathlineto{\pgfqpointpolar{-54}{\pgfplotmarksize}}
\pgfpathlineto{\pgfqpointpolar{-90}{0.5\pgfplotmarksize}}
\pgfpathlineto{\pgfqpointpolar{234}{\pgfplotmarksize}}
\pgfpathlineto{\pgfqpointpolar{198}{0.5\pgfplotmarksize}}
\pgfpathlineto{\pgfqpointpolar{162}{\pgfplotmarksize}}
\pgfpathlineto{\pgfqpointpolar{134}{0.5\pgfplotmarksize}}
\pgfpathclose
\pgfusepathqfillstroke
}
\begin{tikzpicture}[x=0.05\textwidth,y=.05\textwidth]

\definecolor{c}{rgb}{0,0,0};
\draw [c] (2,11.8008) -- (2,22.4216) -- (18,22.4216) -- (18,11.8008) -- (2,11.8008);

\definecolor{c}{rgb}{0,0,0};
\draw [c] (2,11.8008) -- (2,22.4216) -- (18,22.4216) -- (18,11.8008) -- (2,11.8008);
\colorlet{c}{kugray!50};
\draw [c] (2.16,22.0847) -- (2.16,22.0869);
\draw [c] (2.16,22.0869) -- (2.16,22.0891);
\draw [c] (2,22.0869) -- (2.16,22.0869);
\draw [c] (2.16,22.0869) -- (2.32,22.0869);
\draw [c] (2.48,20.989) -- (2.48,20.9954);
\draw [c] (2.48,20.9954) -- (2.48,21.0016);
\draw [c] (2.32,20.9954) -- (2.48,20.9954);
\draw [c] (2.48,20.9954) -- (2.64,20.9954);
\draw [c] (2.8,20.2777) -- (2.8,20.2902);
\draw [c] (2.8,20.2902) -- (2.8,20.3024);
\draw [c] (2.64,20.2902) -- (2.8,20.2902);
\draw [c] (2.8,20.2902) -- (2.96,20.2902);
\draw [c] (3.12,19.7357) -- (3.12,19.7568);
\draw [c] (3.12,19.7568) -- (3.12,19.777);
\draw [c] (2.96,19.7568) -- (3.12,19.7568);
\draw [c] (3.12,19.7568) -- (3.28,19.7568);
\draw [c] (3.44,19.2804) -- (3.44,19.313);
\draw [c] (3.44,19.313) -- (3.44,19.3436);
\draw [c] (3.28,19.313) -- (3.44,19.313);
\draw [c] (3.44,19.313) -- (3.6,19.313);
\draw [c] (3.76,18.9425) -- (3.76,18.9876);
\draw [c] (3.76,18.9876) -- (3.76,19.0291);
\draw [c] (3.6,18.9876) -- (3.76,18.9876);
\draw [c] (3.76,18.9876) -- (3.92,18.9876);
\draw [c] (4.08,18.4927) -- (4.08,18.5621);
\draw [c] (4.08,18.5621) -- (4.08,18.6234);
\draw [c] (3.92,18.5621) -- (4.08,18.5621);
\draw [c] (4.08,18.5621) -- (4.24,18.5621);
\draw [c] (4.4,18.1352) -- (4.4,18.2331);
\draw [c] (4.4,18.2331) -- (4.4,18.3154);
\draw [c] (4.24,18.2331) -- (4.4,18.2331);
\draw [c] (4.4,18.2331) -- (4.56,18.2331);
\draw [c] (4.72,17.8253) -- (4.72,17.957);
\draw [c] (4.72,17.957) -- (4.72,18.062);
\draw [c] (4.56,17.957) -- (4.72,17.957);
\draw [c] (4.72,17.957) -- (4.88,17.957);
\draw [c] (5.04,17.7645) -- (5.04,17.8494);
\draw [c] (5.04,17.8494) -- (5.04,17.9225);
\draw [c] (4.88,17.8494) -- (5.04,17.8494);
\draw [c] (5.04,17.8494) -- (5.2,17.8494);
\draw [c] (5.36,17.4837) -- (5.36,17.4878);
\draw [c] (5.36,17.4878) -- (5.36,17.4919);
\draw [c] (5.2,17.4878) -- (5.36,17.4878);
\draw [c] (5.36,17.4878) -- (5.52,17.4878);
\draw [c] (5.68,17.2475) -- (5.68,17.2526);
\draw [c] (5.68,17.2526) -- (5.68,17.2577);
\draw [c] (5.52,17.2526) -- (5.68,17.2526);
\draw [c] (5.68,17.2526) -- (5.84,17.2526);
\draw [c] (6,16.9982) -- (6,17.0047);
\draw [c] (6,17.0047) -- (6,17.0112);
\draw [c] (5.84,17.0047) -- (6,17.0047);
\draw [c] (6,17.0047) -- (6.16,17.0047);
\draw [c] (6.32,16.7792) -- (6.32,16.7873);
\draw [c] (6.32,16.7873) -- (6.32,16.7952);
\draw [c] (6.16,16.7873) -- (6.32,16.7873);
\draw [c] (6.32,16.7873) -- (6.48,16.7873);
\draw [c] (6.64,16.5651) -- (6.64,16.575);
\draw [c] (6.64,16.575) -- (6.64,16.5847);
\draw [c] (6.48,16.575) -- (6.64,16.575);
\draw [c] (6.64,16.575) -- (6.8,16.575);
\draw [c] (6.96,16.3657) -- (6.96,16.3776);
\draw [c] (6.96,16.3776) -- (6.96,16.3893);
\draw [c] (6.8,16.3776) -- (6.96,16.3776);
\draw [c] (6.96,16.3776) -- (7.12,16.3776);
\draw [c] (7.28,16.1641) -- (7.28,16.1786);
\draw [c] (7.28,16.1786) -- (7.28,16.1927);
\draw [c] (7.12,16.1786) -- (7.28,16.1786);
\draw [c] (7.28,16.1786) -- (7.44,16.1786);
\draw [c] (7.6,15.9868) -- (7.6,16.004);
\draw [c] (7.6,16.004) -- (7.6,16.0206);
\draw [c] (7.44,16.004) -- (7.6,16.004);
\draw [c] (7.6,16.004) -- (7.76,16.004);
\draw [c] (7.92,15.7677) -- (7.92,15.7889);
\draw [c] (7.92,15.7889) -- (7.92,15.8093);
\draw [c] (7.76,15.7889) -- (7.92,15.7889);
\draw [c] (7.92,15.7889) -- (8.08,15.7889);
\draw [c] (8.24,15.4855) -- (8.24,15.5133);
\draw [c] (8.24,15.5133) -- (8.24,15.5398);
\draw [c] (8.08,15.5133) -- (8.24,15.5133);
\draw [c] (8.24,15.5133) -- (8.4,15.5133);
\draw [c] (8.56,15.2517) -- (8.56,15.2866);
\draw [c] (8.56,15.2866) -- (8.56,15.3193);
\draw [c] (8.4,15.2866) -- (8.56,15.2866);
\draw [c] (8.56,15.2866) -- (8.72,15.2866);
\draw [c] (8.88,14.9685) -- (8.88,15.0142);
\draw [c] (8.88,15.0142) -- (8.88,15.0563);
\draw [c] (8.72,15.0142) -- (8.88,15.0142);
\draw [c] (8.88,15.0142) -- (9.04,15.0142);
\draw [c] (9.2,14.9072) -- (9.2,14.9557);
\draw [c] (9.2,14.9557) -- (9.2,15.0001);
\draw [c] (9.04,14.9557) -- (9.2,14.9557);
\draw [c] (9.2,14.9557) -- (9.36,14.9557);
\draw [c] (9.52,14.6847) -- (9.52,14.7448);
\draw [c] (9.52,14.7448) -- (9.52,14.7987);
\draw [c] (9.36,14.7448) -- (9.52,14.7448);
\draw [c] (9.52,14.7448) -- (9.68,14.7448);
\draw [c] (9.84,14.3956) -- (9.84,14.4749);
\draw [c] (9.84,14.4749) -- (9.84,14.5437);
\draw [c] (9.68,14.4749) -- (9.84,14.4749);
\draw [c] (9.84,14.4749) -- (10,14.4749);
\draw [c] (10.16,14.24) -- (10.16,14.3321);
\draw [c] (10.16,14.3321) -- (10.16,14.4103);
\draw [c] (10,14.3321) -- (10.16,14.3321);
\draw [c] (10.16,14.3321) -- (10.32,14.3321);
\draw [c] (10.48,13.9492) -- (10.48,14.0709);
\draw [c] (10.48,14.0709) -- (10.48,14.1694);
\draw [c] (10.32,14.0709) -- (10.48,14.0709);
\draw [c] (10.48,14.0709) -- (10.64,14.0709);
\draw [c] (10.8,13.923) -- (10.8,14.0478);
\draw [c] (10.8,14.0478) -- (10.8,14.1483);
\draw [c] (10.64,14.0478) -- (10.8,14.0478);
\draw [c] (10.8,14.0478) -- (10.96,14.0478);
\draw [c] (11.12,13.6051) -- (11.12,13.7741);
\draw [c] (11.12,13.7741) -- (11.12,13.9014);
\draw [c] (10.96,13.7741) -- (11.12,13.7741);
\draw [c] (11.12,13.7741) -- (11.28,13.7741);
\draw [c] (11.44,13.5552) -- (11.44,13.7324);
\draw [c] (11.44,13.7324) -- (11.44,13.8644);
\draw [c] (11.28,13.7324) -- (11.44,13.7324);
\draw [c] (11.44,13.7324) -- (11.6,13.7324);
\draw [c] (11.76,13.5005) -- (11.76,13.6872);
\draw [c] (11.76,13.6872) -- (11.76,13.8243);
\draw [c] (11.6,13.6872) -- (11.76,13.6872);
\draw [c] (11.76,13.6872) -- (11.92,13.6872);
\draw [c] (12.08,12.8003) -- (12.08,13.1609);
\draw [c] (12.08,13.1609) -- (12.08,13.3718);
\draw [c] (11.92,13.1609) -- (12.08,13.1609);
\draw [c] (12.08,13.1609) -- (12.24,13.1609);
\draw [c] (12.4,13.5005) -- (12.4,13.6872);
\draw [c] (12.4,13.6872) -- (12.4,13.8243);
\draw [c] (12.24,13.6872) -- (12.4,13.6872);
\draw [c] (12.4,13.6872) -- (12.56,13.6872);
\draw [c] (12.72,11.8008) -- (12.72,12.4397);
\draw [c] (12.72,12.4397) -- (12.72,12.8003);
\draw [c] (12.56,12.4397) -- (12.72,12.4397);
\draw [c] (12.72,12.4397) -- (12.88,12.4397);
\draw [c] (13.04,11.8008) -- (13.04,12.4397);
\draw [c] (13.04,12.4397) -- (13.04,12.8003);
\draw [c] (12.88,12.4397) -- (13.04,12.4397);
\draw [c] (13.04,12.4397) -- (13.2,12.4397);
\draw [c] (13.36,11.8008) -- (13.36,12.4397);
\draw [c] (13.36,12.4397) -- (13.36,12.8003);
\draw [c] (13.2,12.4397) -- (13.36,12.4397);
\draw [c] (13.36,12.4397) -- (13.52,12.4397);
\draw [c] (13.68,12.1614) -- (13.68,12.8003);
\draw [c] (13.68,12.8003) -- (13.68,13.0785);
\draw [c] (13.52,12.8003) -- (13.68,12.8003);
\draw [c] (13.68,12.8003) -- (13.84,12.8003);
\draw [c] (14,11.8008) -- (14,12.4397);
\draw [c] (14,12.4397) -- (14,12.8003);
\draw [c] (13.84,12.4397) -- (14,12.4397);
\draw [c] (14,12.4397) -- (14.16,12.4397);
\draw [c] (14.32,11.8008) -- (14.32,12.4397);
\draw [c] (14.32,12.4397) -- (14.32,12.8003);
\draw [c] (14.16,12.4397) -- (14.32,12.4397);
\draw [c] (14.32,12.4397) -- (14.48,12.4397);
\draw [c] (14.64,11.8008) -- (14.64,12.4397);
\draw [c] (14.64,12.4397) -- (14.64,12.8003);
\draw [c] (14.48,12.4397) -- (14.64,12.4397);
\draw [c] (14.64,12.4397) -- (14.8,12.4397);
\draw [c] (15.28,11.8008) -- (15.28,12.4397);
\draw [c] (15.28,12.4397) -- (15.28,12.8003);
\draw [c] (15.12,12.4397) -- (15.28,12.4397);
\draw [c] (15.28,12.4397) -- (15.44,12.4397);
\definecolor{c}{rgb}{0,0,0};
\draw [c] (2,11.8008) -- (18,11.8008);
\draw [c] (3.30612,12.084) -- (3.30612,11.8008);
\draw [c] (3.63265,11.9424) -- (3.63265,11.8008);
\draw [c] (3.95918,11.9424) -- (3.95918,11.8008);
\draw [c] (4.28571,11.9424) -- (4.28571,11.8008);
\draw [c] (4.61225,11.9424) -- (4.61225,11.8008);
\draw [c] (4.93878,12.084) -- (4.93878,11.8008);
\draw [c] (5.26531,11.9424) -- (5.26531,11.8008);
\draw [c] (5.59184,11.9424) -- (5.59184,11.8008);
\draw [c] (5.91837,11.9424) -- (5.91837,11.8008);
\draw [c] (6.2449,11.9424) -- (6.2449,11.8008);
\draw [c] (6.57143,12.084) -- (6.57143,11.8008);
\draw [c] (6.89796,11.9424) -- (6.89796,11.8008);
\draw [c] (7.22449,11.9424) -- (7.22449,11.8008);
\draw [c] (7.55102,11.9424) -- (7.55102,11.8008);
\draw [c] (7.87755,11.9424) -- (7.87755,11.8008);
\draw [c] (8.20408,12.084) -- (8.20408,11.8008);
\draw [c] (8.53061,11.9424) -- (8.53061,11.8008);
\draw [c] (8.85714,11.9424) -- (8.85714,11.8008);
\draw [c] (9.18367,11.9424) -- (9.18367,11.8008);
\draw [c] (9.5102,11.9424) -- (9.5102,11.8008);
\draw [c] (9.83673,12.084) -- (9.83673,11.8008);
\draw [c] (10.1633,11.9424) -- (10.1633,11.8008);
\draw [c] (10.4898,11.9424) -- (10.4898,11.8008);
\draw [c] (10.8163,11.9424) -- (10.8163,11.8008);
\draw [c] (11.1429,11.9424) -- (11.1429,11.8008);
\draw [c] (11.4694,12.084) -- (11.4694,11.8008);
\draw [c] (11.7959,11.9424) -- (11.7959,11.8008);
\draw [c] (12.1224,11.9424) -- (12.1224,11.8008);
\draw [c] (12.449,11.9424) -- (12.449,11.8008);
\draw [c] (12.7755,11.9424) -- (12.7755,11.8008);
\draw [c] (13.102,12.084) -- (13.102,11.8008);
\draw [c] (13.4286,11.9424) -- (13.4286,11.8008);
\draw [c] (13.7551,11.9424) -- (13.7551,11.8008);
\draw [c] (14.0816,11.9424) -- (14.0816,11.8008);
\draw [c] (14.4082,11.9424) -- (14.4082,11.8008);
\draw [c] (14.7347,12.084) -- (14.7347,11.8008);
\draw [c] (15.0612,11.9424) -- (15.0612,11.8008);
\draw [c] (15.3878,11.9424) -- (15.3878,11.8008);
\draw [c] (15.7143,11.9424) -- (15.7143,11.8008);
\draw [c] (16.0408,11.9424) -- (16.0408,11.8008);
\draw [c] (16.3673,12.084) -- (16.3673,11.8008);
\draw [c] (16.6939,11.9424) -- (16.6939,11.8008);
\draw [c] (17.0204,11.9424) -- (17.0204,11.8008);
\draw [c] (17.3469,11.9424) -- (17.3469,11.8008);
\draw [c] (17.6735,11.9424) -- (17.6735,11.8008);
\draw [c] (18,12.084) -- (18,11.8008);
\draw [c] (3.30612,12.084) -- (3.30612,11.8008);
\draw [c] (2.97959,11.9424) -- (2.97959,11.8008);
\draw [c] (2.65306,11.9424) -- (2.65306,11.8008);
\draw [c] (2.32653,11.9424) -- (2.32653,11.8008);

\draw [c] (2,11.8008) -- (2,22.4216);
\draw [anchor= east] (0.1,22.4216) node[ rotate=90]{$\di\sigma/\di M_{\gamma\gamma}$ [pb/GeV]};
\draw [c] (2.27,11.8185) -- (2,11.8185);
\draw [c] (2.27,11.8986) -- (2,11.8986);
\draw [c] (2.27,11.9681) -- (2,11.9681);
\draw [c] (2.27,12.0294) -- (2,12.0294);
\draw [c] (2.54,12.0842) -- (2,12.0842);
\draw [anchor= east] (1.844,12.0842) node[ rotate=0]{$10^{-10}$};
\draw [c] (2.27,12.4448) -- (2,12.4448);
\draw [c] (2.27,12.6558) -- (2,12.6558);
\draw [c] (2.27,12.8054) -- (2,12.8054);
\draw [c] (2.27,12.9215) -- (2,12.9215);
\draw [c] (2.27,13.0164) -- (2,13.0164);
\draw [c] (2.27,13.0966) -- (2,13.0966);
\draw [c] (2.27,13.166) -- (2,13.166);
\draw [c] (2.27,13.2273) -- (2,13.2273);
\draw [c] (2.54,13.2821) -- (2,13.2821);
\draw [anchor= east] (1.844,13.2821) node[ rotate=0]{$10^{-9}$};
\draw [c] (2.27,13.6427) -- (2,13.6427);
\draw [c] (2.27,13.8537) -- (2,13.8537);
\draw [c] (2.27,14.0033) -- (2,14.0033);
\draw [c] (2.27,14.1194) -- (2,14.1194);
\draw [c] (2.27,14.2143) -- (2,14.2143);
\draw [c] (2.27,14.2945) -- (2,14.2945);
\draw [c] (2.27,14.364) -- (2,14.364);
\draw [c] (2.27,14.4252) -- (2,14.4252);
\draw [c] (2.54,14.48) -- (2,14.48);
\draw [anchor= east] (1.844,14.48) node[ rotate=0]{$10^{-8}$};
\draw [c] (2.27,14.8407) -- (2,14.8407);
\draw [c] (2.27,15.0516) -- (2,15.0516);
\draw [c] (2.27,15.2013) -- (2,15.2013);
\draw [c] (2.27,15.3174) -- (2,15.3174);
\draw [c] (2.27,15.4122) -- (2,15.4122);
\draw [c] (2.27,15.4924) -- (2,15.4924);
\draw [c] (2.27,15.5619) -- (2,15.5619);
\draw [c] (2.27,15.6231) -- (2,15.6231);
\draw [c] (2.54,15.678) -- (2,15.678);
\draw [anchor= east] (1.844,15.678) node[ rotate=0]{$10^{-7}$};
\draw [c] (2.27,16.0386) -- (2,16.0386);
\draw [c] (2.27,16.2495) -- (2,16.2495);
\draw [c] (2.27,16.3992) -- (2,16.3992);
\draw [c] (2.27,16.5153) -- (2,16.5153);
\draw [c] (2.27,16.6101) -- (2,16.6101);
\draw [c] (2.27,16.6903) -- (2,16.6903);
\draw [c] (2.27,16.7598) -- (2,16.7598);
\draw [c] (2.27,16.8211) -- (2,16.8211);
\draw [c] (2.54,16.8759) -- (2,16.8759);
\draw [anchor= east] (1.844,16.8759) node[ rotate=0]{$10^{-6}$};
\draw [c] (2.27,17.2365) -- (2,17.2365);
\draw [c] (2.27,17.4474) -- (2,17.4474);
\draw [c] (2.27,17.5971) -- (2,17.5971);
\draw [c] (2.27,17.7132) -- (2,17.7132);
\draw [c] (2.27,17.808) -- (2,17.808);
\draw [c] (2.27,17.8882) -- (2,17.8882);
\draw [c] (2.27,17.9577) -- (2,17.9577);
\draw [c] (2.27,18.019) -- (2,18.019);
\draw [c] (2.54,18.0738) -- (2,18.0738);
\draw [anchor= east] (1.844,18.0738) node[ rotate=0]{$10^{-5}$};
\draw [c] (2.27,18.4344) -- (2,18.4344);
\draw [c] (2.27,18.6453) -- (2,18.6453);
\draw [c] (2.27,18.795) -- (2,18.795);
\draw [c] (2.27,18.9111) -- (2,18.9111);
\draw [c] (2.27,19.006) -- (2,19.006);
\draw [c] (2.27,19.0861) -- (2,19.0861);
\draw [c] (2.27,19.1556) -- (2,19.1556);
\draw [c] (2.27,19.2169) -- (2,19.2169);
\draw [c] (2.54,19.2717) -- (2,19.2717);
\draw [anchor= east] (1.844,19.2717) node[ rotate=0]{$10^{-4}$};
\draw [c] (2.27,19.6323) -- (2,19.6323);
\draw [c] (2.27,19.8433) -- (2,19.8433);
\draw [c] (2.27,19.9929) -- (2,19.9929);
\draw [c] (2.27,20.109) -- (2,20.109);
\draw [c] (2.27,20.2039) -- (2,20.2039);
\draw [c] (2.27,20.2841) -- (2,20.2841);
\draw [c] (2.27,20.3535) -- (2,20.3535);
\draw [c] (2.27,20.4148) -- (2,20.4148);
\draw [c] (2.54,20.4696) -- (2,20.4696);
\draw [anchor= east] (1.844,20.4696) node[ rotate=0]{$10^{-3}$};
\draw [c] (2.27,20.8302) -- (2,20.8302);
\draw [c] (2.27,21.0412) -- (2,21.0412);
\draw [c] (2.27,21.1908) -- (2,21.1908);
\draw [c] (2.27,21.3069) -- (2,21.3069);
\draw [c] (2.27,21.4018) -- (2,21.4018);
\draw [c] (2.27,21.482) -- (2,21.482);
\draw [c] (2.27,21.5515) -- (2,21.5515);
\draw [c] (2.27,21.6127) -- (2,21.6127);
\draw [c] (2.54,21.6675) -- (2,21.6675);
\draw [anchor= east] (1.844,21.6675) node[ rotate=0]{$10^{-2}$};
\draw [c] (2.27,22.0282) -- (2,22.0282);
\draw [c] (2.27,22.2391) -- (2,22.2391);
\draw [c] (2.27,22.3888) -- (2,22.3888);
\colorlet{c}{kugray};
\draw [c] (2.16,21.9886) -- (2.16,21.9911);
\draw [c] (2.16,21.9911) -- (2.16,21.9935);
\draw [c] (2,21.9911) -- (2.16,21.9911);
\draw [c] (2.16,21.9911) -- (2.32,21.9911);
\draw [c] (2.48,20.8999) -- (2.48,20.9067);
\draw [c] (2.48,20.9067) -- (2.48,20.9135);
\draw [c] (2.32,20.9067) -- (2.48,20.9067);
\draw [c] (2.48,20.9067) -- (2.64,20.9067);
\draw [c] (2.8,20.1651) -- (2.8,20.179);
\draw [c] (2.8,20.179) -- (2.8,20.1926);
\draw [c] (2.64,20.179) -- (2.8,20.179);
\draw [c] (2.8,20.179) -- (2.96,20.179);
\draw [c] (3.12,19.6256) -- (3.12,19.649);
\draw [c] (3.12,19.649) -- (3.12,19.6714);
\draw [c] (2.96,19.649) -- (3.12,19.649);
\draw [c] (3.12,19.649) -- (3.28,19.649);
\draw [c] (3.44,19.2037) -- (3.44,19.2388);
\draw [c] (3.44,19.2388) -- (3.44,19.2717);
\draw [c] (3.28,19.2388) -- (3.44,19.2388);
\draw [c] (3.44,19.2388) -- (3.6,19.2388);
\draw [c] (3.76,18.8394) -- (3.76,18.8892);
\draw [c] (3.76,18.8892) -- (3.76,18.9346);
\draw [c] (3.6,18.8892) -- (3.76,18.8892);
\draw [c] (3.76,18.8892) -- (3.92,18.8892);
\draw [c] (4.08,18.3977) -- (4.08,18.4738);
\draw [c] (4.08,18.4738) -- (4.08,18.5401);
\draw [c] (3.92,18.4738) -- (4.08,18.4738);
\draw [c] (4.08,18.4738) -- (4.24,18.4738);
\draw [c] (4.4,18.0231) -- (4.4,18.1321);
\draw [c] (4.4,18.1321) -- (4.4,18.2221);
\draw [c] (4.24,18.1321) -- (4.4,18.1321);
\draw [c] (4.4,18.1321) -- (4.56,18.1321);
\draw [c] (4.72,17.5639) -- (4.72,17.7329);
\draw [c] (4.72,17.7329) -- (4.72,17.8603);
\draw [c] (4.56,17.7329) -- (4.72,17.7329);
\draw [c] (4.72,17.7329) -- (4.88,17.7329);
\draw [c] (5.04,17.6487) -- (5.04,17.7446);
\draw [c] (5.04,17.7446) -- (5.04,17.8255);
\draw [c] (4.88,17.7446) -- (5.04,17.7446);
\draw [c] (5.04,17.7446) -- (5.2,17.7446);
\draw [c] (5.36,17.3748) -- (5.36,17.3793);
\draw [c] (5.36,17.3793) -- (5.36,17.3838);
\draw [c] (5.2,17.3793) -- (5.36,17.3793);
\draw [c] (5.36,17.3793) -- (5.52,17.3793);
\draw [c] (5.68,17.1419) -- (5.68,17.1476);
\draw [c] (5.68,17.1476) -- (5.68,17.1532);
\draw [c] (5.52,17.1476) -- (5.68,17.1476);
\draw [c] (5.68,17.1476) -- (5.84,17.1476);
\draw [c] (6,16.8903) -- (6,16.8975);
\draw [c] (6,16.8975) -- (6,16.9046);
\draw [c] (5.84,16.8975) -- (6,16.8975);
\draw [c] (6,16.8975) -- (6.16,16.8975);
\draw [c] (6.32,16.6718) -- (6.32,16.6807);
\draw [c] (6.32,16.6807) -- (6.32,16.6894);
\draw [c] (6.16,16.6807) -- (6.32,16.6807);
\draw [c] (6.32,16.6807) -- (6.48,16.6807);
\draw [c] (6.64,16.4482) -- (6.64,16.4592);
\draw [c] (6.64,16.4592) -- (6.64,16.47);
\draw [c] (6.48,16.4592) -- (6.64,16.4592);
\draw [c] (6.64,16.4592) -- (6.8,16.4592);
\draw [c] (6.96,16.2518) -- (6.96,16.2651);
\draw [c] (6.96,16.2651) -- (6.96,16.2781);
\draw [c] (6.8,16.2651) -- (6.96,16.2651);
\draw [c] (6.96,16.2651) -- (7.12,16.2651);
\draw [c] (7.28,16.043) -- (7.28,16.0593);
\draw [c] (7.28,16.0593) -- (7.28,16.0751);
\draw [c] (7.12,16.0593) -- (7.28,16.0593);
\draw [c] (7.28,16.0593) -- (7.44,16.0593);
\draw [c] (7.6,15.8872) -- (7.6,15.9061);
\draw [c] (7.6,15.9061) -- (7.6,15.9244);
\draw [c] (7.44,15.9061) -- (7.6,15.9061);
\draw [c] (7.6,15.9061) -- (7.76,15.9061);
\draw [c] (7.92,15.6564) -- (7.92,15.68);
\draw [c] (7.92,15.68) -- (7.92,15.7026);
\draw [c] (7.76,15.68) -- (7.92,15.68);
\draw [c] (7.92,15.68) -- (8.08,15.68);
\draw [c] (8.24,15.3725) -- (8.24,15.4036);
\draw [c] (8.24,15.4036) -- (8.24,15.4329);
\draw [c] (8.08,15.4036) -- (8.24,15.4036);
\draw [c] (8.24,15.4036) -- (8.4,15.4036);
\draw [c] (8.56,15.1099) -- (8.56,15.1499);
\draw [c] (8.56,15.1499) -- (8.56,15.187);
\draw [c] (8.4,15.1499) -- (8.56,15.1499);
\draw [c] (8.56,15.1499) -- (8.72,15.1499);
\draw [c] (8.88,14.862) -- (8.88,14.9127);
\draw [c] (8.88,14.9127) -- (8.88,14.9589);
\draw [c] (8.72,14.9127) -- (8.88,14.9127);
\draw [c] (8.88,14.9127) -- (9.04,14.9127);
\draw [c] (9.2,14.8126) -- (9.2,14.8658);
\draw [c] (9.2,14.8658) -- (9.2,14.914);
\draw [c] (9.04,14.8658) -- (9.2,14.8658);
\draw [c] (9.2,14.8658) -- (9.36,14.8658);
\draw [c] (9.52,14.5837) -- (9.52,14.6499);
\draw [c] (9.52,14.6499) -- (9.52,14.7087);
\draw [c] (9.36,14.6499) -- (9.52,14.6499);
\draw [c] (9.52,14.6499) -- (9.68,14.6499);
\draw [c] (9.84,14.2548) -- (9.84,14.3456);
\draw [c] (9.84,14.3456) -- (9.84,14.4229);
\draw [c] (9.68,14.3456) -- (9.84,14.3456);
\draw [c] (9.84,14.3456) -- (10,14.3456);
\draw [c] (10.16,14.1042) -- (10.16,14.2091);
\draw [c] (10.16,14.2091) -- (10.16,14.2964);
\draw [c] (10,14.2091) -- (10.16,14.2091);
\draw [c] (10.16,14.2091) -- (10.32,14.2091);
\draw [c] (10.48,13.7691) -- (10.48,13.9136);
\draw [c] (10.48,13.9136) -- (10.48,14.0266);
\draw [c] (10.32,13.9136) -- (10.48,13.9136);
\draw [c] (10.48,13.9136) -- (10.64,13.9136);
\draw [c] (10.8,13.7691) -- (10.8,13.9136);
\draw [c] (10.8,13.9136) -- (10.8,14.0266);
\draw [c] (10.64,13.9136) -- (10.8,13.9136);
\draw [c] (10.8,13.9136) -- (10.96,13.9136);
\draw [c] (11.12,13.5005) -- (11.12,13.6872);
\draw [c] (11.12,13.6872) -- (11.12,13.8243);
\draw [c] (10.96,13.6872) -- (11.12,13.6872);
\draw [c] (11.12,13.6872) -- (11.28,13.6872);
\draw [c] (11.44,13.4398) -- (11.44,13.6376);
\draw [c] (11.44,13.6376) -- (11.44,13.7805);
\draw [c] (11.28,13.6376) -- (11.44,13.6376);
\draw [c] (11.44,13.6376) -- (11.6,13.6376);
\draw [c] (11.76,13.3718) -- (11.76,13.5828);
\draw [c] (11.76,13.5828) -- (11.76,13.7324);
\draw [c] (11.6,13.5828) -- (11.76,13.5828);
\draw [c] (11.76,13.5828) -- (11.92,13.5828);
\draw [c] (12.08,12.1614) -- (12.08,12.8003);
\draw [c] (12.08,12.8003) -- (12.08,13.0785);
\draw [c] (11.92,12.8003) -- (12.08,12.8003);
\draw [c] (12.08,12.8003) -- (12.24,12.8003);
\draw [c] (12.4,13.4398) -- (12.4,13.6376);
\draw [c] (12.4,13.6376) -- (12.4,13.7805);
\draw [c] (12.24,13.6376) -- (12.4,13.6376);
\draw [c] (12.4,13.6376) -- (12.56,13.6376);
\draw [c] (13.04,11.8008) -- (13.04,12.4397);
\draw [c] (13.04,12.4397) -- (13.04,12.8003);
\draw [c] (12.88,12.4397) -- (13.04,12.4397);
\draw [c] (13.04,12.4397) -- (13.2,12.4397);
\draw [c] (13.36,11.8008) -- (13.36,12.4397);
\draw [c] (13.36,12.4397) -- (13.36,12.8003);
\draw [c] (13.2,12.4397) -- (13.36,12.4397);
\draw [c] (13.36,12.4397) -- (13.52,12.4397);
\draw [c] (13.68,11.8008) -- (13.68,12.4397);
\draw [c] (13.68,12.4397) -- (13.68,12.8003);
\draw [c] (13.52,12.4397) -- (13.68,12.4397);
\draw [c] (13.68,12.4397) -- (13.84,12.4397);
\colorlet{c}{natgreen!50};
\draw [c] (2.16,22.0854) -- (2.16,22.0876);
\draw [c] (2.16,22.0876) -- (2.16,22.0898);
\draw [c] (2,22.0876) -- (2.16,22.0876);
\draw [c] (2.16,22.0876) -- (2.32,22.0876);
\draw [c] (2.48,21.0058) -- (2.48,21.012);
\draw [c] (2.48,21.012) -- (2.48,21.0181);
\draw [c] (2.32,21.012) -- (2.48,21.012);
\draw [c] (2.48,21.012) -- (2.64,21.012);
\draw [c] (2.8,20.2599) -- (2.8,20.2726);
\draw [c] (2.8,20.2726) -- (2.8,20.2851);
\draw [c] (2.64,20.2726) -- (2.8,20.2726);
\draw [c] (2.8,20.2726) -- (2.96,20.2726);
\draw [c] (3.12,19.7099) -- (3.12,19.7315);
\draw [c] (3.12,19.7315) -- (3.12,19.7522);
\draw [c] (2.96,19.7315) -- (3.12,19.7315);
\draw [c] (3.12,19.7315) -- (3.28,19.7315);
\draw [c] (3.44,19.3211) -- (3.44,19.3525);
\draw [c] (3.44,19.3525) -- (3.44,19.3821);
\draw [c] (3.28,19.3525) -- (3.44,19.3525);
\draw [c] (3.44,19.3525) -- (3.6,19.3525);
\draw [c] (3.76,18.928) -- (3.76,18.9738);
\draw [c] (3.76,18.9738) -- (3.76,19.0158);
\draw [c] (3.6,18.9738) -- (3.76,18.9738);
\draw [c] (3.76,18.9738) -- (3.92,18.9738);
\draw [c] (4.08,18.5665) -- (4.08,18.6313);
\draw [c] (4.08,18.6313) -- (4.08,18.6889);
\draw [c] (3.92,18.6313) -- (4.08,18.6313);
\draw [c] (4.08,18.6313) -- (4.24,18.6313);
\draw [c] (4.4,18.2956) -- (4.4,18.3796);
\draw [c] (4.4,18.3796) -- (4.4,18.4519);
\draw [c] (4.24,18.3796) -- (4.4,18.3796);
\draw [c] (4.4,18.3796) -- (4.56,18.3796);
\draw [c] (4.72,18.0441) -- (4.72,18.151);
\draw [c] (4.72,18.151) -- (4.72,18.2396);
\draw [c] (4.56,18.151) -- (4.72,18.151);
\draw [c] (4.72,18.151) -- (4.88,18.151);
\draw [c] (5.04,17.8055) -- (5.04,17.8627);
\draw [c] (5.04,17.8627) -- (5.04,17.9143);
\draw [c] (4.88,17.8627) -- (5.04,17.8627);
\draw [c] (5.04,17.8627) -- (5.2,17.8627);
\draw [c] (5.36,17.6929) -- (5.36,17.6978);
\draw [c] (5.36,17.6978) -- (5.36,17.7026);
\draw [c] (5.2,17.6978) -- (5.36,17.6978);
\draw [c] (5.36,17.6978) -- (5.52,17.6978);
\draw [c] (5.68,17.5229) -- (5.68,17.5286);
\draw [c] (5.68,17.5286) -- (5.68,17.5343);
\draw [c] (5.52,17.5286) -- (5.68,17.5286);
\draw [c] (5.68,17.5286) -- (5.84,17.5286);
\draw [c] (6,17.3746) -- (6,17.3812);
\draw [c] (6,17.3812) -- (6,17.3877);
\draw [c] (5.84,17.3812) -- (6,17.3812);
\draw [c] (6,17.3812) -- (6.16,17.3812);
\draw [c] (6.32,17.2692) -- (6.32,17.2765);
\draw [c] (6.32,17.2765) -- (6.32,17.2837);
\draw [c] (6.16,17.2765) -- (6.32,17.2765);
\draw [c] (6.32,17.2765) -- (6.48,17.2765);
\draw [c] (6.64,17.1541) -- (6.64,17.1623);
\draw [c] (6.64,17.1623) -- (6.64,17.1703);
\draw [c] (6.48,17.1623) -- (6.64,17.1623);
\draw [c] (6.64,17.1623) -- (6.8,17.1623);
\draw [c] (6.96,17.0606) -- (6.96,17.0695);
\draw [c] (6.96,17.0695) -- (6.96,17.0783);
\draw [c] (6.8,17.0695) -- (6.96,17.0695);
\draw [c] (6.96,17.0695) -- (7.12,17.0695);
\draw [c] (7.28,16.9827) -- (7.28,16.9923);
\draw [c] (7.28,16.9923) -- (7.28,17.0018);
\draw [c] (7.12,16.9923) -- (7.28,16.9923);
\draw [c] (7.28,16.9923) -- (7.44,16.9923);
\draw [c] (7.6,16.9271) -- (7.6,16.9372);
\draw [c] (7.6,16.9372) -- (7.6,16.9472);
\draw [c] (7.44,16.9372) -- (7.6,16.9372);
\draw [c] (7.6,16.9372) -- (7.76,16.9372);
\draw [c] (7.92,16.8537) -- (7.92,16.8646);
\draw [c] (7.92,16.8646) -- (7.92,16.8752);
\draw [c] (7.76,16.8646) -- (7.92,16.8646);
\draw [c] (7.92,16.8646) -- (8.08,16.8646);
\draw [c] (8.24,16.8089) -- (8.24,16.8203);
\draw [c] (8.24,16.8203) -- (8.24,16.8314);
\draw [c] (8.08,16.8203) -- (8.24,16.8203);
\draw [c] (8.24,16.8203) -- (8.4,16.8203);
\draw [c] (8.56,16.7645) -- (8.56,16.7764);
\draw [c] (8.56,16.7764) -- (8.56,16.788);
\draw [c] (8.4,16.7764) -- (8.56,16.7764);
\draw [c] (8.56,16.7764) -- (8.72,16.7764);
\draw [c] (8.88,16.7259) -- (8.88,16.7382);
\draw [c] (8.88,16.7382) -- (8.88,16.7502);
\draw [c] (8.72,16.7382) -- (8.88,16.7382);
\draw [c] (8.88,16.7382) -- (9.04,16.7382);
\draw [c] (9.2,16.6876) -- (9.2,16.7004);
\draw [c] (9.2,16.7004) -- (9.2,16.7129);
\draw [c] (9.04,16.7004) -- (9.2,16.7004);
\draw [c] (9.2,16.7004) -- (9.36,16.7004);
\draw [c] (9.52,16.6177) -- (9.52,16.6314);
\draw [c] (9.52,16.6314) -- (9.52,16.6447);
\draw [c] (9.36,16.6314) -- (9.52,16.6314);
\draw [c] (9.52,16.6314) -- (9.68,16.6314);
\draw [c] (9.84,16.5889) -- (9.84,16.603);
\draw [c] (9.84,16.603) -- (9.84,16.6167);
\draw [c] (9.68,16.603) -- (9.84,16.603);
\draw [c] (9.84,16.603) -- (10,16.603);
\draw [c] (10.16,16.5341) -- (10.16,16.5489);
\draw [c] (10.16,16.5489) -- (10.16,16.5633);
\draw [c] (10,16.5489) -- (10.16,16.5489);
\draw [c] (10.16,16.5489) -- (10.32,16.5489);
\draw [c] (10.48,16.4921) -- (10.48,16.5075);
\draw [c] (10.48,16.5075) -- (10.48,16.5225);
\draw [c] (10.32,16.5075) -- (10.48,16.5075);
\draw [c] (10.48,16.5075) -- (10.64,16.5075);
\draw [c] (10.8,16.4375) -- (10.8,16.4538);
\draw [c] (10.8,16.4538) -- (10.8,16.4696);
\draw [c] (10.64,16.4538) -- (10.8,16.4538);
\draw [c] (10.8,16.4538) -- (10.96,16.4538);
\draw [c] (11.12,16.3965) -- (11.12,16.4134);
\draw [c] (11.12,16.4134) -- (11.12,16.4298);
\draw [c] (10.96,16.4134) -- (11.12,16.4134);
\draw [c] (11.12,16.4134) -- (11.28,16.4134);
\draw [c] (11.44,16.3364) -- (11.44,16.3543);
\draw [c] (11.44,16.3543) -- (11.44,16.3716);
\draw [c] (11.28,16.3543) -- (11.44,16.3543);
\draw [c] (11.44,16.3543) -- (11.6,16.3543);
\draw [c] (11.76,16.263) -- (11.76,16.2822);
\draw [c] (11.76,16.2822) -- (11.76,16.3007);
\draw [c] (11.6,16.2822) -- (11.76,16.2822);
\draw [c] (11.76,16.2822) -- (11.92,16.2822);
\draw [c] (12.08,16.2328) -- (12.08,16.2526);
\draw [c] (12.08,16.2526) -- (12.08,16.2717);
\draw [c] (11.92,16.2526) -- (12.08,16.2526);
\draw [c] (12.08,16.2526) -- (12.24,16.2526);
\draw [c] (12.4,16.1643) -- (12.4,16.1854);
\draw [c] (12.4,16.1854) -- (12.4,16.2057);
\draw [c] (12.24,16.1854) -- (12.4,16.1854);
\draw [c] (12.4,16.1854) -- (12.56,16.1854);
\draw [c] (12.72,16.1018) -- (12.72,16.1242);
\draw [c] (12.72,16.1242) -- (12.72,16.1457);
\draw [c] (12.56,16.1242) -- (12.72,16.1242);
\draw [c] (12.72,16.1242) -- (12.88,16.1242);
\draw [c] (13.04,15.9736) -- (13.04,15.999);
\draw [c] (13.04,15.999) -- (13.04,16.0232);
\draw [c] (12.88,15.999) -- (13.04,15.999);
\draw [c] (13.04,15.999) -- (13.2,15.999);
\draw [c] (13.36,15.9321) -- (13.36,15.9586);
\draw [c] (13.36,15.9586) -- (13.36,15.9837);
\draw [c] (13.2,15.9586) -- (13.36,15.9586);
\draw [c] (13.36,15.9586) -- (13.52,15.9586);
\draw [c] (13.68,15.7989) -- (13.68,15.8289);
\draw [c] (13.68,15.8289) -- (13.68,15.8573);
\draw [c] (13.52,15.8289) -- (13.68,15.8289);
\draw [c] (13.68,15.8289) -- (13.84,15.8289);
\draw [c] (14,15.7439) -- (14,15.7755);
\draw [c] (14,15.7755) -- (14,15.8054);
\draw [c] (13.84,15.7755) -- (14,15.7755);
\draw [c] (14,15.7755) -- (14.16,15.7755);
\draw [c] (14.32,15.6009) -- (14.32,15.6372);
\draw [c] (14.32,15.6372) -- (14.32,15.6712);
\draw [c] (14.16,15.6372) -- (14.32,15.6372);
\draw [c] (14.32,15.6372) -- (14.48,15.6372);
\draw [c] (14.64,15.4985) -- (14.64,15.5386);
\draw [c] (14.64,15.5386) -- (14.64,15.5758);
\draw [c] (14.48,15.5386) -- (14.64,15.5386);
\draw [c] (14.64,15.5386) -- (14.8,15.5386);
\draw [c] (14.96,15.4253) -- (14.96,15.4683);
\draw [c] (14.96,15.4683) -- (14.96,15.508);
\draw [c] (14.8,15.4683) -- (14.96,15.4683);
\draw [c] (14.96,15.4683) -- (15.12,15.4683);
\draw [c] (15.28,15.3282) -- (15.28,15.3754);
\draw [c] (15.28,15.3754) -- (15.28,15.4187);
\draw [c] (15.12,15.3754) -- (15.28,15.3754);
\draw [c] (15.28,15.3754) -- (15.44,15.3754);
\draw [c] (15.6,15.199) -- (15.6,15.2524);
\draw [c] (15.6,15.2524) -- (15.6,15.3009);
\draw [c] (15.44,15.2524) -- (15.6,15.2524);
\draw [c] (15.6,15.2524) -- (15.76,15.2524);
\draw [c] (15.92,15.2042) -- (15.92,15.2574);
\draw [c] (15.92,15.2574) -- (15.92,15.3056);
\draw [c] (15.76,15.2574) -- (15.92,15.2574);
\draw [c] (15.92,15.2574) -- (16.08,15.2574);
\draw [c] (16.24,14.8508) -- (16.24,14.9254);
\draw [c] (16.24,14.9254) -- (16.24,14.9906);
\draw [c] (16.08,14.9254) -- (16.24,14.9254);
\draw [c] (16.24,14.9254) -- (16.4,14.9254);
\draw [c] (16.56,14.7149) -- (16.56,14.7999);
\draw [c] (16.56,14.7999) -- (16.56,14.873);
\draw [c] (16.4,14.7999) -- (16.56,14.7999);
\draw [c] (16.56,14.7999) -- (16.72,14.7999);
\draw [c] (16.88,14.6163) -- (16.88,14.7098);
\draw [c] (16.88,14.7098) -- (16.88,14.789);
\draw [c] (16.72,14.7098) -- (16.88,14.7098);
\draw [c] (16.88,14.7098) -- (17.04,14.7098);
\draw [c] (17.2,14.4347) -- (17.2,14.5459);
\draw [c] (17.2,14.5459) -- (17.2,14.6374);
\draw [c] (17.04,14.5459) -- (17.2,14.5459);
\draw [c] (17.2,14.5459) -- (17.36,14.5459);
\draw [c] (17.52,14.124) -- (17.52,14.2736);
\draw [c] (17.52,14.2736) -- (17.52,14.3897);
\draw [c] (17.36,14.2736) -- (17.52,14.2736);
\draw [c] (17.52,14.2736) -- (17.68,14.2736);
\draw [c] (17.84,13.8314) -- (17.84,14.0291);
\draw [c] (17.84,14.0291) -- (17.84,14.1721);
\draw [c] (17.68,14.0291) -- (17.84,14.0291);
\draw [c] (17.84,14.0291) -- (18,14.0291);
\colorlet{c}{natgreen};
\draw [c] (2.16,21.9864) -- (2.16,21.9888);
\draw [c] (2.16,21.9888) -- (2.16,21.9913);
\draw [c] (2,21.9888) -- (2.16,21.9888);
\draw [c] (2.16,21.9888) -- (2.32,21.9888);
\draw [c] (2.48,20.9164) -- (2.48,20.9232);
\draw [c] (2.48,20.9232) -- (2.48,20.9299);
\draw [c] (2.32,20.9232) -- (2.48,20.9232);
\draw [c] (2.48,20.9232) -- (2.64,20.9232);
\draw [c] (2.8,20.1539) -- (2.8,20.168);
\draw [c] (2.8,20.168) -- (2.8,20.1817);
\draw [c] (2.64,20.168) -- (2.8,20.168);
\draw [c] (2.8,20.168) -- (2.96,20.168);
\draw [c] (3.12,19.6053) -- (3.12,19.6292);
\draw [c] (3.12,19.6292) -- (3.12,19.6521);
\draw [c] (2.96,19.6292) -- (3.12,19.6292);
\draw [c] (3.12,19.6292) -- (3.28,19.6292);
\draw [c] (3.44,19.181) -- (3.44,19.2169);
\draw [c] (3.44,19.2169) -- (3.44,19.2505);
\draw [c] (3.28,19.2169) -- (3.44,19.2169);
\draw [c] (3.44,19.2169) -- (3.6,19.2169);
\draw [c] (3.76,18.7975) -- (3.76,18.8494);
\draw [c] (3.76,18.8494) -- (3.76,18.8965);
\draw [c] (3.6,18.8494) -- (3.76,18.8494);
\draw [c] (3.76,18.8494) -- (3.92,18.8494);
\draw [c] (4.08,18.4188) -- (4.08,18.4934);
\draw [c] (4.08,18.4934) -- (4.08,18.5587);
\draw [c] (3.92,18.4934) -- (4.08,18.4934);
\draw [c] (4.08,18.4934) -- (4.24,18.4934);
\draw [c] (4.4,18.136) -- (4.4,18.2338);
\draw [c] (4.4,18.2338) -- (4.4,18.3162);
\draw [c] (4.24,18.2338) -- (4.4,18.2338);
\draw [c] (4.4,18.2338) -- (4.56,18.2338);
\draw [c] (4.72,17.9807) -- (4.72,18.0943);
\draw [c] (4.72,18.0943) -- (4.72,18.1874);
\draw [c] (4.56,18.0943) -- (4.72,18.0943);
\draw [c] (4.72,18.0943) -- (4.88,18.0943);
\draw [c] (5.04,17.7086) -- (5.04,17.7768);
\draw [c] (5.04,17.7768) -- (5.04,17.837);
\draw [c] (4.88,17.7768) -- (5.04,17.7768);
\draw [c] (5.04,17.7768) -- (5.2,17.7768);
\draw [c] (5.36,17.585) -- (5.36,17.5903);
\draw [c] (5.36,17.5903) -- (5.36,17.5957);
\draw [c] (5.2,17.5903) -- (5.36,17.5903);
\draw [c] (5.36,17.5903) -- (5.52,17.5903);
\draw [c] (5.68,17.4122) -- (5.68,17.4185);
\draw [c] (5.68,17.4185) -- (5.68,17.4248);
\draw [c] (5.52,17.4185) -- (5.68,17.4185);
\draw [c] (5.68,17.4185) -- (5.84,17.4185);
\draw [c] (6,17.2623) -- (6,17.2697);
\draw [c] (6,17.2697) -- (6,17.277);
\draw [c] (5.84,17.2697) -- (6,17.2697);
\draw [c] (6,17.2697) -- (6.16,17.2697);
\draw [c] (6.32,17.1484) -- (6.32,17.1566);
\draw [c] (6.32,17.1566) -- (6.32,17.1647);
\draw [c] (6.16,17.1566) -- (6.32,17.1566);
\draw [c] (6.32,17.1566) -- (6.48,17.1566);
\draw [c] (6.64,17.0378) -- (6.64,17.0469);
\draw [c] (6.64,17.0469) -- (6.64,17.0559);
\draw [c] (6.48,17.0469) -- (6.64,17.0469);
\draw [c] (6.64,17.0469) -- (6.8,17.0469);
\draw [c] (6.96,16.9406) -- (6.96,16.9507);
\draw [c] (6.96,16.9507) -- (6.96,16.9605);
\draw [c] (6.8,16.9507) -- (6.96,16.9507);
\draw [c] (6.96,16.9507) -- (7.12,16.9507);
\draw [c] (7.28,16.8659) -- (7.28,16.8767);
\draw [c] (7.28,16.8767) -- (7.28,16.8873);
\draw [c] (7.12,16.8767) -- (7.28,16.8767);
\draw [c] (7.28,16.8767) -- (7.44,16.8767);
\draw [c] (7.6,16.8118) -- (7.6,16.8232);
\draw [c] (7.6,16.8232) -- (7.6,16.8343);
\draw [c] (7.44,16.8232) -- (7.6,16.8232);
\draw [c] (7.6,16.8232) -- (7.76,16.8232);
\draw [c] (7.92,16.7322) -- (7.92,16.7444);
\draw [c] (7.92,16.7444) -- (7.92,16.7564);
\draw [c] (7.76,16.7444) -- (7.92,16.7444);
\draw [c] (7.92,16.7444) -- (8.08,16.7444);
\draw [c] (8.24,16.7023) -- (8.24,16.7149);
\draw [c] (8.24,16.7149) -- (8.24,16.7272);
\draw [c] (8.08,16.7149) -- (8.24,16.7149);
\draw [c] (8.24,16.7149) -- (8.4,16.7149);
\draw [c] (8.56,16.6426) -- (8.56,16.656);
\draw [c] (8.56,16.656) -- (8.56,16.669);
\draw [c] (8.4,16.656) -- (8.56,16.656);
\draw [c] (8.56,16.656) -- (8.72,16.656);
\draw [c] (8.88,16.6037) -- (8.88,16.6175);
\draw [c] (8.88,16.6175) -- (8.88,16.631);
\draw [c] (8.72,16.6175) -- (8.88,16.6175);
\draw [c] (8.88,16.6175) -- (9.04,16.6175);
\draw [c] (9.2,16.5706) -- (9.2,16.5849);
\draw [c] (9.2,16.5849) -- (9.2,16.5988);
\draw [c] (9.04,16.5849) -- (9.2,16.5849);
\draw [c] (9.2,16.5849) -- (9.36,16.5849);
\draw [c] (9.52,16.4921) -- (9.52,16.5075);
\draw [c] (9.52,16.5075) -- (9.52,16.5225);
\draw [c] (9.36,16.5075) -- (9.52,16.5075);
\draw [c] (9.52,16.5075) -- (9.68,16.5075);
\draw [c] (9.84,16.4677) -- (9.84,16.4834);
\draw [c] (9.84,16.4834) -- (9.84,16.4988);
\draw [c] (9.68,16.4834) -- (9.84,16.4834);
\draw [c] (9.84,16.4834) -- (10,16.4834);
\draw [c] (10.16,16.4156) -- (10.16,16.4322);
\draw [c] (10.16,16.4322) -- (10.16,16.4483);
\draw [c] (10,16.4322) -- (10.16,16.4322);
\draw [c] (10.16,16.4322) -- (10.32,16.4322);
\draw [c] (10.48,16.3822) -- (10.48,16.3994);
\draw [c] (10.48,16.3994) -- (10.48,16.416);
\draw [c] (10.32,16.3994) -- (10.48,16.3994);
\draw [c] (10.48,16.3994) -- (10.64,16.3994);
\draw [c] (10.8,16.3229) -- (10.8,16.341);
\draw [c] (10.8,16.341) -- (10.8,16.3586);
\draw [c] (10.64,16.341) -- (10.8,16.341);
\draw [c] (10.8,16.341) -- (10.96,16.341);
\draw [c] (11.12,16.2713) -- (11.12,16.2903);
\draw [c] (11.12,16.2903) -- (11.12,16.3087);
\draw [c] (10.96,16.2903) -- (11.12,16.2903);
\draw [c] (11.12,16.2903) -- (11.28,16.2903);
\draw [c] (11.44,16.2148) -- (11.44,16.2349);
\draw [c] (11.44,16.2349) -- (11.44,16.2543);
\draw [c] (11.28,16.2349) -- (11.44,16.2349);
\draw [c] (11.44,16.2349) -- (11.6,16.2349);
\draw [c] (11.76,16.1489) -- (11.76,16.1704);
\draw [c] (11.76,16.1704) -- (11.76,16.191);
\draw [c] (11.6,16.1704) -- (11.76,16.1704);
\draw [c] (11.76,16.1704) -- (11.92,16.1704);
\draw [c] (12.08,16.1121) -- (12.08,16.1343);
\draw [c] (12.08,16.1343) -- (12.08,16.1556);
\draw [c] (11.92,16.1343) -- (12.08,16.1343);
\draw [c] (12.08,16.1343) -- (12.24,16.1343);
\draw [c] (12.4,16.0405) -- (12.4,16.0643);
\draw [c] (12.4,16.0643) -- (12.4,16.0871);
\draw [c] (12.24,16.0643) -- (12.4,16.0643);
\draw [c] (12.4,16.0643) -- (12.56,16.0643);
\draw [c] (12.72,16.0042) -- (12.72,16.0288);
\draw [c] (12.72,16.0288) -- (12.72,16.0524);
\draw [c] (12.56,16.0288) -- (12.72,16.0288);
\draw [c] (12.72,16.0288) -- (12.88,16.0288);
\draw [c] (13.04,15.8697) -- (13.04,15.8978);
\draw [c] (13.04,15.8978) -- (13.04,15.9244);
\draw [c] (12.88,15.8978) -- (13.04,15.8978);
\draw [c] (13.04,15.8978) -- (13.2,15.8978);
\draw [c] (13.36,15.8378) -- (13.36,15.8668);
\draw [c] (13.36,15.8668) -- (13.36,15.8942);
\draw [c] (13.2,15.8668) -- (13.36,15.8668);
\draw [c] (13.36,15.8668) -- (13.52,15.8668);
\draw [c] (13.68,15.6825) -- (13.68,15.7161);
\draw [c] (13.68,15.7161) -- (13.68,15.7476);
\draw [c] (13.52,15.7161) -- (13.68,15.7161);
\draw [c] (13.68,15.7161) -- (13.84,15.7161);
\draw [c] (14,15.6295) -- (14,15.6649);
\draw [c] (14,15.6649) -- (14,15.6979);
\draw [c] (13.84,15.6649) -- (14,15.6649);
\draw [c] (14,15.6649) -- (14.16,15.6649);
\draw [c] (14.32,15.4985) -- (14.32,15.5386);
\draw [c] (14.32,15.5386) -- (14.32,15.5758);
\draw [c] (14.16,15.5386) -- (14.32,15.5386);
\draw [c] (14.32,15.5386) -- (14.48,15.5386);
\draw [c] (14.64,15.3444) -- (14.64,15.3908);
\draw [c] (14.64,15.3908) -- (14.64,15.4335);
\draw [c] (14.48,15.3908) -- (14.64,15.3908);
\draw [c] (14.64,15.3908) -- (14.8,15.3908);
\draw [c] (14.96,15.3073) -- (14.96,15.3555);
\draw [c] (14.96,15.3555) -- (14.96,15.3995);
\draw [c] (14.8,15.3555) -- (14.96,15.3555);
\draw [c] (14.96,15.3555) -- (15.12,15.3555);
\draw [c] (15.28,15.2195) -- (15.28,15.2719);
\draw [c] (15.28,15.2719) -- (15.28,15.3195);
\draw [c] (15.12,15.2719) -- (15.28,15.2719);
\draw [c] (15.28,15.2719) -- (15.44,15.2719);
\draw [c] (15.6,15.0828) -- (15.6,15.1425);
\draw [c] (15.6,15.1425) -- (15.6,15.1961);
\draw [c] (15.44,15.1425) -- (15.6,15.1425);
\draw [c] (15.6,15.1425) -- (15.76,15.1425);
\draw [c] (15.92,15.0493) -- (15.92,15.1109);
\draw [c] (15.92,15.1109) -- (15.92,15.1661);
\draw [c] (15.76,15.1109) -- (15.92,15.1109);
\draw [c] (15.92,15.1109) -- (16.08,15.1109);
\draw [c] (16.24,14.7641) -- (16.24,14.8452);
\draw [c] (16.24,14.8452) -- (16.24,14.9153);
\draw [c] (16.08,14.8452) -- (16.24,14.8452);
\draw [c] (16.24,14.8452) -- (16.4,14.8452);
\draw [c] (16.56,14.6608) -- (16.56,14.7503);
\draw [c] (16.56,14.7503) -- (16.56,14.8267);
\draw [c] (16.4,14.7503) -- (16.56,14.7503);
\draw [c] (16.56,14.7503) -- (16.72,14.7503);
\draw [c] (16.88,14.4762) -- (16.88,14.583);
\draw [c] (16.88,14.583) -- (16.88,14.6716);
\draw [c] (16.72,14.583) -- (16.88,14.583);
\draw [c] (16.88,14.583) -- (17.04,14.583);
\draw [c] (17.2,14.3408) -- (17.2,14.4624);
\draw [c] (17.2,14.4624) -- (17.2,14.561);
\draw [c] (17.04,14.4624) -- (17.2,14.4624);
\draw [c] (17.2,14.4624) -- (17.36,14.4624);
\draw [c] (17.52,13.8314) -- (17.52,14.0291);
\draw [c] (17.52,14.0291) -- (17.52,14.1721);
\draw [c] (17.36,14.0291) -- (17.52,14.0291);
\draw [c] (17.52,14.0291) -- (17.68,14.0291);
\draw [c] (17.84,13.7634) -- (17.84,13.9743);
\draw [c] (17.84,13.9743) -- (17.84,14.124);
\draw [c] (17.68,13.9743) -- (17.84,13.9743);
\draw [c] (17.84,13.9743) -- (18,13.9743);
\colorlet{c}{natcomp!50};
\draw [c] (2.16,22.0839) -- (2.16,22.0861);
\draw [c] (2.16,22.0861) -- (2.16,22.0883);
\draw [c] (2,22.0861) -- (2.16,22.0861);
\draw [c] (2.16,22.0861) -- (2.32,22.0861);
\draw [c] (2.48,21.011) -- (2.48,21.0172);
\draw [c] (2.48,21.0172) -- (2.48,21.0233);
\draw [c] (2.32,21.0172) -- (2.48,21.0172);
\draw [c] (2.48,21.0172) -- (2.64,21.0172);
\draw [c] (2.8,20.276) -- (2.8,20.2885);
\draw [c] (2.8,20.2885) -- (2.8,20.3008);
\draw [c] (2.64,20.2885) -- (2.8,20.2885);
\draw [c] (2.8,20.2885) -- (2.96,20.2885);
\draw [c] (3.12,19.7289) -- (3.12,19.7501);
\draw [c] (3.12,19.7501) -- (3.12,19.7705);
\draw [c] (2.96,19.7501) -- (3.12,19.7501);
\draw [c] (3.12,19.7501) -- (3.28,19.7501);
\draw [c] (3.44,19.3257) -- (3.44,19.3569);
\draw [c] (3.44,19.3569) -- (3.44,19.3864);
\draw [c] (3.28,19.3569) -- (3.44,19.3569);
\draw [c] (3.44,19.3569) -- (3.6,19.3569);
\draw [c] (3.76,18.9625) -- (3.76,19.0068);
\draw [c] (3.76,19.0068) -- (3.76,19.0477);
\draw [c] (3.6,19.0068) -- (3.76,19.0068);
\draw [c] (3.76,19.0068) -- (3.92,19.0068);
\draw [c] (4.08,18.575) -- (4.08,18.6392);
\draw [c] (4.08,18.6392) -- (4.08,18.6965);
\draw [c] (3.92,18.6392) -- (4.08,18.6392);
\draw [c] (4.08,18.6392) -- (4.24,18.6392);
\draw [c] (4.4,18.378) -- (4.4,18.4557);
\draw [c] (4.4,18.4557) -- (4.4,18.5232);
\draw [c] (4.24,18.4557) -- (4.4,18.4557);
\draw [c] (4.4,18.4557) -- (4.56,18.4557);
\draw [c] (4.72,18.2004) -- (4.72,18.2925);
\draw [c] (4.72,18.2925) -- (4.72,18.3707);
\draw [c] (4.56,18.2925) -- (4.72,18.2925);
\draw [c] (4.72,18.2925) -- (4.88,18.2925);
\draw [c] (5.04,18.1788) -- (5.04,18.2303);
\draw [c] (5.04,18.2303) -- (5.04,18.2771);
\draw [c] (4.88,18.2303) -- (5.04,18.2303);
\draw [c] (5.04,18.2303) -- (5.2,18.2303);
\draw [c] (5.36,18.056) -- (5.36,18.0627);
\draw [c] (5.36,18.0627) -- (5.36,18.0692);
\draw [c] (5.2,18.0627) -- (5.36,18.0627);
\draw [c] (5.36,18.0627) -- (5.52,18.0627);
\draw [c] (5.68,17.9779) -- (5.68,17.9851);
\draw [c] (5.68,17.9851) -- (5.68,17.9922);
\draw [c] (5.52,17.9851) -- (5.68,17.9851);
\draw [c] (5.68,17.9851) -- (5.84,17.9851);
\draw [c] (6,17.91) -- (6,17.9176);
\draw [c] (6,17.9176) -- (6,17.9252);
\draw [c] (5.84,17.9176) -- (6,17.9176);
\draw [c] (6,17.9176) -- (6.16,17.9176);
\draw [c] (6.32,17.8879) -- (6.32,17.8958);
\draw [c] (6.32,17.8958) -- (6.32,17.9035);
\draw [c] (6.16,17.8958) -- (6.32,17.8958);
\draw [c] (6.32,17.8958) -- (6.48,17.8958);
\draw [c] (6.64,17.8645) -- (6.64,17.8725);
\draw [c] (6.64,17.8725) -- (6.64,17.8804);
\draw [c] (6.48,17.8725) -- (6.64,17.8725);
\draw [c] (6.64,17.8725) -- (6.8,17.8725);
\draw [c] (6.96,17.8303) -- (6.96,17.8386);
\draw [c] (6.96,17.8386) -- (6.96,17.8467);
\draw [c] (6.8,17.8386) -- (6.96,17.8386);
\draw [c] (6.96,17.8386) -- (7.12,17.8386);
\draw [c] (7.28,17.8212) -- (7.28,17.8295);
\draw [c] (7.28,17.8295) -- (7.28,17.8378);
\draw [c] (7.12,17.8295) -- (7.28,17.8295);
\draw [c] (7.28,17.8295) -- (7.44,17.8295);
\draw [c] (7.6,17.8047) -- (7.6,17.8132);
\draw [c] (7.6,17.8132) -- (7.6,17.8215);
\draw [c] (7.44,17.8132) -- (7.6,17.8132);
\draw [c] (7.6,17.8132) -- (7.76,17.8132);
\draw [c] (7.92,17.7853) -- (7.92,17.794);
\draw [c] (7.92,17.794) -- (7.92,17.8025);
\draw [c] (7.76,17.794) -- (7.92,17.794);
\draw [c] (7.92,17.794) -- (8.08,17.794);
\draw [c] (8.24,17.78) -- (8.24,17.7887);
\draw [c] (8.24,17.7887) -- (8.24,17.7973);
\draw [c] (8.08,17.7887) -- (8.24,17.7887);
\draw [c] (8.24,17.7887) -- (8.4,17.7887);
\draw [c] (8.56,17.7561) -- (8.56,17.765);
\draw [c] (8.56,17.765) -- (8.56,17.7737);
\draw [c] (8.4,17.765) -- (8.56,17.765);
\draw [c] (8.56,17.765) -- (8.72,17.765);
\draw [c] (8.88,17.7523) -- (8.88,17.7612);
\draw [c] (8.88,17.7612) -- (8.88,17.77);
\draw [c] (8.72,17.7612) -- (8.88,17.7612);
\draw [c] (8.88,17.7612) -- (9.04,17.7612);
\draw [c] (9.2,17.7282) -- (9.2,17.7373);
\draw [c] (9.2,17.7373) -- (9.2,17.7463);
\draw [c] (9.04,17.7373) -- (9.2,17.7373);
\draw [c] (9.2,17.7373) -- (9.36,17.7373);
\draw [c] (9.52,17.7082) -- (9.52,17.7175);
\draw [c] (9.52,17.7175) -- (9.52,17.7267);
\draw [c] (9.36,17.7175) -- (9.52,17.7175);
\draw [c] (9.52,17.7175) -- (9.68,17.7175);
\draw [c] (9.84,17.684) -- (9.84,17.6935);
\draw [c] (9.84,17.6935) -- (9.84,17.7029);
\draw [c] (9.68,17.6935) -- (9.84,17.6935);
\draw [c] (9.84,17.6935) -- (10,17.6935);
\draw [c] (10.16,17.6742) -- (10.16,17.6838);
\draw [c] (10.16,17.6838) -- (10.16,17.6933);
\draw [c] (10,17.6838) -- (10.16,17.6838);
\draw [c] (10.16,17.6838) -- (10.32,17.6838);
\draw [c] (10.48,17.6116) -- (10.48,17.6218);
\draw [c] (10.48,17.6218) -- (10.48,17.6319);
\draw [c] (10.32,17.6218) -- (10.48,17.6218);
\draw [c] (10.48,17.6218) -- (10.64,17.6218);
\draw [c] (10.8,17.5894) -- (10.8,17.5999);
\draw [c] (10.8,17.5999) -- (10.8,17.6101);
\draw [c] (10.64,17.5999) -- (10.8,17.5999);
\draw [c] (10.8,17.5999) -- (10.96,17.5999);
\draw [c] (11.12,17.5354) -- (11.12,17.5464);
\draw [c] (11.12,17.5464) -- (11.12,17.5572);
\draw [c] (10.96,17.5464) -- (11.12,17.5464);
\draw [c] (11.12,17.5464) -- (11.28,17.5464);
\draw [c] (11.44,17.5116) -- (11.44,17.5229);
\draw [c] (11.44,17.5229) -- (11.44,17.5339);
\draw [c] (11.28,17.5229) -- (11.44,17.5229);
\draw [c] (11.44,17.5229) -- (11.6,17.5229);
\draw [c] (11.76,17.4347) -- (11.76,17.4468);
\draw [c] (11.76,17.4468) -- (11.76,17.4586);
\draw [c] (11.6,17.4468) -- (11.76,17.4468);
\draw [c] (11.76,17.4468) -- (11.92,17.4468);
\draw [c] (12.08,17.3889) -- (12.08,17.4015);
\draw [c] (12.08,17.4015) -- (12.08,17.4139);
\draw [c] (11.92,17.4015) -- (12.08,17.4015);
\draw [c] (12.08,17.4015) -- (12.24,17.4015);
\draw [c] (12.4,17.3285) -- (12.4,17.3419);
\draw [c] (12.4,17.3419) -- (12.4,17.355);
\draw [c] (12.24,17.3419) -- (12.4,17.3419);
\draw [c] (12.4,17.3419) -- (12.56,17.3419);
\draw [c] (12.72,17.2607) -- (12.72,17.275);
\draw [c] (12.72,17.275) -- (12.72,17.289);
\draw [c] (12.56,17.275) -- (12.72,17.275);
\draw [c] (12.72,17.275) -- (12.88,17.275);
\draw [c] (13.04,17.1814) -- (13.04,17.1969);
\draw [c] (13.04,17.1969) -- (13.04,17.2119);
\draw [c] (12.88,17.1969) -- (13.04,17.1969);
\draw [c] (13.04,17.1969) -- (13.2,17.1969);
\draw [c] (13.36,17.1123) -- (13.36,17.1288);
\draw [c] (13.36,17.1288) -- (13.36,17.1448);
\draw [c] (13.2,17.1288) -- (13.36,17.1288);
\draw [c] (13.36,17.1288) -- (13.52,17.1288);
\draw [c] (13.68,17.0504) -- (13.68,17.0679);
\draw [c] (13.68,17.0679) -- (13.68,17.0849);
\draw [c] (13.52,17.0679) -- (13.68,17.0679);
\draw [c] (13.68,17.0679) -- (13.84,17.0679);
\draw [c] (14,16.9278) -- (14,16.9475);
\draw [c] (14,16.9475) -- (14,16.9665);
\draw [c] (13.84,16.9475) -- (14,16.9475);
\draw [c] (14,16.9475) -- (14.16,16.9475);
\draw [c] (14.32,16.8461) -- (14.32,16.8674);
\draw [c] (14.32,16.8674) -- (14.32,16.8879);
\draw [c] (14.16,16.8674) -- (14.32,16.8674);
\draw [c] (14.32,16.8674) -- (14.48,16.8674);
\draw [c] (14.64,16.6858) -- (14.64,16.7107);
\draw [c] (14.64,16.7107) -- (14.64,16.7345);
\draw [c] (14.48,16.7107) -- (14.64,16.7107);
\draw [c] (14.64,16.7107) -- (14.8,16.7107);
\draw [c] (14.96,16.589) -- (14.96,16.6163);
\draw [c] (14.96,16.6163) -- (14.96,16.6423);
\draw [c] (14.8,16.6163) -- (14.96,16.6163);
\draw [c] (14.96,16.6163) -- (15.12,16.6163);
\draw [c] (15.28,16.3911) -- (15.28,16.4241);
\draw [c] (15.28,16.4241) -- (15.28,16.4552);
\draw [c] (15.12,16.4241) -- (15.28,16.4241);
\draw [c] (15.28,16.4241) -- (15.44,16.4241);
\draw [c] (15.6,16.3145) -- (15.6,16.3501);
\draw [c] (15.6,16.3501) -- (15.6,16.3834);
\draw [c] (15.44,16.3501) -- (15.6,16.3501);
\draw [c] (15.6,16.3501) -- (15.76,16.3501);
\draw [c] (15.92,16.1599) -- (15.92,16.2012);
\draw [c] (15.92,16.2012) -- (15.92,16.2394);
\draw [c] (15.76,16.2012) -- (15.92,16.2012);
\draw [c] (15.92,16.2012) -- (16.08,16.2012);
\draw [c] (16.24,16.1) -- (16.24,16.1437);
\draw [c] (16.24,16.1437) -- (16.24,16.184);
\draw [c] (16.08,16.1437) -- (16.24,16.1437);
\draw [c] (16.24,16.1437) -- (16.4,16.1437);
\draw [c] (16.56,15.9731) -- (16.56,16.0225);
\draw [c] (16.56,16.0225) -- (16.56,16.0676);
\draw [c] (16.4,16.0225) -- (16.56,16.0225);
\draw [c] (16.56,16.0225) -- (16.72,16.0225);
\draw [c] (16.88,15.7482) -- (16.88,15.8094);
\draw [c] (16.88,15.8094) -- (16.88,15.8642);
\draw [c] (16.72,15.8094) -- (16.88,15.8094);
\draw [c] (16.88,15.8094) -- (17.04,15.8094);
\draw [c] (17.2,15.7056) -- (17.2,15.7694);
\draw [c] (17.2,15.7694) -- (17.2,15.8262);
\draw [c] (17.04,15.7694) -- (17.2,15.7694);
\draw [c] (17.2,15.7694) -- (17.36,15.7694);
\draw [c] (17.52,15.6174) -- (17.52,15.6869);
\draw [c] (17.52,15.6869) -- (17.52,15.7482);
\draw [c] (17.36,15.6869) -- (17.52,15.6869);
\draw [c] (17.52,15.6869) -- (17.68,15.6869);
\draw [c] (17.84,15.4069) -- (17.84,15.4919);
\draw [c] (17.84,15.4919) -- (17.84,15.565);
\draw [c] (17.68,15.4919) -- (17.84,15.4919);
\draw [c] (17.84,15.4919) -- (18,15.4919);
\colorlet{c}{natcomp};
\draw [c] (2.16,21.987) -- (2.16,21.9894);
\draw [c] (2.16,21.9894) -- (2.16,21.9918);
\draw [c] (2,21.9894) -- (2.16,21.9894);
\draw [c] (2.16,21.9894) -- (2.32,21.9894);
\draw [c] (2.48,20.9235) -- (2.48,20.9302);
\draw [c] (2.48,20.9302) -- (2.48,20.9369);
\draw [c] (2.32,20.9302) -- (2.48,20.9302);
\draw [c] (2.48,20.9302) -- (2.64,20.9302);
\draw [c] (2.8,20.1663) -- (2.8,20.1803);
\draw [c] (2.8,20.1803) -- (2.8,20.1939);
\draw [c] (2.64,20.1803) -- (2.8,20.1803);
\draw [c] (2.8,20.1803) -- (2.96,20.1803);
\draw [c] (3.12,19.62) -- (3.12,19.6436);
\draw [c] (3.12,19.6436) -- (3.12,19.6661);
\draw [c] (2.96,19.6436) -- (3.12,19.6436);
\draw [c] (3.12,19.6436) -- (3.28,19.6436);
\draw [c] (3.44,19.1891) -- (3.44,19.2248);
\draw [c] (3.44,19.2248) -- (3.44,19.2581);
\draw [c] (3.28,19.2248) -- (3.44,19.2248);
\draw [c] (3.44,19.2248) -- (3.6,19.2248);
\draw [c] (3.76,18.8502) -- (3.76,18.8995);
\draw [c] (3.76,18.8995) -- (3.76,18.9446);
\draw [c] (3.6,18.8995) -- (3.76,18.8995);
\draw [c] (3.76,18.8995) -- (3.92,18.8995);
\draw [c] (4.08,18.4298) -- (4.08,18.5037);
\draw [c] (4.08,18.5037) -- (4.08,18.5684);
\draw [c] (3.92,18.5037) -- (4.08,18.5037);
\draw [c] (4.08,18.5037) -- (4.24,18.5037);
\draw [c] (4.4,18.23) -- (4.4,18.3195);
\draw [c] (4.4,18.3195) -- (4.4,18.3958);
\draw [c] (4.24,18.3195) -- (4.4,18.3195);
\draw [c] (4.4,18.3195) -- (4.56,18.3195);
\draw [c] (4.72,18.0251) -- (4.72,18.134);
\draw [c] (4.72,18.134) -- (4.72,18.224);
\draw [c] (4.56,18.134) -- (4.72,18.134);
\draw [c] (4.72,18.134) -- (4.88,18.134);
\draw [c] (5.04,18.0918) -- (5.04,18.1521);
\draw [c] (5.04,18.1521) -- (5.04,18.2061);
\draw [c] (4.88,18.1521) -- (5.04,18.1521);
\draw [c] (5.04,18.1521) -- (5.2,18.1521);
\draw [c] (5.36,17.9391) -- (5.36,17.9465);
\draw [c] (5.36,17.9465) -- (5.36,17.9539);
\draw [c] (5.2,17.9465) -- (5.36,17.9465);
\draw [c] (5.36,17.9465) -- (5.52,17.9465);
\draw [c] (5.68,17.8552) -- (5.68,17.8633);
\draw [c] (5.68,17.8633) -- (5.68,17.8713);
\draw [c] (5.52,17.8633) -- (5.68,17.8633);
\draw [c] (5.68,17.8633) -- (5.84,17.8633);
\draw [c] (6,17.7897) -- (6,17.7983);
\draw [c] (6,17.7983) -- (6,17.8068);
\draw [c] (5.84,17.7983) -- (6,17.7983);
\draw [c] (6,17.7983) -- (6.16,17.7983);
\draw [c] (6.32,17.7698) -- (6.32,17.7786);
\draw [c] (6.32,17.7786) -- (6.32,17.7872);
\draw [c] (6.16,17.7786) -- (6.32,17.7786);
\draw [c] (6.32,17.7786) -- (6.48,17.7786);
\draw [c] (6.64,17.7423) -- (6.64,17.7513);
\draw [c] (6.64,17.7513) -- (6.64,17.7602);
\draw [c] (6.48,17.7513) -- (6.64,17.7513);
\draw [c] (6.64,17.7513) -- (6.8,17.7513);
\draw [c] (6.96,17.7084) -- (6.96,17.7177);
\draw [c] (6.96,17.7177) -- (6.96,17.7269);
\draw [c] (6.8,17.7177) -- (6.96,17.7177);
\draw [c] (6.96,17.7177) -- (7.12,17.7177);
\draw [c] (7.28,17.6994) -- (7.28,17.7088);
\draw [c] (7.28,17.7088) -- (7.28,17.718);
\draw [c] (7.12,17.7088) -- (7.28,17.7088);
\draw [c] (7.28,17.7088) -- (7.44,17.7088);
\draw [c] (7.6,17.6817) -- (7.6,17.6913);
\draw [c] (7.6,17.6913) -- (7.6,17.7007);
\draw [c] (7.44,17.6913) -- (7.6,17.6913);
\draw [c] (7.6,17.6913) -- (7.76,17.6913);
\draw [c] (7.92,17.6626) -- (7.92,17.6723);
\draw [c] (7.92,17.6723) -- (7.92,17.6819);
\draw [c] (7.76,17.6723) -- (7.92,17.6723);
\draw [c] (7.92,17.6723) -- (8.08,17.6723);
\draw [c] (8.24,17.6584) -- (8.24,17.6682);
\draw [c] (8.24,17.6682) -- (8.24,17.6778);
\draw [c] (8.08,17.6682) -- (8.24,17.6682);
\draw [c] (8.24,17.6682) -- (8.4,17.6682);
\draw [c] (8.56,17.6306) -- (8.56,17.6406);
\draw [c] (8.56,17.6406) -- (8.56,17.6505);
\draw [c] (8.4,17.6406) -- (8.56,17.6406);
\draw [c] (8.56,17.6406) -- (8.72,17.6406);
\draw [c] (8.88,17.6357) -- (8.88,17.6457);
\draw [c] (8.88,17.6457) -- (8.88,17.6555);
\draw [c] (8.72,17.6457) -- (8.88,17.6457);
\draw [c] (8.88,17.6457) -- (9.04,17.6457);
\draw [c] (9.2,17.6126) -- (9.2,17.6228);
\draw [c] (9.2,17.6228) -- (9.2,17.6328);
\draw [c] (9.04,17.6228) -- (9.2,17.6228);
\draw [c] (9.2,17.6228) -- (9.36,17.6228);
\draw [c] (9.52,17.5853) -- (9.52,17.5958);
\draw [c] (9.52,17.5958) -- (9.52,17.606);
\draw [c] (9.36,17.5958) -- (9.52,17.5958);
\draw [c] (9.52,17.5958) -- (9.68,17.5958);
\draw [c] (9.84,17.5635) -- (9.84,17.5742);
\draw [c] (9.84,17.5742) -- (9.84,17.5847);
\draw [c] (9.68,17.5742) -- (9.84,17.5742);
\draw [c] (9.84,17.5742) -- (10,17.5742);
\draw [c] (10.16,17.5595) -- (10.16,17.5703);
\draw [c] (10.16,17.5703) -- (10.16,17.5808);
\draw [c] (10,17.5703) -- (10.16,17.5703);
\draw [c] (10.16,17.5703) -- (10.32,17.5703);
\draw [c] (10.48,17.4826) -- (10.48,17.4942);
\draw [c] (10.48,17.4942) -- (10.48,17.5055);
\draw [c] (10.32,17.4942) -- (10.48,17.4942);
\draw [c] (10.48,17.4942) -- (10.64,17.4942);
\draw [c] (10.8,17.4679) -- (10.8,17.4797);
\draw [c] (10.8,17.4797) -- (10.8,17.4911);
\draw [c] (10.64,17.4797) -- (10.8,17.4797);
\draw [c] (10.8,17.4797) -- (10.96,17.4797);
\draw [c] (11.12,17.4202) -- (11.12,17.4325);
\draw [c] (11.12,17.4325) -- (11.12,17.4445);
\draw [c] (10.96,17.4325) -- (11.12,17.4325);
\draw [c] (11.12,17.4325) -- (11.28,17.4325);
\draw [c] (11.44,17.4101) -- (11.44,17.4225);
\draw [c] (11.44,17.4225) -- (11.44,17.4346);
\draw [c] (11.28,17.4225) -- (11.44,17.4225);
\draw [c] (11.44,17.4225) -- (11.6,17.4225);
\draw [c] (11.76,17.3196) -- (11.76,17.3331);
\draw [c] (11.76,17.3331) -- (11.76,17.3463);
\draw [c] (11.6,17.3331) -- (11.76,17.3331);
\draw [c] (11.76,17.3331) -- (11.92,17.3331);
\draw [c] (12.08,17.2832) -- (12.08,17.2972);
\draw [c] (12.08,17.2972) -- (12.08,17.3109);
\draw [c] (11.92,17.2972) -- (12.08,17.2972);
\draw [c] (12.08,17.2972) -- (12.24,17.2972);
\draw [c] (12.4,17.2096) -- (12.4,17.2247);
\draw [c] (12.4,17.2247) -- (12.4,17.2393);
\draw [c] (12.24,17.2247) -- (12.4,17.2247);
\draw [c] (12.4,17.2247) -- (12.56,17.2247);
\draw [c] (12.72,17.1506) -- (12.72,17.1666);
\draw [c] (12.72,17.1666) -- (12.72,17.182);
\draw [c] (12.56,17.1666) -- (12.72,17.1666);
\draw [c] (12.72,17.1666) -- (12.88,17.1666);
\draw [c] (13.04,17.0776) -- (13.04,17.0947);
\draw [c] (13.04,17.0947) -- (13.04,17.1112);
\draw [c] (12.88,17.0947) -- (13.04,17.0947);
\draw [c] (13.04,17.0947) -- (13.2,17.0947);
\draw [c] (13.36,16.9933) -- (13.36,17.0118);
\draw [c] (13.36,17.0118) -- (13.36,17.0297);
\draw [c] (13.2,17.0118) -- (13.36,17.0118);
\draw [c] (13.36,17.0118) -- (13.52,17.0118);
\draw [c] (13.68,16.943) -- (13.68,16.9624);
\draw [c] (13.68,16.9624) -- (13.68,16.9812);
\draw [c] (13.52,16.9624) -- (13.68,16.9624);
\draw [c] (13.68,16.9624) -- (13.84,16.9624);
\draw [c] (14,16.79) -- (14,16.8125);
\draw [c] (14,16.8125) -- (14,16.8341);
\draw [c] (13.84,16.8125) -- (14,16.8125);
\draw [c] (14,16.8125) -- (14.16,16.8125);
\draw [c] (14.32,16.7483) -- (14.32,16.7717);
\draw [c] (14.32,16.7717) -- (14.32,16.7942);
\draw [c] (14.16,16.7717) -- (14.32,16.7717);
\draw [c] (14.32,16.7717) -- (14.48,16.7717);
\draw [c] (14.64,16.5632) -- (14.64,16.5912);
\draw [c] (14.64,16.5912) -- (14.64,16.6178);
\draw [c] (14.48,16.5912) -- (14.64,16.5912);
\draw [c] (14.64,16.5912) -- (14.8,16.5912);
\draw [c] (14.96,16.4738) -- (14.96,16.5043);
\draw [c] (14.96,16.5043) -- (14.96,16.5331);
\draw [c] (14.8,16.5043) -- (14.96,16.5043);
\draw [c] (14.96,16.5043) -- (15.12,16.5043);
\draw [c] (15.28,16.273) -- (15.28,16.31);
\draw [c] (15.28,16.31) -- (15.28,16.3445);
\draw [c] (15.12,16.31) -- (15.28,16.31);
\draw [c] (15.28,16.31) -- (15.44,16.31);
\draw [c] (15.6,16.1815) -- (15.6,16.222);
\draw [c] (15.6,16.222) -- (15.6,16.2595);
\draw [c] (15.44,16.222) -- (15.6,16.222);
\draw [c] (15.6,16.222) -- (15.76,16.222);
\draw [c] (15.92,16.0161) -- (15.92,16.0635);
\draw [c] (15.92,16.0635) -- (15.92,16.1069);
\draw [c] (15.76,16.0635) -- (15.92,16.0635);
\draw [c] (15.92,16.0635) -- (16.08,16.0635);
\draw [c] (16.24,16.0202) -- (16.24,16.0674);
\draw [c] (16.24,16.0674) -- (16.24,16.1107);
\draw [c] (16.08,16.0674) -- (16.24,16.0674);
\draw [c] (16.24,16.0674) -- (16.4,16.0674);
\draw [c] (16.56,15.8475) -- (16.56,15.9032);
\draw [c] (16.56,15.9032) -- (16.56,15.9535);
\draw [c] (16.4,15.9032) -- (16.56,15.9032);
\draw [c] (16.56,15.9032) -- (16.72,15.9032);
\draw [c] (16.88,15.6512) -- (16.88,15.7184);
\draw [c] (16.88,15.7184) -- (16.88,15.778);
\draw [c] (16.72,15.7184) -- (16.88,15.7184);
\draw [c] (16.88,15.7184) -- (17.04,15.7184);
\draw [c] (17.2,15.5624) -- (17.2,15.6357);
\draw [c] (17.2,15.6357) -- (17.2,15.6999);
\draw [c] (17.04,15.6357) -- (17.2,15.6357);
\draw [c] (17.2,15.6357) -- (17.36,15.6357);
\draw [c] (17.52,15.5327) -- (17.52,15.608);
\draw [c] (17.52,15.608) -- (17.52,15.6738);
\draw [c] (17.36,15.608) -- (17.52,15.608);
\draw [c] (17.52,15.608) -- (17.68,15.608);
\draw [c] (17.84,15.2927) -- (17.84,15.3875);
\draw [c] (17.84,15.3875) -- (17.84,15.4677);
\draw [c] (17.68,15.3875) -- (17.84,15.3875);
\draw [c] (17.84,15.3875) -- (18,15.3875);
\definecolor{c}{rgb}{1,1,1};
\draw [c] (10,19.7074) -- (10,22.1855) -- (17.6,22.1855) -- (17.6,19.7074) -- (10,19.7074);
\draw [c] (10,19.7074) -- (17.6,19.7074);
\draw [c] (17.6,19.7074) -- (17.6,22.1855);
\draw [c] (17.6,22.1855) -- (10,22.1855);
\draw [c] (10,22.1855) -- (10,19.7074);
\draw [anchor=base west] (10.1425,21.7364) node[ rotate=0]{Hard event};
\draw [anchor=base west] (13.1476,21.7364) node[ rotate=0]{Extended event};
\draw [anchor=base west] (10.95,21.1168) node[ rotate=0]{};
\colorlet{c}{kugray!20};
\draw [c, fill=c] (10.1425,21.0394) -- (10.8075,21.0394) -- (10.8075,21.4731) -- (10.1425,21.4731);
\colorlet{c}{kugray!50};
\draw [c] (10.1425,21.2562) -- (10.8075,21.2562);
\draw [anchor=base west] (13.9551,21.1168) node[ rotate=0]{Standard Model};
\colorlet{c}{kugray};
\draw [c] (13.1476,21.2562) -- (13.8126,21.2562);
\draw [anchor=base west] (10.95,20.4973) node[ rotate=0]{};
\colorlet{c}{natgreen!20};
\draw [c, fill=c] (10.1425,20.4198) -- (10.8075,20.4198) -- (10.8075,20.8535) -- (10.1425,20.8535);
\colorlet{c}{natgreen!50};
\draw [c] (10.1425,20.6367) -- (10.8075,20.6367);
\draw [anchor=base west] (13.9551,20.4973) node[ rotate=0]{$\Lambda$ = 1.00 TeV};
\colorlet{c}{natgreen};
\draw [c] (13.1476,20.6367) -- (13.8126,20.6367);
\draw [anchor=base west] (10.95,19.8777) node[ rotate=0]{};
\colorlet{c}{natcomp!20};
\draw [c, fill=c] (10.1425,19.8003) -- (10.8075,19.8003) -- (10.8075,20.234) -- (10.1425,20.234);
\colorlet{c}{natcomp!50};
\draw [c] (10.1425,20.0171) -- (10.8075,20.0171);
\draw [anchor=base west] (13.9551,19.8777) node[ rotate=0]{$\Lambda$ = 0.75 TeV};
\colorlet{c}{natcomp};
\draw [c] (13.1476,20.0171) -- (13.8126,20.0171);
%\definecolor{c}{rgb}{1,1,1};
%\draw [color=c, fill=c] (0,8.26057) rectangle (20,11.8008);
%\draw [color=c, fill=c] (1.99181,8.26739) rectangle (17.9809,11.8008);
\definecolor{c}{rgb}{0,0,0};
\draw [c] (1.99181,8.26739) -- (1.99181,11.8008) -- (17.9809,11.8008) -- (17.9809,8.26739) -- (1.99181,8.26739);


\colorlet{c}{natcomp!20};
\draw [color=c, fill=c] (1.99181,11.1767) rectangle (2.3116,11.2472);
\draw [color=c, fill=c] (2.3116,11.1134) rectangle (2.63138,11.3104);
\draw [color=c, fill=c] (2.63138,11.0135) rectangle (2.95116,11.4103);
\draw [color=c, fill=c] (2.95116,10.879) rectangle (3.27094,11.5448);
\draw [color=c, fill=c] (3.27094,10.7262) rectangle (3.59072,11.6976);
\draw [color=c, fill=c] (3.59072,10.5319) rectangle (3.9105,11.8008);
\draw [color=c, fill=c] (3.9105,10.2438) rectangle (4.23029,11.8008);
\draw [color=c, fill=c] (4.23029,10.057) rectangle (4.55007,11.8008);
\draw [color=c, fill=c] (4.55007,9.86088) rectangle (4.86985,11.8008);
\draw [color=c, fill=c] (4.86985,10.4271) rectangle (5.18963,11.8008);
\draw [color=c, fill=c] (5.18963,11.1058) rectangle (5.50941,11.318);
\draw [color=c, fill=c] (5.50941,11.0976) rectangle (5.8292,11.3263);
\draw [color=c, fill=c] (5.8292,11.0899) rectangle (6.14898,11.3339);
\draw [color=c, fill=c] (6.14898,11.0873) rectangle (6.46876,11.3365);
\draw [color=c, fill=c] (6.46876,11.0845) rectangle (6.78854,11.3393);
\draw [color=c, fill=c] (6.78854,11.0803) rectangle (7.10832,11.3435);
\draw [color=c, fill=c] (7.10832,11.0791) rectangle (7.4281,11.3447);
\draw [color=c, fill=c] (7.4281,11.077) rectangle (7.74789,11.3468);
\draw [color=c, fill=c] (7.74789,11.0745) rectangle (8.06767,11.3493);
\draw [color=c, fill=c] (8.06767,11.0738) rectangle (8.38745,11.35);
\draw [color=c, fill=c] (8.38745,11.0706) rectangle (8.70723,11.3532);
\draw [color=c, fill=c] (8.70723,11.0701) rectangle (9.02701,11.3537);
\draw [color=c, fill=c] (9.02701,11.0668) rectangle (9.34679,11.357);
\draw [color=c, fill=c] (9.34679,11.064) rectangle (9.66658,11.3598);
\draw [color=c, fill=c] (9.66658,11.0606) rectangle (9.98636,11.3632);
\draw [color=c, fill=c] (9.98636,11.0592) rectangle (10.3061,11.3647);
\draw [color=c, fill=c] (10.3061,11.0498) rectangle (10.6259,11.374);
\draw [color=c, fill=c] (10.6259,11.0463) rectangle (10.9457,11.3775);
\draw [color=c, fill=c] (10.9457,11.0376) rectangle (11.2655,11.3862);
\draw [color=c, fill=c] (11.2655,11.0336) rectangle (11.5853,11.3902);
\draw [color=c, fill=c] (11.5853,11.0201) rectangle (11.905,11.4037);
\draw [color=c, fill=c] (11.905,11.0116) rectangle (12.2248,11.4123);
\draw [color=c, fill=c] (12.2248,10.9998) rectangle (12.5446,11.4241);
\draw [color=c, fill=c] (12.5446,10.9857) rectangle (12.8644,11.4382);
\draw [color=c, fill=c] (12.8644,10.968) rectangle (13.1842,11.4558);
\draw [color=c, fill=c] (13.1842,10.9515) rectangle (13.504,11.4723);
\draw [color=c, fill=c] (13.504,10.9358) rectangle (13.8237,11.488);
\draw [color=c, fill=c] (13.8237,10.902) rectangle (14.1435,11.5219);
\draw [color=c, fill=c] (14.1435,10.8772) rectangle (14.4633,11.5467);
\draw [color=c, fill=c] (14.4633,10.8228) rectangle (14.7831,11.6011);
\draw [color=c, fill=c] (14.7831,10.7858) rectangle (15.1029,11.638);
\draw [color=c, fill=c] (15.1029,10.6993) rectangle (15.4226,11.7245);
\draw [color=c, fill=c] (15.4226,10.6616) rectangle (15.7424,11.7623);
\draw [color=c, fill=c] (15.7424,10.5769) rectangle (16.0622,11.8008);
\draw [color=c, fill=c] (16.0622,10.5408) rectangle (16.382,11.8008);
\draw [color=c, fill=c] (16.382,10.4579) rectangle (16.7018,11.8008);
\draw [color=c, fill=c] (16.7018,10.2865) rectangle (17.0216,11.8008);
\draw [color=c, fill=c] (17.0216,10.2502) rectangle (17.3413,11.8008);
\draw [color=c, fill=c] (17.3413,10.1709) rectangle (17.6611,11.8008);
\draw [color=c, fill=c] (17.6611,9.95637) rectangle (17.9809,11.8008);
\definecolor{c}{rgb}{0,0,0};
\draw [c] (1.99181,8.26739) -- (1.99181,11.8008) -- (17.9809,11.8008) -- (17.9809,8.26739) -- (1.99181,8.26739);
\definecolor{c}{rgb}{0,0,0};
\draw [c] (1.99181,8.26739) -- (17.9809,8.26739);
\draw [c] (3.29705,8.3523) -- (3.29705,8.26739);
\draw [c] (3.62335,8.30985) -- (3.62335,8.26739);
\draw [c] (3.94966,8.30985) -- (3.94966,8.26739);
\draw [c] (4.27597,8.30985) -- (4.27597,8.26739);
\draw [c] (4.60228,8.30985) -- (4.60228,8.26739);
\draw [c] (4.92859,8.3523) -- (4.92859,8.26739);
\draw [c] (5.25489,8.30985) -- (5.25489,8.26739);
\draw [c] (5.5812,8.30985) -- (5.5812,8.26739);
\draw [c] (5.90751,8.30985) -- (5.90751,8.26739);
\draw [c] (6.23382,8.30985) -- (6.23382,8.26739);
\draw [c] (6.56012,8.3523) -- (6.56012,8.26739);
\draw [c] (6.88643,8.30985) -- (6.88643,8.26739);
\draw [c] (7.21274,8.30985) -- (7.21274,8.26739);
\draw [c] (7.53905,8.30985) -- (7.53905,8.26739);
\draw [c] (7.86536,8.30985) -- (7.86536,8.26739);
\draw [c] (8.19166,8.3523) -- (8.19166,8.26739);
\draw [c] (8.51797,8.30985) -- (8.51797,8.26739);
\draw [c] (8.84428,8.30985) -- (8.84428,8.26739);
\draw [c] (9.17059,8.30985) -- (9.17059,8.26739);
\draw [c] (9.4969,8.30985) -- (9.4969,8.26739);
\draw [c] (9.8232,8.3523) -- (9.8232,8.26739);
\draw [c] (10.1495,8.30985) -- (10.1495,8.26739);
\draw [c] (10.4758,8.30985) -- (10.4758,8.26739);
\draw [c] (10.8021,8.30985) -- (10.8021,8.26739);
\draw [c] (11.1284,8.30985) -- (11.1284,8.26739);
\draw [c] (11.4547,8.3523) -- (11.4547,8.26739);
\draw [c] (11.7811,8.30985) -- (11.7811,8.26739);
\draw [c] (12.1074,8.30985) -- (12.1074,8.26739);
\draw [c] (12.4337,8.30985) -- (12.4337,8.26739);
\draw [c] (12.76,8.30985) -- (12.76,8.26739);
\draw [c] (13.0863,8.3523) -- (13.0863,8.26739);
\draw [c] (13.4126,8.30985) -- (13.4126,8.26739);
\draw [c] (13.7389,8.30985) -- (13.7389,8.26739);
\draw [c] (14.0652,8.30985) -- (14.0652,8.26739);
\draw [c] (14.3915,8.30985) -- (14.3915,8.26739);
\draw [c] (14.7178,8.3523) -- (14.7178,8.26739);
\draw [c] (15.0441,8.30985) -- (15.0441,8.26739);
\draw [c] (15.3704,8.30985) -- (15.3704,8.26739);
\draw [c] (15.6967,8.30985) -- (15.6967,8.26739);
\draw [c] (16.0231,8.30985) -- (16.0231,8.26739);
\draw [c] (16.3494,8.3523) -- (16.3494,8.26739);
\draw [c] (16.6757,8.30985) -- (16.6757,8.26739);
\draw [c] (17.002,8.30985) -- (17.002,8.26739);
\draw [c] (17.3283,8.30985) -- (17.3283,8.26739);
\draw [c] (17.6546,8.30985) -- (17.6546,8.26739);
\draw [c] (17.9809,8.3523) -- (17.9809,8.26739);
\draw [c] (3.29705,8.3523) -- (3.29705,8.26739);
\draw [c] (2.97074,8.30985) -- (2.97074,8.26739);
\draw [c] (2.64443,8.30985) -- (2.64443,8.26739);
\draw [c] (2.31812,8.30985) -- (2.31812,8.26739);

\draw [c] (1.99181,8.26739) -- (1.99181,11.8008);
\draw [anchor= east] (0.1,11.8008) node[ rotate=90]{Ratio};
\draw [c] (2.59066,8.8563) -- (1.99181,8.8563);
\draw [c] (2.29124,9.15075) -- (1.99181,9.15075);
\draw [c] (2.29124,9.4452) -- (1.99181,9.4452);
\draw [c] (2.29124,9.73965) -- (1.99181,9.73965);
\draw [c] (2.59066,10.0341) -- (1.99181,10.0341);
\draw [c] (2.29124,10.3286) -- (1.99181,10.3286);
\draw [c] (2.29124,10.623) -- (1.99181,10.623);
\draw [c] (2.29124,10.9175) -- (1.99181,10.9175);
\draw [c] (2.59066,11.2119) -- (1.99181,11.2119);
\draw [c] (2.59066,8.8563) -- (1.99181,8.8563);
\draw [c] (2.29124,8.56185) -- (1.99181,8.56185);
\draw [c] (2.29124,8.26739) -- (1.99181,8.26739);
\draw [c] (2.59066,11.2119) -- (1.99181,11.2119);
\draw [c] (2.29124,11.5064) -- (1.99181,11.5064);
\draw [anchor= east] (1.89181,8.8563) node[ rotate=0]{0.6};
\draw [anchor= east] (1.89181,10.0341) node[ rotate=0]{0.8};
\draw [anchor= east] (1.89181,11.2119) node[ rotate=0]{1.0};
\colorlet{c}{natcomp};
\draw [c] (2.15171,10.1822) -- (2.15171,10.213);
\draw [c] (2.15171,10.213) -- (2.15171,10.2437);
\draw [c] (1.99181,10.213) -- (2.15171,10.213);
\draw [c] (2.15171,10.213) -- (2.3116,10.213);
\draw [c] (2.47149,10.2181) -- (2.47149,10.3052);
\draw [c] (2.47149,10.3052) -- (2.47149,10.3922);
\draw [c] (2.3116,10.3052) -- (2.47149,10.3052);
\draw [c] (2.47149,10.3052) -- (2.63138,10.3052);
\draw [c] (2.79127,9.93575) -- (2.79127,10.1059);
\draw [c] (2.79127,10.1059) -- (2.79127,10.2762);
\draw [c] (2.63138,10.1059) -- (2.79127,10.1059);
\draw [c] (2.79127,10.1059) -- (2.95116,10.1059);
\draw [c] (3.11105,9.83519) -- (3.11105,10.1214);
\draw [c] (3.11105,10.1214) -- (3.11105,10.4076);
\draw [c] (2.95116,10.1214) -- (3.11105,10.1214);
\draw [c] (3.11105,10.1214) -- (3.27094,10.1214);
\draw [c] (3.43083,9.48778) -- (3.43083,9.89084);
\draw [c] (3.43083,9.89084) -- (3.43083,10.2939);
\draw [c] (3.27094,9.89084) -- (3.43083,9.89084);
\draw [c] (3.43083,9.89084) -- (3.59072,9.89084);
\draw [c] (3.75061,9.53015) -- (3.75061,10.1142);
\draw [c] (3.75061,10.1142) -- (3.75061,10.6982);
\draw [c] (3.59072,10.1142) -- (3.75061,10.1142);
\draw [c] (3.75061,10.1142) -- (3.9105,10.1142);
\draw [c] (4.0704,9.06166) -- (4.0704,9.86124);
\draw [c] (4.0704,9.86124) -- (4.0704,10.6608);
\draw [c] (3.9105,9.86124) -- (4.0704,9.86124);
\draw [c] (4.0704,9.86124) -- (4.23029,9.86124);
\draw [c] (4.39018,8.90223) -- (4.39018,9.85515);
\draw [c] (4.39018,9.85515) -- (4.39018,10.8081);
\draw [c] (4.23029,9.85515) -- (4.39018,9.85515);
\draw [c] (4.39018,9.85515) -- (4.55007,9.85515);
\draw [c] (4.70996,8.58418) -- (4.70996,9.66524);
\draw [c] (4.70996,9.66524) -- (4.70996,10.7463);
\draw [c] (4.55007,9.66524) -- (4.70996,9.66524);
\draw [c] (4.70996,9.66524) -- (4.86985,9.66524);
\draw [c] (5.02974,9.65838) -- (5.02974,10.3901);
\draw [c] (5.02974,10.3901) -- (5.02974,11.1218);
\draw [c] (4.86985,10.3901) -- (5.02974,10.3901);
\draw [c] (5.02974,10.3901) -- (5.18963,10.3901);
\draw [c] (5.34952,9.94367) -- (5.34952,10.0337);
\draw [c] (5.34952,10.0337) -- (5.34952,10.1238);
\draw [c] (5.18963,10.0337) -- (5.34952,10.0337);
\draw [c] (5.34952,10.0337) -- (5.50941,10.0337);
\draw [c] (5.6693,9.88678) -- (5.6693,9.98304);
\draw [c] (5.6693,9.98304) -- (5.6693,10.0793);
\draw [c] (5.50941,9.98304) -- (5.6693,9.98304);
\draw [c] (5.6693,9.98304) -- (5.8292,9.98304);
\draw [c] (5.98909,9.90198) -- (5.98909,10.005);
\draw [c] (5.98909,10.005) -- (5.98909,10.1081);
\draw [c] (5.8292,10.005) -- (5.98909,10.005);
\draw [c] (5.98909,10.005) -- (6.14898,10.005);
\draw [c] (6.30887,9.91879) -- (6.30887,10.0244);
\draw [c] (6.30887,10.0244) -- (6.30887,10.1299);
\draw [c] (6.14898,10.0244) -- (6.30887,10.0244);
\draw [c] (6.30887,10.0244) -- (6.46876,10.0244);
\draw [c] (6.62865,9.88073) -- (6.62865,9.98807);
\draw [c] (6.62865,9.98807) -- (6.62865,10.0954);
\draw [c] (6.46876,9.98807) -- (6.62865,9.98807);
\draw [c] (6.62865,9.98807) -- (6.78854,9.98807);
\draw [c] (6.94843,9.8799) -- (6.94843,9.99085);
\draw [c] (6.94843,9.99085) -- (6.94843,10.1018);
\draw [c] (6.78854,9.99085) -- (6.94843,9.99085);
\draw [c] (6.94843,9.99085) -- (7.10832,9.99085);
\draw [c] (7.26821,9.87995) -- (7.26821,9.99189);
\draw [c] (7.26821,9.99189) -- (7.26821,10.1038);
\draw [c] (7.10832,9.99189) -- (7.26821,9.99189);
\draw [c] (7.26821,9.99189) -- (7.4281,9.99189);
\draw [c] (7.58799,9.86865) -- (7.58799,9.9822);
\draw [c] (7.58799,9.9822) -- (7.58799,10.0957);
\draw [c] (7.4281,9.9822) -- (7.58799,9.9822);
\draw [c] (7.58799,9.9822) -- (7.74789,9.9822);
\draw [c] (7.90778,9.8684) -- (7.90778,9.9841);
\draw [c] (7.90778,9.9841) -- (7.90778,10.0998);
\draw [c] (7.74789,9.9841) -- (7.90778,9.9841);
\draw [c] (7.90778,9.9841) -- (8.06767,9.9841);
\draw [c] (8.22756,9.87781) -- (8.22756,9.99427);
\draw [c] (8.22756,9.99427) -- (8.22756,10.1107);
\draw [c] (8.06767,9.99427) -- (8.22756,9.99427);
\draw [c] (8.22756,9.99427) -- (8.38745,9.99427);
\draw [c] (8.54734,9.84103) -- (8.54734,9.95954);
\draw [c] (8.54734,9.95954) -- (8.54734,10.078);
\draw [c] (8.38745,9.95954) -- (8.54734,9.95954);
\draw [c] (8.54734,9.95954) -- (8.70723,9.95954);
\draw [c] (8.86712,9.91882) -- (8.86712,10.0392);
\draw [c] (8.86712,10.0392) -- (8.86712,10.1596);
\draw [c] (8.70723,10.0392) -- (8.86712,10.0392);
\draw [c] (8.86712,10.0392) -- (9.02701,10.0392);
\draw [c] (9.1869,9.92502) -- (9.1869,10.0484);
\draw [c] (9.1869,10.0484) -- (9.1869,10.1718);
\draw [c] (9.02701,10.0484) -- (9.1869,10.0484);
\draw [c] (9.1869,10.0484) -- (9.34679,10.0484);
\draw [c] (9.50669,9.85838) -- (9.50669,9.98286);
\draw [c] (9.50669,9.98286) -- (9.50669,10.1074);
\draw [c] (9.34679,9.98286) -- (9.50669,9.98286);
\draw [c] (9.50669,9.98286) -- (9.66658,9.98286);
\draw [c] (9.82647,9.87673) -- (9.82647,10.0046);
\draw [c] (9.82647,10.0046) -- (9.82647,10.1324);
\draw [c] (9.66658,10.0046) -- (9.82647,10.0046);
\draw [c] (9.82647,10.0046) -- (9.98636,10.0046);
\draw [c] (10.1462,9.92702) -- (10.1462,10.0571);
\draw [c] (10.1462,10.0571) -- (10.1462,10.1871);
\draw [c] (9.98636,10.0571) -- (10.1462,10.0571);
\draw [c] (10.1462,10.0571) -- (10.3061,10.0571);
\draw [c] (10.466,9.79562) -- (10.466,9.93101);
\draw [c] (10.466,9.93101) -- (10.466,10.0664);
\draw [c] (10.3061,9.93101) -- (10.466,9.93101);
\draw [c] (10.466,9.93101) -- (10.6259,9.93101);
\draw [c] (10.7858,9.85717) -- (10.7858,9.99686);
\draw [c] (10.7858,9.99686) -- (10.7858,10.1366);
\draw [c] (10.6259,9.99686) -- (10.7858,9.99686);
\draw [c] (10.7858,9.99686) -- (10.9457,9.99686);
\draw [c] (11.1056,9.90536) -- (11.1056,10.0537);
\draw [c] (11.1056,10.0537) -- (11.1056,10.2021);
\draw [c] (10.9457,10.0537) -- (11.1056,10.0537);
\draw [c] (11.1056,10.0537) -- (11.2655,10.0537);
\draw [c] (11.4254,10.0236) -- (11.4254,10.1782);
\draw [c] (11.4254,10.1782) -- (11.4254,10.3329);
\draw [c] (11.2655,10.1782) -- (11.4254,10.1782);
\draw [c] (11.4254,10.1782) -- (11.5853,10.1782);
\draw [c] (11.7452,9.89266) -- (11.7452,10.056);
\draw [c] (11.7452,10.056) -- (11.7452,10.2193);
\draw [c] (11.5853,10.056) -- (11.7452,10.056);
\draw [c] (11.7452,10.056) -- (11.905,10.056);
\draw [c] (12.0649,9.96899) -- (12.0649,10.1418);
\draw [c] (12.0649,10.1418) -- (12.0649,10.3146);
\draw [c] (11.905,10.1418) -- (12.0649,10.1418);
\draw [c] (12.0649,10.1418) -- (12.2248,10.1418);
\draw [c] (12.3847,9.84368) -- (12.3847,10.0234);
\draw [c] (12.3847,10.0234) -- (12.3847,10.2031);
\draw [c] (12.2248,10.0234) -- (12.3847,10.0234);
\draw [c] (12.3847,10.0234) -- (12.5446,10.0234);
\draw [c] (12.7045,9.90962) -- (12.7045,10.1036);
\draw [c] (12.7045,10.1036) -- (12.7045,10.2977);
\draw [c] (12.5446,10.1036) -- (12.7045,10.1036);
\draw [c] (12.7045,10.1036) -- (12.8644,10.1036);
\draw [c] (13.0243,9.95039) -- (13.0243,10.1614);
\draw [c] (13.0243,10.1614) -- (13.0243,10.3724);
\draw [c] (12.8644,10.1614) -- (13.0243,10.1614);
\draw [c] (13.0243,10.1614) -- (13.1842,10.1614);
\draw [c] (13.3441,9.80537) -- (13.3441,10.026);
\draw [c] (13.3441,10.026) -- (13.3441,10.2467);
\draw [c] (13.1842,10.026) -- (13.3441,10.026);
\draw [c] (13.3441,10.026) -- (13.504,10.026);
\draw [c] (13.6638,9.89343) -- (13.6638,10.1312);
\draw [c] (13.6638,10.1312) -- (13.6638,10.3689);
\draw [c] (13.504,10.1312) -- (13.6638,10.1312);
\draw [c] (13.6638,10.1312) -- (13.8237,10.1312);
\draw [c] (13.9836,9.60987) -- (13.9836,9.86608);
\draw [c] (13.9836,9.86608) -- (13.9836,10.1223);
\draw [c] (13.8237,9.86608) -- (13.9836,9.86608);
\draw [c] (13.9836,9.86608) -- (14.1435,9.86608);
\draw [c] (14.3034,9.93025) -- (14.3034,10.2225);
\draw [c] (14.3034,10.2225) -- (14.3034,10.5147);
\draw [c] (14.1435,10.2225) -- (14.3034,10.2225);
\draw [c] (14.3034,10.2225) -- (14.4633,10.2225);
\draw [c] (14.6232,9.6746) -- (14.6232,10.0032);
\draw [c] (14.6232,10.0032) -- (14.6232,10.3319);
\draw [c] (14.4633,10.0032) -- (14.6232,10.0032);
\draw [c] (14.6232,10.0032) -- (14.7831,10.0032);
\draw [c] (14.943,9.70748) -- (14.943,10.0711);
\draw [c] (14.943,10.0711) -- (14.943,10.4347);
\draw [c] (14.7831,10.0711) -- (14.943,10.0711);
\draw [c] (14.943,10.0711) -- (15.1029,10.0711);
\draw [c] (15.2628,9.61583) -- (15.2628,10.052);
\draw [c] (15.2628,10.052) -- (15.2628,10.4881);
\draw [c] (15.1029,10.052) -- (15.2628,10.052);
\draw [c] (15.2628,10.052) -- (15.4226,10.052);
\draw [c] (15.5825,9.46685) -- (15.5825,9.9261);
\draw [c] (15.5825,9.9261) -- (15.5825,10.3853);
\draw [c] (15.4226,9.9261) -- (15.5825,9.9261);
\draw [c] (15.5825,9.9261) -- (15.7424,9.9261);
\draw [c] (15.9023,9.3194) -- (15.9023,9.84237);
\draw [c] (15.9023,9.84237) -- (15.9023,10.3653);
\draw [c] (15.7424,9.84237) -- (15.9023,9.84237);
\draw [c] (15.9023,9.84237) -- (16.0622,9.84237);
\draw [c] (16.2221,9.80682) -- (16.2221,10.4089);
\draw [c] (16.2221,10.4089) -- (16.2221,11.0109);
\draw [c] (16.0622,10.4089) -- (16.2221,10.4089);
\draw [c] (16.2221,10.4089) -- (16.382,10.4089);
\draw [c] (16.5419,9.36818) -- (16.5419,10.0051);
\draw [c] (16.5419,10.0051) -- (16.5419,10.6421);
\draw [c] (16.382,10.0051) -- (16.5419,10.0051);
\draw [c] (16.5419,10.0051) -- (16.7018,10.0051);
\draw [c] (16.8617,9.45362) -- (16.8617,10.2668);
\draw [c] (16.8617,10.2668) -- (16.8617,11.0799);
\draw [c] (16.7018,10.2668) -- (16.8617,10.2668);
\draw [c] (16.8617,10.2668) -- (17.0216,10.2668);
\draw [c] (17.1814,9.08074) -- (17.1814,9.87707);
\draw [c] (17.1814,9.87707) -- (17.1814,10.6734);
\draw [c] (17.0216,9.87707) -- (17.1814,9.87707);
\draw [c] (17.1814,9.87707) -- (17.3413,9.87707);
\draw [c] (17.5012,9.45324) -- (17.5012,10.3838);
\draw [c] (17.5012,10.3838) -- (17.5012,11.3143);
\draw [c] (17.3413,10.3838) -- (17.5012,10.3838);
\draw [c] (17.5012,10.3838) -- (17.6611,10.3838);
\draw [c] (17.821,9.05835) -- (17.821,10.1412);
\draw [c] (17.821,10.1412) -- (17.821,11.224);
\draw [c] (17.6611,10.1412) -- (17.821,10.1412);
\draw [c] (17.821,10.1412) -- (17.9809,10.1412);

\definecolor{c}{rgb}{0,0,0};
\draw [c] (1.99181,4.72033) -- (1.99181,8.26057) -- (17.9809,8.26057) -- (17.9809,4.72033) -- (1.99181,4.72033);


\colorlet{c}{natgreen!20};
\draw [color=c, fill=c] (1.99181,7.63529) rectangle (2.3116,7.70578);
\draw [color=c, fill=c] (2.3116,7.57144) rectangle (2.63138,7.76963);
\draw [color=c, fill=c] (2.63138,7.46886) rectangle (2.95116,7.8722);
\draw [color=c, fill=c] (2.95116,7.33128) rectangle (3.27094,8.00978);
\draw [color=c, fill=c] (3.27094,7.18221) rectangle (3.59072,8.15885);
\draw [color=c, fill=c] (3.59072,6.9678) rectangle (3.9105,8.26057);
\draw [color=c, fill=c] (3.9105,6.69389) rectangle (4.23029,8.26057);
\draw [color=c, fill=c] (4.23029,6.42662) rectangle (4.55007,8.26057);
\draw [color=c, fill=c] (4.55007,6.12101) rectangle (4.86985,8.26057);
\draw [color=c, fill=c] (4.86985,6.80106) rectangle (5.18963,8.26057);
\draw [color=c, fill=c] (5.18963,7.59289) rectangle (5.50941,7.74818);
\draw [color=c, fill=c] (5.50941,7.57918) rectangle (5.8292,7.76189);
\draw [color=c, fill=c] (5.8292,7.56527) rectangle (6.14898,7.7758);
\draw [color=c, fill=c] (6.14898,7.55413) rectangle (6.46876,7.78693);
\draw [color=c, fill=c] (6.46876,7.54063) rectangle (6.78854,7.80044);
\draw [color=c, fill=c] (6.78854,7.52851) rectangle (7.10832,7.81256);
\draw [color=c, fill=c] (7.10832,7.51757) rectangle (7.4281,7.82349);
\draw [color=c, fill=c] (7.4281,7.50925) rectangle (7.74789,7.83181);
\draw [color=c, fill=c] (7.74789,7.49759) rectangle (8.06767,7.84348);
\draw [color=c, fill=c] (8.06767,7.49007) rectangle (8.38745,7.851);
\draw [color=c, fill=c] (8.38745,7.48229) rectangle (8.70723,7.85877);
\draw [color=c, fill=c] (8.70723,7.47526) rectangle (9.02701,7.86581);
\draw [color=c, fill=c] (9.02701,7.46803) rectangle (9.34679,7.87303);
\draw [color=c, fill=c] (9.34679,7.45414) rectangle (9.66658,7.88692);
\draw [color=c, fill=c] (9.66658,7.44815) rectangle (9.98636,7.89291);
\draw [color=c, fill=c] (9.98636,7.43629) rectangle (10.3061,7.90477);
\draw [color=c, fill=c] (10.3061,7.42679) rectangle (10.6259,7.91428);
\draw [color=c, fill=c] (10.6259,7.41387) rectangle (10.9457,7.92719);
\draw [color=c, fill=c] (10.9457,7.40371) rectangle (11.2655,7.93736);
\draw [color=c, fill=c] (11.2655,7.38812) rectangle (11.5853,7.95295);
\draw [color=c, fill=c] (11.5853,7.36785) rectangle (11.905,7.97322);
\draw [color=c, fill=c] (11.905,7.35912) rectangle (12.2248,7.98194);
\draw [color=c, fill=c] (12.2248,7.33835) rectangle (12.5446,8.00272);
\draw [color=c, fill=c] (12.5446,7.31823) rectangle (12.8644,8.02283);
\draw [color=c, fill=c] (12.8644,7.27318) rectangle (13.1842,8.06789);
\draw [color=c, fill=c] (13.1842,7.25742) rectangle (13.504,8.08364);
\draw [color=c, fill=c] (13.504,7.2026) rectangle (13.8237,8.13846);
\draw [color=c, fill=c] (13.8237,7.17798) rectangle (14.1435,8.16309);
\draw [color=c, fill=c] (14.1435,7.10795) rectangle (14.4633,8.23311);
\draw [color=c, fill=c] (14.4633,7.052) rectangle (14.7831,8.26057);
\draw [color=c, fill=c] (14.7831,7.00877) rectangle (15.1029,8.26057);
\draw [color=c, fill=c] (15.1029,6.94698) rectangle (15.4226,8.26057);
\draw [color=c, fill=c] (15.4226,6.8562) rectangle (15.7424,8.26057);
\draw [color=c, fill=c] (15.7424,6.86005) rectangle (16.0622,8.26057);
\draw [color=c, fill=c] (16.0622,6.55546) rectangle (16.382,8.26057);
\draw [color=c, fill=c] (16.382,6.41256) rectangle (16.7018,8.26057);
\draw [color=c, fill=c] (16.7018,6.29871) rectangle (17.0216,8.26057);
\draw [color=c, fill=c] (17.0216,6.06464) rectangle (17.3413,8.26057);
\draw [color=c, fill=c] (17.3413,5.58442) rectangle (17.6611,8.26057);
\draw [color=c, fill=c] (17.6611,5.03179) rectangle (17.9809,8.26057);
\definecolor{c}{rgb}{0,0,0};
\draw [c] (1.99181,4.72033) -- (17.9809,4.72033);
\draw [c] (3.29705,4.80524) -- (3.29705,4.72033);
\draw [c] (3.62335,4.76278) -- (3.62335,4.72033);
\draw [c] (3.94966,4.76278) -- (3.94966,4.72033);
\draw [c] (4.27597,4.76278) -- (4.27597,4.72033);
\draw [c] (4.60228,4.76278) -- (4.60228,4.72033);
\draw [c] (4.92859,4.80524) -- (4.92859,4.72033);
\draw [c] (5.25489,4.76278) -- (5.25489,4.72033);
\draw [c] (5.5812,4.76278) -- (5.5812,4.72033);
\draw [c] (5.90751,4.76278) -- (5.90751,4.72033);
\draw [c] (6.23382,4.76278) -- (6.23382,4.72033);
\draw [c] (6.56012,4.80524) -- (6.56012,4.72033);
\draw [c] (6.88643,4.76278) -- (6.88643,4.72033);
\draw [c] (7.21274,4.76278) -- (7.21274,4.72033);
\draw [c] (7.53905,4.76278) -- (7.53905,4.72033);
\draw [c] (7.86536,4.76278) -- (7.86536,4.72033);
\draw [c] (8.19166,4.80524) -- (8.19166,4.72033);
\draw [c] (8.51797,4.76278) -- (8.51797,4.72033);
\draw [c] (8.84428,4.76278) -- (8.84428,4.72033);
\draw [c] (9.17059,4.76278) -- (9.17059,4.72033);
\draw [c] (9.4969,4.76278) -- (9.4969,4.72033);
\draw [c] (9.8232,4.80524) -- (9.8232,4.72033);
\draw [c] (10.1495,4.76278) -- (10.1495,4.72033);
\draw [c] (10.4758,4.76278) -- (10.4758,4.72033);
\draw [c] (10.8021,4.76278) -- (10.8021,4.72033);
\draw [c] (11.1284,4.76278) -- (11.1284,4.72033);
\draw [c] (11.4547,4.80524) -- (11.4547,4.72033);
\draw [c] (11.7811,4.76278) -- (11.7811,4.72033);
\draw [c] (12.1074,4.76278) -- (12.1074,4.72033);
\draw [c] (12.4337,4.76278) -- (12.4337,4.72033);
\draw [c] (12.76,4.76278) -- (12.76,4.72033);
\draw [c] (13.0863,4.80524) -- (13.0863,4.72033);
\draw [c] (13.4126,4.76278) -- (13.4126,4.72033);
\draw [c] (13.7389,4.76278) -- (13.7389,4.72033);
\draw [c] (14.0652,4.76278) -- (14.0652,4.72033);
\draw [c] (14.3915,4.76278) -- (14.3915,4.72033);
\draw [c] (14.7178,4.80524) -- (14.7178,4.72033);
\draw [c] (15.0441,4.76278) -- (15.0441,4.72033);
\draw [c] (15.3704,4.76278) -- (15.3704,4.72033);
\draw [c] (15.6967,4.76278) -- (15.6967,4.72033);
\draw [c] (16.0231,4.76278) -- (16.0231,4.72033);
\draw [c] (16.3494,4.80524) -- (16.3494,4.72033);
\draw [c] (16.6757,4.76278) -- (16.6757,4.72033);
\draw [c] (17.002,4.76278) -- (17.002,4.72033);
\draw [c] (17.3283,4.76278) -- (17.3283,4.72033);
\draw [c] (17.6546,4.76278) -- (17.6546,4.72033);
\draw [c] (17.9809,4.80524) -- (17.9809,4.72033);
\draw [c] (3.29705,4.80524) -- (3.29705,4.72033);
\draw [c] (2.97074,4.76278) -- (2.97074,4.72033);
\draw [c] (2.64443,4.76278) -- (2.64443,4.72033);
\draw [c] (2.31812,4.76278) -- (2.31812,4.72033);

\draw [c] (1.99181,4.72033) -- (1.99181,8.26057);
\draw [anchor= east] (0.1,8.26057) node[ rotate=90]{Ratio};
\draw [c] (2.59181,5.31037) -- (1.99181,5.31037);
\draw [c] (2.29181,5.60539) -- (1.99181,5.60539);
\draw [c] (2.29181,5.90041) -- (1.99181,5.90041);
\draw [c] (2.29181,6.19543) -- (1.99181,6.19543);
\draw [c] (2.59181,6.49045) -- (1.99181,6.49045);
\draw [c] (2.29181,6.78547) -- (1.99181,6.78547);
\draw [c] (2.29181,7.08049) -- (1.99181,7.08049);
\draw [c] (2.29181,7.37551) -- (1.99181,7.37551);
\draw [c] (2.59181,7.67053) -- (1.99181,7.67053);
\draw [c] (2.59181,5.31037) -- (1.99181,5.31037);
\draw [c] (2.29181,5.01535) -- (1.99181,5.01535);
\draw [c] (2.29181,4.72033) -- (1.99181,4.72033);
\draw [c] (2.59181,7.67053) -- (1.99181,7.67053);
\draw [c] (2.29181,7.96555) -- (1.99181,7.96555);
\draw [anchor= east] (1.89181,5.31037) node[ rotate=0]{0.6};
\draw [anchor= east] (1.89181,6.49045) node[ rotate=0]{0.8};
\draw [anchor= east] (1.89181,7.67053) node[ rotate=0]{1.0};
\colorlet{c}{natgreen};
\draw [c] (2.15171,6.61965) -- (2.15171,6.65065);
\draw [c] (2.15171,6.65065) -- (2.15171,6.68165);
\draw [c] (1.99181,6.65065) -- (2.15171,6.65065);
\draw [c] (2.15171,6.65065) -- (2.3116,6.65065);
\draw [c] (2.47149,6.6571) -- (2.47149,6.74444);
\draw [c] (2.47149,6.74444) -- (2.47149,6.83179);
\draw [c] (2.3116,6.74444) -- (2.47149,6.74444);
\draw [c] (2.47149,6.74444) -- (2.63138,6.74444);
\draw [c] (2.79127,6.4215) -- (2.79127,6.59536);
\draw [c] (2.79127,6.59536) -- (2.79127,6.76923);
\draw [c] (2.63138,6.59536) -- (2.79127,6.59536);
\draw [c] (2.79127,6.59536) -- (2.95116,6.59536);
\draw [c] (3.11105,6.3241) -- (3.11105,6.61754);
\draw [c] (3.11105,6.61754) -- (3.11105,6.91099);
\draw [c] (2.95116,6.61754) -- (3.11105,6.61754);
\draw [c] (3.11105,6.61754) -- (3.27094,6.61754);
\draw [c] (3.43083,5.9137) -- (3.43083,6.31703);
\draw [c] (3.43083,6.31703) -- (3.43083,6.72036);
\draw [c] (3.27094,6.31703) -- (3.43083,6.31703);
\draw [c] (3.43083,6.31703) -- (3.59072,6.31703);
\draw [c] (3.75061,5.82618) -- (3.75061,6.41561);
\draw [c] (3.75061,6.41561) -- (3.75061,7.00504);
\draw [c] (3.59072,6.41561) -- (3.75061,6.41561);
\draw [c] (3.75061,6.41561) -- (3.9105,6.41561);
\draw [c] (4.0704,5.49288) -- (4.0704,6.29697);
\draw [c] (4.0704,6.29697) -- (4.0704,7.10107);
\draw [c] (3.9105,6.29697) -- (4.0704,6.29697);
\draw [c] (4.0704,6.29697) -- (4.23029,6.29697);
\draw [c] (4.39018,5.21582) -- (4.39018,6.2289);
\draw [c] (4.39018,6.2289) -- (4.39018,7.24198);
\draw [c] (4.23029,6.2289) -- (4.39018,6.2289);
\draw [c] (4.39018,6.2289) -- (4.55007,6.2289);
\draw [c] (4.70996,5.63215) -- (4.70996,7.061);
\draw [c] (4.70996,7.061) -- (4.70996,8.26057);
\draw [c] (4.55007,7.061) -- (4.70996,7.061);
\draw [c] (4.70996,7.061) -- (4.86985,7.061);
\draw [c] (5.02974,5.96613) -- (5.02974,6.77191);
\draw [c] (5.02974,6.77191) -- (5.02974,7.57769);
\draw [c] (4.86985,6.77191) -- (5.02974,6.77191);
\draw [c] (5.02974,6.77191) -- (5.18963,6.77191);
\draw [c] (5.34952,6.50286) -- (5.34952,6.56954);
\draw [c] (5.34952,6.56954) -- (5.34952,6.63622);
\draw [c] (5.18963,6.56954) -- (5.34952,6.56954);
\draw [c] (5.34952,6.56954) -- (5.50941,6.56954);
\draw [c] (5.6693,6.46716) -- (5.6693,6.54533);
\draw [c] (5.6693,6.54533) -- (5.6693,6.6235);
\draw [c] (5.50941,6.54533) -- (5.6693,6.54533);
\draw [c] (5.6693,6.54533) -- (5.8292,6.54533);
\draw [c] (5.98909,6.44262) -- (5.98909,6.53252);
\draw [c] (5.98909,6.53252) -- (5.98909,6.62241);
\draw [c] (5.8292,6.53252) -- (5.98909,6.53252);
\draw [c] (5.98909,6.53252) -- (6.14898,6.53252);
\draw [c] (6.30887,6.35753) -- (6.30887,6.45578);
\draw [c] (6.30887,6.45578) -- (6.30887,6.55402);
\draw [c] (6.14898,6.45578) -- (6.30887,6.45578);
\draw [c] (6.30887,6.45578) -- (6.46876,6.45578);
\draw [c] (6.62865,6.38613) -- (6.62865,6.49646);
\draw [c] (6.62865,6.49646) -- (6.62865,6.60679);
\draw [c] (6.46876,6.49646) -- (6.62865,6.49646);
\draw [c] (6.62865,6.49646) -- (6.78854,6.49646);
\draw [c] (6.94843,6.34544) -- (6.94843,6.46549);
\draw [c] (6.94843,6.46549) -- (6.94843,6.58555);
\draw [c] (6.78854,6.46549) -- (6.94843,6.46549);
\draw [c] (6.94843,6.46549) -- (7.10832,6.46549);
\draw [c] (7.26821,6.36493) -- (7.26821,6.49481);
\draw [c] (7.26821,6.49481) -- (7.26821,6.62469);
\draw [c] (7.10832,6.49481) -- (7.26821,6.49481);
\draw [c] (7.26821,6.49481) -- (7.4281,6.49481);
\draw [c] (7.58799,6.37173) -- (7.58799,6.50896);
\draw [c] (7.58799,6.50896) -- (7.58799,6.6462);
\draw [c] (7.4281,6.50896) -- (7.58799,6.50896);
\draw [c] (7.58799,6.50896) -- (7.74789,6.50896);
\draw [c] (7.90778,6.30802) -- (7.90778,6.45395);
\draw [c] (7.90778,6.45395) -- (7.90778,6.59988);
\draw [c] (7.74789,6.45395) -- (7.90778,6.45395);
\draw [c] (7.90778,6.45395) -- (8.06767,6.45395);
\draw [c] (8.22756,6.43327) -- (8.22756,6.5887);
\draw [c] (8.22756,6.5887) -- (8.22756,6.74413);
\draw [c] (8.06767,6.5887) -- (8.22756,6.5887);
\draw [c] (8.22756,6.5887) -- (8.38745,6.5887);
\draw [c] (8.54734,6.29264) -- (8.54734,6.45141);
\draw [c] (8.54734,6.45141) -- (8.54734,6.61019);
\draw [c] (8.38745,6.45141) -- (8.54734,6.45141);
\draw [c] (8.54734,6.45141) -- (8.70723,6.45141);
\draw [c] (8.86712,6.28444) -- (8.86712,6.44909);
\draw [c] (8.86712,6.44909) -- (8.86712,6.61374);
\draw [c] (8.70723,6.44909) -- (8.86712,6.44909);
\draw [c] (8.86712,6.44909) -- (9.02701,6.44909);
\draw [c] (9.1869,6.32404) -- (9.1869,6.49601);
\draw [c] (9.1869,6.49601) -- (9.1869,6.66798);
\draw [c] (9.02701,6.49601) -- (9.1869,6.49601);
\draw [c] (9.1869,6.49601) -- (9.34679,6.49601);
\draw [c] (9.50669,6.23896) -- (9.50669,6.42061);
\draw [c] (9.50669,6.42061) -- (9.50669,6.60226);
\draw [c] (9.34679,6.42061) -- (9.50669,6.42061);
\draw [c] (9.50669,6.42061) -- (9.66658,6.42061);
\draw [c] (9.82647,6.27164) -- (9.82647,6.45944);
\draw [c] (9.82647,6.45944) -- (9.82647,6.64724);
\draw [c] (9.66658,6.45944) -- (9.82647,6.45944);
\draw [c] (9.82647,6.45944) -- (9.98636,6.45944);
\draw [c] (10.1462,6.28628) -- (10.1462,6.48487);
\draw [c] (10.1462,6.48487) -- (10.1462,6.68346);
\draw [c] (9.98636,6.48487) -- (10.1462,6.48487);
\draw [c] (10.1462,6.48487) -- (10.3061,6.48487);
\draw [c] (10.466,6.35383) -- (10.466,6.56295);
\draw [c] (10.466,6.56295) -- (10.466,6.77206);
\draw [c] (10.3061,6.56295) -- (10.466,6.56295);
\draw [c] (10.466,6.56295) -- (10.6259,6.56295);
\draw [c] (10.7858,6.30181) -- (10.7858,6.52059);
\draw [c] (10.7858,6.52059) -- (10.7858,6.73938);
\draw [c] (10.6259,6.52059) -- (10.7858,6.52059);
\draw [c] (10.7858,6.52059) -- (10.9457,6.52059);
\draw [c] (11.1056,6.20347) -- (11.1056,6.42771);
\draw [c] (11.1056,6.42771) -- (11.1056,6.65194);
\draw [c] (10.9457,6.42771) -- (11.1056,6.42771);
\draw [c] (11.1056,6.42771) -- (11.2655,6.42771);
\draw [c] (11.4254,6.22216) -- (11.4254,6.46071);
\draw [c] (11.4254,6.46071) -- (11.4254,6.69926);
\draw [c] (11.2655,6.46071) -- (11.4254,6.46071);
\draw [c] (11.4254,6.46071) -- (11.5853,6.46071);
\draw [c] (11.7452,6.27091) -- (11.7452,6.52927);
\draw [c] (11.7452,6.52927) -- (11.7452,6.78763);
\draw [c] (11.5853,6.52927) -- (11.7452,6.52927);
\draw [c] (11.7452,6.52927) -- (11.905,6.52927);
\draw [c] (12.0649,6.20728) -- (12.0649,6.47073);
\draw [c] (12.0649,6.47073) -- (12.0649,6.73417);
\draw [c] (11.905,6.47073) -- (12.0649,6.47073);
\draw [c] (12.0649,6.47073) -- (12.2248,6.47073);
\draw [c] (12.3847,6.16563) -- (12.3847,6.44557);
\draw [c] (12.3847,6.44557) -- (12.3847,6.7255);
\draw [c] (12.2248,6.44557) -- (12.3847,6.44557);
\draw [c] (12.3847,6.44557) -- (12.5446,6.44557);
\draw [c] (12.7045,6.3742) -- (12.7045,6.68187);
\draw [c] (12.7045,6.68187) -- (12.7045,6.98955);
\draw [c] (12.5446,6.68187) -- (12.7045,6.68187);
\draw [c] (12.7045,6.68187) -- (12.8644,6.68187);
\draw [c] (13.0243,6.28273) -- (13.0243,6.62692);
\draw [c] (13.0243,6.62692) -- (13.0243,6.97112);
\draw [c] (12.8644,6.62692) -- (13.0243,6.62692);
\draw [c] (13.0243,6.62692) -- (13.1842,6.62692);
\draw [c] (13.3441,6.35345) -- (13.3441,6.71605);
\draw [c] (13.3441,6.71605) -- (13.3441,7.07866);
\draw [c] (13.1842,6.71605) -- (13.3441,6.71605);
\draw [c] (13.3441,6.71605) -- (13.504,6.71605);
\draw [c] (13.6638,6.12128) -- (13.6638,6.52014);
\draw [c] (13.6638,6.52014) -- (13.6638,6.91899);
\draw [c] (13.504,6.52014) -- (13.6638,6.52014);
\draw [c] (13.6638,6.52014) -- (13.8237,6.52014);
\draw [c] (13.9836,6.11869) -- (13.9836,6.53979);
\draw [c] (13.9836,6.53979) -- (13.9836,6.96089);
\draw [c] (13.8237,6.53979) -- (13.9836,6.53979);
\draw [c] (13.9836,6.53979) -- (14.1435,6.53979);
\draw [c] (14.3034,6.16227) -- (14.3034,6.65137);
\draw [c] (14.3034,6.65137) -- (14.3034,7.14047);
\draw [c] (14.1435,6.65137) -- (14.3034,6.65137);
\draw [c] (14.3034,6.65137) -- (14.4633,6.65137);
\draw [c] (14.6232,5.70926) -- (14.6232,6.21164);
\draw [c] (14.6232,6.21164) -- (14.6232,6.71402);
\draw [c] (14.4633,6.21164) -- (14.6232,6.21164);
\draw [c] (14.6232,6.21164) -- (14.7831,6.21164);
\draw [c] (14.943,5.95607) -- (14.943,6.52014);
\draw [c] (14.943,6.52014) -- (14.943,7.08421);
\draw [c] (14.7831,6.52014) -- (14.943,6.52014);
\draw [c] (14.943,6.52014) -- (15.1029,6.52014);
\draw [c] (15.2628,5.98102) -- (15.2628,6.6058);
\draw [c] (15.2628,6.6058) -- (15.2628,7.23057);
\draw [c] (15.1029,6.6058) -- (15.2628,6.6058);
\draw [c] (15.2628,6.6058) -- (15.4226,6.6058);
\draw [c] (15.5825,5.84972) -- (15.5825,6.54664);
\draw [c] (15.5825,6.54664) -- (15.5825,7.24357);
\draw [c] (15.4226,6.54664) -- (15.5825,6.54664);
\draw [c] (15.5825,6.54664) -- (15.7424,6.54664);
\draw [c] (15.9023,5.56375) -- (15.9023,6.22326);
\draw [c] (15.9023,6.22326) -- (15.9023,6.88278);
\draw [c] (15.7424,6.22326) -- (15.9023,6.22326);
\draw [c] (15.9023,6.22326) -- (16.0622,6.22326);
\draw [c] (16.2221,5.83281) -- (16.2221,6.82762);
\draw [c] (16.2221,6.82762) -- (16.2221,7.82242);
\draw [c] (16.0622,6.82762) -- (16.2221,6.82762);
\draw [c] (16.2221,6.82762) -- (16.382,6.82762);
\draw [c] (16.5419,5.96228) -- (16.5419,7.13413);
\draw [c] (16.5419,7.13413) -- (16.5419,8.26057);
\draw [c] (16.382,7.13413) -- (16.5419,7.13413);
\draw [c] (16.5419,7.13413) -- (16.7018,7.13413);
\draw [c] (16.8617,5.2478) -- (16.8617,6.39477);
\draw [c] (16.8617,6.39477) -- (16.8617,7.54173);
\draw [c] (16.7018,6.39477) -- (16.8617,6.39477);
\draw [c] (16.8617,6.39477) -- (17.0216,6.39477);
\draw [c] (17.1814,5.37018) -- (17.1814,6.7964);
\draw [c] (17.1814,6.7964) -- (17.1814,8.22261);
\draw [c] (17.0216,6.7964) -- (17.1814,6.7964);
\draw [c] (17.1814,6.7964) -- (17.3413,6.7964);
\draw [c] (17.5012,4.72033) -- (17.5012,5.45788);
\draw [c] (17.5012,5.45788) -- (17.5012,6.94446);
\draw [c] (17.3413,5.45788) -- (17.5012,5.45788);
\draw [c] (17.5012,5.45788) -- (17.6611,5.45788);
\draw [c] (17.821,4.72033) -- (17.821,7.08049);
\draw [c] (17.821,7.08049) -- (17.821,8.26057);
\draw [c] (17.6611,7.08049) -- (17.821,7.08049);
\draw [c] (17.821,7.08049) -- (17.9809,7.08049);
\definecolor{c}{rgb}{0,0,0};
\draw [c] (1.99181,4.72033) -- (1.99181,8.26057) -- (17.9809,8.26057) -- (17.9809,4.72033) -- (1.99181,4.72033);
%\definecolor{c}{rgb}{1,1,1};
%\draw [color=c, fill=c] (0,0) rectangle (20,4.72033);
%\draw [color=c, fill=c] (1.99181,0.954979) rectangle (17.9809,4.66576);
\definecolor{c}{rgb}{0,0,0};
\draw [c] (1.99181,0.954979) -- (1.99181,4.72033) -- (17.9809,4.72033) -- (17.9809,0.954979) -- (1.99181,0.954979);
%\definecolor{c}{rgb}{1,1,1};
%\draw [color=c, fill=c] (1.99181,0.954979) rectangle (17.9809,4.66576);
\colorlet{c}{kugray!20};
\draw [color=c, fill=c] (1.99181,4.01035) rectangle (2.3116,4.08424);
\draw [color=c, fill=c] (2.3116,3.94182) rectangle (2.63138,4.15276);
\draw [color=c, fill=c] (2.63138,3.83958) rectangle (2.95116,4.25501);
\draw [color=c, fill=c] (2.95116,3.70048) rectangle (3.27094,4.39411);
\draw [color=c, fill=c] (3.27094,3.51599) rectangle (3.59072,4.5786);
\draw [color=c, fill=c] (3.59072,3.32095) rectangle (3.9105,4.66576);
\draw [color=c, fill=c] (3.9105,2.954) rectangle (4.23029,4.66576);
\draw [color=c, fill=c] (4.23029,2.5473) rectangle (4.55007,4.66576);
\draw [color=c, fill=c] (4.55007,2.09154) rectangle (4.86985,4.66576);
\draw [color=c, fill=c] (4.86985,2.72952) rectangle (5.18963,4.66576);
\draw [color=c, fill=c] (5.18963,3.97894) rectangle (5.50941,4.11565);
\draw [color=c, fill=c] (5.50941,3.9616) rectangle (5.8292,4.13299);
\draw [color=c, fill=c] (5.8292,3.93855) rectangle (6.14898,4.15604);
\draw [color=c, fill=c] (6.14898,3.91327) rectangle (6.46876,4.18132);
\draw [color=c, fill=c] (6.46876,3.88294) rectangle (6.78854,4.21165);
\draw [color=c, fill=c] (6.78854,3.84862) rectangle (7.10832,4.24597);
\draw [color=c, fill=c] (7.10832,3.80674) rectangle (7.4281,4.28785);
\draw [color=c, fill=c] (7.4281,3.76277) rectangle (7.74789,4.33181);
\draw [color=c, fill=c] (7.74789,3.69744) rectangle (8.06767,4.39715);
\draw [color=c, fill=c] (8.06767,3.59136) rectangle (8.38745,4.50323);
\draw [color=c, fill=c] (8.38745,3.48035) rectangle (8.70723,4.61424);
\draw [color=c, fill=c] (8.70723,3.31072) rectangle (9.02701,4.66576);
\draw [color=c, fill=c] (9.02701,3.2681) rectangle (9.34679,4.66576);
\draw [color=c, fill=c] (9.34679,3.09299) rectangle (9.66658,4.66576);
\draw [color=c, fill=c] (9.66658,2.81037) rectangle (9.98636,4.66576);
\draw [color=c, fill=c] (9.98636,2.62844) rectangle (10.3061,4.66576);
\draw [color=c, fill=c] (10.3061,2.22355) rectangle (10.6259,4.66576);
\draw [color=c, fill=c] (10.6259,2.18256) rectangle (10.9457,4.66576);
\draw [color=c, fill=c] (10.9457,1.62148) rectangle (11.2655,4.66576);
\draw [color=c, fill=c] (11.2655,1.52243) rectangle (11.5853,4.66576);
\draw [color=c, fill=c] (11.5853,1.41016) rectangle (11.905,4.66576);
\draw [color=c, fill=c] (11.905,0.954979) rectangle (12.2248,4.66576);
\draw [color=c, fill=c] (12.2248,1.41016) rectangle (12.5446,4.66576);
\draw [color=c, fill=c] (12.5446,0.954979) rectangle (12.8644,4.66576);
\draw [color=c, fill=c] (12.8644,0.954979) rectangle (13.1842,4.66576);
\draw [color=c, fill=c] (13.1842,0.954979) rectangle (13.504,4.66576);
\draw [color=c, fill=c] (13.504,0.954979) rectangle (13.8237,4.66576);
\draw [color=c, fill=c] (13.8237,0.954979) rectangle (14.1435,4.66576);
\draw [color=c, fill=c] (14.1435,0.954979) rectangle (14.4633,4.66576);
\draw [color=c, fill=c] (14.4633,0.954979) rectangle (14.7831,4.66576);
\draw [color=c, fill=c] (14.7831,4.04729) rectangle (15.1029,4.04729);
\draw [color=c, fill=c] (15.1029,0.954979) rectangle (15.4226,4.66576);
\draw [color=c, fill=c] (15.4226,4.04729) rectangle (15.7424,4.04729);
\draw [color=c, fill=c] (15.7424,4.04729) rectangle (16.0622,4.04729);
\draw [color=c, fill=c] (16.0622,4.04729) rectangle (16.382,4.04729);
\draw [color=c, fill=c] (16.382,4.04729) rectangle (16.7018,4.04729);
\draw [color=c, fill=c] (16.7018,4.04729) rectangle (17.0216,4.04729);
\draw [color=c, fill=c] (17.0216,4.04729) rectangle (17.3413,4.04729);
\draw [color=c, fill=c] (17.3413,4.04729) rectangle (17.6611,4.04729);
\draw [color=c, fill=c] (17.6611,4.04729) rectangle (17.9809,4.04729);
\definecolor{c}{rgb}{0,0,0};
\draw [c] (1.99181,0.954979) -- (17.9809,0.954979);
\draw [anchor= east] (17.9809,0.090641) node[ rotate=0]{$M_{\gamma\gamma}$ [GeV]};
\draw [c] (3.29705,1.06819) -- (3.29705,0.954979);
\draw [c] (3.62335,1.01158) -- (3.62335,0.954979);
\draw [c] (3.94966,1.01158) -- (3.94966,0.954979);
\draw [c] (4.27597,1.01158) -- (4.27597,0.954979);
\draw [c] (4.60228,1.01158) -- (4.60228,0.954979);
\draw [c] (4.92859,1.06819) -- (4.92859,0.954979);
\draw [c] (5.25489,1.01158) -- (5.25489,0.954979);
\draw [c] (5.5812,1.01158) -- (5.5812,0.954979);
\draw [c] (5.90751,1.01158) -- (5.90751,0.954979);
\draw [c] (6.23382,1.01158) -- (6.23382,0.954979);
\draw [c] (6.56012,1.06819) -- (6.56012,0.954979);
\draw [c] (6.88643,1.01158) -- (6.88643,0.954979);
\draw [c] (7.21274,1.01158) -- (7.21274,0.954979);
\draw [c] (7.53905,1.01158) -- (7.53905,0.954979);
\draw [c] (7.86536,1.01158) -- (7.86536,0.954979);
\draw [c] (8.19166,1.06819) -- (8.19166,0.954979);
\draw [c] (8.51797,1.01158) -- (8.51797,0.954979);
\draw [c] (8.84428,1.01158) -- (8.84428,0.954979);
\draw [c] (9.17059,1.01158) -- (9.17059,0.954979);
\draw [c] (9.4969,1.01158) -- (9.4969,0.954979);
\draw [c] (9.8232,1.06819) -- (9.8232,0.954979);
\draw [c] (10.1495,1.01158) -- (10.1495,0.954979);
\draw [c] (10.4758,1.01158) -- (10.4758,0.954979);
\draw [c] (10.8021,1.01158) -- (10.8021,0.954979);
\draw [c] (11.1284,1.01158) -- (11.1284,0.954979);
\draw [c] (11.4547,1.06819) -- (11.4547,0.954979);
\draw [c] (11.7811,1.01158) -- (11.7811,0.954979);
\draw [c] (12.1074,1.01158) -- (12.1074,0.954979);
\draw [c] (12.4337,1.01158) -- (12.4337,0.954979);
\draw [c] (12.76,1.01158) -- (12.76,0.954979);
\draw [c] (13.0863,1.06819) -- (13.0863,0.954979);
\draw [c] (13.4126,1.01158) -- (13.4126,0.954979);
\draw [c] (13.7389,1.01158) -- (13.7389,0.954979);
\draw [c] (14.0652,1.01158) -- (14.0652,0.954979);
\draw [c] (14.3915,1.01158) -- (14.3915,0.954979);
\draw [c] (14.7178,1.06819) -- (14.7178,0.954979);
\draw [c] (15.0441,1.01158) -- (15.0441,0.954979);
\draw [c] (15.3704,1.01158) -- (15.3704,0.954979);
\draw [c] (15.6967,1.01158) -- (15.6967,0.954979);
\draw [c] (16.0231,1.01158) -- (16.0231,0.954979);
\draw [c] (16.3494,1.06819) -- (16.3494,0.954979);
\draw [c] (16.6757,1.01158) -- (16.6757,0.954979);
\draw [c] (17.002,1.01158) -- (17.002,0.954979);
\draw [c] (17.3283,1.01158) -- (17.3283,0.954979);
\draw [c] (17.6546,1.01158) -- (17.6546,0.954979);
\draw [c] (17.9809,1.06819) -- (17.9809,0.954979);
\draw [c] (3.29705,1.06819) -- (3.29705,0.954979);
\draw [c] (2.97074,1.01158) -- (2.97074,0.954979);
\draw [c] (2.64443,1.01158) -- (2.64443,0.954979);
\draw [c] (2.31812,1.01158) -- (2.31812,0.954979);
\draw [anchor=base] (3.29705,0.499209) node[ rotate=0]{500};
\draw [anchor=base] (4.92859,0.499209) node[ rotate=0]{1000};
\draw [anchor=base] (6.56012,0.499209) node[ rotate=0]{1500};
\draw [anchor=base] (8.19166,0.499209) node[ rotate=0]{2000};
\draw [anchor=base] (9.8232, 0.499209) node[ rotate=0]{2500};
\draw [anchor=base] (11.4547,0.499209) node[ rotate=0]{3000};
\draw [anchor=base] (13.0863,0.499209) node[ rotate=0]{3500};
\draw [anchor=base] (14.7178,0.499209) node[ rotate=0]{4000};
\draw [anchor=base] (16.3494,0.499209) node[ rotate=0]{4500};
\draw [anchor=base] (17.9809,0.499209) node[ rotate=0]{5000};
\draw [c] (1.99181,0.954979) -- (1.99181,4.66576);
\draw [anchor= east] (0.1,4.66576) node[ rotate=90]{Ratio};
\draw [c] (2.46349,1.57344) -- (1.99181,1.57344);
\draw [c] (2.22765,1.88267) -- (1.99181,1.88267);
\draw [c] (2.22765,2.19191) -- (1.99181,2.19191);
\draw [c] (2.22765,2.50114) -- (1.99181,2.50114);
\draw [c] (2.46349,2.81037) -- (1.99181,2.81037);
\draw [c] (2.22765,3.1196) -- (1.99181,3.1196);
\draw [c] (2.22765,3.42883) -- (1.99181,3.42883);
\draw [c] (2.22765,3.73806) -- (1.99181,3.73806);
\draw [c] (2.46349,4.04729) -- (1.99181,4.04729);
\draw [c] (2.46349,1.57344) -- (1.99181,1.57344);
\draw [c] (2.22765,1.26421) -- (1.99181,1.26421);
\draw [c] (2.22765,0.954979) -- (1.99181,0.954979);
\draw [c] (2.46349,4.04729) -- (1.99181,4.04729);
\draw [c] (2.22765,4.35653) -- (1.99181,4.35653);
\draw [anchor= east] (1.89181,1.57344) node[ rotate=0]{0.6};
\draw [anchor= east] (1.89181,2.81037) node[ rotate=0]{0.8};
\draw [anchor= east] (1.89181,4.04729) node[ rotate=0]{1.0};
\colorlet{c}{kugray};
\draw [c] (2.15171,2.97466) -- (2.15171,3.00691);
\draw [c] (2.15171,3.00691) -- (2.15171,3.03915);
\draw [c] (1.99181,3.00691) -- (2.15171,3.00691);
\draw [c] (2.15171,3.00691) -- (2.3116,3.00691);
\draw [c] (2.47149,2.98576) -- (2.47149,3.07875);
\draw [c] (2.47149,3.07875) -- (2.47149,3.17174);
\draw [c] (2.3116,3.07875) -- (2.47149,3.07875);
\draw [c] (2.47149,3.07875) -- (2.63138,3.07875);
\draw [c] (2.79127,2.6804) -- (2.79127,2.85787);
\draw [c] (2.79127,2.85787) -- (2.79127,3.03535);
\draw [c] (2.63138,2.85787) -- (2.79127,2.85787);
\draw [c] (2.79127,2.85787) -- (2.95116,2.85787);
\draw [c] (3.11105,2.5925) -- (3.11105,2.89021);
\draw [c] (3.11105,2.89021) -- (3.11105,3.18792);
\draw [c] (2.95116,2.89021) -- (3.11105,2.89021);
\draw [c] (3.11105,2.89021) -- (3.27094,2.89021);
\draw [c] (3.43083,2.74779) -- (3.43083,3.22584);
\draw [c] (3.43083,3.22584) -- (3.43083,3.70389);
\draw [c] (3.27094,3.22584) -- (3.43083,3.22584);
\draw [c] (3.43083,3.22584) -- (3.59072,3.22584);
\draw [c] (3.75061,2.34951) -- (3.75061,2.98117);
\draw [c] (3.75061,2.98117) -- (3.75061,3.61283);
\draw [c] (3.59072,2.98117) -- (3.75061,2.98117);
\draw [c] (3.75061,2.98117) -- (3.9105,2.98117);
\draw [c] (4.0704,2.11685) -- (4.0704,3.08109);
\draw [c] (4.0704,3.08109) -- (4.0704,4.04533);
\draw [c] (3.9105,3.08109) -- (4.0704,3.08109);
\draw [c] (4.0704,3.08109) -- (4.23029,3.08109);
\draw [c] (4.39018,1.65635) -- (4.39018,2.95616);
\draw [c] (4.39018,2.95616) -- (4.39018,4.25597);
\draw [c] (4.23029,2.95616) -- (4.39018,2.95616);
\draw [c] (4.39018,2.95616) -- (4.55007,2.95616);
\draw [c] (4.70996,0.954979) -- (4.70996,1.88283);
\draw [c] (4.70996,1.88283) -- (4.70996,3.31503);
\draw [c] (4.55007,1.88283) -- (4.70996,1.88283);
\draw [c] (4.70996,1.88283) -- (4.86985,1.88283);
\draw [c] (5.02974,1.77633) -- (5.02974,2.91812);
\draw [c] (5.02974,2.91812) -- (5.02974,4.05992);
\draw [c] (4.86985,2.91812) -- (5.02974,2.91812);
\draw [c] (5.02974,2.91812) -- (5.18963,2.91812);
\draw [c] (5.34952,2.82443) -- (5.34952,2.88305);
\draw [c] (5.34952,2.88305) -- (5.34952,2.94167);
\draw [c] (5.18963,2.88305) -- (5.34952,2.88305);
\draw [c] (5.34952,2.88305) -- (5.50941,2.88305);
\draw [c] (5.6693,2.84303) -- (5.6693,2.91688);
\draw [c] (5.6693,2.91688) -- (5.6693,2.99072);
\draw [c] (5.50941,2.91688) -- (5.6693,2.91688);
\draw [c] (5.6693,2.91688) -- (5.8292,2.91688);
\draw [c] (5.98909,2.80185) -- (5.98909,2.89526);
\draw [c] (5.98909,2.89526) -- (5.98909,2.98868);
\draw [c] (5.8292,2.89526) -- (5.98909,2.89526);
\draw [c] (5.98909,2.89526) -- (6.14898,2.89526);
\draw [c] (6.30887,2.78633) -- (6.30887,2.90156);
\draw [c] (6.30887,2.90156) -- (6.30887,3.0168);
\draw [c] (6.14898,2.90156) -- (6.30887,2.90156);
\draw [c] (6.30887,2.90156) -- (6.46876,2.90156);
\draw [c] (6.62865,2.6739) -- (6.62865,2.81343);
\draw [c] (6.62865,2.81343) -- (6.62865,2.95295);
\draw [c] (6.46876,2.81343) -- (6.62865,2.81343);
\draw [c] (6.62865,2.81343) -- (6.78854,2.81343);
\draw [c] (6.94843,2.67478) -- (6.94843,2.8442);
\draw [c] (6.94843,2.8442) -- (6.94843,3.01361);
\draw [c] (6.78854,2.8442) -- (6.94843,2.8442);
\draw [c] (6.94843,2.8442) -- (7.10832,2.8442);
\draw [c] (7.26821,2.57629) -- (7.26821,2.77949);
\draw [c] (7.26821,2.77949) -- (7.26821,2.98269);
\draw [c] (7.10832,2.77949) -- (7.26821,2.77949);
\draw [c] (7.26821,2.77949) -- (7.4281,2.77949);
\draw [c] (7.58799,2.73943) -- (7.58799,2.98707);
\draw [c] (7.58799,2.98707) -- (7.58799,3.23471);
\draw [c] (7.4281,2.98707) -- (7.58799,2.98707);
\draw [c] (7.58799,2.98707) -- (7.74789,2.98707);
\draw [c] (7.90778,2.57977) -- (7.90778,2.87964);
\draw [c] (7.90778,2.87964) -- (7.90778,3.1795);
\draw [c] (7.74789,2.87964) -- (7.90778,2.87964);
\draw [c] (7.90778,2.87964) -- (8.06767,2.87964);
\draw [c] (8.22756,2.48058) -- (8.22756,2.87087);
\draw [c] (8.22756,2.87087) -- (8.22756,3.26116);
\draw [c] (8.06767,2.87087) -- (8.22756,2.87087);
\draw [c] (8.22756,2.87087) -- (8.38745,2.87087);
\draw [c] (8.54734,2.15054) -- (8.54734,2.61807);
\draw [c] (8.54734,2.61807) -- (8.54734,3.08561);
\draw [c] (8.38745,2.61807) -- (8.54734,2.61807);
\draw [c] (8.54734,2.61807) -- (8.70723,2.61807);
\draw [c] (8.86712,2.31293) -- (8.86712,2.95073);
\draw [c] (8.86712,2.95073) -- (8.86712,3.58852);
\draw [c] (8.70723,2.95073) -- (8.86712,2.95073);
\draw [c] (8.86712,2.95073) -- (9.02701,2.95073);
\draw [c] (9.1869,2.37987) -- (9.1869,3.06561);
\draw [c] (9.1869,3.06561) -- (9.1869,3.75134);
\draw [c] (9.02701,3.06561) -- (9.1869,3.06561);
\draw [c] (9.1869,3.06561) -- (9.34679,3.06561);
\draw [c] (9.50669,2.18245) -- (9.50669,3.01652);
\draw [c] (9.50669,3.01652) -- (9.50669,3.8506);
\draw [c] (9.34679,3.01652) -- (9.50669,3.01652);
\draw [c] (9.50669,3.01652) -- (9.66658,3.01652);
\draw [c] (9.82647,1.65609) -- (9.82647,2.68668);
\draw [c] (9.82647,2.68668) -- (9.82647,3.71727);
\draw [c] (9.66658,2.68668) -- (9.82647,2.68668);
\draw [c] (9.82647,2.68668) -- (9.98636,2.68668);
\draw [c] (10.1462,1.55278) -- (10.1462,2.74527);
\draw [c] (10.1462,2.74527) -- (10.1462,3.93775);
\draw [c] (9.98636,2.74527) -- (10.1462,2.74527);
\draw [c] (10.1462,2.74527) -- (10.3061,2.74527);
\draw [c] (10.466,0.971815) -- (10.466,2.43391);
\draw [c] (10.466,2.43391) -- (10.466,3.89601);
\draw [c] (10.3061,2.43391) -- (10.466,2.43391);
\draw [c] (10.466,2.43391) -- (10.6259,2.43391);
\draw [c] (10.7858,1.09845) -- (10.7858,2.6417);
\draw [c] (10.7858,2.6417) -- (10.7858,4.18495);
\draw [c] (10.6259,2.6417) -- (10.7858,2.6417);
\draw [c] (10.7858,2.6417) -- (10.9457,2.6417);
\draw [c] (11.1056,0.954979) -- (11.1056,3.09581);
\draw [c] (11.1056,3.09581) -- (11.1056,4.66576);
\draw [c] (10.9457,3.09581) -- (11.1056,3.09581);
\draw [c] (11.1056,3.09581) -- (11.2655,3.09581);
\draw [c] (11.4254,0.954979) -- (11.4254,3.01652);
\draw [c] (11.4254,3.01652) -- (11.4254,4.66576);
\draw [c] (11.2655,3.01652) -- (11.4254,3.01652);
\draw [c] (11.4254,3.01652) -- (11.5853,3.01652);
\draw [c] (11.7452,0.954979) -- (11.7452,2.92282);
\draw [c] (11.7452,2.92282) -- (11.7452,4.66576);
\draw [c] (11.5853,2.92282) -- (11.7452,2.92282);
\draw [c] (11.7452,2.92282) -- (11.905,2.92282);
\draw [c] (12.0649,0.954979) -- (12.0649,3.633);
\draw [c] (12.3847,1.02846) -- (12.3847,3.48505);
\draw [c] (12.3847,3.48505) -- (12.3847,4.66576);
\draw [c] (12.2248,3.48505) -- (12.3847,3.48505);
\draw [c] (12.3847,3.48505) -- (12.5446,3.48505);
\draw [c] (13.0243,0.954979) -- (13.0243,4.04729);
\draw [c] (13.0243,4.04729) -- (13.0243,4.66576);
\draw [c] (12.8644,4.04729) -- (13.0243,4.04729);
\draw [c] (13.0243,4.04729) -- (13.1842,4.04729);
\draw [c] (13.3441,0.954979) -- (13.3441,4.04729);
\draw [c] (13.3441,4.04729) -- (13.3441,4.66576);
\draw [c] (13.1842,4.04729) -- (13.3441,4.04729);
\draw [c] (13.3441,4.04729) -- (13.504,4.04729);
\draw [c] (13.6638,0.954979) -- (13.6638,4.66576);
\definecolor{c}{rgb}{0,0,0};
\draw [c] (1.99181,0.954979) -- (1.99181,4.72033) -- (17.9809,4.72033) -- (17.9809,0.954979) -- (1.99181,0.954979);

\end{tikzpicture}

\end{infilsf}}
\end{minipage}\hfill
%\begin{minipage}[b]{.\textwidth}
\caption{The distribution of invariant masses in the event samples generated by CalcHEP at three values of $\Lambda$, and the same event samples after extending them in surrounding processes with pythia, with ratio plots. Once again, the effect of the use of stratified sampling is visible as a jump in the magnitude of the errors around 1000 GeV.
\label{pythify}}
%\end{minipage}
\end{figure}

Here, the ratio plots make it clear that the effect upon this particular observable of overlaying parton level effects is simply to remove a fraction of the events. This is reasonable, since the effect of adding final state radiation to a hard event is to alter the final state particle content for a fraction of those events, and since we require two final state photons to calculate an invariant mass, final states that do not contain two photons are discarded in the \textsc{rivet} analysis step. We can conclude from this figure that the selection of events with altered final state particle content does not depend on the invariant mass of the photons in the event, nor does the process of overlaying parton level effects alter the distribution of photon invariant masses.

This constant difference in distributions between the process level and parton level event samples does not apply for all observables. Figure~\ref{pythicos} illustrates how the distribution of cos $\theta_{\gamma_1}$, the scattering angle for the leading photon is skewed slightly toward lower values, meaning a greater proportion of large scattering angles. This could be an effect of extended events having their kinematics altered so that the leading photon in the extended event is not the same as the leading photon in the hard event, so that the subleading photon, which scatter in the opposite direction from the leading photon, spill into the leading photon sample. This would also explain why the shape of the invariant mass distribution, which is not sensitive to the identification of a leading photon, is not affected.

\begin{figure}[htp]
\begin{minipage}[b]{.65\textwidth}
\begin{infilsf} \tiny \makebox[0pt][l]{
\hspace{-1em}\pgfdeclareplotmark{cross} {
\pgfpathmoveto{\pgfpoint{-0.3\pgfplotmarksize}{\pgfplotmarksize}}
\pgfpathlineto{\pgfpoint{+0.3\pgfplotmarksize}{\pgfplotmarksize}}
\pgfpathlineto{\pgfpoint{+0.3\pgfplotmarksize}{0.3\pgfplotmarksize}}
\pgfpathlineto{\pgfpoint{+1\pgfplotmarksize}{0.3\pgfplotmarksize}}
\pgfpathlineto{\pgfpoint{+1\pgfplotmarksize}{-0.3\pgfplotmarksize}}
\pgfpathlineto{\pgfpoint{+0.3\pgfplotmarksize}{-0.3\pgfplotmarksize}}
\pgfpathlineto{\pgfpoint{+0.3\pgfplotmarksize}{-1.\pgfplotmarksize}}
\pgfpathlineto{\pgfpoint{-0.3\pgfplotmarksize}{-1.\pgfplotmarksize}}
\pgfpathlineto{\pgfpoint{-0.3\pgfplotmarksize}{-0.3\pgfplotmarksize}}
\pgfpathlineto{\pgfpoint{-1.\pgfplotmarksize}{-0.3\pgfplotmarksize}}
\pgfpathlineto{\pgfpoint{-1.\pgfplotmarksize}{0.3\pgfplotmarksize}}
\pgfpathlineto{\pgfpoint{-0.3\pgfplotmarksize}{0.3\pgfplotmarksize}}
\pgfpathclose
\pgfusepathqstroke
}
\pgfdeclareplotmark{cross*} {
\pgfpathmoveto{\pgfpoint{-0.3\pgfplotmarksize}{\pgfplotmarksize}}
\pgfpathlineto{\pgfpoint{+0.3\pgfplotmarksize}{\pgfplotmarksize}}
\pgfpathlineto{\pgfpoint{+0.3\pgfplotmarksize}{0.3\pgfplotmarksize}}
\pgfpathlineto{\pgfpoint{+1\pgfplotmarksize}{0.3\pgfplotmarksize}}
\pgfpathlineto{\pgfpoint{+1\pgfplotmarksize}{-0.3\pgfplotmarksize}}
\pgfpathlineto{\pgfpoint{+0.3\pgfplotmarksize}{-0.3\pgfplotmarksize}}
\pgfpathlineto{\pgfpoint{+0.3\pgfplotmarksize}{-1.\pgfplotmarksize}}
\pgfpathlineto{\pgfpoint{-0.3\pgfplotmarksize}{-1.\pgfplotmarksize}}
\pgfpathlineto{\pgfpoint{-0.3\pgfplotmarksize}{-0.3\pgfplotmarksize}}
\pgfpathlineto{\pgfpoint{-1.\pgfplotmarksize}{-0.3\pgfplotmarksize}}
\pgfpathlineto{\pgfpoint{-1.\pgfplotmarksize}{0.3\pgfplotmarksize}}
\pgfpathlineto{\pgfpoint{-0.3\pgfplotmarksize}{0.3\pgfplotmarksize}}
\pgfpathclose
\pgfusepathqfillstroke
}
\pgfdeclareplotmark{newstar} {
\pgfpathmoveto{\pgfqpoint{0pt}{\pgfplotmarksize}}
\pgfpathlineto{\pgfqpointpolar{44}{0.5\pgfplotmarksize}}
\pgfpathlineto{\pgfqpointpolar{18}{\pgfplotmarksize}}
\pgfpathlineto{\pgfqpointpolar{-20}{0.5\pgfplotmarksize}}
\pgfpathlineto{\pgfqpointpolar{-54}{\pgfplotmarksize}}
\pgfpathlineto{\pgfqpointpolar{-90}{0.5\pgfplotmarksize}}
\pgfpathlineto{\pgfqpointpolar{234}{\pgfplotmarksize}}
\pgfpathlineto{\pgfqpointpolar{198}{0.5\pgfplotmarksize}}
\pgfpathlineto{\pgfqpointpolar{162}{\pgfplotmarksize}}
\pgfpathlineto{\pgfqpointpolar{134}{0.5\pgfplotmarksize}}
\pgfpathclose
\pgfusepathqstroke
}
\pgfdeclareplotmark{newstar*} {
\pgfpathmoveto{\pgfqpoint{0pt}{\pgfplotmarksize}}
\pgfpathlineto{\pgfqpointpolar{44}{0.5\pgfplotmarksize}}
\pgfpathlineto{\pgfqpointpolar{18}{\pgfplotmarksize}}
\pgfpathlineto{\pgfqpointpolar{-20}{0.5\pgfplotmarksize}}
\pgfpathlineto{\pgfqpointpolar{-54}{\pgfplotmarksize}}
\pgfpathlineto{\pgfqpointpolar{-90}{0.5\pgfplotmarksize}}
\pgfpathlineto{\pgfqpointpolar{234}{\pgfplotmarksize}}
\pgfpathlineto{\pgfqpointpolar{198}{0.5\pgfplotmarksize}}
\pgfpathlineto{\pgfqpointpolar{162}{\pgfplotmarksize}}
\pgfpathlineto{\pgfqpointpolar{134}{0.5\pgfplotmarksize}}
\pgfpathclose
\pgfusepathqfillstroke
}
\begin{tikzpicture}[x=.05\textwidth,y=.05\textwidth]
\definecolor{c}{rgb}{1,1,1};
\draw [color=c, fill=c] (0,0) rectangle (20,13.5632);
\draw [color=c, fill=c] (0,4.74713) rectangle (20,13.5632);
\draw [color=c, fill=c] (2,4.74713) rectangle (19.8,13.4751);
\definecolor{c}{rgb}{0,0,0};
\draw [c] (2,4.74713) -- (2,13.4751) -- (19.8,13.4751) -- (19.8,4.74713) -- (2,4.74713);
\definecolor{c}{rgb}{1,1,1};
\draw [color=c, fill=c] (2,4.74713) rectangle (19.8,13.4751);
\definecolor{c}{rgb}{0,0,0};
\draw [c] (2,4.74713) -- (2,13.4751) -- (19.8,13.4751) -- (19.8,4.74713) -- (2,4.74713);
\colorlet{c}{natcomp};
\draw [c] (5.382,4.74713) -- (5.382,5.66961);
\draw [c] (5.382,5.66961) -- (5.382,6.31742);
\draw [c] (5.204,5.66961) -- (5.382,5.66961);
\draw [c] (5.382,5.66961) -- (5.56,5.66961);
\draw [c] (6.45,4.74713) -- (6.45,5.83932);
\draw [c] (6.45,5.83932) -- (6.45,6.40706);
\draw [c] (6.272,5.83932) -- (6.45,5.83932);
\draw [c] (6.45,5.83932) -- (6.628,5.83932);
\draw [c] (6.806,5.5407) -- (6.806,6.44299);
\draw [c] (6.806,6.44299) -- (6.806,6.8934);
\draw [c] (6.628,6.44299) -- (6.806,6.44299);
\draw [c] (6.806,6.44299) -- (6.984,6.44299);
\draw [c] (7.162,5.51686) -- (7.162,6.43392);
\draw [c] (7.162,6.43392) -- (7.162,6.88775);
\draw [c] (6.984,6.43392) -- (7.162,6.43392);
\draw [c] (7.162,6.43392) -- (7.34,6.43392);
\draw [c] (7.518,6.78948) -- (7.518,7.27183);
\draw [c] (7.518,7.27183) -- (7.518,7.58841);
\draw [c] (7.34,7.27183) -- (7.518,7.27183);
\draw [c] (7.518,7.27183) -- (7.696,7.27183);
\draw [c] (7.874,6.29349) -- (7.874,6.88879);
\draw [c] (7.874,6.88879) -- (7.874,7.24955);
\draw [c] (7.696,6.88879) -- (7.874,6.88879);
\draw [c] (7.874,6.88879) -- (8.052,6.88879);
\draw [c] (8.23,7.50468) -- (8.23,7.83351);
\draw [c] (8.23,7.83351) -- (8.23,8.07627);
\draw [c] (8.052,7.83351) -- (8.23,7.83351);
\draw [c] (8.23,7.83351) -- (8.408,7.83351);
\draw [c] (8.586,7.89404) -- (8.586,8.16341);
\draw [c] (8.586,8.16341) -- (8.586,8.37227);
\draw [c] (8.408,8.16341) -- (8.586,8.16341);
\draw [c] (8.586,8.16341) -- (8.764,8.16341);
\draw [c] (8.942,8.20572) -- (8.942,8.43185);
\draw [c] (8.942,8.43185) -- (8.942,8.61379);
\draw [c] (8.764,8.43185) -- (8.942,8.43185);
\draw [c] (8.942,8.43185) -- (9.12,8.43185);
\draw [c] (9.298,8.72008) -- (9.298,8.8925);
\draw [c] (9.298,8.8925) -- (9.298,9.03801);
\draw [c] (9.12,8.8925) -- (9.298,8.8925);
\draw [c] (9.298,8.8925) -- (9.476,8.8925);
\draw [c] (9.654,8.60305) -- (9.654,8.78328);
\draw [c] (9.654,8.78328) -- (9.654,8.93431);
\draw [c] (9.476,8.78328) -- (9.654,8.78328);
\draw [c] (9.654,8.78328) -- (9.832,8.78328);
\draw [c] (10.01,8.53481) -- (10.01,8.7162);
\draw [c] (10.01,8.7162) -- (10.01,8.86805);
\draw [c] (9.832,8.7162) -- (10.01,8.7162);
\draw [c] (10.01,8.7162) -- (10.188,8.7162);
\draw [c] (10.366,9.18671) -- (10.366,9.32071);
\draw [c] (10.366,9.32071) -- (10.366,9.43789);
\draw [c] (10.188,9.32071) -- (10.366,9.32071);
\draw [c] (10.366,9.32071) -- (10.544,9.32071);
\draw [c] (10.722,9.10211) -- (10.722,9.23999);
\draw [c] (10.722,9.23999) -- (10.722,9.36011);
\draw [c] (10.544,9.23999) -- (10.722,9.23999);
\draw [c] (10.722,9.23999) -- (10.9,9.23999);
\draw [c] (11.078,9.29506) -- (11.078,9.42151);
\draw [c] (11.078,9.42151) -- (11.078,9.53287);
\draw [c] (10.9,9.42151) -- (11.078,9.42151);
\draw [c] (11.078,9.42151) -- (11.256,9.42151);
\draw [c] (11.434,9.20387) -- (11.434,9.33425);
\draw [c] (11.434,9.33425) -- (11.434,9.44865);
\draw [c] (11.256,9.33425) -- (11.434,9.33425);
\draw [c] (11.434,9.33425) -- (11.612,9.33425);
\draw [c] (11.79,9.13256) -- (11.79,9.26636);
\draw [c] (11.79,9.26636) -- (11.79,9.38339);
\draw [c] (11.612,9.26636) -- (11.79,9.26636);
\draw [c] (11.79,9.26636) -- (11.968,9.26636);
\draw [c] (12.146,9.19178) -- (12.146,9.32089);
\draw [c] (12.146,9.32089) -- (12.146,9.43432);
\draw [c] (11.968,9.32089) -- (12.146,9.32089);
\draw [c] (12.146,9.32089) -- (12.324,9.32089);
\draw [c] (12.502,9.5085) -- (12.502,9.61988);
\draw [c] (12.502,9.61988) -- (12.502,9.71938);
\draw [c] (12.324,9.61988) -- (12.502,9.61988);
\draw [c] (12.502,9.61988) -- (12.68,9.61988);
\draw [c] (12.858,9.49437) -- (12.858,9.60644);
\draw [c] (12.858,9.60644) -- (12.858,9.7065);
\draw [c] (12.68,9.60644) -- (12.858,9.60644);
\draw [c] (12.858,9.60644) -- (13.036,9.60644);
\draw [c] (13.214,9.55446) -- (13.214,9.66249);
\draw [c] (13.214,9.66249) -- (13.214,9.75931);
\draw [c] (13.036,9.66249) -- (13.214,9.66249);
\draw [c] (13.214,9.66249) -- (13.392,9.66249);
\draw [c] (13.57,9.42412) -- (13.57,9.53843);
\draw [c] (13.57,9.53843) -- (13.57,9.64028);
\draw [c] (13.392,9.53843) -- (13.57,9.53843);
\draw [c] (13.57,9.53843) -- (13.748,9.53843);
\draw [c] (13.926,9.67863) -- (13.926,9.77962);
\draw [c] (13.926,9.77962) -- (13.926,9.87075);
\draw [c] (13.748,9.77962) -- (13.926,9.77962);
\draw [c] (13.926,9.77962) -- (14.104,9.77962);
\draw [c] (14.282,9.72171) -- (14.282,9.82035);
\draw [c] (14.282,9.82035) -- (14.282,9.90956);
\draw [c] (14.104,9.82035) -- (14.282,9.82035);
\draw [c] (14.282,9.82035) -- (14.46,9.82035);
\draw [c] (14.638,9.85957) -- (14.638,9.9518);
\draw [c] (14.638,9.9518) -- (14.638,10.0358);
\draw [c] (14.46,9.9518) -- (14.638,9.9518);
\draw [c] (14.638,9.9518) -- (14.816,9.9518);
\draw [c] (14.994,9.92296) -- (14.994,10.0118);
\draw [c] (14.994,10.0118) -- (14.994,10.093);
\draw [c] (14.816,10.0118) -- (14.994,10.0118);
\draw [c] (14.994,10.0118) -- (15.172,10.0118);
\draw [c] (15.35,9.94089) -- (15.35,10.0282);
\draw [c] (15.35,10.0282) -- (15.35,10.108);
\draw [c] (15.172,10.0282) -- (15.35,10.0282);
\draw [c] (15.35,10.0282) -- (15.528,10.0282);
\draw [c] (15.706,10.1927) -- (15.706,10.2701);
\draw [c] (15.706,10.2701) -- (15.706,10.3417);
\draw [c] (15.528,10.2701) -- (15.706,10.2701);
\draw [c] (15.706,10.2701) -- (15.884,10.2701);
\draw [c] (16.062,10.1057) -- (16.062,10.1858);
\draw [c] (16.062,10.1858) -- (16.062,10.2597);
\draw [c] (15.884,10.1858) -- (16.062,10.1858);
\draw [c] (16.062,10.1858) -- (16.24,10.1858);
\draw [c] (16.418,10.3707) -- (16.418,10.4412);
\draw [c] (16.418,10.4412) -- (16.418,10.5067);
\draw [c] (16.24,10.4412) -- (16.418,10.4412);
\draw [c] (16.418,10.4412) -- (16.596,10.4412);
\draw [c] (16.774,10.333) -- (16.774,10.4043);
\draw [c] (16.774,10.4043) -- (16.774,10.4705);
\draw [c] (16.596,10.4043) -- (16.774,10.4043);
\draw [c] (16.774,10.4043) -- (16.952,10.4043);
\draw [c] (17.13,10.4948) -- (17.13,10.5601);
\draw [c] (17.13,10.5601) -- (17.13,10.6211);
\draw [c] (16.952,10.5601) -- (17.13,10.5601);
\draw [c] (17.13,10.5601) -- (17.308,10.5601);
\draw [c] (17.486,10.6651) -- (17.486,10.7249);
\draw [c] (17.486,10.7249) -- (17.486,10.7811);
\draw [c] (17.308,10.7249) -- (17.486,10.7249);
\draw [c] (17.486,10.7249) -- (17.664,10.7249);
\draw [c] (17.842,10.8028) -- (17.842,10.8579);
\draw [c] (17.842,10.8579) -- (17.842,10.9098);
\draw [c] (17.664,10.8579) -- (17.842,10.8579);
\draw [c] (17.842,10.8579) -- (18.02,10.8579);
\draw [c] (18.198,11.0062) -- (18.198,11.0558);
\draw [c] (18.198,11.0558) -- (18.198,11.1028);
\draw [c] (18.02,11.0558) -- (18.198,11.0558);
\draw [c] (18.198,11.0558) -- (18.376,11.0558);
\draw [c] (18.554,11.2236) -- (18.554,11.2678);
\draw [c] (18.554,11.2678) -- (18.554,11.3099);
\draw [c] (18.376,11.2678) -- (18.554,11.2678);
\draw [c] (18.554,11.2678) -- (18.732,11.2678);
\draw [c] (18.91,11.5547) -- (18.91,11.5914);
\draw [c] (18.91,11.5914) -- (18.91,11.6267);
\draw [c] (18.732,11.5914) -- (18.91,11.5914);
\draw [c] (18.91,11.5914) -- (19.088,11.5914);
\draw [c] (19.266,11.9365) -- (19.266,11.966);
\draw [c] (19.266,11.966) -- (19.266,11.9946);
\draw [c] (19.088,11.966) -- (19.266,11.966);
\draw [c] (19.266,11.966) -- (19.444,11.966);
\draw [c] (19.622,12.5219) -- (19.622,12.5418);
\draw [c] (19.622,12.5418) -- (19.622,12.5613);
\draw [c] (19.444,12.5418) -- (19.622,12.5418);
\draw [c] (19.622,12.5418) -- (19.8,12.5418);
\definecolor{c}{rgb}{0,0,0};
\draw [c] (2,4.74713) -- (19.8,4.74713);
\draw [c] (2,4.98252) -- (2,4.74713);
\draw [c] (2.445,4.86482) -- (2.445,4.74713);
\draw [c] (2.89,4.86482) -- (2.89,4.74713);
\draw [c] (3.335,4.86482) -- (3.335,4.74713);
\draw [c] (3.78,4.98252) -- (3.78,4.74713);
\draw [c] (4.225,4.86482) -- (4.225,4.74713);
\draw [c] (4.67,4.86482) -- (4.67,4.74713);
\draw [c] (5.115,4.86482) -- (5.115,4.74713);
\draw [c] (5.56,4.98252) -- (5.56,4.74713);
\draw [c] (6.005,4.86482) -- (6.005,4.74713);
\draw [c] (6.45,4.86482) -- (6.45,4.74713);
\draw [c] (6.895,4.86482) -- (6.895,4.74713);
\draw [c] (7.34,4.98252) -- (7.34,4.74713);
\draw [c] (7.785,4.86482) -- (7.785,4.74713);
\draw [c] (8.23,4.86482) -- (8.23,4.74713);
\draw [c] (8.675,4.86482) -- (8.675,4.74713);
\draw [c] (9.12,4.98252) -- (9.12,4.74713);
\draw [c] (9.565,4.86482) -- (9.565,4.74713);
\draw [c] (10.01,4.86482) -- (10.01,4.74713);
\draw [c] (10.455,4.86482) -- (10.455,4.74713);
\draw [c] (10.9,4.98252) -- (10.9,4.74713);
\draw [c] (11.345,4.86482) -- (11.345,4.74713);
\draw [c] (11.79,4.86482) -- (11.79,4.74713);
\draw [c] (12.235,4.86482) -- (12.235,4.74713);
\draw [c] (12.68,4.98252) -- (12.68,4.74713);
\draw [c] (13.125,4.86482) -- (13.125,4.74713);
\draw [c] (13.57,4.86482) -- (13.57,4.74713);
\draw [c] (14.015,4.86482) -- (14.015,4.74713);
\draw [c] (14.46,4.98252) -- (14.46,4.74713);
\draw [c] (14.905,4.86482) -- (14.905,4.74713);
\draw [c] (15.35,4.86482) -- (15.35,4.74713);
\draw [c] (15.795,4.86482) -- (15.795,4.74713);
\draw [c] (16.24,4.98252) -- (16.24,4.74713);
\draw [c] (16.685,4.86482) -- (16.685,4.74713);
\draw [c] (17.13,4.86482) -- (17.13,4.74713);
\draw [c] (17.575,4.86482) -- (17.575,4.74713);
\draw [c] (18.02,4.98252) -- (18.02,4.74713);
\draw [c] (18.465,4.86482) -- (18.465,4.74713);
\draw [c] (18.91,4.86482) -- (18.91,4.74713);
\draw [c] (19.355,4.86482) -- (19.355,4.74713);
\draw [c] (19.8,4.98252) -- (19.8,4.74713);
\draw [c] (2,4.98252) -- (2,4.74713);
%\draw [anchor=base] (2,4.4562) node[ rotate=0]{-1};
%\draw [anchor=base] (3.78,4.4562) node[ rotate=0]{-0.8};
%\draw [anchor=base] (5.56,4.4562) node[ rotate=0]{-0.6};
%\draw [anchor=base] (7.34,4.4562) node[ rotate=0]{-0.4};
%\draw [anchor=base] (9.12,4.4562) node[ rotate=0]{-0.2};
%\draw [anchor=base] (10.9,4.4562) node[ rotate=0]{0};
%\draw [anchor=base] (12.68,4.4562) node[ rotate=0]{0.2};
%\draw [anchor=base] (14.46,4.4562) node[ rotate=0]{0.4};
%\draw [anchor=base] (16.24,4.4562) node[ rotate=0]{0.6};
%\draw [anchor=base] (18.02,4.4562) node[ rotate=0]{0.8};
%\draw [anchor=base] (19.8,4.4562) node[ rotate=0]{1};
\draw [c] (2,4.74713) -- (2,13.4751);
\draw [anchor= east] (0.4,13.4751) node[ rotate=90]{$\di\sigma/\di\text{ cos }\theta_{\gamma_1}\text{ [pb]}$};
\draw [c] (2.297,4.92777) -- (2,4.92777);
\draw [c] (2.297,5.30672) -- (2,5.30672);
\draw [c] (2.297,5.57558) -- (2,5.57558);
\draw [c] (2.297,5.78413) -- (2,5.78413);
\draw [c] (2.297,5.95452) -- (2,5.95452);
\draw [c] (2.297,6.09859) -- (2,6.09859);
\draw [c] (2.297,6.22339) -- (2,6.22339);
\draw [c] (2.297,6.33347) -- (2,6.33347);
\draw [c] (2.594,6.43194) -- (2,6.43194);
\draw [anchor= east] (1.844,6.43194) node[ rotate=0]{$10^{-1}$};
\draw [c] (2.297,7.07975) -- (2,7.07975);
\draw [c] (2.297,7.45869) -- (2,7.45869);
\draw [c] (2.297,7.72756) -- (2,7.72756);
\draw [c] (2.297,7.93611) -- (2,7.93611);
\draw [c] (2.297,8.1065) -- (2,8.1065);
\draw [c] (2.297,8.25057) -- (2,8.25057);
\draw [c] (2.297,8.37537) -- (2,8.37537);
\draw [c] (2.297,8.48545) -- (2,8.48545);
\draw [c] (2.594,8.58391) -- (2,8.58391);
\draw [anchor= east] (1.844,8.58391) node[ rotate=0]{1};
\draw [c] (2.297,9.23172) -- (2,9.23172);
\draw [c] (2.297,9.61067) -- (2,9.61067);
\draw [c] (2.297,9.87953) -- (2,9.87953);
\draw [c] (2.297,10.0881) -- (2,10.0881);
\draw [c] (2.297,10.2585) -- (2,10.2585);
\draw [c] (2.297,10.4025) -- (2,10.4025);
\draw [c] (2.297,10.5273) -- (2,10.5273);
\draw [c] (2.297,10.6374) -- (2,10.6374);
\draw [c] (2.594,10.7359) -- (2,10.7359);
\draw [anchor= east] (1.844,10.7359) node[ rotate=0]{10};
\draw [c] (2.297,11.3837) -- (2,11.3837);
\draw [c] (2.297,11.7626) -- (2,11.7626);
\draw [c] (2.297,12.0315) -- (2,12.0315);
\draw [c] (2.297,12.2401) -- (2,12.2401);
\draw [c] (2.297,12.4105) -- (2,12.4105);
\draw [c] (2.297,12.5545) -- (2,12.5545);
\draw [c] (2.297,12.6793) -- (2,12.6793);
\draw [c] (2.297,12.7894) -- (2,12.7894);
\draw [c] (2.594,12.8879) -- (2,12.8879);
\draw [anchor= east] (1.844,12.8879) node[ rotate=0]{$10^{2}$};
\colorlet{c}{natgreen};
\draw [c] (8.942,8.32752) -- (8.942,8.54389);
\draw [c] (8.942,8.54389) -- (8.942,8.71946);
\draw [c] (8.764,8.54389) -- (8.942,8.54389);
\draw [c] (8.942,8.54389) -- (9.12,8.54389);
\draw [c] (9.298,8.95732) -- (9.298,9.11057);
\draw [c] (9.298,9.11057) -- (9.298,9.2422);
\draw [c] (9.12,9.11057) -- (9.298,9.11057);
\draw [c] (9.298,9.11057) -- (9.476,9.11057);
\draw [c] (9.654,9.05883) -- (9.654,9.20102);
\draw [c] (9.654,9.20102) -- (9.654,9.3244);
\draw [c] (9.476,9.20102) -- (9.654,9.20102);
\draw [c] (9.654,9.20102) -- (9.832,9.20102);
\draw [c] (10.01,9.28346) -- (10.01,9.41137);
\draw [c] (10.01,9.41137) -- (10.01,9.52387);
\draw [c] (9.832,9.41137) -- (10.01,9.41137);
\draw [c] (10.01,9.41137) -- (10.188,9.41137);
\draw [c] (10.366,9.70261) -- (10.366,9.80633);
\draw [c] (10.366,9.80633) -- (10.366,9.89968);
\draw [c] (10.188,9.80633) -- (10.366,9.80633);
\draw [c] (10.366,9.80633) -- (10.544,9.80633);
\draw [c] (10.722,9.54835) -- (10.722,9.65947);
\draw [c] (10.722,9.65947) -- (10.722,9.75877);
\draw [c] (10.544,9.65947) -- (10.722,9.65947);
\draw [c] (10.722,9.65947) -- (10.9,9.65947);
\draw [c] (11.078,9.63068) -- (11.078,9.73663);
\draw [c] (11.078,9.73663) -- (11.078,9.83179);
\draw [c] (10.9,9.73663) -- (11.078,9.73663);
\draw [c] (11.078,9.73663) -- (11.256,9.73663);
\draw [c] (11.434,9.44451) -- (11.434,9.55945);
\draw [c] (11.434,9.55945) -- (11.434,9.66179);
\draw [c] (11.256,9.55945) -- (11.434,9.55945);
\draw [c] (11.434,9.55945) -- (11.612,9.55945);
\draw [c] (11.79,9.75381) -- (11.79,9.85293);
\draw [c] (11.79,9.85293) -- (11.79,9.94254);
\draw [c] (11.612,9.85293) -- (11.79,9.85293);
\draw [c] (11.79,9.85293) -- (11.968,9.85293);
\draw [c] (12.146,9.94753) -- (12.146,10.0378);
\draw [c] (12.146,10.0378) -- (12.146,10.1201);
\draw [c] (11.968,10.0378) -- (12.146,10.0378);
\draw [c] (12.146,10.0378) -- (12.324,10.0378);
\draw [c] (12.502,9.84847) -- (12.502,9.94226);
\draw [c] (12.502,9.94226) -- (12.502,10.0275);
\draw [c] (12.324,9.94226) -- (12.502,9.94226);
\draw [c] (12.502,9.94226) -- (12.68,9.94226);
\draw [c] (12.858,9.80496) -- (12.858,9.90077);
\draw [c] (12.858,9.90077) -- (12.858,9.98767);
\draw [c] (12.68,9.90077) -- (12.858,9.90077);
\draw [c] (12.858,9.90077) -- (13.036,9.90077);
\draw [c] (13.214,9.92635) -- (13.214,10.0163);
\draw [c] (13.214,10.0163) -- (13.214,10.0984);
\draw [c] (13.036,10.0163) -- (13.214,10.0163);
\draw [c] (13.214,10.0163) -- (13.392,10.0163);
\draw [c] (13.57,10.051) -- (13.57,10.1353);
\draw [c] (13.57,10.1353) -- (13.57,10.2127);
\draw [c] (13.392,10.1353) -- (13.57,10.1353);
\draw [c] (13.57,10.1353) -- (13.748,10.1353);
\draw [c] (13.926,9.8832) -- (13.926,9.97426);
\draw [c] (13.926,9.97426) -- (13.926,10.0572);
\draw [c] (13.748,9.97426) -- (13.926,9.97426);
\draw [c] (13.926,9.97426) -- (14.104,9.97426);
\draw [c] (14.282,10.1661) -- (14.282,10.2457);
\draw [c] (14.282,10.2457) -- (14.282,10.3191);
\draw [c] (14.104,10.2457) -- (14.282,10.2457);
\draw [c] (14.282,10.2457) -- (14.46,10.2457);
\draw [c] (14.638,10.0759) -- (14.638,10.1582);
\draw [c] (14.638,10.1582) -- (14.638,10.2338);
\draw [c] (14.46,10.1582) -- (14.638,10.1582);
\draw [c] (14.638,10.1582) -- (14.816,10.1582);
\draw [c] (14.994,10.2932) -- (14.994,10.3674);
\draw [c] (14.994,10.3674) -- (14.994,10.4362);
\draw [c] (14.816,10.3674) -- (14.994,10.3674);
\draw [c] (14.994,10.3674) -- (15.172,10.3674);
\draw [c] (15.35,10.3636) -- (15.35,10.4348);
\draw [c] (15.35,10.4348) -- (15.35,10.501);
\draw [c] (15.172,10.4348) -- (15.35,10.4348);
\draw [c] (15.35,10.4348) -- (15.528,10.4348);
\draw [c] (15.706,10.3869) -- (15.706,10.4569);
\draw [c] (15.706,10.4569) -- (15.706,10.522);
\draw [c] (15.528,10.4569) -- (15.706,10.4569);
\draw [c] (15.706,10.4569) -- (15.884,10.4569);
\draw [c] (16.062,10.4427) -- (16.062,10.5109);
\draw [c] (16.062,10.5109) -- (16.062,10.5745);
\draw [c] (15.884,10.5109) -- (16.062,10.5109);
\draw [c] (16.062,10.5109) -- (16.24,10.5109);
\draw [c] (16.418,10.6101) -- (16.418,10.6725);
\draw [c] (16.418,10.6725) -- (16.418,10.7311);
\draw [c] (16.24,10.6725) -- (16.418,10.6725);
\draw [c] (16.418,10.6725) -- (16.596,10.6725);
\draw [c] (16.774,10.6243) -- (16.774,10.6859);
\draw [c] (16.774,10.6859) -- (16.774,10.7436);
\draw [c] (16.596,10.6859) -- (16.774,10.6859);
\draw [c] (16.774,10.6859) -- (16.952,10.6859);
\draw [c] (17.13,10.9306) -- (17.13,10.9836);
\draw [c] (17.13,10.9836) -- (17.13,11.0337);
\draw [c] (16.952,10.9836) -- (17.13,10.9836);
\draw [c] (17.13,10.9836) -- (17.308,10.9836);
\draw [c] (17.486,11.0221) -- (17.486,11.0723);
\draw [c] (17.486,11.0723) -- (17.486,11.1199);
\draw [c] (17.308,11.0723) -- (17.486,11.0723);
\draw [c] (17.486,11.0723) -- (17.664,11.0723);
\draw [c] (17.842,11.2318) -- (17.842,11.2768);
\draw [c] (17.842,11.2768) -- (17.842,11.3197);
\draw [c] (17.664,11.2768) -- (17.842,11.2768);
\draw [c] (17.842,11.2768) -- (18.02,11.2768);
\draw [c] (18.198,11.3307) -- (18.198,11.3731);
\draw [c] (18.198,11.3731) -- (18.198,11.4136);
\draw [c] (18.02,11.3731) -- (18.198,11.3731);
\draw [c] (18.198,11.3731) -- (18.376,11.3731);
\draw [c] (18.554,11.5436) -- (18.554,11.5814);
\draw [c] (18.554,11.5814) -- (18.554,11.6176);
\draw [c] (18.376,11.5814) -- (18.554,11.5814);
\draw [c] (18.554,11.5814) -- (18.732,11.5814);
\draw [c] (18.91,11.8831) -- (18.91,11.9145);
\draw [c] (18.91,11.9145) -- (18.91,11.9449);
\draw [c] (18.732,11.9145) -- (18.91,11.9145);
\draw [c] (18.91,11.9145) -- (19.088,11.9145);
\draw [c] (19.266,12.22) -- (19.266,12.2458);
\draw [c] (19.266,12.2458) -- (19.266,12.2709);
\draw [c] (19.088,12.2458) -- (19.266,12.2458);
\draw [c] (19.266,12.2458) -- (19.444,12.2458);
\draw [c] (19.622,12.9858) -- (19.622,13.0021);
\draw [c] (19.622,13.0021) -- (19.622,13.018);
\draw [c] (19.444,13.0021) -- (19.622,13.0021);
\draw [c] (19.622,13.0021) -- (19.8,13.0021);
\definecolor{c}{rgb}{1,1,1};
\draw [c] (3.3908,11.3793) -- (3.3908,13.1609) -- (9.39655,13.1609) -- (9.39655,11.3793) -- (3.3908,11.3793);
\draw [c] (3.3908,11.3793) -- (9.39655,11.3793);
\draw [c] (9.39655,11.3793) -- (9.39655,13.1609);
\draw [c] (9.39655,13.1609) -- (3.3908,13.1609);
\draw [c] (3.3908,13.1609) -- (3.3908,11.3793);
\draw [anchor=base west] (4.89224,12.5151) node[ rotate=0]{Hard event};
\colorlet{c}{natgreen!20};
\draw [c, fill=c] (3.61602,12.4037) -- (4.66703,12.4037) -- (4.66703,13.0273) -- (3.61602,13.0273);
\colorlet{c}{natgreen};
\draw [c] (3.61602,12.7155) -- (4.66703,12.7155);
\draw [anchor=base west] (4.89224,11.6243) node[ rotate=0]{Extended event};
\colorlet{c}{natcomp};
\draw [c] (3.61602,11.8247) -- (4.66703,11.8247);
\definecolor{c}{rgb}{1,1,1};
\draw [color=c, fill=c] (0,0) rectangle (20,4.74713);
\draw [color=c, fill=c] (2,0.949425) rectangle (19.8,4.74713);
\definecolor{c}{rgb}{0,0,0};
\draw [c] (2,0.949425) -- (2,4.74713) -- (19.8,4.74713) -- (19.8,0.949425) -- (2,0.949425);
\definecolor{c}{rgb}{1,1,1};
\draw [color=c, fill=c] (2,0.949425) rectangle (19.8,4.74713);
\definecolor{c}{rgb}{0,0,0};
\draw [c] (2,0.949425) -- (2,4.74713) -- (19.8,4.74713) -- (19.8,0.949425) -- (2,0.949425);
\colorlet{c}{natgreen!20};
\draw [color=c, fill=c] (8.764,2.20029) rectangle (9.12,4.5763);
\draw [color=c, fill=c] (9.12,2.51889) rectangle (9.476,4.2577);
\draw [color=c, fill=c] (9.476,2.57703) rectangle (9.832,4.19956);
\draw [color=c, fill=c] (9.832,2.653) rectangle (10.188,4.12359);
\draw [color=c, fill=c] (10.188,2.78445) rectangle (10.544,3.99213);
\draw [color=c, fill=c] (10.544,2.74389) rectangle (10.9,4.0327);
\draw [color=c, fill=c] (10.9,2.77216) rectangle (11.256,4.00442);
\draw [color=c, fill=c] (11.256,2.72307) rectangle (11.612,4.05352);
\draw [color=c, fill=c] (11.612,2.80985) rectangle (11.968,3.96674);
\draw [color=c, fill=c] (11.968,2.85912) rectangle (12.324,3.91747);
\draw [color=c, fill=c] (12.324,2.83943) rectangle (12.68,3.93716);
\draw [color=c, fill=c] (12.68,2.82815) rectangle (13.036,3.94844);
\draw [color=c, fill=c] (13.036,2.86077) rectangle (13.392,3.91582);
\draw [color=c, fill=c] (13.392,2.89215) rectangle (13.748,3.88444);
\draw [color=c, fill=c] (13.748,2.85464) rectangle (14.104,3.92195);
\draw [color=c, fill=c] (14.104,2.919) rectangle (14.46,3.85759);
\draw [color=c, fill=c] (14.46,2.90399) rectangle (14.816,3.8726);
\draw [color=c, fill=c] (14.816,2.94946) rectangle (15.172,3.82713);
\draw [color=c, fill=c] (15.172,2.96654) rectangle (15.528,3.81005);
\draw [color=c, fill=c] (15.528,2.97351) rectangle (15.884,3.80308);
\draw [color=c, fill=c] (15.884,2.98366) rectangle (16.24,3.79293);
\draw [color=c, fill=c] (16.24,3.01662) rectangle (16.596,3.75996);
\draw [color=c, fill=c] (16.596,3.02182) rectangle (16.952,3.75477);
\draw [color=c, fill=c] (16.952,3.07151) rectangle (17.308,3.70508);
\draw [color=c, fill=c] (17.308,3.08783) rectangle (17.664,3.68876);
\draw [color=c, fill=c] (17.664,3.1182) rectangle (18.02,3.65839);
\draw [color=c, fill=c] (18.02,3.1333) rectangle (18.376,3.64329);
\draw [color=c, fill=c] (18.376,3.16084) rectangle (18.732,3.61575);
\draw [color=c, fill=c] (18.732,3.19845) rectangle (19.088,3.57814);
\draw [color=c, fill=c] (19.088,3.23177) rectangle (19.444,3.54482);
\draw [color=c, fill=c] (19.444,3.2892) rectangle (19.8,3.48738);
\definecolor{c}{rgb}{0,0,0};
\draw [c] (2,0.949425) -- (19.8,0.949425);
\draw [anchor= east] (19.8,0.683586) +(0,-1.8em) node[ rotate=0]{$\text{cos }\theta_{\gamma_1}$};
\draw [c] (2,1.07617) -- (2,0.949425);
\draw [c] (2.445,1.0128) -- (2.445,0.949425);
\draw [c] (2.89,1.0128) -- (2.89,0.949425);
\draw [c] (3.335,1.0128) -- (3.335,0.949425);
\draw [c] (3.78,1.07617) -- (3.78,0.949425);
\draw [c] (4.225,1.0128) -- (4.225,0.949425);
\draw [c] (4.67,1.0128) -- (4.67,0.949425);
\draw [c] (5.115,1.0128) -- (5.115,0.949425);
\draw [c] (5.56,1.07617) -- (5.56,0.949425);
\draw [c] (6.005,1.0128) -- (6.005,0.949425);
\draw [c] (6.45,1.0128) -- (6.45,0.949425);
\draw [c] (6.895,1.0128) -- (6.895,0.949425);
\draw [c] (7.34,1.07617) -- (7.34,0.949425);
\draw [c] (7.785,1.0128) -- (7.785,0.949425);
\draw [c] (8.23,1.0128) -- (8.23,0.949425);
\draw [c] (8.675,1.0128) -- (8.675,0.949425);
\draw [c] (9.12,1.07617) -- (9.12,0.949425);
\draw [c] (9.565,1.0128) -- (9.565,0.949425);
\draw [c] (10.01,1.0128) -- (10.01,0.949425);
\draw [c] (10.455,1.0128) -- (10.455,0.949425);
\draw [c] (10.9,1.07617) -- (10.9,0.949425);
\draw [c] (11.345,1.0128) -- (11.345,0.949425);
\draw [c] (11.79,1.0128) -- (11.79,0.949425);
\draw [c] (12.235,1.0128) -- (12.235,0.949425);
\draw [c] (12.68,1.07617) -- (12.68,0.949425);
\draw [c] (13.125,1.0128) -- (13.125,0.949425);
\draw [c] (13.57,1.0128) -- (13.57,0.949425);
\draw [c] (14.015,1.0128) -- (14.015,0.949425);
\draw [c] (14.46,1.07617) -- (14.46,0.949425);
\draw [c] (14.905,1.0128) -- (14.905,0.949425);
\draw [c] (15.35,1.0128) -- (15.35,0.949425);
\draw [c] (15.795,1.0128) -- (15.795,0.949425);
\draw [c] (16.24,1.07617) -- (16.24,0.949425);
\draw [c] (16.685,1.0128) -- (16.685,0.949425);
\draw [c] (17.13,1.0128) -- (17.13,0.949425);
\draw [c] (17.575,1.0128) -- (17.575,0.949425);
\draw [c] (18.02,1.07617) -- (18.02,0.949425);
\draw [c] (18.465,1.0128) -- (18.465,0.949425);
\draw [c] (18.91,1.0128) -- (18.91,0.949425);
\draw [c] (19.355,1.0128) -- (19.355,0.949425);
\draw [c] (19.8,1.07617) -- (19.8,0.949425);
\draw [c] (2,1.07617) -- (2,0.949425);
\draw [anchor=base] (2,    0.79277) +(0,-.8em) node[ rotate=0]{-1};
\draw [anchor=base] (3.78, 0.79277) +(0,-.8em) node[ rotate=0]{-0.8};
\draw [anchor=base] (5.56, 0.79277) +(0,-.8em) node[ rotate=0]{-0.6};
\draw [anchor=base] (7.34, 0.79277) +(0,-.8em) node[ rotate=0]{-0.4};
\draw [anchor=base] (9.12, 0.79277) +(0,-.8em) node[ rotate=0]{-0.2};
\draw [anchor=base] (10.9, 0.79277) +(0,-.8em) node[ rotate=0]{0};
\draw [anchor=base] (12.68,0.79277) +(0,-.8em) node[ rotate=0]{0.2};
\draw [anchor=base] (14.46,0.79277) +(0,-.8em) node[ rotate=0]{0.4};
\draw [anchor=base] (16.24,0.79277) +(0,-.8em) node[ rotate=0]{0.6};
\draw [anchor=base] (18.02,0.79277) +(0,-.8em) node[ rotate=0]{0.8};
\draw [anchor=base] (19.8, 0.79277) +(0,-.8em) node[ rotate=0]{1};
\draw [c] (2,0.949425) -- (2,4.74713);
\draw [anchor= east] (0.4,4.74713) node[ rotate=90]{Ratio};
\draw [c] (2.48,0.949423) -- (2,0.949423);
\draw [c] (2.24,1.15266) -- (2,1.15266);
\draw [c] (2.24,1.3559) -- (2,1.3559);
\draw [c] (2.24,1.55914) -- (2,1.55914);
\draw [c] (2.48,1.76238) -- (2,1.76238);
\draw [c] (2.24,1.96562) -- (2,1.96562);
\draw [c] (2.24,2.16886) -- (2,2.16886);
\draw [c] (2.24,2.3721) -- (2,2.3721);
\draw [c] (2.48,2.57534) -- (2,2.57534);
\draw [c] (2.24,2.77858) -- (2,2.77858);
\draw [c] (2.24,2.98182) -- (2,2.98182);
\draw [c] (2.24,3.18505) -- (2,3.18505);
\draw [c] (2.48,3.38829) -- (2,3.38829);
\draw [c] (2.24,3.59153) -- (2,3.59153);
\draw [c] (2.24,3.79477) -- (2,3.79477);
\draw [c] (2.24,3.99801) -- (2,3.99801);
\draw [c] (2.48,4.20125) -- (2,4.20125);
\draw [c] (2.48,0.949423) -- (2,0.949423);
\draw [c] (2.48,4.20125) -- (2,4.20125);
\draw [c] (2.24,4.40449) -- (2,4.40449);
\draw [c] (2.24,4.60773) -- (2,4.60773);
\draw [anchor= east] (1.9,0.949423) node[ rotate=0]{0.4};
\draw [anchor= east] (1.9,1.76238) node[ rotate=0]{0.6};
\draw [anchor= east] (1.9,2.57534) node[ rotate=0]{0.8};
\draw [anchor= east] (1.9,3.38829) node[ rotate=0]{1.0};
\draw [anchor= east] (1.9,4.20125) node[ rotate=0]{1.2};
\colorlet{c}{natcomp};
\draw [c] (8.942,1.85406) -- (8.942,2.92909);
\draw [c] (8.942,2.92909) -- (8.942,4.00412);
\draw [c] (8.764,2.92909) -- (8.942,2.92909);
\draw [c] (8.942,2.92909) -- (9.12,2.92909);
\draw [c] (9.298,1.81363) -- (9.298,2.54238);
\draw [c] (9.298,2.54238) -- (9.298,3.27112);
\draw [c] (9.12,2.54238) -- (9.298,2.54238);
\draw [c] (9.298,2.54238) -- (9.476,2.54238);
\draw [c] (9.654,1.33795) -- (9.654,1.92319);
\draw [c] (9.654,1.92319) -- (9.654,2.50843);
\draw [c] (9.476,1.92319) -- (9.654,1.92319);
\draw [c] (9.654,1.92319) -- (9.832,1.92319);
\draw [c] (10.01,0.949425) -- (10.01,1.25548);
\draw [c] (10.01,1.25548) -- (10.01,1.67647);
\draw [c] (9.832,1.25548) -- (10.01,1.25548);
\draw [c] (10.01,1.25548) -- (10.188,1.25548);
\draw [c] (10.366,1.33025) -- (10.366,1.74106);
\draw [c] (10.366,1.74106) -- (10.366,2.15188);
\draw [c] (10.188,1.74106) -- (10.366,1.74106);
\draw [c] (10.366,1.74106) -- (10.544,1.74106);
\draw [c] (10.722,1.45869) -- (10.722,1.91834);
\draw [c] (10.722,1.91834) -- (10.722,2.378);
\draw [c] (10.544,1.91834) -- (10.722,1.91834);
\draw [c] (10.722,1.91834) -- (10.9,1.91834);
\draw [c] (11.078,1.74372) -- (11.078,2.22488);
\draw [c] (11.078,2.22488) -- (11.078,2.70604);
\draw [c] (10.9,2.22488) -- (11.078,2.22488);
\draw [c] (11.078,2.22488) -- (11.256,2.22488);
\draw [c] (11.434,1.96144) -- (11.434,2.51791);
\draw [c] (11.434,2.51791) -- (11.434,3.07438);
\draw [c] (11.256,2.51791) -- (11.434,2.51791);
\draw [c] (11.434,2.51791) -- (11.612,2.51791);
\draw [c] (11.79,1.13095) -- (11.79,1.49354);
\draw [c] (11.79,1.49354) -- (11.79,1.85613);
\draw [c] (11.612,1.49354) -- (11.79,1.49354);
\draw [c] (11.79,1.49354) -- (11.968,1.49354);
\draw [c] (12.146,0.949425) -- (12.146,1.2111);
\draw [c] (12.146,1.2111) -- (12.146,1.51029);
\draw [c] (11.968,1.2111) -- (12.146,1.2111);
\draw [c] (12.146,1.2111) -- (12.324,1.2111);
\draw [c] (12.502,1.77798) -- (12.502,2.20244);
\draw [c] (12.502,2.20244) -- (12.502,2.62691);
\draw [c] (12.324,2.20244) -- (12.502,2.20244);
\draw [c] (12.502,2.20244) -- (12.68,2.20244);
\draw [c] (12.858,1.84749) -- (12.858,2.29016);
\draw [c] (12.858,2.29016) -- (12.858,2.73282);
\draw [c] (12.68,2.29016) -- (12.858,2.29016);
\draw [c] (12.858,2.29016) -- (13.036,2.29016);
\draw [c] (13.214,1.71023) -- (13.214,2.10721);
\draw [c] (13.214,2.10721) -- (13.214,2.50418);
\draw [c] (13.036,2.10721) -- (13.214,2.10721);
\draw [c] (13.214,2.10721) -- (13.392,2.10721);
\draw [c] (13.57,1.16088) -- (13.57,1.46969);
\draw [c] (13.57,1.46969) -- (13.57,1.77851);
\draw [c] (13.392,1.46969) -- (13.57,1.46969);
\draw [c] (13.57,1.46969) -- (13.748,1.46969);
\draw [c] (13.926,2.16784) -- (13.926,2.62409);
\draw [c] (13.926,2.62409) -- (13.926,3.08035);
\draw [c] (13.748,2.62409) -- (13.926,2.62409);
\draw [c] (13.926,2.62409) -- (14.104,2.62409);
\draw [c] (14.282,1.56881) -- (14.282,1.90201);
\draw [c] (14.282,1.90201) -- (14.282,2.23521);
\draw [c] (14.104,1.90201) -- (14.282,1.90201);
\draw [c] (14.282,1.90201) -- (14.46,1.90201);
\draw [c] (14.638,2.17149) -- (14.638,2.58286);
\draw [c] (14.638,2.58286) -- (14.638,2.99424);
\draw [c] (14.46,2.58286) -- (14.638,2.58286);
\draw [c] (14.638,2.58286) -- (14.816,2.58286);
\draw [c] (14.994,1.77248) -- (14.994,2.10191);
\draw [c] (14.994,2.10191) -- (14.994,2.43134);
\draw [c] (14.816,2.10191) -- (14.994,2.10191);
\draw [c] (14.994,2.10191) -- (15.172,2.10191);
\draw [c] (15.35,1.65051) -- (15.35,1.9543);
\draw [c] (15.35,1.9543) -- (15.35,2.2581);
\draw [c] (15.172,1.9543) -- (15.35,1.9543);
\draw [c] (15.35,1.9543) -- (15.528,1.9543);
\draw [c] (15.706,2.29443) -- (15.706,2.65192);
\draw [c] (15.706,2.65192) -- (15.706,3.00942);
\draw [c] (15.528,2.65192) -- (15.706,2.65192);
\draw [c] (15.706,2.65192) -- (15.884,2.65192);
\draw [c] (16.062,1.88336) -- (16.062,2.19402);
\draw [c] (16.062,2.19402) -- (16.062,2.50467);
\draw [c] (15.884,2.19402) -- (16.062,2.19402);
\draw [c] (16.062,2.19402) -- (16.24,2.19402);
\draw [c] (16.418,2.18845) -- (16.418,2.49704);
\draw [c] (16.418,2.49704) -- (16.418,2.80562);
\draw [c] (16.24,2.49704) -- (16.418,2.49704);
\draw [c] (16.418,2.49704) -- (16.596,2.49704);
\draw [c] (16.774,2.03858) -- (16.774,2.33102);
\draw [c] (16.774,2.33102) -- (16.774,2.62347);
\draw [c] (16.596,2.33102) -- (16.774,2.33102);
\draw [c] (16.774,2.33102) -- (16.952,2.33102);
\draw [c] (17.13,1.68223) -- (17.13,1.90728);
\draw [c] (17.13,1.90728) -- (17.13,2.13233);
\draw [c] (16.952,1.90728) -- (17.13,1.90728);
\draw [c] (17.13,1.90728) -- (17.308,1.90728);
\draw [c] (17.486,1.89907) -- (17.486,2.1263);
\draw [c] (17.486,2.1263) -- (17.486,2.35353);
\draw [c] (17.308,2.1263) -- (17.486,2.1263);
\draw [c] (17.486,2.1263) -- (17.664,2.1263);
\draw [c] (17.842,1.72777) -- (17.842,1.91992);
\draw [c] (17.842,1.91992) -- (17.842,2.11206);
\draw [c] (17.664,1.91992) -- (17.842,1.91992);
\draw [c] (17.842,1.91992) -- (18.02,1.91992);
\draw [c] (18.198,2.02109) -- (18.198,2.21811);
\draw [c] (18.198,2.21811) -- (18.198,2.41514);
\draw [c] (18.02,2.21811) -- (18.198,2.21811);
\draw [c] (18.198,2.21811) -- (18.376,2.21811);
\draw [c] (18.554,2.05309) -- (18.554,2.22969);
\draw [c] (18.554,2.22969) -- (18.554,2.4063);
\draw [c] (18.376,2.22969) -- (18.554,2.22969);
\draw [c] (18.554,2.22969) -- (18.732,2.22969);
\draw [c] (18.91,2.0543) -- (18.91,2.20026);
\draw [c] (18.91,2.20026) -- (18.91,2.34622);
\draw [c] (18.732,2.20026) -- (18.91,2.20026);
\draw [c] (18.91,2.20026) -- (19.088,2.20026);
\draw [c] (19.266,2.21205) -- (19.266,2.33655);
\draw [c] (19.266,2.33655) -- (19.266,2.46105);
\draw [c] (19.088,2.33655) -- (19.266,2.33655);
\draw [c] (19.266,2.33655) -- (19.444,2.33655);
\draw [c] (19.622,1.73995) -- (19.622,1.80757);
\draw [c] (19.622,1.80757) -- (19.622,1.87519);
\draw [c] (19.444,1.80757) -- (19.622,1.80757);
\draw [c] (19.622,1.80757) -- (19.8,1.80757);
\end{tikzpicture}

}\end{infilsf}
\end{minipage}
\hfill\begin{minipage}[b]{.3\textwidth}
\caption{A plot of the distribution of the cosine of the scattering angle $\theta$ (in the CS frame, as in fig.~\ref{discr}) of the leading photon in the set of events generated by CalcHEP in the SM scenario before (green) and after (red) being extended with pythia. Aside from the same constant faction of lost events as was also seen in fig.~\ref{pythify}, we note that the distribution contains events with lower $\cos\theta_{\gamma_1}$ than were present in the set of generated events.
\label{pythicos}}
\end{minipage}
\end{figure}

Since the main process under study does not involve coloured final states, we do not expect that these surrounding processes will be a significant source of systematic uncertainty or bias, We are supported in this assumption by the findings in figure~\ref{pythify}. To support that expectation beyond what is done here, one might attempt to add the extended processes with a different software package.



\chapter{Experiment}\label{ch.exp}

Having thus formed a clear picture of what observable effects we expect to manifest from our changes to the Standard Model, we can now proceed with the experimental test.

As was perhaps given away by the title, this thesis will use the already established \textsc{atlas}\footnote{\textbf{A} \textbf{T}oroidal \textbf{L}HC \textbf{A}pparatu\textbf{s}.} experiment, which is part of the Large Hadron Collider complex at \textsc{cern}\footnote{\textbf{C}onseil \textbf{E}uropéen pour la \textbf{R}echerche \textbf{N}ucléaire. When the council tasked with creating the european nuclear research laboratory became the organisation tasked with running that laboratory, its name changed to Organisation Européenne pour la Recherche Nucléaire---the European Organisation for Nuclear Research, but the initialism remained. Acronyms, it seems, are not only ubiquitous, but also immutable.} in Geneva.

\section{The Large Hadron Collider}

\begin{figure}[htp]
\begin{center}
\includegraphics[width=.8\textwidth]{Cernrings}
\end{center}
\begin{minipage}[b]{\textwidth}
\caption{A schematic view of the \textsc{cern} accelerator complex \cite{cernbro}, in which protons or heavy ions used in collision experiments are accelerated through several machines to progressively higher energies. The paths protons can take through the machine are marked with light gray triangles. The dark gray triangles mark the paths taken by heavy ions when the Collider runs proton--lead or lead--lead collision experiments. Protons are `created' by ionising hydrogen atoms and then injected by \textcolor{Purple}{LINAC 2} into the \textcolor{Plum}{Booster ring}. From there, protons are accelerated by the Proton Synchrotron (\textcolor{Magenta}{PS}) and then the Super Protron Synchrotron (\textcolor{RoyalBlue}{SPS}) before finally being sent into the \textcolor{MidnightBlue}{LHC} ring. The LHC ring is the largest circular accelerator in the \textsc{cern} complex, and---at time of writing---in the world, measuring approximately 27~km in circumference.}
\label{cernrings}
\end{minipage}
\end{figure}

The \textsc{cern} accelerator complex comprises a number of particle accelerators with a wide variety of sizes, which service several experiments with various aims \cite{cernexps}. The most conspicuous element of the accelerator complex, of which an overview is given in figure~\ref{cernrings}, is the LHC ring, which is 27~km in circumference (which, even by the standards of particle accelerators, is quite large) and designed to contain proton (a species of hadron) beams with energies as high as 7~TeV. Where two such beams cross one another, proton collisions with 14~TeV energy may occur. The LHC's beams cross one another at four points around the ring, each one the site of one of the four main experiments.

The LHC is the last step in a series of particle accelerators---which incorporates two of \textsc{cern}'s previous main accelerators, the Proton Synchrotron and the Super Proton Synchrotron---which is needed to bring protons from rest to a final energy of 7~TeV. Throughout, protons are accelerated by applied electric fields, applied within so--called radiofrequency cavities, and contained within the rings by dipole magnets. Magnets with higher pole counts are used to manipulate the profile of the proton beam. [Fernow, maybe?]

Even though the LHC was designed for an energy per beam of 7~TeV, giving a collision energy of 14~TeV, several accidents during commissioning of the machine have necessitated the use of a lower beam energy for the first runs. In 2012, when the data that will be used in this thesis was taken, the LHC ran at 4~TeV per beam, for a total collision energy of 8~TeV.

Although we speak of proton beams, protons within the beams are arranged in discrete bunches, occurring at 50~ns intervals. As a consequence, proton collisions occur only within known time intervals, which dictate the timing by which detector readout occurs.

\section{The ATLAS detector}
The \textsc{atlas} detector is the largest of the LHC's four detector expermients, and is, as is the CMS\footnote{The \textbf{C}ompact \textbf{M}uon \textbf{S}olenoid.}, a general purpose detector, designed to capture as much information as possible about collision events. To that end, the \textsc{atlas} detector is made up of three distinct detector subsystems, layered concentrically about the interaction point, as illustrated in in figure~\ref{allatlas}. From innermost to outermost, these are: the tracking system, the calorimeter system and the muon spectrometer.

\begin{figure}[htp]
\begin{minipage}[b]{.69\textwidth}
\includegraphics[width=1\textwidth]{AllAtlasBig}
\end{minipage}
\begin{minipage}[b]{.3\textwidth}
\caption{A diagram of the full \textsc{atlas} detector \cite{atlasweb}. The overall structure is of a layered cylinder centred on the interaction point. We refer to those parts of the detector that make up the wall of the cylinder as the barrel section, and to the ends of the cylinder as the endcap. The electromagnetic calorimeter, which is the detector element that we will make the most use of here, is coloured orange in this drawing.}
\end{minipage}
\label{allatlas}
\end{figure}

\textsc{Atlas} defines its own coordinate system, centred on the interaction point, where the position in the angular direction, perpendicular to the beam pipe, is measured by the azimuthal coordinate $\phi$, and the angle to the beam pipe is measured in pseudorapidity $\eta$, which is defined as
\(\eta\equiv-\ln[\tan(\theta/2)]=\half\ln\frac{|\mathbf{p}|+p_z}{|\mathbf{p}|-p_z},\)
where $\theta$ is the polar angle to the beam pipe in radians \cite{green:eta}. The pseudorapidity $\eta$ is a simple transformation of $\theta$---it is 0 at $\theta=\pi/2$, $\infty$ at $\theta=0$ and $-\infty$ at $\theta=\pi$---but is chosen since it is approximately, and for massless particles exactly, equal to rapidity, 
\[ y \equiv \frac{1}{2} \ln \frac{E + p_z }{E - p_z }, \]
which is additive under Lorenz boosts in the $z$ direction \cite{green:y}.

A machine as large and complex as \atlas{} would require a large document, such as~\cite{detectorpaper}, to describe in comprehensive detail. For the present, a decidedly non--comprehensive description of the three detector layers, in order of increasing relevance to this work, follows:

The muon spectrometer forms the outermost, and most voluminous, part of the detector. As a species, muons, along with neutrinos, are the only types of particles that can regularly penetrate both calorimeters. Capturing neutrinos with \textsc{atlas}' mere 90\,000~t of material and 7\,000~m$^3$ detector volume \cite{atlasweb} is something of a lost cause, however.

The muon spectrometer and the inner detector are both tracking detectors, meaning that they are designed to determine the path of a charged particle that passes through them. By immersing such a detector in a magnetic field, which will deflect a charged particle, the charge sign and momentum of the particle can be deduced from the curvature of the track.

The inner detector is also a tracking detector, which uses two distinct detector technologies to localise the tracks of charged particles that pass through it. The innermost layers use silicon semiconductor chips to detect the charge left by a passing charged particle, essentially by allowing the ionisation charge to flip a transistor. The outer part of the inner detector uses drift straws, long tubes with a wire with a high voltage charge in the centre, and filled with an inert gas. The ionisation charge left by a passing high--energy particle is attracted to the wire, and read out by detectors attached to its ends as a voltage change. The silicon detector is compact system with a high resolution, whereas the straw detectors can economically cover a large volume.

Both the straw detector and the outer part of the silicon detector---the silicon microstrip detector---have very long detector elements, which can only report that a hit has occurred somewhere along its length\footnote{Some additional resolution can be gained by studying the drift time in the straw detectors.}. To improve resolution in the long direction of these detectors, successive layers of detector elements are placed at an angle to one another, so that hits on successive detector elements narrow down the possible location of a track. The innermost layer of the silicon detector, the pixel detector, uses a pixel structure rather than a strip structure, and so has adequate resolution in all directions. What is worth noting here is that the tracking detector only directly reports times and---more or less approximate---locations of particle observations. They are only combined into particle tracks during a later analysis step. 

There are two calorimeter systems in \textsc{atlas}: the (inner) electromagnetic calorimeter and the (outer) hadronic calorimeter. In this context, a calorimeter is a device which attempts to absorb energetic particles in some material, and then measures the amount of energy deposited in the process. All calorimeters in \textsc{atlas} are sampling calorimeters, meaning that the absorbing material is divided into layers, with some form of detector inserted between the layers, which sample the flow of energy through the material at various depths. In the barrel section of the hadronic calorimeter---the tile calorimeter, the absorbing material is steel, and the sensitive layers are scintillators, a material that luminesces when exposed to ionising radiation. The light produced there is then guided to photomultipliers for detection.

The remaining calorimeters, which covers both the EM calorimeters and the endcap hadronic calorimeters, are all of a similar design, which has liquid argon as the sensitive material, and are thus called LAr calorimeters. As with the straw detectors, activity in the liquid argon layers are detected by an electrode, which picks up the ionisation left by passing energetic particles. In the barrel LAr calorimeters, the absorbing material is lead encased in thin steel layers. The endcap LAr calorimeters, however, are subjected to much stronger particle fluxes, lead is replaced with copper, and in some cases tungsten, which is more resistant to high temperatures.

\subsection{The electromagnetic calorimeter}

The EM calorimeters are particularly important tools for detecting photons. As such, we devote an entire sub--section to their description.

\begin{figure}[htp]
\begin{minipage}[b]{.69\textwidth}
\begin{infilsf}\footnotesize
\begin{tikzpicture}[x=\textwidth*.95/3.5,y=\textwidth*.95/3.5*1.2]

\tikzset{photon/.style={decorate,decoration={snake,amplitude=1}},
         vert/.style={fill=black,circle,draw,inner sep=0pt,minimum size=2},
         grid/.style={draw=kugray!50}
         }
         
\draw[grid] (0,-.7) -- +(0,1.5) (1,-.7) -- +(0,1.5) (2,-.7) -- +(0,1.5) (3,-.7) -- +(0,1.5);
\draw (-.2,-.7) -- (0,-.7) node[below] {0} -- +(0,.05) +(0,0) -- ++(1,0) node[below] {1} -- +(0,.05) +(0,0) -- ++(1,0) node[below] {2} -- +(0,0.05) +(0,0) -- ++(1,0) node[below] {3} -- +(0,.05) +(0,0) -- ++(.3,0);
\draw (3.2,-.7) +(0,-1.1em) node[below left] {Depth in radiation lengths};

\draw (-.1,0) -- (0.4,0) node[vert] {} -- (1.7,-.2) node[vert] {} -- (2.7,-0.1) node[vert] {} -- (3.2,0) ;
\draw[photon] (0.4,0) -- (1.3,.2) node[vert] {};
\draw (1.3,.2) -- node[above] {\tiny$+$} (2.5,.5) node[vert] {} -- node[below] {\tiny$+$} (3.2,.4);
\draw[photon] (2.5,.5) -- (3.2,.7);
\draw (1.3,.2) -- (2.3,.1) node[vert] {} -- (3.2,.2);
\draw[photon] (2.3,.1) -- (3.2,.05) (2.7,-.1) -- (3.2,-.1) (1.7,-.2) -- (2.3,-.3);
\draw (3.2,-.3) -- (2.3,-.3) node[vert] {} -- node[below] {\tiny$+$} (3.2,-.5);



\end{tikzpicture}
\end{infilsf}
\end{minipage}\hfill
\begin{minipage}[b]{.3\textwidth}
\caption{A schematic description of an EM shower developing in an absorbing material, adapted from \cite{fernow:sampcal}. Here, a wavy line indicates a photon, a straight line indicates an electron, and a straight line with a `$+$' a positron. At each split, the two resulting particles carry away half the energy of the original particle. In a sampling calorimeter, a sensitive layer is typically inserted at intervals of one radiation length. \textsc{Atlas}' LAr calorimeters measure the magnitude of ionisation that is left in the liquid argon by the passage of the particle shower.
\label{emshower}
}
\end{minipage}
\end{figure}

In the presence of matter, high--energy photons loose energy primarily through pair production, while electrons loose energy primarily through bremsstrahlung. The typical lengths travelled by both types of particle before undergoing these respective processes---their radiation lengths---depend on the material traversed, but are roughly equal to one another \cite{fernow:sampcal}. As illustrated in fig.~\ref{emshower}, both types of particles will undergo the same sort of evolution as they travel through an absorbing material, splitting into pairs of particles, each with a fraction of the energy of the parent particle, until the daughter particles no longer have sufficient energy to penetrate the absorbing material. Within the calorimeter, this cascade of particles forms a shower structure, an example of which is sketched in fig.~\ref{shower}. Thus, we can measure the energy of the original particle both by how deep into the absorbing material it penetrates, and by how many daughter particles it produces. In a homogeneous calorimeter, such as the one in CMS, where a single material is used both for absorbtion and detection, only the amount of activity produced by a hit to the calorimeter is measured. In a sampling calorimeter, where the absorbing material is passive, and activity is sampled by inserting sensitive layers into the absorbing material at intervals, can measure both the penetration depth, since the thickness of absorbing material in front of the deepest activated sensitive layer is known, and showering activity, by using proportional detectors in the sensitive layers. The radiation length is then a natural choice for the thickness of each layer of absorber.

\begin{figure}[htp]
\begin{minipage}[t]{.4\textwidth}
\caption{Several figures, from \cite{atlasweb}, that illustrate the structure and use of the LAr calorimeters. These are sampling calorimeters, which have an absorbing medium (primarily lead and steel) with layers of detecting medium (liquid argon) inserted regularly to measure particle flux. In \textsc{atlas}' LAr calorimeters, rather than having flat layers, the absorbing and sensitive materials are interleaved in an accordion shape, visible in \subcaptionref{larpic} and \subcaptionref{shower}, which allows the detector electronics to be inserted along the gaps between the absorbing plates. Thus, the calorimeters do not need to be interrupted by non--sensitive space for signalling connections.
\label{calostruc}}
\end{minipage}\hfill\begin{minipage}[t]{.57\textwidth}
\phantom{p}

\phantom{p}

    \begin{center}
    \includegraphics[width=\textwidth]{larpic}\makebox[0em][r]{\textcolor{natgreen}{\rule{\textwidth}{1pt}}}
    
    \contsubcaption{A section of the LAr calorimeter. \label{larpic}}
    \end{center}
\end{minipage}

\noindent\begin{minipage}[b]{.4\textwidth}
\begin{center}
    \includegraphics[width=\textwidth]{shower}\makebox[0em][r]{\textcolor{natgreen}{\rule{\textwidth}{1pt}}}

    \contsubcaption{Illustration of a particle shower within the LAr calorimeter. \label{shower}}
    \end{center}
\end{minipage}\hfill\begin{minipage}[b]{.57\textwidth}

    \begin{center}    \includegraphics[width=1\textwidth]{figures/CaloDiag.png}
    \contsubcaption{Schematic showing the placement of the LAr calorimeters in \textsc{atlas}.}
    \end{center}
\end{minipage}
\end{figure}

\Atlas{}' barrel LAr calorimeters are divided into three layers, which are split into readout bins as illustrated in fig.~\ref{caldiv}. The first layer is very finely divided in the $\eta$ direction, and its readout cells will on occasion be referred to as `strips' in the `strip layer', as opposed to the `cells' in the other two layers. The detectors function on the same sort of principle as the straw detectors in the inner detector: charged particles leave ionisation trails in the liquid argon, which is picked up on plates suspended in the middle of the LAr gap, which carry a high voltage charge. Each cell functions as a single proportional counter. Thus, the amount of energy deposited in each cell of each layer of the calorimeter is measured by the amount of showering activity in the sensitive layers of that cell.

\begin{figure}[htp]
\begin{minipage}[b]{.69\textwidth}
\includegraphics[width=\textwidth]{caldiv}
\end{minipage}
\begin{minipage}[b]{.3\textwidth}
\caption{The division of the EM calorimeter into detecting cells \cite{egede}. The first layer is divided into thin strips for the greatest resolution in the $\eta$ direction. The second layer is divided into roughly square cells, and comprises the bulk of the depth of the detector. The last layer is presumed to only be reached by the most energetic particles, and can have a coarser division without loosing resolution. This diagram is of the calorimeter at $\eta = 0$, closest to the interaction point. At higher $|\eta|$, the towers are angled so that they are still pointed toward the interaction point.
\label{caldiv}}
\end{minipage}
\end{figure}

The absorbing material in the barrel LAr calorimeter are accordion shaped sheets of lead held in iron sheets, visible in fig.~\ref{calostruc}, inserted radially into the detector, so that radiation leaving the detector is still faced with broadly uniform layers of absorbing material, while gaps between them, which are filled with liquid argon and hold the detector plates, still run contiguously throughout the depth of the calorimeter. Connections between the detectors in the inner part of the calorimeter and the readout electronics outside the calorimeter can be run through these gaps without needing to create holes in the detectors coverage.

The division of the calorimeter into layers gives us some resolution in depth when attempting to ascertain the shape of a shower. Showers initiated by types of particles other than photons or electrons evolve in a slightly different way, which allows us discern the source of a shower by examining its shape. Additionally, with this shower shape information, we can extrapolate the direction from which a particle entered the calorimeter, which is of particular importance when attempting to trace the origin of unconverted photons, which otherwise leave no trail in the detector.

There is quite a bit of material between the calorimeter and the interaction point, any of which may interact with a particle in the same way as the absorbing material in the calorimeter. To attempt to correct for this, the first active layer, called the presampler, sits ahead of the first absorbing layer. Photons that undergo pair production sufficiently deep in the detector for the tracker to resolve at least one of the resulting (anti--)electrons are treated as a separate object type, namely converted photons. Identifying all the detector signatures that may have been left by photons is the first major (relevant) step in analysing the detector's output.

\subsection{Photon identification}
In broad terms, we may view information coming from the detector in terms of energy clusters in the calorimeters or tracks in the inner detector. Assigning calorimeter deposits to discrete clusters is a task that can be approached a number of ways, both subtle and complex, however at the photon ID stage, the straightforward sliding window method is used \cite{atlascluster:sw}. A rectangle of fixed size is defined by combining calorimeter cells, and the energy within is summed. This `window' is then `slid' across the entire calorimeter, and a cluster candidate is proclaimed whenever the energy within the window reaches a local maximum above a predefined threshold value.

Assuming that every charged particle leaves both a track in the inner detector and a cluster in the calorimeter, we interpret every cluster in the calorimeter that can not be associated with a track as the signature of a neutral particle. Each of these neutral particles might be an unconverted photon, and so every cluster without an associated track is added to the list of photon candidates.

As previously mentioned, converted photons are photons that undergo pair production ``in flight.'' As such, we expect the detector signature of such events to be a positively and a negatively charged track with a common vertex away from the interaction point. Any set of measurements that match this signature may be added to the sequence straight away, however since track reconstruction is an imperfect process, especially in the straw detector, we include also any electron candidate, whose track is reconstructed solely from hits in the straw detector.

These steps form a selection algorithm, which creates a list of photon candidates for every event in \atlas{} \cite{phorec}.

This set will of course contain many events that were not photons: $\pi^0$ mesons also create calorimeter hits with no associated track, and QCD jets can mimic converted photons, among other possibilities. To attempt to sort genuine photon events from impostor events, we study the shape of the shower left in the calorimeter. To that end, we define the following shower shape variables:

\begin{itemize}
\item $R_\text{had}$, the ratio of energy deposited in the hadronic calorimeter to the cluster energy in the EM calorimeter. Hadronic showers are expected to penetrate deeper into the hadronic calormieter than EM showers.
\item In the middle EM calorimeter layer, non-EM showers spread wider than electromagnetic ones. The variables that measure the shape of the shower in this layer are:
\begin{itemize}
\item $R_\eta$, the ratio in $\eta$ of cell energies in 3 $\times$ 7 versus 7 $\times$ 7 cells.
\item $R_\phi$, the ratio in $\phi$ of cell energies in 3 $\times$ 7 versus 7 $\times$ 7 cells.
\item $w_{\eta 2}$, the width of the shower in the $\eta$ direction.
\end{itemize}
\item The strip layer, with its greater resolution in $\eta$, can pick out some of the internal structure of a jet. Hadron showers tend to show more than one maximum. Variables that measure the shape in the strip layer are:
\begin{itemize}
\item $w_{s3}$, the shower width for three strips around the maximum strip.
\item $w_{s\text{ tot}}$, the total lateral shower width in the strip layer.
\item $F_\text{side}$, the faction of energy deposited outside a core of 3 central strips, but within 7 strips.
\item $\Delta E$, the difference in energy of the strip with the second largest energy deposited and the strip with the smallest energy deposited between the two leading strips.
\item $E_\text{ratio}$, the ratio of the energy difference associated with the
largest and second largest energy deposits, over the sum of these energies.
\end{itemize}
\end{itemize}

Cuts made in these variables form the selection criteria that are used to divide the sample of photon candidates into those that are to be considered actual or impostor photons. The full list of variables above forms the tight selection criteria, while a shorter list, consisting of $R_\text{had}$, $R_\eta$ and $w_{\eta2}$ form the loose selection criteria. Separate cut values in these variables exist for different $\eta$ ranges. A complete description is available in \cite{Carminati}.

\subsection{Triggering and readout}
While in full operation for the 2012, 8~TeV run, the LHC delivered a bunch crossing in \textsc{atlas}' interaction point every 50~ns, or 20 million events per second. Reading out the whole detector produces 1.6~MB of information, which, if the detector were read out completely with every crossing, would produce a data rate of 34~TB/s.\footnote{For perspective, that is approximately equal to the estimated global IP traffic rate in 2015, according to \cite{wolframip}.} However, since only a fraction of these collisions produce interesting physics events, we can reduce the data rate to less prohibitive levels simply by not recording data from collision that do not produce interesting events. To accomplish this, we need a system that examines events in the detector as they occur, and trigger recording whenever it sees an interesting event. In \textsc{atlas}, this trigger system has three levels, which are described in detail in \cite{detectorpaper}. 

The level--1 trigger examines events in real time as they occur in the detector. To do so, the trigger logic is run on specialised hardware built in to the detectors. As a result, each trigger only has access to information from the detector its hardware is attached to. Chalorimeter triggers, for example, do not have access to information from the tracking system at the first trigger level. Also, computationally intensive tasks, such as track reconstruction, can not be completed in the window of time available to the level--1 trigger, and so are also not available. The next trigger level, level--2, is run on the full set of information from an event, on those events which pass the level--1 filter. The final trigger level works with fully reconstructed events and derived physical observables. This requires more time and processing power than is available at the previous levels, but it also identifies interesting events with the same quality of information as will be used in the subsequent analysis. All three triggers in combination cuts the final event rate to 300 events per second, with a peak rate of 600 events.

For this thesis, we shall use data taken during the 8~TeV run in 2012. The amount of data taken at any given time depends on the conditions of the beam, which can be summarised in the instantaneous luminosity, and the conditions of the detector, which may only capture a fraction of the events produced at any given time. Figure~\ref{intlumi} gives the distribution of integrated and captured luminosity over the course of the year.

\begin{figure}[htp]
\begin{minipage}[b]{.57\textwidth}
\hspace{-1em}\includegraphics[width=\textwidth]{figures/intlumi}
\end{minipage}\hfill\begin{minipage}[b]{.4\textwidth}
\caption{A plot \cite{publiclumi} showing the integrated luminosity delivered by the LHC (green), recorded by \atlas{} (yellow), and fulfilling data quality criteria (blue), over the course of the 8 TeV run in 2012.
\label{intlumi}}
\end{minipage}
\end{figure}

Unfortunately, the triggers that are implemented in \atlas{} do not guarantee that the event rate remains within the technical limitations of the readout system. To stay within those limits, \atlas{} prescales certian triggers, when they originate from a trigger that produces more events than it is considered worth keeping,\footnote{Explaining how it is decided whether data is worth keeping would veer into a discussion of \atlas{} internal politics, which is a topic beyond the scope of this thesis.} which means simply that a fraction of event that pass a trigger are not recorded. The diphoton channel is important to the search for the Higgs boson, however, so the triggers that produce diphotons events have not been prescaled.


\subsection{Triggering and data collection}
While in full operation for the 2012, 8~TeV run, the LHC delivered a bunch crossing in \textsc{atlas}' interaction point every 50~ns. Reading out the whole detector produces 1.6~MB of information, which, if the detector were read out completely with every crossing, would produce a data rate of 34~TB/s.\footnote{For perspective, that is approximately equal to the estimated global IP traffic rate in 2015, according to \cite{wolframip}.} However, since only a fraction of these collisions produce interesting physics events, we can reduce the data rate to less prohibitive levels simply by not recording data from collision that do not produce interesting events. To accomplish this, we need a system that examines events in the detector as they occur, and trigger recording whenever it sees an interesting event. In \textsc{atlas}, this trigger system has three levels, which are described in detail in \cite{detectorpaper}. 

The level--1 trigger genuinely does examine events as they occur in the detector. To do so, it runs on specialised hardware built in to the detectors, and as a result, it only has access to the raw information from the detectors to which it is attached. This means, for example, that track reconstruction is not available when the level--1 trigger deicdes whether or not to record an event. The next trigger level, level--2, is run on the full set of information on an event, on those events which pass the level--1 filter. The final trigger level works with fully reconstructed events and derived physical observables. This requres more time and processing power than is available at the previous levels, but it also identifies interesting events with the same quality of information as will be used in the subsequent analysis. All three triggers in combination cuts the final event rate to 300 events per second, with a peak rate of 600 events.

For this thesis, we shall use data taken during the 8~TeV run in 2012. The amount of data taken at any given time depends on the condiitions of the beam, which can be summarised in the instantaneous luminosity, and the conditions of the detector, which may only capture a fraction of the events produced at any given time. Figure~\ref{intlumi} gives the distribution of integrated and captured luminosity over the course of the year.

\begin{figure}[htp]
\begin{minipage}[b]{.69\textwidth}
\hspace{-1em}\includegraphics[width=\textwidth]{figures/intlumi}
\end{minipage}\hfill\begin{minipage}[b]{.3\textwidth}
\caption{A plot showing the integrated luminosity delivered by the LHC (green), recorded by ATLAS (yellow), and fulfilling data quality criteria (blue), over the course of the 8 TeV run in 2012 \cite{publiclumi}.
\label{intlumi}}
\end{minipage}
\end{figure}

Unfortunately, the triggers that \atlas{} implements do not guarantee that the event rate remains within the technical limitations of the readout system. to stay within those limits, \atlas{} removes a fraction of the events that did pass the triggers, when they originate from a trigger that produces more events than it is considered worth keeping.\footnote{Explaining how it is decided whether data is worth keeping would veer into a discussion of \atlas{} internal politics, which is a topic beyond the scope of this thesis.} The diphoton channel is important to the search for the Higgs boson, however, so the triggers that produce diphotons events are not prescaled in this fashion.


\chapter{Data preparation}

For the present analysis, we will use events that passed the \texttt{2g40\_loose} level--1 trigger, which requires that the EM calorimeter reports two hits with at least 40~GeV of transverse energy that pass the loose selection criteria, described in the previous chapter.

The datasets have been retrieved in the \texttt{NTUP\_PHOTON} format, which is streamlined to contain information relevant to photon analyses, and easily readable by \textsc{root}. The dataset used contains events corresponding to 18.301~fb$^{-1}$ of integrated luminosity.

On each of the prospective photons in this dataset, we impose a series of selection criteria:

\begin{itemize}
\item \textbf{otx and phoCloan cut:} Object quality cuts, which cut out events too close to non-functioning or noisy detector elements, and events taken while the detector was in a non-optimal state.
\item \textbf{ID cut:} Objects that did not pass photon identification, or do not satisfy the loose selection criteria after reconstruction, are eliminated.
\item \textbf{kinematics cut:} Ensures that objects do not have $|\eta|$ greater than 2.37, which is the forward limit of the first layer of the EM calorimeter, or in the range between 1.37 and 1.52, which is the transition region between the barrel and endcap calorimeters. Also ensures $E_T$ greater than 50~GeV, which clears the turn--on curve of the \texttt{2g40\_loose} trigger.
\item \textbf{N\_events cut:} Ensures that each event has at least two photons that pass the above criteria.
%\item \textbf{PV cut:} Ensures that the photon pair selected for analysis have the same primary vertex, and that that vertex has at least three tracks associated with it.

\end{itemize}

The number of objects remaining at each step of the cut procedure is plotted in figure~\ref{cutflow}

\begin{figure}[htp]
\begin{minipage}[b]{.69\textwidth}
\begin{infilsf} \tiny
\begin{tikzpicture}[x=.092\textwidth,y=.092\textwidth]
\pgfdeclareplotmark{cross} {
\pgfpathmoveto{\pgfpoint{-0.3\pgfplotmarksize}{\pgfplotmarksize}}
\pgfpathlineto{\pgfpoint{+0.3\pgfplotmarksize}{\pgfplotmarksize}}
\pgfpathlineto{\pgfpoint{+0.3\pgfplotmarksize}{0.3\pgfplotmarksize}}
\pgfpathlineto{\pgfpoint{+1\pgfplotmarksize}{0.3\pgfplotmarksize}}
\pgfpathlineto{\pgfpoint{+1\pgfplotmarksize}{-0.3\pgfplotmarksize}}
\pgfpathlineto{\pgfpoint{+0.3\pgfplotmarksize}{-0.3\pgfplotmarksize}}
\pgfpathlineto{\pgfpoint{+0.3\pgfplotmarksize}{-1.\pgfplotmarksize}}
\pgfpathlineto{\pgfpoint{-0.3\pgfplotmarksize}{-1.\pgfplotmarksize}}
\pgfpathlineto{\pgfpoint{-0.3\pgfplotmarksize}{-0.3\pgfplotmarksize}}
\pgfpathlineto{\pgfpoint{-1.\pgfplotmarksize}{-0.3\pgfplotmarksize}}
\pgfpathlineto{\pgfpoint{-1.\pgfplotmarksize}{0.3\pgfplotmarksize}}
\pgfpathlineto{\pgfpoint{-0.3\pgfplotmarksize}{0.3\pgfplotmarksize}}
\pgfpathclose
\pgfusepathqstroke
}
\pgfdeclareplotmark{cross*} {
\pgfpathmoveto{\pgfpoint{-0.3\pgfplotmarksize}{\pgfplotmarksize}}
\pgfpathlineto{\pgfpoint{+0.3\pgfplotmarksize}{\pgfplotmarksize}}
\pgfpathlineto{\pgfpoint{+0.3\pgfplotmarksize}{0.3\pgfplotmarksize}}
\pgfpathlineto{\pgfpoint{+1\pgfplotmarksize}{0.3\pgfplotmarksize}}
\pgfpathlineto{\pgfpoint{+1\pgfplotmarksize}{-0.3\pgfplotmarksize}}
\pgfpathlineto{\pgfpoint{+0.3\pgfplotmarksize}{-0.3\pgfplotmarksize}}
\pgfpathlineto{\pgfpoint{+0.3\pgfplotmarksize}{-1.\pgfplotmarksize}}
\pgfpathlineto{\pgfpoint{-0.3\pgfplotmarksize}{-1.\pgfplotmarksize}}
\pgfpathlineto{\pgfpoint{-0.3\pgfplotmarksize}{-0.3\pgfplotmarksize}}
\pgfpathlineto{\pgfpoint{-1.\pgfplotmarksize}{-0.3\pgfplotmarksize}}
\pgfpathlineto{\pgfpoint{-1.\pgfplotmarksize}{0.3\pgfplotmarksize}}
\pgfpathlineto{\pgfpoint{-0.3\pgfplotmarksize}{0.3\pgfplotmarksize}}
\pgfpathclose
\pgfusepathqfillstroke
}
\pgfdeclareplotmark{newstar} {
\pgfpathmoveto{\pgfqpoint{0pt}{\pgfplotmarksize}}
\pgfpathlineto{\pgfqpointpolar{44}{0.5\pgfplotmarksize}}
\pgfpathlineto{\pgfqpointpolar{18}{\pgfplotmarksize}}
\pgfpathlineto{\pgfqpointpolar{-20}{0.5\pgfplotmarksize}}
\pgfpathlineto{\pgfqpointpolar{-54}{\pgfplotmarksize}}
\pgfpathlineto{\pgfqpointpolar{-90}{0.5\pgfplotmarksize}}
\pgfpathlineto{\pgfqpointpolar{234}{\pgfplotmarksize}}
\pgfpathlineto{\pgfqpointpolar{198}{0.5\pgfplotmarksize}}
\pgfpathlineto{\pgfqpointpolar{162}{\pgfplotmarksize}}
\pgfpathlineto{\pgfqpointpolar{134}{0.5\pgfplotmarksize}}
\pgfpathclose
\pgfusepathqstroke
}
\pgfdeclareplotmark{newstar*} {
\pgfpathmoveto{\pgfqpoint{0pt}{\pgfplotmarksize}}
\pgfpathlineto{\pgfqpointpolar{44}{0.5\pgfplotmarksize}}
\pgfpathlineto{\pgfqpointpolar{18}{\pgfplotmarksize}}
\pgfpathlineto{\pgfqpointpolar{-20}{0.5\pgfplotmarksize}}
\pgfpathlineto{\pgfqpointpolar{-54}{\pgfplotmarksize}}
\pgfpathlineto{\pgfqpointpolar{-90}{0.5\pgfplotmarksize}}
\pgfpathlineto{\pgfqpointpolar{234}{\pgfplotmarksize}}
\pgfpathlineto{\pgfqpointpolar{198}{0.5\pgfplotmarksize}}
\pgfpathlineto{\pgfqpointpolar{162}{\pgfplotmarksize}}
\pgfpathlineto{\pgfqpointpolar{134}{0.5\pgfplotmarksize}}
\pgfpathclose
\pgfusepathqfillstroke
}
\definecolor{c}{rgb}{1,1,1};
\draw [color=c, fill=c] (0,0) rectangle (10,6.80516);
\draw [color=c, fill=c] (1,0.680516) rectangle (9.95,6.73711);
\definecolor{c}{rgb}{0,0,0};
\draw [c] (1,0.680516) -- (1,6.73711) -- (9.95,6.73711) -- (9.95,0.680516) -- (1,0.680516);
\definecolor{c}{rgb}{1,1,1};
\draw [color=c, fill=c] (1,0.680516) rectangle (9.95,6.73711);
\definecolor{c}{rgb}{0,0,0};
\draw [c] (1,0.680516) -- (1,6.73711) -- (9.95,6.73711) -- (9.95,0.680516) -- (1,0.680516);
\colorlet{c}{natgreen};
\draw [c] (1.55938,5.10106) -- (1.55938,5.10157);
\draw [c] (1.55938,5.10157) -- (1.55938,5.10208);
\draw [c] (1,5.10157) -- (1.55938,5.10157);
\draw [c] (1.55938,5.10157) -- (2.11875,5.10157);
\definecolor{c}{rgb}{0,0,0};
\colorlet{c}{natgreen};
\draw [c] (2.67813,5.0948) -- (2.67813,5.0953);
\draw [c] (2.67813,5.0953) -- (2.67813,5.09581);
\draw [c] (2.11875,5.0953) -- (2.67813,5.0953);
\draw [c] (2.67813,5.0953) -- (3.2375,5.0953);
\definecolor{c}{rgb}{0,0,0};
\colorlet{c}{natgreen};
\draw [c] (3.79688,5.04737) -- (3.79688,5.04788);
\draw [c] (3.79688,5.04788) -- (3.79688,5.04838);
\draw [c] (3.2375,5.04788) -- (3.79688,5.04788);
\draw [c] (3.79688,5.04788) -- (4.35625,5.04788);
\definecolor{c}{rgb}{0,0,0};
\colorlet{c}{natgreen};
\draw [c] (4.91563,3.57142) -- (4.91563,3.57183);
\draw [c] (4.91563,3.57183) -- (4.91563,3.57225);
\draw [c] (4.35625,3.57183) -- (4.91563,3.57183);
\draw [c] (4.91563,3.57183) -- (5.475,3.57183);
\definecolor{c}{rgb}{0,0,0};
\colorlet{c}{natgreen};
\draw [c] (6.03438,1.14305) -- (6.03438,1.14321);
\draw [c] (6.03438,1.14321) -- (6.03438,1.14337);
\draw [c] (5.475,1.14321) -- (6.03438,1.14321);
\draw [c] (6.03438,1.14321) -- (6.59375,1.14321);
\definecolor{c}{rgb}{0,0,0};
\colorlet{c}{natgreen};
\draw [c] (7.15312,0.8555) -- (7.15312,0.855634);
\draw [c] (7.15312,0.855634) -- (7.15312,0.855767);
\draw [c] (6.59375,0.855634) -- (7.15312,0.855634);
\draw [c] (7.15312,0.855634) -- (7.7125,0.855634);
\definecolor{c}{rgb}{0,0,0};
\colorlet{c}{natgreen};
\draw [c] (7.7125,0.766811) -- (8.27188,0.766811);
\draw [c] (8.27188,0.766811) -- (8.83125,0.766811);
\definecolor{c}{rgb}{0,0,0};
\colorlet{c}{natgreen};
\draw [c] (8.83125,0.765055) -- (9.39062,0.765055);
\draw [c] (9.39062,0.765055) -- (9.95,0.765055);
\definecolor{c}{rgb}{0,0,0};
\draw [c] (1,0.680516) -- (9.95,0.680516);
\draw [anchor= west] (1.55938,0.31712) node[color=c, rotate=-20]{All};
\draw [anchor= west] (2.67813,0.31712) node[color=c, rotate=-20]{otx\_cut};
\draw [anchor= west] (3.79688,0.31712) node[color=c, rotate=-20]{phoClean\_cut};
\draw [anchor= west] (4.91563,0.31712) node[color=c, rotate=-20]{ID\_cut};
\draw [anchor= west] (6.03438,0.31712) node[color=c, rotate=-20]{kinematics\_cut};
\draw [anchor= west] (7.15312,0.31712) node[color=c, rotate=-20]{N\_events};
\draw [anchor= west] (8.27188,0.31712) node[color=c, rotate=-20]{PV\_cut};
\draw [anchor= west] (9.39062,0.31712) node[color=c, rotate=-20]{selected\_photons\_cut};
\draw [c] (1,0.863234) -- (1,0.680516);
\draw [c] (2.11875,0.863234) -- (2.11875,0.680516);
\draw [c] (3.2375,0.863234) -- (3.2375,0.680516);
\draw [c] (4.35625,0.863234) -- (4.35625,0.680516);
\draw [c] (5.475,0.863234) -- (5.475,0.680516);
\draw [c] (6.59375,0.863234) -- (6.59375,0.680516);
\draw [c] (7.7125,0.863234) -- (7.7125,0.680516);
\draw [c] (8.83125,0.863234) -- (8.83125,0.680516);
\draw [c] (9.95,0.863234) -- (9.95,0.680516);
\draw [c] (1,0.680516) -- (1,6.73711);
\draw [anchor= east] (-0.12,6.73711) node[color=c, rotate=90]{Number of accepted events};
\draw [c] (1.267,0.705045) -- (1,0.705045);
\draw [c] (1.1335,0.941649) -- (1,0.941649);
\draw [c] (1.1335,1.17825) -- (1,1.17825);
\draw [c] (1.1335,1.41486) -- (1,1.41486);
\draw [c] (1.1335,1.65146) -- (1,1.65146);
\draw [c] (1.267,1.88806) -- (1,1.88806);
\draw [c] (1.1335,2.12467) -- (1,2.12467);
\draw [c] (1.1335,2.36127) -- (1,2.36127);
\draw [c] (1.1335,2.59788) -- (1,2.59788);
\draw [c] (1.1335,2.83448) -- (1,2.83448);
\draw [c] (1.267,3.07108) -- (1,3.07108);
\draw [c] (1.1335,3.30769) -- (1,3.30769);
\draw [c] (1.1335,3.54429) -- (1,3.54429);
\draw [c] (1.1335,3.7809) -- (1,3.7809);
\draw [c] (1.1335,4.0175) -- (1,4.0175);
\draw [c] (1.267,4.2541) -- (1,4.2541);
\draw [c] (1.1335,4.49071) -- (1,4.49071);
\draw [c] (1.1335,4.72731) -- (1,4.72731);
\draw [c] (1.1335,4.96391) -- (1,4.96391);
\draw [c] (1.1335,5.20052) -- (1,5.20052);
\draw [c] (1.267,5.43712) -- (1,5.43712);
\draw [c] (1.1335,5.67373) -- (1,5.67373);
\draw [c] (1.1335,5.91033) -- (1,5.91033);
\draw [c] (1.1335,6.14693) -- (1,6.14693);
\draw [c] (1.1335,6.38354) -- (1,6.38354);
\draw [c] (1.267,6.62014) -- (1,6.62014);
\draw [c] (1.267,0.705045) -- (1,0.705045);
\draw [c] (1.267,6.62014) -- (1,6.62014);
\draw [anchor= east] (0.95,0.705045) node[color=c, rotate=0]{0};
\draw [anchor= east] (0.95,1.88806) node[color=c, rotate=0]{10};
\draw [anchor= east] (0.95,3.07108) node[color=c, rotate=0]{20};
\draw [anchor= east] (0.95,4.2541) node[color=c, rotate=0]{30};
\draw [anchor= east] (0.95,5.43712) node[color=c, rotate=0]{40};
\draw [anchor= east] (0.95,6.62014) node[color=c, rotate=0]{50};
\draw [anchor=base west] (1,6.76092) node[color=c, rotate=0]{$\times10^{6}$};
\colorlet{c}{natgreen};
\draw [c] (1.55938,5.10106) -- (1.55938,5.10157);
\draw [c] (1.55938,5.10157) -- (1.55938,5.10208);
\draw [c] (1,5.10157) -- (1.55938,5.10157);
\draw [c] (1.55938,5.10157) -- (2.11875,5.10157);
\definecolor{c}{rgb}{0,0,0};
\colorlet{c}{natgreen};
\draw [c] (2.67813,5.0948) -- (2.67813,5.0953);
\draw [c] (2.67813,5.0953) -- (2.67813,5.09581);
\draw [c] (2.11875,5.0953) -- (2.67813,5.0953);
\draw [c] (2.67813,5.0953) -- (3.2375,5.0953);
\definecolor{c}{rgb}{0,0,0};
\colorlet{c}{natgreen};
\draw [c] (3.79688,5.04737) -- (3.79688,5.04788);
\draw [c] (3.79688,5.04788) -- (3.79688,5.04838);
\draw [c] (3.2375,5.04788) -- (3.79688,5.04788);
\draw [c] (3.79688,5.04788) -- (4.35625,5.04788);
\definecolor{c}{rgb}{0,0,0};
\colorlet{c}{natgreen};
\draw [c] (4.91563,3.57142) -- (4.91563,3.57183);
\draw [c] (4.91563,3.57183) -- (4.91563,3.57225);
\draw [c] (4.35625,3.57183) -- (4.91563,3.57183);
\draw [c] (4.91563,3.57183) -- (5.475,3.57183);
\definecolor{c}{rgb}{0,0,0};
\colorlet{c}{natgreen};
\draw [c] (6.03438,1.14305) -- (6.03438,1.14321);
\draw [c] (6.03438,1.14321) -- (6.03438,1.14337);
\draw [c] (5.475,1.14321) -- (6.03438,1.14321);
\draw [c] (6.03438,1.14321) -- (6.59375,1.14321);
\definecolor{c}{rgb}{0,0,0};
\colorlet{c}{natgreen};
\draw [c] (7.15312,0.8555) -- (7.15312,0.855634);
\draw [c] (7.15312,0.855634) -- (7.15312,0.855767);
\draw [c] (6.59375,0.855634) -- (7.15312,0.855634);
\draw [c] (7.15312,0.855634) -- (7.7125,0.855634);
\definecolor{c}{rgb}{0,0,0};
\colorlet{c}{natgreen};
\draw [c] (7.7125,0.766811) -- (8.27188,0.766811);
\draw [c] (8.27188,0.766811) -- (8.83125,0.766811);
\definecolor{c}{rgb}{0,0,0};
\colorlet{c}{natgreen};
\draw [c] (8.83125,0.765055) -- (9.39062,0.765055);
\draw [c] (9.39062,0.765055) -- (9.95,0.765055);
\definecolor{c}{rgb}{0,0,0};
\draw [anchor= west] (1.55938,5.12879) node[color=c, rotate=90]{3.71636e+07};
\draw [anchor= west] (2.67813,5.12253) node[color=c, rotate=90]{3.71106e+07};
\draw [anchor= west] (3.79688,5.0751) node[color=c, rotate=90]{3.67097e+07};
\draw [anchor= west] (4.91563,3.59905) node[color=c, rotate=90]{2.42328e+07};
\draw [anchor= west] (6.03438,1.17043) node[color=c, rotate=90]{3.70375e+06};
\draw [anchor= west] (7.15312,0.882854) node[color=c, rotate=90]{1.27291e+06};
\draw [anchor= west] (8.27188,0.794031) node[color=c, rotate=90]{522099};
\draw [anchor= west] (9.39062,0.792275) node[color=c, rotate=90]{507255};
\end{tikzpicture}

\end{infilsf}
\end{minipage}\hfill\begin{minipage}[b]{.3\textwidth}
\caption{A cutflow diagram, showing how many objects remain in the dataset after each of the selection criteria are imposed. The final number of photons is (something I sould be able to dig up somewhere).
\label{cutflow}}
\end{minipage}
\end{figure}

What remains after these cuts have been applied is a purer sample of photons than we had before, however the sample will still contain a background of events that do not come from the processes that we wish to study. An estimate of this background is required.

\section{Data driven background estimation}
The background that remains in the signal sample after these criteria have been applied can be estimated in a number of ways. In chapter~\ref{ch.mc}, monte  carlo samples that simulate several physical processes that act as background to diphoton events were presented. Here, however, we shall attempt to quantify the magnitude of the background by examining the data.

\subsection{The ABCD method}
Also known as the two--dimensional sideband method, illustrated in figure~\ref{abcd}.

\begin{figure}[hbp]
  \includegraphics[width=\textwidth]{figures/sideband}
  \caption{Illustrating the two--step ABCD method using the `tight' selection criteria and the isolation energy: the full set of diphotons (the L-L sample) is split into four groups---A, B, C and D---according to the discriminating variables for the leading photon. Signal photons are now confined to the A region. The events in the C region can be used to estimate the shape, and the B and D region can be used to estimate the magnitude of the distribution of background events in the signal region. The procedure is then repeated for the subleading partner photons of the events in the A region. This gives an estimate of the distibution of background events in the combined signal (A--A) region. This figure adapted from \cite{fdirect}.}\label{abcd}
\end{figure}

By taking two uncorrelated discriminating variables, we can split the distribution of one of them into two, according to the other discriminating variable, one sample with pure background events, and the other with signal and background events. Assuming that the shape of the distribution of background events is the same in the mixed signal and background sample as it is in the pure background sample, the shape of the distribution in the signal region of the background sample, this is the C region, is the same as the shape of the distribution of background events in the signal region of the mixed sample, the A region. The B and D regions contain only background events, so the magnitude of the background distribution in the A region can be found by comparing the sidebands. For more detail, see \cite{cmsabcd}

For the diphoton sample, we will use the `tight' selection criteria, which were described in chapter~\ref{ch.ex}, and the transverse isolation energy, $E_T^{\text{isol}}$, the energy deposited in the calorimeter in a cone with radius $R\le0.4$, but outside $R\le0.2$, where
\[R=\sqrt{\Delta\phi^2+\Delta\theta^2}.\]
The signal region is cut at $E_T^{\text{isol}}\le3$ GeV. We allow a crosstalk region of 2 GeV, which means the background region is defined with $5\text{ GeV}\le E_T^{\text{isol}}\le25\text{ GeV}$. The distribution of $E_T^{\text{isol}}$ for leading photons and subleading photons in the `A' sample is given in figure~\ref{etiso}.

\begin{figure}[htp]
\begin{minipage}[b]{.69\textwidth}
\hspace{-1em}%\includegraphics[width=\textwidth]{figures/intlumi}
\centering (a very impressive plot.)
\end{minipage}\hfill\begin{minipage}[b]{.3\textwidth}
\caption{The transverse isolation energy $E_T^{\text{isol}}\le3$ for the tight and non--tight photon selection of the leading photons, and for the set of subleading photons with partners in the `A' sample. The non--tight samples have been scaled so that the `D' region contains the same number of events as the `B' region. For both sets of samples, the shapes of the distributions match up match up after the scaling, which supports the assumption that the shape of the background distribution is the same in the signal region as well.
\label{etiso}}
\end{minipage}
\end{figure}

The same procedure is applied to the subleading partner photons to the leading photons in the `A' region, the `A' sample in fig.~\ref{abcd}. This gives us the number of background events in the leading `A' sample and the number of background events in the subleading `A' sample. The final estimated background on the diphoton signal is the estimated fraction of background events in the leading `A' sample multiplied by the `A--A' subleading sample, plus the estimated background in this sample, minus the background fraction in the `A' sample times the estimated background on the `A--A' sample, which was double--counted.

Performing this process for each bin in $M_{\gamma\gamma}$, we obtain the distribution of background events shown in figure~\ref{mggbck}. Note, though, that the plots of $E_T^{\text{isol}}\le3$ above showed the combined distribution for all $M_{\gamma\gamma}$ bins combined.

\begin{figure}[htp]
\begin{minipage}[b]{.69\textwidth}
\hspace{-1em}%\includegraphics[width=\textwidth]{figures/intlumi}
\centering (a very impressive plot.)
\end{minipage}\hfill\begin{minipage}[b]{.3\textwidth}
\caption{The distribution of background events in $M_{\gamma\gamma}$ estimated from data with the ABCD method, along with the distribution of dihpoton events given by the data.
\label{mggbck}}
\end{minipage}
\end{figure}

\section{Total background}
Combining the estimate of the background obtained above with the simulated background samples described in chapter~\ref{ch.mc} ... compare the predicted distributions derived in chapter~\ref{ch.mc}

\chapter{Analysis}\label{ch.an}

In chapter~\ref{ch.mc}, we produced, among others, a sample of Monte Carlo events that gives the SM prediction for the distribution of events. Since we discovered above, that the \atlas{} $\gamma\gamma$ Monte Carlo sample gives a distribution of events that matches data well, we will compare these two samples to asses how well our SM prediction matches data. In doing so, we encounter a few problems.

First, it appears that the procedure which should correct for pileup in the detector simulation procedure has not functioned as intended. During detector simulation, the events produced for this thesis are assigned only a very limited range of values for the number of interactions per bunch crossing. Pileup reweighting produces a weight for each event so that its distribution in a Monte Carlo dataset mathches the distribution found in data.

Since this appears to be a technical glitch, we will substitute a reweighting in the number of reconstructed primary vertices---vertices considered to originate directly from interactions between protons---in each event. The distribution of numbers of primary vertices in the present MC data set is compared to the one found in the \atlas{} MC set in figure~\ref{pvnnone}.

\begin{figure}[htp]
\begin{minipage}[b]{.69\textwidth}
\begin{infilsf} \tiny
\begin{tikzpicture}[x=.092\textwidth,y=.092\textwidth]
\pgfdeclareplotmark{cross} {
\pgfpathmoveto{\pgfpoint{-0.3\pgfplotmarksize}{\pgfplotmarksize}}
\pgfpathlineto{\pgfpoint{+0.3\pgfplotmarksize}{\pgfplotmarksize}}
\pgfpathlineto{\pgfpoint{+0.3\pgfplotmarksize}{0.3\pgfplotmarksize}}
\pgfpathlineto{\pgfpoint{+1\pgfplotmarksize}{0.3\pgfplotmarksize}}
\pgfpathlineto{\pgfpoint{+1\pgfplotmarksize}{-0.3\pgfplotmarksize}}
\pgfpathlineto{\pgfpoint{+0.3\pgfplotmarksize}{-0.3\pgfplotmarksize}}
\pgfpathlineto{\pgfpoint{+0.3\pgfplotmarksize}{-1.\pgfplotmarksize}}
\pgfpathlineto{\pgfpoint{-0.3\pgfplotmarksize}{-1.\pgfplotmarksize}}
\pgfpathlineto{\pgfpoint{-0.3\pgfplotmarksize}{-0.3\pgfplotmarksize}}
\pgfpathlineto{\pgfpoint{-1.\pgfplotmarksize}{-0.3\pgfplotmarksize}}
\pgfpathlineto{\pgfpoint{-1.\pgfplotmarksize}{0.3\pgfplotmarksize}}
\pgfpathlineto{\pgfpoint{-0.3\pgfplotmarksize}{0.3\pgfplotmarksize}}
\pgfpathclose
\pgfusepathqstroke
}
\pgfdeclareplotmark{cross*} {
\pgfpathmoveto{\pgfpoint{-0.3\pgfplotmarksize}{\pgfplotmarksize}}
\pgfpathlineto{\pgfpoint{+0.3\pgfplotmarksize}{\pgfplotmarksize}}
\pgfpathlineto{\pgfpoint{+0.3\pgfplotmarksize}{0.3\pgfplotmarksize}}
\pgfpathlineto{\pgfpoint{+1\pgfplotmarksize}{0.3\pgfplotmarksize}}
\pgfpathlineto{\pgfpoint{+1\pgfplotmarksize}{-0.3\pgfplotmarksize}}
\pgfpathlineto{\pgfpoint{+0.3\pgfplotmarksize}{-0.3\pgfplotmarksize}}
\pgfpathlineto{\pgfpoint{+0.3\pgfplotmarksize}{-1.\pgfplotmarksize}}
\pgfpathlineto{\pgfpoint{-0.3\pgfplotmarksize}{-1.\pgfplotmarksize}}
\pgfpathlineto{\pgfpoint{-0.3\pgfplotmarksize}{-0.3\pgfplotmarksize}}
\pgfpathlineto{\pgfpoint{-1.\pgfplotmarksize}{-0.3\pgfplotmarksize}}
\pgfpathlineto{\pgfpoint{-1.\pgfplotmarksize}{0.3\pgfplotmarksize}}
\pgfpathlineto{\pgfpoint{-0.3\pgfplotmarksize}{0.3\pgfplotmarksize}}
\pgfpathclose
\pgfusepathqfillstroke
}
\pgfdeclareplotmark{newstar} {
\pgfpathmoveto{\pgfqpoint{0pt}{\pgfplotmarksize}}
\pgfpathlineto{\pgfqpointpolar{44}{0.5\pgfplotmarksize}}
\pgfpathlineto{\pgfqpointpolar{18}{\pgfplotmarksize}}
\pgfpathlineto{\pgfqpointpolar{-20}{0.5\pgfplotmarksize}}
\pgfpathlineto{\pgfqpointpolar{-54}{\pgfplotmarksize}}
\pgfpathlineto{\pgfqpointpolar{-90}{0.5\pgfplotmarksize}}
\pgfpathlineto{\pgfqpointpolar{234}{\pgfplotmarksize}}
\pgfpathlineto{\pgfqpointpolar{198}{0.5\pgfplotmarksize}}
\pgfpathlineto{\pgfqpointpolar{162}{\pgfplotmarksize}}
\pgfpathlineto{\pgfqpointpolar{134}{0.5\pgfplotmarksize}}
\pgfpathclose
\pgfusepathqstroke
}
\pgfdeclareplotmark{newstar*} {
\pgfpathmoveto{\pgfqpoint{0pt}{\pgfplotmarksize}}
\pgfpathlineto{\pgfqpointpolar{44}{0.5\pgfplotmarksize}}
\pgfpathlineto{\pgfqpointpolar{18}{\pgfplotmarksize}}
\pgfpathlineto{\pgfqpointpolar{-20}{0.5\pgfplotmarksize}}
\pgfpathlineto{\pgfqpointpolar{-54}{\pgfplotmarksize}}
\pgfpathlineto{\pgfqpointpolar{-90}{0.5\pgfplotmarksize}}
\pgfpathlineto{\pgfqpointpolar{234}{\pgfplotmarksize}}
\pgfpathlineto{\pgfqpointpolar{198}{0.5\pgfplotmarksize}}
\pgfpathlineto{\pgfqpointpolar{162}{\pgfplotmarksize}}
\pgfpathlineto{\pgfqpointpolar{134}{0.5\pgfplotmarksize}}
\pgfpathclose
\pgfusepathqfillstroke
}
\definecolor{c}{rgb}{1,1,1};
\draw [color=c, fill=c] (0,0) rectangle (10,6.80516);
\draw [color=c, fill=c] (1,0.680516) rectangle (9.95,6.73711);
\definecolor{c}{rgb}{0,0,0};
\draw [c] (1,0.680516) -- (1,6.73711) -- (9.95,6.73711) -- (9.95,0.680516) -- (1,0.680516);
\definecolor{c}{rgb}{1,1,1};
\draw [color=c, fill=c] (1,0.680516) rectangle (9.95,6.73711);
\definecolor{c}{rgb}{0,0,0};
\draw [c] (1,0.680516) -- (1,6.73711) -- (9.95,6.73711) -- (9.95,0.680516) -- (1,0.680516);
\colorlet{c}{natcomp!70};
\draw [c] (2.01048,0.686784) -- (2.01048,0.695341);
\draw [c] (2.01048,0.695341) -- (2.01048,0.703897);
\draw [c] (1.86613,0.695341) -- (2.01048,0.695341);
\draw [c] (2.01048,0.695341) -- (2.15484,0.695341);
\definecolor{c}{rgb}{0,0,0};
\colorlet{c}{natcomp!70};
\draw [c] (2.29919,0.680526) -- (2.29919,0.680532);
\draw [c] (2.29919,0.680532) -- (2.29919,0.680538);
\draw [c] (2.15484,0.680532) -- (2.29919,0.680532);
\draw [c] (2.29919,0.680532) -- (2.44355,0.680532);
\definecolor{c}{rgb}{0,0,0};
\colorlet{c}{natcomp!70};
\draw [c] (2.5879,0.752927) -- (2.5879,0.77446);
\draw [c] (2.5879,0.77446) -- (2.5879,0.795993);
\draw [c] (2.44355,0.77446) -- (2.5879,0.77446);
\draw [c] (2.5879,0.77446) -- (2.73226,0.77446);
\definecolor{c}{rgb}{0,0,0};
\colorlet{c}{natcomp!70};
\draw [c] (2.87661,0.943483) -- (2.87661,0.982067);
\draw [c] (2.87661,0.982067) -- (2.87661,1.02065);
\draw [c] (2.73226,0.982067) -- (2.87661,0.982067);
\draw [c] (2.87661,0.982067) -- (3.02097,0.982067);
\definecolor{c}{rgb}{0,0,0};
\colorlet{c}{natcomp!70};
\draw [c] (3.16532,1.29053) -- (3.16532,1.34793);
\draw [c] (3.16532,1.34793) -- (3.16532,1.40533);
\draw [c] (3.02097,1.34793) -- (3.16532,1.34793);
\draw [c] (3.16532,1.34793) -- (3.30968,1.34793);
\definecolor{c}{rgb}{0,0,0};
\colorlet{c}{natcomp!70};
\draw [c] (3.45403,1.93426) -- (3.45403,2.01543);
\draw [c] (3.45403,2.01543) -- (3.45403,2.09661);
\draw [c] (3.30968,2.01543) -- (3.45403,2.01543);
\draw [c] (3.45403,2.01543) -- (3.59839,2.01543);
\definecolor{c}{rgb}{0,0,0};
\colorlet{c}{natcomp!70};
\draw [c] (3.74274,2.71861) -- (3.74274,2.82141);
\draw [c] (3.74274,2.82141) -- (3.74274,2.9242);
\draw [c] (3.59839,2.82141) -- (3.74274,2.82141);
\draw [c] (3.74274,2.82141) -- (3.8871,2.82141);
\definecolor{c}{rgb}{0,0,0};
\colorlet{c}{natcomp!70};
\draw [c] (4.03145,4.11719) -- (4.03145,4.24993);
\draw [c] (4.03145,4.24993) -- (4.03145,4.38267);
\draw [c] (3.8871,4.24993) -- (4.03145,4.24993);
\draw [c] (4.03145,4.24993) -- (4.17581,4.24993);
\definecolor{c}{rgb}{0,0,0};
\colorlet{c}{natcomp!70};
\draw [c] (4.32016,4.63729) -- (4.32016,4.77952);
\draw [c] (4.32016,4.77952) -- (4.32016,4.92176);
\draw [c] (4.17581,4.77952) -- (4.32016,4.77952);
\draw [c] (4.32016,4.77952) -- (4.46452,4.77952);
\definecolor{c}{rgb}{0,0,0};
\colorlet{c}{natcomp!70};
\draw [c] (4.60887,5.6146) -- (4.60887,5.77315);
\draw [c] (4.60887,5.77315) -- (4.60887,5.93169);
\draw [c] (4.46452,5.77315) -- (4.60887,5.77315);
\draw [c] (4.60887,5.77315) -- (4.75323,5.77315);
\definecolor{c}{rgb}{0,0,0};
\colorlet{c}{natcomp!70};
\draw [c] (4.89758,6.11613) -- (4.89758,6.28241);
\draw [c] (4.89758,6.28241) -- (4.89758,6.4487);
\draw [c] (4.75323,6.28241) -- (4.89758,6.28241);
\draw [c] (4.89758,6.28241) -- (5.04194,6.28241);
\definecolor{c}{rgb}{0,0,0};
\colorlet{c}{natcomp!70};
\draw [c] (5.18629,5.576) -- (5.18629,5.73393);
\draw [c] (5.18629,5.73393) -- (5.18629,5.89186);
\draw [c] (5.04194,5.73393) -- (5.18629,5.73393);
\draw [c] (5.18629,5.73393) -- (5.33065,5.73393);
\definecolor{c}{rgb}{0,0,0};
\colorlet{c}{natcomp!70};
\draw [c] (5.475,5.47333) -- (5.475,5.62963);
\draw [c] (5.475,5.62963) -- (5.475,5.78593);
\draw [c] (5.33065,5.62963) -- (5.475,5.62963);
\draw [c] (5.475,5.62963) -- (5.61935,5.62963);
\definecolor{c}{rgb}{0,0,0};
\colorlet{c}{natcomp!70};
\draw [c] (5.76371,4.43808) -- (5.76371,4.57675);
\draw [c] (5.76371,4.57675) -- (5.76371,4.71543);
\draw [c] (5.61935,4.57675) -- (5.76371,4.57675);
\draw [c] (5.76371,4.57675) -- (5.90806,4.57675);
\definecolor{c}{rgb}{0,0,0};
\colorlet{c}{natcomp!70};
\draw [c] (6.05242,3.40517) -- (6.05242,3.52363);
\draw [c] (6.05242,3.52363) -- (6.05242,3.64209);
\draw [c] (5.90806,3.52363) -- (6.05242,3.52363);
\draw [c] (6.05242,3.52363) -- (6.19677,3.52363);
\definecolor{c}{rgb}{0,0,0};
\colorlet{c}{natcomp!70};
\draw [c] (6.34113,2.64159) -- (6.34113,2.74247);
\draw [c] (6.34113,2.74247) -- (6.34113,2.84335);
\draw [c] (6.19677,2.74247) -- (6.34113,2.74247);
\draw [c] (6.34113,2.74247) -- (6.48548,2.74247);
\definecolor{c}{rgb}{0,0,0};
\colorlet{c}{natcomp!70};
\draw [c] (6.62984,2.05919) -- (6.62984,2.14419);
\draw [c] (6.62984,2.14419) -- (6.62984,2.22918);
\draw [c] (6.48548,2.14419) -- (6.62984,2.14419);
\draw [c] (6.62984,2.14419) -- (6.77419,2.14419);
\definecolor{c}{rgb}{0,0,0};
\colorlet{c}{natcomp!70};
\draw [c] (6.91855,1.74752) -- (6.91855,1.8226);
\draw [c] (6.91855,1.8226) -- (6.91855,1.89768);
\draw [c] (6.77419,1.8226) -- (6.91855,1.8226);
\draw [c] (6.91855,1.8226) -- (7.0629,1.8226);
\definecolor{c}{rgb}{0,0,0};
\colorlet{c}{natcomp!70};
\draw [c] (7.20726,1.16326) -- (7.20726,1.2146);
\draw [c] (7.20726,1.2146) -- (7.20726,1.26594);
\draw [c] (7.0629,1.2146) -- (7.20726,1.2146);
\draw [c] (7.20726,1.2146) -- (7.35161,1.2146);
\definecolor{c}{rgb}{0,0,0};
\colorlet{c}{natcomp!70};
\draw [c] (7.49597,0.925145) -- (7.49597,0.962442);
\draw [c] (7.49597,0.962442) -- (7.49597,0.999739);
\draw [c] (7.35161,0.962442) -- (7.49597,0.962442);
\draw [c] (7.49597,0.962442) -- (7.64032,0.962442);
\definecolor{c}{rgb}{0,0,0};
\colorlet{c}{natcomp!70};
\draw [c] (7.78468,0.847054) -- (7.78468,0.878298);
\draw [c] (7.78468,0.878298) -- (7.78468,0.909542);
\draw [c] (7.64032,0.878298) -- (7.78468,0.878298);
\draw [c] (7.78468,0.878298) -- (7.92903,0.878298);
\definecolor{c}{rgb}{0,0,0};
\colorlet{c}{natcomp!70};
\draw [c] (8.07339,0.7399) -- (8.07339,0.759661);
\draw [c] (8.07339,0.759661) -- (8.07339,0.779421);
\draw [c] (7.92903,0.759661) -- (8.07339,0.759661);
\draw [c] (8.07339,0.759661) -- (8.21774,0.759661);
\definecolor{c}{rgb}{0,0,0};
\colorlet{c}{natcomp!70};
\draw [c] (8.3621,0.710206) -- (8.3621,0.725026);
\draw [c] (8.3621,0.725026) -- (8.3621,0.739846);
\draw [c] (8.21774,0.725026) -- (8.3621,0.725026);
\draw [c] (8.3621,0.725026) -- (8.50645,0.725026);
\definecolor{c}{rgb}{0,0,0};
\colorlet{c}{natcomp!70};
\draw [c] (8.65081,0.686798) -- (8.65081,0.695355);
\draw [c] (8.65081,0.695355) -- (8.65081,0.703911);
\draw [c] (8.50645,0.695355) -- (8.65081,0.695355);
\draw [c] (8.65081,0.695355) -- (8.79516,0.695355);
\definecolor{c}{rgb}{0,0,0};
\colorlet{c}{natcomp!70};
\draw [c] (8.93952,0.683424) -- (8.93952,0.69041);
\draw [c] (8.93952,0.69041) -- (8.93952,0.697396);
\draw [c] (8.79516,0.69041) -- (8.93952,0.69041);
\draw [c] (8.93952,0.69041) -- (9.08387,0.69041);
\definecolor{c}{rgb}{0,0,0};
\colorlet{c}{natcomp!70};
\draw [c] (9.22823,0.690398) -- (9.22823,0.700279);
\draw [c] (9.22823,0.700279) -- (9.22823,0.710159);
\draw [c] (9.08387,0.700279) -- (9.22823,0.700279);
\draw [c] (9.22823,0.700279) -- (9.37258,0.700279);
\definecolor{c}{rgb}{0,0,0};
\colorlet{c}{natcomp!70};
\draw [c] (9.80564,0.680517) -- (9.80564,0.68052);
\draw [c] (9.80564,0.68052) -- (9.80564,0.680524);
\draw [c] (9.66129,0.68052) -- (9.80564,0.68052);
\draw [c] (9.80564,0.68052) -- (9.95,0.68052);
\definecolor{c}{rgb}{0,0,0};
\draw [c] (1,0.680516) -- (9.95,0.680516);
\draw [anchor= east] (9.95,0.108883) node[color=c, rotate=0]{Number of primary vertices};
\draw [c] (1.14435,0.863234) -- (1.14435,0.680516);
\draw [c] (1.43306,0.771875) -- (1.43306,0.680516);
\draw [c] (1.72177,0.771875) -- (1.72177,0.680516);
\draw [c] (2.01048,0.771875) -- (2.01048,0.680516);
\draw [c] (2.29919,0.771875) -- (2.29919,0.680516);
\draw [c] (2.5879,0.863234) -- (2.5879,0.680516);
\draw [c] (2.87661,0.771875) -- (2.87661,0.680516);
\draw [c] (3.16532,0.771875) -- (3.16532,0.680516);
\draw [c] (3.45403,0.771875) -- (3.45403,0.680516);
\draw [c] (3.74274,0.771875) -- (3.74274,0.680516);
\draw [c] (4.03145,0.863234) -- (4.03145,0.680516);
\draw [c] (4.32016,0.771875) -- (4.32016,0.680516);
\draw [c] (4.60887,0.771875) -- (4.60887,0.680516);
\draw [c] (4.89758,0.771875) -- (4.89758,0.680516);
\draw [c] (5.18629,0.771875) -- (5.18629,0.680516);
\draw [c] (5.475,0.863234) -- (5.475,0.680516);
\draw [c] (5.76371,0.771875) -- (5.76371,0.680516);
\draw [c] (6.05242,0.771875) -- (6.05242,0.680516);
\draw [c] (6.34113,0.771875) -- (6.34113,0.680516);
\draw [c] (6.62984,0.771875) -- (6.62984,0.680516);
\draw [c] (6.91855,0.863234) -- (6.91855,0.680516);
\draw [c] (7.20726,0.771875) -- (7.20726,0.680516);
\draw [c] (7.49597,0.771875) -- (7.49597,0.680516);
\draw [c] (7.78468,0.771875) -- (7.78468,0.680516);
\draw [c] (8.07339,0.771875) -- (8.07339,0.680516);
\draw [c] (8.3621,0.863234) -- (8.3621,0.680516);
\draw [c] (8.65081,0.771875) -- (8.65081,0.680516);
\draw [c] (8.93952,0.771875) -- (8.93952,0.680516);
\draw [c] (9.22823,0.771875) -- (9.22823,0.680516);
\draw [c] (9.51694,0.771875) -- (9.51694,0.680516);
\draw [c] (9.80564,0.863234) -- (9.80564,0.680516);
\draw [c] (1.14435,0.863234) -- (1.14435,0.680516);
\draw [c] (9.80564,0.863234) -- (9.80564,0.680516);
\draw [anchor=base] (1.14435,0.353868) node[color=c, rotate=0]{0};
\draw [anchor=base] (2.5879,0.353868) node[color=c, rotate=0]{5};
\draw [anchor=base] (4.03145,0.353868) node[color=c, rotate=0]{10};
\draw [anchor=base] (5.475,0.353868) node[color=c, rotate=0]{15};
\draw [anchor=base] (6.91855,0.353868) node[color=c, rotate=0]{20};
\draw [anchor=base] (8.3621,0.353868) node[color=c, rotate=0]{25};
\draw [anchor=base] (9.80564,0.353868) node[color=c, rotate=0]{30};
\draw [c] (1,0.680516) -- (1,6.73711);
\draw [anchor= east] (-0.12,6.73711) node[color=c, rotate=90]{Normalised number of events};
\draw [c] (1.267,0.680516) -- (1,0.680516);
\draw [c] (1.1335,0.878524) -- (1,0.878524);
\draw [c] (1.1335,1.07653) -- (1,1.07653);
\draw [c] (1.1335,1.27454) -- (1,1.27454);
\draw [c] (1.1335,1.47255) -- (1,1.47255);
\draw [c] (1.267,1.67056) -- (1,1.67056);
\draw [c] (1.1335,1.86857) -- (1,1.86857);
\draw [c] (1.1335,2.06657) -- (1,2.06657);
\draw [c] (1.1335,2.26458) -- (1,2.26458);
\draw [c] (1.1335,2.46259) -- (1,2.46259);
\draw [c] (1.267,2.6606) -- (1,2.6606);
\draw [c] (1.1335,2.85861) -- (1,2.85861);
\draw [c] (1.1335,3.05662) -- (1,3.05662);
\draw [c] (1.1335,3.25462) -- (1,3.25462);
\draw [c] (1.1335,3.45263) -- (1,3.45263);
\draw [c] (1.267,3.65064) -- (1,3.65064);
\draw [c] (1.1335,3.84865) -- (1,3.84865);
\draw [c] (1.1335,4.04666) -- (1,4.04666);
\draw [c] (1.1335,4.24467) -- (1,4.24467);
\draw [c] (1.1335,4.44267) -- (1,4.44267);
\draw [c] (1.267,4.64068) -- (1,4.64068);
\draw [c] (1.1335,4.83869) -- (1,4.83869);
\draw [c] (1.1335,5.0367) -- (1,5.0367);
\draw [c] (1.1335,5.23471) -- (1,5.23471);
\draw [c] (1.1335,5.43272) -- (1,5.43272);
\draw [c] (1.267,5.63072) -- (1,5.63072);
\draw [c] (1.1335,5.82873) -- (1,5.82873);
\draw [c] (1.1335,6.02674) -- (1,6.02674);
\draw [c] (1.1335,6.22475) -- (1,6.22475);
\draw [c] (1.1335,6.42276) -- (1,6.42276);
\draw [c] (1.267,6.62077) -- (1,6.62077);
\draw [c] (1.267,6.62077) -- (1,6.62077);
\draw [anchor= east] (0.95,0.680516) node[color=c, rotate=0]{0};
\draw [anchor= east] (0.95,1.67056) node[color=c, rotate=0]{500};
\draw [anchor= east] (0.95,2.6606) node[color=c, rotate=0]{1000};
\draw [anchor= east] (0.95,3.65064) node[color=c, rotate=0]{1500};
\draw [anchor= east] (0.95,4.64068) node[color=c, rotate=0]{2000};
\draw [anchor= east] (0.95,5.63072) node[color=c, rotate=0]{2500};
\draw [anchor= east] (0.95,6.62077) node[color=c, rotate=0]{3000};
\colorlet{c}{natgreen};
\draw [c] (1.72177,0.680523) -- (1.72177,0.681242);
\draw [c] (1.72177,0.681242) -- (1.72177,0.68196);
\draw [c] (1.57742,0.681242) -- (1.72177,0.681242);
\draw [c] (1.72177,0.681242) -- (1.86613,0.681242);
\definecolor{c}{rgb}{0,0,0};
\colorlet{c}{natgreen};
\draw [c] (2.01048,0.689073) -- (2.01048,0.696007);
\draw [c] (2.01048,0.696007) -- (2.01048,0.702941);
\draw [c] (1.86613,0.696007) -- (2.01048,0.696007);
\draw [c] (2.01048,0.696007) -- (2.15484,0.696007);
\definecolor{c}{rgb}{0,0,0};
\colorlet{c}{natgreen};
\draw [c] (2.29919,0.716299) -- (2.29919,0.731686);
\draw [c] (2.29919,0.731686) -- (2.29919,0.747074);
\draw [c] (2.15484,0.731686) -- (2.29919,0.731686);
\draw [c] (2.29919,0.731686) -- (2.44355,0.731686);
\definecolor{c}{rgb}{0,0,0};
\colorlet{c}{natgreen};
\draw [c] (2.5879,0.871864) -- (2.5879,0.902068);
\draw [c] (2.5879,0.902068) -- (2.5879,0.932273);
\draw [c] (2.44355,0.902068) -- (2.5879,0.902068);
\draw [c] (2.5879,0.902068) -- (2.73226,0.902068);
\definecolor{c}{rgb}{0,0,0};
\colorlet{c}{natgreen};
\draw [c] (2.87661,1.14594) -- (2.87661,1.19239);
\draw [c] (2.87661,1.19239) -- (2.87661,1.23883);
\draw [c] (2.73226,1.19239) -- (2.87661,1.19239);
\draw [c] (2.87661,1.19239) -- (3.02097,1.19239);
\definecolor{c}{rgb}{0,0,0};
\colorlet{c}{natgreen};
\draw [c] (3.16532,1.64654) -- (3.16532,1.715);
\draw [c] (3.16532,1.715) -- (3.16532,1.78347);
\draw [c] (3.02097,1.715) -- (3.16532,1.715);
\draw [c] (3.16532,1.715) -- (3.30968,1.715);
\definecolor{c}{rgb}{0,0,0};
\colorlet{c}{natgreen};
\draw [c] (3.45403,2.17071) -- (3.45403,2.25603);
\draw [c] (3.45403,2.25603) -- (3.45403,2.34135);
\draw [c] (3.30968,2.25603) -- (3.45403,2.25603);
\draw [c] (3.45403,2.25603) -- (3.59839,2.25603);
\definecolor{c}{rgb}{0,0,0};
\colorlet{c}{natgreen};
\draw [c] (3.74274,3.01706) -- (3.74274,3.12356);
\draw [c] (3.74274,3.12356) -- (3.74274,3.23006);
\draw [c] (3.59839,3.12356) -- (3.74274,3.12356);
\draw [c] (3.74274,3.12356) -- (3.8871,3.12356);
\definecolor{c}{rgb}{0,0,0};
\colorlet{c}{natgreen};
\draw [c] (4.03145,3.62689) -- (4.03145,3.74661);
\draw [c] (4.03145,3.74661) -- (4.03145,3.86633);
\draw [c] (3.8871,3.74661) -- (4.03145,3.74661);
\draw [c] (4.03145,3.74661) -- (4.17581,3.74661);
\definecolor{c}{rgb}{0,0,0};
\colorlet{c}{natgreen};
\draw [c] (4.32016,4.27462) -- (4.32016,4.40545);
\draw [c] (4.32016,4.40545) -- (4.32016,4.53628);
\draw [c] (4.17581,4.40545) -- (4.32016,4.40545);
\draw [c] (4.32016,4.40545) -- (4.46452,4.40545);
\definecolor{c}{rgb}{0,0,0};
\colorlet{c}{natgreen};
\draw [c] (4.60887,4.55969) -- (4.60887,4.69362);
\draw [c] (4.60887,4.69362) -- (4.60887,4.82755);
\draw [c] (4.46452,4.69362) -- (4.60887,4.69362);
\draw [c] (4.60887,4.69362) -- (4.75323,4.69362);
\definecolor{c}{rgb}{0,0,0};
\colorlet{c}{natgreen};
\draw [c] (4.89758,4.53815) -- (4.89758,4.66935);
\draw [c] (4.89758,4.66935) -- (4.89758,4.80056);
\draw [c] (4.75323,4.66935) -- (4.89758,4.66935);
\draw [c] (4.89758,4.66935) -- (5.04194,4.66935);
\definecolor{c}{rgb}{0,0,0};
\colorlet{c}{natgreen};
\draw [c] (5.18629,4.99176) -- (5.18629,5.12865);
\draw [c] (5.18629,5.12865) -- (5.18629,5.26553);
\draw [c] (5.04194,5.12865) -- (5.18629,5.12865);
\draw [c] (5.18629,5.12865) -- (5.33065,5.12865);
\definecolor{c}{rgb}{0,0,0};
\colorlet{c}{natgreen};
\draw [c] (5.475,4.29737) -- (5.475,4.41993);
\draw [c] (5.475,4.41993) -- (5.475,4.54248);
\draw [c] (5.33065,4.41993) -- (5.475,4.41993);
\draw [c] (5.475,4.41993) -- (5.61935,4.41993);
\definecolor{c}{rgb}{0,0,0};
\colorlet{c}{natgreen};
\draw [c] (5.76371,4.12649) -- (5.76371,4.24387);
\draw [c] (5.76371,4.24387) -- (5.76371,4.36126);
\draw [c] (5.61935,4.24387) -- (5.76371,4.24387);
\draw [c] (5.76371,4.24387) -- (5.90806,4.24387);
\definecolor{c}{rgb}{0,0,0};
\colorlet{c}{natgreen};
\draw [c] (6.05242,3.79458) -- (6.05242,3.90367);
\draw [c] (6.05242,3.90367) -- (6.05242,4.01276);
\draw [c] (5.90806,3.90367) -- (6.05242,3.90367);
\draw [c] (6.05242,3.90367) -- (6.19677,3.90367);
\definecolor{c}{rgb}{0,0,0};
\colorlet{c}{natgreen};
\draw [c] (6.34113,3.46262) -- (6.34113,3.56305);
\draw [c] (6.34113,3.56305) -- (6.34113,3.66348);
\draw [c] (6.19677,3.56305) -- (6.34113,3.56305);
\draw [c] (6.34113,3.56305) -- (6.48548,3.56305);
\definecolor{c}{rgb}{0,0,0};
\colorlet{c}{natgreen};
\draw [c] (6.62984,2.62966) -- (6.62984,2.71154);
\draw [c] (6.62984,2.71154) -- (6.62984,2.79342);
\draw [c] (6.48548,2.71154) -- (6.62984,2.71154);
\draw [c] (6.62984,2.71154) -- (6.77419,2.71154);
\definecolor{c}{rgb}{0,0,0};
\colorlet{c}{natgreen};
\draw [c] (6.91855,2.28926) -- (6.91855,2.36189);
\draw [c] (6.91855,2.36189) -- (6.91855,2.43452);
\draw [c] (6.77419,2.36189) -- (6.91855,2.36189);
\draw [c] (6.91855,2.36189) -- (7.0629,2.36189);
\definecolor{c}{rgb}{0,0,0};
\colorlet{c}{natgreen};
\draw [c] (7.20726,1.66134) -- (7.20726,1.71656);
\draw [c] (7.20726,1.71656) -- (7.20726,1.77178);
\draw [c] (7.0629,1.71656) -- (7.20726,1.71656);
\draw [c] (7.20726,1.71656) -- (7.35161,1.71656);
\definecolor{c}{rgb}{0,0,0};
\colorlet{c}{natgreen};
\draw [c] (7.49597,1.50108) -- (7.49597,1.55148);
\draw [c] (7.49597,1.55148) -- (7.49597,1.60188);
\draw [c] (7.35161,1.55148) -- (7.49597,1.55148);
\draw [c] (7.49597,1.55148) -- (7.64032,1.55148);
\definecolor{c}{rgb}{0,0,0};
\colorlet{c}{natgreen};
\draw [c] (7.78468,1.21206) -- (7.78468,1.25203);
\draw [c] (7.78468,1.25203) -- (7.78468,1.292);
\draw [c] (7.64032,1.25203) -- (7.78468,1.25203);
\draw [c] (7.78468,1.25203) -- (7.92903,1.25203);
\definecolor{c}{rgb}{0,0,0};
\colorlet{c}{natgreen};
\draw [c] (8.07339,1.01304) -- (8.07339,1.04329);
\draw [c] (8.07339,1.04329) -- (8.07339,1.07354);
\draw [c] (7.92903,1.04329) -- (8.07339,1.04329);
\draw [c] (8.07339,1.04329) -- (8.21774,1.04329);
\definecolor{c}{rgb}{0,0,0};
\colorlet{c}{natgreen};
\draw [c] (8.3621,0.847918) -- (8.3621,0.869254);
\draw [c] (8.3621,0.869254) -- (8.3621,0.890591);
\draw [c] (8.21774,0.869254) -- (8.3621,0.869254);
\draw [c] (8.3621,0.869254) -- (8.50645,0.869254);
\definecolor{c}{rgb}{0,0,0};
\colorlet{c}{natgreen};
\draw [c] (8.65081,0.787574) -- (8.65081,0.80412);
\draw [c] (8.65081,0.80412) -- (8.65081,0.820665);
\draw [c] (8.50645,0.80412) -- (8.65081,0.80412);
\draw [c] (8.65081,0.80412) -- (8.79516,0.80412);
\definecolor{c}{rgb}{0,0,0};
\colorlet{c}{natgreen};
\draw [c] (8.93952,0.739471) -- (8.93952,0.751966);
\draw [c] (8.93952,0.751966) -- (8.93952,0.76446);
\draw [c] (8.79516,0.751966) -- (8.93952,0.751966);
\draw [c] (8.93952,0.751966) -- (9.08387,0.751966);
\definecolor{c}{rgb}{0,0,0};
\colorlet{c}{natgreen};
\draw [c] (9.22823,0.713252) -- (9.22823,0.722645);
\draw [c] (9.22823,0.722645) -- (9.22823,0.732039);
\draw [c] (9.08387,0.722645) -- (9.22823,0.722645);
\draw [c] (9.22823,0.722645) -- (9.37258,0.722645);
\definecolor{c}{rgb}{0,0,0};
\colorlet{c}{natgreen};
\draw [c] (9.51694,0.695598) -- (9.51694,0.701995);
\draw [c] (9.51694,0.701995) -- (9.51694,0.708392);
\draw [c] (9.37258,0.701995) -- (9.51694,0.701995);
\draw [c] (9.51694,0.701995) -- (9.66129,0.701995);
\definecolor{c}{rgb}{0,0,0};
\colorlet{c}{natgreen};
\draw [c] (9.80564,0.682123) -- (9.80564,0.68405);
\draw [c] (9.80564,0.68405) -- (9.80564,0.685978);
\draw [c] (9.66129,0.68405) -- (9.80564,0.68405);
\draw [c] (9.80564,0.68405) -- (9.95,0.68405);
\definecolor{c}{rgb}{0,0,0};
\draw [anchor=base west] (6.96633,6.17962) node[color=c, rotate=0]{ATLAS MC};
\colorlet{c}{natgreen};
\draw [c] (6.13521,6.27149) -- (6.81966,6.27149);
\draw [c] (6.47744,6.149) -- (6.47744,6.39398);
\definecolor{c}{rgb}{0,0,0};
\draw [anchor=base west] (6.96633,5.77131) node[color=c, rotate=0]{Our MC};
\colorlet{c}{natcomp!70};
\draw [c] (6.13521,5.86318) -- (6.81966,5.86318);
\draw [c] (6.47744,5.74069) -- (6.47744,5.98567);
\end{tikzpicture}

\end{infilsf}
\end{minipage}
\begin{minipage}[b]{.3\textwidth}
\caption{The distribution of the number of reconstructed primary vertices in the \atlas{} $\gamma\gamma$ MC set and in the CalcHEP MC set produced for this thesis, normalised to the same number of events.}\label{pvnnone}
\end{minipage}
\end{figure}

[Uncertainty...] [Also, effect on Mgg...?]

The second issue we encounter is in the distribution of $E_T^\text{iso}$, which is much broader in the CalcHEP MC set than in the \atlas{} one, as illustrated in figure~\ref{etpv}. Assuming that the distribution in the CalcHEP sample is just the one in the \atlas{} sample, but broadened and shifted slightly, we develop a mapping function to reverse that effect [\~$E_T^\text{iso}/4+0.2$ GeV]. The result of applying this function is shown in fig.~\ref{etmap}.

\begin{figure}[htp]
\begin{minipage}[b]{.49\textwidth}
\begin{infilsf} \tiny 
\begin{tikzpicture}[x=.092\textwidth,y=.092\textwidth]
\pgfdeclareplotmark{cross} {
\pgfpathmoveto{\pgfpoint{-0.3\pgfplotmarksize}{\pgfplotmarksize}}
\pgfpathlineto{\pgfpoint{+0.3\pgfplotmarksize}{\pgfplotmarksize}}
\pgfpathlineto{\pgfpoint{+0.3\pgfplotmarksize}{0.3\pgfplotmarksize}}
\pgfpathlineto{\pgfpoint{+1\pgfplotmarksize}{0.3\pgfplotmarksize}}
\pgfpathlineto{\pgfpoint{+1\pgfplotmarksize}{-0.3\pgfplotmarksize}}
\pgfpathlineto{\pgfpoint{+0.3\pgfplotmarksize}{-0.3\pgfplotmarksize}}
\pgfpathlineto{\pgfpoint{+0.3\pgfplotmarksize}{-1.\pgfplotmarksize}}
\pgfpathlineto{\pgfpoint{-0.3\pgfplotmarksize}{-1.\pgfplotmarksize}}
\pgfpathlineto{\pgfpoint{-0.3\pgfplotmarksize}{-0.3\pgfplotmarksize}}
\pgfpathlineto{\pgfpoint{-1.\pgfplotmarksize}{-0.3\pgfplotmarksize}}
\pgfpathlineto{\pgfpoint{-1.\pgfplotmarksize}{0.3\pgfplotmarksize}}
\pgfpathlineto{\pgfpoint{-0.3\pgfplotmarksize}{0.3\pgfplotmarksize}}
\pgfpathclose
\pgfusepathqstroke
}
\pgfdeclareplotmark{cross*} {
\pgfpathmoveto{\pgfpoint{-0.3\pgfplotmarksize}{\pgfplotmarksize}}
\pgfpathlineto{\pgfpoint{+0.3\pgfplotmarksize}{\pgfplotmarksize}}
\pgfpathlineto{\pgfpoint{+0.3\pgfplotmarksize}{0.3\pgfplotmarksize}}
\pgfpathlineto{\pgfpoint{+1\pgfplotmarksize}{0.3\pgfplotmarksize}}
\pgfpathlineto{\pgfpoint{+1\pgfplotmarksize}{-0.3\pgfplotmarksize}}
\pgfpathlineto{\pgfpoint{+0.3\pgfplotmarksize}{-0.3\pgfplotmarksize}}
\pgfpathlineto{\pgfpoint{+0.3\pgfplotmarksize}{-1.\pgfplotmarksize}}
\pgfpathlineto{\pgfpoint{-0.3\pgfplotmarksize}{-1.\pgfplotmarksize}}
\pgfpathlineto{\pgfpoint{-0.3\pgfplotmarksize}{-0.3\pgfplotmarksize}}
\pgfpathlineto{\pgfpoint{-1.\pgfplotmarksize}{-0.3\pgfplotmarksize}}
\pgfpathlineto{\pgfpoint{-1.\pgfplotmarksize}{0.3\pgfplotmarksize}}
\pgfpathlineto{\pgfpoint{-0.3\pgfplotmarksize}{0.3\pgfplotmarksize}}
\pgfpathclose
\pgfusepathqfillstroke
}
\pgfdeclareplotmark{newstar} {
\pgfpathmoveto{\pgfqpoint{0pt}{\pgfplotmarksize}}
\pgfpathlineto{\pgfqpointpolar{44}{0.5\pgfplotmarksize}}
\pgfpathlineto{\pgfqpointpolar{18}{\pgfplotmarksize}}
\pgfpathlineto{\pgfqpointpolar{-20}{0.5\pgfplotmarksize}}
\pgfpathlineto{\pgfqpointpolar{-54}{\pgfplotmarksize}}
\pgfpathlineto{\pgfqpointpolar{-90}{0.5\pgfplotmarksize}}
\pgfpathlineto{\pgfqpointpolar{234}{\pgfplotmarksize}}
\pgfpathlineto{\pgfqpointpolar{198}{0.5\pgfplotmarksize}}
\pgfpathlineto{\pgfqpointpolar{162}{\pgfplotmarksize}}
\pgfpathlineto{\pgfqpointpolar{134}{0.5\pgfplotmarksize}}
\pgfpathclose
\pgfusepathqstroke
}
\pgfdeclareplotmark{newstar*} {
\pgfpathmoveto{\pgfqpoint{0pt}{\pgfplotmarksize}}
\pgfpathlineto{\pgfqpointpolar{44}{0.5\pgfplotmarksize}}
\pgfpathlineto{\pgfqpointpolar{18}{\pgfplotmarksize}}
\pgfpathlineto{\pgfqpointpolar{-20}{0.5\pgfplotmarksize}}
\pgfpathlineto{\pgfqpointpolar{-54}{\pgfplotmarksize}}
\pgfpathlineto{\pgfqpointpolar{-90}{0.5\pgfplotmarksize}}
\pgfpathlineto{\pgfqpointpolar{234}{\pgfplotmarksize}}
\pgfpathlineto{\pgfqpointpolar{198}{0.5\pgfplotmarksize}}
\pgfpathlineto{\pgfqpointpolar{162}{\pgfplotmarksize}}
\pgfpathlineto{\pgfqpointpolar{134}{0.5\pgfplotmarksize}}
\pgfpathclose
\pgfusepathqfillstroke
}
\definecolor{c}{rgb}{1,1,1};
\draw [color=c, fill=c] (0,0) rectangle (10,6.80516);
\draw [color=c, fill=c] (1,0.680516) rectangle (9.95,6.73711);
\definecolor{c}{rgb}{0,0,0};
\draw [c] (1,0.680516) -- (1,6.73711) -- (9.95,6.73711) -- (9.95,0.680516) -- (1,0.680516);
\definecolor{c}{rgb}{1,1,1};
\draw [color=c, fill=c] (1,0.680516) rectangle (9.95,6.73711);
\definecolor{c}{rgb}{0,0,0};
\draw [c] (1,0.680516) -- (1,6.73711) -- (9.95,6.73711) -- (9.95,0.680516) -- (1,0.680516);
\colorlet{c}{natgreen};
\draw [c] (1.59395,0.686894) -- (1.59395,0.686894);
\draw [c] (1.59395,0.686894) -- (1.59395,0.686894);
\draw [c] (1.58582,0.686894) -- (1.59395,0.686894);
\draw [c] (1.59395,0.686894) -- (1.60209,0.686894);
\definecolor{c}{rgb}{0,0,0};
\colorlet{c}{natgreen};
\draw [c] (1.96823,0.686894) -- (1.96823,0.686894);
\draw [c] (1.96823,0.686894) -- (1.96823,0.686894);
\draw [c] (1.96009,0.686894) -- (1.96823,0.686894);
\draw [c] (1.96823,0.686894) -- (1.97636,0.686894);
\definecolor{c}{rgb}{0,0,0};
\colorlet{c}{natgreen};
\draw [c] (2.24486,0.686894) -- (2.24486,0.686894);
\draw [c] (2.24486,0.686894) -- (2.24486,0.686894);
\draw [c] (2.23673,0.686894) -- (2.24486,0.686894);
\draw [c] (2.24486,0.686894) -- (2.253,0.686894);
\definecolor{c}{rgb}{0,0,0};
\colorlet{c}{natgreen};
\draw [c] (2.94459,0.692107) -- (2.94459,0.702737);
\draw [c] (2.94459,0.702737) -- (2.94459,0.713368);
\draw [c] (2.93645,0.702737) -- (2.94459,0.702737);
\draw [c] (2.94459,0.702737) -- (2.95273,0.702737);
\definecolor{c}{rgb}{0,0,0};
\colorlet{c}{natgreen};
\draw [c] (2.96086,0.691743) -- (2.96086,0.701841);
\draw [c] (2.96086,0.701841) -- (2.96086,0.711939);
\draw [c] (2.95273,0.701841) -- (2.96086,0.701841);
\draw [c] (2.96086,0.701841) -- (2.969,0.701841);
\definecolor{c}{rgb}{0,0,0};
\colorlet{c}{natgreen};
\draw [c] (3.02595,0.68941) -- (3.02595,0.69722);
\draw [c] (3.02595,0.69722) -- (3.02595,0.70503);
\draw [c] (3.01782,0.69722) -- (3.02595,0.69722);
\draw [c] (3.02595,0.69722) -- (3.03409,0.69722);
\definecolor{c}{rgb}{0,0,0};
\colorlet{c}{natgreen};
\draw [c] (3.0585,0.70998) -- (3.0585,0.729499);
\draw [c] (3.0585,0.729499) -- (3.0585,0.749018);
\draw [c] (3.05036,0.729499) -- (3.0585,0.729499);
\draw [c] (3.0585,0.729499) -- (3.06664,0.729499);
\definecolor{c}{rgb}{0,0,0};
\colorlet{c}{natgreen};
\draw [c] (3.07477,0.73884) -- (3.07477,0.766271);
\draw [c] (3.07477,0.766271) -- (3.07477,0.793702);
\draw [c] (3.06664,0.766271) -- (3.07477,0.766271);
\draw [c] (3.07477,0.766271) -- (3.08291,0.766271);
\definecolor{c}{rgb}{0,0,0};
\colorlet{c}{natgreen};
\draw [c] (3.09105,0.70686) -- (3.09105,0.727728);
\draw [c] (3.09105,0.727728) -- (3.09105,0.748596);
\draw [c] (3.08291,0.727728) -- (3.09105,0.727728);
\draw [c] (3.09105,0.727728) -- (3.09918,0.727728);
\definecolor{c}{rgb}{0,0,0};
\colorlet{c}{natgreen};
\draw [c] (3.10732,0.74735) -- (3.10732,0.773852);
\draw [c] (3.10732,0.773852) -- (3.10732,0.800355);
\draw [c] (3.09918,0.773852) -- (3.10732,0.773852);
\draw [c] (3.10732,0.773852) -- (3.11545,0.773852);
\definecolor{c}{rgb}{0,0,0};
\colorlet{c}{natgreen};
\draw [c] (3.12359,0.721184) -- (3.12359,0.744621);
\draw [c] (3.12359,0.744621) -- (3.12359,0.768058);
\draw [c] (3.11545,0.744621) -- (3.12359,0.744621);
\draw [c] (3.12359,0.744621) -- (3.13173,0.744621);
\definecolor{c}{rgb}{0,0,0};
\colorlet{c}{natgreen};
\draw [c] (3.13986,0.758599) -- (3.13986,0.790979);
\draw [c] (3.13986,0.790979) -- (3.13986,0.82336);
\draw [c] (3.13173,0.790979) -- (3.13986,0.790979);
\draw [c] (3.13986,0.790979) -- (3.148,0.790979);
\definecolor{c}{rgb}{0,0,0};
\colorlet{c}{natgreen};
\draw [c] (3.15614,0.770727) -- (3.15614,0.801482);
\draw [c] (3.15614,0.801482) -- (3.15614,0.832236);
\draw [c] (3.148,0.801482) -- (3.15614,0.801482);
\draw [c] (3.15614,0.801482) -- (3.16427,0.801482);
\definecolor{c}{rgb}{0,0,0};
\colorlet{c}{natgreen};
\draw [c] (3.17241,0.793587) -- (3.17241,0.827995);
\draw [c] (3.17241,0.827995) -- (3.17241,0.862404);
\draw [c] (3.16427,0.827995) -- (3.17241,0.827995);
\draw [c] (3.17241,0.827995) -- (3.18055,0.827995);
\definecolor{c}{rgb}{0,0,0};
\colorlet{c}{natgreen};
\draw [c] (3.18868,0.815002) -- (3.18868,0.854342);
\draw [c] (3.18868,0.854342) -- (3.18868,0.893683);
\draw [c] (3.18055,0.854342) -- (3.18868,0.854342);
\draw [c] (3.18868,0.854342) -- (3.19682,0.854342);
\definecolor{c}{rgb}{0,0,0};
\colorlet{c}{natgreen};
\draw [c] (3.20495,0.920998) -- (3.20495,0.973836);
\draw [c] (3.20495,0.973836) -- (3.20495,1.02667);
\draw [c] (3.19682,0.973836) -- (3.20495,0.973836);
\draw [c] (3.20495,0.973836) -- (3.21309,0.973836);
\definecolor{c}{rgb}{0,0,0};
\colorlet{c}{natgreen};
\draw [c] (3.22123,0.915327) -- (3.22123,0.966615);
\draw [c] (3.22123,0.966615) -- (3.22123,1.0179);
\draw [c] (3.21309,0.966615) -- (3.22123,0.966615);
\draw [c] (3.22123,0.966615) -- (3.22936,0.966615);
\definecolor{c}{rgb}{0,0,0};
\colorlet{c}{natgreen};
\draw [c] (3.2375,1.09644) -- (3.2375,1.16438);
\draw [c] (3.2375,1.16438) -- (3.2375,1.23231);
\draw [c] (3.22936,1.16438) -- (3.2375,1.16438);
\draw [c] (3.2375,1.16438) -- (3.24564,1.16438);
\definecolor{c}{rgb}{0,0,0};
\colorlet{c}{natgreen};
\draw [c] (3.25377,1.19172) -- (3.25377,1.26618);
\draw [c] (3.25377,1.26618) -- (3.25377,1.34065);
\draw [c] (3.24564,1.26618) -- (3.25377,1.26618);
\draw [c] (3.25377,1.26618) -- (3.26191,1.26618);
\definecolor{c}{rgb}{0,0,0};
\colorlet{c}{natgreen};
\draw [c] (3.27005,1.3208) -- (3.27005,1.40739);
\draw [c] (3.27005,1.40739) -- (3.27005,1.49398);
\draw [c] (3.26191,1.40739) -- (3.27005,1.40739);
\draw [c] (3.27005,1.40739) -- (3.27818,1.40739);
\definecolor{c}{rgb}{0,0,0};
\colorlet{c}{natgreen};
\draw [c] (3.28632,1.49265) -- (3.28632,1.58839);
\draw [c] (3.28632,1.58839) -- (3.28632,1.68413);
\draw [c] (3.27818,1.58839) -- (3.28632,1.58839);
\draw [c] (3.28632,1.58839) -- (3.29445,1.58839);
\definecolor{c}{rgb}{0,0,0};
\colorlet{c}{natgreen};
\draw [c] (3.30259,1.75408) -- (3.30259,1.865);
\draw [c] (3.30259,1.865) -- (3.30259,1.97593);
\draw [c] (3.29445,1.865) -- (3.30259,1.865);
\draw [c] (3.30259,1.865) -- (3.31073,1.865);
\definecolor{c}{rgb}{0,0,0};
\colorlet{c}{natgreen};
\draw [c] (3.31886,1.97279) -- (3.31886,2.09474);
\draw [c] (3.31886,2.09474) -- (3.31886,2.21669);
\draw [c] (3.31073,2.09474) -- (3.31886,2.09474);
\draw [c] (3.31886,2.09474) -- (3.327,2.09474);
\definecolor{c}{rgb}{0,0,0};
\colorlet{c}{natgreen};
\draw [c] (3.33514,2.31441) -- (3.33514,2.45017);
\draw [c] (3.33514,2.45017) -- (3.33514,2.58593);
\draw [c] (3.327,2.45017) -- (3.33514,2.45017);
\draw [c] (3.33514,2.45017) -- (3.34327,2.45017);
\definecolor{c}{rgb}{0,0,0};
\colorlet{c}{natgreen};
\draw [c] (3.35141,2.64591) -- (3.35141,2.7972);
\draw [c] (3.35141,2.7972) -- (3.35141,2.9485);
\draw [c] (3.34327,2.7972) -- (3.35141,2.7972);
\draw [c] (3.35141,2.7972) -- (3.35955,2.7972);
\definecolor{c}{rgb}{0,0,0};
\colorlet{c}{natgreen};
\draw [c] (3.36768,3.07353) -- (3.36768,3.24084);
\draw [c] (3.36768,3.24084) -- (3.36768,3.40814);
\draw [c] (3.35955,3.24084) -- (3.36768,3.24084);
\draw [c] (3.36768,3.24084) -- (3.37582,3.24084);
\definecolor{c}{rgb}{0,0,0};
\colorlet{c}{natgreen};
\draw [c] (3.38395,3.50243) -- (3.38395,3.68251);
\draw [c] (3.38395,3.68251) -- (3.38395,3.8626);
\draw [c] (3.37582,3.68251) -- (3.38395,3.68251);
\draw [c] (3.38395,3.68251) -- (3.39209,3.68251);
\definecolor{c}{rgb}{0,0,0};
\colorlet{c}{natgreen};
\draw [c] (3.40023,4.04994) -- (3.40023,4.24798);
\draw [c] (3.40023,4.24798) -- (3.40023,4.44601);
\draw [c] (3.39209,4.24798) -- (3.40023,4.24798);
\draw [c] (3.40023,4.24798) -- (3.40836,4.24798);
\definecolor{c}{rgb}{0,0,0};
\colorlet{c}{natgreen};
\draw [c] (3.4165,4.32972) -- (3.4165,4.53659);
\draw [c] (3.4165,4.53659) -- (3.4165,4.74346);
\draw [c] (3.40836,4.53659) -- (3.4165,4.53659);
\draw [c] (3.4165,4.53659) -- (3.42464,4.53659);
\definecolor{c}{rgb}{0,0,0};
\colorlet{c}{natgreen};
\draw [c] (3.43277,5.58375) -- (3.43277,5.82833);
\draw [c] (3.43277,5.82833) -- (3.43277,6.0729);
\draw [c] (3.42464,5.82833) -- (3.43277,5.82833);
\draw [c] (3.43277,5.82833) -- (3.44091,5.82833);
\definecolor{c}{rgb}{0,0,0};
\colorlet{c}{natgreen};
\draw [c] (3.44905,5.85444) -- (3.44905,6.10271);
\draw [c] (3.44905,6.10271) -- (3.44905,6.35098);
\draw [c] (3.44091,6.10271) -- (3.44905,6.10271);
\draw [c] (3.44905,6.10271) -- (3.45718,6.10271);
\definecolor{c}{rgb}{0,0,0};
\colorlet{c}{natgreen};
\draw [c] (3.46532,5.99755) -- (3.46532,6.2519);
\draw [c] (3.46532,6.2519) -- (3.46532,6.50624);
\draw [c] (3.45718,6.2519) -- (3.46532,6.2519);
\draw [c] (3.46532,6.2519) -- (3.47345,6.2519);
\definecolor{c}{rgb}{0,0,0};
\colorlet{c}{natgreen};
\draw [c] (3.48159,5.75888) -- (3.48159,6.00699);
\draw [c] (3.48159,6.00699) -- (3.48159,6.2551);
\draw [c] (3.47345,6.00699) -- (3.48159,6.00699);
\draw [c] (3.48159,6.00699) -- (3.48973,6.00699);
\definecolor{c}{rgb}{0,0,0};
\colorlet{c}{natgreen};
\draw [c] (3.49786,5.99961) -- (3.49786,6.25327);
\draw [c] (3.49786,6.25327) -- (3.49786,6.50693);
\draw [c] (3.48973,6.25327) -- (3.49786,6.25327);
\draw [c] (3.49786,6.25327) -- (3.506,6.25327);
\definecolor{c}{rgb}{0,0,0};
\colorlet{c}{natgreen};
\draw [c] (3.51414,6.07844) -- (3.51414,6.33275);
\draw [c] (3.51414,6.33275) -- (3.51414,6.58705);
\draw [c] (3.506,6.33275) -- (3.51414,6.33275);
\draw [c] (3.51414,6.33275) -- (3.52227,6.33275);
\definecolor{c}{rgb}{0,0,0};
\colorlet{c}{natgreen};
\draw [c] (3.53041,5.86819) -- (3.53041,6.1164);
\draw [c] (3.53041,6.1164) -- (3.53041,6.36461);
\draw [c] (3.52227,6.1164) -- (3.53041,6.1164);
\draw [c] (3.53041,6.1164) -- (3.53855,6.1164);
\definecolor{c}{rgb}{0,0,0};
\colorlet{c}{natgreen};
\draw [c] (3.54668,5.42558) -- (3.54668,5.66604);
\draw [c] (3.54668,5.66604) -- (3.54668,5.90649);
\draw [c] (3.53855,5.66604) -- (3.54668,5.66604);
\draw [c] (3.54668,5.66604) -- (3.55482,5.66604);
\definecolor{c}{rgb}{0,0,0};
\colorlet{c}{natgreen};
\draw [c] (3.56295,5.15763) -- (3.56295,5.3894);
\draw [c] (3.56295,5.3894) -- (3.56295,5.62118);
\draw [c] (3.55482,5.3894) -- (3.56295,5.3894);
\draw [c] (3.56295,5.3894) -- (3.57109,5.3894);
\definecolor{c}{rgb}{0,0,0};
\colorlet{c}{natgreen};
\draw [c] (3.57923,4.48033) -- (3.57923,4.69298);
\draw [c] (3.57923,4.69298) -- (3.57923,4.90562);
\draw [c] (3.57109,4.69298) -- (3.57923,4.69298);
\draw [c] (3.57923,4.69298) -- (3.58736,4.69298);
\definecolor{c}{rgb}{0,0,0};
\colorlet{c}{natgreen};
\draw [c] (3.5955,4.00128) -- (3.5955,4.20052);
\draw [c] (3.5955,4.20052) -- (3.5955,4.39975);
\draw [c] (3.58736,4.20052) -- (3.5955,4.20052);
\draw [c] (3.5955,4.20052) -- (3.60364,4.20052);
\definecolor{c}{rgb}{0,0,0};
\colorlet{c}{natgreen};
\draw [c] (3.61177,4.5575) -- (3.61177,4.7732);
\draw [c] (3.61177,4.7732) -- (3.61177,4.98891);
\draw [c] (3.60364,4.7732) -- (3.61177,4.7732);
\draw [c] (3.61177,4.7732) -- (3.61991,4.7732);
\definecolor{c}{rgb}{0,0,0};
\colorlet{c}{natgreen};
\draw [c] (3.62805,3.69671) -- (3.62805,3.88706);
\draw [c] (3.62805,3.88706) -- (3.62805,4.07741);
\draw [c] (3.61991,3.88706) -- (3.62805,3.88706);
\draw [c] (3.62805,3.88706) -- (3.63618,3.88706);
\definecolor{c}{rgb}{0,0,0};
\colorlet{c}{natgreen};
\draw [c] (3.64432,3.2541) -- (3.64432,3.43157);
\draw [c] (3.64432,3.43157) -- (3.64432,3.60905);
\draw [c] (3.63618,3.43157) -- (3.64432,3.43157);
\draw [c] (3.64432,3.43157) -- (3.65245,3.43157);
\definecolor{c}{rgb}{0,0,0};
\colorlet{c}{natgreen};
\draw [c] (3.66059,3.29397) -- (3.66059,3.46949);
\draw [c] (3.66059,3.46949) -- (3.66059,3.64502);
\draw [c] (3.65245,3.46949) -- (3.66059,3.46949);
\draw [c] (3.66059,3.46949) -- (3.66873,3.46949);
\definecolor{c}{rgb}{0,0,0};
\colorlet{c}{natgreen};
\draw [c] (3.67686,2.88985) -- (3.67686,3.05307);
\draw [c] (3.67686,3.05307) -- (3.67686,3.21628);
\draw [c] (3.66873,3.05307) -- (3.67686,3.05307);
\draw [c] (3.67686,3.05307) -- (3.685,3.05307);
\definecolor{c}{rgb}{0,0,0};
\colorlet{c}{natgreen};
\draw [c] (3.69314,2.89355) -- (3.69314,3.05681);
\draw [c] (3.69314,3.05681) -- (3.69314,3.22008);
\draw [c] (3.685,3.05681) -- (3.69314,3.05681);
\draw [c] (3.69314,3.05681) -- (3.70127,3.05681);
\definecolor{c}{rgb}{0,0,0};
\colorlet{c}{natgreen};
\draw [c] (3.70941,2.51251) -- (3.70941,2.66097);
\draw [c] (3.70941,2.66097) -- (3.70941,2.80943);
\draw [c] (3.70127,2.66097) -- (3.70941,2.66097);
\draw [c] (3.70941,2.66097) -- (3.71755,2.66097);
\definecolor{c}{rgb}{0,0,0};
\colorlet{c}{natgreen};
\draw [c] (3.72568,2.57195) -- (3.72568,2.72513);
\draw [c] (3.72568,2.72513) -- (3.72568,2.87831);
\draw [c] (3.71755,2.72513) -- (3.72568,2.72513);
\draw [c] (3.72568,2.72513) -- (3.73382,2.72513);
\definecolor{c}{rgb}{0,0,0};
\colorlet{c}{natgreen};
\draw [c] (3.74195,2.17915) -- (3.74195,2.31346);
\draw [c] (3.74195,2.31346) -- (3.74195,2.44777);
\draw [c] (3.73382,2.31346) -- (3.74195,2.31346);
\draw [c] (3.74195,2.31346) -- (3.75009,2.31346);
\definecolor{c}{rgb}{0,0,0};
\colorlet{c}{natgreen};
\draw [c] (3.75823,2.19392) -- (3.75823,2.32903);
\draw [c] (3.75823,2.32903) -- (3.75823,2.46413);
\draw [c] (3.75009,2.32903) -- (3.75823,2.32903);
\draw [c] (3.75823,2.32903) -- (3.76636,2.32903);
\definecolor{c}{rgb}{0,0,0};
\colorlet{c}{natgreen};
\draw [c] (3.7745,1.81437) -- (3.7745,1.92945);
\draw [c] (3.7745,1.92945) -- (3.7745,2.04452);
\draw [c] (3.76636,1.92945) -- (3.7745,1.92945);
\draw [c] (3.7745,1.92945) -- (3.78264,1.92945);
\definecolor{c}{rgb}{0,0,0};
\colorlet{c}{natgreen};
\draw [c] (3.79077,1.94943) -- (3.79077,2.07455);
\draw [c] (3.79077,2.07455) -- (3.79077,2.19966);
\draw [c] (3.78264,2.07455) -- (3.79077,2.07455);
\draw [c] (3.79077,2.07455) -- (3.79891,2.07455);
\definecolor{c}{rgb}{0,0,0};
\colorlet{c}{natgreen};
\draw [c] (3.80705,1.89796) -- (3.80705,2.01752);
\draw [c] (3.80705,2.01752) -- (3.80705,2.13708);
\draw [c] (3.79891,2.01752) -- (3.80705,2.01752);
\draw [c] (3.80705,2.01752) -- (3.81518,2.01752);
\definecolor{c}{rgb}{0,0,0};
\colorlet{c}{natgreen};
\draw [c] (3.82332,1.63297) -- (3.82332,1.7425);
\draw [c] (3.82332,1.7425) -- (3.82332,1.85203);
\draw [c] (3.81518,1.7425) -- (3.82332,1.7425);
\draw [c] (3.82332,1.7425) -- (3.83145,1.7425);
\definecolor{c}{rgb}{0,0,0};
\colorlet{c}{natgreen};
\draw [c] (3.83959,1.88026) -- (3.83959,1.99932);
\draw [c] (3.83959,1.99932) -- (3.83959,2.11838);
\draw [c] (3.83145,1.99932) -- (3.83959,1.99932);
\draw [c] (3.83959,1.99932) -- (3.84773,1.99932);
\definecolor{c}{rgb}{0,0,0};
\colorlet{c}{natgreen};
\draw [c] (3.85586,1.55762) -- (3.85586,1.66135);
\draw [c] (3.85586,1.66135) -- (3.85586,1.76508);
\draw [c] (3.84773,1.66135) -- (3.85586,1.66135);
\draw [c] (3.85586,1.66135) -- (3.864,1.66135);
\definecolor{c}{rgb}{0,0,0};
\colorlet{c}{natgreen};
\draw [c] (3.87214,1.37707) -- (3.87214,1.46506);
\draw [c] (3.87214,1.46506) -- (3.87214,1.55305);
\draw [c] (3.864,1.46506) -- (3.87214,1.46506);
\draw [c] (3.87214,1.46506) -- (3.88027,1.46506);
\definecolor{c}{rgb}{0,0,0};
\colorlet{c}{natgreen};
\draw [c] (3.88841,1.38074) -- (3.88841,1.47519);
\draw [c] (3.88841,1.47519) -- (3.88841,1.56964);
\draw [c] (3.88027,1.47519) -- (3.88841,1.47519);
\draw [c] (3.88841,1.47519) -- (3.89655,1.47519);
\definecolor{c}{rgb}{0,0,0};
\colorlet{c}{natgreen};
\draw [c] (3.90468,1.34633) -- (3.90468,1.43759);
\draw [c] (3.90468,1.43759) -- (3.90468,1.52885);
\draw [c] (3.89655,1.43759) -- (3.90468,1.43759);
\draw [c] (3.90468,1.43759) -- (3.91282,1.43759);
\definecolor{c}{rgb}{0,0,0};
\colorlet{c}{natgreen};
\draw [c] (3.92095,1.20184) -- (3.92095,1.28192);
\draw [c] (3.92095,1.28192) -- (3.92095,1.362);
\draw [c] (3.91282,1.28192) -- (3.92095,1.28192);
\draw [c] (3.92095,1.28192) -- (3.92909,1.28192);
\definecolor{c}{rgb}{0,0,0};
\colorlet{c}{natgreen};
\draw [c] (3.93723,1.14683) -- (3.93723,1.22409);
\draw [c] (3.93723,1.22409) -- (3.93723,1.30136);
\draw [c] (3.92909,1.22409) -- (3.93723,1.22409);
\draw [c] (3.93723,1.22409) -- (3.94536,1.22409);
\definecolor{c}{rgb}{0,0,0};
\colorlet{c}{natgreen};
\draw [c] (3.9535,1.09219) -- (3.9535,1.16486);
\draw [c] (3.9535,1.16486) -- (3.9535,1.23754);
\draw [c] (3.94536,1.16486) -- (3.9535,1.16486);
\draw [c] (3.9535,1.16486) -- (3.96164,1.16486);
\definecolor{c}{rgb}{0,0,0};
\colorlet{c}{natgreen};
\draw [c] (3.96977,1.03834) -- (3.96977,1.10508);
\draw [c] (3.96977,1.10508) -- (3.96977,1.17181);
\draw [c] (3.96164,1.10508) -- (3.96977,1.10508);
\draw [c] (3.96977,1.10508) -- (3.97791,1.10508);
\definecolor{c}{rgb}{0,0,0};
\colorlet{c}{natgreen};
\draw [c] (3.98605,1.09536) -- (3.98605,1.16626);
\draw [c] (3.98605,1.16626) -- (3.98605,1.23716);
\draw [c] (3.97791,1.16626) -- (3.98605,1.16626);
\draw [c] (3.98605,1.16626) -- (3.99418,1.16626);
\definecolor{c}{rgb}{0,0,0};
\colorlet{c}{natgreen};
\draw [c] (4.00232,0.977898) -- (4.00232,1.03829);
\draw [c] (4.00232,1.03829) -- (4.00232,1.09867);
\draw [c] (3.99418,1.03829) -- (4.00232,1.03829);
\draw [c] (4.00232,1.03829) -- (4.01045,1.03829);
\definecolor{c}{rgb}{0,0,0};
\colorlet{c}{natgreen};
\draw [c] (4.01859,1.13448) -- (4.01859,1.21092);
\draw [c] (4.01859,1.21092) -- (4.01859,1.28736);
\draw [c] (4.01045,1.21092) -- (4.01859,1.21092);
\draw [c] (4.01859,1.21092) -- (4.02673,1.21092);
\definecolor{c}{rgb}{0,0,0};
\colorlet{c}{natgreen};
\draw [c] (4.03486,0.932526) -- (4.03486,0.989231);
\draw [c] (4.03486,0.989231) -- (4.03486,1.04594);
\draw [c] (4.02673,0.989231) -- (4.03486,0.989231);
\draw [c] (4.03486,0.989231) -- (4.043,0.989231);
\definecolor{c}{rgb}{0,0,0};
\colorlet{c}{natgreen};
\draw [c] (4.05114,0.909038) -- (4.05114,0.964041);
\draw [c] (4.05114,0.964041) -- (4.05114,1.01904);
\draw [c] (4.043,0.964041) -- (4.05114,0.964041);
\draw [c] (4.05114,0.964041) -- (4.05927,0.964041);
\definecolor{c}{rgb}{0,0,0};
\colorlet{c}{natgreen};
\draw [c] (4.06741,0.905278) -- (4.06741,0.955802);
\draw [c] (4.06741,0.955802) -- (4.06741,1.00633);
\draw [c] (4.05927,0.955802) -- (4.06741,0.955802);
\draw [c] (4.06741,0.955802) -- (4.07555,0.955802);
\definecolor{c}{rgb}{0,0,0};
\colorlet{c}{natgreen};
\draw [c] (4.08368,0.910484) -- (4.08368,0.965967);
\draw [c] (4.08368,0.965967) -- (4.08368,1.02145);
\draw [c] (4.07555,0.965967) -- (4.08368,0.965967);
\draw [c] (4.08368,0.965967) -- (4.09182,0.965967);
\definecolor{c}{rgb}{0,0,0};
\colorlet{c}{natgreen};
\draw [c] (4.09995,0.84669) -- (4.09995,0.893411);
\draw [c] (4.09995,0.893411) -- (4.09995,0.940132);
\draw [c] (4.09182,0.893411) -- (4.09995,0.893411);
\draw [c] (4.09995,0.893411) -- (4.10809,0.893411);
\definecolor{c}{rgb}{0,0,0};
\colorlet{c}{natgreen};
\draw [c] (4.11623,0.816785) -- (4.11623,0.859365);
\draw [c] (4.11623,0.859365) -- (4.11623,0.901946);
\draw [c] (4.10809,0.859365) -- (4.11623,0.859365);
\draw [c] (4.11623,0.859365) -- (4.12436,0.859365);
\definecolor{c}{rgb}{0,0,0};
\colorlet{c}{natgreen};
\draw [c] (4.1325,0.820052) -- (4.1325,0.86177);
\draw [c] (4.1325,0.86177) -- (4.1325,0.903487);
\draw [c] (4.12436,0.86177) -- (4.1325,0.86177);
\draw [c] (4.1325,0.86177) -- (4.14064,0.86177);
\definecolor{c}{rgb}{0,0,0};
\colorlet{c}{natgreen};
\draw [c] (4.14877,0.830401) -- (4.14877,0.871992);
\draw [c] (4.14877,0.871992) -- (4.14877,0.913582);
\draw [c] (4.14064,0.871992) -- (4.14877,0.871992);
\draw [c] (4.14877,0.871992) -- (4.15691,0.871992);
\definecolor{c}{rgb}{0,0,0};
\colorlet{c}{natgreen};
\draw [c] (4.16505,0.79867) -- (4.16505,0.837525);
\draw [c] (4.16505,0.837525) -- (4.16505,0.87638);
\draw [c] (4.15691,0.837525) -- (4.16505,0.837525);
\draw [c] (4.16505,0.837525) -- (4.17318,0.837525);
\definecolor{c}{rgb}{0,0,0};
\colorlet{c}{natgreen};
\draw [c] (4.18132,0.784983) -- (4.18132,0.822264);
\draw [c] (4.18132,0.822264) -- (4.18132,0.859546);
\draw [c] (4.17318,0.822264) -- (4.18132,0.822264);
\draw [c] (4.18132,0.822264) -- (4.18945,0.822264);
\definecolor{c}{rgb}{0,0,0};
\colorlet{c}{natgreen};
\draw [c] (4.19759,0.794485) -- (4.19759,0.831406);
\draw [c] (4.19759,0.831406) -- (4.19759,0.868327);
\draw [c] (4.18945,0.831406) -- (4.19759,0.831406);
\draw [c] (4.19759,0.831406) -- (4.20573,0.831406);
\definecolor{c}{rgb}{0,0,0};
\colorlet{c}{natgreen};
\draw [c] (4.21386,0.787833) -- (4.21386,0.826205);
\draw [c] (4.21386,0.826205) -- (4.21386,0.864577);
\draw [c] (4.20573,0.826205) -- (4.21386,0.826205);
\draw [c] (4.21386,0.826205) -- (4.222,0.826205);
\definecolor{c}{rgb}{0,0,0};
\colorlet{c}{natgreen};
\draw [c] (4.23014,0.779584) -- (4.23014,0.816318);
\draw [c] (4.23014,0.816318) -- (4.23014,0.853052);
\draw [c] (4.222,0.816318) -- (4.23014,0.816318);
\draw [c] (4.23014,0.816318) -- (4.23827,0.816318);
\definecolor{c}{rgb}{0,0,0};
\colorlet{c}{natgreen};
\draw [c] (4.24641,0.773341) -- (4.24641,0.809215);
\draw [c] (4.24641,0.809215) -- (4.24641,0.845089);
\draw [c] (4.23827,0.809215) -- (4.24641,0.809215);
\draw [c] (4.24641,0.809215) -- (4.25455,0.809215);
\definecolor{c}{rgb}{0,0,0};
\colorlet{c}{natgreen};
\draw [c] (4.26268,0.754109) -- (4.26268,0.789001);
\draw [c] (4.26268,0.789001) -- (4.26268,0.823894);
\draw [c] (4.25455,0.789001) -- (4.26268,0.789001);
\draw [c] (4.26268,0.789001) -- (4.27082,0.789001);
\definecolor{c}{rgb}{0,0,0};
\colorlet{c}{natgreen};
\draw [c] (4.27895,0.761187) -- (4.27895,0.795352);
\draw [c] (4.27895,0.795352) -- (4.27895,0.829516);
\draw [c] (4.27082,0.795352) -- (4.27895,0.795352);
\draw [c] (4.27895,0.795352) -- (4.28709,0.795352);
\definecolor{c}{rgb}{0,0,0};
\colorlet{c}{natgreen};
\draw [c] (4.29523,0.703056) -- (4.29523,0.723502);
\draw [c] (4.29523,0.723502) -- (4.29523,0.743949);
\draw [c] (4.28709,0.723502) -- (4.29523,0.723502);
\draw [c] (4.29523,0.723502) -- (4.30336,0.723502);
\definecolor{c}{rgb}{0,0,0};
\colorlet{c}{natgreen};
\draw [c] (4.3115,0.731532) -- (4.3115,0.761049);
\draw [c] (4.3115,0.761049) -- (4.3115,0.790567);
\draw [c] (4.30336,0.761049) -- (4.3115,0.761049);
\draw [c] (4.3115,0.761049) -- (4.31964,0.761049);
\definecolor{c}{rgb}{0,0,0};
\colorlet{c}{natgreen};
\draw [c] (4.32777,0.745916) -- (4.32777,0.773891);
\draw [c] (4.32777,0.773891) -- (4.32777,0.801865);
\draw [c] (4.31964,0.773891) -- (4.32777,0.773891);
\draw [c] (4.32777,0.773891) -- (4.33591,0.773891);
\definecolor{c}{rgb}{0,0,0};
\colorlet{c}{natgreen};
\draw [c] (4.34405,0.743554) -- (4.34405,0.773152);
\draw [c] (4.34405,0.773152) -- (4.34405,0.80275);
\draw [c] (4.33591,0.773152) -- (4.34405,0.773152);
\draw [c] (4.34405,0.773152) -- (4.35218,0.773152);
\definecolor{c}{rgb}{0,0,0};
\colorlet{c}{natgreen};
\draw [c] (4.36032,0.692936) -- (4.36032,0.707758);
\draw [c] (4.36032,0.707758) -- (4.36032,0.72258);
\draw [c] (4.35218,0.707758) -- (4.36032,0.707758);
\draw [c] (4.36032,0.707758) -- (4.36845,0.707758);
\definecolor{c}{rgb}{0,0,0};
\colorlet{c}{natgreen};
\draw [c] (4.37659,0.757202) -- (4.37659,0.792096);
\draw [c] (4.37659,0.792096) -- (4.37659,0.826991);
\draw [c] (4.36845,0.792096) -- (4.37659,0.792096);
\draw [c] (4.37659,0.792096) -- (4.38473,0.792096);
\definecolor{c}{rgb}{0,0,0};
\colorlet{c}{natgreen};
\draw [c] (4.39286,0.702945) -- (4.39286,0.719518);
\draw [c] (4.39286,0.719518) -- (4.39286,0.73609);
\draw [c] (4.38473,0.719518) -- (4.39286,0.719518);
\draw [c] (4.39286,0.719518) -- (4.401,0.719518);
\definecolor{c}{rgb}{0,0,0};
\colorlet{c}{natgreen};
\draw [c] (4.40914,0.723316) -- (4.40914,0.749374);
\draw [c] (4.40914,0.749374) -- (4.40914,0.775431);
\draw [c] (4.401,0.749374) -- (4.40914,0.749374);
\draw [c] (4.40914,0.749374) -- (4.41727,0.749374);
\definecolor{c}{rgb}{0,0,0};
\colorlet{c}{natgreen};
\draw [c] (4.42541,0.70816) -- (4.42541,0.729558);
\draw [c] (4.42541,0.729558) -- (4.42541,0.750956);
\draw [c] (4.41727,0.729558) -- (4.42541,0.729558);
\draw [c] (4.42541,0.729558) -- (4.43355,0.729558);
\definecolor{c}{rgb}{0,0,0};
\colorlet{c}{natgreen};
\draw [c] (4.44168,0.731388) -- (4.44168,0.758337);
\draw [c] (4.44168,0.758337) -- (4.44168,0.785287);
\draw [c] (4.43355,0.758337) -- (4.44168,0.758337);
\draw [c] (4.44168,0.758337) -- (4.44982,0.758337);
\definecolor{c}{rgb}{0,0,0};
\colorlet{c}{natgreen};
\draw [c] (4.45795,0.715949) -- (4.45795,0.737013);
\draw [c] (4.45795,0.737013) -- (4.45795,0.758078);
\draw [c] (4.44982,0.737013) -- (4.45795,0.737013);
\draw [c] (4.45795,0.737013) -- (4.46609,0.737013);
\definecolor{c}{rgb}{0,0,0};
\colorlet{c}{natgreen};
\draw [c] (4.47423,0.708102) -- (4.47423,0.72723);
\draw [c] (4.47423,0.72723) -- (4.47423,0.746359);
\draw [c] (4.46609,0.72723) -- (4.47423,0.72723);
\draw [c] (4.47423,0.72723) -- (4.48236,0.72723);
\definecolor{c}{rgb}{0,0,0};
\colorlet{c}{natgreen};
\draw [c] (4.4905,0.695838) -- (4.4905,0.713323);
\draw [c] (4.4905,0.713323) -- (4.4905,0.730808);
\draw [c] (4.48236,0.713323) -- (4.4905,0.713323);
\draw [c] (4.4905,0.713323) -- (4.49864,0.713323);
\definecolor{c}{rgb}{0,0,0};
\colorlet{c}{natgreen};
\draw [c] (4.50677,0.70065) -- (4.50677,0.71825);
\draw [c] (4.50677,0.71825) -- (4.50677,0.735851);
\draw [c] (4.49864,0.71825) -- (4.50677,0.71825);
\draw [c] (4.50677,0.71825) -- (4.51491,0.71825);
\definecolor{c}{rgb}{0,0,0};
\colorlet{c}{natgreen};
\draw [c] (4.52305,0.692273) -- (4.52305,0.70583);
\draw [c] (4.52305,0.70583) -- (4.52305,0.719387);
\draw [c] (4.51491,0.70583) -- (4.52305,0.70583);
\draw [c] (4.52305,0.70583) -- (4.53118,0.70583);
\definecolor{c}{rgb}{0,0,0};
\colorlet{c}{natgreen};
\draw [c] (4.53932,0.708612) -- (4.53932,0.730843);
\draw [c] (4.53932,0.730843) -- (4.53932,0.753075);
\draw [c] (4.53118,0.730843) -- (4.53932,0.730843);
\draw [c] (4.53932,0.730843) -- (4.54745,0.730843);
\definecolor{c}{rgb}{0,0,0};
\colorlet{c}{natgreen};
\draw [c] (4.55559,0.699596) -- (4.55559,0.720953);
\draw [c] (4.55559,0.720953) -- (4.55559,0.74231);
\draw [c] (4.54745,0.720953) -- (4.55559,0.720953);
\draw [c] (4.55559,0.720953) -- (4.56373,0.720953);
\definecolor{c}{rgb}{0,0,0};
\colorlet{c}{natgreen};
\draw [c] (4.57186,0.709471) -- (4.57186,0.731051);
\draw [c] (4.57186,0.731051) -- (4.57186,0.75263);
\draw [c] (4.56373,0.731051) -- (4.57186,0.731051);
\draw [c] (4.57186,0.731051) -- (4.58,0.731051);
\definecolor{c}{rgb}{0,0,0};
\colorlet{c}{natgreen};
\draw [c] (4.58814,0.69136) -- (4.58814,0.704002);
\draw [c] (4.58814,0.704002) -- (4.58814,0.716643);
\draw [c] (4.58,0.704002) -- (4.58814,0.704002);
\draw [c] (4.58814,0.704002) -- (4.59627,0.704002);
\definecolor{c}{rgb}{0,0,0};
\colorlet{c}{natgreen};
\draw [c] (4.60441,0.688612) -- (4.60441,0.700212);
\draw [c] (4.60441,0.700212) -- (4.60441,0.711812);
\draw [c] (4.59627,0.700212) -- (4.60441,0.700212);
\draw [c] (4.60441,0.700212) -- (4.61255,0.700212);
\definecolor{c}{rgb}{0,0,0};
\colorlet{c}{natgreen};
\draw [c] (4.62068,0.691463) -- (4.62068,0.70493);
\draw [c] (4.62068,0.70493) -- (4.62068,0.718398);
\draw [c] (4.61255,0.70493) -- (4.62068,0.70493);
\draw [c] (4.62068,0.70493) -- (4.62882,0.70493);
\definecolor{c}{rgb}{0,0,0};
\colorlet{c}{natgreen};
\draw [c] (4.65323,0.690572) -- (4.65323,0.702891);
\draw [c] (4.65323,0.702891) -- (4.65323,0.71521);
\draw [c] (4.64509,0.702891) -- (4.65323,0.702891);
\draw [c] (4.65323,0.702891) -- (4.66136,0.702891);
\definecolor{c}{rgb}{0,0,0};
\colorlet{c}{natgreen};
\draw [c] (4.68577,0.686894) -- (4.68577,0.69532);
\draw [c] (4.68577,0.69532) -- (4.68577,0.703746);
\draw [c] (4.67764,0.69532) -- (4.68577,0.69532);
\draw [c] (4.68577,0.69532) -- (4.69391,0.69532);
\definecolor{c}{rgb}{0,0,0};
\colorlet{c}{natgreen};
\draw [c] (4.70205,0.702308) -- (4.70205,0.723366);
\draw [c] (4.70205,0.723366) -- (4.70205,0.744423);
\draw [c] (4.69391,0.723366) -- (4.70205,0.723366);
\draw [c] (4.70205,0.723366) -- (4.71018,0.723366);
\definecolor{c}{rgb}{0,0,0};
\colorlet{c}{natgreen};
\draw [c] (4.73459,0.686894) -- (4.73459,0.687013);
\draw [c] (4.73459,0.687013) -- (4.73459,0.687133);
\draw [c] (4.72645,0.687013) -- (4.73459,0.687013);
\draw [c] (4.73459,0.687013) -- (4.74273,0.687013);
\definecolor{c}{rgb}{0,0,0};
\colorlet{c}{natgreen};
\draw [c] (4.75086,0.686894) -- (4.75086,0.699741);
\draw [c] (4.75086,0.699741) -- (4.75086,0.712588);
\draw [c] (4.74273,0.699741) -- (4.75086,0.699741);
\draw [c] (4.75086,0.699741) -- (4.759,0.699741);
\definecolor{c}{rgb}{0,0,0};
\colorlet{c}{natgreen};
\draw [c] (4.76714,0.689559) -- (4.76714,0.695992);
\draw [c] (4.76714,0.695992) -- (4.76714,0.702426);
\draw [c] (4.759,0.695992) -- (4.76714,0.695992);
\draw [c] (4.76714,0.695992) -- (4.77527,0.695992);
\definecolor{c}{rgb}{0,0,0};
\colorlet{c}{natgreen};
\draw [c] (4.78341,0.688608) -- (4.78341,0.699913);
\draw [c] (4.78341,0.699913) -- (4.78341,0.711218);
\draw [c] (4.77527,0.699913) -- (4.78341,0.699913);
\draw [c] (4.78341,0.699913) -- (4.79155,0.699913);
\definecolor{c}{rgb}{0,0,0};
\colorlet{c}{natgreen};
\draw [c] (4.79968,0.697235) -- (4.79968,0.712428);
\draw [c] (4.79968,0.712428) -- (4.79968,0.72762);
\draw [c] (4.79155,0.712428) -- (4.79968,0.712428);
\draw [c] (4.79968,0.712428) -- (4.80782,0.712428);
\definecolor{c}{rgb}{0,0,0};
\colorlet{c}{natgreen};
\draw [c] (4.83223,0.686894) -- (4.83223,0.691443);
\draw [c] (4.83223,0.691443) -- (4.83223,0.695992);
\draw [c] (4.82409,0.691443) -- (4.83223,0.691443);
\draw [c] (4.83223,0.691443) -- (4.84036,0.691443);
\definecolor{c}{rgb}{0,0,0};
\colorlet{c}{natgreen};
\draw [c] (4.8485,0.686894) -- (4.8485,0.692853);
\draw [c] (4.8485,0.692853) -- (4.8485,0.698811);
\draw [c] (4.84036,0.692853) -- (4.8485,0.692853);
\draw [c] (4.8485,0.692853) -- (4.85664,0.692853);
\definecolor{c}{rgb}{0,0,0};
\colorlet{c}{natgreen};
\draw [c] (4.88105,0.691479) -- (4.88105,0.70509);
\draw [c] (4.88105,0.70509) -- (4.88105,0.718702);
\draw [c] (4.87291,0.70509) -- (4.88105,0.70509);
\draw [c] (4.88105,0.70509) -- (4.88918,0.70509);
\definecolor{c}{rgb}{0,0,0};
\colorlet{c}{natgreen};
\draw [c] (4.89732,0.697187) -- (4.89732,0.714402);
\draw [c] (4.89732,0.714402) -- (4.89732,0.731616);
\draw [c] (4.88918,0.714402) -- (4.89732,0.714402);
\draw [c] (4.89732,0.714402) -- (4.90545,0.714402);
\definecolor{c}{rgb}{0,0,0};
\colorlet{c}{natgreen};
\draw [c] (4.94614,0.701168) -- (4.94614,0.722014);
\draw [c] (4.94614,0.722014) -- (4.94614,0.74286);
\draw [c] (4.938,0.722014) -- (4.94614,0.722014);
\draw [c] (4.94614,0.722014) -- (4.95427,0.722014);
\definecolor{c}{rgb}{0,0,0};
\colorlet{c}{natgreen};
\draw [c] (4.96241,0.686894) -- (4.96241,0.70053);
\draw [c] (4.96241,0.70053) -- (4.96241,0.714166);
\draw [c] (4.95427,0.70053) -- (4.96241,0.70053);
\draw [c] (4.96241,0.70053) -- (4.97055,0.70053);
\definecolor{c}{rgb}{0,0,0};
\colorlet{c}{natgreen};
\draw [c] (5.04377,0.686894) -- (5.04377,0.688764);
\draw [c] (5.04377,0.688764) -- (5.04377,0.690634);
\draw [c] (5.03564,0.688764) -- (5.04377,0.688764);
\draw [c] (5.04377,0.688764) -- (5.05191,0.688764);
\definecolor{c}{rgb}{0,0,0};
\colorlet{c}{natgreen};
\draw [c] (5.06005,0.686894) -- (5.06005,0.698043);
\draw [c] (5.06005,0.698043) -- (5.06005,0.709192);
\draw [c] (5.05191,0.698043) -- (5.06005,0.698043);
\draw [c] (5.06005,0.698043) -- (5.06818,0.698043);
\definecolor{c}{rgb}{0,0,0};
\colorlet{c}{natgreen};
\draw [c] (5.10886,0.686894) -- (5.10886,0.697253);
\draw [c] (5.10886,0.697253) -- (5.10886,0.707613);
\draw [c] (5.10073,0.697253) -- (5.10886,0.697253);
\draw [c] (5.10886,0.697253) -- (5.117,0.697253);
\definecolor{c}{rgb}{0,0,0};
\colorlet{c}{natgreen};
\draw [c] (5.15768,0.686894) -- (5.15768,0.699132);
\draw [c] (5.15768,0.699132) -- (5.15768,0.71137);
\draw [c] (5.14955,0.699132) -- (5.15768,0.699132);
\draw [c] (5.15768,0.699132) -- (5.16582,0.699132);
\definecolor{c}{rgb}{0,0,0};
\colorlet{c}{natgreen};
\draw [c] (5.19023,0.686894) -- (5.19023,0.687595);
\draw [c] (5.19023,0.687595) -- (5.19023,0.688296);
\draw [c] (5.18209,0.687595) -- (5.19023,0.687595);
\draw [c] (5.19023,0.687595) -- (5.19836,0.687595);
\definecolor{c}{rgb}{0,0,0};
\colorlet{c}{natgreen};
\draw [c] (5.30414,0.686894) -- (5.30414,0.692853);
\draw [c] (5.30414,0.692853) -- (5.30414,0.698811);
\draw [c] (5.296,0.692853) -- (5.30414,0.692853);
\draw [c] (5.30414,0.692853) -- (5.31227,0.692853);
\definecolor{c}{rgb}{0,0,0};
\colorlet{c}{natgreen};
\draw [c] (5.33668,0.686894) -- (5.33668,0.694017);
\draw [c] (5.33668,0.694017) -- (5.33668,0.70114);
\draw [c] (5.32855,0.694017) -- (5.33668,0.694017);
\draw [c] (5.33668,0.694017) -- (5.34482,0.694017);
\definecolor{c}{rgb}{0,0,0};
\colorlet{c}{natgreen};
\draw [c] (5.36923,0.686894) -- (5.36923,0.69065);
\draw [c] (5.36923,0.69065) -- (5.36923,0.694407);
\draw [c] (5.36109,0.69065) -- (5.36923,0.69065);
\draw [c] (5.36923,0.69065) -- (5.37736,0.69065);
\definecolor{c}{rgb}{0,0,0};
\colorlet{c}{natgreen};
\draw [c] (5.3855,0.686894) -- (5.3855,0.687595);
\draw [c] (5.3855,0.687595) -- (5.3855,0.688296);
\draw [c] (5.37736,0.687595) -- (5.3855,0.687595);
\draw [c] (5.3855,0.687595) -- (5.39364,0.687595);
\definecolor{c}{rgb}{0,0,0};
\colorlet{c}{natgreen};
\draw [c] (5.40177,0.686894) -- (5.40177,0.694017);
\draw [c] (5.40177,0.694017) -- (5.40177,0.70114);
\draw [c] (5.39364,0.694017) -- (5.40177,0.694017);
\draw [c] (5.40177,0.694017) -- (5.40991,0.694017);
\definecolor{c}{rgb}{0,0,0};
\colorlet{c}{natgreen};
\draw [c] (5.41805,0.686894) -- (5.41805,0.694017);
\draw [c] (5.41805,0.694017) -- (5.41805,0.70114);
\draw [c] (5.40991,0.694017) -- (5.41805,0.694017);
\draw [c] (5.41805,0.694017) -- (5.42618,0.694017);
\definecolor{c}{rgb}{0,0,0};
\colorlet{c}{natgreen};
\draw [c] (5.43432,0.686894) -- (5.43432,0.696318);
\draw [c] (5.43432,0.696318) -- (5.43432,0.705742);
\draw [c] (5.42618,0.696318) -- (5.43432,0.696318);
\draw [c] (5.43432,0.696318) -- (5.44245,0.696318);
\definecolor{c}{rgb}{0,0,0};
\colorlet{c}{natgreen};
\draw [c] (5.45059,0.686894) -- (5.45059,0.691443);
\draw [c] (5.45059,0.691443) -- (5.45059,0.695992);
\draw [c] (5.44245,0.691443) -- (5.45059,0.691443);
\draw [c] (5.45059,0.691443) -- (5.45873,0.691443);
\definecolor{c}{rgb}{0,0,0};
\colorlet{c}{natgreen};
\draw [c] (5.48314,0.686894) -- (5.48314,0.698972);
\draw [c] (5.48314,0.698972) -- (5.48314,0.711049);
\draw [c] (5.475,0.698972) -- (5.48314,0.698972);
\draw [c] (5.48314,0.698972) -- (5.49127,0.698972);
\definecolor{c}{rgb}{0,0,0};
\colorlet{c}{natgreen};
\draw [c] (5.71095,0.686894) -- (5.71095,0.70053);
\draw [c] (5.71095,0.70053) -- (5.71095,0.714166);
\draw [c] (5.70282,0.70053) -- (5.71095,0.70053);
\draw [c] (5.71095,0.70053) -- (5.71909,0.70053);
\definecolor{c}{rgb}{0,0,0};
\colorlet{c}{natgreen};
\draw [c] (5.7435,0.686894) -- (5.7435,0.698334);
\draw [c] (5.7435,0.698334) -- (5.7435,0.709774);
\draw [c] (5.73536,0.698334) -- (5.7435,0.698334);
\draw [c] (5.7435,0.698334) -- (5.75164,0.698334);
\definecolor{c}{rgb}{0,0,0};
\colorlet{c}{natgreen};
\draw [c] (5.79232,0.686894) -- (5.79232,0.69532);
\draw [c] (5.79232,0.69532) -- (5.79232,0.703746);
\draw [c] (5.78418,0.69532) -- (5.79232,0.69532);
\draw [c] (5.79232,0.69532) -- (5.80045,0.69532);
\definecolor{c}{rgb}{0,0,0};
\colorlet{c}{natgreen};
\draw [c] (5.80859,0.686894) -- (5.80859,0.694017);
\draw [c] (5.80859,0.694017) -- (5.80859,0.70114);
\draw [c] (5.80045,0.694017) -- (5.80859,0.694017);
\draw [c] (5.80859,0.694017) -- (5.81673,0.694017);
\definecolor{c}{rgb}{0,0,0};
\colorlet{c}{natgreen};
\draw [c] (5.93877,0.686894) -- (5.93877,0.697253);
\draw [c] (5.93877,0.697253) -- (5.93877,0.707613);
\draw [c] (5.93064,0.697253) -- (5.93877,0.697253);
\draw [c] (5.93877,0.697253) -- (5.94691,0.697253);
\definecolor{c}{rgb}{0,0,0};
\colorlet{c}{natgreen};
\draw [c] (6.39441,0.686894) -- (6.39441,0.698489);
\draw [c] (6.39441,0.698489) -- (6.39441,0.710083);
\draw [c] (6.38627,0.698489) -- (6.39441,0.698489);
\draw [c] (6.39441,0.698489) -- (6.40255,0.698489);
\definecolor{c}{rgb}{0,0,0};
\colorlet{c}{natgreen};
\draw [c] (6.57341,0.686894) -- (6.57341,0.698606);
\draw [c] (6.57341,0.698606) -- (6.57341,0.710318);
\draw [c] (6.56527,0.698606) -- (6.57341,0.698606);
\draw [c] (6.57341,0.698606) -- (6.58155,0.698606);
\definecolor{c}{rgb}{0,0,0};
\colorlet{c}{natgreen};
\draw [c] (7.74505,0.686894) -- (7.74505,0.690097);
\draw [c] (7.74505,0.690097) -- (7.74505,0.6933);
\draw [c] (7.73691,0.690097) -- (7.74505,0.690097);
\draw [c] (7.74505,0.690097) -- (7.75318,0.690097);
\definecolor{c}{rgb}{0,0,0};
\draw [c] (1,0.680516) -- (9.95,0.680516);
\draw [anchor= east] (9.95,0.108883) node[color=c, rotate=0]{$E_{T}^{iso} \text{ [GeV]}$};
\draw [c] (1,0.863234) -- (1,0.680516);
\draw [c] (1.25571,0.771875) -- (1.25571,0.680516);
\draw [c] (1.51143,0.771875) -- (1.51143,0.680516);
\draw [c] (1.76714,0.771875) -- (1.76714,0.680516);
\draw [c] (2.02286,0.771875) -- (2.02286,0.680516);
\draw [c] (2.27857,0.863234) -- (2.27857,0.680516);
\draw [c] (2.53429,0.771875) -- (2.53429,0.680516);
\draw [c] (2.79,0.771875) -- (2.79,0.680516);
\draw [c] (3.04571,0.771875) -- (3.04571,0.680516);
\draw [c] (3.30143,0.771875) -- (3.30143,0.680516);
\draw [c] (3.55714,0.863234) -- (3.55714,0.680516);
\draw [c] (3.81286,0.771875) -- (3.81286,0.680516);
\draw [c] (4.06857,0.771875) -- (4.06857,0.680516);
\draw [c] (4.32429,0.771875) -- (4.32429,0.680516);
\draw [c] (4.58,0.771875) -- (4.58,0.680516);
\draw [c] (4.83571,0.863234) -- (4.83571,0.680516);
\draw [c] (5.09143,0.771875) -- (5.09143,0.680516);
\draw [c] (5.34714,0.771875) -- (5.34714,0.680516);
\draw [c] (5.60286,0.771875) -- (5.60286,0.680516);
\draw [c] (5.85857,0.771875) -- (5.85857,0.680516);
\draw [c] (6.11429,0.863234) -- (6.11429,0.680516);
\draw [c] (6.37,0.771875) -- (6.37,0.680516);
\draw [c] (6.62571,0.771875) -- (6.62571,0.680516);
\draw [c] (6.88143,0.771875) -- (6.88143,0.680516);
\draw [c] (7.13714,0.771875) -- (7.13714,0.680516);
\draw [c] (7.39286,0.863234) -- (7.39286,0.680516);
\draw [c] (7.64857,0.771875) -- (7.64857,0.680516);
\draw [c] (7.90429,0.771875) -- (7.90429,0.680516);
\draw [c] (8.16,0.771875) -- (8.16,0.680516);
\draw [c] (8.41571,0.771875) -- (8.41571,0.680516);
\draw [c] (8.67143,0.863234) -- (8.67143,0.680516);
\draw [c] (8.92714,0.771875) -- (8.92714,0.680516);
\draw [c] (9.18286,0.771875) -- (9.18286,0.680516);
\draw [c] (9.43857,0.771875) -- (9.43857,0.680516);
\draw [c] (9.69429,0.771875) -- (9.69429,0.680516);
\draw [c] (9.95,0.863234) -- (9.95,0.680516);
\draw [c] (9.95,0.863234) -- (9.95,0.680516);
\draw [anchor=base] (1,0.353868) node[color=c, rotate=0]{-20};
\draw [anchor=base] (2.27857,0.353868) node[color=c, rotate=0]{-10};
\draw [anchor=base] (3.55714,0.353868) node[color=c, rotate=0]{0};
\draw [anchor=base] (4.83571,0.353868) node[color=c, rotate=0]{10};
\draw [anchor=base] (6.11429,0.353868) node[color=c, rotate=0]{20};
\draw [anchor=base] (7.39286,0.353868) node[color=c, rotate=0]{30};
\draw [anchor=base] (8.67143,0.353868) node[color=c, rotate=0]{40};
\draw [anchor=base] (9.95,0.353868) node[color=c, rotate=0]{50};
\draw [c] (1,0.680516) -- (1,6.73711);
\draw [anchor= east] (-0.12,6.73711) node[color=c, rotate=90]{Normalised number of events};
\draw [c] (1.267,0.686894) -- (1,0.686894);
\draw [c] (1.1335,0.960568) -- (1,0.960568);
\draw [c] (1.1335,1.23424) -- (1,1.23424);
\draw [c] (1.1335,1.50792) -- (1,1.50792);
\draw [c] (1.267,1.78159) -- (1,1.78159);
\draw [c] (1.1335,2.05527) -- (1,2.05527);
\draw [c] (1.1335,2.32894) -- (1,2.32894);
\draw [c] (1.1335,2.60262) -- (1,2.60262);
\draw [c] (1.267,2.87629) -- (1,2.87629);
\draw [c] (1.1335,3.14996) -- (1,3.14996);
\draw [c] (1.1335,3.42364) -- (1,3.42364);
\draw [c] (1.1335,3.69731) -- (1,3.69731);
\draw [c] (1.267,3.97099) -- (1,3.97099);
\draw [c] (1.1335,4.24466) -- (1,4.24466);
\draw [c] (1.1335,4.51834) -- (1,4.51834);
\draw [c] (1.1335,4.79201) -- (1,4.79201);
\draw [c] (1.267,5.06569) -- (1,5.06569);
\draw [c] (1.1335,5.33936) -- (1,5.33936);
\draw [c] (1.1335,5.61304) -- (1,5.61304);
\draw [c] (1.1335,5.88671) -- (1,5.88671);
\draw [c] (1.267,6.16039) -- (1,6.16039);
\draw [c] (1.267,0.686894) -- (1,0.686894);
\draw [c] (1.267,6.16039) -- (1,6.16039);
\draw [c] (1.1335,6.43406) -- (1,6.43406);
\draw [c] (1.1335,6.70773) -- (1,6.70773);
\draw [anchor= east] (0.95,0.686894) node[color=c, rotate=0]{0};
\draw [anchor= east] (0.95,1.78159) node[color=c, rotate=0]{200};
\draw [anchor= east] (0.95,2.87629) node[color=c, rotate=0]{400};
\draw [anchor= east] (0.95,3.97099) node[color=c, rotate=0]{600};
\draw [anchor= east] (0.95,5.06569) node[color=c, rotate=0]{800};
\draw [anchor= east] (0.95,6.16039) node[color=c, rotate=0]{1000};
\colorlet{c}{natcomp!70};
\draw [c] (1.04068,0.686894) -- (1.04068,0.6869);
\draw [c] (1.04068,0.6869) -- (1.04068,0.686907);
\draw [c] (1.03255,0.6869) -- (1.04068,0.6869);
\draw [c] (1.04068,0.6869) -- (1.04882,0.6869);
\definecolor{c}{rgb}{0,0,0};
\colorlet{c}{natcomp!70};
\draw [c] (1.10577,0.686894) -- (1.10577,0.705403);
\draw [c] (1.10577,0.705403) -- (1.10577,0.723913);
\draw [c] (1.09764,0.705403) -- (1.10577,0.705403);
\draw [c] (1.10577,0.705403) -- (1.11391,0.705403);
\definecolor{c}{rgb}{0,0,0};
\colorlet{c}{natcomp!70};
\draw [c] (1.12205,0.6869) -- (1.12205,0.708668);
\draw [c] (1.12205,0.708668) -- (1.12205,0.730436);
\draw [c] (1.11391,0.708668) -- (1.12205,0.708668);
\draw [c] (1.12205,0.708668) -- (1.13018,0.708668);
\definecolor{c}{rgb}{0,0,0};
\colorlet{c}{natcomp!70};
\draw [c] (1.13832,0.686894) -- (1.13832,0.686899);
\draw [c] (1.13832,0.686899) -- (1.13832,0.686904);
\draw [c] (1.13018,0.686899) -- (1.13832,0.686899);
\draw [c] (1.13832,0.686899) -- (1.14645,0.686899);
\definecolor{c}{rgb}{0,0,0};
\colorlet{c}{natcomp!70};
\draw [c] (1.15459,0.686907) -- (1.15459,0.695046);
\draw [c] (1.15459,0.695046) -- (1.15459,0.703185);
\draw [c] (1.14645,0.695046) -- (1.15459,0.695046);
\draw [c] (1.15459,0.695046) -- (1.16273,0.695046);
\definecolor{c}{rgb}{0,0,0};
\colorlet{c}{natcomp!70};
\draw [c] (1.17086,0.686894) -- (1.17086,0.686904);
\draw [c] (1.17086,0.686904) -- (1.17086,0.686914);
\draw [c] (1.16273,0.686904) -- (1.17086,0.686904);
\draw [c] (1.17086,0.686904) -- (1.179,0.686904);
\definecolor{c}{rgb}{0,0,0};
\colorlet{c}{natcomp!70};
\draw [c] (1.18714,0.686897) -- (1.18714,0.686904);
\draw [c] (1.18714,0.686904) -- (1.18714,0.686911);
\draw [c] (1.179,0.686904) -- (1.18714,0.686904);
\draw [c] (1.18714,0.686904) -- (1.19527,0.686904);
\definecolor{c}{rgb}{0,0,0};
\colorlet{c}{natcomp!70};
\draw [c] (1.20341,0.686906) -- (1.20341,0.705415);
\draw [c] (1.20341,0.705415) -- (1.20341,0.723925);
\draw [c] (1.19527,0.705415) -- (1.20341,0.705415);
\draw [c] (1.20341,0.705415) -- (1.21155,0.705415);
\definecolor{c}{rgb}{0,0,0};
\colorlet{c}{natcomp!70};
\draw [c] (1.21968,0.686898) -- (1.21968,0.68691);
\draw [c] (1.21968,0.68691) -- (1.21968,0.686923);
\draw [c] (1.21155,0.68691) -- (1.21968,0.68691);
\draw [c] (1.21968,0.68691) -- (1.22782,0.68691);
\definecolor{c}{rgb}{0,0,0};
\colorlet{c}{natcomp!70};
\draw [c] (1.25223,0.695733) -- (1.25223,0.723286);
\draw [c] (1.25223,0.723286) -- (1.25223,0.75084);
\draw [c] (1.24409,0.723286) -- (1.25223,0.723286);
\draw [c] (1.25223,0.723286) -- (1.26036,0.723286);
\definecolor{c}{rgb}{0,0,0};
\colorlet{c}{natcomp!70};
\draw [c] (1.2685,0.686901) -- (1.2685,0.686911);
\draw [c] (1.2685,0.686911) -- (1.2685,0.686922);
\draw [c] (1.26036,0.686911) -- (1.2685,0.686911);
\draw [c] (1.2685,0.686911) -- (1.27664,0.686911);
\definecolor{c}{rgb}{0,0,0};
\colorlet{c}{natcomp!70};
\draw [c] (1.28477,0.686898) -- (1.28477,0.686915);
\draw [c] (1.28477,0.686915) -- (1.28477,0.686932);
\draw [c] (1.27664,0.686915) -- (1.28477,0.686915);
\draw [c] (1.28477,0.686915) -- (1.29291,0.686915);
\definecolor{c}{rgb}{0,0,0};
\colorlet{c}{natcomp!70};
\draw [c] (1.31732,0.694427) -- (1.31732,0.721188);
\draw [c] (1.31732,0.721188) -- (1.31732,0.747948);
\draw [c] (1.30918,0.721188) -- (1.31732,0.721188);
\draw [c] (1.31732,0.721188) -- (1.32545,0.721188);
\definecolor{c}{rgb}{0,0,0};
\colorlet{c}{natcomp!70};
\draw [c] (1.34986,0.686899) -- (1.34986,0.69462);
\draw [c] (1.34986,0.69462) -- (1.34986,0.702342);
\draw [c] (1.34173,0.69462) -- (1.34986,0.69462);
\draw [c] (1.34986,0.69462) -- (1.358,0.69462);
\definecolor{c}{rgb}{0,0,0};
\colorlet{c}{natcomp!70};
\draw [c] (1.36614,0.686894) -- (1.36614,0.6869);
\draw [c] (1.36614,0.6869) -- (1.36614,0.686907);
\draw [c] (1.358,0.6869) -- (1.36614,0.6869);
\draw [c] (1.36614,0.6869) -- (1.37427,0.6869);
\definecolor{c}{rgb}{0,0,0};
\colorlet{c}{natcomp!70};
\draw [c] (1.38241,0.68691) -- (1.38241,0.71207);
\draw [c] (1.38241,0.71207) -- (1.38241,0.73723);
\draw [c] (1.37427,0.71207) -- (1.38241,0.71207);
\draw [c] (1.38241,0.71207) -- (1.39055,0.71207);
\definecolor{c}{rgb}{0,0,0};
\colorlet{c}{natcomp!70};
\draw [c] (1.39868,0.696781) -- (1.39868,0.722664);
\draw [c] (1.39868,0.722664) -- (1.39868,0.748546);
\draw [c] (1.39055,0.722664) -- (1.39868,0.722664);
\draw [c] (1.39868,0.722664) -- (1.40682,0.722664);
\definecolor{c}{rgb}{0,0,0};
\colorlet{c}{natcomp!70};
\draw [c] (1.41495,0.686894) -- (1.41495,0.697305);
\draw [c] (1.41495,0.697305) -- (1.41495,0.707716);
\draw [c] (1.40682,0.697305) -- (1.41495,0.697305);
\draw [c] (1.41495,0.697305) -- (1.42309,0.697305);
\definecolor{c}{rgb}{0,0,0};
\colorlet{c}{natcomp!70};
\draw [c] (1.43123,0.6869) -- (1.43123,0.700903);
\draw [c] (1.43123,0.700903) -- (1.43123,0.714905);
\draw [c] (1.42309,0.700903) -- (1.43123,0.700903);
\draw [c] (1.43123,0.700903) -- (1.43936,0.700903);
\definecolor{c}{rgb}{0,0,0};
\colorlet{c}{natcomp!70};
\draw [c] (1.4475,0.686909) -- (1.4475,0.705418);
\draw [c] (1.4475,0.705418) -- (1.4475,0.723928);
\draw [c] (1.43936,0.705418) -- (1.4475,0.705418);
\draw [c] (1.4475,0.705418) -- (1.45564,0.705418);
\definecolor{c}{rgb}{0,0,0};
\colorlet{c}{natcomp!70};
\draw [c] (1.46377,0.686915) -- (1.46377,0.705424);
\draw [c] (1.46377,0.705424) -- (1.46377,0.723933);
\draw [c] (1.45564,0.705424) -- (1.46377,0.705424);
\draw [c] (1.46377,0.705424) -- (1.47191,0.705424);
\definecolor{c}{rgb}{0,0,0};
\colorlet{c}{natcomp!70};
\draw [c] (1.48005,0.6869) -- (1.48005,0.686916);
\draw [c] (1.48005,0.686916) -- (1.48005,0.686932);
\draw [c] (1.47191,0.686916) -- (1.48005,0.686916);
\draw [c] (1.48005,0.686916) -- (1.48818,0.686916);
\definecolor{c}{rgb}{0,0,0};
\colorlet{c}{natcomp!70};
\draw [c] (1.49632,0.686904) -- (1.49632,0.700906);
\draw [c] (1.49632,0.700906) -- (1.49632,0.714908);
\draw [c] (1.48818,0.700906) -- (1.49632,0.700906);
\draw [c] (1.49632,0.700906) -- (1.50445,0.700906);
\definecolor{c}{rgb}{0,0,0};
\colorlet{c}{natcomp!70};
\draw [c] (1.51259,0.686912) -- (1.51259,0.698144);
\draw [c] (1.51259,0.698144) -- (1.51259,0.709377);
\draw [c] (1.50445,0.698144) -- (1.51259,0.698144);
\draw [c] (1.51259,0.698144) -- (1.52073,0.698144);
\definecolor{c}{rgb}{0,0,0};
\colorlet{c}{natcomp!70};
\draw [c] (1.52886,0.686905) -- (1.52886,0.697316);
\draw [c] (1.52886,0.697316) -- (1.52886,0.707727);
\draw [c] (1.52073,0.697316) -- (1.52886,0.697316);
\draw [c] (1.52886,0.697316) -- (1.537,0.697316);
\definecolor{c}{rgb}{0,0,0};
\colorlet{c}{natcomp!70};
\draw [c] (1.54514,0.68691) -- (1.54514,0.686925);
\draw [c] (1.54514,0.686925) -- (1.54514,0.68694);
\draw [c] (1.537,0.686925) -- (1.54514,0.686925);
\draw [c] (1.54514,0.686925) -- (1.55327,0.686925);
\definecolor{c}{rgb}{0,0,0};
\colorlet{c}{natcomp!70};
\draw [c] (1.56141,0.686899) -- (1.56141,0.686912);
\draw [c] (1.56141,0.686912) -- (1.56141,0.686926);
\draw [c] (1.55327,0.686912) -- (1.56141,0.686912);
\draw [c] (1.56141,0.686912) -- (1.56955,0.686912);
\definecolor{c}{rgb}{0,0,0};
\colorlet{c}{natcomp!70};
\draw [c] (1.57768,0.686904) -- (1.57768,0.705413);
\draw [c] (1.57768,0.705413) -- (1.57768,0.723923);
\draw [c] (1.56955,0.705413) -- (1.57768,0.705413);
\draw [c] (1.57768,0.705413) -- (1.58582,0.705413);
\definecolor{c}{rgb}{0,0,0};
\colorlet{c}{natcomp!70};
\draw [c] (1.59395,0.686923) -- (1.59395,0.696817);
\draw [c] (1.59395,0.696817) -- (1.59395,0.706711);
\draw [c] (1.58582,0.696817) -- (1.59395,0.696817);
\draw [c] (1.59395,0.696817) -- (1.60209,0.696817);
\definecolor{c}{rgb}{0,0,0};
\colorlet{c}{natcomp!70};
\draw [c] (1.61023,0.686911) -- (1.61023,0.719865);
\draw [c] (1.61023,0.719865) -- (1.61023,0.752819);
\draw [c] (1.60209,0.719865) -- (1.61023,0.719865);
\draw [c] (1.61023,0.719865) -- (1.61836,0.719865);
\definecolor{c}{rgb}{0,0,0};
\colorlet{c}{natcomp!70};
\draw [c] (1.6265,0.692256) -- (1.6265,0.705147);
\draw [c] (1.6265,0.705147) -- (1.6265,0.718038);
\draw [c] (1.61836,0.705147) -- (1.6265,0.705147);
\draw [c] (1.6265,0.705147) -- (1.63464,0.705147);
\definecolor{c}{rgb}{0,0,0};
\colorlet{c}{natcomp!70};
\draw [c] (1.64277,0.686912) -- (1.64277,0.686932);
\draw [c] (1.64277,0.686932) -- (1.64277,0.686952);
\draw [c] (1.63464,0.686932) -- (1.64277,0.686932);
\draw [c] (1.64277,0.686932) -- (1.65091,0.686932);
\definecolor{c}{rgb}{0,0,0};
\colorlet{c}{natcomp!70};
\draw [c] (1.65905,0.692192) -- (1.65905,0.704906);
\draw [c] (1.65905,0.704906) -- (1.65905,0.717619);
\draw [c] (1.65091,0.704906) -- (1.65905,0.704906);
\draw [c] (1.65905,0.704906) -- (1.66718,0.704906);
\definecolor{c}{rgb}{0,0,0};
\colorlet{c}{natcomp!70};
\draw [c] (1.67532,0.700693) -- (1.67532,0.723056);
\draw [c] (1.67532,0.723056) -- (1.67532,0.74542);
\draw [c] (1.66718,0.723056) -- (1.67532,0.723056);
\draw [c] (1.67532,0.723056) -- (1.68345,0.723056);
\definecolor{c}{rgb}{0,0,0};
\colorlet{c}{natcomp!70};
\draw [c] (1.69159,0.686942) -- (1.69159,0.708221);
\draw [c] (1.69159,0.708221) -- (1.69159,0.729499);
\draw [c] (1.68345,0.708221) -- (1.69159,0.708221);
\draw [c] (1.69159,0.708221) -- (1.69973,0.708221);
\definecolor{c}{rgb}{0,0,0};
\colorlet{c}{natcomp!70};
\draw [c] (1.70786,0.705683) -- (1.70786,0.724842);
\draw [c] (1.70786,0.724842) -- (1.70786,0.744);
\draw [c] (1.69973,0.724842) -- (1.70786,0.724842);
\draw [c] (1.70786,0.724842) -- (1.716,0.724842);
\definecolor{c}{rgb}{0,0,0};
\colorlet{c}{natcomp!70};
\draw [c] (1.72414,0.686939) -- (1.72414,0.700941);
\draw [c] (1.72414,0.700941) -- (1.72414,0.714943);
\draw [c] (1.716,0.700941) -- (1.72414,0.700941);
\draw [c] (1.72414,0.700941) -- (1.73227,0.700941);
\definecolor{c}{rgb}{0,0,0};
\colorlet{c}{natcomp!70};
\draw [c] (1.74041,0.686936) -- (1.74041,0.708215);
\draw [c] (1.74041,0.708215) -- (1.74041,0.729493);
\draw [c] (1.73227,0.708215) -- (1.74041,0.708215);
\draw [c] (1.74041,0.708215) -- (1.74855,0.708215);
\definecolor{c}{rgb}{0,0,0};
\colorlet{c}{natcomp!70};
\draw [c] (1.75668,0.686903) -- (1.75668,0.696018);
\draw [c] (1.75668,0.696018) -- (1.75668,0.705133);
\draw [c] (1.74855,0.696018) -- (1.75668,0.696018);
\draw [c] (1.75668,0.696018) -- (1.76482,0.696018);
\definecolor{c}{rgb}{0,0,0};
\colorlet{c}{natcomp!70};
\draw [c] (1.77295,0.714667) -- (1.77295,0.738024);
\draw [c] (1.77295,0.738024) -- (1.77295,0.761381);
\draw [c] (1.76482,0.738024) -- (1.77295,0.738024);
\draw [c] (1.77295,0.738024) -- (1.78109,0.738024);
\definecolor{c}{rgb}{0,0,0};
\colorlet{c}{natcomp!70};
\draw [c] (1.78923,0.713535) -- (1.78923,0.735176);
\draw [c] (1.78923,0.735176) -- (1.78923,0.756817);
\draw [c] (1.78109,0.735176) -- (1.78923,0.735176);
\draw [c] (1.78923,0.735176) -- (1.79736,0.735176);
\definecolor{c}{rgb}{0,0,0};
\colorlet{c}{natcomp!70};
\draw [c] (1.8055,0.706706) -- (1.8055,0.727668);
\draw [c] (1.8055,0.727668) -- (1.8055,0.748629);
\draw [c] (1.79736,0.727668) -- (1.8055,0.727668);
\draw [c] (1.8055,0.727668) -- (1.81364,0.727668);
\definecolor{c}{rgb}{0,0,0};
\colorlet{c}{natcomp!70};
\draw [c] (1.82177,0.71253) -- (1.82177,0.7417);
\draw [c] (1.82177,0.7417) -- (1.82177,0.77087);
\draw [c] (1.81364,0.7417) -- (1.82177,0.7417);
\draw [c] (1.82177,0.7417) -- (1.82991,0.7417);
\definecolor{c}{rgb}{0,0,0};
\colorlet{c}{natcomp!70};
\draw [c] (1.83805,0.686939) -- (1.83805,0.698172);
\draw [c] (1.83805,0.698172) -- (1.83805,0.709404);
\draw [c] (1.82991,0.698172) -- (1.83805,0.698172);
\draw [c] (1.83805,0.698172) -- (1.84618,0.698172);
\definecolor{c}{rgb}{0,0,0};
\colorlet{c}{natcomp!70};
\draw [c] (1.85432,0.693926) -- (1.85432,0.714558);
\draw [c] (1.85432,0.714558) -- (1.85432,0.73519);
\draw [c] (1.84618,0.714558) -- (1.85432,0.714558);
\draw [c] (1.85432,0.714558) -- (1.86245,0.714558);
\definecolor{c}{rgb}{0,0,0};
\colorlet{c}{natcomp!70};
\draw [c] (1.87059,0.702794) -- (1.87059,0.726598);
\draw [c] (1.87059,0.726598) -- (1.87059,0.750403);
\draw [c] (1.86245,0.726598) -- (1.87059,0.726598);
\draw [c] (1.87059,0.726598) -- (1.87873,0.726598);
\definecolor{c}{rgb}{0,0,0};
\colorlet{c}{natcomp!70};
\draw [c] (1.88686,0.723513) -- (1.88686,0.749566);
\draw [c] (1.88686,0.749566) -- (1.88686,0.775619);
\draw [c] (1.87873,0.749566) -- (1.88686,0.749566);
\draw [c] (1.88686,0.749566) -- (1.895,0.749566);
\definecolor{c}{rgb}{0,0,0};
\colorlet{c}{natcomp!70};
\draw [c] (1.90314,0.703651) -- (1.90314,0.728586);
\draw [c] (1.90314,0.728586) -- (1.90314,0.753521);
\draw [c] (1.895,0.728586) -- (1.90314,0.728586);
\draw [c] (1.90314,0.728586) -- (1.91127,0.728586);
\definecolor{c}{rgb}{0,0,0};
\colorlet{c}{natcomp!70};
\draw [c] (1.91941,0.735552) -- (1.91941,0.762435);
\draw [c] (1.91941,0.762435) -- (1.91941,0.789317);
\draw [c] (1.91127,0.762435) -- (1.91941,0.762435);
\draw [c] (1.91941,0.762435) -- (1.92755,0.762435);
\definecolor{c}{rgb}{0,0,0};
\colorlet{c}{natcomp!70};
\draw [c] (1.93568,0.705159) -- (1.93568,0.723496);
\draw [c] (1.93568,0.723496) -- (1.93568,0.741834);
\draw [c] (1.92755,0.723496) -- (1.93568,0.723496);
\draw [c] (1.93568,0.723496) -- (1.94382,0.723496);
\definecolor{c}{rgb}{0,0,0};
\colorlet{c}{natcomp!70};
\draw [c] (1.95195,0.728073) -- (1.95195,0.759278);
\draw [c] (1.95195,0.759278) -- (1.95195,0.790483);
\draw [c] (1.94382,0.759278) -- (1.95195,0.759278);
\draw [c] (1.95195,0.759278) -- (1.96009,0.759278);
\definecolor{c}{rgb}{0,0,0};
\colorlet{c}{natcomp!70};
\draw [c] (1.96823,0.721308) -- (1.96823,0.745166);
\draw [c] (1.96823,0.745166) -- (1.96823,0.769024);
\draw [c] (1.96009,0.745166) -- (1.96823,0.745166);
\draw [c] (1.96823,0.745166) -- (1.97636,0.745166);
\definecolor{c}{rgb}{0,0,0};
\colorlet{c}{natcomp!70};
\draw [c] (1.9845,0.717672) -- (1.9845,0.74667);
\draw [c] (1.9845,0.74667) -- (1.9845,0.775667);
\draw [c] (1.97636,0.74667) -- (1.9845,0.74667);
\draw [c] (1.9845,0.74667) -- (1.99264,0.74667);
\definecolor{c}{rgb}{0,0,0};
\colorlet{c}{natcomp!70};
\draw [c] (2.00077,0.754545) -- (2.00077,0.791474);
\draw [c] (2.00077,0.791474) -- (2.00077,0.828402);
\draw [c] (1.99264,0.791474) -- (2.00077,0.791474);
\draw [c] (2.00077,0.791474) -- (2.00891,0.791474);
\definecolor{c}{rgb}{0,0,0};
\colorlet{c}{natcomp!70};
\draw [c] (2.01705,0.812339) -- (2.01705,0.861065);
\draw [c] (2.01705,0.861065) -- (2.01705,0.909792);
\draw [c] (2.00891,0.861065) -- (2.01705,0.861065);
\draw [c] (2.01705,0.861065) -- (2.02518,0.861065);
\definecolor{c}{rgb}{0,0,0};
\colorlet{c}{natcomp!70};
\draw [c] (2.03332,0.756859) -- (2.03332,0.788069);
\draw [c] (2.03332,0.788069) -- (2.03332,0.819279);
\draw [c] (2.02518,0.788069) -- (2.03332,0.788069);
\draw [c] (2.03332,0.788069) -- (2.04145,0.788069);
\definecolor{c}{rgb}{0,0,0};
\colorlet{c}{natcomp!70};
\draw [c] (2.04959,0.733143) -- (2.04959,0.772578);
\draw [c] (2.04959,0.772578) -- (2.04959,0.812014);
\draw [c] (2.04145,0.772578) -- (2.04959,0.772578);
\draw [c] (2.04959,0.772578) -- (2.05773,0.772578);
\definecolor{c}{rgb}{0,0,0};
\colorlet{c}{natcomp!70};
\draw [c] (2.06586,0.734846) -- (2.06586,0.767565);
\draw [c] (2.06586,0.767565) -- (2.06586,0.800284);
\draw [c] (2.05773,0.767565) -- (2.06586,0.767565);
\draw [c] (2.06586,0.767565) -- (2.074,0.767565);
\definecolor{c}{rgb}{0,0,0};
\colorlet{c}{natcomp!70};
\draw [c] (2.08214,0.789195) -- (2.08214,0.827149);
\draw [c] (2.08214,0.827149) -- (2.08214,0.865104);
\draw [c] (2.074,0.827149) -- (2.08214,0.827149);
\draw [c] (2.08214,0.827149) -- (2.09027,0.827149);
\definecolor{c}{rgb}{0,0,0};
\colorlet{c}{natcomp!70};
\draw [c] (2.09841,0.788521) -- (2.09841,0.828396);
\draw [c] (2.09841,0.828396) -- (2.09841,0.868271);
\draw [c] (2.09027,0.828396) -- (2.09841,0.828396);
\draw [c] (2.09841,0.828396) -- (2.10655,0.828396);
\definecolor{c}{rgb}{0,0,0};
\colorlet{c}{natcomp!70};
\draw [c] (2.11468,0.759161) -- (2.11468,0.799535);
\draw [c] (2.11468,0.799535) -- (2.11468,0.839908);
\draw [c] (2.10655,0.799535) -- (2.11468,0.799535);
\draw [c] (2.11468,0.799535) -- (2.12282,0.799535);
\definecolor{c}{rgb}{0,0,0};
\colorlet{c}{natcomp!70};
\draw [c] (2.13095,0.7679) -- (2.13095,0.805702);
\draw [c] (2.13095,0.805702) -- (2.13095,0.843503);
\draw [c] (2.12282,0.805702) -- (2.13095,0.805702);
\draw [c] (2.13095,0.805702) -- (2.13909,0.805702);
\definecolor{c}{rgb}{0,0,0};
\colorlet{c}{natcomp!70};
\draw [c] (2.14723,0.746911) -- (2.14723,0.783535);
\draw [c] (2.14723,0.783535) -- (2.14723,0.820159);
\draw [c] (2.13909,0.783535) -- (2.14723,0.783535);
\draw [c] (2.14723,0.783535) -- (2.15536,0.783535);
\definecolor{c}{rgb}{0,0,0};
\colorlet{c}{natcomp!70};
\draw [c] (2.1635,0.772802) -- (2.1635,0.808728);
\draw [c] (2.1635,0.808728) -- (2.1635,0.844654);
\draw [c] (2.15536,0.808728) -- (2.1635,0.808728);
\draw [c] (2.1635,0.808728) -- (2.17164,0.808728);
\definecolor{c}{rgb}{0,0,0};
\colorlet{c}{natcomp!70};
\draw [c] (2.17977,0.810552) -- (2.17977,0.849296);
\draw [c] (2.17977,0.849296) -- (2.17977,0.888039);
\draw [c] (2.17164,0.849296) -- (2.17977,0.849296);
\draw [c] (2.17977,0.849296) -- (2.18791,0.849296);
\definecolor{c}{rgb}{0,0,0};
\colorlet{c}{natcomp!70};
\draw [c] (2.19605,0.778252) -- (2.19605,0.813555);
\draw [c] (2.19605,0.813555) -- (2.19605,0.848858);
\draw [c] (2.18791,0.813555) -- (2.19605,0.813555);
\draw [c] (2.19605,0.813555) -- (2.20418,0.813555);
\definecolor{c}{rgb}{0,0,0};
\colorlet{c}{natcomp!70};
\draw [c] (2.21232,0.789468) -- (2.21232,0.827907);
\draw [c] (2.21232,0.827907) -- (2.21232,0.866346);
\draw [c] (2.20418,0.827907) -- (2.21232,0.827907);
\draw [c] (2.21232,0.827907) -- (2.22045,0.827907);
\definecolor{c}{rgb}{0,0,0};
\colorlet{c}{natcomp!70};
\draw [c] (2.22859,0.815059) -- (2.22859,0.859924);
\draw [c] (2.22859,0.859924) -- (2.22859,0.904788);
\draw [c] (2.22045,0.859924) -- (2.22859,0.859924);
\draw [c] (2.22859,0.859924) -- (2.23673,0.859924);
\definecolor{c}{rgb}{0,0,0};
\colorlet{c}{natcomp!70};
\draw [c] (2.24486,0.802412) -- (2.24486,0.839797);
\draw [c] (2.24486,0.839797) -- (2.24486,0.877183);
\draw [c] (2.23673,0.839797) -- (2.24486,0.839797);
\draw [c] (2.24486,0.839797) -- (2.253,0.839797);
\definecolor{c}{rgb}{0,0,0};
\colorlet{c}{natcomp!70};
\draw [c] (2.26114,0.840272) -- (2.26114,0.885266);
\draw [c] (2.26114,0.885266) -- (2.26114,0.93026);
\draw [c] (2.253,0.885266) -- (2.26114,0.885266);
\draw [c] (2.26114,0.885266) -- (2.26927,0.885266);
\definecolor{c}{rgb}{0,0,0};
\colorlet{c}{natcomp!70};
\draw [c] (2.27741,0.805072) -- (2.27741,0.848826);
\draw [c] (2.27741,0.848826) -- (2.27741,0.892579);
\draw [c] (2.26927,0.848826) -- (2.27741,0.848826);
\draw [c] (2.27741,0.848826) -- (2.28555,0.848826);
\definecolor{c}{rgb}{0,0,0};
\colorlet{c}{natcomp!70};
\draw [c] (2.29368,0.791504) -- (2.29368,0.826715);
\draw [c] (2.29368,0.826715) -- (2.29368,0.861925);
\draw [c] (2.28555,0.826715) -- (2.29368,0.826715);
\draw [c] (2.29368,0.826715) -- (2.30182,0.826715);
\definecolor{c}{rgb}{0,0,0};
\colorlet{c}{natcomp!70};
\draw [c] (2.30995,0.856209) -- (2.30995,0.906613);
\draw [c] (2.30995,0.906613) -- (2.30995,0.957017);
\draw [c] (2.30182,0.906613) -- (2.30995,0.906613);
\draw [c] (2.30995,0.906613) -- (2.31809,0.906613);
\definecolor{c}{rgb}{0,0,0};
\colorlet{c}{natcomp!70};
\draw [c] (2.32623,0.828277) -- (2.32623,0.869435);
\draw [c] (2.32623,0.869435) -- (2.32623,0.910592);
\draw [c] (2.31809,0.869435) -- (2.32623,0.869435);
\draw [c] (2.32623,0.869435) -- (2.33436,0.869435);
\definecolor{c}{rgb}{0,0,0};
\colorlet{c}{natcomp!70};
\draw [c] (2.3425,0.88625) -- (2.3425,0.939801);
\draw [c] (2.3425,0.939801) -- (2.3425,0.993352);
\draw [c] (2.33436,0.939801) -- (2.3425,0.939801);
\draw [c] (2.3425,0.939801) -- (2.35064,0.939801);
\definecolor{c}{rgb}{0,0,0};
\colorlet{c}{natcomp!70};
\draw [c] (2.35877,0.916275) -- (2.35877,0.969867);
\draw [c] (2.35877,0.969867) -- (2.35877,1.02346);
\draw [c] (2.35064,0.969867) -- (2.35877,0.969867);
\draw [c] (2.35877,0.969867) -- (2.36691,0.969867);
\definecolor{c}{rgb}{0,0,0};
\colorlet{c}{natcomp!70};
\draw [c] (2.37505,0.907296) -- (2.37505,0.962405);
\draw [c] (2.37505,0.962405) -- (2.37505,1.01751);
\draw [c] (2.36691,0.962405) -- (2.37505,0.962405);
\draw [c] (2.37505,0.962405) -- (2.38318,0.962405);
\definecolor{c}{rgb}{0,0,0};
\colorlet{c}{natcomp!70};
\draw [c] (2.39132,0.827878) -- (2.39132,0.86913);
\draw [c] (2.39132,0.86913) -- (2.39132,0.910383);
\draw [c] (2.38318,0.86913) -- (2.39132,0.86913);
\draw [c] (2.39132,0.86913) -- (2.39945,0.86913);
\definecolor{c}{rgb}{0,0,0};
\colorlet{c}{natcomp!70};
\draw [c] (2.40759,0.857178) -- (2.40759,0.910918);
\draw [c] (2.40759,0.910918) -- (2.40759,0.964658);
\draw [c] (2.39945,0.910918) -- (2.40759,0.910918);
\draw [c] (2.40759,0.910918) -- (2.41573,0.910918);
\definecolor{c}{rgb}{0,0,0};
\colorlet{c}{natcomp!70};
\draw [c] (2.42386,0.919688) -- (2.42386,0.974446);
\draw [c] (2.42386,0.974446) -- (2.42386,1.0292);
\draw [c] (2.41573,0.974446) -- (2.42386,0.974446);
\draw [c] (2.42386,0.974446) -- (2.432,0.974446);
\definecolor{c}{rgb}{0,0,0};
\colorlet{c}{natcomp!70};
\draw [c] (2.44014,0.898282) -- (2.44014,0.948641);
\draw [c] (2.44014,0.948641) -- (2.44014,0.999);
\draw [c] (2.432,0.948641) -- (2.44014,0.948641);
\draw [c] (2.44014,0.948641) -- (2.44827,0.948641);
\definecolor{c}{rgb}{0,0,0};
\colorlet{c}{natcomp!70};
\draw [c] (2.45641,0.910429) -- (2.45641,0.968628);
\draw [c] (2.45641,0.968628) -- (2.45641,1.02683);
\draw [c] (2.44827,0.968628) -- (2.45641,0.968628);
\draw [c] (2.45641,0.968628) -- (2.46455,0.968628);
\definecolor{c}{rgb}{0,0,0};
\colorlet{c}{natcomp!70};
\draw [c] (2.47268,0.884339) -- (2.47268,0.931852);
\draw [c] (2.47268,0.931852) -- (2.47268,0.979366);
\draw [c] (2.46455,0.931852) -- (2.47268,0.931852);
\draw [c] (2.47268,0.931852) -- (2.48082,0.931852);
\definecolor{c}{rgb}{0,0,0};
\colorlet{c}{natcomp!70};
\draw [c] (2.48895,0.969757) -- (2.48895,1.0318);
\draw [c] (2.48895,1.0318) -- (2.48895,1.09385);
\draw [c] (2.48082,1.0318) -- (2.48895,1.0318);
\draw [c] (2.48895,1.0318) -- (2.49709,1.0318);
\definecolor{c}{rgb}{0,0,0};
\colorlet{c}{natcomp!70};
\draw [c] (2.50523,0.971435) -- (2.50523,1.03092);
\draw [c] (2.50523,1.03092) -- (2.50523,1.0904);
\draw [c] (2.49709,1.03092) -- (2.50523,1.03092);
\draw [c] (2.50523,1.03092) -- (2.51336,1.03092);
\definecolor{c}{rgb}{0,0,0};
\colorlet{c}{natcomp!70};
\draw [c] (2.5215,0.990909) -- (2.5215,1.05577);
\draw [c] (2.5215,1.05577) -- (2.5215,1.12064);
\draw [c] (2.51336,1.05577) -- (2.5215,1.05577);
\draw [c] (2.5215,1.05577) -- (2.52964,1.05577);
\definecolor{c}{rgb}{0,0,0};
\colorlet{c}{natcomp!70};
\draw [c] (2.53777,1.00227) -- (2.53777,1.06689);
\draw [c] (2.53777,1.06689) -- (2.53777,1.13151);
\draw [c] (2.52964,1.06689) -- (2.53777,1.06689);
\draw [c] (2.53777,1.06689) -- (2.54591,1.06689);
\definecolor{c}{rgb}{0,0,0};
\colorlet{c}{natcomp!70};
\draw [c] (2.55405,1.04669) -- (2.55405,1.11011);
\draw [c] (2.55405,1.11011) -- (2.55405,1.17353);
\draw [c] (2.54591,1.11011) -- (2.55405,1.11011);
\draw [c] (2.55405,1.11011) -- (2.56218,1.11011);
\definecolor{c}{rgb}{0,0,0};
\colorlet{c}{natcomp!70};
\draw [c] (2.57032,0.995602) -- (2.57032,1.06059);
\draw [c] (2.57032,1.06059) -- (2.57032,1.12558);
\draw [c] (2.56218,1.06059) -- (2.57032,1.06059);
\draw [c] (2.57032,1.06059) -- (2.57845,1.06059);
\definecolor{c}{rgb}{0,0,0};
\colorlet{c}{natcomp!70};
\draw [c] (2.58659,0.99475) -- (2.58659,1.05464);
\draw [c] (2.58659,1.05464) -- (2.58659,1.11453);
\draw [c] (2.57845,1.05464) -- (2.58659,1.05464);
\draw [c] (2.58659,1.05464) -- (2.59473,1.05464);
\definecolor{c}{rgb}{0,0,0};
\colorlet{c}{natcomp!70};
\draw [c] (2.60286,1.07289) -- (2.60286,1.14199);
\draw [c] (2.60286,1.14199) -- (2.60286,1.21109);
\draw [c] (2.59473,1.14199) -- (2.60286,1.14199);
\draw [c] (2.60286,1.14199) -- (2.611,1.14199);
\definecolor{c}{rgb}{0,0,0};
\colorlet{c}{natcomp!70};
\draw [c] (2.61914,1.13773) -- (2.61914,1.21239);
\draw [c] (2.61914,1.21239) -- (2.61914,1.28705);
\draw [c] (2.611,1.21239) -- (2.61914,1.21239);
\draw [c] (2.61914,1.21239) -- (2.62727,1.21239);
\definecolor{c}{rgb}{0,0,0};
\colorlet{c}{natcomp!70};
\draw [c] (2.63541,0.996344) -- (2.63541,1.05637);
\draw [c] (2.63541,1.05637) -- (2.63541,1.1164);
\draw [c] (2.62727,1.05637) -- (2.63541,1.05637);
\draw [c] (2.63541,1.05637) -- (2.64355,1.05637);
\definecolor{c}{rgb}{0,0,0};
\colorlet{c}{natcomp!70};
\draw [c] (2.65168,0.988957) -- (2.65168,1.0492);
\draw [c] (2.65168,1.0492) -- (2.65168,1.10944);
\draw [c] (2.64355,1.0492) -- (2.65168,1.0492);
\draw [c] (2.65168,1.0492) -- (2.65982,1.0492);
\definecolor{c}{rgb}{0,0,0};
\colorlet{c}{natcomp!70};
\draw [c] (2.66795,1.1185) -- (2.66795,1.18712);
\draw [c] (2.66795,1.18712) -- (2.66795,1.25574);
\draw [c] (2.65982,1.18712) -- (2.66795,1.18712);
\draw [c] (2.66795,1.18712) -- (2.67609,1.18712);
\definecolor{c}{rgb}{0,0,0};
\colorlet{c}{natcomp!70};
\draw [c] (2.68423,1.11432) -- (2.68423,1.19207);
\draw [c] (2.68423,1.19207) -- (2.68423,1.26981);
\draw [c] (2.67609,1.19207) -- (2.68423,1.19207);
\draw [c] (2.68423,1.19207) -- (2.69236,1.19207);
\definecolor{c}{rgb}{0,0,0};
\colorlet{c}{natcomp!70};
\draw [c] (2.7005,1.21466) -- (2.7005,1.29003);
\draw [c] (2.7005,1.29003) -- (2.7005,1.3654);
\draw [c] (2.69236,1.29003) -- (2.7005,1.29003);
\draw [c] (2.7005,1.29003) -- (2.70864,1.29003);
\definecolor{c}{rgb}{0,0,0};
\colorlet{c}{natcomp!70};
\draw [c] (2.71677,1.10429) -- (2.71677,1.17543);
\draw [c] (2.71677,1.17543) -- (2.71677,1.24658);
\draw [c] (2.70864,1.17543) -- (2.71677,1.17543);
\draw [c] (2.71677,1.17543) -- (2.72491,1.17543);
\definecolor{c}{rgb}{0,0,0};
\colorlet{c}{natcomp!70};
\draw [c] (2.73305,1.12661) -- (2.73305,1.2009);
\draw [c] (2.73305,1.2009) -- (2.73305,1.27519);
\draw [c] (2.72491,1.2009) -- (2.73305,1.2009);
\draw [c] (2.73305,1.2009) -- (2.74118,1.2009);
\definecolor{c}{rgb}{0,0,0};
\colorlet{c}{natcomp!70};
\draw [c] (2.74932,1.06158) -- (2.74932,1.12477);
\draw [c] (2.74932,1.12477) -- (2.74932,1.18796);
\draw [c] (2.74118,1.12477) -- (2.74932,1.12477);
\draw [c] (2.74932,1.12477) -- (2.75745,1.12477);
\definecolor{c}{rgb}{0,0,0};
\colorlet{c}{natcomp!70};
\draw [c] (2.76559,1.18504) -- (2.76559,1.26123);
\draw [c] (2.76559,1.26123) -- (2.76559,1.33741);
\draw [c] (2.75745,1.26123) -- (2.76559,1.26123);
\draw [c] (2.76559,1.26123) -- (2.77373,1.26123);
\definecolor{c}{rgb}{0,0,0};
\colorlet{c}{natcomp!70};
\draw [c] (2.78186,1.23072) -- (2.78186,1.31228);
\draw [c] (2.78186,1.31228) -- (2.78186,1.39384);
\draw [c] (2.77373,1.31228) -- (2.78186,1.31228);
\draw [c] (2.78186,1.31228) -- (2.79,1.31228);
\definecolor{c}{rgb}{0,0,0};
\colorlet{c}{natcomp!70};
\draw [c] (2.79814,1.28934) -- (2.79814,1.37187);
\draw [c] (2.79814,1.37187) -- (2.79814,1.45439);
\draw [c] (2.79,1.37187) -- (2.79814,1.37187);
\draw [c] (2.79814,1.37187) -- (2.80627,1.37187);
\definecolor{c}{rgb}{0,0,0};
\colorlet{c}{natcomp!70};
\draw [c] (2.81441,1.32982) -- (2.81441,1.41625);
\draw [c] (2.81441,1.41625) -- (2.81441,1.50267);
\draw [c] (2.80627,1.41625) -- (2.81441,1.41625);
\draw [c] (2.81441,1.41625) -- (2.82255,1.41625);
\definecolor{c}{rgb}{0,0,0};
\colorlet{c}{natcomp!70};
\draw [c] (2.83068,1.30347) -- (2.83068,1.38681);
\draw [c] (2.83068,1.38681) -- (2.83068,1.47015);
\draw [c] (2.82255,1.38681) -- (2.83068,1.38681);
\draw [c] (2.83068,1.38681) -- (2.83882,1.38681);
\definecolor{c}{rgb}{0,0,0};
\colorlet{c}{natcomp!70};
\draw [c] (2.84695,1.23426) -- (2.84695,1.31216);
\draw [c] (2.84695,1.31216) -- (2.84695,1.39006);
\draw [c] (2.83882,1.31216) -- (2.84695,1.31216);
\draw [c] (2.84695,1.31216) -- (2.85509,1.31216);
\definecolor{c}{rgb}{0,0,0};
\colorlet{c}{natcomp!70};
\draw [c] (2.86323,1.36782) -- (2.86323,1.45635);
\draw [c] (2.86323,1.45635) -- (2.86323,1.54489);
\draw [c] (2.85509,1.45635) -- (2.86323,1.45635);
\draw [c] (2.86323,1.45635) -- (2.87136,1.45635);
\definecolor{c}{rgb}{0,0,0};
\colorlet{c}{natcomp!70};
\draw [c] (2.8795,1.31293) -- (2.8795,1.3965);
\draw [c] (2.8795,1.3965) -- (2.8795,1.48008);
\draw [c] (2.87136,1.3965) -- (2.8795,1.3965);
\draw [c] (2.8795,1.3965) -- (2.88764,1.3965);
\definecolor{c}{rgb}{0,0,0};
\colorlet{c}{natcomp!70};
\draw [c] (2.89577,1.42357) -- (2.89577,1.51525);
\draw [c] (2.89577,1.51525) -- (2.89577,1.60692);
\draw [c] (2.88764,1.51525) -- (2.89577,1.51525);
\draw [c] (2.89577,1.51525) -- (2.90391,1.51525);
\definecolor{c}{rgb}{0,0,0};
\colorlet{c}{natcomp!70};
\draw [c] (2.91205,1.43079) -- (2.91205,1.52154);
\draw [c] (2.91205,1.52154) -- (2.91205,1.61228);
\draw [c] (2.90391,1.52154) -- (2.91205,1.52154);
\draw [c] (2.91205,1.52154) -- (2.92018,1.52154);
\definecolor{c}{rgb}{0,0,0};
\colorlet{c}{natcomp!70};
\draw [c] (2.92832,1.34088) -- (2.92832,1.42855);
\draw [c] (2.92832,1.42855) -- (2.92832,1.51622);
\draw [c] (2.92018,1.42855) -- (2.92832,1.42855);
\draw [c] (2.92832,1.42855) -- (2.93645,1.42855);
\definecolor{c}{rgb}{0,0,0};
\colorlet{c}{natcomp!70};
\draw [c] (2.94459,1.41547) -- (2.94459,1.50915);
\draw [c] (2.94459,1.50915) -- (2.94459,1.60282);
\draw [c] (2.93645,1.50915) -- (2.94459,1.50915);
\draw [c] (2.94459,1.50915) -- (2.95273,1.50915);
\definecolor{c}{rgb}{0,0,0};
\colorlet{c}{natcomp!70};
\draw [c] (2.96086,1.58909) -- (2.96086,1.69352);
\draw [c] (2.96086,1.69352) -- (2.96086,1.79794);
\draw [c] (2.95273,1.69352) -- (2.96086,1.69352);
\draw [c] (2.96086,1.69352) -- (2.969,1.69352);
\definecolor{c}{rgb}{0,0,0};
\colorlet{c}{natcomp!70};
\draw [c] (2.97714,1.33395) -- (2.97714,1.41959);
\draw [c] (2.97714,1.41959) -- (2.97714,1.50522);
\draw [c] (2.969,1.41959) -- (2.97714,1.41959);
\draw [c] (2.97714,1.41959) -- (2.98527,1.41959);
\definecolor{c}{rgb}{0,0,0};
\colorlet{c}{natcomp!70};
\draw [c] (2.99341,1.59874) -- (2.99341,1.69978);
\draw [c] (2.99341,1.69978) -- (2.99341,1.80083);
\draw [c] (2.98527,1.69978) -- (2.99341,1.69978);
\draw [c] (2.99341,1.69978) -- (3.00155,1.69978);
\definecolor{c}{rgb}{0,0,0};
\colorlet{c}{natcomp!70};
\draw [c] (3.00968,1.52556) -- (3.00968,1.61936);
\draw [c] (3.00968,1.61936) -- (3.00968,1.71316);
\draw [c] (3.00155,1.61936) -- (3.00968,1.61936);
\draw [c] (3.00968,1.61936) -- (3.01782,1.61936);
\definecolor{c}{rgb}{0,0,0};
\colorlet{c}{natcomp!70};
\draw [c] (3.02595,1.6018) -- (3.02595,1.70136);
\draw [c] (3.02595,1.70136) -- (3.02595,1.80091);
\draw [c] (3.01782,1.70136) -- (3.02595,1.70136);
\draw [c] (3.02595,1.70136) -- (3.03409,1.70136);
\definecolor{c}{rgb}{0,0,0};
\colorlet{c}{natcomp!70};
\draw [c] (3.04223,1.58016) -- (3.04223,1.68072);
\draw [c] (3.04223,1.68072) -- (3.04223,1.78129);
\draw [c] (3.03409,1.68072) -- (3.04223,1.68072);
\draw [c] (3.04223,1.68072) -- (3.05036,1.68072);
\definecolor{c}{rgb}{0,0,0};
\colorlet{c}{natcomp!70};
\draw [c] (3.0585,1.72343) -- (3.0585,1.82892);
\draw [c] (3.0585,1.82892) -- (3.0585,1.93442);
\draw [c] (3.05036,1.82892) -- (3.0585,1.82892);
\draw [c] (3.0585,1.82892) -- (3.06664,1.82892);
\definecolor{c}{rgb}{0,0,0};
\colorlet{c}{natcomp!70};
\draw [c] (3.07477,1.80323) -- (3.07477,1.92123);
\draw [c] (3.07477,1.92123) -- (3.07477,2.03923);
\draw [c] (3.06664,1.92123) -- (3.07477,1.92123);
\draw [c] (3.07477,1.92123) -- (3.08291,1.92123);
\definecolor{c}{rgb}{0,0,0};
\colorlet{c}{natcomp!70};
\draw [c] (3.09105,1.72774) -- (3.09105,1.8397);
\draw [c] (3.09105,1.8397) -- (3.09105,1.95166);
\draw [c] (3.08291,1.8397) -- (3.09105,1.8397);
\draw [c] (3.09105,1.8397) -- (3.09918,1.8397);
\definecolor{c}{rgb}{0,0,0};
\colorlet{c}{natcomp!70};
\draw [c] (3.10732,1.5984) -- (3.10732,1.70303);
\draw [c] (3.10732,1.70303) -- (3.10732,1.80766);
\draw [c] (3.09918,1.70303) -- (3.10732,1.70303);
\draw [c] (3.10732,1.70303) -- (3.11545,1.70303);
\definecolor{c}{rgb}{0,0,0};
\colorlet{c}{natcomp!70};
\draw [c] (3.12359,1.72127) -- (3.12359,1.83007);
\draw [c] (3.12359,1.83007) -- (3.12359,1.93888);
\draw [c] (3.11545,1.83007) -- (3.12359,1.83007);
\draw [c] (3.12359,1.83007) -- (3.13173,1.83007);
\definecolor{c}{rgb}{0,0,0};
\colorlet{c}{natcomp!70};
\draw [c] (3.13986,1.64464) -- (3.13986,1.74758);
\draw [c] (3.13986,1.74758) -- (3.13986,1.85053);
\draw [c] (3.13173,1.74758) -- (3.13986,1.74758);
\draw [c] (3.13986,1.74758) -- (3.148,1.74758);
\definecolor{c}{rgb}{0,0,0};
\colorlet{c}{natcomp!70};
\draw [c] (3.15614,1.69808) -- (3.15614,1.80701);
\draw [c] (3.15614,1.80701) -- (3.15614,1.91594);
\draw [c] (3.148,1.80701) -- (3.15614,1.80701);
\draw [c] (3.15614,1.80701) -- (3.16427,1.80701);
\definecolor{c}{rgb}{0,0,0};
\colorlet{c}{natcomp!70};
\draw [c] (3.17241,1.85831) -- (3.17241,1.98447);
\draw [c] (3.17241,1.98447) -- (3.17241,2.11062);
\draw [c] (3.16427,1.98447) -- (3.17241,1.98447);
\draw [c] (3.17241,1.98447) -- (3.18055,1.98447);
\definecolor{c}{rgb}{0,0,0};
\colorlet{c}{natcomp!70};
\draw [c] (3.18868,1.75745) -- (3.18868,1.87281);
\draw [c] (3.18868,1.87281) -- (3.18868,1.98818);
\draw [c] (3.18055,1.87281) -- (3.18868,1.87281);
\draw [c] (3.18868,1.87281) -- (3.19682,1.87281);
\definecolor{c}{rgb}{0,0,0};
\colorlet{c}{natcomp!70};
\draw [c] (3.20495,1.88058) -- (3.20495,2.00281);
\draw [c] (3.20495,2.00281) -- (3.20495,2.12505);
\draw [c] (3.19682,2.00281) -- (3.20495,2.00281);
\draw [c] (3.20495,2.00281) -- (3.21309,2.00281);
\definecolor{c}{rgb}{0,0,0};
\colorlet{c}{natcomp!70};
\draw [c] (3.22123,1.80611) -- (3.22123,1.92753);
\draw [c] (3.22123,1.92753) -- (3.22123,2.04895);
\draw [c] (3.21309,1.92753) -- (3.22123,1.92753);
\draw [c] (3.22123,1.92753) -- (3.22936,1.92753);
\definecolor{c}{rgb}{0,0,0};
\colorlet{c}{natcomp!70};
\draw [c] (3.2375,1.91345) -- (3.2375,2.03616);
\draw [c] (3.2375,2.03616) -- (3.2375,2.15888);
\draw [c] (3.22936,2.03616) -- (3.2375,2.03616);
\draw [c] (3.2375,2.03616) -- (3.24564,2.03616);
\definecolor{c}{rgb}{0,0,0};
\colorlet{c}{natcomp!70};
\draw [c] (3.25377,1.96375) -- (3.25377,2.09043);
\draw [c] (3.25377,2.09043) -- (3.25377,2.21712);
\draw [c] (3.24564,2.09043) -- (3.25377,2.09043);
\draw [c] (3.25377,2.09043) -- (3.26191,2.09043);
\definecolor{c}{rgb}{0,0,0};
\colorlet{c}{natcomp!70};
\draw [c] (3.27005,1.92288) -- (3.27005,2.04725);
\draw [c] (3.27005,2.04725) -- (3.27005,2.17162);
\draw [c] (3.26191,2.04725) -- (3.27005,2.04725);
\draw [c] (3.27005,2.04725) -- (3.27818,2.04725);
\definecolor{c}{rgb}{0,0,0};
\colorlet{c}{natcomp!70};
\draw [c] (3.28632,1.84932) -- (3.28632,1.97407);
\draw [c] (3.28632,1.97407) -- (3.28632,2.09882);
\draw [c] (3.27818,1.97407) -- (3.28632,1.97407);
\draw [c] (3.28632,1.97407) -- (3.29445,1.97407);
\definecolor{c}{rgb}{0,0,0};
\colorlet{c}{natcomp!70};
\draw [c] (3.30259,1.90077) -- (3.30259,2.02007);
\draw [c] (3.30259,2.02007) -- (3.30259,2.13937);
\draw [c] (3.29445,2.02007) -- (3.30259,2.02007);
\draw [c] (3.30259,2.02007) -- (3.31073,2.02007);
\definecolor{c}{rgb}{0,0,0};
\colorlet{c}{natcomp!70};
\draw [c] (3.31886,1.83797) -- (3.31886,1.96045);
\draw [c] (3.31886,1.96045) -- (3.31886,2.08293);
\draw [c] (3.31073,1.96045) -- (3.31886,1.96045);
\draw [c] (3.31886,1.96045) -- (3.327,1.96045);
\definecolor{c}{rgb}{0,0,0};
\colorlet{c}{natcomp!70};
\draw [c] (3.33514,1.88986) -- (3.33514,2.01298);
\draw [c] (3.33514,2.01298) -- (3.33514,2.1361);
\draw [c] (3.327,2.01298) -- (3.33514,2.01298);
\draw [c] (3.33514,2.01298) -- (3.34327,2.01298);
\definecolor{c}{rgb}{0,0,0};
\colorlet{c}{natcomp!70};
\draw [c] (3.35141,1.97995) -- (3.35141,2.10928);
\draw [c] (3.35141,2.10928) -- (3.35141,2.2386);
\draw [c] (3.34327,2.10928) -- (3.35141,2.10928);
\draw [c] (3.35141,2.10928) -- (3.35955,2.10928);
\definecolor{c}{rgb}{0,0,0};
\colorlet{c}{natcomp!70};
\draw [c] (3.36768,1.69146) -- (3.36768,1.80413);
\draw [c] (3.36768,1.80413) -- (3.36768,1.9168);
\draw [c] (3.35955,1.80413) -- (3.36768,1.80413);
\draw [c] (3.36768,1.80413) -- (3.37582,1.80413);
\definecolor{c}{rgb}{0,0,0};
\colorlet{c}{natcomp!70};
\draw [c] (3.38395,1.76713) -- (3.38395,1.8867);
\draw [c] (3.38395,1.8867) -- (3.38395,2.00627);
\draw [c] (3.37582,1.8867) -- (3.38395,1.8867);
\draw [c] (3.38395,1.8867) -- (3.39209,1.8867);
\definecolor{c}{rgb}{0,0,0};
\colorlet{c}{natcomp!70};
\draw [c] (3.40023,1.73703) -- (3.40023,1.85405);
\draw [c] (3.40023,1.85405) -- (3.40023,1.97106);
\draw [c] (3.39209,1.85405) -- (3.40023,1.85405);
\draw [c] (3.40023,1.85405) -- (3.40836,1.85405);
\definecolor{c}{rgb}{0,0,0};
\colorlet{c}{natcomp!70};
\draw [c] (3.4165,1.75427) -- (3.4165,1.87418);
\draw [c] (3.4165,1.87418) -- (3.4165,1.99409);
\draw [c] (3.40836,1.87418) -- (3.4165,1.87418);
\draw [c] (3.4165,1.87418) -- (3.42464,1.87418);
\definecolor{c}{rgb}{0,0,0};
\colorlet{c}{natcomp!70};
\draw [c] (3.43277,1.59071) -- (3.43277,1.70168);
\draw [c] (3.43277,1.70168) -- (3.43277,1.81264);
\draw [c] (3.42464,1.70168) -- (3.43277,1.70168);
\draw [c] (3.43277,1.70168) -- (3.44091,1.70168);
\definecolor{c}{rgb}{0,0,0};
\colorlet{c}{natcomp!70};
\draw [c] (3.44905,2.0116) -- (3.44905,2.1488);
\draw [c] (3.44905,2.1488) -- (3.44905,2.286);
\draw [c] (3.44091,2.1488) -- (3.44905,2.1488);
\draw [c] (3.44905,2.1488) -- (3.45718,2.1488);
\definecolor{c}{rgb}{0,0,0};
\colorlet{c}{natcomp!70};
\draw [c] (3.46532,1.58379) -- (3.46532,1.69731);
\draw [c] (3.46532,1.69731) -- (3.46532,1.81082);
\draw [c] (3.45718,1.69731) -- (3.46532,1.69731);
\draw [c] (3.46532,1.69731) -- (3.47345,1.69731);
\definecolor{c}{rgb}{0,0,0};
\colorlet{c}{natcomp!70};
\draw [c] (3.48159,1.69179) -- (3.48159,1.8072);
\draw [c] (3.48159,1.8072) -- (3.48159,1.92261);
\draw [c] (3.47345,1.8072) -- (3.48159,1.8072);
\draw [c] (3.48159,1.8072) -- (3.48973,1.8072);
\definecolor{c}{rgb}{0,0,0};
\colorlet{c}{natcomp!70};
\draw [c] (3.49786,1.66795) -- (3.49786,1.7778);
\draw [c] (3.49786,1.7778) -- (3.49786,1.88766);
\draw [c] (3.48973,1.7778) -- (3.49786,1.7778);
\draw [c] (3.49786,1.7778) -- (3.506,1.7778);
\definecolor{c}{rgb}{0,0,0};
\colorlet{c}{natcomp!70};
\draw [c] (3.51414,1.54378) -- (3.51414,1.65414);
\draw [c] (3.51414,1.65414) -- (3.51414,1.76451);
\draw [c] (3.506,1.65414) -- (3.51414,1.65414);
\draw [c] (3.51414,1.65414) -- (3.52227,1.65414);
\definecolor{c}{rgb}{0,0,0};
\colorlet{c}{natcomp!70};
\draw [c] (3.53041,1.63429) -- (3.53041,1.75469);
\draw [c] (3.53041,1.75469) -- (3.53041,1.87509);
\draw [c] (3.52227,1.75469) -- (3.53041,1.75469);
\draw [c] (3.53041,1.75469) -- (3.53855,1.75469);
\definecolor{c}{rgb}{0,0,0};
\colorlet{c}{natcomp!70};
\draw [c] (3.54668,1.49617) -- (3.54668,1.60081);
\draw [c] (3.54668,1.60081) -- (3.54668,1.70545);
\draw [c] (3.53855,1.60081) -- (3.54668,1.60081);
\draw [c] (3.54668,1.60081) -- (3.55482,1.60081);
\definecolor{c}{rgb}{0,0,0};
\colorlet{c}{natcomp!70};
\draw [c] (3.56295,1.58314) -- (3.56295,1.69229);
\draw [c] (3.56295,1.69229) -- (3.56295,1.80144);
\draw [c] (3.55482,1.69229) -- (3.56295,1.69229);
\draw [c] (3.56295,1.69229) -- (3.57109,1.69229);
\definecolor{c}{rgb}{0,0,0};
\colorlet{c}{natcomp!70};
\draw [c] (3.57923,1.46417) -- (3.57923,1.56462);
\draw [c] (3.57923,1.56462) -- (3.57923,1.66507);
\draw [c] (3.57109,1.56462) -- (3.57923,1.56462);
\draw [c] (3.57923,1.56462) -- (3.58736,1.56462);
\definecolor{c}{rgb}{0,0,0};
\colorlet{c}{natcomp!70};
\draw [c] (3.5955,1.7506) -- (3.5955,1.87231);
\draw [c] (3.5955,1.87231) -- (3.5955,1.99402);
\draw [c] (3.58736,1.87231) -- (3.5955,1.87231);
\draw [c] (3.5955,1.87231) -- (3.60364,1.87231);
\definecolor{c}{rgb}{0,0,0};
\colorlet{c}{natcomp!70};
\draw [c] (3.61177,1.53631) -- (3.61177,1.63577);
\draw [c] (3.61177,1.63577) -- (3.61177,1.73524);
\draw [c] (3.60364,1.63577) -- (3.61177,1.63577);
\draw [c] (3.61177,1.63577) -- (3.61991,1.63577);
\definecolor{c}{rgb}{0,0,0};
\colorlet{c}{natcomp!70};
\draw [c] (3.62805,1.60831) -- (3.62805,1.71397);
\draw [c] (3.62805,1.71397) -- (3.62805,1.81962);
\draw [c] (3.61991,1.71397) -- (3.62805,1.71397);
\draw [c] (3.62805,1.71397) -- (3.63618,1.71397);
\definecolor{c}{rgb}{0,0,0};
\colorlet{c}{natcomp!70};
\draw [c] (3.64432,1.41401) -- (3.64432,1.5139);
\draw [c] (3.64432,1.5139) -- (3.64432,1.61378);
\draw [c] (3.63618,1.5139) -- (3.64432,1.5139);
\draw [c] (3.64432,1.5139) -- (3.65245,1.5139);
\definecolor{c}{rgb}{0,0,0};
\colorlet{c}{natcomp!70};
\draw [c] (3.66059,1.47867) -- (3.66059,1.57869);
\draw [c] (3.66059,1.57869) -- (3.66059,1.67871);
\draw [c] (3.65245,1.57869) -- (3.66059,1.57869);
\draw [c] (3.66059,1.57869) -- (3.66873,1.57869);
\definecolor{c}{rgb}{0,0,0};
\colorlet{c}{natcomp!70};
\draw [c] (3.67686,1.43725) -- (3.67686,1.53155);
\draw [c] (3.67686,1.53155) -- (3.67686,1.62584);
\draw [c] (3.66873,1.53155) -- (3.67686,1.53155);
\draw [c] (3.67686,1.53155) -- (3.685,1.53155);
\definecolor{c}{rgb}{0,0,0};
\colorlet{c}{natcomp!70};
\draw [c] (3.69314,1.44595) -- (3.69314,1.54196);
\draw [c] (3.69314,1.54196) -- (3.69314,1.63798);
\draw [c] (3.685,1.54196) -- (3.69314,1.54196);
\draw [c] (3.69314,1.54196) -- (3.70127,1.54196);
\definecolor{c}{rgb}{0,0,0};
\colorlet{c}{natcomp!70};
\draw [c] (3.70941,1.31331) -- (3.70941,1.4008);
\draw [c] (3.70941,1.4008) -- (3.70941,1.48828);
\draw [c] (3.70127,1.4008) -- (3.70941,1.4008);
\draw [c] (3.70941,1.4008) -- (3.71755,1.4008);
\definecolor{c}{rgb}{0,0,0};
\colorlet{c}{natcomp!70};
\draw [c] (3.72568,1.38718) -- (3.72568,1.47882);
\draw [c] (3.72568,1.47882) -- (3.72568,1.57046);
\draw [c] (3.71755,1.47882) -- (3.72568,1.47882);
\draw [c] (3.72568,1.47882) -- (3.73382,1.47882);
\definecolor{c}{rgb}{0,0,0};
\colorlet{c}{natcomp!70};
\draw [c] (3.74195,1.30543) -- (3.74195,1.39264);
\draw [c] (3.74195,1.39264) -- (3.74195,1.47984);
\draw [c] (3.73382,1.39264) -- (3.74195,1.39264);
\draw [c] (3.74195,1.39264) -- (3.75009,1.39264);
\definecolor{c}{rgb}{0,0,0};
\colorlet{c}{natcomp!70};
\draw [c] (3.75823,1.33609) -- (3.75823,1.42593);
\draw [c] (3.75823,1.42593) -- (3.75823,1.51577);
\draw [c] (3.75009,1.42593) -- (3.75823,1.42593);
\draw [c] (3.75823,1.42593) -- (3.76636,1.42593);
\definecolor{c}{rgb}{0,0,0};
\colorlet{c}{natcomp!70};
\draw [c] (3.7745,1.48113) -- (3.7745,1.57802);
\draw [c] (3.7745,1.57802) -- (3.7745,1.67491);
\draw [c] (3.76636,1.57802) -- (3.7745,1.57802);
\draw [c] (3.7745,1.57802) -- (3.78264,1.57802);
\definecolor{c}{rgb}{0,0,0};
\colorlet{c}{natcomp!70};
\draw [c] (3.79077,1.36985) -- (3.79077,1.4665);
\draw [c] (3.79077,1.4665) -- (3.79077,1.56316);
\draw [c] (3.78264,1.4665) -- (3.79077,1.4665);
\draw [c] (3.79077,1.4665) -- (3.79891,1.4665);
\definecolor{c}{rgb}{0,0,0};
\colorlet{c}{natcomp!70};
\draw [c] (3.80705,1.3613) -- (3.80705,1.45155);
\draw [c] (3.80705,1.45155) -- (3.80705,1.5418);
\draw [c] (3.79891,1.45155) -- (3.80705,1.45155);
\draw [c] (3.80705,1.45155) -- (3.81518,1.45155);
\definecolor{c}{rgb}{0,0,0};
\colorlet{c}{natcomp!70};
\draw [c] (3.82332,1.34548) -- (3.82332,1.43349);
\draw [c] (3.82332,1.43349) -- (3.82332,1.5215);
\draw [c] (3.81518,1.43349) -- (3.82332,1.43349);
\draw [c] (3.82332,1.43349) -- (3.83145,1.43349);
\definecolor{c}{rgb}{0,0,0};
\colorlet{c}{natcomp!70};
\draw [c] (3.83959,1.34648) -- (3.83959,1.44227);
\draw [c] (3.83959,1.44227) -- (3.83959,1.53807);
\draw [c] (3.83145,1.44227) -- (3.83959,1.44227);
\draw [c] (3.83959,1.44227) -- (3.84773,1.44227);
\definecolor{c}{rgb}{0,0,0};
\colorlet{c}{natcomp!70};
\draw [c] (3.85586,1.26603) -- (3.85586,1.34858);
\draw [c] (3.85586,1.34858) -- (3.85586,1.43114);
\draw [c] (3.84773,1.34858) -- (3.85586,1.34858);
\draw [c] (3.85586,1.34858) -- (3.864,1.34858);
\definecolor{c}{rgb}{0,0,0};
\colorlet{c}{natcomp!70};
\draw [c] (3.87214,1.19945) -- (3.87214,1.28167);
\draw [c] (3.87214,1.28167) -- (3.87214,1.3639);
\draw [c] (3.864,1.28167) -- (3.87214,1.28167);
\draw [c] (3.87214,1.28167) -- (3.88027,1.28167);
\definecolor{c}{rgb}{0,0,0};
\colorlet{c}{natcomp!70};
\draw [c] (3.88841,1.37139) -- (3.88841,1.46647);
\draw [c] (3.88841,1.46647) -- (3.88841,1.56155);
\draw [c] (3.88027,1.46647) -- (3.88841,1.46647);
\draw [c] (3.88841,1.46647) -- (3.89655,1.46647);
\definecolor{c}{rgb}{0,0,0};
\colorlet{c}{natcomp!70};
\draw [c] (3.90468,1.0781) -- (3.90468,1.14498);
\draw [c] (3.90468,1.14498) -- (3.90468,1.21187);
\draw [c] (3.89655,1.14498) -- (3.90468,1.14498);
\draw [c] (3.90468,1.14498) -- (3.91282,1.14498);
\definecolor{c}{rgb}{0,0,0};
\colorlet{c}{natcomp!70};
\draw [c] (3.92095,1.12409) -- (3.92095,1.19694);
\draw [c] (3.92095,1.19694) -- (3.92095,1.26979);
\draw [c] (3.91282,1.19694) -- (3.92095,1.19694);
\draw [c] (3.92095,1.19694) -- (3.92909,1.19694);
\definecolor{c}{rgb}{0,0,0};
\colorlet{c}{natcomp!70};
\draw [c] (3.93723,1.21279) -- (3.93723,1.29838);
\draw [c] (3.93723,1.29838) -- (3.93723,1.38397);
\draw [c] (3.92909,1.29838) -- (3.93723,1.29838);
\draw [c] (3.93723,1.29838) -- (3.94536,1.29838);
\definecolor{c}{rgb}{0,0,0};
\colorlet{c}{natcomp!70};
\draw [c] (3.9535,1.18838) -- (3.9535,1.26484);
\draw [c] (3.9535,1.26484) -- (3.9535,1.3413);
\draw [c] (3.94536,1.26484) -- (3.9535,1.26484);
\draw [c] (3.9535,1.26484) -- (3.96164,1.26484);
\definecolor{c}{rgb}{0,0,0};
\colorlet{c}{natcomp!70};
\draw [c] (3.96977,1.13244) -- (3.96977,1.20537);
\draw [c] (3.96977,1.20537) -- (3.96977,1.2783);
\draw [c] (3.96164,1.20537) -- (3.96977,1.20537);
\draw [c] (3.96977,1.20537) -- (3.97791,1.20537);
\definecolor{c}{rgb}{0,0,0};
\colorlet{c}{natcomp!70};
\draw [c] (3.98605,1.18644) -- (3.98605,1.26695);
\draw [c] (3.98605,1.26695) -- (3.98605,1.34746);
\draw [c] (3.97791,1.26695) -- (3.98605,1.26695);
\draw [c] (3.98605,1.26695) -- (3.99418,1.26695);
\definecolor{c}{rgb}{0,0,0};
\colorlet{c}{natcomp!70};
\draw [c] (4.00232,1.18905) -- (4.00232,1.27326);
\draw [c] (4.00232,1.27326) -- (4.00232,1.35748);
\draw [c] (3.99418,1.27326) -- (4.00232,1.27326);
\draw [c] (4.00232,1.27326) -- (4.01045,1.27326);
\definecolor{c}{rgb}{0,0,0};
\colorlet{c}{natcomp!70};
\draw [c] (4.01859,1.21515) -- (4.01859,1.29654);
\draw [c] (4.01859,1.29654) -- (4.01859,1.37792);
\draw [c] (4.01045,1.29654) -- (4.01859,1.29654);
\draw [c] (4.01859,1.29654) -- (4.02673,1.29654);
\definecolor{c}{rgb}{0,0,0};
\colorlet{c}{natcomp!70};
\draw [c] (4.03486,1.0567) -- (4.03486,1.12126);
\draw [c] (4.03486,1.12126) -- (4.03486,1.18582);
\draw [c] (4.02673,1.12126) -- (4.03486,1.12126);
\draw [c] (4.03486,1.12126) -- (4.043,1.12126);
\definecolor{c}{rgb}{0,0,0};
\colorlet{c}{natcomp!70};
\draw [c] (4.05114,1.0637) -- (4.05114,1.13565);
\draw [c] (4.05114,1.13565) -- (4.05114,1.2076);
\draw [c] (4.043,1.13565) -- (4.05114,1.13565);
\draw [c] (4.05114,1.13565) -- (4.05927,1.13565);
\definecolor{c}{rgb}{0,0,0};
\colorlet{c}{natcomp!70};
\draw [c] (4.06741,1.12543) -- (4.06741,1.19947);
\draw [c] (4.06741,1.19947) -- (4.06741,1.27351);
\draw [c] (4.05927,1.19947) -- (4.06741,1.19947);
\draw [c] (4.06741,1.19947) -- (4.07555,1.19947);
\definecolor{c}{rgb}{0,0,0};
\colorlet{c}{natcomp!70};
\draw [c] (4.08368,1.11089) -- (4.08368,1.18647);
\draw [c] (4.08368,1.18647) -- (4.08368,1.26206);
\draw [c] (4.07555,1.18647) -- (4.08368,1.18647);
\draw [c] (4.08368,1.18647) -- (4.09182,1.18647);
\definecolor{c}{rgb}{0,0,0};
\colorlet{c}{natcomp!70};
\draw [c] (4.09995,1.09755) -- (4.09995,1.16945);
\draw [c] (4.09995,1.16945) -- (4.09995,1.24135);
\draw [c] (4.09182,1.16945) -- (4.09995,1.16945);
\draw [c] (4.09995,1.16945) -- (4.10809,1.16945);
\definecolor{c}{rgb}{0,0,0};
\colorlet{c}{natcomp!70};
\draw [c] (4.11623,1.07487) -- (4.11623,1.14657);
\draw [c] (4.11623,1.14657) -- (4.11623,1.21826);
\draw [c] (4.10809,1.14657) -- (4.11623,1.14657);
\draw [c] (4.11623,1.14657) -- (4.12436,1.14657);
\definecolor{c}{rgb}{0,0,0};
\colorlet{c}{natcomp!70};
\draw [c] (4.1325,1.03624) -- (4.1325,1.10277);
\draw [c] (4.1325,1.10277) -- (4.1325,1.16931);
\draw [c] (4.12436,1.10277) -- (4.1325,1.10277);
\draw [c] (4.1325,1.10277) -- (4.14064,1.10277);
\definecolor{c}{rgb}{0,0,0};
\colorlet{c}{natcomp!70};
\draw [c] (4.14877,0.952458) -- (4.14877,1.00761);
\draw [c] (4.14877,1.00761) -- (4.14877,1.06276);
\draw [c] (4.14064,1.00761) -- (4.14877,1.00761);
\draw [c] (4.14877,1.00761) -- (4.15691,1.00761);
\definecolor{c}{rgb}{0,0,0};
\colorlet{c}{natcomp!70};
\draw [c] (4.16505,1.04998) -- (4.16505,1.12012);
\draw [c] (4.16505,1.12012) -- (4.16505,1.19027);
\draw [c] (4.15691,1.12012) -- (4.16505,1.12012);
\draw [c] (4.16505,1.12012) -- (4.17318,1.12012);
\definecolor{c}{rgb}{0,0,0};
\colorlet{c}{natcomp!70};
\draw [c] (4.18132,1.08985) -- (4.18132,1.16335);
\draw [c] (4.18132,1.16335) -- (4.18132,1.23686);
\draw [c] (4.17318,1.16335) -- (4.18132,1.16335);
\draw [c] (4.18132,1.16335) -- (4.18945,1.16335);
\definecolor{c}{rgb}{0,0,0};
\colorlet{c}{natcomp!70};
\draw [c] (4.19759,1.06558) -- (4.19759,1.13256);
\draw [c] (4.19759,1.13256) -- (4.19759,1.19953);
\draw [c] (4.18945,1.13256) -- (4.19759,1.13256);
\draw [c] (4.19759,1.13256) -- (4.20573,1.13256);
\definecolor{c}{rgb}{0,0,0};
\colorlet{c}{natcomp!70};
\draw [c] (4.21386,0.99441) -- (4.21386,1.05948);
\draw [c] (4.21386,1.05948) -- (4.21386,1.12455);
\draw [c] (4.20573,1.05948) -- (4.21386,1.05948);
\draw [c] (4.21386,1.05948) -- (4.222,1.05948);
\definecolor{c}{rgb}{0,0,0};
\colorlet{c}{natcomp!70};
\draw [c] (4.23014,1.15782) -- (4.23014,1.23917);
\draw [c] (4.23014,1.23917) -- (4.23014,1.32052);
\draw [c] (4.222,1.23917) -- (4.23014,1.23917);
\draw [c] (4.23014,1.23917) -- (4.23827,1.23917);
\definecolor{c}{rgb}{0,0,0};
\colorlet{c}{natcomp!70};
\draw [c] (4.24641,0.983316) -- (4.24641,1.04763);
\draw [c] (4.24641,1.04763) -- (4.24641,1.11194);
\draw [c] (4.23827,1.04763) -- (4.24641,1.04763);
\draw [c] (4.24641,1.04763) -- (4.25455,1.04763);
\definecolor{c}{rgb}{0,0,0};
\colorlet{c}{natcomp!70};
\draw [c] (4.26268,1.03462) -- (4.26268,1.10754);
\draw [c] (4.26268,1.10754) -- (4.26268,1.18046);
\draw [c] (4.25455,1.10754) -- (4.26268,1.10754);
\draw [c] (4.26268,1.10754) -- (4.27082,1.10754);
\definecolor{c}{rgb}{0,0,0};
\colorlet{c}{natcomp!70};
\draw [c] (4.27895,0.980957) -- (4.27895,1.04168);
\draw [c] (4.27895,1.04168) -- (4.27895,1.1024);
\draw [c] (4.27082,1.04168) -- (4.27895,1.04168);
\draw [c] (4.27895,1.04168) -- (4.28709,1.04168);
\definecolor{c}{rgb}{0,0,0};
\colorlet{c}{natcomp!70};
\draw [c] (4.29523,0.98657) -- (4.29523,1.04335);
\draw [c] (4.29523,1.04335) -- (4.29523,1.10013);
\draw [c] (4.28709,1.04335) -- (4.29523,1.04335);
\draw [c] (4.29523,1.04335) -- (4.30336,1.04335);
\definecolor{c}{rgb}{0,0,0};
\colorlet{c}{natcomp!70};
\draw [c] (4.3115,0.987687) -- (4.3115,1.0472);
\draw [c] (4.3115,1.0472) -- (4.3115,1.10672);
\draw [c] (4.30336,1.0472) -- (4.3115,1.0472);
\draw [c] (4.3115,1.0472) -- (4.31964,1.0472);
\definecolor{c}{rgb}{0,0,0};
\colorlet{c}{natcomp!70};
\draw [c] (4.32777,1.01895) -- (4.32777,1.08324);
\draw [c] (4.32777,1.08324) -- (4.32777,1.14754);
\draw [c] (4.31964,1.08324) -- (4.32777,1.08324);
\draw [c] (4.32777,1.08324) -- (4.33591,1.08324);
\definecolor{c}{rgb}{0,0,0};
\colorlet{c}{natcomp!70};
\draw [c] (4.34405,0.968473) -- (4.34405,1.02884);
\draw [c] (4.34405,1.02884) -- (4.34405,1.08921);
\draw [c] (4.33591,1.02884) -- (4.34405,1.02884);
\draw [c] (4.34405,1.02884) -- (4.35218,1.02884);
\definecolor{c}{rgb}{0,0,0};
\colorlet{c}{natcomp!70};
\draw [c] (4.36032,0.974112) -- (4.36032,1.03613);
\draw [c] (4.36032,1.03613) -- (4.36032,1.09814);
\draw [c] (4.35218,1.03613) -- (4.36032,1.03613);
\draw [c] (4.36032,1.03613) -- (4.36845,1.03613);
\definecolor{c}{rgb}{0,0,0};
\colorlet{c}{natcomp!70};
\draw [c] (4.37659,0.980809) -- (4.37659,1.04153);
\draw [c] (4.37659,1.04153) -- (4.37659,1.10224);
\draw [c] (4.36845,1.04153) -- (4.37659,1.04153);
\draw [c] (4.37659,1.04153) -- (4.38473,1.04153);
\definecolor{c}{rgb}{0,0,0};
\colorlet{c}{natcomp!70};
\draw [c] (4.39286,0.971029) -- (4.39286,1.02981);
\draw [c] (4.39286,1.02981) -- (4.39286,1.08859);
\draw [c] (4.38473,1.02981) -- (4.39286,1.02981);
\draw [c] (4.39286,1.02981) -- (4.401,1.02981);
\definecolor{c}{rgb}{0,0,0};
\colorlet{c}{natcomp!70};
\draw [c] (4.40914,0.95256) -- (4.40914,1.00793);
\draw [c] (4.40914,1.00793) -- (4.40914,1.06331);
\draw [c] (4.401,1.00793) -- (4.40914,1.00793);
\draw [c] (4.40914,1.00793) -- (4.41727,1.00793);
\definecolor{c}{rgb}{0,0,0};
\colorlet{c}{natcomp!70};
\draw [c] (4.42541,1.04433) -- (4.42541,1.1097);
\draw [c] (4.42541,1.1097) -- (4.42541,1.17507);
\draw [c] (4.41727,1.1097) -- (4.42541,1.1097);
\draw [c] (4.42541,1.1097) -- (4.43355,1.1097);
\definecolor{c}{rgb}{0,0,0};
\colorlet{c}{natcomp!70};
\draw [c] (4.44168,0.921556) -- (4.44168,0.976194);
\draw [c] (4.44168,0.976194) -- (4.44168,1.03083);
\draw [c] (4.43355,0.976194) -- (4.44168,0.976194);
\draw [c] (4.44168,0.976194) -- (4.44982,0.976194);
\definecolor{c}{rgb}{0,0,0};
\colorlet{c}{natcomp!70};
\draw [c] (4.45795,0.972593) -- (4.45795,1.04013);
\draw [c] (4.45795,1.04013) -- (4.45795,1.10767);
\draw [c] (4.44982,1.04013) -- (4.45795,1.04013);
\draw [c] (4.45795,1.04013) -- (4.46609,1.04013);
\definecolor{c}{rgb}{0,0,0};
\colorlet{c}{natcomp!70};
\draw [c] (4.47423,0.946413) -- (4.47423,0.999826);
\draw [c] (4.47423,0.999826) -- (4.47423,1.05324);
\draw [c] (4.46609,0.999826) -- (4.47423,0.999826);
\draw [c] (4.47423,0.999826) -- (4.48236,0.999826);
\definecolor{c}{rgb}{0,0,0};
\colorlet{c}{natcomp!70};
\draw [c] (4.4905,0.928871) -- (4.4905,0.983045);
\draw [c] (4.4905,0.983045) -- (4.4905,1.03722);
\draw [c] (4.48236,0.983045) -- (4.4905,0.983045);
\draw [c] (4.4905,0.983045) -- (4.49864,0.983045);
\definecolor{c}{rgb}{0,0,0};
\colorlet{c}{natcomp!70};
\draw [c] (4.50677,0.92616) -- (4.50677,0.979906);
\draw [c] (4.50677,0.979906) -- (4.50677,1.03365);
\draw [c] (4.49864,0.979906) -- (4.50677,0.979906);
\draw [c] (4.50677,0.979906) -- (4.51491,0.979906);
\definecolor{c}{rgb}{0,0,0};
\colorlet{c}{natcomp!70};
\draw [c] (4.52305,0.866944) -- (4.52305,0.914198);
\draw [c] (4.52305,0.914198) -- (4.52305,0.961452);
\draw [c] (4.51491,0.914198) -- (4.52305,0.914198);
\draw [c] (4.52305,0.914198) -- (4.53118,0.914198);
\definecolor{c}{rgb}{0,0,0};
\colorlet{c}{natcomp!70};
\draw [c] (4.53932,0.956535) -- (4.53932,1.01368);
\draw [c] (4.53932,1.01368) -- (4.53932,1.07083);
\draw [c] (4.53118,1.01368) -- (4.53932,1.01368);
\draw [c] (4.53932,1.01368) -- (4.54745,1.01368);
\definecolor{c}{rgb}{0,0,0};
\colorlet{c}{natcomp!70};
\draw [c] (4.55559,1.00276) -- (4.55559,1.06462);
\draw [c] (4.55559,1.06462) -- (4.55559,1.12648);
\draw [c] (4.54745,1.06462) -- (4.55559,1.06462);
\draw [c] (4.55559,1.06462) -- (4.56373,1.06462);
\definecolor{c}{rgb}{0,0,0};
\colorlet{c}{natcomp!70};
\draw [c] (4.57186,0.919644) -- (4.57186,0.985824);
\draw [c] (4.57186,0.985824) -- (4.57186,1.052);
\draw [c] (4.56373,0.985824) -- (4.57186,0.985824);
\draw [c] (4.57186,0.985824) -- (4.58,0.985824);
\definecolor{c}{rgb}{0,0,0};
\colorlet{c}{natcomp!70};
\draw [c] (4.58814,0.882513) -- (4.58814,0.934061);
\draw [c] (4.58814,0.934061) -- (4.58814,0.985609);
\draw [c] (4.58,0.934061) -- (4.58814,0.934061);
\draw [c] (4.58814,0.934061) -- (4.59627,0.934061);
\definecolor{c}{rgb}{0,0,0};
\colorlet{c}{natcomp!70};
\draw [c] (4.60441,0.912457) -- (4.60441,0.967236);
\draw [c] (4.60441,0.967236) -- (4.60441,1.02202);
\draw [c] (4.59627,0.967236) -- (4.60441,0.967236);
\draw [c] (4.60441,0.967236) -- (4.61255,0.967236);
\definecolor{c}{rgb}{0,0,0};
\colorlet{c}{natcomp!70};
\draw [c] (4.62068,0.914155) -- (4.62068,0.969166);
\draw [c] (4.62068,0.969166) -- (4.62068,1.02418);
\draw [c] (4.61255,0.969166) -- (4.62068,0.969166);
\draw [c] (4.62068,0.969166) -- (4.62882,0.969166);
\definecolor{c}{rgb}{0,0,0};
\colorlet{c}{natcomp!70};
\draw [c] (4.63695,0.879878) -- (4.63695,0.938137);
\draw [c] (4.63695,0.938137) -- (4.63695,0.996396);
\draw [c] (4.62882,0.938137) -- (4.63695,0.938137);
\draw [c] (4.63695,0.938137) -- (4.64509,0.938137);
\definecolor{c}{rgb}{0,0,0};
\colorlet{c}{natcomp!70};
\draw [c] (4.65323,0.899783) -- (4.65323,0.956467);
\draw [c] (4.65323,0.956467) -- (4.65323,1.01315);
\draw [c] (4.64509,0.956467) -- (4.65323,0.956467);
\draw [c] (4.65323,0.956467) -- (4.66136,0.956467);
\definecolor{c}{rgb}{0,0,0};
\colorlet{c}{natcomp!70};
\draw [c] (4.6695,0.884772) -- (4.6695,0.936176);
\draw [c] (4.6695,0.936176) -- (4.6695,0.987579);
\draw [c] (4.66136,0.936176) -- (4.6695,0.936176);
\draw [c] (4.6695,0.936176) -- (4.67764,0.936176);
\definecolor{c}{rgb}{0,0,0};
\colorlet{c}{natcomp!70};
\draw [c] (4.68577,0.985752) -- (4.68577,1.05675);
\draw [c] (4.68577,1.05675) -- (4.68577,1.12775);
\draw [c] (4.67764,1.05675) -- (4.68577,1.05675);
\draw [c] (4.68577,1.05675) -- (4.69391,1.05675);
\definecolor{c}{rgb}{0,0,0};
\colorlet{c}{natcomp!70};
\draw [c] (4.70205,0.888065) -- (4.70205,0.939341);
\draw [c] (4.70205,0.939341) -- (4.70205,0.990618);
\draw [c] (4.69391,0.939341) -- (4.70205,0.939341);
\draw [c] (4.70205,0.939341) -- (4.71018,0.939341);
\definecolor{c}{rgb}{0,0,0};
\colorlet{c}{natcomp!70};
\draw [c] (4.71832,0.950968) -- (4.71832,1.01404);
\draw [c] (4.71832,1.01404) -- (4.71832,1.0771);
\draw [c] (4.71018,1.01404) -- (4.71832,1.01404);
\draw [c] (4.71832,1.01404) -- (4.72645,1.01404);
\definecolor{c}{rgb}{0,0,0};
\colorlet{c}{natcomp!70};
\draw [c] (4.73459,0.890061) -- (4.73459,0.949548);
\draw [c] (4.73459,0.949548) -- (4.73459,1.00904);
\draw [c] (4.72645,0.949548) -- (4.73459,0.949548);
\draw [c] (4.73459,0.949548) -- (4.74273,0.949548);
\definecolor{c}{rgb}{0,0,0};
\colorlet{c}{natcomp!70};
\draw [c] (4.75086,0.903141) -- (4.75086,0.959014);
\draw [c] (4.75086,0.959014) -- (4.75086,1.01489);
\draw [c] (4.74273,0.959014) -- (4.75086,0.959014);
\draw [c] (4.75086,0.959014) -- (4.759,0.959014);
\definecolor{c}{rgb}{0,0,0};
\colorlet{c}{natcomp!70};
\draw [c] (4.76714,0.878125) -- (4.76714,0.929474);
\draw [c] (4.76714,0.929474) -- (4.76714,0.980823);
\draw [c] (4.759,0.929474) -- (4.76714,0.929474);
\draw [c] (4.76714,0.929474) -- (4.77527,0.929474);
\definecolor{c}{rgb}{0,0,0};
\colorlet{c}{natcomp!70};
\draw [c] (4.78341,0.831227) -- (4.78341,0.880248);
\draw [c] (4.78341,0.880248) -- (4.78341,0.929269);
\draw [c] (4.77527,0.880248) -- (4.78341,0.880248);
\draw [c] (4.78341,0.880248) -- (4.79155,0.880248);
\definecolor{c}{rgb}{0,0,0};
\colorlet{c}{natcomp!70};
\draw [c] (4.79968,0.864439) -- (4.79968,0.917743);
\draw [c] (4.79968,0.917743) -- (4.79968,0.971047);
\draw [c] (4.79155,0.917743) -- (4.79968,0.917743);
\draw [c] (4.79968,0.917743) -- (4.80782,0.917743);
\definecolor{c}{rgb}{0,0,0};
\colorlet{c}{natcomp!70};
\draw [c] (4.81595,0.852991) -- (4.81595,0.902588);
\draw [c] (4.81595,0.902588) -- (4.81595,0.952185);
\draw [c] (4.80782,0.902588) -- (4.81595,0.902588);
\draw [c] (4.81595,0.902588) -- (4.82409,0.902588);
\definecolor{c}{rgb}{0,0,0};
\colorlet{c}{natcomp!70};
\draw [c] (4.83223,0.838268) -- (4.83223,0.879525);
\draw [c] (4.83223,0.879525) -- (4.83223,0.920782);
\draw [c] (4.82409,0.879525) -- (4.83223,0.879525);
\draw [c] (4.83223,0.879525) -- (4.84036,0.879525);
\definecolor{c}{rgb}{0,0,0};
\colorlet{c}{natcomp!70};
\draw [c] (4.8485,0.849755) -- (4.8485,0.901917);
\draw [c] (4.8485,0.901917) -- (4.8485,0.954078);
\draw [c] (4.84036,0.901917) -- (4.8485,0.901917);
\draw [c] (4.8485,0.901917) -- (4.85664,0.901917);
\definecolor{c}{rgb}{0,0,0};
\colorlet{c}{natcomp!70};
\draw [c] (4.86477,0.88973) -- (4.86477,0.942964);
\draw [c] (4.86477,0.942964) -- (4.86477,0.996198);
\draw [c] (4.85664,0.942964) -- (4.86477,0.942964);
\draw [c] (4.86477,0.942964) -- (4.87291,0.942964);
\definecolor{c}{rgb}{0,0,0};
\colorlet{c}{natcomp!70};
\draw [c] (4.88105,0.852361) -- (4.88105,0.898715);
\draw [c] (4.88105,0.898715) -- (4.88105,0.945068);
\draw [c] (4.87291,0.898715) -- (4.88105,0.898715);
\draw [c] (4.88105,0.898715) -- (4.88918,0.898715);
\definecolor{c}{rgb}{0,0,0};
\colorlet{c}{natcomp!70};
\draw [c] (4.89732,0.889338) -- (4.89732,0.94623);
\draw [c] (4.89732,0.94623) -- (4.89732,1.00312);
\draw [c] (4.88918,0.94623) -- (4.89732,0.94623);
\draw [c] (4.89732,0.94623) -- (4.90545,0.94623);
\definecolor{c}{rgb}{0,0,0};
\colorlet{c}{natcomp!70};
\draw [c] (4.91359,0.818285) -- (4.91359,0.857852);
\draw [c] (4.91359,0.857852) -- (4.91359,0.897419);
\draw [c] (4.90545,0.857852) -- (4.91359,0.857852);
\draw [c] (4.91359,0.857852) -- (4.92173,0.857852);
\definecolor{c}{rgb}{0,0,0};
\colorlet{c}{natcomp!70};
\draw [c] (4.92986,0.839442) -- (4.92986,0.884558);
\draw [c] (4.92986,0.884558) -- (4.92986,0.929674);
\draw [c] (4.92173,0.884558) -- (4.92986,0.884558);
\draw [c] (4.92986,0.884558) -- (4.938,0.884558);
\definecolor{c}{rgb}{0,0,0};
\colorlet{c}{natcomp!70};
\draw [c] (4.94614,0.928982) -- (4.94614,0.983751);
\draw [c] (4.94614,0.983751) -- (4.94614,1.03852);
\draw [c] (4.938,0.983751) -- (4.94614,0.983751);
\draw [c] (4.94614,0.983751) -- (4.95427,0.983751);
\definecolor{c}{rgb}{0,0,0};
\colorlet{c}{natcomp!70};
\draw [c] (4.96241,0.756623) -- (4.96241,0.793018);
\draw [c] (4.96241,0.793018) -- (4.96241,0.829412);
\draw [c] (4.95427,0.793018) -- (4.96241,0.793018);
\draw [c] (4.96241,0.793018) -- (4.97055,0.793018);
\definecolor{c}{rgb}{0,0,0};
\colorlet{c}{natcomp!70};
\draw [c] (4.97868,0.863301) -- (4.97868,0.913408);
\draw [c] (4.97868,0.913408) -- (4.97868,0.963516);
\draw [c] (4.97055,0.913408) -- (4.97868,0.913408);
\draw [c] (4.97868,0.913408) -- (4.98682,0.913408);
\definecolor{c}{rgb}{0,0,0};
\colorlet{c}{natcomp!70};
\draw [c] (4.99495,0.820068) -- (4.99495,0.859989);
\draw [c] (4.99495,0.859989) -- (4.99495,0.89991);
\draw [c] (4.98682,0.859989) -- (4.99495,0.859989);
\draw [c] (4.99495,0.859989) -- (5.00309,0.859989);
\definecolor{c}{rgb}{0,0,0};
\colorlet{c}{natcomp!70};
\draw [c] (5.01123,0.781416) -- (5.01123,0.816202);
\draw [c] (5.01123,0.816202) -- (5.01123,0.850987);
\draw [c] (5.00309,0.816202) -- (5.01123,0.816202);
\draw [c] (5.01123,0.816202) -- (5.01936,0.816202);
\definecolor{c}{rgb}{0,0,0};
\colorlet{c}{natcomp!70};
\draw [c] (5.0275,0.790764) -- (5.0275,0.827697);
\draw [c] (5.0275,0.827697) -- (5.0275,0.86463);
\draw [c] (5.01936,0.827697) -- (5.0275,0.827697);
\draw [c] (5.0275,0.827697) -- (5.03564,0.827697);
\definecolor{c}{rgb}{0,0,0};
\colorlet{c}{natcomp!70};
\draw [c] (5.04377,0.750942) -- (5.04377,0.784172);
\draw [c] (5.04377,0.784172) -- (5.04377,0.817401);
\draw [c] (5.03564,0.784172) -- (5.04377,0.784172);
\draw [c] (5.04377,0.784172) -- (5.05191,0.784172);
\definecolor{c}{rgb}{0,0,0};
\colorlet{c}{natcomp!70};
\draw [c] (5.06005,0.881823) -- (5.06005,0.934976);
\draw [c] (5.06005,0.934976) -- (5.06005,0.988128);
\draw [c] (5.05191,0.934976) -- (5.06005,0.934976);
\draw [c] (5.06005,0.934976) -- (5.06818,0.934976);
\definecolor{c}{rgb}{0,0,0};
\colorlet{c}{natcomp!70};
\draw [c] (5.07632,0.879455) -- (5.07632,0.936439);
\draw [c] (5.07632,0.936439) -- (5.07632,0.993423);
\draw [c] (5.06818,0.936439) -- (5.07632,0.936439);
\draw [c] (5.07632,0.936439) -- (5.08445,0.936439);
\definecolor{c}{rgb}{0,0,0};
\colorlet{c}{natcomp!70};
\draw [c] (5.09259,0.765327) -- (5.09259,0.799544);
\draw [c] (5.09259,0.799544) -- (5.09259,0.833762);
\draw [c] (5.08445,0.799544) -- (5.09259,0.799544);
\draw [c] (5.09259,0.799544) -- (5.10073,0.799544);
\definecolor{c}{rgb}{0,0,0};
\colorlet{c}{natcomp!70};
\draw [c] (5.10886,0.816846) -- (5.10886,0.854773);
\draw [c] (5.10886,0.854773) -- (5.10886,0.8927);
\draw [c] (5.10073,0.854773) -- (5.10886,0.854773);
\draw [c] (5.10886,0.854773) -- (5.117,0.854773);
\definecolor{c}{rgb}{0,0,0};
\colorlet{c}{natcomp!70};
\draw [c] (5.12514,0.878289) -- (5.12514,0.930852);
\draw [c] (5.12514,0.930852) -- (5.12514,0.983415);
\draw [c] (5.117,0.930852) -- (5.12514,0.930852);
\draw [c] (5.12514,0.930852) -- (5.13327,0.930852);
\definecolor{c}{rgb}{0,0,0};
\colorlet{c}{natcomp!70};
\draw [c] (5.14141,0.886117) -- (5.14141,0.93627);
\draw [c] (5.14141,0.93627) -- (5.14141,0.986423);
\draw [c] (5.13327,0.93627) -- (5.14141,0.93627);
\draw [c] (5.14141,0.93627) -- (5.14955,0.93627);
\definecolor{c}{rgb}{0,0,0};
\colorlet{c}{natcomp!70};
\draw [c] (5.15768,0.756901) -- (5.15768,0.7883);
\draw [c] (5.15768,0.7883) -- (5.15768,0.819698);
\draw [c] (5.14955,0.7883) -- (5.15768,0.7883);
\draw [c] (5.15768,0.7883) -- (5.16582,0.7883);
\definecolor{c}{rgb}{0,0,0};
\colorlet{c}{natcomp!70};
\draw [c] (5.17395,0.850388) -- (5.17395,0.903041);
\draw [c] (5.17395,0.903041) -- (5.17395,0.955693);
\draw [c] (5.16582,0.903041) -- (5.17395,0.903041);
\draw [c] (5.17395,0.903041) -- (5.18209,0.903041);
\definecolor{c}{rgb}{0,0,0};
\colorlet{c}{natcomp!70};
\draw [c] (5.19023,0.837397) -- (5.19023,0.880272);
\draw [c] (5.19023,0.880272) -- (5.19023,0.923147);
\draw [c] (5.18209,0.880272) -- (5.19023,0.880272);
\draw [c] (5.19023,0.880272) -- (5.19836,0.880272);
\definecolor{c}{rgb}{0,0,0};
\colorlet{c}{natcomp!70};
\draw [c] (5.2065,0.782891) -- (5.2065,0.819807);
\draw [c] (5.2065,0.819807) -- (5.2065,0.856723);
\draw [c] (5.19836,0.819807) -- (5.2065,0.819807);
\draw [c] (5.2065,0.819807) -- (5.21464,0.819807);
\definecolor{c}{rgb}{0,0,0};
\colorlet{c}{natcomp!70};
\draw [c] (5.22277,0.877003) -- (5.22277,0.925132);
\draw [c] (5.22277,0.925132) -- (5.22277,0.973261);
\draw [c] (5.21464,0.925132) -- (5.22277,0.925132);
\draw [c] (5.22277,0.925132) -- (5.23091,0.925132);
\definecolor{c}{rgb}{0,0,0};
\colorlet{c}{natcomp!70};
\draw [c] (5.23905,0.832507) -- (5.23905,0.877913);
\draw [c] (5.23905,0.877913) -- (5.23905,0.923319);
\draw [c] (5.23091,0.877913) -- (5.23905,0.877913);
\draw [c] (5.23905,0.877913) -- (5.24718,0.877913);
\definecolor{c}{rgb}{0,0,0};
\colorlet{c}{natcomp!70};
\draw [c] (5.25532,0.809327) -- (5.25532,0.853797);
\draw [c] (5.25532,0.853797) -- (5.25532,0.898268);
\draw [c] (5.24718,0.853797) -- (5.25532,0.853797);
\draw [c] (5.25532,0.853797) -- (5.26345,0.853797);
\definecolor{c}{rgb}{0,0,0};
\colorlet{c}{natcomp!70};
\draw [c] (5.27159,0.838939) -- (5.27159,0.884592);
\draw [c] (5.27159,0.884592) -- (5.27159,0.930245);
\draw [c] (5.26345,0.884592) -- (5.27159,0.884592);
\draw [c] (5.27159,0.884592) -- (5.27973,0.884592);
\definecolor{c}{rgb}{0,0,0};
\colorlet{c}{natcomp!70};
\draw [c] (5.28786,0.774997) -- (5.28786,0.815419);
\draw [c] (5.28786,0.815419) -- (5.28786,0.85584);
\draw [c] (5.27973,0.815419) -- (5.28786,0.815419);
\draw [c] (5.28786,0.815419) -- (5.296,0.815419);
\definecolor{c}{rgb}{0,0,0};
\colorlet{c}{natcomp!70};
\draw [c] (5.30414,0.802019) -- (5.30414,0.840859);
\draw [c] (5.30414,0.840859) -- (5.30414,0.879699);
\draw [c] (5.296,0.840859) -- (5.30414,0.840859);
\draw [c] (5.30414,0.840859) -- (5.31227,0.840859);
\definecolor{c}{rgb}{0,0,0};
\colorlet{c}{natcomp!70};
\draw [c] (5.32041,0.734849) -- (5.32041,0.765883);
\draw [c] (5.32041,0.765883) -- (5.32041,0.796917);
\draw [c] (5.31227,0.765883) -- (5.32041,0.765883);
\draw [c] (5.32041,0.765883) -- (5.32855,0.765883);
\definecolor{c}{rgb}{0,0,0};
\colorlet{c}{natcomp!70};
\draw [c] (5.33668,0.76708) -- (5.33668,0.802514);
\draw [c] (5.33668,0.802514) -- (5.33668,0.837949);
\draw [c] (5.32855,0.802514) -- (5.33668,0.802514);
\draw [c] (5.33668,0.802514) -- (5.34482,0.802514);
\definecolor{c}{rgb}{0,0,0};
\colorlet{c}{natcomp!70};
\draw [c] (5.35295,0.820219) -- (5.35295,0.864005);
\draw [c] (5.35295,0.864005) -- (5.35295,0.907791);
\draw [c] (5.34482,0.864005) -- (5.35295,0.864005);
\draw [c] (5.35295,0.864005) -- (5.36109,0.864005);
\definecolor{c}{rgb}{0,0,0};
\colorlet{c}{natcomp!70};
\draw [c] (5.36923,0.792286) -- (5.36923,0.832453);
\draw [c] (5.36923,0.832453) -- (5.36923,0.87262);
\draw [c] (5.36109,0.832453) -- (5.36923,0.832453);
\draw [c] (5.36923,0.832453) -- (5.37736,0.832453);
\definecolor{c}{rgb}{0,0,0};
\colorlet{c}{natcomp!70};
\draw [c] (5.3855,0.779486) -- (5.3855,0.817789);
\draw [c] (5.3855,0.817789) -- (5.3855,0.856092);
\draw [c] (5.37736,0.817789) -- (5.3855,0.817789);
\draw [c] (5.3855,0.817789) -- (5.39364,0.817789);
\definecolor{c}{rgb}{0,0,0};
\colorlet{c}{natcomp!70};
\draw [c] (5.40177,0.758942) -- (5.40177,0.79033);
\draw [c] (5.40177,0.79033) -- (5.40177,0.821719);
\draw [c] (5.39364,0.79033) -- (5.40177,0.79033);
\draw [c] (5.40177,0.79033) -- (5.40991,0.79033);
\definecolor{c}{rgb}{0,0,0};
\colorlet{c}{natcomp!70};
\draw [c] (5.41805,0.828335) -- (5.41805,0.872313);
\draw [c] (5.41805,0.872313) -- (5.41805,0.916292);
\draw [c] (5.40991,0.872313) -- (5.41805,0.872313);
\draw [c] (5.41805,0.872313) -- (5.42618,0.872313);
\definecolor{c}{rgb}{0,0,0};
\colorlet{c}{natcomp!70};
\draw [c] (5.43432,0.796227) -- (5.43432,0.833442);
\draw [c] (5.43432,0.833442) -- (5.43432,0.870657);
\draw [c] (5.42618,0.833442) -- (5.43432,0.833442);
\draw [c] (5.43432,0.833442) -- (5.44245,0.833442);
\definecolor{c}{rgb}{0,0,0};
\colorlet{c}{natcomp!70};
\draw [c] (5.45059,0.824412) -- (5.45059,0.86803);
\draw [c] (5.45059,0.86803) -- (5.45059,0.911649);
\draw [c] (5.44245,0.86803) -- (5.45059,0.86803);
\draw [c] (5.45059,0.86803) -- (5.45873,0.86803);
\definecolor{c}{rgb}{0,0,0};
\colorlet{c}{natcomp!70};
\draw [c] (5.46686,0.749011) -- (5.46686,0.778106);
\draw [c] (5.46686,0.778106) -- (5.46686,0.807201);
\draw [c] (5.45873,0.778106) -- (5.46686,0.778106);
\draw [c] (5.46686,0.778106) -- (5.475,0.778106);
\definecolor{c}{rgb}{0,0,0};
\colorlet{c}{natcomp!70};
\draw [c] (5.48314,0.717158) -- (5.48314,0.738269);
\draw [c] (5.48314,0.738269) -- (5.48314,0.75938);
\draw [c] (5.475,0.738269) -- (5.48314,0.738269);
\draw [c] (5.48314,0.738269) -- (5.49127,0.738269);
\definecolor{c}{rgb}{0,0,0};
\colorlet{c}{natcomp!70};
\draw [c] (5.49941,0.760794) -- (5.49941,0.791025);
\draw [c] (5.49941,0.791025) -- (5.49941,0.821255);
\draw [c] (5.49127,0.791025) -- (5.49941,0.791025);
\draw [c] (5.49941,0.791025) -- (5.50755,0.791025);
\definecolor{c}{rgb}{0,0,0};
\colorlet{c}{natcomp!70};
\draw [c] (5.51568,0.780683) -- (5.51568,0.819473);
\draw [c] (5.51568,0.819473) -- (5.51568,0.858263);
\draw [c] (5.50755,0.819473) -- (5.51568,0.819473);
\draw [c] (5.51568,0.819473) -- (5.52382,0.819473);
\definecolor{c}{rgb}{0,0,0};
\colorlet{c}{natcomp!70};
\draw [c] (5.53195,0.767697) -- (5.53195,0.81179);
\draw [c] (5.53195,0.81179) -- (5.53195,0.855883);
\draw [c] (5.52382,0.81179) -- (5.53195,0.81179);
\draw [c] (5.53195,0.81179) -- (5.54009,0.81179);
\definecolor{c}{rgb}{0,0,0};
\colorlet{c}{natcomp!70};
\draw [c] (5.54823,0.789662) -- (5.54823,0.82657);
\draw [c] (5.54823,0.82657) -- (5.54823,0.863477);
\draw [c] (5.54009,0.82657) -- (5.54823,0.82657);
\draw [c] (5.54823,0.82657) -- (5.55636,0.82657);
\definecolor{c}{rgb}{0,0,0};
\colorlet{c}{natcomp!70};
\draw [c] (5.5645,0.836001) -- (5.5645,0.881788);
\draw [c] (5.5645,0.881788) -- (5.5645,0.927576);
\draw [c] (5.55636,0.881788) -- (5.5645,0.881788);
\draw [c] (5.5645,0.881788) -- (5.57264,0.881788);
\definecolor{c}{rgb}{0,0,0};
\colorlet{c}{natcomp!70};
\draw [c] (5.58077,0.77978) -- (5.58077,0.81503);
\draw [c] (5.58077,0.81503) -- (5.58077,0.85028);
\draw [c] (5.57264,0.81503) -- (5.58077,0.81503);
\draw [c] (5.58077,0.81503) -- (5.58891,0.81503);
\definecolor{c}{rgb}{0,0,0};
\colorlet{c}{natcomp!70};
\draw [c] (5.59705,0.765328) -- (5.59705,0.798276);
\draw [c] (5.59705,0.798276) -- (5.59705,0.831225);
\draw [c] (5.58891,0.798276) -- (5.59705,0.798276);
\draw [c] (5.59705,0.798276) -- (5.60518,0.798276);
\definecolor{c}{rgb}{0,0,0};
\colorlet{c}{natcomp!70};
\draw [c] (5.61332,0.713101) -- (5.61332,0.735885);
\draw [c] (5.61332,0.735885) -- (5.61332,0.758668);
\draw [c] (5.60518,0.735885) -- (5.61332,0.735885);
\draw [c] (5.61332,0.735885) -- (5.62145,0.735885);
\definecolor{c}{rgb}{0,0,0};
\colorlet{c}{natcomp!70};
\draw [c] (5.62959,0.77156) -- (5.62959,0.806931);
\draw [c] (5.62959,0.806931) -- (5.62959,0.842301);
\draw [c] (5.62145,0.806931) -- (5.62959,0.806931);
\draw [c] (5.62959,0.806931) -- (5.63773,0.806931);
\definecolor{c}{rgb}{0,0,0};
\colorlet{c}{natcomp!70};
\draw [c] (5.64586,0.767598) -- (5.64586,0.801089);
\draw [c] (5.64586,0.801089) -- (5.64586,0.83458);
\draw [c] (5.63773,0.801089) -- (5.64586,0.801089);
\draw [c] (5.64586,0.801089) -- (5.654,0.801089);
\definecolor{c}{rgb}{0,0,0};
\colorlet{c}{natcomp!70};
\draw [c] (5.66214,0.812558) -- (5.66214,0.852249);
\draw [c] (5.66214,0.852249) -- (5.66214,0.891941);
\draw [c] (5.654,0.852249) -- (5.66214,0.852249);
\draw [c] (5.66214,0.852249) -- (5.67027,0.852249);
\definecolor{c}{rgb}{0,0,0};
\colorlet{c}{natcomp!70};
\draw [c] (5.67841,0.750819) -- (5.67841,0.790319);
\draw [c] (5.67841,0.790319) -- (5.67841,0.829819);
\draw [c] (5.67027,0.790319) -- (5.67841,0.790319);
\draw [c] (5.67841,0.790319) -- (5.68655,0.790319);
\definecolor{c}{rgb}{0,0,0};
\colorlet{c}{natcomp!70};
\draw [c] (5.69468,0.763809) -- (5.69468,0.800161);
\draw [c] (5.69468,0.800161) -- (5.69468,0.836514);
\draw [c] (5.68655,0.800161) -- (5.69468,0.800161);
\draw [c] (5.69468,0.800161) -- (5.70282,0.800161);
\definecolor{c}{rgb}{0,0,0};
\colorlet{c}{natcomp!70};
\draw [c] (5.71095,0.773509) -- (5.71095,0.810299);
\draw [c] (5.71095,0.810299) -- (5.71095,0.84709);
\draw [c] (5.70282,0.810299) -- (5.71095,0.810299);
\draw [c] (5.71095,0.810299) -- (5.71909,0.810299);
\definecolor{c}{rgb}{0,0,0};
\colorlet{c}{natcomp!70};
\draw [c] (5.72723,0.781311) -- (5.72723,0.815949);
\draw [c] (5.72723,0.815949) -- (5.72723,0.850586);
\draw [c] (5.71909,0.815949) -- (5.72723,0.815949);
\draw [c] (5.72723,0.815949) -- (5.73536,0.815949);
\definecolor{c}{rgb}{0,0,0};
\colorlet{c}{natcomp!70};
\draw [c] (5.7435,0.749041) -- (5.7435,0.783543);
\draw [c] (5.7435,0.783543) -- (5.7435,0.818044);
\draw [c] (5.73536,0.783543) -- (5.7435,0.783543);
\draw [c] (5.7435,0.783543) -- (5.75164,0.783543);
\definecolor{c}{rgb}{0,0,0};
\colorlet{c}{natcomp!70};
\draw [c] (5.75977,0.734525) -- (5.75977,0.760913);
\draw [c] (5.75977,0.760913) -- (5.75977,0.787301);
\draw [c] (5.75164,0.760913) -- (5.75977,0.760913);
\draw [c] (5.75977,0.760913) -- (5.76791,0.760913);
\definecolor{c}{rgb}{0,0,0};
\colorlet{c}{natcomp!70};
\draw [c] (5.77605,0.757329) -- (5.77605,0.795295);
\draw [c] (5.77605,0.795295) -- (5.77605,0.833262);
\draw [c] (5.76791,0.795295) -- (5.77605,0.795295);
\draw [c] (5.77605,0.795295) -- (5.78418,0.795295);
\definecolor{c}{rgb}{0,0,0};
\colorlet{c}{natcomp!70};
\draw [c] (5.79232,0.778456) -- (5.79232,0.824701);
\draw [c] (5.79232,0.824701) -- (5.79232,0.870946);
\draw [c] (5.78418,0.824701) -- (5.79232,0.824701);
\draw [c] (5.79232,0.824701) -- (5.80045,0.824701);
\definecolor{c}{rgb}{0,0,0};
\colorlet{c}{natcomp!70};
\draw [c] (5.80859,0.791499) -- (5.80859,0.828233);
\draw [c] (5.80859,0.828233) -- (5.80859,0.864968);
\draw [c] (5.80045,0.828233) -- (5.80859,0.828233);
\draw [c] (5.80859,0.828233) -- (5.81673,0.828233);
\definecolor{c}{rgb}{0,0,0};
\colorlet{c}{natcomp!70};
\draw [c] (5.82486,0.778777) -- (5.82486,0.81849);
\draw [c] (5.82486,0.81849) -- (5.82486,0.858202);
\draw [c] (5.81673,0.81849) -- (5.82486,0.81849);
\draw [c] (5.82486,0.81849) -- (5.833,0.81849);
\definecolor{c}{rgb}{0,0,0};
\colorlet{c}{natcomp!70};
\draw [c] (5.84114,0.766161) -- (5.84114,0.815256);
\draw [c] (5.84114,0.815256) -- (5.84114,0.864352);
\draw [c] (5.833,0.815256) -- (5.84114,0.815256);
\draw [c] (5.84114,0.815256) -- (5.84927,0.815256);
\definecolor{c}{rgb}{0,0,0};
\colorlet{c}{natcomp!70};
\draw [c] (5.85741,0.739717) -- (5.85741,0.766483);
\draw [c] (5.85741,0.766483) -- (5.85741,0.79325);
\draw [c] (5.84927,0.766483) -- (5.85741,0.766483);
\draw [c] (5.85741,0.766483) -- (5.86555,0.766483);
\definecolor{c}{rgb}{0,0,0};
\colorlet{c}{natcomp!70};
\draw [c] (5.87368,0.767167) -- (5.87368,0.798411);
\draw [c] (5.87368,0.798411) -- (5.87368,0.829655);
\draw [c] (5.86555,0.798411) -- (5.87368,0.798411);
\draw [c] (5.87368,0.798411) -- (5.88182,0.798411);
\definecolor{c}{rgb}{0,0,0};
\colorlet{c}{natcomp!70};
\draw [c] (5.88995,0.7537) -- (5.88995,0.78274);
\draw [c] (5.88995,0.78274) -- (5.88995,0.811781);
\draw [c] (5.88182,0.78274) -- (5.88995,0.78274);
\draw [c] (5.88995,0.78274) -- (5.89809,0.78274);
\definecolor{c}{rgb}{0,0,0};
\colorlet{c}{natcomp!70};
\draw [c] (5.90623,0.787736) -- (5.90623,0.823567);
\draw [c] (5.90623,0.823567) -- (5.90623,0.859398);
\draw [c] (5.89809,0.823567) -- (5.90623,0.823567);
\draw [c] (5.90623,0.823567) -- (5.91436,0.823567);
\definecolor{c}{rgb}{0,0,0};
\colorlet{c}{natcomp!70};
\draw [c] (5.9225,0.7657) -- (5.9225,0.797966);
\draw [c] (5.9225,0.797966) -- (5.9225,0.830233);
\draw [c] (5.91436,0.797966) -- (5.9225,0.797966);
\draw [c] (5.9225,0.797966) -- (5.93064,0.797966);
\definecolor{c}{rgb}{0,0,0};
\colorlet{c}{natcomp!70};
\draw [c] (5.93877,0.776906) -- (5.93877,0.812086);
\draw [c] (5.93877,0.812086) -- (5.93877,0.847266);
\draw [c] (5.93064,0.812086) -- (5.93877,0.812086);
\draw [c] (5.93877,0.812086) -- (5.94691,0.812086);
\definecolor{c}{rgb}{0,0,0};
\colorlet{c}{natcomp!70};
\draw [c] (5.95505,0.727034) -- (5.95505,0.751779);
\draw [c] (5.95505,0.751779) -- (5.95505,0.776524);
\draw [c] (5.94691,0.751779) -- (5.95505,0.751779);
\draw [c] (5.95505,0.751779) -- (5.96318,0.751779);
\definecolor{c}{rgb}{0,0,0};
\colorlet{c}{natcomp!70};
\draw [c] (5.97132,0.718677) -- (5.97132,0.7484);
\draw [c] (5.97132,0.7484) -- (5.97132,0.778123);
\draw [c] (5.96318,0.7484) -- (5.97132,0.7484);
\draw [c] (5.97132,0.7484) -- (5.97945,0.7484);
\definecolor{c}{rgb}{0,0,0};
\colorlet{c}{natcomp!70};
\draw [c] (5.98759,0.723722) -- (5.98759,0.746507);
\draw [c] (5.98759,0.746507) -- (5.98759,0.769292);
\draw [c] (5.97945,0.746507) -- (5.98759,0.746507);
\draw [c] (5.98759,0.746507) -- (5.99573,0.746507);
\definecolor{c}{rgb}{0,0,0};
\colorlet{c}{natcomp!70};
\draw [c] (6.00386,0.756977) -- (6.00386,0.792774);
\draw [c] (6.00386,0.792774) -- (6.00386,0.828571);
\draw [c] (5.99573,0.792774) -- (6.00386,0.792774);
\draw [c] (6.00386,0.792774) -- (6.012,0.792774);
\definecolor{c}{rgb}{0,0,0};
\colorlet{c}{natcomp!70};
\draw [c] (6.02014,0.72654) -- (6.02014,0.750922);
\draw [c] (6.02014,0.750922) -- (6.02014,0.775304);
\draw [c] (6.012,0.750922) -- (6.02014,0.750922);
\draw [c] (6.02014,0.750922) -- (6.02827,0.750922);
\definecolor{c}{rgb}{0,0,0};
\colorlet{c}{natcomp!70};
\draw [c] (6.03641,0.765876) -- (6.03641,0.798169);
\draw [c] (6.03641,0.798169) -- (6.03641,0.830462);
\draw [c] (6.02827,0.798169) -- (6.03641,0.798169);
\draw [c] (6.03641,0.798169) -- (6.04455,0.798169);
\definecolor{c}{rgb}{0,0,0};
\colorlet{c}{natcomp!70};
\draw [c] (6.05268,0.723692) -- (6.05268,0.763257);
\draw [c] (6.05268,0.763257) -- (6.05268,0.802821);
\draw [c] (6.04455,0.763257) -- (6.05268,0.763257);
\draw [c] (6.05268,0.763257) -- (6.06082,0.763257);
\definecolor{c}{rgb}{0,0,0};
\colorlet{c}{natcomp!70};
\draw [c] (6.06895,0.735947) -- (6.06895,0.776286);
\draw [c] (6.06895,0.776286) -- (6.06895,0.816626);
\draw [c] (6.06082,0.776286) -- (6.06895,0.776286);
\draw [c] (6.06895,0.776286) -- (6.07709,0.776286);
\definecolor{c}{rgb}{0,0,0};
\colorlet{c}{natcomp!70};
\draw [c] (6.08523,0.768199) -- (6.08523,0.8023);
\draw [c] (6.08523,0.8023) -- (6.08523,0.836401);
\draw [c] (6.07709,0.8023) -- (6.08523,0.8023);
\draw [c] (6.08523,0.8023) -- (6.09336,0.8023);
\definecolor{c}{rgb}{0,0,0};
\colorlet{c}{natcomp!70};
\draw [c] (6.1015,0.710277) -- (6.1015,0.729228);
\draw [c] (6.1015,0.729228) -- (6.1015,0.748179);
\draw [c] (6.09336,0.729228) -- (6.1015,0.729228);
\draw [c] (6.1015,0.729228) -- (6.10964,0.729228);
\definecolor{c}{rgb}{0,0,0};
\colorlet{c}{natcomp!70};
\draw [c] (6.11777,0.740692) -- (6.11777,0.767798);
\draw [c] (6.11777,0.767798) -- (6.11777,0.794904);
\draw [c] (6.10964,0.767798) -- (6.11777,0.767798);
\draw [c] (6.11777,0.767798) -- (6.12591,0.767798);
\definecolor{c}{rgb}{0,0,0};
\colorlet{c}{natcomp!70};
\draw [c] (6.13405,0.720528) -- (6.13405,0.743887);
\draw [c] (6.13405,0.743887) -- (6.13405,0.767247);
\draw [c] (6.12591,0.743887) -- (6.13405,0.743887);
\draw [c] (6.13405,0.743887) -- (6.14218,0.743887);
\definecolor{c}{rgb}{0,0,0};
\colorlet{c}{natcomp!70};
\draw [c] (6.15032,0.69748) -- (6.15032,0.712307);
\draw [c] (6.15032,0.712307) -- (6.15032,0.727135);
\draw [c] (6.14218,0.712307) -- (6.15032,0.712307);
\draw [c] (6.15032,0.712307) -- (6.15845,0.712307);
\definecolor{c}{rgb}{0,0,0};
\colorlet{c}{natcomp!70};
\draw [c] (6.16659,0.709648) -- (6.16659,0.728192);
\draw [c] (6.16659,0.728192) -- (6.16659,0.746736);
\draw [c] (6.15845,0.728192) -- (6.16659,0.728192);
\draw [c] (6.16659,0.728192) -- (6.17473,0.728192);
\definecolor{c}{rgb}{0,0,0};
\colorlet{c}{natcomp!70};
\draw [c] (6.18286,0.7346) -- (6.18286,0.766423);
\draw [c] (6.18286,0.766423) -- (6.18286,0.798247);
\draw [c] (6.17473,0.766423) -- (6.18286,0.766423);
\draw [c] (6.18286,0.766423) -- (6.191,0.766423);
\definecolor{c}{rgb}{0,0,0};
\colorlet{c}{natcomp!70};
\draw [c] (6.19914,0.725682) -- (6.19914,0.753499);
\draw [c] (6.19914,0.753499) -- (6.19914,0.781317);
\draw [c] (6.191,0.753499) -- (6.19914,0.753499);
\draw [c] (6.19914,0.753499) -- (6.20727,0.753499);
\definecolor{c}{rgb}{0,0,0};
\colorlet{c}{natcomp!70};
\draw [c] (6.21541,0.726686) -- (6.21541,0.759284);
\draw [c] (6.21541,0.759284) -- (6.21541,0.791882);
\draw [c] (6.20727,0.759284) -- (6.21541,0.759284);
\draw [c] (6.21541,0.759284) -- (6.22355,0.759284);
\definecolor{c}{rgb}{0,0,0};
\colorlet{c}{natcomp!70};
\draw [c] (6.23168,0.742176) -- (6.23168,0.777624);
\draw [c] (6.23168,0.777624) -- (6.23168,0.813072);
\draw [c] (6.22355,0.777624) -- (6.23168,0.777624);
\draw [c] (6.23168,0.777624) -- (6.23982,0.777624);
\definecolor{c}{rgb}{0,0,0};
\colorlet{c}{natcomp!70};
\draw [c] (6.24795,0.776333) -- (6.24795,0.816985);
\draw [c] (6.24795,0.816985) -- (6.24795,0.857637);
\draw [c] (6.23982,0.816985) -- (6.24795,0.816985);
\draw [c] (6.24795,0.816985) -- (6.25609,0.816985);
\definecolor{c}{rgb}{0,0,0};
\colorlet{c}{natcomp!70};
\draw [c] (6.26423,0.716593) -- (6.26423,0.737387);
\draw [c] (6.26423,0.737387) -- (6.26423,0.758182);
\draw [c] (6.25609,0.737387) -- (6.26423,0.737387);
\draw [c] (6.26423,0.737387) -- (6.27236,0.737387);
\definecolor{c}{rgb}{0,0,0};
\colorlet{c}{natcomp!70};
\draw [c] (6.2805,0.73157) -- (6.2805,0.759414);
\draw [c] (6.2805,0.759414) -- (6.2805,0.787258);
\draw [c] (6.27236,0.759414) -- (6.2805,0.759414);
\draw [c] (6.2805,0.759414) -- (6.28864,0.759414);
\definecolor{c}{rgb}{0,0,0};
\colorlet{c}{natcomp!70};
\draw [c] (6.29677,0.765117) -- (6.29677,0.800819);
\draw [c] (6.29677,0.800819) -- (6.29677,0.83652);
\draw [c] (6.28864,0.800819) -- (6.29677,0.800819);
\draw [c] (6.29677,0.800819) -- (6.30491,0.800819);
\definecolor{c}{rgb}{0,0,0};
\colorlet{c}{natcomp!70};
\draw [c] (6.31305,0.697057) -- (6.31305,0.711255);
\draw [c] (6.31305,0.711255) -- (6.31305,0.725454);
\draw [c] (6.30491,0.711255) -- (6.31305,0.711255);
\draw [c] (6.31305,0.711255) -- (6.32118,0.711255);
\definecolor{c}{rgb}{0,0,0};
\colorlet{c}{natcomp!70};
\draw [c] (6.32932,0.699766) -- (6.32932,0.721515);
\draw [c] (6.32932,0.721515) -- (6.32932,0.743263);
\draw [c] (6.32118,0.721515) -- (6.32932,0.721515);
\draw [c] (6.32932,0.721515) -- (6.33745,0.721515);
\definecolor{c}{rgb}{0,0,0};
\colorlet{c}{natcomp!70};
\draw [c] (6.34559,0.75193) -- (6.34559,0.784287);
\draw [c] (6.34559,0.784287) -- (6.34559,0.816645);
\draw [c] (6.33745,0.784287) -- (6.34559,0.784287);
\draw [c] (6.34559,0.784287) -- (6.35373,0.784287);
\definecolor{c}{rgb}{0,0,0};
\colorlet{c}{natcomp!70};
\draw [c] (6.36186,0.751842) -- (6.36186,0.788755);
\draw [c] (6.36186,0.788755) -- (6.36186,0.825668);
\draw [c] (6.35373,0.788755) -- (6.36186,0.788755);
\draw [c] (6.36186,0.788755) -- (6.37,0.788755);
\definecolor{c}{rgb}{0,0,0};
\colorlet{c}{natcomp!70};
\draw [c] (6.37814,0.740937) -- (6.37814,0.773723);
\draw [c] (6.37814,0.773723) -- (6.37814,0.806508);
\draw [c] (6.37,0.773723) -- (6.37814,0.773723);
\draw [c] (6.37814,0.773723) -- (6.38627,0.773723);
\definecolor{c}{rgb}{0,0,0};
\colorlet{c}{natcomp!70};
\draw [c] (6.39441,0.698759) -- (6.39441,0.715275);
\draw [c] (6.39441,0.715275) -- (6.39441,0.73179);
\draw [c] (6.38627,0.715275) -- (6.39441,0.715275);
\draw [c] (6.39441,0.715275) -- (6.40255,0.715275);
\definecolor{c}{rgb}{0,0,0};
\colorlet{c}{natcomp!70};
\draw [c] (6.41068,0.710571) -- (6.41068,0.730263);
\draw [c] (6.41068,0.730263) -- (6.41068,0.749955);
\draw [c] (6.40255,0.730263) -- (6.41068,0.730263);
\draw [c] (6.41068,0.730263) -- (6.41882,0.730263);
\definecolor{c}{rgb}{0,0,0};
\colorlet{c}{natcomp!70};
\draw [c] (6.42695,0.733223) -- (6.42695,0.762826);
\draw [c] (6.42695,0.762826) -- (6.42695,0.792429);
\draw [c] (6.41882,0.762826) -- (6.42695,0.762826);
\draw [c] (6.42695,0.762826) -- (6.43509,0.762826);
\definecolor{c}{rgb}{0,0,0};
\colorlet{c}{natcomp!70};
\draw [c] (6.44323,0.722776) -- (6.44323,0.747622);
\draw [c] (6.44323,0.747622) -- (6.44323,0.772467);
\draw [c] (6.43509,0.747622) -- (6.44323,0.747622);
\draw [c] (6.44323,0.747622) -- (6.45136,0.747622);
\definecolor{c}{rgb}{0,0,0};
\colorlet{c}{natcomp!70};
\draw [c] (6.4595,0.739345) -- (6.4595,0.766635);
\draw [c] (6.4595,0.766635) -- (6.4595,0.793925);
\draw [c] (6.45136,0.766635) -- (6.4595,0.766635);
\draw [c] (6.4595,0.766635) -- (6.46764,0.766635);
\definecolor{c}{rgb}{0,0,0};
\colorlet{c}{natcomp!70};
\draw [c] (6.47577,0.710389) -- (6.47577,0.729721);
\draw [c] (6.47577,0.729721) -- (6.47577,0.749052);
\draw [c] (6.46764,0.729721) -- (6.47577,0.729721);
\draw [c] (6.47577,0.729721) -- (6.48391,0.729721);
\definecolor{c}{rgb}{0,0,0};
\colorlet{c}{natcomp!70};
\draw [c] (6.49205,0.741601) -- (6.49205,0.774287);
\draw [c] (6.49205,0.774287) -- (6.49205,0.806974);
\draw [c] (6.48391,0.774287) -- (6.49205,0.774287);
\draw [c] (6.49205,0.774287) -- (6.50018,0.774287);
\definecolor{c}{rgb}{0,0,0};
\colorlet{c}{natcomp!70};
\draw [c] (6.50832,0.704581) -- (6.50832,0.722973);
\draw [c] (6.50832,0.722973) -- (6.50832,0.741364);
\draw [c] (6.50018,0.722973) -- (6.50832,0.722973);
\draw [c] (6.50832,0.722973) -- (6.51645,0.722973);
\definecolor{c}{rgb}{0,0,0};
\colorlet{c}{natcomp!70};
\draw [c] (6.52459,0.698445) -- (6.52459,0.714356);
\draw [c] (6.52459,0.714356) -- (6.52459,0.730267);
\draw [c] (6.51645,0.714356) -- (6.52459,0.714356);
\draw [c] (6.52459,0.714356) -- (6.53273,0.714356);
\definecolor{c}{rgb}{0,0,0};
\colorlet{c}{natcomp!70};
\draw [c] (6.54086,0.713893) -- (6.54086,0.736042);
\draw [c] (6.54086,0.736042) -- (6.54086,0.758192);
\draw [c] (6.53273,0.736042) -- (6.54086,0.736042);
\draw [c] (6.54086,0.736042) -- (6.549,0.736042);
\definecolor{c}{rgb}{0,0,0};
\colorlet{c}{natcomp!70};
\draw [c] (6.55714,0.705456) -- (6.55714,0.724441);
\draw [c] (6.55714,0.724441) -- (6.55714,0.743426);
\draw [c] (6.549,0.724441) -- (6.55714,0.724441);
\draw [c] (6.55714,0.724441) -- (6.56527,0.724441);
\definecolor{c}{rgb}{0,0,0};
\colorlet{c}{natcomp!70};
\draw [c] (6.57341,0.708858) -- (6.57341,0.726682);
\draw [c] (6.57341,0.726682) -- (6.57341,0.744506);
\draw [c] (6.56527,0.726682) -- (6.57341,0.726682);
\draw [c] (6.57341,0.726682) -- (6.58155,0.726682);
\definecolor{c}{rgb}{0,0,0};
\colorlet{c}{natcomp!70};
\draw [c] (6.58968,0.744164) -- (6.58968,0.773304);
\draw [c] (6.58968,0.773304) -- (6.58968,0.802443);
\draw [c] (6.58155,0.773304) -- (6.58968,0.773304);
\draw [c] (6.58968,0.773304) -- (6.59782,0.773304);
\definecolor{c}{rgb}{0,0,0};
\colorlet{c}{natcomp!70};
\draw [c] (6.60595,0.728187) -- (6.60595,0.754684);
\draw [c] (6.60595,0.754684) -- (6.60595,0.781181);
\draw [c] (6.59782,0.754684) -- (6.60595,0.754684);
\draw [c] (6.60595,0.754684) -- (6.61409,0.754684);
\definecolor{c}{rgb}{0,0,0};
\colorlet{c}{natcomp!70};
\draw [c] (6.62223,0.716607) -- (6.62223,0.741593);
\draw [c] (6.62223,0.741593) -- (6.62223,0.766578);
\draw [c] (6.61409,0.741593) -- (6.62223,0.741593);
\draw [c] (6.62223,0.741593) -- (6.63036,0.741593);
\definecolor{c}{rgb}{0,0,0};
\colorlet{c}{natcomp!70};
\draw [c] (6.6385,0.699804) -- (6.6385,0.717549);
\draw [c] (6.6385,0.717549) -- (6.6385,0.735295);
\draw [c] (6.63036,0.717549) -- (6.6385,0.717549);
\draw [c] (6.6385,0.717549) -- (6.64664,0.717549);
\definecolor{c}{rgb}{0,0,0};
\colorlet{c}{natcomp!70};
\draw [c] (6.65477,0.753668) -- (6.65477,0.789033);
\draw [c] (6.65477,0.789033) -- (6.65477,0.824398);
\draw [c] (6.64664,0.789033) -- (6.65477,0.789033);
\draw [c] (6.65477,0.789033) -- (6.66291,0.789033);
\definecolor{c}{rgb}{0,0,0};
\colorlet{c}{natcomp!70};
\draw [c] (6.67105,0.708234) -- (6.67105,0.734765);
\draw [c] (6.67105,0.734765) -- (6.67105,0.761297);
\draw [c] (6.66291,0.734765) -- (6.67105,0.734765);
\draw [c] (6.67105,0.734765) -- (6.67918,0.734765);
\definecolor{c}{rgb}{0,0,0};
\colorlet{c}{natcomp!70};
\draw [c] (6.68732,0.713447) -- (6.68732,0.738205);
\draw [c] (6.68732,0.738205) -- (6.68732,0.762963);
\draw [c] (6.67918,0.738205) -- (6.68732,0.738205);
\draw [c] (6.68732,0.738205) -- (6.69545,0.738205);
\definecolor{c}{rgb}{0,0,0};
\colorlet{c}{natcomp!70};
\draw [c] (6.70359,0.733101) -- (6.70359,0.759424);
\draw [c] (6.70359,0.759424) -- (6.70359,0.785747);
\draw [c] (6.69545,0.759424) -- (6.70359,0.759424);
\draw [c] (6.70359,0.759424) -- (6.71173,0.759424);
\definecolor{c}{rgb}{0,0,0};
\colorlet{c}{natcomp!70};
\draw [c] (6.71986,0.703929) -- (6.71986,0.721255);
\draw [c] (6.71986,0.721255) -- (6.71986,0.738582);
\draw [c] (6.71173,0.721255) -- (6.71986,0.721255);
\draw [c] (6.71986,0.721255) -- (6.728,0.721255);
\definecolor{c}{rgb}{0,0,0};
\colorlet{c}{natcomp!70};
\draw [c] (6.73614,0.704397) -- (6.73614,0.721875);
\draw [c] (6.73614,0.721875) -- (6.73614,0.739353);
\draw [c] (6.728,0.721875) -- (6.73614,0.721875);
\draw [c] (6.73614,0.721875) -- (6.74427,0.721875);
\definecolor{c}{rgb}{0,0,0};
\colorlet{c}{natcomp!70};
\draw [c] (6.75241,0.710501) -- (6.75241,0.729699);
\draw [c] (6.75241,0.729699) -- (6.75241,0.748896);
\draw [c] (6.74427,0.729699) -- (6.75241,0.729699);
\draw [c] (6.75241,0.729699) -- (6.76055,0.729699);
\definecolor{c}{rgb}{0,0,0};
\colorlet{c}{natcomp!70};
\draw [c] (6.76868,0.711589) -- (6.76868,0.731751);
\draw [c] (6.76868,0.731751) -- (6.76868,0.751914);
\draw [c] (6.76055,0.731751) -- (6.76868,0.731751);
\draw [c] (6.76868,0.731751) -- (6.77682,0.731751);
\definecolor{c}{rgb}{0,0,0};
\colorlet{c}{natcomp!70};
\draw [c] (6.78495,0.718017) -- (6.78495,0.745245);
\draw [c] (6.78495,0.745245) -- (6.78495,0.772473);
\draw [c] (6.77682,0.745245) -- (6.78495,0.745245);
\draw [c] (6.78495,0.745245) -- (6.79309,0.745245);
\definecolor{c}{rgb}{0,0,0};
\colorlet{c}{natcomp!70};
\draw [c] (6.80123,0.697887) -- (6.80123,0.713515);
\draw [c] (6.80123,0.713515) -- (6.80123,0.729144);
\draw [c] (6.79309,0.713515) -- (6.80123,0.713515);
\draw [c] (6.80123,0.713515) -- (6.80936,0.713515);
\definecolor{c}{rgb}{0,0,0};
\colorlet{c}{natcomp!70};
\draw [c] (6.8175,0.712023) -- (6.8175,0.732788);
\draw [c] (6.8175,0.732788) -- (6.8175,0.753553);
\draw [c] (6.80936,0.732788) -- (6.8175,0.732788);
\draw [c] (6.8175,0.732788) -- (6.82564,0.732788);
\definecolor{c}{rgb}{0,0,0};
\colorlet{c}{natcomp!70};
\draw [c] (6.83377,0.714232) -- (6.83377,0.736446);
\draw [c] (6.83377,0.736446) -- (6.83377,0.758661);
\draw [c] (6.82564,0.736446) -- (6.83377,0.736446);
\draw [c] (6.83377,0.736446) -- (6.84191,0.736446);
\definecolor{c}{rgb}{0,0,0};
\colorlet{c}{natcomp!70};
\draw [c] (6.85005,0.709462) -- (6.85005,0.739161);
\draw [c] (6.85005,0.739161) -- (6.85005,0.768861);
\draw [c] (6.84191,0.739161) -- (6.85005,0.739161);
\draw [c] (6.85005,0.739161) -- (6.85818,0.739161);
\definecolor{c}{rgb}{0,0,0};
\colorlet{c}{natcomp!70};
\draw [c] (6.86632,0.69326) -- (6.86632,0.708575);
\draw [c] (6.86632,0.708575) -- (6.86632,0.72389);
\draw [c] (6.85818,0.708575) -- (6.86632,0.708575);
\draw [c] (6.86632,0.708575) -- (6.87445,0.708575);
\definecolor{c}{rgb}{0,0,0};
\colorlet{c}{natcomp!70};
\draw [c] (6.88259,0.703644) -- (6.88259,0.720612);
\draw [c] (6.88259,0.720612) -- (6.88259,0.73758);
\draw [c] (6.87445,0.720612) -- (6.88259,0.720612);
\draw [c] (6.88259,0.720612) -- (6.89073,0.720612);
\definecolor{c}{rgb}{0,0,0};
\colorlet{c}{natcomp!70};
\draw [c] (6.89886,0.692271) -- (6.89886,0.705485);
\draw [c] (6.89886,0.705485) -- (6.89886,0.7187);
\draw [c] (6.89073,0.705485) -- (6.89886,0.705485);
\draw [c] (6.89886,0.705485) -- (6.907,0.705485);
\definecolor{c}{rgb}{0,0,0};
\colorlet{c}{natcomp!70};
\draw [c] (6.91514,0.692587) -- (6.91514,0.706325);
\draw [c] (6.91514,0.706325) -- (6.91514,0.720063);
\draw [c] (6.907,0.706325) -- (6.91514,0.706325);
\draw [c] (6.91514,0.706325) -- (6.92327,0.706325);
\definecolor{c}{rgb}{0,0,0};
\colorlet{c}{natcomp!70};
\draw [c] (6.93141,0.686927) -- (6.93141,0.695065);
\draw [c] (6.93141,0.695065) -- (6.93141,0.703204);
\draw [c] (6.92327,0.695065) -- (6.93141,0.695065);
\draw [c] (6.93141,0.695065) -- (6.93955,0.695065);
\definecolor{c}{rgb}{0,0,0};
\colorlet{c}{natcomp!70};
\draw [c] (6.94768,0.704333) -- (6.94768,0.7221);
\draw [c] (6.94768,0.7221) -- (6.94768,0.739867);
\draw [c] (6.93955,0.7221) -- (6.94768,0.7221);
\draw [c] (6.94768,0.7221) -- (6.95582,0.7221);
\definecolor{c}{rgb}{0,0,0};
\colorlet{c}{natcomp!70};
\draw [c] (6.96395,0.686909) -- (6.96395,0.68692);
\draw [c] (6.96395,0.68692) -- (6.96395,0.686931);
\draw [c] (6.95582,0.68692) -- (6.96395,0.68692);
\draw [c] (6.96395,0.68692) -- (6.97209,0.68692);
\definecolor{c}{rgb}{0,0,0};
\colorlet{c}{natcomp!70};
\draw [c] (6.98023,0.718409) -- (6.98023,0.740294);
\draw [c] (6.98023,0.740294) -- (6.98023,0.762179);
\draw [c] (6.97209,0.740294) -- (6.98023,0.740294);
\draw [c] (6.98023,0.740294) -- (6.98836,0.740294);
\definecolor{c}{rgb}{0,0,0};
\colorlet{c}{natcomp!70};
\draw [c] (6.9965,0.686917) -- (6.9965,0.68693);
\draw [c] (6.9965,0.68693) -- (6.9965,0.686943);
\draw [c] (6.98836,0.68693) -- (6.9965,0.68693);
\draw [c] (6.9965,0.68693) -- (7.00464,0.68693);
\definecolor{c}{rgb}{0,0,0};
\colorlet{c}{natcomp!70};
\draw [c] (7.01277,0.713906) -- (7.01277,0.736963);
\draw [c] (7.01277,0.736963) -- (7.01277,0.760019);
\draw [c] (7.00464,0.736963) -- (7.01277,0.736963);
\draw [c] (7.01277,0.736963) -- (7.02091,0.736963);
\definecolor{c}{rgb}{0,0,0};
\colorlet{c}{natcomp!70};
\draw [c] (7.02905,0.703869) -- (7.02905,0.721431);
\draw [c] (7.02905,0.721431) -- (7.02905,0.738992);
\draw [c] (7.02091,0.721431) -- (7.02905,0.721431);
\draw [c] (7.02905,0.721431) -- (7.03718,0.721431);
\definecolor{c}{rgb}{0,0,0};
\colorlet{c}{natcomp!70};
\draw [c] (7.04532,0.705655) -- (7.04532,0.724877);
\draw [c] (7.04532,0.724877) -- (7.04532,0.7441);
\draw [c] (7.03718,0.724877) -- (7.04532,0.724877);
\draw [c] (7.04532,0.724877) -- (7.05345,0.724877);
\definecolor{c}{rgb}{0,0,0};
\colorlet{c}{natcomp!70};
\draw [c] (7.06159,0.703801) -- (7.06159,0.72083);
\draw [c] (7.06159,0.72083) -- (7.06159,0.73786);
\draw [c] (7.05345,0.72083) -- (7.06159,0.72083);
\draw [c] (7.06159,0.72083) -- (7.06973,0.72083);
\definecolor{c}{rgb}{0,0,0};
\colorlet{c}{natcomp!70};
\draw [c] (7.07786,0.719124) -- (7.07786,0.741674);
\draw [c] (7.07786,0.741674) -- (7.07786,0.764224);
\draw [c] (7.06973,0.741674) -- (7.07786,0.741674);
\draw [c] (7.07786,0.741674) -- (7.086,0.741674);
\definecolor{c}{rgb}{0,0,0};
\colorlet{c}{natcomp!70};
\draw [c] (7.09414,0.709945) -- (7.09414,0.738016);
\draw [c] (7.09414,0.738016) -- (7.09414,0.766087);
\draw [c] (7.086,0.738016) -- (7.09414,0.738016);
\draw [c] (7.09414,0.738016) -- (7.10227,0.738016);
\definecolor{c}{rgb}{0,0,0};
\colorlet{c}{natcomp!70};
\draw [c] (7.11041,0.691149) -- (7.11041,0.701332);
\draw [c] (7.11041,0.701332) -- (7.11041,0.711516);
\draw [c] (7.10227,0.701332) -- (7.11041,0.701332);
\draw [c] (7.11041,0.701332) -- (7.11855,0.701332);
\definecolor{c}{rgb}{0,0,0};
\colorlet{c}{natcomp!70};
\draw [c] (7.12668,0.706897) -- (7.12668,0.728052);
\draw [c] (7.12668,0.728052) -- (7.12668,0.749206);
\draw [c] (7.11855,0.728052) -- (7.12668,0.728052);
\draw [c] (7.12668,0.728052) -- (7.13482,0.728052);
\definecolor{c}{rgb}{0,0,0};
\colorlet{c}{natcomp!70};
\draw [c] (7.14295,0.691954) -- (7.14295,0.704175);
\draw [c] (7.14295,0.704175) -- (7.14295,0.716395);
\draw [c] (7.13482,0.704175) -- (7.14295,0.704175);
\draw [c] (7.14295,0.704175) -- (7.15109,0.704175);
\definecolor{c}{rgb}{0,0,0};
\colorlet{c}{natcomp!70};
\draw [c] (7.15923,0.686944) -- (7.15923,0.695807);
\draw [c] (7.15923,0.695807) -- (7.15923,0.704669);
\draw [c] (7.15109,0.695807) -- (7.15923,0.695807);
\draw [c] (7.15923,0.695807) -- (7.16736,0.695807);
\definecolor{c}{rgb}{0,0,0};
\colorlet{c}{natcomp!70};
\draw [c] (7.1755,0.713402) -- (7.1755,0.735241);
\draw [c] (7.1755,0.735241) -- (7.1755,0.75708);
\draw [c] (7.16736,0.735241) -- (7.1755,0.735241);
\draw [c] (7.1755,0.735241) -- (7.18364,0.735241);
\definecolor{c}{rgb}{0,0,0};
\colorlet{c}{natcomp!70};
\draw [c] (7.19177,0.70731) -- (7.19177,0.72784);
\draw [c] (7.19177,0.72784) -- (7.19177,0.74837);
\draw [c] (7.18364,0.72784) -- (7.19177,0.72784);
\draw [c] (7.19177,0.72784) -- (7.19991,0.72784);
\definecolor{c}{rgb}{0,0,0};
\colorlet{c}{natcomp!70};
\draw [c] (7.20805,0.694873) -- (7.20805,0.730967);
\draw [c] (7.20805,0.730967) -- (7.20805,0.767061);
\draw [c] (7.19991,0.730967) -- (7.20805,0.730967);
\draw [c] (7.20805,0.730967) -- (7.21618,0.730967);
\definecolor{c}{rgb}{0,0,0};
\colorlet{c}{natcomp!70};
\draw [c] (7.22432,0.692201) -- (7.22432,0.704986);
\draw [c] (7.22432,0.704986) -- (7.22432,0.71777);
\draw [c] (7.21618,0.704986) -- (7.22432,0.704986);
\draw [c] (7.22432,0.704986) -- (7.23245,0.704986);
\definecolor{c}{rgb}{0,0,0};
\colorlet{c}{natcomp!70};
\draw [c] (7.24059,0.694145) -- (7.24059,0.712713);
\draw [c] (7.24059,0.712713) -- (7.24059,0.731281);
\draw [c] (7.23245,0.712713) -- (7.24059,0.712713);
\draw [c] (7.24059,0.712713) -- (7.24873,0.712713);
\definecolor{c}{rgb}{0,0,0};
\colorlet{c}{natcomp!70};
\draw [c] (7.25686,0.686911) -- (7.25686,0.696805);
\draw [c] (7.25686,0.696805) -- (7.25686,0.706699);
\draw [c] (7.24873,0.696805) -- (7.25686,0.696805);
\draw [c] (7.25686,0.696805) -- (7.265,0.696805);
\definecolor{c}{rgb}{0,0,0};
\colorlet{c}{natcomp!70};
\draw [c] (7.27314,0.686939) -- (7.27314,0.698171);
\draw [c] (7.27314,0.698171) -- (7.27314,0.709404);
\draw [c] (7.265,0.698171) -- (7.27314,0.698171);
\draw [c] (7.27314,0.698171) -- (7.28127,0.698171);
\definecolor{c}{rgb}{0,0,0};
\colorlet{c}{natcomp!70};
\draw [c] (7.28941,0.705817) -- (7.28941,0.724823);
\draw [c] (7.28941,0.724823) -- (7.28941,0.743829);
\draw [c] (7.28127,0.724823) -- (7.28941,0.724823);
\draw [c] (7.28941,0.724823) -- (7.29755,0.724823);
\definecolor{c}{rgb}{0,0,0};
\colorlet{c}{natcomp!70};
\draw [c] (7.30568,0.704012) -- (7.30568,0.721633);
\draw [c] (7.30568,0.721633) -- (7.30568,0.739254);
\draw [c] (7.29755,0.721633) -- (7.30568,0.721633);
\draw [c] (7.30568,0.721633) -- (7.31382,0.721633);
\definecolor{c}{rgb}{0,0,0};
\colorlet{c}{natcomp!70};
\draw [c] (7.32195,0.714809) -- (7.32195,0.753099);
\draw [c] (7.32195,0.753099) -- (7.32195,0.791389);
\draw [c] (7.31382,0.753099) -- (7.32195,0.753099);
\draw [c] (7.32195,0.753099) -- (7.33009,0.753099);
\definecolor{c}{rgb}{0,0,0};
\colorlet{c}{natcomp!70};
\draw [c] (7.33823,0.697923) -- (7.33823,0.713018);
\draw [c] (7.33823,0.713018) -- (7.33823,0.728113);
\draw [c] (7.33009,0.713018) -- (7.33823,0.713018);
\draw [c] (7.33823,0.713018) -- (7.34636,0.713018);
\definecolor{c}{rgb}{0,0,0};
\colorlet{c}{natcomp!70};
\draw [c] (7.3545,0.698691) -- (7.3545,0.715314);
\draw [c] (7.3545,0.715314) -- (7.3545,0.731936);
\draw [c] (7.34636,0.715314) -- (7.3545,0.715314);
\draw [c] (7.3545,0.715314) -- (7.36264,0.715314);
\definecolor{c}{rgb}{0,0,0};
\colorlet{c}{natcomp!70};
\draw [c] (7.37077,0.717852) -- (7.37077,0.745539);
\draw [c] (7.37077,0.745539) -- (7.37077,0.773225);
\draw [c] (7.36264,0.745539) -- (7.37077,0.745539);
\draw [c] (7.37077,0.745539) -- (7.37891,0.745539);
\definecolor{c}{rgb}{0,0,0};
\colorlet{c}{natcomp!70};
\draw [c] (7.38705,0.701315) -- (7.38705,0.726433);
\draw [c] (7.38705,0.726433) -- (7.38705,0.751551);
\draw [c] (7.37891,0.726433) -- (7.38705,0.726433);
\draw [c] (7.38705,0.726433) -- (7.39518,0.726433);
\definecolor{c}{rgb}{0,0,0};
\colorlet{c}{natcomp!70};
\draw [c] (7.40332,0.719987) -- (7.40332,0.743126);
\draw [c] (7.40332,0.743126) -- (7.40332,0.766266);
\draw [c] (7.39518,0.743126) -- (7.40332,0.743126);
\draw [c] (7.40332,0.743126) -- (7.41145,0.743126);
\definecolor{c}{rgb}{0,0,0};
\colorlet{c}{natcomp!70};
\draw [c] (7.41959,0.691933) -- (7.41959,0.704153);
\draw [c] (7.41959,0.704153) -- (7.41959,0.716373);
\draw [c] (7.41145,0.704153) -- (7.41959,0.704153);
\draw [c] (7.41959,0.704153) -- (7.42773,0.704153);
\definecolor{c}{rgb}{0,0,0};
\colorlet{c}{natcomp!70};
\draw [c] (7.45214,0.686902) -- (7.45214,0.695866);
\draw [c] (7.45214,0.695866) -- (7.45214,0.704829);
\draw [c] (7.444,0.695866) -- (7.45214,0.695866);
\draw [c] (7.45214,0.695866) -- (7.46027,0.695866);
\definecolor{c}{rgb}{0,0,0};
\colorlet{c}{natcomp!70};
\draw [c] (7.46841,0.686921) -- (7.46841,0.695885);
\draw [c] (7.46841,0.695885) -- (7.46841,0.704848);
\draw [c] (7.46027,0.695885) -- (7.46841,0.695885);
\draw [c] (7.46841,0.695885) -- (7.47655,0.695885);
\definecolor{c}{rgb}{0,0,0};
\colorlet{c}{natcomp!70};
\draw [c] (7.48468,0.686897) -- (7.48468,0.686905);
\draw [c] (7.48468,0.686905) -- (7.48468,0.686914);
\draw [c] (7.47655,0.686905) -- (7.48468,0.686905);
\draw [c] (7.48468,0.686905) -- (7.49282,0.686905);
\definecolor{c}{rgb}{0,0,0};
\colorlet{c}{natcomp!70};
\draw [c] (7.50095,0.686894) -- (7.50095,0.686899);
\draw [c] (7.50095,0.686899) -- (7.50095,0.686904);
\draw [c] (7.49282,0.686899) -- (7.50095,0.686899);
\draw [c] (7.50095,0.686899) -- (7.50909,0.686899);
\definecolor{c}{rgb}{0,0,0};
\colorlet{c}{natcomp!70};
\draw [c] (7.51723,0.702942) -- (7.51723,0.719147);
\draw [c] (7.51723,0.719147) -- (7.51723,0.735351);
\draw [c] (7.50909,0.719147) -- (7.51723,0.719147);
\draw [c] (7.51723,0.719147) -- (7.52536,0.719147);
\definecolor{c}{rgb}{0,0,0};
\colorlet{c}{natcomp!70};
\draw [c] (7.5335,0.691582) -- (7.5335,0.703001);
\draw [c] (7.5335,0.703001) -- (7.5335,0.71442);
\draw [c] (7.52536,0.703001) -- (7.5335,0.703001);
\draw [c] (7.5335,0.703001) -- (7.54164,0.703001);
\definecolor{c}{rgb}{0,0,0};
\colorlet{c}{natcomp!70};
\draw [c] (7.54977,0.699339) -- (7.54977,0.718609);
\draw [c] (7.54977,0.718609) -- (7.54977,0.737879);
\draw [c] (7.54164,0.718609) -- (7.54977,0.718609);
\draw [c] (7.54977,0.718609) -- (7.55791,0.718609);
\definecolor{c}{rgb}{0,0,0};
\colorlet{c}{natcomp!70};
\draw [c] (7.56605,0.705399) -- (7.56605,0.728064);
\draw [c] (7.56605,0.728064) -- (7.56605,0.750729);
\draw [c] (7.55791,0.728064) -- (7.56605,0.728064);
\draw [c] (7.56605,0.728064) -- (7.57418,0.728064);
\definecolor{c}{rgb}{0,0,0};
\colorlet{c}{natcomp!70};
\draw [c] (7.58232,0.699404) -- (7.58232,0.723597);
\draw [c] (7.58232,0.723597) -- (7.58232,0.74779);
\draw [c] (7.57418,0.723597) -- (7.58232,0.723597);
\draw [c] (7.58232,0.723597) -- (7.59045,0.723597);
\definecolor{c}{rgb}{0,0,0};
\colorlet{c}{natcomp!70};
\draw [c] (7.59859,0.686912) -- (7.59859,0.694113);
\draw [c] (7.59859,0.694113) -- (7.59859,0.701313);
\draw [c] (7.59045,0.694113) -- (7.59859,0.694113);
\draw [c] (7.59859,0.694113) -- (7.60673,0.694113);
\definecolor{c}{rgb}{0,0,0};
\colorlet{c}{natcomp!70};
\draw [c] (7.61486,0.686896) -- (7.61486,0.686902);
\draw [c] (7.61486,0.686902) -- (7.61486,0.686909);
\draw [c] (7.60673,0.686902) -- (7.61486,0.686902);
\draw [c] (7.61486,0.686902) -- (7.623,0.686902);
\definecolor{c}{rgb}{0,0,0};
\colorlet{c}{natcomp!70};
\draw [c] (7.63114,0.721485) -- (7.63114,0.74621);
\draw [c] (7.63114,0.74621) -- (7.63114,0.770934);
\draw [c] (7.623,0.74621) -- (7.63114,0.74621);
\draw [c] (7.63114,0.74621) -- (7.63927,0.74621);
\definecolor{c}{rgb}{0,0,0};
\colorlet{c}{natcomp!70};
\draw [c] (7.64741,0.686934) -- (7.64741,0.695073);
\draw [c] (7.64741,0.695073) -- (7.64741,0.703212);
\draw [c] (7.63927,0.695073) -- (7.64741,0.695073);
\draw [c] (7.64741,0.695073) -- (7.65555,0.695073);
\definecolor{c}{rgb}{0,0,0};
\colorlet{c}{natcomp!70};
\draw [c] (7.66368,0.69193) -- (7.66368,0.704888);
\draw [c] (7.66368,0.704888) -- (7.66368,0.717847);
\draw [c] (7.65555,0.704888) -- (7.66368,0.704888);
\draw [c] (7.66368,0.704888) -- (7.67182,0.704888);
\definecolor{c}{rgb}{0,0,0};
\colorlet{c}{natcomp!70};
\draw [c] (7.67995,0.693227) -- (7.67995,0.708478);
\draw [c] (7.67995,0.708478) -- (7.67995,0.723729);
\draw [c] (7.67182,0.708478) -- (7.67995,0.708478);
\draw [c] (7.67995,0.708478) -- (7.68809,0.708478);
\definecolor{c}{rgb}{0,0,0};
\colorlet{c}{natcomp!70};
\draw [c] (7.69623,0.705019) -- (7.69623,0.723124);
\draw [c] (7.69623,0.723124) -- (7.69623,0.74123);
\draw [c] (7.68809,0.723124) -- (7.69623,0.723124);
\draw [c] (7.69623,0.723124) -- (7.70436,0.723124);
\definecolor{c}{rgb}{0,0,0};
\colorlet{c}{natcomp!70};
\draw [c] (7.7125,0.686901) -- (7.7125,0.686914);
\draw [c] (7.7125,0.686914) -- (7.7125,0.686926);
\draw [c] (7.70436,0.686914) -- (7.7125,0.686914);
\draw [c] (7.7125,0.686914) -- (7.72064,0.686914);
\definecolor{c}{rgb}{0,0,0};
\colorlet{c}{natcomp!70};
\draw [c] (7.72877,0.686911) -- (7.72877,0.697705);
\draw [c] (7.72877,0.697705) -- (7.72877,0.7085);
\draw [c] (7.72064,0.697705) -- (7.72877,0.697705);
\draw [c] (7.72877,0.697705) -- (7.73691,0.697705);
\definecolor{c}{rgb}{0,0,0};
\colorlet{c}{natcomp!70};
\draw [c] (7.74505,0.697486) -- (7.74505,0.712097);
\draw [c] (7.74505,0.712097) -- (7.74505,0.726708);
\draw [c] (7.73691,0.712097) -- (7.74505,0.712097);
\draw [c] (7.74505,0.712097) -- (7.75318,0.712097);
\definecolor{c}{rgb}{0,0,0};
\colorlet{c}{natcomp!70};
\draw [c] (7.76132,0.692362) -- (7.76132,0.708107);
\draw [c] (7.76132,0.708107) -- (7.76132,0.723852);
\draw [c] (7.75318,0.708107) -- (7.76132,0.708107);
\draw [c] (7.76132,0.708107) -- (7.76945,0.708107);
\definecolor{c}{rgb}{0,0,0};
\colorlet{c}{natcomp!70};
\draw [c] (7.77759,0.699709) -- (7.77759,0.717404);
\draw [c] (7.77759,0.717404) -- (7.77759,0.735098);
\draw [c] (7.76945,0.717404) -- (7.77759,0.717404);
\draw [c] (7.77759,0.717404) -- (7.78573,0.717404);
\definecolor{c}{rgb}{0,0,0};
\colorlet{c}{natcomp!70};
\draw [c] (7.79386,0.733221) -- (7.79386,0.759074);
\draw [c] (7.79386,0.759074) -- (7.79386,0.784928);
\draw [c] (7.78573,0.759074) -- (7.79386,0.759074);
\draw [c] (7.79386,0.759074) -- (7.802,0.759074);
\definecolor{c}{rgb}{0,0,0};
\colorlet{c}{natcomp!70};
\draw [c] (7.81014,0.697974) -- (7.81014,0.713528);
\draw [c] (7.81014,0.713528) -- (7.81014,0.729082);
\draw [c] (7.802,0.713528) -- (7.81014,0.713528);
\draw [c] (7.81014,0.713528) -- (7.81827,0.713528);
\definecolor{c}{rgb}{0,0,0};
\colorlet{c}{natcomp!70};
\draw [c] (7.82641,0.692082) -- (7.82641,0.705044);
\draw [c] (7.82641,0.705044) -- (7.82641,0.718006);
\draw [c] (7.81827,0.705044) -- (7.82641,0.705044);
\draw [c] (7.82641,0.705044) -- (7.83455,0.705044);
\definecolor{c}{rgb}{0,0,0};
\colorlet{c}{natcomp!70};
\draw [c] (7.84268,0.69139) -- (7.84268,0.702257);
\draw [c] (7.84268,0.702257) -- (7.84268,0.713124);
\draw [c] (7.83455,0.702257) -- (7.84268,0.702257);
\draw [c] (7.84268,0.702257) -- (7.85082,0.702257);
\definecolor{c}{rgb}{0,0,0};
\colorlet{c}{natcomp!70};
\draw [c] (7.85895,0.68692) -- (7.85895,0.696815);
\draw [c] (7.85895,0.696815) -- (7.85895,0.706709);
\draw [c] (7.85082,0.696815) -- (7.85895,0.696815);
\draw [c] (7.85895,0.696815) -- (7.86709,0.696815);
\definecolor{c}{rgb}{0,0,0};
\colorlet{c}{natcomp!70};
\draw [c] (7.87523,0.699254) -- (7.87523,0.718923);
\draw [c] (7.87523,0.718923) -- (7.87523,0.738592);
\draw [c] (7.86709,0.718923) -- (7.87523,0.718923);
\draw [c] (7.87523,0.718923) -- (7.88336,0.718923);
\definecolor{c}{rgb}{0,0,0};
\colorlet{c}{natcomp!70};
\draw [c] (7.8915,0.711611) -- (7.8915,0.732071);
\draw [c] (7.8915,0.732071) -- (7.8915,0.752532);
\draw [c] (7.88336,0.732071) -- (7.8915,0.732071);
\draw [c] (7.8915,0.732071) -- (7.89964,0.732071);
\definecolor{c}{rgb}{0,0,0};
\colorlet{c}{natcomp!70};
\draw [c] (7.90777,0.715085) -- (7.90777,0.738825);
\draw [c] (7.90777,0.738825) -- (7.90777,0.762565);
\draw [c] (7.89964,0.738825) -- (7.90777,0.738825);
\draw [c] (7.90777,0.738825) -- (7.91591,0.738825);
\definecolor{c}{rgb}{0,0,0};
\colorlet{c}{natcomp!70};
\draw [c] (7.92405,0.686922) -- (7.92405,0.696038);
\draw [c] (7.92405,0.696038) -- (7.92405,0.705153);
\draw [c] (7.91591,0.696038) -- (7.92405,0.696038);
\draw [c] (7.92405,0.696038) -- (7.93218,0.696038);
\definecolor{c}{rgb}{0,0,0};
\colorlet{c}{natcomp!70};
\draw [c] (7.94032,0.686907) -- (7.94032,0.696802);
\draw [c] (7.94032,0.696802) -- (7.94032,0.706696);
\draw [c] (7.93218,0.696802) -- (7.94032,0.696802);
\draw [c] (7.94032,0.696802) -- (7.94845,0.696802);
\definecolor{c}{rgb}{0,0,0};
\colorlet{c}{natcomp!70};
\draw [c] (7.95659,0.691941) -- (7.95659,0.704161);
\draw [c] (7.95659,0.704161) -- (7.95659,0.716381);
\draw [c] (7.94845,0.704161) -- (7.95659,0.704161);
\draw [c] (7.95659,0.704161) -- (7.96473,0.704161);
\definecolor{c}{rgb}{0,0,0};
\colorlet{c}{natcomp!70};
\draw [c] (7.97286,0.686912) -- (7.97286,0.69505);
\draw [c] (7.97286,0.69505) -- (7.97286,0.703189);
\draw [c] (7.96473,0.69505) -- (7.97286,0.69505);
\draw [c] (7.97286,0.69505) -- (7.981,0.69505);
\definecolor{c}{rgb}{0,0,0};
\colorlet{c}{natcomp!70};
\draw [c] (7.98914,0.693662) -- (7.98914,0.710807);
\draw [c] (7.98914,0.710807) -- (7.98914,0.727952);
\draw [c] (7.981,0.710807) -- (7.98914,0.710807);
\draw [c] (7.98914,0.710807) -- (7.99727,0.710807);
\definecolor{c}{rgb}{0,0,0};
\colorlet{c}{natcomp!70};
\draw [c] (8.00541,0.686922) -- (8.00541,0.695785);
\draw [c] (8.00541,0.695785) -- (8.00541,0.704647);
\draw [c] (7.99727,0.695785) -- (8.00541,0.695785);
\draw [c] (8.00541,0.695785) -- (8.01355,0.695785);
\definecolor{c}{rgb}{0,0,0};
\colorlet{c}{natcomp!70};
\draw [c] (8.02168,0.691552) -- (8.02168,0.702771);
\draw [c] (8.02168,0.702771) -- (8.02168,0.71399);
\draw [c] (8.01355,0.702771) -- (8.02168,0.702771);
\draw [c] (8.02168,0.702771) -- (8.02982,0.702771);
\definecolor{c}{rgb}{0,0,0};
\colorlet{c}{natcomp!70};
\draw [c] (8.03795,0.686908) -- (8.03795,0.70091);
\draw [c] (8.03795,0.70091) -- (8.03795,0.714913);
\draw [c] (8.02982,0.70091) -- (8.03795,0.70091);
\draw [c] (8.03795,0.70091) -- (8.04609,0.70091);
\definecolor{c}{rgb}{0,0,0};
\colorlet{c}{natcomp!70};
\draw [c] (8.05423,0.686902) -- (8.05423,0.695765);
\draw [c] (8.05423,0.695765) -- (8.05423,0.704627);
\draw [c] (8.04609,0.695765) -- (8.05423,0.695765);
\draw [c] (8.05423,0.695765) -- (8.06236,0.695765);
\definecolor{c}{rgb}{0,0,0};
\colorlet{c}{natcomp!70};
\draw [c] (8.0705,0.686913) -- (8.0705,0.694635);
\draw [c] (8.0705,0.694635) -- (8.0705,0.702356);
\draw [c] (8.06236,0.694635) -- (8.0705,0.694635);
\draw [c] (8.0705,0.694635) -- (8.07864,0.694635);
\definecolor{c}{rgb}{0,0,0};
\colorlet{c}{natcomp!70};
\draw [c] (8.08677,0.691886) -- (8.08677,0.703918);
\draw [c] (8.08677,0.703918) -- (8.08677,0.715951);
\draw [c] (8.07864,0.703918) -- (8.08677,0.703918);
\draw [c] (8.08677,0.703918) -- (8.09491,0.703918);
\definecolor{c}{rgb}{0,0,0};
\colorlet{c}{natcomp!70};
\draw [c] (8.10305,0.699036) -- (8.10305,0.715859);
\draw [c] (8.10305,0.715859) -- (8.10305,0.732682);
\draw [c] (8.09491,0.715859) -- (8.10305,0.715859);
\draw [c] (8.10305,0.715859) -- (8.11118,0.715859);
\definecolor{c}{rgb}{0,0,0};
\colorlet{c}{natcomp!70};
\draw [c] (8.11932,0.70093) -- (8.11932,0.723385);
\draw [c] (8.11932,0.723385) -- (8.11932,0.74584);
\draw [c] (8.11118,0.723385) -- (8.11932,0.723385);
\draw [c] (8.11932,0.723385) -- (8.12745,0.723385);
\definecolor{c}{rgb}{0,0,0};
\colorlet{c}{natcomp!70};
\draw [c] (8.13559,0.691991) -- (8.13559,0.704542);
\draw [c] (8.13559,0.704542) -- (8.13559,0.717092);
\draw [c] (8.12745,0.704542) -- (8.13559,0.704542);
\draw [c] (8.13559,0.704542) -- (8.14373,0.704542);
\definecolor{c}{rgb}{0,0,0};
\colorlet{c}{natcomp!70};
\draw [c] (8.15186,0.704381) -- (8.15186,0.722348);
\draw [c] (8.15186,0.722348) -- (8.15186,0.740315);
\draw [c] (8.14373,0.722348) -- (8.15186,0.722348);
\draw [c] (8.15186,0.722348) -- (8.16,0.722348);
\definecolor{c}{rgb}{0,0,0};
\colorlet{c}{natcomp!70};
\draw [c] (8.16814,0.692514) -- (8.16814,0.706186);
\draw [c] (8.16814,0.706186) -- (8.16814,0.719858);
\draw [c] (8.16,0.706186) -- (8.16814,0.706186);
\draw [c] (8.16814,0.706186) -- (8.17627,0.706186);
\definecolor{c}{rgb}{0,0,0};
\colorlet{c}{natcomp!70};
\draw [c] (8.18441,0.686908) -- (8.18441,0.697319);
\draw [c] (8.18441,0.697319) -- (8.18441,0.70773);
\draw [c] (8.17627,0.697319) -- (8.18441,0.697319);
\draw [c] (8.18441,0.697319) -- (8.19255,0.697319);
\definecolor{c}{rgb}{0,0,0};
\colorlet{c}{natcomp!70};
\draw [c] (8.20068,0.686908) -- (8.20068,0.708675);
\draw [c] (8.20068,0.708675) -- (8.20068,0.730443);
\draw [c] (8.19255,0.708675) -- (8.20068,0.708675);
\draw [c] (8.20068,0.708675) -- (8.20882,0.708675);
\definecolor{c}{rgb}{0,0,0};
\colorlet{c}{natcomp!70};
\draw [c] (8.21695,0.691366) -- (8.21695,0.702233);
\draw [c] (8.21695,0.702233) -- (8.21695,0.7131);
\draw [c] (8.20882,0.702233) -- (8.21695,0.702233);
\draw [c] (8.21695,0.702233) -- (8.22509,0.702233);
\definecolor{c}{rgb}{0,0,0};
\colorlet{c}{natcomp!70};
\draw [c] (8.23323,0.700598) -- (8.23323,0.719335);
\draw [c] (8.23323,0.719335) -- (8.23323,0.738073);
\draw [c] (8.22509,0.719335) -- (8.23323,0.719335);
\draw [c] (8.23323,0.719335) -- (8.24136,0.719335);
\definecolor{c}{rgb}{0,0,0};
\colorlet{c}{natcomp!70};
\draw [c] (8.2495,0.698239) -- (8.2495,0.713998);
\draw [c] (8.2495,0.713998) -- (8.2495,0.729758);
\draw [c] (8.24136,0.713998) -- (8.2495,0.713998);
\draw [c] (8.2495,0.713998) -- (8.25764,0.713998);
\definecolor{c}{rgb}{0,0,0};
\colorlet{c}{natcomp!70};
\draw [c] (8.26577,0.704207) -- (8.26577,0.72189);
\draw [c] (8.26577,0.72189) -- (8.26577,0.739573);
\draw [c] (8.25764,0.72189) -- (8.26577,0.72189);
\draw [c] (8.26577,0.72189) -- (8.27391,0.72189);
\definecolor{c}{rgb}{0,0,0};
\colorlet{c}{natcomp!70};
\draw [c] (8.28205,0.699406) -- (8.28205,0.719236);
\draw [c] (8.28205,0.719236) -- (8.28205,0.739065);
\draw [c] (8.27391,0.719236) -- (8.28205,0.719236);
\draw [c] (8.28205,0.719236) -- (8.29018,0.719236);
\definecolor{c}{rgb}{0,0,0};
\colorlet{c}{natcomp!70};
\draw [c] (8.29832,0.686897) -- (8.29832,0.686904);
\draw [c] (8.29832,0.686904) -- (8.29832,0.686911);
\draw [c] (8.29018,0.686904) -- (8.29832,0.686904);
\draw [c] (8.29832,0.686904) -- (8.30645,0.686904);
\definecolor{c}{rgb}{0,0,0};
\colorlet{c}{natcomp!70};
\draw [c] (8.31459,0.692156) -- (8.31459,0.704967);
\draw [c] (8.31459,0.704967) -- (8.31459,0.717779);
\draw [c] (8.30645,0.704967) -- (8.31459,0.704967);
\draw [c] (8.31459,0.704967) -- (8.32273,0.704967);
\definecolor{c}{rgb}{0,0,0};
\colorlet{c}{natcomp!70};
\draw [c] (8.33086,0.686894) -- (8.33086,0.686898);
\draw [c] (8.33086,0.686898) -- (8.33086,0.686902);
\draw [c] (8.32273,0.686898) -- (8.33086,0.686898);
\draw [c] (8.33086,0.686898) -- (8.339,0.686898);
\definecolor{c}{rgb}{0,0,0};
\colorlet{c}{natcomp!70};
\draw [c] (8.34714,0.686913) -- (8.34714,0.696808);
\draw [c] (8.34714,0.696808) -- (8.34714,0.706702);
\draw [c] (8.339,0.696808) -- (8.34714,0.696808);
\draw [c] (8.34714,0.696808) -- (8.35527,0.696808);
\definecolor{c}{rgb}{0,0,0};
\colorlet{c}{natcomp!70};
\draw [c] (8.36341,0.6869) -- (8.36341,0.686908);
\draw [c] (8.36341,0.686908) -- (8.36341,0.686916);
\draw [c] (8.35527,0.686908) -- (8.36341,0.686908);
\draw [c] (8.36341,0.686908) -- (8.37155,0.686908);
\definecolor{c}{rgb}{0,0,0};
\colorlet{c}{natcomp!70};
\draw [c] (8.37968,0.709537) -- (8.37968,0.733883);
\draw [c] (8.37968,0.733883) -- (8.37968,0.758228);
\draw [c] (8.37155,0.733883) -- (8.37968,0.733883);
\draw [c] (8.37968,0.733883) -- (8.38782,0.733883);
\definecolor{c}{rgb}{0,0,0};
\colorlet{c}{natcomp!70};
\draw [c] (8.39595,0.686897) -- (8.39595,0.696792);
\draw [c] (8.39595,0.696792) -- (8.39595,0.706686);
\draw [c] (8.38782,0.696792) -- (8.39595,0.696792);
\draw [c] (8.39595,0.696792) -- (8.40409,0.696792);
\definecolor{c}{rgb}{0,0,0};
\colorlet{c}{natcomp!70};
\draw [c] (8.41223,0.686907) -- (8.41223,0.696023);
\draw [c] (8.41223,0.696023) -- (8.41223,0.705138);
\draw [c] (8.40409,0.696023) -- (8.41223,0.696023);
\draw [c] (8.41223,0.696023) -- (8.42036,0.696023);
\definecolor{c}{rgb}{0,0,0};
\colorlet{c}{natcomp!70};
\draw [c] (8.4285,0.691882) -- (8.4285,0.703914);
\draw [c] (8.4285,0.703914) -- (8.4285,0.715947);
\draw [c] (8.42036,0.703914) -- (8.4285,0.703914);
\draw [c] (8.4285,0.703914) -- (8.43664,0.703914);
\definecolor{c}{rgb}{0,0,0};
\colorlet{c}{natcomp!70};
\draw [c] (8.44477,0.686908) -- (8.44477,0.69814);
\draw [c] (8.44477,0.69814) -- (8.44477,0.709372);
\draw [c] (8.43664,0.69814) -- (8.44477,0.69814);
\draw [c] (8.44477,0.69814) -- (8.45291,0.69814);
\definecolor{c}{rgb}{0,0,0};
\colorlet{c}{natcomp!70};
\draw [c] (8.46105,0.686896) -- (8.46105,0.686903);
\draw [c] (8.46105,0.686903) -- (8.46105,0.686909);
\draw [c] (8.45291,0.686903) -- (8.46105,0.686903);
\draw [c] (8.46105,0.686903) -- (8.46918,0.686903);
\definecolor{c}{rgb}{0,0,0};
\colorlet{c}{natcomp!70};
\draw [c] (8.47732,0.686902) -- (8.47732,0.686911);
\draw [c] (8.47732,0.686911) -- (8.47732,0.686919);
\draw [c] (8.46918,0.686911) -- (8.47732,0.686911);
\draw [c] (8.47732,0.686911) -- (8.48545,0.686911);
\definecolor{c}{rgb}{0,0,0};
\colorlet{c}{natcomp!70};
\draw [c] (8.49359,0.686915) -- (8.49359,0.712075);
\draw [c] (8.49359,0.712075) -- (8.49359,0.737235);
\draw [c] (8.48545,0.712075) -- (8.49359,0.712075);
\draw [c] (8.49359,0.712075) -- (8.50173,0.712075);
\definecolor{c}{rgb}{0,0,0};
\colorlet{c}{natcomp!70};
\draw [c] (8.50986,0.686899) -- (8.50986,0.686907);
\draw [c] (8.50986,0.686907) -- (8.50986,0.686915);
\draw [c] (8.50173,0.686907) -- (8.50986,0.686907);
\draw [c] (8.50986,0.686907) -- (8.518,0.686907);
\definecolor{c}{rgb}{0,0,0};
\colorlet{c}{natcomp!70};
\draw [c] (8.52614,0.686912) -- (8.52614,0.694113);
\draw [c] (8.52614,0.694113) -- (8.52614,0.701314);
\draw [c] (8.518,0.694113) -- (8.52614,0.694113);
\draw [c] (8.52614,0.694113) -- (8.53427,0.694113);
\definecolor{c}{rgb}{0,0,0};
\colorlet{c}{natcomp!70};
\draw [c] (8.54241,0.70362) -- (8.54241,0.720773);
\draw [c] (8.54241,0.720773) -- (8.54241,0.737927);
\draw [c] (8.53427,0.720773) -- (8.54241,0.720773);
\draw [c] (8.54241,0.720773) -- (8.55055,0.720773);
\definecolor{c}{rgb}{0,0,0};
\colorlet{c}{natcomp!70};
\draw [c] (8.55868,0.686902) -- (8.55868,0.697697);
\draw [c] (8.55868,0.697697) -- (8.55868,0.708492);
\draw [c] (8.55055,0.697697) -- (8.55868,0.697697);
\draw [c] (8.55868,0.697697) -- (8.56682,0.697697);
\definecolor{c}{rgb}{0,0,0};
\colorlet{c}{natcomp!70};
\draw [c] (8.57495,0.692719) -- (8.57495,0.70709);
\draw [c] (8.57495,0.70709) -- (8.57495,0.721461);
\draw [c] (8.56682,0.70709) -- (8.57495,0.70709);
\draw [c] (8.57495,0.70709) -- (8.58309,0.70709);
\definecolor{c}{rgb}{0,0,0};
\colorlet{c}{natcomp!70};
\draw [c] (8.59123,0.686902) -- (8.59123,0.686909);
\draw [c] (8.59123,0.686909) -- (8.59123,0.686917);
\draw [c] (8.58309,0.686909) -- (8.59123,0.686909);
\draw [c] (8.59123,0.686909) -- (8.59936,0.686909);
\definecolor{c}{rgb}{0,0,0};
\colorlet{c}{natcomp!70};
\draw [c] (8.6075,0.686894) -- (8.6075,0.686897);
\draw [c] (8.6075,0.686897) -- (8.6075,0.686901);
\draw [c] (8.59936,0.686897) -- (8.6075,0.686897);
\draw [c] (8.6075,0.686897) -- (8.61564,0.686897);
\definecolor{c}{rgb}{0,0,0};
\colorlet{c}{natcomp!70};
\draw [c] (8.62377,0.686903) -- (8.62377,0.686912);
\draw [c] (8.62377,0.686912) -- (8.62377,0.686922);
\draw [c] (8.61564,0.686912) -- (8.62377,0.686912);
\draw [c] (8.62377,0.686912) -- (8.63191,0.686912);
\definecolor{c}{rgb}{0,0,0};
\colorlet{c}{natcomp!70};
\draw [c] (8.64005,0.706296) -- (8.64005,0.725795);
\draw [c] (8.64005,0.725795) -- (8.64005,0.745294);
\draw [c] (8.63191,0.725795) -- (8.64005,0.725795);
\draw [c] (8.64005,0.725795) -- (8.64818,0.725795);
\definecolor{c}{rgb}{0,0,0};
\colorlet{c}{natcomp!70};
\draw [c] (8.65632,0.686902) -- (8.65632,0.697676);
\draw [c] (8.65632,0.697676) -- (8.65632,0.708449);
\draw [c] (8.64818,0.697676) -- (8.65632,0.697676);
\draw [c] (8.65632,0.697676) -- (8.66445,0.697676);
\definecolor{c}{rgb}{0,0,0};
\colorlet{c}{natcomp!70};
\draw [c] (8.67259,0.697669) -- (8.67259,0.712756);
\draw [c] (8.67259,0.712756) -- (8.67259,0.727844);
\draw [c] (8.66445,0.712756) -- (8.67259,0.712756);
\draw [c] (8.67259,0.712756) -- (8.68073,0.712756);
\definecolor{c}{rgb}{0,0,0};
\colorlet{c}{natcomp!70};
\draw [c] (8.68886,0.686917) -- (8.68886,0.695056);
\draw [c] (8.68886,0.695056) -- (8.68886,0.703195);
\draw [c] (8.68073,0.695056) -- (8.68886,0.695056);
\draw [c] (8.68886,0.695056) -- (8.697,0.695056);
\definecolor{c}{rgb}{0,0,0};
\colorlet{c}{natcomp!70};
\draw [c] (8.70514,0.686894) -- (8.70514,0.686898);
\draw [c] (8.70514,0.686898) -- (8.70514,0.686902);
\draw [c] (8.697,0.686898) -- (8.70514,0.686898);
\draw [c] (8.70514,0.686898) -- (8.71327,0.686898);
\definecolor{c}{rgb}{0,0,0};
\colorlet{c}{natcomp!70};
\draw [c] (8.72141,0.686896) -- (8.72141,0.686903);
\draw [c] (8.72141,0.686903) -- (8.72141,0.686909);
\draw [c] (8.71327,0.686903) -- (8.72141,0.686903);
\draw [c] (8.72141,0.686903) -- (8.72955,0.686903);
\definecolor{c}{rgb}{0,0,0};
\colorlet{c}{natcomp!70};
\draw [c] (8.73768,0.6869) -- (8.73768,0.686909);
\draw [c] (8.73768,0.686909) -- (8.73768,0.686917);
\draw [c] (8.72955,0.686909) -- (8.73768,0.686909);
\draw [c] (8.73768,0.686909) -- (8.74582,0.686909);
\definecolor{c}{rgb}{0,0,0};
\colorlet{c}{natcomp!70};
\draw [c] (8.75395,0.686898) -- (8.75395,0.697671);
\draw [c] (8.75395,0.697671) -- (8.75395,0.708445);
\draw [c] (8.74582,0.697671) -- (8.75395,0.697671);
\draw [c] (8.75395,0.697671) -- (8.76209,0.697671);
\definecolor{c}{rgb}{0,0,0};
\colorlet{c}{natcomp!70};
\draw [c] (8.77023,0.686896) -- (8.77023,0.686902);
\draw [c] (8.77023,0.686902) -- (8.77023,0.686908);
\draw [c] (8.76209,0.686902) -- (8.77023,0.686902);
\draw [c] (8.77023,0.686902) -- (8.77836,0.686902);
\definecolor{c}{rgb}{0,0,0};
\colorlet{c}{natcomp!70};
\draw [c] (8.80277,0.686894) -- (8.80277,0.697305);
\draw [c] (8.80277,0.697305) -- (8.80277,0.707716);
\draw [c] (8.79464,0.697305) -- (8.80277,0.697305);
\draw [c] (8.80277,0.697305) -- (8.81091,0.697305);
\definecolor{c}{rgb}{0,0,0};
\colorlet{c}{natcomp!70};
\draw [c] (8.81905,0.686894) -- (8.81905,0.686899);
\draw [c] (8.81905,0.686899) -- (8.81905,0.686904);
\draw [c] (8.81091,0.686899) -- (8.81905,0.686899);
\draw [c] (8.81905,0.686899) -- (8.82718,0.686899);
\definecolor{c}{rgb}{0,0,0};
\colorlet{c}{natcomp!70};
\draw [c] (8.83532,0.686894) -- (8.83532,0.686899);
\draw [c] (8.83532,0.686899) -- (8.83532,0.686904);
\draw [c] (8.82718,0.686899) -- (8.83532,0.686899);
\draw [c] (8.83532,0.686899) -- (8.84345,0.686899);
\definecolor{c}{rgb}{0,0,0};
\colorlet{c}{natcomp!70};
\draw [c] (8.85159,0.686907) -- (8.85159,0.696022);
\draw [c] (8.85159,0.696022) -- (8.85159,0.705137);
\draw [c] (8.84345,0.696022) -- (8.85159,0.696022);
\draw [c] (8.85159,0.696022) -- (8.85973,0.696022);
\definecolor{c}{rgb}{0,0,0};
\colorlet{c}{natcomp!70};
\draw [c] (8.86786,0.686894) -- (8.86786,0.686897);
\draw [c] (8.86786,0.686897) -- (8.86786,0.686901);
\draw [c] (8.85973,0.686897) -- (8.86786,0.686897);
\draw [c] (8.86786,0.686897) -- (8.876,0.686897);
\definecolor{c}{rgb}{0,0,0};
\colorlet{c}{natcomp!70};
\draw [c] (8.88414,0.700826) -- (8.88414,0.720637);
\draw [c] (8.88414,0.720637) -- (8.88414,0.740448);
\draw [c] (8.876,0.720637) -- (8.88414,0.720637);
\draw [c] (8.88414,0.720637) -- (8.89227,0.720637);
\definecolor{c}{rgb}{0,0,0};
\colorlet{c}{natcomp!70};
\draw [c] (8.90041,0.686894) -- (8.90041,0.686897);
\draw [c] (8.90041,0.686897) -- (8.90041,0.686901);
\draw [c] (8.89227,0.686897) -- (8.90041,0.686897);
\draw [c] (8.90041,0.686897) -- (8.90855,0.686897);
\definecolor{c}{rgb}{0,0,0};
\colorlet{c}{natcomp!70};
\draw [c] (8.91668,0.722516) -- (8.91668,0.755184);
\draw [c] (8.91668,0.755184) -- (8.91668,0.787852);
\draw [c] (8.90855,0.755184) -- (8.91668,0.755184);
\draw [c] (8.91668,0.755184) -- (8.92482,0.755184);
\definecolor{c}{rgb}{0,0,0};
\colorlet{c}{natcomp!70};
\draw [c] (8.93295,0.6869) -- (8.93295,0.694622);
\draw [c] (8.93295,0.694622) -- (8.93295,0.702344);
\draw [c] (8.92482,0.694622) -- (8.93295,0.694622);
\draw [c] (8.93295,0.694622) -- (8.94109,0.694622);
\definecolor{c}{rgb}{0,0,0};
\colorlet{c}{natcomp!70};
\draw [c] (8.94923,0.686894) -- (8.94923,0.686898);
\draw [c] (8.94923,0.686898) -- (8.94923,0.686902);
\draw [c] (8.94109,0.686898) -- (8.94923,0.686898);
\draw [c] (8.94923,0.686898) -- (8.95736,0.686898);
\definecolor{c}{rgb}{0,0,0};
\colorlet{c}{natcomp!70};
\draw [c] (8.9655,0.686894) -- (8.9655,0.686898);
\draw [c] (8.9655,0.686898) -- (8.9655,0.686902);
\draw [c] (8.95736,0.686898) -- (8.9655,0.686898);
\draw [c] (8.9655,0.686898) -- (8.97364,0.686898);
\definecolor{c}{rgb}{0,0,0};
\colorlet{c}{natcomp!70};
\draw [c] (8.98177,0.686896) -- (8.98177,0.686902);
\draw [c] (8.98177,0.686902) -- (8.98177,0.686907);
\draw [c] (8.97364,0.686902) -- (8.98177,0.686902);
\draw [c] (8.98177,0.686902) -- (8.98991,0.686902);
\definecolor{c}{rgb}{0,0,0};
\colorlet{c}{natcomp!70};
\draw [c] (8.99805,0.686904) -- (8.99805,0.686922);
\draw [c] (8.99805,0.686922) -- (8.99805,0.686939);
\draw [c] (8.98991,0.686922) -- (8.99805,0.686922);
\draw [c] (8.99805,0.686922) -- (9.00618,0.686922);
\definecolor{c}{rgb}{0,0,0};
\colorlet{c}{natcomp!70};
\draw [c] (9.01432,0.686897) -- (9.01432,0.686903);
\draw [c] (9.01432,0.686903) -- (9.01432,0.68691);
\draw [c] (9.00618,0.686903) -- (9.01432,0.686903);
\draw [c] (9.01432,0.686903) -- (9.02245,0.686903);
\definecolor{c}{rgb}{0,0,0};
\colorlet{c}{natcomp!70};
\draw [c] (9.04686,0.708993) -- (9.04686,0.731658);
\draw [c] (9.04686,0.731658) -- (9.04686,0.754323);
\draw [c] (9.03873,0.731658) -- (9.04686,0.731658);
\draw [c] (9.04686,0.731658) -- (9.055,0.731658);
\definecolor{c}{rgb}{0,0,0};
\colorlet{c}{natcomp!70};
\draw [c] (9.07941,0.692085) -- (9.07941,0.704618);
\draw [c] (9.07941,0.704618) -- (9.07941,0.717151);
\draw [c] (9.07127,0.704618) -- (9.07941,0.704618);
\draw [c] (9.07941,0.704618) -- (9.08755,0.704618);
\definecolor{c}{rgb}{0,0,0};
\colorlet{c}{natcomp!70};
\draw [c] (9.09568,0.686897) -- (9.09568,0.695861);
\draw [c] (9.09568,0.695861) -- (9.09568,0.704825);
\draw [c] (9.08755,0.695861) -- (9.09568,0.695861);
\draw [c] (9.09568,0.695861) -- (9.10382,0.695861);
\definecolor{c}{rgb}{0,0,0};
\colorlet{c}{natcomp!70};
\draw [c] (9.12823,0.686894) -- (9.12823,0.686897);
\draw [c] (9.12823,0.686897) -- (9.12823,0.686901);
\draw [c] (9.12009,0.686897) -- (9.12823,0.686897);
\draw [c] (9.12823,0.686897) -- (9.13636,0.686897);
\definecolor{c}{rgb}{0,0,0};
\colorlet{c}{natcomp!70};
\draw [c] (9.1445,0.686899) -- (9.1445,0.696014);
\draw [c] (9.1445,0.696014) -- (9.1445,0.70513);
\draw [c] (9.13636,0.696014) -- (9.1445,0.696014);
\draw [c] (9.1445,0.696014) -- (9.15264,0.696014);
\definecolor{c}{rgb}{0,0,0};
\colorlet{c}{natcomp!70};
\draw [c] (9.16077,0.686897) -- (9.16077,0.686903);
\draw [c] (9.16077,0.686903) -- (9.16077,0.68691);
\draw [c] (9.15264,0.686903) -- (9.16077,0.686903);
\draw [c] (9.16077,0.686903) -- (9.16891,0.686903);
\definecolor{c}{rgb}{0,0,0};
\colorlet{c}{natcomp!70};
\draw [c] (9.17705,0.686894) -- (9.17705,0.705403);
\draw [c] (9.17705,0.705403) -- (9.17705,0.723913);
\draw [c] (9.16891,0.705403) -- (9.17705,0.705403);
\draw [c] (9.17705,0.705403) -- (9.18518,0.705403);
\definecolor{c}{rgb}{0,0,0};
\colorlet{c}{natcomp!70};
\draw [c] (9.19332,0.686897) -- (9.19332,0.686906);
\draw [c] (9.19332,0.686906) -- (9.19332,0.686914);
\draw [c] (9.18518,0.686906) -- (9.19332,0.686906);
\draw [c] (9.19332,0.686906) -- (9.20145,0.686906);
\definecolor{c}{rgb}{0,0,0};
\colorlet{c}{natcomp!70};
\draw [c] (9.20959,0.686906) -- (9.20959,0.694628);
\draw [c] (9.20959,0.694628) -- (9.20959,0.70235);
\draw [c] (9.20145,0.694628) -- (9.20959,0.694628);
\draw [c] (9.20959,0.694628) -- (9.21773,0.694628);
\definecolor{c}{rgb}{0,0,0};
\colorlet{c}{natcomp!70};
\draw [c] (9.22586,0.686898) -- (9.22586,0.697671);
\draw [c] (9.22586,0.697671) -- (9.22586,0.708444);
\draw [c] (9.21773,0.697671) -- (9.22586,0.697671);
\draw [c] (9.22586,0.697671) -- (9.234,0.697671);
\definecolor{c}{rgb}{0,0,0};
\colorlet{c}{natcomp!70};
\draw [c] (9.24214,0.697919) -- (9.24214,0.713014);
\draw [c] (9.24214,0.713014) -- (9.24214,0.72811);
\draw [c] (9.234,0.713014) -- (9.24214,0.713014);
\draw [c] (9.24214,0.713014) -- (9.25027,0.713014);
\definecolor{c}{rgb}{0,0,0};
\colorlet{c}{natcomp!70};
\draw [c] (9.25841,0.686896) -- (9.25841,0.686902);
\draw [c] (9.25841,0.686902) -- (9.25841,0.686908);
\draw [c] (9.25027,0.686902) -- (9.25841,0.686902);
\draw [c] (9.25841,0.686902) -- (9.26655,0.686902);
\definecolor{c}{rgb}{0,0,0};
\colorlet{c}{natcomp!70};
\draw [c] (9.27468,0.686897) -- (9.27468,0.686903);
\draw [c] (9.27468,0.686903) -- (9.27468,0.68691);
\draw [c] (9.26655,0.686903) -- (9.27468,0.686903);
\draw [c] (9.27468,0.686903) -- (9.28282,0.686903);
\definecolor{c}{rgb}{0,0,0};
\colorlet{c}{natcomp!70};
\draw [c] (9.30723,0.691421) -- (9.30723,0.702341);
\draw [c] (9.30723,0.702341) -- (9.30723,0.713261);
\draw [c] (9.29909,0.702341) -- (9.30723,0.702341);
\draw [c] (9.30723,0.702341) -- (9.31536,0.702341);
\definecolor{c}{rgb}{0,0,0};
\colorlet{c}{natcomp!70};
\draw [c] (9.33977,0.686902) -- (9.33977,0.69504);
\draw [c] (9.33977,0.69504) -- (9.33977,0.703179);
\draw [c] (9.33164,0.69504) -- (9.33977,0.69504);
\draw [c] (9.33977,0.69504) -- (9.34791,0.69504);
\definecolor{c}{rgb}{0,0,0};
\colorlet{c}{natcomp!70};
\draw [c] (9.35605,0.693227) -- (9.35605,0.708478);
\draw [c] (9.35605,0.708478) -- (9.35605,0.723729);
\draw [c] (9.34791,0.708478) -- (9.35605,0.708478);
\draw [c] (9.35605,0.708478) -- (9.36418,0.708478);
\definecolor{c}{rgb}{0,0,0};
\colorlet{c}{natcomp!70};
\draw [c] (9.37232,0.686896) -- (9.37232,0.686902);
\draw [c] (9.37232,0.686902) -- (9.37232,0.686908);
\draw [c] (9.36418,0.686902) -- (9.37232,0.686902);
\draw [c] (9.37232,0.686902) -- (9.38046,0.686902);
\definecolor{c}{rgb}{0,0,0};
\colorlet{c}{natcomp!70};
\draw [c] (9.38859,0.686897) -- (9.38859,0.696013);
\draw [c] (9.38859,0.696013) -- (9.38859,0.705128);
\draw [c] (9.38046,0.696013) -- (9.38859,0.696013);
\draw [c] (9.38859,0.696013) -- (9.39673,0.696013);
\definecolor{c}{rgb}{0,0,0};
\colorlet{c}{natcomp!70};
\draw [c] (9.43741,0.686902) -- (9.43741,0.697312);
\draw [c] (9.43741,0.697312) -- (9.43741,0.707723);
\draw [c] (9.42927,0.697312) -- (9.43741,0.697312);
\draw [c] (9.43741,0.697312) -- (9.44555,0.697312);
\definecolor{c}{rgb}{0,0,0};
\colorlet{c}{natcomp!70};
\draw [c] (9.45368,0.686897) -- (9.45368,0.686903);
\draw [c] (9.45368,0.686903) -- (9.45368,0.68691);
\draw [c] (9.44555,0.686903) -- (9.45368,0.686903);
\draw [c] (9.45368,0.686903) -- (9.46182,0.686903);
\definecolor{c}{rgb}{0,0,0};
\colorlet{c}{natcomp!70};
\draw [c] (9.46995,0.692681) -- (9.46995,0.706988);
\draw [c] (9.46995,0.706988) -- (9.46995,0.721296);
\draw [c] (9.46182,0.706988) -- (9.46995,0.706988);
\draw [c] (9.46995,0.706988) -- (9.47809,0.706988);
\definecolor{c}{rgb}{0,0,0};
\colorlet{c}{natcomp!70};
\draw [c] (9.48623,0.686894) -- (9.48623,0.686903);
\draw [c] (9.48623,0.686903) -- (9.48623,0.686911);
\draw [c] (9.47809,0.686903) -- (9.48623,0.686903);
\draw [c] (9.48623,0.686903) -- (9.49436,0.686903);
\definecolor{c}{rgb}{0,0,0};
\colorlet{c}{natcomp!70};
\draw [c] (9.5025,0.686898) -- (9.5025,0.696793);
\draw [c] (9.5025,0.696793) -- (9.5025,0.706687);
\draw [c] (9.49436,0.696793) -- (9.5025,0.696793);
\draw [c] (9.5025,0.696793) -- (9.51064,0.696793);
\definecolor{c}{rgb}{0,0,0};
\colorlet{c}{natcomp!70};
\draw [c] (9.51877,0.686894) -- (9.51877,0.697689);
\draw [c] (9.51877,0.697689) -- (9.51877,0.708483);
\draw [c] (9.51064,0.697689) -- (9.51877,0.697689);
\draw [c] (9.51877,0.697689) -- (9.52691,0.697689);
\definecolor{c}{rgb}{0,0,0};
\colorlet{c}{natcomp!70};
\draw [c] (9.53505,0.686896) -- (9.53505,0.686901);
\draw [c] (9.53505,0.686901) -- (9.53505,0.686906);
\draw [c] (9.52691,0.686901) -- (9.53505,0.686901);
\draw [c] (9.53505,0.686901) -- (9.54318,0.686901);
\definecolor{c}{rgb}{0,0,0};
\colorlet{c}{natcomp!70};
\draw [c] (9.55132,0.686896) -- (9.55132,0.686902);
\draw [c] (9.55132,0.686902) -- (9.55132,0.686908);
\draw [c] (9.54318,0.686902) -- (9.55132,0.686902);
\draw [c] (9.55132,0.686902) -- (9.55945,0.686902);
\definecolor{c}{rgb}{0,0,0};
\colorlet{c}{natcomp!70};
\draw [c] (9.56759,0.686894) -- (9.56759,0.696788);
\draw [c] (9.56759,0.696788) -- (9.56759,0.706683);
\draw [c] (9.55945,0.696788) -- (9.56759,0.696788);
\draw [c] (9.56759,0.696788) -- (9.57573,0.696788);
\definecolor{c}{rgb}{0,0,0};
\colorlet{c}{natcomp!70};
\draw [c] (9.58386,0.686894) -- (9.58386,0.686898);
\draw [c] (9.58386,0.686898) -- (9.58386,0.686902);
\draw [c] (9.57573,0.686898) -- (9.58386,0.686898);
\draw [c] (9.58386,0.686898) -- (9.592,0.686898);
\definecolor{c}{rgb}{0,0,0};
\colorlet{c}{natcomp!70};
\draw [c] (9.60014,0.686899) -- (9.60014,0.695761);
\draw [c] (9.60014,0.695761) -- (9.60014,0.704623);
\draw [c] (9.592,0.695761) -- (9.60014,0.695761);
\draw [c] (9.60014,0.695761) -- (9.60827,0.695761);
\definecolor{c}{rgb}{0,0,0};
\colorlet{c}{natcomp!70};
\draw [c] (9.61641,0.691959) -- (9.61641,0.70451);
\draw [c] (9.61641,0.70451) -- (9.61641,0.717061);
\draw [c] (9.60827,0.70451) -- (9.61641,0.70451);
\draw [c] (9.61641,0.70451) -- (9.62455,0.70451);
\definecolor{c}{rgb}{0,0,0};
\colorlet{c}{natcomp!70};
\draw [c] (9.64895,0.686899) -- (9.64895,0.686907);
\draw [c] (9.64895,0.686907) -- (9.64895,0.686914);
\draw [c] (9.64082,0.686907) -- (9.64895,0.686907);
\draw [c] (9.64895,0.686907) -- (9.65709,0.686907);
\definecolor{c}{rgb}{0,0,0};
\colorlet{c}{natcomp!70};
\draw [c] (9.66523,0.686899) -- (9.66523,0.686907);
\draw [c] (9.66523,0.686907) -- (9.66523,0.686914);
\draw [c] (9.65709,0.686907) -- (9.66523,0.686907);
\draw [c] (9.66523,0.686907) -- (9.67336,0.686907);
\definecolor{c}{rgb}{0,0,0};
\colorlet{c}{natcomp!70};
\draw [c] (9.6815,0.686894) -- (9.6815,0.686897);
\draw [c] (9.6815,0.686897) -- (9.6815,0.686901);
\draw [c] (9.67336,0.686897) -- (9.6815,0.686897);
\draw [c] (9.6815,0.686897) -- (9.68964,0.686897);
\definecolor{c}{rgb}{0,0,0};
\colorlet{c}{natcomp!70};
\draw [c] (9.71405,0.692495) -- (9.71405,0.706167);
\draw [c] (9.71405,0.706167) -- (9.71405,0.719839);
\draw [c] (9.70591,0.706167) -- (9.71405,0.706167);
\draw [c] (9.71405,0.706167) -- (9.72218,0.706167);
\definecolor{c}{rgb}{0,0,0};
\colorlet{c}{natcomp!70};
\draw [c] (9.74659,0.686894) -- (9.74659,0.686897);
\draw [c] (9.74659,0.686897) -- (9.74659,0.686901);
\draw [c] (9.73845,0.686897) -- (9.74659,0.686897);
\draw [c] (9.74659,0.686897) -- (9.75473,0.686897);
\definecolor{c}{rgb}{0,0,0};
\colorlet{c}{natcomp!70};
\draw [c] (9.76286,0.686896) -- (9.76286,0.686901);
\draw [c] (9.76286,0.686901) -- (9.76286,0.686906);
\draw [c] (9.75473,0.686901) -- (9.76286,0.686901);
\draw [c] (9.76286,0.686901) -- (9.771,0.686901);
\definecolor{c}{rgb}{0,0,0};
\colorlet{c}{natcomp!70};
\draw [c] (9.77914,0.686896) -- (9.77914,0.686902);
\draw [c] (9.77914,0.686902) -- (9.77914,0.686908);
\draw [c] (9.771,0.686902) -- (9.77914,0.686902);
\draw [c] (9.77914,0.686902) -- (9.78727,0.686902);
\definecolor{c}{rgb}{0,0,0};
\colorlet{c}{natcomp!70};
\draw [c] (9.79541,0.686898) -- (9.79541,0.696014);
\draw [c] (9.79541,0.696014) -- (9.79541,0.705129);
\draw [c] (9.78727,0.696014) -- (9.79541,0.696014);
\draw [c] (9.79541,0.696014) -- (9.80354,0.696014);
\definecolor{c}{rgb}{0,0,0};
\colorlet{c}{natcomp!70};
\draw [c] (9.81168,0.686894) -- (9.81168,0.686899);
\draw [c] (9.81168,0.686899) -- (9.81168,0.686903);
\draw [c] (9.80354,0.686899) -- (9.81168,0.686899);
\draw [c] (9.81168,0.686899) -- (9.81982,0.686899);
\definecolor{c}{rgb}{0,0,0};
\colorlet{c}{natcomp!70};
\draw [c] (9.82795,0.692367) -- (9.82795,0.70565);
\draw [c] (9.82795,0.70565) -- (9.82795,0.718933);
\draw [c] (9.81982,0.70565) -- (9.82795,0.70565);
\draw [c] (9.82795,0.70565) -- (9.83609,0.70565);
\definecolor{c}{rgb}{0,0,0};
\colorlet{c}{natcomp!70};
\draw [c] (9.84423,0.686894) -- (9.84423,0.686898);
\draw [c] (9.84423,0.686898) -- (9.84423,0.686902);
\draw [c] (9.83609,0.686898) -- (9.84423,0.686898);
\draw [c] (9.84423,0.686898) -- (9.85236,0.686898);
\definecolor{c}{rgb}{0,0,0};
\colorlet{c}{natcomp!70};
\draw [c] (9.8605,0.686894) -- (9.8605,0.686901);
\draw [c] (9.8605,0.686901) -- (9.8605,0.686908);
\draw [c] (9.85236,0.686901) -- (9.8605,0.686901);
\draw [c] (9.8605,0.686901) -- (9.86864,0.686901);
\definecolor{c}{rgb}{0,0,0};
\colorlet{c}{natcomp!70};
\draw [c] (9.87677,0.686894) -- (9.87677,0.686899);
\draw [c] (9.87677,0.686899) -- (9.87677,0.686903);
\draw [c] (9.86864,0.686899) -- (9.87677,0.686899);
\draw [c] (9.87677,0.686899) -- (9.88491,0.686899);
\definecolor{c}{rgb}{0,0,0};
\colorlet{c}{natcomp!70};
\draw [c] (9.89305,0.691538) -- (9.89305,0.702957);
\draw [c] (9.89305,0.702957) -- (9.89305,0.714375);
\draw [c] (9.88491,0.702957) -- (9.89305,0.702957);
\draw [c] (9.89305,0.702957) -- (9.90118,0.702957);
\definecolor{c}{rgb}{0,0,0};
\colorlet{c}{natcomp!70};
\draw [c] (9.90932,0.692367) -- (9.90932,0.70565);
\draw [c] (9.90932,0.70565) -- (9.90932,0.718933);
\draw [c] (9.90118,0.70565) -- (9.90932,0.70565);
\draw [c] (9.90932,0.70565) -- (9.91745,0.70565);
\definecolor{c}{rgb}{0,0,0};
\colorlet{c}{natcomp!70};
\draw [c] (9.94186,0.686899) -- (9.94186,0.686907);
\draw [c] (9.94186,0.686907) -- (9.94186,0.686915);
\draw [c] (9.93373,0.686907) -- (9.94186,0.686907);
\draw [c] (9.94186,0.686907) -- (9.95,0.686907);
\definecolor{c}{rgb}{0,0,0};
\colorlet{c}{natcomp!70};
\draw [c] (1.04068,0.686894) -- (1.04068,0.6869);
\draw [c] (1.04068,0.6869) -- (1.04068,0.686907);
\draw [c] (1.03255,0.6869) -- (1.04068,0.6869);
\draw [c] (1.04068,0.6869) -- (1.04882,0.6869);
\definecolor{c}{rgb}{0,0,0};
\colorlet{c}{natcomp!70};
\draw [c] (1.10577,0.686894) -- (1.10577,0.705403);
\draw [c] (1.10577,0.705403) -- (1.10577,0.723913);
\draw [c] (1.09764,0.705403) -- (1.10577,0.705403);
\draw [c] (1.10577,0.705403) -- (1.11391,0.705403);
\definecolor{c}{rgb}{0,0,0};
\colorlet{c}{natcomp!70};
\draw [c] (1.12205,0.6869) -- (1.12205,0.708668);
\draw [c] (1.12205,0.708668) -- (1.12205,0.730436);
\draw [c] (1.11391,0.708668) -- (1.12205,0.708668);
\draw [c] (1.12205,0.708668) -- (1.13018,0.708668);
\definecolor{c}{rgb}{0,0,0};
\colorlet{c}{natcomp!70};
\draw [c] (1.13832,0.686894) -- (1.13832,0.686899);
\draw [c] (1.13832,0.686899) -- (1.13832,0.686904);
\draw [c] (1.13018,0.686899) -- (1.13832,0.686899);
\draw [c] (1.13832,0.686899) -- (1.14645,0.686899);
\definecolor{c}{rgb}{0,0,0};
\colorlet{c}{natcomp!70};
\draw [c] (1.15459,0.686907) -- (1.15459,0.695046);
\draw [c] (1.15459,0.695046) -- (1.15459,0.703185);
\draw [c] (1.14645,0.695046) -- (1.15459,0.695046);
\draw [c] (1.15459,0.695046) -- (1.16273,0.695046);
\definecolor{c}{rgb}{0,0,0};
\colorlet{c}{natcomp!70};
\draw [c] (1.17086,0.686894) -- (1.17086,0.686904);
\draw [c] (1.17086,0.686904) -- (1.17086,0.686914);
\draw [c] (1.16273,0.686904) -- (1.17086,0.686904);
\draw [c] (1.17086,0.686904) -- (1.179,0.686904);
\definecolor{c}{rgb}{0,0,0};
\colorlet{c}{natcomp!70};
\draw [c] (1.18714,0.686897) -- (1.18714,0.686904);
\draw [c] (1.18714,0.686904) -- (1.18714,0.686911);
\draw [c] (1.179,0.686904) -- (1.18714,0.686904);
\draw [c] (1.18714,0.686904) -- (1.19527,0.686904);
\definecolor{c}{rgb}{0,0,0};
\colorlet{c}{natcomp!70};
\draw [c] (1.20341,0.686906) -- (1.20341,0.705415);
\draw [c] (1.20341,0.705415) -- (1.20341,0.723925);
\draw [c] (1.19527,0.705415) -- (1.20341,0.705415);
\draw [c] (1.20341,0.705415) -- (1.21155,0.705415);
\definecolor{c}{rgb}{0,0,0};
\colorlet{c}{natcomp!70};
\draw [c] (1.21968,0.686898) -- (1.21968,0.68691);
\draw [c] (1.21968,0.68691) -- (1.21968,0.686923);
\draw [c] (1.21155,0.68691) -- (1.21968,0.68691);
\draw [c] (1.21968,0.68691) -- (1.22782,0.68691);
\definecolor{c}{rgb}{0,0,0};
\colorlet{c}{natcomp!70};
\draw [c] (1.25223,0.695733) -- (1.25223,0.723286);
\draw [c] (1.25223,0.723286) -- (1.25223,0.75084);
\draw [c] (1.24409,0.723286) -- (1.25223,0.723286);
\draw [c] (1.25223,0.723286) -- (1.26036,0.723286);
\definecolor{c}{rgb}{0,0,0};
\colorlet{c}{natcomp!70};
\draw [c] (1.2685,0.686901) -- (1.2685,0.686911);
\draw [c] (1.2685,0.686911) -- (1.2685,0.686922);
\draw [c] (1.26036,0.686911) -- (1.2685,0.686911);
\draw [c] (1.2685,0.686911) -- (1.27664,0.686911);
\definecolor{c}{rgb}{0,0,0};
\colorlet{c}{natcomp!70};
\draw [c] (1.28477,0.686898) -- (1.28477,0.686915);
\draw [c] (1.28477,0.686915) -- (1.28477,0.686932);
\draw [c] (1.27664,0.686915) -- (1.28477,0.686915);
\draw [c] (1.28477,0.686915) -- (1.29291,0.686915);
\definecolor{c}{rgb}{0,0,0};
\colorlet{c}{natcomp!70};
\draw [c] (1.31732,0.694427) -- (1.31732,0.721188);
\draw [c] (1.31732,0.721188) -- (1.31732,0.747948);
\draw [c] (1.30918,0.721188) -- (1.31732,0.721188);
\draw [c] (1.31732,0.721188) -- (1.32545,0.721188);
\definecolor{c}{rgb}{0,0,0};
\colorlet{c}{natcomp!70};
\draw [c] (1.34986,0.686899) -- (1.34986,0.69462);
\draw [c] (1.34986,0.69462) -- (1.34986,0.702342);
\draw [c] (1.34173,0.69462) -- (1.34986,0.69462);
\draw [c] (1.34986,0.69462) -- (1.358,0.69462);
\definecolor{c}{rgb}{0,0,0};
\colorlet{c}{natcomp!70};
\draw [c] (1.36614,0.686894) -- (1.36614,0.6869);
\draw [c] (1.36614,0.6869) -- (1.36614,0.686907);
\draw [c] (1.358,0.6869) -- (1.36614,0.6869);
\draw [c] (1.36614,0.6869) -- (1.37427,0.6869);
\definecolor{c}{rgb}{0,0,0};
\colorlet{c}{natcomp!70};
\draw [c] (1.38241,0.68691) -- (1.38241,0.71207);
\draw [c] (1.38241,0.71207) -- (1.38241,0.73723);
\draw [c] (1.37427,0.71207) -- (1.38241,0.71207);
\draw [c] (1.38241,0.71207) -- (1.39055,0.71207);
\definecolor{c}{rgb}{0,0,0};
\colorlet{c}{natcomp!70};
\draw [c] (1.39868,0.696781) -- (1.39868,0.722664);
\draw [c] (1.39868,0.722664) -- (1.39868,0.748546);
\draw [c] (1.39055,0.722664) -- (1.39868,0.722664);
\draw [c] (1.39868,0.722664) -- (1.40682,0.722664);
\definecolor{c}{rgb}{0,0,0};
\colorlet{c}{natcomp!70};
\draw [c] (1.41495,0.686894) -- (1.41495,0.697305);
\draw [c] (1.41495,0.697305) -- (1.41495,0.707716);
\draw [c] (1.40682,0.697305) -- (1.41495,0.697305);
\draw [c] (1.41495,0.697305) -- (1.42309,0.697305);
\definecolor{c}{rgb}{0,0,0};
\colorlet{c}{natcomp!70};
\draw [c] (1.43123,0.6869) -- (1.43123,0.700903);
\draw [c] (1.43123,0.700903) -- (1.43123,0.714905);
\draw [c] (1.42309,0.700903) -- (1.43123,0.700903);
\draw [c] (1.43123,0.700903) -- (1.43936,0.700903);
\definecolor{c}{rgb}{0,0,0};
\colorlet{c}{natcomp!70};
\draw [c] (1.4475,0.686909) -- (1.4475,0.705418);
\draw [c] (1.4475,0.705418) -- (1.4475,0.723928);
\draw [c] (1.43936,0.705418) -- (1.4475,0.705418);
\draw [c] (1.4475,0.705418) -- (1.45564,0.705418);
\definecolor{c}{rgb}{0,0,0};
\colorlet{c}{natcomp!70};
\draw [c] (1.46377,0.686915) -- (1.46377,0.705424);
\draw [c] (1.46377,0.705424) -- (1.46377,0.723933);
\draw [c] (1.45564,0.705424) -- (1.46377,0.705424);
\draw [c] (1.46377,0.705424) -- (1.47191,0.705424);
\definecolor{c}{rgb}{0,0,0};
\colorlet{c}{natcomp!70};
\draw [c] (1.48005,0.6869) -- (1.48005,0.686916);
\draw [c] (1.48005,0.686916) -- (1.48005,0.686932);
\draw [c] (1.47191,0.686916) -- (1.48005,0.686916);
\draw [c] (1.48005,0.686916) -- (1.48818,0.686916);
\definecolor{c}{rgb}{0,0,0};
\colorlet{c}{natcomp!70};
\draw [c] (1.49632,0.686904) -- (1.49632,0.700906);
\draw [c] (1.49632,0.700906) -- (1.49632,0.714908);
\draw [c] (1.48818,0.700906) -- (1.49632,0.700906);
\draw [c] (1.49632,0.700906) -- (1.50445,0.700906);
\definecolor{c}{rgb}{0,0,0};
\colorlet{c}{natcomp!70};
\draw [c] (1.51259,0.686912) -- (1.51259,0.698144);
\draw [c] (1.51259,0.698144) -- (1.51259,0.709377);
\draw [c] (1.50445,0.698144) -- (1.51259,0.698144);
\draw [c] (1.51259,0.698144) -- (1.52073,0.698144);
\definecolor{c}{rgb}{0,0,0};
\colorlet{c}{natcomp!70};
\draw [c] (1.52886,0.686905) -- (1.52886,0.697316);
\draw [c] (1.52886,0.697316) -- (1.52886,0.707727);
\draw [c] (1.52073,0.697316) -- (1.52886,0.697316);
\draw [c] (1.52886,0.697316) -- (1.537,0.697316);
\definecolor{c}{rgb}{0,0,0};
\colorlet{c}{natcomp!70};
\draw [c] (1.54514,0.68691) -- (1.54514,0.686925);
\draw [c] (1.54514,0.686925) -- (1.54514,0.68694);
\draw [c] (1.537,0.686925) -- (1.54514,0.686925);
\draw [c] (1.54514,0.686925) -- (1.55327,0.686925);
\definecolor{c}{rgb}{0,0,0};
\colorlet{c}{natcomp!70};
\draw [c] (1.56141,0.686899) -- (1.56141,0.686912);
\draw [c] (1.56141,0.686912) -- (1.56141,0.686926);
\draw [c] (1.55327,0.686912) -- (1.56141,0.686912);
\draw [c] (1.56141,0.686912) -- (1.56955,0.686912);
\definecolor{c}{rgb}{0,0,0};
\colorlet{c}{natcomp!70};
\draw [c] (1.57768,0.686904) -- (1.57768,0.705413);
\draw [c] (1.57768,0.705413) -- (1.57768,0.723923);
\draw [c] (1.56955,0.705413) -- (1.57768,0.705413);
\draw [c] (1.57768,0.705413) -- (1.58582,0.705413);
\definecolor{c}{rgb}{0,0,0};
\colorlet{c}{natcomp!70};
\draw [c] (1.59395,0.686923) -- (1.59395,0.696817);
\draw [c] (1.59395,0.696817) -- (1.59395,0.706711);
\draw [c] (1.58582,0.696817) -- (1.59395,0.696817);
\draw [c] (1.59395,0.696817) -- (1.60209,0.696817);
\definecolor{c}{rgb}{0,0,0};
\colorlet{c}{natcomp!70};
\draw [c] (1.61023,0.686911) -- (1.61023,0.719865);
\draw [c] (1.61023,0.719865) -- (1.61023,0.752819);
\draw [c] (1.60209,0.719865) -- (1.61023,0.719865);
\draw [c] (1.61023,0.719865) -- (1.61836,0.719865);
\definecolor{c}{rgb}{0,0,0};
\colorlet{c}{natcomp!70};
\draw [c] (1.6265,0.692256) -- (1.6265,0.705147);
\draw [c] (1.6265,0.705147) -- (1.6265,0.718038);
\draw [c] (1.61836,0.705147) -- (1.6265,0.705147);
\draw [c] (1.6265,0.705147) -- (1.63464,0.705147);
\definecolor{c}{rgb}{0,0,0};
\colorlet{c}{natcomp!70};
\draw [c] (1.64277,0.686912) -- (1.64277,0.686932);
\draw [c] (1.64277,0.686932) -- (1.64277,0.686952);
\draw [c] (1.63464,0.686932) -- (1.64277,0.686932);
\draw [c] (1.64277,0.686932) -- (1.65091,0.686932);
\definecolor{c}{rgb}{0,0,0};
\colorlet{c}{natcomp!70};
\draw [c] (1.65905,0.692192) -- (1.65905,0.704906);
\draw [c] (1.65905,0.704906) -- (1.65905,0.717619);
\draw [c] (1.65091,0.704906) -- (1.65905,0.704906);
\draw [c] (1.65905,0.704906) -- (1.66718,0.704906);
\definecolor{c}{rgb}{0,0,0};
\colorlet{c}{natcomp!70};
\draw [c] (1.67532,0.700693) -- (1.67532,0.723056);
\draw [c] (1.67532,0.723056) -- (1.67532,0.74542);
\draw [c] (1.66718,0.723056) -- (1.67532,0.723056);
\draw [c] (1.67532,0.723056) -- (1.68345,0.723056);
\definecolor{c}{rgb}{0,0,0};
\colorlet{c}{natcomp!70};
\draw [c] (1.69159,0.686942) -- (1.69159,0.708221);
\draw [c] (1.69159,0.708221) -- (1.69159,0.729499);
\draw [c] (1.68345,0.708221) -- (1.69159,0.708221);
\draw [c] (1.69159,0.708221) -- (1.69973,0.708221);
\definecolor{c}{rgb}{0,0,0};
\colorlet{c}{natcomp!70};
\draw [c] (1.70786,0.705683) -- (1.70786,0.724842);
\draw [c] (1.70786,0.724842) -- (1.70786,0.744);
\draw [c] (1.69973,0.724842) -- (1.70786,0.724842);
\draw [c] (1.70786,0.724842) -- (1.716,0.724842);
\definecolor{c}{rgb}{0,0,0};
\colorlet{c}{natcomp!70};
\draw [c] (1.72414,0.686939) -- (1.72414,0.700941);
\draw [c] (1.72414,0.700941) -- (1.72414,0.714943);
\draw [c] (1.716,0.700941) -- (1.72414,0.700941);
\draw [c] (1.72414,0.700941) -- (1.73227,0.700941);
\definecolor{c}{rgb}{0,0,0};
\colorlet{c}{natcomp!70};
\draw [c] (1.74041,0.686936) -- (1.74041,0.708215);
\draw [c] (1.74041,0.708215) -- (1.74041,0.729493);
\draw [c] (1.73227,0.708215) -- (1.74041,0.708215);
\draw [c] (1.74041,0.708215) -- (1.74855,0.708215);
\definecolor{c}{rgb}{0,0,0};
\colorlet{c}{natcomp!70};
\draw [c] (1.75668,0.686903) -- (1.75668,0.696018);
\draw [c] (1.75668,0.696018) -- (1.75668,0.705133);
\draw [c] (1.74855,0.696018) -- (1.75668,0.696018);
\draw [c] (1.75668,0.696018) -- (1.76482,0.696018);
\definecolor{c}{rgb}{0,0,0};
\colorlet{c}{natcomp!70};
\draw [c] (1.77295,0.714667) -- (1.77295,0.738024);
\draw [c] (1.77295,0.738024) -- (1.77295,0.761381);
\draw [c] (1.76482,0.738024) -- (1.77295,0.738024);
\draw [c] (1.77295,0.738024) -- (1.78109,0.738024);
\definecolor{c}{rgb}{0,0,0};
\colorlet{c}{natcomp!70};
\draw [c] (1.78923,0.713535) -- (1.78923,0.735176);
\draw [c] (1.78923,0.735176) -- (1.78923,0.756817);
\draw [c] (1.78109,0.735176) -- (1.78923,0.735176);
\draw [c] (1.78923,0.735176) -- (1.79736,0.735176);
\definecolor{c}{rgb}{0,0,0};
\colorlet{c}{natcomp!70};
\draw [c] (1.8055,0.706706) -- (1.8055,0.727668);
\draw [c] (1.8055,0.727668) -- (1.8055,0.748629);
\draw [c] (1.79736,0.727668) -- (1.8055,0.727668);
\draw [c] (1.8055,0.727668) -- (1.81364,0.727668);
\definecolor{c}{rgb}{0,0,0};
\colorlet{c}{natcomp!70};
\draw [c] (1.82177,0.71253) -- (1.82177,0.7417);
\draw [c] (1.82177,0.7417) -- (1.82177,0.77087);
\draw [c] (1.81364,0.7417) -- (1.82177,0.7417);
\draw [c] (1.82177,0.7417) -- (1.82991,0.7417);
\definecolor{c}{rgb}{0,0,0};
\colorlet{c}{natcomp!70};
\draw [c] (1.83805,0.686939) -- (1.83805,0.698172);
\draw [c] (1.83805,0.698172) -- (1.83805,0.709404);
\draw [c] (1.82991,0.698172) -- (1.83805,0.698172);
\draw [c] (1.83805,0.698172) -- (1.84618,0.698172);
\definecolor{c}{rgb}{0,0,0};
\colorlet{c}{natcomp!70};
\draw [c] (1.85432,0.693926) -- (1.85432,0.714558);
\draw [c] (1.85432,0.714558) -- (1.85432,0.73519);
\draw [c] (1.84618,0.714558) -- (1.85432,0.714558);
\draw [c] (1.85432,0.714558) -- (1.86245,0.714558);
\definecolor{c}{rgb}{0,0,0};
\colorlet{c}{natcomp!70};
\draw [c] (1.87059,0.702794) -- (1.87059,0.726598);
\draw [c] (1.87059,0.726598) -- (1.87059,0.750403);
\draw [c] (1.86245,0.726598) -- (1.87059,0.726598);
\draw [c] (1.87059,0.726598) -- (1.87873,0.726598);
\definecolor{c}{rgb}{0,0,0};
\colorlet{c}{natcomp!70};
\draw [c] (1.88686,0.723513) -- (1.88686,0.749566);
\draw [c] (1.88686,0.749566) -- (1.88686,0.775619);
\draw [c] (1.87873,0.749566) -- (1.88686,0.749566);
\draw [c] (1.88686,0.749566) -- (1.895,0.749566);
\definecolor{c}{rgb}{0,0,0};
\colorlet{c}{natcomp!70};
\draw [c] (1.90314,0.703651) -- (1.90314,0.728586);
\draw [c] (1.90314,0.728586) -- (1.90314,0.753521);
\draw [c] (1.895,0.728586) -- (1.90314,0.728586);
\draw [c] (1.90314,0.728586) -- (1.91127,0.728586);
\definecolor{c}{rgb}{0,0,0};
\colorlet{c}{natcomp!70};
\draw [c] (1.91941,0.735552) -- (1.91941,0.762435);
\draw [c] (1.91941,0.762435) -- (1.91941,0.789317);
\draw [c] (1.91127,0.762435) -- (1.91941,0.762435);
\draw [c] (1.91941,0.762435) -- (1.92755,0.762435);
\definecolor{c}{rgb}{0,0,0};
\colorlet{c}{natcomp!70};
\draw [c] (1.93568,0.705159) -- (1.93568,0.723496);
\draw [c] (1.93568,0.723496) -- (1.93568,0.741834);
\draw [c] (1.92755,0.723496) -- (1.93568,0.723496);
\draw [c] (1.93568,0.723496) -- (1.94382,0.723496);
\definecolor{c}{rgb}{0,0,0};
\colorlet{c}{natcomp!70};
\draw [c] (1.95195,0.728073) -- (1.95195,0.759278);
\draw [c] (1.95195,0.759278) -- (1.95195,0.790483);
\draw [c] (1.94382,0.759278) -- (1.95195,0.759278);
\draw [c] (1.95195,0.759278) -- (1.96009,0.759278);
\definecolor{c}{rgb}{0,0,0};
\colorlet{c}{natcomp!70};
\draw [c] (1.96823,0.721308) -- (1.96823,0.745166);
\draw [c] (1.96823,0.745166) -- (1.96823,0.769024);
\draw [c] (1.96009,0.745166) -- (1.96823,0.745166);
\draw [c] (1.96823,0.745166) -- (1.97636,0.745166);
\definecolor{c}{rgb}{0,0,0};
\colorlet{c}{natcomp!70};
\draw [c] (1.9845,0.717672) -- (1.9845,0.74667);
\draw [c] (1.9845,0.74667) -- (1.9845,0.775667);
\draw [c] (1.97636,0.74667) -- (1.9845,0.74667);
\draw [c] (1.9845,0.74667) -- (1.99264,0.74667);
\definecolor{c}{rgb}{0,0,0};
\colorlet{c}{natcomp!70};
\draw [c] (2.00077,0.754545) -- (2.00077,0.791474);
\draw [c] (2.00077,0.791474) -- (2.00077,0.828402);
\draw [c] (1.99264,0.791474) -- (2.00077,0.791474);
\draw [c] (2.00077,0.791474) -- (2.00891,0.791474);
\definecolor{c}{rgb}{0,0,0};
\colorlet{c}{natcomp!70};
\draw [c] (2.01705,0.812339) -- (2.01705,0.861065);
\draw [c] (2.01705,0.861065) -- (2.01705,0.909792);
\draw [c] (2.00891,0.861065) -- (2.01705,0.861065);
\draw [c] (2.01705,0.861065) -- (2.02518,0.861065);
\definecolor{c}{rgb}{0,0,0};
\colorlet{c}{natcomp!70};
\draw [c] (2.03332,0.756859) -- (2.03332,0.788069);
\draw [c] (2.03332,0.788069) -- (2.03332,0.819279);
\draw [c] (2.02518,0.788069) -- (2.03332,0.788069);
\draw [c] (2.03332,0.788069) -- (2.04145,0.788069);
\definecolor{c}{rgb}{0,0,0};
\colorlet{c}{natcomp!70};
\draw [c] (2.04959,0.733143) -- (2.04959,0.772578);
\draw [c] (2.04959,0.772578) -- (2.04959,0.812014);
\draw [c] (2.04145,0.772578) -- (2.04959,0.772578);
\draw [c] (2.04959,0.772578) -- (2.05773,0.772578);
\definecolor{c}{rgb}{0,0,0};
\colorlet{c}{natcomp!70};
\draw [c] (2.06586,0.734846) -- (2.06586,0.767565);
\draw [c] (2.06586,0.767565) -- (2.06586,0.800284);
\draw [c] (2.05773,0.767565) -- (2.06586,0.767565);
\draw [c] (2.06586,0.767565) -- (2.074,0.767565);
\definecolor{c}{rgb}{0,0,0};
\colorlet{c}{natcomp!70};
\draw [c] (2.08214,0.789195) -- (2.08214,0.827149);
\draw [c] (2.08214,0.827149) -- (2.08214,0.865104);
\draw [c] (2.074,0.827149) -- (2.08214,0.827149);
\draw [c] (2.08214,0.827149) -- (2.09027,0.827149);
\definecolor{c}{rgb}{0,0,0};
\colorlet{c}{natcomp!70};
\draw [c] (2.09841,0.788521) -- (2.09841,0.828396);
\draw [c] (2.09841,0.828396) -- (2.09841,0.868271);
\draw [c] (2.09027,0.828396) -- (2.09841,0.828396);
\draw [c] (2.09841,0.828396) -- (2.10655,0.828396);
\definecolor{c}{rgb}{0,0,0};
\colorlet{c}{natcomp!70};
\draw [c] (2.11468,0.759161) -- (2.11468,0.799535);
\draw [c] (2.11468,0.799535) -- (2.11468,0.839908);
\draw [c] (2.10655,0.799535) -- (2.11468,0.799535);
\draw [c] (2.11468,0.799535) -- (2.12282,0.799535);
\definecolor{c}{rgb}{0,0,0};
\colorlet{c}{natcomp!70};
\draw [c] (2.13095,0.7679) -- (2.13095,0.805702);
\draw [c] (2.13095,0.805702) -- (2.13095,0.843503);
\draw [c] (2.12282,0.805702) -- (2.13095,0.805702);
\draw [c] (2.13095,0.805702) -- (2.13909,0.805702);
\definecolor{c}{rgb}{0,0,0};
\colorlet{c}{natcomp!70};
\draw [c] (2.14723,0.746911) -- (2.14723,0.783535);
\draw [c] (2.14723,0.783535) -- (2.14723,0.820159);
\draw [c] (2.13909,0.783535) -- (2.14723,0.783535);
\draw [c] (2.14723,0.783535) -- (2.15536,0.783535);
\definecolor{c}{rgb}{0,0,0};
\colorlet{c}{natcomp!70};
\draw [c] (2.1635,0.772802) -- (2.1635,0.808728);
\draw [c] (2.1635,0.808728) -- (2.1635,0.844654);
\draw [c] (2.15536,0.808728) -- (2.1635,0.808728);
\draw [c] (2.1635,0.808728) -- (2.17164,0.808728);
\definecolor{c}{rgb}{0,0,0};
\colorlet{c}{natcomp!70};
\draw [c] (2.17977,0.810552) -- (2.17977,0.849296);
\draw [c] (2.17977,0.849296) -- (2.17977,0.888039);
\draw [c] (2.17164,0.849296) -- (2.17977,0.849296);
\draw [c] (2.17977,0.849296) -- (2.18791,0.849296);
\definecolor{c}{rgb}{0,0,0};
\colorlet{c}{natcomp!70};
\draw [c] (2.19605,0.778252) -- (2.19605,0.813555);
\draw [c] (2.19605,0.813555) -- (2.19605,0.848858);
\draw [c] (2.18791,0.813555) -- (2.19605,0.813555);
\draw [c] (2.19605,0.813555) -- (2.20418,0.813555);
\definecolor{c}{rgb}{0,0,0};
\colorlet{c}{natcomp!70};
\draw [c] (2.21232,0.789468) -- (2.21232,0.827907);
\draw [c] (2.21232,0.827907) -- (2.21232,0.866346);
\draw [c] (2.20418,0.827907) -- (2.21232,0.827907);
\draw [c] (2.21232,0.827907) -- (2.22045,0.827907);
\definecolor{c}{rgb}{0,0,0};
\colorlet{c}{natcomp!70};
\draw [c] (2.22859,0.815059) -- (2.22859,0.859924);
\draw [c] (2.22859,0.859924) -- (2.22859,0.904788);
\draw [c] (2.22045,0.859924) -- (2.22859,0.859924);
\draw [c] (2.22859,0.859924) -- (2.23673,0.859924);
\definecolor{c}{rgb}{0,0,0};
\colorlet{c}{natcomp!70};
\draw [c] (2.24486,0.802412) -- (2.24486,0.839797);
\draw [c] (2.24486,0.839797) -- (2.24486,0.877183);
\draw [c] (2.23673,0.839797) -- (2.24486,0.839797);
\draw [c] (2.24486,0.839797) -- (2.253,0.839797);
\definecolor{c}{rgb}{0,0,0};
\colorlet{c}{natcomp!70};
\draw [c] (2.26114,0.840272) -- (2.26114,0.885266);
\draw [c] (2.26114,0.885266) -- (2.26114,0.93026);
\draw [c] (2.253,0.885266) -- (2.26114,0.885266);
\draw [c] (2.26114,0.885266) -- (2.26927,0.885266);
\definecolor{c}{rgb}{0,0,0};
\colorlet{c}{natcomp!70};
\draw [c] (2.27741,0.805072) -- (2.27741,0.848826);
\draw [c] (2.27741,0.848826) -- (2.27741,0.892579);
\draw [c] (2.26927,0.848826) -- (2.27741,0.848826);
\draw [c] (2.27741,0.848826) -- (2.28555,0.848826);
\definecolor{c}{rgb}{0,0,0};
\colorlet{c}{natcomp!70};
\draw [c] (2.29368,0.791504) -- (2.29368,0.826715);
\draw [c] (2.29368,0.826715) -- (2.29368,0.861925);
\draw [c] (2.28555,0.826715) -- (2.29368,0.826715);
\draw [c] (2.29368,0.826715) -- (2.30182,0.826715);
\definecolor{c}{rgb}{0,0,0};
\colorlet{c}{natcomp!70};
\draw [c] (2.30995,0.856209) -- (2.30995,0.906613);
\draw [c] (2.30995,0.906613) -- (2.30995,0.957017);
\draw [c] (2.30182,0.906613) -- (2.30995,0.906613);
\draw [c] (2.30995,0.906613) -- (2.31809,0.906613);
\definecolor{c}{rgb}{0,0,0};
\colorlet{c}{natcomp!70};
\draw [c] (2.32623,0.828277) -- (2.32623,0.869435);
\draw [c] (2.32623,0.869435) -- (2.32623,0.910592);
\draw [c] (2.31809,0.869435) -- (2.32623,0.869435);
\draw [c] (2.32623,0.869435) -- (2.33436,0.869435);
\definecolor{c}{rgb}{0,0,0};
\colorlet{c}{natcomp!70};
\draw [c] (2.3425,0.88625) -- (2.3425,0.939801);
\draw [c] (2.3425,0.939801) -- (2.3425,0.993352);
\draw [c] (2.33436,0.939801) -- (2.3425,0.939801);
\draw [c] (2.3425,0.939801) -- (2.35064,0.939801);
\definecolor{c}{rgb}{0,0,0};
\colorlet{c}{natcomp!70};
\draw [c] (2.35877,0.916275) -- (2.35877,0.969867);
\draw [c] (2.35877,0.969867) -- (2.35877,1.02346);
\draw [c] (2.35064,0.969867) -- (2.35877,0.969867);
\draw [c] (2.35877,0.969867) -- (2.36691,0.969867);
\definecolor{c}{rgb}{0,0,0};
\colorlet{c}{natcomp!70};
\draw [c] (2.37505,0.907296) -- (2.37505,0.962405);
\draw [c] (2.37505,0.962405) -- (2.37505,1.01751);
\draw [c] (2.36691,0.962405) -- (2.37505,0.962405);
\draw [c] (2.37505,0.962405) -- (2.38318,0.962405);
\definecolor{c}{rgb}{0,0,0};
\colorlet{c}{natcomp!70};
\draw [c] (2.39132,0.827878) -- (2.39132,0.86913);
\draw [c] (2.39132,0.86913) -- (2.39132,0.910383);
\draw [c] (2.38318,0.86913) -- (2.39132,0.86913);
\draw [c] (2.39132,0.86913) -- (2.39945,0.86913);
\definecolor{c}{rgb}{0,0,0};
\colorlet{c}{natcomp!70};
\draw [c] (2.40759,0.857178) -- (2.40759,0.910918);
\draw [c] (2.40759,0.910918) -- (2.40759,0.964658);
\draw [c] (2.39945,0.910918) -- (2.40759,0.910918);
\draw [c] (2.40759,0.910918) -- (2.41573,0.910918);
\definecolor{c}{rgb}{0,0,0};
\colorlet{c}{natcomp!70};
\draw [c] (2.42386,0.919688) -- (2.42386,0.974446);
\draw [c] (2.42386,0.974446) -- (2.42386,1.0292);
\draw [c] (2.41573,0.974446) -- (2.42386,0.974446);
\draw [c] (2.42386,0.974446) -- (2.432,0.974446);
\definecolor{c}{rgb}{0,0,0};
\colorlet{c}{natcomp!70};
\draw [c] (2.44014,0.898282) -- (2.44014,0.948641);
\draw [c] (2.44014,0.948641) -- (2.44014,0.999);
\draw [c] (2.432,0.948641) -- (2.44014,0.948641);
\draw [c] (2.44014,0.948641) -- (2.44827,0.948641);
\definecolor{c}{rgb}{0,0,0};
\colorlet{c}{natcomp!70};
\draw [c] (2.45641,0.910429) -- (2.45641,0.968628);
\draw [c] (2.45641,0.968628) -- (2.45641,1.02683);
\draw [c] (2.44827,0.968628) -- (2.45641,0.968628);
\draw [c] (2.45641,0.968628) -- (2.46455,0.968628);
\definecolor{c}{rgb}{0,0,0};
\colorlet{c}{natcomp!70};
\draw [c] (2.47268,0.884339) -- (2.47268,0.931852);
\draw [c] (2.47268,0.931852) -- (2.47268,0.979366);
\draw [c] (2.46455,0.931852) -- (2.47268,0.931852);
\draw [c] (2.47268,0.931852) -- (2.48082,0.931852);
\definecolor{c}{rgb}{0,0,0};
\colorlet{c}{natcomp!70};
\draw [c] (2.48895,0.969757) -- (2.48895,1.0318);
\draw [c] (2.48895,1.0318) -- (2.48895,1.09385);
\draw [c] (2.48082,1.0318) -- (2.48895,1.0318);
\draw [c] (2.48895,1.0318) -- (2.49709,1.0318);
\definecolor{c}{rgb}{0,0,0};
\colorlet{c}{natcomp!70};
\draw [c] (2.50523,0.971435) -- (2.50523,1.03092);
\draw [c] (2.50523,1.03092) -- (2.50523,1.0904);
\draw [c] (2.49709,1.03092) -- (2.50523,1.03092);
\draw [c] (2.50523,1.03092) -- (2.51336,1.03092);
\definecolor{c}{rgb}{0,0,0};
\colorlet{c}{natcomp!70};
\draw [c] (2.5215,0.990909) -- (2.5215,1.05577);
\draw [c] (2.5215,1.05577) -- (2.5215,1.12064);
\draw [c] (2.51336,1.05577) -- (2.5215,1.05577);
\draw [c] (2.5215,1.05577) -- (2.52964,1.05577);
\definecolor{c}{rgb}{0,0,0};
\colorlet{c}{natcomp!70};
\draw [c] (2.53777,1.00227) -- (2.53777,1.06689);
\draw [c] (2.53777,1.06689) -- (2.53777,1.13151);
\draw [c] (2.52964,1.06689) -- (2.53777,1.06689);
\draw [c] (2.53777,1.06689) -- (2.54591,1.06689);
\definecolor{c}{rgb}{0,0,0};
\colorlet{c}{natcomp!70};
\draw [c] (2.55405,1.04669) -- (2.55405,1.11011);
\draw [c] (2.55405,1.11011) -- (2.55405,1.17353);
\draw [c] (2.54591,1.11011) -- (2.55405,1.11011);
\draw [c] (2.55405,1.11011) -- (2.56218,1.11011);
\definecolor{c}{rgb}{0,0,0};
\colorlet{c}{natcomp!70};
\draw [c] (2.57032,0.995602) -- (2.57032,1.06059);
\draw [c] (2.57032,1.06059) -- (2.57032,1.12558);
\draw [c] (2.56218,1.06059) -- (2.57032,1.06059);
\draw [c] (2.57032,1.06059) -- (2.57845,1.06059);
\definecolor{c}{rgb}{0,0,0};
\colorlet{c}{natcomp!70};
\draw [c] (2.58659,0.99475) -- (2.58659,1.05464);
\draw [c] (2.58659,1.05464) -- (2.58659,1.11453);
\draw [c] (2.57845,1.05464) -- (2.58659,1.05464);
\draw [c] (2.58659,1.05464) -- (2.59473,1.05464);
\definecolor{c}{rgb}{0,0,0};
\colorlet{c}{natcomp!70};
\draw [c] (2.60286,1.07289) -- (2.60286,1.14199);
\draw [c] (2.60286,1.14199) -- (2.60286,1.21109);
\draw [c] (2.59473,1.14199) -- (2.60286,1.14199);
\draw [c] (2.60286,1.14199) -- (2.611,1.14199);
\definecolor{c}{rgb}{0,0,0};
\colorlet{c}{natcomp!70};
\draw [c] (2.61914,1.13773) -- (2.61914,1.21239);
\draw [c] (2.61914,1.21239) -- (2.61914,1.28705);
\draw [c] (2.611,1.21239) -- (2.61914,1.21239);
\draw [c] (2.61914,1.21239) -- (2.62727,1.21239);
\definecolor{c}{rgb}{0,0,0};
\colorlet{c}{natcomp!70};
\draw [c] (2.63541,0.996344) -- (2.63541,1.05637);
\draw [c] (2.63541,1.05637) -- (2.63541,1.1164);
\draw [c] (2.62727,1.05637) -- (2.63541,1.05637);
\draw [c] (2.63541,1.05637) -- (2.64355,1.05637);
\definecolor{c}{rgb}{0,0,0};
\colorlet{c}{natcomp!70};
\draw [c] (2.65168,0.988957) -- (2.65168,1.0492);
\draw [c] (2.65168,1.0492) -- (2.65168,1.10944);
\draw [c] (2.64355,1.0492) -- (2.65168,1.0492);
\draw [c] (2.65168,1.0492) -- (2.65982,1.0492);
\definecolor{c}{rgb}{0,0,0};
\colorlet{c}{natcomp!70};
\draw [c] (2.66795,1.1185) -- (2.66795,1.18712);
\draw [c] (2.66795,1.18712) -- (2.66795,1.25574);
\draw [c] (2.65982,1.18712) -- (2.66795,1.18712);
\draw [c] (2.66795,1.18712) -- (2.67609,1.18712);
\definecolor{c}{rgb}{0,0,0};
\colorlet{c}{natcomp!70};
\draw [c] (2.68423,1.11432) -- (2.68423,1.19207);
\draw [c] (2.68423,1.19207) -- (2.68423,1.26981);
\draw [c] (2.67609,1.19207) -- (2.68423,1.19207);
\draw [c] (2.68423,1.19207) -- (2.69236,1.19207);
\definecolor{c}{rgb}{0,0,0};
\colorlet{c}{natcomp!70};
\draw [c] (2.7005,1.21466) -- (2.7005,1.29003);
\draw [c] (2.7005,1.29003) -- (2.7005,1.3654);
\draw [c] (2.69236,1.29003) -- (2.7005,1.29003);
\draw [c] (2.7005,1.29003) -- (2.70864,1.29003);
\definecolor{c}{rgb}{0,0,0};
\colorlet{c}{natcomp!70};
\draw [c] (2.71677,1.10429) -- (2.71677,1.17543);
\draw [c] (2.71677,1.17543) -- (2.71677,1.24658);
\draw [c] (2.70864,1.17543) -- (2.71677,1.17543);
\draw [c] (2.71677,1.17543) -- (2.72491,1.17543);
\definecolor{c}{rgb}{0,0,0};
\colorlet{c}{natcomp!70};
\draw [c] (2.73305,1.12661) -- (2.73305,1.2009);
\draw [c] (2.73305,1.2009) -- (2.73305,1.27519);
\draw [c] (2.72491,1.2009) -- (2.73305,1.2009);
\draw [c] (2.73305,1.2009) -- (2.74118,1.2009);
\definecolor{c}{rgb}{0,0,0};
\colorlet{c}{natcomp!70};
\draw [c] (2.74932,1.06158) -- (2.74932,1.12477);
\draw [c] (2.74932,1.12477) -- (2.74932,1.18796);
\draw [c] (2.74118,1.12477) -- (2.74932,1.12477);
\draw [c] (2.74932,1.12477) -- (2.75745,1.12477);
\definecolor{c}{rgb}{0,0,0};
\colorlet{c}{natcomp!70};
\draw [c] (2.76559,1.18504) -- (2.76559,1.26123);
\draw [c] (2.76559,1.26123) -- (2.76559,1.33741);
\draw [c] (2.75745,1.26123) -- (2.76559,1.26123);
\draw [c] (2.76559,1.26123) -- (2.77373,1.26123);
\definecolor{c}{rgb}{0,0,0};
\colorlet{c}{natcomp!70};
\draw [c] (2.78186,1.23072) -- (2.78186,1.31228);
\draw [c] (2.78186,1.31228) -- (2.78186,1.39384);
\draw [c] (2.77373,1.31228) -- (2.78186,1.31228);
\draw [c] (2.78186,1.31228) -- (2.79,1.31228);
\definecolor{c}{rgb}{0,0,0};
\colorlet{c}{natcomp!70};
\draw [c] (2.79814,1.28934) -- (2.79814,1.37187);
\draw [c] (2.79814,1.37187) -- (2.79814,1.45439);
\draw [c] (2.79,1.37187) -- (2.79814,1.37187);
\draw [c] (2.79814,1.37187) -- (2.80627,1.37187);
\definecolor{c}{rgb}{0,0,0};
\colorlet{c}{natcomp!70};
\draw [c] (2.81441,1.32982) -- (2.81441,1.41625);
\draw [c] (2.81441,1.41625) -- (2.81441,1.50267);
\draw [c] (2.80627,1.41625) -- (2.81441,1.41625);
\draw [c] (2.81441,1.41625) -- (2.82255,1.41625);
\definecolor{c}{rgb}{0,0,0};
\colorlet{c}{natcomp!70};
\draw [c] (2.83068,1.30347) -- (2.83068,1.38681);
\draw [c] (2.83068,1.38681) -- (2.83068,1.47015);
\draw [c] (2.82255,1.38681) -- (2.83068,1.38681);
\draw [c] (2.83068,1.38681) -- (2.83882,1.38681);
\definecolor{c}{rgb}{0,0,0};
\colorlet{c}{natcomp!70};
\draw [c] (2.84695,1.23426) -- (2.84695,1.31216);
\draw [c] (2.84695,1.31216) -- (2.84695,1.39006);
\draw [c] (2.83882,1.31216) -- (2.84695,1.31216);
\draw [c] (2.84695,1.31216) -- (2.85509,1.31216);
\definecolor{c}{rgb}{0,0,0};
\colorlet{c}{natcomp!70};
\draw [c] (2.86323,1.36782) -- (2.86323,1.45635);
\draw [c] (2.86323,1.45635) -- (2.86323,1.54489);
\draw [c] (2.85509,1.45635) -- (2.86323,1.45635);
\draw [c] (2.86323,1.45635) -- (2.87136,1.45635);
\definecolor{c}{rgb}{0,0,0};
\colorlet{c}{natcomp!70};
\draw [c] (2.8795,1.31293) -- (2.8795,1.3965);
\draw [c] (2.8795,1.3965) -- (2.8795,1.48008);
\draw [c] (2.87136,1.3965) -- (2.8795,1.3965);
\draw [c] (2.8795,1.3965) -- (2.88764,1.3965);
\definecolor{c}{rgb}{0,0,0};
\colorlet{c}{natcomp!70};
\draw [c] (2.89577,1.42357) -- (2.89577,1.51525);
\draw [c] (2.89577,1.51525) -- (2.89577,1.60692);
\draw [c] (2.88764,1.51525) -- (2.89577,1.51525);
\draw [c] (2.89577,1.51525) -- (2.90391,1.51525);
\definecolor{c}{rgb}{0,0,0};
\colorlet{c}{natcomp!70};
\draw [c] (2.91205,1.43079) -- (2.91205,1.52154);
\draw [c] (2.91205,1.52154) -- (2.91205,1.61228);
\draw [c] (2.90391,1.52154) -- (2.91205,1.52154);
\draw [c] (2.91205,1.52154) -- (2.92018,1.52154);
\definecolor{c}{rgb}{0,0,0};
\colorlet{c}{natcomp!70};
\draw [c] (2.92832,1.34088) -- (2.92832,1.42855);
\draw [c] (2.92832,1.42855) -- (2.92832,1.51622);
\draw [c] (2.92018,1.42855) -- (2.92832,1.42855);
\draw [c] (2.92832,1.42855) -- (2.93645,1.42855);
\definecolor{c}{rgb}{0,0,0};
\colorlet{c}{natcomp!70};
\draw [c] (2.94459,1.41547) -- (2.94459,1.50915);
\draw [c] (2.94459,1.50915) -- (2.94459,1.60282);
\draw [c] (2.93645,1.50915) -- (2.94459,1.50915);
\draw [c] (2.94459,1.50915) -- (2.95273,1.50915);
\definecolor{c}{rgb}{0,0,0};
\colorlet{c}{natcomp!70};
\draw [c] (2.96086,1.58909) -- (2.96086,1.69352);
\draw [c] (2.96086,1.69352) -- (2.96086,1.79794);
\draw [c] (2.95273,1.69352) -- (2.96086,1.69352);
\draw [c] (2.96086,1.69352) -- (2.969,1.69352);
\definecolor{c}{rgb}{0,0,0};
\colorlet{c}{natcomp!70};
\draw [c] (2.97714,1.33395) -- (2.97714,1.41959);
\draw [c] (2.97714,1.41959) -- (2.97714,1.50522);
\draw [c] (2.969,1.41959) -- (2.97714,1.41959);
\draw [c] (2.97714,1.41959) -- (2.98527,1.41959);
\definecolor{c}{rgb}{0,0,0};
\colorlet{c}{natcomp!70};
\draw [c] (2.99341,1.59874) -- (2.99341,1.69978);
\draw [c] (2.99341,1.69978) -- (2.99341,1.80083);
\draw [c] (2.98527,1.69978) -- (2.99341,1.69978);
\draw [c] (2.99341,1.69978) -- (3.00155,1.69978);
\definecolor{c}{rgb}{0,0,0};
\colorlet{c}{natcomp!70};
\draw [c] (3.00968,1.52556) -- (3.00968,1.61936);
\draw [c] (3.00968,1.61936) -- (3.00968,1.71316);
\draw [c] (3.00155,1.61936) -- (3.00968,1.61936);
\draw [c] (3.00968,1.61936) -- (3.01782,1.61936);
\definecolor{c}{rgb}{0,0,0};
\colorlet{c}{natcomp!70};
\draw [c] (3.02595,1.6018) -- (3.02595,1.70136);
\draw [c] (3.02595,1.70136) -- (3.02595,1.80091);
\draw [c] (3.01782,1.70136) -- (3.02595,1.70136);
\draw [c] (3.02595,1.70136) -- (3.03409,1.70136);
\definecolor{c}{rgb}{0,0,0};
\colorlet{c}{natcomp!70};
\draw [c] (3.04223,1.58016) -- (3.04223,1.68072);
\draw [c] (3.04223,1.68072) -- (3.04223,1.78129);
\draw [c] (3.03409,1.68072) -- (3.04223,1.68072);
\draw [c] (3.04223,1.68072) -- (3.05036,1.68072);
\definecolor{c}{rgb}{0,0,0};
\colorlet{c}{natcomp!70};
\draw [c] (3.0585,1.72343) -- (3.0585,1.82892);
\draw [c] (3.0585,1.82892) -- (3.0585,1.93442);
\draw [c] (3.05036,1.82892) -- (3.0585,1.82892);
\draw [c] (3.0585,1.82892) -- (3.06664,1.82892);
\definecolor{c}{rgb}{0,0,0};
\colorlet{c}{natcomp!70};
\draw [c] (3.07477,1.80323) -- (3.07477,1.92123);
\draw [c] (3.07477,1.92123) -- (3.07477,2.03923);
\draw [c] (3.06664,1.92123) -- (3.07477,1.92123);
\draw [c] (3.07477,1.92123) -- (3.08291,1.92123);
\definecolor{c}{rgb}{0,0,0};
\colorlet{c}{natcomp!70};
\draw [c] (3.09105,1.72774) -- (3.09105,1.8397);
\draw [c] (3.09105,1.8397) -- (3.09105,1.95166);
\draw [c] (3.08291,1.8397) -- (3.09105,1.8397);
\draw [c] (3.09105,1.8397) -- (3.09918,1.8397);
\definecolor{c}{rgb}{0,0,0};
\colorlet{c}{natcomp!70};
\draw [c] (3.10732,1.5984) -- (3.10732,1.70303);
\draw [c] (3.10732,1.70303) -- (3.10732,1.80766);
\draw [c] (3.09918,1.70303) -- (3.10732,1.70303);
\draw [c] (3.10732,1.70303) -- (3.11545,1.70303);
\definecolor{c}{rgb}{0,0,0};
\colorlet{c}{natcomp!70};
\draw [c] (3.12359,1.72127) -- (3.12359,1.83007);
\draw [c] (3.12359,1.83007) -- (3.12359,1.93888);
\draw [c] (3.11545,1.83007) -- (3.12359,1.83007);
\draw [c] (3.12359,1.83007) -- (3.13173,1.83007);
\definecolor{c}{rgb}{0,0,0};
\colorlet{c}{natcomp!70};
\draw [c] (3.13986,1.64464) -- (3.13986,1.74758);
\draw [c] (3.13986,1.74758) -- (3.13986,1.85053);
\draw [c] (3.13173,1.74758) -- (3.13986,1.74758);
\draw [c] (3.13986,1.74758) -- (3.148,1.74758);
\definecolor{c}{rgb}{0,0,0};
\colorlet{c}{natcomp!70};
\draw [c] (3.15614,1.69808) -- (3.15614,1.80701);
\draw [c] (3.15614,1.80701) -- (3.15614,1.91594);
\draw [c] (3.148,1.80701) -- (3.15614,1.80701);
\draw [c] (3.15614,1.80701) -- (3.16427,1.80701);
\definecolor{c}{rgb}{0,0,0};
\colorlet{c}{natcomp!70};
\draw [c] (3.17241,1.85831) -- (3.17241,1.98447);
\draw [c] (3.17241,1.98447) -- (3.17241,2.11062);
\draw [c] (3.16427,1.98447) -- (3.17241,1.98447);
\draw [c] (3.17241,1.98447) -- (3.18055,1.98447);
\definecolor{c}{rgb}{0,0,0};
\colorlet{c}{natcomp!70};
\draw [c] (3.18868,1.75745) -- (3.18868,1.87281);
\draw [c] (3.18868,1.87281) -- (3.18868,1.98818);
\draw [c] (3.18055,1.87281) -- (3.18868,1.87281);
\draw [c] (3.18868,1.87281) -- (3.19682,1.87281);
\definecolor{c}{rgb}{0,0,0};
\colorlet{c}{natcomp!70};
\draw [c] (3.20495,1.88058) -- (3.20495,2.00281);
\draw [c] (3.20495,2.00281) -- (3.20495,2.12505);
\draw [c] (3.19682,2.00281) -- (3.20495,2.00281);
\draw [c] (3.20495,2.00281) -- (3.21309,2.00281);
\definecolor{c}{rgb}{0,0,0};
\colorlet{c}{natcomp!70};
\draw [c] (3.22123,1.80611) -- (3.22123,1.92753);
\draw [c] (3.22123,1.92753) -- (3.22123,2.04895);
\draw [c] (3.21309,1.92753) -- (3.22123,1.92753);
\draw [c] (3.22123,1.92753) -- (3.22936,1.92753);
\definecolor{c}{rgb}{0,0,0};
\colorlet{c}{natcomp!70};
\draw [c] (3.2375,1.91345) -- (3.2375,2.03616);
\draw [c] (3.2375,2.03616) -- (3.2375,2.15888);
\draw [c] (3.22936,2.03616) -- (3.2375,2.03616);
\draw [c] (3.2375,2.03616) -- (3.24564,2.03616);
\definecolor{c}{rgb}{0,0,0};
\colorlet{c}{natcomp!70};
\draw [c] (3.25377,1.96375) -- (3.25377,2.09043);
\draw [c] (3.25377,2.09043) -- (3.25377,2.21712);
\draw [c] (3.24564,2.09043) -- (3.25377,2.09043);
\draw [c] (3.25377,2.09043) -- (3.26191,2.09043);
\definecolor{c}{rgb}{0,0,0};
\colorlet{c}{natcomp!70};
\draw [c] (3.27005,1.92288) -- (3.27005,2.04725);
\draw [c] (3.27005,2.04725) -- (3.27005,2.17162);
\draw [c] (3.26191,2.04725) -- (3.27005,2.04725);
\draw [c] (3.27005,2.04725) -- (3.27818,2.04725);
\definecolor{c}{rgb}{0,0,0};
\colorlet{c}{natcomp!70};
\draw [c] (3.28632,1.84932) -- (3.28632,1.97407);
\draw [c] (3.28632,1.97407) -- (3.28632,2.09882);
\draw [c] (3.27818,1.97407) -- (3.28632,1.97407);
\draw [c] (3.28632,1.97407) -- (3.29445,1.97407);
\definecolor{c}{rgb}{0,0,0};
\colorlet{c}{natcomp!70};
\draw [c] (3.30259,1.90077) -- (3.30259,2.02007);
\draw [c] (3.30259,2.02007) -- (3.30259,2.13937);
\draw [c] (3.29445,2.02007) -- (3.30259,2.02007);
\draw [c] (3.30259,2.02007) -- (3.31073,2.02007);
\definecolor{c}{rgb}{0,0,0};
\colorlet{c}{natcomp!70};
\draw [c] (3.31886,1.83797) -- (3.31886,1.96045);
\draw [c] (3.31886,1.96045) -- (3.31886,2.08293);
\draw [c] (3.31073,1.96045) -- (3.31886,1.96045);
\draw [c] (3.31886,1.96045) -- (3.327,1.96045);
\definecolor{c}{rgb}{0,0,0};
\colorlet{c}{natcomp!70};
\draw [c] (3.33514,1.88986) -- (3.33514,2.01298);
\draw [c] (3.33514,2.01298) -- (3.33514,2.1361);
\draw [c] (3.327,2.01298) -- (3.33514,2.01298);
\draw [c] (3.33514,2.01298) -- (3.34327,2.01298);
\definecolor{c}{rgb}{0,0,0};
\colorlet{c}{natcomp!70};
\draw [c] (3.35141,1.97995) -- (3.35141,2.10928);
\draw [c] (3.35141,2.10928) -- (3.35141,2.2386);
\draw [c] (3.34327,2.10928) -- (3.35141,2.10928);
\draw [c] (3.35141,2.10928) -- (3.35955,2.10928);
\definecolor{c}{rgb}{0,0,0};
\colorlet{c}{natcomp!70};
\draw [c] (3.36768,1.69146) -- (3.36768,1.80413);
\draw [c] (3.36768,1.80413) -- (3.36768,1.9168);
\draw [c] (3.35955,1.80413) -- (3.36768,1.80413);
\draw [c] (3.36768,1.80413) -- (3.37582,1.80413);
\definecolor{c}{rgb}{0,0,0};
\colorlet{c}{natcomp!70};
\draw [c] (3.38395,1.76713) -- (3.38395,1.8867);
\draw [c] (3.38395,1.8867) -- (3.38395,2.00627);
\draw [c] (3.37582,1.8867) -- (3.38395,1.8867);
\draw [c] (3.38395,1.8867) -- (3.39209,1.8867);
\definecolor{c}{rgb}{0,0,0};
\colorlet{c}{natcomp!70};
\draw [c] (3.40023,1.73703) -- (3.40023,1.85405);
\draw [c] (3.40023,1.85405) -- (3.40023,1.97106);
\draw [c] (3.39209,1.85405) -- (3.40023,1.85405);
\draw [c] (3.40023,1.85405) -- (3.40836,1.85405);
\definecolor{c}{rgb}{0,0,0};
\colorlet{c}{natcomp!70};
\draw [c] (3.4165,1.75427) -- (3.4165,1.87418);
\draw [c] (3.4165,1.87418) -- (3.4165,1.99409);
\draw [c] (3.40836,1.87418) -- (3.4165,1.87418);
\draw [c] (3.4165,1.87418) -- (3.42464,1.87418);
\definecolor{c}{rgb}{0,0,0};
\colorlet{c}{natcomp!70};
\draw [c] (3.43277,1.59071) -- (3.43277,1.70168);
\draw [c] (3.43277,1.70168) -- (3.43277,1.81264);
\draw [c] (3.42464,1.70168) -- (3.43277,1.70168);
\draw [c] (3.43277,1.70168) -- (3.44091,1.70168);
\definecolor{c}{rgb}{0,0,0};
\colorlet{c}{natcomp!70};
\draw [c] (3.44905,2.0116) -- (3.44905,2.1488);
\draw [c] (3.44905,2.1488) -- (3.44905,2.286);
\draw [c] (3.44091,2.1488) -- (3.44905,2.1488);
\draw [c] (3.44905,2.1488) -- (3.45718,2.1488);
\definecolor{c}{rgb}{0,0,0};
\colorlet{c}{natcomp!70};
\draw [c] (3.46532,1.58379) -- (3.46532,1.69731);
\draw [c] (3.46532,1.69731) -- (3.46532,1.81082);
\draw [c] (3.45718,1.69731) -- (3.46532,1.69731);
\draw [c] (3.46532,1.69731) -- (3.47345,1.69731);
\definecolor{c}{rgb}{0,0,0};
\colorlet{c}{natcomp!70};
\draw [c] (3.48159,1.69179) -- (3.48159,1.8072);
\draw [c] (3.48159,1.8072) -- (3.48159,1.92261);
\draw [c] (3.47345,1.8072) -- (3.48159,1.8072);
\draw [c] (3.48159,1.8072) -- (3.48973,1.8072);
\definecolor{c}{rgb}{0,0,0};
\colorlet{c}{natcomp!70};
\draw [c] (3.49786,1.66795) -- (3.49786,1.7778);
\draw [c] (3.49786,1.7778) -- (3.49786,1.88766);
\draw [c] (3.48973,1.7778) -- (3.49786,1.7778);
\draw [c] (3.49786,1.7778) -- (3.506,1.7778);
\definecolor{c}{rgb}{0,0,0};
\colorlet{c}{natcomp!70};
\draw [c] (3.51414,1.54378) -- (3.51414,1.65414);
\draw [c] (3.51414,1.65414) -- (3.51414,1.76451);
\draw [c] (3.506,1.65414) -- (3.51414,1.65414);
\draw [c] (3.51414,1.65414) -- (3.52227,1.65414);
\definecolor{c}{rgb}{0,0,0};
\colorlet{c}{natcomp!70};
\draw [c] (3.53041,1.63429) -- (3.53041,1.75469);
\draw [c] (3.53041,1.75469) -- (3.53041,1.87509);
\draw [c] (3.52227,1.75469) -- (3.53041,1.75469);
\draw [c] (3.53041,1.75469) -- (3.53855,1.75469);
\definecolor{c}{rgb}{0,0,0};
\colorlet{c}{natcomp!70};
\draw [c] (3.54668,1.49617) -- (3.54668,1.60081);
\draw [c] (3.54668,1.60081) -- (3.54668,1.70545);
\draw [c] (3.53855,1.60081) -- (3.54668,1.60081);
\draw [c] (3.54668,1.60081) -- (3.55482,1.60081);
\definecolor{c}{rgb}{0,0,0};
\colorlet{c}{natcomp!70};
\draw [c] (3.56295,1.58314) -- (3.56295,1.69229);
\draw [c] (3.56295,1.69229) -- (3.56295,1.80144);
\draw [c] (3.55482,1.69229) -- (3.56295,1.69229);
\draw [c] (3.56295,1.69229) -- (3.57109,1.69229);
\definecolor{c}{rgb}{0,0,0};
\colorlet{c}{natcomp!70};
\draw [c] (3.57923,1.46417) -- (3.57923,1.56462);
\draw [c] (3.57923,1.56462) -- (3.57923,1.66507);
\draw [c] (3.57109,1.56462) -- (3.57923,1.56462);
\draw [c] (3.57923,1.56462) -- (3.58736,1.56462);
\definecolor{c}{rgb}{0,0,0};
\colorlet{c}{natcomp!70};
\draw [c] (3.5955,1.7506) -- (3.5955,1.87231);
\draw [c] (3.5955,1.87231) -- (3.5955,1.99402);
\draw [c] (3.58736,1.87231) -- (3.5955,1.87231);
\draw [c] (3.5955,1.87231) -- (3.60364,1.87231);
\definecolor{c}{rgb}{0,0,0};
\colorlet{c}{natcomp!70};
\draw [c] (3.61177,1.53631) -- (3.61177,1.63577);
\draw [c] (3.61177,1.63577) -- (3.61177,1.73524);
\draw [c] (3.60364,1.63577) -- (3.61177,1.63577);
\draw [c] (3.61177,1.63577) -- (3.61991,1.63577);
\definecolor{c}{rgb}{0,0,0};
\colorlet{c}{natcomp!70};
\draw [c] (3.62805,1.60831) -- (3.62805,1.71397);
\draw [c] (3.62805,1.71397) -- (3.62805,1.81962);
\draw [c] (3.61991,1.71397) -- (3.62805,1.71397);
\draw [c] (3.62805,1.71397) -- (3.63618,1.71397);
\definecolor{c}{rgb}{0,0,0};
\colorlet{c}{natcomp!70};
\draw [c] (3.64432,1.41401) -- (3.64432,1.5139);
\draw [c] (3.64432,1.5139) -- (3.64432,1.61378);
\draw [c] (3.63618,1.5139) -- (3.64432,1.5139);
\draw [c] (3.64432,1.5139) -- (3.65245,1.5139);
\definecolor{c}{rgb}{0,0,0};
\colorlet{c}{natcomp!70};
\draw [c] (3.66059,1.47867) -- (3.66059,1.57869);
\draw [c] (3.66059,1.57869) -- (3.66059,1.67871);
\draw [c] (3.65245,1.57869) -- (3.66059,1.57869);
\draw [c] (3.66059,1.57869) -- (3.66873,1.57869);
\definecolor{c}{rgb}{0,0,0};
\colorlet{c}{natcomp!70};
\draw [c] (3.67686,1.43725) -- (3.67686,1.53155);
\draw [c] (3.67686,1.53155) -- (3.67686,1.62584);
\draw [c] (3.66873,1.53155) -- (3.67686,1.53155);
\draw [c] (3.67686,1.53155) -- (3.685,1.53155);
\definecolor{c}{rgb}{0,0,0};
\colorlet{c}{natcomp!70};
\draw [c] (3.69314,1.44595) -- (3.69314,1.54196);
\draw [c] (3.69314,1.54196) -- (3.69314,1.63798);
\draw [c] (3.685,1.54196) -- (3.69314,1.54196);
\draw [c] (3.69314,1.54196) -- (3.70127,1.54196);
\definecolor{c}{rgb}{0,0,0};
\colorlet{c}{natcomp!70};
\draw [c] (3.70941,1.31331) -- (3.70941,1.4008);
\draw [c] (3.70941,1.4008) -- (3.70941,1.48828);
\draw [c] (3.70127,1.4008) -- (3.70941,1.4008);
\draw [c] (3.70941,1.4008) -- (3.71755,1.4008);
\definecolor{c}{rgb}{0,0,0};
\colorlet{c}{natcomp!70};
\draw [c] (3.72568,1.38718) -- (3.72568,1.47882);
\draw [c] (3.72568,1.47882) -- (3.72568,1.57046);
\draw [c] (3.71755,1.47882) -- (3.72568,1.47882);
\draw [c] (3.72568,1.47882) -- (3.73382,1.47882);
\definecolor{c}{rgb}{0,0,0};
\colorlet{c}{natcomp!70};
\draw [c] (3.74195,1.30543) -- (3.74195,1.39264);
\draw [c] (3.74195,1.39264) -- (3.74195,1.47984);
\draw [c] (3.73382,1.39264) -- (3.74195,1.39264);
\draw [c] (3.74195,1.39264) -- (3.75009,1.39264);
\definecolor{c}{rgb}{0,0,0};
\colorlet{c}{natcomp!70};
\draw [c] (3.75823,1.33609) -- (3.75823,1.42593);
\draw [c] (3.75823,1.42593) -- (3.75823,1.51577);
\draw [c] (3.75009,1.42593) -- (3.75823,1.42593);
\draw [c] (3.75823,1.42593) -- (3.76636,1.42593);
\definecolor{c}{rgb}{0,0,0};
\colorlet{c}{natcomp!70};
\draw [c] (3.7745,1.48113) -- (3.7745,1.57802);
\draw [c] (3.7745,1.57802) -- (3.7745,1.67491);
\draw [c] (3.76636,1.57802) -- (3.7745,1.57802);
\draw [c] (3.7745,1.57802) -- (3.78264,1.57802);
\definecolor{c}{rgb}{0,0,0};
\colorlet{c}{natcomp!70};
\draw [c] (3.79077,1.36985) -- (3.79077,1.4665);
\draw [c] (3.79077,1.4665) -- (3.79077,1.56316);
\draw [c] (3.78264,1.4665) -- (3.79077,1.4665);
\draw [c] (3.79077,1.4665) -- (3.79891,1.4665);
\definecolor{c}{rgb}{0,0,0};
\colorlet{c}{natcomp!70};
\draw [c] (3.80705,1.3613) -- (3.80705,1.45155);
\draw [c] (3.80705,1.45155) -- (3.80705,1.5418);
\draw [c] (3.79891,1.45155) -- (3.80705,1.45155);
\draw [c] (3.80705,1.45155) -- (3.81518,1.45155);
\definecolor{c}{rgb}{0,0,0};
\colorlet{c}{natcomp!70};
\draw [c] (3.82332,1.34548) -- (3.82332,1.43349);
\draw [c] (3.82332,1.43349) -- (3.82332,1.5215);
\draw [c] (3.81518,1.43349) -- (3.82332,1.43349);
\draw [c] (3.82332,1.43349) -- (3.83145,1.43349);
\definecolor{c}{rgb}{0,0,0};
\colorlet{c}{natcomp!70};
\draw [c] (3.83959,1.34648) -- (3.83959,1.44227);
\draw [c] (3.83959,1.44227) -- (3.83959,1.53807);
\draw [c] (3.83145,1.44227) -- (3.83959,1.44227);
\draw [c] (3.83959,1.44227) -- (3.84773,1.44227);
\definecolor{c}{rgb}{0,0,0};
\colorlet{c}{natcomp!70};
\draw [c] (3.85586,1.26603) -- (3.85586,1.34858);
\draw [c] (3.85586,1.34858) -- (3.85586,1.43114);
\draw [c] (3.84773,1.34858) -- (3.85586,1.34858);
\draw [c] (3.85586,1.34858) -- (3.864,1.34858);
\definecolor{c}{rgb}{0,0,0};
\colorlet{c}{natcomp!70};
\draw [c] (3.87214,1.19945) -- (3.87214,1.28167);
\draw [c] (3.87214,1.28167) -- (3.87214,1.3639);
\draw [c] (3.864,1.28167) -- (3.87214,1.28167);
\draw [c] (3.87214,1.28167) -- (3.88027,1.28167);
\definecolor{c}{rgb}{0,0,0};
\colorlet{c}{natcomp!70};
\draw [c] (3.88841,1.37139) -- (3.88841,1.46647);
\draw [c] (3.88841,1.46647) -- (3.88841,1.56155);
\draw [c] (3.88027,1.46647) -- (3.88841,1.46647);
\draw [c] (3.88841,1.46647) -- (3.89655,1.46647);
\definecolor{c}{rgb}{0,0,0};
\colorlet{c}{natcomp!70};
\draw [c] (3.90468,1.0781) -- (3.90468,1.14498);
\draw [c] (3.90468,1.14498) -- (3.90468,1.21187);
\draw [c] (3.89655,1.14498) -- (3.90468,1.14498);
\draw [c] (3.90468,1.14498) -- (3.91282,1.14498);
\definecolor{c}{rgb}{0,0,0};
\colorlet{c}{natcomp!70};
\draw [c] (3.92095,1.12409) -- (3.92095,1.19694);
\draw [c] (3.92095,1.19694) -- (3.92095,1.26979);
\draw [c] (3.91282,1.19694) -- (3.92095,1.19694);
\draw [c] (3.92095,1.19694) -- (3.92909,1.19694);
\definecolor{c}{rgb}{0,0,0};
\colorlet{c}{natcomp!70};
\draw [c] (3.93723,1.21279) -- (3.93723,1.29838);
\draw [c] (3.93723,1.29838) -- (3.93723,1.38397);
\draw [c] (3.92909,1.29838) -- (3.93723,1.29838);
\draw [c] (3.93723,1.29838) -- (3.94536,1.29838);
\definecolor{c}{rgb}{0,0,0};
\colorlet{c}{natcomp!70};
\draw [c] (3.9535,1.18838) -- (3.9535,1.26484);
\draw [c] (3.9535,1.26484) -- (3.9535,1.3413);
\draw [c] (3.94536,1.26484) -- (3.9535,1.26484);
\draw [c] (3.9535,1.26484) -- (3.96164,1.26484);
\definecolor{c}{rgb}{0,0,0};
\colorlet{c}{natcomp!70};
\draw [c] (3.96977,1.13244) -- (3.96977,1.20537);
\draw [c] (3.96977,1.20537) -- (3.96977,1.2783);
\draw [c] (3.96164,1.20537) -- (3.96977,1.20537);
\draw [c] (3.96977,1.20537) -- (3.97791,1.20537);
\definecolor{c}{rgb}{0,0,0};
\colorlet{c}{natcomp!70};
\draw [c] (3.98605,1.18644) -- (3.98605,1.26695);
\draw [c] (3.98605,1.26695) -- (3.98605,1.34746);
\draw [c] (3.97791,1.26695) -- (3.98605,1.26695);
\draw [c] (3.98605,1.26695) -- (3.99418,1.26695);
\definecolor{c}{rgb}{0,0,0};
\colorlet{c}{natcomp!70};
\draw [c] (4.00232,1.18905) -- (4.00232,1.27326);
\draw [c] (4.00232,1.27326) -- (4.00232,1.35748);
\draw [c] (3.99418,1.27326) -- (4.00232,1.27326);
\draw [c] (4.00232,1.27326) -- (4.01045,1.27326);
\definecolor{c}{rgb}{0,0,0};
\colorlet{c}{natcomp!70};
\draw [c] (4.01859,1.21515) -- (4.01859,1.29654);
\draw [c] (4.01859,1.29654) -- (4.01859,1.37792);
\draw [c] (4.01045,1.29654) -- (4.01859,1.29654);
\draw [c] (4.01859,1.29654) -- (4.02673,1.29654);
\definecolor{c}{rgb}{0,0,0};
\colorlet{c}{natcomp!70};
\draw [c] (4.03486,1.0567) -- (4.03486,1.12126);
\draw [c] (4.03486,1.12126) -- (4.03486,1.18582);
\draw [c] (4.02673,1.12126) -- (4.03486,1.12126);
\draw [c] (4.03486,1.12126) -- (4.043,1.12126);
\definecolor{c}{rgb}{0,0,0};
\colorlet{c}{natcomp!70};
\draw [c] (4.05114,1.0637) -- (4.05114,1.13565);
\draw [c] (4.05114,1.13565) -- (4.05114,1.2076);
\draw [c] (4.043,1.13565) -- (4.05114,1.13565);
\draw [c] (4.05114,1.13565) -- (4.05927,1.13565);
\definecolor{c}{rgb}{0,0,0};
\colorlet{c}{natcomp!70};
\draw [c] (4.06741,1.12543) -- (4.06741,1.19947);
\draw [c] (4.06741,1.19947) -- (4.06741,1.27351);
\draw [c] (4.05927,1.19947) -- (4.06741,1.19947);
\draw [c] (4.06741,1.19947) -- (4.07555,1.19947);
\definecolor{c}{rgb}{0,0,0};
\colorlet{c}{natcomp!70};
\draw [c] (4.08368,1.11089) -- (4.08368,1.18647);
\draw [c] (4.08368,1.18647) -- (4.08368,1.26206);
\draw [c] (4.07555,1.18647) -- (4.08368,1.18647);
\draw [c] (4.08368,1.18647) -- (4.09182,1.18647);
\definecolor{c}{rgb}{0,0,0};
\colorlet{c}{natcomp!70};
\draw [c] (4.09995,1.09755) -- (4.09995,1.16945);
\draw [c] (4.09995,1.16945) -- (4.09995,1.24135);
\draw [c] (4.09182,1.16945) -- (4.09995,1.16945);
\draw [c] (4.09995,1.16945) -- (4.10809,1.16945);
\definecolor{c}{rgb}{0,0,0};
\colorlet{c}{natcomp!70};
\draw [c] (4.11623,1.07487) -- (4.11623,1.14657);
\draw [c] (4.11623,1.14657) -- (4.11623,1.21826);
\draw [c] (4.10809,1.14657) -- (4.11623,1.14657);
\draw [c] (4.11623,1.14657) -- (4.12436,1.14657);
\definecolor{c}{rgb}{0,0,0};
\colorlet{c}{natcomp!70};
\draw [c] (4.1325,1.03624) -- (4.1325,1.10277);
\draw [c] (4.1325,1.10277) -- (4.1325,1.16931);
\draw [c] (4.12436,1.10277) -- (4.1325,1.10277);
\draw [c] (4.1325,1.10277) -- (4.14064,1.10277);
\definecolor{c}{rgb}{0,0,0};
\colorlet{c}{natcomp!70};
\draw [c] (4.14877,0.952458) -- (4.14877,1.00761);
\draw [c] (4.14877,1.00761) -- (4.14877,1.06276);
\draw [c] (4.14064,1.00761) -- (4.14877,1.00761);
\draw [c] (4.14877,1.00761) -- (4.15691,1.00761);
\definecolor{c}{rgb}{0,0,0};
\colorlet{c}{natcomp!70};
\draw [c] (4.16505,1.04998) -- (4.16505,1.12012);
\draw [c] (4.16505,1.12012) -- (4.16505,1.19027);
\draw [c] (4.15691,1.12012) -- (4.16505,1.12012);
\draw [c] (4.16505,1.12012) -- (4.17318,1.12012);
\definecolor{c}{rgb}{0,0,0};
\colorlet{c}{natcomp!70};
\draw [c] (4.18132,1.08985) -- (4.18132,1.16335);
\draw [c] (4.18132,1.16335) -- (4.18132,1.23686);
\draw [c] (4.17318,1.16335) -- (4.18132,1.16335);
\draw [c] (4.18132,1.16335) -- (4.18945,1.16335);
\definecolor{c}{rgb}{0,0,0};
\colorlet{c}{natcomp!70};
\draw [c] (4.19759,1.06558) -- (4.19759,1.13256);
\draw [c] (4.19759,1.13256) -- (4.19759,1.19953);
\draw [c] (4.18945,1.13256) -- (4.19759,1.13256);
\draw [c] (4.19759,1.13256) -- (4.20573,1.13256);
\definecolor{c}{rgb}{0,0,0};
\colorlet{c}{natcomp!70};
\draw [c] (4.21386,0.99441) -- (4.21386,1.05948);
\draw [c] (4.21386,1.05948) -- (4.21386,1.12455);
\draw [c] (4.20573,1.05948) -- (4.21386,1.05948);
\draw [c] (4.21386,1.05948) -- (4.222,1.05948);
\definecolor{c}{rgb}{0,0,0};
\colorlet{c}{natcomp!70};
\draw [c] (4.23014,1.15782) -- (4.23014,1.23917);
\draw [c] (4.23014,1.23917) -- (4.23014,1.32052);
\draw [c] (4.222,1.23917) -- (4.23014,1.23917);
\draw [c] (4.23014,1.23917) -- (4.23827,1.23917);
\definecolor{c}{rgb}{0,0,0};
\colorlet{c}{natcomp!70};
\draw [c] (4.24641,0.983316) -- (4.24641,1.04763);
\draw [c] (4.24641,1.04763) -- (4.24641,1.11194);
\draw [c] (4.23827,1.04763) -- (4.24641,1.04763);
\draw [c] (4.24641,1.04763) -- (4.25455,1.04763);
\definecolor{c}{rgb}{0,0,0};
\colorlet{c}{natcomp!70};
\draw [c] (4.26268,1.03462) -- (4.26268,1.10754);
\draw [c] (4.26268,1.10754) -- (4.26268,1.18046);
\draw [c] (4.25455,1.10754) -- (4.26268,1.10754);
\draw [c] (4.26268,1.10754) -- (4.27082,1.10754);
\definecolor{c}{rgb}{0,0,0};
\colorlet{c}{natcomp!70};
\draw [c] (4.27895,0.980957) -- (4.27895,1.04168);
\draw [c] (4.27895,1.04168) -- (4.27895,1.1024);
\draw [c] (4.27082,1.04168) -- (4.27895,1.04168);
\draw [c] (4.27895,1.04168) -- (4.28709,1.04168);
\definecolor{c}{rgb}{0,0,0};
\colorlet{c}{natcomp!70};
\draw [c] (4.29523,0.98657) -- (4.29523,1.04335);
\draw [c] (4.29523,1.04335) -- (4.29523,1.10013);
\draw [c] (4.28709,1.04335) -- (4.29523,1.04335);
\draw [c] (4.29523,1.04335) -- (4.30336,1.04335);
\definecolor{c}{rgb}{0,0,0};
\colorlet{c}{natcomp!70};
\draw [c] (4.3115,0.987687) -- (4.3115,1.0472);
\draw [c] (4.3115,1.0472) -- (4.3115,1.10672);
\draw [c] (4.30336,1.0472) -- (4.3115,1.0472);
\draw [c] (4.3115,1.0472) -- (4.31964,1.0472);
\definecolor{c}{rgb}{0,0,0};
\colorlet{c}{natcomp!70};
\draw [c] (4.32777,1.01895) -- (4.32777,1.08324);
\draw [c] (4.32777,1.08324) -- (4.32777,1.14754);
\draw [c] (4.31964,1.08324) -- (4.32777,1.08324);
\draw [c] (4.32777,1.08324) -- (4.33591,1.08324);
\definecolor{c}{rgb}{0,0,0};
\colorlet{c}{natcomp!70};
\draw [c] (4.34405,0.968473) -- (4.34405,1.02884);
\draw [c] (4.34405,1.02884) -- (4.34405,1.08921);
\draw [c] (4.33591,1.02884) -- (4.34405,1.02884);
\draw [c] (4.34405,1.02884) -- (4.35218,1.02884);
\definecolor{c}{rgb}{0,0,0};
\colorlet{c}{natcomp!70};
\draw [c] (4.36032,0.974112) -- (4.36032,1.03613);
\draw [c] (4.36032,1.03613) -- (4.36032,1.09814);
\draw [c] (4.35218,1.03613) -- (4.36032,1.03613);
\draw [c] (4.36032,1.03613) -- (4.36845,1.03613);
\definecolor{c}{rgb}{0,0,0};
\colorlet{c}{natcomp!70};
\draw [c] (4.37659,0.980809) -- (4.37659,1.04153);
\draw [c] (4.37659,1.04153) -- (4.37659,1.10224);
\draw [c] (4.36845,1.04153) -- (4.37659,1.04153);
\draw [c] (4.37659,1.04153) -- (4.38473,1.04153);
\definecolor{c}{rgb}{0,0,0};
\colorlet{c}{natcomp!70};
\draw [c] (4.39286,0.971029) -- (4.39286,1.02981);
\draw [c] (4.39286,1.02981) -- (4.39286,1.08859);
\draw [c] (4.38473,1.02981) -- (4.39286,1.02981);
\draw [c] (4.39286,1.02981) -- (4.401,1.02981);
\definecolor{c}{rgb}{0,0,0};
\colorlet{c}{natcomp!70};
\draw [c] (4.40914,0.95256) -- (4.40914,1.00793);
\draw [c] (4.40914,1.00793) -- (4.40914,1.06331);
\draw [c] (4.401,1.00793) -- (4.40914,1.00793);
\draw [c] (4.40914,1.00793) -- (4.41727,1.00793);
\definecolor{c}{rgb}{0,0,0};
\colorlet{c}{natcomp!70};
\draw [c] (4.42541,1.04433) -- (4.42541,1.1097);
\draw [c] (4.42541,1.1097) -- (4.42541,1.17507);
\draw [c] (4.41727,1.1097) -- (4.42541,1.1097);
\draw [c] (4.42541,1.1097) -- (4.43355,1.1097);
\definecolor{c}{rgb}{0,0,0};
\colorlet{c}{natcomp!70};
\draw [c] (4.44168,0.921556) -- (4.44168,0.976194);
\draw [c] (4.44168,0.976194) -- (4.44168,1.03083);
\draw [c] (4.43355,0.976194) -- (4.44168,0.976194);
\draw [c] (4.44168,0.976194) -- (4.44982,0.976194);
\definecolor{c}{rgb}{0,0,0};
\colorlet{c}{natcomp!70};
\draw [c] (4.45795,0.972593) -- (4.45795,1.04013);
\draw [c] (4.45795,1.04013) -- (4.45795,1.10767);
\draw [c] (4.44982,1.04013) -- (4.45795,1.04013);
\draw [c] (4.45795,1.04013) -- (4.46609,1.04013);
\definecolor{c}{rgb}{0,0,0};
\colorlet{c}{natcomp!70};
\draw [c] (4.47423,0.946413) -- (4.47423,0.999826);
\draw [c] (4.47423,0.999826) -- (4.47423,1.05324);
\draw [c] (4.46609,0.999826) -- (4.47423,0.999826);
\draw [c] (4.47423,0.999826) -- (4.48236,0.999826);
\definecolor{c}{rgb}{0,0,0};
\colorlet{c}{natcomp!70};
\draw [c] (4.4905,0.928871) -- (4.4905,0.983045);
\draw [c] (4.4905,0.983045) -- (4.4905,1.03722);
\draw [c] (4.48236,0.983045) -- (4.4905,0.983045);
\draw [c] (4.4905,0.983045) -- (4.49864,0.983045);
\definecolor{c}{rgb}{0,0,0};
\colorlet{c}{natcomp!70};
\draw [c] (4.50677,0.92616) -- (4.50677,0.979906);
\draw [c] (4.50677,0.979906) -- (4.50677,1.03365);
\draw [c] (4.49864,0.979906) -- (4.50677,0.979906);
\draw [c] (4.50677,0.979906) -- (4.51491,0.979906);
\definecolor{c}{rgb}{0,0,0};
\colorlet{c}{natcomp!70};
\draw [c] (4.52305,0.866944) -- (4.52305,0.914198);
\draw [c] (4.52305,0.914198) -- (4.52305,0.961452);
\draw [c] (4.51491,0.914198) -- (4.52305,0.914198);
\draw [c] (4.52305,0.914198) -- (4.53118,0.914198);
\definecolor{c}{rgb}{0,0,0};
\colorlet{c}{natcomp!70};
\draw [c] (4.53932,0.956535) -- (4.53932,1.01368);
\draw [c] (4.53932,1.01368) -- (4.53932,1.07083);
\draw [c] (4.53118,1.01368) -- (4.53932,1.01368);
\draw [c] (4.53932,1.01368) -- (4.54745,1.01368);
\definecolor{c}{rgb}{0,0,0};
\colorlet{c}{natcomp!70};
\draw [c] (4.55559,1.00276) -- (4.55559,1.06462);
\draw [c] (4.55559,1.06462) -- (4.55559,1.12648);
\draw [c] (4.54745,1.06462) -- (4.55559,1.06462);
\draw [c] (4.55559,1.06462) -- (4.56373,1.06462);
\definecolor{c}{rgb}{0,0,0};
\colorlet{c}{natcomp!70};
\draw [c] (4.57186,0.919644) -- (4.57186,0.985824);
\draw [c] (4.57186,0.985824) -- (4.57186,1.052);
\draw [c] (4.56373,0.985824) -- (4.57186,0.985824);
\draw [c] (4.57186,0.985824) -- (4.58,0.985824);
\definecolor{c}{rgb}{0,0,0};
\colorlet{c}{natcomp!70};
\draw [c] (4.58814,0.882513) -- (4.58814,0.934061);
\draw [c] (4.58814,0.934061) -- (4.58814,0.985609);
\draw [c] (4.58,0.934061) -- (4.58814,0.934061);
\draw [c] (4.58814,0.934061) -- (4.59627,0.934061);
\definecolor{c}{rgb}{0,0,0};
\colorlet{c}{natcomp!70};
\draw [c] (4.60441,0.912457) -- (4.60441,0.967236);
\draw [c] (4.60441,0.967236) -- (4.60441,1.02202);
\draw [c] (4.59627,0.967236) -- (4.60441,0.967236);
\draw [c] (4.60441,0.967236) -- (4.61255,0.967236);
\definecolor{c}{rgb}{0,0,0};
\colorlet{c}{natcomp!70};
\draw [c] (4.62068,0.914155) -- (4.62068,0.969166);
\draw [c] (4.62068,0.969166) -- (4.62068,1.02418);
\draw [c] (4.61255,0.969166) -- (4.62068,0.969166);
\draw [c] (4.62068,0.969166) -- (4.62882,0.969166);
\definecolor{c}{rgb}{0,0,0};
\colorlet{c}{natcomp!70};
\draw [c] (4.63695,0.879878) -- (4.63695,0.938137);
\draw [c] (4.63695,0.938137) -- (4.63695,0.996396);
\draw [c] (4.62882,0.938137) -- (4.63695,0.938137);
\draw [c] (4.63695,0.938137) -- (4.64509,0.938137);
\definecolor{c}{rgb}{0,0,0};
\colorlet{c}{natcomp!70};
\draw [c] (4.65323,0.899783) -- (4.65323,0.956467);
\draw [c] (4.65323,0.956467) -- (4.65323,1.01315);
\draw [c] (4.64509,0.956467) -- (4.65323,0.956467);
\draw [c] (4.65323,0.956467) -- (4.66136,0.956467);
\definecolor{c}{rgb}{0,0,0};
\colorlet{c}{natcomp!70};
\draw [c] (4.6695,0.884772) -- (4.6695,0.936176);
\draw [c] (4.6695,0.936176) -- (4.6695,0.987579);
\draw [c] (4.66136,0.936176) -- (4.6695,0.936176);
\draw [c] (4.6695,0.936176) -- (4.67764,0.936176);
\definecolor{c}{rgb}{0,0,0};
\colorlet{c}{natcomp!70};
\draw [c] (4.68577,0.985752) -- (4.68577,1.05675);
\draw [c] (4.68577,1.05675) -- (4.68577,1.12775);
\draw [c] (4.67764,1.05675) -- (4.68577,1.05675);
\draw [c] (4.68577,1.05675) -- (4.69391,1.05675);
\definecolor{c}{rgb}{0,0,0};
\colorlet{c}{natcomp!70};
\draw [c] (4.70205,0.888065) -- (4.70205,0.939341);
\draw [c] (4.70205,0.939341) -- (4.70205,0.990618);
\draw [c] (4.69391,0.939341) -- (4.70205,0.939341);
\draw [c] (4.70205,0.939341) -- (4.71018,0.939341);
\definecolor{c}{rgb}{0,0,0};
\colorlet{c}{natcomp!70};
\draw [c] (4.71832,0.950968) -- (4.71832,1.01404);
\draw [c] (4.71832,1.01404) -- (4.71832,1.0771);
\draw [c] (4.71018,1.01404) -- (4.71832,1.01404);
\draw [c] (4.71832,1.01404) -- (4.72645,1.01404);
\definecolor{c}{rgb}{0,0,0};
\colorlet{c}{natcomp!70};
\draw [c] (4.73459,0.890061) -- (4.73459,0.949548);
\draw [c] (4.73459,0.949548) -- (4.73459,1.00904);
\draw [c] (4.72645,0.949548) -- (4.73459,0.949548);
\draw [c] (4.73459,0.949548) -- (4.74273,0.949548);
\definecolor{c}{rgb}{0,0,0};
\colorlet{c}{natcomp!70};
\draw [c] (4.75086,0.903141) -- (4.75086,0.959014);
\draw [c] (4.75086,0.959014) -- (4.75086,1.01489);
\draw [c] (4.74273,0.959014) -- (4.75086,0.959014);
\draw [c] (4.75086,0.959014) -- (4.759,0.959014);
\definecolor{c}{rgb}{0,0,0};
\colorlet{c}{natcomp!70};
\draw [c] (4.76714,0.878125) -- (4.76714,0.929474);
\draw [c] (4.76714,0.929474) -- (4.76714,0.980823);
\draw [c] (4.759,0.929474) -- (4.76714,0.929474);
\draw [c] (4.76714,0.929474) -- (4.77527,0.929474);
\definecolor{c}{rgb}{0,0,0};
\colorlet{c}{natcomp!70};
\draw [c] (4.78341,0.831227) -- (4.78341,0.880248);
\draw [c] (4.78341,0.880248) -- (4.78341,0.929269);
\draw [c] (4.77527,0.880248) -- (4.78341,0.880248);
\draw [c] (4.78341,0.880248) -- (4.79155,0.880248);
\definecolor{c}{rgb}{0,0,0};
\colorlet{c}{natcomp!70};
\draw [c] (4.79968,0.864439) -- (4.79968,0.917743);
\draw [c] (4.79968,0.917743) -- (4.79968,0.971047);
\draw [c] (4.79155,0.917743) -- (4.79968,0.917743);
\draw [c] (4.79968,0.917743) -- (4.80782,0.917743);
\definecolor{c}{rgb}{0,0,0};
\colorlet{c}{natcomp!70};
\draw [c] (4.81595,0.852991) -- (4.81595,0.902588);
\draw [c] (4.81595,0.902588) -- (4.81595,0.952185);
\draw [c] (4.80782,0.902588) -- (4.81595,0.902588);
\draw [c] (4.81595,0.902588) -- (4.82409,0.902588);
\definecolor{c}{rgb}{0,0,0};
\colorlet{c}{natcomp!70};
\draw [c] (4.83223,0.838268) -- (4.83223,0.879525);
\draw [c] (4.83223,0.879525) -- (4.83223,0.920782);
\draw [c] (4.82409,0.879525) -- (4.83223,0.879525);
\draw [c] (4.83223,0.879525) -- (4.84036,0.879525);
\definecolor{c}{rgb}{0,0,0};
\colorlet{c}{natcomp!70};
\draw [c] (4.8485,0.849755) -- (4.8485,0.901917);
\draw [c] (4.8485,0.901917) -- (4.8485,0.954078);
\draw [c] (4.84036,0.901917) -- (4.8485,0.901917);
\draw [c] (4.8485,0.901917) -- (4.85664,0.901917);
\definecolor{c}{rgb}{0,0,0};
\colorlet{c}{natcomp!70};
\draw [c] (4.86477,0.88973) -- (4.86477,0.942964);
\draw [c] (4.86477,0.942964) -- (4.86477,0.996198);
\draw [c] (4.85664,0.942964) -- (4.86477,0.942964);
\draw [c] (4.86477,0.942964) -- (4.87291,0.942964);
\definecolor{c}{rgb}{0,0,0};
\colorlet{c}{natcomp!70};
\draw [c] (4.88105,0.852361) -- (4.88105,0.898715);
\draw [c] (4.88105,0.898715) -- (4.88105,0.945068);
\draw [c] (4.87291,0.898715) -- (4.88105,0.898715);
\draw [c] (4.88105,0.898715) -- (4.88918,0.898715);
\definecolor{c}{rgb}{0,0,0};
\colorlet{c}{natcomp!70};
\draw [c] (4.89732,0.889338) -- (4.89732,0.94623);
\draw [c] (4.89732,0.94623) -- (4.89732,1.00312);
\draw [c] (4.88918,0.94623) -- (4.89732,0.94623);
\draw [c] (4.89732,0.94623) -- (4.90545,0.94623);
\definecolor{c}{rgb}{0,0,0};
\colorlet{c}{natcomp!70};
\draw [c] (4.91359,0.818285) -- (4.91359,0.857852);
\draw [c] (4.91359,0.857852) -- (4.91359,0.897419);
\draw [c] (4.90545,0.857852) -- (4.91359,0.857852);
\draw [c] (4.91359,0.857852) -- (4.92173,0.857852);
\definecolor{c}{rgb}{0,0,0};
\colorlet{c}{natcomp!70};
\draw [c] (4.92986,0.839442) -- (4.92986,0.884558);
\draw [c] (4.92986,0.884558) -- (4.92986,0.929674);
\draw [c] (4.92173,0.884558) -- (4.92986,0.884558);
\draw [c] (4.92986,0.884558) -- (4.938,0.884558);
\definecolor{c}{rgb}{0,0,0};
\colorlet{c}{natcomp!70};
\draw [c] (4.94614,0.928982) -- (4.94614,0.983751);
\draw [c] (4.94614,0.983751) -- (4.94614,1.03852);
\draw [c] (4.938,0.983751) -- (4.94614,0.983751);
\draw [c] (4.94614,0.983751) -- (4.95427,0.983751);
\definecolor{c}{rgb}{0,0,0};
\colorlet{c}{natcomp!70};
\draw [c] (4.96241,0.756623) -- (4.96241,0.793018);
\draw [c] (4.96241,0.793018) -- (4.96241,0.829412);
\draw [c] (4.95427,0.793018) -- (4.96241,0.793018);
\draw [c] (4.96241,0.793018) -- (4.97055,0.793018);
\definecolor{c}{rgb}{0,0,0};
\colorlet{c}{natcomp!70};
\draw [c] (4.97868,0.863301) -- (4.97868,0.913408);
\draw [c] (4.97868,0.913408) -- (4.97868,0.963516);
\draw [c] (4.97055,0.913408) -- (4.97868,0.913408);
\draw [c] (4.97868,0.913408) -- (4.98682,0.913408);
\definecolor{c}{rgb}{0,0,0};
\colorlet{c}{natcomp!70};
\draw [c] (4.99495,0.820068) -- (4.99495,0.859989);
\draw [c] (4.99495,0.859989) -- (4.99495,0.89991);
\draw [c] (4.98682,0.859989) -- (4.99495,0.859989);
\draw [c] (4.99495,0.859989) -- (5.00309,0.859989);
\definecolor{c}{rgb}{0,0,0};
\colorlet{c}{natcomp!70};
\draw [c] (5.01123,0.781416) -- (5.01123,0.816202);
\draw [c] (5.01123,0.816202) -- (5.01123,0.850987);
\draw [c] (5.00309,0.816202) -- (5.01123,0.816202);
\draw [c] (5.01123,0.816202) -- (5.01936,0.816202);
\definecolor{c}{rgb}{0,0,0};
\colorlet{c}{natcomp!70};
\draw [c] (5.0275,0.790764) -- (5.0275,0.827697);
\draw [c] (5.0275,0.827697) -- (5.0275,0.86463);
\draw [c] (5.01936,0.827697) -- (5.0275,0.827697);
\draw [c] (5.0275,0.827697) -- (5.03564,0.827697);
\definecolor{c}{rgb}{0,0,0};
\colorlet{c}{natcomp!70};
\draw [c] (5.04377,0.750942) -- (5.04377,0.784172);
\draw [c] (5.04377,0.784172) -- (5.04377,0.817401);
\draw [c] (5.03564,0.784172) -- (5.04377,0.784172);
\draw [c] (5.04377,0.784172) -- (5.05191,0.784172);
\definecolor{c}{rgb}{0,0,0};
\colorlet{c}{natcomp!70};
\draw [c] (5.06005,0.881823) -- (5.06005,0.934976);
\draw [c] (5.06005,0.934976) -- (5.06005,0.988128);
\draw [c] (5.05191,0.934976) -- (5.06005,0.934976);
\draw [c] (5.06005,0.934976) -- (5.06818,0.934976);
\definecolor{c}{rgb}{0,0,0};
\colorlet{c}{natcomp!70};
\draw [c] (5.07632,0.879455) -- (5.07632,0.936439);
\draw [c] (5.07632,0.936439) -- (5.07632,0.993423);
\draw [c] (5.06818,0.936439) -- (5.07632,0.936439);
\draw [c] (5.07632,0.936439) -- (5.08445,0.936439);
\definecolor{c}{rgb}{0,0,0};
\colorlet{c}{natcomp!70};
\draw [c] (5.09259,0.765327) -- (5.09259,0.799544);
\draw [c] (5.09259,0.799544) -- (5.09259,0.833762);
\draw [c] (5.08445,0.799544) -- (5.09259,0.799544);
\draw [c] (5.09259,0.799544) -- (5.10073,0.799544);
\definecolor{c}{rgb}{0,0,0};
\colorlet{c}{natcomp!70};
\draw [c] (5.10886,0.816846) -- (5.10886,0.854773);
\draw [c] (5.10886,0.854773) -- (5.10886,0.8927);
\draw [c] (5.10073,0.854773) -- (5.10886,0.854773);
\draw [c] (5.10886,0.854773) -- (5.117,0.854773);
\definecolor{c}{rgb}{0,0,0};
\colorlet{c}{natcomp!70};
\draw [c] (5.12514,0.878289) -- (5.12514,0.930852);
\draw [c] (5.12514,0.930852) -- (5.12514,0.983415);
\draw [c] (5.117,0.930852) -- (5.12514,0.930852);
\draw [c] (5.12514,0.930852) -- (5.13327,0.930852);
\definecolor{c}{rgb}{0,0,0};
\colorlet{c}{natcomp!70};
\draw [c] (5.14141,0.886117) -- (5.14141,0.93627);
\draw [c] (5.14141,0.93627) -- (5.14141,0.986423);
\draw [c] (5.13327,0.93627) -- (5.14141,0.93627);
\draw [c] (5.14141,0.93627) -- (5.14955,0.93627);
\definecolor{c}{rgb}{0,0,0};
\colorlet{c}{natcomp!70};
\draw [c] (5.15768,0.756901) -- (5.15768,0.7883);
\draw [c] (5.15768,0.7883) -- (5.15768,0.819698);
\draw [c] (5.14955,0.7883) -- (5.15768,0.7883);
\draw [c] (5.15768,0.7883) -- (5.16582,0.7883);
\definecolor{c}{rgb}{0,0,0};
\colorlet{c}{natcomp!70};
\draw [c] (5.17395,0.850388) -- (5.17395,0.903041);
\draw [c] (5.17395,0.903041) -- (5.17395,0.955693);
\draw [c] (5.16582,0.903041) -- (5.17395,0.903041);
\draw [c] (5.17395,0.903041) -- (5.18209,0.903041);
\definecolor{c}{rgb}{0,0,0};
\colorlet{c}{natcomp!70};
\draw [c] (5.19023,0.837397) -- (5.19023,0.880272);
\draw [c] (5.19023,0.880272) -- (5.19023,0.923147);
\draw [c] (5.18209,0.880272) -- (5.19023,0.880272);
\draw [c] (5.19023,0.880272) -- (5.19836,0.880272);
\definecolor{c}{rgb}{0,0,0};
\colorlet{c}{natcomp!70};
\draw [c] (5.2065,0.782891) -- (5.2065,0.819807);
\draw [c] (5.2065,0.819807) -- (5.2065,0.856723);
\draw [c] (5.19836,0.819807) -- (5.2065,0.819807);
\draw [c] (5.2065,0.819807) -- (5.21464,0.819807);
\definecolor{c}{rgb}{0,0,0};
\colorlet{c}{natcomp!70};
\draw [c] (5.22277,0.877003) -- (5.22277,0.925132);
\draw [c] (5.22277,0.925132) -- (5.22277,0.973261);
\draw [c] (5.21464,0.925132) -- (5.22277,0.925132);
\draw [c] (5.22277,0.925132) -- (5.23091,0.925132);
\definecolor{c}{rgb}{0,0,0};
\colorlet{c}{natcomp!70};
\draw [c] (5.23905,0.832507) -- (5.23905,0.877913);
\draw [c] (5.23905,0.877913) -- (5.23905,0.923319);
\draw [c] (5.23091,0.877913) -- (5.23905,0.877913);
\draw [c] (5.23905,0.877913) -- (5.24718,0.877913);
\definecolor{c}{rgb}{0,0,0};
\colorlet{c}{natcomp!70};
\draw [c] (5.25532,0.809327) -- (5.25532,0.853797);
\draw [c] (5.25532,0.853797) -- (5.25532,0.898268);
\draw [c] (5.24718,0.853797) -- (5.25532,0.853797);
\draw [c] (5.25532,0.853797) -- (5.26345,0.853797);
\definecolor{c}{rgb}{0,0,0};
\colorlet{c}{natcomp!70};
\draw [c] (5.27159,0.838939) -- (5.27159,0.884592);
\draw [c] (5.27159,0.884592) -- (5.27159,0.930245);
\draw [c] (5.26345,0.884592) -- (5.27159,0.884592);
\draw [c] (5.27159,0.884592) -- (5.27973,0.884592);
\definecolor{c}{rgb}{0,0,0};
\colorlet{c}{natcomp!70};
\draw [c] (5.28786,0.774997) -- (5.28786,0.815419);
\draw [c] (5.28786,0.815419) -- (5.28786,0.85584);
\draw [c] (5.27973,0.815419) -- (5.28786,0.815419);
\draw [c] (5.28786,0.815419) -- (5.296,0.815419);
\definecolor{c}{rgb}{0,0,0};
\colorlet{c}{natcomp!70};
\draw [c] (5.30414,0.802019) -- (5.30414,0.840859);
\draw [c] (5.30414,0.840859) -- (5.30414,0.879699);
\draw [c] (5.296,0.840859) -- (5.30414,0.840859);
\draw [c] (5.30414,0.840859) -- (5.31227,0.840859);
\definecolor{c}{rgb}{0,0,0};
\colorlet{c}{natcomp!70};
\draw [c] (5.32041,0.734849) -- (5.32041,0.765883);
\draw [c] (5.32041,0.765883) -- (5.32041,0.796917);
\draw [c] (5.31227,0.765883) -- (5.32041,0.765883);
\draw [c] (5.32041,0.765883) -- (5.32855,0.765883);
\definecolor{c}{rgb}{0,0,0};
\colorlet{c}{natcomp!70};
\draw [c] (5.33668,0.76708) -- (5.33668,0.802514);
\draw [c] (5.33668,0.802514) -- (5.33668,0.837949);
\draw [c] (5.32855,0.802514) -- (5.33668,0.802514);
\draw [c] (5.33668,0.802514) -- (5.34482,0.802514);
\definecolor{c}{rgb}{0,0,0};
\colorlet{c}{natcomp!70};
\draw [c] (5.35295,0.820219) -- (5.35295,0.864005);
\draw [c] (5.35295,0.864005) -- (5.35295,0.907791);
\draw [c] (5.34482,0.864005) -- (5.35295,0.864005);
\draw [c] (5.35295,0.864005) -- (5.36109,0.864005);
\definecolor{c}{rgb}{0,0,0};
\colorlet{c}{natcomp!70};
\draw [c] (5.36923,0.792286) -- (5.36923,0.832453);
\draw [c] (5.36923,0.832453) -- (5.36923,0.87262);
\draw [c] (5.36109,0.832453) -- (5.36923,0.832453);
\draw [c] (5.36923,0.832453) -- (5.37736,0.832453);
\definecolor{c}{rgb}{0,0,0};
\colorlet{c}{natcomp!70};
\draw [c] (5.3855,0.779486) -- (5.3855,0.817789);
\draw [c] (5.3855,0.817789) -- (5.3855,0.856092);
\draw [c] (5.37736,0.817789) -- (5.3855,0.817789);
\draw [c] (5.3855,0.817789) -- (5.39364,0.817789);
\definecolor{c}{rgb}{0,0,0};
\colorlet{c}{natcomp!70};
\draw [c] (5.40177,0.758942) -- (5.40177,0.79033);
\draw [c] (5.40177,0.79033) -- (5.40177,0.821719);
\draw [c] (5.39364,0.79033) -- (5.40177,0.79033);
\draw [c] (5.40177,0.79033) -- (5.40991,0.79033);
\definecolor{c}{rgb}{0,0,0};
\colorlet{c}{natcomp!70};
\draw [c] (5.41805,0.828335) -- (5.41805,0.872313);
\draw [c] (5.41805,0.872313) -- (5.41805,0.916292);
\draw [c] (5.40991,0.872313) -- (5.41805,0.872313);
\draw [c] (5.41805,0.872313) -- (5.42618,0.872313);
\definecolor{c}{rgb}{0,0,0};
\colorlet{c}{natcomp!70};
\draw [c] (5.43432,0.796227) -- (5.43432,0.833442);
\draw [c] (5.43432,0.833442) -- (5.43432,0.870657);
\draw [c] (5.42618,0.833442) -- (5.43432,0.833442);
\draw [c] (5.43432,0.833442) -- (5.44245,0.833442);
\definecolor{c}{rgb}{0,0,0};
\colorlet{c}{natcomp!70};
\draw [c] (5.45059,0.824412) -- (5.45059,0.86803);
\draw [c] (5.45059,0.86803) -- (5.45059,0.911649);
\draw [c] (5.44245,0.86803) -- (5.45059,0.86803);
\draw [c] (5.45059,0.86803) -- (5.45873,0.86803);
\definecolor{c}{rgb}{0,0,0};
\colorlet{c}{natcomp!70};
\draw [c] (5.46686,0.749011) -- (5.46686,0.778106);
\draw [c] (5.46686,0.778106) -- (5.46686,0.807201);
\draw [c] (5.45873,0.778106) -- (5.46686,0.778106);
\draw [c] (5.46686,0.778106) -- (5.475,0.778106);
\definecolor{c}{rgb}{0,0,0};
\colorlet{c}{natcomp!70};
\draw [c] (5.48314,0.717158) -- (5.48314,0.738269);
\draw [c] (5.48314,0.738269) -- (5.48314,0.75938);
\draw [c] (5.475,0.738269) -- (5.48314,0.738269);
\draw [c] (5.48314,0.738269) -- (5.49127,0.738269);
\definecolor{c}{rgb}{0,0,0};
\colorlet{c}{natcomp!70};
\draw [c] (5.49941,0.760794) -- (5.49941,0.791025);
\draw [c] (5.49941,0.791025) -- (5.49941,0.821255);
\draw [c] (5.49127,0.791025) -- (5.49941,0.791025);
\draw [c] (5.49941,0.791025) -- (5.50755,0.791025);
\definecolor{c}{rgb}{0,0,0};
\colorlet{c}{natcomp!70};
\draw [c] (5.51568,0.780683) -- (5.51568,0.819473);
\draw [c] (5.51568,0.819473) -- (5.51568,0.858263);
\draw [c] (5.50755,0.819473) -- (5.51568,0.819473);
\draw [c] (5.51568,0.819473) -- (5.52382,0.819473);
\definecolor{c}{rgb}{0,0,0};
\colorlet{c}{natcomp!70};
\draw [c] (5.53195,0.767697) -- (5.53195,0.81179);
\draw [c] (5.53195,0.81179) -- (5.53195,0.855883);
\draw [c] (5.52382,0.81179) -- (5.53195,0.81179);
\draw [c] (5.53195,0.81179) -- (5.54009,0.81179);
\definecolor{c}{rgb}{0,0,0};
\colorlet{c}{natcomp!70};
\draw [c] (5.54823,0.789662) -- (5.54823,0.82657);
\draw [c] (5.54823,0.82657) -- (5.54823,0.863477);
\draw [c] (5.54009,0.82657) -- (5.54823,0.82657);
\draw [c] (5.54823,0.82657) -- (5.55636,0.82657);
\definecolor{c}{rgb}{0,0,0};
\colorlet{c}{natcomp!70};
\draw [c] (5.5645,0.836001) -- (5.5645,0.881788);
\draw [c] (5.5645,0.881788) -- (5.5645,0.927576);
\draw [c] (5.55636,0.881788) -- (5.5645,0.881788);
\draw [c] (5.5645,0.881788) -- (5.57264,0.881788);
\definecolor{c}{rgb}{0,0,0};
\colorlet{c}{natcomp!70};
\draw [c] (5.58077,0.77978) -- (5.58077,0.81503);
\draw [c] (5.58077,0.81503) -- (5.58077,0.85028);
\draw [c] (5.57264,0.81503) -- (5.58077,0.81503);
\draw [c] (5.58077,0.81503) -- (5.58891,0.81503);
\definecolor{c}{rgb}{0,0,0};
\colorlet{c}{natcomp!70};
\draw [c] (5.59705,0.765328) -- (5.59705,0.798276);
\draw [c] (5.59705,0.798276) -- (5.59705,0.831225);
\draw [c] (5.58891,0.798276) -- (5.59705,0.798276);
\draw [c] (5.59705,0.798276) -- (5.60518,0.798276);
\definecolor{c}{rgb}{0,0,0};
\colorlet{c}{natcomp!70};
\draw [c] (5.61332,0.713101) -- (5.61332,0.735885);
\draw [c] (5.61332,0.735885) -- (5.61332,0.758668);
\draw [c] (5.60518,0.735885) -- (5.61332,0.735885);
\draw [c] (5.61332,0.735885) -- (5.62145,0.735885);
\definecolor{c}{rgb}{0,0,0};
\colorlet{c}{natcomp!70};
\draw [c] (5.62959,0.77156) -- (5.62959,0.806931);
\draw [c] (5.62959,0.806931) -- (5.62959,0.842301);
\draw [c] (5.62145,0.806931) -- (5.62959,0.806931);
\draw [c] (5.62959,0.806931) -- (5.63773,0.806931);
\definecolor{c}{rgb}{0,0,0};
\colorlet{c}{natcomp!70};
\draw [c] (5.64586,0.767598) -- (5.64586,0.801089);
\draw [c] (5.64586,0.801089) -- (5.64586,0.83458);
\draw [c] (5.63773,0.801089) -- (5.64586,0.801089);
\draw [c] (5.64586,0.801089) -- (5.654,0.801089);
\definecolor{c}{rgb}{0,0,0};
\colorlet{c}{natcomp!70};
\draw [c] (5.66214,0.812558) -- (5.66214,0.852249);
\draw [c] (5.66214,0.852249) -- (5.66214,0.891941);
\draw [c] (5.654,0.852249) -- (5.66214,0.852249);
\draw [c] (5.66214,0.852249) -- (5.67027,0.852249);
\definecolor{c}{rgb}{0,0,0};
\colorlet{c}{natcomp!70};
\draw [c] (5.67841,0.750819) -- (5.67841,0.790319);
\draw [c] (5.67841,0.790319) -- (5.67841,0.829819);
\draw [c] (5.67027,0.790319) -- (5.67841,0.790319);
\draw [c] (5.67841,0.790319) -- (5.68655,0.790319);
\definecolor{c}{rgb}{0,0,0};
\colorlet{c}{natcomp!70};
\draw [c] (5.69468,0.763809) -- (5.69468,0.800161);
\draw [c] (5.69468,0.800161) -- (5.69468,0.836514);
\draw [c] (5.68655,0.800161) -- (5.69468,0.800161);
\draw [c] (5.69468,0.800161) -- (5.70282,0.800161);
\definecolor{c}{rgb}{0,0,0};
\colorlet{c}{natcomp!70};
\draw [c] (5.71095,0.773509) -- (5.71095,0.810299);
\draw [c] (5.71095,0.810299) -- (5.71095,0.84709);
\draw [c] (5.70282,0.810299) -- (5.71095,0.810299);
\draw [c] (5.71095,0.810299) -- (5.71909,0.810299);
\definecolor{c}{rgb}{0,0,0};
\colorlet{c}{natcomp!70};
\draw [c] (5.72723,0.781311) -- (5.72723,0.815949);
\draw [c] (5.72723,0.815949) -- (5.72723,0.850586);
\draw [c] (5.71909,0.815949) -- (5.72723,0.815949);
\draw [c] (5.72723,0.815949) -- (5.73536,0.815949);
\definecolor{c}{rgb}{0,0,0};
\colorlet{c}{natcomp!70};
\draw [c] (5.7435,0.749041) -- (5.7435,0.783543);
\draw [c] (5.7435,0.783543) -- (5.7435,0.818044);
\draw [c] (5.73536,0.783543) -- (5.7435,0.783543);
\draw [c] (5.7435,0.783543) -- (5.75164,0.783543);
\definecolor{c}{rgb}{0,0,0};
\colorlet{c}{natcomp!70};
\draw [c] (5.75977,0.734525) -- (5.75977,0.760913);
\draw [c] (5.75977,0.760913) -- (5.75977,0.787301);
\draw [c] (5.75164,0.760913) -- (5.75977,0.760913);
\draw [c] (5.75977,0.760913) -- (5.76791,0.760913);
\definecolor{c}{rgb}{0,0,0};
\colorlet{c}{natcomp!70};
\draw [c] (5.77605,0.757329) -- (5.77605,0.795295);
\draw [c] (5.77605,0.795295) -- (5.77605,0.833262);
\draw [c] (5.76791,0.795295) -- (5.77605,0.795295);
\draw [c] (5.77605,0.795295) -- (5.78418,0.795295);
\definecolor{c}{rgb}{0,0,0};
\colorlet{c}{natcomp!70};
\draw [c] (5.79232,0.778456) -- (5.79232,0.824701);
\draw [c] (5.79232,0.824701) -- (5.79232,0.870946);
\draw [c] (5.78418,0.824701) -- (5.79232,0.824701);
\draw [c] (5.79232,0.824701) -- (5.80045,0.824701);
\definecolor{c}{rgb}{0,0,0};
\colorlet{c}{natcomp!70};
\draw [c] (5.80859,0.791499) -- (5.80859,0.828233);
\draw [c] (5.80859,0.828233) -- (5.80859,0.864968);
\draw [c] (5.80045,0.828233) -- (5.80859,0.828233);
\draw [c] (5.80859,0.828233) -- (5.81673,0.828233);
\definecolor{c}{rgb}{0,0,0};
\colorlet{c}{natcomp!70};
\draw [c] (5.82486,0.778777) -- (5.82486,0.81849);
\draw [c] (5.82486,0.81849) -- (5.82486,0.858202);
\draw [c] (5.81673,0.81849) -- (5.82486,0.81849);
\draw [c] (5.82486,0.81849) -- (5.833,0.81849);
\definecolor{c}{rgb}{0,0,0};
\colorlet{c}{natcomp!70};
\draw [c] (5.84114,0.766161) -- (5.84114,0.815256);
\draw [c] (5.84114,0.815256) -- (5.84114,0.864352);
\draw [c] (5.833,0.815256) -- (5.84114,0.815256);
\draw [c] (5.84114,0.815256) -- (5.84927,0.815256);
\definecolor{c}{rgb}{0,0,0};
\colorlet{c}{natcomp!70};
\draw [c] (5.85741,0.739717) -- (5.85741,0.766483);
\draw [c] (5.85741,0.766483) -- (5.85741,0.79325);
\draw [c] (5.84927,0.766483) -- (5.85741,0.766483);
\draw [c] (5.85741,0.766483) -- (5.86555,0.766483);
\definecolor{c}{rgb}{0,0,0};
\colorlet{c}{natcomp!70};
\draw [c] (5.87368,0.767167) -- (5.87368,0.798411);
\draw [c] (5.87368,0.798411) -- (5.87368,0.829655);
\draw [c] (5.86555,0.798411) -- (5.87368,0.798411);
\draw [c] (5.87368,0.798411) -- (5.88182,0.798411);
\definecolor{c}{rgb}{0,0,0};
\colorlet{c}{natcomp!70};
\draw [c] (5.88995,0.7537) -- (5.88995,0.78274);
\draw [c] (5.88995,0.78274) -- (5.88995,0.811781);
\draw [c] (5.88182,0.78274) -- (5.88995,0.78274);
\draw [c] (5.88995,0.78274) -- (5.89809,0.78274);
\definecolor{c}{rgb}{0,0,0};
\colorlet{c}{natcomp!70};
\draw [c] (5.90623,0.787736) -- (5.90623,0.823567);
\draw [c] (5.90623,0.823567) -- (5.90623,0.859398);
\draw [c] (5.89809,0.823567) -- (5.90623,0.823567);
\draw [c] (5.90623,0.823567) -- (5.91436,0.823567);
\definecolor{c}{rgb}{0,0,0};
\colorlet{c}{natcomp!70};
\draw [c] (5.9225,0.7657) -- (5.9225,0.797966);
\draw [c] (5.9225,0.797966) -- (5.9225,0.830233);
\draw [c] (5.91436,0.797966) -- (5.9225,0.797966);
\draw [c] (5.9225,0.797966) -- (5.93064,0.797966);
\definecolor{c}{rgb}{0,0,0};
\colorlet{c}{natcomp!70};
\draw [c] (5.93877,0.776906) -- (5.93877,0.812086);
\draw [c] (5.93877,0.812086) -- (5.93877,0.847266);
\draw [c] (5.93064,0.812086) -- (5.93877,0.812086);
\draw [c] (5.93877,0.812086) -- (5.94691,0.812086);
\definecolor{c}{rgb}{0,0,0};
\colorlet{c}{natcomp!70};
\draw [c] (5.95505,0.727034) -- (5.95505,0.751779);
\draw [c] (5.95505,0.751779) -- (5.95505,0.776524);
\draw [c] (5.94691,0.751779) -- (5.95505,0.751779);
\draw [c] (5.95505,0.751779) -- (5.96318,0.751779);
\definecolor{c}{rgb}{0,0,0};
\colorlet{c}{natcomp!70};
\draw [c] (5.97132,0.718677) -- (5.97132,0.7484);
\draw [c] (5.97132,0.7484) -- (5.97132,0.778123);
\draw [c] (5.96318,0.7484) -- (5.97132,0.7484);
\draw [c] (5.97132,0.7484) -- (5.97945,0.7484);
\definecolor{c}{rgb}{0,0,0};
\colorlet{c}{natcomp!70};
\draw [c] (5.98759,0.723722) -- (5.98759,0.746507);
\draw [c] (5.98759,0.746507) -- (5.98759,0.769292);
\draw [c] (5.97945,0.746507) -- (5.98759,0.746507);
\draw [c] (5.98759,0.746507) -- (5.99573,0.746507);
\definecolor{c}{rgb}{0,0,0};
\colorlet{c}{natcomp!70};
\draw [c] (6.00386,0.756977) -- (6.00386,0.792774);
\draw [c] (6.00386,0.792774) -- (6.00386,0.828571);
\draw [c] (5.99573,0.792774) -- (6.00386,0.792774);
\draw [c] (6.00386,0.792774) -- (6.012,0.792774);
\definecolor{c}{rgb}{0,0,0};
\colorlet{c}{natcomp!70};
\draw [c] (6.02014,0.72654) -- (6.02014,0.750922);
\draw [c] (6.02014,0.750922) -- (6.02014,0.775304);
\draw [c] (6.012,0.750922) -- (6.02014,0.750922);
\draw [c] (6.02014,0.750922) -- (6.02827,0.750922);
\definecolor{c}{rgb}{0,0,0};
\colorlet{c}{natcomp!70};
\draw [c] (6.03641,0.765876) -- (6.03641,0.798169);
\draw [c] (6.03641,0.798169) -- (6.03641,0.830462);
\draw [c] (6.02827,0.798169) -- (6.03641,0.798169);
\draw [c] (6.03641,0.798169) -- (6.04455,0.798169);
\definecolor{c}{rgb}{0,0,0};
\colorlet{c}{natcomp!70};
\draw [c] (6.05268,0.723692) -- (6.05268,0.763257);
\draw [c] (6.05268,0.763257) -- (6.05268,0.802821);
\draw [c] (6.04455,0.763257) -- (6.05268,0.763257);
\draw [c] (6.05268,0.763257) -- (6.06082,0.763257);
\definecolor{c}{rgb}{0,0,0};
\colorlet{c}{natcomp!70};
\draw [c] (6.06895,0.735947) -- (6.06895,0.776286);
\draw [c] (6.06895,0.776286) -- (6.06895,0.816626);
\draw [c] (6.06082,0.776286) -- (6.06895,0.776286);
\draw [c] (6.06895,0.776286) -- (6.07709,0.776286);
\definecolor{c}{rgb}{0,0,0};
\colorlet{c}{natcomp!70};
\draw [c] (6.08523,0.768199) -- (6.08523,0.8023);
\draw [c] (6.08523,0.8023) -- (6.08523,0.836401);
\draw [c] (6.07709,0.8023) -- (6.08523,0.8023);
\draw [c] (6.08523,0.8023) -- (6.09336,0.8023);
\definecolor{c}{rgb}{0,0,0};
\colorlet{c}{natcomp!70};
\draw [c] (6.1015,0.710277) -- (6.1015,0.729228);
\draw [c] (6.1015,0.729228) -- (6.1015,0.748179);
\draw [c] (6.09336,0.729228) -- (6.1015,0.729228);
\draw [c] (6.1015,0.729228) -- (6.10964,0.729228);
\definecolor{c}{rgb}{0,0,0};
\colorlet{c}{natcomp!70};
\draw [c] (6.11777,0.740692) -- (6.11777,0.767798);
\draw [c] (6.11777,0.767798) -- (6.11777,0.794904);
\draw [c] (6.10964,0.767798) -- (6.11777,0.767798);
\draw [c] (6.11777,0.767798) -- (6.12591,0.767798);
\definecolor{c}{rgb}{0,0,0};
\colorlet{c}{natcomp!70};
\draw [c] (6.13405,0.720528) -- (6.13405,0.743887);
\draw [c] (6.13405,0.743887) -- (6.13405,0.767247);
\draw [c] (6.12591,0.743887) -- (6.13405,0.743887);
\draw [c] (6.13405,0.743887) -- (6.14218,0.743887);
\definecolor{c}{rgb}{0,0,0};
\colorlet{c}{natcomp!70};
\draw [c] (6.15032,0.69748) -- (6.15032,0.712307);
\draw [c] (6.15032,0.712307) -- (6.15032,0.727135);
\draw [c] (6.14218,0.712307) -- (6.15032,0.712307);
\draw [c] (6.15032,0.712307) -- (6.15845,0.712307);
\definecolor{c}{rgb}{0,0,0};
\colorlet{c}{natcomp!70};
\draw [c] (6.16659,0.709648) -- (6.16659,0.728192);
\draw [c] (6.16659,0.728192) -- (6.16659,0.746736);
\draw [c] (6.15845,0.728192) -- (6.16659,0.728192);
\draw [c] (6.16659,0.728192) -- (6.17473,0.728192);
\definecolor{c}{rgb}{0,0,0};
\colorlet{c}{natcomp!70};
\draw [c] (6.18286,0.7346) -- (6.18286,0.766423);
\draw [c] (6.18286,0.766423) -- (6.18286,0.798247);
\draw [c] (6.17473,0.766423) -- (6.18286,0.766423);
\draw [c] (6.18286,0.766423) -- (6.191,0.766423);
\definecolor{c}{rgb}{0,0,0};
\colorlet{c}{natcomp!70};
\draw [c] (6.19914,0.725682) -- (6.19914,0.753499);
\draw [c] (6.19914,0.753499) -- (6.19914,0.781317);
\draw [c] (6.191,0.753499) -- (6.19914,0.753499);
\draw [c] (6.19914,0.753499) -- (6.20727,0.753499);
\definecolor{c}{rgb}{0,0,0};
\colorlet{c}{natcomp!70};
\draw [c] (6.21541,0.726686) -- (6.21541,0.759284);
\draw [c] (6.21541,0.759284) -- (6.21541,0.791882);
\draw [c] (6.20727,0.759284) -- (6.21541,0.759284);
\draw [c] (6.21541,0.759284) -- (6.22355,0.759284);
\definecolor{c}{rgb}{0,0,0};
\colorlet{c}{natcomp!70};
\draw [c] (6.23168,0.742176) -- (6.23168,0.777624);
\draw [c] (6.23168,0.777624) -- (6.23168,0.813072);
\draw [c] (6.22355,0.777624) -- (6.23168,0.777624);
\draw [c] (6.23168,0.777624) -- (6.23982,0.777624);
\definecolor{c}{rgb}{0,0,0};
\colorlet{c}{natcomp!70};
\draw [c] (6.24795,0.776333) -- (6.24795,0.816985);
\draw [c] (6.24795,0.816985) -- (6.24795,0.857637);
\draw [c] (6.23982,0.816985) -- (6.24795,0.816985);
\draw [c] (6.24795,0.816985) -- (6.25609,0.816985);
\definecolor{c}{rgb}{0,0,0};
\colorlet{c}{natcomp!70};
\draw [c] (6.26423,0.716593) -- (6.26423,0.737387);
\draw [c] (6.26423,0.737387) -- (6.26423,0.758182);
\draw [c] (6.25609,0.737387) -- (6.26423,0.737387);
\draw [c] (6.26423,0.737387) -- (6.27236,0.737387);
\definecolor{c}{rgb}{0,0,0};
\colorlet{c}{natcomp!70};
\draw [c] (6.2805,0.73157) -- (6.2805,0.759414);
\draw [c] (6.2805,0.759414) -- (6.2805,0.787258);
\draw [c] (6.27236,0.759414) -- (6.2805,0.759414);
\draw [c] (6.2805,0.759414) -- (6.28864,0.759414);
\definecolor{c}{rgb}{0,0,0};
\colorlet{c}{natcomp!70};
\draw [c] (6.29677,0.765117) -- (6.29677,0.800819);
\draw [c] (6.29677,0.800819) -- (6.29677,0.83652);
\draw [c] (6.28864,0.800819) -- (6.29677,0.800819);
\draw [c] (6.29677,0.800819) -- (6.30491,0.800819);
\definecolor{c}{rgb}{0,0,0};
\colorlet{c}{natcomp!70};
\draw [c] (6.31305,0.697057) -- (6.31305,0.711255);
\draw [c] (6.31305,0.711255) -- (6.31305,0.725454);
\draw [c] (6.30491,0.711255) -- (6.31305,0.711255);
\draw [c] (6.31305,0.711255) -- (6.32118,0.711255);
\definecolor{c}{rgb}{0,0,0};
\colorlet{c}{natcomp!70};
\draw [c] (6.32932,0.699766) -- (6.32932,0.721515);
\draw [c] (6.32932,0.721515) -- (6.32932,0.743263);
\draw [c] (6.32118,0.721515) -- (6.32932,0.721515);
\draw [c] (6.32932,0.721515) -- (6.33745,0.721515);
\definecolor{c}{rgb}{0,0,0};
\colorlet{c}{natcomp!70};
\draw [c] (6.34559,0.75193) -- (6.34559,0.784287);
\draw [c] (6.34559,0.784287) -- (6.34559,0.816645);
\draw [c] (6.33745,0.784287) -- (6.34559,0.784287);
\draw [c] (6.34559,0.784287) -- (6.35373,0.784287);
\definecolor{c}{rgb}{0,0,0};
\colorlet{c}{natcomp!70};
\draw [c] (6.36186,0.751842) -- (6.36186,0.788755);
\draw [c] (6.36186,0.788755) -- (6.36186,0.825668);
\draw [c] (6.35373,0.788755) -- (6.36186,0.788755);
\draw [c] (6.36186,0.788755) -- (6.37,0.788755);
\definecolor{c}{rgb}{0,0,0};
\colorlet{c}{natcomp!70};
\draw [c] (6.37814,0.740937) -- (6.37814,0.773723);
\draw [c] (6.37814,0.773723) -- (6.37814,0.806508);
\draw [c] (6.37,0.773723) -- (6.37814,0.773723);
\draw [c] (6.37814,0.773723) -- (6.38627,0.773723);
\definecolor{c}{rgb}{0,0,0};
\colorlet{c}{natcomp!70};
\draw [c] (6.39441,0.698759) -- (6.39441,0.715275);
\draw [c] (6.39441,0.715275) -- (6.39441,0.73179);
\draw [c] (6.38627,0.715275) -- (6.39441,0.715275);
\draw [c] (6.39441,0.715275) -- (6.40255,0.715275);
\definecolor{c}{rgb}{0,0,0};
\colorlet{c}{natcomp!70};
\draw [c] (6.41068,0.710571) -- (6.41068,0.730263);
\draw [c] (6.41068,0.730263) -- (6.41068,0.749955);
\draw [c] (6.40255,0.730263) -- (6.41068,0.730263);
\draw [c] (6.41068,0.730263) -- (6.41882,0.730263);
\definecolor{c}{rgb}{0,0,0};
\colorlet{c}{natcomp!70};
\draw [c] (6.42695,0.733223) -- (6.42695,0.762826);
\draw [c] (6.42695,0.762826) -- (6.42695,0.792429);
\draw [c] (6.41882,0.762826) -- (6.42695,0.762826);
\draw [c] (6.42695,0.762826) -- (6.43509,0.762826);
\definecolor{c}{rgb}{0,0,0};
\colorlet{c}{natcomp!70};
\draw [c] (6.44323,0.722776) -- (6.44323,0.747622);
\draw [c] (6.44323,0.747622) -- (6.44323,0.772467);
\draw [c] (6.43509,0.747622) -- (6.44323,0.747622);
\draw [c] (6.44323,0.747622) -- (6.45136,0.747622);
\definecolor{c}{rgb}{0,0,0};
\colorlet{c}{natcomp!70};
\draw [c] (6.4595,0.739345) -- (6.4595,0.766635);
\draw [c] (6.4595,0.766635) -- (6.4595,0.793925);
\draw [c] (6.45136,0.766635) -- (6.4595,0.766635);
\draw [c] (6.4595,0.766635) -- (6.46764,0.766635);
\definecolor{c}{rgb}{0,0,0};
\colorlet{c}{natcomp!70};
\draw [c] (6.47577,0.710389) -- (6.47577,0.729721);
\draw [c] (6.47577,0.729721) -- (6.47577,0.749052);
\draw [c] (6.46764,0.729721) -- (6.47577,0.729721);
\draw [c] (6.47577,0.729721) -- (6.48391,0.729721);
\definecolor{c}{rgb}{0,0,0};
\colorlet{c}{natcomp!70};
\draw [c] (6.49205,0.741601) -- (6.49205,0.774287);
\draw [c] (6.49205,0.774287) -- (6.49205,0.806974);
\draw [c] (6.48391,0.774287) -- (6.49205,0.774287);
\draw [c] (6.49205,0.774287) -- (6.50018,0.774287);
\definecolor{c}{rgb}{0,0,0};
\colorlet{c}{natcomp!70};
\draw [c] (6.50832,0.704581) -- (6.50832,0.722973);
\draw [c] (6.50832,0.722973) -- (6.50832,0.741364);
\draw [c] (6.50018,0.722973) -- (6.50832,0.722973);
\draw [c] (6.50832,0.722973) -- (6.51645,0.722973);
\definecolor{c}{rgb}{0,0,0};
\colorlet{c}{natcomp!70};
\draw [c] (6.52459,0.698445) -- (6.52459,0.714356);
\draw [c] (6.52459,0.714356) -- (6.52459,0.730267);
\draw [c] (6.51645,0.714356) -- (6.52459,0.714356);
\draw [c] (6.52459,0.714356) -- (6.53273,0.714356);
\definecolor{c}{rgb}{0,0,0};
\colorlet{c}{natcomp!70};
\draw [c] (6.54086,0.713893) -- (6.54086,0.736042);
\draw [c] (6.54086,0.736042) -- (6.54086,0.758192);
\draw [c] (6.53273,0.736042) -- (6.54086,0.736042);
\draw [c] (6.54086,0.736042) -- (6.549,0.736042);
\definecolor{c}{rgb}{0,0,0};
\colorlet{c}{natcomp!70};
\draw [c] (6.55714,0.705456) -- (6.55714,0.724441);
\draw [c] (6.55714,0.724441) -- (6.55714,0.743426);
\draw [c] (6.549,0.724441) -- (6.55714,0.724441);
\draw [c] (6.55714,0.724441) -- (6.56527,0.724441);
\definecolor{c}{rgb}{0,0,0};
\colorlet{c}{natcomp!70};
\draw [c] (6.57341,0.708858) -- (6.57341,0.726682);
\draw [c] (6.57341,0.726682) -- (6.57341,0.744506);
\draw [c] (6.56527,0.726682) -- (6.57341,0.726682);
\draw [c] (6.57341,0.726682) -- (6.58155,0.726682);
\definecolor{c}{rgb}{0,0,0};
\colorlet{c}{natcomp!70};
\draw [c] (6.58968,0.744164) -- (6.58968,0.773304);
\draw [c] (6.58968,0.773304) -- (6.58968,0.802443);
\draw [c] (6.58155,0.773304) -- (6.58968,0.773304);
\draw [c] (6.58968,0.773304) -- (6.59782,0.773304);
\definecolor{c}{rgb}{0,0,0};
\colorlet{c}{natcomp!70};
\draw [c] (6.60595,0.728187) -- (6.60595,0.754684);
\draw [c] (6.60595,0.754684) -- (6.60595,0.781181);
\draw [c] (6.59782,0.754684) -- (6.60595,0.754684);
\draw [c] (6.60595,0.754684) -- (6.61409,0.754684);
\definecolor{c}{rgb}{0,0,0};
\colorlet{c}{natcomp!70};
\draw [c] (6.62223,0.716607) -- (6.62223,0.741593);
\draw [c] (6.62223,0.741593) -- (6.62223,0.766578);
\draw [c] (6.61409,0.741593) -- (6.62223,0.741593);
\draw [c] (6.62223,0.741593) -- (6.63036,0.741593);
\definecolor{c}{rgb}{0,0,0};
\colorlet{c}{natcomp!70};
\draw [c] (6.6385,0.699804) -- (6.6385,0.717549);
\draw [c] (6.6385,0.717549) -- (6.6385,0.735295);
\draw [c] (6.63036,0.717549) -- (6.6385,0.717549);
\draw [c] (6.6385,0.717549) -- (6.64664,0.717549);
\definecolor{c}{rgb}{0,0,0};
\colorlet{c}{natcomp!70};
\draw [c] (6.65477,0.753668) -- (6.65477,0.789033);
\draw [c] (6.65477,0.789033) -- (6.65477,0.824398);
\draw [c] (6.64664,0.789033) -- (6.65477,0.789033);
\draw [c] (6.65477,0.789033) -- (6.66291,0.789033);
\definecolor{c}{rgb}{0,0,0};
\colorlet{c}{natcomp!70};
\draw [c] (6.67105,0.708234) -- (6.67105,0.734765);
\draw [c] (6.67105,0.734765) -- (6.67105,0.761297);
\draw [c] (6.66291,0.734765) -- (6.67105,0.734765);
\draw [c] (6.67105,0.734765) -- (6.67918,0.734765);
\definecolor{c}{rgb}{0,0,0};
\colorlet{c}{natcomp!70};
\draw [c] (6.68732,0.713447) -- (6.68732,0.738205);
\draw [c] (6.68732,0.738205) -- (6.68732,0.762963);
\draw [c] (6.67918,0.738205) -- (6.68732,0.738205);
\draw [c] (6.68732,0.738205) -- (6.69545,0.738205);
\definecolor{c}{rgb}{0,0,0};
\colorlet{c}{natcomp!70};
\draw [c] (6.70359,0.733101) -- (6.70359,0.759424);
\draw [c] (6.70359,0.759424) -- (6.70359,0.785747);
\draw [c] (6.69545,0.759424) -- (6.70359,0.759424);
\draw [c] (6.70359,0.759424) -- (6.71173,0.759424);
\definecolor{c}{rgb}{0,0,0};
\colorlet{c}{natcomp!70};
\draw [c] (6.71986,0.703929) -- (6.71986,0.721255);
\draw [c] (6.71986,0.721255) -- (6.71986,0.738582);
\draw [c] (6.71173,0.721255) -- (6.71986,0.721255);
\draw [c] (6.71986,0.721255) -- (6.728,0.721255);
\definecolor{c}{rgb}{0,0,0};
\colorlet{c}{natcomp!70};
\draw [c] (6.73614,0.704397) -- (6.73614,0.721875);
\draw [c] (6.73614,0.721875) -- (6.73614,0.739353);
\draw [c] (6.728,0.721875) -- (6.73614,0.721875);
\draw [c] (6.73614,0.721875) -- (6.74427,0.721875);
\definecolor{c}{rgb}{0,0,0};
\colorlet{c}{natcomp!70};
\draw [c] (6.75241,0.710501) -- (6.75241,0.729699);
\draw [c] (6.75241,0.729699) -- (6.75241,0.748896);
\draw [c] (6.74427,0.729699) -- (6.75241,0.729699);
\draw [c] (6.75241,0.729699) -- (6.76055,0.729699);
\definecolor{c}{rgb}{0,0,0};
\colorlet{c}{natcomp!70};
\draw [c] (6.76868,0.711589) -- (6.76868,0.731751);
\draw [c] (6.76868,0.731751) -- (6.76868,0.751914);
\draw [c] (6.76055,0.731751) -- (6.76868,0.731751);
\draw [c] (6.76868,0.731751) -- (6.77682,0.731751);
\definecolor{c}{rgb}{0,0,0};
\colorlet{c}{natcomp!70};
\draw [c] (6.78495,0.718017) -- (6.78495,0.745245);
\draw [c] (6.78495,0.745245) -- (6.78495,0.772473);
\draw [c] (6.77682,0.745245) -- (6.78495,0.745245);
\draw [c] (6.78495,0.745245) -- (6.79309,0.745245);
\definecolor{c}{rgb}{0,0,0};
\colorlet{c}{natcomp!70};
\draw [c] (6.80123,0.697887) -- (6.80123,0.713515);
\draw [c] (6.80123,0.713515) -- (6.80123,0.729144);
\draw [c] (6.79309,0.713515) -- (6.80123,0.713515);
\draw [c] (6.80123,0.713515) -- (6.80936,0.713515);
\definecolor{c}{rgb}{0,0,0};
\colorlet{c}{natcomp!70};
\draw [c] (6.8175,0.712023) -- (6.8175,0.732788);
\draw [c] (6.8175,0.732788) -- (6.8175,0.753553);
\draw [c] (6.80936,0.732788) -- (6.8175,0.732788);
\draw [c] (6.8175,0.732788) -- (6.82564,0.732788);
\definecolor{c}{rgb}{0,0,0};
\colorlet{c}{natcomp!70};
\draw [c] (6.83377,0.714232) -- (6.83377,0.736446);
\draw [c] (6.83377,0.736446) -- (6.83377,0.758661);
\draw [c] (6.82564,0.736446) -- (6.83377,0.736446);
\draw [c] (6.83377,0.736446) -- (6.84191,0.736446);
\definecolor{c}{rgb}{0,0,0};
\colorlet{c}{natcomp!70};
\draw [c] (6.85005,0.709462) -- (6.85005,0.739161);
\draw [c] (6.85005,0.739161) -- (6.85005,0.768861);
\draw [c] (6.84191,0.739161) -- (6.85005,0.739161);
\draw [c] (6.85005,0.739161) -- (6.85818,0.739161);
\definecolor{c}{rgb}{0,0,0};
\colorlet{c}{natcomp!70};
\draw [c] (6.86632,0.69326) -- (6.86632,0.708575);
\draw [c] (6.86632,0.708575) -- (6.86632,0.72389);
\draw [c] (6.85818,0.708575) -- (6.86632,0.708575);
\draw [c] (6.86632,0.708575) -- (6.87445,0.708575);
\definecolor{c}{rgb}{0,0,0};
\colorlet{c}{natcomp!70};
\draw [c] (6.88259,0.703644) -- (6.88259,0.720612);
\draw [c] (6.88259,0.720612) -- (6.88259,0.73758);
\draw [c] (6.87445,0.720612) -- (6.88259,0.720612);
\draw [c] (6.88259,0.720612) -- (6.89073,0.720612);
\definecolor{c}{rgb}{0,0,0};
\colorlet{c}{natcomp!70};
\draw [c] (6.89886,0.692271) -- (6.89886,0.705485);
\draw [c] (6.89886,0.705485) -- (6.89886,0.7187);
\draw [c] (6.89073,0.705485) -- (6.89886,0.705485);
\draw [c] (6.89886,0.705485) -- (6.907,0.705485);
\definecolor{c}{rgb}{0,0,0};
\colorlet{c}{natcomp!70};
\draw [c] (6.91514,0.692587) -- (6.91514,0.706325);
\draw [c] (6.91514,0.706325) -- (6.91514,0.720063);
\draw [c] (6.907,0.706325) -- (6.91514,0.706325);
\draw [c] (6.91514,0.706325) -- (6.92327,0.706325);
\definecolor{c}{rgb}{0,0,0};
\colorlet{c}{natcomp!70};
\draw [c] (6.93141,0.686927) -- (6.93141,0.695065);
\draw [c] (6.93141,0.695065) -- (6.93141,0.703204);
\draw [c] (6.92327,0.695065) -- (6.93141,0.695065);
\draw [c] (6.93141,0.695065) -- (6.93955,0.695065);
\definecolor{c}{rgb}{0,0,0};
\colorlet{c}{natcomp!70};
\draw [c] (6.94768,0.704333) -- (6.94768,0.7221);
\draw [c] (6.94768,0.7221) -- (6.94768,0.739867);
\draw [c] (6.93955,0.7221) -- (6.94768,0.7221);
\draw [c] (6.94768,0.7221) -- (6.95582,0.7221);
\definecolor{c}{rgb}{0,0,0};
\colorlet{c}{natcomp!70};
\draw [c] (6.96395,0.686909) -- (6.96395,0.68692);
\draw [c] (6.96395,0.68692) -- (6.96395,0.686931);
\draw [c] (6.95582,0.68692) -- (6.96395,0.68692);
\draw [c] (6.96395,0.68692) -- (6.97209,0.68692);
\definecolor{c}{rgb}{0,0,0};
\colorlet{c}{natcomp!70};
\draw [c] (6.98023,0.718409) -- (6.98023,0.740294);
\draw [c] (6.98023,0.740294) -- (6.98023,0.762179);
\draw [c] (6.97209,0.740294) -- (6.98023,0.740294);
\draw [c] (6.98023,0.740294) -- (6.98836,0.740294);
\definecolor{c}{rgb}{0,0,0};
\colorlet{c}{natcomp!70};
\draw [c] (6.9965,0.686917) -- (6.9965,0.68693);
\draw [c] (6.9965,0.68693) -- (6.9965,0.686943);
\draw [c] (6.98836,0.68693) -- (6.9965,0.68693);
\draw [c] (6.9965,0.68693) -- (7.00464,0.68693);
\definecolor{c}{rgb}{0,0,0};
\colorlet{c}{natcomp!70};
\draw [c] (7.01277,0.713906) -- (7.01277,0.736963);
\draw [c] (7.01277,0.736963) -- (7.01277,0.760019);
\draw [c] (7.00464,0.736963) -- (7.01277,0.736963);
\draw [c] (7.01277,0.736963) -- (7.02091,0.736963);
\definecolor{c}{rgb}{0,0,0};
\colorlet{c}{natcomp!70};
\draw [c] (7.02905,0.703869) -- (7.02905,0.721431);
\draw [c] (7.02905,0.721431) -- (7.02905,0.738992);
\draw [c] (7.02091,0.721431) -- (7.02905,0.721431);
\draw [c] (7.02905,0.721431) -- (7.03718,0.721431);
\definecolor{c}{rgb}{0,0,0};
\colorlet{c}{natcomp!70};
\draw [c] (7.04532,0.705655) -- (7.04532,0.724877);
\draw [c] (7.04532,0.724877) -- (7.04532,0.7441);
\draw [c] (7.03718,0.724877) -- (7.04532,0.724877);
\draw [c] (7.04532,0.724877) -- (7.05345,0.724877);
\definecolor{c}{rgb}{0,0,0};
\colorlet{c}{natcomp!70};
\draw [c] (7.06159,0.703801) -- (7.06159,0.72083);
\draw [c] (7.06159,0.72083) -- (7.06159,0.73786);
\draw [c] (7.05345,0.72083) -- (7.06159,0.72083);
\draw [c] (7.06159,0.72083) -- (7.06973,0.72083);
\definecolor{c}{rgb}{0,0,0};
\colorlet{c}{natcomp!70};
\draw [c] (7.07786,0.719124) -- (7.07786,0.741674);
\draw [c] (7.07786,0.741674) -- (7.07786,0.764224);
\draw [c] (7.06973,0.741674) -- (7.07786,0.741674);
\draw [c] (7.07786,0.741674) -- (7.086,0.741674);
\definecolor{c}{rgb}{0,0,0};
\colorlet{c}{natcomp!70};
\draw [c] (7.09414,0.709945) -- (7.09414,0.738016);
\draw [c] (7.09414,0.738016) -- (7.09414,0.766087);
\draw [c] (7.086,0.738016) -- (7.09414,0.738016);
\draw [c] (7.09414,0.738016) -- (7.10227,0.738016);
\definecolor{c}{rgb}{0,0,0};
\colorlet{c}{natcomp!70};
\draw [c] (7.11041,0.691149) -- (7.11041,0.701332);
\draw [c] (7.11041,0.701332) -- (7.11041,0.711516);
\draw [c] (7.10227,0.701332) -- (7.11041,0.701332);
\draw [c] (7.11041,0.701332) -- (7.11855,0.701332);
\definecolor{c}{rgb}{0,0,0};
\colorlet{c}{natcomp!70};
\draw [c] (7.12668,0.706897) -- (7.12668,0.728052);
\draw [c] (7.12668,0.728052) -- (7.12668,0.749206);
\draw [c] (7.11855,0.728052) -- (7.12668,0.728052);
\draw [c] (7.12668,0.728052) -- (7.13482,0.728052);
\definecolor{c}{rgb}{0,0,0};
\colorlet{c}{natcomp!70};
\draw [c] (7.14295,0.691954) -- (7.14295,0.704175);
\draw [c] (7.14295,0.704175) -- (7.14295,0.716395);
\draw [c] (7.13482,0.704175) -- (7.14295,0.704175);
\draw [c] (7.14295,0.704175) -- (7.15109,0.704175);
\definecolor{c}{rgb}{0,0,0};
\colorlet{c}{natcomp!70};
\draw [c] (7.15923,0.686944) -- (7.15923,0.695807);
\draw [c] (7.15923,0.695807) -- (7.15923,0.704669);
\draw [c] (7.15109,0.695807) -- (7.15923,0.695807);
\draw [c] (7.15923,0.695807) -- (7.16736,0.695807);
\definecolor{c}{rgb}{0,0,0};
\colorlet{c}{natcomp!70};
\draw [c] (7.1755,0.713402) -- (7.1755,0.735241);
\draw [c] (7.1755,0.735241) -- (7.1755,0.75708);
\draw [c] (7.16736,0.735241) -- (7.1755,0.735241);
\draw [c] (7.1755,0.735241) -- (7.18364,0.735241);
\definecolor{c}{rgb}{0,0,0};
\colorlet{c}{natcomp!70};
\draw [c] (7.19177,0.70731) -- (7.19177,0.72784);
\draw [c] (7.19177,0.72784) -- (7.19177,0.74837);
\draw [c] (7.18364,0.72784) -- (7.19177,0.72784);
\draw [c] (7.19177,0.72784) -- (7.19991,0.72784);
\definecolor{c}{rgb}{0,0,0};
\colorlet{c}{natcomp!70};
\draw [c] (7.20805,0.694873) -- (7.20805,0.730967);
\draw [c] (7.20805,0.730967) -- (7.20805,0.767061);
\draw [c] (7.19991,0.730967) -- (7.20805,0.730967);
\draw [c] (7.20805,0.730967) -- (7.21618,0.730967);
\definecolor{c}{rgb}{0,0,0};
\colorlet{c}{natcomp!70};
\draw [c] (7.22432,0.692201) -- (7.22432,0.704986);
\draw [c] (7.22432,0.704986) -- (7.22432,0.71777);
\draw [c] (7.21618,0.704986) -- (7.22432,0.704986);
\draw [c] (7.22432,0.704986) -- (7.23245,0.704986);
\definecolor{c}{rgb}{0,0,0};
\colorlet{c}{natcomp!70};
\draw [c] (7.24059,0.694145) -- (7.24059,0.712713);
\draw [c] (7.24059,0.712713) -- (7.24059,0.731281);
\draw [c] (7.23245,0.712713) -- (7.24059,0.712713);
\draw [c] (7.24059,0.712713) -- (7.24873,0.712713);
\definecolor{c}{rgb}{0,0,0};
\colorlet{c}{natcomp!70};
\draw [c] (7.25686,0.686911) -- (7.25686,0.696805);
\draw [c] (7.25686,0.696805) -- (7.25686,0.706699);
\draw [c] (7.24873,0.696805) -- (7.25686,0.696805);
\draw [c] (7.25686,0.696805) -- (7.265,0.696805);
\definecolor{c}{rgb}{0,0,0};
\colorlet{c}{natcomp!70};
\draw [c] (7.27314,0.686939) -- (7.27314,0.698171);
\draw [c] (7.27314,0.698171) -- (7.27314,0.709404);
\draw [c] (7.265,0.698171) -- (7.27314,0.698171);
\draw [c] (7.27314,0.698171) -- (7.28127,0.698171);
\definecolor{c}{rgb}{0,0,0};
\colorlet{c}{natcomp!70};
\draw [c] (7.28941,0.705817) -- (7.28941,0.724823);
\draw [c] (7.28941,0.724823) -- (7.28941,0.743829);
\draw [c] (7.28127,0.724823) -- (7.28941,0.724823);
\draw [c] (7.28941,0.724823) -- (7.29755,0.724823);
\definecolor{c}{rgb}{0,0,0};
\colorlet{c}{natcomp!70};
\draw [c] (7.30568,0.704012) -- (7.30568,0.721633);
\draw [c] (7.30568,0.721633) -- (7.30568,0.739254);
\draw [c] (7.29755,0.721633) -- (7.30568,0.721633);
\draw [c] (7.30568,0.721633) -- (7.31382,0.721633);
\definecolor{c}{rgb}{0,0,0};
\colorlet{c}{natcomp!70};
\draw [c] (7.32195,0.714809) -- (7.32195,0.753099);
\draw [c] (7.32195,0.753099) -- (7.32195,0.791389);
\draw [c] (7.31382,0.753099) -- (7.32195,0.753099);
\draw [c] (7.32195,0.753099) -- (7.33009,0.753099);
\definecolor{c}{rgb}{0,0,0};
\colorlet{c}{natcomp!70};
\draw [c] (7.33823,0.697923) -- (7.33823,0.713018);
\draw [c] (7.33823,0.713018) -- (7.33823,0.728113);
\draw [c] (7.33009,0.713018) -- (7.33823,0.713018);
\draw [c] (7.33823,0.713018) -- (7.34636,0.713018);
\definecolor{c}{rgb}{0,0,0};
\colorlet{c}{natcomp!70};
\draw [c] (7.3545,0.698691) -- (7.3545,0.715314);
\draw [c] (7.3545,0.715314) -- (7.3545,0.731936);
\draw [c] (7.34636,0.715314) -- (7.3545,0.715314);
\draw [c] (7.3545,0.715314) -- (7.36264,0.715314);
\definecolor{c}{rgb}{0,0,0};
\colorlet{c}{natcomp!70};
\draw [c] (7.37077,0.717852) -- (7.37077,0.745539);
\draw [c] (7.37077,0.745539) -- (7.37077,0.773225);
\draw [c] (7.36264,0.745539) -- (7.37077,0.745539);
\draw [c] (7.37077,0.745539) -- (7.37891,0.745539);
\definecolor{c}{rgb}{0,0,0};
\colorlet{c}{natcomp!70};
\draw [c] (7.38705,0.701315) -- (7.38705,0.726433);
\draw [c] (7.38705,0.726433) -- (7.38705,0.751551);
\draw [c] (7.37891,0.726433) -- (7.38705,0.726433);
\draw [c] (7.38705,0.726433) -- (7.39518,0.726433);
\definecolor{c}{rgb}{0,0,0};
\colorlet{c}{natcomp!70};
\draw [c] (7.40332,0.719987) -- (7.40332,0.743126);
\draw [c] (7.40332,0.743126) -- (7.40332,0.766266);
\draw [c] (7.39518,0.743126) -- (7.40332,0.743126);
\draw [c] (7.40332,0.743126) -- (7.41145,0.743126);
\definecolor{c}{rgb}{0,0,0};
\colorlet{c}{natcomp!70};
\draw [c] (7.41959,0.691933) -- (7.41959,0.704153);
\draw [c] (7.41959,0.704153) -- (7.41959,0.716373);
\draw [c] (7.41145,0.704153) -- (7.41959,0.704153);
\draw [c] (7.41959,0.704153) -- (7.42773,0.704153);
\definecolor{c}{rgb}{0,0,0};
\colorlet{c}{natcomp!70};
\draw [c] (7.45214,0.686902) -- (7.45214,0.695866);
\draw [c] (7.45214,0.695866) -- (7.45214,0.704829);
\draw [c] (7.444,0.695866) -- (7.45214,0.695866);
\draw [c] (7.45214,0.695866) -- (7.46027,0.695866);
\definecolor{c}{rgb}{0,0,0};
\colorlet{c}{natcomp!70};
\draw [c] (7.46841,0.686921) -- (7.46841,0.695885);
\draw [c] (7.46841,0.695885) -- (7.46841,0.704848);
\draw [c] (7.46027,0.695885) -- (7.46841,0.695885);
\draw [c] (7.46841,0.695885) -- (7.47655,0.695885);
\definecolor{c}{rgb}{0,0,0};
\colorlet{c}{natcomp!70};
\draw [c] (7.48468,0.686897) -- (7.48468,0.686905);
\draw [c] (7.48468,0.686905) -- (7.48468,0.686914);
\draw [c] (7.47655,0.686905) -- (7.48468,0.686905);
\draw [c] (7.48468,0.686905) -- (7.49282,0.686905);
\definecolor{c}{rgb}{0,0,0};
\colorlet{c}{natcomp!70};
\draw [c] (7.50095,0.686894) -- (7.50095,0.686899);
\draw [c] (7.50095,0.686899) -- (7.50095,0.686904);
\draw [c] (7.49282,0.686899) -- (7.50095,0.686899);
\draw [c] (7.50095,0.686899) -- (7.50909,0.686899);
\definecolor{c}{rgb}{0,0,0};
\colorlet{c}{natcomp!70};
\draw [c] (7.51723,0.702942) -- (7.51723,0.719147);
\draw [c] (7.51723,0.719147) -- (7.51723,0.735351);
\draw [c] (7.50909,0.719147) -- (7.51723,0.719147);
\draw [c] (7.51723,0.719147) -- (7.52536,0.719147);
\definecolor{c}{rgb}{0,0,0};
\colorlet{c}{natcomp!70};
\draw [c] (7.5335,0.691582) -- (7.5335,0.703001);
\draw [c] (7.5335,0.703001) -- (7.5335,0.71442);
\draw [c] (7.52536,0.703001) -- (7.5335,0.703001);
\draw [c] (7.5335,0.703001) -- (7.54164,0.703001);
\definecolor{c}{rgb}{0,0,0};
\colorlet{c}{natcomp!70};
\draw [c] (7.54977,0.699339) -- (7.54977,0.718609);
\draw [c] (7.54977,0.718609) -- (7.54977,0.737879);
\draw [c] (7.54164,0.718609) -- (7.54977,0.718609);
\draw [c] (7.54977,0.718609) -- (7.55791,0.718609);
\definecolor{c}{rgb}{0,0,0};
\colorlet{c}{natcomp!70};
\draw [c] (7.56605,0.705399) -- (7.56605,0.728064);
\draw [c] (7.56605,0.728064) -- (7.56605,0.750729);
\draw [c] (7.55791,0.728064) -- (7.56605,0.728064);
\draw [c] (7.56605,0.728064) -- (7.57418,0.728064);
\definecolor{c}{rgb}{0,0,0};
\colorlet{c}{natcomp!70};
\draw [c] (7.58232,0.699404) -- (7.58232,0.723597);
\draw [c] (7.58232,0.723597) -- (7.58232,0.74779);
\draw [c] (7.57418,0.723597) -- (7.58232,0.723597);
\draw [c] (7.58232,0.723597) -- (7.59045,0.723597);
\definecolor{c}{rgb}{0,0,0};
\colorlet{c}{natcomp!70};
\draw [c] (7.59859,0.686912) -- (7.59859,0.694113);
\draw [c] (7.59859,0.694113) -- (7.59859,0.701313);
\draw [c] (7.59045,0.694113) -- (7.59859,0.694113);
\draw [c] (7.59859,0.694113) -- (7.60673,0.694113);
\definecolor{c}{rgb}{0,0,0};
\colorlet{c}{natcomp!70};
\draw [c] (7.61486,0.686896) -- (7.61486,0.686902);
\draw [c] (7.61486,0.686902) -- (7.61486,0.686909);
\draw [c] (7.60673,0.686902) -- (7.61486,0.686902);
\draw [c] (7.61486,0.686902) -- (7.623,0.686902);
\definecolor{c}{rgb}{0,0,0};
\colorlet{c}{natcomp!70};
\draw [c] (7.63114,0.721485) -- (7.63114,0.74621);
\draw [c] (7.63114,0.74621) -- (7.63114,0.770934);
\draw [c] (7.623,0.74621) -- (7.63114,0.74621);
\draw [c] (7.63114,0.74621) -- (7.63927,0.74621);
\definecolor{c}{rgb}{0,0,0};
\colorlet{c}{natcomp!70};
\draw [c] (7.64741,0.686934) -- (7.64741,0.695073);
\draw [c] (7.64741,0.695073) -- (7.64741,0.703212);
\draw [c] (7.63927,0.695073) -- (7.64741,0.695073);
\draw [c] (7.64741,0.695073) -- (7.65555,0.695073);
\definecolor{c}{rgb}{0,0,0};
\colorlet{c}{natcomp!70};
\draw [c] (7.66368,0.69193) -- (7.66368,0.704888);
\draw [c] (7.66368,0.704888) -- (7.66368,0.717847);
\draw [c] (7.65555,0.704888) -- (7.66368,0.704888);
\draw [c] (7.66368,0.704888) -- (7.67182,0.704888);
\definecolor{c}{rgb}{0,0,0};
\colorlet{c}{natcomp!70};
\draw [c] (7.67995,0.693227) -- (7.67995,0.708478);
\draw [c] (7.67995,0.708478) -- (7.67995,0.723729);
\draw [c] (7.67182,0.708478) -- (7.67995,0.708478);
\draw [c] (7.67995,0.708478) -- (7.68809,0.708478);
\definecolor{c}{rgb}{0,0,0};
\colorlet{c}{natcomp!70};
\draw [c] (7.69623,0.705019) -- (7.69623,0.723124);
\draw [c] (7.69623,0.723124) -- (7.69623,0.74123);
\draw [c] (7.68809,0.723124) -- (7.69623,0.723124);
\draw [c] (7.69623,0.723124) -- (7.70436,0.723124);
\definecolor{c}{rgb}{0,0,0};
\colorlet{c}{natcomp!70};
\draw [c] (7.7125,0.686901) -- (7.7125,0.686914);
\draw [c] (7.7125,0.686914) -- (7.7125,0.686926);
\draw [c] (7.70436,0.686914) -- (7.7125,0.686914);
\draw [c] (7.7125,0.686914) -- (7.72064,0.686914);
\definecolor{c}{rgb}{0,0,0};
\colorlet{c}{natcomp!70};
\draw [c] (7.72877,0.686911) -- (7.72877,0.697705);
\draw [c] (7.72877,0.697705) -- (7.72877,0.7085);
\draw [c] (7.72064,0.697705) -- (7.72877,0.697705);
\draw [c] (7.72877,0.697705) -- (7.73691,0.697705);
\definecolor{c}{rgb}{0,0,0};
\colorlet{c}{natcomp!70};
\draw [c] (7.74505,0.697486) -- (7.74505,0.712097);
\draw [c] (7.74505,0.712097) -- (7.74505,0.726708);
\draw [c] (7.73691,0.712097) -- (7.74505,0.712097);
\draw [c] (7.74505,0.712097) -- (7.75318,0.712097);
\definecolor{c}{rgb}{0,0,0};
\colorlet{c}{natcomp!70};
\draw [c] (7.76132,0.692362) -- (7.76132,0.708107);
\draw [c] (7.76132,0.708107) -- (7.76132,0.723852);
\draw [c] (7.75318,0.708107) -- (7.76132,0.708107);
\draw [c] (7.76132,0.708107) -- (7.76945,0.708107);
\definecolor{c}{rgb}{0,0,0};
\colorlet{c}{natcomp!70};
\draw [c] (7.77759,0.699709) -- (7.77759,0.717404);
\draw [c] (7.77759,0.717404) -- (7.77759,0.735098);
\draw [c] (7.76945,0.717404) -- (7.77759,0.717404);
\draw [c] (7.77759,0.717404) -- (7.78573,0.717404);
\definecolor{c}{rgb}{0,0,0};
\colorlet{c}{natcomp!70};
\draw [c] (7.79386,0.733221) -- (7.79386,0.759074);
\draw [c] (7.79386,0.759074) -- (7.79386,0.784928);
\draw [c] (7.78573,0.759074) -- (7.79386,0.759074);
\draw [c] (7.79386,0.759074) -- (7.802,0.759074);
\definecolor{c}{rgb}{0,0,0};
\colorlet{c}{natcomp!70};
\draw [c] (7.81014,0.697974) -- (7.81014,0.713528);
\draw [c] (7.81014,0.713528) -- (7.81014,0.729082);
\draw [c] (7.802,0.713528) -- (7.81014,0.713528);
\draw [c] (7.81014,0.713528) -- (7.81827,0.713528);
\definecolor{c}{rgb}{0,0,0};
\colorlet{c}{natcomp!70};
\draw [c] (7.82641,0.692082) -- (7.82641,0.705044);
\draw [c] (7.82641,0.705044) -- (7.82641,0.718006);
\draw [c] (7.81827,0.705044) -- (7.82641,0.705044);
\draw [c] (7.82641,0.705044) -- (7.83455,0.705044);
\definecolor{c}{rgb}{0,0,0};
\colorlet{c}{natcomp!70};
\draw [c] (7.84268,0.69139) -- (7.84268,0.702257);
\draw [c] (7.84268,0.702257) -- (7.84268,0.713124);
\draw [c] (7.83455,0.702257) -- (7.84268,0.702257);
\draw [c] (7.84268,0.702257) -- (7.85082,0.702257);
\definecolor{c}{rgb}{0,0,0};
\colorlet{c}{natcomp!70};
\draw [c] (7.85895,0.68692) -- (7.85895,0.696815);
\draw [c] (7.85895,0.696815) -- (7.85895,0.706709);
\draw [c] (7.85082,0.696815) -- (7.85895,0.696815);
\draw [c] (7.85895,0.696815) -- (7.86709,0.696815);
\definecolor{c}{rgb}{0,0,0};
\colorlet{c}{natcomp!70};
\draw [c] (7.87523,0.699254) -- (7.87523,0.718923);
\draw [c] (7.87523,0.718923) -- (7.87523,0.738592);
\draw [c] (7.86709,0.718923) -- (7.87523,0.718923);
\draw [c] (7.87523,0.718923) -- (7.88336,0.718923);
\definecolor{c}{rgb}{0,0,0};
\colorlet{c}{natcomp!70};
\draw [c] (7.8915,0.711611) -- (7.8915,0.732071);
\draw [c] (7.8915,0.732071) -- (7.8915,0.752532);
\draw [c] (7.88336,0.732071) -- (7.8915,0.732071);
\draw [c] (7.8915,0.732071) -- (7.89964,0.732071);
\definecolor{c}{rgb}{0,0,0};
\colorlet{c}{natcomp!70};
\draw [c] (7.90777,0.715085) -- (7.90777,0.738825);
\draw [c] (7.90777,0.738825) -- (7.90777,0.762565);
\draw [c] (7.89964,0.738825) -- (7.90777,0.738825);
\draw [c] (7.90777,0.738825) -- (7.91591,0.738825);
\definecolor{c}{rgb}{0,0,0};
\colorlet{c}{natcomp!70};
\draw [c] (7.92405,0.686922) -- (7.92405,0.696038);
\draw [c] (7.92405,0.696038) -- (7.92405,0.705153);
\draw [c] (7.91591,0.696038) -- (7.92405,0.696038);
\draw [c] (7.92405,0.696038) -- (7.93218,0.696038);
\definecolor{c}{rgb}{0,0,0};
\colorlet{c}{natcomp!70};
\draw [c] (7.94032,0.686907) -- (7.94032,0.696802);
\draw [c] (7.94032,0.696802) -- (7.94032,0.706696);
\draw [c] (7.93218,0.696802) -- (7.94032,0.696802);
\draw [c] (7.94032,0.696802) -- (7.94845,0.696802);
\definecolor{c}{rgb}{0,0,0};
\colorlet{c}{natcomp!70};
\draw [c] (7.95659,0.691941) -- (7.95659,0.704161);
\draw [c] (7.95659,0.704161) -- (7.95659,0.716381);
\draw [c] (7.94845,0.704161) -- (7.95659,0.704161);
\draw [c] (7.95659,0.704161) -- (7.96473,0.704161);
\definecolor{c}{rgb}{0,0,0};
\colorlet{c}{natcomp!70};
\draw [c] (7.97286,0.686912) -- (7.97286,0.69505);
\draw [c] (7.97286,0.69505) -- (7.97286,0.703189);
\draw [c] (7.96473,0.69505) -- (7.97286,0.69505);
\draw [c] (7.97286,0.69505) -- (7.981,0.69505);
\definecolor{c}{rgb}{0,0,0};
\colorlet{c}{natcomp!70};
\draw [c] (7.98914,0.693662) -- (7.98914,0.710807);
\draw [c] (7.98914,0.710807) -- (7.98914,0.727952);
\draw [c] (7.981,0.710807) -- (7.98914,0.710807);
\draw [c] (7.98914,0.710807) -- (7.99727,0.710807);
\definecolor{c}{rgb}{0,0,0};
\colorlet{c}{natcomp!70};
\draw [c] (8.00541,0.686922) -- (8.00541,0.695785);
\draw [c] (8.00541,0.695785) -- (8.00541,0.704647);
\draw [c] (7.99727,0.695785) -- (8.00541,0.695785);
\draw [c] (8.00541,0.695785) -- (8.01355,0.695785);
\definecolor{c}{rgb}{0,0,0};
\colorlet{c}{natcomp!70};
\draw [c] (8.02168,0.691552) -- (8.02168,0.702771);
\draw [c] (8.02168,0.702771) -- (8.02168,0.71399);
\draw [c] (8.01355,0.702771) -- (8.02168,0.702771);
\draw [c] (8.02168,0.702771) -- (8.02982,0.702771);
\definecolor{c}{rgb}{0,0,0};
\colorlet{c}{natcomp!70};
\draw [c] (8.03795,0.686908) -- (8.03795,0.70091);
\draw [c] (8.03795,0.70091) -- (8.03795,0.714913);
\draw [c] (8.02982,0.70091) -- (8.03795,0.70091);
\draw [c] (8.03795,0.70091) -- (8.04609,0.70091);
\definecolor{c}{rgb}{0,0,0};
\colorlet{c}{natcomp!70};
\draw [c] (8.05423,0.686902) -- (8.05423,0.695765);
\draw [c] (8.05423,0.695765) -- (8.05423,0.704627);
\draw [c] (8.04609,0.695765) -- (8.05423,0.695765);
\draw [c] (8.05423,0.695765) -- (8.06236,0.695765);
\definecolor{c}{rgb}{0,0,0};
\colorlet{c}{natcomp!70};
\draw [c] (8.0705,0.686913) -- (8.0705,0.694635);
\draw [c] (8.0705,0.694635) -- (8.0705,0.702356);
\draw [c] (8.06236,0.694635) -- (8.0705,0.694635);
\draw [c] (8.0705,0.694635) -- (8.07864,0.694635);
\definecolor{c}{rgb}{0,0,0};
\colorlet{c}{natcomp!70};
\draw [c] (8.08677,0.691886) -- (8.08677,0.703918);
\draw [c] (8.08677,0.703918) -- (8.08677,0.715951);
\draw [c] (8.07864,0.703918) -- (8.08677,0.703918);
\draw [c] (8.08677,0.703918) -- (8.09491,0.703918);
\definecolor{c}{rgb}{0,0,0};
\colorlet{c}{natcomp!70};
\draw [c] (8.10305,0.699036) -- (8.10305,0.715859);
\draw [c] (8.10305,0.715859) -- (8.10305,0.732682);
\draw [c] (8.09491,0.715859) -- (8.10305,0.715859);
\draw [c] (8.10305,0.715859) -- (8.11118,0.715859);
\definecolor{c}{rgb}{0,0,0};
\colorlet{c}{natcomp!70};
\draw [c] (8.11932,0.70093) -- (8.11932,0.723385);
\draw [c] (8.11932,0.723385) -- (8.11932,0.74584);
\draw [c] (8.11118,0.723385) -- (8.11932,0.723385);
\draw [c] (8.11932,0.723385) -- (8.12745,0.723385);
\definecolor{c}{rgb}{0,0,0};
\colorlet{c}{natcomp!70};
\draw [c] (8.13559,0.691991) -- (8.13559,0.704542);
\draw [c] (8.13559,0.704542) -- (8.13559,0.717092);
\draw [c] (8.12745,0.704542) -- (8.13559,0.704542);
\draw [c] (8.13559,0.704542) -- (8.14373,0.704542);
\definecolor{c}{rgb}{0,0,0};
\colorlet{c}{natcomp!70};
\draw [c] (8.15186,0.704381) -- (8.15186,0.722348);
\draw [c] (8.15186,0.722348) -- (8.15186,0.740315);
\draw [c] (8.14373,0.722348) -- (8.15186,0.722348);
\draw [c] (8.15186,0.722348) -- (8.16,0.722348);
\definecolor{c}{rgb}{0,0,0};
\colorlet{c}{natcomp!70};
\draw [c] (8.16814,0.692514) -- (8.16814,0.706186);
\draw [c] (8.16814,0.706186) -- (8.16814,0.719858);
\draw [c] (8.16,0.706186) -- (8.16814,0.706186);
\draw [c] (8.16814,0.706186) -- (8.17627,0.706186);
\definecolor{c}{rgb}{0,0,0};
\colorlet{c}{natcomp!70};
\draw [c] (8.18441,0.686908) -- (8.18441,0.697319);
\draw [c] (8.18441,0.697319) -- (8.18441,0.70773);
\draw [c] (8.17627,0.697319) -- (8.18441,0.697319);
\draw [c] (8.18441,0.697319) -- (8.19255,0.697319);
\definecolor{c}{rgb}{0,0,0};
\colorlet{c}{natcomp!70};
\draw [c] (8.20068,0.686908) -- (8.20068,0.708675);
\draw [c] (8.20068,0.708675) -- (8.20068,0.730443);
\draw [c] (8.19255,0.708675) -- (8.20068,0.708675);
\draw [c] (8.20068,0.708675) -- (8.20882,0.708675);
\definecolor{c}{rgb}{0,0,0};
\colorlet{c}{natcomp!70};
\draw [c] (8.21695,0.691366) -- (8.21695,0.702233);
\draw [c] (8.21695,0.702233) -- (8.21695,0.7131);
\draw [c] (8.20882,0.702233) -- (8.21695,0.702233);
\draw [c] (8.21695,0.702233) -- (8.22509,0.702233);
\definecolor{c}{rgb}{0,0,0};
\colorlet{c}{natcomp!70};
\draw [c] (8.23323,0.700598) -- (8.23323,0.719335);
\draw [c] (8.23323,0.719335) -- (8.23323,0.738073);
\draw [c] (8.22509,0.719335) -- (8.23323,0.719335);
\draw [c] (8.23323,0.719335) -- (8.24136,0.719335);
\definecolor{c}{rgb}{0,0,0};
\colorlet{c}{natcomp!70};
\draw [c] (8.2495,0.698239) -- (8.2495,0.713998);
\draw [c] (8.2495,0.713998) -- (8.2495,0.729758);
\draw [c] (8.24136,0.713998) -- (8.2495,0.713998);
\draw [c] (8.2495,0.713998) -- (8.25764,0.713998);
\definecolor{c}{rgb}{0,0,0};
\colorlet{c}{natcomp!70};
\draw [c] (8.26577,0.704207) -- (8.26577,0.72189);
\draw [c] (8.26577,0.72189) -- (8.26577,0.739573);
\draw [c] (8.25764,0.72189) -- (8.26577,0.72189);
\draw [c] (8.26577,0.72189) -- (8.27391,0.72189);
\definecolor{c}{rgb}{0,0,0};
\colorlet{c}{natcomp!70};
\draw [c] (8.28205,0.699406) -- (8.28205,0.719236);
\draw [c] (8.28205,0.719236) -- (8.28205,0.739065);
\draw [c] (8.27391,0.719236) -- (8.28205,0.719236);
\draw [c] (8.28205,0.719236) -- (8.29018,0.719236);
\definecolor{c}{rgb}{0,0,0};
\colorlet{c}{natcomp!70};
\draw [c] (8.29832,0.686897) -- (8.29832,0.686904);
\draw [c] (8.29832,0.686904) -- (8.29832,0.686911);
\draw [c] (8.29018,0.686904) -- (8.29832,0.686904);
\draw [c] (8.29832,0.686904) -- (8.30645,0.686904);
\definecolor{c}{rgb}{0,0,0};
\colorlet{c}{natcomp!70};
\draw [c] (8.31459,0.692156) -- (8.31459,0.704967);
\draw [c] (8.31459,0.704967) -- (8.31459,0.717779);
\draw [c] (8.30645,0.704967) -- (8.31459,0.704967);
\draw [c] (8.31459,0.704967) -- (8.32273,0.704967);
\definecolor{c}{rgb}{0,0,0};
\colorlet{c}{natcomp!70};
\draw [c] (8.33086,0.686894) -- (8.33086,0.686898);
\draw [c] (8.33086,0.686898) -- (8.33086,0.686902);
\draw [c] (8.32273,0.686898) -- (8.33086,0.686898);
\draw [c] (8.33086,0.686898) -- (8.339,0.686898);
\definecolor{c}{rgb}{0,0,0};
\colorlet{c}{natcomp!70};
\draw [c] (8.34714,0.686913) -- (8.34714,0.696808);
\draw [c] (8.34714,0.696808) -- (8.34714,0.706702);
\draw [c] (8.339,0.696808) -- (8.34714,0.696808);
\draw [c] (8.34714,0.696808) -- (8.35527,0.696808);
\definecolor{c}{rgb}{0,0,0};
\colorlet{c}{natcomp!70};
\draw [c] (8.36341,0.6869) -- (8.36341,0.686908);
\draw [c] (8.36341,0.686908) -- (8.36341,0.686916);
\draw [c] (8.35527,0.686908) -- (8.36341,0.686908);
\draw [c] (8.36341,0.686908) -- (8.37155,0.686908);
\definecolor{c}{rgb}{0,0,0};
\colorlet{c}{natcomp!70};
\draw [c] (8.37968,0.709537) -- (8.37968,0.733883);
\draw [c] (8.37968,0.733883) -- (8.37968,0.758228);
\draw [c] (8.37155,0.733883) -- (8.37968,0.733883);
\draw [c] (8.37968,0.733883) -- (8.38782,0.733883);
\definecolor{c}{rgb}{0,0,0};
\colorlet{c}{natcomp!70};
\draw [c] (8.39595,0.686897) -- (8.39595,0.696792);
\draw [c] (8.39595,0.696792) -- (8.39595,0.706686);
\draw [c] (8.38782,0.696792) -- (8.39595,0.696792);
\draw [c] (8.39595,0.696792) -- (8.40409,0.696792);
\definecolor{c}{rgb}{0,0,0};
\colorlet{c}{natcomp!70};
\draw [c] (8.41223,0.686907) -- (8.41223,0.696023);
\draw [c] (8.41223,0.696023) -- (8.41223,0.705138);
\draw [c] (8.40409,0.696023) -- (8.41223,0.696023);
\draw [c] (8.41223,0.696023) -- (8.42036,0.696023);
\definecolor{c}{rgb}{0,0,0};
\colorlet{c}{natcomp!70};
\draw [c] (8.4285,0.691882) -- (8.4285,0.703914);
\draw [c] (8.4285,0.703914) -- (8.4285,0.715947);
\draw [c] (8.42036,0.703914) -- (8.4285,0.703914);
\draw [c] (8.4285,0.703914) -- (8.43664,0.703914);
\definecolor{c}{rgb}{0,0,0};
\colorlet{c}{natcomp!70};
\draw [c] (8.44477,0.686908) -- (8.44477,0.69814);
\draw [c] (8.44477,0.69814) -- (8.44477,0.709372);
\draw [c] (8.43664,0.69814) -- (8.44477,0.69814);
\draw [c] (8.44477,0.69814) -- (8.45291,0.69814);
\definecolor{c}{rgb}{0,0,0};
\colorlet{c}{natcomp!70};
\draw [c] (8.46105,0.686896) -- (8.46105,0.686903);
\draw [c] (8.46105,0.686903) -- (8.46105,0.686909);
\draw [c] (8.45291,0.686903) -- (8.46105,0.686903);
\draw [c] (8.46105,0.686903) -- (8.46918,0.686903);
\definecolor{c}{rgb}{0,0,0};
\colorlet{c}{natcomp!70};
\draw [c] (8.47732,0.686902) -- (8.47732,0.686911);
\draw [c] (8.47732,0.686911) -- (8.47732,0.686919);
\draw [c] (8.46918,0.686911) -- (8.47732,0.686911);
\draw [c] (8.47732,0.686911) -- (8.48545,0.686911);
\definecolor{c}{rgb}{0,0,0};
\colorlet{c}{natcomp!70};
\draw [c] (8.49359,0.686915) -- (8.49359,0.712075);
\draw [c] (8.49359,0.712075) -- (8.49359,0.737235);
\draw [c] (8.48545,0.712075) -- (8.49359,0.712075);
\draw [c] (8.49359,0.712075) -- (8.50173,0.712075);
\definecolor{c}{rgb}{0,0,0};
\colorlet{c}{natcomp!70};
\draw [c] (8.50986,0.686899) -- (8.50986,0.686907);
\draw [c] (8.50986,0.686907) -- (8.50986,0.686915);
\draw [c] (8.50173,0.686907) -- (8.50986,0.686907);
\draw [c] (8.50986,0.686907) -- (8.518,0.686907);
\definecolor{c}{rgb}{0,0,0};
\colorlet{c}{natcomp!70};
\draw [c] (8.52614,0.686912) -- (8.52614,0.694113);
\draw [c] (8.52614,0.694113) -- (8.52614,0.701314);
\draw [c] (8.518,0.694113) -- (8.52614,0.694113);
\draw [c] (8.52614,0.694113) -- (8.53427,0.694113);
\definecolor{c}{rgb}{0,0,0};
\colorlet{c}{natcomp!70};
\draw [c] (8.54241,0.70362) -- (8.54241,0.720773);
\draw [c] (8.54241,0.720773) -- (8.54241,0.737927);
\draw [c] (8.53427,0.720773) -- (8.54241,0.720773);
\draw [c] (8.54241,0.720773) -- (8.55055,0.720773);
\definecolor{c}{rgb}{0,0,0};
\colorlet{c}{natcomp!70};
\draw [c] (8.55868,0.686902) -- (8.55868,0.697697);
\draw [c] (8.55868,0.697697) -- (8.55868,0.708492);
\draw [c] (8.55055,0.697697) -- (8.55868,0.697697);
\draw [c] (8.55868,0.697697) -- (8.56682,0.697697);
\definecolor{c}{rgb}{0,0,0};
\colorlet{c}{natcomp!70};
\draw [c] (8.57495,0.692719) -- (8.57495,0.70709);
\draw [c] (8.57495,0.70709) -- (8.57495,0.721461);
\draw [c] (8.56682,0.70709) -- (8.57495,0.70709);
\draw [c] (8.57495,0.70709) -- (8.58309,0.70709);
\definecolor{c}{rgb}{0,0,0};
\colorlet{c}{natcomp!70};
\draw [c] (8.59123,0.686902) -- (8.59123,0.686909);
\draw [c] (8.59123,0.686909) -- (8.59123,0.686917);
\draw [c] (8.58309,0.686909) -- (8.59123,0.686909);
\draw [c] (8.59123,0.686909) -- (8.59936,0.686909);
\definecolor{c}{rgb}{0,0,0};
\colorlet{c}{natcomp!70};
\draw [c] (8.6075,0.686894) -- (8.6075,0.686897);
\draw [c] (8.6075,0.686897) -- (8.6075,0.686901);
\draw [c] (8.59936,0.686897) -- (8.6075,0.686897);
\draw [c] (8.6075,0.686897) -- (8.61564,0.686897);
\definecolor{c}{rgb}{0,0,0};
\colorlet{c}{natcomp!70};
\draw [c] (8.62377,0.686903) -- (8.62377,0.686912);
\draw [c] (8.62377,0.686912) -- (8.62377,0.686922);
\draw [c] (8.61564,0.686912) -- (8.62377,0.686912);
\draw [c] (8.62377,0.686912) -- (8.63191,0.686912);
\definecolor{c}{rgb}{0,0,0};
\colorlet{c}{natcomp!70};
\draw [c] (8.64005,0.706296) -- (8.64005,0.725795);
\draw [c] (8.64005,0.725795) -- (8.64005,0.745294);
\draw [c] (8.63191,0.725795) -- (8.64005,0.725795);
\draw [c] (8.64005,0.725795) -- (8.64818,0.725795);
\definecolor{c}{rgb}{0,0,0};
\colorlet{c}{natcomp!70};
\draw [c] (8.65632,0.686902) -- (8.65632,0.697676);
\draw [c] (8.65632,0.697676) -- (8.65632,0.708449);
\draw [c] (8.64818,0.697676) -- (8.65632,0.697676);
\draw [c] (8.65632,0.697676) -- (8.66445,0.697676);
\definecolor{c}{rgb}{0,0,0};
\colorlet{c}{natcomp!70};
\draw [c] (8.67259,0.697669) -- (8.67259,0.712756);
\draw [c] (8.67259,0.712756) -- (8.67259,0.727844);
\draw [c] (8.66445,0.712756) -- (8.67259,0.712756);
\draw [c] (8.67259,0.712756) -- (8.68073,0.712756);
\definecolor{c}{rgb}{0,0,0};
\colorlet{c}{natcomp!70};
\draw [c] (8.68886,0.686917) -- (8.68886,0.695056);
\draw [c] (8.68886,0.695056) -- (8.68886,0.703195);
\draw [c] (8.68073,0.695056) -- (8.68886,0.695056);
\draw [c] (8.68886,0.695056) -- (8.697,0.695056);
\definecolor{c}{rgb}{0,0,0};
\colorlet{c}{natcomp!70};
\draw [c] (8.70514,0.686894) -- (8.70514,0.686898);
\draw [c] (8.70514,0.686898) -- (8.70514,0.686902);
\draw [c] (8.697,0.686898) -- (8.70514,0.686898);
\draw [c] (8.70514,0.686898) -- (8.71327,0.686898);
\definecolor{c}{rgb}{0,0,0};
\colorlet{c}{natcomp!70};
\draw [c] (8.72141,0.686896) -- (8.72141,0.686903);
\draw [c] (8.72141,0.686903) -- (8.72141,0.686909);
\draw [c] (8.71327,0.686903) -- (8.72141,0.686903);
\draw [c] (8.72141,0.686903) -- (8.72955,0.686903);
\definecolor{c}{rgb}{0,0,0};
\colorlet{c}{natcomp!70};
\draw [c] (8.73768,0.6869) -- (8.73768,0.686909);
\draw [c] (8.73768,0.686909) -- (8.73768,0.686917);
\draw [c] (8.72955,0.686909) -- (8.73768,0.686909);
\draw [c] (8.73768,0.686909) -- (8.74582,0.686909);
\definecolor{c}{rgb}{0,0,0};
\colorlet{c}{natcomp!70};
\draw [c] (8.75395,0.686898) -- (8.75395,0.697671);
\draw [c] (8.75395,0.697671) -- (8.75395,0.708445);
\draw [c] (8.74582,0.697671) -- (8.75395,0.697671);
\draw [c] (8.75395,0.697671) -- (8.76209,0.697671);
\definecolor{c}{rgb}{0,0,0};
\colorlet{c}{natcomp!70};
\draw [c] (8.77023,0.686896) -- (8.77023,0.686902);
\draw [c] (8.77023,0.686902) -- (8.77023,0.686908);
\draw [c] (8.76209,0.686902) -- (8.77023,0.686902);
\draw [c] (8.77023,0.686902) -- (8.77836,0.686902);
\definecolor{c}{rgb}{0,0,0};
\colorlet{c}{natcomp!70};
\draw [c] (8.80277,0.686894) -- (8.80277,0.697305);
\draw [c] (8.80277,0.697305) -- (8.80277,0.707716);
\draw [c] (8.79464,0.697305) -- (8.80277,0.697305);
\draw [c] (8.80277,0.697305) -- (8.81091,0.697305);
\definecolor{c}{rgb}{0,0,0};
\colorlet{c}{natcomp!70};
\draw [c] (8.81905,0.686894) -- (8.81905,0.686899);
\draw [c] (8.81905,0.686899) -- (8.81905,0.686904);
\draw [c] (8.81091,0.686899) -- (8.81905,0.686899);
\draw [c] (8.81905,0.686899) -- (8.82718,0.686899);
\definecolor{c}{rgb}{0,0,0};
\colorlet{c}{natcomp!70};
\draw [c] (8.83532,0.686894) -- (8.83532,0.686899);
\draw [c] (8.83532,0.686899) -- (8.83532,0.686904);
\draw [c] (8.82718,0.686899) -- (8.83532,0.686899);
\draw [c] (8.83532,0.686899) -- (8.84345,0.686899);
\definecolor{c}{rgb}{0,0,0};
\colorlet{c}{natcomp!70};
\draw [c] (8.85159,0.686907) -- (8.85159,0.696022);
\draw [c] (8.85159,0.696022) -- (8.85159,0.705137);
\draw [c] (8.84345,0.696022) -- (8.85159,0.696022);
\draw [c] (8.85159,0.696022) -- (8.85973,0.696022);
\definecolor{c}{rgb}{0,0,0};
\colorlet{c}{natcomp!70};
\draw [c] (8.86786,0.686894) -- (8.86786,0.686897);
\draw [c] (8.86786,0.686897) -- (8.86786,0.686901);
\draw [c] (8.85973,0.686897) -- (8.86786,0.686897);
\draw [c] (8.86786,0.686897) -- (8.876,0.686897);
\definecolor{c}{rgb}{0,0,0};
\colorlet{c}{natcomp!70};
\draw [c] (8.88414,0.700826) -- (8.88414,0.720637);
\draw [c] (8.88414,0.720637) -- (8.88414,0.740448);
\draw [c] (8.876,0.720637) -- (8.88414,0.720637);
\draw [c] (8.88414,0.720637) -- (8.89227,0.720637);
\definecolor{c}{rgb}{0,0,0};
\colorlet{c}{natcomp!70};
\draw [c] (8.90041,0.686894) -- (8.90041,0.686897);
\draw [c] (8.90041,0.686897) -- (8.90041,0.686901);
\draw [c] (8.89227,0.686897) -- (8.90041,0.686897);
\draw [c] (8.90041,0.686897) -- (8.90855,0.686897);
\definecolor{c}{rgb}{0,0,0};
\colorlet{c}{natcomp!70};
\draw [c] (8.91668,0.722516) -- (8.91668,0.755184);
\draw [c] (8.91668,0.755184) -- (8.91668,0.787852);
\draw [c] (8.90855,0.755184) -- (8.91668,0.755184);
\draw [c] (8.91668,0.755184) -- (8.92482,0.755184);
\definecolor{c}{rgb}{0,0,0};
\colorlet{c}{natcomp!70};
\draw [c] (8.93295,0.6869) -- (8.93295,0.694622);
\draw [c] (8.93295,0.694622) -- (8.93295,0.702344);
\draw [c] (8.92482,0.694622) -- (8.93295,0.694622);
\draw [c] (8.93295,0.694622) -- (8.94109,0.694622);
\definecolor{c}{rgb}{0,0,0};
\colorlet{c}{natcomp!70};
\draw [c] (8.94923,0.686894) -- (8.94923,0.686898);
\draw [c] (8.94923,0.686898) -- (8.94923,0.686902);
\draw [c] (8.94109,0.686898) -- (8.94923,0.686898);
\draw [c] (8.94923,0.686898) -- (8.95736,0.686898);
\definecolor{c}{rgb}{0,0,0};
\colorlet{c}{natcomp!70};
\draw [c] (8.9655,0.686894) -- (8.9655,0.686898);
\draw [c] (8.9655,0.686898) -- (8.9655,0.686902);
\draw [c] (8.95736,0.686898) -- (8.9655,0.686898);
\draw [c] (8.9655,0.686898) -- (8.97364,0.686898);
\definecolor{c}{rgb}{0,0,0};
\colorlet{c}{natcomp!70};
\draw [c] (8.98177,0.686896) -- (8.98177,0.686902);
\draw [c] (8.98177,0.686902) -- (8.98177,0.686907);
\draw [c] (8.97364,0.686902) -- (8.98177,0.686902);
\draw [c] (8.98177,0.686902) -- (8.98991,0.686902);
\definecolor{c}{rgb}{0,0,0};
\colorlet{c}{natcomp!70};
\draw [c] (8.99805,0.686904) -- (8.99805,0.686922);
\draw [c] (8.99805,0.686922) -- (8.99805,0.686939);
\draw [c] (8.98991,0.686922) -- (8.99805,0.686922);
\draw [c] (8.99805,0.686922) -- (9.00618,0.686922);
\definecolor{c}{rgb}{0,0,0};
\colorlet{c}{natcomp!70};
\draw [c] (9.01432,0.686897) -- (9.01432,0.686903);
\draw [c] (9.01432,0.686903) -- (9.01432,0.68691);
\draw [c] (9.00618,0.686903) -- (9.01432,0.686903);
\draw [c] (9.01432,0.686903) -- (9.02245,0.686903);
\definecolor{c}{rgb}{0,0,0};
\colorlet{c}{natcomp!70};
\draw [c] (9.04686,0.708993) -- (9.04686,0.731658);
\draw [c] (9.04686,0.731658) -- (9.04686,0.754323);
\draw [c] (9.03873,0.731658) -- (9.04686,0.731658);
\draw [c] (9.04686,0.731658) -- (9.055,0.731658);
\definecolor{c}{rgb}{0,0,0};
\colorlet{c}{natcomp!70};
\draw [c] (9.07941,0.692085) -- (9.07941,0.704618);
\draw [c] (9.07941,0.704618) -- (9.07941,0.717151);
\draw [c] (9.07127,0.704618) -- (9.07941,0.704618);
\draw [c] (9.07941,0.704618) -- (9.08755,0.704618);
\definecolor{c}{rgb}{0,0,0};
\colorlet{c}{natcomp!70};
\draw [c] (9.09568,0.686897) -- (9.09568,0.695861);
\draw [c] (9.09568,0.695861) -- (9.09568,0.704825);
\draw [c] (9.08755,0.695861) -- (9.09568,0.695861);
\draw [c] (9.09568,0.695861) -- (9.10382,0.695861);
\definecolor{c}{rgb}{0,0,0};
\colorlet{c}{natcomp!70};
\draw [c] (9.12823,0.686894) -- (9.12823,0.686897);
\draw [c] (9.12823,0.686897) -- (9.12823,0.686901);
\draw [c] (9.12009,0.686897) -- (9.12823,0.686897);
\draw [c] (9.12823,0.686897) -- (9.13636,0.686897);
\definecolor{c}{rgb}{0,0,0};
\colorlet{c}{natcomp!70};
\draw [c] (9.1445,0.686899) -- (9.1445,0.696014);
\draw [c] (9.1445,0.696014) -- (9.1445,0.70513);
\draw [c] (9.13636,0.696014) -- (9.1445,0.696014);
\draw [c] (9.1445,0.696014) -- (9.15264,0.696014);
\definecolor{c}{rgb}{0,0,0};
\colorlet{c}{natcomp!70};
\draw [c] (9.16077,0.686897) -- (9.16077,0.686903);
\draw [c] (9.16077,0.686903) -- (9.16077,0.68691);
\draw [c] (9.15264,0.686903) -- (9.16077,0.686903);
\draw [c] (9.16077,0.686903) -- (9.16891,0.686903);
\definecolor{c}{rgb}{0,0,0};
\colorlet{c}{natcomp!70};
\draw [c] (9.17705,0.686894) -- (9.17705,0.705403);
\draw [c] (9.17705,0.705403) -- (9.17705,0.723913);
\draw [c] (9.16891,0.705403) -- (9.17705,0.705403);
\draw [c] (9.17705,0.705403) -- (9.18518,0.705403);
\definecolor{c}{rgb}{0,0,0};
\colorlet{c}{natcomp!70};
\draw [c] (9.19332,0.686897) -- (9.19332,0.686906);
\draw [c] (9.19332,0.686906) -- (9.19332,0.686914);
\draw [c] (9.18518,0.686906) -- (9.19332,0.686906);
\draw [c] (9.19332,0.686906) -- (9.20145,0.686906);
\definecolor{c}{rgb}{0,0,0};
\colorlet{c}{natcomp!70};
\draw [c] (9.20959,0.686906) -- (9.20959,0.694628);
\draw [c] (9.20959,0.694628) -- (9.20959,0.70235);
\draw [c] (9.20145,0.694628) -- (9.20959,0.694628);
\draw [c] (9.20959,0.694628) -- (9.21773,0.694628);
\definecolor{c}{rgb}{0,0,0};
\colorlet{c}{natcomp!70};
\draw [c] (9.22586,0.686898) -- (9.22586,0.697671);
\draw [c] (9.22586,0.697671) -- (9.22586,0.708444);
\draw [c] (9.21773,0.697671) -- (9.22586,0.697671);
\draw [c] (9.22586,0.697671) -- (9.234,0.697671);
\definecolor{c}{rgb}{0,0,0};
\colorlet{c}{natcomp!70};
\draw [c] (9.24214,0.697919) -- (9.24214,0.713014);
\draw [c] (9.24214,0.713014) -- (9.24214,0.72811);
\draw [c] (9.234,0.713014) -- (9.24214,0.713014);
\draw [c] (9.24214,0.713014) -- (9.25027,0.713014);
\definecolor{c}{rgb}{0,0,0};
\colorlet{c}{natcomp!70};
\draw [c] (9.25841,0.686896) -- (9.25841,0.686902);
\draw [c] (9.25841,0.686902) -- (9.25841,0.686908);
\draw [c] (9.25027,0.686902) -- (9.25841,0.686902);
\draw [c] (9.25841,0.686902) -- (9.26655,0.686902);
\definecolor{c}{rgb}{0,0,0};
\colorlet{c}{natcomp!70};
\draw [c] (9.27468,0.686897) -- (9.27468,0.686903);
\draw [c] (9.27468,0.686903) -- (9.27468,0.68691);
\draw [c] (9.26655,0.686903) -- (9.27468,0.686903);
\draw [c] (9.27468,0.686903) -- (9.28282,0.686903);
\definecolor{c}{rgb}{0,0,0};
\colorlet{c}{natcomp!70};
\draw [c] (9.30723,0.691421) -- (9.30723,0.702341);
\draw [c] (9.30723,0.702341) -- (9.30723,0.713261);
\draw [c] (9.29909,0.702341) -- (9.30723,0.702341);
\draw [c] (9.30723,0.702341) -- (9.31536,0.702341);
\definecolor{c}{rgb}{0,0,0};
\colorlet{c}{natcomp!70};
\draw [c] (9.33977,0.686902) -- (9.33977,0.69504);
\draw [c] (9.33977,0.69504) -- (9.33977,0.703179);
\draw [c] (9.33164,0.69504) -- (9.33977,0.69504);
\draw [c] (9.33977,0.69504) -- (9.34791,0.69504);
\definecolor{c}{rgb}{0,0,0};
\colorlet{c}{natcomp!70};
\draw [c] (9.35605,0.693227) -- (9.35605,0.708478);
\draw [c] (9.35605,0.708478) -- (9.35605,0.723729);
\draw [c] (9.34791,0.708478) -- (9.35605,0.708478);
\draw [c] (9.35605,0.708478) -- (9.36418,0.708478);
\definecolor{c}{rgb}{0,0,0};
\colorlet{c}{natcomp!70};
\draw [c] (9.37232,0.686896) -- (9.37232,0.686902);
\draw [c] (9.37232,0.686902) -- (9.37232,0.686908);
\draw [c] (9.36418,0.686902) -- (9.37232,0.686902);
\draw [c] (9.37232,0.686902) -- (9.38046,0.686902);
\definecolor{c}{rgb}{0,0,0};
\colorlet{c}{natcomp!70};
\draw [c] (9.38859,0.686897) -- (9.38859,0.696013);
\draw [c] (9.38859,0.696013) -- (9.38859,0.705128);
\draw [c] (9.38046,0.696013) -- (9.38859,0.696013);
\draw [c] (9.38859,0.696013) -- (9.39673,0.696013);
\definecolor{c}{rgb}{0,0,0};
\colorlet{c}{natcomp!70};
\draw [c] (9.43741,0.686902) -- (9.43741,0.697312);
\draw [c] (9.43741,0.697312) -- (9.43741,0.707723);
\draw [c] (9.42927,0.697312) -- (9.43741,0.697312);
\draw [c] (9.43741,0.697312) -- (9.44555,0.697312);
\definecolor{c}{rgb}{0,0,0};
\colorlet{c}{natcomp!70};
\draw [c] (9.45368,0.686897) -- (9.45368,0.686903);
\draw [c] (9.45368,0.686903) -- (9.45368,0.68691);
\draw [c] (9.44555,0.686903) -- (9.45368,0.686903);
\draw [c] (9.45368,0.686903) -- (9.46182,0.686903);
\definecolor{c}{rgb}{0,0,0};
\colorlet{c}{natcomp!70};
\draw [c] (9.46995,0.692681) -- (9.46995,0.706988);
\draw [c] (9.46995,0.706988) -- (9.46995,0.721296);
\draw [c] (9.46182,0.706988) -- (9.46995,0.706988);
\draw [c] (9.46995,0.706988) -- (9.47809,0.706988);
\definecolor{c}{rgb}{0,0,0};
\colorlet{c}{natcomp!70};
\draw [c] (9.48623,0.686894) -- (9.48623,0.686903);
\draw [c] (9.48623,0.686903) -- (9.48623,0.686911);
\draw [c] (9.47809,0.686903) -- (9.48623,0.686903);
\draw [c] (9.48623,0.686903) -- (9.49436,0.686903);
\definecolor{c}{rgb}{0,0,0};
\colorlet{c}{natcomp!70};
\draw [c] (9.5025,0.686898) -- (9.5025,0.696793);
\draw [c] (9.5025,0.696793) -- (9.5025,0.706687);
\draw [c] (9.49436,0.696793) -- (9.5025,0.696793);
\draw [c] (9.5025,0.696793) -- (9.51064,0.696793);
\definecolor{c}{rgb}{0,0,0};
\colorlet{c}{natcomp!70};
\draw [c] (9.51877,0.686894) -- (9.51877,0.697689);
\draw [c] (9.51877,0.697689) -- (9.51877,0.708483);
\draw [c] (9.51064,0.697689) -- (9.51877,0.697689);
\draw [c] (9.51877,0.697689) -- (9.52691,0.697689);
\definecolor{c}{rgb}{0,0,0};
\colorlet{c}{natcomp!70};
\draw [c] (9.53505,0.686896) -- (9.53505,0.686901);
\draw [c] (9.53505,0.686901) -- (9.53505,0.686906);
\draw [c] (9.52691,0.686901) -- (9.53505,0.686901);
\draw [c] (9.53505,0.686901) -- (9.54318,0.686901);
\definecolor{c}{rgb}{0,0,0};
\colorlet{c}{natcomp!70};
\draw [c] (9.55132,0.686896) -- (9.55132,0.686902);
\draw [c] (9.55132,0.686902) -- (9.55132,0.686908);
\draw [c] (9.54318,0.686902) -- (9.55132,0.686902);
\draw [c] (9.55132,0.686902) -- (9.55945,0.686902);
\definecolor{c}{rgb}{0,0,0};
\colorlet{c}{natcomp!70};
\draw [c] (9.56759,0.686894) -- (9.56759,0.696788);
\draw [c] (9.56759,0.696788) -- (9.56759,0.706683);
\draw [c] (9.55945,0.696788) -- (9.56759,0.696788);
\draw [c] (9.56759,0.696788) -- (9.57573,0.696788);
\definecolor{c}{rgb}{0,0,0};
\colorlet{c}{natcomp!70};
\draw [c] (9.58386,0.686894) -- (9.58386,0.686898);
\draw [c] (9.58386,0.686898) -- (9.58386,0.686902);
\draw [c] (9.57573,0.686898) -- (9.58386,0.686898);
\draw [c] (9.58386,0.686898) -- (9.592,0.686898);
\definecolor{c}{rgb}{0,0,0};
\colorlet{c}{natcomp!70};
\draw [c] (9.60014,0.686899) -- (9.60014,0.695761);
\draw [c] (9.60014,0.695761) -- (9.60014,0.704623);
\draw [c] (9.592,0.695761) -- (9.60014,0.695761);
\draw [c] (9.60014,0.695761) -- (9.60827,0.695761);
\definecolor{c}{rgb}{0,0,0};
\colorlet{c}{natcomp!70};
\draw [c] (9.61641,0.691959) -- (9.61641,0.70451);
\draw [c] (9.61641,0.70451) -- (9.61641,0.717061);
\draw [c] (9.60827,0.70451) -- (9.61641,0.70451);
\draw [c] (9.61641,0.70451) -- (9.62455,0.70451);
\definecolor{c}{rgb}{0,0,0};
\colorlet{c}{natcomp!70};
\draw [c] (9.64895,0.686899) -- (9.64895,0.686907);
\draw [c] (9.64895,0.686907) -- (9.64895,0.686914);
\draw [c] (9.64082,0.686907) -- (9.64895,0.686907);
\draw [c] (9.64895,0.686907) -- (9.65709,0.686907);
\definecolor{c}{rgb}{0,0,0};
\colorlet{c}{natcomp!70};
\draw [c] (9.66523,0.686899) -- (9.66523,0.686907);
\draw [c] (9.66523,0.686907) -- (9.66523,0.686914);
\draw [c] (9.65709,0.686907) -- (9.66523,0.686907);
\draw [c] (9.66523,0.686907) -- (9.67336,0.686907);
\definecolor{c}{rgb}{0,0,0};
\colorlet{c}{natcomp!70};
\draw [c] (9.6815,0.686894) -- (9.6815,0.686897);
\draw [c] (9.6815,0.686897) -- (9.6815,0.686901);
\draw [c] (9.67336,0.686897) -- (9.6815,0.686897);
\draw [c] (9.6815,0.686897) -- (9.68964,0.686897);
\definecolor{c}{rgb}{0,0,0};
\colorlet{c}{natcomp!70};
\draw [c] (9.71405,0.692495) -- (9.71405,0.706167);
\draw [c] (9.71405,0.706167) -- (9.71405,0.719839);
\draw [c] (9.70591,0.706167) -- (9.71405,0.706167);
\draw [c] (9.71405,0.706167) -- (9.72218,0.706167);
\definecolor{c}{rgb}{0,0,0};
\colorlet{c}{natcomp!70};
\draw [c] (9.74659,0.686894) -- (9.74659,0.686897);
\draw [c] (9.74659,0.686897) -- (9.74659,0.686901);
\draw [c] (9.73845,0.686897) -- (9.74659,0.686897);
\draw [c] (9.74659,0.686897) -- (9.75473,0.686897);
\definecolor{c}{rgb}{0,0,0};
\colorlet{c}{natcomp!70};
\draw [c] (9.76286,0.686896) -- (9.76286,0.686901);
\draw [c] (9.76286,0.686901) -- (9.76286,0.686906);
\draw [c] (9.75473,0.686901) -- (9.76286,0.686901);
\draw [c] (9.76286,0.686901) -- (9.771,0.686901);
\definecolor{c}{rgb}{0,0,0};
\colorlet{c}{natcomp!70};
\draw [c] (9.77914,0.686896) -- (9.77914,0.686902);
\draw [c] (9.77914,0.686902) -- (9.77914,0.686908);
\draw [c] (9.771,0.686902) -- (9.77914,0.686902);
\draw [c] (9.77914,0.686902) -- (9.78727,0.686902);
\definecolor{c}{rgb}{0,0,0};
\colorlet{c}{natcomp!70};
\draw [c] (9.79541,0.686898) -- (9.79541,0.696014);
\draw [c] (9.79541,0.696014) -- (9.79541,0.705129);
\draw [c] (9.78727,0.696014) -- (9.79541,0.696014);
\draw [c] (9.79541,0.696014) -- (9.80354,0.696014);
\definecolor{c}{rgb}{0,0,0};
\colorlet{c}{natcomp!70};
\draw [c] (9.81168,0.686894) -- (9.81168,0.686899);
\draw [c] (9.81168,0.686899) -- (9.81168,0.686903);
\draw [c] (9.80354,0.686899) -- (9.81168,0.686899);
\draw [c] (9.81168,0.686899) -- (9.81982,0.686899);
\definecolor{c}{rgb}{0,0,0};
\colorlet{c}{natcomp!70};
\draw [c] (9.82795,0.692367) -- (9.82795,0.70565);
\draw [c] (9.82795,0.70565) -- (9.82795,0.718933);
\draw [c] (9.81982,0.70565) -- (9.82795,0.70565);
\draw [c] (9.82795,0.70565) -- (9.83609,0.70565);
\definecolor{c}{rgb}{0,0,0};
\colorlet{c}{natcomp!70};
\draw [c] (9.84423,0.686894) -- (9.84423,0.686898);
\draw [c] (9.84423,0.686898) -- (9.84423,0.686902);
\draw [c] (9.83609,0.686898) -- (9.84423,0.686898);
\draw [c] (9.84423,0.686898) -- (9.85236,0.686898);
\definecolor{c}{rgb}{0,0,0};
\colorlet{c}{natcomp!70};
\draw [c] (9.8605,0.686894) -- (9.8605,0.686901);
\draw [c] (9.8605,0.686901) -- (9.8605,0.686908);
\draw [c] (9.85236,0.686901) -- (9.8605,0.686901);
\draw [c] (9.8605,0.686901) -- (9.86864,0.686901);
\definecolor{c}{rgb}{0,0,0};
\colorlet{c}{natcomp!70};
\draw [c] (9.87677,0.686894) -- (9.87677,0.686899);
\draw [c] (9.87677,0.686899) -- (9.87677,0.686903);
\draw [c] (9.86864,0.686899) -- (9.87677,0.686899);
\draw [c] (9.87677,0.686899) -- (9.88491,0.686899);
\definecolor{c}{rgb}{0,0,0};
\colorlet{c}{natcomp!70};
\draw [c] (9.89305,0.691538) -- (9.89305,0.702957);
\draw [c] (9.89305,0.702957) -- (9.89305,0.714375);
\draw [c] (9.88491,0.702957) -- (9.89305,0.702957);
\draw [c] (9.89305,0.702957) -- (9.90118,0.702957);
\definecolor{c}{rgb}{0,0,0};
\colorlet{c}{natcomp!70};
\draw [c] (9.90932,0.692367) -- (9.90932,0.70565);
\draw [c] (9.90932,0.70565) -- (9.90932,0.718933);
\draw [c] (9.90118,0.70565) -- (9.90932,0.70565);
\draw [c] (9.90932,0.70565) -- (9.91745,0.70565);
\definecolor{c}{rgb}{0,0,0};
\colorlet{c}{natcomp!70};
\draw [c] (9.94186,0.686899) -- (9.94186,0.686907);
\draw [c] (9.94186,0.686907) -- (9.94186,0.686915);
\draw [c] (9.93373,0.686907) -- (9.94186,0.686907);
\draw [c] (9.94186,0.686907) -- (9.95,0.686907);
\definecolor{c}{rgb}{0,0,0};
\draw [anchor=base west] (6.62249,6.01827) node[color=c, rotate=0]{ALTAS MC};
\colorlet{c}{natgreen};
\draw [c] (5.87661,6.12464) -- (6.49087,6.12464);
\draw [c] (6.18374,5.98281) -- (6.18374,6.26648);
\definecolor{c}{rgb}{0,0,0};
\draw [anchor=base west] (6.62249,5.54549) node[color=c, rotate=0]{CalcHEP MC};
\colorlet{c}{natcomp!70};
\draw [c] (5.87661,5.65186) -- (6.49087,5.65186);
\draw [c] (6.18374,5.51003) -- (6.18374,5.7937);
\end{tikzpicture}

\end{infilsf}
\end{minipage}
\hfill
\begin{minipage}[b]{.49\textwidth}
\begin{infilsf} \tiny 
\begin{tikzpicture}[x=.092\textwidth,y=.092\textwidth]
\pgfdeclareplotmark{cross} {
\pgfpathmoveto{\pgfpoint{-0.3\pgfplotmarksize}{\pgfplotmarksize}}
\pgfpathlineto{\pgfpoint{+0.3\pgfplotmarksize}{\pgfplotmarksize}}
\pgfpathlineto{\pgfpoint{+0.3\pgfplotmarksize}{0.3\pgfplotmarksize}}
\pgfpathlineto{\pgfpoint{+1\pgfplotmarksize}{0.3\pgfplotmarksize}}
\pgfpathlineto{\pgfpoint{+1\pgfplotmarksize}{-0.3\pgfplotmarksize}}
\pgfpathlineto{\pgfpoint{+0.3\pgfplotmarksize}{-0.3\pgfplotmarksize}}
\pgfpathlineto{\pgfpoint{+0.3\pgfplotmarksize}{-1.\pgfplotmarksize}}
\pgfpathlineto{\pgfpoint{-0.3\pgfplotmarksize}{-1.\pgfplotmarksize}}
\pgfpathlineto{\pgfpoint{-0.3\pgfplotmarksize}{-0.3\pgfplotmarksize}}
\pgfpathlineto{\pgfpoint{-1.\pgfplotmarksize}{-0.3\pgfplotmarksize}}
\pgfpathlineto{\pgfpoint{-1.\pgfplotmarksize}{0.3\pgfplotmarksize}}
\pgfpathlineto{\pgfpoint{-0.3\pgfplotmarksize}{0.3\pgfplotmarksize}}
\pgfpathclose
\pgfusepathqstroke
}
\pgfdeclareplotmark{cross*} {
\pgfpathmoveto{\pgfpoint{-0.3\pgfplotmarksize}{\pgfplotmarksize}}
\pgfpathlineto{\pgfpoint{+0.3\pgfplotmarksize}{\pgfplotmarksize}}
\pgfpathlineto{\pgfpoint{+0.3\pgfplotmarksize}{0.3\pgfplotmarksize}}
\pgfpathlineto{\pgfpoint{+1\pgfplotmarksize}{0.3\pgfplotmarksize}}
\pgfpathlineto{\pgfpoint{+1\pgfplotmarksize}{-0.3\pgfplotmarksize}}
\pgfpathlineto{\pgfpoint{+0.3\pgfplotmarksize}{-0.3\pgfplotmarksize}}
\pgfpathlineto{\pgfpoint{+0.3\pgfplotmarksize}{-1.\pgfplotmarksize}}
\pgfpathlineto{\pgfpoint{-0.3\pgfplotmarksize}{-1.\pgfplotmarksize}}
\pgfpathlineto{\pgfpoint{-0.3\pgfplotmarksize}{-0.3\pgfplotmarksize}}
\pgfpathlineto{\pgfpoint{-1.\pgfplotmarksize}{-0.3\pgfplotmarksize}}
\pgfpathlineto{\pgfpoint{-1.\pgfplotmarksize}{0.3\pgfplotmarksize}}
\pgfpathlineto{\pgfpoint{-0.3\pgfplotmarksize}{0.3\pgfplotmarksize}}
\pgfpathclose
\pgfusepathqfillstroke
}
\pgfdeclareplotmark{newstar} {
\pgfpathmoveto{\pgfqpoint{0pt}{\pgfplotmarksize}}
\pgfpathlineto{\pgfqpointpolar{44}{0.5\pgfplotmarksize}}
\pgfpathlineto{\pgfqpointpolar{18}{\pgfplotmarksize}}
\pgfpathlineto{\pgfqpointpolar{-20}{0.5\pgfplotmarksize}}
\pgfpathlineto{\pgfqpointpolar{-54}{\pgfplotmarksize}}
\pgfpathlineto{\pgfqpointpolar{-90}{0.5\pgfplotmarksize}}
\pgfpathlineto{\pgfqpointpolar{234}{\pgfplotmarksize}}
\pgfpathlineto{\pgfqpointpolar{198}{0.5\pgfplotmarksize}}
\pgfpathlineto{\pgfqpointpolar{162}{\pgfplotmarksize}}
\pgfpathlineto{\pgfqpointpolar{134}{0.5\pgfplotmarksize}}
\pgfpathclose
\pgfusepathqstroke
}
\pgfdeclareplotmark{newstar*} {
\pgfpathmoveto{\pgfqpoint{0pt}{\pgfplotmarksize}}
\pgfpathlineto{\pgfqpointpolar{44}{0.5\pgfplotmarksize}}
\pgfpathlineto{\pgfqpointpolar{18}{\pgfplotmarksize}}
\pgfpathlineto{\pgfqpointpolar{-20}{0.5\pgfplotmarksize}}
\pgfpathlineto{\pgfqpointpolar{-54}{\pgfplotmarksize}}
\pgfpathlineto{\pgfqpointpolar{-90}{0.5\pgfplotmarksize}}
\pgfpathlineto{\pgfqpointpolar{234}{\pgfplotmarksize}}
\pgfpathlineto{\pgfqpointpolar{198}{0.5\pgfplotmarksize}}
\pgfpathlineto{\pgfqpointpolar{162}{\pgfplotmarksize}}
\pgfpathlineto{\pgfqpointpolar{134}{0.5\pgfplotmarksize}}
\pgfpathclose
\pgfusepathqfillstroke
}
\definecolor{c}{rgb}{1,1,1};
\draw [color=c, fill=c] (0,0) rectangle (10,6.80516);
\draw [color=c, fill=c] (1,0.680516) rectangle (9.95,6.73711);
\definecolor{c}{rgb}{0,0,0};
\draw [c] (1,0.680516) -- (1,6.73711) -- (9.95,6.73711) -- (9.95,0.680516) -- (1,0.680516);
\definecolor{c}{rgb}{1,1,1};
\draw [color=c, fill=c] (1,0.680516) rectangle (9.95,6.73711);
\definecolor{c}{rgb}{0,0,0};
\draw [c] (1,0.680516) -- (1,6.73711) -- (9.95,6.73711) -- (9.95,0.680516) -- (1,0.680516);
\colorlet{c}{natgreen};
\draw [c] (1.09022,0.740799) -- (1.09022,0.753375);
\draw [c] (1.09022,0.753375) -- (1.09022,0.76595);
\draw [c] (1.07218,0.753375) -- (1.09022,0.753375);
\draw [c] (1.09022,0.753375) -- (1.10827,0.753375);
\definecolor{c}{rgb}{0,0,0};
\colorlet{c}{natgreen};
\draw [c] (1.12631,0.740369) -- (1.12631,0.752314);
\draw [c] (1.12631,0.752314) -- (1.12631,0.76426);
\draw [c] (1.10827,0.752314) -- (1.12631,0.752314);
\draw [c] (1.12631,0.752314) -- (1.14435,0.752314);
\definecolor{c}{rgb}{0,0,0};
\colorlet{c}{natgreen};
\draw [c] (1.30675,0.737608) -- (1.30675,0.746848);
\draw [c] (1.30675,0.746848) -- (1.30675,0.756087);
\draw [c] (1.28871,0.746848) -- (1.30675,0.746848);
\draw [c] (1.30675,0.746848) -- (1.3248,0.746848);
\definecolor{c}{rgb}{0,0,0};
\colorlet{c}{natgreen};
\draw [c] (1.37893,0.755182) -- (1.37893,0.777154);
\draw [c] (1.37893,0.777154) -- (1.37893,0.799127);
\draw [c] (1.36089,0.777154) -- (1.37893,0.777154);
\draw [c] (1.37893,0.777154) -- (1.39698,0.777154);
\definecolor{c}{rgb}{0,0,0};
\colorlet{c}{natgreen};
\draw [c] (1.41502,0.755947) -- (1.41502,0.779138);
\draw [c] (1.41502,0.779138) -- (1.41502,0.802329);
\draw [c] (1.39698,0.779138) -- (1.41502,0.779138);
\draw [c] (1.41502,0.779138) -- (1.43306,0.779138);
\definecolor{c}{rgb}{0,0,0};
\colorlet{c}{natgreen};
\draw [c] (1.45111,0.768124) -- (1.45111,0.791905);
\draw [c] (1.45111,0.791905) -- (1.45111,0.815686);
\draw [c] (1.43306,0.791905) -- (1.45111,0.791905);
\draw [c] (1.45111,0.791905) -- (1.46915,0.791905);
\definecolor{c}{rgb}{0,0,0};
\colorlet{c}{natgreen};
\draw [c] (1.4872,0.758252) -- (1.4872,0.782938);
\draw [c] (1.4872,0.782938) -- (1.4872,0.807624);
\draw [c] (1.46915,0.782938) -- (1.4872,0.782938);
\draw [c] (1.4872,0.782938) -- (1.50524,0.782938);
\definecolor{c}{rgb}{0,0,0};
\colorlet{c}{natgreen};
\draw [c] (1.52329,0.78146) -- (1.52329,0.808304);
\draw [c] (1.52329,0.808304) -- (1.52329,0.835148);
\draw [c] (1.50524,0.808304) -- (1.52329,0.808304);
\draw [c] (1.52329,0.808304) -- (1.54133,0.808304);
\definecolor{c}{rgb}{0,0,0};
\colorlet{c}{natgreen};
\draw [c] (1.55938,0.77573) -- (1.55938,0.802259);
\draw [c] (1.55938,0.802259) -- (1.55938,0.828788);
\draw [c] (1.54133,0.802259) -- (1.55938,0.802259);
\draw [c] (1.55938,0.802259) -- (1.57742,0.802259);
\definecolor{c}{rgb}{0,0,0};
\colorlet{c}{natgreen};
\draw [c] (1.59546,0.761209) -- (1.59546,0.784068);
\draw [c] (1.59546,0.784068) -- (1.59546,0.806926);
\draw [c] (1.57742,0.784068) -- (1.59546,0.784068);
\draw [c] (1.59546,0.784068) -- (1.61351,0.784068);
\definecolor{c}{rgb}{0,0,0};
\colorlet{c}{natgreen};
\draw [c] (1.63155,0.802522) -- (1.63155,0.838187);
\draw [c] (1.63155,0.838187) -- (1.63155,0.873852);
\draw [c] (1.61351,0.838187) -- (1.63155,0.838187);
\draw [c] (1.63155,0.838187) -- (1.6496,0.838187);
\definecolor{c}{rgb}{0,0,0};
\colorlet{c}{natgreen};
\draw [c] (1.66764,0.829492) -- (1.66764,0.865601);
\draw [c] (1.66764,0.865601) -- (1.66764,0.901709);
\draw [c] (1.6496,0.865601) -- (1.66764,0.865601);
\draw [c] (1.66764,0.865601) -- (1.68569,0.865601);
\definecolor{c}{rgb}{0,0,0};
\colorlet{c}{natgreen};
\draw [c] (1.70373,0.80146) -- (1.70373,0.833741);
\draw [c] (1.70373,0.833741) -- (1.70373,0.866023);
\draw [c] (1.68569,0.833741) -- (1.70373,0.833741);
\draw [c] (1.70373,0.833741) -- (1.72177,0.833741);
\definecolor{c}{rgb}{0,0,0};
\colorlet{c}{natgreen};
\draw [c] (1.73982,0.803179) -- (1.73982,0.835265);
\draw [c] (1.73982,0.835265) -- (1.73982,0.867351);
\draw [c] (1.72177,0.835265) -- (1.73982,0.835265);
\draw [c] (1.73982,0.835265) -- (1.75786,0.835265);
\definecolor{c}{rgb}{0,0,0};
\colorlet{c}{natgreen};
\draw [c] (1.77591,0.898175) -- (1.77591,0.946579);
\draw [c] (1.77591,0.946579) -- (1.77591,0.994983);
\draw [c] (1.75786,0.946579) -- (1.77591,0.946579);
\draw [c] (1.77591,0.946579) -- (1.79395,0.946579);
\definecolor{c}{rgb}{0,0,0};
\colorlet{c}{natgreen};
\draw [c] (1.812,0.953061) -- (1.812,1.00914);
\draw [c] (1.812,1.00914) -- (1.812,1.06522);
\draw [c] (1.79395,1.00914) -- (1.812,1.00914);
\draw [c] (1.812,1.00914) -- (1.83004,1.00914);
\definecolor{c}{rgb}{0,0,0};
\colorlet{c}{natgreen};
\draw [c] (1.84808,0.939044) -- (1.84808,0.99126);
\draw [c] (1.84808,0.99126) -- (1.84808,1.04348);
\draw [c] (1.83004,0.99126) -- (1.84808,0.99126);
\draw [c] (1.84808,0.99126) -- (1.86613,0.99126);
\definecolor{c}{rgb}{0,0,0};
\colorlet{c}{natgreen};
\draw [c] (1.88417,1.08459) -- (1.88417,1.15614);
\draw [c] (1.88417,1.15614) -- (1.88417,1.22769);
\draw [c] (1.86613,1.15614) -- (1.88417,1.15614);
\draw [c] (1.88417,1.15614) -- (1.90222,1.15614);
\definecolor{c}{rgb}{0,0,0};
\colorlet{c}{natgreen};
\draw [c] (1.92026,1.07599) -- (1.92026,1.14389);
\draw [c] (1.92026,1.14389) -- (1.92026,1.21179);
\draw [c] (1.90222,1.14389) -- (1.92026,1.14389);
\draw [c] (1.92026,1.14389) -- (1.93831,1.14389);
\definecolor{c}{rgb}{0,0,0};
\colorlet{c}{natgreen};
\draw [c] (1.95635,1.20977) -- (1.95635,1.28799);
\draw [c] (1.95635,1.28799) -- (1.95635,1.36621);
\draw [c] (1.93831,1.28799) -- (1.95635,1.28799);
\draw [c] (1.95635,1.28799) -- (1.9744,1.28799);
\definecolor{c}{rgb}{0,0,0};
\colorlet{c}{natgreen};
\draw [c] (1.99244,1.34676) -- (1.99244,1.43945);
\draw [c] (1.99244,1.43945) -- (1.99244,1.53215);
\draw [c] (1.9744,1.43945) -- (1.99244,1.43945);
\draw [c] (1.99244,1.43945) -- (2.01048,1.43945);
\definecolor{c}{rgb}{0,0,0};
\colorlet{c}{natgreen};
\draw [c] (2.02853,1.38683) -- (2.02853,1.48215);
\draw [c] (2.02853,1.48215) -- (2.02853,1.57747);
\draw [c] (2.01048,1.48215) -- (2.02853,1.48215);
\draw [c] (2.02853,1.48215) -- (2.04657,1.48215);
\definecolor{c}{rgb}{0,0,0};
\colorlet{c}{natgreen};
\draw [c] (2.06462,1.59757) -- (2.06462,1.70779);
\draw [c] (2.06462,1.70779) -- (2.06462,1.81801);
\draw [c] (2.04657,1.70779) -- (2.06462,1.70779);
\draw [c] (2.06462,1.70779) -- (2.08266,1.70779);
\definecolor{c}{rgb}{0,0,0};
\colorlet{c}{natgreen};
\draw [c] (2.10071,1.84589) -- (2.10071,1.96852);
\draw [c] (2.10071,1.96852) -- (2.10071,2.09115);
\draw [c] (2.08266,1.96852) -- (2.10071,1.96852);
\draw [c] (2.10071,1.96852) -- (2.11875,1.96852);
\definecolor{c}{rgb}{0,0,0};
\colorlet{c}{natgreen};
\draw [c] (2.13679,1.88974) -- (2.13679,2.01623);
\draw [c] (2.13679,2.01623) -- (2.13679,2.14272);
\draw [c] (2.11875,2.01623) -- (2.13679,2.01623);
\draw [c] (2.13679,2.01623) -- (2.15484,2.01623);
\definecolor{c}{rgb}{0,0,0};
\colorlet{c}{natgreen};
\draw [c] (2.17288,2.27127) -- (2.17288,2.41567);
\draw [c] (2.17288,2.41567) -- (2.17288,2.56008);
\draw [c] (2.15484,2.41567) -- (2.17288,2.41567);
\draw [c] (2.17288,2.41567) -- (2.19093,2.41567);
\definecolor{c}{rgb}{0,0,0};
\colorlet{c}{natgreen};
\draw [c] (2.20897,2.40762) -- (2.20897,2.55977);
\draw [c] (2.20897,2.55977) -- (2.20897,2.71191);
\draw [c] (2.19093,2.55977) -- (2.20897,2.55977);
\draw [c] (2.20897,2.55977) -- (2.22702,2.55977);
\definecolor{c}{rgb}{0,0,0};
\colorlet{c}{natgreen};
\draw [c] (2.24506,2.6453) -- (2.24506,2.80823);
\draw [c] (2.24506,2.80823) -- (2.24506,2.97115);
\draw [c] (2.22702,2.80823) -- (2.24506,2.80823);
\draw [c] (2.24506,2.80823) -- (2.2631,2.80823);
\definecolor{c}{rgb}{0,0,0};
\colorlet{c}{natgreen};
\draw [c] (2.28115,3.10578) -- (2.28115,3.28792);
\draw [c] (2.28115,3.28792) -- (2.28115,3.47006);
\draw [c] (2.2631,3.28792) -- (2.28115,3.28792);
\draw [c] (2.28115,3.28792) -- (2.29919,3.28792);
\definecolor{c}{rgb}{0,0,0};
\colorlet{c}{natgreen};
\draw [c] (2.31724,3.25624) -- (2.31724,3.44303);
\draw [c] (2.31724,3.44303) -- (2.31724,3.62982);
\draw [c] (2.29919,3.44303) -- (2.31724,3.44303);
\draw [c] (2.31724,3.44303) -- (2.33528,3.44303);
\definecolor{c}{rgb}{0,0,0};
\colorlet{c}{natgreen};
\draw [c] (2.35333,3.66962) -- (2.35333,3.87126);
\draw [c] (2.35333,3.87126) -- (2.35333,4.07289);
\draw [c] (2.33528,3.87126) -- (2.35333,3.87126);
\draw [c] (2.35333,3.87126) -- (2.37137,3.87126);
\definecolor{c}{rgb}{0,0,0};
\colorlet{c}{natgreen};
\draw [c] (2.38942,4.0976) -- (2.38942,4.3147);
\draw [c] (2.38942,4.3147) -- (2.38942,4.5318);
\draw [c] (2.37137,4.3147) -- (2.38942,4.3147);
\draw [c] (2.38942,4.3147) -- (2.40746,4.3147);
\definecolor{c}{rgb}{0,0,0};
\colorlet{c}{natgreen};
\draw [c] (2.4255,4.48404) -- (2.4255,4.71294);
\draw [c] (2.4255,4.71294) -- (2.4255,4.94183);
\draw [c] (2.40746,4.71294) -- (2.4255,4.71294);
\draw [c] (2.4255,4.71294) -- (2.44355,4.71294);
\definecolor{c}{rgb}{0,0,0};
\colorlet{c}{natgreen};
\draw [c] (2.46159,5.32319) -- (2.46159,5.58255);
\draw [c] (2.46159,5.58255) -- (2.46159,5.84191);
\draw [c] (2.44355,5.58255) -- (2.46159,5.58255);
\draw [c] (2.46159,5.58255) -- (2.47964,5.58255);
\definecolor{c}{rgb}{0,0,0};
\colorlet{c}{natgreen};
\draw [c] (2.49768,5.48167) -- (2.49768,5.74099);
\draw [c] (2.49768,5.74099) -- (2.49768,6.00031);
\draw [c] (2.47964,5.74099) -- (2.49768,5.74099);
\draw [c] (2.49768,5.74099) -- (2.51573,5.74099);
\definecolor{c}{rgb}{0,0,0};
\colorlet{c}{natgreen};
\draw [c] (2.53377,5.79817) -- (2.53377,6.06855);
\draw [c] (2.53377,6.06855) -- (2.53377,6.33893);
\draw [c] (2.51573,6.06855) -- (2.53377,6.06855);
\draw [c] (2.53377,6.06855) -- (2.55181,6.06855);
\definecolor{c}{rgb}{0,0,0};
\colorlet{c}{natgreen};
\draw [c] (2.56986,5.60846) -- (2.56986,5.87462);
\draw [c] (2.56986,5.87462) -- (2.56986,6.14078);
\draw [c] (2.55181,5.87462) -- (2.56986,5.87462);
\draw [c] (2.56986,5.87462) -- (2.5879,5.87462);
\definecolor{c}{rgb}{0,0,0};
\colorlet{c}{natgreen};
\draw [c] (2.60595,5.15422) -- (2.60595,5.4071);
\draw [c] (2.60595,5.4071) -- (2.60595,5.65997);
\draw [c] (2.5879,5.4071) -- (2.60595,5.4071);
\draw [c] (2.60595,5.4071) -- (2.62399,5.4071);
\definecolor{c}{rgb}{0,0,0};
\colorlet{c}{natgreen};
\draw [c] (2.64204,5.79256) -- (2.64204,6.06206);
\draw [c] (2.64204,6.06206) -- (2.64204,6.33155);
\draw [c] (2.62399,6.06206) -- (2.64204,6.06206);
\draw [c] (2.64204,6.06206) -- (2.66008,6.06206);
\definecolor{c}{rgb}{0,0,0};
\colorlet{c}{natgreen};
\draw [c] (2.67813,5.8431) -- (2.67813,6.11415);
\draw [c] (2.67813,6.11415) -- (2.67813,6.38519);
\draw [c] (2.66008,6.11415) -- (2.67813,6.11415);
\draw [c] (2.67813,6.11415) -- (2.69617,6.11415);
\definecolor{c}{rgb}{0,0,0};
\colorlet{c}{natgreen};
\draw [c] (2.71421,5.63737) -- (2.71421,5.90035);
\draw [c] (2.71421,5.90035) -- (2.71421,6.16332);
\draw [c] (2.69617,5.90035) -- (2.71421,5.90035);
\draw [c] (2.71421,5.90035) -- (2.73226,5.90035);
\definecolor{c}{rgb}{0,0,0};
\colorlet{c}{natgreen};
\draw [c] (2.7503,5.25438) -- (2.7503,5.50765);
\draw [c] (2.7503,5.50765) -- (2.7503,5.76092);
\draw [c] (2.73226,5.50765) -- (2.7503,5.50765);
\draw [c] (2.7503,5.50765) -- (2.76835,5.50765);
\definecolor{c}{rgb}{0,0,0};
\colorlet{c}{natgreen};
\draw [c] (2.78639,5.03639) -- (2.78639,5.2877);
\draw [c] (2.78639,5.2877) -- (2.78639,5.53901);
\draw [c] (2.76835,5.2877) -- (2.78639,5.2877);
\draw [c] (2.78639,5.2877) -- (2.80444,5.2877);
\definecolor{c}{rgb}{0,0,0};
\colorlet{c}{natgreen};
\draw [c] (2.82248,4.89291) -- (2.82248,5.13628);
\draw [c] (2.82248,5.13628) -- (2.82248,5.37966);
\draw [c] (2.80444,5.13628) -- (2.82248,5.13628);
\draw [c] (2.82248,5.13628) -- (2.84052,5.13628);
\definecolor{c}{rgb}{0,0,0};
\colorlet{c}{natgreen};
\draw [c] (2.85857,4.3619) -- (2.85857,4.58945);
\draw [c] (2.85857,4.58945) -- (2.85857,4.81699);
\draw [c] (2.84052,4.58945) -- (2.85857,4.58945);
\draw [c] (2.85857,4.58945) -- (2.87661,4.58945);
\definecolor{c}{rgb}{0,0,0};
\colorlet{c}{natgreen};
\draw [c] (2.89466,3.86365) -- (2.89466,4.07372);
\draw [c] (2.89466,4.07372) -- (2.89466,4.28379);
\draw [c] (2.87661,4.07372) -- (2.89466,4.07372);
\draw [c] (2.89466,4.07372) -- (2.9127,4.07372);
\definecolor{c}{rgb}{0,0,0};
\colorlet{c}{natgreen};
\draw [c] (2.93075,3.79834) -- (2.93075,4.0075);
\draw [c] (2.93075,4.0075) -- (2.93075,4.21665);
\draw [c] (2.9127,4.0075) -- (2.93075,4.0075);
\draw [c] (2.93075,4.0075) -- (2.94879,4.0075);
\definecolor{c}{rgb}{0,0,0};
\colorlet{c}{natgreen};
\draw [c] (2.96683,4.24582) -- (2.96683,4.47087);
\draw [c] (2.96683,4.47087) -- (2.96683,4.69591);
\draw [c] (2.94879,4.47087) -- (2.96683,4.47087);
\draw [c] (2.96683,4.47087) -- (2.98488,4.47087);
\definecolor{c}{rgb}{0,0,0};
\colorlet{c}{natgreen};
\draw [c] (3.00292,3.50606) -- (3.00292,3.70501);
\draw [c] (3.00292,3.70501) -- (3.00292,3.90397);
\draw [c] (2.98488,3.70501) -- (3.00292,3.70501);
\draw [c] (3.00292,3.70501) -- (3.02097,3.70501);
\definecolor{c}{rgb}{0,0,0};
\colorlet{c}{natgreen};
\draw [c] (3.03901,3.34697) -- (3.03901,3.54204);
\draw [c] (3.03901,3.54204) -- (3.03901,3.73711);
\draw [c] (3.02097,3.54204) -- (3.03901,3.54204);
\draw [c] (3.03901,3.54204) -- (3.05706,3.54204);
\definecolor{c}{rgb}{0,0,0};
\colorlet{c}{natgreen};
\draw [c] (3.0751,3.09791) -- (3.0751,3.28161);
\draw [c] (3.0751,3.28161) -- (3.0751,3.46531);
\draw [c] (3.05706,3.28161) -- (3.0751,3.28161);
\draw [c] (3.0751,3.28161) -- (3.09315,3.28161);
\definecolor{c}{rgb}{0,0,0};
\colorlet{c}{natgreen};
\draw [c] (3.11119,2.9437) -- (3.11119,3.12193);
\draw [c] (3.11119,3.12193) -- (3.11119,3.30016);
\draw [c] (3.09315,3.12193) -- (3.11119,3.12193);
\draw [c] (3.11119,3.12193) -- (3.12923,3.12193);
\definecolor{c}{rgb}{0,0,0};
\colorlet{c}{natgreen};
\draw [c] (3.14728,2.78324) -- (3.14728,2.95447);
\draw [c] (3.14728,2.95447) -- (3.14728,3.1257);
\draw [c] (3.12923,2.95447) -- (3.14728,2.95447);
\draw [c] (3.14728,2.95447) -- (3.16532,2.95447);
\definecolor{c}{rgb}{0,0,0};
\colorlet{c}{natgreen};
\draw [c] (3.18337,2.77691) -- (3.18337,2.94856);
\draw [c] (3.18337,2.94856) -- (3.18337,3.1202);
\draw [c] (3.16532,2.94856) -- (3.18337,2.94856);
\draw [c] (3.18337,2.94856) -- (3.20141,2.94856);
\definecolor{c}{rgb}{0,0,0};
\colorlet{c}{natgreen};
\draw [c] (3.21946,2.48398) -- (3.21946,2.64232);
\draw [c] (3.21946,2.64232) -- (3.21946,2.80067);
\draw [c] (3.20141,2.64232) -- (3.21946,2.64232);
\draw [c] (3.21946,2.64232) -- (3.2375,2.64232);
\definecolor{c}{rgb}{0,0,0};
\colorlet{c}{natgreen};
\draw [c] (3.25554,2.42662) -- (3.25554,2.58401);
\draw [c] (3.25554,2.58401) -- (3.25554,2.7414);
\draw [c] (3.2375,2.58401) -- (3.25554,2.58401);
\draw [c] (3.25554,2.58401) -- (3.27359,2.58401);
\definecolor{c}{rgb}{0,0,0};
\colorlet{c}{natgreen};
\draw [c] (3.29163,2.45462) -- (3.29163,2.61483);
\draw [c] (3.29163,2.61483) -- (3.29163,2.77503);
\draw [c] (3.27359,2.61483) -- (3.29163,2.61483);
\draw [c] (3.29163,2.61483) -- (3.30968,2.61483);
\definecolor{c}{rgb}{0,0,0};
\colorlet{c}{natgreen};
\draw [c] (3.32772,2.09209) -- (3.32772,2.23211);
\draw [c] (3.32772,2.23211) -- (3.32772,2.37213);
\draw [c] (3.30968,2.23211) -- (3.32772,2.23211);
\draw [c] (3.32772,2.23211) -- (3.34577,2.23211);
\definecolor{c}{rgb}{0,0,0};
\colorlet{c}{natgreen};
\draw [c] (3.36381,2.05646) -- (3.36381,2.19661);
\draw [c] (3.36381,2.19661) -- (3.36381,2.33676);
\draw [c] (3.34577,2.19661) -- (3.36381,2.19661);
\draw [c] (3.36381,2.19661) -- (3.38185,2.19661);
\definecolor{c}{rgb}{0,0,0};
\colorlet{c}{natgreen};
\draw [c] (3.3999,1.90968) -- (3.3999,2.0376);
\draw [c] (3.3999,2.0376) -- (3.3999,2.16552);
\draw [c] (3.38185,2.0376) -- (3.3999,2.0376);
\draw [c] (3.3999,2.0376) -- (3.41794,2.0376);
\definecolor{c}{rgb}{0,0,0};
\colorlet{c}{natgreen};
\draw [c] (3.43599,1.8398) -- (3.43599,1.96403);
\draw [c] (3.43599,1.96403) -- (3.43599,2.08826);
\draw [c] (3.41794,1.96403) -- (3.43599,1.96403);
\draw [c] (3.43599,1.96403) -- (3.45403,1.96403);
\definecolor{c}{rgb}{0,0,0};
\colorlet{c}{natgreen};
\draw [c] (3.47208,1.89732) -- (3.47208,2.02992);
\draw [c] (3.47208,2.02992) -- (3.47208,2.16253);
\draw [c] (3.45403,2.02992) -- (3.47208,2.02992);
\draw [c] (3.47208,2.02992) -- (3.49012,2.02992);
\definecolor{c}{rgb}{0,0,0};
\colorlet{c}{natgreen};
\draw [c] (3.50817,1.76818) -- (3.50817,1.88907);
\draw [c] (3.50817,1.88907) -- (3.50817,2.00995);
\draw [c] (3.49012,1.88907) -- (3.50817,1.88907);
\draw [c] (3.50817,1.88907) -- (3.52621,1.88907);
\definecolor{c}{rgb}{0,0,0};
\colorlet{c}{natgreen};
\draw [c] (3.54425,1.65461) -- (3.54425,1.77173);
\draw [c] (3.54425,1.77173) -- (3.54425,1.88885);
\draw [c] (3.52621,1.77173) -- (3.54425,1.77173);
\draw [c] (3.54425,1.77173) -- (3.5623,1.77173);
\definecolor{c}{rgb}{0,0,0};
\colorlet{c}{natgreen};
\draw [c] (3.58034,1.83018) -- (3.58034,1.9561);
\draw [c] (3.58034,1.9561) -- (3.58034,2.08202);
\draw [c] (3.5623,1.9561) -- (3.58034,1.9561);
\draw [c] (3.58034,1.9561) -- (3.59839,1.9561);
\definecolor{c}{rgb}{0,0,0};
\colorlet{c}{natgreen};
\draw [c] (3.61643,1.73218) -- (3.61643,1.85168);
\draw [c] (3.61643,1.85168) -- (3.61643,1.97118);
\draw [c] (3.59839,1.85168) -- (3.61643,1.85168);
\draw [c] (3.61643,1.85168) -- (3.63448,1.85168);
\definecolor{c}{rgb}{0,0,0};
\colorlet{c}{natgreen};
\draw [c] (3.65252,1.5139) -- (3.65252,1.62264);
\draw [c] (3.65252,1.62264) -- (3.65252,1.73139);
\draw [c] (3.63448,1.62264) -- (3.65252,1.62264);
\draw [c] (3.65252,1.62264) -- (3.67056,1.62264);
\definecolor{c}{rgb}{0,0,0};
\colorlet{c}{natgreen};
\draw [c] (3.68861,1.3318) -- (3.68861,1.42091);
\draw [c] (3.68861,1.42091) -- (3.68861,1.51002);
\draw [c] (3.67056,1.42091) -- (3.68861,1.42091);
\draw [c] (3.68861,1.42091) -- (3.70665,1.42091);
\definecolor{c}{rgb}{0,0,0};
\colorlet{c}{natgreen};
\draw [c] (3.7247,1.36919) -- (3.7247,1.46681);
\draw [c] (3.7247,1.46681) -- (3.7247,1.56444);
\draw [c] (3.70665,1.46681) -- (3.7247,1.46681);
\draw [c] (3.7247,1.46681) -- (3.74274,1.46681);
\definecolor{c}{rgb}{0,0,0};
\colorlet{c}{natgreen};
\draw [c] (3.76079,1.42997) -- (3.76079,1.53404);
\draw [c] (3.76079,1.53404) -- (3.76079,1.63811);
\draw [c] (3.74274,1.53404) -- (3.76079,1.53404);
\draw [c] (3.76079,1.53404) -- (3.77883,1.53404);
\definecolor{c}{rgb}{0,0,0};
\colorlet{c}{natgreen};
\draw [c] (3.79688,1.27256) -- (3.79688,1.36142);
\draw [c] (3.79688,1.36142) -- (3.79688,1.45027);
\draw [c] (3.77883,1.36142) -- (3.79688,1.36142);
\draw [c] (3.79688,1.36142) -- (3.81492,1.36142);
\definecolor{c}{rgb}{0,0,0};
\colorlet{c}{natgreen};
\draw [c] (3.83296,1.21638) -- (3.83296,1.30191);
\draw [c] (3.83296,1.30191) -- (3.83296,1.38745);
\draw [c] (3.81492,1.30191) -- (3.83296,1.30191);
\draw [c] (3.83296,1.30191) -- (3.85101,1.30191);
\definecolor{c}{rgb}{0,0,0};
\colorlet{c}{natgreen};
\draw [c] (3.86905,1.12283) -- (3.86905,1.20232);
\draw [c] (3.86905,1.20232) -- (3.86905,1.28182);
\draw [c] (3.85101,1.20232) -- (3.86905,1.20232);
\draw [c] (3.86905,1.20232) -- (3.8871,1.20232);
\definecolor{c}{rgb}{0,0,0};
\colorlet{c}{natgreen};
\draw [c] (3.90514,1.13958) -- (3.90514,1.21936);
\draw [c] (3.90514,1.21936) -- (3.90514,1.29914);
\draw [c] (3.8871,1.21936) -- (3.90514,1.21936);
\draw [c] (3.90514,1.21936) -- (3.92319,1.21936);
\definecolor{c}{rgb}{0,0,0};
\colorlet{c}{natgreen};
\draw [c] (3.94123,1.05205) -- (3.94123,1.1216);
\draw [c] (3.94123,1.1216) -- (3.94123,1.19116);
\draw [c] (3.92319,1.1216) -- (3.94123,1.1216);
\draw [c] (3.94123,1.1216) -- (3.95927,1.1216);
\definecolor{c}{rgb}{0,0,0};
\colorlet{c}{natgreen};
\draw [c] (3.97732,1.07434) -- (3.97732,1.14602);
\draw [c] (3.97732,1.14602) -- (3.97732,1.2177);
\draw [c] (3.95927,1.14602) -- (3.97732,1.14602);
\draw [c] (3.97732,1.14602) -- (3.99536,1.14602);
\definecolor{c}{rgb}{0,0,0};
\colorlet{c}{natgreen};
\draw [c] (4.01341,1.1326) -- (4.01341,1.20974);
\draw [c] (4.01341,1.20974) -- (4.01341,1.28687);
\draw [c] (3.99536,1.20974) -- (4.01341,1.20974);
\draw [c] (4.01341,1.20974) -- (4.03145,1.20974);
\definecolor{c}{rgb}{0,0,0};
\colorlet{c}{natgreen};
\draw [c] (4.0495,0.939214) -- (4.0495,0.995054);
\draw [c] (4.0495,0.995054) -- (4.0495,1.05089);
\draw [c] (4.03145,0.995054) -- (4.0495,0.995054);
\draw [c] (4.0495,0.995054) -- (4.06754,0.995054);
\definecolor{c}{rgb}{0,0,0};
\colorlet{c}{natgreen};
\draw [c] (4.08558,1.20029) -- (4.08558,1.28407);
\draw [c] (4.08558,1.28407) -- (4.08558,1.36784);
\draw [c] (4.06754,1.28407) -- (4.08558,1.28407);
\draw [c] (4.08558,1.28407) -- (4.10363,1.28407);
\definecolor{c}{rgb}{0,0,0};
\colorlet{c}{natgreen};
\draw [c] (4.12167,1.0113) -- (4.12167,1.07931);
\draw [c] (4.12167,1.07931) -- (4.12167,1.14733);
\draw [c] (4.10363,1.07931) -- (4.12167,1.07931);
\draw [c] (4.12167,1.07931) -- (4.13972,1.07931);
\definecolor{c}{rgb}{0,0,0};
\colorlet{c}{natgreen};
\draw [c] (4.15776,0.981008) -- (4.15776,1.04309);
\draw [c] (4.15776,1.04309) -- (4.15776,1.10517);
\draw [c] (4.13972,1.04309) -- (4.15776,1.04309);
\draw [c] (4.15776,1.04309) -- (4.17581,1.04309);
\definecolor{c}{rgb}{0,0,0};
\colorlet{c}{natgreen};
\draw [c] (4.19385,0.886966) -- (4.19385,0.937799);
\draw [c] (4.19385,0.937799) -- (4.19385,0.988632);
\draw [c] (4.17581,0.937799) -- (4.19385,0.937799);
\draw [c] (4.19385,0.937799) -- (4.21189,0.937799);
\definecolor{c}{rgb}{0,0,0};
\colorlet{c}{natgreen};
\draw [c] (4.22994,0.996004) -- (4.22994,1.0585);
\draw [c] (4.22994,1.0585) -- (4.22994,1.121);
\draw [c] (4.21189,1.0585) -- (4.22994,1.0585);
\draw [c] (4.22994,1.0585) -- (4.24798,1.0585);
\definecolor{c}{rgb}{0,0,0};
\colorlet{c}{natgreen};
\draw [c] (4.26603,0.916623) -- (4.26603,0.96946);
\draw [c] (4.26603,0.96946) -- (4.26603,1.0223);
\draw [c] (4.24798,0.96946) -- (4.26603,0.96946);
\draw [c] (4.26603,0.96946) -- (4.28407,0.96946);
\definecolor{c}{rgb}{0,0,0};
\colorlet{c}{natgreen};
\draw [c] (4.30212,0.950391) -- (4.30212,1.01253);
\draw [c] (4.30212,1.01253) -- (4.30212,1.07467);
\draw [c] (4.28407,1.01253) -- (4.30212,1.01253);
\draw [c] (4.30212,1.01253) -- (4.32016,1.01253);
\definecolor{c}{rgb}{0,0,0};
\colorlet{c}{natgreen};
\draw [c] (4.33821,0.870271) -- (4.33821,0.915271);
\draw [c] (4.33821,0.915271) -- (4.33821,0.960271);
\draw [c] (4.32016,0.915271) -- (4.33821,0.915271);
\draw [c] (4.33821,0.915271) -- (4.35625,0.915271);
\definecolor{c}{rgb}{0,0,0};
\colorlet{c}{natgreen};
\draw [c] (4.37429,0.832558) -- (4.37429,0.87354);
\draw [c] (4.37429,0.87354) -- (4.37429,0.914522);
\draw [c] (4.35625,0.87354) -- (4.37429,0.87354);
\draw [c] (4.37429,0.87354) -- (4.39234,0.87354);
\definecolor{c}{rgb}{0,0,0};
\colorlet{c}{natgreen};
\draw [c] (4.41038,0.865532) -- (4.41038,0.913695);
\draw [c] (4.41038,0.913695) -- (4.41038,0.961857);
\draw [c] (4.39234,0.913695) -- (4.41038,0.913695);
\draw [c] (4.41038,0.913695) -- (4.42843,0.913695);
\definecolor{c}{rgb}{0,0,0};
\colorlet{c}{natgreen};
\draw [c] (4.44647,0.891549) -- (4.44647,0.93864);
\draw [c] (4.44647,0.93864) -- (4.44647,0.985731);
\draw [c] (4.42843,0.93864) -- (4.44647,0.93864);
\draw [c] (4.44647,0.93864) -- (4.46452,0.93864);
\definecolor{c}{rgb}{0,0,0};
\colorlet{c}{natgreen};
\draw [c] (4.48256,0.808217) -- (4.48256,0.84252);
\draw [c] (4.48256,0.84252) -- (4.48256,0.876822);
\draw [c] (4.46452,0.84252) -- (4.48256,0.84252);
\draw [c] (4.48256,0.84252) -- (4.5006,0.84252);
\definecolor{c}{rgb}{0,0,0};
\colorlet{c}{natgreen};
\draw [c] (4.51865,0.844653) -- (4.51865,0.886753);
\draw [c] (4.51865,0.886753) -- (4.51865,0.928852);
\draw [c] (4.5006,0.886753) -- (4.51865,0.886753);
\draw [c] (4.51865,0.886753) -- (4.53669,0.886753);
\definecolor{c}{rgb}{0,0,0};
\colorlet{c}{natgreen};
\draw [c] (4.55474,0.817059) -- (4.55474,0.854363);
\draw [c] (4.55474,0.854363) -- (4.55474,0.891668);
\draw [c] (4.53669,0.854363) -- (4.55474,0.854363);
\draw [c] (4.55474,0.854363) -- (4.57278,0.854363);
\definecolor{c}{rgb}{0,0,0};
\colorlet{c}{natgreen};
\draw [c] (4.59083,0.798305) -- (4.59083,0.832897);
\draw [c] (4.59083,0.832897) -- (4.59083,0.86749);
\draw [c] (4.57278,0.832897) -- (4.59083,0.832897);
\draw [c] (4.59083,0.832897) -- (4.60887,0.832897);
\definecolor{c}{rgb}{0,0,0};
\colorlet{c}{natgreen};
\draw [c] (4.62692,0.850203) -- (4.62692,0.892336);
\draw [c] (4.62692,0.892336) -- (4.62692,0.934469);
\draw [c] (4.60887,0.892336) -- (4.62692,0.892336);
\draw [c] (4.62692,0.892336) -- (4.64496,0.892336);
\definecolor{c}{rgb}{0,0,0};
\colorlet{c}{natgreen};
\draw [c] (4.663,0.837422) -- (4.663,0.880898);
\draw [c] (4.663,0.880898) -- (4.663,0.924374);
\draw [c] (4.64496,0.880898) -- (4.663,0.880898);
\draw [c] (4.663,0.880898) -- (4.68105,0.880898);
\definecolor{c}{rgb}{0,0,0};
\colorlet{c}{natgreen};
\draw [c] (4.69909,0.79697) -- (4.69909,0.828716);
\draw [c] (4.69909,0.828716) -- (4.69909,0.860462);
\draw [c] (4.68105,0.828716) -- (4.69909,0.828716);
\draw [c] (4.69909,0.828716) -- (4.71714,0.828716);
\definecolor{c}{rgb}{0,0,0};
\colorlet{c}{natgreen};
\draw [c] (4.73518,0.840524) -- (4.73518,0.884851);
\draw [c] (4.73518,0.884851) -- (4.73518,0.929178);
\draw [c] (4.71714,0.884851) -- (4.73518,0.884851);
\draw [c] (4.73518,0.884851) -- (4.75323,0.884851);
\definecolor{c}{rgb}{0,0,0};
\colorlet{c}{natgreen};
\draw [c] (4.77127,0.780186) -- (4.77127,0.813231);
\draw [c] (4.77127,0.813231) -- (4.77127,0.846275);
\draw [c] (4.75323,0.813231) -- (4.77127,0.813231);
\draw [c] (4.77127,0.813231) -- (4.78931,0.813231);
\definecolor{c}{rgb}{0,0,0};
\colorlet{c}{natgreen};
\draw [c] (4.80736,0.813031) -- (4.80736,0.853508);
\draw [c] (4.80736,0.853508) -- (4.80736,0.893985);
\draw [c] (4.78931,0.853508) -- (4.80736,0.853508);
\draw [c] (4.80736,0.853508) -- (4.8254,0.853508);
\definecolor{c}{rgb}{0,0,0};
\colorlet{c}{natgreen};
\draw [c] (4.84345,0.793976) -- (4.84345,0.828606);
\draw [c] (4.84345,0.828606) -- (4.84345,0.863236);
\draw [c] (4.8254,0.828606) -- (4.84345,0.828606);
\draw [c] (4.84345,0.828606) -- (4.86149,0.828606);
\definecolor{c}{rgb}{0,0,0};
\colorlet{c}{natgreen};
\draw [c] (4.87954,0.758384) -- (4.87954,0.784644);
\draw [c] (4.87954,0.784644) -- (4.87954,0.810903);
\draw [c] (4.86149,0.784644) -- (4.87954,0.784644);
\draw [c] (4.87954,0.784644) -- (4.89758,0.784644);
\definecolor{c}{rgb}{0,0,0};
\colorlet{c}{natgreen};
\draw [c] (4.91563,0.758837) -- (4.91563,0.787174);
\draw [c] (4.91563,0.787174) -- (4.91563,0.815512);
\draw [c] (4.89758,0.787174) -- (4.91563,0.787174);
\draw [c] (4.91563,0.787174) -- (4.93367,0.787174);
\definecolor{c}{rgb}{0,0,0};
\colorlet{c}{natgreen};
\draw [c] (4.95171,0.761736) -- (4.95171,0.785406);
\draw [c] (4.95171,0.785406) -- (4.95171,0.809077);
\draw [c] (4.93367,0.785406) -- (4.95171,0.785406);
\draw [c] (4.95171,0.785406) -- (4.96976,0.785406);
\definecolor{c}{rgb}{0,0,0};
\colorlet{c}{natgreen};
\draw [c] (4.9878,0.800978) -- (4.9878,0.832609);
\draw [c] (4.9878,0.832609) -- (4.9878,0.86424);
\draw [c] (4.96976,0.832609) -- (4.9878,0.832609);
\draw [c] (4.9878,0.832609) -- (5.00585,0.832609);
\definecolor{c}{rgb}{0,0,0};
\colorlet{c}{natgreen};
\draw [c] (5.02389,0.775133) -- (5.02389,0.80439);
\draw [c] (5.02389,0.80439) -- (5.02389,0.833647);
\draw [c] (5.00585,0.80439) -- (5.02389,0.80439);
\draw [c] (5.02389,0.80439) -- (5.04194,0.80439);
\definecolor{c}{rgb}{0,0,0};
\colorlet{c}{natgreen};
\draw [c] (5.05998,0.742255) -- (5.05998,0.760704);
\draw [c] (5.05998,0.760704) -- (5.05998,0.779153);
\draw [c] (5.04194,0.760704) -- (5.05998,0.760704);
\draw [c] (5.05998,0.760704) -- (5.07802,0.760704);
\definecolor{c}{rgb}{0,0,0};
\colorlet{c}{natgreen};
\draw [c] (5.09607,0.751018) -- (5.09607,0.773695);
\draw [c] (5.09607,0.773695) -- (5.09607,0.796372);
\draw [c] (5.07802,0.773695) -- (5.09607,0.773695);
\draw [c] (5.09607,0.773695) -- (5.11411,0.773695);
\definecolor{c}{rgb}{0,0,0};
\colorlet{c}{natgreen};
\draw [c] (5.13216,0.812429) -- (5.13216,0.851758);
\draw [c] (5.13216,0.851758) -- (5.13216,0.891088);
\draw [c] (5.11411,0.851758) -- (5.13216,0.851758);
\draw [c] (5.13216,0.851758) -- (5.1502,0.851758);
\definecolor{c}{rgb}{0,0,0};
\colorlet{c}{natgreen};
\draw [c] (5.16825,0.747876) -- (5.16825,0.76617);
\draw [c] (5.16825,0.76617) -- (5.16825,0.784464);
\draw [c] (5.1502,0.76617) -- (5.16825,0.76617);
\draw [c] (5.16825,0.76617) -- (5.18629,0.76617);
\definecolor{c}{rgb}{0,0,0};
\colorlet{c}{natgreen};
\draw [c] (5.20433,0.757096) -- (5.20433,0.780973);
\draw [c] (5.20433,0.780973) -- (5.20433,0.804849);
\draw [c] (5.18629,0.780973) -- (5.20433,0.780973);
\draw [c] (5.20433,0.780973) -- (5.22238,0.780973);
\definecolor{c}{rgb}{0,0,0};
\colorlet{c}{natgreen};
\draw [c] (5.24042,0.780723) -- (5.24042,0.812674);
\draw [c] (5.24042,0.812674) -- (5.24042,0.844625);
\draw [c] (5.22238,0.812674) -- (5.24042,0.812674);
\draw [c] (5.24042,0.812674) -- (5.25847,0.812674);
\definecolor{c}{rgb}{0,0,0};
\colorlet{c}{natgreen};
\draw [c] (5.27651,0.746262) -- (5.27651,0.76547);
\draw [c] (5.27651,0.76547) -- (5.27651,0.784677);
\draw [c] (5.25847,0.76547) -- (5.27651,0.76547);
\draw [c] (5.27651,0.76547) -- (5.29456,0.76547);
\definecolor{c}{rgb}{0,0,0};
\colorlet{c}{natgreen};
\draw [c] (5.3126,0.77607) -- (5.3126,0.803785);
\draw [c] (5.3126,0.803785) -- (5.3126,0.8315);
\draw [c] (5.29456,0.803785) -- (5.3126,0.803785);
\draw [c] (5.3126,0.803785) -- (5.33065,0.803785);
\definecolor{c}{rgb}{0,0,0};
\colorlet{c}{natgreen};
\draw [c] (5.34869,0.75608) -- (5.34869,0.778446);
\draw [c] (5.34869,0.778446) -- (5.34869,0.800813);
\draw [c] (5.33065,0.778446) -- (5.34869,0.778446);
\draw [c] (5.34869,0.778446) -- (5.36673,0.778446);
\definecolor{c}{rgb}{0,0,0};
\colorlet{c}{natgreen};
\draw [c] (5.38478,0.752922) -- (5.38478,0.773923);
\draw [c] (5.38478,0.773923) -- (5.38478,0.794924);
\draw [c] (5.36673,0.773923) -- (5.38478,0.773923);
\draw [c] (5.38478,0.773923) -- (5.40282,0.773923);
\definecolor{c}{rgb}{0,0,0};
\colorlet{c}{natgreen};
\draw [c] (5.42087,0.742951) -- (5.42087,0.760468);
\draw [c] (5.42087,0.760468) -- (5.42087,0.777986);
\draw [c] (5.40282,0.760468) -- (5.42087,0.760468);
\draw [c] (5.42087,0.760468) -- (5.43891,0.760468);
\definecolor{c}{rgb}{0,0,0};
\colorlet{c}{natgreen};
\draw [c] (5.45696,0.742899) -- (5.45696,0.762868);
\draw [c] (5.45696,0.762868) -- (5.45696,0.782838);
\draw [c] (5.43891,0.762868) -- (5.45696,0.762868);
\draw [c] (5.45696,0.762868) -- (5.475,0.762868);
\definecolor{c}{rgb}{0,0,0};
\colorlet{c}{natgreen};
\draw [c] (5.49304,0.750905) -- (5.49304,0.771726);
\draw [c] (5.49304,0.771726) -- (5.49304,0.792547);
\draw [c] (5.475,0.771726) -- (5.49304,0.771726);
\draw [c] (5.49304,0.771726) -- (5.51109,0.771726);
\definecolor{c}{rgb}{0,0,0};
\colorlet{c}{natgreen};
\draw [c] (5.52913,0.735553) -- (5.52913,0.742652);
\draw [c] (5.52913,0.742652) -- (5.52913,0.749751);
\draw [c] (5.51109,0.742652) -- (5.52913,0.742652);
\draw [c] (5.52913,0.742652) -- (5.54718,0.742652);
\definecolor{c}{rgb}{0,0,0};
\colorlet{c}{natgreen};
\draw [c] (5.56522,0.752318) -- (5.56522,0.776656);
\draw [c] (5.56522,0.776656) -- (5.56522,0.800993);
\draw [c] (5.54718,0.776656) -- (5.56522,0.776656);
\draw [c] (5.56522,0.776656) -- (5.58327,0.776656);
\definecolor{c}{rgb}{0,0,0};
\colorlet{c}{natgreen};
\draw [c] (5.60131,0.752888) -- (5.60131,0.779509);
\draw [c] (5.60131,0.779509) -- (5.60131,0.80613);
\draw [c] (5.58327,0.779509) -- (5.60131,0.779509);
\draw [c] (5.60131,0.779509) -- (5.61935,0.779509);
\definecolor{c}{rgb}{0,0,0};
\colorlet{c}{natgreen};
\draw [c] (5.6374,0.754683) -- (5.6374,0.778493);
\draw [c] (5.6374,0.778493) -- (5.6374,0.802304);
\draw [c] (5.61935,0.778493) -- (5.6374,0.778493);
\draw [c] (5.6374,0.778493) -- (5.65544,0.778493);
\definecolor{c}{rgb}{0,0,0};
\colorlet{c}{natgreen};
\draw [c] (5.67349,0.737725) -- (5.67349,0.748389);
\draw [c] (5.67349,0.748389) -- (5.67349,0.759052);
\draw [c] (5.65544,0.748389) -- (5.67349,0.748389);
\draw [c] (5.67349,0.748389) -- (5.69153,0.748389);
\definecolor{c}{rgb}{0,0,0};
\colorlet{c}{natgreen};
\draw [c] (5.70958,0.734632) -- (5.70958,0.741681);
\draw [c] (5.70958,0.741681) -- (5.70958,0.74873);
\draw [c] (5.69153,0.741681) -- (5.70958,0.741681);
\draw [c] (5.70958,0.741681) -- (5.72762,0.741681);
\definecolor{c}{rgb}{0,0,0};
\colorlet{c}{natgreen};
\draw [c] (5.74567,0.744543) -- (5.74567,0.763576);
\draw [c] (5.74567,0.763576) -- (5.74567,0.782609);
\draw [c] (5.72762,0.763576) -- (5.74567,0.763576);
\draw [c] (5.74567,0.763576) -- (5.76371,0.763576);
\definecolor{c}{rgb}{0,0,0};
\colorlet{c}{natgreen};
\draw [c] (5.81784,0.740037) -- (5.81784,0.755969);
\draw [c] (5.81784,0.755969) -- (5.81784,0.771901);
\draw [c] (5.7998,0.755969) -- (5.81784,0.755969);
\draw [c] (5.81784,0.755969) -- (5.83589,0.755969);
\definecolor{c}{rgb}{0,0,0};
\colorlet{c}{natgreen};
\draw [c] (5.89002,0.738984) -- (5.89002,0.753556);
\draw [c] (5.89002,0.753556) -- (5.89002,0.768129);
\draw [c] (5.87198,0.753556) -- (5.89002,0.753556);
\draw [c] (5.89002,0.753556) -- (5.90806,0.753556);
\definecolor{c}{rgb}{0,0,0};
\colorlet{c}{natgreen};
\draw [c] (5.99829,0.734632) -- (5.99829,0.7446);
\draw [c] (5.99829,0.7446) -- (5.99829,0.754568);
\draw [c] (5.98024,0.7446) -- (5.99829,0.7446);
\draw [c] (5.99829,0.7446) -- (6.01633,0.7446);
\definecolor{c}{rgb}{0,0,0};
\colorlet{c}{natgreen};
\draw [c] (6.03438,0.752867) -- (6.03438,0.777778);
\draw [c] (6.03438,0.777778) -- (6.03438,0.802688);
\draw [c] (6.01633,0.777778) -- (6.03438,0.777778);
\draw [c] (6.03438,0.777778) -- (6.05242,0.777778);
\definecolor{c}{rgb}{0,0,0};
\colorlet{c}{natgreen};
\draw [c] (6.14264,0.734633) -- (6.14264,0.734774);
\draw [c] (6.14264,0.734774) -- (6.14264,0.734916);
\draw [c] (6.1246,0.734774) -- (6.14264,0.734774);
\draw [c] (6.14264,0.734774) -- (6.16069,0.734774);
\definecolor{c}{rgb}{0,0,0};
\colorlet{c}{natgreen};
\draw [c] (6.17873,0.734632) -- (6.17873,0.74983);
\draw [c] (6.17873,0.74983) -- (6.17873,0.765027);
\draw [c] (6.16069,0.74983) -- (6.17873,0.74983);
\draw [c] (6.17873,0.74983) -- (6.19677,0.74983);
\definecolor{c}{rgb}{0,0,0};
\colorlet{c}{natgreen};
\draw [c] (6.21482,0.737785) -- (6.21482,0.745395);
\draw [c] (6.21482,0.745395) -- (6.21482,0.753006);
\draw [c] (6.19677,0.745395) -- (6.21482,0.745395);
\draw [c] (6.21482,0.745395) -- (6.23286,0.745395);
\definecolor{c}{rgb}{0,0,0};
\colorlet{c}{natgreen};
\draw [c] (6.25091,0.73666) -- (6.25091,0.750033);
\draw [c] (6.25091,0.750033) -- (6.25091,0.763407);
\draw [c] (6.23286,0.750033) -- (6.25091,0.750033);
\draw [c] (6.25091,0.750033) -- (6.26895,0.750033);
\definecolor{c}{rgb}{0,0,0};
\colorlet{c}{natgreen};
\draw [c] (6.287,0.734632) -- (6.287,0.747821);
\draw [c] (6.287,0.747821) -- (6.287,0.76101);
\draw [c] (6.26895,0.747821) -- (6.287,0.747821);
\draw [c] (6.287,0.747821) -- (6.30504,0.747821);
\definecolor{c}{rgb}{0,0,0};
\colorlet{c}{natgreen};
\draw [c] (6.32308,0.739441) -- (6.32308,0.751649);
\draw [c] (6.32308,0.751649) -- (6.32308,0.763857);
\draw [c] (6.30504,0.751649) -- (6.32308,0.751649);
\draw [c] (6.32308,0.751649) -- (6.34113,0.751649);
\definecolor{c}{rgb}{0,0,0};
\colorlet{c}{natgreen};
\draw [c] (6.39526,0.734632) -- (6.39526,0.740014);
\draw [c] (6.39526,0.740014) -- (6.39526,0.745395);
\draw [c] (6.37722,0.740014) -- (6.39526,0.740014);
\draw [c] (6.39526,0.740014) -- (6.41331,0.740014);
\definecolor{c}{rgb}{0,0,0};
\colorlet{c}{natgreen};
\draw [c] (6.43135,0.734632) -- (6.43135,0.741681);
\draw [c] (6.43135,0.741681) -- (6.43135,0.74873);
\draw [c] (6.41331,0.741681) -- (6.43135,0.741681);
\draw [c] (6.43135,0.741681) -- (6.4494,0.741681);
\definecolor{c}{rgb}{0,0,0};
\colorlet{c}{natgreen};
\draw [c] (6.53962,0.740056) -- (6.53962,0.756158);
\draw [c] (6.53962,0.756158) -- (6.53962,0.77226);
\draw [c] (6.52157,0.756158) -- (6.53962,0.756158);
\draw [c] (6.53962,0.756158) -- (6.55766,0.756158);
\definecolor{c}{rgb}{0,0,0};
\colorlet{c}{natgreen};
\draw [c] (6.57571,0.746809) -- (6.57571,0.767173);
\draw [c] (6.57571,0.767173) -- (6.57571,0.787537);
\draw [c] (6.55766,0.767173) -- (6.57571,0.767173);
\draw [c] (6.57571,0.767173) -- (6.59375,0.767173);
\definecolor{c}{rgb}{0,0,0};
\colorlet{c}{natgreen};
\draw [c] (6.72006,0.74203) -- (6.72006,0.762645);
\draw [c] (6.72006,0.762645) -- (6.72006,0.78326);
\draw [c] (6.70202,0.762645) -- (6.72006,0.762645);
\draw [c] (6.72006,0.762645) -- (6.7381,0.762645);
\definecolor{c}{rgb}{0,0,0};
\colorlet{c}{natgreen};
\draw [c] (6.75615,0.74324) -- (6.75615,0.764297);
\draw [c] (6.75615,0.764297) -- (6.75615,0.785353);
\draw [c] (6.7381,0.764297) -- (6.75615,0.764297);
\draw [c] (6.75615,0.764297) -- (6.77419,0.764297);
\definecolor{c}{rgb}{0,0,0};
\colorlet{c}{natgreen};
\draw [c] (7.00877,0.734632) -- (7.00877,0.736845);
\draw [c] (7.00877,0.736845) -- (7.00877,0.739057);
\draw [c] (6.99073,0.736845) -- (7.00877,0.736845);
\draw [c] (7.00877,0.736845) -- (7.02681,0.736845);
\definecolor{c}{rgb}{0,0,0};
\colorlet{c}{natgreen};
\draw [c] (7.08095,0.734632) -- (7.08095,0.747821);
\draw [c] (7.08095,0.747821) -- (7.08095,0.76101);
\draw [c] (7.0629,0.747821) -- (7.08095,0.747821);
\draw [c] (7.08095,0.747821) -- (7.09899,0.747821);
\definecolor{c}{rgb}{0,0,0};
\colorlet{c}{natgreen};
\draw [c] (7.18921,0.734632) -- (7.18921,0.746887);
\draw [c] (7.18921,0.746887) -- (7.18921,0.759142);
\draw [c] (7.17117,0.746887) -- (7.18921,0.746887);
\draw [c] (7.18921,0.746887) -- (7.20726,0.746887);
\definecolor{c}{rgb}{0,0,0};
\colorlet{c}{natgreen};
\draw [c] (7.33357,0.734632) -- (7.33357,0.749109);
\draw [c] (7.33357,0.749109) -- (7.33357,0.763586);
\draw [c] (7.31552,0.749109) -- (7.33357,0.749109);
\draw [c] (7.33357,0.749109) -- (7.35161,0.749109);
\definecolor{c}{rgb}{0,0,0};
\colorlet{c}{natgreen};
\draw [c] (7.44183,0.734632) -- (7.44183,0.735461);
\draw [c] (7.44183,0.735461) -- (7.44183,0.736291);
\draw [c] (7.42379,0.735461) -- (7.44183,0.735461);
\draw [c] (7.44183,0.735461) -- (7.45988,0.735461);
\definecolor{c}{rgb}{0,0,0};
\colorlet{c}{natgreen};
\draw [c] (7.73054,0.734632) -- (7.73054,0.741681);
\draw [c] (7.73054,0.741681) -- (7.73054,0.74873);
\draw [c] (7.7125,0.741681) -- (7.73054,0.741681);
\draw [c] (7.73054,0.741681) -- (7.74859,0.741681);
\definecolor{c}{rgb}{0,0,0};
\colorlet{c}{natgreen};
\draw [c] (7.83881,0.734632) -- (7.83881,0.743059);
\draw [c] (7.83881,0.743059) -- (7.83881,0.751485);
\draw [c] (7.82077,0.743059) -- (7.83881,0.743059);
\draw [c] (7.83881,0.743059) -- (7.85685,0.743059);
\definecolor{c}{rgb}{0,0,0};
\colorlet{c}{natgreen};
\draw [c] (7.91099,0.734632) -- (7.91099,0.739076);
\draw [c] (7.91099,0.739076) -- (7.91099,0.74352);
\draw [c] (7.89294,0.739076) -- (7.91099,0.739076);
\draw [c] (7.91099,0.739076) -- (7.92903,0.739076);
\definecolor{c}{rgb}{0,0,0};
\colorlet{c}{natgreen};
\draw [c] (7.94708,0.734632) -- (7.94708,0.735461);
\draw [c] (7.94708,0.735461) -- (7.94708,0.736291);
\draw [c] (7.92903,0.735461) -- (7.94708,0.735461);
\draw [c] (7.94708,0.735461) -- (7.96512,0.735461);
\definecolor{c}{rgb}{0,0,0};
\colorlet{c}{natgreen};
\draw [c] (8.01925,0.734632) -- (8.01925,0.743059);
\draw [c] (8.01925,0.743059) -- (8.01925,0.751485);
\draw [c] (8.00121,0.743059) -- (8.01925,0.743059);
\draw [c] (8.01925,0.743059) -- (8.0373,0.743059);
\definecolor{c}{rgb}{0,0,0};
\colorlet{c}{natgreen};
\draw [c] (8.05534,0.734632) -- (8.05534,0.743059);
\draw [c] (8.05534,0.743059) -- (8.05534,0.751485);
\draw [c] (8.0373,0.743059) -- (8.05534,0.743059);
\draw [c] (8.05534,0.743059) -- (8.07339,0.743059);
\definecolor{c}{rgb}{0,0,0};
\colorlet{c}{natgreen};
\draw [c] (8.09143,0.734632) -- (8.09143,0.745781);
\draw [c] (8.09143,0.745781) -- (8.09143,0.756929);
\draw [c] (8.07339,0.745781) -- (8.09143,0.745781);
\draw [c] (8.09143,0.745781) -- (8.10948,0.745781);
\definecolor{c}{rgb}{0,0,0};
\colorlet{c}{natgreen};
\draw [c] (8.12752,0.734632) -- (8.12752,0.740014);
\draw [c] (8.12752,0.740014) -- (8.12752,0.745395);
\draw [c] (8.10948,0.740014) -- (8.12752,0.740014);
\draw [c] (8.12752,0.740014) -- (8.14556,0.740014);
\definecolor{c}{rgb}{0,0,0};
\colorlet{c}{natgreen};
\draw [c] (8.23579,0.734632) -- (8.23579,0.74892);
\draw [c] (8.23579,0.74892) -- (8.23579,0.763207);
\draw [c] (8.21774,0.74892) -- (8.23579,0.74892);
\draw [c] (8.23579,0.74892) -- (8.25383,0.74892);
\definecolor{c}{rgb}{0,0,0};
\colorlet{c}{natgreen};
\draw [c] (8.88538,0.734632) -- (8.88538,0.750764);
\draw [c] (8.88538,0.750764) -- (8.88538,0.766895);
\draw [c] (8.86734,0.750764) -- (8.88538,0.750764);
\draw [c] (8.88538,0.750764) -- (8.90343,0.750764);
\definecolor{c}{rgb}{0,0,0};
\colorlet{c}{natgreen};
\draw [c] (8.95756,0.734632) -- (8.95756,0.748166);
\draw [c] (8.95756,0.748166) -- (8.95756,0.761699);
\draw [c] (8.93952,0.748166) -- (8.95756,0.748166);
\draw [c] (8.95756,0.748166) -- (8.97561,0.748166);
\definecolor{c}{rgb}{0,0,0};
\colorlet{c}{natgreen};
\draw [c] (9.10192,0.734632) -- (9.10192,0.7446);
\draw [c] (9.10192,0.7446) -- (9.10192,0.754568);
\draw [c] (9.08387,0.7446) -- (9.10192,0.7446);
\draw [c] (9.10192,0.7446) -- (9.11996,0.7446);
\definecolor{c}{rgb}{0,0,0};
\colorlet{c}{natgreen};
\draw [c] (9.17409,0.734632) -- (9.17409,0.743059);
\draw [c] (9.17409,0.743059) -- (9.17409,0.751485);
\draw [c] (9.15605,0.743059) -- (9.17409,0.743059);
\draw [c] (9.17409,0.743059) -- (9.19214,0.743059);
\definecolor{c}{rgb}{0,0,0};
\colorlet{c}{natgreen};
\draw [c] (9.53498,0.734632) -- (9.53498,0.746887);
\draw [c] (9.53498,0.746887) -- (9.53498,0.759142);
\draw [c] (9.51694,0.746887) -- (9.53498,0.746887);
\draw [c] (9.53498,0.746887) -- (9.55302,0.746887);
\definecolor{c}{rgb}{0,0,0};
\draw [c] (1,0.680516) -- (9.95,0.680516);
\draw [anchor= east] (9.95,-0.0816619) node[color=c, rotate=0]{$E_{T}^{\text{iso}}\text{ [GeV]}$};
\draw [c] (1,0.863234) -- (1,0.680516);
\draw [c] (1.36089,0.771875) -- (1.36089,0.680516);
\draw [c] (1.72177,0.771875) -- (1.72177,0.680516);
\draw [c] (2.08266,0.771875) -- (2.08266,0.680516);
\draw [c] (2.44355,0.771875) -- (2.44355,0.680516);
\draw [c] (2.80444,0.863234) -- (2.80444,0.680516);
\draw [c] (3.16532,0.771875) -- (3.16532,0.680516);
\draw [c] (3.52621,0.771875) -- (3.52621,0.680516);
\draw [c] (3.8871,0.771875) -- (3.8871,0.680516);
\draw [c] (4.24798,0.771875) -- (4.24798,0.680516);
\draw [c] (4.60887,0.863234) -- (4.60887,0.680516);
\draw [c] (4.96976,0.771875) -- (4.96976,0.680516);
\draw [c] (5.33065,0.771875) -- (5.33065,0.680516);
\draw [c] (5.69153,0.771875) -- (5.69153,0.680516);
\draw [c] (6.05242,0.771875) -- (6.05242,0.680516);
\draw [c] (6.41331,0.863234) -- (6.41331,0.680516);
\draw [c] (6.77419,0.771875) -- (6.77419,0.680516);
\draw [c] (7.13508,0.771875) -- (7.13508,0.680516);
\draw [c] (7.49597,0.771875) -- (7.49597,0.680516);
\draw [c] (7.85685,0.771875) -- (7.85685,0.680516);
\draw [c] (8.21774,0.863234) -- (8.21774,0.680516);
\draw [c] (8.21774,0.863234) -- (8.21774,0.680516);
\draw [c] (8.57863,0.771875) -- (8.57863,0.680516);
\draw [c] (8.93952,0.771875) -- (8.93952,0.680516);
\draw [c] (9.3004,0.771875) -- (9.3004,0.680516);
\draw [c] (9.66129,0.771875) -- (9.66129,0.680516);
\draw [anchor=base] (1,0.285817) node[color=c, rotate=0]{-5};
\draw [anchor=base] (2.80444,0.285817) node[color=c, rotate=0]{0};
\draw [anchor=base] (4.60887,0.285817) node[color=c, rotate=0]{5};
\draw [anchor=base] (6.41331,0.285817) node[color=c, rotate=0]{10};
\draw [anchor=base] (8.21774,0.285817) node[color=c, rotate=0]{15};
\draw [c] (1,0.680516) -- (1,6.73711);
\draw [anchor= east] (-0.4,6.73711) node[color=c, rotate=90]{Normalised number of events};
\draw [c] (1.267,0.734632) -- (1,0.734632);
\draw [c] (1.1335,0.864132) -- (1,0.864132);
\draw [c] (1.1335,0.993631) -- (1,0.993631);
\draw [c] (1.1335,1.12313) -- (1,1.12313);
\draw [c] (1.1335,1.25263) -- (1,1.25263);
\draw [c] (1.267,1.38213) -- (1,1.38213);
\draw [c] (1.1335,1.51163) -- (1,1.51163);
\draw [c] (1.1335,1.64113) -- (1,1.64113);
\draw [c] (1.1335,1.77063) -- (1,1.77063);
\draw [c] (1.1335,1.90013) -- (1,1.90013);
\draw [c] (1.267,2.02963) -- (1,2.02963);
\draw [c] (1.1335,2.15913) -- (1,2.15913);
\draw [c] (1.1335,2.28863) -- (1,2.28863);
\draw [c] (1.1335,2.41812) -- (1,2.41812);
\draw [c] (1.1335,2.54762) -- (1,2.54762);
\draw [c] (1.267,2.67712) -- (1,2.67712);
\draw [c] (1.1335,2.80662) -- (1,2.80662);
\draw [c] (1.1335,2.93612) -- (1,2.93612);
\draw [c] (1.1335,3.06562) -- (1,3.06562);
\draw [c] (1.1335,3.19512) -- (1,3.19512);
\draw [c] (1.267,3.32462) -- (1,3.32462);
\draw [c] (1.1335,3.45412) -- (1,3.45412);
\draw [c] (1.1335,3.58362) -- (1,3.58362);
\draw [c] (1.1335,3.71312) -- (1,3.71312);
\draw [c] (1.1335,3.84262) -- (1,3.84262);
\draw [c] (1.267,3.97212) -- (1,3.97212);
\draw [c] (1.1335,4.10162) -- (1,4.10162);
\draw [c] (1.1335,4.23112) -- (1,4.23112);
\draw [c] (1.1335,4.36062) -- (1,4.36062);
\draw [c] (1.1335,4.49012) -- (1,4.49012);
\draw [c] (1.267,4.61962) -- (1,4.61962);
\draw [c] (1.1335,4.74911) -- (1,4.74911);
\draw [c] (1.1335,4.87861) -- (1,4.87861);
\draw [c] (1.1335,5.00811) -- (1,5.00811);
\draw [c] (1.1335,5.13761) -- (1,5.13761);
\draw [c] (1.267,5.26711) -- (1,5.26711);
\draw [c] (1.1335,5.39661) -- (1,5.39661);
\draw [c] (1.1335,5.52611) -- (1,5.52611);
\draw [c] (1.1335,5.65561) -- (1,5.65561);
\draw [c] (1.1335,5.78511) -- (1,5.78511);
\draw [c] (1.267,5.91461) -- (1,5.91461);
\draw [c] (1.1335,6.04411) -- (1,6.04411);
\draw [c] (1.1335,6.17361) -- (1,6.17361);
\draw [c] (1.1335,6.30311) -- (1,6.30311);
\draw [c] (1.1335,6.43261) -- (1,6.43261);
\draw [c] (1.267,6.56211) -- (1,6.56211);
\draw [c] (1.267,0.734632) -- (1,0.734632);
\draw [c] (1.267,6.56211) -- (1,6.56211);
\draw [c] (1.1335,6.69161) -- (1,6.69161);
\draw [anchor= east] (0.95,0.734632) node[color=c, rotate=0]{0};
\draw [anchor= east] (0.95,1.38213) node[color=c, rotate=0]{100};
\draw [anchor= east] (0.95,2.02963) node[color=c, rotate=0]{200};
\draw [anchor= east] (0.95,2.67712) node[color=c, rotate=0]{300};
\draw [anchor= east] (0.95,3.32462) node[color=c, rotate=0]{400};
\draw [anchor= east] (0.95,3.97212) node[color=c, rotate=0]{500};
\draw [anchor= east] (0.95,4.61962) node[color=c, rotate=0]{600};
\draw [anchor= east] (0.95,5.26711) node[color=c, rotate=0]{700};
\draw [anchor= east] (0.95,5.91461) node[color=c, rotate=0]{800};
\draw [anchor= east] (0.95,6.56211) node[color=c, rotate=0]{900};
\colorlet{c}{natcomp!70};
\draw [c] (1.01804,0.734643) -- (1.01804,0.756364);
\draw [c] (1.01804,0.756364) -- (1.01804,0.778085);
\draw [c] (1,0.756364) -- (1.01804,0.756364);
\draw [c] (1.01804,0.756364) -- (1.03609,0.756364);
\definecolor{c}{rgb}{0,0,0};
\colorlet{c}{natcomp!70};
\draw [c] (1.05413,0.734632) -- (1.05413,0.734644);
\draw [c] (1.05413,0.734644) -- (1.05413,0.734656);
\draw [c] (1.03609,0.734644) -- (1.05413,0.734644);
\draw [c] (1.05413,0.734644) -- (1.07218,0.734644);
\definecolor{c}{rgb}{0,0,0};
\colorlet{c}{natcomp!70};
\draw [c] (1.19849,0.734643) -- (1.19849,0.764169);
\draw [c] (1.19849,0.764169) -- (1.19849,0.793695);
\draw [c] (1.18044,0.764169) -- (1.19849,0.764169);
\draw [c] (1.19849,0.764169) -- (1.21653,0.764169);
\definecolor{c}{rgb}{0,0,0};
\colorlet{c}{natcomp!70};
\draw [c] (1.23458,0.734636) -- (1.23458,0.734645);
\draw [c] (1.23458,0.734645) -- (1.23458,0.734654);
\draw [c] (1.21653,0.734645) -- (1.23458,0.734645);
\draw [c] (1.23458,0.734645) -- (1.25262,0.734645);
\definecolor{c}{rgb}{0,0,0};
\colorlet{c}{natcomp!70};
\draw [c] (1.27067,0.73464) -- (1.27067,0.756361);
\draw [c] (1.27067,0.756361) -- (1.27067,0.778083);
\draw [c] (1.25262,0.756361) -- (1.27067,0.756361);
\draw [c] (1.27067,0.756361) -- (1.28871,0.756361);
\definecolor{c}{rgb}{0,0,0};
\colorlet{c}{natcomp!70};
\draw [c] (1.30675,0.74249) -- (1.30675,0.769762);
\draw [c] (1.30675,0.769762) -- (1.30675,0.797034);
\draw [c] (1.28871,0.769762) -- (1.30675,0.769762);
\draw [c] (1.30675,0.769762) -- (1.3248,0.769762);
\definecolor{c}{rgb}{0,0,0};
\colorlet{c}{natcomp!70};
\draw [c] (1.34284,0.734678) -- (1.34284,0.756399);
\draw [c] (1.34284,0.756399) -- (1.34284,0.77812);
\draw [c] (1.3248,0.756399) -- (1.34284,0.756399);
\draw [c] (1.34284,0.756399) -- (1.36089,0.756399);
\definecolor{c}{rgb}{0,0,0};
\colorlet{c}{natcomp!70};
\draw [c] (1.37893,0.74505) -- (1.37893,0.777385);
\draw [c] (1.37893,0.777385) -- (1.37893,0.80972);
\draw [c] (1.36089,0.777385) -- (1.37893,0.777385);
\draw [c] (1.37893,0.777385) -- (1.39698,0.777385);
\definecolor{c}{rgb}{0,0,0};
\colorlet{c}{natcomp!70};
\draw [c] (1.41502,0.751247) -- (1.41502,0.783932);
\draw [c] (1.41502,0.783932) -- (1.41502,0.816618);
\draw [c] (1.39698,0.783932) -- (1.41502,0.783932);
\draw [c] (1.41502,0.783932) -- (1.43306,0.783932);
\definecolor{c}{rgb}{0,0,0};
\colorlet{c}{natcomp!70};
\draw [c] (1.45111,0.774293) -- (1.45111,0.81838);
\draw [c] (1.45111,0.81838) -- (1.45111,0.862466);
\draw [c] (1.43306,0.81838) -- (1.45111,0.81838);
\draw [c] (1.45111,0.81838) -- (1.46915,0.81838);
\definecolor{c}{rgb}{0,0,0};
\colorlet{c}{natcomp!70};
\draw [c] (1.4872,0.759731) -- (1.4872,0.794569);
\draw [c] (1.4872,0.794569) -- (1.4872,0.829406);
\draw [c] (1.46915,0.794569) -- (1.4872,0.794569);
\draw [c] (1.4872,0.794569) -- (1.50524,0.794569);
\definecolor{c}{rgb}{0,0,0};
\colorlet{c}{natcomp!70};
\draw [c] (1.52329,0.752172) -- (1.52329,0.776524);
\draw [c] (1.52329,0.776524) -- (1.52329,0.800876);
\draw [c] (1.50524,0.776524) -- (1.52329,0.776524);
\draw [c] (1.52329,0.776524) -- (1.54133,0.776524);
\definecolor{c}{rgb}{0,0,0};
\colorlet{c}{natcomp!70};
\draw [c] (1.55938,0.734713) -- (1.55938,0.756435);
\draw [c] (1.55938,0.756435) -- (1.55938,0.778156);
\draw [c] (1.54133,0.756435) -- (1.55938,0.756435);
\draw [c] (1.55938,0.756435) -- (1.57742,0.756435);
\definecolor{c}{rgb}{0,0,0};
\colorlet{c}{natcomp!70};
\draw [c] (1.59546,0.763306) -- (1.59546,0.806425);
\draw [c] (1.59546,0.806425) -- (1.59546,0.849545);
\draw [c] (1.57742,0.806425) -- (1.59546,0.806425);
\draw [c] (1.59546,0.806425) -- (1.61351,0.806425);
\definecolor{c}{rgb}{0,0,0};
\colorlet{c}{natcomp!70};
\draw [c] (1.63155,0.784078) -- (1.63155,0.823256);
\draw [c] (1.63155,0.823256) -- (1.63155,0.862434);
\draw [c] (1.61351,0.823256) -- (1.63155,0.823256);
\draw [c] (1.63155,0.823256) -- (1.6496,0.823256);
\definecolor{c}{rgb}{0,0,0};
\colorlet{c}{natcomp!70};
\draw [c] (1.66764,0.800855) -- (1.66764,0.840914);
\draw [c] (1.66764,0.840914) -- (1.66764,0.880972);
\draw [c] (1.6496,0.840914) -- (1.66764,0.840914);
\draw [c] (1.66764,0.840914) -- (1.68569,0.840914);
\definecolor{c}{rgb}{0,0,0};
\colorlet{c}{natcomp!70};
\draw [c] (1.70373,0.898383) -- (1.70373,0.953898);
\draw [c] (1.70373,0.953898) -- (1.70373,1.00941);
\draw [c] (1.68569,0.953898) -- (1.70373,0.953898);
\draw [c] (1.70373,0.953898) -- (1.72177,0.953898);
\definecolor{c}{rgb}{0,0,0};
\colorlet{c}{natcomp!70};
\draw [c] (1.73982,0.798717) -- (1.73982,0.84012);
\draw [c] (1.73982,0.84012) -- (1.73982,0.881522);
\draw [c] (1.72177,0.84012) -- (1.73982,0.84012);
\draw [c] (1.73982,0.84012) -- (1.75786,0.84012);
\definecolor{c}{rgb}{0,0,0};
\colorlet{c}{natcomp!70};
\draw [c] (1.77591,0.920053) -- (1.77591,0.975511);
\draw [c] (1.77591,0.975511) -- (1.77591,1.03097);
\draw [c] (1.75786,0.975511) -- (1.77591,0.975511);
\draw [c] (1.77591,0.975511) -- (1.79395,0.975511);
\definecolor{c}{rgb}{0,0,0};
\colorlet{c}{natcomp!70};
\draw [c] (1.812,0.936516) -- (1.812,0.998493);
\draw [c] (1.812,0.998493) -- (1.812,1.06047);
\draw [c] (1.79395,0.998493) -- (1.812,0.998493);
\draw [c] (1.812,0.998493) -- (1.83004,0.998493);
\definecolor{c}{rgb}{0,0,0};
\colorlet{c}{natcomp!70};
\draw [c] (1.84808,1.18919) -- (1.84808,1.28414);
\draw [c] (1.84808,1.28414) -- (1.84808,1.3791);
\draw [c] (1.83004,1.28414) -- (1.84808,1.28414);
\draw [c] (1.84808,1.28414) -- (1.86613,1.28414);
\definecolor{c}{rgb}{0,0,0};
\colorlet{c}{natcomp!70};
\draw [c] (1.88417,1.12118) -- (1.88417,1.20085);
\draw [c] (1.88417,1.20085) -- (1.88417,1.28052);
\draw [c] (1.86613,1.20085) -- (1.88417,1.20085);
\draw [c] (1.88417,1.20085) -- (1.90222,1.20085);
\definecolor{c}{rgb}{0,0,0};
\colorlet{c}{natcomp!70};
\draw [c] (1.92026,1.14624) -- (1.92026,1.22811);
\draw [c] (1.92026,1.22811) -- (1.92026,1.30998);
\draw [c] (1.90222,1.22811) -- (1.92026,1.22811);
\draw [c] (1.92026,1.22811) -- (1.93831,1.22811);
\definecolor{c}{rgb}{0,0,0};
\colorlet{c}{natcomp!70};
\draw [c] (1.95635,1.30269) -- (1.95635,1.39258);
\draw [c] (1.95635,1.39258) -- (1.95635,1.48248);
\draw [c] (1.93831,1.39258) -- (1.95635,1.39258);
\draw [c] (1.95635,1.39258) -- (1.9744,1.39258);
\definecolor{c}{rgb}{0,0,0};
\colorlet{c}{natcomp!70};
\draw [c] (1.99244,1.33385) -- (1.99244,1.42513);
\draw [c] (1.99244,1.42513) -- (1.99244,1.51641);
\draw [c] (1.9744,1.42513) -- (1.99244,1.42513);
\draw [c] (1.99244,1.42513) -- (2.01048,1.42513);
\definecolor{c}{rgb}{0,0,0};
\colorlet{c}{natcomp!70};
\draw [c] (2.02853,1.50746) -- (2.02853,1.6117);
\draw [c] (2.02853,1.6117) -- (2.02853,1.71594);
\draw [c] (2.01048,1.6117) -- (2.02853,1.6117);
\draw [c] (2.02853,1.6117) -- (2.04657,1.6117);
\definecolor{c}{rgb}{0,0,0};
\colorlet{c}{natcomp!70};
\draw [c] (2.06462,1.66914) -- (2.06462,1.7831);
\draw [c] (2.06462,1.7831) -- (2.06462,1.89706);
\draw [c] (2.04657,1.7831) -- (2.06462,1.7831);
\draw [c] (2.06462,1.7831) -- (2.08266,1.7831);
\definecolor{c}{rgb}{0,0,0};
\colorlet{c}{natcomp!70};
\draw [c] (2.10071,1.71336) -- (2.10071,1.83223);
\draw [c] (2.10071,1.83223) -- (2.10071,1.9511);
\draw [c] (2.08266,1.83223) -- (2.10071,1.83223);
\draw [c] (2.10071,1.83223) -- (2.11875,1.83223);
\definecolor{c}{rgb}{0,0,0};
\colorlet{c}{natcomp!70};
\draw [c] (2.13679,2.01604) -- (2.13679,2.14978);
\draw [c] (2.13679,2.14978) -- (2.13679,2.28352);
\draw [c] (2.11875,2.14978) -- (2.13679,2.14978);
\draw [c] (2.13679,2.14978) -- (2.15484,2.14978);
\definecolor{c}{rgb}{0,0,0};
\colorlet{c}{natcomp!70};
\draw [c] (2.17288,2.19175) -- (2.17288,2.33083);
\draw [c] (2.17288,2.33083) -- (2.17288,2.46991);
\draw [c] (2.15484,2.33083) -- (2.17288,2.33083);
\draw [c] (2.17288,2.33083) -- (2.19093,2.33083);
\definecolor{c}{rgb}{0,0,0};
\colorlet{c}{natcomp!70};
\draw [c] (2.20897,2.44823) -- (2.20897,2.5985);
\draw [c] (2.20897,2.5985) -- (2.20897,2.74878);
\draw [c] (2.19093,2.5985) -- (2.20897,2.5985);
\draw [c] (2.20897,2.5985) -- (2.22702,2.5985);
\definecolor{c}{rgb}{0,0,0};
\colorlet{c}{natcomp!70};
\draw [c] (2.24506,2.63778) -- (2.24506,2.79551);
\draw [c] (2.24506,2.79551) -- (2.24506,2.95324);
\draw [c] (2.22702,2.79551) -- (2.24506,2.79551);
\draw [c] (2.24506,2.79551) -- (2.2631,2.79551);
\definecolor{c}{rgb}{0,0,0};
\colorlet{c}{natcomp!70};
\draw [c] (2.28115,2.66197) -- (2.28115,2.81917);
\draw [c] (2.28115,2.81917) -- (2.28115,2.97637);
\draw [c] (2.2631,2.81917) -- (2.28115,2.81917);
\draw [c] (2.28115,2.81917) -- (2.29919,2.81917);
\definecolor{c}{rgb}{0,0,0};
\colorlet{c}{natcomp!70};
\draw [c] (2.31724,3.3542) -- (2.31724,3.53769);
\draw [c] (2.31724,3.53769) -- (2.31724,3.72117);
\draw [c] (2.29919,3.53769) -- (2.31724,3.53769);
\draw [c] (2.31724,3.53769) -- (2.33528,3.53769);
\definecolor{c}{rgb}{0,0,0};
\colorlet{c}{natcomp!70};
\draw [c] (2.35333,3.50262) -- (2.35333,3.68753);
\draw [c] (2.35333,3.68753) -- (2.35333,3.87244);
\draw [c] (2.33528,3.68753) -- (2.35333,3.68753);
\draw [c] (2.35333,3.68753) -- (2.37137,3.68753);
\definecolor{c}{rgb}{0,0,0};
\colorlet{c}{natcomp!70};
\draw [c] (2.38942,3.94866) -- (2.38942,4.15279);
\draw [c] (2.38942,4.15279) -- (2.38942,4.35691);
\draw [c] (2.37137,4.15279) -- (2.38942,4.15279);
\draw [c] (2.38942,4.15279) -- (2.40746,4.15279);
\definecolor{c}{rgb}{0,0,0};
\colorlet{c}{natcomp!70};
\draw [c] (2.4255,4.34127) -- (2.4255,4.55624);
\draw [c] (2.4255,4.55624) -- (2.4255,4.77121);
\draw [c] (2.40746,4.55624) -- (2.4255,4.55624);
\draw [c] (2.4255,4.55624) -- (2.44355,4.55624);
\definecolor{c}{rgb}{0,0,0};
\colorlet{c}{natcomp!70};
\draw [c] (2.46159,4.60059) -- (2.46159,4.81922);
\draw [c] (2.46159,4.81922) -- (2.46159,5.03785);
\draw [c] (2.44355,4.81922) -- (2.46159,4.81922);
\draw [c] (2.46159,4.81922) -- (2.47964,4.81922);
\definecolor{c}{rgb}{0,0,0};
\colorlet{c}{natcomp!70};
\draw [c] (2.49768,5.39548) -- (2.49768,5.64371);
\draw [c] (2.49768,5.64371) -- (2.49768,5.89194);
\draw [c] (2.47964,5.64371) -- (2.49768,5.64371);
\draw [c] (2.49768,5.64371) -- (2.51573,5.64371);
\definecolor{c}{rgb}{0,0,0};
\colorlet{c}{natcomp!70};
\draw [c] (2.53377,5.06867) -- (2.53377,5.3066);
\draw [c] (2.53377,5.3066) -- (2.53377,5.54453);
\draw [c] (2.51573,5.3066) -- (2.53377,5.3066);
\draw [c] (2.53377,5.3066) -- (2.55181,5.3066);
\definecolor{c}{rgb}{0,0,0};
\colorlet{c}{natcomp!70};
\draw [c] (2.56986,5.88276) -- (2.56986,6.15509);
\draw [c] (2.56986,6.15509) -- (2.56986,6.42741);
\draw [c] (2.55181,6.15509) -- (2.56986,6.15509);
\draw [c] (2.56986,6.15509) -- (2.5879,6.15509);
\definecolor{c}{rgb}{0,0,0};
\colorlet{c}{natcomp!70};
\draw [c] (2.60595,6.14439) -- (2.60595,6.41923);
\draw [c] (2.60595,6.41923) -- (2.60595,6.69406);
\draw [c] (2.5879,6.41923) -- (2.60595,6.41923);
\draw [c] (2.60595,6.41923) -- (2.62399,6.41923);
\definecolor{c}{rgb}{0,0,0};
\colorlet{c}{natcomp!70};
\draw [c] (2.64204,6.13672) -- (2.64204,6.41349);
\draw [c] (2.64204,6.41349) -- (2.64204,6.69026);
\draw [c] (2.62399,6.41349) -- (2.64204,6.41349);
\draw [c] (2.64204,6.41349) -- (2.66008,6.41349);
\definecolor{c}{rgb}{0,0,0};
\colorlet{c}{natcomp!70};
\draw [c] (2.67813,5.66473) -- (2.67813,5.93277);
\draw [c] (2.67813,5.93277) -- (2.67813,6.20081);
\draw [c] (2.66008,5.93277) -- (2.67813,5.93277);
\draw [c] (2.67813,5.93277) -- (2.69617,5.93277);
\definecolor{c}{rgb}{0,0,0};
\colorlet{c}{natcomp!70};
\draw [c] (2.71421,5.38953) -- (2.71421,5.65592);
\draw [c] (2.71421,5.65592) -- (2.71421,5.92231);
\draw [c] (2.69617,5.65592) -- (2.71421,5.65592);
\draw [c] (2.71421,5.65592) -- (2.73226,5.65592);
\definecolor{c}{rgb}{0,0,0};
\colorlet{c}{natcomp!70};
\draw [c] (2.7503,4.96565) -- (2.7503,5.21733);
\draw [c] (2.7503,5.21733) -- (2.7503,5.46902);
\draw [c] (2.73226,5.21733) -- (2.7503,5.21733);
\draw [c] (2.7503,5.21733) -- (2.76835,5.21733);
\definecolor{c}{rgb}{0,0,0};
\colorlet{c}{natcomp!70};
\draw [c] (2.78639,4.72809) -- (2.78639,4.97622);
\draw [c] (2.78639,4.97622) -- (2.78639,5.22435);
\draw [c] (2.76835,4.97622) -- (2.78639,4.97622);
\draw [c] (2.78639,4.97622) -- (2.80444,4.97622);
\definecolor{c}{rgb}{0,0,0};
\colorlet{c}{natcomp!70};
\draw [c] (2.82248,4.72571) -- (2.82248,4.96387);
\draw [c] (2.82248,4.96387) -- (2.82248,5.20204);
\draw [c] (2.80444,4.96387) -- (2.82248,4.96387);
\draw [c] (2.82248,4.96387) -- (2.84052,4.96387);
\definecolor{c}{rgb}{0,0,0};
\colorlet{c}{natcomp!70};
\draw [c] (2.85857,4.18523) -- (2.85857,4.403);
\draw [c] (2.85857,4.403) -- (2.85857,4.62076);
\draw [c] (2.84052,4.403) -- (2.85857,4.403);
\draw [c] (2.85857,4.403) -- (2.87661,4.403);
\definecolor{c}{rgb}{0,0,0};
\colorlet{c}{natcomp!70};
\draw [c] (2.89466,3.65899) -- (2.89466,3.85657);
\draw [c] (2.89466,3.85657) -- (2.89466,4.05415);
\draw [c] (2.87661,3.85657) -- (2.89466,3.85657);
\draw [c] (2.89466,3.85657) -- (2.9127,3.85657);
\definecolor{c}{rgb}{0,0,0};
\colorlet{c}{natcomp!70};
\draw [c] (2.93075,3.93162) -- (2.93075,4.14189);
\draw [c] (2.93075,4.14189) -- (2.93075,4.35217);
\draw [c] (2.9127,4.14189) -- (2.93075,4.14189);
\draw [c] (2.93075,4.14189) -- (2.94879,4.14189);
\definecolor{c}{rgb}{0,0,0};
\colorlet{c}{natcomp!70};
\draw [c] (2.96683,3.4435) -- (2.96683,3.63714);
\draw [c] (2.96683,3.63714) -- (2.96683,3.83078);
\draw [c] (2.94879,3.63714) -- (2.96683,3.63714);
\draw [c] (2.96683,3.63714) -- (2.98488,3.63714);
\definecolor{c}{rgb}{0,0,0};
\colorlet{c}{natcomp!70};
\draw [c] (3.00292,3.01529) -- (3.00292,3.1925);
\draw [c] (3.00292,3.1925) -- (3.00292,3.36972);
\draw [c] (2.98488,3.1925) -- (3.00292,3.1925);
\draw [c] (3.00292,3.1925) -- (3.02097,3.1925);
\definecolor{c}{rgb}{0,0,0};
\colorlet{c}{natcomp!70};
\draw [c] (3.03901,3.01986) -- (3.03901,3.19798);
\draw [c] (3.03901,3.19798) -- (3.03901,3.3761);
\draw [c] (3.02097,3.19798) -- (3.03901,3.19798);
\draw [c] (3.03901,3.19798) -- (3.05706,3.19798);
\definecolor{c}{rgb}{0,0,0};
\colorlet{c}{natcomp!70};
\draw [c] (3.0751,2.88122) -- (3.0751,3.05473);
\draw [c] (3.0751,3.05473) -- (3.0751,3.22824);
\draw [c] (3.05706,3.05473) -- (3.0751,3.05473);
\draw [c] (3.0751,3.05473) -- (3.09315,3.05473);
\definecolor{c}{rgb}{0,0,0};
\colorlet{c}{natcomp!70};
\draw [c] (3.11119,2.58374) -- (3.11119,2.74401);
\draw [c] (3.11119,2.74401) -- (3.11119,2.90429);
\draw [c] (3.09315,2.74401) -- (3.11119,2.74401);
\draw [c] (3.11119,2.74401) -- (3.12923,2.74401);
\definecolor{c}{rgb}{0,0,0};
\colorlet{c}{natcomp!70};
\draw [c] (3.14728,2.30093) -- (3.14728,2.44861);
\draw [c] (3.14728,2.44861) -- (3.14728,2.59629);
\draw [c] (3.12923,2.44861) -- (3.14728,2.44861);
\draw [c] (3.14728,2.44861) -- (3.16532,2.44861);
\definecolor{c}{rgb}{0,0,0};
\colorlet{c}{natcomp!70};
\draw [c] (3.18337,2.3983) -- (3.18337,2.55001);
\draw [c] (3.18337,2.55001) -- (3.18337,2.70172);
\draw [c] (3.16532,2.55001) -- (3.18337,2.55001);
\draw [c] (3.18337,2.55001) -- (3.20141,2.55001);
\definecolor{c}{rgb}{0,0,0};
\colorlet{c}{natcomp!70};
\draw [c] (3.21946,2.43359) -- (3.21946,2.5937);
\draw [c] (3.21946,2.5937) -- (3.21946,2.75381);
\draw [c] (3.20141,2.5937) -- (3.21946,2.5937);
\draw [c] (3.21946,2.5937) -- (3.2375,2.5937);
\definecolor{c}{rgb}{0,0,0};
\colorlet{c}{natcomp!70};
\draw [c] (3.25554,2.18533) -- (3.25554,2.32113);
\draw [c] (3.25554,2.32113) -- (3.25554,2.45694);
\draw [c] (3.2375,2.32113) -- (3.25554,2.32113);
\draw [c] (3.25554,2.32113) -- (3.27359,2.32113);
\definecolor{c}{rgb}{0,0,0};
\colorlet{c}{natcomp!70};
\draw [c] (3.29163,2.0191) -- (3.29163,2.15263);
\draw [c] (3.29163,2.15263) -- (3.29163,2.28615);
\draw [c] (3.27359,2.15263) -- (3.29163,2.15263);
\draw [c] (3.29163,2.15263) -- (3.30968,2.15263);
\definecolor{c}{rgb}{0,0,0};
\colorlet{c}{natcomp!70};
\draw [c] (3.32772,2.11944) -- (3.32772,2.256);
\draw [c] (3.32772,2.256) -- (3.32772,2.39256);
\draw [c] (3.30968,2.256) -- (3.32772,2.256);
\draw [c] (3.32772,2.256) -- (3.34577,2.256);
\definecolor{c}{rgb}{0,0,0};
\colorlet{c}{natcomp!70};
\draw [c] (3.36381,1.88428) -- (3.36381,2.00683);
\draw [c] (3.36381,2.00683) -- (3.36381,2.12938);
\draw [c] (3.34577,2.00683) -- (3.36381,2.00683);
\draw [c] (3.36381,2.00683) -- (3.38185,2.00683);
\definecolor{c}{rgb}{0,0,0};
\colorlet{c}{natcomp!70};
\draw [c] (3.3999,1.92813) -- (3.3999,2.05525);
\draw [c] (3.3999,2.05525) -- (3.3999,2.18238);
\draw [c] (3.38185,2.05525) -- (3.3999,2.05525);
\draw [c] (3.3999,2.05525) -- (3.41794,2.05525);
\definecolor{c}{rgb}{0,0,0};
\colorlet{c}{natcomp!70};
\draw [c] (3.43599,1.74122) -- (3.43599,1.86638);
\draw [c] (3.43599,1.86638) -- (3.43599,1.99155);
\draw [c] (3.41794,1.86638) -- (3.43599,1.86638);
\draw [c] (3.43599,1.86638) -- (3.45403,1.86638);
\definecolor{c}{rgb}{0,0,0};
\colorlet{c}{natcomp!70};
\draw [c] (3.47208,1.81489) -- (3.47208,1.94716);
\draw [c] (3.47208,1.94716) -- (3.47208,2.07944);
\draw [c] (3.45403,1.94716) -- (3.47208,1.94716);
\draw [c] (3.47208,1.94716) -- (3.49012,1.94716);
\definecolor{c}{rgb}{0,0,0};
\colorlet{c}{natcomp!70};
\draw [c] (3.50817,1.85856) -- (3.50817,1.99328);
\draw [c] (3.50817,1.99328) -- (3.50817,2.128);
\draw [c] (3.49012,1.99328) -- (3.50817,1.99328);
\draw [c] (3.50817,1.99328) -- (3.52621,1.99328);
\definecolor{c}{rgb}{0,0,0};
\colorlet{c}{natcomp!70};
\draw [c] (3.54425,1.51474) -- (3.54425,1.62467);
\draw [c] (3.54425,1.62467) -- (3.54425,1.7346);
\draw [c] (3.52621,1.62467) -- (3.54425,1.62467);
\draw [c] (3.54425,1.62467) -- (3.5623,1.62467);
\definecolor{c}{rgb}{0,0,0};
\colorlet{c}{natcomp!70};
\draw [c] (3.58034,1.58337) -- (3.58034,1.69582);
\draw [c] (3.58034,1.69582) -- (3.58034,1.80827);
\draw [c] (3.5623,1.69582) -- (3.58034,1.69582);
\draw [c] (3.58034,1.69582) -- (3.59839,1.69582);
\definecolor{c}{rgb}{0,0,0};
\colorlet{c}{natcomp!70};
\draw [c] (3.61643,1.54794) -- (3.61643,1.65528);
\draw [c] (3.61643,1.65528) -- (3.61643,1.76262);
\draw [c] (3.59839,1.65528) -- (3.61643,1.65528);
\draw [c] (3.61643,1.65528) -- (3.63448,1.65528);
\definecolor{c}{rgb}{0,0,0};
\colorlet{c}{natcomp!70};
\draw [c] (3.65252,1.42522) -- (3.65252,1.52375);
\draw [c] (3.65252,1.52375) -- (3.65252,1.62228);
\draw [c] (3.63448,1.52375) -- (3.65252,1.52375);
\draw [c] (3.65252,1.52375) -- (3.67056,1.52375);
\definecolor{c}{rgb}{0,0,0};
\colorlet{c}{natcomp!70};
\draw [c] (3.68861,1.31417) -- (3.68861,1.40383);
\draw [c] (3.68861,1.40383) -- (3.68861,1.4935);
\draw [c] (3.67056,1.40383) -- (3.68861,1.40383);
\draw [c] (3.68861,1.40383) -- (3.70665,1.40383);
\definecolor{c}{rgb}{0,0,0};
\colorlet{c}{natcomp!70};
\draw [c] (3.7247,1.39879) -- (3.7247,1.50121);
\draw [c] (3.7247,1.50121) -- (3.7247,1.60364);
\draw [c] (3.70665,1.50121) -- (3.7247,1.50121);
\draw [c] (3.7247,1.50121) -- (3.74274,1.50121);
\definecolor{c}{rgb}{0,0,0};
\colorlet{c}{natcomp!70};
\draw [c] (3.76079,1.47938) -- (3.76079,1.57989);
\draw [c] (3.76079,1.57989) -- (3.76079,1.68041);
\draw [c] (3.74274,1.57989) -- (3.76079,1.57989);
\draw [c] (3.76079,1.57989) -- (3.77883,1.57989);
\definecolor{c}{rgb}{0,0,0};
\colorlet{c}{natcomp!70};
\draw [c] (3.79688,1.37064) -- (3.79688,1.46625);
\draw [c] (3.79688,1.46625) -- (3.79688,1.56186);
\draw [c] (3.77883,1.46625) -- (3.79688,1.46625);
\draw [c] (3.79688,1.46625) -- (3.81492,1.46625);
\definecolor{c}{rgb}{0,0,0};
\colorlet{c}{natcomp!70};
\draw [c] (3.83296,1.48496) -- (3.83296,1.58921);
\draw [c] (3.83296,1.58921) -- (3.83296,1.69345);
\draw [c] (3.81492,1.58921) -- (3.83296,1.58921);
\draw [c] (3.83296,1.58921) -- (3.85101,1.58921);
\definecolor{c}{rgb}{0,0,0};
\colorlet{c}{natcomp!70};
\draw [c] (3.86905,1.16297) -- (3.86905,1.24409);
\draw [c] (3.86905,1.24409) -- (3.86905,1.3252);
\draw [c] (3.85101,1.24409) -- (3.86905,1.24409);
\draw [c] (3.86905,1.24409) -- (3.8871,1.24409);
\definecolor{c}{rgb}{0,0,0};
\colorlet{c}{natcomp!70};
\draw [c] (3.90514,1.26804) -- (3.90514,1.35812);
\draw [c] (3.90514,1.35812) -- (3.90514,1.4482);
\draw [c] (3.8871,1.35812) -- (3.90514,1.35812);
\draw [c] (3.90514,1.35812) -- (3.92319,1.35812);
\definecolor{c}{rgb}{0,0,0};
\colorlet{c}{natcomp!70};
\draw [c] (3.94123,1.22039) -- (3.94123,1.30135);
\draw [c] (3.94123,1.30135) -- (3.94123,1.38232);
\draw [c] (3.92319,1.30135) -- (3.94123,1.30135);
\draw [c] (3.94123,1.30135) -- (3.95927,1.30135);
\definecolor{c}{rgb}{0,0,0};
\colorlet{c}{natcomp!70};
\draw [c] (3.97732,1.12845) -- (3.97732,1.20076);
\draw [c] (3.97732,1.20076) -- (3.97732,1.27306);
\draw [c] (3.95927,1.20076) -- (3.97732,1.20076);
\draw [c] (3.97732,1.20076) -- (3.99536,1.20076);
\definecolor{c}{rgb}{0,0,0};
\colorlet{c}{natcomp!70};
\draw [c] (4.01341,1.29099) -- (4.01341,1.38407);
\draw [c] (4.01341,1.38407) -- (4.01341,1.47715);
\draw [c] (3.99536,1.38407) -- (4.01341,1.38407);
\draw [c] (4.01341,1.38407) -- (4.03145,1.38407);
\definecolor{c}{rgb}{0,0,0};
\colorlet{c}{natcomp!70};
\draw [c] (4.0495,1.08142) -- (4.0495,1.15343);
\draw [c] (4.0495,1.15343) -- (4.0495,1.22545);
\draw [c] (4.03145,1.15343) -- (4.0495,1.15343);
\draw [c] (4.0495,1.15343) -- (4.06754,1.15343);
\definecolor{c}{rgb}{0,0,0};
\colorlet{c}{natcomp!70};
\draw [c] (4.08558,1.18677) -- (4.08558,1.27028);
\draw [c] (4.08558,1.27028) -- (4.08558,1.35379);
\draw [c] (4.06754,1.27028) -- (4.08558,1.27028);
\draw [c] (4.08558,1.27028) -- (4.10363,1.27028);
\definecolor{c}{rgb}{0,0,0};
\colorlet{c}{natcomp!70};
\draw [c] (4.12167,1.14251) -- (4.12167,1.21997);
\draw [c] (4.12167,1.21997) -- (4.12167,1.29744);
\draw [c] (4.10363,1.21997) -- (4.12167,1.21997);
\draw [c] (4.12167,1.21997) -- (4.13972,1.21997);
\definecolor{c}{rgb}{0,0,0};
\colorlet{c}{natcomp!70};
\draw [c] (4.15776,1.11761) -- (4.15776,1.20294);
\draw [c] (4.15776,1.20294) -- (4.15776,1.28827);
\draw [c] (4.13972,1.20294) -- (4.15776,1.20294);
\draw [c] (4.15776,1.20294) -- (4.17581,1.20294);
\definecolor{c}{rgb}{0,0,0};
\colorlet{c}{natcomp!70};
\draw [c] (4.19385,1.17105) -- (4.19385,1.25955);
\draw [c] (4.19385,1.25955) -- (4.19385,1.34804);
\draw [c] (4.17581,1.25955) -- (4.19385,1.25955);
\draw [c] (4.19385,1.25955) -- (4.21189,1.25955);
\definecolor{c}{rgb}{0,0,0};
\colorlet{c}{natcomp!70};
\draw [c] (4.22994,1.10263) -- (4.22994,1.17065);
\draw [c] (4.22994,1.17065) -- (4.22994,1.23867);
\draw [c] (4.21189,1.17065) -- (4.22994,1.17065);
\draw [c] (4.22994,1.17065) -- (4.24798,1.17065);
\definecolor{c}{rgb}{0,0,0};
\colorlet{c}{natcomp!70};
\draw [c] (4.26603,1.06985) -- (4.26603,1.1406);
\draw [c] (4.26603,1.1406) -- (4.26603,1.21135);
\draw [c] (4.24798,1.1406) -- (4.26603,1.1406);
\draw [c] (4.26603,1.1406) -- (4.28407,1.1406);
\definecolor{c}{rgb}{0,0,0};
\colorlet{c}{natcomp!70};
\draw [c] (4.30212,1.04355) -- (4.30212,1.1102);
\draw [c] (4.30212,1.1102) -- (4.30212,1.17684);
\draw [c] (4.28407,1.1102) -- (4.30212,1.1102);
\draw [c] (4.30212,1.1102) -- (4.32016,1.1102);
\definecolor{c}{rgb}{0,0,0};
\colorlet{c}{natcomp!70};
\draw [c] (4.33821,1.01865) -- (4.33821,1.09854);
\draw [c] (4.33821,1.09854) -- (4.33821,1.17843);
\draw [c] (4.32016,1.09854) -- (4.33821,1.09854);
\draw [c] (4.33821,1.09854) -- (4.35625,1.09854);
\definecolor{c}{rgb}{0,0,0};
\colorlet{c}{natcomp!70};
\draw [c] (4.37429,0.906879) -- (4.37429,0.955253);
\draw [c] (4.37429,0.955253) -- (4.37429,1.00363);
\draw [c] (4.35625,0.955253) -- (4.37429,0.955253);
\draw [c] (4.37429,0.955253) -- (4.39234,0.955253);
\definecolor{c}{rgb}{0,0,0};
\colorlet{c}{natcomp!70};
\draw [c] (4.41038,0.926954) -- (4.41038,0.983752);
\draw [c] (4.41038,0.983752) -- (4.41038,1.04055);
\draw [c] (4.39234,0.983752) -- (4.41038,0.983752);
\draw [c] (4.41038,0.983752) -- (4.42843,0.983752);
\definecolor{c}{rgb}{0,0,0};
\colorlet{c}{natcomp!70};
\draw [c] (4.44647,1.03078) -- (4.44647,1.10645);
\draw [c] (4.44647,1.10645) -- (4.44647,1.18211);
\draw [c] (4.42843,1.10645) -- (4.44647,1.10645);
\draw [c] (4.44647,1.10645) -- (4.46452,1.10645);
\definecolor{c}{rgb}{0,0,0};
\colorlet{c}{natcomp!70};
\draw [c] (4.48256,0.96438) -- (4.48256,1.02395);
\draw [c] (4.48256,1.02395) -- (4.48256,1.08353);
\draw [c] (4.46452,1.02395) -- (4.48256,1.02395);
\draw [c] (4.48256,1.02395) -- (4.5006,1.02395);
\definecolor{c}{rgb}{0,0,0};
\colorlet{c}{natcomp!70};
\draw [c] (4.51865,1.01953) -- (4.51865,1.09213);
\draw [c] (4.51865,1.09213) -- (4.51865,1.16473);
\draw [c] (4.5006,1.09213) -- (4.51865,1.09213);
\draw [c] (4.51865,1.09213) -- (4.53669,1.09213);
\definecolor{c}{rgb}{0,0,0};
\colorlet{c}{natcomp!70};
\draw [c] (4.55474,0.89496) -- (4.55474,0.945912);
\draw [c] (4.55474,0.945912) -- (4.55474,0.996863);
\draw [c] (4.53669,0.945912) -- (4.55474,0.945912);
\draw [c] (4.55474,0.945912) -- (4.57278,0.945912);
\definecolor{c}{rgb}{0,0,0};
\colorlet{c}{natcomp!70};
\draw [c] (4.59083,0.986726) -- (4.59083,1.04818);
\draw [c] (4.59083,1.04818) -- (4.59083,1.10964);
\draw [c] (4.57278,1.04818) -- (4.59083,1.04818);
\draw [c] (4.59083,1.04818) -- (4.60887,1.04818);
\definecolor{c}{rgb}{0,0,0};
\colorlet{c}{natcomp!70};
\draw [c] (4.62692,0.828778) -- (4.62692,0.867519);
\draw [c] (4.62692,0.867519) -- (4.62692,0.90626);
\draw [c] (4.60887,0.867519) -- (4.62692,0.867519);
\draw [c] (4.62692,0.867519) -- (4.64496,0.867519);
\definecolor{c}{rgb}{0,0,0};
\colorlet{c}{natcomp!70};
\draw [c] (4.663,0.929865) -- (4.663,0.982585);
\draw [c] (4.663,0.982585) -- (4.663,1.03531);
\draw [c] (4.64496,0.982585) -- (4.663,0.982585);
\draw [c] (4.663,0.982585) -- (4.68105,0.982585);
\definecolor{c}{rgb}{0,0,0};
\colorlet{c}{natcomp!70};
\draw [c] (4.69909,0.937887) -- (4.69909,1.0001);
\draw [c] (4.69909,1.0001) -- (4.69909,1.06232);
\draw [c] (4.68105,1.0001) -- (4.69909,1.0001);
\draw [c] (4.69909,1.0001) -- (4.71714,1.0001);
\definecolor{c}{rgb}{0,0,0};
\colorlet{c}{natcomp!70};
\draw [c] (4.73518,0.910001) -- (4.73518,0.961309);
\draw [c] (4.73518,0.961309) -- (4.73518,1.01262);
\draw [c] (4.71714,0.961309) -- (4.73518,0.961309);
\draw [c] (4.73518,0.961309) -- (4.75323,0.961309);
\definecolor{c}{rgb}{0,0,0};
\colorlet{c}{natcomp!70};
\draw [c] (4.77127,0.868019) -- (4.77127,0.914493);
\draw [c] (4.77127,0.914493) -- (4.77127,0.960967);
\draw [c] (4.75323,0.914493) -- (4.77127,0.914493);
\draw [c] (4.77127,0.914493) -- (4.78931,0.914493);
\definecolor{c}{rgb}{0,0,0};
\colorlet{c}{natcomp!70};
\draw [c] (4.80736,0.878272) -- (4.80736,0.930742);
\draw [c] (4.80736,0.930742) -- (4.80736,0.983212);
\draw [c] (4.78931,0.930742) -- (4.80736,0.930742);
\draw [c] (4.80736,0.930742) -- (4.8254,0.930742);
\definecolor{c}{rgb}{0,0,0};
\colorlet{c}{natcomp!70};
\draw [c] (4.84345,0.809312) -- (4.84345,0.84424);
\draw [c] (4.84345,0.84424) -- (4.84345,0.879168);
\draw [c] (4.8254,0.84424) -- (4.84345,0.84424);
\draw [c] (4.84345,0.84424) -- (4.86149,0.84424);
\definecolor{c}{rgb}{0,0,0};
\colorlet{c}{natcomp!70};
\draw [c] (4.87954,0.762677) -- (4.87954,0.78561);
\draw [c] (4.87954,0.78561) -- (4.87954,0.808544);
\draw [c] (4.86149,0.78561) -- (4.87954,0.78561);
\draw [c] (4.87954,0.78561) -- (4.89758,0.78561);
\definecolor{c}{rgb}{0,0,0};
\colorlet{c}{natcomp!70};
\draw [c] (4.91563,0.835344) -- (4.91563,0.874984);
\draw [c] (4.91563,0.874984) -- (4.91563,0.914624);
\draw [c] (4.89758,0.874984) -- (4.91563,0.874984);
\draw [c] (4.91563,0.874984) -- (4.93367,0.874984);
\definecolor{c}{rgb}{0,0,0};
\colorlet{c}{natcomp!70};
\draw [c] (4.95171,0.87042) -- (4.95171,0.914521);
\draw [c] (4.95171,0.914521) -- (4.95171,0.958622);
\draw [c] (4.93367,0.914521) -- (4.95171,0.914521);
\draw [c] (4.95171,0.914521) -- (4.96976,0.914521);
\definecolor{c}{rgb}{0,0,0};
\colorlet{c}{natcomp!70};
\draw [c] (4.9878,0.826334) -- (4.9878,0.870592);
\draw [c] (4.9878,0.870592) -- (4.9878,0.91485);
\draw [c] (4.96976,0.870592) -- (4.9878,0.870592);
\draw [c] (4.9878,0.870592) -- (5.00585,0.870592);
\definecolor{c}{rgb}{0,0,0};
\colorlet{c}{natcomp!70};
\draw [c] (5.02389,0.822381) -- (5.02389,0.86059);
\draw [c] (5.02389,0.86059) -- (5.02389,0.898798);
\draw [c] (5.00585,0.86059) -- (5.02389,0.86059);
\draw [c] (5.02389,0.86059) -- (5.04194,0.86059);
\definecolor{c}{rgb}{0,0,0};
\colorlet{c}{natcomp!70};
\draw [c] (5.05998,0.788662) -- (5.05998,0.838806);
\draw [c] (5.05998,0.838806) -- (5.05998,0.88895);
\draw [c] (5.04194,0.838806) -- (5.05998,0.838806);
\draw [c] (5.05998,0.838806) -- (5.07802,0.838806);
\definecolor{c}{rgb}{0,0,0};
\colorlet{c}{natcomp!70};
\draw [c] (5.09607,0.81529) -- (5.09607,0.850679);
\draw [c] (5.09607,0.850679) -- (5.09607,0.886068);
\draw [c] (5.07802,0.850679) -- (5.09607,0.850679);
\draw [c] (5.09607,0.850679) -- (5.11411,0.850679);
\definecolor{c}{rgb}{0,0,0};
\colorlet{c}{natcomp!70};
\draw [c] (5.13216,0.877018) -- (5.13216,0.93804);
\draw [c] (5.13216,0.93804) -- (5.13216,0.999062);
\draw [c] (5.11411,0.93804) -- (5.13216,0.93804);
\draw [c] (5.13216,0.93804) -- (5.1502,0.93804);
\definecolor{c}{rgb}{0,0,0};
\colorlet{c}{natcomp!70};
\draw [c] (5.16825,0.815589) -- (5.16825,0.856866);
\draw [c] (5.16825,0.856866) -- (5.16825,0.898142);
\draw [c] (5.1502,0.856866) -- (5.16825,0.856866);
\draw [c] (5.16825,0.856866) -- (5.18629,0.856866);
\definecolor{c}{rgb}{0,0,0};
\colorlet{c}{natcomp!70};
\draw [c] (5.20433,0.740854) -- (5.20433,0.755731);
\draw [c] (5.20433,0.755731) -- (5.20433,0.770607);
\draw [c] (5.18629,0.755731) -- (5.20433,0.755731);
\draw [c] (5.20433,0.755731) -- (5.22238,0.755731);
\definecolor{c}{rgb}{0,0,0};
\colorlet{c}{natcomp!70};
\draw [c] (5.24042,0.812639) -- (5.24042,0.85358);
\draw [c] (5.24042,0.85358) -- (5.24042,0.89452);
\draw [c] (5.22238,0.85358) -- (5.24042,0.85358);
\draw [c] (5.24042,0.85358) -- (5.25847,0.85358);
\definecolor{c}{rgb}{0,0,0};
\colorlet{c}{natcomp!70};
\draw [c] (5.27651,0.803111) -- (5.27651,0.844052);
\draw [c] (5.27651,0.844052) -- (5.27651,0.884993);
\draw [c] (5.25847,0.844052) -- (5.27651,0.844052);
\draw [c] (5.27651,0.844052) -- (5.29456,0.844052);
\definecolor{c}{rgb}{0,0,0};
\colorlet{c}{natcomp!70};
\draw [c] (5.3126,0.800844) -- (5.3126,0.834967);
\draw [c] (5.3126,0.834967) -- (5.3126,0.869089);
\draw [c] (5.29456,0.834967) -- (5.3126,0.834967);
\draw [c] (5.3126,0.834967) -- (5.33065,0.834967);
\definecolor{c}{rgb}{0,0,0};
\colorlet{c}{natcomp!70};
\draw [c] (5.34869,0.784644) -- (5.34869,0.815177);
\draw [c] (5.34869,0.815177) -- (5.34869,0.845709);
\draw [c] (5.33065,0.815177) -- (5.34869,0.815177);
\draw [c] (5.34869,0.815177) -- (5.36673,0.815177);
\definecolor{c}{rgb}{0,0,0};
\colorlet{c}{natcomp!70};
\draw [c] (5.38478,0.802543) -- (5.38478,0.837564);
\draw [c] (5.38478,0.837564) -- (5.38478,0.872585);
\draw [c] (5.36673,0.837564) -- (5.38478,0.837564);
\draw [c] (5.38478,0.837564) -- (5.40282,0.837564);
\definecolor{c}{rgb}{0,0,0};
\colorlet{c}{natcomp!70};
\draw [c] (5.42087,0.838708) -- (5.42087,0.877281);
\draw [c] (5.42087,0.877281) -- (5.42087,0.915853);
\draw [c] (5.40282,0.877281) -- (5.42087,0.877281);
\draw [c] (5.42087,0.877281) -- (5.43891,0.877281);
\definecolor{c}{rgb}{0,0,0};
\colorlet{c}{natcomp!70};
\draw [c] (5.45696,0.828845) -- (5.45696,0.868725);
\draw [c] (5.45696,0.868725) -- (5.45696,0.908606);
\draw [c] (5.43891,0.868725) -- (5.45696,0.868725);
\draw [c] (5.45696,0.868725) -- (5.475,0.868725);
\definecolor{c}{rgb}{0,0,0};
\colorlet{c}{natcomp!70};
\draw [c] (5.49304,0.776612) -- (5.49304,0.806357);
\draw [c] (5.49304,0.806357) -- (5.49304,0.836102);
\draw [c] (5.475,0.806357) -- (5.49304,0.806357);
\draw [c] (5.49304,0.806357) -- (5.51109,0.806357);
\definecolor{c}{rgb}{0,0,0};
\colorlet{c}{natcomp!70};
\draw [c] (5.52913,0.765895) -- (5.52913,0.792266);
\draw [c] (5.52913,0.792266) -- (5.52913,0.818637);
\draw [c] (5.51109,0.792266) -- (5.52913,0.792266);
\draw [c] (5.52913,0.792266) -- (5.54718,0.792266);
\definecolor{c}{rgb}{0,0,0};
\colorlet{c}{natcomp!70};
\draw [c] (5.56522,0.772359) -- (5.56522,0.799594);
\draw [c] (5.56522,0.799594) -- (5.56522,0.82683);
\draw [c] (5.54718,0.799594) -- (5.56522,0.799594);
\draw [c] (5.56522,0.799594) -- (5.58327,0.799594);
\definecolor{c}{rgb}{0,0,0};
\colorlet{c}{natcomp!70};
\draw [c] (5.60131,0.796115) -- (5.60131,0.832186);
\draw [c] (5.60131,0.832186) -- (5.60131,0.868257);
\draw [c] (5.58327,0.832186) -- (5.60131,0.832186);
\draw [c] (5.60131,0.832186) -- (5.61935,0.832186);
\definecolor{c}{rgb}{0,0,0};
\colorlet{c}{natcomp!70};
\draw [c] (5.6374,0.781953) -- (5.6374,0.811129);
\draw [c] (5.6374,0.811129) -- (5.6374,0.840305);
\draw [c] (5.61935,0.811129) -- (5.6374,0.811129);
\draw [c] (5.6374,0.811129) -- (5.65544,0.811129);
\definecolor{c}{rgb}{0,0,0};
\colorlet{c}{natcomp!70};
\draw [c] (5.67349,0.81513) -- (5.67349,0.856508);
\draw [c] (5.67349,0.856508) -- (5.67349,0.897886);
\draw [c] (5.65544,0.856508) -- (5.67349,0.856508);
\draw [c] (5.67349,0.856508) -- (5.69153,0.856508);
\definecolor{c}{rgb}{0,0,0};
\colorlet{c}{natcomp!70};
\draw [c] (5.70958,0.792858) -- (5.70958,0.826425);
\draw [c] (5.70958,0.826425) -- (5.70958,0.859993);
\draw [c] (5.69153,0.826425) -- (5.70958,0.826425);
\draw [c] (5.70958,0.826425) -- (5.72762,0.826425);
\definecolor{c}{rgb}{0,0,0};
\colorlet{c}{natcomp!70};
\draw [c] (5.74567,0.749354) -- (5.74567,0.771177);
\draw [c] (5.74567,0.771177) -- (5.74567,0.793);
\draw [c] (5.72762,0.771177) -- (5.74567,0.771177);
\draw [c] (5.74567,0.771177) -- (5.76371,0.771177);
\definecolor{c}{rgb}{0,0,0};
\colorlet{c}{natcomp!70};
\draw [c] (5.78175,0.776801) -- (5.78175,0.80719);
\draw [c] (5.78175,0.80719) -- (5.78175,0.837579);
\draw [c] (5.76371,0.80719) -- (5.78175,0.80719);
\draw [c] (5.78175,0.80719) -- (5.7998,0.80719);
\definecolor{c}{rgb}{0,0,0};
\colorlet{c}{natcomp!70};
\draw [c] (5.81784,0.745067) -- (5.81784,0.777402);
\draw [c] (5.81784,0.777402) -- (5.81784,0.809737);
\draw [c] (5.7998,0.777402) -- (5.81784,0.777402);
\draw [c] (5.81784,0.777402) -- (5.83589,0.777402);
\definecolor{c}{rgb}{0,0,0};
\colorlet{c}{natcomp!70};
\draw [c] (5.85393,0.761061) -- (5.85393,0.782893);
\draw [c] (5.85393,0.782893) -- (5.85393,0.804725);
\draw [c] (5.83589,0.782893) -- (5.85393,0.782893);
\draw [c] (5.85393,0.782893) -- (5.87198,0.782893);
\definecolor{c}{rgb}{0,0,0};
\colorlet{c}{natcomp!70};
\draw [c] (5.89002,0.749941) -- (5.89002,0.771033);
\draw [c] (5.89002,0.771033) -- (5.89002,0.792126);
\draw [c] (5.87198,0.771033) -- (5.89002,0.771033);
\draw [c] (5.89002,0.771033) -- (5.90806,0.771033);
\definecolor{c}{rgb}{0,0,0};
\colorlet{c}{natcomp!70};
\draw [c] (5.92611,0.77436) -- (5.92611,0.80203);
\draw [c] (5.92611,0.80203) -- (5.92611,0.829699);
\draw [c] (5.90806,0.80203) -- (5.92611,0.80203);
\draw [c] (5.92611,0.80203) -- (5.94415,0.80203);
\definecolor{c}{rgb}{0,0,0};
\colorlet{c}{natcomp!70};
\draw [c] (5.9622,0.74754) -- (5.9622,0.765501);
\draw [c] (5.9622,0.765501) -- (5.9622,0.783462);
\draw [c] (5.94415,0.765501) -- (5.9622,0.765501);
\draw [c] (5.9622,0.765501) -- (5.98024,0.765501);
\definecolor{c}{rgb}{0,0,0};
\colorlet{c}{natcomp!70};
\draw [c] (5.99829,0.734669) -- (5.99829,0.747312);
\draw [c] (5.99829,0.747312) -- (5.99829,0.759955);
\draw [c] (5.98024,0.747312) -- (5.99829,0.747312);
\draw [c] (5.99829,0.747312) -- (6.01633,0.747312);
\definecolor{c}{rgb}{0,0,0};
\colorlet{c}{natcomp!70};
\draw [c] (6.03438,0.734644) -- (6.03438,0.746862);
\draw [c] (6.03438,0.746862) -- (6.03438,0.759079);
\draw [c] (6.01633,0.746862) -- (6.03438,0.746862);
\draw [c] (6.03438,0.746862) -- (6.05242,0.746862);
\definecolor{c}{rgb}{0,0,0};
\colorlet{c}{natcomp!70};
\draw [c] (6.07046,0.75936) -- (6.07046,0.784951);
\draw [c] (6.07046,0.784951) -- (6.07046,0.810543);
\draw [c] (6.05242,0.784951) -- (6.07046,0.784951);
\draw [c] (6.07046,0.784951) -- (6.08851,0.784951);
\definecolor{c}{rgb}{0,0,0};
\colorlet{c}{natcomp!70};
\draw [c] (6.10655,0.784459) -- (6.10655,0.823852);
\draw [c] (6.10655,0.823852) -- (6.10655,0.863245);
\draw [c] (6.08851,0.823852) -- (6.10655,0.823852);
\draw [c] (6.10655,0.823852) -- (6.1246,0.823852);
\definecolor{c}{rgb}{0,0,0};
\colorlet{c}{natcomp!70};
\draw [c] (6.14264,0.734662) -- (6.14264,0.734685);
\draw [c] (6.14264,0.734685) -- (6.14264,0.734709);
\draw [c] (6.1246,0.734685) -- (6.14264,0.734685);
\draw [c] (6.14264,0.734685) -- (6.16069,0.734685);
\definecolor{c}{rgb}{0,0,0};
\colorlet{c}{natcomp!70};
\draw [c] (6.17873,0.769006) -- (6.17873,0.797565);
\draw [c] (6.17873,0.797565) -- (6.17873,0.826124);
\draw [c] (6.16069,0.797565) -- (6.17873,0.797565);
\draw [c] (6.17873,0.797565) -- (6.19677,0.797565);
\definecolor{c}{rgb}{0,0,0};
\colorlet{c}{natcomp!70};
\draw [c] (6.21482,0.740767) -- (6.21482,0.75556);
\draw [c] (6.21482,0.75556) -- (6.21482,0.770352);
\draw [c] (6.19677,0.75556) -- (6.21482,0.75556);
\draw [c] (6.21482,0.75556) -- (6.23286,0.75556);
\definecolor{c}{rgb}{0,0,0};
\colorlet{c}{natcomp!70};
\draw [c] (6.25091,0.742869) -- (6.25091,0.767082);
\draw [c] (6.25091,0.767082) -- (6.25091,0.791294);
\draw [c] (6.23286,0.767082) -- (6.25091,0.767082);
\draw [c] (6.25091,0.767082) -- (6.26895,0.767082);
\definecolor{c}{rgb}{0,0,0};
\colorlet{c}{natcomp!70};
\draw [c] (6.287,0.763439) -- (6.287,0.787014);
\draw [c] (6.287,0.787014) -- (6.287,0.810589);
\draw [c] (6.26895,0.787014) -- (6.287,0.787014);
\draw [c] (6.287,0.787014) -- (6.30504,0.787014);
\definecolor{c}{rgb}{0,0,0};
\colorlet{c}{natcomp!70};
\draw [c] (6.32308,0.739956) -- (6.32308,0.752771);
\draw [c] (6.32308,0.752771) -- (6.32308,0.765586);
\draw [c] (6.30504,0.752771) -- (6.32308,0.752771);
\draw [c] (6.32308,0.752771) -- (6.34113,0.752771);
\definecolor{c}{rgb}{0,0,0};
\colorlet{c}{natcomp!70};
\draw [c] (6.35917,0.749245) -- (6.35917,0.769531);
\draw [c] (6.35917,0.769531) -- (6.35917,0.789818);
\draw [c] (6.34113,0.769531) -- (6.35917,0.769531);
\draw [c] (6.35917,0.769531) -- (6.37722,0.769531);
\definecolor{c}{rgb}{0,0,0};
\colorlet{c}{natcomp!70};
\draw [c] (6.39526,0.734642) -- (6.39526,0.745339);
\draw [c] (6.39526,0.745339) -- (6.39526,0.756036);
\draw [c] (6.37722,0.745339) -- (6.39526,0.745339);
\draw [c] (6.39526,0.745339) -- (6.41331,0.745339);
\definecolor{c}{rgb}{0,0,0};
\colorlet{c}{natcomp!70};
\draw [c] (6.43135,0.749697) -- (6.43135,0.770462);
\draw [c] (6.43135,0.770462) -- (6.43135,0.791227);
\draw [c] (6.41331,0.770462) -- (6.43135,0.770462);
\draw [c] (6.43135,0.770462) -- (6.4494,0.770462);
\definecolor{c}{rgb}{0,0,0};
\colorlet{c}{natcomp!70};
\draw [c] (6.46744,0.741741) -- (6.46744,0.758925);
\draw [c] (6.46744,0.758925) -- (6.46744,0.77611);
\draw [c] (6.4494,0.758925) -- (6.46744,0.758925);
\draw [c] (6.46744,0.758925) -- (6.48548,0.758925);
\definecolor{c}{rgb}{0,0,0};
\colorlet{c}{natcomp!70};
\draw [c] (6.50353,0.741076) -- (6.50353,0.756664);
\draw [c] (6.50353,0.756664) -- (6.50353,0.772252);
\draw [c] (6.48548,0.756664) -- (6.50353,0.756664);
\draw [c] (6.50353,0.756664) -- (6.52157,0.756664);
\definecolor{c}{rgb}{0,0,0};
\colorlet{c}{natcomp!70};
\draw [c] (6.53962,0.740597) -- (6.53962,0.755326);
\draw [c] (6.53962,0.755326) -- (6.53962,0.770055);
\draw [c] (6.52157,0.755326) -- (6.53962,0.755326);
\draw [c] (6.53962,0.755326) -- (6.55766,0.755326);
\definecolor{c}{rgb}{0,0,0};
\colorlet{c}{natcomp!70};
\draw [c] (6.57571,0.741224) -- (6.57571,0.757269);
\draw [c] (6.57571,0.757269) -- (6.57571,0.773313);
\draw [c] (6.55766,0.757269) -- (6.57571,0.757269);
\draw [c] (6.57571,0.757269) -- (6.59375,0.757269);
\definecolor{c}{rgb}{0,0,0};
\colorlet{c}{natcomp!70};
\draw [c] (6.61179,0.734641) -- (6.61179,0.73465);
\draw [c] (6.61179,0.73465) -- (6.61179,0.734658);
\draw [c] (6.59375,0.73465) -- (6.61179,0.73465);
\draw [c] (6.61179,0.73465) -- (6.62984,0.73465);
\definecolor{c}{rgb}{0,0,0};
\colorlet{c}{natcomp!70};
\draw [c] (6.64788,0.748451) -- (6.64788,0.767356);
\draw [c] (6.64788,0.767356) -- (6.64788,0.786261);
\draw [c] (6.62984,0.767356) -- (6.64788,0.767356);
\draw [c] (6.64788,0.767356) -- (6.66593,0.767356);
\definecolor{c}{rgb}{0,0,0};
\colorlet{c}{natcomp!70};
\draw [c] (6.68397,0.734651) -- (6.68397,0.745051);
\draw [c] (6.68397,0.745051) -- (6.68397,0.755451);
\draw [c] (6.66593,0.745051) -- (6.68397,0.745051);
\draw [c] (6.68397,0.745051) -- (6.70202,0.745051);
\definecolor{c}{rgb}{0,0,0};
\colorlet{c}{natcomp!70};
\draw [c] (6.72006,0.747383) -- (6.72006,0.765114);
\draw [c] (6.72006,0.765114) -- (6.72006,0.782845);
\draw [c] (6.70202,0.765114) -- (6.72006,0.765114);
\draw [c] (6.72006,0.765114) -- (6.7381,0.765114);
\definecolor{c}{rgb}{0,0,0};
\colorlet{c}{natcomp!70};
\draw [c] (6.75615,0.734642) -- (6.75615,0.744194);
\draw [c] (6.75615,0.744194) -- (6.75615,0.753745);
\draw [c] (6.7381,0.744194) -- (6.75615,0.744194);
\draw [c] (6.75615,0.744194) -- (6.77419,0.744194);
\definecolor{c}{rgb}{0,0,0};
\colorlet{c}{natcomp!70};
\draw [c] (6.79224,0.745364) -- (6.79224,0.760001);
\draw [c] (6.79224,0.760001) -- (6.79224,0.774637);
\draw [c] (6.77419,0.760001) -- (6.79224,0.760001);
\draw [c] (6.79224,0.760001) -- (6.81028,0.760001);
\definecolor{c}{rgb}{0,0,0};
\colorlet{c}{natcomp!70};
\draw [c] (6.82833,0.748794) -- (6.82833,0.768589);
\draw [c] (6.82833,0.768589) -- (6.82833,0.788384);
\draw [c] (6.81028,0.768589) -- (6.82833,0.768589);
\draw [c] (6.82833,0.768589) -- (6.84637,0.768589);
\definecolor{c}{rgb}{0,0,0};
\colorlet{c}{natcomp!70};
\draw [c] (6.86442,0.734632) -- (6.86442,0.744183);
\draw [c] (6.86442,0.744183) -- (6.86442,0.753735);
\draw [c] (6.84637,0.744183) -- (6.86442,0.744183);
\draw [c] (6.86442,0.744183) -- (6.88246,0.744183);
\definecolor{c}{rgb}{0,0,0};
\colorlet{c}{natcomp!70};
\draw [c] (6.9005,0.74204) -- (6.9005,0.761487);
\draw [c] (6.9005,0.761487) -- (6.9005,0.780933);
\draw [c] (6.88246,0.761487) -- (6.9005,0.761487);
\draw [c] (6.9005,0.761487) -- (6.91855,0.761487);
\definecolor{c}{rgb}{0,0,0};
\colorlet{c}{natcomp!70};
\draw [c] (6.93659,0.734641) -- (6.93659,0.745338);
\draw [c] (6.93659,0.745338) -- (6.93659,0.756035);
\draw [c] (6.91855,0.745338) -- (6.93659,0.745338);
\draw [c] (6.93659,0.745338) -- (6.95464,0.745338);
\definecolor{c}{rgb}{0,0,0};
\colorlet{c}{natcomp!70};
\draw [c] (6.97268,0.734636) -- (6.97268,0.734649);
\draw [c] (6.97268,0.734649) -- (6.97268,0.734662);
\draw [c] (6.95464,0.734649) -- (6.97268,0.734649);
\draw [c] (6.97268,0.734649) -- (6.99073,0.734649);
\definecolor{c}{rgb}{0,0,0};
\colorlet{c}{natcomp!70};
\draw [c] (7.00877,0.743265) -- (7.00877,0.76582);
\draw [c] (7.00877,0.76582) -- (7.00877,0.788375);
\draw [c] (6.99073,0.76582) -- (7.00877,0.76582);
\draw [c] (7.00877,0.76582) -- (7.02681,0.76582);
\definecolor{c}{rgb}{0,0,0};
\colorlet{c}{natcomp!70};
\draw [c] (7.04486,0.740152) -- (7.04486,0.753784);
\draw [c] (7.04486,0.753784) -- (7.04486,0.767416);
\draw [c] (7.02681,0.753784) -- (7.04486,0.753784);
\draw [c] (7.04486,0.753784) -- (7.0629,0.753784);
\definecolor{c}{rgb}{0,0,0};
\colorlet{c}{natcomp!70};
\draw [c] (7.08095,0.73465) -- (7.08095,0.746868);
\draw [c] (7.08095,0.746868) -- (7.08095,0.759085);
\draw [c] (7.0629,0.746868) -- (7.08095,0.746868);
\draw [c] (7.08095,0.746868) -- (7.09899,0.746868);
\definecolor{c}{rgb}{0,0,0};
\colorlet{c}{natcomp!70};
\draw [c] (7.11704,0.734641) -- (7.11704,0.744192);
\draw [c] (7.11704,0.744192) -- (7.11704,0.753744);
\draw [c] (7.09899,0.744192) -- (7.11704,0.744192);
\draw [c] (7.11704,0.744192) -- (7.13508,0.744192);
\definecolor{c}{rgb}{0,0,0};
\colorlet{c}{natcomp!70};
\draw [c] (7.15312,0.734653) -- (7.15312,0.746265);
\draw [c] (7.15312,0.746265) -- (7.15312,0.757876);
\draw [c] (7.13508,0.746265) -- (7.15312,0.746265);
\draw [c] (7.15312,0.746265) -- (7.17117,0.746265);
\definecolor{c}{rgb}{0,0,0};
\colorlet{c}{natcomp!70};
\draw [c] (7.18921,0.734635) -- (7.18921,0.734641);
\draw [c] (7.18921,0.734641) -- (7.18921,0.734648);
\draw [c] (7.17117,0.734641) -- (7.18921,0.734641);
\draw [c] (7.18921,0.734641) -- (7.20726,0.734641);
\definecolor{c}{rgb}{0,0,0};
\colorlet{c}{natcomp!70};
\draw [c] (7.2253,0.734641) -- (7.2253,0.760185);
\draw [c] (7.2253,0.760185) -- (7.2253,0.78573);
\draw [c] (7.20726,0.760185) -- (7.2253,0.760185);
\draw [c] (7.2253,0.760185) -- (7.24335,0.760185);
\definecolor{c}{rgb}{0,0,0};
\colorlet{c}{natcomp!70};
\draw [c] (7.26139,0.734632) -- (7.26139,0.734637);
\draw [c] (7.26139,0.734637) -- (7.26139,0.734642);
\draw [c] (7.24335,0.734637) -- (7.26139,0.734637);
\draw [c] (7.26139,0.734637) -- (7.27944,0.734637);
\definecolor{c}{rgb}{0,0,0};
\colorlet{c}{natcomp!70};
\draw [c] (7.29748,0.734642) -- (7.29748,0.745339);
\draw [c] (7.29748,0.745339) -- (7.29748,0.756036);
\draw [c] (7.27944,0.745339) -- (7.29748,0.745339);
\draw [c] (7.29748,0.745339) -- (7.31552,0.745339);
\definecolor{c}{rgb}{0,0,0};
\colorlet{c}{natcomp!70};
\draw [c] (7.33357,0.734637) -- (7.33357,0.745037);
\draw [c] (7.33357,0.745037) -- (7.33357,0.755437);
\draw [c] (7.31552,0.745037) -- (7.33357,0.745037);
\draw [c] (7.33357,0.745037) -- (7.35161,0.745037);
\definecolor{c}{rgb}{0,0,0};
\colorlet{c}{natcomp!70};
\draw [c] (7.36966,0.734641) -- (7.36966,0.746252);
\draw [c] (7.36966,0.746252) -- (7.36966,0.757864);
\draw [c] (7.35161,0.746252) -- (7.36966,0.746252);
\draw [c] (7.36966,0.746252) -- (7.3877,0.746252);
\definecolor{c}{rgb}{0,0,0};
\colorlet{c}{natcomp!70};
\draw [c] (7.40575,0.734656) -- (7.40575,0.743718);
\draw [c] (7.40575,0.743718) -- (7.40575,0.752779);
\draw [c] (7.3877,0.743718) -- (7.40575,0.743718);
\draw [c] (7.40575,0.743718) -- (7.42379,0.743718);
\definecolor{c}{rgb}{0,0,0};
\colorlet{c}{natcomp!70};
\draw [c] (7.44183,0.734632) -- (7.44183,0.734637);
\draw [c] (7.44183,0.734637) -- (7.44183,0.734641);
\draw [c] (7.42379,0.734637) -- (7.44183,0.734637);
\draw [c] (7.44183,0.734637) -- (7.45988,0.734637);
\definecolor{c}{rgb}{0,0,0};
\colorlet{c}{natcomp!70};
\draw [c] (7.58619,0.734637) -- (7.58619,0.73465);
\draw [c] (7.58619,0.73465) -- (7.58619,0.734663);
\draw [c] (7.56815,0.73465) -- (7.58619,0.73465);
\draw [c] (7.58619,0.73465) -- (7.60423,0.73465);
\definecolor{c}{rgb}{0,0,0};
\colorlet{c}{natcomp!70};
\draw [c] (7.62228,0.740087) -- (7.62228,0.753488);
\draw [c] (7.62228,0.753488) -- (7.62228,0.766888);
\draw [c] (7.60423,0.753488) -- (7.62228,0.753488);
\draw [c] (7.62228,0.753488) -- (7.64032,0.753488);
\definecolor{c}{rgb}{0,0,0};
\colorlet{c}{natcomp!70};
\draw [c] (7.73054,0.740987) -- (7.73054,0.756833);
\draw [c] (7.73054,0.756833) -- (7.73054,0.772678);
\draw [c] (7.7125,0.756833) -- (7.73054,0.756833);
\draw [c] (7.73054,0.756833) -- (7.74859,0.756833);
\definecolor{c}{rgb}{0,0,0};
\colorlet{c}{natcomp!70};
\draw [c] (7.76663,0.74081) -- (7.76663,0.755729);
\draw [c] (7.76663,0.755729) -- (7.76663,0.770649);
\draw [c] (7.74859,0.755729) -- (7.76663,0.755729);
\draw [c] (7.76663,0.755729) -- (7.78468,0.755729);
\definecolor{c}{rgb}{0,0,0};
\colorlet{c}{natcomp!70};
\draw [c] (7.80272,0.734636) -- (7.80272,0.734645);
\draw [c] (7.80272,0.734645) -- (7.80272,0.734655);
\draw [c] (7.78468,0.734645) -- (7.80272,0.734645);
\draw [c] (7.80272,0.734645) -- (7.82077,0.734645);
\definecolor{c}{rgb}{0,0,0};
\colorlet{c}{natcomp!70};
\draw [c] (7.83881,0.734638) -- (7.83881,0.734652);
\draw [c] (7.83881,0.734652) -- (7.83881,0.734666);
\draw [c] (7.82077,0.734652) -- (7.83881,0.734652);
\draw [c] (7.83881,0.734652) -- (7.85685,0.734652);
\definecolor{c}{rgb}{0,0,0};
\colorlet{c}{natcomp!70};
\draw [c] (7.8749,0.734638) -- (7.8749,0.744189);
\draw [c] (7.8749,0.744189) -- (7.8749,0.75374);
\draw [c] (7.85685,0.744189) -- (7.8749,0.744189);
\draw [c] (7.8749,0.744189) -- (7.89294,0.744189);
\definecolor{c}{rgb}{0,0,0};
\colorlet{c}{natcomp!70};
\draw [c] (7.91099,0.734635) -- (7.91099,0.734642);
\draw [c] (7.91099,0.734642) -- (7.91099,0.734649);
\draw [c] (7.89294,0.734642) -- (7.91099,0.734642);
\draw [c] (7.91099,0.734642) -- (7.92903,0.734642);
\definecolor{c}{rgb}{0,0,0};
\colorlet{c}{natcomp!70};
\draw [c] (7.94708,0.734632) -- (7.94708,0.746244);
\draw [c] (7.94708,0.746244) -- (7.94708,0.757855);
\draw [c] (7.92903,0.746244) -- (7.94708,0.746244);
\draw [c] (7.94708,0.746244) -- (7.96512,0.746244);
\definecolor{c}{rgb}{0,0,0};
\colorlet{c}{natcomp!70};
\draw [c] (7.98317,0.734657) -- (7.98317,0.745354);
\draw [c] (7.98317,0.745354) -- (7.98317,0.756051);
\draw [c] (7.96512,0.745354) -- (7.98317,0.745354);
\draw [c] (7.98317,0.745354) -- (8.00121,0.745354);
\definecolor{c}{rgb}{0,0,0};
\colorlet{c}{natcomp!70};
\draw [c] (8.01925,0.734632) -- (8.01925,0.734638);
\draw [c] (8.01925,0.734638) -- (8.01925,0.734644);
\draw [c] (8.00121,0.734638) -- (8.01925,0.734638);
\draw [c] (8.01925,0.734638) -- (8.0373,0.734638);
\definecolor{c}{rgb}{0,0,0};
\colorlet{c}{natcomp!70};
\draw [c] (8.05534,0.734632) -- (8.05534,0.734637);
\draw [c] (8.05534,0.734637) -- (8.05534,0.734642);
\draw [c] (8.0373,0.734637) -- (8.05534,0.734637);
\draw [c] (8.05534,0.734637) -- (8.07339,0.734637);
\definecolor{c}{rgb}{0,0,0};
\colorlet{c}{natcomp!70};
\draw [c] (8.09143,0.734632) -- (8.09143,0.7473);
\draw [c] (8.09143,0.7473) -- (8.09143,0.759968);
\draw [c] (8.07339,0.7473) -- (8.09143,0.7473);
\draw [c] (8.09143,0.7473) -- (8.10948,0.7473);
\definecolor{c}{rgb}{0,0,0};
\colorlet{c}{natcomp!70};
\draw [c] (8.16361,0.734632) -- (8.16361,0.743694);
\draw [c] (8.16361,0.743694) -- (8.16361,0.752755);
\draw [c] (8.14556,0.743694) -- (8.16361,0.743694);
\draw [c] (8.16361,0.743694) -- (8.18165,0.743694);
\definecolor{c}{rgb}{0,0,0};
\colorlet{c}{natcomp!70};
\draw [c] (8.23579,0.734632) -- (8.23579,0.734637);
\draw [c] (8.23579,0.734637) -- (8.23579,0.734642);
\draw [c] (8.21774,0.734637) -- (8.23579,0.734637);
\draw [c] (8.23579,0.734637) -- (8.25383,0.734637);
\definecolor{c}{rgb}{0,0,0};
\colorlet{c}{natcomp!70};
\draw [c] (8.27188,0.734637) -- (8.27188,0.745156);
\draw [c] (8.27188,0.745156) -- (8.27188,0.755676);
\draw [c] (8.25383,0.745156) -- (8.27188,0.745156);
\draw [c] (8.27188,0.745156) -- (8.28992,0.745156);
\definecolor{c}{rgb}{0,0,0};
\colorlet{c}{natcomp!70};
\draw [c] (8.34405,0.734651) -- (8.34405,0.747294);
\draw [c] (8.34405,0.747294) -- (8.34405,0.759937);
\draw [c] (8.32601,0.747294) -- (8.34405,0.747294);
\draw [c] (8.34405,0.747294) -- (8.3621,0.747294);
\definecolor{c}{rgb}{0,0,0};
\colorlet{c}{natcomp!70};
\draw [c] (8.45232,0.734632) -- (8.45232,0.734637);
\draw [c] (8.45232,0.734637) -- (8.45232,0.734642);
\draw [c] (8.43427,0.734637) -- (8.45232,0.734637);
\draw [c] (8.45232,0.734637) -- (8.47036,0.734637);
\definecolor{c}{rgb}{0,0,0};
\colorlet{c}{natcomp!70};
\draw [c] (8.48841,0.734637) -- (8.48841,0.744188);
\draw [c] (8.48841,0.744188) -- (8.48841,0.75374);
\draw [c] (8.47036,0.744188) -- (8.48841,0.744188);
\draw [c] (8.48841,0.744188) -- (8.50645,0.744188);
\definecolor{c}{rgb}{0,0,0};
\colorlet{c}{natcomp!70};
\draw [c] (8.74103,0.734632) -- (8.74103,0.734637);
\draw [c] (8.74103,0.734637) -- (8.74103,0.734641);
\draw [c] (8.72298,0.734637) -- (8.74103,0.734637);
\draw [c] (8.74103,0.734637) -- (8.75907,0.734637);
\definecolor{c}{rgb}{0,0,0};
\colorlet{c}{natcomp!70};
\draw [c] (8.77712,0.734632) -- (8.77712,0.734636);
\draw [c] (8.77712,0.734636) -- (8.77712,0.73464);
\draw [c] (8.75907,0.734636) -- (8.77712,0.734636);
\draw [c] (8.77712,0.734636) -- (8.79516,0.734636);
\definecolor{c}{rgb}{0,0,0};
\colorlet{c}{natcomp!70};
\draw [c] (8.81321,0.734632) -- (8.81321,0.734637);
\draw [c] (8.81321,0.734637) -- (8.81321,0.734642);
\draw [c] (8.79516,0.734637) -- (8.81321,0.734637);
\draw [c] (8.81321,0.734637) -- (8.83125,0.734637);
\definecolor{c}{rgb}{0,0,0};
\colorlet{c}{natcomp!70};
\draw [c] (8.88538,0.734632) -- (8.88538,0.743082);
\draw [c] (8.88538,0.743082) -- (8.88538,0.751533);
\draw [c] (8.86734,0.743082) -- (8.88538,0.743082);
\draw [c] (8.88538,0.743082) -- (8.90343,0.743082);
\definecolor{c}{rgb}{0,0,0};
\colorlet{c}{natcomp!70};
\draw [c] (8.95756,0.734632) -- (8.95756,0.734638);
\draw [c] (8.95756,0.734638) -- (8.95756,0.734644);
\draw [c] (8.93952,0.734638) -- (8.95756,0.734638);
\draw [c] (8.95756,0.734638) -- (8.97561,0.734638);
\definecolor{c}{rgb}{0,0,0};
\colorlet{c}{natcomp!70};
\draw [c] (9.10192,0.734632) -- (9.10192,0.734637);
\draw [c] (9.10192,0.734637) -- (9.10192,0.734641);
\draw [c] (9.08387,0.734637) -- (9.10192,0.734637);
\draw [c] (9.10192,0.734637) -- (9.11996,0.734637);
\definecolor{c}{rgb}{0,0,0};
\colorlet{c}{natcomp!70};
\draw [c] (9.21018,0.734632) -- (9.21018,0.753592);
\draw [c] (9.21018,0.753592) -- (9.21018,0.772551);
\draw [c] (9.19214,0.753592) -- (9.21018,0.753592);
\draw [c] (9.21018,0.753592) -- (9.22823,0.753592);
\definecolor{c}{rgb}{0,0,0};
\colorlet{c}{natcomp!70};
\draw [c] (9.28236,0.734632) -- (9.28236,0.745329);
\draw [c] (9.28236,0.745329) -- (9.28236,0.756027);
\draw [c] (9.26431,0.745329) -- (9.28236,0.745329);
\draw [c] (9.28236,0.745329) -- (9.3004,0.745329);
\definecolor{c}{rgb}{0,0,0};
\colorlet{c}{natcomp!70};
\draw [c] (9.31845,0.734632) -- (9.31845,0.73464);
\draw [c] (9.31845,0.73464) -- (9.31845,0.734648);
\draw [c] (9.3004,0.73464) -- (9.31845,0.73464);
\draw [c] (9.31845,0.73464) -- (9.33649,0.73464);
\definecolor{c}{rgb}{0,0,0};
\colorlet{c}{natcomp!70};
\draw [c] (9.4628,0.734632) -- (9.4628,0.746244);
\draw [c] (9.4628,0.746244) -- (9.4628,0.757855);
\draw [c] (9.44476,0.746244) -- (9.4628,0.746244);
\draw [c] (9.4628,0.746244) -- (9.48085,0.746244);
\definecolor{c}{rgb}{0,0,0};
\colorlet{c}{natcomp!70};
\draw [c] (9.49889,0.734632) -- (9.49889,0.744183);
\draw [c] (9.49889,0.744183) -- (9.49889,0.753735);
\draw [c] (9.48085,0.744183) -- (9.49889,0.744183);
\draw [c] (9.49889,0.744183) -- (9.51694,0.744183);
\definecolor{c}{rgb}{0,0,0};
\colorlet{c}{natcomp!70};
\draw [c] (9.57107,0.734632) -- (9.57107,0.745032);
\draw [c] (9.57107,0.745032) -- (9.57107,0.755432);
\draw [c] (9.55302,0.745032) -- (9.57107,0.745032);
\draw [c] (9.57107,0.745032) -- (9.58911,0.745032);
\definecolor{c}{rgb}{0,0,0};
\colorlet{c}{natcomp!70};
\draw [c] (9.67933,0.734632) -- (9.67933,0.734636);
\draw [c] (9.67933,0.734636) -- (9.67933,0.73464);
\draw [c] (9.66129,0.734636) -- (9.67933,0.734636);
\draw [c] (9.67933,0.734636) -- (9.69738,0.734636);
\definecolor{c}{rgb}{0,0,0};
\colorlet{c}{natcomp!70};
\draw [c] (9.7876,0.734632) -- (9.7876,0.734636);
\draw [c] (9.7876,0.734636) -- (9.7876,0.73464);
\draw [c] (9.76956,0.734636) -- (9.7876,0.734636);
\draw [c] (9.7876,0.734636) -- (9.80564,0.734636);
\definecolor{c}{rgb}{0,0,0};
\colorlet{c}{natcomp!70};
\draw [c] (9.82369,0.734632) -- (9.82369,0.734636);
\draw [c] (9.82369,0.734636) -- (9.82369,0.73464);
\draw [c] (9.80564,0.734636) -- (9.82369,0.734636);
\draw [c] (9.82369,0.734636) -- (9.84173,0.734636);
\definecolor{c}{rgb}{0,0,0};
\colorlet{c}{natcomp!70};
\draw [c] (9.93196,0.734632) -- (9.93196,0.746244);
\draw [c] (9.93196,0.746244) -- (9.93196,0.757855);
\draw [c] (9.91391,0.746244) -- (9.93196,0.746244);
\draw [c] (9.93196,0.746244) -- (9.95,0.746244);
\definecolor{c}{rgb}{0,0,0};
\draw [anchor=base west] (6.09599,5.83614) node[color=c, rotate=0]{ATLAS MC};
\colorlet{c}{natgreen};
\draw [c] (5.14004,5.96347) -- (5.92729,5.96347);
\draw [c] (5.53367,5.7937) -- (5.53367,6.13324);
\definecolor{c}{rgb}{0,0,0};
\draw [anchor=base west] (6.09599,5.27024) node[color=c, rotate=0]{CalcHEP MC};
\colorlet{c}{natcomp!70};
\draw [c] (5.14004,5.39756) -- (5.92729,5.39756);
\draw [c] (5.53367,5.22779) -- (5.53367,5.56734);
\end{tikzpicture}

\end{infilsf}
\end{minipage}
\begin{minipage}[b]{.49\textwidth}
\subcaption{Before application of mapping function.\label{etpv}}
\end{minipage}\hfill
\begin{minipage}[b]{.49\textwidth}
\subcaption{After application of mapping function.\label{etmap}}
\end{minipage}
\caption{The distribution of $E_T^\text{iso}$ in the \atlas{} MC set compared with the distribution in the CalcHEP MC set. In \subcaptionref{etmap}, a mapping function which applies a scale and offset to the values for the CalcHEP MC has been applied.}
\end{figure}

These distributions are now very close to being identical. We correct the remaining discrepancy by reweighting the CalcHEP sample [uncertainty, effect on Mgg...?]

Finally, the CalcHEP sample only included events produced by the tree level process, whereas the \atlas{} sample also includes the contribution from the box diagram shown in fig.~\ref{hiorder}. So, to meaningfully compare the two, we must know the contribution from the box diagram. This, we glean from another \atlas{} MC sample\footnote{that would be that appendix again}, which provides a $M_{\gamma\gamma}$ distribution illustrated in fig.~\ref{boxmgg}. As with the estimated background, this distribution has insufficient statistics to accurately represent the shape of the distribution in the interesting region above 1\,000 GeV, forcing us to extrapolate the shape of this distribution as well.

\begin{figure}[htp]
\begin{minipage}[b]{.69\textwidth}
\begin{infilsf} \tiny
\begin{tikzpicture}[x=.092\textwidth,y=.092\textwidth]
\pgfdeclareplotmark{cross} {
\pgfpathmoveto{\pgfpoint{-0.3\pgfplotmarksize}{\pgfplotmarksize}}
\pgfpathlineto{\pgfpoint{+0.3\pgfplotmarksize}{\pgfplotmarksize}}
\pgfpathlineto{\pgfpoint{+0.3\pgfplotmarksize}{0.3\pgfplotmarksize}}
\pgfpathlineto{\pgfpoint{+1\pgfplotmarksize}{0.3\pgfplotmarksize}}
\pgfpathlineto{\pgfpoint{+1\pgfplotmarksize}{-0.3\pgfplotmarksize}}
\pgfpathlineto{\pgfpoint{+0.3\pgfplotmarksize}{-0.3\pgfplotmarksize}}
\pgfpathlineto{\pgfpoint{+0.3\pgfplotmarksize}{-1.\pgfplotmarksize}}
\pgfpathlineto{\pgfpoint{-0.3\pgfplotmarksize}{-1.\pgfplotmarksize}}
\pgfpathlineto{\pgfpoint{-0.3\pgfplotmarksize}{-0.3\pgfplotmarksize}}
\pgfpathlineto{\pgfpoint{-1.\pgfplotmarksize}{-0.3\pgfplotmarksize}}
\pgfpathlineto{\pgfpoint{-1.\pgfplotmarksize}{0.3\pgfplotmarksize}}
\pgfpathlineto{\pgfpoint{-0.3\pgfplotmarksize}{0.3\pgfplotmarksize}}
\pgfpathclose
\pgfusepathqstroke
}
\pgfdeclareplotmark{cross*} {
\pgfpathmoveto{\pgfpoint{-0.3\pgfplotmarksize}{\pgfplotmarksize}}
\pgfpathlineto{\pgfpoint{+0.3\pgfplotmarksize}{\pgfplotmarksize}}
\pgfpathlineto{\pgfpoint{+0.3\pgfplotmarksize}{0.3\pgfplotmarksize}}
\pgfpathlineto{\pgfpoint{+1\pgfplotmarksize}{0.3\pgfplotmarksize}}
\pgfpathlineto{\pgfpoint{+1\pgfplotmarksize}{-0.3\pgfplotmarksize}}
\pgfpathlineto{\pgfpoint{+0.3\pgfplotmarksize}{-0.3\pgfplotmarksize}}
\pgfpathlineto{\pgfpoint{+0.3\pgfplotmarksize}{-1.\pgfplotmarksize}}
\pgfpathlineto{\pgfpoint{-0.3\pgfplotmarksize}{-1.\pgfplotmarksize}}
\pgfpathlineto{\pgfpoint{-0.3\pgfplotmarksize}{-0.3\pgfplotmarksize}}
\pgfpathlineto{\pgfpoint{-1.\pgfplotmarksize}{-0.3\pgfplotmarksize}}
\pgfpathlineto{\pgfpoint{-1.\pgfplotmarksize}{0.3\pgfplotmarksize}}
\pgfpathlineto{\pgfpoint{-0.3\pgfplotmarksize}{0.3\pgfplotmarksize}}
\pgfpathclose
\pgfusepathqfillstroke
}
\pgfdeclareplotmark{newstar} {
\pgfpathmoveto{\pgfqpoint{0pt}{\pgfplotmarksize}}
\pgfpathlineto{\pgfqpointpolar{44}{0.5\pgfplotmarksize}}
\pgfpathlineto{\pgfqpointpolar{18}{\pgfplotmarksize}}
\pgfpathlineto{\pgfqpointpolar{-20}{0.5\pgfplotmarksize}}
\pgfpathlineto{\pgfqpointpolar{-54}{\pgfplotmarksize}}
\pgfpathlineto{\pgfqpointpolar{-90}{0.5\pgfplotmarksize}}
\pgfpathlineto{\pgfqpointpolar{234}{\pgfplotmarksize}}
\pgfpathlineto{\pgfqpointpolar{198}{0.5\pgfplotmarksize}}
\pgfpathlineto{\pgfqpointpolar{162}{\pgfplotmarksize}}
\pgfpathlineto{\pgfqpointpolar{134}{0.5\pgfplotmarksize}}
\pgfpathclose
\pgfusepathqstroke
}
\pgfdeclareplotmark{newstar*} {
\pgfpathmoveto{\pgfqpoint{0pt}{\pgfplotmarksize}}
\pgfpathlineto{\pgfqpointpolar{44}{0.5\pgfplotmarksize}}
\pgfpathlineto{\pgfqpointpolar{18}{\pgfplotmarksize}}
\pgfpathlineto{\pgfqpointpolar{-20}{0.5\pgfplotmarksize}}
\pgfpathlineto{\pgfqpointpolar{-54}{\pgfplotmarksize}}
\pgfpathlineto{\pgfqpointpolar{-90}{0.5\pgfplotmarksize}}
\pgfpathlineto{\pgfqpointpolar{234}{\pgfplotmarksize}}
\pgfpathlineto{\pgfqpointpolar{198}{0.5\pgfplotmarksize}}
\pgfpathlineto{\pgfqpointpolar{162}{\pgfplotmarksize}}
\pgfpathlineto{\pgfqpointpolar{134}{0.5\pgfplotmarksize}}
\pgfpathclose
\pgfusepathqfillstroke
}
\definecolor{c}{rgb}{1,1,1};
\draw [color=c, fill=c] (0,0) rectangle (10,6.80516);
\draw [color=c, fill=c] (1,0.680516) rectangle (9.95,6.73711);
\definecolor{c}{rgb}{0,0,0};
\draw [c] (1,0.680516) -- (1,6.73711) -- (9.95,6.73711) -- (9.95,0.680516) -- (1,0.680516);
\definecolor{c}{rgb}{1,1,1};
\draw [color=c, fill=c] (1,0.680516) rectangle (9.95,6.73711);
\definecolor{c}{rgb}{0,0,0};
\draw [c] (1,0.680516) -- (1,6.73711) -- (9.95,6.73711) -- (9.95,0.680516) -- (1,0.680516);
\colorlet{c}{natgreen};
\draw [c] (1.6646,0.680516) -- (1.6646,2.3049);
\draw [c] (1.6646,2.3049) -- (1.6646,2.69635);
\draw [c] (1.6203,2.3049) -- (1.6646,2.3049);
\draw [c] (1.6646,2.3049) -- (1.70891,2.3049);
\definecolor{c}{rgb}{0,0,0};
\colorlet{c}{natgreen};
\draw [c] (1.75322,2.38132) -- (1.75322,2.95036);
\draw [c] (1.75322,2.95036) -- (1.75322,3.22799);
\draw [c] (1.70891,2.95036) -- (1.75322,2.95036);
\draw [c] (1.75322,2.95036) -- (1.79752,2.95036);
\definecolor{c}{rgb}{0,0,0};
\colorlet{c}{natgreen};
\draw [c] (1.84183,4.31431) -- (1.84183,4.43557);
\draw [c] (1.84183,4.43557) -- (1.84183,4.53534);
\draw [c] (1.79752,4.43557) -- (1.84183,4.43557);
\draw [c] (1.84183,4.43557) -- (1.88614,4.43557);
\definecolor{c}{rgb}{0,0,0};
\colorlet{c}{natgreen};
\draw [c] (1.93045,6.01844) -- (1.93045,6.04598);
\draw [c] (1.93045,6.04598) -- (1.93045,6.07223);
\draw [c] (1.88614,6.04598) -- (1.93045,6.04598);
\draw [c] (1.93045,6.04598) -- (1.97475,6.04598);
\definecolor{c}{rgb}{0,0,0};
\colorlet{c}{natgreen};
\draw [c] (2.01906,6.4381) -- (2.01906,6.45705);
\draw [c] (2.01906,6.45705) -- (2.01906,6.47539);
\draw [c] (1.97475,6.45705) -- (2.01906,6.45705);
\draw [c] (2.01906,6.45705) -- (2.06337,6.45705);
\definecolor{c}{rgb}{0,0,0};
\colorlet{c}{natgreen};
\draw [c] (2.10767,6.48095) -- (2.10767,6.49923);
\draw [c] (2.10767,6.49923) -- (2.10767,6.51694);
\draw [c] (2.06337,6.49923) -- (2.10767,6.49923);
\draw [c] (2.10767,6.49923) -- (2.15198,6.49923);
\definecolor{c}{rgb}{0,0,0};
\colorlet{c}{natgreen};
\draw [c] (2.19629,6.42817) -- (2.19629,6.44727);
\draw [c] (2.19629,6.44727) -- (2.19629,6.46573);
\draw [c] (2.15198,6.44727) -- (2.19629,6.44727);
\draw [c] (2.19629,6.44727) -- (2.24059,6.44727);
\definecolor{c}{rgb}{0,0,0};
\colorlet{c}{natgreen};
\draw [c] (2.2849,6.34203) -- (2.2849,6.36277);
\draw [c] (2.2849,6.36277) -- (2.2849,6.38277);
\draw [c] (2.24059,6.36277) -- (2.2849,6.36277);
\draw [c] (2.2849,6.36277) -- (2.32921,6.36277);
\definecolor{c}{rgb}{0,0,0};
\colorlet{c}{natgreen};
\draw [c] (2.37351,6.22705) -- (2.37351,6.24993);
\draw [c] (2.37351,6.24993) -- (2.37351,6.27192);
\draw [c] (2.32921,6.24993) -- (2.37351,6.24993);
\draw [c] (2.37351,6.24993) -- (2.41782,6.24993);
\definecolor{c}{rgb}{0,0,0};
\colorlet{c}{natgreen};
\draw [c] (2.46213,6.10396) -- (2.46213,6.12959);
\draw [c] (2.46213,6.12959) -- (2.46213,6.15411);
\draw [c] (2.41782,6.12959) -- (2.46213,6.12959);
\draw [c] (2.46213,6.12959) -- (2.50644,6.12959);
\definecolor{c}{rgb}{0,0,0};
\colorlet{c}{natgreen};
\draw [c] (2.55074,5.923) -- (2.55074,5.95292);
\draw [c] (2.55074,5.95292) -- (2.55074,5.98134);
\draw [c] (2.50644,5.95292) -- (2.55074,5.95292);
\draw [c] (2.55074,5.95292) -- (2.59505,5.95292);
\definecolor{c}{rgb}{0,0,0};
\colorlet{c}{natgreen};
\draw [c] (2.63936,5.81438) -- (2.63936,5.84748);
\draw [c] (2.63936,5.84748) -- (2.63936,5.87875);
\draw [c] (2.59505,5.84748) -- (2.63936,5.84748);
\draw [c] (2.63936,5.84748) -- (2.68366,5.84748);
\definecolor{c}{rgb}{0,0,0};
\colorlet{c}{natgreen};
\draw [c] (2.72797,5.6363) -- (2.72797,5.67497);
\draw [c] (2.72797,5.67497) -- (2.72797,5.71115);
\draw [c] (2.68366,5.67497) -- (2.72797,5.67497);
\draw [c] (2.72797,5.67497) -- (2.77228,5.67497);
\definecolor{c}{rgb}{0,0,0};
\colorlet{c}{natgreen};
\draw [c] (2.81658,5.53226) -- (2.81658,5.57507);
\draw [c] (2.81658,5.57507) -- (2.81658,5.61487);
\draw [c] (2.77228,5.57507) -- (2.81658,5.57507);
\draw [c] (2.81658,5.57507) -- (2.86089,5.57507);
\definecolor{c}{rgb}{0,0,0};
\colorlet{c}{natgreen};
\draw [c] (2.9052,5.35892) -- (2.9052,5.4081);
\draw [c] (2.9052,5.4081) -- (2.9052,5.45334);
\draw [c] (2.86089,5.4081) -- (2.9052,5.4081);
\draw [c] (2.9052,5.4081) -- (2.94951,5.4081);
\definecolor{c}{rgb}{0,0,0};
\colorlet{c}{natgreen};
\draw [c] (2.99381,5.24072) -- (2.99381,5.29613);
\draw [c] (2.99381,5.29613) -- (2.99381,5.34658);
\draw [c] (2.94951,5.29613) -- (2.99381,5.29613);
\draw [c] (2.99381,5.29613) -- (3.03812,5.29613);
\definecolor{c}{rgb}{0,0,0};
\colorlet{c}{natgreen};
\draw [c] (3.08243,5.21899) -- (3.08243,5.27515);
\draw [c] (3.08243,5.27515) -- (3.08243,5.32623);
\draw [c] (3.03812,5.27515) -- (3.08243,5.27515);
\draw [c] (3.08243,5.27515) -- (3.12673,5.27515);
\definecolor{c}{rgb}{0,0,0};
\colorlet{c}{natgreen};
\draw [c] (3.17104,5.09339) -- (3.17104,5.1552);
\draw [c] (3.17104,5.1552) -- (3.17104,5.21091);
\draw [c] (3.12673,5.1552) -- (3.17104,5.1552);
\draw [c] (3.17104,5.1552) -- (3.21535,5.1552);
\definecolor{c}{rgb}{0,0,0};
\colorlet{c}{natgreen};
\draw [c] (3.25965,4.96656) -- (3.25965,5.03664);
\draw [c] (3.25965,5.03664) -- (3.25965,5.09898);
\draw [c] (3.21535,5.03664) -- (3.25965,5.03664);
\draw [c] (3.25965,5.03664) -- (3.30396,5.03664);
\definecolor{c}{rgb}{0,0,0};
\colorlet{c}{natgreen};
\draw [c] (3.34827,4.84513) -- (3.34827,4.92243);
\draw [c] (3.34827,4.92243) -- (3.34827,4.99041);
\draw [c] (3.30396,4.92243) -- (3.34827,4.92243);
\draw [c] (3.34827,4.92243) -- (3.39257,4.92243);
\definecolor{c}{rgb}{0,0,0};
\colorlet{c}{natgreen};
\draw [c] (3.43688,4.68676) -- (3.43688,4.7774);
\draw [c] (3.43688,4.7774) -- (3.43688,4.85548);
\draw [c] (3.39257,4.7774) -- (3.43688,4.7774);
\draw [c] (3.43688,4.7774) -- (3.48119,4.7774);
\definecolor{c}{rgb}{0,0,0};
\colorlet{c}{natgreen};
\draw [c] (3.5255,4.69232) -- (3.5255,4.78254);
\draw [c] (3.5255,4.78254) -- (3.5255,4.86032);
\draw [c] (3.48119,4.78254) -- (3.5255,4.78254);
\draw [c] (3.5255,4.78254) -- (3.5698,4.78254);
\definecolor{c}{rgb}{0,0,0};
\colorlet{c}{natgreen};
\draw [c] (3.61411,4.44789) -- (3.61411,4.55535);
\draw [c] (3.61411,4.55535) -- (3.61411,4.64559);
\draw [c] (3.5698,4.55535) -- (3.61411,4.55535);
\draw [c] (3.61411,4.55535) -- (3.65842,4.55535);
\definecolor{c}{rgb}{0,0,0};
\colorlet{c}{natgreen};
\draw [c] (3.70272,4.38517) -- (3.70272,4.50101);
\draw [c] (3.70272,4.50101) -- (3.70272,4.59709);
\draw [c] (3.65842,4.50101) -- (3.70272,4.50101);
\draw [c] (3.70272,4.50101) -- (3.74703,4.50101);
\definecolor{c}{rgb}{0,0,0};
\colorlet{c}{natgreen};
\draw [c] (3.79134,4.43828) -- (3.79134,4.54987);
\draw [c] (3.79134,4.54987) -- (3.79134,4.64301);
\draw [c] (3.74703,4.54987) -- (3.79134,4.54987);
\draw [c] (3.79134,4.54987) -- (3.83564,4.54987);
\definecolor{c}{rgb}{0,0,0};
\colorlet{c}{natgreen};
\draw [c] (3.87995,4.17957) -- (3.87995,4.32023);
\draw [c] (3.87995,4.32023) -- (3.87995,4.43275);
\draw [c] (3.83564,4.32023) -- (3.87995,4.32023);
\draw [c] (3.87995,4.32023) -- (3.92426,4.32023);
\definecolor{c}{rgb}{0,0,0};
\colorlet{c}{natgreen};
\draw [c] (3.96856,3.71331) -- (3.96856,3.92083);
\draw [c] (3.96856,3.92083) -- (3.96856,4.07224);
\draw [c] (3.92426,3.92083) -- (3.96856,3.92083);
\draw [c] (3.96856,3.92083) -- (4.01287,3.92083);
\definecolor{c}{rgb}{0,0,0};
\colorlet{c}{natgreen};
\draw [c] (4.05718,3.73488) -- (4.05718,3.93623);
\draw [c] (4.05718,3.93623) -- (4.05718,4.08436);
\draw [c] (4.01287,3.93623) -- (4.05718,3.93623);
\draw [c] (4.05718,3.93623) -- (4.10149,3.93623);
\definecolor{c}{rgb}{0,0,0};
\colorlet{c}{natgreen};
\draw [c] (4.14579,3.98827) -- (4.14579,4.14568);
\draw [c] (4.14579,4.14568) -- (4.14579,4.26864);
\draw [c] (4.10149,4.14568) -- (4.14579,4.14568);
\draw [c] (4.14579,4.14568) -- (4.1901,4.14568);
\definecolor{c}{rgb}{0,0,0};
\colorlet{c}{natgreen};
\draw [c] (4.23441,3.70996) -- (4.23441,3.92222);
\draw [c] (4.23441,3.92222) -- (4.23441,4.07613);
\draw [c] (4.1901,3.92222) -- (4.23441,3.92222);
\draw [c] (4.23441,3.92222) -- (4.27871,3.92222);
\definecolor{c}{rgb}{0,0,0};
\colorlet{c}{natgreen};
\draw [c] (4.32302,3.33856) -- (4.32302,3.6243);
\draw [c] (4.32302,3.6243) -- (4.32302,3.81313);
\draw [c] (4.27871,3.6243) -- (4.32302,3.6243);
\draw [c] (4.32302,3.6243) -- (4.36733,3.6243);
\definecolor{c}{rgb}{0,0,0};
\colorlet{c}{natgreen};
\draw [c] (4.41163,3.32705) -- (4.41163,3.6277);
\draw [c] (4.41163,3.6277) -- (4.41163,3.82286);
\draw [c] (4.36733,3.6277) -- (4.41163,3.6277);
\draw [c] (4.41163,3.6277) -- (4.45594,3.6277);
\definecolor{c}{rgb}{0,0,0};
\colorlet{c}{natgreen};
\draw [c] (4.50025,3.83238) -- (4.50025,4.02204);
\draw [c] (4.50025,4.02204) -- (4.50025,4.16377);
\draw [c] (4.45594,4.02204) -- (4.50025,4.02204);
\draw [c] (4.50025,4.02204) -- (4.54455,4.02204);
\definecolor{c}{rgb}{0,0,0};
\colorlet{c}{natgreen};
\draw [c] (4.58886,3.27564) -- (4.58886,3.5691);
\draw [c] (4.58886,3.5691) -- (4.58886,3.76124);
\draw [c] (4.54455,3.5691) -- (4.58886,3.5691);
\draw [c] (4.58886,3.5691) -- (4.63317,3.5691);
\definecolor{c}{rgb}{0,0,0};
\colorlet{c}{natgreen};
\draw [c] (4.67748,2.74147) -- (4.67748,3.24921);
\draw [c] (4.67748,3.24921) -- (4.67748,3.51219);
\draw [c] (4.63317,3.24921) -- (4.67748,3.24921);
\draw [c] (4.67748,3.24921) -- (4.72178,3.24921);
\definecolor{c}{rgb}{0,0,0};
\colorlet{c}{natgreen};
\draw [c] (4.76609,3.60758) -- (4.76609,3.85031);
\draw [c] (4.76609,3.85031) -- (4.76609,4.01953);
\draw [c] (4.72178,3.85031) -- (4.76609,3.85031);
\draw [c] (4.76609,3.85031) -- (4.8104,3.85031);
\definecolor{c}{rgb}{0,0,0};
\colorlet{c}{natgreen};
\draw [c] (4.94332,2.30513) -- (4.94332,3.00293);
\draw [c] (4.94332,3.00293) -- (4.94332,3.30569);
\draw [c] (4.89901,3.00293) -- (4.94332,3.00293);
\draw [c] (4.94332,3.00293) -- (4.98762,3.00293);
\definecolor{c}{rgb}{0,0,0};
\colorlet{c}{natgreen};
\draw [c] (5.12054,3.19183) -- (5.12054,3.50284);
\draw [c] (5.12054,3.50284) -- (5.12054,3.70225);
\draw [c] (5.07624,3.50284) -- (5.12054,3.50284);
\draw [c] (5.12054,3.50284) -- (5.16485,3.50284);
\definecolor{c}{rgb}{0,0,0};
\colorlet{c}{natgreen};
\draw [c] (5.20916,2.95994) -- (5.20916,3.3768);
\draw [c] (5.20916,3.3768) -- (5.20916,3.61401);
\draw [c] (5.16485,3.3768) -- (5.20916,3.3768);
\draw [c] (5.20916,3.3768) -- (5.25347,3.3768);
\definecolor{c}{rgb}{0,0,0};
\colorlet{c}{natgreen};
\draw [c] (5.29777,2.67687) -- (5.29777,3.18022);
\draw [c] (5.29777,3.18022) -- (5.29777,3.44206);
\draw [c] (5.25347,3.18022) -- (5.29777,3.18022);
\draw [c] (5.29777,3.18022) -- (5.34208,3.18022);
\definecolor{c}{rgb}{0,0,0};
\colorlet{c}{natgreen};
\draw [c] (5.38639,0.680516) -- (5.38639,2.77962);
\draw [c] (5.38639,2.77962) -- (5.38639,3.17107);
\draw [c] (5.34208,2.77962) -- (5.38639,2.77962);
\draw [c] (5.38639,2.77962) -- (5.43069,2.77962);
\definecolor{c}{rgb}{0,0,0};
\colorlet{c}{natgreen};
\draw [c] (5.475,2.32505) -- (5.475,3.01853);
\draw [c] (5.475,3.01853) -- (5.475,3.32055);
\draw [c] (5.43069,3.01853) -- (5.475,3.01853);
\draw [c] (5.475,3.01853) -- (5.51931,3.01853);
\definecolor{c}{rgb}{0,0,0};
\colorlet{c}{natgreen};
\draw [c] (5.56361,0.680516) -- (5.56361,2.25224);
\draw [c] (5.56361,2.25224) -- (5.56361,2.64369);
\draw [c] (5.51931,2.25224) -- (5.56361,2.25224);
\draw [c] (5.56361,2.25224) -- (5.60792,2.25224);
\definecolor{c}{rgb}{0,0,0};
\colorlet{c}{natgreen};
\draw [c] (5.65223,2.72587) -- (5.65223,3.2156);
\draw [c] (5.65223,3.2156) -- (5.65223,3.47388);
\draw [c] (5.60792,3.2156) -- (5.65223,3.2156);
\draw [c] (5.65223,3.2156) -- (5.69653,3.2156);
\definecolor{c}{rgb}{0,0,0};
\colorlet{c}{natgreen};
\draw [c] (5.74084,2.27391) -- (5.74084,2.96766);
\draw [c] (5.74084,2.96766) -- (5.74084,3.26973);
\draw [c] (5.69653,2.96766) -- (5.74084,2.96766);
\draw [c] (5.74084,2.96766) -- (5.78515,2.96766);
\definecolor{c}{rgb}{0,0,0};
\colorlet{c}{natgreen};
\draw [c] (5.82946,0.680516) -- (5.82946,2.51889);
\draw [c] (5.82946,2.51889) -- (5.82946,2.91034);
\draw [c] (5.78515,2.51889) -- (5.82946,2.51889);
\draw [c] (5.82946,2.51889) -- (5.87376,2.51889);
\definecolor{c}{rgb}{0,0,0};
\colorlet{c}{natgreen};
\draw [c] (6.00668,0.680516) -- (6.00668,2.62777);
\draw [c] (6.00668,2.62777) -- (6.00668,3.01922);
\draw [c] (5.96238,2.62777) -- (6.00668,2.62777);
\draw [c] (6.00668,2.62777) -- (6.05099,2.62777);
\definecolor{c}{rgb}{0,0,0};
\colorlet{c}{natgreen};
\draw [c] (6.0953,0.680516) -- (6.0953,2.62558);
\draw [c] (6.0953,2.62558) -- (6.0953,3.01404);
\draw [c] (6.05099,2.62558) -- (6.0953,2.62558);
\draw [c] (6.0953,2.62558) -- (6.1396,2.62558);
\definecolor{c}{rgb}{0,0,0};
\colorlet{c}{natgreen};
\draw [c] (6.36114,0.680516) -- (6.36114,1.60554);
\draw [c] (6.36114,1.60554) -- (6.36114,1.99699);
\draw [c] (6.31683,1.60554) -- (6.36114,1.60554);
\draw [c] (6.36114,1.60554) -- (6.40545,1.60554);
\definecolor{c}{rgb}{0,0,0};
\colorlet{c}{natgreen};
\draw [c] (6.62698,0.680516) -- (6.62698,2.65744);
\draw [c] (6.62698,2.65744) -- (6.62698,3.04889);
\draw [c] (6.58267,2.65744) -- (6.62698,2.65744);
\draw [c] (6.62698,2.65744) -- (6.67129,2.65744);
\definecolor{c}{rgb}{0,0,0};
\colorlet{c}{natgreen};
\draw [c] (6.71559,2.78195) -- (6.71559,3.27658);
\draw [c] (6.71559,3.27658) -- (6.71559,3.53615);
\draw [c] (6.67129,3.27658) -- (6.71559,3.27658);
\draw [c] (6.71559,3.27658) -- (6.7599,3.27658);
\definecolor{c}{rgb}{0,0,0};
\colorlet{c}{natgreen};
\draw [c] (6.80421,0.680516) -- (6.80421,2.68178);
\draw [c] (6.80421,2.68178) -- (6.80421,3.07323);
\draw [c] (6.7599,2.68178) -- (6.80421,2.68178);
\draw [c] (6.80421,2.68178) -- (6.84852,2.68178);
\definecolor{c}{rgb}{0,0,0};
\colorlet{c}{natgreen};
\draw [c] (7.07005,0.680516) -- (7.07005,2.5647);
\draw [c] (7.07005,2.5647) -- (7.07005,2.95615);
\draw [c] (7.02574,2.5647) -- (7.07005,2.5647);
\draw [c] (7.07005,2.5647) -- (7.11436,2.5647);
\definecolor{c}{rgb}{0,0,0};
\colorlet{c}{natgreen};
\draw [c] (7.4245,0.680516) -- (7.4245,2.60897);
\draw [c] (7.4245,2.60897) -- (7.4245,3.00042);
\draw [c] (7.3802,2.60897) -- (7.4245,2.60897);
\draw [c] (7.4245,2.60897) -- (7.46881,2.60897);
\definecolor{c}{rgb}{0,0,0};
\colorlet{c}{natgreen};
\draw [c] (8.39926,0.680516) -- (8.39926,1.05001);
\draw [c] (8.39926,1.05001) -- (8.39926,1.44146);
\draw [c] (8.35495,1.05001) -- (8.39926,1.05001);
\draw [c] (8.39926,1.05001) -- (8.44356,1.05001);
\definecolor{c}{rgb}{0,0,0};
\draw [c] (1,0.680516) -- (9.95,0.680516);
\draw [anchor= east] (9.95,0.108883) node[color=c, rotate=0]{$M_{\gamma\gamma}\text{ [GeV]}$};
\draw [c] (1,0.863234) -- (1,0.680516);
\draw [c] (1.44307,0.771875) -- (1.44307,0.680516);
\draw [c] (1.88614,0.771875) -- (1.88614,0.680516);
\draw [c] (2.32921,0.771875) -- (2.32921,0.680516);
\draw [c] (2.77228,0.863234) -- (2.77228,0.680516);
\draw [c] (3.21535,0.771875) -- (3.21535,0.680516);
\draw [c] (3.65842,0.771875) -- (3.65842,0.680516);
\draw [c] (4.10149,0.771875) -- (4.10149,0.680516);
\draw [c] (4.54455,0.863234) -- (4.54455,0.680516);
\draw [c] (4.98762,0.771875) -- (4.98762,0.680516);
\draw [c] (5.43069,0.771875) -- (5.43069,0.680516);
\draw [c] (5.87376,0.771875) -- (5.87376,0.680516);
\draw [c] (6.31683,0.863234) -- (6.31683,0.680516);
\draw [c] (6.7599,0.771875) -- (6.7599,0.680516);
\draw [c] (7.20297,0.771875) -- (7.20297,0.680516);
\draw [c] (7.64604,0.771875) -- (7.64604,0.680516);
\draw [c] (8.08911,0.863234) -- (8.08911,0.680516);
\draw [c] (8.53218,0.771875) -- (8.53218,0.680516);
\draw [c] (8.97525,0.771875) -- (8.97525,0.680516);
\draw [c] (9.41832,0.771875) -- (9.41832,0.680516);
\draw [c] (9.86139,0.863234) -- (9.86139,0.680516);
\draw [c] (9.86139,0.863234) -- (9.86139,0.680516);
\draw [anchor=base] (1,0.353868) node[color=c, rotate=0]{0};
\draw [anchor=base] (2.77228,0.353868) node[color=c, rotate=0]{200};
\draw [anchor=base] (4.54455,0.353868) node[color=c, rotate=0]{400};
\draw [anchor=base] (6.31683,0.353868) node[color=c, rotate=0]{600};
\draw [anchor=base] (8.08911,0.353868) node[color=c, rotate=0]{800};
\draw [anchor=base] (9.86139,0.353868) node[color=c, rotate=0]{1000};
\draw [c] (1,0.680516) -- (1,6.73711);
\draw [anchor= east] (-0.12,6.73711) node[color=c, rotate=90]{Number of events};
\draw [c] (1.1335,0.7327) -- (1,0.7327);
\draw [c] (1.1335,0.895167) -- (1,0.895167);
\draw [c] (1.1335,1.02119) -- (1,1.02119);
\draw [c] (1.1335,1.12415) -- (1,1.12415);
\draw [c] (1.1335,1.21121) -- (1,1.21121);
\draw [c] (1.1335,1.28662) -- (1,1.28662);
\draw [c] (1.1335,1.35314) -- (1,1.35314);
\draw [c] (1.267,1.41264) -- (1,1.41264);
\draw [anchor= east] (0.922,1.41264) node[color=c, rotate=0]{$10^{-1}$};
\draw [c] (1.1335,1.80409) -- (1,1.80409);
\draw [c] (1.1335,2.03307) -- (1,2.03307);
\draw [c] (1.1335,2.19554) -- (1,2.19554);
\draw [c] (1.1335,2.32156) -- (1,2.32156);
\draw [c] (1.1335,2.42453) -- (1,2.42453);
\draw [c] (1.1335,2.51158) -- (1,2.51158);
\draw [c] (1.1335,2.58699) -- (1,2.58699);
\draw [c] (1.1335,2.65351) -- (1,2.65351);
\draw [c] (1.267,2.71301) -- (1,2.71301);
\draw [anchor= east] (0.922,2.71301) node[color=c, rotate=0]{1};
\draw [c] (1.1335,3.10446) -- (1,3.10446);
\draw [c] (1.1335,3.33345) -- (1,3.33345);
\draw [c] (1.1335,3.49592) -- (1,3.49592);
\draw [c] (1.1335,3.62193) -- (1,3.62193);
\draw [c] (1.1335,3.7249) -- (1,3.7249);
\draw [c] (1.1335,3.81196) -- (1,3.81196);
\draw [c] (1.1335,3.88737) -- (1,3.88737);
\draw [c] (1.1335,3.95388) -- (1,3.95388);
\draw [c] (1.267,4.01339) -- (1,4.01339);
\draw [anchor= east] (0.922,4.01339) node[color=c, rotate=0]{10};
\draw [c] (1.1335,4.40484) -- (1,4.40484);
\draw [c] (1.1335,4.63382) -- (1,4.63382);
\draw [c] (1.1335,4.79629) -- (1,4.79629);
\draw [c] (1.1335,4.92231) -- (1,4.92231);
\draw [c] (1.1335,5.02527) -- (1,5.02527);
\draw [c] (1.1335,5.11233) -- (1,5.11233);
\draw [c] (1.1335,5.18774) -- (1,5.18774);
\draw [c] (1.1335,5.25426) -- (1,5.25426);
\draw [c] (1.267,5.31376) -- (1,5.31376);
\draw [anchor= east] (0.922,5.31376) node[color=c, rotate=0]{$10^{2}$};
\draw [c] (1.1335,5.70521) -- (1,5.70521);
\draw [c] (1.1335,5.9342) -- (1,5.9342);
\draw [c] (1.1335,6.09666) -- (1,6.09666);
\draw [c] (1.1335,6.22268) -- (1,6.22268);
\draw [c] (1.1335,6.32565) -- (1,6.32565);
\draw [c] (1.1335,6.4127) -- (1,6.4127);
\draw [c] (1.1335,6.48812) -- (1,6.48812);
\draw [c] (1.1335,6.55463) -- (1,6.55463);
\draw [c] (1.267,6.61414) -- (1,6.61414);
\draw [anchor= east] (0.922,6.61414) node[color=c, rotate=0]{$10^{3}$};
\end{tikzpicture}

\end{infilsf}
\end{minipage}
\begin{minipage}[b]{.3\textwidth}
\caption{The distribution of invariant masses in the \atlas{} box diagram data set.}\label{boxmgg}
\end{minipage}
\end{figure}

Adding this to the CalcHEP sample, we get the distribution in fig.~\ref{ggcomp}. As should be evident, there is still a deficit in the CalcHEP sample. This might be because the box diagram contribution added to the CalcHEP sample was produced by a different generator than was used to generate the \atlas{} $\gamma\gamma$ sample, or it might be due to interference between the diagrams which simply adding together samples does not capture. In any case, the difference will be obscure by the uncertainty on the $\gamma$jet sample, which must be added to both.

\begin{figure}[htp]
\begin{minipage}[b]{.69\textwidth}
\begin{infilsf} \tiny
\begin{tikzpicture}[x=.092\textwidth,y=.092\textwidth]
\pgfdeclareplotmark{cross} {
\pgfpathmoveto{\pgfpoint{-0.3\pgfplotmarksize}{\pgfplotmarksize}}
\pgfpathlineto{\pgfpoint{+0.3\pgfplotmarksize}{\pgfplotmarksize}}
\pgfpathlineto{\pgfpoint{+0.3\pgfplotmarksize}{0.3\pgfplotmarksize}}
\pgfpathlineto{\pgfpoint{+1\pgfplotmarksize}{0.3\pgfplotmarksize}}
\pgfpathlineto{\pgfpoint{+1\pgfplotmarksize}{-0.3\pgfplotmarksize}}
\pgfpathlineto{\pgfpoint{+0.3\pgfplotmarksize}{-0.3\pgfplotmarksize}}
\pgfpathlineto{\pgfpoint{+0.3\pgfplotmarksize}{-1.\pgfplotmarksize}}
\pgfpathlineto{\pgfpoint{-0.3\pgfplotmarksize}{-1.\pgfplotmarksize}}
\pgfpathlineto{\pgfpoint{-0.3\pgfplotmarksize}{-0.3\pgfplotmarksize}}
\pgfpathlineto{\pgfpoint{-1.\pgfplotmarksize}{-0.3\pgfplotmarksize}}
\pgfpathlineto{\pgfpoint{-1.\pgfplotmarksize}{0.3\pgfplotmarksize}}
\pgfpathlineto{\pgfpoint{-0.3\pgfplotmarksize}{0.3\pgfplotmarksize}}
\pgfpathclose
\pgfusepathqstroke
}
\pgfdeclareplotmark{cross*} {
\pgfpathmoveto{\pgfpoint{-0.3\pgfplotmarksize}{\pgfplotmarksize}}
\pgfpathlineto{\pgfpoint{+0.3\pgfplotmarksize}{\pgfplotmarksize}}
\pgfpathlineto{\pgfpoint{+0.3\pgfplotmarksize}{0.3\pgfplotmarksize}}
\pgfpathlineto{\pgfpoint{+1\pgfplotmarksize}{0.3\pgfplotmarksize}}
\pgfpathlineto{\pgfpoint{+1\pgfplotmarksize}{-0.3\pgfplotmarksize}}
\pgfpathlineto{\pgfpoint{+0.3\pgfplotmarksize}{-0.3\pgfplotmarksize}}
\pgfpathlineto{\pgfpoint{+0.3\pgfplotmarksize}{-1.\pgfplotmarksize}}
\pgfpathlineto{\pgfpoint{-0.3\pgfplotmarksize}{-1.\pgfplotmarksize}}
\pgfpathlineto{\pgfpoint{-0.3\pgfplotmarksize}{-0.3\pgfplotmarksize}}
\pgfpathlineto{\pgfpoint{-1.\pgfplotmarksize}{-0.3\pgfplotmarksize}}
\pgfpathlineto{\pgfpoint{-1.\pgfplotmarksize}{0.3\pgfplotmarksize}}
\pgfpathlineto{\pgfpoint{-0.3\pgfplotmarksize}{0.3\pgfplotmarksize}}
\pgfpathclose
\pgfusepathqfillstroke
}
\pgfdeclareplotmark{newstar} {
\pgfpathmoveto{\pgfqpoint{0pt}{\pgfplotmarksize}}
\pgfpathlineto{\pgfqpointpolar{44}{0.5\pgfplotmarksize}}
\pgfpathlineto{\pgfqpointpolar{18}{\pgfplotmarksize}}
\pgfpathlineto{\pgfqpointpolar{-20}{0.5\pgfplotmarksize}}
\pgfpathlineto{\pgfqpointpolar{-54}{\pgfplotmarksize}}
\pgfpathlineto{\pgfqpointpolar{-90}{0.5\pgfplotmarksize}}
\pgfpathlineto{\pgfqpointpolar{234}{\pgfplotmarksize}}
\pgfpathlineto{\pgfqpointpolar{198}{0.5\pgfplotmarksize}}
\pgfpathlineto{\pgfqpointpolar{162}{\pgfplotmarksize}}
\pgfpathlineto{\pgfqpointpolar{134}{0.5\pgfplotmarksize}}
\pgfpathclose
\pgfusepathqstroke
}
\pgfdeclareplotmark{newstar*} {
\pgfpathmoveto{\pgfqpoint{0pt}{\pgfplotmarksize}}
\pgfpathlineto{\pgfqpointpolar{44}{0.5\pgfplotmarksize}}
\pgfpathlineto{\pgfqpointpolar{18}{\pgfplotmarksize}}
\pgfpathlineto{\pgfqpointpolar{-20}{0.5\pgfplotmarksize}}
\pgfpathlineto{\pgfqpointpolar{-54}{\pgfplotmarksize}}
\pgfpathlineto{\pgfqpointpolar{-90}{0.5\pgfplotmarksize}}
\pgfpathlineto{\pgfqpointpolar{234}{\pgfplotmarksize}}
\pgfpathlineto{\pgfqpointpolar{198}{0.5\pgfplotmarksize}}
\pgfpathlineto{\pgfqpointpolar{162}{\pgfplotmarksize}}
\pgfpathlineto{\pgfqpointpolar{134}{0.5\pgfplotmarksize}}
\pgfpathclose
\pgfusepathqfillstroke
}
\definecolor{c}{rgb}{1,1,1};
\draw [color=c, fill=c] (0,0) rectangle (10,6.80516);
\draw [color=c, fill=c] (1,0.680516) rectangle (9.95,6.73711);
\definecolor{c}{rgb}{0,0,0};
\draw [c] (1,0.680516) -- (1,6.73711) -- (9.95,6.73711) -- (9.95,0.680516) -- (1,0.680516);
\definecolor{c}{rgb}{1,1,1};
\draw [color=c, fill=c] (1,0.680516) rectangle (9.95,6.73711);
\definecolor{c}{rgb}{0,0,0};
\draw [c] (1,0.680516) -- (1,6.73711) -- (9.95,6.73711) -- (9.95,0.680516) -- (1,0.680516);
\colorlet{c}{natcomp!70};
\draw [c] (1.34131,1.78352) -- (1.34131,2.51541);
\draw [c] (1.34131,2.51541) -- (1.34131,2.82687);
\draw [c] (1.30339,2.51541) -- (1.34131,2.51541);
\draw [c] (1.34131,2.51541) -- (1.37924,2.51541);
\definecolor{c}{rgb}{0,0,0};
\colorlet{c}{natcomp!70};
\draw [c] (1.41716,1.62511) -- (1.41716,2.33052);
\draw [c] (1.41716,2.33052) -- (1.41716,2.63755);
\draw [c] (1.37924,2.33052) -- (1.41716,2.33052);
\draw [c] (1.41716,2.33052) -- (1.45508,2.33052);
\definecolor{c}{rgb}{0,0,0};
\colorlet{c}{natcomp!70};
\draw [c] (1.56886,1.90599) -- (1.56886,2.61743);
\draw [c] (1.56886,2.61743) -- (1.56886,2.92549);
\draw [c] (1.53093,2.61743) -- (1.56886,2.61743);
\draw [c] (1.56886,2.61743) -- (1.60678,2.61743);
\definecolor{c}{rgb}{0,0,0};
\colorlet{c}{natcomp!70};
\draw [c] (1.6447,2.52941) -- (1.6447,2.8612);
\draw [c] (1.6447,2.8612) -- (1.6447,3.07008);
\draw [c] (1.60678,2.8612) -- (1.6447,2.8612);
\draw [c] (1.6447,2.8612) -- (1.68263,2.8612);
\definecolor{c}{rgb}{0,0,0};
\colorlet{c}{natcomp!70};
\draw [c] (1.72055,3.76948) -- (1.72055,3.89028);
\draw [c] (1.72055,3.89028) -- (1.72055,3.99002);
\draw [c] (1.68263,3.89028) -- (1.72055,3.89028);
\draw [c] (1.72055,3.89028) -- (1.75847,3.89028);
\definecolor{c}{rgb}{0,0,0};
\colorlet{c}{natcomp!70};
\draw [c] (1.7964,5.80305) -- (1.7964,5.82466);
\draw [c] (1.7964,5.82466) -- (1.7964,5.84548);
\draw [c] (1.75847,5.82466) -- (1.7964,5.82466);
\draw [c] (1.7964,5.82466) -- (1.83432,5.82466);
\definecolor{c}{rgb}{0,0,0};
\colorlet{c}{natcomp!70};
\draw [c] (1.87225,6.24998) -- (1.87225,6.26438);
\draw [c] (1.87225,6.26438) -- (1.87225,6.27841);
\draw [c] (1.83432,6.26438) -- (1.87225,6.26438);
\draw [c] (1.87225,6.26438) -- (1.91017,6.26438);
\definecolor{c}{rgb}{0,0,0};
\colorlet{c}{natcomp!70};
\draw [c] (1.94809,6.34388) -- (1.94809,6.35726);
\draw [c] (1.94809,6.35726) -- (1.94809,6.37035);
\draw [c] (1.91017,6.35726) -- (1.94809,6.35726);
\draw [c] (1.94809,6.35726) -- (1.98602,6.35726);
\definecolor{c}{rgb}{0,0,0};
\colorlet{c}{natcomp!70};
\draw [c] (2.02394,6.30879) -- (2.02394,6.32269);
\draw [c] (2.02394,6.32269) -- (2.02394,6.33626);
\draw [c] (1.98602,6.32269) -- (2.02394,6.32269);
\draw [c] (2.02394,6.32269) -- (2.06186,6.32269);
\definecolor{c}{rgb}{0,0,0};
\colorlet{c}{natcomp!70};
\draw [c] (2.09979,6.25019) -- (2.09979,6.26512);
\draw [c] (2.09979,6.26512) -- (2.09979,6.27967);
\draw [c] (2.06186,6.26512) -- (2.09979,6.26512);
\draw [c] (2.09979,6.26512) -- (2.13771,6.26512);
\definecolor{c}{rgb}{0,0,0};
\colorlet{c}{natcomp!70};
\draw [c] (2.17564,6.16207) -- (2.17564,6.17792);
\draw [c] (2.17564,6.17792) -- (2.17564,6.19334);
\draw [c] (2.13771,6.17792) -- (2.17564,6.17792);
\draw [c] (2.17564,6.17792) -- (2.21356,6.17792);
\definecolor{c}{rgb}{0,0,0};
\colorlet{c}{natcomp!70};
\draw [c] (2.25148,6.07193) -- (2.25148,6.08917);
\draw [c] (2.25148,6.08917) -- (2.25148,6.10592);
\draw [c] (2.21356,6.08917) -- (2.25148,6.08917);
\draw [c] (2.25148,6.08917) -- (2.28941,6.08917);
\definecolor{c}{rgb}{0,0,0};
\colorlet{c}{natcomp!70};
\draw [c] (2.32733,5.94806) -- (2.32733,5.96764);
\draw [c] (2.32733,5.96764) -- (2.32733,5.98657);
\draw [c] (2.28941,5.96764) -- (2.32733,5.96764);
\draw [c] (2.32733,5.96764) -- (2.36525,5.96764);
\definecolor{c}{rgb}{0,0,0};
\colorlet{c}{natcomp!70};
\draw [c] (2.40318,5.82449) -- (2.40318,5.846);
\draw [c] (2.40318,5.846) -- (2.40318,5.86674);
\draw [c] (2.36525,5.846) -- (2.40318,5.846);
\draw [c] (2.40318,5.846) -- (2.4411,5.846);
\definecolor{c}{rgb}{0,0,0};
\colorlet{c}{natcomp!70};
\draw [c] (2.47903,5.71169) -- (2.47903,5.73544);
\draw [c] (2.47903,5.73544) -- (2.47903,5.75825);
\draw [c] (2.4411,5.73544) -- (2.47903,5.73544);
\draw [c] (2.47903,5.73544) -- (2.51695,5.73544);
\definecolor{c}{rgb}{0,0,0};
\colorlet{c}{natcomp!70};
\draw [c] (2.55487,5.64532) -- (2.55487,5.67137);
\draw [c] (2.55487,5.67137) -- (2.55487,5.69629);
\draw [c] (2.51695,5.67137) -- (2.55487,5.67137);
\draw [c] (2.55487,5.67137) -- (2.5928,5.67137);
\definecolor{c}{rgb}{0,0,0};
\colorlet{c}{natcomp!70};
\draw [c] (2.63072,5.52352) -- (2.63072,5.55176);
\draw [c] (2.63072,5.55176) -- (2.63072,5.57869);
\draw [c] (2.5928,5.55176) -- (2.63072,5.55176);
\draw [c] (2.63072,5.55176) -- (2.66864,5.55176);
\definecolor{c}{rgb}{0,0,0};
\colorlet{c}{natcomp!70};
\draw [c] (2.70657,5.50217) -- (2.70657,5.53122);
\draw [c] (2.70657,5.53122) -- (2.70657,5.55886);
\draw [c] (2.66864,5.53122) -- (2.70657,5.53122);
\draw [c] (2.70657,5.53122) -- (2.74449,5.53122);
\definecolor{c}{rgb}{0,0,0};
\colorlet{c}{natcomp!70};
\draw [c] (2.78242,5.40445) -- (2.78242,5.43645);
\draw [c] (2.78242,5.43645) -- (2.78242,5.46676);
\draw [c] (2.74449,5.43645) -- (2.78242,5.43645);
\draw [c] (2.78242,5.43645) -- (2.82034,5.43645);
\definecolor{c}{rgb}{0,0,0};
\colorlet{c}{natcomp!70};
\draw [c] (2.85826,5.28652) -- (2.85826,5.32084);
\draw [c] (2.85826,5.32084) -- (2.85826,5.35322);
\draw [c] (2.82034,5.32084) -- (2.85826,5.32084);
\draw [c] (2.85826,5.32084) -- (2.89619,5.32084);
\definecolor{c}{rgb}{0,0,0};
\colorlet{c}{natcomp!70};
\draw [c] (2.93411,5.09026) -- (2.93411,5.13066);
\draw [c] (2.93411,5.13066) -- (2.93411,5.16841);
\draw [c] (2.89619,5.13066) -- (2.93411,5.13066);
\draw [c] (2.93411,5.13066) -- (2.97203,5.13066);
\definecolor{c}{rgb}{0,0,0};
\colorlet{c}{natcomp!70};
\draw [c] (3.00996,5.07889) -- (3.00996,5.12198);
\draw [c] (3.00996,5.12198) -- (3.00996,5.16205);
\draw [c] (2.97203,5.12198) -- (3.00996,5.12198);
\draw [c] (3.00996,5.12198) -- (3.04788,5.12198);
\definecolor{c}{rgb}{0,0,0};
\colorlet{c}{natcomp!70};
\draw [c] (3.08581,4.99109) -- (3.08581,5.03784);
\draw [c] (3.08581,5.03784) -- (3.08581,5.08107);
\draw [c] (3.04788,5.03784) -- (3.08581,5.03784);
\draw [c] (3.08581,5.03784) -- (3.12373,5.03784);
\definecolor{c}{rgb}{0,0,0};
\colorlet{c}{natcomp!70};
\draw [c] (3.16165,4.99427) -- (3.16165,5.04101);
\draw [c] (3.16165,5.04101) -- (3.16165,5.08423);
\draw [c] (3.12373,5.04101) -- (3.16165,5.04101);
\draw [c] (3.16165,5.04101) -- (3.19958,5.04101);
\definecolor{c}{rgb}{0,0,0};
\colorlet{c}{natcomp!70};
\draw [c] (3.2375,4.84683) -- (3.2375,4.89782);
\draw [c] (3.2375,4.89782) -- (3.2375,4.94464);
\draw [c] (3.19958,4.89782) -- (3.2375,4.89782);
\draw [c] (3.2375,4.89782) -- (3.27542,4.89782);
\definecolor{c}{rgb}{0,0,0};
\colorlet{c}{natcomp!70};
\draw [c] (3.31335,4.77703) -- (3.31335,4.8307);
\draw [c] (3.31335,4.8307) -- (3.31335,4.87977);
\draw [c] (3.27542,4.8307) -- (3.31335,4.8307);
\draw [c] (3.31335,4.8307) -- (3.35127,4.8307);
\definecolor{c}{rgb}{0,0,0};
\colorlet{c}{natcomp!70};
\draw [c] (3.38919,4.63709) -- (3.38919,4.69689);
\draw [c] (3.38919,4.69689) -- (3.38919,4.75105);
\draw [c] (3.35127,4.69689) -- (3.38919,4.69689);
\draw [c] (3.38919,4.69689) -- (3.42712,4.69689);
\definecolor{c}{rgb}{0,0,0};
\colorlet{c}{natcomp!70};
\draw [c] (3.46504,4.62899) -- (3.46504,4.69223);
\draw [c] (3.46504,4.69223) -- (3.46504,4.74918);
\draw [c] (3.42712,4.69223) -- (3.46504,4.69223);
\draw [c] (3.46504,4.69223) -- (3.50297,4.69223);
\definecolor{c}{rgb}{0,0,0};
\colorlet{c}{natcomp!70};
\draw [c] (3.54089,4.5342) -- (3.54089,4.60227);
\draw [c] (3.54089,4.60227) -- (3.54089,4.66313);
\draw [c] (3.50297,4.60227) -- (3.54089,4.60227);
\draw [c] (3.54089,4.60227) -- (3.57881,4.60227);
\definecolor{c}{rgb}{0,0,0};
\colorlet{c}{natcomp!70};
\draw [c] (3.61674,4.40663) -- (3.61674,4.47821);
\draw [c] (3.61674,4.47821) -- (3.61674,4.54185);
\draw [c] (3.57881,4.47821) -- (3.61674,4.47821);
\draw [c] (3.61674,4.47821) -- (3.65466,4.47821);
\definecolor{c}{rgb}{0,0,0};
\colorlet{c}{natcomp!70};
\draw [c] (3.69258,4.3837) -- (3.69258,4.46006);
\draw [c] (3.69258,4.46006) -- (3.69258,4.52744);
\draw [c] (3.65466,4.46006) -- (3.69258,4.46006);
\draw [c] (3.69258,4.46006) -- (3.73051,4.46006);
\definecolor{c}{rgb}{0,0,0};
\colorlet{c}{natcomp!70};
\draw [c] (3.76843,4.26121) -- (3.76843,4.35345);
\draw [c] (3.76843,4.35345) -- (3.76843,4.43289);
\draw [c] (3.73051,4.35345) -- (3.76843,4.35345);
\draw [c] (3.76843,4.35345) -- (3.80636,4.35345);
\definecolor{c}{rgb}{0,0,0};
\colorlet{c}{natcomp!70};
\draw [c] (3.84428,4.14632) -- (3.84428,4.24826);
\draw [c] (3.84428,4.24826) -- (3.84428,4.3348);
\draw [c] (3.80636,4.24826) -- (3.84428,4.24826);
\draw [c] (3.84428,4.24826) -- (3.8822,4.24826);
\definecolor{c}{rgb}{0,0,0};
\colorlet{c}{natcomp!70};
\draw [c] (3.92013,4.18657) -- (3.92013,4.28491);
\draw [c] (3.92013,4.28491) -- (3.92013,4.36884);
\draw [c] (3.8822,4.28491) -- (3.92013,4.28491);
\draw [c] (3.92013,4.28491) -- (3.95805,4.28491);
\definecolor{c}{rgb}{0,0,0};
\colorlet{c}{natcomp!70};
\draw [c] (3.99597,4.31132) -- (3.99597,4.39416);
\draw [c] (3.99597,4.39416) -- (3.99597,4.46653);
\draw [c] (3.95805,4.39416) -- (3.99597,4.39416);
\draw [c] (3.99597,4.39416) -- (4.0339,4.39416);
\definecolor{c}{rgb}{0,0,0};
\colorlet{c}{natcomp!70};
\draw [c] (4.07182,3.79936) -- (4.07182,3.92106);
\draw [c] (4.07182,3.92106) -- (4.07182,4.02142);
\draw [c] (4.0339,3.92106) -- (4.07182,3.92106);
\draw [c] (4.07182,3.92106) -- (4.10975,3.92106);
\definecolor{c}{rgb}{0,0,0};
\colorlet{c}{natcomp!70};
\draw [c] (4.14767,4.04423) -- (4.14767,4.15122);
\draw [c] (4.14767,4.15122) -- (4.14767,4.24136);
\draw [c] (4.10975,4.15122) -- (4.14767,4.15122);
\draw [c] (4.14767,4.15122) -- (4.18559,4.15122);
\definecolor{c}{rgb}{0,0,0};
\colorlet{c}{natcomp!70};
\draw [c] (4.22352,4.17213) -- (4.22352,4.27444);
\draw [c] (4.22352,4.27444) -- (4.22352,4.36124);
\draw [c] (4.18559,4.27444) -- (4.22352,4.27444);
\draw [c] (4.22352,4.27444) -- (4.26144,4.27444);
\definecolor{c}{rgb}{0,0,0};
\colorlet{c}{natcomp!70};
\draw [c] (4.29936,3.93105) -- (4.29936,3.95136);
\draw [c] (4.29936,3.95136) -- (4.29936,3.97098);
\draw [c] (4.26144,3.95136) -- (4.29936,3.95136);
\draw [c] (4.29936,3.95136) -- (4.33729,3.95136);
\definecolor{c}{rgb}{0,0,0};
\colorlet{c}{natcomp!70};
\draw [c] (4.37521,3.86494) -- (4.37521,3.89839);
\draw [c] (4.37521,3.89839) -- (4.37521,3.92999);
\draw [c] (4.33729,3.89839) -- (4.37521,3.89839);
\draw [c] (4.37521,3.89839) -- (4.41314,3.89839);
\definecolor{c}{rgb}{0,0,0};
\colorlet{c}{natcomp!70};
\draw [c] (4.45106,3.84202) -- (4.45106,3.86382);
\draw [c] (4.45106,3.86382) -- (4.45106,3.88482);
\draw [c] (4.41314,3.86382) -- (4.45106,3.86382);
\draw [c] (4.45106,3.86382) -- (4.48898,3.86382);
\definecolor{c}{rgb}{0,0,0};
\colorlet{c}{natcomp!70};
\draw [c] (4.52691,3.84293) -- (4.52691,3.8867);
\draw [c] (4.52691,3.8867) -- (4.52691,3.92736);
\draw [c] (4.48898,3.8867) -- (4.52691,3.8867);
\draw [c] (4.52691,3.8867) -- (4.56483,3.8867);
\definecolor{c}{rgb}{0,0,0};
\colorlet{c}{natcomp!70};
\draw [c] (4.60275,3.75875) -- (4.60275,3.80805);
\draw [c] (4.60275,3.80805) -- (4.60275,3.85345);
\draw [c] (4.56483,3.80805) -- (4.60275,3.80805);
\draw [c] (4.60275,3.80805) -- (4.64068,3.80805);
\definecolor{c}{rgb}{0,0,0};
\colorlet{c}{natcomp!70};
\draw [c] (4.6786,3.66225) -- (4.6786,3.71021);
\draw [c] (4.6786,3.71021) -- (4.6786,3.75446);
\draw [c] (4.64068,3.71021) -- (4.6786,3.71021);
\draw [c] (4.6786,3.71021) -- (4.71653,3.71021);
\definecolor{c}{rgb}{0,0,0};
\colorlet{c}{natcomp!70};
\draw [c] (4.75445,3.57524) -- (4.75445,3.62382);
\draw [c] (4.75445,3.62382) -- (4.75445,3.6686);
\draw [c] (4.71653,3.62382) -- (4.75445,3.62382);
\draw [c] (4.75445,3.62382) -- (4.79237,3.62382);
\definecolor{c}{rgb}{0,0,0};
\colorlet{c}{natcomp!70};
\draw [c] (4.8303,3.55668) -- (4.8303,3.60867);
\draw [c] (4.8303,3.60867) -- (4.8303,3.65634);
\draw [c] (4.79237,3.60867) -- (4.8303,3.60867);
\draw [c] (4.8303,3.60867) -- (4.86822,3.60867);
\definecolor{c}{rgb}{0,0,0};
\colorlet{c}{natcomp!70};
\draw [c] (4.90614,3.51853) -- (4.90614,3.55247);
\draw [c] (4.90614,3.55247) -- (4.90614,3.58452);
\draw [c] (4.86822,3.55247) -- (4.90614,3.55247);
\draw [c] (4.90614,3.55247) -- (4.94407,3.55247);
\definecolor{c}{rgb}{0,0,0};
\colorlet{c}{natcomp!70};
\draw [c] (4.98199,3.52238) -- (4.98199,3.58359);
\draw [c] (4.98199,3.58359) -- (4.98199,3.6389);
\draw [c] (4.94407,3.58359) -- (4.98199,3.58359);
\draw [c] (4.98199,3.58359) -- (5.01992,3.58359);
\definecolor{c}{rgb}{0,0,0};
\colorlet{c}{natcomp!70};
\draw [c] (5.05784,3.43016) -- (5.05784,3.48934);
\draw [c] (5.05784,3.48934) -- (5.05784,3.54298);
\draw [c] (5.01992,3.48934) -- (5.05784,3.48934);
\draw [c] (5.05784,3.48934) -- (5.09576,3.48934);
\definecolor{c}{rgb}{0,0,0};
\colorlet{c}{natcomp!70};
\draw [c] (5.13369,3.32771) -- (5.13369,3.37927);
\draw [c] (5.13369,3.37927) -- (5.13369,3.42658);
\draw [c] (5.09576,3.37927) -- (5.13369,3.37927);
\draw [c] (5.13369,3.37927) -- (5.17161,3.37927);
\definecolor{c}{rgb}{0,0,0};
\colorlet{c}{natcomp!70};
\draw [c] (5.20953,3.23557) -- (5.20953,3.27213);
\draw [c] (5.20953,3.27213) -- (5.20953,3.3065);
\draw [c] (5.17161,3.27213) -- (5.20953,3.27213);
\draw [c] (5.20953,3.27213) -- (5.24746,3.27213);
\definecolor{c}{rgb}{0,0,0};
\colorlet{c}{natcomp!70};
\draw [c] (5.28538,3.29755) -- (5.28538,3.35759);
\draw [c] (5.28538,3.35759) -- (5.28538,3.41194);
\draw [c] (5.24746,3.35759) -- (5.28538,3.35759);
\draw [c] (5.28538,3.35759) -- (5.32331,3.35759);
\definecolor{c}{rgb}{0,0,0};
\colorlet{c}{natcomp!70};
\draw [c] (5.36123,3.1904) -- (5.36123,3.25923);
\draw [c] (5.36123,3.25923) -- (5.36123,3.32068);
\draw [c] (5.32331,3.25923) -- (5.36123,3.25923);
\draw [c] (5.36123,3.25923) -- (5.39915,3.25923);
\definecolor{c}{rgb}{0,0,0};
\colorlet{c}{natcomp!70};
\draw [c] (5.43708,3.14764) -- (5.43708,3.18783);
\draw [c] (5.43708,3.18783) -- (5.43708,3.22538);
\draw [c] (5.39915,3.18783) -- (5.43708,3.18783);
\draw [c] (5.43708,3.18783) -- (5.475,3.18783);
\definecolor{c}{rgb}{0,0,0};
\colorlet{c}{natcomp!70};
\draw [c] (5.51292,3.10655) -- (5.51292,3.14909);
\draw [c] (5.51292,3.14909) -- (5.51292,3.18868);
\draw [c] (5.475,3.14909) -- (5.51292,3.14909);
\draw [c] (5.51292,3.14909) -- (5.55085,3.14909);
\definecolor{c}{rgb}{0,0,0};
\colorlet{c}{natcomp!70};
\draw [c] (5.58877,3.12957) -- (5.58877,3.17285);
\draw [c] (5.58877,3.17285) -- (5.58877,3.21309);
\draw [c] (5.55085,3.17285) -- (5.58877,3.17285);
\draw [c] (5.58877,3.17285) -- (5.6267,3.17285);
\definecolor{c}{rgb}{0,0,0};
\colorlet{c}{natcomp!70};
\draw [c] (5.66462,2.98182) -- (5.66462,3.0269);
\draw [c] (5.66462,3.0269) -- (5.66462,3.0687);
\draw [c] (5.6267,3.0269) -- (5.66462,3.0269);
\draw [c] (5.66462,3.0269) -- (5.70254,3.0269);
\definecolor{c}{rgb}{0,0,0};
\colorlet{c}{natcomp!70};
\draw [c] (5.74047,2.95982) -- (5.74047,3.00795);
\draw [c] (5.74047,3.00795) -- (5.74047,3.05235);
\draw [c] (5.70254,3.00795) -- (5.74047,3.00795);
\draw [c] (5.74047,3.00795) -- (5.77839,3.00795);
\definecolor{c}{rgb}{0,0,0};
\colorlet{c}{natcomp!70};
\draw [c] (5.81631,2.9584) -- (5.81631,3.05897);
\draw [c] (5.81631,3.05897) -- (5.81631,3.14452);
\draw [c] (5.77839,3.05897) -- (5.81631,3.05897);
\draw [c] (5.81631,3.05897) -- (5.85424,3.05897);
\definecolor{c}{rgb}{0,0,0};
\colorlet{c}{natcomp!70};
\draw [c] (5.89216,3.10587) -- (5.89216,3.23102);
\draw [c] (5.89216,3.23102) -- (5.89216,3.3337);
\draw [c] (5.85424,3.23102) -- (5.89216,3.23102);
\draw [c] (5.89216,3.23102) -- (5.93008,3.23102);
\definecolor{c}{rgb}{0,0,0};
\colorlet{c}{natcomp!70};
\draw [c] (5.96801,2.82054) -- (5.96801,2.94787);
\draw [c] (5.96801,2.94787) -- (5.96801,3.05201);
\draw [c] (5.93008,2.94787) -- (5.96801,2.94787);
\draw [c] (5.96801,2.94787) -- (6.00593,2.94787);
\definecolor{c}{rgb}{0,0,0};
\colorlet{c}{natcomp!70};
\draw [c] (6.04386,2.72523) -- (6.04386,2.78281);
\draw [c] (6.04386,2.78281) -- (6.04386,2.83514);
\draw [c] (6.00593,2.78281) -- (6.04386,2.78281);
\draw [c] (6.04386,2.78281) -- (6.08178,2.78281);
\definecolor{c}{rgb}{0,0,0};
\colorlet{c}{natcomp!70};
\draw [c] (6.1197,2.78902) -- (6.1197,2.8457);
\draw [c] (6.1197,2.8457) -- (6.1197,2.89729);
\draw [c] (6.08178,2.8457) -- (6.1197,2.8457);
\draw [c] (6.1197,2.8457) -- (6.15763,2.8457);
\definecolor{c}{rgb}{0,0,0};
\colorlet{c}{natcomp!70};
\draw [c] (6.19555,2.76443) -- (6.19555,2.88354);
\draw [c] (6.19555,2.88354) -- (6.19555,2.98212);
\draw [c] (6.15763,2.88354) -- (6.19555,2.88354);
\draw [c] (6.19555,2.88354) -- (6.23347,2.88354);
\definecolor{c}{rgb}{0,0,0};
\colorlet{c}{natcomp!70};
\draw [c] (6.2714,2.71588) -- (6.2714,2.77508);
\draw [c] (6.2714,2.77508) -- (6.2714,2.82873);
\draw [c] (6.23347,2.77508) -- (6.2714,2.77508);
\draw [c] (6.2714,2.77508) -- (6.30932,2.77508);
\definecolor{c}{rgb}{0,0,0};
\colorlet{c}{natcomp!70};
\draw [c] (6.34725,2.55336) -- (6.34725,2.61892);
\draw [c] (6.34725,2.61892) -- (6.34725,2.67775);
\draw [c] (6.30932,2.61892) -- (6.34725,2.61892);
\draw [c] (6.34725,2.61892) -- (6.38517,2.61892);
\definecolor{c}{rgb}{0,0,0};
\colorlet{c}{natcomp!70};
\draw [c] (6.42309,2.54103) -- (6.42309,2.60764);
\draw [c] (6.42309,2.60764) -- (6.42309,2.66732);
\draw [c] (6.38517,2.60764) -- (6.42309,2.60764);
\draw [c] (6.42309,2.60764) -- (6.46102,2.60764);
\definecolor{c}{rgb}{0,0,0};
\colorlet{c}{natcomp!70};
\draw [c] (6.49894,2.57973) -- (6.49894,2.7434);
\draw [c] (6.49894,2.7434) -- (6.49894,2.87061);
\draw [c] (6.46102,2.7434) -- (6.49894,2.7434);
\draw [c] (6.49894,2.7434) -- (6.53686,2.7434);
\definecolor{c}{rgb}{0,0,0};
\colorlet{c}{natcomp!70};
\draw [c] (6.57479,2.28663) -- (6.57479,2.36458);
\draw [c] (6.57479,2.36458) -- (6.57479,2.4332);
\draw [c] (6.53686,2.36458) -- (6.57479,2.36458);
\draw [c] (6.57479,2.36458) -- (6.61271,2.36458);
\definecolor{c}{rgb}{0,0,0};
\colorlet{c}{natcomp!70};
\draw [c] (6.65064,2.40822) -- (6.65064,2.48613);
\draw [c] (6.65064,2.48613) -- (6.65064,2.55472);
\draw [c] (6.61271,2.48613) -- (6.65064,2.48613);
\draw [c] (6.65064,2.48613) -- (6.68856,2.48613);
\definecolor{c}{rgb}{0,0,0};
\colorlet{c}{natcomp!70};
\draw [c] (6.72648,2.47109) -- (6.72648,2.54861);
\draw [c] (6.72648,2.54861) -- (6.72648,2.6169);
\draw [c] (6.68856,2.54861) -- (6.72648,2.54861);
\draw [c] (6.72648,2.54861) -- (6.76441,2.54861);
\definecolor{c}{rgb}{0,0,0};
\colorlet{c}{natcomp!70};
\draw [c] (6.80233,2.34361) -- (6.80233,2.42923);
\draw [c] (6.80233,2.42923) -- (6.80233,2.50371);
\draw [c] (6.76441,2.42923) -- (6.80233,2.42923);
\draw [c] (6.80233,2.42923) -- (6.84025,2.42923);
\definecolor{c}{rgb}{0,0,0};
\colorlet{c}{natcomp!70};
\draw [c] (6.87818,2.39771) -- (6.87818,2.47467);
\draw [c] (6.87818,2.47467) -- (6.87818,2.54252);
\draw [c] (6.84025,2.47467) -- (6.87818,2.47467);
\draw [c] (6.87818,2.47467) -- (6.9161,2.47467);
\definecolor{c}{rgb}{0,0,0};
\colorlet{c}{natcomp!70};
\draw [c] (6.95403,2.13205) -- (6.95403,2.22216);
\draw [c] (6.95403,2.22216) -- (6.95403,2.30002);
\draw [c] (6.9161,2.22216) -- (6.95403,2.22216);
\draw [c] (6.95403,2.22216) -- (6.99195,2.22216);
\definecolor{c}{rgb}{0,0,0};
\colorlet{c}{natcomp!70};
\draw [c] (7.02987,2.25139) -- (7.02987,2.33724);
\draw [c] (7.02987,2.33724) -- (7.02987,2.41191);
\draw [c] (6.99195,2.33724) -- (7.02987,2.33724);
\draw [c] (7.02987,2.33724) -- (7.0678,2.33724);
\definecolor{c}{rgb}{0,0,0};
\colorlet{c}{natcomp!70};
\draw [c] (7.10572,2.32922) -- (7.10572,2.41443);
\draw [c] (7.10572,2.41443) -- (7.10572,2.48861);
\draw [c] (7.0678,2.41443) -- (7.10572,2.41443);
\draw [c] (7.10572,2.41443) -- (7.14364,2.41443);
\definecolor{c}{rgb}{0,0,0};
\colorlet{c}{natcomp!70};
\draw [c] (7.18157,2.12343) -- (7.18157,2.21978);
\draw [c] (7.18157,2.21978) -- (7.18157,2.30226);
\draw [c] (7.14364,2.21978) -- (7.18157,2.21978);
\draw [c] (7.18157,2.21978) -- (7.21949,2.21978);
\definecolor{c}{rgb}{0,0,0};
\colorlet{c}{natcomp!70};
\draw [c] (7.25742,2.03344) -- (7.25742,2.13853);
\draw [c] (7.25742,2.13853) -- (7.25742,2.22732);
\draw [c] (7.21949,2.13853) -- (7.25742,2.13853);
\draw [c] (7.25742,2.13853) -- (7.29534,2.13853);
\definecolor{c}{rgb}{0,0,0};
\colorlet{c}{natcomp!70};
\draw [c] (7.33326,2.02634) -- (7.33326,2.12717);
\draw [c] (7.33326,2.12717) -- (7.33326,2.2129);
\draw [c] (7.29534,2.12717) -- (7.33326,2.12717);
\draw [c] (7.33326,2.12717) -- (7.37119,2.12717);
\definecolor{c}{rgb}{0,0,0};
\colorlet{c}{natcomp!70};
\draw [c] (7.40911,1.83146) -- (7.40911,1.95577);
\draw [c] (7.40911,1.95577) -- (7.40911,2.05788);
\draw [c] (7.37119,1.95577) -- (7.40911,1.95577);
\draw [c] (7.40911,1.95577) -- (7.44703,1.95577);
\definecolor{c}{rgb}{0,0,0};
\colorlet{c}{natcomp!70};
\draw [c] (7.48496,2.08552) -- (7.48496,2.18317);
\draw [c] (7.48496,2.18317) -- (7.48496,2.26659);
\draw [c] (7.44703,2.18317) -- (7.48496,2.18317);
\draw [c] (7.48496,2.18317) -- (7.52288,2.18317);
\definecolor{c}{rgb}{0,0,0};
\colorlet{c}{natcomp!70};
\draw [c] (7.56081,2.1242) -- (7.56081,2.21961);
\draw [c] (7.56081,2.21961) -- (7.56081,2.3014);
\draw [c] (7.52288,2.21961) -- (7.56081,2.21961);
\draw [c] (7.56081,2.21961) -- (7.59873,2.21961);
\definecolor{c}{rgb}{0,0,0};
\colorlet{c}{natcomp!70};
\draw [c] (7.63665,1.9047) -- (7.63665,2.02183);
\draw [c] (7.63665,2.02183) -- (7.63665,2.11905);
\draw [c] (7.59873,2.02183) -- (7.63665,2.02183);
\draw [c] (7.63665,2.02183) -- (7.67458,2.02183);
\definecolor{c}{rgb}{0,0,0};
\colorlet{c}{natcomp!70};
\draw [c] (7.7125,1.94031) -- (7.7125,2.06341);
\draw [c] (7.7125,2.06341) -- (7.7125,2.16472);
\draw [c] (7.67458,2.06341) -- (7.7125,2.06341);
\draw [c] (7.7125,2.06341) -- (7.75042,2.06341);
\definecolor{c}{rgb}{0,0,0};
\colorlet{c}{natcomp!70};
\draw [c] (7.78835,1.72392) -- (7.78835,1.86154);
\draw [c] (7.78835,1.86154) -- (7.78835,1.97246);
\draw [c] (7.75042,1.86154) -- (7.78835,1.86154);
\draw [c] (7.78835,1.86154) -- (7.82627,1.86154);
\definecolor{c}{rgb}{0,0,0};
\colorlet{c}{natcomp!70};
\draw [c] (7.86419,1.73133) -- (7.86419,1.86386);
\draw [c] (7.86419,1.86386) -- (7.86419,1.97145);
\draw [c] (7.82627,1.86386) -- (7.86419,1.86386);
\draw [c] (7.86419,1.86386) -- (7.90212,1.86386);
\definecolor{c}{rgb}{0,0,0};
\colorlet{c}{natcomp!70};
\draw [c] (7.94004,1.51339) -- (7.94004,1.66499);
\draw [c] (7.94004,1.66499) -- (7.94004,1.78479);
\draw [c] (7.90212,1.66499) -- (7.94004,1.66499);
\draw [c] (7.94004,1.66499) -- (7.97797,1.66499);
\definecolor{c}{rgb}{0,0,0};
\colorlet{c}{natcomp!70};
\draw [c] (8.01589,1.80733) -- (8.01589,1.9271);
\draw [c] (8.01589,1.9271) -- (8.01589,2.02614);
\draw [c] (7.97797,1.9271) -- (8.01589,1.9271);
\draw [c] (8.01589,1.9271) -- (8.05381,1.9271);
\definecolor{c}{rgb}{0,0,0};
\colorlet{c}{natcomp!70};
\draw [c] (8.09174,1.74933) -- (8.09174,1.87737);
\draw [c] (8.09174,1.87737) -- (8.09174,1.982);
\draw [c] (8.05381,1.87737) -- (8.09174,1.87737);
\draw [c] (8.09174,1.87737) -- (8.12966,1.87737);
\definecolor{c}{rgb}{0,0,0};
\colorlet{c}{natcomp!70};
\draw [c] (8.16758,1.58389) -- (8.16758,1.7371);
\draw [c] (8.16758,1.7371) -- (8.16758,1.85791);
\draw [c] (8.12966,1.7371) -- (8.16758,1.7371);
\draw [c] (8.16758,1.7371) -- (8.20551,1.7371);
\definecolor{c}{rgb}{0,0,0};
\colorlet{c}{natcomp!70};
\draw [c] (8.24343,1.69622) -- (8.24343,1.83434);
\draw [c] (8.24343,1.83434) -- (8.24343,1.94559);
\draw [c] (8.20551,1.83434) -- (8.24343,1.83434);
\draw [c] (8.24343,1.83434) -- (8.28136,1.83434);
\definecolor{c}{rgb}{0,0,0};
\colorlet{c}{natcomp!70};
\draw [c] (8.31928,1.69369) -- (8.31928,1.82769);
\draw [c] (8.31928,1.82769) -- (8.31928,1.93625);
\draw [c] (8.28136,1.82769) -- (8.31928,1.82769);
\draw [c] (8.31928,1.82769) -- (8.3572,1.82769);
\definecolor{c}{rgb}{0,0,0};
\colorlet{c}{natcomp!70};
\draw [c] (8.39513,1.35148) -- (8.39513,1.52488);
\draw [c] (8.39513,1.52488) -- (8.39513,1.65787);
\draw [c] (8.3572,1.52488) -- (8.39513,1.52488);
\draw [c] (8.39513,1.52488) -- (8.43305,1.52488);
\definecolor{c}{rgb}{0,0,0};
\colorlet{c}{natcomp!70};
\draw [c] (8.47097,1.69673) -- (8.47097,1.81679);
\draw [c] (8.47097,1.81679) -- (8.47097,1.91602);
\draw [c] (8.43305,1.81679) -- (8.47097,1.81679);
\draw [c] (8.47097,1.81679) -- (8.5089,1.81679);
\definecolor{c}{rgb}{0,0,0};
\colorlet{c}{natcomp!70};
\draw [c] (8.54682,1.53076) -- (8.54682,1.65084);
\draw [c] (8.54682,1.65084) -- (8.54682,1.75008);
\draw [c] (8.5089,1.65084) -- (8.54682,1.65084);
\draw [c] (8.54682,1.65084) -- (8.58475,1.65084);
\definecolor{c}{rgb}{0,0,0};
\colorlet{c}{natcomp!70};
\draw [c] (8.62267,1.44173) -- (8.62267,1.5394);
\draw [c] (8.62267,1.5394) -- (8.62267,1.62283);
\draw [c] (8.58475,1.5394) -- (8.62267,1.5394);
\draw [c] (8.62267,1.5394) -- (8.66059,1.5394);
\definecolor{c}{rgb}{0,0,0};
\colorlet{c}{natcomp!70};
\draw [c] (8.69852,1.44392) -- (8.69852,1.46659);
\draw [c] (8.69852,1.46659) -- (8.69852,1.4884);
\draw [c] (8.66059,1.46659) -- (8.69852,1.46659);
\draw [c] (8.69852,1.46659) -- (8.73644,1.46659);
\definecolor{c}{rgb}{0,0,0};
\colorlet{c}{natcomp!70};
\draw [c] (8.77436,1.45513) -- (8.77436,1.47712);
\draw [c] (8.77436,1.47712) -- (8.77436,1.4983);
\draw [c] (8.73644,1.47712) -- (8.77436,1.47712);
\draw [c] (8.77436,1.47712) -- (8.81229,1.47712);
\definecolor{c}{rgb}{0,0,0};
\colorlet{c}{natcomp!70};
\draw [c] (8.85021,1.45819) -- (8.85021,1.48066);
\draw [c] (8.85021,1.48066) -- (8.85021,1.50228);
\draw [c] (8.81229,1.48066) -- (8.85021,1.48066);
\draw [c] (8.85021,1.48066) -- (8.88814,1.48066);
\definecolor{c}{rgb}{0,0,0};
\colorlet{c}{natcomp!70};
\draw [c] (8.92606,1.43318) -- (8.92606,1.45542);
\draw [c] (8.92606,1.45542) -- (8.92606,1.47683);
\draw [c] (8.88814,1.45542) -- (8.92606,1.45542);
\draw [c] (8.92606,1.45542) -- (8.96398,1.45542);
\definecolor{c}{rgb}{0,0,0};
\colorlet{c}{natcomp!70};
\draw [c] (9.00191,1.40407) -- (9.00191,1.42727);
\draw [c] (9.00191,1.42727) -- (9.00191,1.44958);
\draw [c] (8.96398,1.42727) -- (9.00191,1.42727);
\draw [c] (9.00191,1.42727) -- (9.03983,1.42727);
\definecolor{c}{rgb}{0,0,0};
\colorlet{c}{natcomp!70};
\draw [c] (9.07775,1.34916) -- (9.07775,1.37304);
\draw [c] (9.07775,1.37304) -- (9.07775,1.39597);
\draw [c] (9.03983,1.37304) -- (9.07775,1.37304);
\draw [c] (9.07775,1.37304) -- (9.11568,1.37304);
\definecolor{c}{rgb}{0,0,0};
\colorlet{c}{natcomp!70};
\draw [c] (9.1536,1.349) -- (9.1536,1.37351);
\draw [c] (9.1536,1.37351) -- (9.1536,1.39702);
\draw [c] (9.11568,1.37351) -- (9.1536,1.37351);
\draw [c] (9.1536,1.37351) -- (9.19153,1.37351);
\definecolor{c}{rgb}{0,0,0};
\colorlet{c}{natcomp!70};
\draw [c] (9.22945,1.28331) -- (9.22945,1.30947);
\draw [c] (9.22945,1.30947) -- (9.22945,1.3345);
\draw [c] (9.19153,1.30947) -- (9.22945,1.30947);
\draw [c] (9.22945,1.30947) -- (9.26737,1.30947);
\definecolor{c}{rgb}{0,0,0};
\colorlet{c}{natcomp!70};
\draw [c] (9.3053,1.32295) -- (9.3053,1.34851);
\draw [c] (9.3053,1.34851) -- (9.3053,1.37299);
\draw [c] (9.26737,1.34851) -- (9.3053,1.34851);
\draw [c] (9.3053,1.34851) -- (9.34322,1.34851);
\definecolor{c}{rgb}{0,0,0};
\colorlet{c}{natcomp!70};
\draw [c] (9.38114,1.21349) -- (9.38114,1.24116);
\draw [c] (9.38114,1.24116) -- (9.38114,1.26756);
\draw [c] (9.34322,1.24116) -- (9.38114,1.24116);
\draw [c] (9.38114,1.24116) -- (9.41907,1.24116);
\definecolor{c}{rgb}{0,0,0};
\colorlet{c}{natcomp!70};
\draw [c] (9.45699,1.22504) -- (9.45699,1.25181);
\draw [c] (9.45699,1.25181) -- (9.45699,1.27738);
\draw [c] (9.41907,1.25181) -- (9.45699,1.25181);
\draw [c] (9.45699,1.25181) -- (9.49492,1.25181);
\definecolor{c}{rgb}{0,0,0};
\colorlet{c}{natcomp!70};
\draw [c] (9.53284,1.2621) -- (9.53284,1.28829);
\draw [c] (9.53284,1.28829) -- (9.53284,1.31333);
\draw [c] (9.49492,1.28829) -- (9.53284,1.28829);
\draw [c] (9.53284,1.28829) -- (9.57076,1.28829);
\definecolor{c}{rgb}{0,0,0};
\colorlet{c}{natcomp!70};
\draw [c] (9.60869,1.1794) -- (9.60869,1.20761);
\draw [c] (9.60869,1.20761) -- (9.60869,1.23449);
\draw [c] (9.57076,1.20761) -- (9.60869,1.20761);
\draw [c] (9.60869,1.20761) -- (9.64661,1.20761);
\definecolor{c}{rgb}{0,0,0};
\colorlet{c}{natcomp!70};
\draw [c] (9.68453,1.17427) -- (9.68453,1.20235);
\draw [c] (9.68453,1.20235) -- (9.68453,1.22911);
\draw [c] (9.64661,1.20235) -- (9.68453,1.20235);
\draw [c] (9.68453,1.20235) -- (9.72246,1.20235);
\definecolor{c}{rgb}{0,0,0};
\colorlet{c}{natcomp!70};
\draw [c] (9.76038,1.13334) -- (9.76038,1.1631);
\draw [c] (9.76038,1.1631) -- (9.76038,1.19139);
\draw [c] (9.72246,1.1631) -- (9.76038,1.1631);
\draw [c] (9.76038,1.1631) -- (9.79831,1.1631);
\definecolor{c}{rgb}{0,0,0};
\colorlet{c}{natcomp!70};
\draw [c] (9.83623,1.0783) -- (9.83623,1.10911);
\draw [c] (9.83623,1.10911) -- (9.83623,1.13834);
\draw [c] (9.79831,1.10911) -- (9.83623,1.10911);
\draw [c] (9.83623,1.10911) -- (9.87415,1.10911);
\definecolor{c}{rgb}{0,0,0};
\colorlet{c}{natcomp!70};
\draw [c] (9.91208,1.11033) -- (9.91208,1.14063);
\draw [c] (9.91208,1.14063) -- (9.91208,1.16941);
\draw [c] (9.87415,1.14063) -- (9.91208,1.14063);
\draw [c] (9.91208,1.14063) -- (9.95,1.14063);
\definecolor{c}{rgb}{0,0,0};
\draw [c] (1,0.680516) -- (9.95,0.680516);
\draw [anchor= east] (9.95,0.108883) node[color=c, rotate=0]{$M_{\gamma\gamma}\text{ [GeV]}$};
\draw [c] (1,0.863234) -- (1,0.680516);
\draw [c] (1.37924,0.771875) -- (1.37924,0.680516);
\draw [c] (1.75847,0.771875) -- (1.75847,0.680516);
\draw [c] (2.13771,0.771875) -- (2.13771,0.680516);
\draw [c] (2.51695,0.863234) -- (2.51695,0.680516);
\draw [c] (2.89619,0.771875) -- (2.89619,0.680516);
\draw [c] (3.27542,0.771875) -- (3.27542,0.680516);
\draw [c] (3.65466,0.771875) -- (3.65466,0.680516);
\draw [c] (4.0339,0.863234) -- (4.0339,0.680516);
\draw [c] (4.41314,0.771875) -- (4.41314,0.680516);
\draw [c] (4.79237,0.771875) -- (4.79237,0.680516);
\draw [c] (5.17161,0.771875) -- (5.17161,0.680516);
\draw [c] (5.55085,0.863234) -- (5.55085,0.680516);
\draw [c] (5.93008,0.771875) -- (5.93008,0.680516);
\draw [c] (6.30932,0.771875) -- (6.30932,0.680516);
\draw [c] (6.68856,0.771875) -- (6.68856,0.680516);
\draw [c] (7.0678,0.863234) -- (7.0678,0.680516);
\draw [c] (7.44703,0.771875) -- (7.44703,0.680516);
\draw [c] (7.82627,0.771875) -- (7.82627,0.680516);
\draw [c] (8.20551,0.771875) -- (8.20551,0.680516);
\draw [c] (8.58475,0.863234) -- (8.58475,0.680516);
\draw [c] (8.58475,0.863234) -- (8.58475,0.680516);
\draw [c] (8.96398,0.771875) -- (8.96398,0.680516);
\draw [c] (9.34322,0.771875) -- (9.34322,0.680516);
\draw [c] (9.72246,0.771875) -- (9.72246,0.680516);
\draw [anchor=base] (1,0.353868) node[color=c, rotate=0]{0};
\draw [anchor=base] (2.51695,0.353868) node[color=c, rotate=0]{200};
\draw [anchor=base] (4.0339,0.353868) node[color=c, rotate=0]{400};
\draw [anchor=base] (5.55085,0.353868) node[color=c, rotate=0]{600};
\draw [anchor=base] (7.0678,0.353868) node[color=c, rotate=0]{800};
\draw [anchor=base] (8.58475,0.353868) node[color=c, rotate=0]{1000};
\draw [c] (1,0.680516) -- (1,6.73711);
\draw [anchor= east] (-0.12,6.73711) node[color=c, rotate=90]{Number of events};
\draw [c] (1.267,0.688075) -- (1,0.688075);
\draw [anchor= east] (0.922,0.688075) node[color=c, rotate=0]{$10^{-1}$};
\draw [c] (1.1335,1.08586) -- (1,1.08586);
\draw [c] (1.1335,1.31855) -- (1,1.31855);
\draw [c] (1.1335,1.48365) -- (1,1.48365);
\draw [c] (1.1335,1.61171) -- (1,1.61171);
\draw [c] (1.1335,1.71634) -- (1,1.71634);
\draw [c] (1.1335,1.80481) -- (1,1.80481);
\draw [c] (1.1335,1.88144) -- (1,1.88144);
\draw [c] (1.1335,1.94903) -- (1,1.94903);
\draw [c] (1.267,2.0095) -- (1,2.0095);
\draw [anchor= east] (0.922,2.0095) node[color=c, rotate=0]{1};
\draw [c] (1.1335,2.40729) -- (1,2.40729);
\draw [c] (1.1335,2.63998) -- (1,2.63998);
\draw [c] (1.1335,2.80507) -- (1,2.80507);
\draw [c] (1.1335,2.93313) -- (1,2.93313);
\draw [c] (1.1335,3.03776) -- (1,3.03776);
\draw [c] (1.1335,3.12623) -- (1,3.12623);
\draw [c] (1.1335,3.20286) -- (1,3.20286);
\draw [c] (1.1335,3.27046) -- (1,3.27046);
\draw [c] (1.267,3.33092) -- (1,3.33092);
\draw [anchor= east] (0.922,3.33092) node[color=c, rotate=0]{10};
\draw [c] (1.1335,3.72871) -- (1,3.72871);
\draw [c] (1.1335,3.9614) -- (1,3.9614);
\draw [c] (1.1335,4.1265) -- (1,4.1265);
\draw [c] (1.1335,4.25456) -- (1,4.25456);
\draw [c] (1.1335,4.35919) -- (1,4.35919);
\draw [c] (1.1335,4.44765) -- (1,4.44765);
\draw [c] (1.1335,4.52428) -- (1,4.52428);
\draw [c] (1.1335,4.59188) -- (1,4.59188);
\draw [c] (1.267,4.65234) -- (1,4.65234);
\draw [anchor= east] (0.922,4.65234) node[color=c, rotate=0]{$10^{2}$};
\draw [c] (1.1335,5.05013) -- (1,5.05013);
\draw [c] (1.1335,5.28282) -- (1,5.28282);
\draw [c] (1.1335,5.44792) -- (1,5.44792);
\draw [c] (1.1335,5.57598) -- (1,5.57598);
\draw [c] (1.1335,5.68061) -- (1,5.68061);
\draw [c] (1.1335,5.76908) -- (1,5.76908);
\draw [c] (1.1335,5.84571) -- (1,5.84571);
\draw [c] (1.1335,5.9133) -- (1,5.9133);
\draw [c] (1.267,5.97377) -- (1,5.97377);
\draw [anchor= east] (0.922,5.97377) node[color=c, rotate=0]{$10^{3}$};
\draw [c] (1.1335,6.37155) -- (1,6.37155);
\draw [c] (1.1335,6.60425) -- (1,6.60425);
\colorlet{c}{natgreen};
\draw [c] (1.56886,0.680516) -- (1.56886,2.32133);
\draw [c] (1.56886,2.32133) -- (1.56886,2.71912);
\draw [c] (1.53093,2.32133) -- (1.56886,2.32133);
\draw [c] (1.56886,2.32133) -- (1.60678,2.32133);
\definecolor{c}{rgb}{0,0,0};
\colorlet{c}{natgreen};
\draw [c] (1.6447,2.15154) -- (1.6447,2.77477);
\draw [c] (1.6447,2.77477) -- (1.6447,3.06646);
\draw [c] (1.60678,2.77477) -- (1.6447,2.77477);
\draw [c] (1.6447,2.77477) -- (1.68263,2.77477);
\definecolor{c}{rgb}{0,0,0};
\colorlet{c}{natgreen};
\draw [c] (1.72055,4.09914) -- (1.72055,4.23107);
\draw [c] (1.72055,4.23107) -- (1.72055,4.33827);
\draw [c] (1.68263,4.23107) -- (1.72055,4.23107);
\draw [c] (1.72055,4.23107) -- (1.75847,4.23107);
\definecolor{c}{rgb}{0,0,0};
\colorlet{c}{natgreen};
\draw [c] (1.7964,5.88673) -- (1.7964,5.91485);
\draw [c] (1.7964,5.91485) -- (1.7964,5.94166);
\draw [c] (1.75847,5.91485) -- (1.7964,5.91485);
\draw [c] (1.7964,5.91485) -- (1.83432,5.91485);
\definecolor{c}{rgb}{0,0,0};
\colorlet{c}{natgreen};
\draw [c] (1.87225,6.36331) -- (1.87225,6.38193);
\draw [c] (1.87225,6.38193) -- (1.87225,6.39997);
\draw [c] (1.83432,6.38193) -- (1.87225,6.38193);
\draw [c] (1.87225,6.38193) -- (1.91017,6.38193);
\definecolor{c}{rgb}{0,0,0};
\colorlet{c}{natgreen};
\draw [c] (1.94809,6.44614) -- (1.94809,6.46344);
\draw [c] (1.94809,6.46344) -- (1.94809,6.48024);
\draw [c] (1.91017,6.46344) -- (1.94809,6.46344);
\draw [c] (1.94809,6.46344) -- (1.98602,6.46344);
\definecolor{c}{rgb}{0,0,0};
\colorlet{c}{natgreen};
\draw [c] (2.02394,6.37812) -- (2.02394,6.39641);
\draw [c] (2.02394,6.39641) -- (2.02394,6.41414);
\draw [c] (1.98602,6.39641) -- (2.02394,6.39641);
\draw [c] (2.02394,6.39641) -- (2.06186,6.39641);
\definecolor{c}{rgb}{0,0,0};
\colorlet{c}{natgreen};
\draw [c] (2.09979,6.3115) -- (2.09979,6.3308);
\draw [c] (2.09979,6.3308) -- (2.09979,6.34947);
\draw [c] (2.06186,6.3308) -- (2.09979,6.3308);
\draw [c] (2.09979,6.3308) -- (2.13771,6.3308);
\definecolor{c}{rgb}{0,0,0};
\colorlet{c}{natgreen};
\draw [c] (2.17564,6.22934) -- (2.17564,6.25001);
\draw [c] (2.17564,6.25001) -- (2.17564,6.26996);
\draw [c] (2.13771,6.25001) -- (2.17564,6.25001);
\draw [c] (2.17564,6.25001) -- (2.21356,6.25001);
\definecolor{c}{rgb}{0,0,0};
\colorlet{c}{natgreen};
\draw [c] (2.25148,6.12923) -- (2.25148,6.15189);
\draw [c] (2.25148,6.15189) -- (2.25148,6.17369);
\draw [c] (2.21356,6.15189) -- (2.25148,6.15189);
\draw [c] (2.25148,6.15189) -- (2.28941,6.15189);
\definecolor{c}{rgb}{0,0,0};
\colorlet{c}{natgreen};
\draw [c] (2.32733,5.99974) -- (2.32733,6.02512);
\draw [c] (2.32733,6.02512) -- (2.32733,6.04941);
\draw [c] (2.28941,6.02512) -- (2.32733,6.02512);
\draw [c] (2.32733,6.02512) -- (2.36525,6.02512);
\definecolor{c}{rgb}{0,0,0};
\colorlet{c}{natgreen};
\draw [c] (2.40318,5.93984) -- (2.40318,5.96669);
\draw [c] (2.40318,5.96669) -- (2.40318,5.99234);
\draw [c] (2.36525,5.96669) -- (2.40318,5.96669);
\draw [c] (2.40318,5.96669) -- (2.4411,5.96669);
\definecolor{c}{rgb}{0,0,0};
\colorlet{c}{natgreen};
\draw [c] (2.47903,5.86198) -- (2.47903,5.89044);
\draw [c] (2.47903,5.89044) -- (2.47903,5.91755);
\draw [c] (2.4411,5.89044) -- (2.47903,5.89044);
\draw [c] (2.47903,5.89044) -- (2.51695,5.89044);
\definecolor{c}{rgb}{0,0,0};
\colorlet{c}{natgreen};
\draw [c] (2.55487,5.728) -- (2.55487,5.76007);
\draw [c] (2.55487,5.76007) -- (2.55487,5.79044);
\draw [c] (2.51695,5.76007) -- (2.55487,5.76007);
\draw [c] (2.55487,5.76007) -- (2.5928,5.76007);
\definecolor{c}{rgb}{0,0,0};
\colorlet{c}{natgreen};
\draw [c] (2.63072,5.662) -- (2.63072,5.69603);
\draw [c] (2.63072,5.69603) -- (2.63072,5.72815);
\draw [c] (2.5928,5.69603) -- (2.63072,5.69603);
\draw [c] (2.63072,5.69603) -- (2.66864,5.69603);
\definecolor{c}{rgb}{0,0,0};
\colorlet{c}{natgreen};
\draw [c] (2.70657,5.52065) -- (2.70657,5.55932);
\draw [c] (2.70657,5.55932) -- (2.70657,5.59556);
\draw [c] (2.66864,5.55932) -- (2.70657,5.55932);
\draw [c] (2.70657,5.55932) -- (2.74449,5.55932);
\definecolor{c}{rgb}{0,0,0};
\colorlet{c}{natgreen};
\draw [c] (2.78242,5.39535) -- (2.78242,5.43811);
\draw [c] (2.78242,5.43811) -- (2.78242,5.4779);
\draw [c] (2.74449,5.43811) -- (2.78242,5.43811);
\draw [c] (2.78242,5.43811) -- (2.82034,5.43811);
\definecolor{c}{rgb}{0,0,0};
\colorlet{c}{natgreen};
\draw [c] (2.85826,5.25104) -- (2.85826,5.2996);
\draw [c] (2.85826,5.2996) -- (2.85826,5.34438);
\draw [c] (2.82034,5.2996) -- (2.85826,5.2996);
\draw [c] (2.85826,5.2996) -- (2.89619,5.2996);
\definecolor{c}{rgb}{0,0,0};
\colorlet{c}{natgreen};
\draw [c] (2.93411,5.28231) -- (2.93411,5.32982);
\draw [c] (2.93411,5.32982) -- (2.93411,5.37369);
\draw [c] (2.89619,5.32982) -- (2.93411,5.32982);
\draw [c] (2.93411,5.32982) -- (2.97203,5.32982);
\definecolor{c}{rgb}{0,0,0};
\colorlet{c}{natgreen};
\draw [c] (3.00996,5.21897) -- (3.00996,5.2689);
\draw [c] (3.00996,5.2689) -- (3.00996,5.31483);
\draw [c] (2.97203,5.2689) -- (3.00996,5.2689);
\draw [c] (3.00996,5.2689) -- (3.04788,5.2689);
\definecolor{c}{rgb}{0,0,0};
\colorlet{c}{natgreen};
\draw [c] (3.08581,5.21023) -- (3.08581,5.2608);
\draw [c] (3.08581,5.2608) -- (3.08581,5.30728);
\draw [c] (3.04788,5.2608) -- (3.08581,5.2608);
\draw [c] (3.08581,5.2608) -- (3.12373,5.2608);
\definecolor{c}{rgb}{0,0,0};
\colorlet{c}{natgreen};
\draw [c] (3.16165,4.96886) -- (3.16165,5.03107);
\draw [c] (3.16165,5.03107) -- (3.16165,5.08718);
\draw [c] (3.12373,5.03107) -- (3.16165,5.03107);
\draw [c] (3.16165,5.03107) -- (3.19958,5.03107);
\definecolor{c}{rgb}{0,0,0};
\colorlet{c}{natgreen};
\draw [c] (3.2375,4.94124) -- (3.2375,5.00492);
\draw [c] (3.2375,5.00492) -- (3.2375,5.06224);
\draw [c] (3.19958,5.00492) -- (3.2375,5.00492);
\draw [c] (3.2375,5.00492) -- (3.27542,5.00492);
\definecolor{c}{rgb}{0,0,0};
\colorlet{c}{natgreen};
\draw [c] (3.31335,4.69119) -- (3.31335,4.76971);
\draw [c] (3.31335,4.76971) -- (3.31335,4.83877);
\draw [c] (3.27542,4.76971) -- (3.31335,4.76971);
\draw [c] (3.31335,4.76971) -- (3.35127,4.76971);
\definecolor{c}{rgb}{0,0,0};
\colorlet{c}{natgreen};
\draw [c] (3.38919,4.67612) -- (3.38919,4.75548);
\draw [c] (3.38919,4.75548) -- (3.38919,4.82519);
\draw [c] (3.35127,4.75548) -- (3.38919,4.75548);
\draw [c] (3.38919,4.75548) -- (3.42712,4.75548);
\definecolor{c}{rgb}{0,0,0};
\colorlet{c}{natgreen};
\draw [c] (3.46504,4.74061) -- (3.46504,4.81591);
\draw [c] (3.46504,4.81591) -- (3.46504,4.88246);
\draw [c] (3.42712,4.81591) -- (3.46504,4.81591);
\draw [c] (3.46504,4.81591) -- (3.50297,4.81591);
\definecolor{c}{rgb}{0,0,0};
\colorlet{c}{natgreen};
\draw [c] (3.54089,4.54627) -- (3.54089,4.63667);
\draw [c] (3.54089,4.63667) -- (3.54089,4.71475);
\draw [c] (3.50297,4.63667) -- (3.54089,4.63667);
\draw [c] (3.54089,4.63667) -- (3.57881,4.63667);
\definecolor{c}{rgb}{0,0,0};
\colorlet{c}{natgreen};
\draw [c] (3.61674,4.44837) -- (3.61674,4.54601);
\draw [c] (3.61674,4.54601) -- (3.61674,4.62944);
\draw [c] (3.57881,4.54601) -- (3.61674,4.54601);
\draw [c] (3.61674,4.54601) -- (3.65466,4.54601);
\definecolor{c}{rgb}{0,0,0};
\colorlet{c}{natgreen};
\draw [c] (3.69258,4.55513) -- (3.69258,4.64683);
\draw [c] (3.69258,4.64683) -- (3.69258,4.72588);
\draw [c] (3.65466,4.64683) -- (3.69258,4.64683);
\draw [c] (3.69258,4.64683) -- (3.73051,4.64683);
\definecolor{c}{rgb}{0,0,0};
\colorlet{c}{natgreen};
\draw [c] (3.76843,4.48937) -- (3.76843,4.58443);
\draw [c] (3.76843,4.58443) -- (3.76843,4.66596);
\draw [c] (3.73051,4.58443) -- (3.76843,4.58443);
\draw [c] (3.76843,4.58443) -- (3.80636,4.58443);
\definecolor{c}{rgb}{0,0,0};
\colorlet{c}{natgreen};
\draw [c] (3.84428,4.1449) -- (3.84428,4.26971);
\draw [c] (3.84428,4.26971) -- (3.84428,4.37217);
\draw [c] (3.80636,4.26971) -- (3.84428,4.26971);
\draw [c] (3.84428,4.26971) -- (3.8822,4.26971);
\definecolor{c}{rgb}{0,0,0};
\colorlet{c}{natgreen};
\draw [c] (3.92013,4.28662) -- (3.92013,4.40131);
\draw [c] (3.92013,4.40131) -- (3.92013,4.49685);
\draw [c] (3.8822,4.40131) -- (3.92013,4.40131);
\draw [c] (3.92013,4.40131) -- (3.95805,4.40131);
\definecolor{c}{rgb}{0,0,0};
\colorlet{c}{natgreen};
\draw [c] (3.99597,4.2373) -- (3.99597,4.35773);
\draw [c] (3.99597,4.35773) -- (3.99597,4.45722);
\draw [c] (3.95805,4.35773) -- (3.99597,4.35773);
\draw [c] (3.99597,4.35773) -- (4.0339,4.35773);
\definecolor{c}{rgb}{0,0,0};
\colorlet{c}{natgreen};
\draw [c] (4.07182,4.17735) -- (4.07182,4.30431);
\draw [c] (4.07182,4.30431) -- (4.07182,4.40821);
\draw [c] (4.0339,4.30431) -- (4.07182,4.30431);
\draw [c] (4.07182,4.30431) -- (4.10975,4.30431);
\definecolor{c}{rgb}{0,0,0};
\colorlet{c}{natgreen};
\draw [c] (4.14767,3.58812) -- (4.14767,3.79113);
\draw [c] (4.14767,3.79113) -- (4.14767,3.94079);
\draw [c] (4.10975,3.79113) -- (4.14767,3.79113);
\draw [c] (4.14767,3.79113) -- (4.18559,3.79113);
\definecolor{c}{rgb}{0,0,0};
\colorlet{c}{natgreen};
\draw [c] (4.22352,3.94542) -- (4.22352,4.0949);
\draw [c] (4.22352,4.0949) -- (4.22352,4.21338);
\draw [c] (4.18559,4.0949) -- (4.22352,4.0949);
\draw [c] (4.22352,4.0949) -- (4.26144,4.0949);
\definecolor{c}{rgb}{0,0,0};
\colorlet{c}{natgreen};
\draw [c] (4.29936,4.19993) -- (4.29936,4.3255);
\draw [c] (4.29936,4.3255) -- (4.29936,4.42847);
\draw [c] (4.26144,4.3255) -- (4.29936,4.3255);
\draw [c] (4.29936,4.3255) -- (4.33729,4.3255);
\definecolor{c}{rgb}{0,0,0};
\colorlet{c}{natgreen};
\draw [c] (4.37521,3.55265) -- (4.37521,3.76314);
\draw [c] (4.37521,3.76314) -- (4.37521,3.91681);
\draw [c] (4.33729,3.76314) -- (4.37521,3.76314);
\draw [c] (4.37521,3.76314) -- (4.41314,3.76314);
\definecolor{c}{rgb}{0,0,0};
\colorlet{c}{natgreen};
\draw [c] (4.45106,3.92155) -- (4.45106,4.07489);
\draw [c] (4.45106,4.07489) -- (4.45106,4.19578);
\draw [c] (4.41314,4.07489) -- (4.45106,4.07489);
\draw [c] (4.45106,4.07489) -- (4.48898,4.07489);
\definecolor{c}{rgb}{0,0,0};
\colorlet{c}{natgreen};
\draw [c] (4.52691,3.54663) -- (4.52691,3.75691);
\draw [c] (4.52691,3.75691) -- (4.52691,3.91046);
\draw [c] (4.48898,3.75691) -- (4.52691,3.75691);
\draw [c] (4.52691,3.75691) -- (4.56483,3.75691);
\definecolor{c}{rgb}{0,0,0};
\colorlet{c}{natgreen};
\draw [c] (4.60275,3.66877) -- (4.60275,3.8548);
\draw [c] (4.60275,3.8548) -- (4.60275,3.99507);
\draw [c] (4.56483,3.8548) -- (4.60275,3.8548);
\draw [c] (4.60275,3.8548) -- (4.64068,3.8548);
\definecolor{c}{rgb}{0,0,0};
\colorlet{c}{natgreen};
\draw [c] (4.6786,3.38407) -- (4.6786,3.62745);
\draw [c] (4.6786,3.62745) -- (4.6786,3.79781);
\draw [c] (4.64068,3.62745) -- (4.6786,3.62745);
\draw [c] (4.6786,3.62745) -- (4.71653,3.62745);
\definecolor{c}{rgb}{0,0,0};
\colorlet{c}{natgreen};
\draw [c] (4.75445,3.33733) -- (4.75445,3.56304);
\draw [c] (4.75445,3.56304) -- (4.75445,3.72461);
\draw [c] (4.71653,3.56304) -- (4.75445,3.56304);
\draw [c] (4.75445,3.56304) -- (4.79237,3.56304);
\definecolor{c}{rgb}{0,0,0};
\colorlet{c}{natgreen};
\draw [c] (4.8303,3.82832) -- (4.8303,3.99795);
\draw [c] (4.8303,3.99795) -- (4.8303,4.12871);
\draw [c] (4.79237,3.99795) -- (4.8303,3.99795);
\draw [c] (4.8303,3.99795) -- (4.86822,3.99795);
\definecolor{c}{rgb}{0,0,0};
\colorlet{c}{natgreen};
\draw [c] (4.90614,3.33704) -- (4.90614,3.58231);
\draw [c] (4.90614,3.58231) -- (4.90614,3.7536);
\draw [c] (4.86822,3.58231) -- (4.90614,3.58231);
\draw [c] (4.90614,3.58231) -- (4.94407,3.58231);
\definecolor{c}{rgb}{0,0,0};
\colorlet{c}{natgreen};
\draw [c] (4.98199,3.5684) -- (4.98199,3.77461);
\draw [c] (4.98199,3.77461) -- (4.98199,3.92599);
\draw [c] (4.94407,3.77461) -- (4.98199,3.77461);
\draw [c] (4.98199,3.77461) -- (5.01992,3.77461);
\definecolor{c}{rgb}{0,0,0};
\colorlet{c}{natgreen};
\draw [c] (5.05784,3.37991) -- (5.05784,3.63276);
\draw [c] (5.05784,3.63276) -- (5.05784,3.80768);
\draw [c] (5.01992,3.63276) -- (5.05784,3.63276);
\draw [c] (5.05784,3.63276) -- (5.09576,3.63276);
\definecolor{c}{rgb}{0,0,0};
\colorlet{c}{natgreen};
\draw [c] (5.13369,3.09137) -- (5.13369,3.38505);
\draw [c] (5.13369,3.38505) -- (5.13369,3.57837);
\draw [c] (5.09576,3.38505) -- (5.13369,3.38505);
\draw [c] (5.13369,3.38505) -- (5.17161,3.38505);
\definecolor{c}{rgb}{0,0,0};
\colorlet{c}{natgreen};
\draw [c] (5.20953,2.62672) -- (5.20953,3.12268);
\draw [c] (5.20953,3.12268) -- (5.20953,3.38468);
\draw [c] (5.17161,3.12268) -- (5.20953,3.12268);
\draw [c] (5.20953,3.12268) -- (5.24746,3.12268);
\definecolor{c}{rgb}{0,0,0};
\colorlet{c}{natgreen};
\draw [c] (5.28538,3.30387) -- (5.28538,3.56756);
\draw [c] (5.28538,3.56756) -- (5.28538,3.74755);
\draw [c] (5.24746,3.56756) -- (5.28538,3.56756);
\draw [c] (5.28538,3.56756) -- (5.32331,3.56756);
\definecolor{c}{rgb}{0,0,0};
\colorlet{c}{natgreen};
\draw [c] (5.36123,3.06923) -- (5.36123,3.39873);
\draw [c] (5.36123,3.39873) -- (5.36123,3.60673);
\draw [c] (5.32331,3.39873) -- (5.36123,3.39873);
\draw [c] (5.36123,3.39873) -- (5.39915,3.39873);
\definecolor{c}{rgb}{0,0,0};
\colorlet{c}{natgreen};
\draw [c] (5.43708,2.75701) -- (5.43708,3.13388);
\draw [c] (5.43708,3.13388) -- (5.43708,3.35943);
\draw [c] (5.39915,3.13388) -- (5.43708,3.13388);
\draw [c] (5.43708,3.13388) -- (5.475,3.13388);
\definecolor{c}{rgb}{0,0,0};
\colorlet{c}{natgreen};
\draw [c] (5.58877,2.38329) -- (5.58877,2.91028);
\draw [c] (5.58877,2.91028) -- (5.58877,3.18029);
\draw [c] (5.55085,2.91028) -- (5.58877,2.91028);
\draw [c] (5.58877,2.91028) -- (5.6267,2.91028);
\definecolor{c}{rgb}{0,0,0};
\colorlet{c}{natgreen};
\draw [c] (5.66462,2.31794) -- (5.66462,2.87603);
\draw [c] (5.66462,2.87603) -- (5.66462,3.15354);
\draw [c] (5.6267,2.87603) -- (5.66462,2.87603);
\draw [c] (5.66462,2.87603) -- (5.70254,2.87603);
\definecolor{c}{rgb}{0,0,0};
\colorlet{c}{natgreen};
\draw [c] (5.74047,2.64865) -- (5.74047,3.14387);
\draw [c] (5.74047,3.14387) -- (5.74047,3.40568);
\draw [c] (5.70254,3.14387) -- (5.74047,3.14387);
\draw [c] (5.74047,3.14387) -- (5.77839,3.14387);
\definecolor{c}{rgb}{0,0,0};
\colorlet{c}{natgreen};
\draw [c] (5.81631,2.63248) -- (5.81631,3.01946);
\draw [c] (5.81631,3.01946) -- (5.81631,3.2485);
\draw [c] (5.77839,3.01946) -- (5.81631,3.01946);
\draw [c] (5.81631,3.01946) -- (5.85424,3.01946);
\definecolor{c}{rgb}{0,0,0};
\colorlet{c}{natgreen};
\draw [c] (5.89216,3.02506) -- (5.89216,3.35228);
\draw [c] (5.89216,3.35228) -- (5.89216,3.55938);
\draw [c] (5.85424,3.35228) -- (5.89216,3.35228);
\draw [c] (5.89216,3.35228) -- (5.93008,3.35228);
\definecolor{c}{rgb}{0,0,0};
\colorlet{c}{natgreen};
\draw [c] (5.96801,2.96904) -- (5.96801,3.3376);
\draw [c] (5.96801,3.3376) -- (5.96801,3.5602);
\draw [c] (5.93008,3.3376) -- (5.96801,3.3376);
\draw [c] (5.96801,3.3376) -- (6.00593,3.3376);
\definecolor{c}{rgb}{0,0,0};
\colorlet{c}{natgreen};
\draw [c] (6.04386,2.13625) -- (6.04386,2.84097);
\draw [c] (6.04386,2.84097) -- (6.04386,3.14788);
\draw [c] (6.00593,2.84097) -- (6.04386,2.84097);
\draw [c] (6.04386,2.84097) -- (6.08178,2.84097);
\definecolor{c}{rgb}{0,0,0};
\colorlet{c}{natgreen};
\draw [c] (6.1197,0.680516) -- (6.1197,2.38875);
\draw [c] (6.1197,2.38875) -- (6.1197,2.78654);
\draw [c] (6.08178,2.38875) -- (6.1197,2.38875);
\draw [c] (6.1197,2.38875) -- (6.15763,2.38875);
\definecolor{c}{rgb}{0,0,0};
\colorlet{c}{natgreen};
\draw [c] (6.19555,2.54233) -- (6.19555,3.0409);
\draw [c] (6.19555,3.0409) -- (6.19555,3.3036);
\draw [c] (6.15763,3.0409) -- (6.19555,3.0409);
\draw [c] (6.19555,3.0409) -- (6.23347,3.0409);
\definecolor{c}{rgb}{0,0,0};
\colorlet{c}{natgreen};
\draw [c] (6.2714,2.75147) -- (6.2714,3.15831);
\draw [c] (6.2714,3.15831) -- (6.2714,3.39399);
\draw [c] (6.23347,3.15831) -- (6.2714,3.15831);
\draw [c] (6.2714,3.15831) -- (6.30932,3.15831);
\definecolor{c}{rgb}{0,0,0};
\colorlet{c}{natgreen};
\draw [c] (6.34725,1.87883) -- (6.34725,2.64933);
\draw [c] (6.34725,2.64933) -- (6.34725,2.96681);
\draw [c] (6.30932,2.64933) -- (6.34725,2.64933);
\draw [c] (6.34725,2.64933) -- (6.38517,2.64933);
\definecolor{c}{rgb}{0,0,0};
\colorlet{c}{natgreen};
\draw [c] (6.42309,2.61419) -- (6.42309,3.10871);
\draw [c] (6.42309,3.10871) -- (6.42309,3.37033);
\draw [c] (6.38517,3.10871) -- (6.42309,3.10871);
\draw [c] (6.42309,3.10871) -- (6.46102,3.10871);
\definecolor{c}{rgb}{0,0,0};
\colorlet{c}{natgreen};
\draw [c] (6.57479,1.35007) -- (6.57479,2.57712);
\draw [c] (6.57479,2.57712) -- (6.57479,2.94005);
\draw [c] (6.53686,2.57712) -- (6.57479,2.57712);
\draw [c] (6.57479,2.57712) -- (6.61271,2.57712);
\definecolor{c}{rgb}{0,0,0};
\colorlet{c}{natgreen};
\draw [c] (6.65064,2.86211) -- (6.65064,3.23882);
\draw [c] (6.65064,3.23882) -- (6.65064,3.4643);
\draw [c] (6.61271,3.23882) -- (6.65064,3.23882);
\draw [c] (6.65064,3.23882) -- (6.68856,3.23882);
\definecolor{c}{rgb}{0,0,0};
\colorlet{c}{natgreen};
\draw [c] (6.87818,0.680516) -- (6.87818,1.90334);
\draw [c] (6.87818,1.90334) -- (6.87818,2.30113);
\draw [c] (6.84025,1.90334) -- (6.87818,1.90334);
\draw [c] (6.87818,1.90334) -- (6.9161,1.90334);
\definecolor{c}{rgb}{0,0,0};
\colorlet{c}{natgreen};
\draw [c] (7.02987,1.97897) -- (7.02987,2.7321);
\draw [c] (7.02987,2.7321) -- (7.02987,3.04692);
\draw [c] (6.99195,2.7321) -- (7.02987,2.7321);
\draw [c] (7.02987,2.7321) -- (7.0678,2.7321);
\definecolor{c}{rgb}{0,0,0};
\colorlet{c}{natgreen};
\draw [c] (7.18157,0.680516) -- (7.18157,2.58979);
\draw [c] (7.18157,2.58979) -- (7.18157,2.98757);
\draw [c] (7.14364,2.58979) -- (7.18157,2.58979);
\draw [c] (7.18157,2.58979) -- (7.21949,2.58979);
\definecolor{c}{rgb}{0,0,0};
\colorlet{c}{natgreen};
\draw [c] (7.33326,0.680516) -- (7.33326,2.24583);
\draw [c] (7.33326,2.24583) -- (7.33326,2.64362);
\draw [c] (7.29534,2.24583) -- (7.33326,2.24583);
\draw [c] (7.33326,2.24583) -- (7.37119,2.24583);
\definecolor{c}{rgb}{0,0,0};
\colorlet{c}{natgreen};
\draw [c] (7.63665,0.680516) -- (7.63665,2.43257);
\draw [c] (7.63665,2.43257) -- (7.63665,2.83036);
\draw [c] (7.59873,2.43257) -- (7.63665,2.43257);
\draw [c] (7.63665,2.43257) -- (7.67458,2.43257);
\definecolor{c}{rgb}{0,0,0};
\colorlet{c}{natgreen};
\draw [c] (7.78835,0.680516) -- (7.78835,2.25707);
\draw [c] (7.78835,2.25707) -- (7.78835,2.65486);
\draw [c] (7.75042,2.25707) -- (7.78835,2.25707);
\draw [c] (7.78835,2.25707) -- (7.82627,2.25707);
\definecolor{c}{rgb}{0,0,0};
\colorlet{c}{natgreen};
\draw [c] (7.86419,0.680516) -- (7.86419,2.4637);
\draw [c] (7.86419,2.4637) -- (7.86419,2.86148);
\draw [c] (7.82627,2.4637) -- (7.86419,2.4637);
\draw [c] (7.86419,2.4637) -- (7.90212,2.4637);
\definecolor{c}{rgb}{0,0,0};
\colorlet{c}{natgreen};
\draw [c] (8.01589,2.10948) -- (8.01589,2.81591);
\draw [c] (8.01589,2.81591) -- (8.01589,3.12312);
\draw [c] (7.97797,2.81591) -- (8.01589,2.81591);
\draw [c] (8.01589,2.81591) -- (8.05381,2.81591);
\definecolor{c}{rgb}{0,0,0};
\colorlet{c}{natgreen};
\draw [c] (8.09174,2.16055) -- (8.09174,2.86528);
\draw [c] (8.09174,2.86528) -- (8.09174,3.1722);
\draw [c] (8.05381,2.86528) -- (8.09174,2.86528);
\draw [c] (8.09174,2.86528) -- (8.12966,2.86528);
\definecolor{c}{rgb}{0,0,0};
\colorlet{c}{natgreen};
\draw [c] (8.24343,1.97776) -- (8.24343,2.72975);
\draw [c] (8.24343,2.72975) -- (8.24343,3.04439);
\draw [c] (8.20551,2.72975) -- (8.24343,2.72975);
\draw [c] (8.24343,2.72975) -- (8.28136,2.72975);
\definecolor{c}{rgb}{0,0,0};
\colorlet{c}{natgreen};
\draw [c] (8.39513,0.680516) -- (8.39513,2.49913);
\draw [c] (8.39513,2.49913) -- (8.39513,2.89692);
\draw [c] (8.3572,2.49913) -- (8.39513,2.49913);
\draw [c] (8.39513,2.49913) -- (8.43305,2.49913);
\definecolor{c}{rgb}{0,0,0};
\colorlet{c}{natgreen};
\draw [c] (9.60869,0.680516) -- (9.60869,2.58979);
\draw [c] (9.60869,2.58979) -- (9.60869,2.98757);
\draw [c] (9.57076,2.58979) -- (9.60869,2.58979);
\draw [c] (9.60869,2.58979) -- (9.64661,2.58979);
\definecolor{c}{rgb}{0,0,0};
\draw [anchor= west] (6.52221,6.06375) node[color=c, rotate=0]{ATLAS MC};
\colorlet{c}{natgreen};
\draw [c] (5.77633,6.06375) -- (6.39058,6.06375);
\draw [c] (6.08345,5.87679) -- (6.08345,6.25072);
\definecolor{c}{rgb}{0,0,0};
\draw [anchor= west] (6.52221,5.44054) node[color=c, rotate=0]{CalcHEP + box};
\colorlet{c}{natcomp!70};
\draw [c] (5.77633,5.44054) -- (6.39058,5.44054);
\draw [c] (6.08345,5.25358) -- (6.08345,5.62751);
\end{tikzpicture}

\end{infilsf}
\end{minipage}
\begin{minipage}[b]{.3\textwidth}
\caption{Comparing the \atlas{} distribution with the one produced by CalcHEP, combined with a distribution for the box diagram contribution.}\label{ggcomp}
\end{minipage}
\end{figure}

As the data background and box diagram distributions have the same sort of shape, it is natural to combine them before attempting to extrapolate a shape. The function we fit is [some power function... $\chi^2$ fit with probability \~$10^{-5}$, I think...] result in fig.~\ref{bckfit}.

\begin{figure}[htp]
\begin{minipage}[b]{.69\textwidth}
\begin{infilsf} \tiny
\begin{tikzpicture}[x=.092\textwidth,y=.092\textwidth]
\pgfdeclareplotmark{cross} {
\pgfpathmoveto{\pgfpoint{-0.3\pgfplotmarksize}{\pgfplotmarksize}}
\pgfpathlineto{\pgfpoint{+0.3\pgfplotmarksize}{\pgfplotmarksize}}
\pgfpathlineto{\pgfpoint{+0.3\pgfplotmarksize}{0.3\pgfplotmarksize}}
\pgfpathlineto{\pgfpoint{+1\pgfplotmarksize}{0.3\pgfplotmarksize}}
\pgfpathlineto{\pgfpoint{+1\pgfplotmarksize}{-0.3\pgfplotmarksize}}
\pgfpathlineto{\pgfpoint{+0.3\pgfplotmarksize}{-0.3\pgfplotmarksize}}
\pgfpathlineto{\pgfpoint{+0.3\pgfplotmarksize}{-1.\pgfplotmarksize}}
\pgfpathlineto{\pgfpoint{-0.3\pgfplotmarksize}{-1.\pgfplotmarksize}}
\pgfpathlineto{\pgfpoint{-0.3\pgfplotmarksize}{-0.3\pgfplotmarksize}}
\pgfpathlineto{\pgfpoint{-1.\pgfplotmarksize}{-0.3\pgfplotmarksize}}
\pgfpathlineto{\pgfpoint{-1.\pgfplotmarksize}{0.3\pgfplotmarksize}}
\pgfpathlineto{\pgfpoint{-0.3\pgfplotmarksize}{0.3\pgfplotmarksize}}
\pgfpathclose
\pgfusepathqstroke
}
\pgfdeclareplotmark{cross*} {
\pgfpathmoveto{\pgfpoint{-0.3\pgfplotmarksize}{\pgfplotmarksize}}
\pgfpathlineto{\pgfpoint{+0.3\pgfplotmarksize}{\pgfplotmarksize}}
\pgfpathlineto{\pgfpoint{+0.3\pgfplotmarksize}{0.3\pgfplotmarksize}}
\pgfpathlineto{\pgfpoint{+1\pgfplotmarksize}{0.3\pgfplotmarksize}}
\pgfpathlineto{\pgfpoint{+1\pgfplotmarksize}{-0.3\pgfplotmarksize}}
\pgfpathlineto{\pgfpoint{+0.3\pgfplotmarksize}{-0.3\pgfplotmarksize}}
\pgfpathlineto{\pgfpoint{+0.3\pgfplotmarksize}{-1.\pgfplotmarksize}}
\pgfpathlineto{\pgfpoint{-0.3\pgfplotmarksize}{-1.\pgfplotmarksize}}
\pgfpathlineto{\pgfpoint{-0.3\pgfplotmarksize}{-0.3\pgfplotmarksize}}
\pgfpathlineto{\pgfpoint{-1.\pgfplotmarksize}{-0.3\pgfplotmarksize}}
\pgfpathlineto{\pgfpoint{-1.\pgfplotmarksize}{0.3\pgfplotmarksize}}
\pgfpathlineto{\pgfpoint{-0.3\pgfplotmarksize}{0.3\pgfplotmarksize}}
\pgfpathclose
\pgfusepathqfillstroke
}
\pgfdeclareplotmark{newstar} {
\pgfpathmoveto{\pgfqpoint{0pt}{\pgfplotmarksize}}
\pgfpathlineto{\pgfqpointpolar{44}{0.5\pgfplotmarksize}}
\pgfpathlineto{\pgfqpointpolar{18}{\pgfplotmarksize}}
\pgfpathlineto{\pgfqpointpolar{-20}{0.5\pgfplotmarksize}}
\pgfpathlineto{\pgfqpointpolar{-54}{\pgfplotmarksize}}
\pgfpathlineto{\pgfqpointpolar{-90}{0.5\pgfplotmarksize}}
\pgfpathlineto{\pgfqpointpolar{234}{\pgfplotmarksize}}
\pgfpathlineto{\pgfqpointpolar{198}{0.5\pgfplotmarksize}}
\pgfpathlineto{\pgfqpointpolar{162}{\pgfplotmarksize}}
\pgfpathlineto{\pgfqpointpolar{134}{0.5\pgfplotmarksize}}
\pgfpathclose
\pgfusepathqstroke
}
\pgfdeclareplotmark{newstar*} {
\pgfpathmoveto{\pgfqpoint{0pt}{\pgfplotmarksize}}
\pgfpathlineto{\pgfqpointpolar{44}{0.5\pgfplotmarksize}}
\pgfpathlineto{\pgfqpointpolar{18}{\pgfplotmarksize}}
\pgfpathlineto{\pgfqpointpolar{-20}{0.5\pgfplotmarksize}}
\pgfpathlineto{\pgfqpointpolar{-54}{\pgfplotmarksize}}
\pgfpathlineto{\pgfqpointpolar{-90}{0.5\pgfplotmarksize}}
\pgfpathlineto{\pgfqpointpolar{234}{\pgfplotmarksize}}
\pgfpathlineto{\pgfqpointpolar{198}{0.5\pgfplotmarksize}}
\pgfpathlineto{\pgfqpointpolar{162}{\pgfplotmarksize}}
\pgfpathlineto{\pgfqpointpolar{134}{0.5\pgfplotmarksize}}
\pgfpathclose
\pgfusepathqfillstroke
}
\definecolor{c}{rgb}{1,1,1};
\draw [color=c, fill=c] (0,0) rectangle (10,6.80516);
\draw [color=c, fill=c] (1,0.680516) rectangle (9.95,6.73711);
\definecolor{c}{rgb}{0,0,0};
\draw [c] (1,0.680516) -- (1,6.73711) -- (9.95,6.73711) -- (9.95,0.680516) -- (1,0.680516);
\definecolor{c}{rgb}{1,1,1};
\draw [color=c, fill=c] (1,0.680516) rectangle (9.95,6.73711);
\definecolor{c}{rgb}{0,0,0};
\draw [c] (1,0.680516) -- (1,6.73711) -- (9.95,6.73711) -- (9.95,0.680516) -- (1,0.680516);
\colorlet{c}{natgreen};
\draw [c] (1.04431,3.58796) -- (1.04431,3.68717);
\draw [c] (1.04431,3.68717) -- (1.04431,3.77009);
\draw [c] (1,3.68717) -- (1.04431,3.68717);
\draw [c] (1.04431,3.68717) -- (1.08861,3.68717);
\definecolor{c}{rgb}{0,0,0};
\colorlet{c}{natgreen};
\draw [c] (1.13292,4.52557) -- (1.13292,4.56494);
\draw [c] (1.13292,4.56494) -- (1.13292,4.60147);
\draw [c] (1.08861,4.56494) -- (1.13292,4.56494);
\draw [c] (1.13292,4.56494) -- (1.17723,4.56494);
\definecolor{c}{rgb}{0,0,0};
\colorlet{c}{natgreen};
\draw [c] (1.22153,4.55305) -- (1.22153,4.59136);
\draw [c] (1.22153,4.59136) -- (1.22153,4.62698);
\draw [c] (1.17723,4.59136) -- (1.22153,4.59136);
\draw [c] (1.22153,4.59136) -- (1.26584,4.59136);
\definecolor{c}{rgb}{0,0,0};
\colorlet{c}{natgreen};
\draw [c] (1.31015,4.84177) -- (1.31015,4.87058);
\draw [c] (1.31015,4.87058) -- (1.31015,4.89784);
\draw [c] (1.26584,4.87058) -- (1.31015,4.87058);
\draw [c] (1.31015,4.87058) -- (1.35446,4.87058);
\definecolor{c}{rgb}{0,0,0};
\colorlet{c}{natgreen};
\draw [c] (1.39876,4.85866) -- (1.39876,4.887);
\draw [c] (1.39876,4.887) -- (1.39876,4.91384);
\draw [c] (1.35446,4.887) -- (1.39876,4.887);
\draw [c] (1.39876,4.887) -- (1.44307,4.887);
\definecolor{c}{rgb}{0,0,0};
\colorlet{c}{natgreen};
\draw [c] (1.48738,4.90421) -- (1.48738,4.9313);
\draw [c] (1.48738,4.9313) -- (1.48738,4.95702);
\draw [c] (1.44307,4.9313) -- (1.48738,4.9313);
\draw [c] (1.48738,4.9313) -- (1.53168,4.9313);
\definecolor{c}{rgb}{0,0,0};
\colorlet{c}{natgreen};
\draw [c] (1.57599,4.86959) -- (1.57599,4.89763);
\draw [c] (1.57599,4.89763) -- (1.57599,4.92419);
\draw [c] (1.53168,4.89763) -- (1.57599,4.89763);
\draw [c] (1.57599,4.89763) -- (1.6203,4.89763);
\definecolor{c}{rgb}{0,0,0};
\colorlet{c}{natgreen};
\draw [c] (1.6646,4.99569) -- (1.6646,5.02044);
\draw [c] (1.6646,5.02044) -- (1.6646,5.04403);
\draw [c] (1.6203,5.02044) -- (1.6646,5.02044);
\draw [c] (1.6646,5.02044) -- (1.70891,5.02044);
\definecolor{c}{rgb}{0,0,0};
\colorlet{c}{natgreen};
\draw [c] (1.75322,5.16291) -- (1.75322,5.18389);
\draw [c] (1.75322,5.18389) -- (1.75322,5.20403);
\draw [c] (1.70891,5.18389) -- (1.75322,5.18389);
\draw [c] (1.75322,5.18389) -- (1.79752,5.18389);
\definecolor{c}{rgb}{0,0,0};
\colorlet{c}{natgreen};
\draw [c] (1.84183,5.12662) -- (1.84183,5.14828);
\draw [c] (1.84183,5.14828) -- (1.84183,5.16906);
\draw [c] (1.79752,5.14828) -- (1.84183,5.14828);
\draw [c] (1.84183,5.14828) -- (1.88614,5.14828);
\definecolor{c}{rgb}{0,0,0};
\colorlet{c}{natgreen};
\draw [c] (1.93045,5.86287) -- (1.93045,5.87324);
\draw [c] (1.93045,5.87324) -- (1.93045,5.88341);
\draw [c] (1.88614,5.87324) -- (1.93045,5.87324);
\draw [c] (1.93045,5.87324) -- (1.97475,5.87324);
\definecolor{c}{rgb}{0,0,0};
\colorlet{c}{natgreen};
\draw [c] (2.01906,6.27256) -- (2.01906,6.27949);
\draw [c] (2.01906,6.27949) -- (2.01906,6.28633);
\draw [c] (1.97475,6.27949) -- (2.01906,6.27949);
\draw [c] (2.01906,6.27949) -- (2.06337,6.27949);
\definecolor{c}{rgb}{0,0,0};
\colorlet{c}{natgreen};
\draw [c] (2.10767,6.37754) -- (2.10767,6.3838);
\draw [c] (2.10767,6.3838) -- (2.10767,6.38998);
\draw [c] (2.06337,6.3838) -- (2.10767,6.3838);
\draw [c] (2.10767,6.3838) -- (2.15198,6.3838);
\definecolor{c}{rgb}{0,0,0};
\colorlet{c}{natgreen};
\draw [c] (2.19629,6.40128) -- (2.19629,6.4074);
\draw [c] (2.19629,6.4074) -- (2.19629,6.41345);
\draw [c] (2.15198,6.4074) -- (2.19629,6.4074);
\draw [c] (2.19629,6.4074) -- (2.24059,6.4074);
\definecolor{c}{rgb}{0,0,0};
\colorlet{c}{natgreen};
\draw [c] (2.2849,6.36434) -- (2.2849,6.3707);
\draw [c] (2.2849,6.3707) -- (2.2849,6.37697);
\draw [c] (2.24059,6.3707) -- (2.2849,6.3707);
\draw [c] (2.2849,6.3707) -- (2.32921,6.3707);
\definecolor{c}{rgb}{0,0,0};
\colorlet{c}{natgreen};
\draw [c] (2.37351,6.33688) -- (2.37351,6.34342);
\draw [c] (2.37351,6.34342) -- (2.37351,6.34987);
\draw [c] (2.32921,6.34342) -- (2.37351,6.34342);
\draw [c] (2.37351,6.34342) -- (2.41782,6.34342);
\definecolor{c}{rgb}{0,0,0};
\colorlet{c}{natgreen};
\draw [c] (2.46213,6.23385) -- (2.46213,6.24109);
\draw [c] (2.46213,6.24109) -- (2.46213,6.24822);
\draw [c] (2.41782,6.24109) -- (2.46213,6.24109);
\draw [c] (2.46213,6.24109) -- (2.50644,6.24109);
\definecolor{c}{rgb}{0,0,0};
\colorlet{c}{natgreen};
\draw [c] (2.55074,6.16569) -- (2.55074,6.17343);
\draw [c] (2.55074,6.17343) -- (2.55074,6.18107);
\draw [c] (2.50644,6.17343) -- (2.55074,6.17343);
\draw [c] (2.55074,6.17343) -- (2.59505,6.17343);
\definecolor{c}{rgb}{0,0,0};
\colorlet{c}{natgreen};
\draw [c] (2.63936,6.09344) -- (2.63936,6.10177);
\draw [c] (2.63936,6.10177) -- (2.63936,6.10996);
\draw [c] (2.59505,6.10177) -- (2.63936,6.10177);
\draw [c] (2.63936,6.10177) -- (2.68366,6.10177);
\definecolor{c}{rgb}{0,0,0};
\colorlet{c}{natgreen};
\draw [c] (2.72797,6.00711) -- (2.72797,6.01618);
\draw [c] (2.72797,6.01618) -- (2.72797,6.02509);
\draw [c] (2.68366,6.01618) -- (2.72797,6.01618);
\draw [c] (2.72797,6.01618) -- (2.77228,6.01618);
\definecolor{c}{rgb}{0,0,0};
\colorlet{c}{natgreen};
\draw [c] (2.81658,5.93403) -- (2.81658,5.94378);
\draw [c] (2.81658,5.94378) -- (2.81658,5.95336);
\draw [c] (2.77228,5.94378) -- (2.81658,5.94378);
\draw [c] (2.81658,5.94378) -- (2.86089,5.94378);
\definecolor{c}{rgb}{0,0,0};
\colorlet{c}{natgreen};
\draw [c] (2.9052,5.84107) -- (2.9052,5.85176);
\draw [c] (2.9052,5.85176) -- (2.9052,5.86224);
\draw [c] (2.86089,5.85176) -- (2.9052,5.85176);
\draw [c] (2.9052,5.85176) -- (2.94951,5.85176);
\definecolor{c}{rgb}{0,0,0};
\colorlet{c}{natgreen};
\draw [c] (2.99381,5.75037) -- (2.99381,5.76207);
\draw [c] (2.99381,5.76207) -- (2.99381,5.77351);
\draw [c] (2.94951,5.76207) -- (2.99381,5.76207);
\draw [c] (2.99381,5.76207) -- (3.03812,5.76207);
\definecolor{c}{rgb}{0,0,0};
\colorlet{c}{natgreen};
\draw [c] (3.08243,5.71391) -- (3.08243,5.72604);
\draw [c] (3.08243,5.72604) -- (3.08243,5.73789);
\draw [c] (3.03812,5.72604) -- (3.08243,5.72604);
\draw [c] (3.08243,5.72604) -- (3.12673,5.72604);
\definecolor{c}{rgb}{0,0,0};
\colorlet{c}{natgreen};
\draw [c] (3.17104,5.61884) -- (3.17104,5.63216);
\draw [c] (3.17104,5.63216) -- (3.17104,5.64513);
\draw [c] (3.12673,5.63216) -- (3.17104,5.63216);
\draw [c] (3.17104,5.63216) -- (3.21535,5.63216);
\definecolor{c}{rgb}{0,0,0};
\colorlet{c}{natgreen};
\draw [c] (3.25965,5.53981) -- (3.25965,5.55422);
\draw [c] (3.25965,5.55422) -- (3.25965,5.56823);
\draw [c] (3.21535,5.55422) -- (3.25965,5.55422);
\draw [c] (3.25965,5.55422) -- (3.30396,5.55422);
\definecolor{c}{rgb}{0,0,0};
\colorlet{c}{natgreen};
\draw [c] (3.34827,5.4505) -- (3.34827,5.46623);
\draw [c] (3.34827,5.46623) -- (3.34827,5.48148);
\draw [c] (3.30396,5.46623) -- (3.34827,5.46623);
\draw [c] (3.34827,5.46623) -- (3.39257,5.46623);
\definecolor{c}{rgb}{0,0,0};
\colorlet{c}{natgreen};
\draw [c] (3.43688,5.3618) -- (3.43688,5.37899);
\draw [c] (3.43688,5.37899) -- (3.43688,5.39562);
\draw [c] (3.39257,5.37899) -- (3.43688,5.37899);
\draw [c] (3.43688,5.37899) -- (3.48119,5.37899);
\definecolor{c}{rgb}{0,0,0};
\colorlet{c}{natgreen};
\draw [c] (3.5255,5.3192) -- (3.5255,5.33712);
\draw [c] (3.5255,5.33712) -- (3.5255,5.35443);
\draw [c] (3.48119,5.33712) -- (3.5255,5.33712);
\draw [c] (3.5255,5.33712) -- (3.5698,5.33712);
\definecolor{c}{rgb}{0,0,0};
\colorlet{c}{natgreen};
\draw [c] (3.61411,5.29761) -- (3.61411,5.31592);
\draw [c] (3.61411,5.31592) -- (3.61411,5.33358);
\draw [c] (3.5698,5.31592) -- (3.61411,5.31592);
\draw [c] (3.61411,5.31592) -- (3.65842,5.31592);
\definecolor{c}{rgb}{0,0,0};
\colorlet{c}{natgreen};
\draw [c] (3.70272,5.20597) -- (3.70272,5.22601);
\draw [c] (3.70272,5.22601) -- (3.70272,5.2453);
\draw [c] (3.65842,5.22601) -- (3.70272,5.22601);
\draw [c] (3.70272,5.22601) -- (3.74703,5.22601);
\definecolor{c}{rgb}{0,0,0};
\colorlet{c}{natgreen};
\draw [c] (3.79134,5.13236) -- (3.79134,5.15391);
\draw [c] (3.79134,5.15391) -- (3.79134,5.17457);
\draw [c] (3.74703,5.15391) -- (3.79134,5.15391);
\draw [c] (3.79134,5.15391) -- (3.83564,5.15391);
\definecolor{c}{rgb}{0,0,0};
\colorlet{c}{natgreen};
\draw [c] (3.87995,5.04025) -- (3.87995,5.06387);
\draw [c] (3.87995,5.06387) -- (3.87995,5.08644);
\draw [c] (3.83564,5.06387) -- (3.87995,5.06387);
\draw [c] (3.87995,5.06387) -- (3.92426,5.06387);
\definecolor{c}{rgb}{0,0,0};
\colorlet{c}{natgreen};
\draw [c] (3.96856,5.03979) -- (3.96856,5.06345);
\draw [c] (3.96856,5.06345) -- (3.96856,5.08605);
\draw [c] (3.92426,5.06345) -- (3.96856,5.06345);
\draw [c] (3.96856,5.06345) -- (4.01287,5.06345);
\definecolor{c}{rgb}{0,0,0};
\colorlet{c}{natgreen};
\draw [c] (4.05718,4.9159) -- (4.05718,4.94262);
\draw [c] (4.05718,4.94262) -- (4.05718,4.96799);
\draw [c] (4.01287,4.94262) -- (4.05718,4.94262);
\draw [c] (4.05718,4.94262) -- (4.10149,4.94262);
\definecolor{c}{rgb}{0,0,0};
\colorlet{c}{natgreen};
\draw [c] (4.14579,4.77187) -- (4.14579,4.80257);
\draw [c] (4.14579,4.80257) -- (4.14579,4.83152);
\draw [c] (4.10149,4.80257) -- (4.14579,4.80257);
\draw [c] (4.14579,4.80257) -- (4.1901,4.80257);
\definecolor{c}{rgb}{0,0,0};
\colorlet{c}{natgreen};
\draw [c] (4.23441,4.76339) -- (4.23441,4.79444);
\draw [c] (4.23441,4.79444) -- (4.23441,4.8237);
\draw [c] (4.1901,4.79444) -- (4.23441,4.79444);
\draw [c] (4.23441,4.79444) -- (4.27871,4.79444);
\definecolor{c}{rgb}{0,0,0};
\colorlet{c}{natgreen};
\draw [c] (4.32302,4.75203) -- (4.32302,4.78345);
\draw [c] (4.32302,4.78345) -- (4.32302,4.81304);
\draw [c] (4.27871,4.78345) -- (4.32302,4.78345);
\draw [c] (4.32302,4.78345) -- (4.36733,4.78345);
\definecolor{c}{rgb}{0,0,0};
\colorlet{c}{natgreen};
\draw [c] (4.41163,4.63179) -- (4.41163,4.66718);
\draw [c] (4.41163,4.66718) -- (4.41163,4.70026);
\draw [c] (4.36733,4.66718) -- (4.41163,4.66718);
\draw [c] (4.41163,4.66718) -- (4.45594,4.66718);
\definecolor{c}{rgb}{0,0,0};
\colorlet{c}{natgreen};
\draw [c] (4.50025,4.62157) -- (4.50025,4.65723);
\draw [c] (4.50025,4.65723) -- (4.50025,4.69055);
\draw [c] (4.45594,4.65723) -- (4.50025,4.65723);
\draw [c] (4.50025,4.65723) -- (4.54455,4.65723);
\definecolor{c}{rgb}{0,0,0};
\colorlet{c}{natgreen};
\draw [c] (4.58886,4.64707) -- (4.58886,4.6819);
\draw [c] (4.58886,4.6819) -- (4.58886,4.71448);
\draw [c] (4.54455,4.6819) -- (4.58886,4.6819);
\draw [c] (4.58886,4.6819) -- (4.63317,4.6819);
\definecolor{c}{rgb}{0,0,0};
\colorlet{c}{natgreen};
\draw [c] (4.67748,4.39624) -- (4.67748,4.44093);
\draw [c] (4.67748,4.44093) -- (4.67748,4.48199);
\draw [c] (4.63317,4.44093) -- (4.67748,4.44093);
\draw [c] (4.67748,4.44093) -- (4.72178,4.44093);
\definecolor{c}{rgb}{0,0,0};
\colorlet{c}{natgreen};
\draw [c] (4.76609,4.37532) -- (4.76609,4.42087);
\draw [c] (4.76609,4.42087) -- (4.76609,4.46266);
\draw [c] (4.72178,4.42087) -- (4.76609,4.42087);
\draw [c] (4.76609,4.42087) -- (4.8104,4.42087);
\definecolor{c}{rgb}{0,0,0};
\colorlet{c}{natgreen};
\draw [c] (4.8547,4.42175) -- (4.8547,4.46537);
\draw [c] (4.8547,4.46537) -- (4.8547,4.50552);
\draw [c] (4.8104,4.46537) -- (4.8547,4.46537);
\draw [c] (4.8547,4.46537) -- (4.89901,4.46537);
\definecolor{c}{rgb}{0,0,0};
\colorlet{c}{natgreen};
\draw [c] (4.94332,4.31787) -- (4.94332,4.36614);
\draw [c] (4.94332,4.36614) -- (4.94332,4.4102);
\draw [c] (4.89901,4.36614) -- (4.94332,4.36614);
\draw [c] (4.94332,4.36614) -- (4.98762,4.36614);
\definecolor{c}{rgb}{0,0,0};
\colorlet{c}{natgreen};
\draw [c] (5.03193,4.39308) -- (5.03193,4.43794);
\draw [c] (5.03193,4.43794) -- (5.03193,4.47915);
\draw [c] (4.98762,4.43794) -- (5.03193,4.43794);
\draw [c] (5.03193,4.43794) -- (5.07624,4.43794);
\definecolor{c}{rgb}{0,0,0};
\colorlet{c}{natgreen};
\draw [c] (5.12054,4.17587) -- (5.12054,4.23112);
\draw [c] (5.12054,4.23112) -- (5.12054,4.28093);
\draw [c] (5.07624,4.23112) -- (5.12054,4.23112);
\draw [c] (5.12054,4.23112) -- (5.16485,4.23112);
\definecolor{c}{rgb}{0,0,0};
\colorlet{c}{natgreen};
\draw [c] (5.20916,4.1169) -- (5.20916,4.17568);
\draw [c] (5.20916,4.17568) -- (5.20916,4.22834);
\draw [c] (5.16485,4.17568) -- (5.20916,4.17568);
\draw [c] (5.20916,4.17568) -- (5.25347,4.17568);
\definecolor{c}{rgb}{0,0,0};
\colorlet{c}{natgreen};
\draw [c] (5.29777,4.12189) -- (5.29777,4.18035);
\draw [c] (5.29777,4.18035) -- (5.29777,4.23276);
\draw [c] (5.25347,4.18035) -- (5.29777,4.18035);
\draw [c] (5.29777,4.18035) -- (5.34208,4.18035);
\definecolor{c}{rgb}{0,0,0};
\colorlet{c}{natgreen};
\draw [c] (5.38639,4.21247) -- (5.38639,4.26613);
\draw [c] (5.38639,4.26613) -- (5.38639,4.31463);
\draw [c] (5.34208,4.26613) -- (5.38639,4.26613);
\draw [c] (5.38639,4.26613) -- (5.43069,4.26613);
\definecolor{c}{rgb}{0,0,0};
\colorlet{c}{natgreen};
\draw [c] (5.475,4.14696) -- (5.475,4.20408);
\draw [c] (5.475,4.20408) -- (5.475,4.2554);
\draw [c] (5.43069,4.20408) -- (5.475,4.20408);
\draw [c] (5.475,4.20408) -- (5.51931,4.20408);
\definecolor{c}{rgb}{0,0,0};
\colorlet{c}{natgreen};
\draw [c] (5.56361,3.9376) -- (5.56361,4.00776);
\draw [c] (5.56361,4.00776) -- (5.56361,4.06937);
\draw [c] (5.51931,4.00776) -- (5.56361,4.00776);
\draw [c] (5.56361,4.00776) -- (5.60792,4.00776);
\definecolor{c}{rgb}{0,0,0};
\colorlet{c}{natgreen};
\draw [c] (5.65223,3.87671) -- (5.65223,3.95106);
\draw [c] (5.65223,3.95106) -- (5.65223,4.01587);
\draw [c] (5.60792,3.95106) -- (5.65223,3.95106);
\draw [c] (5.65223,3.95106) -- (5.69653,3.95106);
\definecolor{c}{rgb}{0,0,0};
\colorlet{c}{natgreen};
\draw [c] (5.74084,4.07352) -- (5.74084,4.13487);
\draw [c] (5.74084,4.13487) -- (5.74084,4.18959);
\draw [c] (5.69653,4.13487) -- (5.74084,4.13487);
\draw [c] (5.74084,4.13487) -- (5.78515,4.13487);
\definecolor{c}{rgb}{0,0,0};
\colorlet{c}{natgreen};
\draw [c] (5.82946,3.67762) -- (5.82946,3.7682);
\draw [c] (5.82946,3.7682) -- (5.82946,3.84501);
\draw [c] (5.78515,3.7682) -- (5.82946,3.7682);
\draw [c] (5.82946,3.7682) -- (5.87376,3.7682);
\definecolor{c}{rgb}{0,0,0};
\colorlet{c}{natgreen};
\draw [c] (5.91807,3.94968) -- (5.91807,4.01916);
\draw [c] (5.91807,4.01916) -- (5.91807,4.08024);
\draw [c] (5.87376,4.01916) -- (5.91807,4.01916);
\draw [c] (5.91807,4.01916) -- (5.96238,4.01916);
\definecolor{c}{rgb}{0,0,0};
\colorlet{c}{natgreen};
\draw [c] (6.00668,3.75403) -- (6.00668,3.83816);
\draw [c] (6.00668,3.83816) -- (6.00668,3.91028);
\draw [c] (5.96238,3.83816) -- (6.00668,3.83816);
\draw [c] (6.00668,3.83816) -- (6.05099,3.83816);
\definecolor{c}{rgb}{0,0,0};
\colorlet{c}{natgreen};
\draw [c] (6.0953,3.74843) -- (6.0953,3.83301);
\draw [c] (6.0953,3.83301) -- (6.0953,3.90546);
\draw [c] (6.05099,3.83301) -- (6.0953,3.83301);
\draw [c] (6.0953,3.83301) -- (6.1396,3.83301);
\definecolor{c}{rgb}{0,0,0};
\colorlet{c}{natgreen};
\draw [c] (6.18391,3.4963) -- (6.18391,3.60488);
\draw [c] (6.18391,3.60488) -- (6.18391,3.69423);
\draw [c] (6.1396,3.60488) -- (6.18391,3.60488);
\draw [c] (6.18391,3.60488) -- (6.22822,3.60488);
\definecolor{c}{rgb}{0,0,0};
\colorlet{c}{natgreen};
\draw [c] (6.27252,3.50919) -- (6.27252,3.6164);
\draw [c] (6.27252,3.6164) -- (6.27252,3.70483);
\draw [c] (6.22822,3.6164) -- (6.27252,3.6164);
\draw [c] (6.27252,3.6164) -- (6.31683,3.6164);
\definecolor{c}{rgb}{0,0,0};
\colorlet{c}{natgreen};
\draw [c] (6.36114,3.2093) -- (6.36114,3.3527);
\draw [c] (6.36114,3.3527) -- (6.36114,3.46433);
\draw [c] (6.31683,3.3527) -- (6.36114,3.3527);
\draw [c] (6.36114,3.3527) -- (6.40545,3.3527);
\definecolor{c}{rgb}{0,0,0};
\colorlet{c}{natgreen};
\draw [c] (6.44975,3.97014) -- (6.44975,4.03823);
\draw [c] (6.44975,4.03823) -- (6.44975,4.09824);
\draw [c] (6.40545,4.03823) -- (6.44975,4.03823);
\draw [c] (6.44975,4.03823) -- (6.49406,4.03823);
\definecolor{c}{rgb}{0,0,0};
\colorlet{c}{natgreen};
\draw [c] (6.53837,2.88887) -- (6.53837,3.08578);
\draw [c] (6.53837,3.08578) -- (6.53837,3.2272);
\draw [c] (6.49406,3.08578) -- (6.53837,3.08578);
\draw [c] (6.53837,3.08578) -- (6.58267,3.08578);
\definecolor{c}{rgb}{0,0,0};
\colorlet{c}{natgreen};
\draw [c] (6.62698,3.59236) -- (6.62698,3.69101);
\draw [c] (6.62698,3.69101) -- (6.62698,3.77354);
\draw [c] (6.58267,3.69101) -- (6.62698,3.69101);
\draw [c] (6.62698,3.69101) -- (6.67129,3.69101);
\definecolor{c}{rgb}{0,0,0};
\colorlet{c}{natgreen};
\draw [c] (6.71559,3.44265) -- (6.71559,3.55662);
\draw [c] (6.71559,3.55662) -- (6.71559,3.6496);
\draw [c] (6.67129,3.55662) -- (6.71559,3.55662);
\draw [c] (6.71559,3.55662) -- (6.7599,3.55662);
\definecolor{c}{rgb}{0,0,0};
\colorlet{c}{natgreen};
\draw [c] (6.80421,3.38595) -- (6.80421,3.50683);
\draw [c] (6.80421,3.50683) -- (6.80421,3.60434);
\draw [c] (6.7599,3.50683) -- (6.80421,3.50683);
\draw [c] (6.80421,3.50683) -- (6.84852,3.50683);
\definecolor{c}{rgb}{0,0,0};
\colorlet{c}{natgreen};
\draw [c] (6.89282,3.38837) -- (6.89282,3.5091);
\draw [c] (6.89282,3.5091) -- (6.89282,3.60651);
\draw [c] (6.84852,3.5091) -- (6.89282,3.5091);
\draw [c] (6.89282,3.5091) -- (6.93713,3.5091);
\definecolor{c}{rgb}{0,0,0};
\colorlet{c}{natgreen};
\draw [c] (6.98144,3.3475) -- (6.98144,3.47318);
\draw [c] (6.98144,3.47318) -- (6.98144,3.57378);
\draw [c] (6.93713,3.47318) -- (6.98144,3.47318);
\draw [c] (6.98144,3.47318) -- (7.02574,3.47318);
\definecolor{c}{rgb}{0,0,0};
\colorlet{c}{natgreen};
\draw [c] (7.07005,2.39988) -- (7.07005,2.70986);
\draw [c] (7.07005,2.70986) -- (7.07005,2.90074);
\draw [c] (7.02574,2.70986) -- (7.07005,2.70986);
\draw [c] (7.07005,2.70986) -- (7.11436,2.70986);
\definecolor{c}{rgb}{0,0,0};
\colorlet{c}{natgreen};
\draw [c] (7.15866,3.27796) -- (7.15866,3.41252);
\draw [c] (7.15866,3.41252) -- (7.15866,3.51872);
\draw [c] (7.11436,3.41252) -- (7.15866,3.41252);
\draw [c] (7.15866,3.41252) -- (7.20297,3.41252);
\definecolor{c}{rgb}{0,0,0};
\colorlet{c}{natgreen};
\draw [c] (7.24728,3.27734) -- (7.24728,3.41197);
\draw [c] (7.24728,3.41197) -- (7.24728,3.51823);
\draw [c] (7.20297,3.41197) -- (7.24728,3.41197);
\draw [c] (7.24728,3.41197) -- (7.29158,3.41197);
\definecolor{c}{rgb}{0,0,0};
\colorlet{c}{natgreen};
\draw [c] (7.33589,3.29526) -- (7.33589,3.42755);
\draw [c] (7.33589,3.42755) -- (7.33589,3.53234);
\draw [c] (7.29158,3.42755) -- (7.33589,3.42755);
\draw [c] (7.33589,3.42755) -- (7.3802,3.42755);
\definecolor{c}{rgb}{0,0,0};
\colorlet{c}{natgreen};
\draw [c] (7.4245,2.95689) -- (7.4245,3.13997);
\draw [c] (7.4245,3.13997) -- (7.4245,3.27415);
\draw [c] (7.3802,3.13997) -- (7.4245,3.13997);
\draw [c] (7.4245,3.13997) -- (7.46881,3.13997);
\definecolor{c}{rgb}{0,0,0};
\colorlet{c}{natgreen};
\draw [c] (7.51312,2.63192) -- (7.51312,2.88466);
\draw [c] (7.51312,2.88466) -- (7.51312,3.05249);
\draw [c] (7.46881,2.88466) -- (7.51312,2.88466);
\draw [c] (7.51312,2.88466) -- (7.55743,2.88466);
\definecolor{c}{rgb}{0,0,0};
\colorlet{c}{natgreen};
\draw [c] (7.60173,3.10883) -- (7.60173,3.26766);
\draw [c] (7.60173,3.26766) -- (7.60173,3.38838);
\draw [c] (7.55743,3.26766) -- (7.60173,3.26766);
\draw [c] (7.60173,3.26766) -- (7.64604,3.26766);
\definecolor{c}{rgb}{0,0,0};
\colorlet{c}{natgreen};
\draw [c] (7.69035,1.68208) -- (7.69035,2.29607);
\draw [c] (7.69035,2.29607) -- (7.69035,2.56555);
\draw [c] (7.64604,2.29607) -- (7.69035,2.29607);
\draw [c] (7.69035,2.29607) -- (7.73465,2.29607);
\definecolor{c}{rgb}{0,0,0};
\colorlet{c}{natgreen};
\draw [c] (7.77896,3.37021) -- (7.77896,3.49311);
\draw [c] (7.77896,3.49311) -- (7.77896,3.59193);
\draw [c] (7.73465,3.49311) -- (7.77896,3.49311);
\draw [c] (7.77896,3.49311) -- (7.82327,3.49311);
\definecolor{c}{rgb}{0,0,0};
\colorlet{c}{natgreen};
\draw [c] (7.86757,2.70656) -- (7.86757,2.94167);
\draw [c] (7.86757,2.94167) -- (7.86757,3.10162);
\draw [c] (7.82327,2.94167) -- (7.86757,2.94167);
\draw [c] (7.86757,2.94167) -- (7.91188,2.94167);
\definecolor{c}{rgb}{0,0,0};
\colorlet{c}{natgreen};
\draw [c] (7.95619,1.24756) -- (7.95619,2.13237);
\draw [c] (7.95619,2.13237) -- (7.95619,2.43724);
\draw [c] (7.91188,2.13237) -- (7.95619,2.13237);
\draw [c] (7.95619,2.13237) -- (8.00049,2.13237);
\definecolor{c}{rgb}{0,0,0};
\colorlet{c}{natgreen};
\draw [c] (8.0448,2.33128) -- (8.0448,2.66863);
\draw [c] (8.0448,2.66863) -- (8.0448,2.86933);
\draw [c] (8.00049,2.66863) -- (8.0448,2.66863);
\draw [c] (8.0448,2.66863) -- (8.08911,2.66863);
\definecolor{c}{rgb}{0,0,0};
\colorlet{c}{natgreen};
\draw [c] (8.13342,1.23328) -- (8.13342,2.12817);
\draw [c] (8.13342,2.12817) -- (8.13342,2.43399);
\draw [c] (8.08911,2.12817) -- (8.13342,2.12817);
\draw [c] (8.13342,2.12817) -- (8.17772,2.12817);
\definecolor{c}{rgb}{0,0,0};
\colorlet{c}{natgreen};
\draw [c] (8.22203,2.37726) -- (8.22203,2.70014);
\draw [c] (8.22203,2.70014) -- (8.22203,2.89574);
\draw [c] (8.17772,2.70014) -- (8.22203,2.70014);
\draw [c] (8.22203,2.70014) -- (8.26634,2.70014);
\definecolor{c}{rgb}{0,0,0};
\colorlet{c}{natgreen};
\draw [c] (8.31064,2.28945) -- (8.31064,2.64049);
\draw [c] (8.31064,2.64049) -- (8.31064,2.84583);
\draw [c] (8.26634,2.64049) -- (8.31064,2.64049);
\draw [c] (8.31064,2.64049) -- (8.35495,2.64049);
\definecolor{c}{rgb}{0,0,0};
\colorlet{c}{natgreen};
\draw [c] (8.39926,1.52931) -- (8.39926,2.2217);
\draw [c] (8.39926,2.2217) -- (8.39926,2.50372);
\draw [c] (8.35495,2.2217) -- (8.39926,2.2217);
\draw [c] (8.39926,2.2217) -- (8.44356,2.2217);
\definecolor{c}{rgb}{0,0,0};
\colorlet{c}{natgreen};
\draw [c] (8.48787,2.69897) -- (8.48787,2.93582);
\draw [c] (8.48787,2.93582) -- (8.48787,3.09656);
\draw [c] (8.44356,2.93582) -- (8.48787,2.93582);
\draw [c] (8.48787,2.93582) -- (8.53218,2.93582);
\definecolor{c}{rgb}{0,0,0};
\colorlet{c}{natgreen};
\draw [c] (8.6651,1.23778) -- (8.6651,2.12949);
\draw [c] (8.6651,2.12949) -- (8.6651,2.43501);
\draw [c] (8.62079,2.12949) -- (8.6651,2.12949);
\draw [c] (8.6651,2.12949) -- (8.70941,2.12949);
\definecolor{c}{rgb}{0,0,0};
\colorlet{c}{natgreen};
\draw [c] (8.75371,0.680516) -- (8.75371,1.90572);
\draw [c] (8.75371,1.90572) -- (8.75371,2.265);
\draw [c] (8.70941,1.90572) -- (8.75371,1.90572);
\draw [c] (8.75371,1.90572) -- (8.79802,1.90572);
\definecolor{c}{rgb}{0,0,0};
\colorlet{c}{natgreen};
\draw [c] (8.84233,2.55842) -- (8.84233,2.82975);
\draw [c] (8.84233,2.82975) -- (8.84233,3.00548);
\draw [c] (8.79802,2.82975) -- (8.84233,2.82975);
\draw [c] (8.84233,2.82975) -- (8.88663,2.82975);
\definecolor{c}{rgb}{0,0,0};
\colorlet{c}{natgreen};
\draw [c] (8.93094,1.03156) -- (8.93094,2.07613);
\draw [c] (8.93094,2.07613) -- (8.93094,2.39391);
\draw [c] (8.88663,2.07613) -- (8.93094,2.07613);
\draw [c] (8.93094,2.07613) -- (8.97525,2.07613);
\definecolor{c}{rgb}{0,0,0};
\colorlet{c}{natgreen};
\draw [c] (9.01955,2.12561) -- (9.01955,2.53533);
\draw [c] (9.01955,2.53533) -- (9.01955,2.75882);
\draw [c] (8.97525,2.53533) -- (9.01955,2.53533);
\draw [c] (9.01955,2.53533) -- (9.06386,2.53533);
\definecolor{c}{rgb}{0,0,0};
\colorlet{c}{natgreen};
\draw [c] (9.10817,0.680516) -- (9.10817,1.8254);
\draw [c] (9.10817,1.8254) -- (9.10817,2.20548);
\draw [c] (9.06386,1.8254) -- (9.10817,1.8254);
\draw [c] (9.10817,1.8254) -- (9.15248,1.8254);
\definecolor{c}{rgb}{0,0,0};
\colorlet{c}{natgreen};
\draw [c] (9.19678,1.66756) -- (9.19678,2.28945);
\draw [c] (9.19678,2.28945) -- (9.19678,2.56029);
\draw [c] (9.15248,2.28945) -- (9.19678,2.28945);
\draw [c] (9.19678,2.28945) -- (9.24109,2.28945);
\definecolor{c}{rgb}{0,0,0};
\colorlet{c}{natgreen};
\draw [c] (9.46262,0.680516) -- (9.46262,1.93841);
\draw [c] (9.46262,1.93841) -- (9.46262,2.28945);
\draw [c] (9.41832,1.93841) -- (9.46262,1.93841);
\draw [c] (9.46262,1.93841) -- (9.50693,1.93841);
\definecolor{c}{rgb}{0,0,0};
\colorlet{c}{natgreen};
\draw [c] (9.90569,0.680516) -- (9.90569,1.38202);
\draw [c] (9.90569,1.38202) -- (9.90569,1.89102);
\draw [c] (9.86139,1.38202) -- (9.90569,1.38202);
\draw [c] (9.90569,1.38202) -- (9.95,1.38202);
\definecolor{c}{rgb}{0,0,0};
\draw [c] (1,0.680516) -- (9.95,0.680516);
\draw [anchor= east] (9.95,0.108883) node[color=c, rotate=0]{$M_{\gamma\gamma}\text{ [GeV]}$};
\draw [c] (1,0.863234) -- (1,0.680516);
\draw [c] (1.44307,0.771875) -- (1.44307,0.680516);
\draw [c] (1.88614,0.771875) -- (1.88614,0.680516);
\draw [c] (2.32921,0.771875) -- (2.32921,0.680516);
\draw [c] (2.77228,0.863234) -- (2.77228,0.680516);
\draw [c] (3.21535,0.771875) -- (3.21535,0.680516);
\draw [c] (3.65842,0.771875) -- (3.65842,0.680516);
\draw [c] (4.10149,0.771875) -- (4.10149,0.680516);
\draw [c] (4.54455,0.863234) -- (4.54455,0.680516);
\draw [c] (4.98762,0.771875) -- (4.98762,0.680516);
\draw [c] (5.43069,0.771875) -- (5.43069,0.680516);
\draw [c] (5.87376,0.771875) -- (5.87376,0.680516);
\draw [c] (6.31683,0.863234) -- (6.31683,0.680516);
\draw [c] (6.7599,0.771875) -- (6.7599,0.680516);
\draw [c] (7.20297,0.771875) -- (7.20297,0.680516);
\draw [c] (7.64604,0.771875) -- (7.64604,0.680516);
\draw [c] (8.08911,0.863234) -- (8.08911,0.680516);
\draw [c] (8.53218,0.771875) -- (8.53218,0.680516);
\draw [c] (8.97525,0.771875) -- (8.97525,0.680516);
\draw [c] (9.41832,0.771875) -- (9.41832,0.680516);
\draw [c] (9.86139,0.863234) -- (9.86139,0.680516);
\draw [c] (9.86139,0.863234) -- (9.86139,0.680516);
\draw [anchor=base] (1,0.353868) node[color=c, rotate=0]{0};
\draw [anchor=base] (2.77228,0.353868) node[color=c, rotate=0]{200};
\draw [anchor=base] (4.54455,0.353868) node[color=c, rotate=0]{400};
\draw [anchor=base] (6.31683,0.353868) node[color=c, rotate=0]{600};
\draw [anchor=base] (8.08911,0.353868) node[color=c, rotate=0]{800};
\draw [anchor=base] (9.86139,0.353868) node[color=c, rotate=0]{1000};
\draw [c] (1,0.680516) -- (1,6.73711);
\draw [anchor= east] (-0.12,6.73711) node[color=c, rotate=90]{Number of events};
\draw [c] (1.1335,0.718918) -- (1,0.718918);
\draw [c] (1.267,0.772277) -- (1,0.772277);
\draw [anchor= east] (0.922,0.772277) node[color=c, rotate=0]{$10^{-1}$};
\draw [c] (1.1335,1.12332) -- (1,1.12332);
\draw [c] (1.1335,1.32866) -- (1,1.32866);
\draw [c] (1.1335,1.47436) -- (1,1.47436);
\draw [c] (1.1335,1.58737) -- (1,1.58737);
\draw [c] (1.1335,1.6797) -- (1,1.6797);
\draw [c] (1.1335,1.75777) -- (1,1.75777);
\draw [c] (1.1335,1.8254) -- (1,1.8254);
\draw [c] (1.1335,1.88505) -- (1,1.88505);
\draw [c] (1.267,1.93841) -- (1,1.93841);
\draw [anchor= east] (0.922,1.93841) node[color=c, rotate=0]{1};
\draw [c] (1.1335,2.28945) -- (1,2.28945);
\draw [c] (1.1335,2.49479) -- (1,2.49479);
\draw [c] (1.1335,2.64049) -- (1,2.64049);
\draw [c] (1.1335,2.7535) -- (1,2.7535);
\draw [c] (1.1335,2.84583) -- (1,2.84583);
\draw [c] (1.1335,2.9239) -- (1,2.9239);
\draw [c] (1.1335,2.99153) -- (1,2.99153);
\draw [c] (1.1335,3.05118) -- (1,3.05118);
\draw [c] (1.267,3.10454) -- (1,3.10454);
\draw [anchor= east] (0.922,3.10454) node[color=c, rotate=0]{10};
\draw [c] (1.1335,3.45558) -- (1,3.45558);
\draw [c] (1.1335,3.66092) -- (1,3.66092);
\draw [c] (1.1335,3.80662) -- (1,3.80662);
\draw [c] (1.1335,3.91963) -- (1,3.91963);
\draw [c] (1.1335,4.01196) -- (1,4.01196);
\draw [c] (1.1335,4.09003) -- (1,4.09003);
\draw [c] (1.1335,4.15766) -- (1,4.15766);
\draw [c] (1.1335,4.21731) -- (1,4.21731);
\draw [c] (1.267,4.27067) -- (1,4.27067);
\draw [anchor= east] (0.922,4.27067) node[color=c, rotate=0]{$10^{2}$};
\draw [c] (1.1335,4.62171) -- (1,4.62171);
\draw [c] (1.1335,4.82705) -- (1,4.82705);
\draw [c] (1.1335,4.97275) -- (1,4.97275);
\draw [c] (1.1335,5.08576) -- (1,5.08576);
\draw [c] (1.1335,5.1781) -- (1,5.1781);
\draw [c] (1.1335,5.25616) -- (1,5.25616);
\draw [c] (1.1335,5.32379) -- (1,5.32379);
\draw [c] (1.1335,5.38344) -- (1,5.38344);
\draw [c] (1.267,5.4368) -- (1,5.4368);
\draw [anchor= east] (0.922,5.4368) node[color=c, rotate=0]{$10^{3}$};
\draw [c] (1.1335,5.78784) -- (1,5.78784);
\draw [c] (1.1335,5.99319) -- (1,5.99319);
\draw [c] (1.1335,6.13888) -- (1,6.13888);
\draw [c] (1.1335,6.25189) -- (1,6.25189);
\draw [c] (1.1335,6.34423) -- (1,6.34423);
\draw [c] (1.1335,6.42229) -- (1,6.42229);
\draw [c] (1.1335,6.48992) -- (1,6.48992);
\draw [c] (1.1335,6.54957) -- (1,6.54957);
\draw [c] (1.267,6.60293) -- (1,6.60293);
\draw [anchor= east] (0.922,6.60293) node[color=c, rotate=0]{$10^{4}$};
\colorlet{c}{natcomp!70};
\draw [c] (2.36687,6.34851) -- (2.44219,6.27507) -- (2.51751,6.20302) -- (2.59283,6.1323) -- (2.66816,6.06287) -- (2.74348,5.99466) -- (2.8188,5.92763) -- (2.89412,5.86173) -- (2.96944,5.7969) -- (3.04476,5.73311) -- (3.12009,5.6703)
 -- (3.19541,5.60845) -- (3.27073,5.5475) -- (3.34605,5.48743) -- (3.42137,5.42819) -- (3.4967,5.36977) -- (3.57202,5.31211) -- (3.64734,5.25521) -- (3.72266,5.19902) -- (3.79798,5.14352) -- (3.8733,5.08869) -- (3.94863,5.03451) -- (4.02395,4.98094)
 -- (4.09927,4.92798) -- (4.17459,4.87559) -- (4.24991,4.82377) -- (4.32524,4.77249) -- (4.40056,4.72173) -- (4.47588,4.67148) -- (4.5512,4.62172) -- (4.62652,4.57244) -- (4.70184,4.52363) -- (4.77717,4.47526) -- (4.85249,4.42733) --
 (4.92781,4.37982) -- (5.00313,4.33272) -- (5.07845,4.28602) -- (5.15377,4.23971) -- (5.2291,4.19378) -- (5.30442,4.14821) -- (5.37974,4.103) -- (5.45506,4.05814) -- (5.53038,4.01362) -- (5.60571,3.96943) -- (5.68103,3.92556) -- (5.75635,3.88201) --
 (5.83167,3.83876) -- (5.90699,3.79581) -- (5.98231,3.75315) -- (6.05764,3.71077);
\draw [c] (6.05764,3.71077) -- (6.13296,3.66868) -- (6.20828,3.62685) -- (6.2836,3.58529) -- (6.35892,3.54399) -- (6.43425,3.50294) -- (6.50957,3.46213) -- (6.58489,3.42157) -- (6.66021,3.38125) -- (6.73553,3.34115) --
 (6.81085,3.30129) -- (6.88618,3.26164) -- (6.9615,3.22221) -- (7.03682,3.18299) -- (7.11214,3.14398) -- (7.18746,3.10517) -- (7.26278,3.06656) -- (7.33811,3.02815) -- (7.41343,2.98993) -- (7.48875,2.95189) -- (7.56407,2.91404) -- (7.63939,2.87637)
 -- (7.71472,2.83887) -- (7.79004,2.80155) -- (7.86536,2.7644) -- (7.94068,2.72741) -- (8.016,2.69059) -- (8.09132,2.65392) -- (8.16665,2.61742) -- (8.24197,2.58107) -- (8.31729,2.54487) -- (8.39261,2.50882) -- (8.46793,2.47291) -- (8.54325,2.43715)
 -- (8.61858,2.40153) -- (8.6939,2.36605) -- (8.76922,2.3307) -- (8.84454,2.29549) -- (8.91986,2.26041) -- (8.99519,2.22545) -- (9.07051,2.19063) -- (9.14583,2.15593) -- (9.22115,2.12135) -- (9.29647,2.08689) -- (9.37179,2.05255) -- (9.44712,2.01832)
 -- (9.52244,1.98421) -- (9.59776,1.95021) -- (9.67308,1.91633) -- (9.7484,1.88255);
\draw [c] (9.7484,1.88255) -- (9.82373,1.84888);
\draw [c] (2.36687,6.34851) -- (2.44219,6.27507) -- (2.51751,6.20302) -- (2.59283,6.1323) -- (2.66816,6.06287) -- (2.74348,5.99466) -- (2.8188,5.92763) -- (2.89412,5.86173) -- (2.96944,5.7969) -- (3.04476,5.73311) -- (3.12009,5.6703)
 -- (3.19541,5.60845) -- (3.27073,5.5475) -- (3.34605,5.48743) -- (3.42137,5.42819) -- (3.4967,5.36977) -- (3.57202,5.31211) -- (3.64734,5.25521) -- (3.72266,5.19902) -- (3.79798,5.14352) -- (3.8733,5.08869) -- (3.94863,5.03451) -- (4.02395,4.98094)
 -- (4.09927,4.92798) -- (4.17459,4.87559) -- (4.24991,4.82377) -- (4.32524,4.77249) -- (4.40056,4.72173) -- (4.47588,4.67148) -- (4.5512,4.62172) -- (4.62652,4.57244) -- (4.70184,4.52363) -- (4.77717,4.47526) -- (4.85249,4.42733) --
 (4.92781,4.37982) -- (5.00313,4.33272) -- (5.07845,4.28602) -- (5.15377,4.23971) -- (5.2291,4.19378) -- (5.30442,4.14821) -- (5.37974,4.103) -- (5.45506,4.05814) -- (5.53038,4.01362) -- (5.60571,3.96943) -- (5.68103,3.92556) -- (5.75635,3.88201) --
 (5.83167,3.83876) -- (5.90699,3.79581) -- (5.98231,3.75315) -- (6.05764,3.71077);
\draw [c] (6.05764,3.71077) -- (6.13296,3.66868) -- (6.20828,3.62685) -- (6.2836,3.58529) -- (6.35892,3.54399) -- (6.43425,3.50294) -- (6.50957,3.46213) -- (6.58489,3.42157) -- (6.66021,3.38125) -- (6.73553,3.34115) --
 (6.81085,3.30129) -- (6.88618,3.26164) -- (6.9615,3.22221) -- (7.03682,3.18299) -- (7.11214,3.14398) -- (7.18746,3.10517) -- (7.26278,3.06656) -- (7.33811,3.02815) -- (7.41343,2.98993) -- (7.48875,2.95189) -- (7.56407,2.91404) -- (7.63939,2.87637)
 -- (7.71472,2.83887) -- (7.79004,2.80155) -- (7.86536,2.7644) -- (7.94068,2.72741) -- (8.016,2.69059) -- (8.09132,2.65392) -- (8.16665,2.61742) -- (8.24197,2.58107) -- (8.31729,2.54487) -- (8.39261,2.50882) -- (8.46793,2.47291) -- (8.54325,2.43715)
 -- (8.61858,2.40153) -- (8.6939,2.36605) -- (8.76922,2.3307) -- (8.84454,2.29549) -- (8.91986,2.26041) -- (8.99519,2.22545) -- (9.07051,2.19063) -- (9.14583,2.15593) -- (9.22115,2.12135) -- (9.29647,2.08689) -- (9.37179,2.05255) -- (9.44712,2.01832)
 -- (9.52244,1.98421) -- (9.59776,1.95021) -- (9.67308,1.91633) -- (9.7484,1.88255);
\draw [c] (9.7484,1.88255) -- (9.82373,1.84888);
\end{tikzpicture}

\end{infilsf}
\end{minipage}
\begin{minipage}[b]{.3\textwidth}
\caption{Extrapolating the truncated backgrounds by fitting a function.}\label{bckfit}
\end{minipage}
\end{figure}

\section{ Fit}
Recalling eq./eqref{rizzo}, we note that the contribution from the new term is proportional to a constant ($\Lambda^{-2}$) squared. Thus, the content of any bin must be expressible as a second order polynomial [...this should be in the theory chapter, or the sim chapter...]

[Something about the least likelihood method]

Because the fit method is very sensitive to the statistical fluctuations from bin to bin in our predicted invariant mass distributions, we opt to remove those fluctuations by fitting the distributions with a function. Since we take it as a given that the shape of the distribution consists of a base shape plus an excess from the contact interaction, we will fit them with a common power law, plus a gaussian plus an exponential for each of the non--SM distributions [I have a formula around here somewhere...] the result of the fit is illustrated in fig.~\ref{simfit}

\begin{figure}[hbt]
\begin{infilsf}\tiny
\begin{tikzpicture}[x=.092\textwidth,y=.092\textwidth]
\pgfdeclareplotmark{cross} {
\pgfpathmoveto{\pgfpoint{-0.3\pgfplotmarksize}{\pgfplotmarksize}}
\pgfpathlineto{\pgfpoint{+0.3\pgfplotmarksize}{\pgfplotmarksize}}
\pgfpathlineto{\pgfpoint{+0.3\pgfplotmarksize}{0.3\pgfplotmarksize}}
\pgfpathlineto{\pgfpoint{+1\pgfplotmarksize}{0.3\pgfplotmarksize}}
\pgfpathlineto{\pgfpoint{+1\pgfplotmarksize}{-0.3\pgfplotmarksize}}
\pgfpathlineto{\pgfpoint{+0.3\pgfplotmarksize}{-0.3\pgfplotmarksize}}
\pgfpathlineto{\pgfpoint{+0.3\pgfplotmarksize}{-1.\pgfplotmarksize}}
\pgfpathlineto{\pgfpoint{-0.3\pgfplotmarksize}{-1.\pgfplotmarksize}}
\pgfpathlineto{\pgfpoint{-0.3\pgfplotmarksize}{-0.3\pgfplotmarksize}}
\pgfpathlineto{\pgfpoint{-1.\pgfplotmarksize}{-0.3\pgfplotmarksize}}
\pgfpathlineto{\pgfpoint{-1.\pgfplotmarksize}{0.3\pgfplotmarksize}}
\pgfpathlineto{\pgfpoint{-0.3\pgfplotmarksize}{0.3\pgfplotmarksize}}
\pgfpathclose
\pgfusepathqstroke
}
\pgfdeclareplotmark{cross*} {
\pgfpathmoveto{\pgfpoint{-0.3\pgfplotmarksize}{\pgfplotmarksize}}
\pgfpathlineto{\pgfpoint{+0.3\pgfplotmarksize}{\pgfplotmarksize}}
\pgfpathlineto{\pgfpoint{+0.3\pgfplotmarksize}{0.3\pgfplotmarksize}}
\pgfpathlineto{\pgfpoint{+1\pgfplotmarksize}{0.3\pgfplotmarksize}}
\pgfpathlineto{\pgfpoint{+1\pgfplotmarksize}{-0.3\pgfplotmarksize}}
\pgfpathlineto{\pgfpoint{+0.3\pgfplotmarksize}{-0.3\pgfplotmarksize}}
\pgfpathlineto{\pgfpoint{+0.3\pgfplotmarksize}{-1.\pgfplotmarksize}}
\pgfpathlineto{\pgfpoint{-0.3\pgfplotmarksize}{-1.\pgfplotmarksize}}
\pgfpathlineto{\pgfpoint{-0.3\pgfplotmarksize}{-0.3\pgfplotmarksize}}
\pgfpathlineto{\pgfpoint{-1.\pgfplotmarksize}{-0.3\pgfplotmarksize}}
\pgfpathlineto{\pgfpoint{-1.\pgfplotmarksize}{0.3\pgfplotmarksize}}
\pgfpathlineto{\pgfpoint{-0.3\pgfplotmarksize}{0.3\pgfplotmarksize}}
\pgfpathclose
\pgfusepathqfillstroke
}
\pgfdeclareplotmark{newstar} {
\pgfpathmoveto{\pgfqpoint{0pt}{\pgfplotmarksize}}
\pgfpathlineto{\pgfqpointpolar{44}{0.5\pgfplotmarksize}}
\pgfpathlineto{\pgfqpointpolar{18}{\pgfplotmarksize}}
\pgfpathlineto{\pgfqpointpolar{-20}{0.5\pgfplotmarksize}}
\pgfpathlineto{\pgfqpointpolar{-54}{\pgfplotmarksize}}
\pgfpathlineto{\pgfqpointpolar{-90}{0.5\pgfplotmarksize}}
\pgfpathlineto{\pgfqpointpolar{234}{\pgfplotmarksize}}
\pgfpathlineto{\pgfqpointpolar{198}{0.5\pgfplotmarksize}}
\pgfpathlineto{\pgfqpointpolar{162}{\pgfplotmarksize}}
\pgfpathlineto{\pgfqpointpolar{134}{0.5\pgfplotmarksize}}
\pgfpathclose
\pgfusepathqstroke
}
\pgfdeclareplotmark{newstar*} {
\pgfpathmoveto{\pgfqpoint{0pt}{\pgfplotmarksize}}
\pgfpathlineto{\pgfqpointpolar{44}{0.5\pgfplotmarksize}}
\pgfpathlineto{\pgfqpointpolar{18}{\pgfplotmarksize}}
\pgfpathlineto{\pgfqpointpolar{-20}{0.5\pgfplotmarksize}}
\pgfpathlineto{\pgfqpointpolar{-54}{\pgfplotmarksize}}
\pgfpathlineto{\pgfqpointpolar{-90}{0.5\pgfplotmarksize}}
\pgfpathlineto{\pgfqpointpolar{234}{\pgfplotmarksize}}
\pgfpathlineto{\pgfqpointpolar{198}{0.5\pgfplotmarksize}}
\pgfpathlineto{\pgfqpointpolar{162}{\pgfplotmarksize}}
\pgfpathlineto{\pgfqpointpolar{134}{0.5\pgfplotmarksize}}
\pgfpathclose
\pgfusepathqfillstroke
}
\definecolor{c}{rgb}{1,1,1};
\draw [color=c, fill=c] (0,0) rectangle (10,5.96817);
\draw [color=c, fill=c] (1,0.596817) rectangle (9.95,5.90849);
\definecolor{c}{rgb}{0,0,0};
\draw [c] (1,0.596817) -- (1,5.90849) -- (9.95,5.90849) -- (9.95,0.596817) -- (1,0.596817);
\definecolor{c}{rgb}{1,1,1};
\draw [color=c, fill=c] (1,0.596817) rectangle (9.95,5.90849);
\definecolor{c}{rgb}{0,0,0};
\draw [c] (1,0.596817) -- (1,5.90849) -- (9.95,5.90849) -- (9.95,0.596817) -- (1,0.596817);
\colorlet{c}{kugray};
\draw [c] (1.13336,3.41516) -- (1.13336,3.80777);
\draw [c] (1.13336,3.80777) -- (1.13336,3.97485);
\draw [c] (1.11854,3.80777) -- (1.13336,3.80777);
\draw [c] (1.13336,3.80777) -- (1.14818,3.80777);
\definecolor{c}{rgb}{0,0,0};
\colorlet{c}{kugray};
\draw [c] (1.163,3.33018) -- (1.163,3.70859);
\draw [c] (1.163,3.70859) -- (1.163,3.87329);
\draw [c] (1.14818,3.70859) -- (1.163,3.70859);
\draw [c] (1.163,3.70859) -- (1.17781,3.70859);
\definecolor{c}{rgb}{0,0,0};
\colorlet{c}{kugray};
\draw [c] (1.22227,3.26026) -- (1.22227,3.80576);
\draw [c] (1.22227,3.80576) -- (1.22227,3.99181);
\draw [c] (1.20745,3.80576) -- (1.22227,3.80576);
\draw [c] (1.22227,3.80576) -- (1.23709,3.80576);
\definecolor{c}{rgb}{0,0,0};
\colorlet{c}{kugray};
\draw [c] (1.2519,3.59515) -- (1.2519,3.86294);
\draw [c] (1.2519,3.86294) -- (1.2519,4.00395);
\draw [c] (1.23709,3.86294) -- (1.2519,3.86294);
\draw [c] (1.2519,3.86294) -- (1.26672,3.86294);
\definecolor{c}{rgb}{0,0,0};
\colorlet{c}{kugray};
\draw [c] (1.28154,3.81131) -- (1.28154,4.05468);
\draw [c] (1.28154,4.05468) -- (1.28154,4.18889);
\draw [c] (1.26672,4.05468) -- (1.28154,4.05468);
\draw [c] (1.28154,4.05468) -- (1.29636,4.05468);
\definecolor{c}{rgb}{0,0,0};
\colorlet{c}{kugray};
\draw [c] (1.31118,5.36749) -- (1.31118,5.38512);
\draw [c] (1.31118,5.38512) -- (1.31118,5.40179);
\draw [c] (1.29636,5.38512) -- (1.31118,5.38512);
\draw [c] (1.31118,5.38512) -- (1.32599,5.38512);
\definecolor{c}{rgb}{0,0,0};
\colorlet{c}{kugray};
\draw [c] (1.34081,5.61993) -- (1.34081,5.63126);
\draw [c] (1.34081,5.63126) -- (1.34081,5.6422);
\draw [c] (1.32599,5.63126) -- (1.34081,5.63126);
\draw [c] (1.34081,5.63126) -- (1.35563,5.63126);
\definecolor{c}{rgb}{0,0,0};
\colorlet{c}{kugray};
\draw [c] (1.37045,5.69182) -- (1.37045,5.70194);
\draw [c] (1.37045,5.70194) -- (1.37045,5.71174);
\draw [c] (1.35563,5.70194) -- (1.37045,5.70194);
\draw [c] (1.37045,5.70194) -- (1.38526,5.70194);
\definecolor{c}{rgb}{0,0,0};
\colorlet{c}{kugray};
\draw [c] (1.40008,5.67982) -- (1.40008,5.69024);
\draw [c] (1.40008,5.69024) -- (1.40008,5.70031);
\draw [c] (1.38526,5.69024) -- (1.40008,5.69024);
\draw [c] (1.40008,5.69024) -- (1.4149,5.69024);
\definecolor{c}{rgb}{0,0,0};
\colorlet{c}{kugray};
\draw [c] (1.42972,5.65827) -- (1.42972,5.66928);
\draw [c] (1.42972,5.66928) -- (1.42972,5.67991);
\draw [c] (1.4149,5.66928) -- (1.42972,5.66928);
\draw [c] (1.42972,5.66928) -- (1.44454,5.66928);
\definecolor{c}{rgb}{0,0,0};
\colorlet{c}{kugray};
\draw [c] (1.45935,5.62) -- (1.45935,5.63141);
\draw [c] (1.45935,5.63141) -- (1.45935,5.6424);
\draw [c] (1.44454,5.63141) -- (1.45935,5.63141);
\draw [c] (1.45935,5.63141) -- (1.47417,5.63141);
\definecolor{c}{rgb}{0,0,0};
\colorlet{c}{kugray};
\draw [c] (1.48899,5.58164) -- (1.48899,5.5938);
\draw [c] (1.48899,5.5938) -- (1.48899,5.60549);
\draw [c] (1.47417,5.5938) -- (1.48899,5.5938);
\draw [c] (1.48899,5.5938) -- (1.50381,5.5938);
\definecolor{c}{rgb}{0,0,0};
\colorlet{c}{kugray};
\draw [c] (1.51863,5.53046) -- (1.51863,5.54392);
\draw [c] (1.51863,5.54392) -- (1.51863,5.55682);
\draw [c] (1.50381,5.54392) -- (1.51863,5.54392);
\draw [c] (1.51863,5.54392) -- (1.53344,5.54392);
\definecolor{c}{rgb}{0,0,0};
\colorlet{c}{kugray};
\draw [c] (1.54826,5.46022) -- (1.54826,5.47505);
\draw [c] (1.54826,5.47505) -- (1.54826,5.48921);
\draw [c] (1.53344,5.47505) -- (1.54826,5.47505);
\draw [c] (1.54826,5.47505) -- (1.56308,5.47505);
\definecolor{c}{rgb}{0,0,0};
\colorlet{c}{kugray};
\draw [c] (1.5779,5.41506) -- (1.5779,5.43093);
\draw [c] (1.5779,5.43093) -- (1.5779,5.44602);
\draw [c] (1.56308,5.43093) -- (1.5779,5.43093);
\draw [c] (1.5779,5.43093) -- (1.59272,5.43093);
\definecolor{c}{rgb}{0,0,0};
\colorlet{c}{kugray};
\draw [c] (1.60753,5.38705) -- (1.60753,5.40429);
\draw [c] (1.60753,5.40429) -- (1.60753,5.42061);
\draw [c] (1.59272,5.40429) -- (1.60753,5.40429);
\draw [c] (1.60753,5.40429) -- (1.62235,5.40429);
\definecolor{c}{rgb}{0,0,0};
\colorlet{c}{kugray};
\draw [c] (1.63717,5.33118) -- (1.63717,5.34945);
\draw [c] (1.63717,5.34945) -- (1.63717,5.36669);
\draw [c] (1.62235,5.34945) -- (1.63717,5.34945);
\draw [c] (1.63717,5.34945) -- (1.65199,5.34945);
\definecolor{c}{rgb}{0,0,0};
\colorlet{c}{kugray};
\draw [c] (1.6668,5.33501) -- (1.6668,5.35318);
\draw [c] (1.6668,5.35318) -- (1.6668,5.37033);
\draw [c] (1.65199,5.35318) -- (1.6668,5.35318);
\draw [c] (1.6668,5.35318) -- (1.68162,5.35318);
\definecolor{c}{rgb}{0,0,0};
\colorlet{c}{kugray};
\draw [c] (1.69644,5.27038) -- (1.69644,5.29099);
\draw [c] (1.69644,5.29099) -- (1.69644,5.3103);
\draw [c] (1.68162,5.29099) -- (1.69644,5.29099);
\draw [c] (1.69644,5.29099) -- (1.71126,5.29099);
\definecolor{c}{rgb}{0,0,0};
\colorlet{c}{kugray};
\draw [c] (1.72608,5.20804) -- (1.72608,5.23001);
\draw [c] (1.72608,5.23001) -- (1.72608,5.25052);
\draw [c] (1.71126,5.23001) -- (1.72608,5.23001);
\draw [c] (1.72608,5.23001) -- (1.74089,5.23001);
\definecolor{c}{rgb}{0,0,0};
\colorlet{c}{kugray};
\draw [c] (1.75571,5.08901) -- (1.75571,5.11549);
\draw [c] (1.75571,5.11549) -- (1.75571,5.13987);
\draw [c] (1.74089,5.11549) -- (1.75571,5.11549);
\draw [c] (1.75571,5.11549) -- (1.77053,5.11549);
\definecolor{c}{rgb}{0,0,0};
\colorlet{c}{kugray};
\draw [c] (1.78535,5.10181) -- (1.78535,5.12931);
\draw [c] (1.78535,5.12931) -- (1.78535,5.15455);
\draw [c] (1.77053,5.12931) -- (1.78535,5.12931);
\draw [c] (1.78535,5.12931) -- (1.80017,5.12931);
\definecolor{c}{rgb}{0,0,0};
\colorlet{c}{kugray};
\draw [c] (1.81498,5.06396) -- (1.81498,5.09313);
\draw [c] (1.81498,5.09313) -- (1.81498,5.11978);
\draw [c] (1.80017,5.09313) -- (1.81498,5.09313);
\draw [c] (1.81498,5.09313) -- (1.8298,5.09313);
\definecolor{c}{rgb}{0,0,0};
\colorlet{c}{kugray};
\draw [c] (1.84462,5.06537) -- (1.84462,5.09456);
\draw [c] (1.84462,5.09456) -- (1.84462,5.12123);
\draw [c] (1.8298,5.09456) -- (1.84462,5.09456);
\draw [c] (1.84462,5.09456) -- (1.85944,5.09456);
\definecolor{c}{rgb}{0,0,0};
\colorlet{c}{kugray};
\draw [c] (1.87425,4.99725) -- (1.87425,5.02835);
\draw [c] (1.87425,5.02835) -- (1.87425,5.05659);
\draw [c] (1.85944,5.02835) -- (1.87425,5.02835);
\draw [c] (1.87425,5.02835) -- (1.88907,5.02835);
\definecolor{c}{rgb}{0,0,0};
\colorlet{c}{kugray};
\draw [c] (1.90389,4.95832) -- (1.90389,4.99102);
\draw [c] (1.90389,4.99102) -- (1.90389,5.02058);
\draw [c] (1.88907,4.99102) -- (1.90389,4.99102);
\draw [c] (1.90389,4.99102) -- (1.91871,4.99102);
\definecolor{c}{rgb}{0,0,0};
\colorlet{c}{kugray};
\draw [c] (1.93353,4.85544) -- (1.93353,4.89383);
\draw [c] (1.93353,4.89383) -- (1.93353,4.92796);
\draw [c] (1.91871,4.89383) -- (1.93353,4.89383);
\draw [c] (1.93353,4.89383) -- (1.94834,4.89383);
\definecolor{c}{rgb}{0,0,0};
\colorlet{c}{kugray};
\draw [c] (1.96316,4.88333) -- (1.96316,4.92161);
\draw [c] (1.96316,4.92161) -- (1.96316,4.95565);
\draw [c] (1.94834,4.92161) -- (1.96316,4.92161);
\draw [c] (1.96316,4.92161) -- (1.97798,4.92161);
\definecolor{c}{rgb}{0,0,0};
\colorlet{c}{kugray};
\draw [c] (1.9928,4.85825) -- (1.9928,4.89733);
\draw [c] (1.9928,4.89733) -- (1.9928,4.93201);
\draw [c] (1.97798,4.89733) -- (1.9928,4.89733);
\draw [c] (1.9928,4.89733) -- (2.00762,4.89733);
\definecolor{c}{rgb}{0,0,0};
\colorlet{c}{kugray};
\draw [c] (2.02243,4.78026) -- (2.02243,4.82199);
\draw [c] (2.02243,4.82199) -- (2.02243,4.85874);
\draw [c] (2.00762,4.82199) -- (2.02243,4.82199);
\draw [c] (2.02243,4.82199) -- (2.03725,4.82199);
\definecolor{c}{rgb}{0,0,0};
\colorlet{c}{kugray};
\draw [c] (2.05207,4.74394) -- (2.05207,4.7911);
\draw [c] (2.05207,4.7911) -- (2.05207,4.83199);
\draw [c] (2.03725,4.7911) -- (2.05207,4.7911);
\draw [c] (2.05207,4.7911) -- (2.06689,4.7911);
\definecolor{c}{rgb}{0,0,0};
\colorlet{c}{kugray};
\draw [c] (2.08171,4.69059) -- (2.08171,4.74618);
\draw [c] (2.08171,4.74618) -- (2.08171,4.79324);
\draw [c] (2.06689,4.74618) -- (2.08171,4.74618);
\draw [c] (2.08171,4.74618) -- (2.09652,4.74618);
\definecolor{c}{rgb}{0,0,0};
\colorlet{c}{kugray};
\draw [c] (2.11134,4.64498) -- (2.11134,4.70439);
\draw [c] (2.11134,4.70439) -- (2.11134,4.75417);
\draw [c] (2.09652,4.70439) -- (2.11134,4.70439);
\draw [c] (2.11134,4.70439) -- (2.12616,4.70439);
\definecolor{c}{rgb}{0,0,0};
\colorlet{c}{kugray};
\draw [c] (2.14098,4.66916) -- (2.14098,4.726);
\draw [c] (2.14098,4.726) -- (2.14098,4.77396);
\draw [c] (2.12616,4.726) -- (2.14098,4.726);
\draw [c] (2.14098,4.726) -- (2.15579,4.726);
\definecolor{c}{rgb}{0,0,0};
\colorlet{c}{kugray};
\draw [c] (2.17061,4.71202) -- (2.17061,4.76221);
\draw [c] (2.17061,4.76221) -- (2.17061,4.80536);
\draw [c] (2.15579,4.76221) -- (2.17061,4.76221);
\draw [c] (2.17061,4.76221) -- (2.18543,4.76221);
\definecolor{c}{rgb}{0,0,0};
\colorlet{c}{kugray};
\draw [c] (2.20025,4.43289) -- (2.20025,4.50711);
\draw [c] (2.20025,4.50711) -- (2.20025,4.56686);
\draw [c] (2.18543,4.50711) -- (2.20025,4.50711);
\draw [c] (2.20025,4.50711) -- (2.21507,4.50711);
\definecolor{c}{rgb}{0,0,0};
\colorlet{c}{kugray};
\draw [c] (2.22988,4.60564) -- (2.22988,4.66562);
\draw [c] (2.22988,4.66562) -- (2.22988,4.7158);
\draw [c] (2.21507,4.66562) -- (2.22988,4.66562);
\draw [c] (2.22988,4.66562) -- (2.2447,4.66562);
\definecolor{c}{rgb}{0,0,0};
\colorlet{c}{kugray};
\draw [c] (2.25952,4.64159) -- (2.25952,4.70326);
\draw [c] (2.25952,4.70326) -- (2.25952,4.75461);
\draw [c] (2.2447,4.70326) -- (2.25952,4.70326);
\draw [c] (2.25952,4.70326) -- (2.27434,4.70326);
\definecolor{c}{rgb}{0,0,0};
\colorlet{c}{kugray};
\draw [c] (2.28916,4.56718) -- (2.28916,4.57808);
\draw [c] (2.28916,4.57808) -- (2.28916,4.5886);
\draw [c] (2.27434,4.57808) -- (2.28916,4.57808);
\draw [c] (2.28916,4.57808) -- (2.30397,4.57808);
\definecolor{c}{rgb}{0,0,0};
\colorlet{c}{kugray};
\draw [c] (2.31879,4.51802) -- (2.31879,4.5299);
\draw [c] (2.31879,4.5299) -- (2.31879,4.54134);
\draw [c] (2.30397,4.5299) -- (2.31879,4.5299);
\draw [c] (2.31879,4.5299) -- (2.33361,4.5299);
\definecolor{c}{rgb}{0,0,0};
\colorlet{c}{kugray};
\draw [c] (2.34843,4.51942) -- (2.34843,4.53112);
\draw [c] (2.34843,4.53112) -- (2.34843,4.54238);
\draw [c] (2.33361,4.53112) -- (2.34843,4.53112);
\draw [c] (2.34843,4.53112) -- (2.36325,4.53112);
\definecolor{c}{rgb}{0,0,0};
\colorlet{c}{kugray};
\draw [c] (2.37806,4.47939) -- (2.37806,4.492);
\draw [c] (2.37806,4.492) -- (2.37806,4.50411);
\draw [c] (2.36325,4.492) -- (2.37806,4.492);
\draw [c] (2.37806,4.492) -- (2.39288,4.492);
\definecolor{c}{rgb}{0,0,0};
\colorlet{c}{kugray};
\draw [c] (2.4077,4.44091) -- (2.4077,4.45439);
\draw [c] (2.4077,4.45439) -- (2.4077,4.46731);
\draw [c] (2.39288,4.45439) -- (2.4077,4.45439);
\draw [c] (2.4077,4.45439) -- (2.42252,4.45439);
\definecolor{c}{rgb}{0,0,0};
\colorlet{c}{kugray};
\draw [c] (2.43733,4.39585) -- (2.43733,4.41002);
\draw [c] (2.43733,4.41002) -- (2.43733,4.42357);
\draw [c] (2.42252,4.41002) -- (2.43733,4.41002);
\draw [c] (2.43733,4.41002) -- (2.45215,4.41002);
\definecolor{c}{rgb}{0,0,0};
\colorlet{c}{kugray};
\draw [c] (2.46697,4.36562) -- (2.46697,4.38084);
\draw [c] (2.46697,4.38084) -- (2.46697,4.39535);
\draw [c] (2.45215,4.38084) -- (2.46697,4.38084);
\draw [c] (2.46697,4.38084) -- (2.48179,4.38084);
\definecolor{c}{rgb}{0,0,0};
\colorlet{c}{kugray};
\draw [c] (2.49661,4.34442) -- (2.49661,4.3598);
\draw [c] (2.49661,4.3598) -- (2.49661,4.37444);
\draw [c] (2.48179,4.3598) -- (2.49661,4.3598);
\draw [c] (2.49661,4.3598) -- (2.51142,4.3598);
\definecolor{c}{rgb}{0,0,0};
\colorlet{c}{kugray};
\draw [c] (2.52624,4.33876) -- (2.52624,4.3547);
\draw [c] (2.52624,4.3547) -- (2.52624,4.36985);
\draw [c] (2.51142,4.3547) -- (2.52624,4.3547);
\draw [c] (2.52624,4.3547) -- (2.54106,4.3547);
\definecolor{c}{rgb}{0,0,0};
\colorlet{c}{kugray};
\draw [c] (2.55588,4.31164) -- (2.55588,4.32829);
\draw [c] (2.55588,4.32829) -- (2.55588,4.34409);
\draw [c] (2.54106,4.32829) -- (2.55588,4.32829);
\draw [c] (2.55588,4.32829) -- (2.5707,4.32829);
\definecolor{c}{rgb}{0,0,0};
\colorlet{c}{kugray};
\draw [c] (2.58551,4.27319) -- (2.58551,4.29119);
\draw [c] (2.58551,4.29119) -- (2.58551,4.30819);
\draw [c] (2.5707,4.29119) -- (2.58551,4.29119);
\draw [c] (2.58551,4.29119) -- (2.60033,4.29119);
\definecolor{c}{rgb}{0,0,0};
\colorlet{c}{kugray};
\draw [c] (2.61515,4.23142) -- (2.61515,4.25043);
\draw [c] (2.61515,4.25043) -- (2.61515,4.26834);
\draw [c] (2.60033,4.25043) -- (2.61515,4.25043);
\draw [c] (2.61515,4.25043) -- (2.62997,4.25043);
\definecolor{c}{rgb}{0,0,0};
\colorlet{c}{kugray};
\draw [c] (2.64478,4.1941) -- (2.64478,4.21371);
\draw [c] (2.64478,4.21371) -- (2.64478,4.23215);
\draw [c] (2.62997,4.21371) -- (2.64478,4.21371);
\draw [c] (2.64478,4.21371) -- (2.6596,4.21371);
\definecolor{c}{rgb}{0,0,0};
\colorlet{c}{kugray};
\draw [c] (2.67442,4.21383) -- (2.67442,4.23319);
\draw [c] (2.67442,4.23319) -- (2.67442,4.2514);
\draw [c] (2.6596,4.23319) -- (2.67442,4.23319);
\draw [c] (2.67442,4.23319) -- (2.68924,4.23319);
\definecolor{c}{rgb}{0,0,0};
\colorlet{c}{kugray};
\draw [c] (2.70406,4.15434) -- (2.70406,4.17536);
\draw [c] (2.70406,4.17536) -- (2.70406,4.19503);
\draw [c] (2.68924,4.17536) -- (2.70406,4.17536);
\draw [c] (2.70406,4.17536) -- (2.71887,4.17536);
\definecolor{c}{rgb}{0,0,0};
\colorlet{c}{kugray};
\draw [c] (2.73369,4.14693) -- (2.73369,4.16849);
\draw [c] (2.73369,4.16849) -- (2.73369,4.18863);
\draw [c] (2.71887,4.16849) -- (2.73369,4.16849);
\draw [c] (2.73369,4.16849) -- (2.74851,4.16849);
\definecolor{c}{rgb}{0,0,0};
\colorlet{c}{kugray};
\draw [c] (2.76333,4.12489) -- (2.76333,4.1477);
\draw [c] (2.76333,4.1477) -- (2.76333,4.16894);
\draw [c] (2.74851,4.1477) -- (2.76333,4.1477);
\draw [c] (2.76333,4.1477) -- (2.77815,4.1477);
\definecolor{c}{rgb}{0,0,0};
\colorlet{c}{kugray};
\draw [c] (2.79296,4.13183) -- (2.79296,4.15469);
\draw [c] (2.79296,4.15469) -- (2.79296,4.17597);
\draw [c] (2.77815,4.15469) -- (2.79296,4.15469);
\draw [c] (2.79296,4.15469) -- (2.80778,4.15469);
\definecolor{c}{rgb}{0,0,0};
\colorlet{c}{kugray};
\draw [c] (2.8226,4.05797) -- (2.8226,4.08216);
\draw [c] (2.8226,4.08216) -- (2.8226,4.10458);
\draw [c] (2.80778,4.08216) -- (2.8226,4.08216);
\draw [c] (2.8226,4.08216) -- (2.83742,4.08216);
\definecolor{c}{rgb}{0,0,0};
\colorlet{c}{kugray};
\draw [c] (2.85224,4.04617) -- (2.85224,4.07199);
\draw [c] (2.85224,4.07199) -- (2.85224,4.09581);
\draw [c] (2.83742,4.07199) -- (2.85224,4.07199);
\draw [c] (2.85224,4.07199) -- (2.86705,4.07199);
\definecolor{c}{rgb}{0,0,0};
\colorlet{c}{kugray};
\draw [c] (2.88187,4.02533) -- (2.88187,4.05093);
\draw [c] (2.88187,4.05093) -- (2.88187,4.07457);
\draw [c] (2.86705,4.05093) -- (2.88187,4.05093);
\draw [c] (2.88187,4.05093) -- (2.89669,4.05093);
\definecolor{c}{rgb}{0,0,0};
\colorlet{c}{kugray};
\draw [c] (2.91151,4.0461) -- (2.91151,4.07164);
\draw [c] (2.91151,4.07164) -- (2.91151,4.09522);
\draw [c] (2.89669,4.07164) -- (2.91151,4.07164);
\draw [c] (2.91151,4.07164) -- (2.92632,4.07164);
\definecolor{c}{rgb}{0,0,0};
\colorlet{c}{kugray};
\draw [c] (2.94114,3.94743) -- (2.94114,3.97699);
\draw [c] (2.94114,3.97699) -- (2.94114,4.00396);
\draw [c] (2.92632,3.97699) -- (2.94114,3.97699);
\draw [c] (2.94114,3.97699) -- (2.95596,3.97699);
\definecolor{c}{rgb}{0,0,0};
\colorlet{c}{kugray};
\draw [c] (2.97078,3.92033) -- (2.97078,3.95122);
\draw [c] (2.97078,3.95122) -- (2.97078,3.97929);
\draw [c] (2.95596,3.95122) -- (2.97078,3.95122);
\draw [c] (2.97078,3.95122) -- (2.9856,3.95122);
\definecolor{c}{rgb}{0,0,0};
\colorlet{c}{kugray};
\draw [c] (3.00041,3.95455) -- (3.00041,3.98496);
\draw [c] (3.00041,3.98496) -- (3.00041,4.01263);
\draw [c] (2.9856,3.98496) -- (3.00041,3.98496);
\draw [c] (3.00041,3.98496) -- (3.01523,3.98496);
\definecolor{c}{rgb}{0,0,0};
\colorlet{c}{kugray};
\draw [c] (3.03005,3.91612) -- (3.03005,3.94874);
\draw [c] (3.03005,3.94874) -- (3.03005,3.97824);
\draw [c] (3.01523,3.94874) -- (3.03005,3.94874);
\draw [c] (3.03005,3.94874) -- (3.04487,3.94874);
\definecolor{c}{rgb}{0,0,0};
\colorlet{c}{kugray};
\draw [c] (3.05969,3.91532) -- (3.05969,3.94707);
\draw [c] (3.05969,3.94707) -- (3.05969,3.97585);
\draw [c] (3.04487,3.94707) -- (3.05969,3.94707);
\draw [c] (3.05969,3.94707) -- (3.0745,3.94707);
\definecolor{c}{rgb}{0,0,0};
\colorlet{c}{kugray};
\draw [c] (3.08932,3.82813) -- (3.08932,3.8633);
\draw [c] (3.08932,3.8633) -- (3.08932,3.89486);
\draw [c] (3.0745,3.8633) -- (3.08932,3.8633);
\draw [c] (3.08932,3.8633) -- (3.10414,3.8633);
\definecolor{c}{rgb}{0,0,0};
\colorlet{c}{kugray};
\draw [c] (3.11896,3.82152) -- (3.11896,3.85725);
\draw [c] (3.11896,3.85725) -- (3.11896,3.88926);
\draw [c] (3.10414,3.85725) -- (3.11896,3.85725);
\draw [c] (3.11896,3.85725) -- (3.13377,3.85725);
\definecolor{c}{rgb}{0,0,0};
\colorlet{c}{kugray};
\draw [c] (3.14859,3.81099) -- (3.14859,3.849);
\draw [c] (3.14859,3.849) -- (3.14859,3.88283);
\draw [c] (3.13377,3.849) -- (3.14859,3.849);
\draw [c] (3.14859,3.849) -- (3.16341,3.849);
\definecolor{c}{rgb}{0,0,0};
\colorlet{c}{kugray};
\draw [c] (3.17823,3.68505) -- (3.17823,3.72686);
\draw [c] (3.17823,3.72686) -- (3.17823,3.76367);
\draw [c] (3.16341,3.72686) -- (3.17823,3.72686);
\draw [c] (3.17823,3.72686) -- (3.19305,3.72686);
\definecolor{c}{rgb}{0,0,0};
\colorlet{c}{kugray};
\draw [c] (3.20786,3.75027) -- (3.20786,3.79207);
\draw [c] (3.20786,3.79207) -- (3.20786,3.82886);
\draw [c] (3.19305,3.79207) -- (3.20786,3.79207);
\draw [c] (3.20786,3.79207) -- (3.22268,3.79207);
\definecolor{c}{rgb}{0,0,0};
\colorlet{c}{kugray};
\draw [c] (3.2375,3.78399) -- (3.2375,3.82558);
\draw [c] (3.2375,3.82558) -- (3.2375,3.86222);
\draw [c] (3.22268,3.82558) -- (3.2375,3.82558);
\draw [c] (3.2375,3.82558) -- (3.25232,3.82558);
\definecolor{c}{rgb}{0,0,0};
\colorlet{c}{kugray};
\draw [c] (3.26714,3.71561) -- (3.26714,3.76154);
\draw [c] (3.26714,3.76154) -- (3.26714,3.80149);
\draw [c] (3.25232,3.76154) -- (3.26714,3.76154);
\draw [c] (3.26714,3.76154) -- (3.28195,3.76154);
\definecolor{c}{rgb}{0,0,0};
\colorlet{c}{kugray};
\draw [c] (3.29677,3.74464) -- (3.29677,3.78592);
\draw [c] (3.29677,3.78592) -- (3.29677,3.82231);
\draw [c] (3.28195,3.78592) -- (3.29677,3.78592);
\draw [c] (3.29677,3.78592) -- (3.31159,3.78592);
\definecolor{c}{rgb}{0,0,0};
\colorlet{c}{kugray};
\draw [c] (3.32641,3.60212) -- (3.32641,3.65046);
\draw [c] (3.32641,3.65046) -- (3.32641,3.69223);
\draw [c] (3.31159,3.65046) -- (3.32641,3.65046);
\draw [c] (3.32641,3.65046) -- (3.34123,3.65046);
\definecolor{c}{rgb}{0,0,0};
\colorlet{c}{kugray};
\draw [c] (3.35604,3.66614) -- (3.35604,3.71219);
\draw [c] (3.35604,3.71219) -- (3.35604,3.75225);
\draw [c] (3.34123,3.71219) -- (3.35604,3.71219);
\draw [c] (3.35604,3.71219) -- (3.37086,3.71219);
\definecolor{c}{rgb}{0,0,0};
\colorlet{c}{kugray};
\draw [c] (3.38568,3.70789) -- (3.38568,3.7536);
\draw [c] (3.38568,3.7536) -- (3.38568,3.7934);
\draw [c] (3.37086,3.7536) -- (3.38568,3.7536);
\draw [c] (3.38568,3.7536) -- (3.4005,3.7536);
\definecolor{c}{rgb}{0,0,0};
\colorlet{c}{kugray};
\draw [c] (3.41531,3.59749) -- (3.41531,3.64919);
\draw [c] (3.41531,3.64919) -- (3.41531,3.69343);
\draw [c] (3.4005,3.64919) -- (3.41531,3.64919);
\draw [c] (3.41531,3.64919) -- (3.43013,3.64919);
\definecolor{c}{rgb}{0,0,0};
\colorlet{c}{kugray};
\draw [c] (3.44495,3.54922) -- (3.44495,3.6056);
\draw [c] (3.44495,3.6056) -- (3.44495,3.65323);
\draw [c] (3.43013,3.6056) -- (3.44495,3.6056);
\draw [c] (3.44495,3.6056) -- (3.45977,3.6056);
\definecolor{c}{rgb}{0,0,0};
\colorlet{c}{kugray};
\draw [c] (3.47459,3.5315) -- (3.47459,3.58602);
\draw [c] (3.47459,3.58602) -- (3.47459,3.63231);
\draw [c] (3.45977,3.58602) -- (3.47459,3.58602);
\draw [c] (3.47459,3.58602) -- (3.4894,3.58602);
\definecolor{c}{rgb}{0,0,0};
\colorlet{c}{kugray};
\draw [c] (3.50422,3.44087) -- (3.50422,3.50756);
\draw [c] (3.50422,3.50756) -- (3.50422,3.56234);
\draw [c] (3.4894,3.50756) -- (3.50422,3.50756);
\draw [c] (3.50422,3.50756) -- (3.51904,3.50756);
\definecolor{c}{rgb}{0,0,0};
\colorlet{c}{kugray};
\draw [c] (3.53386,3.57716) -- (3.53386,3.62954);
\draw [c] (3.53386,3.62954) -- (3.53386,3.67429);
\draw [c] (3.51904,3.62954) -- (3.53386,3.62954);
\draw [c] (3.53386,3.62954) -- (3.54868,3.62954);
\definecolor{c}{rgb}{0,0,0};
\colorlet{c}{kugray};
\draw [c] (3.56349,3.59791) -- (3.56349,3.64909);
\draw [c] (3.56349,3.64909) -- (3.56349,3.69297);
\draw [c] (3.54868,3.64909) -- (3.56349,3.64909);
\draw [c] (3.56349,3.64909) -- (3.57831,3.64909);
\definecolor{c}{rgb}{0,0,0};
\colorlet{c}{kugray};
\draw [c] (3.59313,3.48016) -- (3.59313,3.54299);
\draw [c] (3.59313,3.54299) -- (3.59313,3.59515);
\draw [c] (3.57831,3.54299) -- (3.59313,3.54299);
\draw [c] (3.59313,3.54299) -- (3.60795,3.54299);
\definecolor{c}{rgb}{0,0,0};
\colorlet{c}{kugray};
\draw [c] (3.62276,3.49926) -- (3.62276,3.5653);
\draw [c] (3.62276,3.5653) -- (3.62276,3.61965);
\draw [c] (3.60795,3.5653) -- (3.62276,3.5653);
\draw [c] (3.62276,3.5653) -- (3.63758,3.5653);
\definecolor{c}{rgb}{0,0,0};
\colorlet{c}{kugray};
\draw [c] (3.6524,3.38318) -- (3.6524,3.45701);
\draw [c] (3.6524,3.45701) -- (3.6524,3.51651);
\draw [c] (3.63758,3.45701) -- (3.6524,3.45701);
\draw [c] (3.6524,3.45701) -- (3.66722,3.45701);
\definecolor{c}{rgb}{0,0,0};
\colorlet{c}{kugray};
\draw [c] (3.68204,3.38716) -- (3.68204,3.45825);
\draw [c] (3.68204,3.45825) -- (3.68204,3.51597);
\draw [c] (3.66722,3.45825) -- (3.68204,3.45825);
\draw [c] (3.68204,3.45825) -- (3.69685,3.45825);
\definecolor{c}{rgb}{0,0,0};
\colorlet{c}{kugray};
\draw [c] (3.71167,3.27025) -- (3.71167,3.35157);
\draw [c] (3.71167,3.35157) -- (3.71167,3.41584);
\draw [c] (3.69685,3.35157) -- (3.71167,3.35157);
\draw [c] (3.71167,3.35157) -- (3.72649,3.35157);
\definecolor{c}{rgb}{0,0,0};
\colorlet{c}{kugray};
\draw [c] (3.74131,3.42793) -- (3.74131,3.49218);
\draw [c] (3.74131,3.49218) -- (3.74131,3.54531);
\draw [c] (3.72649,3.49218) -- (3.74131,3.49218);
\draw [c] (3.74131,3.49218) -- (3.75613,3.49218);
\definecolor{c}{rgb}{0,0,0};
\colorlet{c}{kugray};
\draw [c] (3.77094,3.39681) -- (3.77094,3.4655);
\draw [c] (3.77094,3.4655) -- (3.77094,3.52163);
\draw [c] (3.75613,3.4655) -- (3.77094,3.4655);
\draw [c] (3.77094,3.4655) -- (3.78576,3.4655);
\definecolor{c}{rgb}{0,0,0};
\colorlet{c}{kugray};
\draw [c] (3.80058,3.30807) -- (3.80058,3.39026);
\draw [c] (3.80058,3.39026) -- (3.80058,3.45506);
\draw [c] (3.78576,3.39026) -- (3.80058,3.39026);
\draw [c] (3.80058,3.39026) -- (3.8154,3.39026);
\definecolor{c}{rgb}{0,0,0};
\colorlet{c}{kugray};
\draw [c] (3.83022,3.36833) -- (3.83022,3.44242);
\draw [c] (3.83022,3.44242) -- (3.83022,3.50209);
\draw [c] (3.8154,3.44242) -- (3.83022,3.44242);
\draw [c] (3.83022,3.44242) -- (3.84503,3.44242);
\definecolor{c}{rgb}{0,0,0};
\colorlet{c}{kugray};
\draw [c] (3.85985,3.36697) -- (3.85985,3.43885);
\draw [c] (3.85985,3.43885) -- (3.85985,3.49708);
\draw [c] (3.84503,3.43885) -- (3.85985,3.43885);
\draw [c] (3.85985,3.43885) -- (3.87467,3.43885);
\definecolor{c}{rgb}{0,0,0};
\colorlet{c}{kugray};
\draw [c] (3.88949,3.18339) -- (3.88949,3.27641);
\draw [c] (3.88949,3.27641) -- (3.88949,3.34775);
\draw [c] (3.87467,3.27641) -- (3.88949,3.27641);
\draw [c] (3.88949,3.27641) -- (3.9043,3.27641);
\definecolor{c}{rgb}{0,0,0};
\colorlet{c}{kugray};
\draw [c] (3.91912,3.36859) -- (3.91912,3.433);
\draw [c] (3.91912,3.433) -- (3.91912,3.48624);
\draw [c] (3.9043,3.433) -- (3.91912,3.433);
\draw [c] (3.91912,3.433) -- (3.93394,3.433);
\definecolor{c}{rgb}{0,0,0};
\colorlet{c}{kugray};
\draw [c] (3.94876,3.27957) -- (3.94876,3.34398);
\draw [c] (3.94876,3.34398) -- (3.94876,3.39722);
\draw [c] (3.93394,3.34398) -- (3.94876,3.34398);
\draw [c] (3.94876,3.34398) -- (3.96358,3.34398);
\definecolor{c}{rgb}{0,0,0};
\colorlet{c}{kugray};
\draw [c] (3.97839,3.23181) -- (3.97839,3.2842);
\draw [c] (3.97839,3.2842) -- (3.97839,3.32896);
\draw [c] (3.96358,3.2842) -- (3.97839,3.2842);
\draw [c] (3.97839,3.2842) -- (3.99321,3.2842);
\definecolor{c}{rgb}{0,0,0};
\colorlet{c}{kugray};
\draw [c] (4.00803,3.23298) -- (4.00803,3.24514);
\draw [c] (4.00803,3.24514) -- (4.00803,3.25684);
\draw [c] (3.99321,3.24514) -- (4.00803,3.24514);
\draw [c] (4.00803,3.24514) -- (4.02285,3.24514);
\definecolor{c}{rgb}{0,0,0};
\colorlet{c}{kugray};
\draw [c] (4.03767,3.23899) -- (4.03767,3.25079);
\draw [c] (4.03767,3.25079) -- (4.03767,3.26215);
\draw [c] (4.02285,3.25079) -- (4.03767,3.25079);
\draw [c] (4.03767,3.25079) -- (4.05248,3.25079);
\definecolor{c}{rgb}{0,0,0};
\colorlet{c}{kugray};
\draw [c] (4.0673,3.24063) -- (4.0673,3.25269);
\draw [c] (4.0673,3.25269) -- (4.0673,3.26429);
\draw [c] (4.05248,3.25269) -- (4.0673,3.25269);
\draw [c] (4.0673,3.25269) -- (4.08212,3.25269);
\definecolor{c}{rgb}{0,0,0};
\colorlet{c}{kugray};
\draw [c] (4.09694,3.22722) -- (4.09694,3.23915);
\draw [c] (4.09694,3.23915) -- (4.09694,3.25063);
\draw [c] (4.08212,3.23915) -- (4.09694,3.23915);
\draw [c] (4.09694,3.23915) -- (4.11175,3.23915);
\definecolor{c}{rgb}{0,0,0};
\colorlet{c}{kugray};
\draw [c] (4.12657,3.2116) -- (4.12657,3.22405);
\draw [c] (4.12657,3.22405) -- (4.12657,3.23601);
\draw [c] (4.11175,3.22405) -- (4.12657,3.22405);
\draw [c] (4.12657,3.22405) -- (4.14139,3.22405);
\definecolor{c}{rgb}{0,0,0};
\colorlet{c}{kugray};
\draw [c] (4.15621,3.18215) -- (4.15621,3.19496);
\draw [c] (4.15621,3.19496) -- (4.15621,3.20726);
\draw [c] (4.14139,3.19496) -- (4.15621,3.19496);
\draw [c] (4.15621,3.19496) -- (4.17103,3.19496);
\definecolor{c}{rgb}{0,0,0};
\colorlet{c}{kugray};
\draw [c] (4.18584,3.18206) -- (4.18584,3.19521);
\draw [c] (4.18584,3.19521) -- (4.18584,3.20782);
\draw [c] (4.17103,3.19521) -- (4.18584,3.19521);
\draw [c] (4.18584,3.19521) -- (4.20066,3.19521);
\definecolor{c}{rgb}{0,0,0};
\colorlet{c}{kugray};
\draw [c] (4.21548,3.14682) -- (4.21548,3.16086);
\draw [c] (4.21548,3.16086) -- (4.21548,3.17428);
\draw [c] (4.20066,3.16086) -- (4.21548,3.16086);
\draw [c] (4.21548,3.16086) -- (4.2303,3.16086);
\definecolor{c}{rgb}{0,0,0};
\colorlet{c}{kugray};
\draw [c] (4.24512,3.16809) -- (4.24512,3.1818);
\draw [c] (4.24512,3.1818) -- (4.24512,3.19493);
\draw [c] (4.2303,3.1818) -- (4.24512,3.1818);
\draw [c] (4.24512,3.1818) -- (4.25993,3.1818);
\definecolor{c}{rgb}{0,0,0};
\colorlet{c}{kugray};
\draw [c] (4.27475,3.10937) -- (4.27475,3.12421);
\draw [c] (4.27475,3.12421) -- (4.27475,3.13837);
\draw [c] (4.25993,3.12421) -- (4.27475,3.12421);
\draw [c] (4.27475,3.12421) -- (4.28957,3.12421);
\definecolor{c}{rgb}{0,0,0};
\colorlet{c}{kugray};
\draw [c] (4.30439,3.11556) -- (4.30439,3.12992);
\draw [c] (4.30439,3.12992) -- (4.30439,3.14364);
\draw [c] (4.28957,3.12992) -- (4.30439,3.12992);
\draw [c] (4.30439,3.12992) -- (4.31921,3.12992);
\definecolor{c}{rgb}{0,0,0};
\colorlet{c}{kugray};
\draw [c] (4.33402,3.13544) -- (4.33402,3.14949);
\draw [c] (4.33402,3.14949) -- (4.33402,3.16293);
\draw [c] (4.31921,3.14949) -- (4.33402,3.14949);
\draw [c] (4.33402,3.14949) -- (4.34884,3.14949);
\definecolor{c}{rgb}{0,0,0};
\colorlet{c}{kugray};
\draw [c] (4.36366,3.09108) -- (4.36366,3.10621);
\draw [c] (4.36366,3.10621) -- (4.36366,3.12063);
\draw [c] (4.34884,3.10621) -- (4.36366,3.10621);
\draw [c] (4.36366,3.10621) -- (4.37848,3.10621);
\definecolor{c}{rgb}{0,0,0};
\colorlet{c}{kugray};
\draw [c] (4.39329,3.08833) -- (4.39329,3.10339);
\draw [c] (4.39329,3.10339) -- (4.39329,3.11775);
\draw [c] (4.37848,3.10339) -- (4.39329,3.10339);
\draw [c] (4.39329,3.10339) -- (4.40811,3.10339);
\definecolor{c}{rgb}{0,0,0};
\colorlet{c}{kugray};
\draw [c] (4.42293,3.06637) -- (4.42293,3.08234);
\draw [c] (4.42293,3.08234) -- (4.42293,3.09751);
\draw [c] (4.40811,3.08234) -- (4.42293,3.08234);
\draw [c] (4.42293,3.08234) -- (4.43775,3.08234);
\definecolor{c}{rgb}{0,0,0};
\colorlet{c}{kugray};
\draw [c] (4.45257,3.03685) -- (4.45257,3.05337);
\draw [c] (4.45257,3.05337) -- (4.45257,3.06905);
\draw [c] (4.43775,3.05337) -- (4.45257,3.05337);
\draw [c] (4.45257,3.05337) -- (4.46738,3.05337);
\definecolor{c}{rgb}{0,0,0};
\colorlet{c}{kugray};
\draw [c] (4.4822,3.05403) -- (4.4822,3.07028);
\draw [c] (4.4822,3.07028) -- (4.4822,3.08572);
\draw [c] (4.46738,3.07028) -- (4.4822,3.07028);
\draw [c] (4.4822,3.07028) -- (4.49702,3.07028);
\definecolor{c}{rgb}{0,0,0};
\colorlet{c}{kugray};
\draw [c] (4.51184,3.02475) -- (4.51184,3.04166);
\draw [c] (4.51184,3.04166) -- (4.51184,3.05768);
\draw [c] (4.49702,3.04166) -- (4.51184,3.04166);
\draw [c] (4.51184,3.04166) -- (4.52666,3.04166);
\definecolor{c}{rgb}{0,0,0};
\colorlet{c}{kugray};
\draw [c] (4.54147,2.97429) -- (4.54147,2.99215);
\draw [c] (4.54147,2.99215) -- (4.54147,3.00903);
\draw [c] (4.52666,2.99215) -- (4.54147,2.99215);
\draw [c] (4.54147,2.99215) -- (4.55629,2.99215);
\definecolor{c}{rgb}{0,0,0};
\colorlet{c}{kugray};
\draw [c] (4.57111,2.97145) -- (4.57111,2.9901);
\draw [c] (4.57111,2.9901) -- (4.57111,3.0077);
\draw [c] (4.55629,2.9901) -- (4.57111,2.9901);
\draw [c] (4.57111,2.9901) -- (4.58593,2.9901);
\definecolor{c}{rgb}{0,0,0};
\colorlet{c}{kugray};
\draw [c] (4.60075,2.99637) -- (4.60075,3.01434);
\draw [c] (4.60075,3.01434) -- (4.60075,3.03131);
\draw [c] (4.58593,3.01434) -- (4.60075,3.01434);
\draw [c] (4.60075,3.01434) -- (4.61556,3.01434);
\definecolor{c}{rgb}{0,0,0};
\colorlet{c}{kugray};
\draw [c] (4.63038,2.94705) -- (4.63038,2.96575);
\draw [c] (4.63038,2.96575) -- (4.63038,2.98338);
\draw [c] (4.61556,2.96575) -- (4.63038,2.96575);
\draw [c] (4.63038,2.96575) -- (4.6452,2.96575);
\definecolor{c}{rgb}{0,0,0};
\colorlet{c}{kugray};
\draw [c] (4.66002,2.94213) -- (4.66002,2.96119);
\draw [c] (4.66002,2.96119) -- (4.66002,2.97915);
\draw [c] (4.6452,2.96119) -- (4.66002,2.96119);
\draw [c] (4.66002,2.96119) -- (4.67483,2.96119);
\definecolor{c}{rgb}{0,0,0};
\colorlet{c}{kugray};
\draw [c] (4.68965,2.92699) -- (4.68965,2.94611);
\draw [c] (4.68965,2.94611) -- (4.68965,2.96412);
\draw [c] (4.67483,2.94611) -- (4.68965,2.94611);
\draw [c] (4.68965,2.94611) -- (4.70447,2.94611);
\definecolor{c}{rgb}{0,0,0};
\colorlet{c}{kugray};
\draw [c] (4.71929,2.90533) -- (4.71929,2.92537);
\draw [c] (4.71929,2.92537) -- (4.71929,2.94418);
\draw [c] (4.70447,2.92537) -- (4.71929,2.92537);
\draw [c] (4.71929,2.92537) -- (4.73411,2.92537);
\definecolor{c}{rgb}{0,0,0};
\colorlet{c}{kugray};
\draw [c] (4.74892,2.89803) -- (4.74892,2.9188);
\draw [c] (4.74892,2.9188) -- (4.74892,2.93826);
\draw [c] (4.73411,2.9188) -- (4.74892,2.9188);
\draw [c] (4.74892,2.9188) -- (4.76374,2.9188);
\definecolor{c}{rgb}{0,0,0};
\colorlet{c}{kugray};
\draw [c] (4.77856,2.88491) -- (4.77856,2.90571);
\draw [c] (4.77856,2.90571) -- (4.77856,2.92519);
\draw [c] (4.76374,2.90571) -- (4.77856,2.90571);
\draw [c] (4.77856,2.90571) -- (4.79338,2.90571);
\definecolor{c}{rgb}{0,0,0};
\colorlet{c}{kugray};
\draw [c] (4.8082,2.83295) -- (4.8082,2.85717);
\draw [c] (4.8082,2.85717) -- (4.8082,2.87962);
\draw [c] (4.79338,2.85717) -- (4.8082,2.85717);
\draw [c] (4.8082,2.85717) -- (4.82301,2.85717);
\definecolor{c}{rgb}{0,0,0};
\colorlet{c}{kugray};
\draw [c] (4.83783,2.84099) -- (4.83783,2.86366);
\draw [c] (4.83783,2.86366) -- (4.83783,2.88478);
\draw [c] (4.82301,2.86366) -- (4.83783,2.86366);
\draw [c] (4.83783,2.86366) -- (4.85265,2.86366);
\definecolor{c}{rgb}{0,0,0};
\colorlet{c}{kugray};
\draw [c] (4.86747,2.84516) -- (4.86747,2.86807);
\draw [c] (4.86747,2.86807) -- (4.86747,2.88939);
\draw [c] (4.85265,2.86807) -- (4.86747,2.86807);
\draw [c] (4.86747,2.86807) -- (4.88228,2.86807);
\definecolor{c}{rgb}{0,0,0};
\colorlet{c}{kugray};
\draw [c] (4.8971,2.82427) -- (4.8971,2.84804);
\draw [c] (4.8971,2.84804) -- (4.8971,2.87011);
\draw [c] (4.88228,2.84804) -- (4.8971,2.84804);
\draw [c] (4.8971,2.84804) -- (4.91192,2.84804);
\definecolor{c}{rgb}{0,0,0};
\colorlet{c}{kugray};
\draw [c] (4.92674,2.7958) -- (4.92674,2.82013);
\draw [c] (4.92674,2.82013) -- (4.92674,2.84268);
\draw [c] (4.91192,2.82013) -- (4.92674,2.82013);
\draw [c] (4.92674,2.82013) -- (4.94156,2.82013);
\definecolor{c}{rgb}{0,0,0};
\colorlet{c}{kugray};
\draw [c] (4.95637,2.72162) -- (4.95637,2.74908);
\draw [c] (4.95637,2.74908) -- (4.95637,2.7743);
\draw [c] (4.94156,2.74908) -- (4.95637,2.74908);
\draw [c] (4.95637,2.74908) -- (4.97119,2.74908);
\definecolor{c}{rgb}{0,0,0};
\colorlet{c}{kugray};
\draw [c] (4.98601,2.74467) -- (4.98601,2.77317);
\draw [c] (4.98601,2.77317) -- (4.98601,2.79924);
\draw [c] (4.97119,2.77317) -- (4.98601,2.77317);
\draw [c] (4.98601,2.77317) -- (5.00083,2.77317);
\definecolor{c}{rgb}{0,0,0};
\colorlet{c}{kugray};
\draw [c] (5.01565,2.72133) -- (5.01565,2.74919);
\draw [c] (5.01565,2.74919) -- (5.01565,2.77473);
\draw [c] (5.00083,2.74919) -- (5.01565,2.74919);
\draw [c] (5.01565,2.74919) -- (5.03046,2.74919);
\definecolor{c}{rgb}{0,0,0};
\colorlet{c}{kugray};
\draw [c] (5.04528,2.68622) -- (5.04528,2.71565);
\draw [c] (5.04528,2.71565) -- (5.04528,2.74251);
\draw [c] (5.03046,2.71565) -- (5.04528,2.71565);
\draw [c] (5.04528,2.71565) -- (5.0601,2.71565);
\definecolor{c}{rgb}{0,0,0};
\colorlet{c}{kugray};
\draw [c] (5.07492,2.69413) -- (5.07492,2.72282);
\draw [c] (5.07492,2.72282) -- (5.07492,2.74907);
\draw [c] (5.0601,2.72282) -- (5.07492,2.72282);
\draw [c] (5.07492,2.72282) -- (5.08974,2.72282);
\definecolor{c}{rgb}{0,0,0};
\colorlet{c}{kugray};
\draw [c] (5.10455,2.68206) -- (5.10455,2.71119);
\draw [c] (5.10455,2.71119) -- (5.10455,2.73781);
\draw [c] (5.08974,2.71119) -- (5.10455,2.71119);
\draw [c] (5.10455,2.71119) -- (5.11937,2.71119);
\definecolor{c}{rgb}{0,0,0};
\colorlet{c}{kugray};
\draw [c] (5.13419,2.68307) -- (5.13419,2.7116);
\draw [c] (5.13419,2.7116) -- (5.13419,2.7377);
\draw [c] (5.11937,2.7116) -- (5.13419,2.7116);
\draw [c] (5.13419,2.7116) -- (5.14901,2.7116);
\definecolor{c}{rgb}{0,0,0};
\colorlet{c}{kugray};
\draw [c] (5.16382,2.72714) -- (5.16382,2.75436);
\draw [c] (5.16382,2.75436) -- (5.16382,2.77936);
\draw [c] (5.14901,2.75436) -- (5.16382,2.75436);
\draw [c] (5.16382,2.75436) -- (5.17864,2.75436);
\definecolor{c}{rgb}{0,0,0};
\colorlet{c}{kugray};
\draw [c] (5.19346,2.64861) -- (5.19346,2.67901);
\draw [c] (5.19346,2.67901) -- (5.19346,2.70667);
\draw [c] (5.17864,2.67901) -- (5.19346,2.67901);
\draw [c] (5.19346,2.67901) -- (5.20828,2.67901);
\definecolor{c}{rgb}{0,0,0};
\colorlet{c}{kugray};
\draw [c] (5.2231,2.688) -- (5.2231,2.71811);
\draw [c] (5.2231,2.71811) -- (5.2231,2.74554);
\draw [c] (5.20828,2.71811) -- (5.2231,2.71811);
\draw [c] (5.2231,2.71811) -- (5.23791,2.71811);
\definecolor{c}{rgb}{0,0,0};
\colorlet{c}{kugray};
\draw [c] (5.25273,2.56862) -- (5.25273,2.60483);
\draw [c] (5.25273,2.60483) -- (5.25273,2.63722);
\draw [c] (5.23791,2.60483) -- (5.25273,2.60483);
\draw [c] (5.25273,2.60483) -- (5.26755,2.60483);
\definecolor{c}{rgb}{0,0,0};
\colorlet{c}{kugray};
\draw [c] (5.28237,2.61362) -- (5.28237,2.64718);
\draw [c] (5.28237,2.64718) -- (5.28237,2.67745);
\draw [c] (5.26755,2.64718) -- (5.28237,2.64718);
\draw [c] (5.28237,2.64718) -- (5.29719,2.64718);
\definecolor{c}{rgb}{0,0,0};
\colorlet{c}{kugray};
\draw [c] (5.312,2.58142) -- (5.312,2.61649);
\draw [c] (5.312,2.61649) -- (5.312,2.64798);
\draw [c] (5.29719,2.61649) -- (5.312,2.61649);
\draw [c] (5.312,2.61649) -- (5.32682,2.61649);
\definecolor{c}{rgb}{0,0,0};
\colorlet{c}{kugray};
\draw [c] (5.34164,2.59936) -- (5.34164,2.63325);
\draw [c] (5.34164,2.63325) -- (5.34164,2.66377);
\draw [c] (5.32682,2.63325) -- (5.34164,2.63325);
\draw [c] (5.34164,2.63325) -- (5.35646,2.63325);
\definecolor{c}{rgb}{0,0,0};
\colorlet{c}{kugray};
\draw [c] (5.37127,2.59301) -- (5.37127,2.62735);
\draw [c] (5.37127,2.62735) -- (5.37127,2.65824);
\draw [c] (5.35646,2.62735) -- (5.37127,2.62735);
\draw [c] (5.37127,2.62735) -- (5.38609,2.62735);
\definecolor{c}{rgb}{0,0,0};
\colorlet{c}{kugray};
\draw [c] (5.40091,2.57979) -- (5.40091,2.61666);
\draw [c] (5.40091,2.61666) -- (5.40091,2.64958);
\draw [c] (5.38609,2.61666) -- (5.40091,2.61666);
\draw [c] (5.40091,2.61666) -- (5.41573,2.61666);
\definecolor{c}{rgb}{0,0,0};
\colorlet{c}{kugray};
\draw [c] (5.43055,2.5888) -- (5.43055,2.62522);
\draw [c] (5.43055,2.62522) -- (5.43055,2.65779);
\draw [c] (5.41573,2.62522) -- (5.43055,2.62522);
\draw [c] (5.43055,2.62522) -- (5.44536,2.62522);
\definecolor{c}{rgb}{0,0,0};
\colorlet{c}{kugray};
\draw [c] (5.46018,2.49892) -- (5.46018,2.53839);
\draw [c] (5.46018,2.53839) -- (5.46018,2.57337);
\draw [c] (5.44536,2.53839) -- (5.46018,2.53839);
\draw [c] (5.46018,2.53839) -- (5.475,2.53839);
\definecolor{c}{rgb}{0,0,0};
\colorlet{c}{kugray};
\draw [c] (5.48982,2.51137) -- (5.48982,2.5487);
\draw [c] (5.48982,2.5487) -- (5.48982,2.58199);
\draw [c] (5.475,2.5487) -- (5.48982,2.5487);
\draw [c] (5.48982,2.5487) -- (5.50464,2.5487);
\definecolor{c}{rgb}{0,0,0};
\colorlet{c}{kugray};
\draw [c] (5.51945,2.5345) -- (5.51945,2.57323);
\draw [c] (5.51945,2.57323) -- (5.51945,2.60762);
\draw [c] (5.50464,2.57323) -- (5.51945,2.57323);
\draw [c] (5.51945,2.57323) -- (5.53427,2.57323);
\definecolor{c}{rgb}{0,0,0};
\colorlet{c}{kugray};
\draw [c] (5.54909,2.49349) -- (5.54909,2.533);
\draw [c] (5.54909,2.533) -- (5.54909,2.56801);
\draw [c] (5.53427,2.533) -- (5.54909,2.533);
\draw [c] (5.54909,2.533) -- (5.56391,2.533);
\definecolor{c}{rgb}{0,0,0};
\colorlet{c}{kugray};
\draw [c] (5.57873,2.46256) -- (5.57873,2.50333);
\draw [c] (5.57873,2.50333) -- (5.57873,2.53932);
\draw [c] (5.56391,2.50333) -- (5.57873,2.50333);
\draw [c] (5.57873,2.50333) -- (5.59354,2.50333);
\definecolor{c}{rgb}{0,0,0};
\colorlet{c}{kugray};
\draw [c] (5.60836,2.50823) -- (5.60836,2.54684);
\draw [c] (5.60836,2.54684) -- (5.60836,2.58115);
\draw [c] (5.59354,2.54684) -- (5.60836,2.54684);
\draw [c] (5.60836,2.54684) -- (5.62318,2.54684);
\definecolor{c}{rgb}{0,0,0};
\colorlet{c}{kugray};
\draw [c] (5.638,2.50569) -- (5.638,2.54708);
\draw [c] (5.638,2.54708) -- (5.638,2.58356);
\draw [c] (5.62318,2.54708) -- (5.638,2.54708);
\draw [c] (5.638,2.54708) -- (5.65281,2.54708);
\definecolor{c}{rgb}{0,0,0};
\colorlet{c}{kugray};
\draw [c] (5.66763,2.48101) -- (5.66763,2.5229);
\draw [c] (5.66763,2.5229) -- (5.66763,2.55976);
\draw [c] (5.65281,2.5229) -- (5.66763,2.5229);
\draw [c] (5.66763,2.5229) -- (5.68245,2.5229);
\definecolor{c}{rgb}{0,0,0};
\colorlet{c}{kugray};
\draw [c] (5.69727,2.3796) -- (5.69727,2.42684);
\draw [c] (5.69727,2.42684) -- (5.69727,2.46778);
\draw [c] (5.68245,2.42684) -- (5.69727,2.42684);
\draw [c] (5.69727,2.42684) -- (5.71209,2.42684);
\definecolor{c}{rgb}{0,0,0};
\colorlet{c}{kugray};
\draw [c] (5.7269,2.30877) -- (5.7269,2.35869);
\draw [c] (5.7269,2.35869) -- (5.7269,2.40163);
\draw [c] (5.71209,2.35869) -- (5.7269,2.35869);
\draw [c] (5.7269,2.35869) -- (5.74172,2.35869);
\definecolor{c}{rgb}{0,0,0};
\colorlet{c}{kugray};
\draw [c] (5.75654,2.43196) -- (5.75654,2.47986);
\draw [c] (5.75654,2.47986) -- (5.75654,2.5213);
\draw [c] (5.74172,2.47986) -- (5.75654,2.47986);
\draw [c] (5.75654,2.47986) -- (5.77136,2.47986);
\definecolor{c}{rgb}{0,0,0};
\colorlet{c}{kugray};
\draw [c] (5.78618,2.43169) -- (5.78618,2.47526);
\draw [c] (5.78618,2.47526) -- (5.78618,2.51342);
\draw [c] (5.77136,2.47526) -- (5.78618,2.47526);
\draw [c] (5.78618,2.47526) -- (5.80099,2.47526);
\definecolor{c}{rgb}{0,0,0};
\colorlet{c}{kugray};
\draw [c] (5.81581,2.39907) -- (5.81581,2.44526);
\draw [c] (5.81581,2.44526) -- (5.81581,2.48541);
\draw [c] (5.80099,2.44526) -- (5.81581,2.44526);
\draw [c] (5.81581,2.44526) -- (5.83063,2.44526);
\definecolor{c}{rgb}{0,0,0};
\colorlet{c}{kugray};
\draw [c] (5.84545,2.43121) -- (5.84545,2.47744);
\draw [c] (5.84545,2.47744) -- (5.84545,2.51762);
\draw [c] (5.83063,2.47744) -- (5.84545,2.47744);
\draw [c] (5.84545,2.47744) -- (5.86026,2.47744);
\definecolor{c}{rgb}{0,0,0};
\colorlet{c}{kugray};
\draw [c] (5.87508,2.29827) -- (5.87508,2.35564);
\draw [c] (5.87508,2.35564) -- (5.87508,2.40397);
\draw [c] (5.86026,2.35564) -- (5.87508,2.35564);
\draw [c] (5.87508,2.35564) -- (5.8899,2.35564);
\definecolor{c}{rgb}{0,0,0};
\colorlet{c}{kugray};
\draw [c] (5.90472,2.35345) -- (5.90472,2.40527);
\draw [c] (5.90472,2.40527) -- (5.90472,2.44961);
\draw [c] (5.8899,2.40527) -- (5.90472,2.40527);
\draw [c] (5.90472,2.40527) -- (5.91954,2.40527);
\definecolor{c}{rgb}{0,0,0};
\colorlet{c}{kugray};
\draw [c] (5.93435,2.28972) -- (5.93435,2.34179);
\draw [c] (5.93435,2.34179) -- (5.93435,2.38632);
\draw [c] (5.91954,2.34179) -- (5.93435,2.34179);
\draw [c] (5.93435,2.34179) -- (5.94917,2.34179);
\definecolor{c}{rgb}{0,0,0};
\colorlet{c}{kugray};
\draw [c] (5.96399,2.28856) -- (5.96399,2.34875);
\draw [c] (5.96399,2.34875) -- (5.96399,2.39907);
\draw [c] (5.94917,2.34875) -- (5.96399,2.34875);
\draw [c] (5.96399,2.34875) -- (5.97881,2.34875);
\definecolor{c}{rgb}{0,0,0};
\colorlet{c}{kugray};
\draw [c] (5.99363,2.34369) -- (5.99363,2.39484);
\draw [c] (5.99363,2.39484) -- (5.99363,2.43869);
\draw [c] (5.97881,2.39484) -- (5.99363,2.39484);
\draw [c] (5.99363,2.39484) -- (6.00844,2.39484);
\definecolor{c}{rgb}{0,0,0};
\colorlet{c}{kugray};
\draw [c] (6.02326,2.22004) -- (6.02326,2.27835);
\draw [c] (6.02326,2.27835) -- (6.02326,2.32735);
\draw [c] (6.00844,2.27835) -- (6.02326,2.27835);
\draw [c] (6.02326,2.27835) -- (6.03808,2.27835);
\definecolor{c}{rgb}{0,0,0};
\colorlet{c}{kugray};
\draw [c] (6.0529,2.24812) -- (6.0529,2.30663);
\draw [c] (6.0529,2.30663) -- (6.0529,2.35577);
\draw [c] (6.03808,2.30663) -- (6.0529,2.30663);
\draw [c] (6.0529,2.30663) -- (6.06772,2.30663);
\definecolor{c}{rgb}{0,0,0};
\colorlet{c}{kugray};
\draw [c] (6.08253,2.21392) -- (6.08253,2.27979);
\draw [c] (6.08253,2.27979) -- (6.08253,2.33402);
\draw [c] (6.06772,2.27979) -- (6.08253,2.27979);
\draw [c] (6.08253,2.27979) -- (6.09735,2.27979);
\definecolor{c}{rgb}{0,0,0};
\colorlet{c}{kugray};
\draw [c] (6.11217,2.28422) -- (6.11217,2.34425);
\draw [c] (6.11217,2.34425) -- (6.11217,2.39446);
\draw [c] (6.09735,2.34425) -- (6.11217,2.34425);
\draw [c] (6.11217,2.34425) -- (6.12699,2.34425);
\definecolor{c}{rgb}{0,0,0};
\colorlet{c}{kugray};
\draw [c] (6.1418,2.13526) -- (6.1418,2.20119);
\draw [c] (6.1418,2.20119) -- (6.1418,2.25547);
\draw [c] (6.12699,2.20119) -- (6.1418,2.20119);
\draw [c] (6.1418,2.20119) -- (6.15662,2.20119);
\definecolor{c}{rgb}{0,0,0};
\colorlet{c}{kugray};
\draw [c] (6.17144,2.21632) -- (6.17144,2.28052);
\draw [c] (6.17144,2.28052) -- (6.17144,2.33362);
\draw [c] (6.15662,2.28052) -- (6.17144,2.28052);
\draw [c] (6.17144,2.28052) -- (6.18626,2.28052);
\definecolor{c}{rgb}{0,0,0};
\colorlet{c}{kugray};
\draw [c] (6.20108,2.25826) -- (6.20108,2.32268);
\draw [c] (6.20108,2.32268) -- (6.20108,2.37592);
\draw [c] (6.18626,2.32268) -- (6.20108,2.32268);
\draw [c] (6.20108,2.32268) -- (6.21589,2.32268);
\definecolor{c}{rgb}{0,0,0};
\colorlet{c}{kugray};
\draw [c] (6.23071,2.21696) -- (6.23071,2.28153);
\draw [c] (6.23071,2.28153) -- (6.23071,2.33487);
\draw [c] (6.21589,2.28153) -- (6.23071,2.28153);
\draw [c] (6.23071,2.28153) -- (6.24553,2.28153);
\definecolor{c}{rgb}{0,0,0};
\colorlet{c}{kugray};
\draw [c] (6.26035,2.11717) -- (6.26035,2.18912);
\draw [c] (6.26035,2.18912) -- (6.26035,2.2474);
\draw [c] (6.24553,2.18912) -- (6.26035,2.18912);
\draw [c] (6.26035,2.18912) -- (6.27517,2.18912);
\definecolor{c}{rgb}{0,0,0};
\colorlet{c}{kugray};
\draw [c] (6.28998,2.21915) -- (6.28998,2.27945);
\draw [c] (6.28998,2.27945) -- (6.28998,2.32985);
\draw [c] (6.27517,2.27945) -- (6.28998,2.27945);
\draw [c] (6.28998,2.27945) -- (6.3048,2.27945);
\definecolor{c}{rgb}{0,0,0};
\colorlet{c}{kugray};
\draw [c] (6.31962,2.15067) -- (6.31962,2.22057);
\draw [c] (6.31962,2.22057) -- (6.31962,2.2775);
\draw [c] (6.3048,2.22057) -- (6.31962,2.22057);
\draw [c] (6.31962,2.22057) -- (6.33444,2.22057);
\definecolor{c}{rgb}{0,0,0};
\colorlet{c}{kugray};
\draw [c] (6.34926,2.08129) -- (6.34926,2.15905);
\draw [c] (6.34926,2.15905) -- (6.34926,2.22107);
\draw [c] (6.33444,2.15905) -- (6.34926,2.15905);
\draw [c] (6.34926,2.15905) -- (6.36407,2.15905);
\definecolor{c}{rgb}{0,0,0};
\colorlet{c}{kugray};
\draw [c] (6.37889,2.26481) -- (6.37889,2.32632);
\draw [c] (6.37889,2.32632) -- (6.37889,2.37756);
\draw [c] (6.36407,2.32632) -- (6.37889,2.32632);
\draw [c] (6.37889,2.32632) -- (6.39371,2.32632);
\definecolor{c}{rgb}{0,0,0};
\colorlet{c}{kugray};
\draw [c] (6.40853,2.10103) -- (6.40853,2.17287);
\draw [c] (6.40853,2.17287) -- (6.40853,2.23108);
\draw [c] (6.39371,2.17287) -- (6.40853,2.17287);
\draw [c] (6.40853,2.17287) -- (6.42334,2.17287);
\definecolor{c}{rgb}{0,0,0};
\colorlet{c}{kugray};
\draw [c] (6.43816,2.01805) -- (6.43816,2.10352);
\draw [c] (6.43816,2.10352) -- (6.43816,2.17034);
\draw [c] (6.42334,2.10352) -- (6.43816,2.10352);
\draw [c] (6.43816,2.10352) -- (6.45298,2.10352);
\definecolor{c}{rgb}{0,0,0};
\colorlet{c}{kugray};
\draw [c] (6.4678,2.13798) -- (6.4678,2.21314);
\draw [c] (6.4678,2.21314) -- (6.4678,2.27351);
\draw [c] (6.45298,2.21314) -- (6.4678,2.21314);
\draw [c] (6.4678,2.21314) -- (6.48262,2.21314);
\definecolor{c}{rgb}{0,0,0};
\colorlet{c}{kugray};
\draw [c] (6.49743,2.0628) -- (6.49743,2.13789);
\draw [c] (6.49743,2.13789) -- (6.49743,2.19821);
\draw [c] (6.48262,2.13789) -- (6.49743,2.13789);
\draw [c] (6.49743,2.13789) -- (6.51225,2.13789);
\definecolor{c}{rgb}{0,0,0};
\colorlet{c}{kugray};
\draw [c] (6.52707,2.03036) -- (6.52707,2.11014);
\draw [c] (6.52707,2.11014) -- (6.52707,2.17345);
\draw [c] (6.51225,2.11014) -- (6.52707,2.11014);
\draw [c] (6.52707,2.11014) -- (6.54189,2.11014);
\definecolor{c}{rgb}{0,0,0};
\colorlet{c}{kugray};
\draw [c] (6.55671,1.95941) -- (6.55671,2.05375);
\draw [c] (6.55671,2.05375) -- (6.55671,2.12586);
\draw [c] (6.54189,2.05375) -- (6.55671,2.05375);
\draw [c] (6.55671,2.05375) -- (6.57152,2.05375);
\definecolor{c}{rgb}{0,0,0};
\colorlet{c}{kugray};
\draw [c] (6.58634,2.0092) -- (6.58634,2.09944);
\draw [c] (6.58634,2.09944) -- (6.58634,2.16914);
\draw [c] (6.57152,2.09944) -- (6.58634,2.09944);
\draw [c] (6.58634,2.09944) -- (6.60116,2.09944);
\definecolor{c}{rgb}{0,0,0};
\colorlet{c}{kugray};
\draw [c] (6.61598,2.06823) -- (6.61598,2.14649);
\draw [c] (6.61598,2.14649) -- (6.61598,2.20883);
\draw [c] (6.60116,2.14649) -- (6.61598,2.14649);
\draw [c] (6.61598,2.14649) -- (6.63079,2.14649);
\definecolor{c}{rgb}{0,0,0};
\colorlet{c}{kugray};
\draw [c] (6.64561,1.91105) -- (6.64561,2.01133);
\draw [c] (6.64561,2.01133) -- (6.64561,2.08685);
\draw [c] (6.63079,2.01133) -- (6.64561,2.01133);
\draw [c] (6.64561,2.01133) -- (6.66043,2.01133);
\definecolor{c}{rgb}{0,0,0};
\colorlet{c}{kugray};
\draw [c] (6.67525,1.90644) -- (6.67525,2.01096);
\draw [c] (6.67525,2.01096) -- (6.67525,2.08884);
\draw [c] (6.66043,2.01096) -- (6.67525,2.01096);
\draw [c] (6.67525,2.01096) -- (6.69007,2.01096);
\definecolor{c}{rgb}{0,0,0};
\colorlet{c}{kugray};
\draw [c] (6.70488,1.98499) -- (6.70488,2.07197);
\draw [c] (6.70488,2.07197) -- (6.70488,2.13971);
\draw [c] (6.69007,2.07197) -- (6.70488,2.07197);
\draw [c] (6.70488,2.07197) -- (6.7197,2.07197);
\definecolor{c}{rgb}{0,0,0};
\colorlet{c}{kugray};
\draw [c] (6.73452,1.98323) -- (6.73452,2.07762);
\draw [c] (6.73452,2.07762) -- (6.73452,2.14976);
\draw [c] (6.7197,2.07762) -- (6.73452,2.07762);
\draw [c] (6.73452,2.07762) -- (6.74934,2.07762);
\definecolor{c}{rgb}{0,0,0};
\colorlet{c}{kugray};
\draw [c] (6.76416,1.8404) -- (6.76416,1.94614);
\draw [c] (6.76416,1.94614) -- (6.76416,2.0247);
\draw [c] (6.74934,1.94614) -- (6.76416,1.94614);
\draw [c] (6.76416,1.94614) -- (6.77897,1.94614);
\definecolor{c}{rgb}{0,0,0};
\colorlet{c}{kugray};
\draw [c] (6.79379,1.95326) -- (6.79379,2.04637);
\draw [c] (6.79379,2.04637) -- (6.79379,2.11775);
\draw [c] (6.77897,2.04637) -- (6.79379,2.04637);
\draw [c] (6.79379,2.04637) -- (6.80861,2.04637);
\definecolor{c}{rgb}{0,0,0};
\colorlet{c}{kugray};
\draw [c] (6.82343,1.84998) -- (6.82343,1.96142);
\draw [c] (6.82343,1.96142) -- (6.82343,2.04307);
\draw [c] (6.80861,1.96142) -- (6.82343,1.96142);
\draw [c] (6.82343,1.96142) -- (6.83824,1.96142);
\definecolor{c}{rgb}{0,0,0};
\colorlet{c}{kugray};
\draw [c] (6.85306,1.97353) -- (6.85306,2.06824);
\draw [c] (6.85306,2.06824) -- (6.85306,2.14058);
\draw [c] (6.83824,2.06824) -- (6.85306,2.06824);
\draw [c] (6.85306,2.06824) -- (6.86788,2.06824);
\definecolor{c}{rgb}{0,0,0};
\colorlet{c}{kugray};
\draw [c] (6.8827,1.94959) -- (6.8827,2.05345);
\draw [c] (6.8827,2.05345) -- (6.8827,2.13097);
\draw [c] (6.86788,2.05345) -- (6.8827,2.05345);
\draw [c] (6.8827,2.05345) -- (6.89752,2.05345);
\definecolor{c}{rgb}{0,0,0};
\colorlet{c}{kugray};
\draw [c] (6.91233,1.64738) -- (6.91233,1.796);
\draw [c] (6.91233,1.796) -- (6.91233,1.8958);
\draw [c] (6.89752,1.796) -- (6.91233,1.796);
\draw [c] (6.91233,1.796) -- (6.92715,1.796);
\definecolor{c}{rgb}{0,0,0};
\colorlet{c}{kugray};
\draw [c] (6.94197,1.81861) -- (6.94197,1.94633);
\draw [c] (6.94197,1.94633) -- (6.94197,2.03632);
\draw [c] (6.92715,1.94633) -- (6.94197,1.94633);
\draw [c] (6.94197,1.94633) -- (6.95679,1.94633);
\definecolor{c}{rgb}{0,0,0};
\colorlet{c}{kugray};
\draw [c] (6.97161,1.45467) -- (6.97161,1.67503);
\draw [c] (6.97161,1.67503) -- (6.97161,1.80215);
\draw [c] (6.95679,1.67503) -- (6.97161,1.67503);
\draw [c] (6.97161,1.67503) -- (6.98642,1.67503);
\definecolor{c}{rgb}{0,0,0};
\colorlet{c}{kugray};
\draw [c] (7.00124,1.92607) -- (7.00124,2.03252);
\draw [c] (7.00124,2.03252) -- (7.00124,2.11147);
\draw [c] (6.98642,2.03252) -- (7.00124,2.03252);
\draw [c] (7.00124,2.03252) -- (7.01606,2.03252);
\definecolor{c}{rgb}{0,0,0};
\colorlet{c}{kugray};
\draw [c] (7.03088,1.70142) -- (7.03088,1.83023);
\draw [c] (7.03088,1.83023) -- (7.03088,1.92076);
\draw [c] (7.01606,1.83023) -- (7.03088,1.83023);
\draw [c] (7.03088,1.83023) -- (7.0457,1.83023);
\definecolor{c}{rgb}{0,0,0};
\colorlet{c}{kugray};
\draw [c] (7.06051,1.69915) -- (7.06051,1.84157);
\draw [c] (7.06051,1.84157) -- (7.06051,1.93856);
\draw [c] (7.0457,1.84157) -- (7.06051,1.84157);
\draw [c] (7.06051,1.84157) -- (7.07533,1.84157);
\definecolor{c}{rgb}{0,0,0};
\colorlet{c}{kugray};
\draw [c] (7.09015,1.30062) -- (7.09015,1.58241);
\draw [c] (7.09015,1.58241) -- (7.09015,1.72703);
\draw [c] (7.07533,1.58241) -- (7.09015,1.58241);
\draw [c] (7.09015,1.58241) -- (7.10497,1.58241);
\definecolor{c}{rgb}{0,0,0};
\colorlet{c}{kugray};
\draw [c] (7.11978,1.6185) -- (7.11978,1.77873);
\draw [c] (7.11978,1.77873) -- (7.11978,1.88358);
\draw [c] (7.10497,1.77873) -- (7.11978,1.77873);
\draw [c] (7.11978,1.77873) -- (7.1346,1.77873);
\definecolor{c}{rgb}{0,0,0};
\colorlet{c}{kugray};
\draw [c] (7.14942,1.83806) -- (7.14942,1.95138);
\draw [c] (7.14942,1.95138) -- (7.14942,2.03403);
\draw [c] (7.1346,1.95138) -- (7.14942,1.95138);
\draw [c] (7.14942,1.95138) -- (7.16424,1.95138);
\definecolor{c}{rgb}{0,0,0};
\colorlet{c}{kugray};
\draw [c] (7.17906,1.68124) -- (7.17906,1.87494);
\draw [c] (7.17906,1.87494) -- (7.17906,1.99292);
\draw [c] (7.16424,1.87494) -- (7.17906,1.87494);
\draw [c] (7.17906,1.87494) -- (7.19387,1.87494);
\definecolor{c}{rgb}{0,0,0};
\colorlet{c}{kugray};
\draw [c] (7.20869,1.63815) -- (7.20869,1.80529);
\draw [c] (7.20869,1.80529) -- (7.20869,1.91302);
\draw [c] (7.19387,1.80529) -- (7.20869,1.80529);
\draw [c] (7.20869,1.80529) -- (7.22351,1.80529);
\definecolor{c}{rgb}{0,0,0};
\colorlet{c}{kugray};
\draw [c] (7.23833,1.534) -- (7.23833,1.72717);
\draw [c] (7.23833,1.72717) -- (7.23833,1.84495);
\draw [c] (7.22351,1.72717) -- (7.23833,1.72717);
\draw [c] (7.23833,1.72717) -- (7.25315,1.72717);
\definecolor{c}{rgb}{0,0,0};
\colorlet{c}{kugray};
\draw [c] (7.26796,1.44669) -- (7.26796,1.66186);
\draw [c] (7.26796,1.66186) -- (7.26796,1.78728);
\draw [c] (7.25315,1.66186) -- (7.26796,1.66186);
\draw [c] (7.26796,1.66186) -- (7.28278,1.66186);
\definecolor{c}{rgb}{0,0,0};
\colorlet{c}{kugray};
\draw [c] (7.2976,1.67008) -- (7.2976,1.80777);
\draw [c] (7.2976,1.80777) -- (7.2976,1.90257);
\draw [c] (7.28278,1.80777) -- (7.2976,1.80777);
\draw [c] (7.2976,1.80777) -- (7.31242,1.80777);
\definecolor{c}{rgb}{0,0,0};
\colorlet{c}{kugray};
\draw [c] (7.32724,1.55019) -- (7.32724,1.78423);
\draw [c] (7.32724,1.78423) -- (7.32724,1.91564);
\draw [c] (7.31242,1.78423) -- (7.32724,1.78423);
\draw [c] (7.32724,1.78423) -- (7.34205,1.78423);
\definecolor{c}{rgb}{0,0,0};
\colorlet{c}{kugray};
\draw [c] (7.35687,1.70986) -- (7.35687,1.88005);
\draw [c] (7.35687,1.88005) -- (7.35687,1.98902);
\draw [c] (7.34205,1.88005) -- (7.35687,1.88005);
\draw [c] (7.35687,1.88005) -- (7.37169,1.88005);
\definecolor{c}{rgb}{0,0,0};
\colorlet{c}{kugray};
\draw [c] (7.38651,1.39569) -- (7.38651,1.61683);
\draw [c] (7.38651,1.61683) -- (7.38651,1.7442);
\draw [c] (7.37169,1.61683) -- (7.38651,1.61683);
\draw [c] (7.38651,1.61683) -- (7.40132,1.61683);
\definecolor{c}{rgb}{0,0,0};
\colorlet{c}{kugray};
\draw [c] (7.41614,1.65215) -- (7.41614,1.82337);
\draw [c] (7.41614,1.82337) -- (7.41614,1.93274);
\draw [c] (7.40132,1.82337) -- (7.41614,1.82337);
\draw [c] (7.41614,1.82337) -- (7.43096,1.82337);
\definecolor{c}{rgb}{0,0,0};
\colorlet{c}{kugray};
\draw [c] (7.44578,1.61682) -- (7.44578,1.76694);
\draw [c] (7.44578,1.76694) -- (7.44578,1.86741);
\draw [c] (7.43096,1.76694) -- (7.44578,1.76694);
\draw [c] (7.44578,1.76694) -- (7.4606,1.76694);
\definecolor{c}{rgb}{0,0,0};
\colorlet{c}{kugray};
\draw [c] (7.47541,0.903192) -- (7.47541,1.30472);
\draw [c] (7.47541,1.30472) -- (7.47541,1.47322);
\draw [c] (7.4606,1.30472) -- (7.47541,1.30472);
\draw [c] (7.47541,1.30472) -- (7.49023,1.30472);
\definecolor{c}{rgb}{0,0,0};
\colorlet{c}{kugray};
\draw [c] (7.50505,1.17257) -- (7.50505,1.55106);
\draw [c] (7.50505,1.55106) -- (7.50505,1.71578);
\draw [c] (7.49023,1.55106) -- (7.50505,1.55106);
\draw [c] (7.50505,1.55106) -- (7.51987,1.55106);
\definecolor{c}{rgb}{0,0,0};
\colorlet{c}{kugray};
\draw [c] (7.53469,1.26039) -- (7.53469,1.5276);
\draw [c] (7.53469,1.5276) -- (7.53469,1.66846);
\draw [c] (7.51987,1.5276) -- (7.53469,1.5276);
\draw [c] (7.53469,1.5276) -- (7.5495,1.5276);
\definecolor{c}{rgb}{0,0,0};
\colorlet{c}{kugray};
\draw [c] (7.56432,1.35089) -- (7.56432,1.65389);
\draw [c] (7.56432,1.65389) -- (7.56432,1.80359);
\draw [c] (7.5495,1.65389) -- (7.56432,1.65389);
\draw [c] (7.56432,1.65389) -- (7.57914,1.65389);
\definecolor{c}{rgb}{0,0,0};
\colorlet{c}{kugray};
\draw [c] (7.59396,1.63685) -- (7.59396,1.81433);
\draw [c] (7.59396,1.81433) -- (7.59396,1.92619);
\draw [c] (7.57914,1.81433) -- (7.59396,1.81433);
\draw [c] (7.59396,1.81433) -- (7.60877,1.81433);
\definecolor{c}{rgb}{0,0,0};
\colorlet{c}{kugray};
\draw [c] (7.62359,1.2207) -- (7.62359,1.48705);
\draw [c] (7.62359,1.48705) -- (7.62359,1.62768);
\draw [c] (7.60877,1.48705) -- (7.62359,1.48705);
\draw [c] (7.62359,1.48705) -- (7.63841,1.48705);
\definecolor{c}{rgb}{0,0,0};
\colorlet{c}{kugray};
\draw [c] (7.65323,1.15339) -- (7.65323,1.44836);
\draw [c] (7.65323,1.44836) -- (7.65323,1.5962);
\draw [c] (7.63841,1.44836) -- (7.65323,1.44836);
\draw [c] (7.65323,1.44836) -- (7.66805,1.44836);
\definecolor{c}{rgb}{0,0,0};
\colorlet{c}{kugray};
\draw [c] (7.68286,0.824113) -- (7.68286,1.32118);
\draw [c] (7.68286,1.32118) -- (7.68286,1.50231);
\draw [c] (7.66805,1.32118) -- (7.68286,1.32118);
\draw [c] (7.68286,1.32118) -- (7.69768,1.32118);
\definecolor{c}{rgb}{0,0,0};
\colorlet{c}{kugray};
\draw [c] (7.7125,1.26126) -- (7.7125,1.5339);
\draw [c] (7.7125,1.5339) -- (7.7125,1.67619);
\draw [c] (7.69768,1.5339) -- (7.7125,1.5339);
\draw [c] (7.7125,1.5339) -- (7.72732,1.5339);
\definecolor{c}{rgb}{0,0,0};
\colorlet{c}{kugray};
\draw [c] (7.74214,1.48045) -- (7.74214,1.72479);
\draw [c] (7.74214,1.72479) -- (7.74214,1.85928);
\draw [c] (7.72732,1.72479) -- (7.74214,1.72479);
\draw [c] (7.74214,1.72479) -- (7.75695,1.72479);
\definecolor{c}{rgb}{0,0,0};
\colorlet{c}{kugray};
\draw [c] (7.77177,1.11205) -- (7.77177,1.40937);
\draw [c] (7.77177,1.40937) -- (7.77177,1.55776);
\draw [c] (7.75695,1.40937) -- (7.77177,1.40937);
\draw [c] (7.77177,1.40937) -- (7.78659,1.40937);
\definecolor{c}{rgb}{0,0,0};
\colorlet{c}{kugray};
\draw [c] (7.80141,1.34931) -- (7.80141,1.63981);
\draw [c] (7.80141,1.63981) -- (7.80141,1.78657);
\draw [c] (7.78659,1.63981) -- (7.80141,1.63981);
\draw [c] (7.80141,1.63981) -- (7.81623,1.63981);
\definecolor{c}{rgb}{0,0,0};
\colorlet{c}{kugray};
\draw [c] (7.83104,1.0292) -- (7.83104,1.41463);
\draw [c] (7.83104,1.41463) -- (7.83104,1.58052);
\draw [c] (7.81623,1.41463) -- (7.83104,1.41463);
\draw [c] (7.83104,1.41463) -- (7.84586,1.41463);
\definecolor{c}{rgb}{0,0,0};
\colorlet{c}{kugray};
\draw [c] (7.86068,0.596817) -- (7.86068,1.16974);
\draw [c] (7.86068,1.16974) -- (7.86068,1.38313);
\draw [c] (7.84586,1.16974) -- (7.86068,1.16974);
\draw [c] (7.86068,1.16974) -- (7.8755,1.16974);
\definecolor{c}{rgb}{0,0,0};
\colorlet{c}{kugray};
\draw [c] (7.89031,1.22506) -- (7.89031,1.49977);
\draw [c] (7.89031,1.49977) -- (7.89031,1.64259);
\draw [c] (7.8755,1.49977) -- (7.89031,1.49977);
\draw [c] (7.89031,1.49977) -- (7.90513,1.49977);
\definecolor{c}{rgb}{0,0,0};
\colorlet{c}{kugray};
\draw [c] (7.91995,1.37554) -- (7.91995,1.58913);
\draw [c] (7.91995,1.58913) -- (7.91995,1.71402);
\draw [c] (7.90513,1.58913) -- (7.91995,1.58913);
\draw [c] (7.91995,1.58913) -- (7.93477,1.58913);
\definecolor{c}{rgb}{0,0,0};
\colorlet{c}{kugray};
\draw [c] (7.94959,0.970556) -- (7.94959,1.35289);
\draw [c] (7.94959,1.35289) -- (7.94959,1.51826);
\draw [c] (7.93477,1.35289) -- (7.94959,1.35289);
\draw [c] (7.94959,1.35289) -- (7.9644,1.35289);
\definecolor{c}{rgb}{0,0,0};
\colorlet{c}{kugray};
\draw [c] (7.97922,1.42663) -- (7.97922,1.62932);
\draw [c] (7.97922,1.62932) -- (7.97922,1.75049);
\draw [c] (7.9644,1.62932) -- (7.97922,1.62932);
\draw [c] (7.97922,1.62932) -- (7.99404,1.62932);
\definecolor{c}{rgb}{0,0,0};
\colorlet{c}{kugray};
\draw [c] (8.00886,1.24387) -- (8.00886,1.51218);
\draw [c] (8.00886,1.51218) -- (8.00886,1.65334);
\draw [c] (7.99404,1.51218) -- (8.00886,1.51218);
\draw [c] (8.00886,1.51218) -- (8.02368,1.51218);
\definecolor{c}{rgb}{0,0,0};
\colorlet{c}{kugray};
\draw [c] (8.03849,1.45206) -- (8.03849,1.64214);
\draw [c] (8.03849,1.64214) -- (8.03849,1.75879);
\draw [c] (8.02368,1.64214) -- (8.03849,1.64214);
\draw [c] (8.03849,1.64214) -- (8.05331,1.64214);
\definecolor{c}{rgb}{0,0,0};
\colorlet{c}{kugray};
\draw [c] (8.09776,1.35157) -- (8.09776,1.5665);
\draw [c] (8.09776,1.5665) -- (8.09776,1.69184);
\draw [c] (8.08295,1.5665) -- (8.09776,1.5665);
\draw [c] (8.09776,1.5665) -- (8.11258,1.5665);
\definecolor{c}{rgb}{0,0,0};
\colorlet{c}{kugray};
\draw [c] (8.1274,1.02943) -- (8.1274,1.41864);
\draw [c] (8.1274,1.41864) -- (8.1274,1.58516);
\draw [c] (8.11258,1.41864) -- (8.1274,1.41864);
\draw [c] (8.1274,1.41864) -- (8.14222,1.41864);
\definecolor{c}{rgb}{0,0,0};
\colorlet{c}{kugray};
\draw [c] (8.18667,1.3217) -- (8.18667,1.60896);
\draw [c] (8.18667,1.60896) -- (8.18667,1.75493);
\draw [c] (8.17185,1.60896) -- (8.18667,1.60896);
\draw [c] (8.18667,1.60896) -- (8.20149,1.60896);
\definecolor{c}{rgb}{0,0,0};
\colorlet{c}{kugray};
\draw [c] (8.21631,1.44991) -- (8.21631,1.72488);
\draw [c] (8.21631,1.72488) -- (8.21631,1.86777);
\draw [c] (8.20149,1.72488) -- (8.21631,1.72488);
\draw [c] (8.21631,1.72488) -- (8.23113,1.72488);
\definecolor{c}{rgb}{0,0,0};
\colorlet{c}{kugray};
\draw [c] (8.24594,0.596817) -- (8.24594,1.18479);
\draw [c] (8.24594,1.18479) -- (8.24594,1.39818);
\draw [c] (8.23113,1.18479) -- (8.24594,1.18479);
\draw [c] (8.24594,1.18479) -- (8.26076,1.18479);
\definecolor{c}{rgb}{0,0,0};
\colorlet{c}{kugray};
\draw [c] (8.27558,1.33422) -- (8.27558,1.70646);
\draw [c] (8.27558,1.70646) -- (8.27558,1.8701);
\draw [c] (8.26076,1.70646) -- (8.27558,1.70646);
\draw [c] (8.27558,1.70646) -- (8.2904,1.70646);
\definecolor{c}{rgb}{0,0,0};
\colorlet{c}{kugray};
\draw [c] (8.30521,1.01154) -- (8.30521,1.39412);
\draw [c] (8.30521,1.39412) -- (8.30521,1.55954);
\draw [c] (8.2904,1.39412) -- (8.30521,1.39412);
\draw [c] (8.30521,1.39412) -- (8.32003,1.39412);
\definecolor{c}{rgb}{0,0,0};
\colorlet{c}{kugray};
\draw [c] (8.33485,1.16538) -- (8.33485,1.451);
\draw [c] (8.33485,1.451) -- (8.33485,1.59657);
\draw [c] (8.32003,1.451) -- (8.33485,1.451);
\draw [c] (8.33485,1.451) -- (8.34967,1.451);
\definecolor{c}{rgb}{0,0,0};
\colorlet{c}{kugray};
\draw [c] (8.36449,0.596817) -- (8.36449,1.27192);
\draw [c] (8.36449,1.27192) -- (8.36449,1.48532);
\draw [c] (8.34967,1.27192) -- (8.36449,1.27192);
\draw [c] (8.36449,1.27192) -- (8.3793,1.27192);
\definecolor{c}{rgb}{0,0,0};
\colorlet{c}{kugray};
\draw [c] (8.39412,0.596817) -- (8.39412,1.1669);
\draw [c] (8.39412,1.1669) -- (8.39412,1.38029);
\draw [c] (8.3793,1.1669) -- (8.39412,1.1669);
\draw [c] (8.39412,1.1669) -- (8.40894,1.1669);
\definecolor{c}{rgb}{0,0,0};
\colorlet{c}{kugray};
\draw [c] (8.45339,1.13633) -- (8.45339,1.56977);
\draw [c] (8.45339,1.56977) -- (8.45339,1.74299);
\draw [c] (8.43858,1.56977) -- (8.45339,1.56977);
\draw [c] (8.45339,1.56977) -- (8.46821,1.56977);
\definecolor{c}{rgb}{0,0,0};
\colorlet{c}{kugray};
\draw [c] (8.48303,0.596817) -- (8.48303,1.22594);
\draw [c] (8.48303,1.22594) -- (8.48303,1.43933);
\draw [c] (8.46821,1.22594) -- (8.48303,1.22594);
\draw [c] (8.48303,1.22594) -- (8.49785,1.22594);
\definecolor{c}{rgb}{0,0,0};
\colorlet{c}{kugray};
\draw [c] (8.5423,0.596817) -- (8.5423,1.08304);
\draw [c] (8.5423,1.08304) -- (8.5423,1.29643);
\draw [c] (8.52748,1.08304) -- (8.5423,1.08304);
\draw [c] (8.5423,1.08304) -- (8.55712,1.08304);
\definecolor{c}{rgb}{0,0,0};
\colorlet{c}{kugray};
\draw [c] (8.57194,0.596817) -- (8.57194,1.19595);
\draw [c] (8.57194,1.19595) -- (8.57194,1.40934);
\draw [c] (8.55712,1.19595) -- (8.57194,1.19595);
\draw [c] (8.57194,1.19595) -- (8.58675,1.19595);
\definecolor{c}{rgb}{0,0,0};
\colorlet{c}{kugray};
\draw [c] (8.66084,0.884632) -- (8.66084,1.27717);
\draw [c] (8.66084,1.27717) -- (8.66084,1.44423);
\draw [c] (8.64603,1.27717) -- (8.66084,1.27717);
\draw [c] (8.66084,1.27717) -- (8.67566,1.27717);
\definecolor{c}{rgb}{0,0,0};
\colorlet{c}{kugray};
\draw [c] (8.69048,0.596817) -- (8.69048,1.18479);
\draw [c] (8.69048,1.18479) -- (8.69048,1.39818);
\draw [c] (8.67566,1.18479) -- (8.69048,1.18479);
\draw [c] (8.69048,1.18479) -- (8.7053,1.18479);
\definecolor{c}{rgb}{0,0,0};
\colorlet{c}{kugray};
\draw [c] (8.72012,0.596817) -- (8.72012,1.21188);
\draw [c] (8.72012,1.21188) -- (8.72012,1.42527);
\draw [c] (8.7053,1.21188) -- (8.72012,1.21188);
\draw [c] (8.72012,1.21188) -- (8.73493,1.21188);
\definecolor{c}{rgb}{0,0,0};
\colorlet{c}{kugray};
\draw [c] (8.74975,0.596817) -- (8.74975,1.16598);
\draw [c] (8.74975,1.16598) -- (8.74975,1.37937);
\draw [c] (8.73493,1.16598) -- (8.74975,1.16598);
\draw [c] (8.74975,1.16598) -- (8.76457,1.16598);
\definecolor{c}{rgb}{0,0,0};
\colorlet{c}{kugray};
\draw [c] (8.77939,0.895648) -- (8.77939,1.31039);
\draw [c] (8.77939,1.31039) -- (8.77939,1.48091);
\draw [c] (8.76457,1.31039) -- (8.77939,1.31039);
\draw [c] (8.77939,1.31039) -- (8.79421,1.31039);
\definecolor{c}{rgb}{0,0,0};
\colorlet{c}{kugray};
\draw [c] (8.80902,1.29252) -- (8.80902,1.5583);
\draw [c] (8.80902,1.5583) -- (8.80902,1.69878);
\draw [c] (8.79421,1.5583) -- (8.80902,1.5583);
\draw [c] (8.80902,1.5583) -- (8.82384,1.5583);
\definecolor{c}{rgb}{0,0,0};
\colorlet{c}{kugray};
\draw [c] (8.83866,0.596817) -- (8.83866,1.24072);
\draw [c] (8.83866,1.24072) -- (8.83866,1.45411);
\draw [c] (8.82384,1.24072) -- (8.83866,1.24072);
\draw [c] (8.83866,1.24072) -- (8.85348,1.24072);
\definecolor{c}{rgb}{0,0,0};
\colorlet{c}{kugray};
\draw [c] (8.86829,0.596817) -- (8.86829,0.976443);
\draw [c] (8.86829,0.976443) -- (8.86829,1.18983);
\draw [c] (8.85348,0.976443) -- (8.86829,0.976443);
\draw [c] (8.86829,0.976443) -- (8.88311,0.976443);
\definecolor{c}{rgb}{0,0,0};
\colorlet{c}{kugray};
\draw [c] (8.92757,0.596817) -- (8.92757,1.19654);
\draw [c] (8.92757,1.19654) -- (8.92757,1.40993);
\draw [c] (8.91275,1.19654) -- (8.92757,1.19654);
\draw [c] (8.92757,1.19654) -- (8.94238,1.19654);
\definecolor{c}{rgb}{0,0,0};
\colorlet{c}{kugray};
\draw [c] (9.01647,0.596817) -- (9.01647,1.23526);
\draw [c] (9.01647,1.23526) -- (9.01647,1.44865);
\draw [c] (9.00166,1.23526) -- (9.01647,1.23526);
\draw [c] (9.01647,1.23526) -- (9.03129,1.23526);
\definecolor{c}{rgb}{0,0,0};
\colorlet{c}{kugray};
\draw [c] (9.07574,0.596817) -- (9.07574,1.10612);
\draw [c] (9.07574,1.10612) -- (9.07574,1.31951);
\draw [c] (9.06093,1.10612) -- (9.07574,1.10612);
\draw [c] (9.07574,1.10612) -- (9.09056,1.10612);
\definecolor{c}{rgb}{0,0,0};
\colorlet{c}{kugray};
\draw [c] (9.10538,0.596817) -- (9.10538,1.18963);
\draw [c] (9.10538,1.18963) -- (9.10538,1.40302);
\draw [c] (9.09056,1.18963) -- (9.10538,1.18963);
\draw [c] (9.10538,1.18963) -- (9.1202,1.18963);
\definecolor{c}{rgb}{0,0,0};
\colorlet{c}{kugray};
\draw [c] (9.13502,0.596817) -- (9.13502,1.14382);
\draw [c] (9.13502,1.14382) -- (9.13502,1.35721);
\draw [c] (9.1202,1.14382) -- (9.13502,1.14382);
\draw [c] (9.13502,1.14382) -- (9.14983,1.14382);
\definecolor{c}{rgb}{0,0,0};
\colorlet{c}{kugray};
\draw [c] (9.22392,1.0928) -- (9.22392,1.47755);
\draw [c] (9.22392,1.47755) -- (9.22392,1.64333);
\draw [c] (9.20911,1.47755) -- (9.22392,1.47755);
\draw [c] (9.22392,1.47755) -- (9.23874,1.47755);
\definecolor{c}{rgb}{0,0,0};
\colorlet{c}{kugray};
\draw [c] (9.2832,0.596817) -- (9.2832,1.19105);
\draw [c] (9.2832,1.19105) -- (9.2832,1.40444);
\draw [c] (9.26838,1.19105) -- (9.2832,1.19105);
\draw [c] (9.2832,1.19105) -- (9.29801,1.19105);
\definecolor{c}{rgb}{0,0,0};
\colorlet{c}{kugray};
\draw [c] (9.34247,0.596817) -- (9.34247,1.0722);
\draw [c] (9.34247,1.0722) -- (9.34247,1.28559);
\draw [c] (9.32765,1.0722) -- (9.34247,1.0722);
\draw [c] (9.34247,1.0722) -- (9.35728,1.0722);
\definecolor{c}{rgb}{0,0,0};
\colorlet{c}{kugray};
\draw [c] (9.54992,0.596817) -- (9.54992,1.17593);
\draw [c] (9.54992,1.17593) -- (9.54992,1.38932);
\draw [c] (9.5351,1.17593) -- (9.54992,1.17593);
\draw [c] (9.54992,1.17593) -- (9.56474,1.17593);
\definecolor{c}{rgb}{0,0,0};
\colorlet{c}{kugray};
\draw [c] (9.60919,0.596817) -- (9.60919,1.45456);
\draw [c] (9.60919,1.45456) -- (9.60919,1.66795);
\draw [c] (9.59437,1.45456) -- (9.60919,1.45456);
\draw [c] (9.60919,1.45456) -- (9.62401,1.45456);
\definecolor{c}{rgb}{0,0,0};
\colorlet{c}{kugray};
\draw [c] (9.63882,0.596817) -- (9.63882,1.11893);
\draw [c] (9.63882,1.11893) -- (9.63882,1.33232);
\draw [c] (9.62401,1.11893) -- (9.63882,1.11893);
\draw [c] (9.63882,1.11893) -- (9.65364,1.11893);
\definecolor{c}{rgb}{0,0,0};
\colorlet{c}{kugray};
\draw [c] (9.72773,0.596817) -- (9.72773,1.59979);
\draw [c] (9.72773,1.59979) -- (9.72773,1.81318);
\draw [c] (9.71291,1.59979) -- (9.72773,1.59979);
\draw [c] (9.72773,1.59979) -- (9.74255,1.59979);
\definecolor{c}{rgb}{0,0,0};
\colorlet{c}{kugray};
\draw [c] (9.787,0.596817) -- (9.787,1.11893);
\draw [c] (9.787,1.11893) -- (9.787,1.33232);
\draw [c] (9.77219,1.11893) -- (9.787,1.11893);
\draw [c] (9.787,1.11893) -- (9.80182,1.11893);
\definecolor{c}{rgb}{0,0,0};
\colorlet{c}{kugray};
\draw [c] (9.90555,0.596817) -- (9.90555,1.17321);
\draw [c] (9.90555,1.17321) -- (9.90555,1.3866);
\draw [c] (9.89073,1.17321) -- (9.90555,1.17321);
\draw [c] (9.90555,1.17321) -- (9.92036,1.17321);
\definecolor{c}{rgb}{0,0,0};
\colorlet{c}{kugray};
\draw [c] (9.93518,0.596817) -- (9.93518,1.17077);
\draw [c] (9.93518,1.17077) -- (9.93518,1.38417);
\draw [c] (9.92036,1.17077) -- (9.93518,1.17077);
\draw [c] (9.93518,1.17077) -- (9.95,1.17077);
\definecolor{c}{rgb}{0,0,0};
\colorlet{c}{natgreen};
\draw [c] (1.51655,5.54533) -- (1.60131,5.40419) -- (1.68607,5.2747) -- (1.77083,5.15478) -- (1.85558,5.04288) -- (1.94034,4.93778) -- (2.0251,4.83855) -- (2.10986,4.74443) -- (2.19462,4.65481) -- (2.27938,4.56919) -- (2.36413,4.48712)
 -- (2.44889,4.40827) -- (2.53365,4.33233) -- (2.61841,4.25902) -- (2.70317,4.18813) -- (2.78793,4.11945) -- (2.87268,4.05281) -- (2.95744,3.98806) -- (3.0422,3.92505) -- (3.12696,3.86367) -- (3.21172,3.80381) -- (3.29648,3.74536) --
 (3.38123,3.68824) -- (3.46599,3.63236) -- (3.55075,3.57765) -- (3.63551,3.52404) -- (3.72027,3.47148) -- (3.80502,3.4199) -- (3.88978,3.36925) -- (3.97454,3.31949) -- (4.0593,3.27056) -- (4.14406,3.22243) -- (4.22882,3.17506) -- (4.31357,3.12841) --
 (4.39833,3.08245) -- (4.48309,3.03714) -- (4.56785,2.99247) -- (4.65261,2.94839) -- (4.73737,2.90489) -- (4.82212,2.86194) -- (4.90688,2.81952) -- (4.99164,2.7776) -- (5.0764,2.73617) -- (5.16116,2.6952) -- (5.24592,2.65468) -- (5.33067,2.6146) --
 (5.41543,2.57492) -- (5.50019,2.53565) -- (5.58495,2.49676) -- (5.66971,2.45824);
\draw [c] (5.66971,2.45824) -- (5.75447,2.42007) -- (5.83922,2.38225) -- (5.92398,2.34476) -- (6.00874,2.30759) -- (6.0935,2.27073) -- (6.17826,2.23416) -- (6.26301,2.19789) -- (6.34777,2.16188) -- (6.43253,2.12615) --
 (6.51729,2.09068) -- (6.60205,2.05545) -- (6.68681,2.02047) -- (6.77156,1.98572) -- (6.85632,1.9512) -- (6.94108,1.91689) -- (7.02584,1.88279) -- (7.1106,1.8489) -- (7.19536,1.81521) -- (7.28011,1.78171) -- (7.36487,1.74839) -- (7.44963,1.71525) --
 (7.53439,1.68228) -- (7.61915,1.64948) -- (7.70391,1.61684) -- (7.78866,1.58435) -- (7.87342,1.55202) -- (7.95818,1.51983) -- (8.04294,1.48778) -- (8.1277,1.45586) -- (8.21246,1.42408) -- (8.29721,1.39242) -- (8.38197,1.36088) -- (8.46673,1.32946)
 -- (8.55149,1.29816) -- (8.63625,1.26696) -- (8.72101,1.23587) -- (8.80576,1.20487) -- (8.89052,1.17398) -- (8.97528,1.14317) -- (9.06004,1.11246) -- (9.1448,1.08183) -- (9.22955,1.05129) -- (9.31431,1.02082) -- (9.39907,0.990431) --
 (9.48383,0.960111) -- (9.56859,0.929859) -- (9.65335,0.899673) -- (9.7381,0.869549) -- (9.82286,0.839483);
\draw [c] (9.82286,0.839483) -- (9.90762,0.809473);
\definecolor{c}{rgb}{0,0,0};
\draw [c] (1,0.596817) -- (9.95,0.596817);
\draw [anchor= east] (9.95,0.0954907) node[color=c, rotate=0]{$M_{\gamma\gamma}\text{ [GeV]}$};
\draw [c] (1.02964,0.757062) -- (1.02964,0.596817);
\draw [c] (1.32599,0.67694) -- (1.32599,0.596817);
\draw [c] (1.62235,0.67694) -- (1.62235,0.596817);
\draw [c] (1.91871,0.67694) -- (1.91871,0.596817);
\draw [c] (2.21507,0.67694) -- (2.21507,0.596817);
\draw [c] (2.51142,0.757062) -- (2.51142,0.596817);
\draw [c] (2.80778,0.67694) -- (2.80778,0.596817);
\draw [c] (3.10414,0.67694) -- (3.10414,0.596817);
\draw [c] (3.4005,0.67694) -- (3.4005,0.596817);
\draw [c] (3.69685,0.67694) -- (3.69685,0.596817);
\draw [c] (3.99321,0.757062) -- (3.99321,0.596817);
\draw [c] (4.28957,0.67694) -- (4.28957,0.596817);
\draw [c] (4.58593,0.67694) -- (4.58593,0.596817);
\draw [c] (4.88228,0.67694) -- (4.88228,0.596817);
\draw [c] (5.17864,0.67694) -- (5.17864,0.596817);
\draw [c] (5.475,0.757062) -- (5.475,0.596817);
\draw [c] (5.77136,0.67694) -- (5.77136,0.596817);
\draw [c] (6.06772,0.67694) -- (6.06772,0.596817);
\draw [c] (6.36407,0.67694) -- (6.36407,0.596817);
\draw [c] (6.66043,0.67694) -- (6.66043,0.596817);
\draw [c] (6.95679,0.757062) -- (6.95679,0.596817);
\draw [c] (7.25315,0.67694) -- (7.25315,0.596817);
\draw [c] (7.5495,0.67694) -- (7.5495,0.596817);
\draw [c] (7.84586,0.67694) -- (7.84586,0.596817);
\draw [c] (8.14222,0.67694) -- (8.14222,0.596817);
\draw [c] (8.43858,0.757062) -- (8.43858,0.596817);
\draw [c] (8.73493,0.67694) -- (8.73493,0.596817);
\draw [c] (9.03129,0.67694) -- (9.03129,0.596817);
\draw [c] (9.32765,0.67694) -- (9.32765,0.596817);
\draw [c] (9.62401,0.67694) -- (9.62401,0.596817);
\draw [c] (9.92036,0.757062) -- (9.92036,0.596817);
\draw [c] (1.02964,0.757062) -- (1.02964,0.596817);
\draw [c] (9.92036,0.757062) -- (9.92036,0.596817);
\draw [anchor=base] (1.02964,0.310345) node[color=c, rotate=0]{0};
\draw [anchor=base] (2.51142,0.310345) node[color=c, rotate=0]{500};
\draw [anchor=base] (3.99321,0.310345) node[color=c, rotate=0]{1000};
\draw [anchor=base] (5.475,0.310345) node[color=c, rotate=0]{1500};
\draw [anchor=base] (6.95679,0.310345) node[color=c, rotate=0]{2000};
\draw [anchor=base] (8.43858,0.310345) node[color=c, rotate=0]{2500};
\draw [anchor=base] (9.92036,0.310345) node[color=c, rotate=0]{3000};
\draw [c] (1,0.596817) -- (1,5.90849);
\draw [anchor= east] (-0.12,5.90849) node[color=c, rotate=90]{Number of events};
\draw [c] (1.1335,0.632213) -- (1,0.632213);
\draw [c] (1.1335,0.668474) -- (1,0.668474);
\draw [c] (1.267,0.70091) -- (1,0.70091);
\draw [anchor= east] (0.922,0.70091) node[color=c, rotate=0]{$10^{-4}$};
\draw [c] (1.1335,0.9143) -- (1,0.9143);
\draw [c] (1.1335,1.03913) -- (1,1.03913);
\draw [c] (1.1335,1.12769) -- (1,1.12769);
\draw [c] (1.1335,1.19639) -- (1,1.19639);
\draw [c] (1.1335,1.25252) -- (1,1.25252);
\draw [c] (1.1335,1.29997) -- (1,1.29997);
\draw [c] (1.1335,1.34108) -- (1,1.34108);
\draw [c] (1.1335,1.37734) -- (1,1.37734);
\draw [c] (1.267,1.40978) -- (1,1.40978);
\draw [anchor= east] (0.922,1.40978) node[color=c, rotate=0]{$10^{-3}$};
\draw [c] (1.1335,1.62317) -- (1,1.62317);
\draw [c] (1.1335,1.74799) -- (1,1.74799);
\draw [c] (1.1335,1.83656) -- (1,1.83656);
\draw [c] (1.1335,1.90525) -- (1,1.90525);
\draw [c] (1.1335,1.96138) -- (1,1.96138);
\draw [c] (1.1335,2.00884) -- (1,2.00884);
\draw [c] (1.1335,2.04995) -- (1,2.04995);
\draw [c] (1.1335,2.08621) -- (1,2.08621);
\draw [c] (1.267,2.11864) -- (1,2.11864);
\draw [anchor= east] (0.922,2.11864) node[color=c, rotate=0]{$10^{-2}$};
\draw [c] (1.1335,2.33203) -- (1,2.33203);
\draw [c] (1.1335,2.45686) -- (1,2.45686);
\draw [c] (1.1335,2.54542) -- (1,2.54542);
\draw [c] (1.1335,2.61412) -- (1,2.61412);
\draw [c] (1.1335,2.67025) -- (1,2.67025);
\draw [c] (1.1335,2.71771) -- (1,2.71771);
\draw [c] (1.1335,2.75882) -- (1,2.75882);
\draw [c] (1.1335,2.79508) -- (1,2.79508);
\draw [c] (1.267,2.82751) -- (1,2.82751);
\draw [anchor= east] (0.922,2.82751) node[color=c, rotate=0]{$10^{-1}$};
\draw [c] (1.1335,3.0409) -- (1,3.0409);
\draw [c] (1.1335,3.16573) -- (1,3.16573);
\draw [c] (1.1335,3.25429) -- (1,3.25429);
\draw [c] (1.1335,3.32299) -- (1,3.32299);
\draw [c] (1.1335,3.37912) -- (1,3.37912);
\draw [c] (1.1335,3.42657) -- (1,3.42657);
\draw [c] (1.1335,3.46768) -- (1,3.46768);
\draw [c] (1.1335,3.50394) -- (1,3.50394);
\draw [c] (1.267,3.53638) -- (1,3.53638);
\draw [anchor= east] (0.922,3.53638) node[color=c, rotate=0]{1};
\draw [c] (1.1335,3.74977) -- (1,3.74977);
\draw [c] (1.1335,3.87459) -- (1,3.87459);
\draw [c] (1.1335,3.96316) -- (1,3.96316);
\draw [c] (1.1335,4.03186) -- (1,4.03186);
\draw [c] (1.1335,4.08799) -- (1,4.08799);
\draw [c] (1.1335,4.13544) -- (1,4.13544);
\draw [c] (1.1335,4.17655) -- (1,4.17655);
\draw [c] (1.1335,4.21281) -- (1,4.21281);
\draw [c] (1.267,4.24525) -- (1,4.24525);
\draw [anchor= east] (0.922,4.24525) node[color=c, rotate=0]{10};
\draw [c] (1.1335,4.45864) -- (1,4.45864);
\draw [c] (1.1335,4.58346) -- (1,4.58346);
\draw [c] (1.1335,4.67203) -- (1,4.67203);
\draw [c] (1.1335,4.74072) -- (1,4.74072);
\draw [c] (1.1335,4.79685) -- (1,4.79685);
\draw [c] (1.1335,4.84431) -- (1,4.84431);
\draw [c] (1.1335,4.88542) -- (1,4.88542);
\draw [c] (1.1335,4.92168) -- (1,4.92168);
\draw [c] (1.267,4.95411) -- (1,4.95411);
\draw [anchor= east] (0.922,4.95411) node[color=c, rotate=0]{$10^{2}$};
\draw [c] (1.1335,5.1675) -- (1,5.1675);
\draw [c] (1.1335,5.29233) -- (1,5.29233);
\draw [c] (1.1335,5.38089) -- (1,5.38089);
\draw [c] (1.1335,5.44959) -- (1,5.44959);
\draw [c] (1.1335,5.50572) -- (1,5.50572);
\draw [c] (1.1335,5.55318) -- (1,5.55318);
\draw [c] (1.1335,5.59428) -- (1,5.59428);
\draw [c] (1.1335,5.63055) -- (1,5.63055);
\draw [c] (1.267,5.66298) -- (1,5.66298);
\draw [anchor= east] (0.922,5.66298) node[color=c, rotate=0]{$10^{3}$};
\draw [c] (1.1335,5.87637) -- (1,5.87637);
\colorlet{c}{natgreen!40};
\draw [c, fill=c] (1.45951,0.596817) -- (1.48904,4.38696) -- (1.51857,5.54721) -- (1.54811,5.49618) -- (1.57764,5.44687) -- (1.60717,5.39917) -- (1.6367,5.35293) -- (1.66623,5.30805) -- (1.69576,5.26443) -- (1.7253,5.22198) -- (1.75483,5.18061) --
 (1.78436,5.14026) -- (1.81389,5.10086) -- (1.84342,5.06235) -- (1.87296,5.02469) -- (1.90249,4.98782) -- (1.93202,4.9517) -- (1.96155,4.91631) -- (1.99108,4.88159) -- (2.02061,4.84752) -- (2.05015,4.81408) -- (2.07968,4.78122) -- (2.10921,4.74894)
 -- (2.13874,4.71719) -- (2.16827,4.68597) -- (2.1978,4.65524) -- (2.22734,4.625) -- (2.25687,4.59522) -- (2.2864,4.56589) -- (2.31593,4.53698) -- (2.34546,4.50849) -- (2.375,4.48039) -- (2.40453,4.45268) -- (2.43406,4.42535) -- (2.46359,4.39838) --
 (2.49312,4.37175) -- (2.52265,4.34546) -- (2.55219,4.3195) -- (2.58172,4.29385) -- (2.61125,4.26852) -- (2.64078,4.24348) -- (2.67031,4.21873) -- (2.69984,4.19426) -- (2.72938,4.17006) -- (2.75891,4.14613) -- (2.78844,4.12245) -- (2.81797,4.09903)
 -- (2.8475,4.07584) -- (2.87704,4.0529) -- (2.90657,4.03018) -- (2.9361,4.00769) -- (2.96563,3.98541) -- (2.99516,3.96335) -- (3.02469,3.94149) -- (3.05423,3.91984) -- (3.08376,3.89838) -- (3.11329,3.87712) -- (3.14282,3.85604) -- (3.17235,3.83514)
 -- (3.20189,3.81442) -- (3.23142,3.79387) -- (3.26095,3.7735) -- (3.29048,3.75328) -- (3.32001,3.73323) -- (3.34954,3.71334) -- (3.37908,3.6936) -- (3.40861,3.67402) -- (3.43814,3.65458) -- (3.46767,3.63528) -- (3.4972,3.61613) -- (3.52673,3.59712)
 -- (3.55627,3.57824) -- (3.5858,3.55949) -- (3.61533,3.54087) -- (3.64486,3.52238) -- (3.67439,3.50402) -- (3.70393,3.48578) -- (3.73346,3.46765) -- (3.76299,3.44965) -- (3.79252,3.43176) -- (3.82205,3.41399) -- (3.85158,3.39632) --
 (3.88112,3.37876) -- (3.91065,3.36132) -- (3.94018,3.34397) -- (3.96971,3.32673) -- (3.99924,3.30959) -- (4.02877,3.29255) -- (4.05831,3.27561) -- (4.08784,3.25877) -- (4.11737,3.24202) -- (4.1469,3.22536) -- (4.17643,3.2088) -- (4.20597,3.19232) --
 (4.2355,3.17594) -- (4.26503,3.15964) -- (4.29456,3.14343) -- (4.32409,3.1273) -- (4.35362,3.11126) -- (4.38316,3.09529) -- (4.41269,3.07941) -- (4.44222,3.06361) -- (4.47175,3.04789) -- (4.50128,3.03225) -- (4.53082,3.01668) -- (4.56035,3.00119) --
 (4.58988,2.98577) -- (4.61941,2.97043) -- (4.64894,2.95515) -- (4.67847,2.93995) -- (4.70801,2.92482) -- (4.73754,2.90977) -- (4.76707,2.89477) -- (4.7966,2.87985) -- (4.82613,2.865) -- (4.85566,2.85021) -- (4.8852,2.83548) -- (4.91473,2.82083) --
 (4.94426,2.80623) -- (4.97379,2.7917) -- (5.00332,2.77723) -- (5.03286,2.76282) -- (5.06239,2.74848) -- (5.09192,2.73419) -- (5.12145,2.71996) -- (5.15098,2.70579) -- (5.18051,2.69168) -- (5.21005,2.67763) -- (5.23958,2.66364) -- (5.26911,2.6497) --
 (5.29864,2.63581) -- (5.32817,2.62198) -- (5.3577,2.60821) -- (5.38724,2.59448) -- (5.41677,2.58081) -- (5.4463,2.5672) -- (5.47583,2.55363) -- (5.50536,2.54011) -- (5.5349,2.52665) -- (5.56443,2.51323) -- (5.59396,2.49987) -- (5.62349,2.48655) --
 (5.65302,2.47328) -- (5.68255,2.46005) -- (5.71209,2.44688) -- (5.74162,2.43375) -- (5.77115,2.42066) -- (5.80068,2.40762) -- (5.83021,2.39462) -- (5.85975,2.38167) -- (5.88928,2.36876) -- (5.91881,2.3559) -- (5.94834,2.34307) -- (5.97787,2.33029)
 -- (6.0074,2.31755) -- (6.03694,2.30485) -- (6.06647,2.29219) -- (6.096,2.27957) -- (6.12553,2.26699) -- (6.15506,2.25444) -- (6.18459,2.24194) -- (6.21413,2.22947) -- (6.24366,2.21704) -- (6.27319,2.20465) -- (6.30272,2.1923) -- (6.33225,2.17998)
 -- (6.36179,2.16769) -- (6.39132,2.15544) -- (6.42085,2.14323) -- (6.45038,2.13105) -- (6.47991,2.1189) -- (6.50944,2.10679) -- (6.53898,2.09471) -- (6.56851,2.08267) -- (6.59804,2.07065) -- (6.62757,2.05867) -- (6.6571,2.04672) -- (6.68663,2.03481)
 -- (6.71617,2.02292) -- (6.7457,2.01106) -- (6.77523,1.99923) -- (6.80476,1.98744) -- (6.83429,1.97567) -- (6.86383,1.96393) -- (6.89336,1.95222) -- (6.92289,1.94054) -- (6.95242,1.92889) -- (6.98195,1.91727) -- (7.01148,1.90567) -- (7.04102,1.8941)
 -- (7.07055,1.88256) -- (7.10008,1.87104) -- (7.12961,1.85955) -- (7.15914,1.84809) -- (7.18867,1.83665) -- (7.21821,1.82524) -- (7.24774,1.81385) -- (7.27727,1.80249) -- (7.3068,1.79115) -- (7.33633,1.77984) -- (7.36587,1.76855) -- (7.3954,1.75728)
 -- (7.42493,1.74604) -- (7.45446,1.73482) -- (7.48399,1.72363) -- (7.51352,1.71245) -- (7.54306,1.7013) -- (7.57259,1.69018) -- (7.60212,1.67907) -- (7.63165,1.66798) -- (7.66118,1.65692) -- (7.69072,1.64588) -- (7.72025,1.63486) --
 (7.74978,1.62386) -- (7.77931,1.61288) -- (7.80884,1.60192) -- (7.83837,1.59098) -- (7.86791,1.58006) -- (7.89744,1.56916) -- (7.92697,1.55828) -- (7.9565,1.54742) -- (7.98603,1.53658) -- (8.01556,1.52576) -- (8.0451,1.51495) -- (8.07463,1.50417) --
 (8.10416,1.4934) -- (8.13369,1.48265) -- (8.16322,1.47191) -- (8.19276,1.4612) -- (8.22229,1.4505) -- (8.25182,1.43982) -- (8.28135,1.42915) -- (8.31088,1.41851) -- (8.34041,1.40788) -- (8.36995,1.39726) -- (8.39948,1.38666) -- (8.42901,1.37608) --
 (8.45854,1.36551) -- (8.48807,1.35496) -- (8.5176,1.34443) -- (8.54714,1.3339) -- (8.57667,1.3234) -- (8.6062,1.31291) -- (8.63573,1.30243) -- (8.66526,1.29197) -- (8.6948,1.28152) -- (8.72433,1.27109) -- (8.75386,1.26067) -- (8.78339,1.25026) --
 (8.81292,1.23987) -- (8.84245,1.22949) -- (8.87199,1.21912) -- (8.90152,1.20877) -- (8.93105,1.19843) -- (8.96058,1.1881) -- (8.99011,1.17779) -- (9.01964,1.16748) -- (9.04918,1.15719) -- (9.07871,1.14692) -- (9.10824,1.13665) -- (9.13777,1.12639)
 -- (9.1673,1.11615) -- (9.19684,1.10592) -- (9.22637,1.0957) -- (9.2559,1.08549) -- (9.28543,1.07529) -- (9.31496,1.06511) -- (9.34449,1.05493) -- (9.37403,1.04476) -- (9.40356,1.03461) -- (9.43309,1.02446) -- (9.46262,1.01433) -- (9.49215,1.0042)
 -- (9.52169,0.994085) -- (9.55122,0.983979) -- (9.58075,0.973882) -- (9.61028,0.963795) -- (9.63981,0.953717) -- (9.66934,0.943648) -- (9.69888,0.933588) -- (9.72841,0.923537) -- (9.75794,0.913495) -- (9.78747,0.90346) -- (9.817,0.893435) --
 (9.84653,0.883417) -- (9.87607,0.873408) -- (9.9056,0.810189) -- (9.93513,0.596817) -- (9.93513,0.596817) -- (9.9056,0.810189) -- (9.87607,0.756916) -- (9.84653,0.768025) -- (9.817,0.779137) -- (9.78747,0.790251) -- (9.75794,0.801367) --
 (9.72841,0.812487) -- (9.69888,0.823609) -- (9.66934,0.834734) -- (9.63981,0.845862) -- (9.61028,0.856994) -- (9.58075,0.868129) -- (9.55122,0.879268) -- (9.52169,0.89041) -- (9.49215,0.901557) -- (9.46262,0.912708) -- (9.43309,0.923863) --
 (9.40356,0.935022) -- (9.37403,0.946186) -- (9.34449,0.957355) -- (9.31496,0.968529) -- (9.28543,0.979708) -- (9.2559,0.990893) -- (9.22637,1.00208) -- (9.19684,1.01328) -- (9.1673,1.02448) -- (9.13777,1.03569) -- (9.10824,1.0469) --
 (9.07871,1.05812) -- (9.04918,1.06935) -- (9.01964,1.08058) -- (8.99011,1.09182) -- (8.96058,1.10306) -- (8.93105,1.11432) -- (8.90152,1.12558) -- (8.87199,1.13685) -- (8.84245,1.14813) -- (8.81292,1.15941) -- (8.78339,1.1707) -- (8.75386,1.182) --
 (8.72433,1.19331) -- (8.6948,1.20463) -- (8.66526,1.21596) -- (8.63573,1.22729) -- (8.6062,1.23863) -- (8.57667,1.24999) -- (8.54714,1.26135) -- (8.5176,1.27272) -- (8.48807,1.28411) -- (8.45854,1.2955) -- (8.42901,1.3069) -- (8.39948,1.31832) --
 (8.36995,1.32974) -- (8.34041,1.34118) -- (8.31088,1.35262) -- (8.28135,1.36408) -- (8.25182,1.37555) -- (8.22229,1.38703) -- (8.19276,1.39852) -- (8.16322,1.41003) -- (8.13369,1.42154) -- (8.10416,1.43307) -- (8.07463,1.44461) -- (8.0451,1.45617)
 -- (8.01556,1.46774) -- (7.98603,1.47932) -- (7.9565,1.49091) -- (7.92697,1.50252) -- (7.89744,1.51414) -- (7.86791,1.52578) -- (7.83837,1.53743) -- (7.80884,1.5491) -- (7.77931,1.56078) -- (7.74978,1.57247) -- (7.72025,1.58418) -- (7.69072,1.59591)
 -- (7.66118,1.60765) -- (7.63165,1.61941) -- (7.60212,1.63118) -- (7.57259,1.64297) -- (7.54306,1.65478) -- (7.51352,1.6666) -- (7.48399,1.67844) -- (7.45446,1.6903) -- (7.42493,1.70217) -- (7.3954,1.71407) -- (7.36587,1.72598) -- (7.33633,1.73791)
 -- (7.3068,1.74986) -- (7.27727,1.76183) -- (7.24774,1.77381) -- (7.21821,1.78582) -- (7.18867,1.79785) -- (7.15914,1.80989) -- (7.12961,1.82196) -- (7.10008,1.83405) -- (7.07055,1.84615) -- (7.04102,1.85828) -- (7.01148,1.87043) --
 (6.98195,1.88261) -- (6.95242,1.8948) -- (6.92289,1.90701) -- (6.89336,1.91925) -- (6.86383,1.93151) -- (6.83429,1.9438) -- (6.80476,1.95611) -- (6.77523,1.96844) -- (6.7457,1.9808) -- (6.71617,1.99318) -- (6.68663,2.00558) -- (6.6571,2.01801) --
 (6.62757,2.03047) -- (6.59804,2.04295) -- (6.56851,2.05545) -- (6.53898,2.06799) -- (6.50944,2.08055) -- (6.47991,2.09313) -- (6.45038,2.10575) -- (6.42085,2.11839) -- (6.39132,2.13106) -- (6.36179,2.14376) -- (6.33225,2.15649) -- (6.30272,2.16925)
 -- (6.27319,2.18203) -- (6.24366,2.19485) -- (6.21413,2.2077) -- (6.18459,2.22057) -- (6.15506,2.23348) -- (6.12553,2.24642) -- (6.096,2.2594) -- (6.06647,2.2724) -- (6.03694,2.28544) -- (6.0074,2.29851) -- (5.97787,2.31161) -- (5.94834,2.32475) --
 (5.91881,2.33793) -- (5.88928,2.35113) -- (5.85975,2.36438) -- (5.83021,2.37766) -- (5.80068,2.39098) -- (5.77115,2.40433) -- (5.74162,2.41772) -- (5.71209,2.43115) -- (5.68255,2.44462) -- (5.65302,2.45812) -- (5.62349,2.47167) -- (5.59396,2.48526)
 -- (5.56443,2.49888) -- (5.5349,2.51255) -- (5.50536,2.52626) -- (5.47583,2.54002) -- (5.4463,2.55381) -- (5.41677,2.56765) -- (5.38724,2.58154) -- (5.3577,2.59546) -- (5.32817,2.60944) -- (5.29864,2.62346) -- (5.26911,2.63753) -- (5.23958,2.65165)
 -- (5.21005,2.66581) -- (5.18051,2.68003) -- (5.15098,2.69429) -- (5.12145,2.70861) -- (5.09192,2.72297) -- (5.06239,2.73739) -- (5.03286,2.75187) -- (5.00332,2.7664) -- (4.97379,2.78098) -- (4.94426,2.79562) -- (4.91473,2.81032) -- (4.8852,2.82507)
 -- (4.85566,2.83989) -- (4.82613,2.85476) -- (4.7966,2.8697) -- (4.76707,2.8847) -- (4.73754,2.89976) -- (4.70801,2.91488) -- (4.67847,2.93008) -- (4.64894,2.94534) -- (4.61941,2.96066) -- (4.58988,2.97606) -- (4.56035,2.99153) -- (4.53082,3.00707)
 -- (4.50128,3.02268) -- (4.47175,3.03837) -- (4.44222,3.05413) -- (4.41269,3.06997) -- (4.38316,3.08589) -- (4.35362,3.10188) -- (4.32409,3.11796) -- (4.29456,3.13412) -- (4.26503,3.15037) -- (4.2355,3.1667) -- (4.20597,3.18312) -- (4.17643,3.19962)
 -- (4.1469,3.21622) -- (4.11737,3.23291) -- (4.08784,3.2497) -- (4.05831,3.26657) -- (4.02877,3.28355) -- (3.99924,3.30062) -- (3.96971,3.3178) -- (3.94018,3.33508) -- (3.91065,3.35246) -- (3.88112,3.36995) -- (3.85158,3.38755) -- (3.82205,3.40525)
 -- (3.79252,3.42308) -- (3.76299,3.44101) -- (3.73346,3.45906) -- (3.70393,3.47723) -- (3.67439,3.49553) -- (3.64486,3.51394) -- (3.61533,3.53249) -- (3.5858,3.55116) -- (3.55627,3.56996) -- (3.52673,3.5889) -- (3.4972,3.60797) -- (3.46767,3.62719)
 -- (3.43814,3.64654) -- (3.40861,3.66604) -- (3.37908,3.68569) -- (3.34954,3.70549) -- (3.32001,3.72545) -- (3.29048,3.74556) -- (3.26095,3.76584) -- (3.23142,3.78628) -- (3.20189,3.80689) -- (3.17235,3.82767) -- (3.14282,3.84863) --
 (3.11329,3.86977) -- (3.08376,3.8911) -- (3.05423,3.91261) -- (3.02469,3.93432) -- (2.99516,3.95623) -- (2.96563,3.97834) -- (2.9361,4.00066) -- (2.90657,4.0232) -- (2.87704,4.04595) -- (2.8475,4.06893) -- (2.81797,4.09214) -- (2.78844,4.11559) --
 (2.75891,4.13928) -- (2.72938,4.16323) -- (2.69984,4.18743) -- (2.67031,4.2119) -- (2.64078,4.23665) -- (2.61125,4.26168) -- (2.58172,4.287) -- (2.55219,4.31262) -- (2.52265,4.33855) -- (2.49312,4.3648) -- (2.46359,4.39139) -- (2.43406,4.41832) --
 (2.40453,4.44561) -- (2.375,4.47326) -- (2.34546,4.5013) -- (2.31593,4.52973) -- (2.2864,4.55858) -- (2.25687,4.58785) -- (2.22734,4.61757) -- (2.1978,4.64776) -- (2.16827,4.67842) -- (2.13874,4.70959) -- (2.10921,4.74129) -- (2.07968,4.77353) --
 (2.05015,4.80635) -- (2.02061,4.83976) -- (1.99108,4.87381) -- (1.96155,4.90851) -- (1.93202,4.9439) -- (1.90249,4.98002) -- (1.87296,5.0169) -- (1.84342,5.05458) -- (1.81389,5.0931) -- (1.78436,5.13251) -- (1.75483,5.17286) -- (1.7253,5.2142) --
 (1.69576,5.25657) -- (1.66623,5.30005) -- (1.6367,5.34468) -- (1.60717,5.39055) -- (1.57764,5.43771) -- (1.54811,5.48626) -- (1.51857,5.5363) -- (1.48904,0.596817) -- (1.45951,0.596817);
\definecolor{c}{rgb}{0,0,0};
\colorlet{c}{natcomp!40};
\draw [c, fill=c] (1.45951,0.596817) -- (1.48904,4.38693) -- (1.51857,5.54742) -- (1.54811,5.49642) -- (1.57764,5.44715) -- (1.60717,5.39949) -- (1.6367,5.3533) -- (1.66623,5.30848) -- (1.69576,5.26492) -- (1.7253,5.22253) -- (1.75483,5.18123) --
 (1.78436,5.14096) -- (1.81389,5.10164) -- (1.84342,5.06323) -- (1.87296,5.02567) -- (1.90249,4.98891) -- (1.93202,4.95292) -- (1.96155,4.91766) -- (1.99108,4.88309) -- (2.02061,4.84917) -- (2.05015,4.8159) -- (2.07968,4.78322) -- (2.10921,4.75113)
 -- (2.13874,4.7196) -- (2.16827,4.6886) -- (2.1978,4.65812) -- (2.22734,4.62814) -- (2.25687,4.59863) -- (2.2864,4.56959) -- (2.31593,4.541) -- (2.34546,4.51285) -- (2.375,4.48512) -- (2.40453,4.45779) -- (2.43406,4.43087) -- (2.46359,4.40432) --
 (2.49312,4.37816) -- (2.52265,4.35236) -- (2.55219,4.32691) -- (2.58172,4.30181) -- (2.61125,4.27705) -- (2.64078,4.25262) -- (2.67031,4.22852) -- (2.69984,4.20473) -- (2.72938,4.18125) -- (2.75891,4.15807) -- (2.78844,4.13519) -- (2.81797,4.1126)
 -- (2.8475,4.0903) -- (2.87704,4.06828) -- (2.90657,4.04653) -- (2.9361,4.02506) -- (2.96563,4.00385) -- (2.99516,3.98291) -- (3.02469,3.96222) -- (3.05423,3.94179) -- (3.08376,3.92162) -- (3.11329,3.90169) -- (3.14282,3.88201) -- (3.17235,3.86257)
 -- (3.20189,3.84337) -- (3.23142,3.82441) -- (3.26095,3.80568) -- (3.29048,3.78719) -- (3.32001,3.76893) -- (3.34954,3.7509) -- (3.37908,3.7331) -- (3.40861,3.71552) -- (3.43814,3.69817) -- (3.46767,3.68103) -- (3.4972,3.66412) -- (3.52673,3.64743)
 -- (3.55627,3.63095) -- (3.5858,3.6147) -- (3.61533,3.59865) -- (3.64486,3.58282) -- (3.67439,3.5672) -- (3.70393,3.5518) -- (3.73346,3.5366) -- (3.76299,3.52161) -- (3.79252,3.50683) -- (3.82205,3.49226) -- (3.85158,3.47789) -- (3.88112,3.46372) --
 (3.91065,3.44975) -- (3.94018,3.43599) -- (3.96971,3.42242) -- (3.99924,3.40905) -- (4.02877,3.39587) -- (4.05831,3.3829) -- (4.08784,3.37011) -- (4.11737,3.35751) -- (4.1469,3.34511) -- (4.17643,3.33289) -- (4.20597,3.32085) -- (4.2355,3.309) --
 (4.26503,3.29733) -- (4.29456,3.28584) -- (4.32409,3.27453) -- (4.35362,3.26339) -- (4.38316,3.25243) -- (4.41269,3.24163) -- (4.44222,3.23101) -- (4.47175,3.22055) -- (4.50128,3.21026) -- (4.53082,3.20013) -- (4.56035,3.19015) -- (4.58988,3.18034)
 -- (4.61941,3.17067) -- (4.64894,3.16116) -- (4.67847,3.1518) -- (4.70801,3.14259) -- (4.73754,3.13352) -- (4.76707,3.12459) -- (4.7966,3.1158) -- (4.82613,3.10714) -- (4.85566,3.09862) -- (4.8852,3.09023) -- (4.91473,3.08197) -- (4.94426,3.07384)
 -- (4.97379,3.06583) -- (5.00332,3.05794) -- (5.03286,3.05016) -- (5.06239,3.04251) -- (5.09192,3.03496) -- (5.12145,3.02753) -- (5.15098,3.02021) -- (5.18051,3.01299) -- (5.21005,3.00587) -- (5.23958,2.99886) -- (5.26911,2.99194) --
 (5.29864,2.98512) -- (5.32817,2.97839) -- (5.3577,2.97176) -- (5.38724,2.96521) -- (5.41677,2.95876) -- (5.4463,2.95238) -- (5.47583,2.94609) -- (5.50536,2.93988) -- (5.5349,2.93375) -- (5.56443,2.9277) -- (5.59396,2.92172) -- (5.62349,2.91582) --
 (5.65302,2.90998) -- (5.68255,2.90422) -- (5.71209,2.89852) -- (5.74162,2.89289) -- (5.77115,2.88732) -- (5.80068,2.88181) -- (5.83021,2.87636) -- (5.85975,2.87098) -- (5.88928,2.86565) -- (5.91881,2.86037) -- (5.94834,2.85515) -- (5.97787,2.84999)
 -- (6.0074,2.84487) -- (6.03694,2.83981) -- (6.06647,2.83479) -- (6.096,2.82982) -- (6.12553,2.8249) -- (6.15506,2.82002) -- (6.18459,2.81519) -- (6.21413,2.8104) -- (6.24366,2.80565) -- (6.27319,2.80094) -- (6.30272,2.79627) -- (6.33225,2.79163) --
 (6.36179,2.78704) -- (6.39132,2.78248) -- (6.42085,2.77796) -- (6.45038,2.77347) -- (6.47991,2.76901) -- (6.50944,2.76459) -- (6.53898,2.7602) -- (6.56851,2.75583) -- (6.59804,2.7515) -- (6.62757,2.7472) -- (6.6571,2.74292) -- (6.68663,2.73868) --
 (6.71617,2.73446) -- (6.7457,2.73026) -- (6.77523,2.72609) -- (6.80476,2.72195) -- (6.83429,2.71783) -- (6.86383,2.71373) -- (6.89336,2.70966) -- (6.92289,2.7056) -- (6.95242,2.70157) -- (6.98195,2.69756) -- (7.01148,2.69357) -- (7.04102,2.6896) --
 (7.07055,2.68565) -- (7.10008,2.68172) -- (7.12961,2.6778) -- (7.15914,2.67391) -- (7.18867,2.67003) -- (7.21821,2.66616) -- (7.24774,2.66231) -- (7.27727,2.65848) -- (7.3068,2.65467) -- (7.33633,2.65086) -- (7.36587,2.64708) -- (7.3954,2.6433) --
 (7.42493,2.63954) -- (7.45446,2.6358) -- (7.48399,2.63206) -- (7.51352,2.62834) -- (7.54306,2.62463) -- (7.57259,2.62093) -- (7.60212,2.61724) -- (7.63165,2.61357) -- (7.66118,2.6099) -- (7.69072,2.60625) -- (7.72025,2.6026) -- (7.74978,2.59897) --
 (7.77931,2.59534) -- (7.80884,2.59173) -- (7.83837,2.58812) -- (7.86791,2.58452) -- (7.89744,2.58093) -- (7.92697,2.57735) -- (7.9565,2.57377) -- (7.98603,2.5702) -- (8.01556,2.56664) -- (8.0451,2.56309) -- (8.07463,2.55954) -- (8.10416,2.556) --
 (8.13369,2.55247) -- (8.16322,2.54894) -- (8.19276,2.54541) -- (8.22229,2.5419) -- (8.25182,2.53838) -- (8.28135,2.53488) -- (8.31088,2.53138) -- (8.34041,2.52788) -- (8.36995,2.52438) -- (8.39948,2.52089) -- (8.42901,2.51741) -- (8.45854,2.51392)
 -- (8.48807,2.51045) -- (8.5176,2.50697) -- (8.54714,2.50349) -- (8.57667,2.50002) -- (8.6062,2.49655) -- (8.63573,2.49309) -- (8.66526,2.48962) -- (8.6948,2.48616) -- (8.72433,2.48269) -- (8.75386,2.47923) -- (8.78339,2.47577) -- (8.81292,2.47231)
 -- (8.84245,2.46884) -- (8.87199,2.46538) -- (8.90152,2.46192) -- (8.93105,2.45845) -- (8.96058,2.45499) -- (8.99011,2.45152) -- (9.01964,2.44805) -- (9.04918,2.44457) -- (9.07871,2.4411) -- (9.10824,2.43762) -- (9.13777,2.43413) -- (9.1673,2.43064)
 -- (9.19684,2.42715) -- (9.22637,2.42365) -- (9.2559,2.42015) -- (9.28543,2.41664) -- (9.31496,2.41313) -- (9.34449,2.4096) -- (9.37403,2.40607) -- (9.40356,2.40253) -- (9.43309,2.39899) -- (9.46262,2.39543) -- (9.49215,2.39187) -- (9.52169,2.3883)
 -- (9.55122,2.38471) -- (9.58075,2.38112) -- (9.61028,2.37752) -- (9.63981,2.3739) -- (9.66934,2.37027) -- (9.69888,2.36663) -- (9.72841,2.36298) -- (9.75794,2.35931) -- (9.78747,2.35564) -- (9.817,2.35194) -- (9.84653,2.34824) -- (9.87607,2.34452)
 -- (9.9056,2.32667) -- (9.93513,0.596817) -- (9.93513,0.596817) -- (9.9056,2.32667) -- (9.87607,2.31564) -- (9.84653,2.3194) -- (9.817,2.32313) -- (9.78747,2.32685) -- (9.75794,2.33056) -- (9.72841,2.33425) -- (9.69888,2.33793) -- (9.66934,2.3416)
 -- (9.63981,2.34526) -- (9.61028,2.34891) -- (9.58075,2.35256) -- (9.55122,2.35619) -- (9.52169,2.35983) -- (9.49215,2.36346) -- (9.46262,2.36708) -- (9.43309,2.3707) -- (9.40356,2.37432) -- (9.37403,2.37794) -- (9.34449,2.38155) --
 (9.31496,2.38516) -- (9.28543,2.38878) -- (9.2559,2.39239) -- (9.22637,2.396) -- (9.19684,2.39962) -- (9.1673,2.40323) -- (9.13777,2.40684) -- (9.10824,2.41046) -- (9.07871,2.41408) -- (9.04918,2.4177) -- (9.01964,2.42132) -- (8.99011,2.42494) --
 (8.96058,2.42857) -- (8.93105,2.4322) -- (8.90152,2.43583) -- (8.87199,2.43946) -- (8.84245,2.4431) -- (8.81292,2.44673) -- (8.78339,2.45038) -- (8.75386,2.45402) -- (8.72433,2.45767) -- (8.6948,2.46132) -- (8.66526,2.46498) -- (8.63573,2.46863) --
 (8.6062,2.4723) -- (8.57667,2.47596) -- (8.54714,2.47963) -- (8.5176,2.4833) -- (8.48807,2.48698) -- (8.45854,2.49066) -- (8.42901,2.49435) -- (8.39948,2.49804) -- (8.36995,2.50173) -- (8.34041,2.50543) -- (8.31088,2.50913) -- (8.28135,2.51284) --
 (8.25182,2.51655) -- (8.22229,2.52027) -- (8.19276,2.52399) -- (8.16322,2.52771) -- (8.13369,2.53145) -- (8.10416,2.53518) -- (8.07463,2.53892) -- (8.0451,2.54267) -- (8.01556,2.54643) -- (7.98603,2.55019) -- (7.9565,2.55395) -- (7.92697,2.55772) --
 (7.89744,2.5615) -- (7.86791,2.56528) -- (7.83837,2.56907) -- (7.80884,2.57287) -- (7.77931,2.57668) -- (7.74978,2.58049) -- (7.72025,2.58431) -- (7.69072,2.58814) -- (7.66118,2.59197) -- (7.63165,2.59581) -- (7.60212,2.59967) -- (7.57259,2.60353)
 -- (7.54306,2.60739) -- (7.51352,2.61127) -- (7.48399,2.61516) -- (7.45446,2.61906) -- (7.42493,2.62296) -- (7.3954,2.62688) -- (7.36587,2.63081) -- (7.33633,2.63475) -- (7.3068,2.6387) -- (7.27727,2.64266) -- (7.24774,2.64664) -- (7.21821,2.65063)
 -- (7.18867,2.65463) -- (7.15914,2.65864) -- (7.12961,2.66267) -- (7.10008,2.66671) -- (7.07055,2.67076) -- (7.04102,2.67483) -- (7.01148,2.67892) -- (6.98195,2.68302) -- (6.95242,2.68714) -- (6.92289,2.69128) -- (6.89336,2.69543) --
 (6.86383,2.6996) -- (6.83429,2.7038) -- (6.80476,2.70801) -- (6.77523,2.71224) -- (6.7457,2.71649) -- (6.71617,2.72077) -- (6.68663,2.72506) -- (6.6571,2.72938) -- (6.62757,2.73373) -- (6.59804,2.7381) -- (6.56851,2.74249) -- (6.53898,2.74691) --
 (6.50944,2.75136) -- (6.47991,2.75584) -- (6.45038,2.76034) -- (6.42085,2.76488) -- (6.39132,2.76945) -- (6.36179,2.77405) -- (6.33225,2.77868) -- (6.30272,2.78335) -- (6.27319,2.78805) -- (6.24366,2.79279) -- (6.21413,2.79757) -- (6.18459,2.80239)
 -- (6.15506,2.80725) -- (6.12553,2.81215) -- (6.096,2.81709) -- (6.06647,2.82208) -- (6.03694,2.82711) -- (6.0074,2.8322) -- (5.97787,2.83733) -- (5.94834,2.84251) -- (5.91881,2.84774) -- (5.88928,2.85303) -- (5.85975,2.85838) -- (5.83021,2.86378)
 -- (5.80068,2.86924) -- (5.77115,2.87476) -- (5.74162,2.88034) -- (5.71209,2.88599) -- (5.68255,2.89171) -- (5.65302,2.89749) -- (5.62349,2.90334) -- (5.59396,2.90927) -- (5.56443,2.91527) -- (5.5349,2.92134) -- (5.50536,2.9275) -- (5.47583,2.93373)
 -- (5.4463,2.94005) -- (5.41677,2.94645) -- (5.38724,2.95294) -- (5.3577,2.95952) -- (5.32817,2.96619) -- (5.29864,2.97296) -- (5.26911,2.97982) -- (5.23958,2.98678) -- (5.21005,2.99385) -- (5.18051,3.00101) -- (5.15098,3.00828) -- (5.12145,3.01567)
 -- (5.09192,3.02316) -- (5.06239,3.03076) -- (5.03286,3.03849) -- (5.00332,3.04633) -- (4.97379,3.05429) -- (4.94426,3.06238) -- (4.91473,3.07059) -- (4.8852,3.07893) -- (4.85566,3.0874) -- (4.82613,3.09601) -- (4.7966,3.10475) -- (4.76707,3.11363)
 -- (4.73754,3.12265) -- (4.70801,3.13182) -- (4.67847,3.14113) -- (4.64894,3.15059) -- (4.61941,3.1602) -- (4.58988,3.16997) -- (4.56035,3.17988) -- (4.53082,3.18996) -- (4.50128,3.2002) -- (4.47175,3.2106) -- (4.44222,3.22116) -- (4.41269,3.23189)
 -- (4.38316,3.24279) -- (4.35362,3.25386) -- (4.32409,3.26511) -- (4.29456,3.27652) -- (4.26503,3.28812) -- (4.2355,3.29989) -- (4.20597,3.31184) -- (4.17643,3.32397) -- (4.1469,3.33629) -- (4.11737,3.34879) -- (4.08784,3.36148) -- (4.05831,3.37436)
 -- (4.02877,3.38743) -- (3.99924,3.40069) -- (3.96971,3.41414) -- (3.94018,3.42779) -- (3.91065,3.44164) -- (3.88112,3.45568) -- (3.85158,3.46992) -- (3.82205,3.48436) -- (3.79252,3.49901) -- (3.76299,3.51386) -- (3.73346,3.52891) --
 (3.70393,3.54417) -- (3.67439,3.55963) -- (3.64486,3.57531) -- (3.61533,3.5912) -- (3.5858,3.60729) -- (3.55627,3.6236) -- (3.52673,3.64013) -- (3.4972,3.65687) -- (3.46767,3.67383) -- (3.43814,3.69101) -- (3.40861,3.70841) -- (3.37908,3.72603) --
 (3.34954,3.74388) -- (3.32001,3.76195) -- (3.29048,3.78025) -- (3.26095,3.79878) -- (3.23142,3.81755) -- (3.20189,3.83654) -- (3.17235,3.85578) -- (3.14282,3.87525) -- (3.11329,3.89496) -- (3.08376,3.91492) -- (3.05423,3.93513) -- (3.02469,3.95559)
 -- (2.99516,3.97629) -- (2.96563,3.99726) -- (2.9361,4.01849) -- (2.90657,4.03998) -- (2.87704,4.06173) -- (2.8475,4.08376) -- (2.81797,4.10607) -- (2.78844,4.12866) -- (2.75891,4.15154) -- (2.72938,4.1747) -- (2.69984,4.19817) -- (2.67031,4.22194)
 -- (2.64078,4.24602) -- (2.61125,4.27042) -- (2.58172,4.29515) -- (2.55219,4.32021) -- (2.52265,4.34561) -- (2.49312,4.37136) -- (2.46359,4.39747) -- (2.43406,4.42396) -- (2.40453,4.45083) -- (2.375,4.47809) -- (2.34546,4.50575) -- (2.31593,4.53384)
 -- (2.2864,4.56236) -- (2.25687,4.59133) -- (2.22734,4.62077) -- (2.1978,4.65069) -- (2.16827,4.68111) -- (2.13874,4.71204) -- (2.10921,4.74352) -- (2.07968,4.77557) -- (2.05015,4.8082) -- (2.02061,4.84144) -- (1.99108,4.87533) -- (1.96155,4.90988)
 -- (1.93202,4.94514) -- (1.90249,4.98113) -- (1.87296,5.0179) -- (1.84342,5.05547) -- (1.81389,5.0939) -- (1.78436,5.13322) -- (1.75483,5.17349) -- (1.7253,5.21475) -- (1.69576,5.25706) -- (1.66623,5.30048) -- (1.6367,5.34506) -- (1.60717,5.39088)
 -- (1.57764,5.438) -- (1.54811,5.48651) -- (1.51857,5.53651) -- (1.48904,0.596817) -- (1.45951,0.596817);
\definecolor{c}{rgb}{0,0,0};
\colorlet{c}{natblue!40};
\draw [c, fill=c] (1.45951,0.596817) -- (1.48904,4.38722) -- (1.51857,5.54801) -- (1.54811,5.4971) -- (1.57764,5.44794) -- (1.60717,5.40039) -- (1.6367,5.35433) -- (1.66623,5.30964) -- (1.69576,5.26623) -- (1.7253,5.22401) -- (1.75483,5.1829) --
 (1.78436,5.14283) -- (1.81389,5.10373) -- (1.84342,5.06555) -- (1.87296,5.02825) -- (1.90249,4.99178) -- (1.93202,4.95609) -- (1.96155,4.92115) -- (1.99108,4.88693) -- (2.02061,4.85341) -- (2.05015,4.82054) -- (2.07968,4.78831) -- (2.10921,4.75669)
 -- (2.13874,4.72567) -- (2.16827,4.69522) -- (2.1978,4.66532) -- (2.22734,4.63596) -- (2.25687,4.60712) -- (2.2864,4.5788) -- (2.31593,4.55096) -- (2.34546,4.52361) -- (2.375,4.49674) -- (2.40453,4.47032) -- (2.43406,4.44436) -- (2.46359,4.41883) --
 (2.49312,4.39374) -- (2.52265,4.36908) -- (2.55219,4.34483) -- (2.58172,4.321) -- (2.61125,4.29757) -- (2.64078,4.27454) -- (2.67031,4.25189) -- (2.69984,4.22964) -- (2.72938,4.20776) -- (2.75891,4.18626) -- (2.78844,4.16512) -- (2.81797,4.14436) --
 (2.8475,4.12395) -- (2.87704,4.10389) -- (2.90657,4.08419) -- (2.9361,4.06484) -- (2.96563,4.04583) -- (2.99516,4.02716) -- (3.02469,4.00883) -- (3.05423,3.99083) -- (3.08376,3.97316) -- (3.11329,3.95582) -- (3.14282,3.93881) -- (3.17235,3.92211) --
 (3.20189,3.90573) -- (3.23142,3.88967) -- (3.26095,3.87392) -- (3.29048,3.85848) -- (3.32001,3.84335) -- (3.34954,3.82852) -- (3.37908,3.81399) -- (3.40861,3.79976) -- (3.43814,3.78582) -- (3.46767,3.77218) -- (3.4972,3.75882) -- (3.52673,3.74575)
 -- (3.55627,3.73297) -- (3.5858,3.72046) -- (3.61533,3.70823) -- (3.64486,3.69627) -- (3.67439,3.68459) -- (3.70393,3.67317) -- (3.73346,3.66201) -- (3.76299,3.65112) -- (3.79252,3.64047) -- (3.82205,3.63009) -- (3.85158,3.61995) --
 (3.88112,3.61005) -- (3.91065,3.6004) -- (3.94018,3.59099) -- (3.96971,3.58181) -- (3.99924,3.57285) -- (4.02877,3.56413) -- (4.05831,3.55563) -- (4.08784,3.54735) -- (4.11737,3.53928) -- (4.1469,3.53142) -- (4.17643,3.52378) -- (4.20597,3.51633) --
 (4.2355,3.50908) -- (4.26503,3.50203) -- (4.29456,3.49518) -- (4.32409,3.48851) -- (4.35362,3.48202) -- (4.38316,3.47572) -- (4.41269,3.46959) -- (4.44222,3.46364) -- (4.47175,3.45785) -- (4.50128,3.45223) -- (4.53082,3.44678) -- (4.56035,3.44148)
 -- (4.58988,3.43634) -- (4.61941,3.43135) -- (4.64894,3.42651) -- (4.67847,3.42181) -- (4.70801,3.41726) -- (4.73754,3.41284) -- (4.76707,3.40856) -- (4.7966,3.40441) -- (4.82613,3.40039) -- (4.85566,3.39649) -- (4.8852,3.39271) -- (4.91473,3.38906)
 -- (4.94426,3.38552) -- (4.97379,3.38209) -- (5.00332,3.37877) -- (5.03286,3.37555) -- (5.06239,3.37244) -- (5.09192,3.36942) -- (5.12145,3.3665) -- (5.15098,3.36368) -- (5.18051,3.36094) -- (5.21005,3.35829) -- (5.23958,3.35573) --
 (5.26911,3.35324) -- (5.29864,3.35084) -- (5.32817,3.34851) -- (5.3577,3.34625) -- (5.38724,3.34406) -- (5.41677,3.34193) -- (5.4463,3.33987) -- (5.47583,3.33787) -- (5.50536,3.33593) -- (5.5349,3.33404) -- (5.56443,3.33221) -- (5.59396,3.33043) --
 (5.62349,3.3287) -- (5.65302,3.32701) -- (5.68255,3.32537) -- (5.71209,3.32377) -- (5.74162,3.32221) -- (5.77115,3.32069) -- (5.80068,3.3192) -- (5.83021,3.31775) -- (5.85975,3.31633) -- (5.88928,3.31494) -- (5.91881,3.31358) -- (5.94834,3.31224) --
 (5.97787,3.31094) -- (6.0074,3.30965) -- (6.03694,3.30838) -- (6.06647,3.30714) -- (6.096,3.30591) -- (6.12553,3.30471) -- (6.15506,3.30351) -- (6.18459,3.30234) -- (6.21413,3.30117) -- (6.24366,3.30002) -- (6.27319,3.29888) -- (6.30272,3.29775) --
 (6.33225,3.29663) -- (6.36179,3.29551) -- (6.39132,3.2944) -- (6.42085,3.2933) -- (6.45038,3.2922) -- (6.47991,3.2911) -- (6.50944,3.29001) -- (6.53898,3.28892) -- (6.56851,3.28783) -- (6.59804,3.28673) -- (6.62757,3.28564) -- (6.6571,3.28455) --
 (6.68663,3.28345) -- (6.71617,3.28235) -- (6.7457,3.28124) -- (6.77523,3.28013) -- (6.80476,3.27902) -- (6.83429,3.2779) -- (6.86383,3.27677) -- (6.89336,3.27563) -- (6.92289,3.27449) -- (6.95242,3.27334) -- (6.98195,3.27217) -- (7.01148,3.271) --
 (7.04102,3.26982) -- (7.07055,3.26863) -- (7.10008,3.26742) -- (7.12961,3.2662) -- (7.15914,3.26497) -- (7.18867,3.26373) -- (7.21821,3.26248) -- (7.24774,3.26121) -- (7.27727,3.25992) -- (7.3068,3.25862) -- (7.33633,3.25731) -- (7.36587,3.25598) --
 (7.3954,3.25464) -- (7.42493,3.25327) -- (7.45446,3.25189) -- (7.48399,3.2505) -- (7.51352,3.24908) -- (7.54306,3.24765) -- (7.57259,3.2462) -- (7.60212,3.24473) -- (7.63165,3.24324) -- (7.66118,3.24173) -- (7.69072,3.2402) -- (7.72025,3.23865) --
 (7.74978,3.23708) -- (7.77931,3.23549) -- (7.80884,3.23388) -- (7.83837,3.23224) -- (7.86791,3.23059) -- (7.89744,3.22891) -- (7.92697,3.2272) -- (7.9565,3.22548) -- (7.98603,3.22373) -- (8.01556,3.22195) -- (8.0451,3.22015) -- (8.07463,3.21833) --
 (8.10416,3.21648) -- (8.13369,3.2146) -- (8.16322,3.2127) -- (8.19276,3.21078) -- (8.22229,3.20882) -- (8.25182,3.20684) -- (8.28135,3.20484) -- (8.31088,3.2028) -- (8.34041,3.20074) -- (8.36995,3.19865) -- (8.39948,3.19653) -- (8.42901,3.19439) --
 (8.45854,3.19221) -- (8.48807,3.19) -- (8.5176,3.18777) -- (8.54714,3.18551) -- (8.57667,3.18321) -- (8.6062,3.18089) -- (8.63573,3.17854) -- (8.66526,3.17615) -- (8.6948,3.17374) -- (8.72433,3.17129) -- (8.75386,3.16881) -- (8.78339,3.1663) --
 (8.81292,3.16376) -- (8.84245,3.16119) -- (8.87199,3.15859) -- (8.90152,3.15595) -- (8.93105,3.15328) -- (8.96058,3.15058) -- (8.99011,3.14785) -- (9.01964,3.14508) -- (9.04918,3.14228) -- (9.07871,3.13945) -- (9.10824,3.13658) -- (9.13777,3.13368)
 -- (9.1673,3.13075) -- (9.19684,3.12779) -- (9.22637,3.12479) -- (9.2559,3.12176) -- (9.28543,3.11869) -- (9.31496,3.11559) -- (9.34449,3.11245) -- (9.37403,3.10928) -- (9.40356,3.10608) -- (9.43309,3.10285) -- (9.46262,3.09957) -- (9.49215,3.09627)
 -- (9.52169,3.09293) -- (9.55122,3.08955) -- (9.58075,3.08615) -- (9.61028,3.0827) -- (9.63981,3.07922) -- (9.66934,3.07571) -- (9.69888,3.07217) -- (9.72841,3.06858) -- (9.75794,3.06497) -- (9.78747,3.06132) -- (9.817,3.05763) -- (9.84653,3.05391)
 -- (9.87607,3.05016) -- (9.9056,3.03778) -- (9.93513,0.596817) -- (9.93513,0.596817) -- (9.9056,3.03778) -- (9.87607,3.03276) -- (9.84653,3.03655) -- (9.817,3.0403) -- (9.78747,3.04402) -- (9.75794,3.04771) -- (9.72841,3.05136) -- (9.69888,3.05498)
 -- (9.66934,3.05857) -- (9.63981,3.06212) -- (9.61028,3.06564) -- (9.58075,3.06912) -- (9.55122,3.07257) -- (9.52169,3.07599) -- (9.49215,3.07937) -- (9.46262,3.08273) -- (9.43309,3.08605) -- (9.40356,3.08933) -- (9.37403,3.09258) --
 (9.34449,3.0958) -- (9.31496,3.09899) -- (9.28543,3.10215) -- (9.2559,3.10527) -- (9.22637,3.10836) -- (9.19684,3.11142) -- (9.1673,3.11445) -- (9.13777,3.11744) -- (9.10824,3.1204) -- (9.07871,3.12333) -- (9.04918,3.12623) -- (9.01964,3.1291) --
 (8.99011,3.13194) -- (8.96058,3.13474) -- (8.93105,3.13751) -- (8.90152,3.14025) -- (8.87199,3.14296) -- (8.84245,3.14564) -- (8.81292,3.14829) -- (8.78339,3.15091) -- (8.75386,3.1535) -- (8.72433,3.15605) -- (8.6948,3.15858) -- (8.66526,3.16107) --
 (8.63573,3.16354) -- (8.6062,3.16597) -- (8.57667,3.16837) -- (8.54714,3.17075) -- (8.5176,3.17309) -- (8.48807,3.1754) -- (8.45854,3.17769) -- (8.42901,3.17994) -- (8.39948,3.18216) -- (8.36995,3.18436) -- (8.34041,3.18653) -- (8.31088,3.18866) --
 (8.28135,3.19077) -- (8.25182,3.19285) -- (8.22229,3.1949) -- (8.19276,3.19692) -- (8.16322,3.19891) -- (8.13369,3.20088) -- (8.10416,3.20281) -- (8.07463,3.20472) -- (8.0451,3.2066) -- (8.01556,3.20845) -- (7.98603,3.21028) -- (7.9565,3.21208) --
 (7.92697,3.21385) -- (7.89744,3.21559) -- (7.86791,3.21731) -- (7.83837,3.219) -- (7.80884,3.22067) -- (7.77931,3.22231) -- (7.74978,3.22392) -- (7.72025,3.22552) -- (7.69072,3.22708) -- (7.66118,3.22862) -- (7.63165,3.23014) -- (7.60212,3.23163) --
 (7.57259,3.2331) -- (7.54306,3.23455) -- (7.51352,3.23598) -- (7.48399,3.23738) -- (7.45446,3.23876) -- (7.42493,3.24012) -- (7.3954,3.24146) -- (7.36587,3.24279) -- (7.33633,3.24409) -- (7.3068,3.24537) -- (7.27727,3.24663) -- (7.24774,3.24788) --
 (7.21821,3.24911) -- (7.18867,3.25032) -- (7.15914,3.25152) -- (7.12961,3.25271) -- (7.10008,3.25387) -- (7.07055,3.25503) -- (7.04102,3.25617) -- (7.01148,3.2573) -- (6.98195,3.25842) -- (6.95242,3.25953) -- (6.92289,3.26062) -- (6.89336,3.26171)
 -- (6.86383,3.26279) -- (6.83429,3.26387) -- (6.80476,3.26493) -- (6.77523,3.266) -- (6.7457,3.26705) -- (6.71617,3.26811) -- (6.68663,3.26916) -- (6.6571,3.2702) -- (6.62757,3.27125) -- (6.59804,3.2723) -- (6.56851,3.27335) -- (6.53898,3.2744) --
 (6.50944,3.27546) -- (6.47991,3.27652) -- (6.45038,3.27758) -- (6.42085,3.27865) -- (6.39132,3.27973) -- (6.36179,3.28082) -- (6.33225,3.28192) -- (6.30272,3.28303) -- (6.27319,3.28415) -- (6.24366,3.28529) -- (6.21413,3.28644) -- (6.18459,3.28761)
 -- (6.15506,3.2888) -- (6.12553,3.29) -- (6.096,3.29123) -- (6.06647,3.29247) -- (6.03694,3.29374) -- (6.0074,3.29504) -- (5.97787,3.29636) -- (5.94834,3.29771) -- (5.91881,3.29909) -- (5.88928,3.3005) -- (5.85975,3.30194) -- (5.83021,3.30341) --
 (5.80068,3.30492) -- (5.77115,3.30647) -- (5.74162,3.30805) -- (5.71209,3.30968) -- (5.68255,3.31134) -- (5.65302,3.31305) -- (5.62349,3.31481) -- (5.59396,3.31661) -- (5.56443,3.31847) -- (5.5349,3.32037) -- (5.50536,3.32232) -- (5.47583,3.32433)
 -- (5.4463,3.3264) -- (5.41677,3.32853) -- (5.38724,3.33071) -- (5.3577,3.33296) -- (5.32817,3.33528) -- (5.29864,3.33766) -- (5.26911,3.34011) -- (5.23958,3.34263) -- (5.21005,3.34523) -- (5.18051,3.3479) -- (5.15098,3.35065) -- (5.12145,3.35349)
 -- (5.09192,3.35641) -- (5.06239,3.35941) -- (5.03286,3.36251) -- (5.00332,3.3657) -- (4.97379,3.36898) -- (4.94426,3.37237) -- (4.91473,3.37585) -- (4.8852,3.37945) -- (4.85566,3.38315) -- (4.82613,3.38697) -- (4.7966,3.3909) -- (4.76707,3.39496)
 -- (4.73754,3.39913) -- (4.70801,3.40344) -- (4.67847,3.40788) -- (4.64894,3.41246) -- (4.61941,3.41718) -- (4.58988,3.42204) -- (4.56035,3.42705) -- (4.53082,3.43222) -- (4.50128,3.43754) -- (4.47175,3.44303) -- (4.44222,3.44869) --
 (4.41269,3.45451) -- (4.38316,3.46052) -- (4.35362,3.4667) -- (4.32409,3.47307) -- (4.29456,3.47963) -- (4.26503,3.48639) -- (4.2355,3.49334) -- (4.20597,3.5005) -- (4.17643,3.50787) -- (4.1469,3.51545) -- (4.11737,3.52325) -- (4.08784,3.53127) --
 (4.05831,3.53951) -- (4.02877,3.54799) -- (3.99924,3.5567) -- (3.96971,3.56565) -- (3.94018,3.57484) -- (3.91065,3.58429) -- (3.88112,3.59398) -- (3.85158,3.60392) -- (3.82205,3.61413) -- (3.79252,3.6246) -- (3.76299,3.63533) -- (3.73346,3.64634) --
 (3.70393,3.65762) -- (3.67439,3.66917) -- (3.64486,3.681) -- (3.61533,3.69312) -- (3.5858,3.70552) -- (3.55627,3.71821) -- (3.52673,3.73119) -- (3.4972,3.74446) -- (3.46767,3.75803) -- (3.43814,3.7719) -- (3.40861,3.78607) -- (3.37908,3.80055) --
 (3.34954,3.81533) -- (3.32001,3.83041) -- (3.29048,3.84581) -- (3.26095,3.86152) -- (3.23142,3.87754) -- (3.20189,3.89388) -- (3.17235,3.91054) -- (3.14282,3.92751) -- (3.11329,3.94481) -- (3.08376,3.96243) -- (3.05423,3.98038) -- (3.02469,3.99865)
 -- (2.99516,4.01726) -- (2.96563,4.0362) -- (2.9361,4.05547) -- (2.90657,4.07508) -- (2.87704,4.09502) -- (2.8475,4.11531) -- (2.81797,4.13595) -- (2.78844,4.15693) -- (2.75891,4.17827) -- (2.72938,4.19996) -- (2.69984,4.22201) -- (2.67031,4.24443)
 -- (2.64078,4.26721) -- (2.61125,4.29037) -- (2.58172,4.3139) -- (2.55219,4.33783) -- (2.52265,4.36214) -- (2.49312,4.38686) -- (2.46359,4.41198) -- (2.43406,4.43752) -- (2.40453,4.46349) -- (2.375,4.4899) -- (2.34546,4.51675) -- (2.31593,4.54407)
 -- (2.2864,4.57186) -- (2.25687,4.60013) -- (2.22734,4.62891) -- (2.1978,4.65821) -- (2.16827,4.68804) -- (2.13874,4.71843) -- (2.10921,4.74939) -- (2.07968,4.78094) -- (2.05015,4.81312) -- (2.02061,4.84593) -- (1.99108,4.87942) -- (1.96155,4.9136)
 -- (1.93202,4.94851) -- (1.90249,4.98418) -- (1.87296,5.02064) -- (1.84342,5.05794) -- (1.81389,5.09611) -- (1.78436,5.1352) -- (1.75483,5.17525) -- (1.7253,5.21632) -- (1.69576,5.25845) -- (1.66623,5.3017) -- (1.6367,5.34613) -- (1.60717,5.39181)
 -- (1.57764,5.43881) -- (1.54811,5.48721) -- (1.51857,5.53712) -- (1.48904,0.596817) -- (1.45951,0.596817);
\colorlet{c}{kugray};
\draw [c] (1.13336,3.41516) -- (1.13336,3.80777);
\draw [c] (1.13336,3.80777) -- (1.13336,3.97485);
\draw [c] (1.11854,3.80777) -- (1.13336,3.80777);
\draw [c] (1.13336,3.80777) -- (1.14818,3.80777);
\definecolor{c}{rgb}{0,0,0};
\colorlet{c}{kugray};
\draw [c] (1.163,3.33018) -- (1.163,3.70859);
\draw [c] (1.163,3.70859) -- (1.163,3.87329);
\draw [c] (1.14818,3.70859) -- (1.163,3.70859);
\draw [c] (1.163,3.70859) -- (1.17781,3.70859);
\definecolor{c}{rgb}{0,0,0};
\colorlet{c}{kugray};
\draw [c] (1.22227,3.26026) -- (1.22227,3.80576);
\draw [c] (1.22227,3.80576) -- (1.22227,3.99181);
\draw [c] (1.20745,3.80576) -- (1.22227,3.80576);
\draw [c] (1.22227,3.80576) -- (1.23709,3.80576);
\definecolor{c}{rgb}{0,0,0};
\colorlet{c}{kugray};
\draw [c] (1.2519,3.59515) -- (1.2519,3.86294);
\draw [c] (1.2519,3.86294) -- (1.2519,4.00395);
\draw [c] (1.23709,3.86294) -- (1.2519,3.86294);
\draw [c] (1.2519,3.86294) -- (1.26672,3.86294);
\definecolor{c}{rgb}{0,0,0};
\colorlet{c}{kugray};
\draw [c] (1.28154,3.81131) -- (1.28154,4.05468);
\draw [c] (1.28154,4.05468) -- (1.28154,4.18889);
\draw [c] (1.26672,4.05468) -- (1.28154,4.05468);
\draw [c] (1.28154,4.05468) -- (1.29636,4.05468);
\definecolor{c}{rgb}{0,0,0};
\colorlet{c}{kugray};
\draw [c] (1.31118,5.36749) -- (1.31118,5.38512);
\draw [c] (1.31118,5.38512) -- (1.31118,5.40179);
\draw [c] (1.29636,5.38512) -- (1.31118,5.38512);
\draw [c] (1.31118,5.38512) -- (1.32599,5.38512);
\definecolor{c}{rgb}{0,0,0};
\colorlet{c}{kugray};
\draw [c] (1.34081,5.61993) -- (1.34081,5.63126);
\draw [c] (1.34081,5.63126) -- (1.34081,5.6422);
\draw [c] (1.32599,5.63126) -- (1.34081,5.63126);
\draw [c] (1.34081,5.63126) -- (1.35563,5.63126);
\definecolor{c}{rgb}{0,0,0};
\colorlet{c}{kugray};
\draw [c] (1.37045,5.69182) -- (1.37045,5.70194);
\draw [c] (1.37045,5.70194) -- (1.37045,5.71174);
\draw [c] (1.35563,5.70194) -- (1.37045,5.70194);
\draw [c] (1.37045,5.70194) -- (1.38526,5.70194);
\definecolor{c}{rgb}{0,0,0};
\colorlet{c}{kugray};
\draw [c] (1.40008,5.67982) -- (1.40008,5.69024);
\draw [c] (1.40008,5.69024) -- (1.40008,5.70031);
\draw [c] (1.38526,5.69024) -- (1.40008,5.69024);
\draw [c] (1.40008,5.69024) -- (1.4149,5.69024);
\definecolor{c}{rgb}{0,0,0};
\colorlet{c}{kugray};
\draw [c] (1.42972,5.65827) -- (1.42972,5.66928);
\draw [c] (1.42972,5.66928) -- (1.42972,5.67991);
\draw [c] (1.4149,5.66928) -- (1.42972,5.66928);
\draw [c] (1.42972,5.66928) -- (1.44454,5.66928);
\definecolor{c}{rgb}{0,0,0};
\colorlet{c}{kugray};
\draw [c] (1.45935,5.62) -- (1.45935,5.63141);
\draw [c] (1.45935,5.63141) -- (1.45935,5.6424);
\draw [c] (1.44454,5.63141) -- (1.45935,5.63141);
\draw [c] (1.45935,5.63141) -- (1.47417,5.63141);
\definecolor{c}{rgb}{0,0,0};
\colorlet{c}{kugray};
\draw [c] (1.48899,5.58164) -- (1.48899,5.5938);
\draw [c] (1.48899,5.5938) -- (1.48899,5.60549);
\draw [c] (1.47417,5.5938) -- (1.48899,5.5938);
\draw [c] (1.48899,5.5938) -- (1.50381,5.5938);
\definecolor{c}{rgb}{0,0,0};
\colorlet{c}{kugray};
\draw [c] (1.51863,5.53046) -- (1.51863,5.54392);
\draw [c] (1.51863,5.54392) -- (1.51863,5.55682);
\draw [c] (1.50381,5.54392) -- (1.51863,5.54392);
\draw [c] (1.51863,5.54392) -- (1.53344,5.54392);
\definecolor{c}{rgb}{0,0,0};
\colorlet{c}{kugray};
\draw [c] (1.54826,5.46022) -- (1.54826,5.47505);
\draw [c] (1.54826,5.47505) -- (1.54826,5.48921);
\draw [c] (1.53344,5.47505) -- (1.54826,5.47505);
\draw [c] (1.54826,5.47505) -- (1.56308,5.47505);
\definecolor{c}{rgb}{0,0,0};
\colorlet{c}{kugray};
\draw [c] (1.5779,5.41506) -- (1.5779,5.43093);
\draw [c] (1.5779,5.43093) -- (1.5779,5.44602);
\draw [c] (1.56308,5.43093) -- (1.5779,5.43093);
\draw [c] (1.5779,5.43093) -- (1.59272,5.43093);
\definecolor{c}{rgb}{0,0,0};
\colorlet{c}{kugray};
\draw [c] (1.60753,5.38705) -- (1.60753,5.40429);
\draw [c] (1.60753,5.40429) -- (1.60753,5.42061);
\draw [c] (1.59272,5.40429) -- (1.60753,5.40429);
\draw [c] (1.60753,5.40429) -- (1.62235,5.40429);
\definecolor{c}{rgb}{0,0,0};
\colorlet{c}{kugray};
\draw [c] (1.63717,5.33118) -- (1.63717,5.34945);
\draw [c] (1.63717,5.34945) -- (1.63717,5.36669);
\draw [c] (1.62235,5.34945) -- (1.63717,5.34945);
\draw [c] (1.63717,5.34945) -- (1.65199,5.34945);
\definecolor{c}{rgb}{0,0,0};
\colorlet{c}{kugray};
\draw [c] (1.6668,5.33501) -- (1.6668,5.35318);
\draw [c] (1.6668,5.35318) -- (1.6668,5.37033);
\draw [c] (1.65199,5.35318) -- (1.6668,5.35318);
\draw [c] (1.6668,5.35318) -- (1.68162,5.35318);
\definecolor{c}{rgb}{0,0,0};
\colorlet{c}{kugray};
\draw [c] (1.69644,5.27038) -- (1.69644,5.29099);
\draw [c] (1.69644,5.29099) -- (1.69644,5.3103);
\draw [c] (1.68162,5.29099) -- (1.69644,5.29099);
\draw [c] (1.69644,5.29099) -- (1.71126,5.29099);
\definecolor{c}{rgb}{0,0,0};
\colorlet{c}{kugray};
\draw [c] (1.72608,5.20804) -- (1.72608,5.23001);
\draw [c] (1.72608,5.23001) -- (1.72608,5.25052);
\draw [c] (1.71126,5.23001) -- (1.72608,5.23001);
\draw [c] (1.72608,5.23001) -- (1.74089,5.23001);
\definecolor{c}{rgb}{0,0,0};
\colorlet{c}{kugray};
\draw [c] (1.75571,5.08901) -- (1.75571,5.11549);
\draw [c] (1.75571,5.11549) -- (1.75571,5.13987);
\draw [c] (1.74089,5.11549) -- (1.75571,5.11549);
\draw [c] (1.75571,5.11549) -- (1.77053,5.11549);
\definecolor{c}{rgb}{0,0,0};
\colorlet{c}{kugray};
\draw [c] (1.78535,5.10181) -- (1.78535,5.12931);
\draw [c] (1.78535,5.12931) -- (1.78535,5.15455);
\draw [c] (1.77053,5.12931) -- (1.78535,5.12931);
\draw [c] (1.78535,5.12931) -- (1.80017,5.12931);
\definecolor{c}{rgb}{0,0,0};
\colorlet{c}{kugray};
\draw [c] (1.81498,5.06396) -- (1.81498,5.09313);
\draw [c] (1.81498,5.09313) -- (1.81498,5.11978);
\draw [c] (1.80017,5.09313) -- (1.81498,5.09313);
\draw [c] (1.81498,5.09313) -- (1.8298,5.09313);
\definecolor{c}{rgb}{0,0,0};
\colorlet{c}{kugray};
\draw [c] (1.84462,5.06537) -- (1.84462,5.09456);
\draw [c] (1.84462,5.09456) -- (1.84462,5.12123);
\draw [c] (1.8298,5.09456) -- (1.84462,5.09456);
\draw [c] (1.84462,5.09456) -- (1.85944,5.09456);
\definecolor{c}{rgb}{0,0,0};
\colorlet{c}{kugray};
\draw [c] (1.87425,4.99725) -- (1.87425,5.02835);
\draw [c] (1.87425,5.02835) -- (1.87425,5.05659);
\draw [c] (1.85944,5.02835) -- (1.87425,5.02835);
\draw [c] (1.87425,5.02835) -- (1.88907,5.02835);
\definecolor{c}{rgb}{0,0,0};
\colorlet{c}{kugray};
\draw [c] (1.90389,4.95832) -- (1.90389,4.99102);
\draw [c] (1.90389,4.99102) -- (1.90389,5.02058);
\draw [c] (1.88907,4.99102) -- (1.90389,4.99102);
\draw [c] (1.90389,4.99102) -- (1.91871,4.99102);
\definecolor{c}{rgb}{0,0,0};
\colorlet{c}{kugray};
\draw [c] (1.93353,4.85544) -- (1.93353,4.89383);
\draw [c] (1.93353,4.89383) -- (1.93353,4.92796);
\draw [c] (1.91871,4.89383) -- (1.93353,4.89383);
\draw [c] (1.93353,4.89383) -- (1.94834,4.89383);
\definecolor{c}{rgb}{0,0,0};
\colorlet{c}{kugray};
\draw [c] (1.96316,4.88333) -- (1.96316,4.92161);
\draw [c] (1.96316,4.92161) -- (1.96316,4.95565);
\draw [c] (1.94834,4.92161) -- (1.96316,4.92161);
\draw [c] (1.96316,4.92161) -- (1.97798,4.92161);
\definecolor{c}{rgb}{0,0,0};
\colorlet{c}{kugray};
\draw [c] (1.9928,4.85825) -- (1.9928,4.89733);
\draw [c] (1.9928,4.89733) -- (1.9928,4.93201);
\draw [c] (1.97798,4.89733) -- (1.9928,4.89733);
\draw [c] (1.9928,4.89733) -- (2.00762,4.89733);
\definecolor{c}{rgb}{0,0,0};
\colorlet{c}{kugray};
\draw [c] (2.02243,4.78026) -- (2.02243,4.82199);
\draw [c] (2.02243,4.82199) -- (2.02243,4.85874);
\draw [c] (2.00762,4.82199) -- (2.02243,4.82199);
\draw [c] (2.02243,4.82199) -- (2.03725,4.82199);
\definecolor{c}{rgb}{0,0,0};
\colorlet{c}{kugray};
\draw [c] (2.05207,4.74394) -- (2.05207,4.7911);
\draw [c] (2.05207,4.7911) -- (2.05207,4.83199);
\draw [c] (2.03725,4.7911) -- (2.05207,4.7911);
\draw [c] (2.05207,4.7911) -- (2.06689,4.7911);
\definecolor{c}{rgb}{0,0,0};
\colorlet{c}{kugray};
\draw [c] (2.08171,4.69059) -- (2.08171,4.74618);
\draw [c] (2.08171,4.74618) -- (2.08171,4.79324);
\draw [c] (2.06689,4.74618) -- (2.08171,4.74618);
\draw [c] (2.08171,4.74618) -- (2.09652,4.74618);
\definecolor{c}{rgb}{0,0,0};
\colorlet{c}{kugray};
\draw [c] (2.11134,4.64498) -- (2.11134,4.70439);
\draw [c] (2.11134,4.70439) -- (2.11134,4.75417);
\draw [c] (2.09652,4.70439) -- (2.11134,4.70439);
\draw [c] (2.11134,4.70439) -- (2.12616,4.70439);
\definecolor{c}{rgb}{0,0,0};
\colorlet{c}{kugray};
\draw [c] (2.14098,4.66916) -- (2.14098,4.726);
\draw [c] (2.14098,4.726) -- (2.14098,4.77396);
\draw [c] (2.12616,4.726) -- (2.14098,4.726);
\draw [c] (2.14098,4.726) -- (2.15579,4.726);
\definecolor{c}{rgb}{0,0,0};
\colorlet{c}{kugray};
\draw [c] (2.17061,4.71202) -- (2.17061,4.76221);
\draw [c] (2.17061,4.76221) -- (2.17061,4.80536);
\draw [c] (2.15579,4.76221) -- (2.17061,4.76221);
\draw [c] (2.17061,4.76221) -- (2.18543,4.76221);
\definecolor{c}{rgb}{0,0,0};
\colorlet{c}{kugray};
\draw [c] (2.20025,4.43289) -- (2.20025,4.50711);
\draw [c] (2.20025,4.50711) -- (2.20025,4.56686);
\draw [c] (2.18543,4.50711) -- (2.20025,4.50711);
\draw [c] (2.20025,4.50711) -- (2.21507,4.50711);
\definecolor{c}{rgb}{0,0,0};
\colorlet{c}{kugray};
\draw [c] (2.22988,4.60564) -- (2.22988,4.66562);
\draw [c] (2.22988,4.66562) -- (2.22988,4.7158);
\draw [c] (2.21507,4.66562) -- (2.22988,4.66562);
\draw [c] (2.22988,4.66562) -- (2.2447,4.66562);
\definecolor{c}{rgb}{0,0,0};
\colorlet{c}{kugray};
\draw [c] (2.25952,4.64159) -- (2.25952,4.70326);
\draw [c] (2.25952,4.70326) -- (2.25952,4.75461);
\draw [c] (2.2447,4.70326) -- (2.25952,4.70326);
\draw [c] (2.25952,4.70326) -- (2.27434,4.70326);
\definecolor{c}{rgb}{0,0,0};
\colorlet{c}{kugray};
\draw [c] (2.28916,4.56718) -- (2.28916,4.57808);
\draw [c] (2.28916,4.57808) -- (2.28916,4.5886);
\draw [c] (2.27434,4.57808) -- (2.28916,4.57808);
\draw [c] (2.28916,4.57808) -- (2.30397,4.57808);
\definecolor{c}{rgb}{0,0,0};
\colorlet{c}{kugray};
\draw [c] (2.31879,4.51802) -- (2.31879,4.5299);
\draw [c] (2.31879,4.5299) -- (2.31879,4.54134);
\draw [c] (2.30397,4.5299) -- (2.31879,4.5299);
\draw [c] (2.31879,4.5299) -- (2.33361,4.5299);
\definecolor{c}{rgb}{0,0,0};
\colorlet{c}{kugray};
\draw [c] (2.34843,4.51942) -- (2.34843,4.53112);
\draw [c] (2.34843,4.53112) -- (2.34843,4.54238);
\draw [c] (2.33361,4.53112) -- (2.34843,4.53112);
\draw [c] (2.34843,4.53112) -- (2.36325,4.53112);
\definecolor{c}{rgb}{0,0,0};
\colorlet{c}{kugray};
\draw [c] (2.37806,4.47939) -- (2.37806,4.492);
\draw [c] (2.37806,4.492) -- (2.37806,4.50411);
\draw [c] (2.36325,4.492) -- (2.37806,4.492);
\draw [c] (2.37806,4.492) -- (2.39288,4.492);
\definecolor{c}{rgb}{0,0,0};
\colorlet{c}{kugray};
\draw [c] (2.4077,4.44091) -- (2.4077,4.45439);
\draw [c] (2.4077,4.45439) -- (2.4077,4.46731);
\draw [c] (2.39288,4.45439) -- (2.4077,4.45439);
\draw [c] (2.4077,4.45439) -- (2.42252,4.45439);
\definecolor{c}{rgb}{0,0,0};
\colorlet{c}{kugray};
\draw [c] (2.43733,4.39585) -- (2.43733,4.41002);
\draw [c] (2.43733,4.41002) -- (2.43733,4.42357);
\draw [c] (2.42252,4.41002) -- (2.43733,4.41002);
\draw [c] (2.43733,4.41002) -- (2.45215,4.41002);
\definecolor{c}{rgb}{0,0,0};
\colorlet{c}{kugray};
\draw [c] (2.46697,4.36562) -- (2.46697,4.38084);
\draw [c] (2.46697,4.38084) -- (2.46697,4.39535);
\draw [c] (2.45215,4.38084) -- (2.46697,4.38084);
\draw [c] (2.46697,4.38084) -- (2.48179,4.38084);
\definecolor{c}{rgb}{0,0,0};
\colorlet{c}{kugray};
\draw [c] (2.49661,4.34442) -- (2.49661,4.3598);
\draw [c] (2.49661,4.3598) -- (2.49661,4.37444);
\draw [c] (2.48179,4.3598) -- (2.49661,4.3598);
\draw [c] (2.49661,4.3598) -- (2.51142,4.3598);
\definecolor{c}{rgb}{0,0,0};
\colorlet{c}{kugray};
\draw [c] (2.52624,4.33876) -- (2.52624,4.3547);
\draw [c] (2.52624,4.3547) -- (2.52624,4.36985);
\draw [c] (2.51142,4.3547) -- (2.52624,4.3547);
\draw [c] (2.52624,4.3547) -- (2.54106,4.3547);
\definecolor{c}{rgb}{0,0,0};
\colorlet{c}{kugray};
\draw [c] (2.55588,4.31164) -- (2.55588,4.32829);
\draw [c] (2.55588,4.32829) -- (2.55588,4.34409);
\draw [c] (2.54106,4.32829) -- (2.55588,4.32829);
\draw [c] (2.55588,4.32829) -- (2.5707,4.32829);
\definecolor{c}{rgb}{0,0,0};
\colorlet{c}{kugray};
\draw [c] (2.58551,4.27319) -- (2.58551,4.29119);
\draw [c] (2.58551,4.29119) -- (2.58551,4.30819);
\draw [c] (2.5707,4.29119) -- (2.58551,4.29119);
\draw [c] (2.58551,4.29119) -- (2.60033,4.29119);
\definecolor{c}{rgb}{0,0,0};
\colorlet{c}{kugray};
\draw [c] (2.61515,4.23142) -- (2.61515,4.25043);
\draw [c] (2.61515,4.25043) -- (2.61515,4.26834);
\draw [c] (2.60033,4.25043) -- (2.61515,4.25043);
\draw [c] (2.61515,4.25043) -- (2.62997,4.25043);
\definecolor{c}{rgb}{0,0,0};
\colorlet{c}{kugray};
\draw [c] (2.64478,4.1941) -- (2.64478,4.21371);
\draw [c] (2.64478,4.21371) -- (2.64478,4.23215);
\draw [c] (2.62997,4.21371) -- (2.64478,4.21371);
\draw [c] (2.64478,4.21371) -- (2.6596,4.21371);
\definecolor{c}{rgb}{0,0,0};
\colorlet{c}{kugray};
\draw [c] (2.67442,4.21383) -- (2.67442,4.23319);
\draw [c] (2.67442,4.23319) -- (2.67442,4.2514);
\draw [c] (2.6596,4.23319) -- (2.67442,4.23319);
\draw [c] (2.67442,4.23319) -- (2.68924,4.23319);
\definecolor{c}{rgb}{0,0,0};
\colorlet{c}{kugray};
\draw [c] (2.70406,4.15434) -- (2.70406,4.17536);
\draw [c] (2.70406,4.17536) -- (2.70406,4.19503);
\draw [c] (2.68924,4.17536) -- (2.70406,4.17536);
\draw [c] (2.70406,4.17536) -- (2.71887,4.17536);
\definecolor{c}{rgb}{0,0,0};
\colorlet{c}{kugray};
\draw [c] (2.73369,4.14693) -- (2.73369,4.16849);
\draw [c] (2.73369,4.16849) -- (2.73369,4.18863);
\draw [c] (2.71887,4.16849) -- (2.73369,4.16849);
\draw [c] (2.73369,4.16849) -- (2.74851,4.16849);
\definecolor{c}{rgb}{0,0,0};
\colorlet{c}{kugray};
\draw [c] (2.76333,4.12489) -- (2.76333,4.1477);
\draw [c] (2.76333,4.1477) -- (2.76333,4.16894);
\draw [c] (2.74851,4.1477) -- (2.76333,4.1477);
\draw [c] (2.76333,4.1477) -- (2.77815,4.1477);
\definecolor{c}{rgb}{0,0,0};
\colorlet{c}{kugray};
\draw [c] (2.79296,4.13183) -- (2.79296,4.15469);
\draw [c] (2.79296,4.15469) -- (2.79296,4.17597);
\draw [c] (2.77815,4.15469) -- (2.79296,4.15469);
\draw [c] (2.79296,4.15469) -- (2.80778,4.15469);
\definecolor{c}{rgb}{0,0,0};
\colorlet{c}{kugray};
\draw [c] (2.8226,4.05797) -- (2.8226,4.08216);
\draw [c] (2.8226,4.08216) -- (2.8226,4.10458);
\draw [c] (2.80778,4.08216) -- (2.8226,4.08216);
\draw [c] (2.8226,4.08216) -- (2.83742,4.08216);
\definecolor{c}{rgb}{0,0,0};
\colorlet{c}{kugray};
\draw [c] (2.85224,4.04617) -- (2.85224,4.07199);
\draw [c] (2.85224,4.07199) -- (2.85224,4.09581);
\draw [c] (2.83742,4.07199) -- (2.85224,4.07199);
\draw [c] (2.85224,4.07199) -- (2.86705,4.07199);
\definecolor{c}{rgb}{0,0,0};
\colorlet{c}{kugray};
\draw [c] (2.88187,4.02533) -- (2.88187,4.05093);
\draw [c] (2.88187,4.05093) -- (2.88187,4.07457);
\draw [c] (2.86705,4.05093) -- (2.88187,4.05093);
\draw [c] (2.88187,4.05093) -- (2.89669,4.05093);
\definecolor{c}{rgb}{0,0,0};
\colorlet{c}{kugray};
\draw [c] (2.91151,4.0461) -- (2.91151,4.07164);
\draw [c] (2.91151,4.07164) -- (2.91151,4.09522);
\draw [c] (2.89669,4.07164) -- (2.91151,4.07164);
\draw [c] (2.91151,4.07164) -- (2.92632,4.07164);
\definecolor{c}{rgb}{0,0,0};
\colorlet{c}{kugray};
\draw [c] (2.94114,3.94743) -- (2.94114,3.97699);
\draw [c] (2.94114,3.97699) -- (2.94114,4.00396);
\draw [c] (2.92632,3.97699) -- (2.94114,3.97699);
\draw [c] (2.94114,3.97699) -- (2.95596,3.97699);
\definecolor{c}{rgb}{0,0,0};
\colorlet{c}{kugray};
\draw [c] (2.97078,3.92033) -- (2.97078,3.95122);
\draw [c] (2.97078,3.95122) -- (2.97078,3.97929);
\draw [c] (2.95596,3.95122) -- (2.97078,3.95122);
\draw [c] (2.97078,3.95122) -- (2.9856,3.95122);
\definecolor{c}{rgb}{0,0,0};
\colorlet{c}{kugray};
\draw [c] (3.00041,3.95455) -- (3.00041,3.98496);
\draw [c] (3.00041,3.98496) -- (3.00041,4.01263);
\draw [c] (2.9856,3.98496) -- (3.00041,3.98496);
\draw [c] (3.00041,3.98496) -- (3.01523,3.98496);
\definecolor{c}{rgb}{0,0,0};
\colorlet{c}{kugray};
\draw [c] (3.03005,3.91612) -- (3.03005,3.94874);
\draw [c] (3.03005,3.94874) -- (3.03005,3.97824);
\draw [c] (3.01523,3.94874) -- (3.03005,3.94874);
\draw [c] (3.03005,3.94874) -- (3.04487,3.94874);
\definecolor{c}{rgb}{0,0,0};
\colorlet{c}{kugray};
\draw [c] (3.05969,3.91532) -- (3.05969,3.94707);
\draw [c] (3.05969,3.94707) -- (3.05969,3.97585);
\draw [c] (3.04487,3.94707) -- (3.05969,3.94707);
\draw [c] (3.05969,3.94707) -- (3.0745,3.94707);
\definecolor{c}{rgb}{0,0,0};
\colorlet{c}{kugray};
\draw [c] (3.08932,3.82813) -- (3.08932,3.8633);
\draw [c] (3.08932,3.8633) -- (3.08932,3.89486);
\draw [c] (3.0745,3.8633) -- (3.08932,3.8633);
\draw [c] (3.08932,3.8633) -- (3.10414,3.8633);
\definecolor{c}{rgb}{0,0,0};
\colorlet{c}{kugray};
\draw [c] (3.11896,3.82152) -- (3.11896,3.85725);
\draw [c] (3.11896,3.85725) -- (3.11896,3.88926);
\draw [c] (3.10414,3.85725) -- (3.11896,3.85725);
\draw [c] (3.11896,3.85725) -- (3.13377,3.85725);
\definecolor{c}{rgb}{0,0,0};
\colorlet{c}{kugray};
\draw [c] (3.14859,3.81099) -- (3.14859,3.849);
\draw [c] (3.14859,3.849) -- (3.14859,3.88283);
\draw [c] (3.13377,3.849) -- (3.14859,3.849);
\draw [c] (3.14859,3.849) -- (3.16341,3.849);
\definecolor{c}{rgb}{0,0,0};
\colorlet{c}{kugray};
\draw [c] (3.17823,3.68505) -- (3.17823,3.72686);
\draw [c] (3.17823,3.72686) -- (3.17823,3.76367);
\draw [c] (3.16341,3.72686) -- (3.17823,3.72686);
\draw [c] (3.17823,3.72686) -- (3.19305,3.72686);
\definecolor{c}{rgb}{0,0,0};
\colorlet{c}{kugray};
\draw [c] (3.20786,3.75027) -- (3.20786,3.79207);
\draw [c] (3.20786,3.79207) -- (3.20786,3.82886);
\draw [c] (3.19305,3.79207) -- (3.20786,3.79207);
\draw [c] (3.20786,3.79207) -- (3.22268,3.79207);
\definecolor{c}{rgb}{0,0,0};
\colorlet{c}{kugray};
\draw [c] (3.2375,3.78399) -- (3.2375,3.82558);
\draw [c] (3.2375,3.82558) -- (3.2375,3.86222);
\draw [c] (3.22268,3.82558) -- (3.2375,3.82558);
\draw [c] (3.2375,3.82558) -- (3.25232,3.82558);
\definecolor{c}{rgb}{0,0,0};
\colorlet{c}{kugray};
\draw [c] (3.26714,3.71561) -- (3.26714,3.76154);
\draw [c] (3.26714,3.76154) -- (3.26714,3.80149);
\draw [c] (3.25232,3.76154) -- (3.26714,3.76154);
\draw [c] (3.26714,3.76154) -- (3.28195,3.76154);
\definecolor{c}{rgb}{0,0,0};
\colorlet{c}{kugray};
\draw [c] (3.29677,3.74464) -- (3.29677,3.78592);
\draw [c] (3.29677,3.78592) -- (3.29677,3.82231);
\draw [c] (3.28195,3.78592) -- (3.29677,3.78592);
\draw [c] (3.29677,3.78592) -- (3.31159,3.78592);
\definecolor{c}{rgb}{0,0,0};
\colorlet{c}{kugray};
\draw [c] (3.32641,3.60212) -- (3.32641,3.65046);
\draw [c] (3.32641,3.65046) -- (3.32641,3.69223);
\draw [c] (3.31159,3.65046) -- (3.32641,3.65046);
\draw [c] (3.32641,3.65046) -- (3.34123,3.65046);
\definecolor{c}{rgb}{0,0,0};
\colorlet{c}{kugray};
\draw [c] (3.35604,3.66614) -- (3.35604,3.71219);
\draw [c] (3.35604,3.71219) -- (3.35604,3.75225);
\draw [c] (3.34123,3.71219) -- (3.35604,3.71219);
\draw [c] (3.35604,3.71219) -- (3.37086,3.71219);
\definecolor{c}{rgb}{0,0,0};
\colorlet{c}{kugray};
\draw [c] (3.38568,3.70789) -- (3.38568,3.7536);
\draw [c] (3.38568,3.7536) -- (3.38568,3.7934);
\draw [c] (3.37086,3.7536) -- (3.38568,3.7536);
\draw [c] (3.38568,3.7536) -- (3.4005,3.7536);
\definecolor{c}{rgb}{0,0,0};
\colorlet{c}{kugray};
\draw [c] (3.41531,3.59749) -- (3.41531,3.64919);
\draw [c] (3.41531,3.64919) -- (3.41531,3.69343);
\draw [c] (3.4005,3.64919) -- (3.41531,3.64919);
\draw [c] (3.41531,3.64919) -- (3.43013,3.64919);
\definecolor{c}{rgb}{0,0,0};
\colorlet{c}{kugray};
\draw [c] (3.44495,3.54922) -- (3.44495,3.6056);
\draw [c] (3.44495,3.6056) -- (3.44495,3.65323);
\draw [c] (3.43013,3.6056) -- (3.44495,3.6056);
\draw [c] (3.44495,3.6056) -- (3.45977,3.6056);
\definecolor{c}{rgb}{0,0,0};
\colorlet{c}{kugray};
\draw [c] (3.47459,3.5315) -- (3.47459,3.58602);
\draw [c] (3.47459,3.58602) -- (3.47459,3.63231);
\draw [c] (3.45977,3.58602) -- (3.47459,3.58602);
\draw [c] (3.47459,3.58602) -- (3.4894,3.58602);
\definecolor{c}{rgb}{0,0,0};
\colorlet{c}{kugray};
\draw [c] (3.50422,3.44087) -- (3.50422,3.50756);
\draw [c] (3.50422,3.50756) -- (3.50422,3.56234);
\draw [c] (3.4894,3.50756) -- (3.50422,3.50756);
\draw [c] (3.50422,3.50756) -- (3.51904,3.50756);
\definecolor{c}{rgb}{0,0,0};
\colorlet{c}{kugray};
\draw [c] (3.53386,3.57716) -- (3.53386,3.62954);
\draw [c] (3.53386,3.62954) -- (3.53386,3.67429);
\draw [c] (3.51904,3.62954) -- (3.53386,3.62954);
\draw [c] (3.53386,3.62954) -- (3.54868,3.62954);
\definecolor{c}{rgb}{0,0,0};
\colorlet{c}{kugray};
\draw [c] (3.56349,3.59791) -- (3.56349,3.64909);
\draw [c] (3.56349,3.64909) -- (3.56349,3.69297);
\draw [c] (3.54868,3.64909) -- (3.56349,3.64909);
\draw [c] (3.56349,3.64909) -- (3.57831,3.64909);
\definecolor{c}{rgb}{0,0,0};
\colorlet{c}{kugray};
\draw [c] (3.59313,3.48016) -- (3.59313,3.54299);
\draw [c] (3.59313,3.54299) -- (3.59313,3.59515);
\draw [c] (3.57831,3.54299) -- (3.59313,3.54299);
\draw [c] (3.59313,3.54299) -- (3.60795,3.54299);
\definecolor{c}{rgb}{0,0,0};
\colorlet{c}{kugray};
\draw [c] (3.62276,3.49926) -- (3.62276,3.5653);
\draw [c] (3.62276,3.5653) -- (3.62276,3.61965);
\draw [c] (3.60795,3.5653) -- (3.62276,3.5653);
\draw [c] (3.62276,3.5653) -- (3.63758,3.5653);
\definecolor{c}{rgb}{0,0,0};
\colorlet{c}{kugray};
\draw [c] (3.6524,3.38318) -- (3.6524,3.45701);
\draw [c] (3.6524,3.45701) -- (3.6524,3.51651);
\draw [c] (3.63758,3.45701) -- (3.6524,3.45701);
\draw [c] (3.6524,3.45701) -- (3.66722,3.45701);
\definecolor{c}{rgb}{0,0,0};
\colorlet{c}{kugray};
\draw [c] (3.68204,3.38716) -- (3.68204,3.45825);
\draw [c] (3.68204,3.45825) -- (3.68204,3.51597);
\draw [c] (3.66722,3.45825) -- (3.68204,3.45825);
\draw [c] (3.68204,3.45825) -- (3.69685,3.45825);
\definecolor{c}{rgb}{0,0,0};
\colorlet{c}{kugray};
\draw [c] (3.71167,3.27025) -- (3.71167,3.35157);
\draw [c] (3.71167,3.35157) -- (3.71167,3.41584);
\draw [c] (3.69685,3.35157) -- (3.71167,3.35157);
\draw [c] (3.71167,3.35157) -- (3.72649,3.35157);
\definecolor{c}{rgb}{0,0,0};
\colorlet{c}{kugray};
\draw [c] (3.74131,3.42793) -- (3.74131,3.49218);
\draw [c] (3.74131,3.49218) -- (3.74131,3.54531);
\draw [c] (3.72649,3.49218) -- (3.74131,3.49218);
\draw [c] (3.74131,3.49218) -- (3.75613,3.49218);
\definecolor{c}{rgb}{0,0,0};
\colorlet{c}{kugray};
\draw [c] (3.77094,3.39681) -- (3.77094,3.4655);
\draw [c] (3.77094,3.4655) -- (3.77094,3.52163);
\draw [c] (3.75613,3.4655) -- (3.77094,3.4655);
\draw [c] (3.77094,3.4655) -- (3.78576,3.4655);
\definecolor{c}{rgb}{0,0,0};
\colorlet{c}{kugray};
\draw [c] (3.80058,3.30807) -- (3.80058,3.39026);
\draw [c] (3.80058,3.39026) -- (3.80058,3.45506);
\draw [c] (3.78576,3.39026) -- (3.80058,3.39026);
\draw [c] (3.80058,3.39026) -- (3.8154,3.39026);
\definecolor{c}{rgb}{0,0,0};
\colorlet{c}{kugray};
\draw [c] (3.83022,3.36833) -- (3.83022,3.44242);
\draw [c] (3.83022,3.44242) -- (3.83022,3.50209);
\draw [c] (3.8154,3.44242) -- (3.83022,3.44242);
\draw [c] (3.83022,3.44242) -- (3.84503,3.44242);
\definecolor{c}{rgb}{0,0,0};
\colorlet{c}{kugray};
\draw [c] (3.85985,3.36697) -- (3.85985,3.43885);
\draw [c] (3.85985,3.43885) -- (3.85985,3.49708);
\draw [c] (3.84503,3.43885) -- (3.85985,3.43885);
\draw [c] (3.85985,3.43885) -- (3.87467,3.43885);
\definecolor{c}{rgb}{0,0,0};
\colorlet{c}{kugray};
\draw [c] (3.88949,3.18339) -- (3.88949,3.27641);
\draw [c] (3.88949,3.27641) -- (3.88949,3.34775);
\draw [c] (3.87467,3.27641) -- (3.88949,3.27641);
\draw [c] (3.88949,3.27641) -- (3.9043,3.27641);
\definecolor{c}{rgb}{0,0,0};
\colorlet{c}{kugray};
\draw [c] (3.91912,3.36859) -- (3.91912,3.433);
\draw [c] (3.91912,3.433) -- (3.91912,3.48624);
\draw [c] (3.9043,3.433) -- (3.91912,3.433);
\draw [c] (3.91912,3.433) -- (3.93394,3.433);
\definecolor{c}{rgb}{0,0,0};
\colorlet{c}{kugray};
\draw [c] (3.94876,3.27957) -- (3.94876,3.34398);
\draw [c] (3.94876,3.34398) -- (3.94876,3.39722);
\draw [c] (3.93394,3.34398) -- (3.94876,3.34398);
\draw [c] (3.94876,3.34398) -- (3.96358,3.34398);
\definecolor{c}{rgb}{0,0,0};
\colorlet{c}{kugray};
\draw [c] (3.97839,3.23181) -- (3.97839,3.2842);
\draw [c] (3.97839,3.2842) -- (3.97839,3.32896);
\draw [c] (3.96358,3.2842) -- (3.97839,3.2842);
\draw [c] (3.97839,3.2842) -- (3.99321,3.2842);
\definecolor{c}{rgb}{0,0,0};
\colorlet{c}{kugray};
\draw [c] (4.00803,3.23298) -- (4.00803,3.24514);
\draw [c] (4.00803,3.24514) -- (4.00803,3.25684);
\draw [c] (3.99321,3.24514) -- (4.00803,3.24514);
\draw [c] (4.00803,3.24514) -- (4.02285,3.24514);
\definecolor{c}{rgb}{0,0,0};
\colorlet{c}{kugray};
\draw [c] (4.03767,3.23899) -- (4.03767,3.25079);
\draw [c] (4.03767,3.25079) -- (4.03767,3.26215);
\draw [c] (4.02285,3.25079) -- (4.03767,3.25079);
\draw [c] (4.03767,3.25079) -- (4.05248,3.25079);
\definecolor{c}{rgb}{0,0,0};
\colorlet{c}{kugray};
\draw [c] (4.0673,3.24063) -- (4.0673,3.25269);
\draw [c] (4.0673,3.25269) -- (4.0673,3.26429);
\draw [c] (4.05248,3.25269) -- (4.0673,3.25269);
\draw [c] (4.0673,3.25269) -- (4.08212,3.25269);
\definecolor{c}{rgb}{0,0,0};
\colorlet{c}{kugray};
\draw [c] (4.09694,3.22722) -- (4.09694,3.23915);
\draw [c] (4.09694,3.23915) -- (4.09694,3.25063);
\draw [c] (4.08212,3.23915) -- (4.09694,3.23915);
\draw [c] (4.09694,3.23915) -- (4.11175,3.23915);
\definecolor{c}{rgb}{0,0,0};
\colorlet{c}{kugray};
\draw [c] (4.12657,3.2116) -- (4.12657,3.22405);
\draw [c] (4.12657,3.22405) -- (4.12657,3.23601);
\draw [c] (4.11175,3.22405) -- (4.12657,3.22405);
\draw [c] (4.12657,3.22405) -- (4.14139,3.22405);
\definecolor{c}{rgb}{0,0,0};
\colorlet{c}{kugray};
\draw [c] (4.15621,3.18215) -- (4.15621,3.19496);
\draw [c] (4.15621,3.19496) -- (4.15621,3.20726);
\draw [c] (4.14139,3.19496) -- (4.15621,3.19496);
\draw [c] (4.15621,3.19496) -- (4.17103,3.19496);
\definecolor{c}{rgb}{0,0,0};
\colorlet{c}{kugray};
\draw [c] (4.18584,3.18206) -- (4.18584,3.19521);
\draw [c] (4.18584,3.19521) -- (4.18584,3.20782);
\draw [c] (4.17103,3.19521) -- (4.18584,3.19521);
\draw [c] (4.18584,3.19521) -- (4.20066,3.19521);
\definecolor{c}{rgb}{0,0,0};
\colorlet{c}{kugray};
\draw [c] (4.21548,3.14682) -- (4.21548,3.16086);
\draw [c] (4.21548,3.16086) -- (4.21548,3.17428);
\draw [c] (4.20066,3.16086) -- (4.21548,3.16086);
\draw [c] (4.21548,3.16086) -- (4.2303,3.16086);
\definecolor{c}{rgb}{0,0,0};
\colorlet{c}{kugray};
\draw [c] (4.24512,3.16809) -- (4.24512,3.1818);
\draw [c] (4.24512,3.1818) -- (4.24512,3.19493);
\draw [c] (4.2303,3.1818) -- (4.24512,3.1818);
\draw [c] (4.24512,3.1818) -- (4.25993,3.1818);
\definecolor{c}{rgb}{0,0,0};
\colorlet{c}{kugray};
\draw [c] (4.27475,3.10937) -- (4.27475,3.12421);
\draw [c] (4.27475,3.12421) -- (4.27475,3.13837);
\draw [c] (4.25993,3.12421) -- (4.27475,3.12421);
\draw [c] (4.27475,3.12421) -- (4.28957,3.12421);
\definecolor{c}{rgb}{0,0,0};
\colorlet{c}{kugray};
\draw [c] (4.30439,3.11556) -- (4.30439,3.12992);
\draw [c] (4.30439,3.12992) -- (4.30439,3.14364);
\draw [c] (4.28957,3.12992) -- (4.30439,3.12992);
\draw [c] (4.30439,3.12992) -- (4.31921,3.12992);
\definecolor{c}{rgb}{0,0,0};
\colorlet{c}{kugray};
\draw [c] (4.33402,3.13544) -- (4.33402,3.14949);
\draw [c] (4.33402,3.14949) -- (4.33402,3.16293);
\draw [c] (4.31921,3.14949) -- (4.33402,3.14949);
\draw [c] (4.33402,3.14949) -- (4.34884,3.14949);
\definecolor{c}{rgb}{0,0,0};
\colorlet{c}{kugray};
\draw [c] (4.36366,3.09108) -- (4.36366,3.10621);
\draw [c] (4.36366,3.10621) -- (4.36366,3.12063);
\draw [c] (4.34884,3.10621) -- (4.36366,3.10621);
\draw [c] (4.36366,3.10621) -- (4.37848,3.10621);
\definecolor{c}{rgb}{0,0,0};
\colorlet{c}{kugray};
\draw [c] (4.39329,3.08833) -- (4.39329,3.10339);
\draw [c] (4.39329,3.10339) -- (4.39329,3.11775);
\draw [c] (4.37848,3.10339) -- (4.39329,3.10339);
\draw [c] (4.39329,3.10339) -- (4.40811,3.10339);
\definecolor{c}{rgb}{0,0,0};
\colorlet{c}{kugray};
\draw [c] (4.42293,3.06637) -- (4.42293,3.08234);
\draw [c] (4.42293,3.08234) -- (4.42293,3.09751);
\draw [c] (4.40811,3.08234) -- (4.42293,3.08234);
\draw [c] (4.42293,3.08234) -- (4.43775,3.08234);
\definecolor{c}{rgb}{0,0,0};
\colorlet{c}{kugray};
\draw [c] (4.45257,3.03685) -- (4.45257,3.05337);
\draw [c] (4.45257,3.05337) -- (4.45257,3.06905);
\draw [c] (4.43775,3.05337) -- (4.45257,3.05337);
\draw [c] (4.45257,3.05337) -- (4.46738,3.05337);
\definecolor{c}{rgb}{0,0,0};
\colorlet{c}{kugray};
\draw [c] (4.4822,3.05403) -- (4.4822,3.07028);
\draw [c] (4.4822,3.07028) -- (4.4822,3.08572);
\draw [c] (4.46738,3.07028) -- (4.4822,3.07028);
\draw [c] (4.4822,3.07028) -- (4.49702,3.07028);
\definecolor{c}{rgb}{0,0,0};
\colorlet{c}{kugray};
\draw [c] (4.51184,3.02475) -- (4.51184,3.04166);
\draw [c] (4.51184,3.04166) -- (4.51184,3.05768);
\draw [c] (4.49702,3.04166) -- (4.51184,3.04166);
\draw [c] (4.51184,3.04166) -- (4.52666,3.04166);
\definecolor{c}{rgb}{0,0,0};
\colorlet{c}{kugray};
\draw [c] (4.54147,2.97429) -- (4.54147,2.99215);
\draw [c] (4.54147,2.99215) -- (4.54147,3.00903);
\draw [c] (4.52666,2.99215) -- (4.54147,2.99215);
\draw [c] (4.54147,2.99215) -- (4.55629,2.99215);
\definecolor{c}{rgb}{0,0,0};
\colorlet{c}{kugray};
\draw [c] (4.57111,2.97145) -- (4.57111,2.9901);
\draw [c] (4.57111,2.9901) -- (4.57111,3.0077);
\draw [c] (4.55629,2.9901) -- (4.57111,2.9901);
\draw [c] (4.57111,2.9901) -- (4.58593,2.9901);
\definecolor{c}{rgb}{0,0,0};
\colorlet{c}{kugray};
\draw [c] (4.60075,2.99637) -- (4.60075,3.01434);
\draw [c] (4.60075,3.01434) -- (4.60075,3.03131);
\draw [c] (4.58593,3.01434) -- (4.60075,3.01434);
\draw [c] (4.60075,3.01434) -- (4.61556,3.01434);
\definecolor{c}{rgb}{0,0,0};
\colorlet{c}{kugray};
\draw [c] (4.63038,2.94705) -- (4.63038,2.96575);
\draw [c] (4.63038,2.96575) -- (4.63038,2.98338);
\draw [c] (4.61556,2.96575) -- (4.63038,2.96575);
\draw [c] (4.63038,2.96575) -- (4.6452,2.96575);
\definecolor{c}{rgb}{0,0,0};
\colorlet{c}{kugray};
\draw [c] (4.66002,2.94213) -- (4.66002,2.96119);
\draw [c] (4.66002,2.96119) -- (4.66002,2.97915);
\draw [c] (4.6452,2.96119) -- (4.66002,2.96119);
\draw [c] (4.66002,2.96119) -- (4.67483,2.96119);
\definecolor{c}{rgb}{0,0,0};
\colorlet{c}{kugray};
\draw [c] (4.68965,2.92699) -- (4.68965,2.94611);
\draw [c] (4.68965,2.94611) -- (4.68965,2.96412);
\draw [c] (4.67483,2.94611) -- (4.68965,2.94611);
\draw [c] (4.68965,2.94611) -- (4.70447,2.94611);
\definecolor{c}{rgb}{0,0,0};
\colorlet{c}{kugray};
\draw [c] (4.71929,2.90533) -- (4.71929,2.92537);
\draw [c] (4.71929,2.92537) -- (4.71929,2.94418);
\draw [c] (4.70447,2.92537) -- (4.71929,2.92537);
\draw [c] (4.71929,2.92537) -- (4.73411,2.92537);
\definecolor{c}{rgb}{0,0,0};
\colorlet{c}{kugray};
\draw [c] (4.74892,2.89803) -- (4.74892,2.9188);
\draw [c] (4.74892,2.9188) -- (4.74892,2.93826);
\draw [c] (4.73411,2.9188) -- (4.74892,2.9188);
\draw [c] (4.74892,2.9188) -- (4.76374,2.9188);
\definecolor{c}{rgb}{0,0,0};
\colorlet{c}{kugray};
\draw [c] (4.77856,2.88491) -- (4.77856,2.90571);
\draw [c] (4.77856,2.90571) -- (4.77856,2.92519);
\draw [c] (4.76374,2.90571) -- (4.77856,2.90571);
\draw [c] (4.77856,2.90571) -- (4.79338,2.90571);
\definecolor{c}{rgb}{0,0,0};
\colorlet{c}{kugray};
\draw [c] (4.8082,2.83295) -- (4.8082,2.85717);
\draw [c] (4.8082,2.85717) -- (4.8082,2.87962);
\draw [c] (4.79338,2.85717) -- (4.8082,2.85717);
\draw [c] (4.8082,2.85717) -- (4.82301,2.85717);
\definecolor{c}{rgb}{0,0,0};
\colorlet{c}{kugray};
\draw [c] (4.83783,2.84099) -- (4.83783,2.86366);
\draw [c] (4.83783,2.86366) -- (4.83783,2.88478);
\draw [c] (4.82301,2.86366) -- (4.83783,2.86366);
\draw [c] (4.83783,2.86366) -- (4.85265,2.86366);
\definecolor{c}{rgb}{0,0,0};
\colorlet{c}{kugray};
\draw [c] (4.86747,2.84516) -- (4.86747,2.86807);
\draw [c] (4.86747,2.86807) -- (4.86747,2.88939);
\draw [c] (4.85265,2.86807) -- (4.86747,2.86807);
\draw [c] (4.86747,2.86807) -- (4.88228,2.86807);
\definecolor{c}{rgb}{0,0,0};
\colorlet{c}{kugray};
\draw [c] (4.8971,2.82427) -- (4.8971,2.84804);
\draw [c] (4.8971,2.84804) -- (4.8971,2.87011);
\draw [c] (4.88228,2.84804) -- (4.8971,2.84804);
\draw [c] (4.8971,2.84804) -- (4.91192,2.84804);
\definecolor{c}{rgb}{0,0,0};
\colorlet{c}{kugray};
\draw [c] (4.92674,2.7958) -- (4.92674,2.82013);
\draw [c] (4.92674,2.82013) -- (4.92674,2.84268);
\draw [c] (4.91192,2.82013) -- (4.92674,2.82013);
\draw [c] (4.92674,2.82013) -- (4.94156,2.82013);
\definecolor{c}{rgb}{0,0,0};
\colorlet{c}{kugray};
\draw [c] (4.95637,2.72162) -- (4.95637,2.74908);
\draw [c] (4.95637,2.74908) -- (4.95637,2.7743);
\draw [c] (4.94156,2.74908) -- (4.95637,2.74908);
\draw [c] (4.95637,2.74908) -- (4.97119,2.74908);
\definecolor{c}{rgb}{0,0,0};
\colorlet{c}{kugray};
\draw [c] (4.98601,2.74467) -- (4.98601,2.77317);
\draw [c] (4.98601,2.77317) -- (4.98601,2.79924);
\draw [c] (4.97119,2.77317) -- (4.98601,2.77317);
\draw [c] (4.98601,2.77317) -- (5.00083,2.77317);
\definecolor{c}{rgb}{0,0,0};
\colorlet{c}{kugray};
\draw [c] (5.01565,2.72133) -- (5.01565,2.74919);
\draw [c] (5.01565,2.74919) -- (5.01565,2.77473);
\draw [c] (5.00083,2.74919) -- (5.01565,2.74919);
\draw [c] (5.01565,2.74919) -- (5.03046,2.74919);
\definecolor{c}{rgb}{0,0,0};
\colorlet{c}{kugray};
\draw [c] (5.04528,2.68622) -- (5.04528,2.71565);
\draw [c] (5.04528,2.71565) -- (5.04528,2.74251);
\draw [c] (5.03046,2.71565) -- (5.04528,2.71565);
\draw [c] (5.04528,2.71565) -- (5.0601,2.71565);
\definecolor{c}{rgb}{0,0,0};
\colorlet{c}{kugray};
\draw [c] (5.07492,2.69413) -- (5.07492,2.72282);
\draw [c] (5.07492,2.72282) -- (5.07492,2.74907);
\draw [c] (5.0601,2.72282) -- (5.07492,2.72282);
\draw [c] (5.07492,2.72282) -- (5.08974,2.72282);
\definecolor{c}{rgb}{0,0,0};
\colorlet{c}{kugray};
\draw [c] (5.10455,2.68206) -- (5.10455,2.71119);
\draw [c] (5.10455,2.71119) -- (5.10455,2.73781);
\draw [c] (5.08974,2.71119) -- (5.10455,2.71119);
\draw [c] (5.10455,2.71119) -- (5.11937,2.71119);
\definecolor{c}{rgb}{0,0,0};
\colorlet{c}{kugray};
\draw [c] (5.13419,2.68307) -- (5.13419,2.7116);
\draw [c] (5.13419,2.7116) -- (5.13419,2.7377);
\draw [c] (5.11937,2.7116) -- (5.13419,2.7116);
\draw [c] (5.13419,2.7116) -- (5.14901,2.7116);
\definecolor{c}{rgb}{0,0,0};
\colorlet{c}{kugray};
\draw [c] (5.16382,2.72714) -- (5.16382,2.75436);
\draw [c] (5.16382,2.75436) -- (5.16382,2.77936);
\draw [c] (5.14901,2.75436) -- (5.16382,2.75436);
\draw [c] (5.16382,2.75436) -- (5.17864,2.75436);
\definecolor{c}{rgb}{0,0,0};
\colorlet{c}{kugray};
\draw [c] (5.19346,2.64861) -- (5.19346,2.67901);
\draw [c] (5.19346,2.67901) -- (5.19346,2.70667);
\draw [c] (5.17864,2.67901) -- (5.19346,2.67901);
\draw [c] (5.19346,2.67901) -- (5.20828,2.67901);
\definecolor{c}{rgb}{0,0,0};
\colorlet{c}{kugray};
\draw [c] (5.2231,2.688) -- (5.2231,2.71811);
\draw [c] (5.2231,2.71811) -- (5.2231,2.74554);
\draw [c] (5.20828,2.71811) -- (5.2231,2.71811);
\draw [c] (5.2231,2.71811) -- (5.23791,2.71811);
\definecolor{c}{rgb}{0,0,0};
\colorlet{c}{kugray};
\draw [c] (5.25273,2.56862) -- (5.25273,2.60483);
\draw [c] (5.25273,2.60483) -- (5.25273,2.63722);
\draw [c] (5.23791,2.60483) -- (5.25273,2.60483);
\draw [c] (5.25273,2.60483) -- (5.26755,2.60483);
\definecolor{c}{rgb}{0,0,0};
\colorlet{c}{kugray};
\draw [c] (5.28237,2.61362) -- (5.28237,2.64718);
\draw [c] (5.28237,2.64718) -- (5.28237,2.67745);
\draw [c] (5.26755,2.64718) -- (5.28237,2.64718);
\draw [c] (5.28237,2.64718) -- (5.29719,2.64718);
\definecolor{c}{rgb}{0,0,0};
\colorlet{c}{kugray};
\draw [c] (5.312,2.58142) -- (5.312,2.61649);
\draw [c] (5.312,2.61649) -- (5.312,2.64798);
\draw [c] (5.29719,2.61649) -- (5.312,2.61649);
\draw [c] (5.312,2.61649) -- (5.32682,2.61649);
\definecolor{c}{rgb}{0,0,0};
\colorlet{c}{kugray};
\draw [c] (5.34164,2.59936) -- (5.34164,2.63325);
\draw [c] (5.34164,2.63325) -- (5.34164,2.66377);
\draw [c] (5.32682,2.63325) -- (5.34164,2.63325);
\draw [c] (5.34164,2.63325) -- (5.35646,2.63325);
\definecolor{c}{rgb}{0,0,0};
\colorlet{c}{kugray};
\draw [c] (5.37127,2.59301) -- (5.37127,2.62735);
\draw [c] (5.37127,2.62735) -- (5.37127,2.65824);
\draw [c] (5.35646,2.62735) -- (5.37127,2.62735);
\draw [c] (5.37127,2.62735) -- (5.38609,2.62735);
\definecolor{c}{rgb}{0,0,0};
\colorlet{c}{kugray};
\draw [c] (5.40091,2.57979) -- (5.40091,2.61666);
\draw [c] (5.40091,2.61666) -- (5.40091,2.64958);
\draw [c] (5.38609,2.61666) -- (5.40091,2.61666);
\draw [c] (5.40091,2.61666) -- (5.41573,2.61666);
\definecolor{c}{rgb}{0,0,0};
\colorlet{c}{kugray};
\draw [c] (5.43055,2.5888) -- (5.43055,2.62522);
\draw [c] (5.43055,2.62522) -- (5.43055,2.65779);
\draw [c] (5.41573,2.62522) -- (5.43055,2.62522);
\draw [c] (5.43055,2.62522) -- (5.44536,2.62522);
\definecolor{c}{rgb}{0,0,0};
\colorlet{c}{kugray};
\draw [c] (5.46018,2.49892) -- (5.46018,2.53839);
\draw [c] (5.46018,2.53839) -- (5.46018,2.57337);
\draw [c] (5.44536,2.53839) -- (5.46018,2.53839);
\draw [c] (5.46018,2.53839) -- (5.475,2.53839);
\definecolor{c}{rgb}{0,0,0};
\colorlet{c}{kugray};
\draw [c] (5.48982,2.51137) -- (5.48982,2.5487);
\draw [c] (5.48982,2.5487) -- (5.48982,2.58199);
\draw [c] (5.475,2.5487) -- (5.48982,2.5487);
\draw [c] (5.48982,2.5487) -- (5.50464,2.5487);
\definecolor{c}{rgb}{0,0,0};
\colorlet{c}{kugray};
\draw [c] (5.51945,2.5345) -- (5.51945,2.57323);
\draw [c] (5.51945,2.57323) -- (5.51945,2.60762);
\draw [c] (5.50464,2.57323) -- (5.51945,2.57323);
\draw [c] (5.51945,2.57323) -- (5.53427,2.57323);
\definecolor{c}{rgb}{0,0,0};
\colorlet{c}{kugray};
\draw [c] (5.54909,2.49349) -- (5.54909,2.533);
\draw [c] (5.54909,2.533) -- (5.54909,2.56801);
\draw [c] (5.53427,2.533) -- (5.54909,2.533);
\draw [c] (5.54909,2.533) -- (5.56391,2.533);
\definecolor{c}{rgb}{0,0,0};
\colorlet{c}{kugray};
\draw [c] (5.57873,2.46256) -- (5.57873,2.50333);
\draw [c] (5.57873,2.50333) -- (5.57873,2.53932);
\draw [c] (5.56391,2.50333) -- (5.57873,2.50333);
\draw [c] (5.57873,2.50333) -- (5.59354,2.50333);
\definecolor{c}{rgb}{0,0,0};
\colorlet{c}{kugray};
\draw [c] (5.60836,2.50823) -- (5.60836,2.54684);
\draw [c] (5.60836,2.54684) -- (5.60836,2.58115);
\draw [c] (5.59354,2.54684) -- (5.60836,2.54684);
\draw [c] (5.60836,2.54684) -- (5.62318,2.54684);
\definecolor{c}{rgb}{0,0,0};
\colorlet{c}{kugray};
\draw [c] (5.638,2.50569) -- (5.638,2.54708);
\draw [c] (5.638,2.54708) -- (5.638,2.58356);
\draw [c] (5.62318,2.54708) -- (5.638,2.54708);
\draw [c] (5.638,2.54708) -- (5.65281,2.54708);
\definecolor{c}{rgb}{0,0,0};
\colorlet{c}{kugray};
\draw [c] (5.66763,2.48101) -- (5.66763,2.5229);
\draw [c] (5.66763,2.5229) -- (5.66763,2.55976);
\draw [c] (5.65281,2.5229) -- (5.66763,2.5229);
\draw [c] (5.66763,2.5229) -- (5.68245,2.5229);
\definecolor{c}{rgb}{0,0,0};
\colorlet{c}{kugray};
\draw [c] (5.69727,2.3796) -- (5.69727,2.42684);
\draw [c] (5.69727,2.42684) -- (5.69727,2.46778);
\draw [c] (5.68245,2.42684) -- (5.69727,2.42684);
\draw [c] (5.69727,2.42684) -- (5.71209,2.42684);
\definecolor{c}{rgb}{0,0,0};
\colorlet{c}{kugray};
\draw [c] (5.7269,2.30877) -- (5.7269,2.35869);
\draw [c] (5.7269,2.35869) -- (5.7269,2.40163);
\draw [c] (5.71209,2.35869) -- (5.7269,2.35869);
\draw [c] (5.7269,2.35869) -- (5.74172,2.35869);
\definecolor{c}{rgb}{0,0,0};
\colorlet{c}{kugray};
\draw [c] (5.75654,2.43196) -- (5.75654,2.47986);
\draw [c] (5.75654,2.47986) -- (5.75654,2.5213);
\draw [c] (5.74172,2.47986) -- (5.75654,2.47986);
\draw [c] (5.75654,2.47986) -- (5.77136,2.47986);
\definecolor{c}{rgb}{0,0,0};
\colorlet{c}{kugray};
\draw [c] (5.78618,2.43169) -- (5.78618,2.47526);
\draw [c] (5.78618,2.47526) -- (5.78618,2.51342);
\draw [c] (5.77136,2.47526) -- (5.78618,2.47526);
\draw [c] (5.78618,2.47526) -- (5.80099,2.47526);
\definecolor{c}{rgb}{0,0,0};
\colorlet{c}{kugray};
\draw [c] (5.81581,2.39907) -- (5.81581,2.44526);
\draw [c] (5.81581,2.44526) -- (5.81581,2.48541);
\draw [c] (5.80099,2.44526) -- (5.81581,2.44526);
\draw [c] (5.81581,2.44526) -- (5.83063,2.44526);
\definecolor{c}{rgb}{0,0,0};
\colorlet{c}{kugray};
\draw [c] (5.84545,2.43121) -- (5.84545,2.47744);
\draw [c] (5.84545,2.47744) -- (5.84545,2.51762);
\draw [c] (5.83063,2.47744) -- (5.84545,2.47744);
\draw [c] (5.84545,2.47744) -- (5.86026,2.47744);
\definecolor{c}{rgb}{0,0,0};
\colorlet{c}{kugray};
\draw [c] (5.87508,2.29827) -- (5.87508,2.35564);
\draw [c] (5.87508,2.35564) -- (5.87508,2.40397);
\draw [c] (5.86026,2.35564) -- (5.87508,2.35564);
\draw [c] (5.87508,2.35564) -- (5.8899,2.35564);
\definecolor{c}{rgb}{0,0,0};
\colorlet{c}{kugray};
\draw [c] (5.90472,2.35345) -- (5.90472,2.40527);
\draw [c] (5.90472,2.40527) -- (5.90472,2.44961);
\draw [c] (5.8899,2.40527) -- (5.90472,2.40527);
\draw [c] (5.90472,2.40527) -- (5.91954,2.40527);
\definecolor{c}{rgb}{0,0,0};
\colorlet{c}{kugray};
\draw [c] (5.93435,2.28972) -- (5.93435,2.34179);
\draw [c] (5.93435,2.34179) -- (5.93435,2.38632);
\draw [c] (5.91954,2.34179) -- (5.93435,2.34179);
\draw [c] (5.93435,2.34179) -- (5.94917,2.34179);
\definecolor{c}{rgb}{0,0,0};
\colorlet{c}{kugray};
\draw [c] (5.96399,2.28856) -- (5.96399,2.34875);
\draw [c] (5.96399,2.34875) -- (5.96399,2.39907);
\draw [c] (5.94917,2.34875) -- (5.96399,2.34875);
\draw [c] (5.96399,2.34875) -- (5.97881,2.34875);
\definecolor{c}{rgb}{0,0,0};
\colorlet{c}{kugray};
\draw [c] (5.99363,2.34369) -- (5.99363,2.39484);
\draw [c] (5.99363,2.39484) -- (5.99363,2.43869);
\draw [c] (5.97881,2.39484) -- (5.99363,2.39484);
\draw [c] (5.99363,2.39484) -- (6.00844,2.39484);
\definecolor{c}{rgb}{0,0,0};
\colorlet{c}{kugray};
\draw [c] (6.02326,2.22004) -- (6.02326,2.27835);
\draw [c] (6.02326,2.27835) -- (6.02326,2.32735);
\draw [c] (6.00844,2.27835) -- (6.02326,2.27835);
\draw [c] (6.02326,2.27835) -- (6.03808,2.27835);
\definecolor{c}{rgb}{0,0,0};
\colorlet{c}{kugray};
\draw [c] (6.0529,2.24812) -- (6.0529,2.30663);
\draw [c] (6.0529,2.30663) -- (6.0529,2.35577);
\draw [c] (6.03808,2.30663) -- (6.0529,2.30663);
\draw [c] (6.0529,2.30663) -- (6.06772,2.30663);
\definecolor{c}{rgb}{0,0,0};
\colorlet{c}{kugray};
\draw [c] (6.08253,2.21392) -- (6.08253,2.27979);
\draw [c] (6.08253,2.27979) -- (6.08253,2.33402);
\draw [c] (6.06772,2.27979) -- (6.08253,2.27979);
\draw [c] (6.08253,2.27979) -- (6.09735,2.27979);
\definecolor{c}{rgb}{0,0,0};
\colorlet{c}{kugray};
\draw [c] (6.11217,2.28422) -- (6.11217,2.34425);
\draw [c] (6.11217,2.34425) -- (6.11217,2.39446);
\draw [c] (6.09735,2.34425) -- (6.11217,2.34425);
\draw [c] (6.11217,2.34425) -- (6.12699,2.34425);
\definecolor{c}{rgb}{0,0,0};
\colorlet{c}{kugray};
\draw [c] (6.1418,2.13526) -- (6.1418,2.20119);
\draw [c] (6.1418,2.20119) -- (6.1418,2.25547);
\draw [c] (6.12699,2.20119) -- (6.1418,2.20119);
\draw [c] (6.1418,2.20119) -- (6.15662,2.20119);
\definecolor{c}{rgb}{0,0,0};
\colorlet{c}{kugray};
\draw [c] (6.17144,2.21632) -- (6.17144,2.28052);
\draw [c] (6.17144,2.28052) -- (6.17144,2.33362);
\draw [c] (6.15662,2.28052) -- (6.17144,2.28052);
\draw [c] (6.17144,2.28052) -- (6.18626,2.28052);
\definecolor{c}{rgb}{0,0,0};
\colorlet{c}{kugray};
\draw [c] (6.20108,2.25826) -- (6.20108,2.32268);
\draw [c] (6.20108,2.32268) -- (6.20108,2.37592);
\draw [c] (6.18626,2.32268) -- (6.20108,2.32268);
\draw [c] (6.20108,2.32268) -- (6.21589,2.32268);
\definecolor{c}{rgb}{0,0,0};
\colorlet{c}{kugray};
\draw [c] (6.23071,2.21696) -- (6.23071,2.28153);
\draw [c] (6.23071,2.28153) -- (6.23071,2.33487);
\draw [c] (6.21589,2.28153) -- (6.23071,2.28153);
\draw [c] (6.23071,2.28153) -- (6.24553,2.28153);
\definecolor{c}{rgb}{0,0,0};
\colorlet{c}{kugray};
\draw [c] (6.26035,2.11717) -- (6.26035,2.18912);
\draw [c] (6.26035,2.18912) -- (6.26035,2.2474);
\draw [c] (6.24553,2.18912) -- (6.26035,2.18912);
\draw [c] (6.26035,2.18912) -- (6.27517,2.18912);
\definecolor{c}{rgb}{0,0,0};
\colorlet{c}{kugray};
\draw [c] (6.28998,2.21915) -- (6.28998,2.27945);
\draw [c] (6.28998,2.27945) -- (6.28998,2.32985);
\draw [c] (6.27517,2.27945) -- (6.28998,2.27945);
\draw [c] (6.28998,2.27945) -- (6.3048,2.27945);
\definecolor{c}{rgb}{0,0,0};
\colorlet{c}{kugray};
\draw [c] (6.31962,2.15067) -- (6.31962,2.22057);
\draw [c] (6.31962,2.22057) -- (6.31962,2.2775);
\draw [c] (6.3048,2.22057) -- (6.31962,2.22057);
\draw [c] (6.31962,2.22057) -- (6.33444,2.22057);
\definecolor{c}{rgb}{0,0,0};
\colorlet{c}{kugray};
\draw [c] (6.34926,2.08129) -- (6.34926,2.15905);
\draw [c] (6.34926,2.15905) -- (6.34926,2.22107);
\draw [c] (6.33444,2.15905) -- (6.34926,2.15905);
\draw [c] (6.34926,2.15905) -- (6.36407,2.15905);
\definecolor{c}{rgb}{0,0,0};
\colorlet{c}{kugray};
\draw [c] (6.37889,2.26481) -- (6.37889,2.32632);
\draw [c] (6.37889,2.32632) -- (6.37889,2.37756);
\draw [c] (6.36407,2.32632) -- (6.37889,2.32632);
\draw [c] (6.37889,2.32632) -- (6.39371,2.32632);
\definecolor{c}{rgb}{0,0,0};
\colorlet{c}{kugray};
\draw [c] (6.40853,2.10103) -- (6.40853,2.17287);
\draw [c] (6.40853,2.17287) -- (6.40853,2.23108);
\draw [c] (6.39371,2.17287) -- (6.40853,2.17287);
\draw [c] (6.40853,2.17287) -- (6.42334,2.17287);
\definecolor{c}{rgb}{0,0,0};
\colorlet{c}{kugray};
\draw [c] (6.43816,2.01805) -- (6.43816,2.10352);
\draw [c] (6.43816,2.10352) -- (6.43816,2.17034);
\draw [c] (6.42334,2.10352) -- (6.43816,2.10352);
\draw [c] (6.43816,2.10352) -- (6.45298,2.10352);
\definecolor{c}{rgb}{0,0,0};
\colorlet{c}{kugray};
\draw [c] (6.4678,2.13798) -- (6.4678,2.21314);
\draw [c] (6.4678,2.21314) -- (6.4678,2.27351);
\draw [c] (6.45298,2.21314) -- (6.4678,2.21314);
\draw [c] (6.4678,2.21314) -- (6.48262,2.21314);
\definecolor{c}{rgb}{0,0,0};
\colorlet{c}{kugray};
\draw [c] (6.49743,2.0628) -- (6.49743,2.13789);
\draw [c] (6.49743,2.13789) -- (6.49743,2.19821);
\draw [c] (6.48262,2.13789) -- (6.49743,2.13789);
\draw [c] (6.49743,2.13789) -- (6.51225,2.13789);
\definecolor{c}{rgb}{0,0,0};
\colorlet{c}{kugray};
\draw [c] (6.52707,2.03036) -- (6.52707,2.11014);
\draw [c] (6.52707,2.11014) -- (6.52707,2.17345);
\draw [c] (6.51225,2.11014) -- (6.52707,2.11014);
\draw [c] (6.52707,2.11014) -- (6.54189,2.11014);
\definecolor{c}{rgb}{0,0,0};
\colorlet{c}{kugray};
\draw [c] (6.55671,1.95941) -- (6.55671,2.05375);
\draw [c] (6.55671,2.05375) -- (6.55671,2.12586);
\draw [c] (6.54189,2.05375) -- (6.55671,2.05375);
\draw [c] (6.55671,2.05375) -- (6.57152,2.05375);
\definecolor{c}{rgb}{0,0,0};
\colorlet{c}{kugray};
\draw [c] (6.58634,2.0092) -- (6.58634,2.09944);
\draw [c] (6.58634,2.09944) -- (6.58634,2.16914);
\draw [c] (6.57152,2.09944) -- (6.58634,2.09944);
\draw [c] (6.58634,2.09944) -- (6.60116,2.09944);
\definecolor{c}{rgb}{0,0,0};
\colorlet{c}{kugray};
\draw [c] (6.61598,2.06823) -- (6.61598,2.14649);
\draw [c] (6.61598,2.14649) -- (6.61598,2.20883);
\draw [c] (6.60116,2.14649) -- (6.61598,2.14649);
\draw [c] (6.61598,2.14649) -- (6.63079,2.14649);
\definecolor{c}{rgb}{0,0,0};
\colorlet{c}{kugray};
\draw [c] (6.64561,1.91105) -- (6.64561,2.01133);
\draw [c] (6.64561,2.01133) -- (6.64561,2.08685);
\draw [c] (6.63079,2.01133) -- (6.64561,2.01133);
\draw [c] (6.64561,2.01133) -- (6.66043,2.01133);
\definecolor{c}{rgb}{0,0,0};
\colorlet{c}{kugray};
\draw [c] (6.67525,1.90644) -- (6.67525,2.01096);
\draw [c] (6.67525,2.01096) -- (6.67525,2.08884);
\draw [c] (6.66043,2.01096) -- (6.67525,2.01096);
\draw [c] (6.67525,2.01096) -- (6.69007,2.01096);
\definecolor{c}{rgb}{0,0,0};
\colorlet{c}{kugray};
\draw [c] (6.70488,1.98499) -- (6.70488,2.07197);
\draw [c] (6.70488,2.07197) -- (6.70488,2.13971);
\draw [c] (6.69007,2.07197) -- (6.70488,2.07197);
\draw [c] (6.70488,2.07197) -- (6.7197,2.07197);
\definecolor{c}{rgb}{0,0,0};
\colorlet{c}{kugray};
\draw [c] (6.73452,1.98323) -- (6.73452,2.07762);
\draw [c] (6.73452,2.07762) -- (6.73452,2.14976);
\draw [c] (6.7197,2.07762) -- (6.73452,2.07762);
\draw [c] (6.73452,2.07762) -- (6.74934,2.07762);
\definecolor{c}{rgb}{0,0,0};
\colorlet{c}{kugray};
\draw [c] (6.76416,1.8404) -- (6.76416,1.94614);
\draw [c] (6.76416,1.94614) -- (6.76416,2.0247);
\draw [c] (6.74934,1.94614) -- (6.76416,1.94614);
\draw [c] (6.76416,1.94614) -- (6.77897,1.94614);
\definecolor{c}{rgb}{0,0,0};
\colorlet{c}{kugray};
\draw [c] (6.79379,1.95326) -- (6.79379,2.04637);
\draw [c] (6.79379,2.04637) -- (6.79379,2.11775);
\draw [c] (6.77897,2.04637) -- (6.79379,2.04637);
\draw [c] (6.79379,2.04637) -- (6.80861,2.04637);
\definecolor{c}{rgb}{0,0,0};
\colorlet{c}{kugray};
\draw [c] (6.82343,1.84998) -- (6.82343,1.96142);
\draw [c] (6.82343,1.96142) -- (6.82343,2.04307);
\draw [c] (6.80861,1.96142) -- (6.82343,1.96142);
\draw [c] (6.82343,1.96142) -- (6.83824,1.96142);
\definecolor{c}{rgb}{0,0,0};
\colorlet{c}{kugray};
\draw [c] (6.85306,1.97353) -- (6.85306,2.06824);
\draw [c] (6.85306,2.06824) -- (6.85306,2.14058);
\draw [c] (6.83824,2.06824) -- (6.85306,2.06824);
\draw [c] (6.85306,2.06824) -- (6.86788,2.06824);
\definecolor{c}{rgb}{0,0,0};
\colorlet{c}{kugray};
\draw [c] (6.8827,1.94959) -- (6.8827,2.05345);
\draw [c] (6.8827,2.05345) -- (6.8827,2.13097);
\draw [c] (6.86788,2.05345) -- (6.8827,2.05345);
\draw [c] (6.8827,2.05345) -- (6.89752,2.05345);
\definecolor{c}{rgb}{0,0,0};
\colorlet{c}{kugray};
\draw [c] (6.91233,1.64738) -- (6.91233,1.796);
\draw [c] (6.91233,1.796) -- (6.91233,1.8958);
\draw [c] (6.89752,1.796) -- (6.91233,1.796);
\draw [c] (6.91233,1.796) -- (6.92715,1.796);
\definecolor{c}{rgb}{0,0,0};
\colorlet{c}{kugray};
\draw [c] (6.94197,1.81861) -- (6.94197,1.94633);
\draw [c] (6.94197,1.94633) -- (6.94197,2.03632);
\draw [c] (6.92715,1.94633) -- (6.94197,1.94633);
\draw [c] (6.94197,1.94633) -- (6.95679,1.94633);
\definecolor{c}{rgb}{0,0,0};
\colorlet{c}{kugray};
\draw [c] (6.97161,1.45467) -- (6.97161,1.67503);
\draw [c] (6.97161,1.67503) -- (6.97161,1.80215);
\draw [c] (6.95679,1.67503) -- (6.97161,1.67503);
\draw [c] (6.97161,1.67503) -- (6.98642,1.67503);
\definecolor{c}{rgb}{0,0,0};
\colorlet{c}{kugray};
\draw [c] (7.00124,1.92607) -- (7.00124,2.03252);
\draw [c] (7.00124,2.03252) -- (7.00124,2.11147);
\draw [c] (6.98642,2.03252) -- (7.00124,2.03252);
\draw [c] (7.00124,2.03252) -- (7.01606,2.03252);
\definecolor{c}{rgb}{0,0,0};
\colorlet{c}{kugray};
\draw [c] (7.03088,1.70142) -- (7.03088,1.83023);
\draw [c] (7.03088,1.83023) -- (7.03088,1.92076);
\draw [c] (7.01606,1.83023) -- (7.03088,1.83023);
\draw [c] (7.03088,1.83023) -- (7.0457,1.83023);
\definecolor{c}{rgb}{0,0,0};
\colorlet{c}{kugray};
\draw [c] (7.06051,1.69915) -- (7.06051,1.84157);
\draw [c] (7.06051,1.84157) -- (7.06051,1.93856);
\draw [c] (7.0457,1.84157) -- (7.06051,1.84157);
\draw [c] (7.06051,1.84157) -- (7.07533,1.84157);
\definecolor{c}{rgb}{0,0,0};
\colorlet{c}{kugray};
\draw [c] (7.09015,1.30062) -- (7.09015,1.58241);
\draw [c] (7.09015,1.58241) -- (7.09015,1.72703);
\draw [c] (7.07533,1.58241) -- (7.09015,1.58241);
\draw [c] (7.09015,1.58241) -- (7.10497,1.58241);
\definecolor{c}{rgb}{0,0,0};
\colorlet{c}{kugray};
\draw [c] (7.11978,1.6185) -- (7.11978,1.77873);
\draw [c] (7.11978,1.77873) -- (7.11978,1.88358);
\draw [c] (7.10497,1.77873) -- (7.11978,1.77873);
\draw [c] (7.11978,1.77873) -- (7.1346,1.77873);
\definecolor{c}{rgb}{0,0,0};
\colorlet{c}{kugray};
\draw [c] (7.14942,1.83806) -- (7.14942,1.95138);
\draw [c] (7.14942,1.95138) -- (7.14942,2.03403);
\draw [c] (7.1346,1.95138) -- (7.14942,1.95138);
\draw [c] (7.14942,1.95138) -- (7.16424,1.95138);
\definecolor{c}{rgb}{0,0,0};
\colorlet{c}{kugray};
\draw [c] (7.17906,1.68124) -- (7.17906,1.87494);
\draw [c] (7.17906,1.87494) -- (7.17906,1.99292);
\draw [c] (7.16424,1.87494) -- (7.17906,1.87494);
\draw [c] (7.17906,1.87494) -- (7.19387,1.87494);
\definecolor{c}{rgb}{0,0,0};
\colorlet{c}{kugray};
\draw [c] (7.20869,1.63815) -- (7.20869,1.80529);
\draw [c] (7.20869,1.80529) -- (7.20869,1.91302);
\draw [c] (7.19387,1.80529) -- (7.20869,1.80529);
\draw [c] (7.20869,1.80529) -- (7.22351,1.80529);
\definecolor{c}{rgb}{0,0,0};
\colorlet{c}{kugray};
\draw [c] (7.23833,1.534) -- (7.23833,1.72717);
\draw [c] (7.23833,1.72717) -- (7.23833,1.84495);
\draw [c] (7.22351,1.72717) -- (7.23833,1.72717);
\draw [c] (7.23833,1.72717) -- (7.25315,1.72717);
\definecolor{c}{rgb}{0,0,0};
\colorlet{c}{kugray};
\draw [c] (7.26796,1.44669) -- (7.26796,1.66186);
\draw [c] (7.26796,1.66186) -- (7.26796,1.78728);
\draw [c] (7.25315,1.66186) -- (7.26796,1.66186);
\draw [c] (7.26796,1.66186) -- (7.28278,1.66186);
\definecolor{c}{rgb}{0,0,0};
\colorlet{c}{kugray};
\draw [c] (7.2976,1.67008) -- (7.2976,1.80777);
\draw [c] (7.2976,1.80777) -- (7.2976,1.90257);
\draw [c] (7.28278,1.80777) -- (7.2976,1.80777);
\draw [c] (7.2976,1.80777) -- (7.31242,1.80777);
\definecolor{c}{rgb}{0,0,0};
\colorlet{c}{kugray};
\draw [c] (7.32724,1.55019) -- (7.32724,1.78423);
\draw [c] (7.32724,1.78423) -- (7.32724,1.91564);
\draw [c] (7.31242,1.78423) -- (7.32724,1.78423);
\draw [c] (7.32724,1.78423) -- (7.34205,1.78423);
\definecolor{c}{rgb}{0,0,0};
\colorlet{c}{kugray};
\draw [c] (7.35687,1.70986) -- (7.35687,1.88005);
\draw [c] (7.35687,1.88005) -- (7.35687,1.98902);
\draw [c] (7.34205,1.88005) -- (7.35687,1.88005);
\draw [c] (7.35687,1.88005) -- (7.37169,1.88005);
\definecolor{c}{rgb}{0,0,0};
\colorlet{c}{kugray};
\draw [c] (7.38651,1.39569) -- (7.38651,1.61683);
\draw [c] (7.38651,1.61683) -- (7.38651,1.7442);
\draw [c] (7.37169,1.61683) -- (7.38651,1.61683);
\draw [c] (7.38651,1.61683) -- (7.40132,1.61683);
\definecolor{c}{rgb}{0,0,0};
\colorlet{c}{kugray};
\draw [c] (7.41614,1.65215) -- (7.41614,1.82337);
\draw [c] (7.41614,1.82337) -- (7.41614,1.93274);
\draw [c] (7.40132,1.82337) -- (7.41614,1.82337);
\draw [c] (7.41614,1.82337) -- (7.43096,1.82337);
\definecolor{c}{rgb}{0,0,0};
\colorlet{c}{kugray};
\draw [c] (7.44578,1.61682) -- (7.44578,1.76694);
\draw [c] (7.44578,1.76694) -- (7.44578,1.86741);
\draw [c] (7.43096,1.76694) -- (7.44578,1.76694);
\draw [c] (7.44578,1.76694) -- (7.4606,1.76694);
\definecolor{c}{rgb}{0,0,0};
\colorlet{c}{kugray};
\draw [c] (7.47541,0.903192) -- (7.47541,1.30472);
\draw [c] (7.47541,1.30472) -- (7.47541,1.47322);
\draw [c] (7.4606,1.30472) -- (7.47541,1.30472);
\draw [c] (7.47541,1.30472) -- (7.49023,1.30472);
\definecolor{c}{rgb}{0,0,0};
\colorlet{c}{kugray};
\draw [c] (7.50505,1.17257) -- (7.50505,1.55106);
\draw [c] (7.50505,1.55106) -- (7.50505,1.71578);
\draw [c] (7.49023,1.55106) -- (7.50505,1.55106);
\draw [c] (7.50505,1.55106) -- (7.51987,1.55106);
\definecolor{c}{rgb}{0,0,0};
\colorlet{c}{kugray};
\draw [c] (7.53469,1.26039) -- (7.53469,1.5276);
\draw [c] (7.53469,1.5276) -- (7.53469,1.66846);
\draw [c] (7.51987,1.5276) -- (7.53469,1.5276);
\draw [c] (7.53469,1.5276) -- (7.5495,1.5276);
\definecolor{c}{rgb}{0,0,0};
\colorlet{c}{kugray};
\draw [c] (7.56432,1.35089) -- (7.56432,1.65389);
\draw [c] (7.56432,1.65389) -- (7.56432,1.80359);
\draw [c] (7.5495,1.65389) -- (7.56432,1.65389);
\draw [c] (7.56432,1.65389) -- (7.57914,1.65389);
\definecolor{c}{rgb}{0,0,0};
\colorlet{c}{kugray};
\draw [c] (7.59396,1.63685) -- (7.59396,1.81433);
\draw [c] (7.59396,1.81433) -- (7.59396,1.92619);
\draw [c] (7.57914,1.81433) -- (7.59396,1.81433);
\draw [c] (7.59396,1.81433) -- (7.60877,1.81433);
\definecolor{c}{rgb}{0,0,0};
\colorlet{c}{kugray};
\draw [c] (7.62359,1.2207) -- (7.62359,1.48705);
\draw [c] (7.62359,1.48705) -- (7.62359,1.62768);
\draw [c] (7.60877,1.48705) -- (7.62359,1.48705);
\draw [c] (7.62359,1.48705) -- (7.63841,1.48705);
\definecolor{c}{rgb}{0,0,0};
\colorlet{c}{kugray};
\draw [c] (7.65323,1.15339) -- (7.65323,1.44836);
\draw [c] (7.65323,1.44836) -- (7.65323,1.5962);
\draw [c] (7.63841,1.44836) -- (7.65323,1.44836);
\draw [c] (7.65323,1.44836) -- (7.66805,1.44836);
\definecolor{c}{rgb}{0,0,0};
\colorlet{c}{kugray};
\draw [c] (7.68286,0.824113) -- (7.68286,1.32118);
\draw [c] (7.68286,1.32118) -- (7.68286,1.50231);
\draw [c] (7.66805,1.32118) -- (7.68286,1.32118);
\draw [c] (7.68286,1.32118) -- (7.69768,1.32118);
\definecolor{c}{rgb}{0,0,0};
\colorlet{c}{kugray};
\draw [c] (7.7125,1.26126) -- (7.7125,1.5339);
\draw [c] (7.7125,1.5339) -- (7.7125,1.67619);
\draw [c] (7.69768,1.5339) -- (7.7125,1.5339);
\draw [c] (7.7125,1.5339) -- (7.72732,1.5339);
\definecolor{c}{rgb}{0,0,0};
\colorlet{c}{kugray};
\draw [c] (7.74214,1.48045) -- (7.74214,1.72479);
\draw [c] (7.74214,1.72479) -- (7.74214,1.85928);
\draw [c] (7.72732,1.72479) -- (7.74214,1.72479);
\draw [c] (7.74214,1.72479) -- (7.75695,1.72479);
\definecolor{c}{rgb}{0,0,0};
\colorlet{c}{kugray};
\draw [c] (7.77177,1.11205) -- (7.77177,1.40937);
\draw [c] (7.77177,1.40937) -- (7.77177,1.55776);
\draw [c] (7.75695,1.40937) -- (7.77177,1.40937);
\draw [c] (7.77177,1.40937) -- (7.78659,1.40937);
\definecolor{c}{rgb}{0,0,0};
\colorlet{c}{kugray};
\draw [c] (7.80141,1.34931) -- (7.80141,1.63981);
\draw [c] (7.80141,1.63981) -- (7.80141,1.78657);
\draw [c] (7.78659,1.63981) -- (7.80141,1.63981);
\draw [c] (7.80141,1.63981) -- (7.81623,1.63981);
\definecolor{c}{rgb}{0,0,0};
\colorlet{c}{kugray};
\draw [c] (7.83104,1.0292) -- (7.83104,1.41463);
\draw [c] (7.83104,1.41463) -- (7.83104,1.58052);
\draw [c] (7.81623,1.41463) -- (7.83104,1.41463);
\draw [c] (7.83104,1.41463) -- (7.84586,1.41463);
\definecolor{c}{rgb}{0,0,0};
\colorlet{c}{kugray};
\draw [c] (7.86068,0.596817) -- (7.86068,1.16974);
\draw [c] (7.86068,1.16974) -- (7.86068,1.38313);
\draw [c] (7.84586,1.16974) -- (7.86068,1.16974);
\draw [c] (7.86068,1.16974) -- (7.8755,1.16974);
\definecolor{c}{rgb}{0,0,0};
\colorlet{c}{kugray};
\draw [c] (7.89031,1.22506) -- (7.89031,1.49977);
\draw [c] (7.89031,1.49977) -- (7.89031,1.64259);
\draw [c] (7.8755,1.49977) -- (7.89031,1.49977);
\draw [c] (7.89031,1.49977) -- (7.90513,1.49977);
\definecolor{c}{rgb}{0,0,0};
\colorlet{c}{kugray};
\draw [c] (7.91995,1.37554) -- (7.91995,1.58913);
\draw [c] (7.91995,1.58913) -- (7.91995,1.71402);
\draw [c] (7.90513,1.58913) -- (7.91995,1.58913);
\draw [c] (7.91995,1.58913) -- (7.93477,1.58913);
\definecolor{c}{rgb}{0,0,0};
\colorlet{c}{kugray};
\draw [c] (7.94959,0.970556) -- (7.94959,1.35289);
\draw [c] (7.94959,1.35289) -- (7.94959,1.51826);
\draw [c] (7.93477,1.35289) -- (7.94959,1.35289);
\draw [c] (7.94959,1.35289) -- (7.9644,1.35289);
\definecolor{c}{rgb}{0,0,0};
\colorlet{c}{kugray};
\draw [c] (7.97922,1.42663) -- (7.97922,1.62932);
\draw [c] (7.97922,1.62932) -- (7.97922,1.75049);
\draw [c] (7.9644,1.62932) -- (7.97922,1.62932);
\draw [c] (7.97922,1.62932) -- (7.99404,1.62932);
\definecolor{c}{rgb}{0,0,0};
\colorlet{c}{kugray};
\draw [c] (8.00886,1.24387) -- (8.00886,1.51218);
\draw [c] (8.00886,1.51218) -- (8.00886,1.65334);
\draw [c] (7.99404,1.51218) -- (8.00886,1.51218);
\draw [c] (8.00886,1.51218) -- (8.02368,1.51218);
\definecolor{c}{rgb}{0,0,0};
\colorlet{c}{kugray};
\draw [c] (8.03849,1.45206) -- (8.03849,1.64214);
\draw [c] (8.03849,1.64214) -- (8.03849,1.75879);
\draw [c] (8.02368,1.64214) -- (8.03849,1.64214);
\draw [c] (8.03849,1.64214) -- (8.05331,1.64214);
\definecolor{c}{rgb}{0,0,0};
\colorlet{c}{kugray};
\draw [c] (8.09776,1.35157) -- (8.09776,1.5665);
\draw [c] (8.09776,1.5665) -- (8.09776,1.69184);
\draw [c] (8.08295,1.5665) -- (8.09776,1.5665);
\draw [c] (8.09776,1.5665) -- (8.11258,1.5665);
\definecolor{c}{rgb}{0,0,0};
\colorlet{c}{kugray};
\draw [c] (8.1274,1.02943) -- (8.1274,1.41864);
\draw [c] (8.1274,1.41864) -- (8.1274,1.58516);
\draw [c] (8.11258,1.41864) -- (8.1274,1.41864);
\draw [c] (8.1274,1.41864) -- (8.14222,1.41864);
\definecolor{c}{rgb}{0,0,0};
\colorlet{c}{kugray};
\draw [c] (8.18667,1.3217) -- (8.18667,1.60896);
\draw [c] (8.18667,1.60896) -- (8.18667,1.75493);
\draw [c] (8.17185,1.60896) -- (8.18667,1.60896);
\draw [c] (8.18667,1.60896) -- (8.20149,1.60896);
\definecolor{c}{rgb}{0,0,0};
\colorlet{c}{kugray};
\draw [c] (8.21631,1.44991) -- (8.21631,1.72488);
\draw [c] (8.21631,1.72488) -- (8.21631,1.86777);
\draw [c] (8.20149,1.72488) -- (8.21631,1.72488);
\draw [c] (8.21631,1.72488) -- (8.23113,1.72488);
\definecolor{c}{rgb}{0,0,0};
\colorlet{c}{kugray};
\draw [c] (8.24594,0.596817) -- (8.24594,1.18479);
\draw [c] (8.24594,1.18479) -- (8.24594,1.39818);
\draw [c] (8.23113,1.18479) -- (8.24594,1.18479);
\draw [c] (8.24594,1.18479) -- (8.26076,1.18479);
\definecolor{c}{rgb}{0,0,0};
\colorlet{c}{kugray};
\draw [c] (8.27558,1.33422) -- (8.27558,1.70646);
\draw [c] (8.27558,1.70646) -- (8.27558,1.8701);
\draw [c] (8.26076,1.70646) -- (8.27558,1.70646);
\draw [c] (8.27558,1.70646) -- (8.2904,1.70646);
\definecolor{c}{rgb}{0,0,0};
\colorlet{c}{kugray};
\draw [c] (8.30521,1.01154) -- (8.30521,1.39412);
\draw [c] (8.30521,1.39412) -- (8.30521,1.55954);
\draw [c] (8.2904,1.39412) -- (8.30521,1.39412);
\draw [c] (8.30521,1.39412) -- (8.32003,1.39412);
\definecolor{c}{rgb}{0,0,0};
\colorlet{c}{kugray};
\draw [c] (8.33485,1.16538) -- (8.33485,1.451);
\draw [c] (8.33485,1.451) -- (8.33485,1.59657);
\draw [c] (8.32003,1.451) -- (8.33485,1.451);
\draw [c] (8.33485,1.451) -- (8.34967,1.451);
\definecolor{c}{rgb}{0,0,0};
\colorlet{c}{kugray};
\draw [c] (8.36449,0.596817) -- (8.36449,1.27192);
\draw [c] (8.36449,1.27192) -- (8.36449,1.48532);
\draw [c] (8.34967,1.27192) -- (8.36449,1.27192);
\draw [c] (8.36449,1.27192) -- (8.3793,1.27192);
\definecolor{c}{rgb}{0,0,0};
\colorlet{c}{kugray};
\draw [c] (8.39412,0.596817) -- (8.39412,1.1669);
\draw [c] (8.39412,1.1669) -- (8.39412,1.38029);
\draw [c] (8.3793,1.1669) -- (8.39412,1.1669);
\draw [c] (8.39412,1.1669) -- (8.40894,1.1669);
\definecolor{c}{rgb}{0,0,0};
\colorlet{c}{kugray};
\draw [c] (8.45339,1.13633) -- (8.45339,1.56977);
\draw [c] (8.45339,1.56977) -- (8.45339,1.74299);
\draw [c] (8.43858,1.56977) -- (8.45339,1.56977);
\draw [c] (8.45339,1.56977) -- (8.46821,1.56977);
\definecolor{c}{rgb}{0,0,0};
\colorlet{c}{kugray};
\draw [c] (8.48303,0.596817) -- (8.48303,1.22594);
\draw [c] (8.48303,1.22594) -- (8.48303,1.43933);
\draw [c] (8.46821,1.22594) -- (8.48303,1.22594);
\draw [c] (8.48303,1.22594) -- (8.49785,1.22594);
\definecolor{c}{rgb}{0,0,0};
\colorlet{c}{kugray};
\draw [c] (8.5423,0.596817) -- (8.5423,1.08304);
\draw [c] (8.5423,1.08304) -- (8.5423,1.29643);
\draw [c] (8.52748,1.08304) -- (8.5423,1.08304);
\draw [c] (8.5423,1.08304) -- (8.55712,1.08304);
\definecolor{c}{rgb}{0,0,0};
\colorlet{c}{kugray};
\draw [c] (8.57194,0.596817) -- (8.57194,1.19595);
\draw [c] (8.57194,1.19595) -- (8.57194,1.40934);
\draw [c] (8.55712,1.19595) -- (8.57194,1.19595);
\draw [c] (8.57194,1.19595) -- (8.58675,1.19595);
\definecolor{c}{rgb}{0,0,0};
\colorlet{c}{kugray};
\draw [c] (8.66084,0.884632) -- (8.66084,1.27717);
\draw [c] (8.66084,1.27717) -- (8.66084,1.44423);
\draw [c] (8.64603,1.27717) -- (8.66084,1.27717);
\draw [c] (8.66084,1.27717) -- (8.67566,1.27717);
\definecolor{c}{rgb}{0,0,0};
\colorlet{c}{kugray};
\draw [c] (8.69048,0.596817) -- (8.69048,1.18479);
\draw [c] (8.69048,1.18479) -- (8.69048,1.39818);
\draw [c] (8.67566,1.18479) -- (8.69048,1.18479);
\draw [c] (8.69048,1.18479) -- (8.7053,1.18479);
\definecolor{c}{rgb}{0,0,0};
\colorlet{c}{kugray};
\draw [c] (8.72012,0.596817) -- (8.72012,1.21188);
\draw [c] (8.72012,1.21188) -- (8.72012,1.42527);
\draw [c] (8.7053,1.21188) -- (8.72012,1.21188);
\draw [c] (8.72012,1.21188) -- (8.73493,1.21188);
\definecolor{c}{rgb}{0,0,0};
\colorlet{c}{kugray};
\draw [c] (8.74975,0.596817) -- (8.74975,1.16598);
\draw [c] (8.74975,1.16598) -- (8.74975,1.37937);
\draw [c] (8.73493,1.16598) -- (8.74975,1.16598);
\draw [c] (8.74975,1.16598) -- (8.76457,1.16598);
\definecolor{c}{rgb}{0,0,0};
\colorlet{c}{kugray};
\draw [c] (8.77939,0.895648) -- (8.77939,1.31039);
\draw [c] (8.77939,1.31039) -- (8.77939,1.48091);
\draw [c] (8.76457,1.31039) -- (8.77939,1.31039);
\draw [c] (8.77939,1.31039) -- (8.79421,1.31039);
\definecolor{c}{rgb}{0,0,0};
\colorlet{c}{kugray};
\draw [c] (8.80902,1.29252) -- (8.80902,1.5583);
\draw [c] (8.80902,1.5583) -- (8.80902,1.69878);
\draw [c] (8.79421,1.5583) -- (8.80902,1.5583);
\draw [c] (8.80902,1.5583) -- (8.82384,1.5583);
\definecolor{c}{rgb}{0,0,0};
\colorlet{c}{kugray};
\draw [c] (8.83866,0.596817) -- (8.83866,1.24072);
\draw [c] (8.83866,1.24072) -- (8.83866,1.45411);
\draw [c] (8.82384,1.24072) -- (8.83866,1.24072);
\draw [c] (8.83866,1.24072) -- (8.85348,1.24072);
\definecolor{c}{rgb}{0,0,0};
\colorlet{c}{kugray};
\draw [c] (8.86829,0.596817) -- (8.86829,0.976443);
\draw [c] (8.86829,0.976443) -- (8.86829,1.18983);
\draw [c] (8.85348,0.976443) -- (8.86829,0.976443);
\draw [c] (8.86829,0.976443) -- (8.88311,0.976443);
\definecolor{c}{rgb}{0,0,0};
\colorlet{c}{kugray};
\draw [c] (8.92757,0.596817) -- (8.92757,1.19654);
\draw [c] (8.92757,1.19654) -- (8.92757,1.40993);
\draw [c] (8.91275,1.19654) -- (8.92757,1.19654);
\draw [c] (8.92757,1.19654) -- (8.94238,1.19654);
\definecolor{c}{rgb}{0,0,0};
\colorlet{c}{kugray};
\draw [c] (9.01647,0.596817) -- (9.01647,1.23526);
\draw [c] (9.01647,1.23526) -- (9.01647,1.44865);
\draw [c] (9.00166,1.23526) -- (9.01647,1.23526);
\draw [c] (9.01647,1.23526) -- (9.03129,1.23526);
\definecolor{c}{rgb}{0,0,0};
\colorlet{c}{kugray};
\draw [c] (9.07574,0.596817) -- (9.07574,1.10612);
\draw [c] (9.07574,1.10612) -- (9.07574,1.31951);
\draw [c] (9.06093,1.10612) -- (9.07574,1.10612);
\draw [c] (9.07574,1.10612) -- (9.09056,1.10612);
\definecolor{c}{rgb}{0,0,0};
\colorlet{c}{kugray};
\draw [c] (9.10538,0.596817) -- (9.10538,1.18963);
\draw [c] (9.10538,1.18963) -- (9.10538,1.40302);
\draw [c] (9.09056,1.18963) -- (9.10538,1.18963);
\draw [c] (9.10538,1.18963) -- (9.1202,1.18963);
\definecolor{c}{rgb}{0,0,0};
\colorlet{c}{kugray};
\draw [c] (9.13502,0.596817) -- (9.13502,1.14382);
\draw [c] (9.13502,1.14382) -- (9.13502,1.35721);
\draw [c] (9.1202,1.14382) -- (9.13502,1.14382);
\draw [c] (9.13502,1.14382) -- (9.14983,1.14382);
\definecolor{c}{rgb}{0,0,0};
\colorlet{c}{kugray};
\draw [c] (9.22392,1.0928) -- (9.22392,1.47755);
\draw [c] (9.22392,1.47755) -- (9.22392,1.64333);
\draw [c] (9.20911,1.47755) -- (9.22392,1.47755);
\draw [c] (9.22392,1.47755) -- (9.23874,1.47755);
\definecolor{c}{rgb}{0,0,0};
\colorlet{c}{kugray};
\draw [c] (9.2832,0.596817) -- (9.2832,1.19105);
\draw [c] (9.2832,1.19105) -- (9.2832,1.40444);
\draw [c] (9.26838,1.19105) -- (9.2832,1.19105);
\draw [c] (9.2832,1.19105) -- (9.29801,1.19105);
\definecolor{c}{rgb}{0,0,0};
\colorlet{c}{kugray};
\draw [c] (9.34247,0.596817) -- (9.34247,1.0722);
\draw [c] (9.34247,1.0722) -- (9.34247,1.28559);
\draw [c] (9.32765,1.0722) -- (9.34247,1.0722);
\draw [c] (9.34247,1.0722) -- (9.35728,1.0722);
\definecolor{c}{rgb}{0,0,0};
\colorlet{c}{kugray};
\draw [c] (9.54992,0.596817) -- (9.54992,1.17593);
\draw [c] (9.54992,1.17593) -- (9.54992,1.38932);
\draw [c] (9.5351,1.17593) -- (9.54992,1.17593);
\draw [c] (9.54992,1.17593) -- (9.56474,1.17593);
\definecolor{c}{rgb}{0,0,0};
\colorlet{c}{kugray};
\draw [c] (9.60919,0.596817) -- (9.60919,1.45456);
\draw [c] (9.60919,1.45456) -- (9.60919,1.66795);
\draw [c] (9.59437,1.45456) -- (9.60919,1.45456);
\draw [c] (9.60919,1.45456) -- (9.62401,1.45456);
\definecolor{c}{rgb}{0,0,0};
\colorlet{c}{kugray};
\draw [c] (9.63882,0.596817) -- (9.63882,1.11893);
\draw [c] (9.63882,1.11893) -- (9.63882,1.33232);
\draw [c] (9.62401,1.11893) -- (9.63882,1.11893);
\draw [c] (9.63882,1.11893) -- (9.65364,1.11893);
\definecolor{c}{rgb}{0,0,0};
\colorlet{c}{kugray};
\draw [c] (9.72773,0.596817) -- (9.72773,1.59979);
\draw [c] (9.72773,1.59979) -- (9.72773,1.81318);
\draw [c] (9.71291,1.59979) -- (9.72773,1.59979);
\draw [c] (9.72773,1.59979) -- (9.74255,1.59979);
\definecolor{c}{rgb}{0,0,0};
\colorlet{c}{kugray};
\draw [c] (9.787,0.596817) -- (9.787,1.11893);
\draw [c] (9.787,1.11893) -- (9.787,1.33232);
\draw [c] (9.77219,1.11893) -- (9.787,1.11893);
\draw [c] (9.787,1.11893) -- (9.80182,1.11893);
\definecolor{c}{rgb}{0,0,0};
\colorlet{c}{kugray};
\draw [c] (9.90555,0.596817) -- (9.90555,1.17321);
\draw [c] (9.90555,1.17321) -- (9.90555,1.3866);
\draw [c] (9.89073,1.17321) -- (9.90555,1.17321);
\draw [c] (9.90555,1.17321) -- (9.92036,1.17321);
\definecolor{c}{rgb}{0,0,0};
\colorlet{c}{kugray};
\draw [c] (9.93518,0.596817) -- (9.93518,1.17077);
\draw [c] (9.93518,1.17077) -- (9.93518,1.38417);
\draw [c] (9.92036,1.17077) -- (9.93518,1.17077);
\draw [c] (9.93518,1.17077) -- (9.95,1.17077);
\definecolor{c}{rgb}{0,0,0};
\colorlet{c}{natgreen};
\draw [c] (1.51655,5.54533) -- (1.60131,5.40419) -- (1.68607,5.2747) -- (1.77083,5.15478) -- (1.85558,5.04288) -- (1.94034,4.93778) -- (2.0251,4.83855) -- (2.10986,4.74443) -- (2.19462,4.65481) -- (2.27938,4.56919) -- (2.36413,4.48712)
 -- (2.44889,4.40827) -- (2.53365,4.33233) -- (2.61841,4.25902) -- (2.70317,4.18813) -- (2.78793,4.11945) -- (2.87268,4.05281) -- (2.95744,3.98806) -- (3.0422,3.92505) -- (3.12696,3.86367) -- (3.21172,3.80381) -- (3.29648,3.74536) --
 (3.38123,3.68824) -- (3.46599,3.63236) -- (3.55075,3.57765) -- (3.63551,3.52404) -- (3.72027,3.47148) -- (3.80502,3.4199) -- (3.88978,3.36925) -- (3.97454,3.31949) -- (4.0593,3.27056) -- (4.14406,3.22243) -- (4.22882,3.17506) -- (4.31357,3.12841) --
 (4.39833,3.08245) -- (4.48309,3.03714) -- (4.56785,2.99247) -- (4.65261,2.94839) -- (4.73737,2.90489) -- (4.82212,2.86194) -- (4.90688,2.81952) -- (4.99164,2.7776) -- (5.0764,2.73617) -- (5.16116,2.6952) -- (5.24592,2.65468) -- (5.33067,2.6146) --
 (5.41543,2.57492) -- (5.50019,2.53565) -- (5.58495,2.49676) -- (5.66971,2.45824);
\draw [c] (5.66971,2.45824) -- (5.75447,2.42007) -- (5.83922,2.38225) -- (5.92398,2.34476) -- (6.00874,2.30759) -- (6.0935,2.27073) -- (6.17826,2.23416) -- (6.26301,2.19789) -- (6.34777,2.16188) -- (6.43253,2.12615) --
 (6.51729,2.09068) -- (6.60205,2.05545) -- (6.68681,2.02047) -- (6.77156,1.98572) -- (6.85632,1.9512) -- (6.94108,1.91689) -- (7.02584,1.88279) -- (7.1106,1.8489) -- (7.19536,1.81521) -- (7.28011,1.78171) -- (7.36487,1.74839) -- (7.44963,1.71525) --
 (7.53439,1.68228) -- (7.61915,1.64948) -- (7.70391,1.61684) -- (7.78866,1.58435) -- (7.87342,1.55202) -- (7.95818,1.51983) -- (8.04294,1.48778) -- (8.1277,1.45586) -- (8.21246,1.42408) -- (8.29721,1.39242) -- (8.38197,1.36088) -- (8.46673,1.32946)
 -- (8.55149,1.29816) -- (8.63625,1.26696) -- (8.72101,1.23587) -- (8.80576,1.20487) -- (8.89052,1.17398) -- (8.97528,1.14317) -- (9.06004,1.11246) -- (9.1448,1.08183) -- (9.22955,1.05129) -- (9.31431,1.02082) -- (9.39907,0.990431) --
 (9.48383,0.960111) -- (9.56859,0.929859) -- (9.65335,0.899673) -- (9.7381,0.869549) -- (9.82286,0.839483);
\draw [c] (9.82286,0.839483) -- (9.90762,0.809473);
\colorlet{c}{kugray};
\draw [c] (1.07409,0.596817) -- (1.07409,1.40897);
\draw [c] (1.07409,1.40897) -- (1.07409,1.62236);
\draw [c] (1.05927,1.40897) -- (1.07409,1.40897);
\draw [c] (1.07409,1.40897) -- (1.08891,1.40897);
\definecolor{c}{rgb}{0,0,0};
\colorlet{c}{kugray};
\draw [c] (1.10373,0.596817) -- (1.10373,3.52791);
\draw [c] (1.10373,3.52791) -- (1.10373,3.7413);
\draw [c] (1.08891,3.52791) -- (1.10373,3.52791);
\draw [c] (1.10373,3.52791) -- (1.11854,3.52791);
\definecolor{c}{rgb}{0,0,0};
\colorlet{c}{kugray};
\draw [c] (1.13336,3.30584) -- (1.13336,3.68425);
\draw [c] (1.13336,3.68425) -- (1.13336,3.84896);
\draw [c] (1.11854,3.68425) -- (1.13336,3.68425);
\draw [c] (1.13336,3.68425) -- (1.14818,3.68425);
\definecolor{c}{rgb}{0,0,0};
\colorlet{c}{kugray};
\draw [c] (1.163,0.596817) -- (1.163,3.53823);
\draw [c] (1.163,3.53823) -- (1.163,3.75162);
\draw [c] (1.14818,3.53823) -- (1.163,3.53823);
\draw [c] (1.163,3.53823) -- (1.17781,3.53823);
\definecolor{c}{rgb}{0,0,0};
\colorlet{c}{kugray};
\draw [c] (1.19263,3.61695) -- (1.19263,3.89008);
\draw [c] (1.19263,3.89008) -- (1.19263,4.03249);
\draw [c] (1.17781,3.89008) -- (1.19263,3.89008);
\draw [c] (1.19263,3.89008) -- (1.20745,3.89008);
\definecolor{c}{rgb}{0,0,0};
\colorlet{c}{kugray};
\draw [c] (1.22227,3.83075) -- (1.22227,4.01677);
\draw [c] (1.22227,4.01677) -- (1.22227,4.13191);
\draw [c] (1.20745,4.01677) -- (1.22227,4.01677);
\draw [c] (1.22227,4.01677) -- (1.23709,4.01677);
\definecolor{c}{rgb}{0,0,0};
\colorlet{c}{kugray};
\draw [c] (1.2519,3.71523) -- (1.2519,3.93196);
\draw [c] (1.2519,3.93196) -- (1.2519,4.05789);
\draw [c] (1.23709,3.93196) -- (1.2519,3.93196);
\draw [c] (1.2519,3.93196) -- (1.26672,3.93196);
\definecolor{c}{rgb}{0,0,0};
\colorlet{c}{kugray};
\draw [c] (1.28154,3.75372) -- (1.28154,3.97484);
\draw [c] (1.28154,3.97484) -- (1.28154,4.1022);
\draw [c] (1.26672,3.97484) -- (1.28154,3.97484);
\draw [c] (1.28154,3.97484) -- (1.29636,3.97484);
\definecolor{c}{rgb}{0,0,0};
\colorlet{c}{kugray};
\draw [c] (1.31118,5.36683) -- (1.31118,5.38443);
\draw [c] (1.31118,5.38443) -- (1.31118,5.40108);
\draw [c] (1.29636,5.38443) -- (1.31118,5.38443);
\draw [c] (1.31118,5.38443) -- (1.32599,5.38443);
\definecolor{c}{rgb}{0,0,0};
\colorlet{c}{kugray};
\draw [c] (1.34081,5.64934) -- (1.34081,5.66037);
\draw [c] (1.34081,5.66037) -- (1.34081,5.67102);
\draw [c] (1.32599,5.66037) -- (1.34081,5.66037);
\draw [c] (1.34081,5.66037) -- (1.35563,5.66037);
\definecolor{c}{rgb}{0,0,0};
\colorlet{c}{kugray};
\draw [c] (1.37045,5.69165) -- (1.37045,5.70189);
\draw [c] (1.37045,5.70189) -- (1.37045,5.7118);
\draw [c] (1.35563,5.70189) -- (1.37045,5.70189);
\draw [c] (1.37045,5.70189) -- (1.38526,5.70189);
\definecolor{c}{rgb}{0,0,0};
\colorlet{c}{kugray};
\draw [c] (1.40008,5.69477) -- (1.40008,5.70516);
\draw [c] (1.40008,5.70516) -- (1.40008,5.71522);
\draw [c] (1.38526,5.70516) -- (1.40008,5.70516);
\draw [c] (1.40008,5.70516) -- (1.4149,5.70516);
\definecolor{c}{rgb}{0,0,0};
\colorlet{c}{kugray};
\draw [c] (1.42972,5.6476) -- (1.42972,5.65871);
\draw [c] (1.42972,5.65871) -- (1.42972,5.66943);
\draw [c] (1.4149,5.65871) -- (1.42972,5.65871);
\draw [c] (1.42972,5.65871) -- (1.44454,5.65871);
\definecolor{c}{rgb}{0,0,0};
\colorlet{c}{kugray};
\draw [c] (1.45935,5.62032) -- (1.45935,5.63176);
\draw [c] (1.45935,5.63176) -- (1.45935,5.6428);
\draw [c] (1.44454,5.63176) -- (1.45935,5.63176);
\draw [c] (1.45935,5.63176) -- (1.47417,5.63176);
\definecolor{c}{rgb}{0,0,0};
\colorlet{c}{kugray};
\draw [c] (1.48899,5.56116) -- (1.48899,5.57382);
\draw [c] (1.48899,5.57382) -- (1.48899,5.58597);
\draw [c] (1.47417,5.57382) -- (1.48899,5.57382);
\draw [c] (1.48899,5.57382) -- (1.50381,5.57382);
\definecolor{c}{rgb}{0,0,0};
\colorlet{c}{kugray};
\draw [c] (1.51863,5.51885) -- (1.51863,5.53274);
\draw [c] (1.51863,5.53274) -- (1.51863,5.54604);
\draw [c] (1.50381,5.53274) -- (1.51863,5.53274);
\draw [c] (1.51863,5.53274) -- (1.53344,5.53274);
\definecolor{c}{rgb}{0,0,0};
\colorlet{c}{kugray};
\draw [c] (1.54826,5.46762) -- (1.54826,5.48249);
\draw [c] (1.54826,5.48249) -- (1.54826,5.49667);
\draw [c] (1.53344,5.48249) -- (1.54826,5.48249);
\draw [c] (1.54826,5.48249) -- (1.56308,5.48249);
\definecolor{c}{rgb}{0,0,0};
\colorlet{c}{kugray};
\draw [c] (1.5779,5.40546) -- (1.5779,5.42176);
\draw [c] (1.5779,5.42176) -- (1.5779,5.43724);
\draw [c] (1.56308,5.42176) -- (1.5779,5.42176);
\draw [c] (1.5779,5.42176) -- (1.59272,5.42176);
\definecolor{c}{rgb}{0,0,0};
\colorlet{c}{kugray};
\draw [c] (1.60753,5.41122) -- (1.60753,5.42791);
\draw [c] (1.60753,5.42791) -- (1.60753,5.44375);
\draw [c] (1.59272,5.42791) -- (1.60753,5.42791);
\draw [c] (1.60753,5.42791) -- (1.62235,5.42791);
\definecolor{c}{rgb}{0,0,0};
\colorlet{c}{kugray};
\draw [c] (1.63717,5.33643) -- (1.63717,5.35499);
\draw [c] (1.63717,5.35499) -- (1.63717,5.37248);
\draw [c] (1.62235,5.35499) -- (1.63717,5.35499);
\draw [c] (1.63717,5.35499) -- (1.65199,5.35499);
\definecolor{c}{rgb}{0,0,0};
\colorlet{c}{kugray};
\draw [c] (1.6668,5.32522) -- (1.6668,5.34431);
\draw [c] (1.6668,5.34431) -- (1.6668,5.36228);
\draw [c] (1.65199,5.34431) -- (1.6668,5.34431);
\draw [c] (1.6668,5.34431) -- (1.68162,5.34431);
\definecolor{c}{rgb}{0,0,0};
\colorlet{c}{kugray};
\draw [c] (1.69644,5.24802) -- (1.69644,5.26862);
\draw [c] (1.69644,5.26862) -- (1.69644,5.28793);
\draw [c] (1.68162,5.26862) -- (1.69644,5.26862);
\draw [c] (1.69644,5.26862) -- (1.71126,5.26862);
\definecolor{c}{rgb}{0,0,0};
\colorlet{c}{kugray};
\draw [c] (1.72608,5.18795) -- (1.72608,5.21178);
\draw [c] (1.72608,5.21178) -- (1.72608,5.23389);
\draw [c] (1.71126,5.21178) -- (1.72608,5.21178);
\draw [c] (1.72608,5.21178) -- (1.74089,5.21178);
\definecolor{c}{rgb}{0,0,0};
\colorlet{c}{kugray};
\draw [c] (1.75571,5.164) -- (1.75571,5.18723);
\draw [c] (1.75571,5.18723) -- (1.75571,5.20882);
\draw [c] (1.74089,5.18723) -- (1.75571,5.18723);
\draw [c] (1.75571,5.18723) -- (1.77053,5.18723);
\definecolor{c}{rgb}{0,0,0};
\colorlet{c}{kugray};
\draw [c] (1.78535,5.12784) -- (1.78535,5.15344);
\draw [c] (1.78535,5.15344) -- (1.78535,5.17707);
\draw [c] (1.77053,5.15344) -- (1.78535,5.15344);
\draw [c] (1.78535,5.15344) -- (1.80017,5.15344);
\definecolor{c}{rgb}{0,0,0};
\colorlet{c}{kugray};
\draw [c] (1.81498,5.07595) -- (1.81498,5.10297);
\draw [c] (1.81498,5.10297) -- (1.81498,5.1278);
\draw [c] (1.80017,5.10297) -- (1.81498,5.10297);
\draw [c] (1.81498,5.10297) -- (1.8298,5.10297);
\definecolor{c}{rgb}{0,0,0};
\colorlet{c}{kugray};
\draw [c] (1.84462,5.05921) -- (1.84462,5.0876);
\draw [c] (1.84462,5.0876) -- (1.84462,5.11359);
\draw [c] (1.8298,5.0876) -- (1.84462,5.0876);
\draw [c] (1.84462,5.0876) -- (1.85944,5.0876);
\definecolor{c}{rgb}{0,0,0};
\colorlet{c}{kugray};
\draw [c] (1.87425,4.98897) -- (1.87425,5.02142);
\draw [c] (1.87425,5.02142) -- (1.87425,5.05078);
\draw [c] (1.85944,5.02142) -- (1.87425,5.02142);
\draw [c] (1.87425,5.02142) -- (1.88907,5.02142);
\definecolor{c}{rgb}{0,0,0};
\colorlet{c}{kugray};
\draw [c] (1.90389,4.87515) -- (1.90389,4.914);
\draw [c] (1.90389,4.914) -- (1.90389,4.94849);
\draw [c] (1.88907,4.914) -- (1.90389,4.914);
\draw [c] (1.90389,4.914) -- (1.91871,4.914);
\definecolor{c}{rgb}{0,0,0};
\colorlet{c}{kugray};
\draw [c] (1.93353,4.89763) -- (1.93353,4.9339);
\draw [c] (1.93353,4.9339) -- (1.93353,4.96634);
\draw [c] (1.91871,4.9339) -- (1.93353,4.9339);
\draw [c] (1.93353,4.9339) -- (1.94834,4.9339);
\definecolor{c}{rgb}{0,0,0};
\colorlet{c}{kugray};
\draw [c] (1.96316,4.88868) -- (1.96316,4.92763);
\draw [c] (1.96316,4.92763) -- (1.96316,4.96221);
\draw [c] (1.94834,4.92763) -- (1.96316,4.92763);
\draw [c] (1.96316,4.92763) -- (1.97798,4.92763);
\definecolor{c}{rgb}{0,0,0};
\colorlet{c}{kugray};
\draw [c] (1.9928,4.78948) -- (1.9928,4.83208);
\draw [c] (1.9928,4.83208) -- (1.9928,4.8695);
\draw [c] (1.97798,4.83208) -- (1.9928,4.83208);
\draw [c] (1.9928,4.83208) -- (2.00762,4.83208);
\definecolor{c}{rgb}{0,0,0};
\colorlet{c}{kugray};
\draw [c] (2.02243,4.71758) -- (2.02243,4.76449);
\draw [c] (2.02243,4.76449) -- (2.02243,4.80519);
\draw [c] (2.00762,4.76449) -- (2.02243,4.76449);
\draw [c] (2.02243,4.76449) -- (2.03725,4.76449);
\definecolor{c}{rgb}{0,0,0};
\colorlet{c}{kugray};
\draw [c] (2.05207,4.70827) -- (2.05207,4.75577);
\draw [c] (2.05207,4.75577) -- (2.05207,4.79692);
\draw [c] (2.03725,4.75577) -- (2.05207,4.75577);
\draw [c] (2.05207,4.75577) -- (2.06689,4.75577);
\definecolor{c}{rgb}{0,0,0};
\colorlet{c}{kugray};
\draw [c] (2.08171,4.68783) -- (2.08171,4.73949);
\draw [c] (2.08171,4.73949) -- (2.08171,4.78371);
\draw [c] (2.06689,4.73949) -- (2.08171,4.73949);
\draw [c] (2.08171,4.73949) -- (2.09652,4.73949);
\definecolor{c}{rgb}{0,0,0};
\colorlet{c}{kugray};
\draw [c] (2.11134,4.66301) -- (2.11134,4.71819);
\draw [c] (2.11134,4.71819) -- (2.11134,4.76496);
\draw [c] (2.09652,4.71819) -- (2.11134,4.71819);
\draw [c] (2.11134,4.71819) -- (2.12616,4.71819);
\definecolor{c}{rgb}{0,0,0};
\colorlet{c}{kugray};
\draw [c] (2.14098,4.66355) -- (2.14098,4.71683);
\draw [c] (2.14098,4.71683) -- (2.14098,4.76224);
\draw [c] (2.12616,4.71683) -- (2.14098,4.71683);
\draw [c] (2.14098,4.71683) -- (2.15579,4.71683);
\definecolor{c}{rgb}{0,0,0};
\colorlet{c}{kugray};
\draw [c] (2.17061,4.50979) -- (2.17061,4.57305);
\draw [c] (2.17061,4.57305) -- (2.17061,4.6255);
\draw [c] (2.15579,4.57305) -- (2.17061,4.57305);
\draw [c] (2.17061,4.57305) -- (2.18543,4.57305);
\definecolor{c}{rgb}{0,0,0};
\colorlet{c}{kugray};
\draw [c] (2.20025,4.5238) -- (2.20025,4.5936);
\draw [c] (2.20025,4.5936) -- (2.20025,4.65046);
\draw [c] (2.18543,4.5936) -- (2.20025,4.5936);
\draw [c] (2.20025,4.5936) -- (2.21507,4.5936);
\definecolor{c}{rgb}{0,0,0};
\colorlet{c}{kugray};
\draw [c] (2.22988,4.49841) -- (2.22988,4.56586);
\draw [c] (2.22988,4.56586) -- (2.22988,4.62114);
\draw [c] (2.21507,4.56586) -- (2.22988,4.56586);
\draw [c] (2.22988,4.56586) -- (2.2447,4.56586);
\definecolor{c}{rgb}{0,0,0};
\colorlet{c}{kugray};
\draw [c] (2.25952,4.51369) -- (2.25952,4.58591);
\draw [c] (2.25952,4.58591) -- (2.25952,4.64436);
\draw [c] (2.2447,4.58591) -- (2.25952,4.58591);
\draw [c] (2.25952,4.58591) -- (2.27434,4.58591);
\definecolor{c}{rgb}{0,0,0};
\colorlet{c}{kugray};
\draw [c] (2.28916,4.55805) -- (2.28916,4.56892);
\draw [c] (2.28916,4.56892) -- (2.28916,4.57942);
\draw [c] (2.27434,4.56892) -- (2.28916,4.56892);
\draw [c] (2.28916,4.56892) -- (2.30397,4.56892);
\definecolor{c}{rgb}{0,0,0};
\colorlet{c}{kugray};
\draw [c] (2.31879,4.50775) -- (2.31879,4.51917);
\draw [c] (2.31879,4.51917) -- (2.31879,4.53019);
\draw [c] (2.30397,4.51917) -- (2.31879,4.51917);
\draw [c] (2.31879,4.51917) -- (2.33361,4.51917);
\definecolor{c}{rgb}{0,0,0};
\colorlet{c}{kugray};
\draw [c] (2.34843,4.49289) -- (2.34843,4.50481);
\draw [c] (2.34843,4.50481) -- (2.34843,4.5163);
\draw [c] (2.33361,4.50481) -- (2.34843,4.50481);
\draw [c] (2.34843,4.50481) -- (2.36325,4.50481);
\definecolor{c}{rgb}{0,0,0};
\colorlet{c}{kugray};
\draw [c] (2.37806,4.46995) -- (2.37806,4.48228);
\draw [c] (2.37806,4.48228) -- (2.37806,4.49413);
\draw [c] (2.36325,4.48228) -- (2.37806,4.48228);
\draw [c] (2.37806,4.48228) -- (2.39288,4.48228);
\definecolor{c}{rgb}{0,0,0};
\colorlet{c}{kugray};
\draw [c] (2.4077,4.4553) -- (2.4077,4.46803);
\draw [c] (2.4077,4.46803) -- (2.4077,4.48025);
\draw [c] (2.39288,4.46803) -- (2.4077,4.46803);
\draw [c] (2.4077,4.46803) -- (2.42252,4.46803);
\definecolor{c}{rgb}{0,0,0};
\colorlet{c}{kugray};
\draw [c] (2.43733,4.4243) -- (2.43733,4.43793);
\draw [c] (2.43733,4.43793) -- (2.43733,4.45097);
\draw [c] (2.42252,4.43793) -- (2.43733,4.43793);
\draw [c] (2.43733,4.43793) -- (2.45215,4.43793);
\definecolor{c}{rgb}{0,0,0};
\colorlet{c}{kugray};
\draw [c] (2.46697,4.40165) -- (2.46697,4.41557);
\draw [c] (2.46697,4.41557) -- (2.46697,4.42889);
\draw [c] (2.45215,4.41557) -- (2.46697,4.41557);
\draw [c] (2.46697,4.41557) -- (2.48179,4.41557);
\definecolor{c}{rgb}{0,0,0};
\colorlet{c}{kugray};
\draw [c] (2.49661,4.3514) -- (2.49661,4.36606);
\draw [c] (2.49661,4.36606) -- (2.49661,4.38005);
\draw [c] (2.48179,4.36606) -- (2.49661,4.36606);
\draw [c] (2.49661,4.36606) -- (2.51142,4.36606);
\definecolor{c}{rgb}{0,0,0};
\colorlet{c}{kugray};
\draw [c] (2.52624,4.33895) -- (2.52624,4.3543);
\draw [c] (2.52624,4.3543) -- (2.52624,4.36892);
\draw [c] (2.51142,4.3543) -- (2.52624,4.3543);
\draw [c] (2.52624,4.3543) -- (2.54106,4.3543);
\definecolor{c}{rgb}{0,0,0};
\colorlet{c}{kugray};
\draw [c] (2.55588,4.30184) -- (2.55588,4.31852);
\draw [c] (2.55588,4.31852) -- (2.55588,4.33435);
\draw [c] (2.54106,4.31852) -- (2.55588,4.31852);
\draw [c] (2.55588,4.31852) -- (2.5707,4.31852);
\definecolor{c}{rgb}{0,0,0};
\colorlet{c}{kugray};
\draw [c] (2.58551,4.29419) -- (2.58551,4.31073);
\draw [c] (2.58551,4.31073) -- (2.58551,4.32643);
\draw [c] (2.5707,4.31073) -- (2.58551,4.31073);
\draw [c] (2.58551,4.31073) -- (2.60033,4.31073);
\definecolor{c}{rgb}{0,0,0};
\colorlet{c}{kugray};
\draw [c] (2.61515,4.23639) -- (2.61515,4.25491);
\draw [c] (2.61515,4.25491) -- (2.61515,4.27238);
\draw [c] (2.60033,4.25491) -- (2.61515,4.25491);
\draw [c] (2.61515,4.25491) -- (2.62997,4.25491);
\definecolor{c}{rgb}{0,0,0};
\colorlet{c}{kugray};
\draw [c] (2.64478,4.2342) -- (2.64478,4.2521);
\draw [c] (2.64478,4.2521) -- (2.64478,4.26903);
\draw [c] (2.62997,4.2521) -- (2.64478,4.2521);
\draw [c] (2.64478,4.2521) -- (2.6596,4.2521);
\definecolor{c}{rgb}{0,0,0};
\colorlet{c}{kugray};
\draw [c] (2.67442,4.19709) -- (2.67442,4.21658);
\draw [c] (2.67442,4.21658) -- (2.67442,4.23491);
\draw [c] (2.6596,4.21658) -- (2.67442,4.21658);
\draw [c] (2.67442,4.21658) -- (2.68924,4.21658);
\definecolor{c}{rgb}{0,0,0};
\colorlet{c}{kugray};
\draw [c] (2.70406,4.15645) -- (2.70406,4.17742);
\draw [c] (2.70406,4.17742) -- (2.70406,4.19706);
\draw [c] (2.68924,4.17742) -- (2.70406,4.17742);
\draw [c] (2.70406,4.17742) -- (2.71887,4.17742);
\definecolor{c}{rgb}{0,0,0};
\colorlet{c}{kugray};
\draw [c] (2.73369,4.10839) -- (2.73369,4.13042);
\draw [c] (2.73369,4.13042) -- (2.73369,4.15097);
\draw [c] (2.71887,4.13042) -- (2.73369,4.13042);
\draw [c] (2.73369,4.13042) -- (2.74851,4.13042);
\definecolor{c}{rgb}{0,0,0};
\colorlet{c}{kugray};
\draw [c] (2.76333,4.13783) -- (2.76333,4.15915);
\draw [c] (2.76333,4.15915) -- (2.76333,4.1791);
\draw [c] (2.74851,4.15915) -- (2.76333,4.15915);
\draw [c] (2.76333,4.15915) -- (2.77815,4.15915);
\definecolor{c}{rgb}{0,0,0};
\colorlet{c}{kugray};
\draw [c] (2.79296,4.09048) -- (2.79296,4.11309);
\draw [c] (2.79296,4.11309) -- (2.79296,4.13416);
\draw [c] (2.77815,4.11309) -- (2.79296,4.11309);
\draw [c] (2.79296,4.11309) -- (2.80778,4.11309);
\definecolor{c}{rgb}{0,0,0};
\colorlet{c}{kugray};
\draw [c] (2.8226,4.09688) -- (2.8226,4.11923);
\draw [c] (2.8226,4.11923) -- (2.8226,4.14007);
\draw [c] (2.80778,4.11923) -- (2.8226,4.11923);
\draw [c] (2.8226,4.11923) -- (2.83742,4.11923);
\definecolor{c}{rgb}{0,0,0};
\colorlet{c}{kugray};
\draw [c] (2.85224,4.00498) -- (2.85224,4.03051);
\draw [c] (2.85224,4.03051) -- (2.85224,4.05409);
\draw [c] (2.83742,4.03051) -- (2.85224,4.03051);
\draw [c] (2.85224,4.03051) -- (2.86705,4.03051);
\definecolor{c}{rgb}{0,0,0};
\colorlet{c}{kugray};
\draw [c] (2.88187,4.03101) -- (2.88187,4.05663);
\draw [c] (2.88187,4.05663) -- (2.88187,4.08028);
\draw [c] (2.86705,4.05663) -- (2.88187,4.05663);
\draw [c] (2.88187,4.05663) -- (2.89669,4.05663);
\definecolor{c}{rgb}{0,0,0};
\colorlet{c}{kugray};
\draw [c] (2.91151,4.01624) -- (2.91151,4.04198);
\draw [c] (2.91151,4.04198) -- (2.91151,4.06572);
\draw [c] (2.89669,4.04198) -- (2.91151,4.04198);
\draw [c] (2.91151,4.04198) -- (2.92632,4.04198);
\definecolor{c}{rgb}{0,0,0};
\colorlet{c}{kugray};
\draw [c] (2.94114,4.00464) -- (2.94114,4.03106);
\draw [c] (2.94114,4.03106) -- (2.94114,4.0554);
\draw [c] (2.92632,4.03106) -- (2.94114,4.03106);
\draw [c] (2.94114,4.03106) -- (2.95596,4.03106);
\definecolor{c}{rgb}{0,0,0};
\colorlet{c}{kugray};
\draw [c] (2.97078,3.992) -- (2.97078,4.01847);
\draw [c] (2.97078,4.01847) -- (2.97078,4.04285);
\draw [c] (2.95596,4.01847) -- (2.97078,4.01847);
\draw [c] (2.97078,4.01847) -- (2.9856,4.01847);
\definecolor{c}{rgb}{0,0,0};
\colorlet{c}{kugray};
\draw [c] (3.00041,3.95995) -- (3.00041,3.9878);
\draw [c] (3.00041,3.9878) -- (3.00041,4.01334);
\draw [c] (2.9856,3.9878) -- (3.00041,3.9878);
\draw [c] (3.00041,3.9878) -- (3.01523,3.9878);
\definecolor{c}{rgb}{0,0,0};
\colorlet{c}{kugray};
\draw [c] (3.03005,3.92381) -- (3.03005,3.95391);
\draw [c] (3.03005,3.95391) -- (3.03005,3.98133);
\draw [c] (3.01523,3.95391) -- (3.03005,3.95391);
\draw [c] (3.03005,3.95391) -- (3.04487,3.95391);
\definecolor{c}{rgb}{0,0,0};
\colorlet{c}{kugray};
\draw [c] (3.05969,3.92723) -- (3.05969,3.9572);
\draw [c] (3.05969,3.9572) -- (3.05969,3.9845);
\draw [c] (3.04487,3.9572) -- (3.05969,3.9572);
\draw [c] (3.05969,3.9572) -- (3.0745,3.9572);
\definecolor{c}{rgb}{0,0,0};
\colorlet{c}{kugray};
\draw [c] (3.08932,3.89686) -- (3.08932,3.92737);
\draw [c] (3.08932,3.92737) -- (3.08932,3.95512);
\draw [c] (3.0745,3.92737) -- (3.08932,3.92737);
\draw [c] (3.08932,3.92737) -- (3.10414,3.92737);
\definecolor{c}{rgb}{0,0,0};
\colorlet{c}{kugray};
\draw [c] (3.11896,3.87075) -- (3.11896,3.9025);
\draw [c] (3.11896,3.9025) -- (3.11896,3.93128);
\draw [c] (3.10414,3.9025) -- (3.11896,3.9025);
\draw [c] (3.11896,3.9025) -- (3.13377,3.9025);
\definecolor{c}{rgb}{0,0,0};
\colorlet{c}{kugray};
\draw [c] (3.14859,3.84537) -- (3.14859,3.88076);
\draw [c] (3.14859,3.88076) -- (3.14859,3.9125);
\draw [c] (3.13377,3.88076) -- (3.14859,3.88076);
\draw [c] (3.14859,3.88076) -- (3.16341,3.88076);
\definecolor{c}{rgb}{0,0,0};
\colorlet{c}{kugray};
\draw [c] (3.17823,3.8829) -- (3.17823,3.91571);
\draw [c] (3.17823,3.91571) -- (3.17823,3.94536);
\draw [c] (3.16341,3.91571) -- (3.17823,3.91571);
\draw [c] (3.17823,3.91571) -- (3.19305,3.91571);
\definecolor{c}{rgb}{0,0,0};
\colorlet{c}{kugray};
\draw [c] (3.20786,3.84639) -- (3.20786,3.8804);
\draw [c] (3.20786,3.8804) -- (3.20786,3.91102);
\draw [c] (3.19305,3.8804) -- (3.20786,3.8804);
\draw [c] (3.20786,3.8804) -- (3.22268,3.8804);
\definecolor{c}{rgb}{0,0,0};
\colorlet{c}{kugray};
\draw [c] (3.2375,3.74035) -- (3.2375,3.78287);
\draw [c] (3.2375,3.78287) -- (3.2375,3.82022);
\draw [c] (3.22268,3.78287) -- (3.2375,3.78287);
\draw [c] (3.2375,3.78287) -- (3.25232,3.78287);
\definecolor{c}{rgb}{0,0,0};
\colorlet{c}{kugray};
\draw [c] (3.26714,3.68918) -- (3.26714,3.73432);
\draw [c] (3.26714,3.73432) -- (3.26714,3.77368);
\draw [c] (3.25232,3.73432) -- (3.26714,3.73432);
\draw [c] (3.26714,3.73432) -- (3.28195,3.73432);
\definecolor{c}{rgb}{0,0,0};
\colorlet{c}{kugray};
\draw [c] (3.29677,3.79443) -- (3.29677,3.83114);
\draw [c] (3.29677,3.83114) -- (3.29677,3.86393);
\draw [c] (3.28195,3.83114) -- (3.29677,3.83114);
\draw [c] (3.29677,3.83114) -- (3.31159,3.83114);
\definecolor{c}{rgb}{0,0,0};
\colorlet{c}{kugray};
\draw [c] (3.32641,3.68348) -- (3.32641,3.72745);
\draw [c] (3.32641,3.72745) -- (3.32641,3.76591);
\draw [c] (3.31159,3.72745) -- (3.32641,3.72745);
\draw [c] (3.32641,3.72745) -- (3.34123,3.72745);
\definecolor{c}{rgb}{0,0,0};
\colorlet{c}{kugray};
\draw [c] (3.35604,3.6893) -- (3.35604,3.73394);
\draw [c] (3.35604,3.73394) -- (3.35604,3.77293);
\draw [c] (3.34123,3.73394) -- (3.35604,3.73394);
\draw [c] (3.35604,3.73394) -- (3.37086,3.73394);
\definecolor{c}{rgb}{0,0,0};
\colorlet{c}{kugray};
\draw [c] (3.38568,3.69206) -- (3.38568,3.73661);
\draw [c] (3.38568,3.73661) -- (3.38568,3.77553);
\draw [c] (3.37086,3.73661) -- (3.38568,3.73661);
\draw [c] (3.38568,3.73661) -- (3.4005,3.73661);
\definecolor{c}{rgb}{0,0,0};
\colorlet{c}{kugray};
\draw [c] (3.41531,3.63598) -- (3.41531,3.68186);
\draw [c] (3.41531,3.68186) -- (3.41531,3.72179);
\draw [c] (3.4005,3.68186) -- (3.41531,3.68186);
\draw [c] (3.41531,3.68186) -- (3.43013,3.68186);
\definecolor{c}{rgb}{0,0,0};
\colorlet{c}{kugray};
\draw [c] (3.44495,3.6432) -- (3.44495,3.68882);
\draw [c] (3.44495,3.68882) -- (3.44495,3.72854);
\draw [c] (3.43013,3.68882) -- (3.44495,3.68882);
\draw [c] (3.44495,3.68882) -- (3.45977,3.68882);
\definecolor{c}{rgb}{0,0,0};
\colorlet{c}{kugray};
\draw [c] (3.47459,3.70732) -- (3.47459,3.7514);
\draw [c] (3.47459,3.7514) -- (3.47459,3.78995);
\draw [c] (3.45977,3.7514) -- (3.47459,3.7514);
\draw [c] (3.47459,3.7514) -- (3.4894,3.7514);
\definecolor{c}{rgb}{0,0,0};
\colorlet{c}{kugray};
\draw [c] (3.50422,3.66603) -- (3.50422,3.71303);
\draw [c] (3.50422,3.71303) -- (3.50422,3.7538);
\draw [c] (3.4894,3.71303) -- (3.50422,3.71303);
\draw [c] (3.50422,3.71303) -- (3.51904,3.71303);
\definecolor{c}{rgb}{0,0,0};
\colorlet{c}{kugray};
\draw [c] (3.53386,3.56775) -- (3.53386,3.61999);
\draw [c] (3.53386,3.61999) -- (3.53386,3.66463);
\draw [c] (3.51904,3.61999) -- (3.53386,3.61999);
\draw [c] (3.53386,3.61999) -- (3.54868,3.61999);
\definecolor{c}{rgb}{0,0,0};
\colorlet{c}{kugray};
\draw [c] (3.56349,3.5261) -- (3.56349,3.58634);
\draw [c] (3.56349,3.58634) -- (3.56349,3.6367);
\draw [c] (3.54868,3.58634) -- (3.56349,3.58634);
\draw [c] (3.56349,3.58634) -- (3.57831,3.58634);
\definecolor{c}{rgb}{0,0,0};
\colorlet{c}{kugray};
\draw [c] (3.59313,3.53612) -- (3.59313,3.59351);
\draw [c] (3.59313,3.59351) -- (3.59313,3.64187);
\draw [c] (3.57831,3.59351) -- (3.59313,3.59351);
\draw [c] (3.59313,3.59351) -- (3.60795,3.59351);
\definecolor{c}{rgb}{0,0,0};
\colorlet{c}{kugray};
\draw [c] (3.62276,3.60477) -- (3.62276,3.65738);
\draw [c] (3.62276,3.65738) -- (3.62276,3.70229);
\draw [c] (3.60795,3.65738) -- (3.62276,3.65738);
\draw [c] (3.62276,3.65738) -- (3.63758,3.65738);
\definecolor{c}{rgb}{0,0,0};
\colorlet{c}{kugray};
\draw [c] (3.6524,3.56534) -- (3.6524,3.6218);
\draw [c] (3.6524,3.6218) -- (3.6524,3.66949);
\draw [c] (3.63758,3.6218) -- (3.6524,3.6218);
\draw [c] (3.6524,3.6218) -- (3.66722,3.6218);
\definecolor{c}{rgb}{0,0,0};
\colorlet{c}{kugray};
\draw [c] (3.68204,3.48401) -- (3.68204,3.54474);
\draw [c] (3.68204,3.54474) -- (3.68204,3.59544);
\draw [c] (3.66722,3.54474) -- (3.68204,3.54474);
\draw [c] (3.68204,3.54474) -- (3.69685,3.54474);
\definecolor{c}{rgb}{0,0,0};
\colorlet{c}{kugray};
\draw [c] (3.71167,3.47019) -- (3.71167,3.52933);
\draw [c] (3.71167,3.52933) -- (3.71167,3.57892);
\draw [c] (3.69685,3.52933) -- (3.71167,3.52933);
\draw [c] (3.71167,3.52933) -- (3.72649,3.52933);
\definecolor{c}{rgb}{0,0,0};
\colorlet{c}{kugray};
\draw [c] (3.74131,3.47175) -- (3.74131,3.53495);
\draw [c] (3.74131,3.53495) -- (3.74131,3.58736);
\draw [c] (3.72649,3.53495) -- (3.74131,3.53495);
\draw [c] (3.74131,3.53495) -- (3.75613,3.53495);
\definecolor{c}{rgb}{0,0,0};
\colorlet{c}{kugray};
\draw [c] (3.77094,3.33813) -- (3.77094,3.41569);
\draw [c] (3.77094,3.41569) -- (3.77094,3.47758);
\draw [c] (3.75613,3.41569) -- (3.77094,3.41569);
\draw [c] (3.77094,3.41569) -- (3.78576,3.41569);
\definecolor{c}{rgb}{0,0,0};
\colorlet{c}{kugray};
\draw [c] (3.80058,3.43206) -- (3.80058,3.5008);
\draw [c] (3.80058,3.5008) -- (3.80058,3.55696);
\draw [c] (3.78576,3.5008) -- (3.80058,3.5008);
\draw [c] (3.80058,3.5008) -- (3.8154,3.5008);
\definecolor{c}{rgb}{0,0,0};
\colorlet{c}{kugray};
\draw [c] (3.83022,3.33768) -- (3.83022,3.41833);
\draw [c] (3.83022,3.41833) -- (3.83022,3.48217);
\draw [c] (3.8154,3.41833) -- (3.83022,3.41833);
\draw [c] (3.83022,3.41833) -- (3.84503,3.41833);
\definecolor{c}{rgb}{0,0,0};
\colorlet{c}{kugray};
\draw [c] (3.85985,3.50616) -- (3.85985,3.56185);
\draw [c] (3.85985,3.56185) -- (3.85985,3.60899);
\draw [c] (3.84503,3.56185) -- (3.85985,3.56185);
\draw [c] (3.85985,3.56185) -- (3.87467,3.56185);
\definecolor{c}{rgb}{0,0,0};
\colorlet{c}{kugray};
\draw [c] (3.88949,3.33314) -- (3.88949,3.40343);
\draw [c] (3.88949,3.40343) -- (3.88949,3.46061);
\draw [c] (3.87467,3.40343) -- (3.88949,3.40343);
\draw [c] (3.88949,3.40343) -- (3.9043,3.40343);
\definecolor{c}{rgb}{0,0,0};
\colorlet{c}{kugray};
\draw [c] (3.91912,3.32482) -- (3.91912,3.38647);
\draw [c] (3.91912,3.38647) -- (3.91912,3.43781);
\draw [c] (3.9043,3.38647) -- (3.91912,3.38647);
\draw [c] (3.91912,3.38647) -- (3.93394,3.38647);
\definecolor{c}{rgb}{0,0,0};
\colorlet{c}{kugray};
\draw [c] (3.94876,3.35526) -- (3.94876,3.41219);
\draw [c] (3.94876,3.41219) -- (3.94876,3.46022);
\draw [c] (3.93394,3.41219) -- (3.94876,3.41219);
\draw [c] (3.94876,3.41219) -- (3.96358,3.41219);
\definecolor{c}{rgb}{0,0,0};
\colorlet{c}{kugray};
\draw [c] (3.97839,3.3937) -- (3.97839,3.43987);
\draw [c] (3.97839,3.43987) -- (3.97839,3.48);
\draw [c] (3.96358,3.43987) -- (3.97839,3.43987);
\draw [c] (3.97839,3.43987) -- (3.99321,3.43987);
\definecolor{c}{rgb}{0,0,0};
\colorlet{c}{kugray};
\draw [c] (4.00803,3.39966) -- (4.00803,3.4339);
\draw [c] (4.00803,3.4339) -- (4.00803,3.4647);
\draw [c] (3.99321,3.4339) -- (4.00803,3.4339);
\draw [c] (4.00803,3.4339) -- (4.02285,3.4339);
\definecolor{c}{rgb}{0,0,0};
\colorlet{c}{kugray};
\draw [c] (4.03767,3.3675) -- (4.03767,3.38466);
\draw [c] (4.03767,3.38466) -- (4.03767,3.4009);
\draw [c] (4.02285,3.38466) -- (4.03767,3.38466);
\draw [c] (4.03767,3.38466) -- (4.05248,3.38466);
\definecolor{c}{rgb}{0,0,0};
\colorlet{c}{kugray};
\draw [c] (4.0673,3.36697) -- (4.0673,3.3818);
\draw [c] (4.0673,3.3818) -- (4.0673,3.39594);
\draw [c] (4.05248,3.3818) -- (4.0673,3.3818);
\draw [c] (4.0673,3.3818) -- (4.08212,3.3818);
\definecolor{c}{rgb}{0,0,0};
\colorlet{c}{kugray};
\draw [c] (4.09694,3.34139) -- (4.09694,3.35685);
\draw [c] (4.09694,3.35685) -- (4.09694,3.37158);
\draw [c] (4.08212,3.35685) -- (4.09694,3.35685);
\draw [c] (4.09694,3.35685) -- (4.11175,3.35685);
\definecolor{c}{rgb}{0,0,0};
\colorlet{c}{kugray};
\draw [c] (4.12657,3.34259) -- (4.12657,3.35775);
\draw [c] (4.12657,3.35775) -- (4.12657,3.37219);
\draw [c] (4.11175,3.35775) -- (4.12657,3.35775);
\draw [c] (4.12657,3.35775) -- (4.14139,3.35775);
\definecolor{c}{rgb}{0,0,0};
\colorlet{c}{kugray};
\draw [c] (4.15621,3.33872) -- (4.15621,3.35421);
\draw [c] (4.15621,3.35421) -- (4.15621,3.36896);
\draw [c] (4.14139,3.35421) -- (4.15621,3.35421);
\draw [c] (4.15621,3.35421) -- (4.17103,3.35421);
\definecolor{c}{rgb}{0,0,0};
\colorlet{c}{kugray};
\draw [c] (4.18584,3.34948) -- (4.18584,3.36456);
\draw [c] (4.18584,3.36456) -- (4.18584,3.37895);
\draw [c] (4.17103,3.36456) -- (4.18584,3.36456);
\draw [c] (4.18584,3.36456) -- (4.20066,3.36456);
\definecolor{c}{rgb}{0,0,0};
\colorlet{c}{kugray};
\draw [c] (4.21548,3.30398) -- (4.21548,3.32047);
\draw [c] (4.21548,3.32047) -- (4.21548,3.33611);
\draw [c] (4.20066,3.32047) -- (4.21548,3.32047);
\draw [c] (4.21548,3.32047) -- (4.2303,3.32047);
\definecolor{c}{rgb}{0,0,0};
\colorlet{c}{kugray};
\draw [c] (4.24512,3.29408) -- (4.24512,3.31043);
\draw [c] (4.24512,3.31043) -- (4.24512,3.32595);
\draw [c] (4.2303,3.31043) -- (4.24512,3.31043);
\draw [c] (4.24512,3.31043) -- (4.25993,3.31043);
\definecolor{c}{rgb}{0,0,0};
\colorlet{c}{kugray};
\draw [c] (4.27475,3.3132) -- (4.27475,3.32917);
\draw [c] (4.27475,3.32917) -- (4.27475,3.34436);
\draw [c] (4.25993,3.32917) -- (4.27475,3.32917);
\draw [c] (4.27475,3.32917) -- (4.28957,3.32917);
\definecolor{c}{rgb}{0,0,0};
\colorlet{c}{kugray};
\draw [c] (4.30439,3.28435) -- (4.30439,3.30161);
\draw [c] (4.30439,3.30161) -- (4.30439,3.31795);
\draw [c] (4.28957,3.30161) -- (4.30439,3.30161);
\draw [c] (4.30439,3.30161) -- (4.31921,3.30161);
\definecolor{c}{rgb}{0,0,0};
\colorlet{c}{kugray};
\draw [c] (4.33402,3.2635) -- (4.33402,3.28087);
\draw [c] (4.33402,3.28087) -- (4.33402,3.29731);
\draw [c] (4.31921,3.28087) -- (4.33402,3.28087);
\draw [c] (4.33402,3.28087) -- (4.34884,3.28087);
\definecolor{c}{rgb}{0,0,0};
\colorlet{c}{kugray};
\draw [c] (4.36366,3.26382) -- (4.36366,3.28126);
\draw [c] (4.36366,3.28126) -- (4.36366,3.29776);
\draw [c] (4.34884,3.28126) -- (4.36366,3.28126);
\draw [c] (4.36366,3.28126) -- (4.37848,3.28126);
\definecolor{c}{rgb}{0,0,0};
\colorlet{c}{kugray};
\draw [c] (4.39329,3.26487) -- (4.39329,3.28278);
\draw [c] (4.39329,3.28278) -- (4.39329,3.29971);
\draw [c] (4.37848,3.28278) -- (4.39329,3.28278);
\draw [c] (4.39329,3.28278) -- (4.40811,3.28278);
\definecolor{c}{rgb}{0,0,0};
\colorlet{c}{kugray};
\draw [c] (4.42293,3.21601) -- (4.42293,3.23471);
\draw [c] (4.42293,3.23471) -- (4.42293,3.25234);
\draw [c] (4.40811,3.23471) -- (4.42293,3.23471);
\draw [c] (4.42293,3.23471) -- (4.43775,3.23471);
\definecolor{c}{rgb}{0,0,0};
\colorlet{c}{kugray};
\draw [c] (4.45257,3.20125) -- (4.45257,3.2207);
\draw [c] (4.45257,3.2207) -- (4.45257,3.239);
\draw [c] (4.43775,3.2207) -- (4.45257,3.2207);
\draw [c] (4.45257,3.2207) -- (4.46738,3.2207);
\definecolor{c}{rgb}{0,0,0};
\colorlet{c}{kugray};
\draw [c] (4.4822,3.21568) -- (4.4822,3.23454);
\draw [c] (4.4822,3.23454) -- (4.4822,3.25231);
\draw [c] (4.46738,3.23454) -- (4.4822,3.23454);
\draw [c] (4.4822,3.23454) -- (4.49702,3.23454);
\definecolor{c}{rgb}{0,0,0};
\colorlet{c}{kugray};
\draw [c] (4.51184,3.21203) -- (4.51184,3.23093);
\draw [c] (4.51184,3.23093) -- (4.51184,3.24874);
\draw [c] (4.49702,3.23093) -- (4.51184,3.23093);
\draw [c] (4.51184,3.23093) -- (4.52666,3.23093);
\definecolor{c}{rgb}{0,0,0};
\colorlet{c}{kugray};
\draw [c] (4.54147,3.1933) -- (4.54147,3.21207);
\draw [c] (4.54147,3.21207) -- (4.54147,3.22977);
\draw [c] (4.52666,3.21207) -- (4.54147,3.21207);
\draw [c] (4.54147,3.21207) -- (4.55629,3.21207);
\definecolor{c}{rgb}{0,0,0};
\colorlet{c}{kugray};
\draw [c] (4.57111,3.16612) -- (4.57111,3.18637);
\draw [c] (4.57111,3.18637) -- (4.57111,3.20537);
\draw [c] (4.55629,3.18637) -- (4.57111,3.18637);
\draw [c] (4.57111,3.18637) -- (4.58593,3.18637);
\definecolor{c}{rgb}{0,0,0};
\colorlet{c}{kugray};
\draw [c] (4.60075,3.144) -- (4.60075,3.1653);
\draw [c] (4.60075,3.1653) -- (4.60075,3.18522);
\draw [c] (4.58593,3.1653) -- (4.60075,3.1653);
\draw [c] (4.60075,3.1653) -- (4.61556,3.1653);
\definecolor{c}{rgb}{0,0,0};
\colorlet{c}{kugray};
\draw [c] (4.63038,3.13842) -- (4.63038,3.15957);
\draw [c] (4.63038,3.15957) -- (4.63038,3.17935);
\draw [c] (4.61556,3.15957) -- (4.63038,3.15957);
\draw [c] (4.63038,3.15957) -- (4.6452,3.15957);
\definecolor{c}{rgb}{0,0,0};
\colorlet{c}{kugray};
\draw [c] (4.66002,3.10825) -- (4.66002,3.12968);
\draw [c] (4.66002,3.12968) -- (4.66002,3.14972);
\draw [c] (4.6452,3.12968) -- (4.66002,3.12968);
\draw [c] (4.66002,3.12968) -- (4.67483,3.12968);
\definecolor{c}{rgb}{0,0,0};
\colorlet{c}{kugray};
\draw [c] (4.68965,3.12507) -- (4.68965,3.14699);
\draw [c] (4.68965,3.14699) -- (4.68965,3.16746);
\draw [c] (4.67483,3.14699) -- (4.68965,3.14699);
\draw [c] (4.68965,3.14699) -- (4.70447,3.14699);
\definecolor{c}{rgb}{0,0,0};
\colorlet{c}{kugray};
\draw [c] (4.71929,3.07502) -- (4.71929,3.09766);
\draw [c] (4.71929,3.09766) -- (4.71929,3.11875);
\draw [c] (4.70447,3.09766) -- (4.71929,3.09766);
\draw [c] (4.71929,3.09766) -- (4.73411,3.09766);
\definecolor{c}{rgb}{0,0,0};
\colorlet{c}{kugray};
\draw [c] (4.74892,3.12089) -- (4.74892,3.1424);
\draw [c] (4.74892,3.1424) -- (4.74892,3.1625);
\draw [c] (4.73411,3.1424) -- (4.74892,3.1424);
\draw [c] (4.74892,3.1424) -- (4.76374,3.1424);
\definecolor{c}{rgb}{0,0,0};
\colorlet{c}{kugray};
\draw [c] (4.77856,3.07673) -- (4.77856,3.09957);
\draw [c] (4.77856,3.09957) -- (4.77856,3.12084);
\draw [c] (4.76374,3.09957) -- (4.77856,3.09957);
\draw [c] (4.77856,3.09957) -- (4.79338,3.09957);
\definecolor{c}{rgb}{0,0,0};
\colorlet{c}{kugray};
\draw [c] (4.8082,3.11216) -- (4.8082,3.13437);
\draw [c] (4.8082,3.13437) -- (4.8082,3.15507);
\draw [c] (4.79338,3.13437) -- (4.8082,3.13437);
\draw [c] (4.8082,3.13437) -- (4.82301,3.13437);
\definecolor{c}{rgb}{0,0,0};
\colorlet{c}{kugray};
\draw [c] (4.83783,3.12021) -- (4.83783,3.14174);
\draw [c] (4.83783,3.14174) -- (4.83783,3.16186);
\draw [c] (4.82301,3.14174) -- (4.83783,3.14174);
\draw [c] (4.83783,3.14174) -- (4.85265,3.14174);
\definecolor{c}{rgb}{0,0,0};
\colorlet{c}{kugray};
\draw [c] (4.86747,3.04958) -- (4.86747,3.07475);
\draw [c] (4.86747,3.07475) -- (4.86747,3.09801);
\draw [c] (4.85265,3.07475) -- (4.86747,3.07475);
\draw [c] (4.86747,3.07475) -- (4.88228,3.07475);
\definecolor{c}{rgb}{0,0,0};
\colorlet{c}{kugray};
\draw [c] (4.8971,3.02902) -- (4.8971,3.05448);
\draw [c] (4.8971,3.05448) -- (4.8971,3.078);
\draw [c] (4.88228,3.05448) -- (4.8971,3.05448);
\draw [c] (4.8971,3.05448) -- (4.91192,3.05448);
\definecolor{c}{rgb}{0,0,0};
\colorlet{c}{kugray};
\draw [c] (4.92674,3.05931) -- (4.92674,3.08346);
\draw [c] (4.92674,3.08346) -- (4.92674,3.10584);
\draw [c] (4.91192,3.08346) -- (4.92674,3.08346);
\draw [c] (4.92674,3.08346) -- (4.94156,3.08346);
\definecolor{c}{rgb}{0,0,0};
\colorlet{c}{kugray};
\draw [c] (4.95637,3.04459) -- (4.95637,3.06898);
\draw [c] (4.95637,3.06898) -- (4.95637,3.09157);
\draw [c] (4.94156,3.06898) -- (4.95637,3.06898);
\draw [c] (4.95637,3.06898) -- (4.97119,3.06898);
\definecolor{c}{rgb}{0,0,0};
\colorlet{c}{kugray};
\draw [c] (4.98601,3.02119) -- (4.98601,3.04833);
\draw [c] (4.98601,3.04833) -- (4.98601,3.07328);
\draw [c] (4.97119,3.04833) -- (4.98601,3.04833);
\draw [c] (4.98601,3.04833) -- (5.00083,3.04833);
\definecolor{c}{rgb}{0,0,0};
\colorlet{c}{kugray};
\draw [c] (5.01565,3.03957) -- (5.01565,3.06478);
\draw [c] (5.01565,3.06478) -- (5.01565,3.08808);
\draw [c] (5.00083,3.06478) -- (5.01565,3.06478);
\draw [c] (5.01565,3.06478) -- (5.03046,3.06478);
\definecolor{c}{rgb}{0,0,0};
\colorlet{c}{kugray};
\draw [c] (5.04528,2.9954) -- (5.04528,3.02095);
\draw [c] (5.04528,3.02095) -- (5.04528,3.04453);
\draw [c] (5.03046,3.02095) -- (5.04528,3.02095);
\draw [c] (5.04528,3.02095) -- (5.0601,3.02095);
\definecolor{c}{rgb}{0,0,0};
\colorlet{c}{kugray};
\draw [c] (5.07492,3.00475) -- (5.07492,3.03087);
\draw [c] (5.07492,3.03087) -- (5.07492,3.05495);
\draw [c] (5.0601,3.03087) -- (5.07492,3.03087);
\draw [c] (5.07492,3.03087) -- (5.08974,3.03087);
\definecolor{c}{rgb}{0,0,0};
\colorlet{c}{kugray};
\draw [c] (5.10455,3.04579) -- (5.10455,3.07035);
\draw [c] (5.10455,3.07035) -- (5.10455,3.0931);
\draw [c] (5.08974,3.07035) -- (5.10455,3.07035);
\draw [c] (5.10455,3.07035) -- (5.11937,3.07035);
\definecolor{c}{rgb}{0,0,0};
\colorlet{c}{kugray};
\draw [c] (5.13419,2.99892) -- (5.13419,3.02565);
\draw [c] (5.13419,3.02565) -- (5.13419,3.05025);
\draw [c] (5.11937,3.02565) -- (5.13419,3.02565);
\draw [c] (5.13419,3.02565) -- (5.14901,3.02565);
\definecolor{c}{rgb}{0,0,0};
\colorlet{c}{kugray};
\draw [c] (5.16382,2.99685) -- (5.16382,3.02306);
\draw [c] (5.16382,3.02306) -- (5.16382,3.0472);
\draw [c] (5.14901,3.02306) -- (5.16382,3.02306);
\draw [c] (5.16382,3.02306) -- (5.17864,3.02306);
\definecolor{c}{rgb}{0,0,0};
\colorlet{c}{kugray};
\draw [c] (5.19346,2.98544) -- (5.19346,3.0123);
\draw [c] (5.19346,3.0123) -- (5.19346,3.03701);
\draw [c] (5.17864,3.0123) -- (5.19346,3.0123);
\draw [c] (5.19346,3.0123) -- (5.20828,3.0123);
\definecolor{c}{rgb}{0,0,0};
\colorlet{c}{kugray};
\draw [c] (5.2231,3.00482) -- (5.2231,3.0317);
\draw [c] (5.2231,3.0317) -- (5.2231,3.05642);
\draw [c] (5.20828,3.0317) -- (5.2231,3.0317);
\draw [c] (5.2231,3.0317) -- (5.23791,3.0317);
\definecolor{c}{rgb}{0,0,0};
\colorlet{c}{kugray};
\draw [c] (5.25273,2.9865) -- (5.25273,3.01355);
\draw [c] (5.25273,3.01355) -- (5.25273,3.03841);
\draw [c] (5.23791,3.01355) -- (5.25273,3.01355);
\draw [c] (5.25273,3.01355) -- (5.26755,3.01355);
\definecolor{c}{rgb}{0,0,0};
\colorlet{c}{kugray};
\draw [c] (5.28237,2.99411) -- (5.28237,3.02124);
\draw [c] (5.28237,3.02124) -- (5.28237,3.04618);
\draw [c] (5.26755,3.02124) -- (5.28237,3.02124);
\draw [c] (5.28237,3.02124) -- (5.29719,3.02124);
\definecolor{c}{rgb}{0,0,0};
\colorlet{c}{kugray};
\draw [c] (5.312,2.94471) -- (5.312,2.97506);
\draw [c] (5.312,2.97506) -- (5.312,3.00268);
\draw [c] (5.29719,2.97506) -- (5.312,2.97506);
\draw [c] (5.312,2.97506) -- (5.32682,2.97506);
\definecolor{c}{rgb}{0,0,0};
\colorlet{c}{kugray};
\draw [c] (5.34164,2.92698) -- (5.34164,2.95685);
\draw [c] (5.34164,2.95685) -- (5.34164,2.98407);
\draw [c] (5.32682,2.95685) -- (5.34164,2.95685);
\draw [c] (5.34164,2.95685) -- (5.35646,2.95685);
\definecolor{c}{rgb}{0,0,0};
\colorlet{c}{kugray};
\draw [c] (5.37127,2.88114) -- (5.37127,2.91303);
\draw [c] (5.37127,2.91303) -- (5.37127,2.94192);
\draw [c] (5.35646,2.91303) -- (5.37127,2.91303);
\draw [c] (5.37127,2.91303) -- (5.38609,2.91303);
\definecolor{c}{rgb}{0,0,0};
\colorlet{c}{kugray};
\draw [c] (5.40091,2.91147) -- (5.40091,2.942);
\draw [c] (5.40091,2.942) -- (5.40091,2.96977);
\draw [c] (5.38609,2.942) -- (5.40091,2.942);
\draw [c] (5.40091,2.942) -- (5.41573,2.942);
\definecolor{c}{rgb}{0,0,0};
\colorlet{c}{kugray};
\draw [c] (5.43055,2.89934) -- (5.43055,2.93053);
\draw [c] (5.43055,2.93053) -- (5.43055,2.95885);
\draw [c] (5.41573,2.93053) -- (5.43055,2.93053);
\draw [c] (5.43055,2.93053) -- (5.44536,2.93053);
\definecolor{c}{rgb}{0,0,0};
\colorlet{c}{kugray};
\draw [c] (5.46018,2.83663) -- (5.46018,2.86962);
\draw [c] (5.46018,2.86962) -- (5.46018,2.89942);
\draw [c] (5.44536,2.86962) -- (5.46018,2.86962);
\draw [c] (5.46018,2.86962) -- (5.475,2.86962);
\definecolor{c}{rgb}{0,0,0};
\colorlet{c}{kugray};
\draw [c] (5.48982,2.89582) -- (5.48982,2.92714);
\draw [c] (5.48982,2.92714) -- (5.48982,2.95557);
\draw [c] (5.475,2.92714) -- (5.48982,2.92714);
\draw [c] (5.48982,2.92714) -- (5.50464,2.92714);
\definecolor{c}{rgb}{0,0,0};
\colorlet{c}{kugray};
\draw [c] (5.51945,2.86857) -- (5.51945,2.90037);
\draw [c] (5.51945,2.90037) -- (5.51945,2.9292);
\draw [c] (5.50464,2.90037) -- (5.51945,2.90037);
\draw [c] (5.51945,2.90037) -- (5.53427,2.90037);
\definecolor{c}{rgb}{0,0,0};
\colorlet{c}{kugray};
\draw [c] (5.54909,2.91849) -- (5.54909,2.94886);
\draw [c] (5.54909,2.94886) -- (5.54909,2.97651);
\draw [c] (5.53427,2.94886) -- (5.54909,2.94886);
\draw [c] (5.54909,2.94886) -- (5.56391,2.94886);
\definecolor{c}{rgb}{0,0,0};
\colorlet{c}{kugray};
\draw [c] (5.57873,2.84325) -- (5.57873,2.87551);
\draw [c] (5.57873,2.87551) -- (5.57873,2.90471);
\draw [c] (5.56391,2.87551) -- (5.57873,2.87551);
\draw [c] (5.57873,2.87551) -- (5.59354,2.87551);
\definecolor{c}{rgb}{0,0,0};
\colorlet{c}{kugray};
\draw [c] (5.60836,2.86623) -- (5.60836,2.89815);
\draw [c] (5.60836,2.89815) -- (5.60836,2.92707);
\draw [c] (5.59354,2.89815) -- (5.60836,2.89815);
\draw [c] (5.60836,2.89815) -- (5.62318,2.89815);
\definecolor{c}{rgb}{0,0,0};
\colorlet{c}{kugray};
\draw [c] (5.638,2.91304) -- (5.638,2.94486);
\draw [c] (5.638,2.94486) -- (5.638,2.97369);
\draw [c] (5.62318,2.94486) -- (5.638,2.94486);
\draw [c] (5.638,2.94486) -- (5.65281,2.94486);
\definecolor{c}{rgb}{0,0,0};
\colorlet{c}{kugray};
\draw [c] (5.66763,2.90882) -- (5.66763,2.93949);
\draw [c] (5.66763,2.93949) -- (5.66763,2.96738);
\draw [c] (5.65281,2.93949) -- (5.66763,2.93949);
\draw [c] (5.66763,2.93949) -- (5.68245,2.93949);
\definecolor{c}{rgb}{0,0,0};
\colorlet{c}{kugray};
\draw [c] (5.69727,2.89105) -- (5.69727,2.92252);
\draw [c] (5.69727,2.92252) -- (5.69727,2.95107);
\draw [c] (5.68245,2.92252) -- (5.69727,2.92252);
\draw [c] (5.69727,2.92252) -- (5.71209,2.92252);
\definecolor{c}{rgb}{0,0,0};
\colorlet{c}{kugray};
\draw [c] (5.7269,2.77815) -- (5.7269,2.81636);
\draw [c] (5.7269,2.81636) -- (5.7269,2.85035);
\draw [c] (5.71209,2.81636) -- (5.7269,2.81636);
\draw [c] (5.7269,2.81636) -- (5.74172,2.81636);
\definecolor{c}{rgb}{0,0,0};
\colorlet{c}{kugray};
\draw [c] (5.75654,2.94321) -- (5.75654,2.97327);
\draw [c] (5.75654,2.97327) -- (5.75654,3.00065);
\draw [c] (5.74172,2.97327) -- (5.75654,2.97327);
\draw [c] (5.75654,2.97327) -- (5.77136,2.97327);
\definecolor{c}{rgb}{0,0,0};
\colorlet{c}{kugray};
\draw [c] (5.78618,2.8141) -- (5.78618,2.84956);
\draw [c] (5.78618,2.84956) -- (5.78618,2.88135);
\draw [c] (5.77136,2.84956) -- (5.78618,2.84956);
\draw [c] (5.78618,2.84956) -- (5.80099,2.84956);
\definecolor{c}{rgb}{0,0,0};
\colorlet{c}{kugray};
\draw [c] (5.81581,2.84857) -- (5.81581,2.88346);
\draw [c] (5.81581,2.88346) -- (5.81581,2.91479);
\draw [c] (5.80099,2.88346) -- (5.81581,2.88346);
\draw [c] (5.81581,2.88346) -- (5.83063,2.88346);
\definecolor{c}{rgb}{0,0,0};
\colorlet{c}{kugray};
\draw [c] (5.84545,2.83643) -- (5.84545,2.87155);
\draw [c] (5.84545,2.87155) -- (5.84545,2.90308);
\draw [c] (5.83063,2.87155) -- (5.84545,2.87155);
\draw [c] (5.84545,2.87155) -- (5.86026,2.87155);
\definecolor{c}{rgb}{0,0,0};
\colorlet{c}{kugray};
\draw [c] (5.87508,2.75529) -- (5.87508,2.79332);
\draw [c] (5.87508,2.79332) -- (5.87508,2.82715);
\draw [c] (5.86026,2.79332) -- (5.87508,2.79332);
\draw [c] (5.87508,2.79332) -- (5.8899,2.79332);
\definecolor{c}{rgb}{0,0,0};
\colorlet{c}{kugray};
\draw [c] (5.90472,2.77226) -- (5.90472,2.81175);
\draw [c] (5.90472,2.81175) -- (5.90472,2.84675);
\draw [c] (5.8899,2.81175) -- (5.90472,2.81175);
\draw [c] (5.90472,2.81175) -- (5.91954,2.81175);
\definecolor{c}{rgb}{0,0,0};
\colorlet{c}{kugray};
\draw [c] (5.93435,2.82058) -- (5.93435,2.85588);
\draw [c] (5.93435,2.85588) -- (5.93435,2.88754);
\draw [c] (5.91954,2.85588) -- (5.93435,2.85588);
\draw [c] (5.93435,2.85588) -- (5.94917,2.85588);
\definecolor{c}{rgb}{0,0,0};
\colorlet{c}{kugray};
\draw [c] (5.96399,2.83477) -- (5.96399,2.87069);
\draw [c] (5.96399,2.87069) -- (5.96399,2.90285);
\draw [c] (5.94917,2.87069) -- (5.96399,2.87069);
\draw [c] (5.96399,2.87069) -- (5.97881,2.87069);
\definecolor{c}{rgb}{0,0,0};
\colorlet{c}{kugray};
\draw [c] (5.99363,2.78914) -- (5.99363,2.8266);
\draw [c] (5.99363,2.8266) -- (5.99363,2.85999);
\draw [c] (5.97881,2.8266) -- (5.99363,2.8266);
\draw [c] (5.99363,2.8266) -- (6.00844,2.8266);
\definecolor{c}{rgb}{0,0,0};
\colorlet{c}{kugray};
\draw [c] (6.02326,2.82633) -- (6.02326,2.86318);
\draw [c] (6.02326,2.86318) -- (6.02326,2.89609);
\draw [c] (6.00844,2.86318) -- (6.02326,2.86318);
\draw [c] (6.02326,2.86318) -- (6.03808,2.86318);
\definecolor{c}{rgb}{0,0,0};
\colorlet{c}{kugray};
\draw [c] (6.0529,2.72961) -- (6.0529,2.7694);
\draw [c] (6.0529,2.7694) -- (6.0529,2.80462);
\draw [c] (6.03808,2.7694) -- (6.0529,2.7694);
\draw [c] (6.0529,2.7694) -- (6.06772,2.7694);
\definecolor{c}{rgb}{0,0,0};
\colorlet{c}{kugray};
\draw [c] (6.08253,2.84391) -- (6.08253,2.87833);
\draw [c] (6.08253,2.87833) -- (6.08253,2.90928);
\draw [c] (6.06772,2.87833) -- (6.08253,2.87833);
\draw [c] (6.08253,2.87833) -- (6.09735,2.87833);
\definecolor{c}{rgb}{0,0,0};
\colorlet{c}{kugray};
\draw [c] (6.11217,2.75513) -- (6.11217,2.79539);
\draw [c] (6.11217,2.79539) -- (6.11217,2.83099);
\draw [c] (6.09735,2.79539) -- (6.11217,2.79539);
\draw [c] (6.11217,2.79539) -- (6.12699,2.79539);
\definecolor{c}{rgb}{0,0,0};
\colorlet{c}{kugray};
\draw [c] (6.1418,2.67424) -- (6.1418,2.71934);
\draw [c] (6.1418,2.71934) -- (6.1418,2.75867);
\draw [c] (6.12699,2.71934) -- (6.1418,2.71934);
\draw [c] (6.1418,2.71934) -- (6.15662,2.71934);
\definecolor{c}{rgb}{0,0,0};
\colorlet{c}{kugray};
\draw [c] (6.17144,2.76048) -- (6.17144,2.79878);
\draw [c] (6.17144,2.79878) -- (6.17144,2.83284);
\draw [c] (6.15662,2.79878) -- (6.17144,2.79878);
\draw [c] (6.17144,2.79878) -- (6.18626,2.79878);
\definecolor{c}{rgb}{0,0,0};
\colorlet{c}{kugray};
\draw [c] (6.20108,2.73144) -- (6.20108,2.77033);
\draw [c] (6.20108,2.77033) -- (6.20108,2.80485);
\draw [c] (6.18626,2.77033) -- (6.20108,2.77033);
\draw [c] (6.20108,2.77033) -- (6.21589,2.77033);
\definecolor{c}{rgb}{0,0,0};
\colorlet{c}{kugray};
\draw [c] (6.23071,2.71313) -- (6.23071,2.75551);
\draw [c] (6.23071,2.75551) -- (6.23071,2.79276);
\draw [c] (6.21589,2.75551) -- (6.23071,2.75551);
\draw [c] (6.23071,2.75551) -- (6.24553,2.75551);
\definecolor{c}{rgb}{0,0,0};
\colorlet{c}{kugray};
\draw [c] (6.26035,2.73793) -- (6.26035,2.77947);
\draw [c] (6.26035,2.77947) -- (6.26035,2.81607);
\draw [c] (6.24553,2.77947) -- (6.26035,2.77947);
\draw [c] (6.26035,2.77947) -- (6.27517,2.77947);
\definecolor{c}{rgb}{0,0,0};
\colorlet{c}{kugray};
\draw [c] (6.28998,2.74513) -- (6.28998,2.7844);
\draw [c] (6.28998,2.7844) -- (6.28998,2.81923);
\draw [c] (6.27517,2.7844) -- (6.28998,2.7844);
\draw [c] (6.28998,2.7844) -- (6.3048,2.7844);
\definecolor{c}{rgb}{0,0,0};
\colorlet{c}{kugray};
\draw [c] (6.31962,2.7653) -- (6.31962,2.80355);
\draw [c] (6.31962,2.80355) -- (6.31962,2.83757);
\draw [c] (6.3048,2.80355) -- (6.31962,2.80355);
\draw [c] (6.31962,2.80355) -- (6.33444,2.80355);
\definecolor{c}{rgb}{0,0,0};
\colorlet{c}{kugray};
\draw [c] (6.34926,2.73889) -- (6.34926,2.78116);
\draw [c] (6.34926,2.78116) -- (6.34926,2.81833);
\draw [c] (6.33444,2.78116) -- (6.34926,2.78116);
\draw [c] (6.34926,2.78116) -- (6.36407,2.78116);
\definecolor{c}{rgb}{0,0,0};
\colorlet{c}{kugray};
\draw [c] (6.37889,2.70662) -- (6.37889,2.74897);
\draw [c] (6.37889,2.74897) -- (6.37889,2.7862);
\draw [c] (6.36407,2.74897) -- (6.37889,2.74897);
\draw [c] (6.37889,2.74897) -- (6.39371,2.74897);
\definecolor{c}{rgb}{0,0,0};
\colorlet{c}{kugray};
\draw [c] (6.40853,2.70883) -- (6.40853,2.74986);
\draw [c] (6.40853,2.74986) -- (6.40853,2.78606);
\draw [c] (6.39371,2.74986) -- (6.40853,2.74986);
\draw [c] (6.40853,2.74986) -- (6.42334,2.74986);
\definecolor{c}{rgb}{0,0,0};
\colorlet{c}{kugray};
\draw [c] (6.43816,2.74777) -- (6.43816,2.79061);
\draw [c] (6.43816,2.79061) -- (6.43816,2.82822);
\draw [c] (6.42334,2.79061) -- (6.43816,2.79061);
\draw [c] (6.43816,2.79061) -- (6.45298,2.79061);
\definecolor{c}{rgb}{0,0,0};
\colorlet{c}{kugray};
\draw [c] (6.4678,2.73848) -- (6.4678,2.78112);
\draw [c] (6.4678,2.78112) -- (6.4678,2.81857);
\draw [c] (6.45298,2.78112) -- (6.4678,2.78112);
\draw [c] (6.4678,2.78112) -- (6.48262,2.78112);
\definecolor{c}{rgb}{0,0,0};
\colorlet{c}{kugray};
\draw [c] (6.49743,2.68882) -- (6.49743,2.73113);
\draw [c] (6.49743,2.73113) -- (6.49743,2.76832);
\draw [c] (6.48262,2.73113) -- (6.49743,2.73113);
\draw [c] (6.49743,2.73113) -- (6.51225,2.73113);
\definecolor{c}{rgb}{0,0,0};
\colorlet{c}{kugray};
\draw [c] (6.52707,2.69846) -- (6.52707,2.74161);
\draw [c] (6.52707,2.74161) -- (6.52707,2.77946);
\draw [c] (6.51225,2.74161) -- (6.52707,2.74161);
\draw [c] (6.52707,2.74161) -- (6.54189,2.74161);
\definecolor{c}{rgb}{0,0,0};
\colorlet{c}{kugray};
\draw [c] (6.55671,2.70662) -- (6.55671,2.74955);
\draw [c] (6.55671,2.74955) -- (6.55671,2.78723);
\draw [c] (6.54189,2.74955) -- (6.55671,2.74955);
\draw [c] (6.55671,2.74955) -- (6.57152,2.74955);
\definecolor{c}{rgb}{0,0,0};
\colorlet{c}{kugray};
\draw [c] (6.58634,2.70574) -- (6.58634,2.75106);
\draw [c] (6.58634,2.75106) -- (6.58634,2.79057);
\draw [c] (6.57152,2.75106) -- (6.58634,2.75106);
\draw [c] (6.58634,2.75106) -- (6.60116,2.75106);
\definecolor{c}{rgb}{0,0,0};
\colorlet{c}{kugray};
\draw [c] (6.61598,2.68906) -- (6.61598,2.73402);
\draw [c] (6.61598,2.73402) -- (6.61598,2.77325);
\draw [c] (6.60116,2.73402) -- (6.61598,2.73402);
\draw [c] (6.61598,2.73402) -- (6.63079,2.73402);
\definecolor{c}{rgb}{0,0,0};
\colorlet{c}{kugray};
\draw [c] (6.64561,2.74536) -- (6.64561,2.78499);
\draw [c] (6.64561,2.78499) -- (6.64561,2.82009);
\draw [c] (6.63079,2.78499) -- (6.64561,2.78499);
\draw [c] (6.64561,2.78499) -- (6.66043,2.78499);
\definecolor{c}{rgb}{0,0,0};
\colorlet{c}{kugray};
\draw [c] (6.67525,2.72189) -- (6.67525,2.76667);
\draw [c] (6.67525,2.76667) -- (6.67525,2.80576);
\draw [c] (6.66043,2.76667) -- (6.67525,2.76667);
\draw [c] (6.67525,2.76667) -- (6.69007,2.76667);
\definecolor{c}{rgb}{0,0,0};
\colorlet{c}{kugray};
\draw [c] (6.70488,2.68289) -- (6.70488,2.72648);
\draw [c] (6.70488,2.72648) -- (6.70488,2.76466);
\draw [c] (6.69007,2.72648) -- (6.70488,2.72648);
\draw [c] (6.70488,2.72648) -- (6.7197,2.72648);
\definecolor{c}{rgb}{0,0,0};
\colorlet{c}{kugray};
\draw [c] (6.73452,2.59638) -- (6.73452,2.64724);
\draw [c] (6.73452,2.64724) -- (6.73452,2.69088);
\draw [c] (6.7197,2.64724) -- (6.73452,2.64724);
\draw [c] (6.73452,2.64724) -- (6.74934,2.64724);
\definecolor{c}{rgb}{0,0,0};
\colorlet{c}{kugray};
\draw [c] (6.76416,2.67667) -- (6.76416,2.72138);
\draw [c] (6.76416,2.72138) -- (6.76416,2.76041);
\draw [c] (6.74934,2.72138) -- (6.76416,2.72138);
\draw [c] (6.76416,2.72138) -- (6.77897,2.72138);
\definecolor{c}{rgb}{0,0,0};
\colorlet{c}{kugray};
\draw [c] (6.79379,2.66082) -- (6.79379,2.70714);
\draw [c] (6.79379,2.70714) -- (6.79379,2.74739);
\draw [c] (6.77897,2.70714) -- (6.79379,2.70714);
\draw [c] (6.79379,2.70714) -- (6.80861,2.70714);
\definecolor{c}{rgb}{0,0,0};
\colorlet{c}{kugray};
\draw [c] (6.82343,2.63019) -- (6.82343,2.67693);
\draw [c] (6.82343,2.67693) -- (6.82343,2.7175);
\draw [c] (6.80861,2.67693) -- (6.82343,2.67693);
\draw [c] (6.82343,2.67693) -- (6.83824,2.67693);
\definecolor{c}{rgb}{0,0,0};
\colorlet{c}{kugray};
\draw [c] (6.85306,2.55239) -- (6.85306,2.60372);
\draw [c] (6.85306,2.60372) -- (6.85306,2.64771);
\draw [c] (6.83824,2.60372) -- (6.85306,2.60372);
\draw [c] (6.85306,2.60372) -- (6.86788,2.60372);
\definecolor{c}{rgb}{0,0,0};
\colorlet{c}{kugray};
\draw [c] (6.8827,2.62294) -- (6.8827,2.67113);
\draw [c] (6.8827,2.67113) -- (6.8827,2.71279);
\draw [c] (6.86788,2.67113) -- (6.8827,2.67113);
\draw [c] (6.8827,2.67113) -- (6.89752,2.67113);
\definecolor{c}{rgb}{0,0,0};
\colorlet{c}{kugray};
\draw [c] (6.91233,2.6117) -- (6.91233,2.65948);
\draw [c] (6.91233,2.65948) -- (6.91233,2.70082);
\draw [c] (6.89752,2.65948) -- (6.91233,2.65948);
\draw [c] (6.91233,2.65948) -- (6.92715,2.65948);
\definecolor{c}{rgb}{0,0,0};
\colorlet{c}{kugray};
\draw [c] (6.94197,2.73243) -- (6.94197,2.77935);
\draw [c] (6.94197,2.77935) -- (6.94197,2.82005);
\draw [c] (6.92715,2.77935) -- (6.94197,2.77935);
\draw [c] (6.94197,2.77935) -- (6.95679,2.77935);
\definecolor{c}{rgb}{0,0,0};
\colorlet{c}{kugray};
\draw [c] (6.97161,2.48019) -- (6.97161,2.53947);
\draw [c] (6.97161,2.53947) -- (6.97161,2.58916);
\draw [c] (6.95679,2.53947) -- (6.97161,2.53947);
\draw [c] (6.97161,2.53947) -- (6.98642,2.53947);
\definecolor{c}{rgb}{0,0,0};
\colorlet{c}{kugray};
\draw [c] (7.00124,2.61681) -- (7.00124,2.66764);
\draw [c] (7.00124,2.66764) -- (7.00124,2.71126);
\draw [c] (6.98642,2.66764) -- (7.00124,2.66764);
\draw [c] (7.00124,2.66764) -- (7.01606,2.66764);
\definecolor{c}{rgb}{0,0,0};
\colorlet{c}{kugray};
\draw [c] (7.03088,2.47861) -- (7.03088,2.53761);
\draw [c] (7.03088,2.53761) -- (7.03088,2.5871);
\draw [c] (7.01606,2.53761) -- (7.03088,2.53761);
\draw [c] (7.03088,2.53761) -- (7.0457,2.53761);
\definecolor{c}{rgb}{0,0,0};
\colorlet{c}{kugray};
\draw [c] (7.06051,2.59337) -- (7.06051,2.64276);
\draw [c] (7.06051,2.64276) -- (7.06051,2.68531);
\draw [c] (7.0457,2.64276) -- (7.06051,2.64276);
\draw [c] (7.06051,2.64276) -- (7.07533,2.64276);
\definecolor{c}{rgb}{0,0,0};
\colorlet{c}{kugray};
\draw [c] (7.09015,2.57461) -- (7.09015,2.62915);
\draw [c] (7.09015,2.62915) -- (7.09015,2.67546);
\draw [c] (7.07533,2.62915) -- (7.09015,2.62915);
\draw [c] (7.09015,2.62915) -- (7.10497,2.62915);
\definecolor{c}{rgb}{0,0,0};
\colorlet{c}{kugray};
\draw [c] (7.11978,2.6974) -- (7.11978,2.74332);
\draw [c] (7.11978,2.74332) -- (7.11978,2.78328);
\draw [c] (7.10497,2.74332) -- (7.11978,2.74332);
\draw [c] (7.11978,2.74332) -- (7.1346,2.74332);
\definecolor{c}{rgb}{0,0,0};
\colorlet{c}{kugray};
\draw [c] (7.14942,2.66302) -- (7.14942,2.70937);
\draw [c] (7.14942,2.70937) -- (7.14942,2.74965);
\draw [c] (7.1346,2.70937) -- (7.14942,2.70937);
\draw [c] (7.14942,2.70937) -- (7.16424,2.70937);
\definecolor{c}{rgb}{0,0,0};
\colorlet{c}{kugray};
\draw [c] (7.17906,2.58815) -- (7.17906,2.64597);
\draw [c] (7.17906,2.64597) -- (7.17906,2.69464);
\draw [c] (7.16424,2.64597) -- (7.17906,2.64597);
\draw [c] (7.17906,2.64597) -- (7.19387,2.64597);
\definecolor{c}{rgb}{0,0,0};
\colorlet{c}{kugray};
\draw [c] (7.20869,2.50752) -- (7.20869,2.5657);
\draw [c] (7.20869,2.5657) -- (7.20869,2.61462);
\draw [c] (7.19387,2.5657) -- (7.20869,2.5657);
\draw [c] (7.20869,2.5657) -- (7.22351,2.5657);
\definecolor{c}{rgb}{0,0,0};
\colorlet{c}{kugray};
\draw [c] (7.23833,2.52123) -- (7.23833,2.57436);
\draw [c] (7.23833,2.57436) -- (7.23833,2.61966);
\draw [c] (7.22351,2.57436) -- (7.23833,2.57436);
\draw [c] (7.23833,2.57436) -- (7.25315,2.57436);
\definecolor{c}{rgb}{0,0,0};
\colorlet{c}{kugray};
\draw [c] (7.26796,2.55873) -- (7.26796,2.6113);
\draw [c] (7.26796,2.6113) -- (7.26796,2.65619);
\draw [c] (7.25315,2.6113) -- (7.26796,2.6113);
\draw [c] (7.26796,2.6113) -- (7.28278,2.6113);
\definecolor{c}{rgb}{0,0,0};
\colorlet{c}{kugray};
\draw [c] (7.2976,2.59094) -- (7.2976,2.64085);
\draw [c] (7.2976,2.64085) -- (7.2976,2.68379);
\draw [c] (7.28278,2.64085) -- (7.2976,2.64085);
\draw [c] (7.2976,2.64085) -- (7.31242,2.64085);
\definecolor{c}{rgb}{0,0,0};
\colorlet{c}{kugray};
\draw [c] (7.32724,2.66777) -- (7.32724,2.71706);
\draw [c] (7.32724,2.71706) -- (7.32724,2.75953);
\draw [c] (7.31242,2.71706) -- (7.32724,2.71706);
\draw [c] (7.32724,2.71706) -- (7.34205,2.71706);
\definecolor{c}{rgb}{0,0,0};
\colorlet{c}{kugray};
\draw [c] (7.35687,2.59037) -- (7.35687,2.6421);
\draw [c] (7.35687,2.6421) -- (7.35687,2.68637);
\draw [c] (7.34205,2.6421) -- (7.35687,2.6421);
\draw [c] (7.35687,2.6421) -- (7.37169,2.6421);
\definecolor{c}{rgb}{0,0,0};
\colorlet{c}{kugray};
\draw [c] (7.38651,2.5739) -- (7.38651,2.62653);
\draw [c] (7.38651,2.62653) -- (7.38651,2.67146);
\draw [c] (7.37169,2.62653) -- (7.38651,2.62653);
\draw [c] (7.38651,2.62653) -- (7.40132,2.62653);
\definecolor{c}{rgb}{0,0,0};
\colorlet{c}{kugray};
\draw [c] (7.41614,2.56099) -- (7.41614,2.61279);
\draw [c] (7.41614,2.61279) -- (7.41614,2.65711);
\draw [c] (7.40132,2.61279) -- (7.41614,2.61279);
\draw [c] (7.41614,2.61279) -- (7.43096,2.61279);
\definecolor{c}{rgb}{0,0,0};
\colorlet{c}{kugray};
\draw [c] (7.44578,2.61977) -- (7.44578,2.67016);
\draw [c] (7.44578,2.67016) -- (7.44578,2.71346);
\draw [c] (7.43096,2.67016) -- (7.44578,2.67016);
\draw [c] (7.44578,2.67016) -- (7.4606,2.67016);
\definecolor{c}{rgb}{0,0,0};
\colorlet{c}{kugray};
\draw [c] (7.47541,2.56963) -- (7.47541,2.62141);
\draw [c] (7.47541,2.62141) -- (7.47541,2.66572);
\draw [c] (7.4606,2.62141) -- (7.47541,2.62141);
\draw [c] (7.47541,2.62141) -- (7.49023,2.62141);
\definecolor{c}{rgb}{0,0,0};
\colorlet{c}{kugray};
\draw [c] (7.50505,2.56848) -- (7.50505,2.62408);
\draw [c] (7.50505,2.62408) -- (7.50505,2.67115);
\draw [c] (7.49023,2.62408) -- (7.50505,2.62408);
\draw [c] (7.50505,2.62408) -- (7.51987,2.62408);
\definecolor{c}{rgb}{0,0,0};
\colorlet{c}{kugray};
\draw [c] (7.53469,2.58512) -- (7.53469,2.63799);
\draw [c] (7.53469,2.63799) -- (7.53469,2.6831);
\draw [c] (7.51987,2.63799) -- (7.53469,2.63799);
\draw [c] (7.53469,2.63799) -- (7.5495,2.63799);
\definecolor{c}{rgb}{0,0,0};
\colorlet{c}{kugray};
\draw [c] (7.56432,2.57531) -- (7.56432,2.63381);
\draw [c] (7.56432,2.63381) -- (7.56432,2.68295);
\draw [c] (7.5495,2.63381) -- (7.56432,2.63381);
\draw [c] (7.56432,2.63381) -- (7.57914,2.63381);
\definecolor{c}{rgb}{0,0,0};
\colorlet{c}{kugray};
\draw [c] (7.59396,2.59086) -- (7.59396,2.64416);
\draw [c] (7.59396,2.64416) -- (7.59396,2.68958);
\draw [c] (7.57914,2.64416) -- (7.59396,2.64416);
\draw [c] (7.59396,2.64416) -- (7.60877,2.64416);
\definecolor{c}{rgb}{0,0,0};
\colorlet{c}{kugray};
\draw [c] (7.62359,2.69067) -- (7.62359,2.73229);
\draw [c] (7.62359,2.73229) -- (7.62359,2.76895);
\draw [c] (7.60877,2.73229) -- (7.62359,2.73229);
\draw [c] (7.62359,2.73229) -- (7.63841,2.73229);
\definecolor{c}{rgb}{0,0,0};
\colorlet{c}{kugray};
\draw [c] (7.65323,2.54373) -- (7.65323,2.60191);
\draw [c] (7.65323,2.60191) -- (7.65323,2.65082);
\draw [c] (7.63841,2.60191) -- (7.65323,2.60191);
\draw [c] (7.65323,2.60191) -- (7.66805,2.60191);
\definecolor{c}{rgb}{0,0,0};
\colorlet{c}{kugray};
\draw [c] (7.68286,2.62652) -- (7.68286,2.67485);
\draw [c] (7.68286,2.67485) -- (7.68286,2.71662);
\draw [c] (7.66805,2.67485) -- (7.68286,2.67485);
\draw [c] (7.68286,2.67485) -- (7.69768,2.67485);
\definecolor{c}{rgb}{0,0,0};
\colorlet{c}{kugray};
\draw [c] (7.7125,2.5223) -- (7.7125,2.58056);
\draw [c] (7.7125,2.58056) -- (7.7125,2.62954);
\draw [c] (7.69768,2.58056) -- (7.7125,2.58056);
\draw [c] (7.7125,2.58056) -- (7.72732,2.58056);
\definecolor{c}{rgb}{0,0,0};
\colorlet{c}{kugray};
\draw [c] (7.74214,2.52932) -- (7.74214,2.58746);
\draw [c] (7.74214,2.58746) -- (7.74214,2.63635);
\draw [c] (7.72732,2.58746) -- (7.74214,2.58746);
\draw [c] (7.74214,2.58746) -- (7.75695,2.58746);
\definecolor{c}{rgb}{0,0,0};
\colorlet{c}{kugray};
\draw [c] (7.77177,2.60875) -- (7.77177,2.65775);
\draw [c] (7.77177,2.65775) -- (7.77177,2.70001);
\draw [c] (7.75695,2.65775) -- (7.77177,2.65775);
\draw [c] (7.77177,2.65775) -- (7.78659,2.65775);
\definecolor{c}{rgb}{0,0,0};
\colorlet{c}{kugray};
\draw [c] (7.80141,2.53381) -- (7.80141,2.58535);
\draw [c] (7.80141,2.58535) -- (7.80141,2.62948);
\draw [c] (7.78659,2.58535) -- (7.80141,2.58535);
\draw [c] (7.80141,2.58535) -- (7.81623,2.58535);
\definecolor{c}{rgb}{0,0,0};
\colorlet{c}{kugray};
\draw [c] (7.83104,2.56688) -- (7.83104,2.62207);
\draw [c] (7.83104,2.62207) -- (7.83104,2.66885);
\draw [c] (7.81623,2.62207) -- (7.83104,2.62207);
\draw [c] (7.83104,2.62207) -- (7.84586,2.62207);
\definecolor{c}{rgb}{0,0,0};
\colorlet{c}{kugray};
\draw [c] (7.86068,2.53374) -- (7.86068,2.58902);
\draw [c] (7.86068,2.58902) -- (7.86068,2.63587);
\draw [c] (7.84586,2.58902) -- (7.86068,2.58902);
\draw [c] (7.86068,2.58902) -- (7.8755,2.58902);
\definecolor{c}{rgb}{0,0,0};
\colorlet{c}{kugray};
\draw [c] (7.89031,2.54177) -- (7.89031,2.59754);
\draw [c] (7.89031,2.59754) -- (7.89031,2.64474);
\draw [c] (7.8755,2.59754) -- (7.89031,2.59754);
\draw [c] (7.89031,2.59754) -- (7.90513,2.59754);
\definecolor{c}{rgb}{0,0,0};
\colorlet{c}{kugray};
\draw [c] (7.91995,2.66944) -- (7.91995,2.71223);
\draw [c] (7.91995,2.71223) -- (7.91995,2.74979);
\draw [c] (7.90513,2.71223) -- (7.91995,2.71223);
\draw [c] (7.91995,2.71223) -- (7.93477,2.71223);
\definecolor{c}{rgb}{0,0,0};
\colorlet{c}{kugray};
\draw [c] (7.94959,2.58273) -- (7.94959,2.63198);
\draw [c] (7.94959,2.63198) -- (7.94959,2.67442);
\draw [c] (7.93477,2.63198) -- (7.94959,2.63198);
\draw [c] (7.94959,2.63198) -- (7.9644,2.63198);
\definecolor{c}{rgb}{0,0,0};
\colorlet{c}{kugray};
\draw [c] (7.97922,2.51141) -- (7.97922,2.56884);
\draw [c] (7.97922,2.56884) -- (7.97922,2.61723);
\draw [c] (7.9644,2.56884) -- (7.97922,2.56884);
\draw [c] (7.97922,2.56884) -- (7.99404,2.56884);
\definecolor{c}{rgb}{0,0,0};
\colorlet{c}{kugray};
\draw [c] (8.00886,2.4778) -- (8.00886,2.5402);
\draw [c] (8.00886,2.5402) -- (8.00886,2.59206);
\draw [c] (7.99404,2.5402) -- (8.00886,2.5402);
\draw [c] (8.00886,2.5402) -- (8.02368,2.5402);
\definecolor{c}{rgb}{0,0,0};
\colorlet{c}{kugray};
\draw [c] (8.03849,2.52697) -- (8.03849,2.58159);
\draw [c] (8.03849,2.58159) -- (8.03849,2.62797);
\draw [c] (8.02368,2.58159) -- (8.03849,2.58159);
\draw [c] (8.03849,2.58159) -- (8.05331,2.58159);
\definecolor{c}{rgb}{0,0,0};
\colorlet{c}{kugray};
\draw [c] (8.06813,2.51298) -- (8.06813,2.57152);
\draw [c] (8.06813,2.57152) -- (8.06813,2.6207);
\draw [c] (8.05331,2.57152) -- (8.06813,2.57152);
\draw [c] (8.06813,2.57152) -- (8.08295,2.57152);
\definecolor{c}{rgb}{0,0,0};
\colorlet{c}{kugray};
\draw [c] (8.09776,2.42785) -- (8.09776,2.49484);
\draw [c] (8.09776,2.49484) -- (8.09776,2.54982);
\draw [c] (8.08295,2.49484) -- (8.09776,2.49484);
\draw [c] (8.09776,2.49484) -- (8.11258,2.49484);
\definecolor{c}{rgb}{0,0,0};
\colorlet{c}{kugray};
\draw [c] (8.1274,2.54013) -- (8.1274,2.59386);
\draw [c] (8.1274,2.59386) -- (8.1274,2.6396);
\draw [c] (8.11258,2.59386) -- (8.1274,2.59386);
\draw [c] (8.1274,2.59386) -- (8.14222,2.59386);
\definecolor{c}{rgb}{0,0,0};
\colorlet{c}{kugray};
\draw [c] (8.15704,2.5211) -- (8.15704,2.5755);
\draw [c] (8.15704,2.5755) -- (8.15704,2.62172);
\draw [c] (8.14222,2.5755) -- (8.15704,2.5755);
\draw [c] (8.15704,2.5755) -- (8.17185,2.5755);
\definecolor{c}{rgb}{0,0,0};
\colorlet{c}{kugray};
\draw [c] (8.18667,2.48828) -- (8.18667,2.54723);
\draw [c] (8.18667,2.54723) -- (8.18667,2.59668);
\draw [c] (8.17185,2.54723) -- (8.18667,2.54723);
\draw [c] (8.18667,2.54723) -- (8.20149,2.54723);
\definecolor{c}{rgb}{0,0,0};
\colorlet{c}{kugray};
\draw [c] (8.21631,2.51647) -- (8.21631,2.57511);
\draw [c] (8.21631,2.57511) -- (8.21631,2.62434);
\draw [c] (8.20149,2.57511) -- (8.21631,2.57511);
\draw [c] (8.21631,2.57511) -- (8.23113,2.57511);
\definecolor{c}{rgb}{0,0,0};
\colorlet{c}{kugray};
\draw [c] (8.24594,2.4989) -- (8.24594,2.55998);
\draw [c] (8.24594,2.55998) -- (8.24594,2.61092);
\draw [c] (8.23113,2.55998) -- (8.24594,2.55998);
\draw [c] (8.24594,2.55998) -- (8.26076,2.55998);
\definecolor{c}{rgb}{0,0,0};
\colorlet{c}{kugray};
\draw [c] (8.27558,2.43272) -- (8.27558,2.4971);
\draw [c] (8.27558,2.4971) -- (8.27558,2.55033);
\draw [c] (8.26076,2.4971) -- (8.27558,2.4971);
\draw [c] (8.27558,2.4971) -- (8.2904,2.4971);
\definecolor{c}{rgb}{0,0,0};
\colorlet{c}{kugray};
\draw [c] (8.30521,2.47691) -- (8.30521,2.53483);
\draw [c] (8.30521,2.53483) -- (8.30521,2.58355);
\draw [c] (8.2904,2.53483) -- (8.30521,2.53483);
\draw [c] (8.30521,2.53483) -- (8.32003,2.53483);
\definecolor{c}{rgb}{0,0,0};
\colorlet{c}{kugray};
\draw [c] (8.33485,2.49351) -- (8.33485,2.55549);
\draw [c] (8.33485,2.55549) -- (8.33485,2.60705);
\draw [c] (8.32003,2.55549) -- (8.33485,2.55549);
\draw [c] (8.33485,2.55549) -- (8.34967,2.55549);
\definecolor{c}{rgb}{0,0,0};
\colorlet{c}{kugray};
\draw [c] (8.36449,2.4196) -- (8.36449,2.48391);
\draw [c] (8.36449,2.48391) -- (8.36449,2.53708);
\draw [c] (8.34967,2.48391) -- (8.36449,2.48391);
\draw [c] (8.36449,2.48391) -- (8.3793,2.48391);
\definecolor{c}{rgb}{0,0,0};
\colorlet{c}{kugray};
\draw [c] (8.39412,2.39842) -- (8.39412,2.46762);
\draw [c] (8.39412,2.46762) -- (8.39412,2.52409);
\draw [c] (8.3793,2.46762) -- (8.39412,2.46762);
\draw [c] (8.39412,2.46762) -- (8.40894,2.46762);
\definecolor{c}{rgb}{0,0,0};
\colorlet{c}{kugray};
\draw [c] (8.42376,2.50236) -- (8.42376,2.56041);
\draw [c] (8.42376,2.56041) -- (8.42376,2.60923);
\draw [c] (8.40894,2.56041) -- (8.42376,2.56041);
\draw [c] (8.42376,2.56041) -- (8.43858,2.56041);
\definecolor{c}{rgb}{0,0,0};
\colorlet{c}{kugray};
\draw [c] (8.45339,2.4599) -- (8.45339,2.5192);
\draw [c] (8.45339,2.5192) -- (8.45339,2.56891);
\draw [c] (8.43858,2.5192) -- (8.45339,2.5192);
\draw [c] (8.45339,2.5192) -- (8.46821,2.5192);
\definecolor{c}{rgb}{0,0,0};
\colorlet{c}{kugray};
\draw [c] (8.48303,2.55597) -- (8.48303,2.61113);
\draw [c] (8.48303,2.61113) -- (8.48303,2.65789);
\draw [c] (8.46821,2.61113) -- (8.48303,2.61113);
\draw [c] (8.48303,2.61113) -- (8.49785,2.61113);
\definecolor{c}{rgb}{0,0,0};
\colorlet{c}{kugray};
\draw [c] (8.51267,2.3936) -- (8.51267,2.46748);
\draw [c] (8.51267,2.46748) -- (8.51267,2.52702);
\draw [c] (8.49785,2.46748) -- (8.51267,2.46748);
\draw [c] (8.51267,2.46748) -- (8.52748,2.46748);
\definecolor{c}{rgb}{0,0,0};
\colorlet{c}{kugray};
\draw [c] (8.5423,2.42909) -- (8.5423,2.50139);
\draw [c] (8.5423,2.50139) -- (8.5423,2.5599);
\draw [c] (8.52748,2.50139) -- (8.5423,2.50139);
\draw [c] (8.5423,2.50139) -- (8.55712,2.50139);
\definecolor{c}{rgb}{0,0,0};
\colorlet{c}{kugray};
\draw [c] (8.57194,2.35401) -- (8.57194,2.42763);
\draw [c] (8.57194,2.42763) -- (8.57194,2.487);
\draw [c] (8.55712,2.42763) -- (8.57194,2.42763);
\draw [c] (8.57194,2.42763) -- (8.58675,2.42763);
\definecolor{c}{rgb}{0,0,0};
\colorlet{c}{kugray};
\draw [c] (8.60157,2.37123) -- (8.60157,2.4445);
\draw [c] (8.60157,2.4445) -- (8.60157,2.50364);
\draw [c] (8.58675,2.4445) -- (8.60157,2.4445);
\draw [c] (8.60157,2.4445) -- (8.61639,2.4445);
\definecolor{c}{rgb}{0,0,0};
\colorlet{c}{kugray};
\draw [c] (8.63121,2.43393) -- (8.63121,2.50355);
\draw [c] (8.63121,2.50355) -- (8.63121,2.5603);
\draw [c] (8.61639,2.50355) -- (8.63121,2.50355);
\draw [c] (8.63121,2.50355) -- (8.64603,2.50355);
\definecolor{c}{rgb}{0,0,0};
\colorlet{c}{kugray};
\draw [c] (8.66084,2.47327) -- (8.66084,2.53303);
\draw [c] (8.66084,2.53303) -- (8.66084,2.58305);
\draw [c] (8.64603,2.53303) -- (8.66084,2.53303);
\draw [c] (8.66084,2.53303) -- (8.67566,2.53303);
\definecolor{c}{rgb}{0,0,0};
\colorlet{c}{kugray};
\draw [c] (8.69048,2.46275) -- (8.69048,2.52243);
\draw [c] (8.69048,2.52243) -- (8.69048,2.5724);
\draw [c] (8.67566,2.52243) -- (8.69048,2.52243);
\draw [c] (8.69048,2.52243) -- (8.7053,2.52243);
\definecolor{c}{rgb}{0,0,0};
\colorlet{c}{kugray};
\draw [c] (8.72012,2.33119) -- (8.72012,2.41226);
\draw [c] (8.72012,2.41226) -- (8.72012,2.47637);
\draw [c] (8.7053,2.41226) -- (8.72012,2.41226);
\draw [c] (8.72012,2.41226) -- (8.73493,2.41226);
\definecolor{c}{rgb}{0,0,0};
\colorlet{c}{kugray};
\draw [c] (8.74975,2.3572) -- (8.74975,2.4275);
\draw [c] (8.74975,2.4275) -- (8.74975,2.48469);
\draw [c] (8.73493,2.4275) -- (8.74975,2.4275);
\draw [c] (8.74975,2.4275) -- (8.76457,2.4275);
\definecolor{c}{rgb}{0,0,0};
\colorlet{c}{kugray};
\draw [c] (8.77939,2.41881) -- (8.77939,2.49045);
\draw [c] (8.77939,2.49045) -- (8.77939,2.54853);
\draw [c] (8.76457,2.49045) -- (8.77939,2.49045);
\draw [c] (8.77939,2.49045) -- (8.79421,2.49045);
\definecolor{c}{rgb}{0,0,0};
\colorlet{c}{kugray};
\draw [c] (8.80902,2.52182) -- (8.80902,2.57585);
\draw [c] (8.80902,2.57585) -- (8.80902,2.62181);
\draw [c] (8.79421,2.57585) -- (8.80902,2.57585);
\draw [c] (8.80902,2.57585) -- (8.82384,2.57585);
\definecolor{c}{rgb}{0,0,0};
\colorlet{c}{kugray};
\draw [c] (8.83866,2.47132) -- (8.83866,2.53141);
\draw [c] (8.83866,2.53141) -- (8.83866,2.58166);
\draw [c] (8.82384,2.53141) -- (8.83866,2.53141);
\draw [c] (8.83866,2.53141) -- (8.85348,2.53141);
\definecolor{c}{rgb}{0,0,0};
\colorlet{c}{kugray};
\draw [c] (8.86829,2.49235) -- (8.86829,2.5502);
\draw [c] (8.86829,2.5502) -- (8.86829,2.59887);
\draw [c] (8.85348,2.5502) -- (8.86829,2.5502);
\draw [c] (8.86829,2.5502) -- (8.88311,2.5502);
\definecolor{c}{rgb}{0,0,0};
\colorlet{c}{kugray};
\draw [c] (8.89793,2.47756) -- (8.89793,2.54006);
\draw [c] (8.89793,2.54006) -- (8.89793,2.59198);
\draw [c] (8.88311,2.54006) -- (8.89793,2.54006);
\draw [c] (8.89793,2.54006) -- (8.91275,2.54006);
\definecolor{c}{rgb}{0,0,0};
\colorlet{c}{kugray};
\draw [c] (8.92757,2.39734) -- (8.92757,2.46961);
\draw [c] (8.92757,2.46961) -- (8.92757,2.52811);
\draw [c] (8.91275,2.46961) -- (8.92757,2.46961);
\draw [c] (8.92757,2.46961) -- (8.94238,2.46961);
\definecolor{c}{rgb}{0,0,0};
\colorlet{c}{kugray};
\draw [c] (8.9572,2.42281) -- (8.9572,2.48693);
\draw [c] (8.9572,2.48693) -- (8.9572,2.53997);
\draw [c] (8.94238,2.48693) -- (8.9572,2.48693);
\draw [c] (8.9572,2.48693) -- (8.97202,2.48693);
\definecolor{c}{rgb}{0,0,0};
\colorlet{c}{kugray};
\draw [c] (8.98684,2.45799) -- (8.98684,2.51852);
\draw [c] (8.98684,2.51852) -- (8.98684,2.56909);
\draw [c] (8.97202,2.51852) -- (8.98684,2.51852);
\draw [c] (8.98684,2.51852) -- (9.00166,2.51852);
\definecolor{c}{rgb}{0,0,0};
\colorlet{c}{kugray};
\draw [c] (9.01647,2.30805) -- (9.01647,2.39137);
\draw [c] (9.01647,2.39137) -- (9.01647,2.45687);
\draw [c] (9.00166,2.39137) -- (9.01647,2.39137);
\draw [c] (9.01647,2.39137) -- (9.03129,2.39137);
\definecolor{c}{rgb}{0,0,0};
\colorlet{c}{kugray};
\draw [c] (9.04611,2.42762) -- (9.04611,2.49931);
\draw [c] (9.04611,2.49931) -- (9.04611,2.55742);
\draw [c] (9.03129,2.49931) -- (9.04611,2.49931);
\draw [c] (9.04611,2.49931) -- (9.06093,2.49931);
\definecolor{c}{rgb}{0,0,0};
\colorlet{c}{kugray};
\draw [c] (9.07574,2.43243) -- (9.07574,2.5203);
\draw [c] (9.07574,2.5203) -- (9.07574,2.58858);
\draw [c] (9.06093,2.5203) -- (9.07574,2.5203);
\draw [c] (9.07574,2.5203) -- (9.09056,2.5203);
\definecolor{c}{rgb}{0,0,0};
\colorlet{c}{kugray};
\draw [c] (9.10538,2.33902) -- (9.10538,2.41416);
\draw [c] (9.10538,2.41416) -- (9.10538,2.47451);
\draw [c] (9.09056,2.41416) -- (9.10538,2.41416);
\draw [c] (9.10538,2.41416) -- (9.1202,2.41416);
\definecolor{c}{rgb}{0,0,0};
\colorlet{c}{kugray};
\draw [c] (9.13502,2.4421) -- (9.13502,2.50747);
\draw [c] (9.13502,2.50747) -- (9.13502,2.56135);
\draw [c] (9.1202,2.50747) -- (9.13502,2.50747);
\draw [c] (9.13502,2.50747) -- (9.14983,2.50747);
\definecolor{c}{rgb}{0,0,0};
\colorlet{c}{kugray};
\draw [c] (9.16465,2.44597) -- (9.16465,2.51695);
\draw [c] (9.16465,2.51695) -- (9.16465,2.5746);
\draw [c] (9.14983,2.51695) -- (9.16465,2.51695);
\draw [c] (9.16465,2.51695) -- (9.17947,2.51695);
\definecolor{c}{rgb}{0,0,0};
\colorlet{c}{kugray};
\draw [c] (9.19429,2.3813) -- (9.19429,2.45249);
\draw [c] (9.19429,2.45249) -- (9.19429,2.51027);
\draw [c] (9.17947,2.45249) -- (9.19429,2.45249);
\draw [c] (9.19429,2.45249) -- (9.20911,2.45249);
\definecolor{c}{rgb}{0,0,0};
\colorlet{c}{kugray};
\draw [c] (9.22392,2.26959) -- (9.22392,2.35173);
\draw [c] (9.22392,2.35173) -- (9.22392,2.4165);
\draw [c] (9.20911,2.35173) -- (9.22392,2.35173);
\draw [c] (9.22392,2.35173) -- (9.23874,2.35173);
\definecolor{c}{rgb}{0,0,0};
\colorlet{c}{kugray};
\draw [c] (9.25356,2.47023) -- (9.25356,2.53233);
\draw [c] (9.25356,2.53233) -- (9.25356,2.58399);
\draw [c] (9.23874,2.53233) -- (9.25356,2.53233);
\draw [c] (9.25356,2.53233) -- (9.26838,2.53233);
\definecolor{c}{rgb}{0,0,0};
\colorlet{c}{kugray};
\draw [c] (9.2832,2.25761) -- (9.2832,2.34259);
\draw [c] (9.2832,2.34259) -- (9.2832,2.40911);
\draw [c] (9.26838,2.34259) -- (9.2832,2.34259);
\draw [c] (9.2832,2.34259) -- (9.29801,2.34259);
\definecolor{c}{rgb}{0,0,0};
\colorlet{c}{kugray};
\draw [c] (9.31283,2.38074) -- (9.31283,2.45146);
\draw [c] (9.31283,2.45146) -- (9.31283,2.50893);
\draw [c] (9.29801,2.45146) -- (9.31283,2.45146);
\draw [c] (9.31283,2.45146) -- (9.32765,2.45146);
\definecolor{c}{rgb}{0,0,0};
\colorlet{c}{kugray};
\draw [c] (9.34247,2.30157) -- (9.34247,2.38622);
\draw [c] (9.34247,2.38622) -- (9.34247,2.45254);
\draw [c] (9.32765,2.38622) -- (9.34247,2.38622);
\draw [c] (9.34247,2.38622) -- (9.35728,2.38622);
\definecolor{c}{rgb}{0,0,0};
\colorlet{c}{kugray};
\draw [c] (9.3721,2.44079) -- (9.3721,2.50347);
\draw [c] (9.3721,2.50347) -- (9.3721,2.55552);
\draw [c] (9.35728,2.50347) -- (9.3721,2.50347);
\draw [c] (9.3721,2.50347) -- (9.38692,2.50347);
\definecolor{c}{rgb}{0,0,0};
\colorlet{c}{kugray};
\draw [c] (9.40174,2.35916) -- (9.40174,2.42967);
\draw [c] (9.40174,2.42967) -- (9.40174,2.487);
\draw [c] (9.38692,2.42967) -- (9.40174,2.42967);
\draw [c] (9.40174,2.42967) -- (9.41656,2.42967);
\definecolor{c}{rgb}{0,0,0};
\colorlet{c}{kugray};
\draw [c] (9.43137,2.34873) -- (9.43137,2.418);
\draw [c] (9.43137,2.418) -- (9.43137,2.4745);
\draw [c] (9.41656,2.418) -- (9.43137,2.418);
\draw [c] (9.43137,2.418) -- (9.44619,2.418);
\definecolor{c}{rgb}{0,0,0};
\colorlet{c}{kugray};
\draw [c] (9.46101,2.48472) -- (9.46101,2.54922);
\draw [c] (9.46101,2.54922) -- (9.46101,2.60252);
\draw [c] (9.44619,2.54922) -- (9.46101,2.54922);
\draw [c] (9.46101,2.54922) -- (9.47583,2.54922);
\definecolor{c}{rgb}{0,0,0};
\colorlet{c}{kugray};
\draw [c] (9.49065,2.28137) -- (9.49065,2.3622);
\draw [c] (9.49065,2.3622) -- (9.49065,2.42615);
\draw [c] (9.47583,2.3622) -- (9.49065,2.3622);
\draw [c] (9.49065,2.3622) -- (9.50546,2.3622);
\definecolor{c}{rgb}{0,0,0};
\colorlet{c}{kugray};
\draw [c] (9.52028,2.44204) -- (9.52028,2.50736);
\draw [c] (9.52028,2.50736) -- (9.52028,2.56121);
\draw [c] (9.50546,2.50736) -- (9.52028,2.50736);
\draw [c] (9.52028,2.50736) -- (9.5351,2.50736);
\definecolor{c}{rgb}{0,0,0};
\colorlet{c}{kugray};
\draw [c] (9.54992,2.16356) -- (9.54992,2.26322);
\draw [c] (9.54992,2.26322) -- (9.54992,2.33839);
\draw [c] (9.5351,2.26322) -- (9.54992,2.26322);
\draw [c] (9.54992,2.26322) -- (9.56474,2.26322);
\definecolor{c}{rgb}{0,0,0};
\colorlet{c}{kugray};
\draw [c] (9.57955,2.38059) -- (9.57955,2.45112);
\draw [c] (9.57955,2.45112) -- (9.57955,2.50847);
\draw [c] (9.56474,2.45112) -- (9.57955,2.45112);
\draw [c] (9.57955,2.45112) -- (9.59437,2.45112);
\definecolor{c}{rgb}{0,0,0};
\colorlet{c}{kugray};
\draw [c] (9.60919,2.39358) -- (9.60919,2.4702);
\draw [c] (9.60919,2.4702) -- (9.60919,2.5315);
\draw [c] (9.59437,2.4702) -- (9.60919,2.4702);
\draw [c] (9.60919,2.4702) -- (9.62401,2.4702);
\definecolor{c}{rgb}{0,0,0};
\colorlet{c}{kugray};
\draw [c] (9.63882,2.215) -- (9.63882,2.30298);
\draw [c] (9.63882,2.30298) -- (9.63882,2.37132);
\draw [c] (9.62401,2.30298) -- (9.63882,2.30298);
\draw [c] (9.63882,2.30298) -- (9.65364,2.30298);
\definecolor{c}{rgb}{0,0,0};
\colorlet{c}{kugray};
\draw [c] (9.66846,2.31943) -- (9.66846,2.39972);
\draw [c] (9.66846,2.39972) -- (9.66846,2.46334);
\draw [c] (9.65364,2.39972) -- (9.66846,2.39972);
\draw [c] (9.66846,2.39972) -- (9.68328,2.39972);
\definecolor{c}{rgb}{0,0,0};
\colorlet{c}{kugray};
\draw [c] (9.6981,2.36691) -- (9.6981,2.44427);
\draw [c] (9.6981,2.44427) -- (9.6981,2.50605);
\draw [c] (9.68328,2.44427) -- (9.6981,2.44427);
\draw [c] (9.6981,2.44427) -- (9.71291,2.44427);
\definecolor{c}{rgb}{0,0,0};
\colorlet{c}{kugray};
\draw [c] (9.72773,2.36518) -- (9.72773,2.44376);
\draw [c] (9.72773,2.44376) -- (9.72773,2.50632);
\draw [c] (9.71291,2.44376) -- (9.72773,2.44376);
\draw [c] (9.72773,2.44376) -- (9.74255,2.44376);
\definecolor{c}{rgb}{0,0,0};
\colorlet{c}{kugray};
\draw [c] (9.75737,2.17203) -- (9.75737,2.267);
\draw [c] (9.75737,2.267) -- (9.75737,2.33947);
\draw [c] (9.74255,2.267) -- (9.75737,2.267);
\draw [c] (9.75737,2.267) -- (9.77219,2.267);
\definecolor{c}{rgb}{0,0,0};
\colorlet{c}{kugray};
\draw [c] (9.787,2.27887) -- (9.787,2.35679);
\draw [c] (9.787,2.35679) -- (9.787,2.41892);
\draw [c] (9.77219,2.35679) -- (9.787,2.35679);
\draw [c] (9.787,2.35679) -- (9.80182,2.35679);
\definecolor{c}{rgb}{0,0,0};
\colorlet{c}{kugray};
\draw [c] (9.81664,2.11839) -- (9.81664,2.23312);
\draw [c] (9.81664,2.23312) -- (9.81664,2.31651);
\draw [c] (9.80182,2.23312) -- (9.81664,2.23312);
\draw [c] (9.81664,2.23312) -- (9.83146,2.23312);
\definecolor{c}{rgb}{0,0,0};
\colorlet{c}{kugray};
\draw [c] (9.84627,2.36042) -- (9.84627,2.42873);
\draw [c] (9.84627,2.42873) -- (9.84627,2.4846);
\draw [c] (9.83146,2.42873) -- (9.84627,2.42873);
\draw [c] (9.84627,2.42873) -- (9.86109,2.42873);
\definecolor{c}{rgb}{0,0,0};
\colorlet{c}{kugray};
\draw [c] (9.87591,2.33174) -- (9.87591,2.40637);
\draw [c] (9.87591,2.40637) -- (9.87591,2.46639);
\draw [c] (9.86109,2.40637) -- (9.87591,2.40637);
\draw [c] (9.87591,2.40637) -- (9.89073,2.40637);
\definecolor{c}{rgb}{0,0,0};
\colorlet{c}{kugray};
\draw [c] (9.90555,2.36337) -- (9.90555,2.43811);
\draw [c] (9.90555,2.43811) -- (9.90555,2.4982);
\draw [c] (9.89073,2.43811) -- (9.90555,2.43811);
\draw [c] (9.90555,2.43811) -- (9.92036,2.43811);
\definecolor{c}{rgb}{0,0,0};
\colorlet{c}{kugray};
\draw [c] (9.93518,2.22165) -- (9.93518,2.328);
\draw [c] (9.93518,2.328) -- (9.93518,2.4069);
\draw [c] (9.92036,2.328) -- (9.93518,2.328);
\draw [c] (9.93518,2.328) -- (9.95,2.328);
\definecolor{c}{rgb}{0,0,0};
\colorlet{c}{natcomp};
\draw [c] (1.51655,5.54553) -- (1.60131,5.4045) -- (1.68607,5.27517) -- (1.77083,5.15545) -- (1.85558,5.0438) -- (1.94034,4.93904) -- (2.0251,4.84024) -- (2.10986,4.74665) -- (2.19462,4.65769) -- (2.27938,4.57286) -- (2.36413,4.49176)
 -- (2.44889,4.41407) -- (2.53365,4.33949) -- (2.61841,4.26781) -- (2.70317,4.19881) -- (2.78793,4.13234) -- (2.87268,4.06825) -- (2.95744,4.00643) -- (3.0422,3.94677) -- (3.12696,3.8892) -- (3.21172,3.83363) -- (3.29648,3.78001) -- (3.38123,3.72829)
 -- (3.46599,3.67842) -- (3.55075,3.63037) -- (3.63551,3.58409) -- (3.72027,3.53955) -- (3.80502,3.49673) -- (3.88978,3.45559) -- (3.97454,3.4161) -- (4.0593,3.37822) -- (4.14406,3.34192) -- (4.22882,3.30715) -- (4.31357,3.27388) -- (4.39833,3.24205)
 -- (4.48309,3.21162) -- (4.56785,3.18254) -- (4.65261,3.15475) -- (4.73737,3.12818) -- (4.82212,3.1028) -- (4.90688,3.07853) -- (4.99164,3.05531) -- (5.0764,3.03308) -- (5.16116,3.01179) -- (5.24592,2.99138) -- (5.33067,2.97179) -- (5.41543,2.95296)
 -- (5.50019,2.93484) -- (5.58495,2.91738) -- (5.66971,2.90053);
\draw [c] (5.66971,2.90053) -- (5.75447,2.88424) -- (5.83922,2.86848) -- (5.92398,2.8532) -- (6.00874,2.83837) -- (6.0935,2.82394) -- (6.17826,2.80989) -- (6.26301,2.79618) -- (6.34777,2.7828) -- (6.43253,2.7697) -- (6.51729,2.75687)
 -- (6.60205,2.74428) -- (6.68681,2.73192) -- (6.77156,2.71977) -- (6.85632,2.7078) -- (6.94108,2.69601) -- (7.02584,2.68437) -- (7.1106,2.67289) -- (7.19536,2.66153) -- (7.28011,2.6503) -- (7.36487,2.63918) -- (7.44963,2.62817) -- (7.53439,2.61724)
 -- (7.61915,2.60641) -- (7.70391,2.59566) -- (7.78866,2.58498) -- (7.87342,2.57436) -- (7.95818,2.56381) -- (8.04294,2.55332) -- (8.1277,2.54287) -- (8.21246,2.53247) -- (8.29721,2.52212) -- (8.38197,2.5118) -- (8.46673,2.50152) -- (8.55149,2.49127)
 -- (8.63625,2.48104) -- (8.72101,2.47084) -- (8.80576,2.46065) -- (8.89052,2.45047) -- (8.97528,2.4403) -- (9.06004,2.43013) -- (9.1448,2.41995) -- (9.22955,2.40976) -- (9.31431,2.39954) -- (9.39907,2.38929) -- (9.48383,2.379) -- (9.56859,2.36866)
 -- (9.65335,2.35824) -- (9.7381,2.34774) -- (9.82286,2.33714);
\draw [c] (9.82286,2.33714) -- (9.90762,2.32641);
\colorlet{c}{kugray};
\draw [c] (1.01482,0.596817) -- (1.01482,3.61122);
\draw [c] (1.01482,3.61122) -- (1.01482,3.82461);
\draw [c] (1,3.61122) -- (1.01482,3.61122);
\draw [c] (1.01482,3.61122) -- (1.02964,3.61122);
\definecolor{c}{rgb}{0,0,0};
\colorlet{c}{kugray};
\draw [c] (1.04445,0.596817) -- (1.04445,1.81958);
\draw [c] (1.04445,1.81958) -- (1.04445,2.03297);
\draw [c] (1.02964,1.81958) -- (1.04445,1.81958);
\draw [c] (1.04445,1.81958) -- (1.05927,1.81958);
\definecolor{c}{rgb}{0,0,0};
\colorlet{c}{kugray};
\draw [c] (1.07409,0.596817) -- (1.07409,3.48891);
\draw [c] (1.07409,3.48891) -- (1.07409,3.7023);
\draw [c] (1.05927,3.48891) -- (1.07409,3.48891);
\draw [c] (1.07409,3.48891) -- (1.08891,3.48891);
\definecolor{c}{rgb}{0,0,0};
\colorlet{c}{kugray};
\draw [c] (1.10373,3.60302) -- (1.10373,3.87778);
\draw [c] (1.10373,3.87778) -- (1.10373,4.02061);
\draw [c] (1.08891,3.87778) -- (1.10373,3.87778);
\draw [c] (1.10373,3.87778) -- (1.11854,3.87778);
\definecolor{c}{rgb}{0,0,0};
\colorlet{c}{kugray};
\draw [c] (1.13336,3.42057) -- (1.13336,3.80196);
\draw [c] (1.13336,3.80196) -- (1.13336,3.96717);
\draw [c] (1.11854,3.80196) -- (1.13336,3.80196);
\draw [c] (1.13336,3.80196) -- (1.14818,3.80196);
\definecolor{c}{rgb}{0,0,0};
\colorlet{c}{kugray};
\draw [c] (1.163,0.596817) -- (1.163,2.01527);
\draw [c] (1.163,2.01527) -- (1.163,2.22866);
\draw [c] (1.14818,2.01527) -- (1.163,2.01527);
\draw [c] (1.163,2.01527) -- (1.17781,2.01527);
\definecolor{c}{rgb}{0,0,0};
\colorlet{c}{kugray};
\draw [c] (1.19263,0.596817) -- (1.19263,3.51401);
\draw [c] (1.19263,3.51401) -- (1.19263,3.7274);
\draw [c] (1.17781,3.51401) -- (1.19263,3.51401);
\draw [c] (1.19263,3.51401) -- (1.20745,3.51401);
\definecolor{c}{rgb}{0,0,0};
\colorlet{c}{kugray};
\draw [c] (1.22227,3.43062) -- (1.22227,3.83166);
\draw [c] (1.22227,3.83166) -- (1.22227,4.00008);
\draw [c] (1.20745,3.83166) -- (1.22227,3.83166);
\draw [c] (1.22227,3.83166) -- (1.23709,3.83166);
\definecolor{c}{rgb}{0,0,0};
\colorlet{c}{kugray};
\draw [c] (1.2519,3.8397) -- (1.2519,4.0877);
\draw [c] (1.2519,4.0877) -- (1.2519,4.22325);
\draw [c] (1.23709,4.0877) -- (1.2519,4.0877);
\draw [c] (1.2519,4.0877) -- (1.26672,4.0877);
\definecolor{c}{rgb}{0,0,0};
\colorlet{c}{kugray};
\draw [c] (1.28154,3.98744) -- (1.28154,4.15251);
\draw [c] (1.28154,4.15251) -- (1.28154,4.25938);
\draw [c] (1.26672,4.15251) -- (1.28154,4.15251);
\draw [c] (1.28154,4.15251) -- (1.29636,4.15251);
\definecolor{c}{rgb}{0,0,0};
\colorlet{c}{kugray};
\draw [c] (1.31118,5.3327) -- (1.31118,5.34989);
\draw [c] (1.31118,5.34989) -- (1.31118,5.36617);
\draw [c] (1.29636,5.34989) -- (1.31118,5.34989);
\draw [c] (1.31118,5.34989) -- (1.32599,5.34989);
\definecolor{c}{rgb}{0,0,0};
\colorlet{c}{kugray};
\draw [c] (1.34081,5.6418) -- (1.34081,5.6522);
\draw [c] (1.34081,5.6522) -- (1.34081,5.66227);
\draw [c] (1.32599,5.6522) -- (1.34081,5.6522);
\draw [c] (1.34081,5.6522) -- (1.35563,5.6522);
\definecolor{c}{rgb}{0,0,0};
\colorlet{c}{kugray};
\draw [c] (1.37045,5.68448) -- (1.37045,5.69437);
\draw [c] (1.37045,5.69437) -- (1.37045,5.70394);
\draw [c] (1.35563,5.69437) -- (1.37045,5.69437);
\draw [c] (1.37045,5.69437) -- (1.38526,5.69437);
\definecolor{c}{rgb}{0,0,0};
\colorlet{c}{kugray};
\draw [c] (1.40008,5.69232) -- (1.40008,5.70203);
\draw [c] (1.40008,5.70203) -- (1.40008,5.71145);
\draw [c] (1.38526,5.70203) -- (1.40008,5.70203);
\draw [c] (1.40008,5.70203) -- (1.4149,5.70203);
\definecolor{c}{rgb}{0,0,0};
\colorlet{c}{kugray};
\draw [c] (1.42972,5.65625) -- (1.42972,5.66643);
\draw [c] (1.42972,5.66643) -- (1.42972,5.67629);
\draw [c] (1.4149,5.66643) -- (1.42972,5.66643);
\draw [c] (1.42972,5.66643) -- (1.44454,5.66643);
\definecolor{c}{rgb}{0,0,0};
\colorlet{c}{kugray};
\draw [c] (1.45935,5.61758) -- (1.45935,5.62829);
\draw [c] (1.45935,5.62829) -- (1.45935,5.63863);
\draw [c] (1.44454,5.62829) -- (1.45935,5.62829);
\draw [c] (1.45935,5.62829) -- (1.47417,5.62829);
\definecolor{c}{rgb}{0,0,0};
\colorlet{c}{kugray};
\draw [c] (1.48899,5.57445) -- (1.48899,5.58624);
\draw [c] (1.48899,5.58624) -- (1.48899,5.5976);
\draw [c] (1.47417,5.58624) -- (1.48899,5.58624);
\draw [c] (1.48899,5.58624) -- (1.50381,5.58624);
\definecolor{c}{rgb}{0,0,0};
\colorlet{c}{kugray};
\draw [c] (1.51863,5.53272) -- (1.51863,5.54514);
\draw [c] (1.51863,5.54514) -- (1.51863,5.55708);
\draw [c] (1.50381,5.54514) -- (1.51863,5.54514);
\draw [c] (1.51863,5.54514) -- (1.53344,5.54514);
\definecolor{c}{rgb}{0,0,0};
\colorlet{c}{kugray};
\draw [c] (1.54826,5.48596) -- (1.54826,5.49969);
\draw [c] (1.54826,5.49969) -- (1.54826,5.51284);
\draw [c] (1.53344,5.49969) -- (1.54826,5.49969);
\draw [c] (1.54826,5.49969) -- (1.56308,5.49969);
\definecolor{c}{rgb}{0,0,0};
\colorlet{c}{kugray};
\draw [c] (1.5779,5.4138) -- (1.5779,5.42885);
\draw [c] (1.5779,5.42885) -- (1.5779,5.44319);
\draw [c] (1.56308,5.42885) -- (1.5779,5.42885);
\draw [c] (1.5779,5.42885) -- (1.59272,5.42885);
\definecolor{c}{rgb}{0,0,0};
\colorlet{c}{kugray};
\draw [c] (1.60753,5.41345) -- (1.60753,5.42901);
\draw [c] (1.60753,5.42901) -- (1.60753,5.44382);
\draw [c] (1.59272,5.42901) -- (1.60753,5.42901);
\draw [c] (1.60753,5.42901) -- (1.62235,5.42901);
\definecolor{c}{rgb}{0,0,0};
\colorlet{c}{kugray};
\draw [c] (1.63717,5.34137) -- (1.63717,5.35821);
\draw [c] (1.63717,5.35821) -- (1.63717,5.37418);
\draw [c] (1.62235,5.35821) -- (1.63717,5.35821);
\draw [c] (1.63717,5.35821) -- (1.65199,5.35821);
\definecolor{c}{rgb}{0,0,0};
\colorlet{c}{kugray};
\draw [c] (1.6668,5.31592) -- (1.6668,5.33358);
\draw [c] (1.6668,5.33358) -- (1.6668,5.35028);
\draw [c] (1.65199,5.33358) -- (1.6668,5.33358);
\draw [c] (1.6668,5.33358) -- (1.68162,5.33358);
\definecolor{c}{rgb}{0,0,0};
\colorlet{c}{kugray};
\draw [c] (1.69644,5.25517) -- (1.69644,5.2753);
\draw [c] (1.69644,5.2753) -- (1.69644,5.2942);
\draw [c] (1.68162,5.2753) -- (1.69644,5.2753);
\draw [c] (1.69644,5.2753) -- (1.71126,5.2753);
\definecolor{c}{rgb}{0,0,0};
\colorlet{c}{kugray};
\draw [c] (1.72608,5.20002) -- (1.72608,5.2216);
\draw [c] (1.72608,5.2216) -- (1.72608,5.24177);
\draw [c] (1.71126,5.2216) -- (1.72608,5.2216);
\draw [c] (1.72608,5.2216) -- (1.74089,5.2216);
\definecolor{c}{rgb}{0,0,0};
\colorlet{c}{kugray};
\draw [c] (1.75571,5.14158) -- (1.75571,5.16518);
\draw [c] (1.75571,5.16518) -- (1.75571,5.18709);
\draw [c] (1.74089,5.16518) -- (1.75571,5.16518);
\draw [c] (1.75571,5.16518) -- (1.77053,5.16518);
\definecolor{c}{rgb}{0,0,0};
\colorlet{c}{kugray};
\draw [c] (1.78535,5.11657) -- (1.78535,5.14067);
\draw [c] (1.78535,5.14067) -- (1.78535,5.16302);
\draw [c] (1.77053,5.14067) -- (1.78535,5.14067);
\draw [c] (1.78535,5.14067) -- (1.80017,5.14067);
\definecolor{c}{rgb}{0,0,0};
\colorlet{c}{kugray};
\draw [c] (1.81498,5.10912) -- (1.81498,5.13446);
\draw [c] (1.81498,5.13446) -- (1.81498,5.15788);
\draw [c] (1.80017,5.13446) -- (1.81498,5.13446);
\draw [c] (1.81498,5.13446) -- (1.8298,5.13446);
\definecolor{c}{rgb}{0,0,0};
\colorlet{c}{kugray};
\draw [c] (1.84462,5.02683) -- (1.84462,5.05542);
\draw [c] (1.84462,5.05542) -- (1.84462,5.08158);
\draw [c] (1.8298,5.05542) -- (1.84462,5.05542);
\draw [c] (1.84462,5.05542) -- (1.85944,5.05542);
\definecolor{c}{rgb}{0,0,0};
\colorlet{c}{kugray};
\draw [c] (1.87425,4.99299) -- (1.87425,5.02669);
\draw [c] (1.87425,5.02669) -- (1.87425,5.05705);
\draw [c] (1.85944,5.02669) -- (1.87425,5.02669);
\draw [c] (1.87425,5.02669) -- (1.88907,5.02669);
\definecolor{c}{rgb}{0,0,0};
\colorlet{c}{kugray};
\draw [c] (1.90389,4.94322) -- (1.90389,4.97626);
\draw [c] (1.90389,4.97626) -- (1.90389,5.00609);
\draw [c] (1.88907,4.97626) -- (1.90389,4.97626);
\draw [c] (1.90389,4.97626) -- (1.91871,4.97626);
\definecolor{c}{rgb}{0,0,0};
\colorlet{c}{kugray};
\draw [c] (1.93353,4.96266) -- (1.93353,4.9937);
\draw [c] (1.93353,4.9937) -- (1.93353,5.02189);
\draw [c] (1.91871,4.9937) -- (1.93353,4.9937);
\draw [c] (1.93353,4.9937) -- (1.94834,4.9937);
\definecolor{c}{rgb}{0,0,0};
\colorlet{c}{kugray};
\draw [c] (1.96316,4.86451) -- (1.96316,4.90192);
\draw [c] (1.96316,4.90192) -- (1.96316,4.93528);
\draw [c] (1.94834,4.90192) -- (1.96316,4.90192);
\draw [c] (1.96316,4.90192) -- (1.97798,4.90192);
\definecolor{c}{rgb}{0,0,0};
\colorlet{c}{kugray};
\draw [c] (1.9928,4.89806) -- (1.9928,4.93394);
\draw [c] (1.9928,4.93394) -- (1.9928,4.96608);
\draw [c] (1.97798,4.93394) -- (1.9928,4.93394);
\draw [c] (1.9928,4.93394) -- (2.00762,4.93394);
\definecolor{c}{rgb}{0,0,0};
\colorlet{c}{kugray};
\draw [c] (2.02243,4.68898) -- (2.02243,4.73645);
\draw [c] (2.02243,4.73645) -- (2.02243,4.77757);
\draw [c] (2.00762,4.73645) -- (2.02243,4.73645);
\draw [c] (2.02243,4.73645) -- (2.03725,4.73645);
\definecolor{c}{rgb}{0,0,0};
\colorlet{c}{kugray};
\draw [c] (2.05207,4.74386) -- (2.05207,4.78796);
\draw [c] (2.05207,4.78796) -- (2.05207,4.82652);
\draw [c] (2.03725,4.78796) -- (2.05207,4.78796);
\draw [c] (2.05207,4.78796) -- (2.06689,4.78796);
\definecolor{c}{rgb}{0,0,0};
\colorlet{c}{kugray};
\draw [c] (2.08171,4.73578) -- (2.08171,4.78026);
\draw [c] (2.08171,4.78026) -- (2.08171,4.81912);
\draw [c] (2.06689,4.78026) -- (2.08171,4.78026);
\draw [c] (2.08171,4.78026) -- (2.09652,4.78026);
\definecolor{c}{rgb}{0,0,0};
\colorlet{c}{kugray};
\draw [c] (2.11134,4.6862) -- (2.11134,4.73798);
\draw [c] (2.11134,4.73798) -- (2.11134,4.7823);
\draw [c] (2.09652,4.73798) -- (2.11134,4.73798);
\draw [c] (2.11134,4.73798) -- (2.12616,4.73798);
\definecolor{c}{rgb}{0,0,0};
\colorlet{c}{kugray};
\draw [c] (2.14098,4.54138) -- (2.14098,4.6098);
\draw [c] (2.14098,4.6098) -- (2.14098,4.66574);
\draw [c] (2.12616,4.6098) -- (2.14098,4.6098);
\draw [c] (2.14098,4.6098) -- (2.15579,4.6098);
\definecolor{c}{rgb}{0,0,0};
\colorlet{c}{kugray};
\draw [c] (2.17061,4.64226) -- (2.17061,4.69994);
\draw [c] (2.17061,4.69994) -- (2.17061,4.74849);
\draw [c] (2.15579,4.69994) -- (2.17061,4.69994);
\draw [c] (2.17061,4.69994) -- (2.18543,4.69994);
\definecolor{c}{rgb}{0,0,0};
\colorlet{c}{kugray};
\draw [c] (2.20025,4.53677) -- (2.20025,4.60067);
\draw [c] (2.20025,4.60067) -- (2.20025,4.65356);
\draw [c] (2.18543,4.60067) -- (2.20025,4.60067);
\draw [c] (2.20025,4.60067) -- (2.21507,4.60067);
\definecolor{c}{rgb}{0,0,0};
\colorlet{c}{kugray};
\draw [c] (2.22988,4.61221) -- (2.22988,4.66583);
\draw [c] (2.22988,4.66583) -- (2.22988,4.71148);
\draw [c] (2.21507,4.66583) -- (2.22988,4.66583);
\draw [c] (2.22988,4.66583) -- (2.2447,4.66583);
\definecolor{c}{rgb}{0,0,0};
\colorlet{c}{kugray};
\draw [c] (2.25952,4.53829) -- (2.25952,4.60341);
\draw [c] (2.25952,4.60341) -- (2.25952,4.65714);
\draw [c] (2.2447,4.60341) -- (2.25952,4.60341);
\draw [c] (2.25952,4.60341) -- (2.27434,4.60341);
\definecolor{c}{rgb}{0,0,0};
\colorlet{c}{kugray};
\draw [c] (2.28916,4.56974) -- (2.28916,4.58026);
\draw [c] (2.28916,4.58026) -- (2.28916,4.59044);
\draw [c] (2.27434,4.58026) -- (2.28916,4.58026);
\draw [c] (2.28916,4.58026) -- (2.30397,4.58026);
\definecolor{c}{rgb}{0,0,0};
\colorlet{c}{kugray};
\draw [c] (2.31879,4.53602) -- (2.31879,4.54723);
\draw [c] (2.31879,4.54723) -- (2.31879,4.55804);
\draw [c] (2.30397,4.54723) -- (2.31879,4.54723);
\draw [c] (2.31879,4.54723) -- (2.33361,4.54723);
\definecolor{c}{rgb}{0,0,0};
\colorlet{c}{kugray};
\draw [c] (2.34843,4.52541) -- (2.34843,4.53687);
\draw [c] (2.34843,4.53687) -- (2.34843,4.54792);
\draw [c] (2.33361,4.53687) -- (2.34843,4.53687);
\draw [c] (2.34843,4.53687) -- (2.36325,4.53687);
\definecolor{c}{rgb}{0,0,0};
\colorlet{c}{kugray};
\draw [c] (2.37806,4.47644) -- (2.37806,4.48842);
\draw [c] (2.37806,4.48842) -- (2.37806,4.49994);
\draw [c] (2.36325,4.48842) -- (2.37806,4.48842);
\draw [c] (2.37806,4.48842) -- (2.39288,4.48842);
\definecolor{c}{rgb}{0,0,0};
\colorlet{c}{kugray};
\draw [c] (2.4077,4.45724) -- (2.4077,4.46981);
\draw [c] (2.4077,4.46981) -- (2.4077,4.48188);
\draw [c] (2.39288,4.46981) -- (2.4077,4.46981);
\draw [c] (2.4077,4.46981) -- (2.42252,4.46981);
\definecolor{c}{rgb}{0,0,0};
\colorlet{c}{kugray};
\draw [c] (2.43733,4.43036) -- (2.43733,4.44342);
\draw [c] (2.43733,4.44342) -- (2.43733,4.45594);
\draw [c] (2.42252,4.44342) -- (2.43733,4.44342);
\draw [c] (2.43733,4.44342) -- (2.45215,4.44342);
\definecolor{c}{rgb}{0,0,0};
\colorlet{c}{kugray};
\draw [c] (2.46697,4.40509) -- (2.46697,4.41861);
\draw [c] (2.46697,4.41861) -- (2.46697,4.43156);
\draw [c] (2.45215,4.41861) -- (2.46697,4.41861);
\draw [c] (2.46697,4.41861) -- (2.48179,4.41861);
\definecolor{c}{rgb}{0,0,0};
\colorlet{c}{kugray};
\draw [c] (2.49661,4.37004) -- (2.49661,4.38467);
\draw [c] (2.49661,4.38467) -- (2.49661,4.39863);
\draw [c] (2.48179,4.38467) -- (2.49661,4.38467);
\draw [c] (2.49661,4.38467) -- (2.51142,4.38467);
\definecolor{c}{rgb}{0,0,0};
\colorlet{c}{kugray};
\draw [c] (2.52624,4.32084) -- (2.52624,4.33659);
\draw [c] (2.52624,4.33659) -- (2.52624,4.35157);
\draw [c] (2.51142,4.33659) -- (2.52624,4.33659);
\draw [c] (2.52624,4.33659) -- (2.54106,4.33659);
\definecolor{c}{rgb}{0,0,0};
\colorlet{c}{kugray};
\draw [c] (2.55588,4.30399) -- (2.55588,4.32008);
\draw [c] (2.55588,4.32008) -- (2.55588,4.33538);
\draw [c] (2.54106,4.32008) -- (2.55588,4.32008);
\draw [c] (2.55588,4.32008) -- (2.5707,4.32008);
\definecolor{c}{rgb}{0,0,0};
\colorlet{c}{kugray};
\draw [c] (2.58551,4.30302) -- (2.58551,4.31916);
\draw [c] (2.58551,4.31916) -- (2.58551,4.3345);
\draw [c] (2.5707,4.31916) -- (2.58551,4.31916);
\draw [c] (2.58551,4.31916) -- (2.60033,4.31916);
\definecolor{c}{rgb}{0,0,0};
\colorlet{c}{kugray};
\draw [c] (2.61515,4.25843) -- (2.61515,4.27581);
\draw [c] (2.61515,4.27581) -- (2.61515,4.29226);
\draw [c] (2.60033,4.27581) -- (2.61515,4.27581);
\draw [c] (2.61515,4.27581) -- (2.62997,4.27581);
\definecolor{c}{rgb}{0,0,0};
\colorlet{c}{kugray};
\draw [c] (2.64478,4.24047) -- (2.64478,4.25814);
\draw [c] (2.64478,4.25814) -- (2.64478,4.27485);
\draw [c] (2.62997,4.25814) -- (2.64478,4.25814);
\draw [c] (2.64478,4.25814) -- (2.6596,4.25814);
\definecolor{c}{rgb}{0,0,0};
\colorlet{c}{kugray};
\draw [c] (2.67442,4.23829) -- (2.67442,4.2564);
\draw [c] (2.67442,4.2564) -- (2.67442,4.27351);
\draw [c] (2.6596,4.2564) -- (2.67442,4.2564);
\draw [c] (2.67442,4.2564) -- (2.68924,4.2564);
\definecolor{c}{rgb}{0,0,0};
\colorlet{c}{kugray};
\draw [c] (2.70406,4.23694) -- (2.70406,4.25493);
\draw [c] (2.70406,4.25493) -- (2.70406,4.27193);
\draw [c] (2.68924,4.25493) -- (2.70406,4.25493);
\draw [c] (2.70406,4.25493) -- (2.71887,4.25493);
\definecolor{c}{rgb}{0,0,0};
\colorlet{c}{kugray};
\draw [c] (2.73369,4.17951) -- (2.73369,4.19937);
\draw [c] (2.73369,4.19937) -- (2.73369,4.21803);
\draw [c] (2.71887,4.19937) -- (2.73369,4.19937);
\draw [c] (2.73369,4.19937) -- (2.74851,4.19937);
\definecolor{c}{rgb}{0,0,0};
\colorlet{c}{kugray};
\draw [c] (2.76333,4.15517) -- (2.76333,4.17566);
\draw [c] (2.76333,4.17566) -- (2.76333,4.19487);
\draw [c] (2.74851,4.17566) -- (2.76333,4.17566);
\draw [c] (2.76333,4.17566) -- (2.77815,4.17566);
\definecolor{c}{rgb}{0,0,0};
\colorlet{c}{kugray};
\draw [c] (2.79296,4.13183) -- (2.79296,4.15328);
\draw [c] (2.79296,4.15328) -- (2.79296,4.17334);
\draw [c] (2.77815,4.15328) -- (2.79296,4.15328);
\draw [c] (2.79296,4.15328) -- (2.80778,4.15328);
\definecolor{c}{rgb}{0,0,0};
\colorlet{c}{kugray};
\draw [c] (2.8226,4.09575) -- (2.8226,4.11831);
\draw [c] (2.8226,4.11831) -- (2.8226,4.13932);
\draw [c] (2.80778,4.11831) -- (2.8226,4.11831);
\draw [c] (2.8226,4.11831) -- (2.83742,4.11831);
\definecolor{c}{rgb}{0,0,0};
\colorlet{c}{kugray};
\draw [c] (2.85224,4.0827) -- (2.85224,4.10635);
\draw [c] (2.85224,4.10635) -- (2.85224,4.12832);
\draw [c] (2.83742,4.10635) -- (2.85224,4.10635);
\draw [c] (2.85224,4.10635) -- (2.86705,4.10635);
\definecolor{c}{rgb}{0,0,0};
\colorlet{c}{kugray};
\draw [c] (2.88187,4.07789) -- (2.88187,4.1009);
\draw [c] (2.88187,4.1009) -- (2.88187,4.12231);
\draw [c] (2.86705,4.1009) -- (2.88187,4.1009);
\draw [c] (2.88187,4.1009) -- (2.89669,4.1009);
\definecolor{c}{rgb}{0,0,0};
\colorlet{c}{kugray};
\draw [c] (2.91151,4.04064) -- (2.91151,4.06493);
\draw [c] (2.91151,4.06493) -- (2.91151,4.08745);
\draw [c] (2.89669,4.06493) -- (2.91151,4.06493);
\draw [c] (2.91151,4.06493) -- (2.92632,4.06493);
\definecolor{c}{rgb}{0,0,0};
\colorlet{c}{kugray};
\draw [c] (2.94114,3.99628) -- (2.94114,4.02251);
\draw [c] (2.94114,4.02251) -- (2.94114,4.04669);
\draw [c] (2.92632,4.02251) -- (2.94114,4.02251);
\draw [c] (2.94114,4.02251) -- (2.95596,4.02251);
\definecolor{c}{rgb}{0,0,0};
\colorlet{c}{kugray};
\draw [c] (2.97078,4.01184) -- (2.97078,4.0375);
\draw [c] (2.97078,4.0375) -- (2.97078,4.06119);
\draw [c] (2.95596,4.0375) -- (2.97078,4.0375);
\draw [c] (2.97078,4.0375) -- (2.9856,4.0375);
\definecolor{c}{rgb}{0,0,0};
\colorlet{c}{kugray};
\draw [c] (3.00041,3.96161) -- (3.00041,3.99102);
\draw [c] (3.00041,3.99102) -- (3.00041,4.01787);
\draw [c] (2.9856,3.99102) -- (3.00041,3.99102);
\draw [c] (3.00041,3.99102) -- (3.01523,3.99102);
\definecolor{c}{rgb}{0,0,0};
\colorlet{c}{kugray};
\draw [c] (3.03005,3.96889) -- (3.03005,3.99707);
\draw [c] (3.03005,3.99707) -- (3.03005,4.02289);
\draw [c] (3.01523,3.99707) -- (3.03005,3.99707);
\draw [c] (3.03005,3.99707) -- (3.04487,3.99707);
\definecolor{c}{rgb}{0,0,0};
\colorlet{c}{kugray};
\draw [c] (3.05969,3.98359) -- (3.05969,4.01138);
\draw [c] (3.05969,4.01138) -- (3.05969,4.03686);
\draw [c] (3.04487,4.01138) -- (3.05969,4.01138);
\draw [c] (3.05969,4.01138) -- (3.0745,4.01138);
\definecolor{c}{rgb}{0,0,0};
\colorlet{c}{kugray};
\draw [c] (3.08932,3.94149) -- (3.08932,3.97026);
\draw [c] (3.08932,3.97026) -- (3.08932,3.99657);
\draw [c] (3.0745,3.97026) -- (3.08932,3.97026);
\draw [c] (3.08932,3.97026) -- (3.10414,3.97026);
\definecolor{c}{rgb}{0,0,0};
\colorlet{c}{kugray};
\draw [c] (3.11896,3.91737) -- (3.11896,3.94858);
\draw [c] (3.11896,3.94858) -- (3.11896,3.97692);
\draw [c] (3.10414,3.94858) -- (3.11896,3.94858);
\draw [c] (3.11896,3.94858) -- (3.13377,3.94858);
\definecolor{c}{rgb}{0,0,0};
\colorlet{c}{kugray};
\draw [c] (3.14859,3.9151) -- (3.14859,3.945);
\draw [c] (3.14859,3.945) -- (3.14859,3.97226);
\draw [c] (3.13377,3.945) -- (3.14859,3.945);
\draw [c] (3.14859,3.945) -- (3.16341,3.945);
\definecolor{c}{rgb}{0,0,0};
\colorlet{c}{kugray};
\draw [c] (3.17823,3.85367) -- (3.17823,3.88787);
\draw [c] (3.17823,3.88787) -- (3.17823,3.91865);
\draw [c] (3.16341,3.88787) -- (3.17823,3.88787);
\draw [c] (3.17823,3.88787) -- (3.19305,3.88787);
\definecolor{c}{rgb}{0,0,0};
\colorlet{c}{kugray};
\draw [c] (3.20786,3.83015) -- (3.20786,3.86574);
\draw [c] (3.20786,3.86574) -- (3.20786,3.89765);
\draw [c] (3.19305,3.86574) -- (3.20786,3.86574);
\draw [c] (3.20786,3.86574) -- (3.22268,3.86574);
\definecolor{c}{rgb}{0,0,0};
\colorlet{c}{kugray};
\draw [c] (3.2375,3.83715) -- (3.2375,3.87247);
\draw [c] (3.2375,3.87247) -- (3.2375,3.90415);
\draw [c] (3.22268,3.87247) -- (3.2375,3.87247);
\draw [c] (3.2375,3.87247) -- (3.25232,3.87247);
\definecolor{c}{rgb}{0,0,0};
\colorlet{c}{kugray};
\draw [c] (3.26714,3.73623) -- (3.26714,3.77648);
\draw [c] (3.26714,3.77648) -- (3.26714,3.81206);
\draw [c] (3.25232,3.77648) -- (3.26714,3.77648);
\draw [c] (3.26714,3.77648) -- (3.28195,3.77648);
\definecolor{c}{rgb}{0,0,0};
\colorlet{c}{kugray};
\draw [c] (3.29677,3.80384) -- (3.29677,3.84266);
\draw [c] (3.29677,3.84266) -- (3.29677,3.87713);
\draw [c] (3.28195,3.84266) -- (3.29677,3.84266);
\draw [c] (3.29677,3.84266) -- (3.31159,3.84266);
\definecolor{c}{rgb}{0,0,0};
\colorlet{c}{kugray};
\draw [c] (3.32641,3.76887) -- (3.32641,3.80715);
\draw [c] (3.32641,3.80715) -- (3.32641,3.84119);
\draw [c] (3.31159,3.80715) -- (3.32641,3.80715);
\draw [c] (3.32641,3.80715) -- (3.34123,3.80715);
\definecolor{c}{rgb}{0,0,0};
\colorlet{c}{kugray};
\draw [c] (3.35604,3.75484) -- (3.35604,3.79665);
\draw [c] (3.35604,3.79665) -- (3.35604,3.83345);
\draw [c] (3.34123,3.79665) -- (3.35604,3.79665);
\draw [c] (3.35604,3.79665) -- (3.37086,3.79665);
\definecolor{c}{rgb}{0,0,0};
\colorlet{c}{kugray};
\draw [c] (3.38568,3.74591) -- (3.38568,3.7847);
\draw [c] (3.38568,3.7847) -- (3.38568,3.81915);
\draw [c] (3.37086,3.7847) -- (3.38568,3.7847);
\draw [c] (3.38568,3.7847) -- (3.4005,3.7847);
\definecolor{c}{rgb}{0,0,0};
\colorlet{c}{kugray};
\draw [c] (3.41531,3.79899) -- (3.41531,3.83877);
\draw [c] (3.41531,3.83877) -- (3.41531,3.874);
\draw [c] (3.4005,3.83877) -- (3.41531,3.83877);
\draw [c] (3.41531,3.83877) -- (3.43013,3.83877);
\definecolor{c}{rgb}{0,0,0};
\colorlet{c}{kugray};
\draw [c] (3.44495,3.68988) -- (3.44495,3.73418);
\draw [c] (3.44495,3.73418) -- (3.44495,3.7729);
\draw [c] (3.43013,3.73418) -- (3.44495,3.73418);
\draw [c] (3.44495,3.73418) -- (3.45977,3.73418);
\definecolor{c}{rgb}{0,0,0};
\colorlet{c}{kugray};
\draw [c] (3.47459,3.63044) -- (3.47459,3.67993);
\draw [c] (3.47459,3.67993) -- (3.47459,3.72255);
\draw [c] (3.45977,3.67993) -- (3.47459,3.67993);
\draw [c] (3.47459,3.67993) -- (3.4894,3.67993);
\definecolor{c}{rgb}{0,0,0};
\colorlet{c}{kugray};
\draw [c] (3.50422,3.78508) -- (3.50422,3.82542);
\draw [c] (3.50422,3.82542) -- (3.50422,3.86107);
\draw [c] (3.4894,3.82542) -- (3.50422,3.82542);
\draw [c] (3.50422,3.82542) -- (3.51904,3.82542);
\definecolor{c}{rgb}{0,0,0};
\colorlet{c}{kugray};
\draw [c] (3.53386,3.72102) -- (3.53386,3.76148);
\draw [c] (3.53386,3.76148) -- (3.53386,3.79724);
\draw [c] (3.51904,3.76148) -- (3.53386,3.76148);
\draw [c] (3.53386,3.76148) -- (3.54868,3.76148);
\definecolor{c}{rgb}{0,0,0};
\colorlet{c}{kugray};
\draw [c] (3.56349,3.66874) -- (3.56349,3.71592);
\draw [c] (3.56349,3.71592) -- (3.56349,3.75683);
\draw [c] (3.54868,3.71592) -- (3.56349,3.71592);
\draw [c] (3.56349,3.71592) -- (3.57831,3.71592);
\definecolor{c}{rgb}{0,0,0};
\colorlet{c}{kugray};
\draw [c] (3.59313,3.62349) -- (3.59313,3.66941);
\draw [c] (3.59313,3.66941) -- (3.59313,3.70935);
\draw [c] (3.57831,3.66941) -- (3.59313,3.66941);
\draw [c] (3.59313,3.66941) -- (3.60795,3.66941);
\definecolor{c}{rgb}{0,0,0};
\colorlet{c}{kugray};
\draw [c] (3.62276,3.72066) -- (3.62276,3.7665);
\draw [c] (3.62276,3.7665) -- (3.62276,3.80639);
\draw [c] (3.60795,3.7665) -- (3.62276,3.7665);
\draw [c] (3.62276,3.7665) -- (3.63758,3.7665);
\definecolor{c}{rgb}{0,0,0};
\colorlet{c}{kugray};
\draw [c] (3.6524,3.60108) -- (3.6524,3.65117);
\draw [c] (3.6524,3.65117) -- (3.6524,3.69424);
\draw [c] (3.63758,3.65117) -- (3.6524,3.65117);
\draw [c] (3.6524,3.65117) -- (3.66722,3.65117);
\definecolor{c}{rgb}{0,0,0};
\colorlet{c}{kugray};
\draw [c] (3.68204,3.59622) -- (3.68204,3.64503);
\draw [c] (3.68204,3.64503) -- (3.68204,3.68715);
\draw [c] (3.66722,3.64503) -- (3.68204,3.64503);
\draw [c] (3.68204,3.64503) -- (3.69685,3.64503);
\definecolor{c}{rgb}{0,0,0};
\colorlet{c}{kugray};
\draw [c] (3.71167,3.61286) -- (3.71167,3.66979);
\draw [c] (3.71167,3.66979) -- (3.71167,3.71782);
\draw [c] (3.69685,3.66979) -- (3.71167,3.66979);
\draw [c] (3.71167,3.66979) -- (3.72649,3.66979);
\definecolor{c}{rgb}{0,0,0};
\colorlet{c}{kugray};
\draw [c] (3.74131,3.57945) -- (3.74131,3.62832);
\draw [c] (3.74131,3.62832) -- (3.74131,3.67048);
\draw [c] (3.72649,3.62832) -- (3.74131,3.62832);
\draw [c] (3.74131,3.62832) -- (3.75613,3.62832);
\definecolor{c}{rgb}{0,0,0};
\colorlet{c}{kugray};
\draw [c] (3.77094,3.5864) -- (3.77094,3.6364);
\draw [c] (3.77094,3.6364) -- (3.77094,3.67939);
\draw [c] (3.75613,3.6364) -- (3.77094,3.6364);
\draw [c] (3.77094,3.6364) -- (3.78576,3.6364);
\definecolor{c}{rgb}{0,0,0};
\colorlet{c}{kugray};
\draw [c] (3.80058,3.58352) -- (3.80058,3.63583);
\draw [c] (3.80058,3.63583) -- (3.80058,3.68052);
\draw [c] (3.78576,3.63583) -- (3.80058,3.63583);
\draw [c] (3.80058,3.63583) -- (3.8154,3.63583);
\definecolor{c}{rgb}{0,0,0};
\colorlet{c}{kugray};
\draw [c] (3.83022,3.54338) -- (3.83022,3.60015);
\draw [c] (3.83022,3.60015) -- (3.83022,3.64806);
\draw [c] (3.8154,3.60015) -- (3.83022,3.60015);
\draw [c] (3.83022,3.60015) -- (3.84503,3.60015);
\definecolor{c}{rgb}{0,0,0};
\colorlet{c}{kugray};
\draw [c] (3.85985,3.54491) -- (3.85985,3.59885);
\draw [c] (3.85985,3.59885) -- (3.85985,3.64473);
\draw [c] (3.84503,3.59885) -- (3.85985,3.59885);
\draw [c] (3.85985,3.59885) -- (3.87467,3.59885);
\definecolor{c}{rgb}{0,0,0};
\colorlet{c}{kugray};
\draw [c] (3.88949,3.51526) -- (3.88949,3.57482);
\draw [c] (3.88949,3.57482) -- (3.88949,3.62471);
\draw [c] (3.87467,3.57482) -- (3.88949,3.57482);
\draw [c] (3.88949,3.57482) -- (3.9043,3.57482);
\definecolor{c}{rgb}{0,0,0};
\colorlet{c}{kugray};
\draw [c] (3.91912,3.60926) -- (3.91912,3.65245);
\draw [c] (3.91912,3.65245) -- (3.91912,3.69032);
\draw [c] (3.9043,3.65245) -- (3.91912,3.65245);
\draw [c] (3.91912,3.65245) -- (3.93394,3.65245);
\definecolor{c}{rgb}{0,0,0};
\colorlet{c}{kugray};
\draw [c] (3.94876,3.62613) -- (3.94876,3.66623);
\draw [c] (3.94876,3.66623) -- (3.94876,3.7017);
\draw [c] (3.93394,3.66623) -- (3.94876,3.66623);
\draw [c] (3.94876,3.66623) -- (3.96358,3.66623);
\definecolor{c}{rgb}{0,0,0};
\colorlet{c}{kugray};
\draw [c] (3.97839,3.57716) -- (3.97839,3.60885);
\draw [c] (3.97839,3.60885) -- (3.97839,3.63758);
\draw [c] (3.96358,3.60885) -- (3.97839,3.60885);
\draw [c] (3.97839,3.60885) -- (3.99321,3.60885);
\definecolor{c}{rgb}{0,0,0};
\colorlet{c}{kugray};
\draw [c] (4.00803,3.51864) -- (4.00803,3.54777);
\draw [c] (4.00803,3.54777) -- (4.00803,3.57438);
\draw [c] (3.99321,3.54777) -- (4.00803,3.54777);
\draw [c] (4.00803,3.54777) -- (4.02285,3.54777);
\definecolor{c}{rgb}{0,0,0};
\colorlet{c}{kugray};
\draw [c] (4.03767,3.54945) -- (4.03767,3.5713);
\draw [c] (4.03767,3.5713) -- (4.03767,3.59171);
\draw [c] (4.02285,3.5713) -- (4.03767,3.5713);
\draw [c] (4.03767,3.5713) -- (4.05248,3.5713);
\definecolor{c}{rgb}{0,0,0};
\colorlet{c}{kugray};
\draw [c] (4.0673,3.52301) -- (4.0673,3.54604);
\draw [c] (4.0673,3.54604) -- (4.0673,3.56746);
\draw [c] (4.05248,3.54604) -- (4.0673,3.54604);
\draw [c] (4.0673,3.54604) -- (4.08212,3.54604);
\definecolor{c}{rgb}{0,0,0};
\colorlet{c}{kugray};
\draw [c] (4.09694,3.48934) -- (4.09694,3.51203);
\draw [c] (4.09694,3.51203) -- (4.09694,3.53317);
\draw [c] (4.08212,3.51203) -- (4.09694,3.51203);
\draw [c] (4.09694,3.51203) -- (4.11175,3.51203);
\definecolor{c}{rgb}{0,0,0};
\colorlet{c}{kugray};
\draw [c] (4.12657,3.50377) -- (4.12657,3.52636);
\draw [c] (4.12657,3.52636) -- (4.12657,3.5474);
\draw [c] (4.11175,3.52636) -- (4.12657,3.52636);
\draw [c] (4.12657,3.52636) -- (4.14139,3.52636);
\definecolor{c}{rgb}{0,0,0};
\colorlet{c}{kugray};
\draw [c] (4.15621,3.51321) -- (4.15621,3.53483);
\draw [c] (4.15621,3.53483) -- (4.15621,3.55503);
\draw [c] (4.14139,3.53483) -- (4.15621,3.53483);
\draw [c] (4.15621,3.53483) -- (4.17103,3.53483);
\definecolor{c}{rgb}{0,0,0};
\colorlet{c}{kugray};
\draw [c] (4.18584,3.5001) -- (4.18584,3.52281);
\draw [c] (4.18584,3.52281) -- (4.18584,3.54395);
\draw [c] (4.17103,3.52281) -- (4.18584,3.52281);
\draw [c] (4.18584,3.52281) -- (4.20066,3.52281);
\definecolor{c}{rgb}{0,0,0};
\colorlet{c}{kugray};
\draw [c] (4.21548,3.46277) -- (4.21548,3.48687);
\draw [c] (4.21548,3.48687) -- (4.21548,3.50921);
\draw [c] (4.20066,3.48687) -- (4.21548,3.48687);
\draw [c] (4.21548,3.48687) -- (4.2303,3.48687);
\definecolor{c}{rgb}{0,0,0};
\colorlet{c}{kugray};
\draw [c] (4.24512,3.50438) -- (4.24512,3.52639);
\draw [c] (4.24512,3.52639) -- (4.24512,3.54694);
\draw [c] (4.2303,3.52639) -- (4.24512,3.52639);
\draw [c] (4.24512,3.52639) -- (4.25993,3.52639);
\definecolor{c}{rgb}{0,0,0};
\colorlet{c}{kugray};
\draw [c] (4.27475,3.47759) -- (4.27475,3.50131);
\draw [c] (4.27475,3.50131) -- (4.27475,3.52334);
\draw [c] (4.25993,3.50131) -- (4.27475,3.50131);
\draw [c] (4.27475,3.50131) -- (4.28957,3.50131);
\definecolor{c}{rgb}{0,0,0};
\colorlet{c}{kugray};
\draw [c] (4.30439,3.4529) -- (4.30439,3.47803);
\draw [c] (4.30439,3.47803) -- (4.30439,3.50126);
\draw [c] (4.28957,3.47803) -- (4.30439,3.47803);
\draw [c] (4.30439,3.47803) -- (4.31921,3.47803);
\definecolor{c}{rgb}{0,0,0};
\colorlet{c}{kugray};
\draw [c] (4.33402,3.41852) -- (4.33402,3.44434);
\draw [c] (4.33402,3.44434) -- (4.33402,3.46816);
\draw [c] (4.31921,3.44434) -- (4.33402,3.44434);
\draw [c] (4.33402,3.44434) -- (4.34884,3.44434);
\definecolor{c}{rgb}{0,0,0};
\colorlet{c}{kugray};
\draw [c] (4.36366,3.41395) -- (4.36366,3.4399);
\draw [c] (4.36366,3.4399) -- (4.36366,3.46384);
\draw [c] (4.34884,3.4399) -- (4.36366,3.4399);
\draw [c] (4.36366,3.4399) -- (4.37848,3.4399);
\definecolor{c}{rgb}{0,0,0};
\colorlet{c}{kugray};
\draw [c] (4.39329,3.46193) -- (4.39329,3.48588);
\draw [c] (4.39329,3.48588) -- (4.39329,3.5081);
\draw [c] (4.37848,3.48588) -- (4.39329,3.48588);
\draw [c] (4.39329,3.48588) -- (4.40811,3.48588);
\definecolor{c}{rgb}{0,0,0};
\colorlet{c}{kugray};
\draw [c] (4.42293,3.43862) -- (4.42293,3.46556);
\draw [c] (4.42293,3.46556) -- (4.42293,3.49033);
\draw [c] (4.40811,3.46556) -- (4.42293,3.46556);
\draw [c] (4.42293,3.46556) -- (4.43775,3.46556);
\definecolor{c}{rgb}{0,0,0};
\colorlet{c}{kugray};
\draw [c] (4.45257,3.40413) -- (4.45257,3.43062);
\draw [c] (4.45257,3.43062) -- (4.45257,3.45502);
\draw [c] (4.43775,3.43062) -- (4.45257,3.43062);
\draw [c] (4.45257,3.43062) -- (4.46738,3.43062);
\definecolor{c}{rgb}{0,0,0};
\colorlet{c}{kugray};
\draw [c] (4.4822,3.45965) -- (4.4822,3.48371);
\draw [c] (4.4822,3.48371) -- (4.4822,3.50602);
\draw [c] (4.46738,3.48371) -- (4.4822,3.48371);
\draw [c] (4.4822,3.48371) -- (4.49702,3.48371);
\definecolor{c}{rgb}{0,0,0};
\colorlet{c}{kugray};
\draw [c] (4.51184,3.43161) -- (4.51184,3.45645);
\draw [c] (4.51184,3.45645) -- (4.51184,3.47944);
\draw [c] (4.49702,3.45645) -- (4.51184,3.45645);
\draw [c] (4.51184,3.45645) -- (4.52666,3.45645);
\definecolor{c}{rgb}{0,0,0};
\colorlet{c}{kugray};
\draw [c] (4.54147,3.41238) -- (4.54147,3.43844);
\draw [c] (4.54147,3.43844) -- (4.54147,3.46247);
\draw [c] (4.52666,3.43844) -- (4.54147,3.43844);
\draw [c] (4.54147,3.43844) -- (4.55629,3.43844);
\definecolor{c}{rgb}{0,0,0};
\colorlet{c}{kugray};
\draw [c] (4.57111,3.40389) -- (4.57111,3.43019);
\draw [c] (4.57111,3.43019) -- (4.57111,3.45442);
\draw [c] (4.55629,3.43019) -- (4.57111,3.43019);
\draw [c] (4.57111,3.43019) -- (4.58593,3.43019);
\definecolor{c}{rgb}{0,0,0};
\colorlet{c}{kugray};
\draw [c] (4.60075,3.40524) -- (4.60075,3.43191);
\draw [c] (4.60075,3.43191) -- (4.60075,3.45646);
\draw [c] (4.58593,3.43191) -- (4.60075,3.43191);
\draw [c] (4.60075,3.43191) -- (4.61556,3.43191);
\definecolor{c}{rgb}{0,0,0};
\colorlet{c}{kugray};
\draw [c] (4.63038,3.40284) -- (4.63038,3.42898);
\draw [c] (4.63038,3.42898) -- (4.63038,3.45308);
\draw [c] (4.61556,3.42898) -- (4.63038,3.42898);
\draw [c] (4.63038,3.42898) -- (4.6452,3.42898);
\definecolor{c}{rgb}{0,0,0};
\colorlet{c}{kugray};
\draw [c] (4.66002,3.40215) -- (4.66002,3.42856);
\draw [c] (4.66002,3.42856) -- (4.66002,3.45289);
\draw [c] (4.6452,3.42856) -- (4.66002,3.42856);
\draw [c] (4.66002,3.42856) -- (4.67483,3.42856);
\definecolor{c}{rgb}{0,0,0};
\colorlet{c}{kugray};
\draw [c] (4.68965,3.33528) -- (4.68965,3.36407);
\draw [c] (4.68965,3.36407) -- (4.68965,3.3904);
\draw [c] (4.67483,3.36407) -- (4.68965,3.36407);
\draw [c] (4.68965,3.36407) -- (4.70447,3.36407);
\definecolor{c}{rgb}{0,0,0};
\colorlet{c}{kugray};
\draw [c] (4.71929,3.39699) -- (4.71929,3.42459);
\draw [c] (4.71929,3.42459) -- (4.71929,3.44992);
\draw [c] (4.70447,3.42459) -- (4.71929,3.42459);
\draw [c] (4.71929,3.42459) -- (4.73411,3.42459);
\definecolor{c}{rgb}{0,0,0};
\colorlet{c}{kugray};
\draw [c] (4.74892,3.37826) -- (4.74892,3.40524);
\draw [c] (4.74892,3.40524) -- (4.74892,3.43005);
\draw [c] (4.73411,3.40524) -- (4.74892,3.40524);
\draw [c] (4.74892,3.40524) -- (4.76374,3.40524);
\definecolor{c}{rgb}{0,0,0};
\colorlet{c}{kugray};
\draw [c] (4.77856,3.42016) -- (4.77856,3.44594);
\draw [c] (4.77856,3.44594) -- (4.77856,3.46973);
\draw [c] (4.76374,3.44594) -- (4.77856,3.44594);
\draw [c] (4.77856,3.44594) -- (4.79338,3.44594);
\definecolor{c}{rgb}{0,0,0};
\colorlet{c}{kugray};
\draw [c] (4.8082,3.38648) -- (4.8082,3.41303);
\draw [c] (4.8082,3.41303) -- (4.8082,3.43746);
\draw [c] (4.79338,3.41303) -- (4.8082,3.41303);
\draw [c] (4.8082,3.41303) -- (4.82301,3.41303);
\definecolor{c}{rgb}{0,0,0};
\colorlet{c}{kugray};
\draw [c] (4.83783,3.39861) -- (4.83783,3.42469);
\draw [c] (4.83783,3.42469) -- (4.83783,3.44873);
\draw [c] (4.82301,3.42469) -- (4.83783,3.42469);
\draw [c] (4.83783,3.42469) -- (4.85265,3.42469);
\definecolor{c}{rgb}{0,0,0};
\colorlet{c}{kugray};
\draw [c] (4.86747,3.30875) -- (4.86747,3.33911);
\draw [c] (4.86747,3.33911) -- (4.86747,3.36675);
\draw [c] (4.85265,3.33911) -- (4.86747,3.33911);
\draw [c] (4.86747,3.33911) -- (4.88228,3.33911);
\definecolor{c}{rgb}{0,0,0};
\colorlet{c}{kugray};
\draw [c] (4.8971,3.35508) -- (4.8971,3.38333);
\draw [c] (4.8971,3.38333) -- (4.8971,3.4092);
\draw [c] (4.88228,3.38333) -- (4.8971,3.38333);
\draw [c] (4.8971,3.38333) -- (4.91192,3.38333);
\definecolor{c}{rgb}{0,0,0};
\colorlet{c}{kugray};
\draw [c] (4.92674,3.30415) -- (4.92674,3.33311);
\draw [c] (4.92674,3.33311) -- (4.92674,3.35958);
\draw [c] (4.91192,3.33311) -- (4.92674,3.33311);
\draw [c] (4.92674,3.33311) -- (4.94156,3.33311);
\definecolor{c}{rgb}{0,0,0};
\colorlet{c}{kugray};
\draw [c] (4.95637,3.33255) -- (4.95637,3.36143);
\draw [c] (4.95637,3.36143) -- (4.95637,3.38782);
\draw [c] (4.94156,3.36143) -- (4.95637,3.36143);
\draw [c] (4.95637,3.36143) -- (4.97119,3.36143);
\definecolor{c}{rgb}{0,0,0};
\colorlet{c}{kugray};
\draw [c] (4.98601,3.37358) -- (4.98601,3.40123);
\draw [c] (4.98601,3.40123) -- (4.98601,3.42661);
\draw [c] (4.97119,3.40123) -- (4.98601,3.40123);
\draw [c] (4.98601,3.40123) -- (5.00083,3.40123);
\definecolor{c}{rgb}{0,0,0};
\colorlet{c}{kugray};
\draw [c] (5.01565,3.35642) -- (5.01565,3.38431);
\draw [c] (5.01565,3.38431) -- (5.01565,3.40989);
\draw [c] (5.00083,3.38431) -- (5.01565,3.38431);
\draw [c] (5.01565,3.38431) -- (5.03046,3.38431);
\definecolor{c}{rgb}{0,0,0};
\colorlet{c}{kugray};
\draw [c] (5.04528,3.31958) -- (5.04528,3.34899);
\draw [c] (5.04528,3.34899) -- (5.04528,3.37583);
\draw [c] (5.03046,3.34899) -- (5.04528,3.34899);
\draw [c] (5.04528,3.34899) -- (5.0601,3.34899);
\definecolor{c}{rgb}{0,0,0};
\colorlet{c}{kugray};
\draw [c] (5.07492,3.3449) -- (5.07492,3.37413);
\draw [c] (5.07492,3.37413) -- (5.07492,3.40082);
\draw [c] (5.0601,3.37413) -- (5.07492,3.37413);
\draw [c] (5.07492,3.37413) -- (5.08974,3.37413);
\definecolor{c}{rgb}{0,0,0};
\colorlet{c}{kugray};
\draw [c] (5.10455,3.35313) -- (5.10455,3.38196);
\draw [c] (5.10455,3.38196) -- (5.10455,3.40832);
\draw [c] (5.08974,3.38196) -- (5.10455,3.38196);
\draw [c] (5.10455,3.38196) -- (5.11937,3.38196);
\definecolor{c}{rgb}{0,0,0};
\colorlet{c}{kugray};
\draw [c] (5.13419,3.35694) -- (5.13419,3.38661);
\draw [c] (5.13419,3.38661) -- (5.13419,3.41367);
\draw [c] (5.11937,3.38661) -- (5.13419,3.38661);
\draw [c] (5.13419,3.38661) -- (5.14901,3.38661);
\definecolor{c}{rgb}{0,0,0};
\colorlet{c}{kugray};
\draw [c] (5.16382,3.34144) -- (5.16382,3.36991);
\draw [c] (5.16382,3.36991) -- (5.16382,3.39596);
\draw [c] (5.14901,3.36991) -- (5.16382,3.36991);
\draw [c] (5.16382,3.36991) -- (5.17864,3.36991);
\definecolor{c}{rgb}{0,0,0};
\colorlet{c}{kugray};
\draw [c] (5.19346,3.37738) -- (5.19346,3.40395);
\draw [c] (5.19346,3.40395) -- (5.19346,3.42841);
\draw [c] (5.17864,3.40395) -- (5.19346,3.40395);
\draw [c] (5.19346,3.40395) -- (5.20828,3.40395);
\definecolor{c}{rgb}{0,0,0};
\colorlet{c}{kugray};
\draw [c] (5.2231,3.33251) -- (5.2231,3.36208);
\draw [c] (5.2231,3.36208) -- (5.2231,3.38905);
\draw [c] (5.20828,3.36208) -- (5.2231,3.36208);
\draw [c] (5.2231,3.36208) -- (5.23791,3.36208);
\definecolor{c}{rgb}{0,0,0};
\colorlet{c}{kugray};
\draw [c] (5.25273,3.32212) -- (5.25273,3.3512);
\draw [c] (5.25273,3.3512) -- (5.25273,3.37777);
\draw [c] (5.23791,3.3512) -- (5.25273,3.3512);
\draw [c] (5.25273,3.3512) -- (5.26755,3.3512);
\definecolor{c}{rgb}{0,0,0};
\colorlet{c}{kugray};
\draw [c] (5.28237,3.37237) -- (5.28237,3.40019);
\draw [c] (5.28237,3.40019) -- (5.28237,3.42569);
\draw [c] (5.26755,3.40019) -- (5.28237,3.40019);
\draw [c] (5.28237,3.40019) -- (5.29719,3.40019);
\definecolor{c}{rgb}{0,0,0};
\colorlet{c}{kugray};
\draw [c] (5.312,3.32807) -- (5.312,3.3584);
\draw [c] (5.312,3.3584) -- (5.312,3.38601);
\draw [c] (5.29719,3.3584) -- (5.312,3.3584);
\draw [c] (5.312,3.3584) -- (5.32682,3.3584);
\definecolor{c}{rgb}{0,0,0};
\colorlet{c}{kugray};
\draw [c] (5.34164,3.30394) -- (5.34164,3.33512);
\draw [c] (5.34164,3.33512) -- (5.34164,3.36344);
\draw [c] (5.32682,3.33512) -- (5.34164,3.33512);
\draw [c] (5.34164,3.33512) -- (5.35646,3.33512);
\definecolor{c}{rgb}{0,0,0};
\colorlet{c}{kugray};
\draw [c] (5.37127,3.27081) -- (5.37127,3.30238);
\draw [c] (5.37127,3.30238) -- (5.37127,3.33101);
\draw [c] (5.35646,3.30238) -- (5.37127,3.30238);
\draw [c] (5.37127,3.30238) -- (5.38609,3.30238);
\definecolor{c}{rgb}{0,0,0};
\colorlet{c}{kugray};
\draw [c] (5.40091,3.34778) -- (5.40091,3.37729);
\draw [c] (5.40091,3.37729) -- (5.40091,3.40421);
\draw [c] (5.38609,3.37729) -- (5.40091,3.37729);
\draw [c] (5.40091,3.37729) -- (5.41573,3.37729);
\definecolor{c}{rgb}{0,0,0};
\colorlet{c}{kugray};
\draw [c] (5.43055,3.31745) -- (5.43055,3.34765);
\draw [c] (5.43055,3.34765) -- (5.43055,3.37514);
\draw [c] (5.41573,3.34765) -- (5.43055,3.34765);
\draw [c] (5.43055,3.34765) -- (5.44536,3.34765);
\definecolor{c}{rgb}{0,0,0};
\colorlet{c}{kugray};
\draw [c] (5.46018,3.27009) -- (5.46018,3.30118);
\draw [c] (5.46018,3.30118) -- (5.46018,3.32942);
\draw [c] (5.44536,3.30118) -- (5.46018,3.30118);
\draw [c] (5.46018,3.30118) -- (5.475,3.30118);
\definecolor{c}{rgb}{0,0,0};
\colorlet{c}{kugray};
\draw [c] (5.48982,3.30023) -- (5.48982,3.32999);
\draw [c] (5.48982,3.32999) -- (5.48982,3.35712);
\draw [c] (5.475,3.32999) -- (5.48982,3.32999);
\draw [c] (5.48982,3.32999) -- (5.50464,3.32999);
\definecolor{c}{rgb}{0,0,0};
\colorlet{c}{kugray};
\draw [c] (5.51945,3.27827) -- (5.51945,3.30971);
\draw [c] (5.51945,3.30971) -- (5.51945,3.33824);
\draw [c] (5.50464,3.30971) -- (5.51945,3.30971);
\draw [c] (5.51945,3.30971) -- (5.53427,3.30971);
\definecolor{c}{rgb}{0,0,0};
\colorlet{c}{kugray};
\draw [c] (5.54909,3.27339) -- (5.54909,3.30456);
\draw [c] (5.54909,3.30456) -- (5.54909,3.33287);
\draw [c] (5.53427,3.30456) -- (5.54909,3.30456);
\draw [c] (5.54909,3.30456) -- (5.56391,3.30456);
\definecolor{c}{rgb}{0,0,0};
\colorlet{c}{kugray};
\draw [c] (5.57873,3.2928) -- (5.57873,3.32215);
\draw [c] (5.57873,3.32215) -- (5.57873,3.34893);
\draw [c] (5.56391,3.32215) -- (5.57873,3.32215);
\draw [c] (5.57873,3.32215) -- (5.59354,3.32215);
\definecolor{c}{rgb}{0,0,0};
\colorlet{c}{kugray};
\draw [c] (5.60836,3.32728) -- (5.60836,3.35666);
\draw [c] (5.60836,3.35666) -- (5.60836,3.38348);
\draw [c] (5.59354,3.35666) -- (5.60836,3.35666);
\draw [c] (5.60836,3.35666) -- (5.62318,3.35666);
\definecolor{c}{rgb}{0,0,0};
\colorlet{c}{kugray};
\draw [c] (5.638,3.35064) -- (5.638,3.3789);
\draw [c] (5.638,3.3789) -- (5.638,3.40479);
\draw [c] (5.62318,3.3789) -- (5.638,3.3789);
\draw [c] (5.638,3.3789) -- (5.65281,3.3789);
\definecolor{c}{rgb}{0,0,0};
\colorlet{c}{kugray};
\draw [c] (5.66763,3.27119) -- (5.66763,3.30362);
\draw [c] (5.66763,3.30362) -- (5.66763,3.33296);
\draw [c] (5.65281,3.30362) -- (5.66763,3.30362);
\draw [c] (5.66763,3.30362) -- (5.68245,3.30362);
\definecolor{c}{rgb}{0,0,0};
\colorlet{c}{kugray};
\draw [c] (5.69727,3.30087) -- (5.69727,3.33324);
\draw [c] (5.69727,3.33324) -- (5.69727,3.36252);
\draw [c] (5.68245,3.33324) -- (5.69727,3.33324);
\draw [c] (5.69727,3.33324) -- (5.71209,3.33324);
\definecolor{c}{rgb}{0,0,0};
\colorlet{c}{kugray};
\draw [c] (5.7269,3.35191) -- (5.7269,3.37987);
\draw [c] (5.7269,3.37987) -- (5.7269,3.4055);
\draw [c] (5.71209,3.37987) -- (5.7269,3.37987);
\draw [c] (5.7269,3.37987) -- (5.74172,3.37987);
\definecolor{c}{rgb}{0,0,0};
\colorlet{c}{kugray};
\draw [c] (5.75654,3.32022) -- (5.75654,3.34998);
\draw [c] (5.75654,3.34998) -- (5.75654,3.37711);
\draw [c] (5.74172,3.34998) -- (5.75654,3.34998);
\draw [c] (5.75654,3.34998) -- (5.77136,3.34998);
\definecolor{c}{rgb}{0,0,0};
\colorlet{c}{kugray};
\draw [c] (5.78618,3.29616) -- (5.78618,3.32751);
\draw [c] (5.78618,3.32751) -- (5.78618,3.35595);
\draw [c] (5.77136,3.32751) -- (5.78618,3.32751);
\draw [c] (5.78618,3.32751) -- (5.80099,3.32751);
\definecolor{c}{rgb}{0,0,0};
\colorlet{c}{kugray};
\draw [c] (5.81581,3.25342) -- (5.81581,3.28535);
\draw [c] (5.81581,3.28535) -- (5.81581,3.31427);
\draw [c] (5.80099,3.28535) -- (5.81581,3.28535);
\draw [c] (5.81581,3.28535) -- (5.83063,3.28535);
\definecolor{c}{rgb}{0,0,0};
\colorlet{c}{kugray};
\draw [c] (5.84545,3.26747) -- (5.84545,3.30131);
\draw [c] (5.84545,3.30131) -- (5.84545,3.3318);
\draw [c] (5.83063,3.30131) -- (5.84545,3.30131);
\draw [c] (5.84545,3.30131) -- (5.86026,3.30131);
\definecolor{c}{rgb}{0,0,0};
\colorlet{c}{kugray};
\draw [c] (5.87508,3.27773) -- (5.87508,3.31106);
\draw [c] (5.87508,3.31106) -- (5.87508,3.34112);
\draw [c] (5.86026,3.31106) -- (5.87508,3.31106);
\draw [c] (5.87508,3.31106) -- (5.8899,3.31106);
\definecolor{c}{rgb}{0,0,0};
\colorlet{c}{kugray};
\draw [c] (5.90472,3.33645) -- (5.90472,3.36537);
\draw [c] (5.90472,3.36537) -- (5.90472,3.3918);
\draw [c] (5.8899,3.36537) -- (5.90472,3.36537);
\draw [c] (5.90472,3.36537) -- (5.91954,3.36537);
\definecolor{c}{rgb}{0,0,0};
\colorlet{c}{kugray};
\draw [c] (5.93435,3.324) -- (5.93435,3.35624);
\draw [c] (5.93435,3.35624) -- (5.93435,3.38541);
\draw [c] (5.91954,3.35624) -- (5.93435,3.35624);
\draw [c] (5.93435,3.35624) -- (5.94917,3.35624);
\definecolor{c}{rgb}{0,0,0};
\colorlet{c}{kugray};
\draw [c] (5.96399,3.26085) -- (5.96399,3.29364);
\draw [c] (5.96399,3.29364) -- (5.96399,3.32327);
\draw [c] (5.94917,3.29364) -- (5.96399,3.29364);
\draw [c] (5.96399,3.29364) -- (5.97881,3.29364);
\definecolor{c}{rgb}{0,0,0};
\colorlet{c}{kugray};
\draw [c] (5.99363,3.26893) -- (5.99363,3.30106);
\draw [c] (5.99363,3.30106) -- (5.99363,3.33016);
\draw [c] (5.97881,3.30106) -- (5.99363,3.30106);
\draw [c] (5.99363,3.30106) -- (6.00844,3.30106);
\definecolor{c}{rgb}{0,0,0};
\colorlet{c}{kugray};
\draw [c] (6.02326,3.32056) -- (6.02326,3.35257);
\draw [c] (6.02326,3.35257) -- (6.02326,3.38156);
\draw [c] (6.00844,3.35257) -- (6.02326,3.35257);
\draw [c] (6.02326,3.35257) -- (6.03808,3.35257);
\definecolor{c}{rgb}{0,0,0};
\colorlet{c}{kugray};
\draw [c] (6.0529,3.27284) -- (6.0529,3.30377);
\draw [c] (6.0529,3.30377) -- (6.0529,3.33188);
\draw [c] (6.03808,3.30377) -- (6.0529,3.30377);
\draw [c] (6.0529,3.30377) -- (6.06772,3.30377);
\definecolor{c}{rgb}{0,0,0};
\colorlet{c}{kugray};
\draw [c] (6.08253,3.28251) -- (6.08253,3.31642);
\draw [c] (6.08253,3.31642) -- (6.08253,3.34697);
\draw [c] (6.06772,3.31642) -- (6.08253,3.31642);
\draw [c] (6.08253,3.31642) -- (6.09735,3.31642);
\definecolor{c}{rgb}{0,0,0};
\colorlet{c}{kugray};
\draw [c] (6.11217,3.31512) -- (6.11217,3.34443);
\draw [c] (6.11217,3.34443) -- (6.11217,3.3712);
\draw [c] (6.09735,3.34443) -- (6.11217,3.34443);
\draw [c] (6.11217,3.34443) -- (6.12699,3.34443);
\definecolor{c}{rgb}{0,0,0};
\colorlet{c}{kugray};
\draw [c] (6.1418,3.24222) -- (6.1418,3.27469);
\draw [c] (6.1418,3.27469) -- (6.1418,3.30406);
\draw [c] (6.12699,3.27469) -- (6.1418,3.27469);
\draw [c] (6.1418,3.27469) -- (6.15662,3.27469);
\definecolor{c}{rgb}{0,0,0};
\colorlet{c}{kugray};
\draw [c] (6.17144,3.19853) -- (6.17144,3.23317);
\draw [c] (6.17144,3.23317) -- (6.17144,3.26431);
\draw [c] (6.15662,3.23317) -- (6.17144,3.23317);
\draw [c] (6.17144,3.23317) -- (6.18626,3.23317);
\definecolor{c}{rgb}{0,0,0};
\colorlet{c}{kugray};
\draw [c] (6.20108,3.26243) -- (6.20108,3.29519);
\draw [c] (6.20108,3.29519) -- (6.20108,3.32479);
\draw [c] (6.18626,3.29519) -- (6.20108,3.29519);
\draw [c] (6.20108,3.29519) -- (6.21589,3.29519);
\definecolor{c}{rgb}{0,0,0};
\colorlet{c}{kugray};
\draw [c] (6.23071,3.26744) -- (6.23071,3.30118);
\draw [c] (6.23071,3.30118) -- (6.23071,3.33159);
\draw [c] (6.21589,3.30118) -- (6.23071,3.30118);
\draw [c] (6.23071,3.30118) -- (6.24553,3.30118);
\definecolor{c}{rgb}{0,0,0};
\colorlet{c}{kugray};
\draw [c] (6.26035,3.29406) -- (6.26035,3.32675);
\draw [c] (6.26035,3.32675) -- (6.26035,3.35629);
\draw [c] (6.24553,3.32675) -- (6.26035,3.32675);
\draw [c] (6.26035,3.32675) -- (6.27517,3.32675);
\definecolor{c}{rgb}{0,0,0};
\colorlet{c}{kugray};
\draw [c] (6.28998,3.23898) -- (6.28998,3.27334);
\draw [c] (6.28998,3.27334) -- (6.28998,3.30426);
\draw [c] (6.27517,3.27334) -- (6.28998,3.27334);
\draw [c] (6.28998,3.27334) -- (6.3048,3.27334);
\definecolor{c}{rgb}{0,0,0};
\colorlet{c}{kugray};
\draw [c] (6.31962,3.22624) -- (6.31962,3.26036);
\draw [c] (6.31962,3.26036) -- (6.31962,3.29107);
\draw [c] (6.3048,3.26036) -- (6.31962,3.26036);
\draw [c] (6.31962,3.26036) -- (6.33444,3.26036);
\definecolor{c}{rgb}{0,0,0};
\colorlet{c}{kugray};
\draw [c] (6.34926,3.26468) -- (6.34926,3.29684);
\draw [c] (6.34926,3.29684) -- (6.34926,3.32595);
\draw [c] (6.33444,3.29684) -- (6.34926,3.29684);
\draw [c] (6.34926,3.29684) -- (6.36407,3.29684);
\definecolor{c}{rgb}{0,0,0};
\colorlet{c}{kugray};
\draw [c] (6.37889,3.30966) -- (6.37889,3.34032);
\draw [c] (6.37889,3.34032) -- (6.37889,3.36821);
\draw [c] (6.36407,3.34032) -- (6.37889,3.34032);
\draw [c] (6.37889,3.34032) -- (6.39371,3.34032);
\definecolor{c}{rgb}{0,0,0};
\colorlet{c}{kugray};
\draw [c] (6.40853,3.23431) -- (6.40853,3.26778);
\draw [c] (6.40853,3.26778) -- (6.40853,3.29797);
\draw [c] (6.39371,3.26778) -- (6.40853,3.26778);
\draw [c] (6.40853,3.26778) -- (6.42334,3.26778);
\definecolor{c}{rgb}{0,0,0};
\colorlet{c}{kugray};
\draw [c] (6.43816,3.26143) -- (6.43816,3.29357);
\draw [c] (6.43816,3.29357) -- (6.43816,3.32268);
\draw [c] (6.42334,3.29357) -- (6.43816,3.29357);
\draw [c] (6.43816,3.29357) -- (6.45298,3.29357);
\definecolor{c}{rgb}{0,0,0};
\colorlet{c}{kugray};
\draw [c] (6.4678,3.21815) -- (6.4678,3.25281);
\draw [c] (6.4678,3.25281) -- (6.4678,3.28397);
\draw [c] (6.45298,3.25281) -- (6.4678,3.25281);
\draw [c] (6.4678,3.25281) -- (6.48262,3.25281);
\definecolor{c}{rgb}{0,0,0};
\colorlet{c}{kugray};
\draw [c] (6.49743,3.25782) -- (6.49743,3.29048);
\draw [c] (6.49743,3.29048) -- (6.49743,3.32);
\draw [c] (6.48262,3.29048) -- (6.49743,3.29048);
\draw [c] (6.49743,3.29048) -- (6.51225,3.29048);
\definecolor{c}{rgb}{0,0,0};
\colorlet{c}{kugray};
\draw [c] (6.52707,3.29629) -- (6.52707,3.32741);
\draw [c] (6.52707,3.32741) -- (6.52707,3.35568);
\draw [c] (6.51225,3.32741) -- (6.52707,3.32741);
\draw [c] (6.52707,3.32741) -- (6.54189,3.32741);
\definecolor{c}{rgb}{0,0,0};
\colorlet{c}{kugray};
\draw [c] (6.55671,3.28852) -- (6.55671,3.32075);
\draw [c] (6.55671,3.32075) -- (6.55671,3.34993);
\draw [c] (6.54189,3.32075) -- (6.55671,3.32075);
\draw [c] (6.55671,3.32075) -- (6.57152,3.32075);
\definecolor{c}{rgb}{0,0,0};
\colorlet{c}{kugray};
\draw [c] (6.58634,3.11775) -- (6.58634,3.15987);
\draw [c] (6.58634,3.15987) -- (6.58634,3.19692);
\draw [c] (6.57152,3.15987) -- (6.58634,3.15987);
\draw [c] (6.58634,3.15987) -- (6.60116,3.15987);
\definecolor{c}{rgb}{0,0,0};
\colorlet{c}{kugray};
\draw [c] (6.61598,3.23027) -- (6.61598,3.2646);
\draw [c] (6.61598,3.2646) -- (6.61598,3.29549);
\draw [c] (6.60116,3.2646) -- (6.61598,3.2646);
\draw [c] (6.61598,3.2646) -- (6.63079,3.2646);
\definecolor{c}{rgb}{0,0,0};
\colorlet{c}{kugray};
\draw [c] (6.64561,3.24953) -- (6.64561,3.28391);
\draw [c] (6.64561,3.28391) -- (6.64561,3.31483);
\draw [c] (6.63079,3.28391) -- (6.64561,3.28391);
\draw [c] (6.64561,3.28391) -- (6.66043,3.28391);
\definecolor{c}{rgb}{0,0,0};
\colorlet{c}{kugray};
\draw [c] (6.67525,3.22925) -- (6.67525,3.26504);
\draw [c] (6.67525,3.26504) -- (6.67525,3.2971);
\draw [c] (6.66043,3.26504) -- (6.67525,3.26504);
\draw [c] (6.67525,3.26504) -- (6.69007,3.26504);
\definecolor{c}{rgb}{0,0,0};
\colorlet{c}{kugray};
\draw [c] (6.70488,3.27333) -- (6.70488,3.30459);
\draw [c] (6.70488,3.30459) -- (6.70488,3.33297);
\draw [c] (6.69007,3.30459) -- (6.70488,3.30459);
\draw [c] (6.70488,3.30459) -- (6.7197,3.30459);
\definecolor{c}{rgb}{0,0,0};
\colorlet{c}{kugray};
\draw [c] (6.73452,3.2188) -- (6.73452,3.25383);
\draw [c] (6.73452,3.25383) -- (6.73452,3.28527);
\draw [c] (6.7197,3.25383) -- (6.73452,3.25383);
\draw [c] (6.73452,3.25383) -- (6.74934,3.25383);
\definecolor{c}{rgb}{0,0,0};
\colorlet{c}{kugray};
\draw [c] (6.76416,3.19407) -- (6.76416,3.23035);
\draw [c] (6.76416,3.23035) -- (6.76416,3.26281);
\draw [c] (6.74934,3.23035) -- (6.76416,3.23035);
\draw [c] (6.76416,3.23035) -- (6.77897,3.23035);
\definecolor{c}{rgb}{0,0,0};
\colorlet{c}{kugray};
\draw [c] (6.79379,3.2118) -- (6.79379,3.24685);
\draw [c] (6.79379,3.24685) -- (6.79379,3.2783);
\draw [c] (6.77897,3.24685) -- (6.79379,3.24685);
\draw [c] (6.79379,3.24685) -- (6.80861,3.24685);
\definecolor{c}{rgb}{0,0,0};
\colorlet{c}{kugray};
\draw [c] (6.82343,3.20129) -- (6.82343,3.23637);
\draw [c] (6.82343,3.23637) -- (6.82343,3.26785);
\draw [c] (6.80861,3.23637) -- (6.82343,3.23637);
\draw [c] (6.82343,3.23637) -- (6.83824,3.23637);
\definecolor{c}{rgb}{0,0,0};
\colorlet{c}{kugray};
\draw [c] (6.85306,3.27615) -- (6.85306,3.30874);
\draw [c] (6.85306,3.30874) -- (6.85306,3.33821);
\draw [c] (6.83824,3.30874) -- (6.85306,3.30874);
\draw [c] (6.85306,3.30874) -- (6.86788,3.30874);
\definecolor{c}{rgb}{0,0,0};
\colorlet{c}{kugray};
\draw [c] (6.8827,3.28993) -- (6.8827,3.32159);
\draw [c] (6.8827,3.32159) -- (6.8827,3.3503);
\draw [c] (6.86788,3.32159) -- (6.8827,3.32159);
\draw [c] (6.8827,3.32159) -- (6.89752,3.32159);
\definecolor{c}{rgb}{0,0,0};
\colorlet{c}{kugray};
\draw [c] (6.91233,3.22502) -- (6.91233,3.25976);
\draw [c] (6.91233,3.25976) -- (6.91233,3.29097);
\draw [c] (6.89752,3.25976) -- (6.91233,3.25976);
\draw [c] (6.91233,3.25976) -- (6.92715,3.25976);
\definecolor{c}{rgb}{0,0,0};
\colorlet{c}{kugray};
\draw [c] (6.94197,3.27047) -- (6.94197,3.30221);
\draw [c] (6.94197,3.30221) -- (6.94197,3.33098);
\draw [c] (6.92715,3.30221) -- (6.94197,3.30221);
\draw [c] (6.94197,3.30221) -- (6.95679,3.30221);
\definecolor{c}{rgb}{0,0,0};
\colorlet{c}{kugray};
\draw [c] (6.97161,3.23938) -- (6.97161,3.27409);
\draw [c] (6.97161,3.27409) -- (6.97161,3.30529);
\draw [c] (6.95679,3.27409) -- (6.97161,3.27409);
\draw [c] (6.97161,3.27409) -- (6.98642,3.27409);
\definecolor{c}{rgb}{0,0,0};
\colorlet{c}{kugray};
\draw [c] (7.00124,3.2166) -- (7.00124,3.25185);
\draw [c] (7.00124,3.25185) -- (7.00124,3.28348);
\draw [c] (6.98642,3.25185) -- (7.00124,3.25185);
\draw [c] (7.00124,3.25185) -- (7.01606,3.25185);
\definecolor{c}{rgb}{0,0,0};
\colorlet{c}{kugray};
\draw [c] (7.03088,3.27665) -- (7.03088,3.3089);
\draw [c] (7.03088,3.3089) -- (7.03088,3.33809);
\draw [c] (7.01606,3.3089) -- (7.03088,3.3089);
\draw [c] (7.03088,3.3089) -- (7.0457,3.3089);
\definecolor{c}{rgb}{0,0,0};
\colorlet{c}{kugray};
\draw [c] (7.06051,3.14522) -- (7.06051,3.18511);
\draw [c] (7.06051,3.18511) -- (7.06051,3.22041);
\draw [c] (7.0457,3.18511) -- (7.06051,3.18511);
\draw [c] (7.06051,3.18511) -- (7.07533,3.18511);
\definecolor{c}{rgb}{0,0,0};
\colorlet{c}{kugray};
\draw [c] (7.09015,3.25803) -- (7.09015,3.29279);
\draw [c] (7.09015,3.29279) -- (7.09015,3.32402);
\draw [c] (7.07533,3.29279) -- (7.09015,3.29279);
\draw [c] (7.09015,3.29279) -- (7.10497,3.29279);
\definecolor{c}{rgb}{0,0,0};
\colorlet{c}{kugray};
\draw [c] (7.11978,3.1251) -- (7.11978,3.16549);
\draw [c] (7.11978,3.16549) -- (7.11978,3.2012);
\draw [c] (7.10497,3.16549) -- (7.11978,3.16549);
\draw [c] (7.11978,3.16549) -- (7.1346,3.16549);
\definecolor{c}{rgb}{0,0,0};
\colorlet{c}{kugray};
\draw [c] (7.14942,3.20086) -- (7.14942,3.23706);
\draw [c] (7.14942,3.23706) -- (7.14942,3.26946);
\draw [c] (7.1346,3.23706) -- (7.14942,3.23706);
\draw [c] (7.14942,3.23706) -- (7.16424,3.23706);
\definecolor{c}{rgb}{0,0,0};
\colorlet{c}{kugray};
\draw [c] (7.17906,3.23931) -- (7.17906,3.27217);
\draw [c] (7.17906,3.27217) -- (7.17906,3.30187);
\draw [c] (7.16424,3.27217) -- (7.17906,3.27217);
\draw [c] (7.17906,3.27217) -- (7.19387,3.27217);
\definecolor{c}{rgb}{0,0,0};
\colorlet{c}{kugray};
\draw [c] (7.20869,3.22963) -- (7.20869,3.26378);
\draw [c] (7.20869,3.26378) -- (7.20869,3.29452);
\draw [c] (7.19387,3.26378) -- (7.20869,3.26378);
\draw [c] (7.20869,3.26378) -- (7.22351,3.26378);
\definecolor{c}{rgb}{0,0,0};
\colorlet{c}{kugray};
\draw [c] (7.23833,3.17041) -- (7.23833,3.20638);
\draw [c] (7.23833,3.20638) -- (7.23833,3.23859);
\draw [c] (7.22351,3.20638) -- (7.23833,3.20638);
\draw [c] (7.23833,3.20638) -- (7.25315,3.20638);
\definecolor{c}{rgb}{0,0,0};
\colorlet{c}{kugray};
\draw [c] (7.26796,3.20336) -- (7.26796,3.2391);
\draw [c] (7.26796,3.2391) -- (7.26796,3.27111);
\draw [c] (7.25315,3.2391) -- (7.26796,3.2391);
\draw [c] (7.26796,3.2391) -- (7.28278,3.2391);
\definecolor{c}{rgb}{0,0,0};
\colorlet{c}{kugray};
\draw [c] (7.2976,3.13789) -- (7.2976,3.17615);
\draw [c] (7.2976,3.17615) -- (7.2976,3.21018);
\draw [c] (7.28278,3.17615) -- (7.2976,3.17615);
\draw [c] (7.2976,3.17615) -- (7.31242,3.17615);
\definecolor{c}{rgb}{0,0,0};
\colorlet{c}{kugray};
\draw [c] (7.32724,3.18107) -- (7.32724,3.21995);
\draw [c] (7.32724,3.21995) -- (7.32724,3.25447);
\draw [c] (7.31242,3.21995) -- (7.32724,3.21995);
\draw [c] (7.32724,3.21995) -- (7.34205,3.21995);
\definecolor{c}{rgb}{0,0,0};
\colorlet{c}{kugray};
\draw [c] (7.35687,3.17771) -- (7.35687,3.214);
\draw [c] (7.35687,3.214) -- (7.35687,3.24646);
\draw [c] (7.34205,3.214) -- (7.35687,3.214);
\draw [c] (7.35687,3.214) -- (7.37169,3.214);
\definecolor{c}{rgb}{0,0,0};
\colorlet{c}{kugray};
\draw [c] (7.38651,3.20963) -- (7.38651,3.24614);
\draw [c] (7.38651,3.24614) -- (7.38651,3.27878);
\draw [c] (7.37169,3.24614) -- (7.38651,3.24614);
\draw [c] (7.38651,3.24614) -- (7.40132,3.24614);
\definecolor{c}{rgb}{0,0,0};
\colorlet{c}{kugray};
\draw [c] (7.41614,3.23434) -- (7.41614,3.26861);
\draw [c] (7.41614,3.26861) -- (7.41614,3.29944);
\draw [c] (7.40132,3.26861) -- (7.41614,3.26861);
\draw [c] (7.41614,3.26861) -- (7.43096,3.26861);
\definecolor{c}{rgb}{0,0,0};
\colorlet{c}{kugray};
\draw [c] (7.44578,3.26607) -- (7.44578,3.29811);
\draw [c] (7.44578,3.29811) -- (7.44578,3.32713);
\draw [c] (7.43096,3.29811) -- (7.44578,3.29811);
\draw [c] (7.44578,3.29811) -- (7.4606,3.29811);
\definecolor{c}{rgb}{0,0,0};
\colorlet{c}{kugray};
\draw [c] (7.47541,3.23088) -- (7.47541,3.26608);
\draw [c] (7.47541,3.26608) -- (7.47541,3.29768);
\draw [c] (7.4606,3.26608) -- (7.47541,3.26608);
\draw [c] (7.47541,3.26608) -- (7.49023,3.26608);
\definecolor{c}{rgb}{0,0,0};
\colorlet{c}{kugray};
\draw [c] (7.50505,3.19492) -- (7.50505,3.23154);
\draw [c] (7.50505,3.23154) -- (7.50505,3.26427);
\draw [c] (7.49023,3.23154) -- (7.50505,3.23154);
\draw [c] (7.50505,3.23154) -- (7.51987,3.23154);
\definecolor{c}{rgb}{0,0,0};
\colorlet{c}{kugray};
\draw [c] (7.53469,3.16974) -- (7.53469,3.20818);
\draw [c] (7.53469,3.20818) -- (7.53469,3.24234);
\draw [c] (7.51987,3.20818) -- (7.53469,3.20818);
\draw [c] (7.53469,3.20818) -- (7.5495,3.20818);
\definecolor{c}{rgb}{0,0,0};
\colorlet{c}{kugray};
\draw [c] (7.56432,3.23775) -- (7.56432,3.27182);
\draw [c] (7.56432,3.27182) -- (7.56432,3.3025);
\draw [c] (7.5495,3.27182) -- (7.56432,3.27182);
\draw [c] (7.56432,3.27182) -- (7.57914,3.27182);
\definecolor{c}{rgb}{0,0,0};
\colorlet{c}{kugray};
\draw [c] (7.59396,3.27348) -- (7.59396,3.30657);
\draw [c] (7.59396,3.30657) -- (7.59396,3.33644);
\draw [c] (7.57914,3.30657) -- (7.59396,3.30657);
\draw [c] (7.59396,3.30657) -- (7.60877,3.30657);
\definecolor{c}{rgb}{0,0,0};
\colorlet{c}{kugray};
\draw [c] (7.62359,3.20408) -- (7.62359,3.24244);
\draw [c] (7.62359,3.24244) -- (7.62359,3.27655);
\draw [c] (7.60877,3.24244) -- (7.62359,3.24244);
\draw [c] (7.62359,3.24244) -- (7.63841,3.24244);
\definecolor{c}{rgb}{0,0,0};
\colorlet{c}{kugray};
\draw [c] (7.65323,3.20046) -- (7.65323,3.23614);
\draw [c] (7.65323,3.23614) -- (7.65323,3.26812);
\draw [c] (7.63841,3.23614) -- (7.65323,3.23614);
\draw [c] (7.65323,3.23614) -- (7.66805,3.23614);
\definecolor{c}{rgb}{0,0,0};
\colorlet{c}{kugray};
\draw [c] (7.68286,3.11491) -- (7.68286,3.15626);
\draw [c] (7.68286,3.15626) -- (7.68286,3.19271);
\draw [c] (7.66805,3.15626) -- (7.68286,3.15626);
\draw [c] (7.68286,3.15626) -- (7.69768,3.15626);
\definecolor{c}{rgb}{0,0,0};
\colorlet{c}{kugray};
\draw [c] (7.7125,3.20878) -- (7.7125,3.24334);
\draw [c] (7.7125,3.24334) -- (7.7125,3.2744);
\draw [c] (7.69768,3.24334) -- (7.7125,3.24334);
\draw [c] (7.7125,3.24334) -- (7.72732,3.24334);
\definecolor{c}{rgb}{0,0,0};
\colorlet{c}{kugray};
\draw [c] (7.74214,3.18668) -- (7.74214,3.22275);
\draw [c] (7.74214,3.22275) -- (7.74214,3.25502);
\draw [c] (7.72732,3.22275) -- (7.74214,3.22275);
\draw [c] (7.74214,3.22275) -- (7.75695,3.22275);
\definecolor{c}{rgb}{0,0,0};
\colorlet{c}{kugray};
\draw [c] (7.77177,3.17887) -- (7.77177,3.21749);
\draw [c] (7.77177,3.21749) -- (7.77177,3.2518);
\draw [c] (7.75695,3.21749) -- (7.77177,3.21749);
\draw [c] (7.77177,3.21749) -- (7.78659,3.21749);
\definecolor{c}{rgb}{0,0,0};
\colorlet{c}{kugray};
\draw [c] (7.80141,3.1287) -- (7.80141,3.1705);
\draw [c] (7.80141,3.1705) -- (7.80141,3.2073);
\draw [c] (7.78659,3.1705) -- (7.80141,3.1705);
\draw [c] (7.80141,3.1705) -- (7.81623,3.1705);
\definecolor{c}{rgb}{0,0,0};
\colorlet{c}{kugray};
\draw [c] (7.83104,3.19387) -- (7.83104,3.23296);
\draw [c] (7.83104,3.23296) -- (7.83104,3.26765);
\draw [c] (7.81623,3.23296) -- (7.83104,3.23296);
\draw [c] (7.83104,3.23296) -- (7.84586,3.23296);
\definecolor{c}{rgb}{0,0,0};
\colorlet{c}{kugray};
\draw [c] (7.86068,3.19817) -- (7.86068,3.23466);
\draw [c] (7.86068,3.23466) -- (7.86068,3.26728);
\draw [c] (7.84586,3.23466) -- (7.86068,3.23466);
\draw [c] (7.86068,3.23466) -- (7.8755,3.23466);
\definecolor{c}{rgb}{0,0,0};
\colorlet{c}{kugray};
\draw [c] (7.89031,3.19947) -- (7.89031,3.23529);
\draw [c] (7.89031,3.23529) -- (7.89031,3.26737);
\draw [c] (7.8755,3.23529) -- (7.89031,3.23529);
\draw [c] (7.89031,3.23529) -- (7.90513,3.23529);
\definecolor{c}{rgb}{0,0,0};
\colorlet{c}{kugray};
\draw [c] (7.91995,3.17289) -- (7.91995,3.21165);
\draw [c] (7.91995,3.21165) -- (7.91995,3.24608);
\draw [c] (7.90513,3.21165) -- (7.91995,3.21165);
\draw [c] (7.91995,3.21165) -- (7.93477,3.21165);
\definecolor{c}{rgb}{0,0,0};
\colorlet{c}{kugray};
\draw [c] (7.94959,3.16056) -- (7.94959,3.20019);
\draw [c] (7.94959,3.20019) -- (7.94959,3.23529);
\draw [c] (7.93477,3.20019) -- (7.94959,3.20019);
\draw [c] (7.94959,3.20019) -- (7.9644,3.20019);
\definecolor{c}{rgb}{0,0,0};
\colorlet{c}{kugray};
\draw [c] (7.97922,3.17291) -- (7.97922,3.2128);
\draw [c] (7.97922,3.2128) -- (7.97922,3.2481);
\draw [c] (7.9644,3.2128) -- (7.97922,3.2128);
\draw [c] (7.97922,3.2128) -- (7.99404,3.2128);
\definecolor{c}{rgb}{0,0,0};
\colorlet{c}{kugray};
\draw [c] (8.00886,3.21345) -- (8.00886,3.24811);
\draw [c] (8.00886,3.24811) -- (8.00886,3.27926);
\draw [c] (7.99404,3.24811) -- (8.00886,3.24811);
\draw [c] (8.00886,3.24811) -- (8.02368,3.24811);
\definecolor{c}{rgb}{0,0,0};
\colorlet{c}{kugray};
\draw [c] (8.03849,3.16384) -- (8.03849,3.20119);
\draw [c] (8.03849,3.20119) -- (8.03849,3.2345);
\draw [c] (8.02368,3.20119) -- (8.03849,3.20119);
\draw [c] (8.03849,3.20119) -- (8.05331,3.20119);
\definecolor{c}{rgb}{0,0,0};
\colorlet{c}{kugray};
\draw [c] (8.06813,3.2368) -- (8.06813,3.27102);
\draw [c] (8.06813,3.27102) -- (8.06813,3.30182);
\draw [c] (8.05331,3.27102) -- (8.06813,3.27102);
\draw [c] (8.06813,3.27102) -- (8.08295,3.27102);
\definecolor{c}{rgb}{0,0,0};
\colorlet{c}{kugray};
\draw [c] (8.09776,3.20237) -- (8.09776,3.23814);
\draw [c] (8.09776,3.23814) -- (8.09776,3.27017);
\draw [c] (8.08295,3.23814) -- (8.09776,3.23814);
\draw [c] (8.09776,3.23814) -- (8.11258,3.23814);
\definecolor{c}{rgb}{0,0,0};
\colorlet{c}{kugray};
\draw [c] (8.1274,3.16214) -- (8.1274,3.20061);
\draw [c] (8.1274,3.20061) -- (8.1274,3.2348);
\draw [c] (8.11258,3.20061) -- (8.1274,3.20061);
\draw [c] (8.1274,3.20061) -- (8.14222,3.20061);
\definecolor{c}{rgb}{0,0,0};
\colorlet{c}{kugray};
\draw [c] (8.15704,3.17943) -- (8.15704,3.21673);
\draw [c] (8.15704,3.21673) -- (8.15704,3.24999);
\draw [c] (8.14222,3.21673) -- (8.15704,3.21673);
\draw [c] (8.15704,3.21673) -- (8.17185,3.21673);
\definecolor{c}{rgb}{0,0,0};
\colorlet{c}{kugray};
\draw [c] (8.18667,3.2077) -- (8.18667,3.24191);
\draw [c] (8.18667,3.24191) -- (8.18667,3.2727);
\draw [c] (8.17185,3.24191) -- (8.18667,3.24191);
\draw [c] (8.18667,3.24191) -- (8.20149,3.24191);
\definecolor{c}{rgb}{0,0,0};
\colorlet{c}{kugray};
\draw [c] (8.21631,3.13546) -- (8.21631,3.17622);
\draw [c] (8.21631,3.17622) -- (8.21631,3.21221);
\draw [c] (8.20149,3.17622) -- (8.21631,3.17622);
\draw [c] (8.21631,3.17622) -- (8.23113,3.17622);
\definecolor{c}{rgb}{0,0,0};
\colorlet{c}{kugray};
\draw [c] (8.24594,3.14151) -- (8.24594,3.18151);
\draw [c] (8.24594,3.18151) -- (8.24594,3.2169);
\draw [c] (8.23113,3.18151) -- (8.24594,3.18151);
\draw [c] (8.24594,3.18151) -- (8.26076,3.18151);
\definecolor{c}{rgb}{0,0,0};
\colorlet{c}{kugray};
\draw [c] (8.27558,3.17226) -- (8.27558,3.21036);
\draw [c] (8.27558,3.21036) -- (8.27558,3.24426);
\draw [c] (8.26076,3.21036) -- (8.27558,3.21036);
\draw [c] (8.27558,3.21036) -- (8.2904,3.21036);
\definecolor{c}{rgb}{0,0,0};
\colorlet{c}{kugray};
\draw [c] (8.30521,3.21664) -- (8.30521,3.25163);
\draw [c] (8.30521,3.25163) -- (8.30521,3.28304);
\draw [c] (8.2904,3.25163) -- (8.30521,3.25163);
\draw [c] (8.30521,3.25163) -- (8.32003,3.25163);
\definecolor{c}{rgb}{0,0,0};
\colorlet{c}{kugray};
\draw [c] (8.33485,3.18633) -- (8.33485,3.22273);
\draw [c] (8.33485,3.22273) -- (8.33485,3.25528);
\draw [c] (8.32003,3.22273) -- (8.33485,3.22273);
\draw [c] (8.33485,3.22273) -- (8.34967,3.22273);
\definecolor{c}{rgb}{0,0,0};
\colorlet{c}{kugray};
\draw [c] (8.36449,3.10352) -- (8.36449,3.14557);
\draw [c] (8.36449,3.14557) -- (8.36449,3.18257);
\draw [c] (8.34967,3.14557) -- (8.36449,3.14557);
\draw [c] (8.36449,3.14557) -- (8.3793,3.14557);
\definecolor{c}{rgb}{0,0,0};
\colorlet{c}{kugray};
\draw [c] (8.39412,3.14212) -- (8.39412,3.18236);
\draw [c] (8.39412,3.18236) -- (8.39412,3.21794);
\draw [c] (8.3793,3.18236) -- (8.39412,3.18236);
\draw [c] (8.39412,3.18236) -- (8.40894,3.18236);
\definecolor{c}{rgb}{0,0,0};
\colorlet{c}{kugray};
\draw [c] (8.42376,3.14268) -- (8.42376,3.18228);
\draw [c] (8.42376,3.18228) -- (8.42376,3.21737);
\draw [c] (8.40894,3.18228) -- (8.42376,3.18228);
\draw [c] (8.42376,3.18228) -- (8.43858,3.18228);
\definecolor{c}{rgb}{0,0,0};
\colorlet{c}{kugray};
\draw [c] (8.45339,3.17027) -- (8.45339,3.21011);
\draw [c] (8.45339,3.21011) -- (8.45339,3.24537);
\draw [c] (8.43858,3.21011) -- (8.45339,3.21011);
\draw [c] (8.45339,3.21011) -- (8.46821,3.21011);
\definecolor{c}{rgb}{0,0,0};
\colorlet{c}{kugray};
\draw [c] (8.48303,3.19033) -- (8.48303,3.22695);
\draw [c] (8.48303,3.22695) -- (8.48303,3.25968);
\draw [c] (8.46821,3.22695) -- (8.48303,3.22695);
\draw [c] (8.48303,3.22695) -- (8.49785,3.22695);
\definecolor{c}{rgb}{0,0,0};
\colorlet{c}{kugray};
\draw [c] (8.51267,3.1697) -- (8.51267,3.20663);
\draw [c] (8.51267,3.20663) -- (8.51267,3.2396);
\draw [c] (8.49785,3.20663) -- (8.51267,3.20663);
\draw [c] (8.51267,3.20663) -- (8.52748,3.20663);
\definecolor{c}{rgb}{0,0,0};
\colorlet{c}{kugray};
\draw [c] (8.5423,3.16479) -- (8.5423,3.20274);
\draw [c] (8.5423,3.20274) -- (8.5423,3.23651);
\draw [c] (8.52748,3.20274) -- (8.5423,3.20274);
\draw [c] (8.5423,3.20274) -- (8.55712,3.20274);
\definecolor{c}{rgb}{0,0,0};
\colorlet{c}{kugray};
\draw [c] (8.57194,3.12526) -- (8.57194,3.1667);
\draw [c] (8.57194,3.1667) -- (8.57194,3.20322);
\draw [c] (8.55712,3.1667) -- (8.57194,3.1667);
\draw [c] (8.57194,3.1667) -- (8.58675,3.1667);
\definecolor{c}{rgb}{0,0,0};
\colorlet{c}{kugray};
\draw [c] (8.60157,3.15802) -- (8.60157,3.19823);
\draw [c] (8.60157,3.19823) -- (8.60157,3.23378);
\draw [c] (8.58675,3.19823) -- (8.60157,3.19823);
\draw [c] (8.60157,3.19823) -- (8.61639,3.19823);
\definecolor{c}{rgb}{0,0,0};
\colorlet{c}{kugray};
\draw [c] (8.63121,3.10131) -- (8.63121,3.14082);
\draw [c] (8.63121,3.14082) -- (8.63121,3.17583);
\draw [c] (8.61639,3.14082) -- (8.63121,3.14082);
\draw [c] (8.63121,3.14082) -- (8.64603,3.14082);
\definecolor{c}{rgb}{0,0,0};
\colorlet{c}{kugray};
\draw [c] (8.66084,3.10207) -- (8.66084,3.14645);
\draw [c] (8.66084,3.14645) -- (8.66084,3.18523);
\draw [c] (8.64603,3.14645) -- (8.66084,3.14645);
\draw [c] (8.66084,3.14645) -- (8.67566,3.14645);
\definecolor{c}{rgb}{0,0,0};
\colorlet{c}{kugray};
\draw [c] (8.69048,3.15812) -- (8.69048,3.19621);
\draw [c] (8.69048,3.19621) -- (8.69048,3.2301);
\draw [c] (8.67566,3.19621) -- (8.69048,3.19621);
\draw [c] (8.69048,3.19621) -- (8.7053,3.19621);
\definecolor{c}{rgb}{0,0,0};
\colorlet{c}{kugray};
\draw [c] (8.72012,3.0956) -- (8.72012,3.14019);
\draw [c] (8.72012,3.14019) -- (8.72012,3.17912);
\draw [c] (8.7053,3.14019) -- (8.72012,3.14019);
\draw [c] (8.72012,3.14019) -- (8.73493,3.14019);
\definecolor{c}{rgb}{0,0,0};
\colorlet{c}{kugray};
\draw [c] (8.74975,3.08632) -- (8.74975,3.12806);
\draw [c] (8.74975,3.12806) -- (8.74975,3.1648);
\draw [c] (8.73493,3.12806) -- (8.74975,3.12806);
\draw [c] (8.74975,3.12806) -- (8.76457,3.12806);
\definecolor{c}{rgb}{0,0,0};
\colorlet{c}{kugray};
\draw [c] (8.77939,3.12605) -- (8.77939,3.16744);
\draw [c] (8.77939,3.16744) -- (8.77939,3.20392);
\draw [c] (8.76457,3.16744) -- (8.77939,3.16744);
\draw [c] (8.77939,3.16744) -- (8.79421,3.16744);
\definecolor{c}{rgb}{0,0,0};
\colorlet{c}{kugray};
\draw [c] (8.80902,3.19836) -- (8.80902,3.23362);
\draw [c] (8.80902,3.23362) -- (8.80902,3.26525);
\draw [c] (8.79421,3.23362) -- (8.80902,3.23362);
\draw [c] (8.80902,3.23362) -- (8.82384,3.23362);
\definecolor{c}{rgb}{0,0,0};
\colorlet{c}{kugray};
\draw [c] (8.83866,3.1022) -- (8.83866,3.14697);
\draw [c] (8.83866,3.14697) -- (8.83866,3.18604);
\draw [c] (8.82384,3.14697) -- (8.83866,3.14697);
\draw [c] (8.83866,3.14697) -- (8.85348,3.14697);
\definecolor{c}{rgb}{0,0,0};
\colorlet{c}{kugray};
\draw [c] (8.86829,3.20731) -- (8.86829,3.2428);
\draw [c] (8.86829,3.2428) -- (8.86829,3.27462);
\draw [c] (8.85348,3.2428) -- (8.86829,3.2428);
\draw [c] (8.86829,3.2428) -- (8.88311,3.2428);
\definecolor{c}{rgb}{0,0,0};
\colorlet{c}{kugray};
\draw [c] (8.89793,3.19795) -- (8.89793,3.23912);
\draw [c] (8.89793,3.23912) -- (8.89793,3.27543);
\draw [c] (8.88311,3.23912) -- (8.89793,3.23912);
\draw [c] (8.89793,3.23912) -- (8.91275,3.23912);
\definecolor{c}{rgb}{0,0,0};
\colorlet{c}{kugray};
\draw [c] (8.92757,3.07413) -- (8.92757,3.11815);
\draw [c] (8.92757,3.11815) -- (8.92757,3.15665);
\draw [c] (8.91275,3.11815) -- (8.92757,3.11815);
\draw [c] (8.92757,3.11815) -- (8.94238,3.11815);
\definecolor{c}{rgb}{0,0,0};
\colorlet{c}{kugray};
\draw [c] (8.9572,3.05408) -- (8.9572,3.10088);
\draw [c] (8.9572,3.10088) -- (8.9572,3.14149);
\draw [c] (8.94238,3.10088) -- (8.9572,3.10088);
\draw [c] (8.9572,3.10088) -- (8.97202,3.10088);
\definecolor{c}{rgb}{0,0,0};
\colorlet{c}{kugray};
\draw [c] (8.98684,3.16696) -- (8.98684,3.20827);
\draw [c] (8.98684,3.20827) -- (8.98684,3.24469);
\draw [c] (8.97202,3.20827) -- (8.98684,3.20827);
\draw [c] (8.98684,3.20827) -- (9.00166,3.20827);
\definecolor{c}{rgb}{0,0,0};
\colorlet{c}{kugray};
\draw [c] (9.01647,3.06066) -- (9.01647,3.10415);
\draw [c] (9.01647,3.10415) -- (9.01647,3.14225);
\draw [c] (9.00166,3.10415) -- (9.01647,3.10415);
\draw [c] (9.01647,3.10415) -- (9.03129,3.10415);
\definecolor{c}{rgb}{0,0,0};
\colorlet{c}{kugray};
\draw [c] (9.04611,3.03321) -- (9.04611,3.07793);
\draw [c] (9.04611,3.07793) -- (9.04611,3.11697);
\draw [c] (9.03129,3.07793) -- (9.04611,3.07793);
\draw [c] (9.04611,3.07793) -- (9.06093,3.07793);
\definecolor{c}{rgb}{0,0,0};
\colorlet{c}{kugray};
\draw [c] (9.07574,3.01544) -- (9.07574,3.0622);
\draw [c] (9.07574,3.0622) -- (9.07574,3.10279);
\draw [c] (9.06093,3.0622) -- (9.07574,3.0622);
\draw [c] (9.07574,3.0622) -- (9.09056,3.0622);
\definecolor{c}{rgb}{0,0,0};
\colorlet{c}{kugray};
\draw [c] (9.10538,3.11751) -- (9.10538,3.16063);
\draw [c] (9.10538,3.16063) -- (9.10538,3.19845);
\draw [c] (9.09056,3.16063) -- (9.10538,3.16063);
\draw [c] (9.10538,3.16063) -- (9.1202,3.16063);
\definecolor{c}{rgb}{0,0,0};
\colorlet{c}{kugray};
\draw [c] (9.13502,3.05438) -- (9.13502,3.10259);
\draw [c] (9.13502,3.10259) -- (9.13502,3.14427);
\draw [c] (9.1202,3.10259) -- (9.13502,3.10259);
\draw [c] (9.13502,3.10259) -- (9.14983,3.10259);
\definecolor{c}{rgb}{0,0,0};
\colorlet{c}{kugray};
\draw [c] (9.16465,3.01975) -- (9.16465,3.07036);
\draw [c] (9.16465,3.07036) -- (9.16465,3.11381);
\draw [c] (9.14983,3.07036) -- (9.16465,3.07036);
\draw [c] (9.16465,3.07036) -- (9.17947,3.07036);
\definecolor{c}{rgb}{0,0,0};
\colorlet{c}{kugray};
\draw [c] (9.19429,3.09316) -- (9.19429,3.13551);
\draw [c] (9.19429,3.13551) -- (9.19429,3.17274);
\draw [c] (9.17947,3.13551) -- (9.19429,3.13551);
\draw [c] (9.19429,3.13551) -- (9.20911,3.13551);
\definecolor{c}{rgb}{0,0,0};
\colorlet{c}{kugray};
\draw [c] (9.22392,3.06278) -- (9.22392,3.109);
\draw [c] (9.22392,3.109) -- (9.22392,3.14918);
\draw [c] (9.20911,3.109) -- (9.22392,3.109);
\draw [c] (9.22392,3.109) -- (9.23874,3.109);
\definecolor{c}{rgb}{0,0,0};
\colorlet{c}{kugray};
\draw [c] (9.25356,3.08447) -- (9.25356,3.12834);
\draw [c] (9.25356,3.12834) -- (9.25356,3.16674);
\draw [c] (9.23874,3.12834) -- (9.25356,3.12834);
\draw [c] (9.25356,3.12834) -- (9.26838,3.12834);
\definecolor{c}{rgb}{0,0,0};
\colorlet{c}{kugray};
\draw [c] (9.2832,2.99447) -- (9.2832,3.04509);
\draw [c] (9.2832,3.04509) -- (9.2832,3.08855);
\draw [c] (9.26838,3.04509) -- (9.2832,3.04509);
\draw [c] (9.2832,3.04509) -- (9.29801,3.04509);
\definecolor{c}{rgb}{0,0,0};
\colorlet{c}{kugray};
\draw [c] (9.31283,3.07942) -- (9.31283,3.12376);
\draw [c] (9.31283,3.12376) -- (9.31283,3.16252);
\draw [c] (9.29801,3.12376) -- (9.31283,3.12376);
\draw [c] (9.31283,3.12376) -- (9.32765,3.12376);
\definecolor{c}{rgb}{0,0,0};
\colorlet{c}{kugray};
\draw [c] (9.34247,2.98071) -- (9.34247,3.03135);
\draw [c] (9.34247,3.03135) -- (9.34247,3.07482);
\draw [c] (9.32765,3.03135) -- (9.34247,3.03135);
\draw [c] (9.34247,3.03135) -- (9.35728,3.03135);
\definecolor{c}{rgb}{0,0,0};
\colorlet{c}{kugray};
\draw [c] (9.3721,3.09749) -- (9.3721,3.13869);
\draw [c] (9.3721,3.13869) -- (9.3721,3.17501);
\draw [c] (9.35728,3.13869) -- (9.3721,3.13869);
\draw [c] (9.3721,3.13869) -- (9.38692,3.13869);
\definecolor{c}{rgb}{0,0,0};
\colorlet{c}{kugray};
\draw [c] (9.40174,3.0393) -- (9.40174,3.08577);
\draw [c] (9.40174,3.08577) -- (9.40174,3.12613);
\draw [c] (9.38692,3.08577) -- (9.40174,3.08577);
\draw [c] (9.40174,3.08577) -- (9.41656,3.08577);
\definecolor{c}{rgb}{0,0,0};
\colorlet{c}{kugray};
\draw [c] (9.43137,2.98225) -- (9.43137,3.03202);
\draw [c] (9.43137,3.03202) -- (9.43137,3.07485);
\draw [c] (9.41656,3.03202) -- (9.43137,3.03202);
\draw [c] (9.43137,3.03202) -- (9.44619,3.03202);
\definecolor{c}{rgb}{0,0,0};
\colorlet{c}{kugray};
\draw [c] (9.46101,3.05321) -- (9.46101,3.0968);
\draw [c] (9.46101,3.0968) -- (9.46101,3.13497);
\draw [c] (9.44619,3.0968) -- (9.46101,3.0968);
\draw [c] (9.46101,3.0968) -- (9.47583,3.0968);
\definecolor{c}{rgb}{0,0,0};
\colorlet{c}{kugray};
\draw [c] (9.49065,3.04863) -- (9.49065,3.09461);
\draw [c] (9.49065,3.09461) -- (9.49065,3.1346);
\draw [c] (9.47583,3.09461) -- (9.49065,3.09461);
\draw [c] (9.49065,3.09461) -- (9.50546,3.09461);
\definecolor{c}{rgb}{0,0,0};
\colorlet{c}{kugray};
\draw [c] (9.52028,3.06634) -- (9.52028,3.11124);
\draw [c] (9.52028,3.11124) -- (9.52028,3.15042);
\draw [c] (9.50546,3.11124) -- (9.52028,3.11124);
\draw [c] (9.52028,3.11124) -- (9.5351,3.11124);
\definecolor{c}{rgb}{0,0,0};
\colorlet{c}{kugray};
\draw [c] (9.54992,3.06028) -- (9.54992,3.1061);
\draw [c] (9.54992,3.1061) -- (9.54992,3.14597);
\draw [c] (9.5351,3.1061) -- (9.54992,3.1061);
\draw [c] (9.54992,3.1061) -- (9.56474,3.1061);
\definecolor{c}{rgb}{0,0,0};
\colorlet{c}{kugray};
\draw [c] (9.57955,2.92876) -- (9.57955,2.98234);
\draw [c] (9.57955,2.98234) -- (9.57955,3.02797);
\draw [c] (9.56474,2.98234) -- (9.57955,2.98234);
\draw [c] (9.57955,2.98234) -- (9.59437,2.98234);
\definecolor{c}{rgb}{0,0,0};
\colorlet{c}{kugray};
\draw [c] (9.60919,3.05453) -- (9.60919,3.10277);
\draw [c] (9.60919,3.10277) -- (9.60919,3.14446);
\draw [c] (9.59437,3.10277) -- (9.60919,3.10277);
\draw [c] (9.60919,3.10277) -- (9.62401,3.10277);
\definecolor{c}{rgb}{0,0,0};
\colorlet{c}{kugray};
\draw [c] (9.63882,3.06402) -- (9.63882,3.10887);
\draw [c] (9.63882,3.10887) -- (9.63882,3.14802);
\draw [c] (9.62401,3.10887) -- (9.63882,3.10887);
\draw [c] (9.63882,3.10887) -- (9.65364,3.10887);
\definecolor{c}{rgb}{0,0,0};
\colorlet{c}{kugray};
\draw [c] (9.66846,3.03821) -- (9.66846,3.08666);
\draw [c] (9.66846,3.08666) -- (9.66846,3.12851);
\draw [c] (9.65364,3.08666) -- (9.66846,3.08666);
\draw [c] (9.66846,3.08666) -- (9.68328,3.08666);
\definecolor{c}{rgb}{0,0,0};
\colorlet{c}{kugray};
\draw [c] (9.6981,3.05196) -- (9.6981,3.09725);
\draw [c] (9.6981,3.09725) -- (9.6981,3.13673);
\draw [c] (9.68328,3.09725) -- (9.6981,3.09725);
\draw [c] (9.6981,3.09725) -- (9.71291,3.09725);
\definecolor{c}{rgb}{0,0,0};
\colorlet{c}{kugray};
\draw [c] (9.72773,2.99179) -- (9.72773,3.04189);
\draw [c] (9.72773,3.04189) -- (9.72773,3.08497);
\draw [c] (9.71291,3.04189) -- (9.72773,3.04189);
\draw [c] (9.72773,3.04189) -- (9.74255,3.04189);
\definecolor{c}{rgb}{0,0,0};
\colorlet{c}{kugray};
\draw [c] (9.75737,2.97816) -- (9.75737,3.02871);
\draw [c] (9.75737,3.02871) -- (9.75737,3.07212);
\draw [c] (9.74255,3.02871) -- (9.75737,3.02871);
\draw [c] (9.75737,3.02871) -- (9.77219,3.02871);
\definecolor{c}{rgb}{0,0,0};
\colorlet{c}{kugray};
\draw [c] (9.787,3.07791) -- (9.787,3.12253);
\draw [c] (9.787,3.12253) -- (9.787,3.1615);
\draw [c] (9.77219,3.12253) -- (9.787,3.12253);
\draw [c] (9.787,3.12253) -- (9.80182,3.12253);
\definecolor{c}{rgb}{0,0,0};
\colorlet{c}{kugray};
\draw [c] (9.81664,2.90394) -- (9.81664,2.95902);
\draw [c] (9.81664,2.95902) -- (9.81664,3.00573);
\draw [c] (9.80182,2.95902) -- (9.81664,2.95902);
\draw [c] (9.81664,2.95902) -- (9.83146,2.95902);
\definecolor{c}{rgb}{0,0,0};
\colorlet{c}{kugray};
\draw [c] (9.84627,3.10832) -- (9.84627,3.15092);
\draw [c] (9.84627,3.15092) -- (9.84627,3.18834);
\draw [c] (9.83146,3.15092) -- (9.84627,3.15092);
\draw [c] (9.84627,3.15092) -- (9.86109,3.15092);
\definecolor{c}{rgb}{0,0,0};
\colorlet{c}{kugray};
\draw [c] (9.87591,2.95007) -- (9.87591,3.00342);
\draw [c] (9.87591,3.00342) -- (9.87591,3.04887);
\draw [c] (9.86109,3.00342) -- (9.87591,3.00342);
\draw [c] (9.87591,3.00342) -- (9.89073,3.00342);
\definecolor{c}{rgb}{0,0,0};
\colorlet{c}{kugray};
\draw [c] (9.90555,2.93927) -- (9.90555,2.99525);
\draw [c] (9.90555,2.99525) -- (9.90555,3.0426);
\draw [c] (9.89073,2.99525) -- (9.90555,2.99525);
\draw [c] (9.90555,2.99525) -- (9.92036,2.99525);
\definecolor{c}{rgb}{0,0,0};
\colorlet{c}{kugray};
\draw [c] (9.93518,2.97877) -- (9.93518,3.0325);
\draw [c] (9.93518,3.0325) -- (9.93518,3.07823);
\draw [c] (9.92036,3.0325) -- (9.93518,3.0325);
\draw [c] (9.93518,3.0325) -- (9.95,3.0325);
\definecolor{c}{rgb}{0,0,0};
\definecolor{c}{rgb}{1,0.8,0};
\draw [c] (1.51655,5.54612) -- (1.60131,5.40539) -- (1.68607,5.27646) -- (1.77083,5.15727) -- (1.85558,5.04631) -- (1.94034,4.9424) -- (2.0251,4.84466) -- (2.10986,4.75237) -- (2.19462,4.66498) -- (2.27938,4.58203) -- (2.36413,4.50316)
 -- (2.44889,4.42808) -- (2.53365,4.35654) -- (2.61841,4.28835) -- (2.70317,4.22336) -- (2.78793,4.16142) -- (2.87268,4.10244) -- (2.95744,4.04632) -- (3.0422,3.99299) -- (3.12696,3.94238) -- (3.21172,3.89444) -- (3.29648,3.84909) --
 (3.38123,3.80628) -- (3.46599,3.76596) -- (3.55075,3.72806) -- (3.63551,3.69251) -- (3.72027,3.65925) -- (3.80502,3.62819) -- (3.88978,3.59926) -- (3.97454,3.57235) -- (4.0593,3.5474) -- (4.14406,3.52429) -- (4.22882,3.50293) -- (4.31357,3.48322) --
 (4.39833,3.46507) -- (4.48309,3.44838) -- (4.56785,3.43305) -- (4.65261,3.41898) -- (4.73737,3.40609) -- (4.82212,3.39428) -- (4.90688,3.38348) -- (4.99164,3.37359) -- (5.0764,3.36455) -- (5.16116,3.35628) -- (5.24592,3.34871) -- (5.33067,3.34177)
 -- (5.41543,3.3354) -- (5.50019,3.32954) -- (5.58495,3.32415) -- (5.66971,3.31916);
\draw [c] (5.66971,3.31916) -- (5.75447,3.31453) -- (5.83922,3.31022) -- (5.92398,3.30618) -- (6.00874,3.30237) -- (6.0935,3.29876) -- (6.17826,3.29531) -- (6.26301,3.292) -- (6.34777,3.28878) -- (6.43253,3.28563) -- (6.51729,3.28253)
 -- (6.60205,3.27946) -- (6.68681,3.27638) -- (6.77156,3.27328) -- (6.85632,3.27014) -- (6.94108,3.26694) -- (7.02584,3.26367) -- (7.1106,3.2603) -- (7.19536,3.25682) -- (7.28011,3.25323) -- (7.36487,3.2495) -- (7.44963,3.24562) -- (7.53439,3.24159)
 -- (7.61915,3.23739) -- (7.70391,3.23302) -- (7.78866,3.22846) -- (7.87342,3.2237) -- (7.95818,3.21875) -- (8.04294,3.21358) -- (8.1277,3.20821) -- (8.21246,3.2026) -- (8.29721,3.19678) -- (8.38197,3.19071) -- (8.46673,3.18441) -- (8.55149,3.17787)
 -- (8.63625,3.17108) -- (8.72101,3.16405) -- (8.80576,3.15675) -- (8.89052,3.1492) -- (8.97528,3.14139) -- (9.06004,3.13331) -- (9.1448,3.12497) -- (9.22955,3.11636) -- (9.31431,3.10747) -- (9.39907,3.09831) -- (9.48383,3.08888) -- (9.56859,3.07917)
 -- (9.65335,3.06918) -- (9.7381,3.0589) -- (9.82286,3.04835);
\draw [c] (9.82286,3.04835) -- (9.90762,3.03751);
\end{tikzpicture}

\end{infilsf}
\caption{The functions fitted to each of the CalcHEP distributions with different $\Lambda$ values, which were produced in chapter~\ref{ch.mc}.}\label{simfit}
\end{figure}

[Without proper errors: Most likely $\Lambda$ is 1.37 TeV, lower limit is 1.04 TeV. The Standard Model survives.]



\chapter{Conclusion}
In this thesis, we attempted to estimate the effect of a $qq\rightarrow\gamma\gamma$ contact interaction, with associated energy scale $\Lambda$, upon the distribution of $M_{\gamma\gamma}$ in diphoton events produced in proton--proton collisions at the LHC.

The new interaction was introduced in a modified Standard Model through the effective Lagrangian approach.

This is a highly non--trivial endeavour, as there are no suitable \atlas{} Monte Carlo sets on which to base this attempt. We found that even the Monte Carlo set created to simulate the Standard Model $qq\rightarrow\gamma\gamma$ process did not satisfactorily cover the sensitive region at high invariant masses. Thus, we were required to go through every step needed to create a suitable MC data set. This process involved long processing times, using large and cumbersome software packages. In the end, there is room for improvement in the outcome of this process.

We exploited the quadratic form of the expression for the strength of the new interaction to derive a shape for the distribution of events at arbitrary $\Lambda$ from distributions created at two discrete values of $\Lambda$ and the Standard Model distribution. This involved describing the shape of the distribution by performing a simultaneous fit over all three distributions.

A data driven background estimation technique was applied to data taken from the \atlas{} detector at the LHC. The resulting estimated background distribution was combined with appropriate distributions from Monte Carlo, and its shape was extrapolated into the sensitive invariant mass region above 1\,000 GeV.

A most likely value of $\Lambda$, with accompanying confidence interval was found by comparing the data distribution of invariant mass, $M_{\gamma\gamma}$, with the Monte Carlo distribution, combined with appropriate backgrounds, using a maximum profile likelihood procedure, which took systematic uncertainties into account. The invariant mass was chosen as the discriminating variable among several other variables, for its discriminating power and robustness.

The profile likelihood fit give as a result a 95\% confidence interval for $\Lambda$ between 763 and 900 GeV. This result is influenced by the poor statistics and high uncertainty in the signal region of the \atlas{} $\gamma$jet Monte Carlo sample. The result could be improved if a sample with better statistics were available. We have also pointed a direction for better accounting for the uncertainty associated with this sample in the context of a likelihood profile fit.

The analysis already included methods that produced errors for bins individually, rather than assuming correlation across all bins. This process is easily extended to accept per--bin or per--event uncertainties for detector effects or Monte Carlo uncertainties. It can also be extended to account for correlated errors.

Lastly, an analysis similar to this one on the data from 13 TeV collisions being taken at the LHC this year (2015) would of course have substantially better sensitivity due to the strong enhancement with energy.

\section{Addendum}
An additional estimate for the confidence limits on $\Lambda$ became available before the thesis defence. A profile likelihood fit which included uncertainties on the background distribution finds a confidence interval for $\Lambda$ which agrees with the Standard Model, and sets a lower limit on $\Lambda$ of 779 GeV.

\printbibliography[title=References]

\appendix

\chapter{ATLAS Monte Carlo data sets}
This appendix contains the full identifiers of the \atlas{} Monte Carlo data sets used in this thesis.

\section{Gamma--gamma set}\label{ax.gg}
\hangpara{1em}{1}\_8TeV.129180.Pythia8\_AU2CTEQ6L1\_gammagamma\_2DP20.merge. \\NTUP\_PHOTON.e1199\_s1482\_s1473\_r3542\_r3549\_p1344

\section{Box gamma--gamma set}\label{ax.ggb}
\hangpara{1em}{1}\_8TeV.146800.Pythia8\_AU2CTEQ6L1\_GamGamBox\_pT35pT20.merge. \\NTUP\_PHOTON.e1222\_s1469\_s1470\_r3542\_r3549\_p1344

\section{Gamma--jet sets}\label{ax.gj}
\hangpara{1em}{1}\_8TeV.129170.Pythia8\_AU2CTEQ6L1\_gammajet\_DP17.merge. \\NTUP\_PHOTON.e1146\_s1468\_s1470\_r3542\_r3549\_p1344

\hangpara{1em}{1}\_8TeV.129171.Pythia8\_AU2CTEQ6L1\_gammajet\_DP35.merge. \\NTUP\_PHOTON.e1146\_s1468\_s1470\_r3542\_r3549\_p1344

\hangpara{1em}{1}\_8TeV.129172.Pythia8\_AU2CTEQ6L1\_gammajet\_DP70.merge. \\NTUP\_PHOTON.e1146\_s1468\_s1470\_r3542\_r3549\_p1344

\hangpara{1em}{1}\_8TeV.129173.Pythia8\_AU2CTEQ6L1\_gammajet\_DP140.merge. \\NTUP\_PHOTON.e1146\_s1468\_s1470\_r3542\_r3549\_p1344

\hangpara{1em}{1}\_8TeV.129174.Pythia8\_AU2CTEQ6L1\_gammajet\_DP280.merge. \\NTUP\_PHOTON.e1146\_s1468\_s1470\_r3542\_r3549\_p1344

\hangpara{1em}{1}\_8TeV.129175.Pythia8\_AU2CTEQ6L1\_gammajet\_DP500.merge. \\NTUP\_PHOTON.e1146\_s1468\_s1470\_r3542\_r3549\_p1344

\section{Z to ee set}\label{ax.zee}

\hangpara{1em}{1}\_8TeV.147770.Sherpa\_CT10\_Zee.merge. \\NTUP\_PHOTON.e1434\_s1499\_s1504\_r3658\_r3549\_p1344.

\section{Sets examined for potential backgrounds}\label{ax.other}
\hangpara{1em}{1}\_8TeV.107680.AlpgenJimmy\_AUET2CTEQ6L1\_WenuNp0.merge. \\NTUP\_PHOTON.e1218\_s1469\_s1470\_r3542\_r3549\_p1344.

\hangpara{1em}{1}\_8TeV.107684.AlpgenJimmy\_AUET2CTEQ6L1\_WenuNp4.merge. \\NTUP\_PHOTON.e1218\_s1469\_s1470\_r3542\_r3549\_p1344.

\hangpara{1em}{1}\_8TeV.113714.Sherpa\_CT10\_SinglePhotonPt35.merge. \\NTUP\_PHOTON.e1249\_s1469\_s1470\_r3542\_r3549\_p1344.31.05.130531160824

\hangpara{1em}{1}\_8TeV.113715.Sherpa\_CT10\_SinglePhotonPt70.merge. \\NTUP\_PHOTON.e1249\_s1469\_s1470\_r3542\_r3549\_p1344.31.05.130531160856

\hangpara{1em}{1}\_8TeV.113716.Sherpa\_CT10\_SinglePhotonPt140.merge. \\NTUP\_PHOTON.e1249\_s1469\_s1470\_r3542\_r3549\_p1344.31.05.130531160931

\hangpara{1em}{1}\_8TeV.113717.Sherpa\_CT10\_SinglePhotonPt280.merge. \\NTUP\_PHOTON.e1249\_s1469\_s1470\_r3542\_r3549\_p1344.31.05.130531161030

\hangpara{1em}{1}\_8TeV.126371.Sherpa\_CT10\_SinglePhotonPt500.merge. \\NTUP\_PHOTON.e1249\_s1469\_s1470\_r3542\_r3549\_p1344.31.05.130531161224

\hangpara{1em}{1}\_8TeV.126372.Sherpa\_CT10\_SinglePhotonPt15.merge. \\NTUP\_PHOTON.e1249\_s1469\_s1470\_r3542\_r3549\_p1344.31.05.130531161146

\hangpara{1em}{1}\_8TeV.126955.Sherpa\_CT10\_SinglePhotonPt800.merge. \\NTUP\_PHOTON.e1249\_s1469\_s1470\_r3542\_r3549\_p1344.31.05.130531161311

\hangpara{1em}{1}\_8TeV.126956.Sherpa\_CT10\_SinglePhotonPt1000.merge. \\NTUP\_PHOTON.e1249\_s1469\_s1470\_r3752\_r3549\_p1344.31.05.130531161341

\hangpara{1em}{1}\_8TeV.128495.AlpgenPythia\_P2011CCTEQ6L1\_GamGamMass50GeVNp0.merge. \\NTUP\_PHOTON.e1222\_s1469\_s1470\_r3542\_r3549\_p1344.

\hangpara{1em}{1}\_8TeV.128496.AlpgenPythia\_P2011CCTEQ6L1\_GamGamMass50GeVNp1.merge. \\NTUP\_PHOTON.e1222\_s1469\_s1470\_r3542\_r3549\_p1344.

\hangpara{1em}{1}\_8TeV.128497.AlpgenPythia\_P2011CCTEQ6L1\_GamGamMass50GeVNp2.merge. \\NTUP\_PHOTON.e1222\_s1469\_s1470\_r3542\_r3549\_p1344.

\hangpara{1em}{1}\_8TeV.128498.AlpgenPythia\_P2011CCTEQ6L1\_GamGamMass50GeVNp3.merge. \\NTUP\_PHOTON.e1222\_s1469\_s1470\_r3542\_r3549\_p1344.

\hangpara{1em}{1}\_8TeV.146431.AlpgenJimmy\_AUET2CTEQ6L1\_WgammaNp1.merge. \\NTUP\_PHOTON.e1260\_s1469\_s1470\_r3542\_r3549\_p1344.

\hangpara{1em}{1}\_8TeV.146433.AlpgenJimmy\_AUET2CTEQ6L1\_WgammaNp3.merge. \\NTUP\_PHOTON.e1293\_s1469\_s1470\_r3542\_r3549\_p1344.

\hangpara{1em}{1}\_8TeV.146434.AlpgenJimmy\_AUET2CTEQ6L1\_WgammaNp4.merge. \\NTUP\_PHOTON.e1293\_s1469\_s1470\_r3542\_r3549\_p1344.

\hangpara{1em}{1}\_8TeV.146435.AlpgenJimmy\_AUET2CTEQ6L1\_WgammaNp5.merge. \\NTUP\_PHOTON.e1293\_s1469\_s1470\_r3542\_r3549\_p1344.

\hangpara{1em}{1}\_8TeV.146824.Sherpa\_CT10\_2DP20\_3j\_Myy80GeV.merge. \\NTUP\_PHOTON.e1434\_s1499\_s1504\_r3658\_r3549\_p1344.

\hangpara{1em}{1}\_8TeV.146825.Sherpa\_CT10\_SinglePhotonPt15\_2JetEt35Et20.merge. \\NTUP\_PHOTON.e1310\_s1469\_s1470\_r3542\_r3549\_p1344.31.05.130531161104

\hangpara{1em}{1}\_8TeV.147364.Sherpa\_CT10\_Zeegammagamma.merge. \\NTUP\_PHOTON.e1972\_s1581\_s1586\_r3658\_r3549\_p1344.

\hangpara{1em}{1}\_8TeV.147771.Sherpa\_CT10\_Zmumu.merge. \\NTUP\_PHOTON.e1434\_s1499\_s1504\_r3658\_r3549\_p1344.

\hangpara{1em}{1}\_8TeV.147800.PowhegPythia8\_AU2CT10\_Wplusenu.merge. \\NTUP\_PHOTON.e1169\_s1482\_s1473\_r3542\_r3549\_p1344.31.05.130531160433

\hangpara{1em}{1}\_8TeV.147803.PowhegPythia8\_AU2CT10\_Wminenu.merge. \\NTUP\_PHOTON.e1169\_s1482\_s1473\_r3542\_r3549\_p1344.31.05.130531160506

\hangpara{1em}{1}\_8TeV.147806.PowhegPythia8\_AU2CT10\_Zee.merge. \\NTUP\_PHOTON.e1169\_s1482\_s1473\_r3542\_r3549\_p1344.31.05.130531160538

\hangpara{1em}{1}\_8TeV.147807.PowhegPythia8\_AU2CT10\_Zmumu.merge. \\NTUP\_PHOTON.e1169\_s1469\_s1470\_r3542\_r3549\_p1344.

\hangpara{1em}{1}\_8TeV.147808.PowhegPythia8\_AU2CT10\_Ztautau.merge. \\NTUP\_PHOTON.e1169\_s1469\_s1470\_r3542\_r3549\_p1344.

\hangpara{1em}{1}\_8TeV.158728.Sherpa\_CT10\_ZeegammaPt70.merge. \\NTUP\_PHOTON.e1518\_s1499\_s1504\_r3658\_r3549\_p1344.

\hangpara{1em}{1}\_8TeV.167476.Sherpa\_CT10\_Zeegammagamma.merge. \\NTUP\_PHOTON.e1531\_s1499\_s1504\_r3658\_r3549\_p1344.

\hangpara{1em}{1}\_8TeV.167477.Sherpa\_CT10\_Zmumugammagamma.merge. \\NTUP\_PHOTON.e1531\_s1499\_s1504\_r3658\_r3549\_p1344.

\hangpara{1em}{1}\_8TeV.167478.Sherpa\_CT10\_Ztautaugammagamma.merge. \\NTUP\_PHOTON.e1573\_s1499\_s1504\_r3658\_r3549\_p1344.

\hangpara{1em}{1}\_8TeV.167479.Sherpa\_CT10\_Znunugammagamma.merge. \\NTUP\_PHOTON.e1531\_s1499\_s1504\_r3658\_r3549\_p1344.

\hangpara{1em}{1}\_8TeV.169707.MadgraphPythia\_AUET2B\_CTEQ6L1\_Wgam0jj\_fullme.merge. \\NTUP\_PHOTON.e1788\_s1581\_s1586\_r3658\_r3549\_p1344.

\hangpara{1em}{1}\_8TeV.181080.Sherpa\_CT10\_ZeegammaPt25.merge. \\NTUP\_PHOTON.e1978\_s1581\_s1586\_r3658\_r3549\_p1344.


\phantom{.}

\thispagestyle{empty}
\vspace{\fill}
\begin{center}
This is version 2 of the thesis, which contains minor corrections and additional results which became available before the thesis defence on 30 october 2015.
\end{center}

\end{english}
\end{document}
