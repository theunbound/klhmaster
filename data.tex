\chapter{Data preparation \label{ch.data}}

For the present analysis, we will use events that passed the \texttt{2g40\_loose} level--1 trigger, which requires that the EM calorimeter reports two energetic regions with at least 40~GeV of transverse energy that pass the loose selection criteria, described previously in chapter~\ref{ch.exp}.

The datasets have been retrieved in the \texttt{NTUP\_PHOTON} format, which is streamlined to contain information relevant to photon analyses, and easily readable by \textsc{root}. The dataset used contains events corresponding to 18.301~fb$^{-1}$ of integrated luminosity.

On each of the prospective photons in this dataset, we impose a series of selection criteria:

\begin{itemize}
\item \textbf{otx and phoClean cut:} Object quality cuts, which cut out events too close to non-functioning or noisy detector elements, and events taken while the detector was in a non-optimal state.
\item \textbf{ID cut:} Objects that did not pass photon identification, or do not satisfy the loose selection criteria after reconstruction, are eliminated.
\item \textbf{kinematics cut:} Ensures that objects do not have $|\eta|$ greater than 2.37, which is the forward limit of the first layer of the EM calorimeter, or in the range between 1.37 and 1.52, which is the transition region between the barrel and endcap calorimeters. Also ensures $E_T$ greater than 50~GeV, which clears the turn--on curve of the \texttt{2g40\_loose} trigger.
\item \textbf{N\_events cut:} Ensures that each event has at least two photons that pass the above criteria.
\item \textbf{PV cut:} Ensures that a photon pair can be associated with a primary vertex---that is, the must be at least one primary vertex within 3$\sigma$ of the reconstructed $z$ coordinate of the common vertex of the photon pair---which, additionally, must have at least 3 tracks associated with it, and be within 3$\sigma$ of the beam spot in the $x$ and $y$ coordinates.

\end{itemize}

If more than one photon pair are selected in a single event, the pair with the most energetic leading, and if necessary, then most energetic subleadng, photon is selected.

\begin{figure}[htp]
\begin{minipage}[b]{.69\textwidth}
\begin{infilsf} \tiny
\begin{tikzpicture}[x=.092\textwidth,y=.092\textwidth]
\pgfdeclareplotmark{cross} {
\pgfpathmoveto{\pgfpoint{-0.3\pgfplotmarksize}{\pgfplotmarksize}}
\pgfpathlineto{\pgfpoint{+0.3\pgfplotmarksize}{\pgfplotmarksize}}
\pgfpathlineto{\pgfpoint{+0.3\pgfplotmarksize}{0.3\pgfplotmarksize}}
\pgfpathlineto{\pgfpoint{+1\pgfplotmarksize}{0.3\pgfplotmarksize}}
\pgfpathlineto{\pgfpoint{+1\pgfplotmarksize}{-0.3\pgfplotmarksize}}
\pgfpathlineto{\pgfpoint{+0.3\pgfplotmarksize}{-0.3\pgfplotmarksize}}
\pgfpathlineto{\pgfpoint{+0.3\pgfplotmarksize}{-1.\pgfplotmarksize}}
\pgfpathlineto{\pgfpoint{-0.3\pgfplotmarksize}{-1.\pgfplotmarksize}}
\pgfpathlineto{\pgfpoint{-0.3\pgfplotmarksize}{-0.3\pgfplotmarksize}}
\pgfpathlineto{\pgfpoint{-1.\pgfplotmarksize}{-0.3\pgfplotmarksize}}
\pgfpathlineto{\pgfpoint{-1.\pgfplotmarksize}{0.3\pgfplotmarksize}}
\pgfpathlineto{\pgfpoint{-0.3\pgfplotmarksize}{0.3\pgfplotmarksize}}
\pgfpathclose
\pgfusepathqstroke
}
\pgfdeclareplotmark{cross*} {
\pgfpathmoveto{\pgfpoint{-0.3\pgfplotmarksize}{\pgfplotmarksize}}
\pgfpathlineto{\pgfpoint{+0.3\pgfplotmarksize}{\pgfplotmarksize}}
\pgfpathlineto{\pgfpoint{+0.3\pgfplotmarksize}{0.3\pgfplotmarksize}}
\pgfpathlineto{\pgfpoint{+1\pgfplotmarksize}{0.3\pgfplotmarksize}}
\pgfpathlineto{\pgfpoint{+1\pgfplotmarksize}{-0.3\pgfplotmarksize}}
\pgfpathlineto{\pgfpoint{+0.3\pgfplotmarksize}{-0.3\pgfplotmarksize}}
\pgfpathlineto{\pgfpoint{+0.3\pgfplotmarksize}{-1.\pgfplotmarksize}}
\pgfpathlineto{\pgfpoint{-0.3\pgfplotmarksize}{-1.\pgfplotmarksize}}
\pgfpathlineto{\pgfpoint{-0.3\pgfplotmarksize}{-0.3\pgfplotmarksize}}
\pgfpathlineto{\pgfpoint{-1.\pgfplotmarksize}{-0.3\pgfplotmarksize}}
\pgfpathlineto{\pgfpoint{-1.\pgfplotmarksize}{0.3\pgfplotmarksize}}
\pgfpathlineto{\pgfpoint{-0.3\pgfplotmarksize}{0.3\pgfplotmarksize}}
\pgfpathclose
\pgfusepathqfillstroke
}
\pgfdeclareplotmark{newstar} {
\pgfpathmoveto{\pgfqpoint{0pt}{\pgfplotmarksize}}
\pgfpathlineto{\pgfqpointpolar{44}{0.5\pgfplotmarksize}}
\pgfpathlineto{\pgfqpointpolar{18}{\pgfplotmarksize}}
\pgfpathlineto{\pgfqpointpolar{-20}{0.5\pgfplotmarksize}}
\pgfpathlineto{\pgfqpointpolar{-54}{\pgfplotmarksize}}
\pgfpathlineto{\pgfqpointpolar{-90}{0.5\pgfplotmarksize}}
\pgfpathlineto{\pgfqpointpolar{234}{\pgfplotmarksize}}
\pgfpathlineto{\pgfqpointpolar{198}{0.5\pgfplotmarksize}}
\pgfpathlineto{\pgfqpointpolar{162}{\pgfplotmarksize}}
\pgfpathlineto{\pgfqpointpolar{134}{0.5\pgfplotmarksize}}
\pgfpathclose
\pgfusepathqstroke
}
\pgfdeclareplotmark{newstar*} {
\pgfpathmoveto{\pgfqpoint{0pt}{\pgfplotmarksize}}
\pgfpathlineto{\pgfqpointpolar{44}{0.5\pgfplotmarksize}}
\pgfpathlineto{\pgfqpointpolar{18}{\pgfplotmarksize}}
\pgfpathlineto{\pgfqpointpolar{-20}{0.5\pgfplotmarksize}}
\pgfpathlineto{\pgfqpointpolar{-54}{\pgfplotmarksize}}
\pgfpathlineto{\pgfqpointpolar{-90}{0.5\pgfplotmarksize}}
\pgfpathlineto{\pgfqpointpolar{234}{\pgfplotmarksize}}
\pgfpathlineto{\pgfqpointpolar{198}{0.5\pgfplotmarksize}}
\pgfpathlineto{\pgfqpointpolar{162}{\pgfplotmarksize}}
\pgfpathlineto{\pgfqpointpolar{134}{0.5\pgfplotmarksize}}
\pgfpathclose
\pgfusepathqfillstroke
}
\definecolor{c}{rgb}{1,1,1};
\draw [color=c, fill=c] (0,0) rectangle (10,6.80516);
\draw [color=c, fill=c] (1,0.680516) rectangle (9.95,6.73711);
\definecolor{c}{rgb}{0,0,0};
\draw [c] (1,0.680516) -- (1,6.73711) -- (9.95,6.73711) -- (9.95,0.680516) -- (1,0.680516);
\definecolor{c}{rgb}{1,1,1};
\draw [color=c, fill=c] (1,0.680516) rectangle (9.95,6.73711);
\definecolor{c}{rgb}{0,0,0};
\draw [c] (1,0.680516) -- (1,6.73711) -- (9.95,6.73711) -- (9.95,0.680516) -- (1,0.680516);
\colorlet{c}{natgreen};
\draw [c] (1.55938,5.10106) -- (1.55938,5.10157);
\draw [c] (1.55938,5.10157) -- (1.55938,5.10208);
\draw [c] (1,5.10157) -- (1.55938,5.10157);
\draw [c] (1.55938,5.10157) -- (2.11875,5.10157);
\definecolor{c}{rgb}{0,0,0};
\colorlet{c}{natgreen};
\draw [c] (2.67813,5.0948) -- (2.67813,5.0953);
\draw [c] (2.67813,5.0953) -- (2.67813,5.09581);
\draw [c] (2.11875,5.0953) -- (2.67813,5.0953);
\draw [c] (2.67813,5.0953) -- (3.2375,5.0953);
\definecolor{c}{rgb}{0,0,0};
\colorlet{c}{natgreen};
\draw [c] (3.79688,5.04737) -- (3.79688,5.04788);
\draw [c] (3.79688,5.04788) -- (3.79688,5.04838);
\draw [c] (3.2375,5.04788) -- (3.79688,5.04788);
\draw [c] (3.79688,5.04788) -- (4.35625,5.04788);
\definecolor{c}{rgb}{0,0,0};
\colorlet{c}{natgreen};
\draw [c] (4.91563,3.57142) -- (4.91563,3.57183);
\draw [c] (4.91563,3.57183) -- (4.91563,3.57225);
\draw [c] (4.35625,3.57183) -- (4.91563,3.57183);
\draw [c] (4.91563,3.57183) -- (5.475,3.57183);
\definecolor{c}{rgb}{0,0,0};
\colorlet{c}{natgreen};
\draw [c] (6.03438,1.14305) -- (6.03438,1.14321);
\draw [c] (6.03438,1.14321) -- (6.03438,1.14337);
\draw [c] (5.475,1.14321) -- (6.03438,1.14321);
\draw [c] (6.03438,1.14321) -- (6.59375,1.14321);
\definecolor{c}{rgb}{0,0,0};
\colorlet{c}{natgreen};
\draw [c] (7.15312,0.8555) -- (7.15312,0.855634);
\draw [c] (7.15312,0.855634) -- (7.15312,0.855767);
\draw [c] (6.59375,0.855634) -- (7.15312,0.855634);
\draw [c] (7.15312,0.855634) -- (7.7125,0.855634);
\definecolor{c}{rgb}{0,0,0};
\colorlet{c}{natgreen};
\draw [c] (7.7125,0.766811) -- (8.27188,0.766811);
\draw [c] (8.27188,0.766811) -- (8.83125,0.766811);
\definecolor{c}{rgb}{0,0,0};
\colorlet{c}{natgreen};
\draw [c] (8.83125,0.765055) -- (9.39062,0.765055);
\draw [c] (9.39062,0.765055) -- (9.95,0.765055);
\definecolor{c}{rgb}{0,0,0};
\draw [c] (1,0.680516) -- (9.95,0.680516);
\draw [anchor= west] (1.55938,0.31712) node[color=c, rotate=-20]{All};
\draw [anchor= west] (2.67813,0.31712) node[color=c, rotate=-20]{otx\_cut};
\draw [anchor= west] (3.79688,0.31712) node[color=c, rotate=-20]{phoClean\_cut};
\draw [anchor= west] (4.91563,0.31712) node[color=c, rotate=-20]{ID\_cut};
\draw [anchor= west] (6.03438,0.31712) node[color=c, rotate=-20]{kinematics\_cut};
\draw [anchor= west] (7.15312,0.31712) node[color=c, rotate=-20]{N\_events};
\draw [anchor= west] (8.27188,0.31712) node[color=c, rotate=-20]{PV\_cut};
\draw [anchor= west] (9.39062,0.31712) node[color=c, rotate=-20]{selected\_photons\_cut};
\draw [c] (1,0.863234) -- (1,0.680516);
\draw [c] (2.11875,0.863234) -- (2.11875,0.680516);
\draw [c] (3.2375,0.863234) -- (3.2375,0.680516);
\draw [c] (4.35625,0.863234) -- (4.35625,0.680516);
\draw [c] (5.475,0.863234) -- (5.475,0.680516);
\draw [c] (6.59375,0.863234) -- (6.59375,0.680516);
\draw [c] (7.7125,0.863234) -- (7.7125,0.680516);
\draw [c] (8.83125,0.863234) -- (8.83125,0.680516);
\draw [c] (9.95,0.863234) -- (9.95,0.680516);
\draw [c] (1,0.680516) -- (1,6.73711);
\draw [anchor= east] (-0.12,6.73711) node[color=c, rotate=90]{Number of accepted events};
\draw [c] (1.267,0.705045) -- (1,0.705045);
\draw [c] (1.1335,0.941649) -- (1,0.941649);
\draw [c] (1.1335,1.17825) -- (1,1.17825);
\draw [c] (1.1335,1.41486) -- (1,1.41486);
\draw [c] (1.1335,1.65146) -- (1,1.65146);
\draw [c] (1.267,1.88806) -- (1,1.88806);
\draw [c] (1.1335,2.12467) -- (1,2.12467);
\draw [c] (1.1335,2.36127) -- (1,2.36127);
\draw [c] (1.1335,2.59788) -- (1,2.59788);
\draw [c] (1.1335,2.83448) -- (1,2.83448);
\draw [c] (1.267,3.07108) -- (1,3.07108);
\draw [c] (1.1335,3.30769) -- (1,3.30769);
\draw [c] (1.1335,3.54429) -- (1,3.54429);
\draw [c] (1.1335,3.7809) -- (1,3.7809);
\draw [c] (1.1335,4.0175) -- (1,4.0175);
\draw [c] (1.267,4.2541) -- (1,4.2541);
\draw [c] (1.1335,4.49071) -- (1,4.49071);
\draw [c] (1.1335,4.72731) -- (1,4.72731);
\draw [c] (1.1335,4.96391) -- (1,4.96391);
\draw [c] (1.1335,5.20052) -- (1,5.20052);
\draw [c] (1.267,5.43712) -- (1,5.43712);
\draw [c] (1.1335,5.67373) -- (1,5.67373);
\draw [c] (1.1335,5.91033) -- (1,5.91033);
\draw [c] (1.1335,6.14693) -- (1,6.14693);
\draw [c] (1.1335,6.38354) -- (1,6.38354);
\draw [c] (1.267,6.62014) -- (1,6.62014);
\draw [c] (1.267,0.705045) -- (1,0.705045);
\draw [c] (1.267,6.62014) -- (1,6.62014);
\draw [anchor= east] (0.95,0.705045) node[color=c, rotate=0]{0};
\draw [anchor= east] (0.95,1.88806) node[color=c, rotate=0]{10};
\draw [anchor= east] (0.95,3.07108) node[color=c, rotate=0]{20};
\draw [anchor= east] (0.95,4.2541) node[color=c, rotate=0]{30};
\draw [anchor= east] (0.95,5.43712) node[color=c, rotate=0]{40};
\draw [anchor= east] (0.95,6.62014) node[color=c, rotate=0]{50};
\draw [anchor=base west] (1,6.76092) node[color=c, rotate=0]{$\times10^{6}$};
\colorlet{c}{natgreen};
\draw [c] (1.55938,5.10106) -- (1.55938,5.10157);
\draw [c] (1.55938,5.10157) -- (1.55938,5.10208);
\draw [c] (1,5.10157) -- (1.55938,5.10157);
\draw [c] (1.55938,5.10157) -- (2.11875,5.10157);
\definecolor{c}{rgb}{0,0,0};
\colorlet{c}{natgreen};
\draw [c] (2.67813,5.0948) -- (2.67813,5.0953);
\draw [c] (2.67813,5.0953) -- (2.67813,5.09581);
\draw [c] (2.11875,5.0953) -- (2.67813,5.0953);
\draw [c] (2.67813,5.0953) -- (3.2375,5.0953);
\definecolor{c}{rgb}{0,0,0};
\colorlet{c}{natgreen};
\draw [c] (3.79688,5.04737) -- (3.79688,5.04788);
\draw [c] (3.79688,5.04788) -- (3.79688,5.04838);
\draw [c] (3.2375,5.04788) -- (3.79688,5.04788);
\draw [c] (3.79688,5.04788) -- (4.35625,5.04788);
\definecolor{c}{rgb}{0,0,0};
\colorlet{c}{natgreen};
\draw [c] (4.91563,3.57142) -- (4.91563,3.57183);
\draw [c] (4.91563,3.57183) -- (4.91563,3.57225);
\draw [c] (4.35625,3.57183) -- (4.91563,3.57183);
\draw [c] (4.91563,3.57183) -- (5.475,3.57183);
\definecolor{c}{rgb}{0,0,0};
\colorlet{c}{natgreen};
\draw [c] (6.03438,1.14305) -- (6.03438,1.14321);
\draw [c] (6.03438,1.14321) -- (6.03438,1.14337);
\draw [c] (5.475,1.14321) -- (6.03438,1.14321);
\draw [c] (6.03438,1.14321) -- (6.59375,1.14321);
\definecolor{c}{rgb}{0,0,0};
\colorlet{c}{natgreen};
\draw [c] (7.15312,0.8555) -- (7.15312,0.855634);
\draw [c] (7.15312,0.855634) -- (7.15312,0.855767);
\draw [c] (6.59375,0.855634) -- (7.15312,0.855634);
\draw [c] (7.15312,0.855634) -- (7.7125,0.855634);
\definecolor{c}{rgb}{0,0,0};
\colorlet{c}{natgreen};
\draw [c] (7.7125,0.766811) -- (8.27188,0.766811);
\draw [c] (8.27188,0.766811) -- (8.83125,0.766811);
\definecolor{c}{rgb}{0,0,0};
\colorlet{c}{natgreen};
\draw [c] (8.83125,0.765055) -- (9.39062,0.765055);
\draw [c] (9.39062,0.765055) -- (9.95,0.765055);
\definecolor{c}{rgb}{0,0,0};
\draw [anchor= west] (1.55938,5.12879) node[color=c, rotate=90]{3.71636e+07};
\draw [anchor= west] (2.67813,5.12253) node[color=c, rotate=90]{3.71106e+07};
\draw [anchor= west] (3.79688,5.0751) node[color=c, rotate=90]{3.67097e+07};
\draw [anchor= west] (4.91563,3.59905) node[color=c, rotate=90]{2.42328e+07};
\draw [anchor= west] (6.03438,1.17043) node[color=c, rotate=90]{3.70375e+06};
\draw [anchor= west] (7.15312,0.882854) node[color=c, rotate=90]{1.27291e+06};
\draw [anchor= west] (8.27188,0.794031) node[color=c, rotate=90]{522099};
\draw [anchor= west] (9.39062,0.792275) node[color=c, rotate=90]{507255};
\end{tikzpicture}

\end{infilsf}
\end{minipage}\hfill\begin{minipage}[b]{.3\textwidth}
\caption{A cutflow diagram, showing how many objects remain in the dataset after each of the selection criteria are imposed. The final number of photons is (something I sould be able to dig up somewhere).
\label{cutflow}}
\end{minipage}
\end{figure}

The number of objects remaining at each step of the cut procedure is plotted in figure~\ref{cutflow}

What remains after these cuts have been applied is a purer sample of photons than we had before, however the sample will still contain a background of events that do not come from the processes that we wish to study. An estimate of this background is required.

\section{Data driven background estimation}
The background that remains in the signal sample after these criteria have been applied can be estimated in a number of ways. In chapter~\ref{ch.mc}, Monte Carlo samples that simulate several physical processes that act as background to diphoton events were [will be have been: WIP] presented. Here, however, we shall attempt to quantify the magnitude of the background by examining the data.

\subsection{The ABCD method}
We assume that it is possible to extrapolate the shape of the distribution of background events that pollute the signal sample from the distribution of events that occur well away from the signal region. The \textsc{abcd} method assumes that the distribution of background events has the same shape in the signal region as it does in a control region. There will still be a scale difference between the two, which can be determined by examining a different control region.

\begin{figure}[hbp]
  \includegraphics[width=\textwidth]{figures/sideband}
  \caption{Illustrating the two--step ABCD method, adapted from \cite{fdirect} using the `tight' selection criteria and the isolation energy: the full set of diphotons (the L-L sample) is split into four groups---A, B, C and D---according to the discriminating variables for the leading photon. Signal photons are now confined to the A region. The events in the C region can be used to estimate the shape, and the B and D region can be used to estimate the magnitude of the distribution of background events in the signal region. The procedure is then repeated for the subleading partner photons of the events in the A region. This gives an estimate of the distibution of background events in the combined signal (AA) region.}\label{abcd}
\end{figure}

This is also known as the two--dimensional sideband method \cite{cmsabcd}. The following description may be aided by the illustration in figure~\ref{abcd}.

To reiterate, we need to examine our sample of signal and background data points in terms of two uncorrelated discriminating variables, call them $x$ and $y$. With this, we can split the data set into four regions:

\begin{itemize}
\item[{\bfseries\sffamily\color{natgreen}A:}] The signal region in both discriminating variables. This region should contain all signal events.
\item[{\bfseries\sffamily\color{natgreen}B:}] The signal region in the $y$ variable, but not in $x$. We assume that the distribution of background events in $y$ in this regions will have the same shape as the distribution of background events in $y$ in the signal region, A.
\item[{\bfseries\sffamily\color{natgreen}C:}] As above, but with $x$ and $y$ exchanged.
\item[{\bfseries\sffamily\color{natgreen}D:}] The control region for both variables. Once again, we assume that the distribution of background events has the same shape for either variable in the control region as it has in its signal region. We expect the distribution in $x$ to have the same shape in the D region as it does in the B region, for example.
\end{itemize}

Thus, the distribution in $x$ of background events in the A region, $A_{bck}$, is assumed to be the shape of the distribution of events in the C region, scaled so that the distribution of events in $x$ in the B and D regions have the same magnitude:
\(A_{bck}=C\frac BD.\label{abckfind}\)
The signal distribution must then be given as
\(A_{sig}=A-A_{bck}.\label{asig}\)

As we are working with events with two photons, both of which give rise to independent backgrounds, this procedure must be repeated for the subleading photons as well. For the subleading photon candidate, we look at the sample of photon candidates that are the subleading partner of a selected photon in the signal, A, region of the distribution of leading photon candidates. Carrying out the ABCD procedure on the sample of subleading photons gives us $AA_{bck}$, an estimate of the number of selected photon pairs where the leading photon is in the A region and the subleading photon is a background event in the A region of the distribution of subleading photons with a partner in the A region (call this the AA region for short)\footnote{AAAAAAAAAAA!}. As above, the number of events in the signal--signal region must be given by
\(A_{sig}A_{sig} = A_{sig}A-A_{sig}A_{bck}.\)
Inserting \eqref{asig} into this gives
\(\begin{aligned}
A_{sig}A_{sig} &=(A-A_{bck})A-(A-A_{bck})A_{bck}\\ &=AA-A_{bck}A-AA_{bck}+A_{bck}A_{bck}\\&=AA-[AA]_{bck}.
\end{aligned}\)
$AA$ is the number of events in the AA region, which is readily available. Of the terms that contribute to the total background $[AA]_{bkc}$, $AA_{bck}$, the total number of background subleading photons in the AA region, is determined by taking
\(AA_{bck}=AC\frac{AB}{AD},\)
analogous to how $A_{bck}$ sample was found in eq.~\eqref{abckfind}. We find $A_{bck}A$, the total number background leading photons in the AA region, and $A_{bck}A_{bck}$, the number of photon pairs in the AA region where both members are background photons, by multiplying $AA$ and $AA_{bck}$ by
\(f_{bck}=\frac{A_{bck}}{A},\)
the fraction of background events in the leading photon A sample. Thus, the total estimated background in the AA region is given by
\([AA]_{bck}=\frac{A_{bck}}{A}(AA)+AA_{bck}-\frac{A_{bck}}{A}(AA_{bck}),\)
which we can interpret as the number of data points where the leading photon was a background plus the number of data points where the subleading photon was a background, subtracted the number of data points where both photons were background events, which would have been double counted.

For the diphoton sample, we will use the `tight' selection criteria, which were described in chapter~\ref{ch.exp}, and the transverse isolation energy, $E_T^{\text{isol}}$, the energy deposited in the calorimeter in a cone with radius $R\le0.4$, but outside $R\le0.2$, where
\[R=\sqrt{\Delta\phi^2+\Delta\theta^2}.\]
The signal region is cut at $E_T^{\text{isol}}\le3$ GeV. We allow a crosstalk region of 2 GeV, which means the background region is defined with $5\text{ GeV}\le E_T^{\text{isol}}\le25\text{ GeV}$ [wait a minute...]. The distribution of $E_T^{\text{isol}}$ for leading photons and subleading photons in the `A' sample is given in figure~\ref{etiso}.

\begin{figure}[htp]
\begin{minipage}[b]{.69\textwidth}
\begin{infilsf} \tiny 
\begin{tikzpicture}[x=.1\textwidth,y=.1\textwidth]
\pgfdeclareplotmark{cross} {
\pgfpathmoveto{\pgfpoint{-0.3\pgfplotmarksize}{\pgfplotmarksize}}
\pgfpathlineto{\pgfpoint{+0.3\pgfplotmarksize}{\pgfplotmarksize}}
\pgfpathlineto{\pgfpoint{+0.3\pgfplotmarksize}{0.3\pgfplotmarksize}}
\pgfpathlineto{\pgfpoint{+1\pgfplotmarksize}{0.3\pgfplotmarksize}}
\pgfpathlineto{\pgfpoint{+1\pgfplotmarksize}{-0.3\pgfplotmarksize}}
\pgfpathlineto{\pgfpoint{+0.3\pgfplotmarksize}{-0.3\pgfplotmarksize}}
\pgfpathlineto{\pgfpoint{+0.3\pgfplotmarksize}{-1.\pgfplotmarksize}}
\pgfpathlineto{\pgfpoint{-0.3\pgfplotmarksize}{-1.\pgfplotmarksize}}
\pgfpathlineto{\pgfpoint{-0.3\pgfplotmarksize}{-0.3\pgfplotmarksize}}
\pgfpathlineto{\pgfpoint{-1.\pgfplotmarksize}{-0.3\pgfplotmarksize}}
\pgfpathlineto{\pgfpoint{-1.\pgfplotmarksize}{0.3\pgfplotmarksize}}
\pgfpathlineto{\pgfpoint{-0.3\pgfplotmarksize}{0.3\pgfplotmarksize}}
\pgfpathclose
\pgfusepathqstroke
}
\pgfdeclareplotmark{cross*} {
\pgfpathmoveto{\pgfpoint{-0.3\pgfplotmarksize}{\pgfplotmarksize}}
\pgfpathlineto{\pgfpoint{+0.3\pgfplotmarksize}{\pgfplotmarksize}}
\pgfpathlineto{\pgfpoint{+0.3\pgfplotmarksize}{0.3\pgfplotmarksize}}
\pgfpathlineto{\pgfpoint{+1\pgfplotmarksize}{0.3\pgfplotmarksize}}
\pgfpathlineto{\pgfpoint{+1\pgfplotmarksize}{-0.3\pgfplotmarksize}}
\pgfpathlineto{\pgfpoint{+0.3\pgfplotmarksize}{-0.3\pgfplotmarksize}}
\pgfpathlineto{\pgfpoint{+0.3\pgfplotmarksize}{-1.\pgfplotmarksize}}
\pgfpathlineto{\pgfpoint{-0.3\pgfplotmarksize}{-1.\pgfplotmarksize}}
\pgfpathlineto{\pgfpoint{-0.3\pgfplotmarksize}{-0.3\pgfplotmarksize}}
\pgfpathlineto{\pgfpoint{-1.\pgfplotmarksize}{-0.3\pgfplotmarksize}}
\pgfpathlineto{\pgfpoint{-1.\pgfplotmarksize}{0.3\pgfplotmarksize}}
\pgfpathlineto{\pgfpoint{-0.3\pgfplotmarksize}{0.3\pgfplotmarksize}}
\pgfpathclose
\pgfusepathqfillstroke
}
\pgfdeclareplotmark{newstar} {
\pgfpathmoveto{\pgfqpoint{0pt}{\pgfplotmarksize}}
\pgfpathlineto{\pgfqpointpolar{44}{0.5\pgfplotmarksize}}
\pgfpathlineto{\pgfqpointpolar{18}{\pgfplotmarksize}}
\pgfpathlineto{\pgfqpointpolar{-20}{0.5\pgfplotmarksize}}
\pgfpathlineto{\pgfqpointpolar{-54}{\pgfplotmarksize}}
\pgfpathlineto{\pgfqpointpolar{-90}{0.5\pgfplotmarksize}}
\pgfpathlineto{\pgfqpointpolar{234}{\pgfplotmarksize}}
\pgfpathlineto{\pgfqpointpolar{198}{0.5\pgfplotmarksize}}
\pgfpathlineto{\pgfqpointpolar{162}{\pgfplotmarksize}}
\pgfpathlineto{\pgfqpointpolar{134}{0.5\pgfplotmarksize}}
\pgfpathclose
\pgfusepathqstroke
}
\pgfdeclareplotmark{newstar*} {
\pgfpathmoveto{\pgfqpoint{0pt}{\pgfplotmarksize}}
\pgfpathlineto{\pgfqpointpolar{44}{0.5\pgfplotmarksize}}
\pgfpathlineto{\pgfqpointpolar{18}{\pgfplotmarksize}}
\pgfpathlineto{\pgfqpointpolar{-20}{0.5\pgfplotmarksize}}
\pgfpathlineto{\pgfqpointpolar{-54}{\pgfplotmarksize}}
\pgfpathlineto{\pgfqpointpolar{-90}{0.5\pgfplotmarksize}}
\pgfpathlineto{\pgfqpointpolar{234}{\pgfplotmarksize}}
\pgfpathlineto{\pgfqpointpolar{198}{0.5\pgfplotmarksize}}
\pgfpathlineto{\pgfqpointpolar{162}{\pgfplotmarksize}}
\pgfpathlineto{\pgfqpointpolar{134}{0.5\pgfplotmarksize}}
\pgfpathclose
\pgfusepathqfillstroke
}
\definecolor{c}{rgb}{1,1,1};
\draw [color=c, fill=c] (1,0.679598) rectangle (9,6.11638);
\definecolor{c}{rgb}{0,0,0};
\draw [c] (1,0.679598) -- (1,6.11638) -- (9,6.11638) -- (9,0.679598) -- (1,0.679598);
\definecolor{c}{rgb}{1,1,1};
\draw [color=c, fill=c] (1,0.679598) rectangle (9,6.11638);
\definecolor{c}{rgb}{0,0,0};
\draw [c] (1,0.679598) -- (1,6.11638) -- (9,6.11638) -- (9,0.679598) -- (1,0.679598);
\colorlet{c}{natgreen}
\draw [c] (1,0.680318) -- (1.0293,0.680318) -- (1.0293,0.681757) -- (1.05861,0.681757) -- (1.05861,0.681757) -- (1.08791,0.681757) -- (1.08791,0.683197) -- (1.11722,0.683197) -- (1.11722,0.682837) -- (1.14652,0.682837) -- (1.14652,0.682837) --
 (1.17582,0.682837) -- (1.17582,0.684637) -- (1.20513,0.684637) -- (1.20513,0.686437) -- (1.23443,0.686437) -- (1.23443,0.689676) -- (1.26374,0.689676) -- (1.26374,0.691476) -- (1.29304,0.691476) -- (1.29304,0.692916) -- (1.32234,0.692916) --
 (1.32234,0.700115) -- (1.35165,0.700115) -- (1.35165,0.706594) -- (1.38095,0.706594) -- (1.38095,0.718832) -- (1.41026,0.718832) -- (1.41026,0.730711) -- (1.43956,0.730711) -- (1.43956,0.73251) -- (1.46886,0.73251) -- (1.46886,0.749428) --
 (1.49817,0.749428) -- (1.49817,0.765986) -- (1.52747,0.765986) -- (1.52747,0.803421) -- (1.55678,0.803421) -- (1.55678,0.825378) -- (1.58608,0.825378) -- (1.58608,0.872891) -- (1.61538,0.872891) -- (1.61538,0.916085) -- (1.64469,0.916085) --
 (1.64469,0.980156) -- (1.67399,0.980156) -- (1.67399,1.02983) -- (1.7033,1.02983) -- (1.7033,1.13098) -- (1.7326,1.13098) -- (1.7326,1.24976) -- (1.7619,1.24976) -- (1.7619,1.36998) -- (1.79121,1.36998) -- (1.79121,1.47797) -- (1.82051,1.47797) --
 (1.82051,1.70258) -- (1.84982,1.70258) -- (1.84982,1.88471) -- (1.87912,1.88471) -- (1.87912,2.13668) -- (1.90842,2.13668) -- (1.90842,2.44264) -- (1.93773,2.44264) -- (1.93773,2.74931) -- (1.96703,2.74931) -- (1.96703,3.11538) -- (1.99634,3.11538)
 -- (1.99634,3.51025) -- (2.02564,3.51025) -- (2.02564,3.91339) -- (2.05494,3.91339) -- (2.05494,4.39969) -- (2.08425,4.39969) -- (2.08425,4.83019) -- (2.11355,4.83019) -- (2.11355,5.25133) -- (2.14286,5.25133) -- (2.14286,5.5742) -- (2.17216,5.5742)
 -- (2.17216,5.74302) -- (2.20147,5.74302) -- (2.20147,5.85748) -- (2.23077,5.85748) -- (2.23077,5.71314) -- (2.26007,5.71314) -- (2.26007,5.43346) -- (2.28938,5.43346) -- (2.28938,5.30064) -- (2.31868,5.30064) -- (2.31868,5.13362) --
 (2.34799,5.13362) -- (2.34799,4.86654) -- (2.37729,4.86654) -- (2.37729,4.65165) -- (2.40659,4.65165) -- (2.40659,4.41552) -- (2.4359,4.41552) -- (2.4359,4.16608) -- (2.4652,4.16608) -- (2.4652,4.03974) -- (2.49451,4.03974) -- (2.49451,3.90655) --
 (2.52381,3.90655) -- (2.52381,3.68158) -- (2.55311,3.68158) -- (2.55311,3.57216) -- (2.58242,3.57216) -- (2.58242,3.35511) -- (2.61172,3.35511) -- (2.61172,3.31767) -- (2.64103,3.31767) -- (2.64103,3.12546) -- (2.67033,3.12546) -- (2.67033,2.98292)
 -- (2.69963,2.98292) -- (2.69963,2.8429) -- (2.72894,2.8429) -- (2.72894,2.79683) -- (2.75824,2.79683) -- (2.75824,2.59309) -- (2.78755,2.59309) -- (2.78755,2.60389) -- (2.81685,2.60389) -- (2.81685,2.42572) -- (2.84615,2.42572) -- (2.84615,2.37208)
 -- (2.87546,2.37208) -- (2.87546,2.2821) -- (2.90476,2.2821) -- (2.90476,2.19823) -- (2.93407,2.19823) -- (2.93407,2.105) -- (2.96337,2.105) -- (2.96337,2.07477) -- (2.99267,2.07477) -- (2.99267,1.9513) -- (3.02198,1.9513) -- (3.02198,1.91567) --
 (3.05128,1.91567) -- (3.05128,1.8552) -- (3.08059,1.8552) -- (3.08059,1.80192) -- (3.10989,1.80192) -- (3.10989,1.72813) -- (3.13919,1.72813) -- (3.13919,1.70582) -- (3.1685,1.70582) -- (3.1685,1.66298) -- (3.1978,1.66298) -- (3.1978,1.60575) --
 (3.22711,1.60575) -- (3.22711,1.58127) -- (3.25641,1.58127) -- (3.25641,1.53772) -- (3.28571,1.53772) -- (3.28571,1.48373) -- (3.31502,1.48373) -- (3.31502,1.43441) -- (3.34432,1.43441) -- (3.34432,1.44557) -- (3.37363,1.44557) -- (3.37363,1.37754)
 -- (3.40293,1.37754) -- (3.40293,1.35594) -- (3.43223,1.35594) -- (3.43223,1.32931) -- (3.46154,1.32931) -- (3.46154,1.29151) -- (3.49084,1.29151) -- (3.49084,1.30159) -- (3.52015,1.30159) -- (3.52015,1.23608) -- (3.54945,1.23608) --
 (3.54945,1.2476) -- (3.57875,1.2476) -- (3.57875,1.20405) -- (3.60806,1.20405) -- (3.60806,1.19361) -- (3.63736,1.19361) -- (3.63736,1.19181) -- (3.66667,1.19181) -- (3.66667,1.18497) -- (3.69597,1.18497) -- (3.69597,1.14357) -- (3.72527,1.14357) --
 (3.72527,1.11766) -- (3.75458,1.11766) -- (3.75458,1.10074) -- (3.78388,1.10074) -- (3.78388,1.0705) -- (3.81319,1.0705) -- (3.81319,1.08454) -- (3.84249,1.08454) -- (3.84249,1.04675) -- (3.87179,1.04675) -- (3.87179,1.03523) -- (3.9011,1.03523) --
 (3.9011,1.01651) -- (3.9304,1.01651) -- (3.9304,1.01939) -- (3.95971,1.01939) -- (3.95971,1.01039) -- (3.98901,1.01039) -- (3.98901,1.01363) -- (4.01831,1.01363) -- (4.01831,0.979796) -- (4.04762,0.979796) -- (4.04762,0.969358) -- (4.07692,0.969358)
 -- (4.07692,0.944881) -- (4.10623,0.944881) -- (4.10623,0.95316) -- (4.13553,0.95316) -- (4.13553,0.927604) -- (4.16483,0.927604) -- (4.16483,0.927244) -- (4.19414,0.927244) -- (4.19414,0.899168) -- (4.22344,0.899168) -- (4.22344,0.907446) --
 (4.25275,0.907446) -- (4.25275,0.914285) -- (4.28205,0.914285) -- (4.28205,0.890529) -- (4.31136,0.890529) -- (4.31136,0.901687) -- (4.34066,0.901687) -- (4.34066,0.898088) -- (4.36996,0.898088) -- (4.36996,0.88369) -- (4.39927,0.88369) --
 (4.39927,0.87793) -- (4.42857,0.87793) -- (4.42857,0.867852) -- (4.45788,0.867852) -- (4.45788,0.859573) -- (4.48718,0.859573) -- (4.48718,0.850574) -- (4.51648,0.850574) -- (4.51648,0.849854) -- (4.54579,0.849854) -- (4.54579,0.854174) --
 (4.57509,0.854174) -- (4.57509,0.852014) -- (4.6044,0.852014) -- (4.6044,0.847695) -- (4.6337,0.847695) -- (4.6337,0.823578) -- (4.663,0.823578) -- (4.663,0.833297) -- (4.69231,0.833297) -- (4.69231,0.818539) -- (4.72161,0.818539) --
 (4.72161,0.802701) -- (4.75092,0.802701) -- (4.75092,0.818539) -- (4.78022,0.818539) -- (4.78022,0.813499) -- (4.80952,0.813499) -- (4.80952,0.81134) -- (4.83883,0.81134) -- (4.83883,0.799101) -- (4.86813,0.799101) -- (4.86813,0.789743) --
 (4.89744,0.789743) -- (4.89744,0.796942) -- (4.92674,0.796942) -- (4.92674,0.792262) -- (4.95604,0.792262) -- (4.95604,0.795502) -- (4.98535,0.795502) -- (4.98535,0.797661) -- (5.01465,0.797661) -- (5.01465,0.777144) -- (5.04396,0.777144) --
 (5.04396,0.783983) -- (5.07326,0.783983) -- (5.07326,0.776784) -- (5.10256,0.776784) -- (5.10256,0.770305) -- (5.13187,0.770305) -- (5.13187,0.773545) -- (5.16117,0.773545) -- (5.16117,0.778584) -- (5.19048,0.778584) -- (5.19048,0.772465) --
 (5.21978,0.772465) -- (5.21978,0.764186) -- (5.24908,0.764186) -- (5.24908,0.763106) -- (5.27839,0.763106) -- (5.27839,0.760227) -- (5.30769,0.760227) -- (5.30769,0.762026) -- (5.337,0.762026) -- (5.337,0.756627) -- (5.3663,0.756627) --
 (5.3663,0.762746) -- (5.3956,0.762746) -- (5.3956,0.757707) -- (5.42491,0.757707) -- (5.42491,0.755187) -- (5.45421,0.755187) -- (5.45421,0.747628) -- (5.48352,0.747628) -- (5.48352,0.745469) -- (5.51282,0.745469) -- (5.51282,0.747268) --
 (5.54212,0.747268) -- (5.54212,0.753388) -- (5.57143,0.753388) -- (5.57143,0.746908) -- (5.60073,0.746908) -- (5.60073,0.73827) -- (5.63004,0.73827) -- (5.63004,0.741869) -- (5.65934,0.741869) -- (5.65934,0.742229) -- (5.68864,0.742229) --
 (5.68864,0.739709) -- (5.71795,0.739709) -- (5.71795,0.73863) -- (5.74725,0.73863) -- (5.74725,0.731431) -- (5.77656,0.731431) -- (5.77656,0.728551) -- (5.80586,0.728551) -- (5.80586,0.731431) -- (5.83517,0.731431) -- (5.83517,0.731791) --
 (5.86447,0.731791) -- (5.86447,0.730351) -- (5.89377,0.730351) -- (5.89377,0.725671) -- (5.92308,0.725671) -- (5.92308,0.73935) -- (5.95238,0.73935) -- (5.95238,0.731791) -- (5.98169,0.731791) -- (5.98169,0.727111) -- (6.01099,0.727111) --
 (6.01099,0.728551) -- (6.04029,0.728551) -- (6.04029,0.73467) -- (6.0696,0.73467) -- (6.0696,0.723152) -- (6.0989,0.723152) -- (6.0989,0.728191) -- (6.12821,0.728191) -- (6.12821,0.722792) -- (6.15751,0.722792) -- (6.15751,0.726391) --
 (6.18681,0.726391) -- (6.18681,0.720632) -- (6.21612,0.720632) -- (6.21612,0.717033) -- (6.24542,0.717033) -- (6.24542,0.722432) -- (6.27473,0.722432) -- (6.27473,0.715593) -- (6.30403,0.715593) -- (6.30403,0.719552) -- (6.33333,0.719552) --
 (6.33333,0.717753) -- (6.36264,0.717753) -- (6.36264,0.717393) -- (6.39194,0.717393) -- (6.39194,0.714153) -- (6.42125,0.714153) -- (6.42125,0.716673) -- (6.45055,0.716673) -- (6.45055,0.713433) -- (6.47985,0.713433) -- (6.47985,0.719192) --
 (6.50916,0.719192) -- (6.50916,0.712353) -- (6.53846,0.712353) -- (6.53846,0.713793) -- (6.56777,0.713793) -- (6.56777,0.715593) -- (6.59707,0.715593) -- (6.59707,0.710194) -- (6.62637,0.710194) -- (6.62637,0.718832) -- (6.65568,0.718832) --
 (6.65568,0.713793) -- (6.68498,0.713793) -- (6.68498,0.706954) -- (6.71429,0.706954) -- (6.71429,0.715593) -- (6.74359,0.715593) -- (6.74359,0.708394) -- (6.77289,0.708394) -- (6.77289,0.711633) -- (6.8022,0.711633) -- (6.8022,0.708394) --
 (6.8315,0.708394) -- (6.8315,0.705154) -- (6.86081,0.705154) -- (6.86081,0.707314) -- (6.89011,0.707314) -- (6.89011,0.711633) -- (6.91941,0.711633) -- (6.91941,0.709834) -- (6.94872,0.709834) -- (6.94872,0.707314) -- (6.97802,0.707314) --
 (6.97802,0.707314) -- (7.00733,0.707314) -- (7.00733,0.704434) -- (7.03663,0.704434) -- (7.03663,0.707314) -- (7.06593,0.707314) -- (7.06593,0.706954) -- (7.09524,0.706954) -- (7.09524,0.703354) -- (7.12454,0.703354) -- (7.12454,0.704794) --
 (7.15385,0.704794) -- (7.15385,0.708754) -- (7.18315,0.708754) -- (7.18315,0.703714) -- (7.21245,0.703714) -- (7.21245,0.709834) -- (7.24176,0.709834) -- (7.24176,0.699755) -- (7.27106,0.699755) -- (7.27106,0.702275) -- (7.30037,0.702275) --
 (7.30037,0.704074) -- (7.32967,0.704074) -- (7.32967,0.700835) -- (7.35897,0.700835) -- (7.35897,0.703354) -- (7.38828,0.703354) -- (7.38828,0.705514) -- (7.41758,0.705514) -- (7.41758,0.696155) -- (7.44689,0.696155) -- (7.44689,0.697595) --
 (7.47619,0.697595) -- (7.47619,0.708034) -- (7.50549,0.708034) -- (7.50549,0.700835) -- (7.5348,0.700835) -- (7.5348,0.700115) -- (7.5641,0.700115) -- (7.5641,0.698315) -- (7.59341,0.698315) -- (7.59341,0.697955) -- (7.62271,0.697955) --
 (7.62271,0.700835) -- (7.65201,0.700835) -- (7.65201,0.702635) -- (7.68132,0.702635) -- (7.68132,0.694716) -- (7.71062,0.694716) -- (7.71062,0.698315) -- (7.73993,0.698315) -- (7.73993,0.703354) -- (7.76923,0.703354) -- (7.76923,0.694716) --
 (7.79853,0.694716) -- (7.79853,0.699035) -- (7.82784,0.699035) -- (7.82784,0.697955) -- (7.85714,0.697955) -- (7.85714,0.697955) -- (7.88645,0.697955) -- (7.88645,0.699755) -- (7.91575,0.699755) -- (7.91575,0.701195) -- (7.94506,0.701195) --
 (7.94506,0.699755) -- (7.97436,0.699755) -- (7.97436,0.694356) -- (8.00366,0.694356) -- (8.00366,0.697955) -- (8.03297,0.697955) -- (8.03297,0.696515) -- (8.06227,0.696515) -- (8.06227,0.697595) -- (8.09157,0.697595) -- (8.09157,0.694356) --
 (8.12088,0.694356) -- (8.12088,0.699395) -- (8.15018,0.699395) -- (8.15018,0.695076) -- (8.17949,0.695076) -- (8.17949,0.695796) -- (8.20879,0.695796) -- (8.20879,0.695436) -- (8.2381,0.695436) -- (8.2381,0.697235) -- (8.2674,0.697235) --
 (8.2674,0.694716) -- (8.2967,0.694716) -- (8.2967,0.694356) -- (8.32601,0.694356) -- (8.32601,0.692556) -- (8.35531,0.692556) -- (8.35531,0.692196) -- (8.38461,0.692196) -- (8.38461,0.695076) -- (8.41392,0.695076) -- (8.41392,0.697235) --
 (8.44322,0.697235) -- (8.44322,0.694716) -- (8.47253,0.694716) -- (8.47253,0.696155) -- (8.50183,0.696155) -- (8.50183,0.694716) -- (8.53114,0.694716) -- (8.53114,0.696155) -- (8.56044,0.696155) -- (8.56044,0.692916) -- (8.58974,0.692916) --
 (8.58974,0.693276) -- (8.61905,0.693276) -- (8.61905,0.688596) -- (8.64835,0.688596) -- (8.64835,0.691476) -- (8.67766,0.691476) -- (8.67766,0.690756) -- (8.70696,0.690756) -- (8.70696,0.688956) -- (8.73626,0.688956) -- (8.73626,0.691476) --
 (8.76557,0.691476) -- (8.76557,0.691476) -- (8.79487,0.691476) -- (8.79487,0.695796) -- (8.82418,0.695796) -- (8.82418,0.692556) -- (8.85348,0.692556) -- (8.85348,0.692556) -- (8.88278,0.692556) -- (8.88278,0.693996) -- (8.91209,0.693996) --
 (8.91209,0.695076) -- (8.94139,0.695076) -- (8.94139,0.689676) -- (8.9707,0.689676) -- (8.9707,0.691116) -- (9,0.691116);
\definecolor{c}{rgb}{0,0,0};
\draw [c] (1,0.679598) -- (9,0.679598);
\draw [anchor= east] (9,0.299023) node[color=c, rotate=0]{$E_{T}^{\text{isol}}$ [GeV]};
\draw [c] (1,0.842701) -- (1,0.679598);
\draw [c] (1.2664,0.761149) -- (1.2664,0.679598);
\draw [c] (1.5328,0.761149) -- (1.5328,0.679598);
\draw [c] (1.7992,0.761149) -- (1.7992,0.679598);
\draw [c] (2.0656,0.761149) -- (2.0656,0.679598);
\draw [c] (2.332,0.842701) -- (2.332,0.679598);
\draw [c] (2.5984,0.761149) -- (2.5984,0.679598);
\draw [c] (2.8648,0.761149) -- (2.8648,0.679598);
\draw [c] (3.1312,0.761149) -- (3.1312,0.679598);
\draw [c] (3.3976,0.761149) -- (3.3976,0.679598);
\draw [c] (3.664,0.842701) -- (3.664,0.679598);
\draw [c] (3.9304,0.761149) -- (3.9304,0.679598);
\draw [c] (4.1968,0.761149) -- (4.1968,0.679598);
\draw [c] (4.4632,0.761149) -- (4.4632,0.679598);
\draw [c] (4.7296,0.761149) -- (4.7296,0.679598);
\draw [c] (4.996,0.842701) -- (4.996,0.679598);
\draw [c] (5.2624,0.761149) -- (5.2624,0.679598);
\draw [c] (5.5288,0.761149) -- (5.5288,0.679598);
\draw [c] (5.7952,0.761149) -- (5.7952,0.679598);
\draw [c] (6.0616,0.761149) -- (6.0616,0.679598);
\draw [c] (6.32801,0.842701) -- (6.32801,0.679598);
\draw [c] (6.59441,0.761149) -- (6.59441,0.679598);
\draw [c] (6.86081,0.761149) -- (6.86081,0.679598);
\draw [c] (7.12721,0.761149) -- (7.12721,0.679598);
\draw [c] (7.39361,0.761149) -- (7.39361,0.679598);
\draw [c] (7.66001,0.842701) -- (7.66001,0.679598);
\draw [c] (7.92641,0.761149) -- (7.92641,0.679598);
\draw [c] (8.19281,0.761149) -- (8.19281,0.679598);
\draw [c] (8.45921,0.761149) -- (8.45921,0.679598);
\draw [c] (8.72561,0.761149) -- (8.72561,0.679598);
\draw [c] (8.99201,0.842701) -- (8.99201,0.679598);
\draw [c] (8.99201,0.842701) -- (8.99201,0.679598);
\draw [anchor=base] (1,0.455331) node[color=c, rotate=0]{-5};
\draw [anchor=base] (2.332,0.455331) node[color=c, rotate=0]{0};
\draw [anchor=base] (3.664,0.455331) node[color=c, rotate=0]{5};
\draw [anchor=base] (4.996,0.455331) node[color=c, rotate=0]{10};
\draw [anchor=base] (6.32801,0.455331) node[color=c, rotate=0]{15};
\draw [anchor=base] (7.66001,0.455331) node[color=c, rotate=0]{20};
\draw [anchor=base] (8.99201,0.455331) node[color=c, rotate=0]{25};
\draw [c] (1,0.679598) -- (1,6.11638);
\draw [anchor= east] (0.0,6.11638) node[color=c, rotate=90]{Number of events};
\draw [c] (1.24,0.679598) -- (1,0.679598);
\draw [c] (1.12,0.859573) -- (1,0.859573);
\draw [c] (1.12,1.03955) -- (1,1.03955);
\draw [c] (1.12,1.21952) -- (1,1.21952);
\draw [c] (1.24,1.3995) -- (1,1.3995);
\draw [c] (1.12,1.57947) -- (1,1.57947);
\draw [c] (1.12,1.75945) -- (1,1.75945);
\draw [c] (1.12,1.93942) -- (1,1.93942);
\draw [c] (1.24,2.1194) -- (1,2.1194);
\draw [c] (1.12,2.29937) -- (1,2.29937);
\draw [c] (1.12,2.47935) -- (1,2.47935);
\draw [c] (1.12,2.65933) -- (1,2.65933);
\draw [c] (1.24,2.8393) -- (1,2.8393);
\draw [c] (1.12,3.01928) -- (1,3.01928);
\draw [c] (1.12,3.19925) -- (1,3.19925);
\draw [c] (1.12,3.37923) -- (1,3.37923);
\draw [c] (1.24,3.5592) -- (1,3.5592);
\draw [c] (1.12,3.73918) -- (1,3.73918);
\draw [c] (1.12,3.91915) -- (1,3.91915);
\draw [c] (1.12,4.09913) -- (1,4.09913);
\draw [c] (1.24,4.2791) -- (1,4.2791);
\draw [c] (1.12,4.45908) -- (1,4.45908);
\draw [c] (1.12,4.63905) -- (1,4.63905);
\draw [c] (1.12,4.81903) -- (1,4.81903);
\draw [c] (1.24,4.999) -- (1,4.999);
\draw [c] (1.12,5.17898) -- (1,5.17898);
\draw [c] (1.12,5.35895) -- (1,5.35895);
\draw [c] (1.12,5.53893) -- (1,5.53893);
\draw [c] (1.24,5.7189) -- (1,5.7189);
\draw [c] (1.24,5.7189) -- (1,5.7189);
\draw [c] (1.12,5.89888) -- (1,5.89888);
\draw [c] (1.12,6.07885) -- (1,6.07885);
\draw [anchor= east] (0.95,0.679598) node[color=c, rotate=0]{0};
\draw [anchor= east] (0.95,1.3995) node[color=c, rotate=0]{2000};
\draw [anchor= east] (0.95,2.1194) node[color=c, rotate=0]{4000};
\draw [anchor= east] (0.95,2.8393) node[color=c, rotate=0]{6000};
\draw [anchor= east] (0.95,3.5592) node[color=c, rotate=0]{8000};
\draw [anchor= east] (0.95,4.2791) node[color=c, rotate=0]{10000};
\draw [anchor= east] (0.95,4.999) node[color=c, rotate=0]{12000};
\draw [anchor= east] (0.95,5.7189) node[color=c, rotate=0]{14000};
\colorlet{c}{natgreen!50}
\draw [c] (1,0.679598) -- (1.0293,0.679598) -- (1.0293,0.679598) -- (1.05861,0.679598) -- (1.05861,0.679598) -- (1.08791,0.679598) -- (1.08791,0.679598) -- (1.11722,0.679598) -- (1.11722,0.679598) -- (1.14652,0.679598) -- (1.14652,0.679598) --
 (1.17582,0.679598) -- (1.17582,0.679598) -- (1.20513,0.679598) -- (1.20513,0.680546) -- (1.23443,0.680546) -- (1.23443,0.680546) -- (1.26374,0.680546) -- (1.26374,0.681359) -- (1.29304,0.681359) -- (1.29304,0.682307) -- (1.32234,0.682307) --
 (1.32234,0.683526) -- (1.35165,0.683526) -- (1.35165,0.684745) -- (1.38095,0.684745) -- (1.38095,0.683932) -- (1.41026,0.683932) -- (1.41026,0.684339) -- (1.43956,0.684339) -- (1.43956,0.688267) -- (1.46886,0.688267) -- (1.46886,0.688809) --
 (1.49817,0.688809) -- (1.49817,0.694227) -- (1.52747,0.694227) -- (1.52747,0.694633) -- (1.55678,0.694633) -- (1.55678,0.700457) -- (1.58608,0.700457) -- (1.58608,0.707366) -- (1.61538,0.707366) -- (1.61538,0.715899) -- (1.64469,0.715899) --
 (1.64469,0.72362) -- (1.67399,0.72362) -- (1.67399,0.733643) -- (1.7033,0.733643) -- (1.7033,0.745834) -- (1.7326,0.745834) -- (1.7326,0.760327) -- (1.7619,0.760327) -- (1.7619,0.777665) -- (1.79121,0.777665) -- (1.79121,0.800963) --
 (1.82051,0.800963) -- (1.82051,0.835368) -- (1.84982,0.835368) -- (1.84982,0.859885) -- (1.87912,0.859885) -- (1.87912,0.894425) -- (1.90842,0.894425) -- (1.90842,0.9333) -- (1.93773,0.9333) -- (1.93773,0.997098) -- (1.96703,0.997098) --
 (1.96703,1.04207) -- (1.99634,1.04207) -- (1.99634,1.10776) -- (2.02564,1.10776) -- (2.02564,1.17413) -- (2.05494,1.17413) -- (2.05494,1.23211) -- (2.08425,1.23211) -- (2.08425,1.31907) -- (2.11355,1.31907) -- (2.11355,1.41091) -- (2.14286,1.41091)
 -- (2.14286,1.46563) -- (2.17216,1.46563) -- (2.17216,1.54798) -- (2.20147,1.54798) -- (2.20147,1.60487) -- (2.23077,1.60487) -- (2.23077,1.64267) -- (2.26007,1.64267) -- (2.26007,1.69346) -- (2.28938,1.69346) -- (2.28938,1.7471) -- (2.31868,1.7471)
 -- (2.31868,1.76268) -- (2.34799,1.76268) -- (2.34799,1.76769) -- (2.37729,1.76769) -- (2.37729,1.81401) -- (2.40659,1.81401) -- (2.40659,1.84151) -- (2.4359,1.84151) -- (2.4359,1.82675) -- (2.4652,1.82675) -- (2.4652,1.85478) -- (2.49451,1.85478)
 -- (2.49451,1.8407) -- (2.52381,1.8407) -- (2.52381,1.83988) -- (2.55311,1.83988) -- (2.55311,1.82796) -- (2.58242,1.82796) -- (2.58242,1.85668) -- (2.61172,1.85668) -- (2.61172,1.84584) -- (2.64103,1.84584) -- (2.64103,1.82295) -- (2.67033,1.82295)
 -- (2.67033,1.79099) -- (2.69963,1.79099) -- (2.69963,1.79884) -- (2.72894,1.79884) -- (2.72894,1.78286) -- (2.75824,1.78286) -- (2.75824,1.75401) -- (2.78755,1.75401) -- (2.78755,1.75645) -- (2.81685,1.75645) -- (2.81685,1.73829) --
 (2.84615,1.73829) -- (2.84615,1.70877) -- (2.87546,1.70877) -- (2.87546,1.67368) -- (2.90476,1.67368) -- (2.90476,1.6745) -- (2.93407,1.6745) -- (2.93407,1.65161) -- (2.96337,1.65161) -- (2.96337,1.62763) -- (2.99267,1.62763) -- (2.99267,1.60691) --
 (3.02198,1.60691) -- (3.02198,1.59566) -- (3.05128,1.59566) -- (3.05128,1.55841) -- (3.08059,1.55841) -- (3.08059,1.53782) -- (3.10989,1.53782) -- (3.10989,1.51981) -- (3.13919,1.51981) -- (3.13919,1.47863) -- (3.1685,1.47863) -- (3.1685,1.47863) --
 (3.1978,1.47863) -- (3.1978,1.45764) -- (3.22711,1.45764) -- (3.22711,1.43163) -- (3.25641,1.43163) -- (3.25641,1.40197) -- (3.28571,1.40197) -- (3.28571,1.38693) -- (3.31502,1.38693) -- (3.31502,1.36648) -- (3.34432,1.36648) -- (3.34432,1.33844) --
 (3.37363,1.33844) -- (3.37363,1.32665) -- (3.40293,1.32665) -- (3.40293,1.30078) -- (3.43223,1.30078) -- (3.43223,1.28575) -- (3.46154,1.28575) -- (3.46154,1.27545) -- (3.49084,1.27545) -- (3.49084,1.22601) -- (3.52015,1.22601) -- (3.52015,1.22452)
 -- (3.54945,1.22452) -- (3.54945,1.21003) -- (3.57875,1.21003) -- (3.57875,1.19256) -- (3.60806,1.19256) -- (3.60806,1.18158) -- (3.63736,1.18158) -- (3.63736,1.14285) -- (3.66667,1.14285) -- (3.66667,1.14027) -- (3.69597,1.14027) --
 (3.69597,1.13011) -- (3.72527,1.13011) -- (3.72527,1.11819) -- (3.75458,1.11819) -- (3.75458,1.08243) -- (3.78388,1.08243) -- (3.78388,1.08487) -- (3.81319,1.08487) -- (3.81319,1.07525) -- (3.84249,1.07525) -- (3.84249,1.06225) -- (3.87179,1.06225)
 -- (3.87179,1.05914) -- (3.9011,1.05914) -- (3.9011,1.0315) -- (3.9304,1.0315) -- (3.9304,1.0376) -- (3.95971,1.0376) -- (3.95971,1.01985) -- (3.98901,1.01985) -- (3.98901,1.01037) -- (4.01831,1.01037) -- (4.01831,0.988565) -- (4.04762,0.988565) --
 (4.04762,0.991139) -- (4.07692,0.991139) -- (4.07692,0.977458) -- (4.10623,0.977458) -- (4.10623,0.954295) -- (4.13553,0.954295) -- (4.13553,0.941292) -- (4.16483,0.941292) -- (4.16483,0.93926) -- (4.19414,0.93926) -- (4.19414,0.940886) --
 (4.22344,0.940886) -- (4.22344,0.919078) -- (4.25275,0.919078) -- (4.25275,0.91068) -- (4.28205,0.91068) -- (4.28205,0.893613) -- (4.31136,0.893613) -- (4.31136,0.911763) -- (4.34066,0.911763) -- (4.34066,0.899031) -- (4.36996,0.899031) --
 (4.36996,0.890633) -- (4.39927,0.890633) -- (4.39927,0.878713) -- (4.42857,0.878713) -- (4.42857,0.881016) -- (4.45788,0.881016) -- (4.45788,0.874243) -- (4.48718,0.874243) -- (4.48718,0.863949) -- (4.51648,0.863949) -- (4.51648,0.865845) --
 (4.54579,0.865845) -- (4.54579,0.850403) -- (4.57509,0.850403) -- (4.57509,0.845933) -- (4.6044,0.845933) -- (4.6044,0.84187) -- (4.6337,0.84187) -- (4.6337,0.842683) -- (4.663,0.842683) -- (4.663,0.837671) -- (4.69231,0.837671) --
 (4.69231,0.831711) -- (4.72161,0.831711) -- (4.72161,0.827512) -- (4.75092,0.827512) -- (4.75092,0.813831) -- (4.78022,0.813831) -- (4.78022,0.819249) -- (4.80952,0.819249) -- (4.80952,0.805162) -- (4.83883,0.805162) -- (4.83883,0.810987) --
 (4.86813,0.810987) -- (4.86813,0.809497) -- (4.89744,0.809497) -- (4.89744,0.80462) -- (4.92674,0.80462) -- (4.92674,0.801776) -- (4.95604,0.801776) -- (4.95604,0.794191) -- (4.98535,0.794191) -- (4.98535,0.80313) -- (5.01465,0.80313) --
 (5.01465,0.788231) -- (5.04396,0.788231) -- (5.04396,0.78051) -- (5.07326,0.78051) -- (5.07326,0.781458) -- (5.10256,0.781458) -- (5.10256,0.78349) -- (5.13187,0.78349) -- (5.13187,0.786741) -- (5.16117,0.786741) -- (5.16117,0.775634) --
 (5.19048,0.775634) -- (5.19048,0.775363) -- (5.21978,0.775363) -- (5.21978,0.770757) -- (5.24908,0.770757) -- (5.24908,0.76561) -- (5.27839,0.76561) -- (5.27839,0.761817) -- (5.30769,0.761817) -- (5.30769,0.766694) -- (5.337,0.766694) --
 (5.337,0.759108) -- (5.3663,0.759108) -- (5.3663,0.762359) -- (5.3956,0.762359) -- (5.3956,0.756535) -- (5.42491,0.756535) -- (5.42491,0.757889) -- (5.45421,0.757889) -- (5.45421,0.751929) -- (5.48352,0.751929) -- (5.48352,0.750846) --
 (5.51282,0.750846) -- (5.51282,0.754368) -- (5.54212,0.754368) -- (5.54212,0.747595) -- (5.57143,0.747595) -- (5.57143,0.742854) -- (5.60073,0.742854) -- (5.60073,0.746918) -- (5.63004,0.746918) -- (5.63004,0.744886) -- (5.65934,0.744886) --
 (5.65934,0.741906) -- (5.68864,0.741906) -- (5.68864,0.737301) -- (5.71795,0.737301) -- (5.71795,0.740281) -- (5.74725,0.740281) -- (5.74725,0.734591) -- (5.77656,0.734591) -- (5.77656,0.734727) -- (5.80586,0.734727) -- (5.80586,0.736352) --
 (5.83517,0.736352) -- (5.83517,0.733914) -- (5.86447,0.733914) -- (5.86447,0.734185) -- (5.89377,0.734185) -- (5.89377,0.726464) -- (5.92308,0.726464) -- (5.92308,0.732424) -- (5.95238,0.732424) -- (5.95238,0.730392) -- (5.98169,0.730392) --
 (5.98169,0.726871) -- (6.01099,0.726871) -- (6.01099,0.725516) -- (6.04029,0.725516) -- (6.04029,0.721588) -- (6.0696,0.721588) -- (6.0696,0.727277) -- (6.0989,0.727277) -- (6.0989,0.72362) -- (6.12821,0.72362) -- (6.12821,0.72064) --
 (6.15751,0.72064) -- (6.15751,0.720504) -- (6.18681,0.720504) -- (6.18681,0.721859) -- (6.21612,0.721859) -- (6.21612,0.720775) -- (6.24542,0.720775) -- (6.24542,0.718879) -- (6.27473,0.718879) -- (6.27473,0.715222) -- (6.30403,0.715222) --
 (6.30403,0.717389) -- (6.33333,0.717389) -- (6.33333,0.718473) -- (6.36264,0.718473) -- (6.36264,0.713732) -- (6.39194,0.713732) -- (6.39194,0.714544) -- (6.42125,0.714544) -- (6.42125,0.717931) -- (6.45055,0.717931) -- (6.45055,0.709939) --
 (6.47985,0.709939) -- (6.47985,0.711158) -- (6.50916,0.711158) -- (6.50916,0.715357) -- (6.53846,0.715357) -- (6.53846,0.714544) -- (6.56777,0.714544) -- (6.56777,0.712919) -- (6.59707,0.712919) -- (6.59707,0.711158) -- (6.62637,0.711158) --
 (6.62637,0.711835) -- (6.65568,0.711835) -- (6.65568,0.707095) -- (6.68498,0.707095) -- (6.68498,0.709397) -- (6.71429,0.709397) -- (6.71429,0.710481) -- (6.74359,0.710481) -- (6.74359,0.706146) -- (6.77289,0.706146) -- (6.77289,0.709939) --
 (6.8022,0.709939) -- (6.8022,0.706824) -- (6.8315,0.706824) -- (6.8315,0.707095) -- (6.86081,0.707095) -- (6.86081,0.703979) -- (6.89011,0.703979) -- (6.89011,0.70276) -- (6.91941,0.70276) -- (6.91941,0.703437) -- (6.94872,0.703437) --
 (6.94872,0.703708) -- (6.97802,0.703708) -- (6.97802,0.701541) -- (7.00733,0.701541) -- (7.00733,0.704656) -- (7.03663,0.704656) -- (7.03663,0.703031) -- (7.06593,0.703031) -- (7.06593,0.700999) -- (7.09524,0.700999) -- (7.09524,0.703979) --
 (7.12454,0.703979) -- (7.12454,0.700728) -- (7.15385,0.700728) -- (7.15385,0.700864) -- (7.18315,0.700864) -- (7.18315,0.702354) -- (7.21245,0.702354) -- (7.21245,0.701676) -- (7.24176,0.701676) -- (7.24176,0.699645) -- (7.27106,0.699645) --
 (7.27106,0.699238) -- (7.30037,0.699238) -- (7.30037,0.700051) -- (7.32967,0.700051) -- (7.32967,0.698426) -- (7.35897,0.698426) -- (7.35897,0.698697) -- (7.38828,0.698697) -- (7.38828,0.698019) -- (7.41758,0.698019) -- (7.41758,0.699103) --
 (7.44689,0.699103) -- (7.44689,0.697207) -- (7.47619,0.697207) -- (7.47619,0.695988) -- (7.50549,0.695988) -- (7.50549,0.696394) -- (7.5348,0.696394) -- (7.5348,0.693956) -- (7.5641,0.693956) -- (7.5641,0.696529) -- (7.59341,0.696529) --
 (7.59341,0.695852) -- (7.62271,0.695852) -- (7.62271,0.695988) -- (7.65201,0.695988) -- (7.65201,0.696123) -- (7.68132,0.696123) -- (7.68132,0.694633) -- (7.71062,0.694633) -- (7.71062,0.693956) -- (7.73993,0.693956) -- (7.73993,0.695717) --
 (7.76923,0.695717) -- (7.76923,0.69531) -- (7.79853,0.69531) -- (7.79853,0.692601) -- (7.82784,0.692601) -- (7.82784,0.693008) -- (7.85714,0.693008) -- (7.85714,0.694633) -- (7.88645,0.694633) -- (7.88645,0.693414) -- (7.91575,0.693414) --
 (7.91575,0.692601) -- (7.94506,0.692601) -- (7.94506,0.694768) -- (7.97436,0.694768) -- (7.97436,0.692601) -- (8.00366,0.692601) -- (8.00366,0.69233) -- (8.03297,0.69233) -- (8.03297,0.691924) -- (8.06227,0.691924) -- (8.06227,0.692466) --
 (8.09157,0.692466) -- (8.09157,0.69382) -- (8.12088,0.69382) -- (8.12088,0.691518) -- (8.15018,0.691518) -- (8.15018,0.694091) -- (8.17949,0.694091) -- (8.17949,0.69084) -- (8.20879,0.69084) -- (8.20879,0.691518) -- (8.2381,0.691518) --
 (8.2381,0.689215) -- (8.2674,0.689215) -- (8.2674,0.692737) -- (8.2967,0.692737) -- (8.2967,0.692737) -- (8.32601,0.692737) -- (8.32601,0.689892) -- (8.35531,0.689892) -- (8.35531,0.691382) -- (8.38461,0.691382) -- (8.38461,0.690163) --
 (8.41392,0.690163) -- (8.41392,0.69084) -- (8.44322,0.69084) -- (8.44322,0.689079) -- (8.47253,0.689079) -- (8.47253,0.69084) -- (8.50183,0.69084) -- (8.50183,0.690298) -- (8.53114,0.690298) -- (8.53114,0.690569) -- (8.56044,0.690569) --
 (8.56044,0.689757) -- (8.58974,0.689757) -- (8.58974,0.68786) -- (8.61905,0.68786) -- (8.61905,0.689215) -- (8.64835,0.689215) -- (8.64835,0.689079) -- (8.67766,0.689079) -- (8.67766,0.686912) -- (8.70696,0.686912) -- (8.70696,0.687319) --
 (8.73626,0.687319) -- (8.73626,0.688809) -- (8.76557,0.688809) -- (8.76557,0.68786) -- (8.79487,0.68786) -- (8.79487,0.68786) -- (8.82418,0.68786) -- (8.82418,0.688673) -- (8.85348,0.688673) -- (8.85348,0.687996) -- (8.88278,0.687996) --
 (8.88278,0.687319) -- (8.91209,0.687319) -- (8.91209,0.686912) -- (8.94139,0.686912) -- (8.94139,0.68786) -- (8.9707,0.68786) -- (8.9707,0.688538) -- (9,0.688538);
\colorlet{c}{natcomp}
\draw [c] (1,0.680318) -- (1.0293,0.680318) -- (1.0293,0.679598) -- (1.05861,0.679598) -- (1.05861,0.679598) -- (1.08791,0.679598) -- (1.08791,0.680318) -- (1.11722,0.680318) -- (1.11722,0.679598) -- (1.14652,0.679598) -- (1.14652,0.679598) --
 (1.17582,0.679598) -- (1.17582,0.680318) -- (1.20513,0.680318) -- (1.20513,0.682837) -- (1.23443,0.682837) -- (1.23443,0.680318) -- (1.26374,0.680318) -- (1.26374,0.683197) -- (1.29304,0.683197) -- (1.29304,0.681757) -- (1.32234,0.681757) --
 (1.32234,0.685717) -- (1.35165,0.685717) -- (1.35165,0.685357) -- (1.38095,0.685357) -- (1.38095,0.688956) -- (1.41026,0.688956) -- (1.41026,0.687877) -- (1.43956,0.687877) -- (1.43956,0.691836) -- (1.46886,0.691836) -- (1.46886,0.692196) --
 (1.49817,0.692196) -- (1.49817,0.700475) -- (1.52747,0.700475) -- (1.52747,0.708754) -- (1.55678,0.708754) -- (1.55678,0.710913) -- (1.58608,0.710913) -- (1.58608,0.717033) -- (1.61538,0.717033) -- (1.61538,0.73863) -- (1.64469,0.73863) --
 (1.64469,0.744389) -- (1.67399,0.744389) -- (1.67399,0.758787) -- (1.7033,0.758787) -- (1.7033,0.778584) -- (1.7326,0.778584) -- (1.7326,0.8117) -- (1.7619,0.8117) -- (1.7619,0.837256) -- (1.79121,0.837256) -- (1.79121,0.88117) -- (1.82051,0.88117)
 -- (1.82051,0.911046) -- (1.84982,0.911046) -- (1.84982,0.963959) -- (1.87912,0.963959) -- (1.87912,1.01687) -- (1.90842,1.01687) -- (1.90842,1.07518) -- (1.93773,1.07518) -- (1.93773,1.16121) -- (1.96703,1.16121) -- (1.96703,1.2476) --
 (1.99634,1.2476) -- (1.99634,1.34335) -- (2.02564,1.34335) -- (2.02564,1.41786) -- (2.05494,1.41786) -- (2.05494,1.57551) -- (2.08425,1.57551) -- (2.08425,1.66406) -- (2.11355,1.66406) -- (2.11355,1.78177) -- (2.14286,1.78177) -- (2.14286,1.83828)
 -- (2.17216,1.83828) -- (2.17216,1.90127) -- (2.20147,1.90127) -- (2.20147,1.94338) -- (2.23077,1.94338) -- (2.23077,1.90451) -- (2.26007,1.90451) -- (2.26007,1.88831) -- (2.28938,1.88831) -- (2.28938,1.848) -- (2.31868,1.848) -- (2.31868,1.78105)
 -- (2.34799,1.78105) -- (2.34799,1.73065) -- (2.37729,1.73065) -- (2.37729,1.71985) -- (2.40659,1.71985) -- (2.40659,1.6547) -- (2.4359,1.6547) -- (2.4359,1.58127) -- (2.4652,1.58127) -- (2.4652,1.55896) -- (2.49451,1.55896) -- (2.49451,1.49633) --
 (2.52381,1.49633) -- (2.52381,1.45529) -- (2.55311,1.45529) -- (2.55311,1.48049) -- (2.58242,1.48049) -- (2.58242,1.37682) -- (2.61172,1.37682) -- (2.61172,1.37862) -- (2.64103,1.37862) -- (2.64103,1.34047) -- (2.67033,1.34047) -- (2.67033,1.30555)
 -- (2.69963,1.30555) -- (2.69963,1.29511) -- (2.72894,1.29511) -- (2.72894,1.23248) -- (2.75824,1.23248) -- (2.75824,1.2116) -- (2.78755,1.2116) -- (2.78755,1.17597) -- (2.81685,1.17597) -- (2.81685,1.17309) -- (2.84615,1.17309) -- (2.84615,1.13709)
 -- (2.87546,1.13709) -- (2.87546,1.13997) -- (2.90476,1.13997) -- (2.90476,1.11586) -- (2.93407,1.11586) -- (2.93407,1.07446) -- (2.96337,1.07446) -- (2.96337,1.05467) -- (2.99267,1.05467) -- (2.99267,1.06906) -- (3.02198,1.06906) --
 (3.02198,1.02623) -- (3.05128,1.02623) -- (3.05128,1.02479) -- (3.08059,1.02479) -- (3.08059,1.00679) -- (3.10989,1.00679) -- (3.10989,0.981596) -- (3.13919,0.981596) -- (3.13919,0.977997) -- (3.1685,0.977997) -- (3.1685,0.961439) --
 (3.1978,0.961439) -- (3.1978,0.959639) -- (3.22711,0.959639) -- (3.22711,0.940562) -- (3.25641,0.940562) -- (3.25641,0.929403) -- (3.28571,0.929403) -- (3.28571,0.929403) -- (3.31502,0.929403) -- (3.31502,0.909246) -- (3.34432,0.909246) --
 (3.34432,0.907806) -- (3.37363,0.907806) -- (3.37363,0.904207) -- (3.40293,0.904207) -- (3.40293,0.886569) -- (3.43223,0.886569) -- (3.43223,0.871451) -- (3.46154,0.871451) -- (3.46154,0.866052) -- (3.49084,0.866052) -- (3.49084,0.874331) --
 (3.52015,0.874331) -- (3.52015,0.868212) -- (3.54945,0.868212) -- (3.54945,0.853454) -- (3.57875,0.853454) -- (3.57875,0.836536) -- (3.60806,0.836536) -- (3.60806,0.851294) -- (3.63736,0.851294) -- (3.63736,0.836536) -- (3.66667,0.836536) --
 (3.66667,0.834376) -- (3.69597,0.834376) -- (3.69597,0.823578) -- (3.72527,0.823578) -- (3.72527,0.803781) -- (3.75458,0.803781) -- (3.75458,0.81062) -- (3.78388,0.81062) -- (3.78388,0.8117) -- (3.81319,0.8117) -- (3.81319,0.814219) --
 (3.84249,0.814219) -- (3.84249,0.792262) -- (3.87179,0.792262) -- (3.87179,0.783264) -- (3.9011,0.783264) -- (3.9011,0.792622) -- (3.9304,0.792622) -- (3.9304,0.774985) -- (3.95971,0.774985) -- (3.95971,0.774985) -- (3.98901,0.774985) --
 (3.98901,0.785783) -- (4.01831,0.785783) -- (4.01831,0.768146) -- (4.04762,0.768146) -- (4.04762,0.772105) -- (4.07692,0.772105) -- (4.07692,0.765266) -- (4.10623,0.765266) -- (4.10623,0.768865) -- (4.13553,0.768865) -- (4.13553,0.769225) --
 (4.16483,0.769225) -- (4.16483,0.769585) -- (4.19414,0.769585) -- (4.19414,0.762386) -- (4.22344,0.762386) -- (4.22344,0.757347) -- (4.25275,0.757347) -- (4.25275,0.747268) -- (4.28205,0.747268) -- (4.28205,0.749788) -- (4.31136,0.749788) --
 (4.31136,0.742949) -- (4.34066,0.742949) -- (4.34066,0.73899) -- (4.36996,0.73899) -- (4.36996,0.745829) -- (4.39927,0.745829) -- (4.39927,0.73719) -- (4.42857,0.73719) -- (4.42857,0.745829) -- (4.45788,0.745829) -- (4.45788,0.727471) --
 (4.48718,0.727471) -- (4.48718,0.742589) -- (4.51648,0.742589) -- (4.51648,0.727471) -- (4.54579,0.727471) -- (4.54579,0.730711) -- (4.57509,0.730711) -- (4.57509,0.731791) -- (4.6044,0.731791) -- (4.6044,0.727471) -- (4.6337,0.727471) --
 (4.6337,0.728191) -- (4.663,0.728191) -- (4.663,0.726391) -- (4.69231,0.726391) -- (4.69231,0.730351) -- (4.72161,0.730351) -- (4.72161,0.725311) -- (4.75092,0.725311) -- (4.75092,0.721352) -- (4.78022,0.721352) -- (4.78022,0.728911) --
 (4.80952,0.728911) -- (4.80952,0.719192) -- (4.83883,0.719192) -- (4.83883,0.711633) -- (4.86813,0.711633) -- (4.86813,0.720992) -- (4.89744,0.720992) -- (4.89744,0.711273) -- (4.92674,0.711273) -- (4.92674,0.713073) -- (4.95604,0.713073) --
 (4.95604,0.711273) -- (4.98535,0.711273) -- (4.98535,0.718112) -- (5.01465,0.718112) -- (5.01465,0.715233) -- (5.04396,0.715233) -- (5.04396,0.715953) -- (5.07326,0.715953) -- (5.07326,0.716313) -- (5.10256,0.716313) -- (5.10256,0.711273) --
 (5.13187,0.711273) -- (5.13187,0.714153) -- (5.16117,0.714153) -- (5.16117,0.710194) -- (5.19048,0.710194) -- (5.19048,0.707314) -- (5.21978,0.707314) -- (5.21978,0.706594) -- (5.24908,0.706594) -- (5.24908,0.712713) -- (5.27839,0.712713) --
 (5.27839,0.702635) -- (5.30769,0.702635) -- (5.30769,0.705874) -- (5.337,0.705874) -- (5.337,0.706234) -- (5.3663,0.706234) -- (5.3663,0.704074) -- (5.3956,0.704074) -- (5.3956,0.714153) -- (5.42491,0.714153) -- (5.42491,0.710553) --
 (5.45421,0.710553) -- (5.45421,0.698315) -- (5.48352,0.698315) -- (5.48352,0.700835) -- (5.51282,0.700835) -- (5.51282,0.705514) -- (5.54212,0.705514) -- (5.54212,0.703714) -- (5.57143,0.703714) -- (5.57143,0.703354) -- (5.60073,0.703354) --
 (5.60073,0.704434) -- (5.63004,0.704434) -- (5.63004,0.701195) -- (5.65934,0.701195) -- (5.65934,0.702275) -- (5.68864,0.702275) -- (5.68864,0.702275) -- (5.71795,0.702275) -- (5.71795,0.700475) -- (5.74725,0.700475) -- (5.74725,0.698315) --
 (5.77656,0.698315) -- (5.77656,0.701915) -- (5.80586,0.701915) -- (5.80586,0.695076) -- (5.83517,0.695076) -- (5.83517,0.701915) -- (5.86447,0.701915) -- (5.86447,0.696515) -- (5.89377,0.696515) -- (5.89377,0.696155) -- (5.92308,0.696155) --
 (5.92308,0.701195) -- (5.95238,0.701195) -- (5.95238,0.696155) -- (5.98169,0.696155) -- (5.98169,0.693996) -- (6.01099,0.693996) -- (6.01099,0.699035) -- (6.04029,0.699035) -- (6.04029,0.702275) -- (6.0696,0.702275) -- (6.0696,0.697235) --
 (6.0989,0.697235) -- (6.0989,0.690396) -- (6.12821,0.690396) -- (6.12821,0.697595) -- (6.15751,0.697595) -- (6.15751,0.694716) -- (6.18681,0.694716) -- (6.18681,0.694356) -- (6.21612,0.694356) -- (6.21612,0.697235) -- (6.24542,0.697235) --
 (6.24542,0.695436) -- (6.27473,0.695436) -- (6.27473,0.693996) -- (6.30403,0.693996) -- (6.30403,0.693636) -- (6.33333,0.693636) -- (6.33333,0.694356) -- (6.36264,0.694356) -- (6.36264,0.693996) -- (6.39194,0.693996) -- (6.39194,0.693636) --
 (6.42125,0.693636) -- (6.42125,0.693636) -- (6.45055,0.693636) -- (6.45055,0.690756) -- (6.47985,0.690756) -- (6.47985,0.692556) -- (6.50916,0.692556) -- (6.50916,0.688956) -- (6.53846,0.688956) -- (6.53846,0.695436) -- (6.56777,0.695436) --
 (6.56777,0.690756) -- (6.59707,0.690756) -- (6.59707,0.692916) -- (6.62637,0.692916) -- (6.62637,0.692916) -- (6.65568,0.692916) -- (6.65568,0.693996) -- (6.68498,0.693996) -- (6.68498,0.691116) -- (6.71429,0.691116) -- (6.71429,0.692196) --
 (6.74359,0.692196) -- (6.74359,0.692556) -- (6.77289,0.692556) -- (6.77289,0.693996) -- (6.8022,0.693996) -- (6.8022,0.693636) -- (6.8315,0.693636) -- (6.8315,0.688956) -- (6.86081,0.688956) -- (6.86081,0.693996) -- (6.89011,0.693996) --
 (6.89011,0.689676) -- (6.91941,0.689676) -- (6.91941,0.691836) -- (6.94872,0.691836) -- (6.94872,0.692556) -- (6.97802,0.692556) -- (6.97802,0.689316) -- (7.00733,0.689316) -- (7.00733,0.689676) -- (7.03663,0.689676) -- (7.03663,0.690756) --
 (7.06593,0.690756) -- (7.06593,0.690036) -- (7.09524,0.690036) -- (7.09524,0.686077) -- (7.12454,0.686077) -- (7.12454,0.693276) -- (7.15385,0.693276) -- (7.15385,0.685717) -- (7.18315,0.685717) -- (7.18315,0.693276) -- (7.21245,0.693276) --
 (7.21245,0.687877) -- (7.24176,0.687877) -- (7.24176,0.689316) -- (7.27106,0.689316) -- (7.27106,0.689316) -- (7.30037,0.689316) -- (7.30037,0.688237) -- (7.32967,0.688237) -- (7.32967,0.689316) -- (7.35897,0.689316) -- (7.35897,0.689316) --
 (7.38828,0.689316) -- (7.38828,0.687157) -- (7.41758,0.687157) -- (7.41758,0.692556) -- (7.44689,0.692556) -- (7.44689,0.686797) -- (7.47619,0.686797) -- (7.47619,0.685357) -- (7.50549,0.685357) -- (7.50549,0.685357) -- (7.5348,0.685357) --
 (7.5348,0.683917) -- (7.5641,0.683917) -- (7.5641,0.690756) -- (7.59341,0.690756) -- (7.59341,0.690396) -- (7.62271,0.690396) -- (7.62271,0.687877) -- (7.65201,0.687877) -- (7.65201,0.684997) -- (7.68132,0.684997) -- (7.68132,0.688956) --
 (7.71062,0.688956) -- (7.71062,0.687157) -- (7.73993,0.687157) -- (7.73993,0.687517) -- (7.76923,0.687517) -- (7.76923,0.689316) -- (7.79853,0.689316) -- (7.79853,0.686437) -- (7.82784,0.686437) -- (7.82784,0.686797) -- (7.85714,0.686797) --
 (7.85714,0.687157) -- (7.88645,0.687157) -- (7.88645,0.691116) -- (7.91575,0.691116) -- (7.91575,0.688237) -- (7.94506,0.688237) -- (7.94506,0.686437) -- (7.97436,0.686437) -- (7.97436,0.688596) -- (8.00366,0.688596) -- (8.00366,0.686077) --
 (8.03297,0.686077) -- (8.03297,0.687517) -- (8.06227,0.687517) -- (8.06227,0.686077) -- (8.09157,0.686077) -- (8.09157,0.687877) -- (8.12088,0.687877) -- (8.12088,0.688237) -- (8.15018,0.688237) -- (8.15018,0.684277) -- (8.17949,0.684277) --
 (8.17949,0.685357) -- (8.20879,0.685357) -- (8.20879,0.684637) -- (8.2381,0.684637) -- (8.2381,0.685357) -- (8.2674,0.685357) -- (8.2674,0.684997) -- (8.2967,0.684997) -- (8.2967,0.684637) -- (8.32601,0.684637) -- (8.32601,0.687157) --
 (8.35531,0.687157) -- (8.35531,0.684637) -- (8.38461,0.684637) -- (8.38461,0.684277) -- (8.41392,0.684277) -- (8.41392,0.685717) -- (8.44322,0.685717) -- (8.44322,0.683917) -- (8.47253,0.683917) -- (8.47253,0.683557) -- (8.50183,0.683557) --
 (8.50183,0.684277) -- (8.53114,0.684277) -- (8.53114,0.684997) -- (8.56044,0.684997) -- (8.56044,0.685357) -- (8.58974,0.685357) -- (8.58974,0.684277) -- (8.61905,0.684277) -- (8.61905,0.686077) -- (8.64835,0.686077) -- (8.64835,0.686437) --
 (8.67766,0.686437) -- (8.67766,0.683557) -- (8.70696,0.683557) -- (8.70696,0.684277) -- (8.73626,0.684277) -- (8.73626,0.683197) -- (8.76557,0.683197) -- (8.76557,0.684637) -- (8.79487,0.684637) -- (8.79487,0.686437) -- (8.82418,0.686437) --
 (8.82418,0.684277) -- (8.85348,0.684277) -- (8.85348,0.684277) -- (8.88278,0.684277) -- (8.88278,0.683557) -- (8.91209,0.683557) -- (8.91209,0.684997) -- (8.94139,0.684997) -- (8.94139,0.684997) -- (8.9707,0.684997) -- (8.9707,0.684997) --
 (9,0.684997) -- (9,0.685357) -- (9,0.685357);
\colorlet{c}{natcomp!50}
\draw [c] (1,0.679598) -- (1.0293,0.679598) -- (1.0293,0.679598) -- (1.05861,0.679598) -- (1.05861,0.679598) -- (1.08791,0.679598) -- (1.08791,0.679598) -- (1.11722,0.679598) -- (1.11722,0.679598) -- (1.14652,0.679598) -- (1.14652,0.679598) --
 (1.17582,0.679598) -- (1.17582,0.679598) -- (1.20513,0.679598) -- (1.20513,0.679598) -- (1.23443,0.679598) -- (1.23443,0.679598) -- (1.26374,0.679598) -- (1.26374,0.680208) -- (1.29304,0.680208) -- (1.29304,0.680819) -- (1.32234,0.680819) --
 (1.32234,0.680295) -- (1.35165,0.680295) -- (1.35165,0.68108) -- (1.38095,0.68108) -- (1.38095,0.680557) -- (1.41026,0.680557) -- (1.41026,0.681429) -- (1.43956,0.681429) -- (1.43956,0.681952) -- (1.46886,0.681952) -- (1.46886,0.681429) --
 (1.49817,0.681429) -- (1.49817,0.682301) -- (1.52747,0.682301) -- (1.52747,0.683347) -- (1.55678,0.683347) -- (1.55678,0.684743) -- (1.58608,0.684743) -- (1.58608,0.686225) -- (1.61538,0.686225) -- (1.61538,0.689015) -- (1.64469,0.689015) --
 (1.64469,0.69198) -- (1.67399,0.69198) -- (1.67399,0.692939) -- (1.7033,0.692939) -- (1.7033,0.694335) -- (1.7326,0.694335) -- (1.7326,0.70166) -- (1.7619,0.70166) -- (1.7619,0.704886) -- (1.79121,0.704886) -- (1.79121,0.709944) --
 (1.82051,0.709944) -- (1.82051,0.715001) -- (1.84982,0.715001) -- (1.84982,0.728081) -- (1.87912,0.728081) -- (1.87912,0.738284) -- (1.90842,0.738284) -- (1.90842,0.744388) -- (1.93773,0.744388) -- (1.93773,0.757206) -- (1.96703,0.757206) --
 (1.96703,0.777786) -- (1.99634,0.777786) -- (1.99634,0.79322) -- (2.02564,0.79322) -- (2.02564,0.812666) -- (2.05494,0.812666) -- (2.05494,0.84057) -- (2.08425,0.84057) -- (2.08425,0.859406) -- (2.11355,0.859406) -- (2.11355,0.877107) --
 (2.14286,0.877107) -- (2.14286,0.901436) -- (2.17216,0.901436) -- (2.17216,0.918179) -- (2.20147,0.918179) -- (2.20147,0.937799) -- (2.23077,0.937799) -- (2.23077,0.951664) -- (2.26007,0.951664) -- (2.26007,0.971807) -- (2.28938,0.971807) --
 (2.28938,0.975819) -- (2.31868,0.975819) -- (2.31868,0.983231) -- (2.34799,0.983231) -- (2.34799,0.998316) -- (2.37729,0.998316) -- (2.37729,1.00957) -- (2.40659,1.00957) -- (2.40659,1.00529) -- (2.4359,1.00529) -- (2.4359,1.01087) --
 (2.4652,1.01087) -- (2.4652,1.02134) -- (2.49451,1.02134) -- (2.49451,1.01567) -- (2.52381,1.01567) -- (2.52381,1.02378) -- (2.55311,1.02378) -- (2.55311,1.01942) -- (2.58242,1.01942) -- (2.58242,1.02666) -- (2.61172,1.02666) -- (2.61172,1.01602) --
 (2.64103,1.01602) -- (2.64103,1.0189) -- (2.67033,1.0189) -- (2.67033,1.01637) -- (2.69963,1.01637) -- (2.69963,1.0209) -- (2.72894,1.0209) -- (2.72894,1.00677) -- (2.75824,1.00677) -- (2.75824,1.00756) -- (2.78755,1.00756) -- (2.78755,0.99352) --
 (2.81685,0.99352) -- (2.81685,0.993084) -- (2.84615,0.993084) -- (2.84615,0.991602) -- (2.87546,0.991602) -- (2.87546,0.985498) -- (2.90476,0.985498) -- (2.90476,0.981574) -- (2.93407,0.981574) -- (2.93407,0.969976) -- (2.96337,0.969976) --
 (2.96337,0.963262) -- (2.99267,0.963262) -- (2.99267,0.965267) -- (3.02198,0.965267) -- (3.02198,0.948961) -- (3.05128,0.948961) -- (3.05128,0.946955) -- (3.08059,0.946955) -- (3.08059,0.937886) -- (3.10989,0.937886) -- (3.10989,0.926637) --
 (3.13919,0.926637) -- (3.13919,0.92498) -- (3.1685,0.92498) -- (3.1685,0.915824) -- (3.1978,0.915824) -- (3.1978,0.90475) -- (3.22711,0.90475) -- (3.22711,0.908325) -- (3.25641,0.908325) -- (3.25641,0.891408) -- (3.28571,0.891408) --
 (3.28571,0.891495) -- (3.31502,0.891495) -- (3.31502,0.887223) -- (3.34432,0.887223) -- (3.34432,0.880159) -- (3.37363,0.880159) -- (3.37363,0.875102) -- (3.40293,0.875102) -- (3.40293,0.862806) -- (3.43223,0.862806) -- (3.43223,0.860801) --
 (3.46154,0.860801) -- (3.46154,0.86176) -- (3.49084,0.86176) -- (3.49084,0.854174) -- (3.52015,0.854174) -- (3.52015,0.851296) -- (3.54945,0.851296) -- (3.54945,0.840919) -- (3.57875,0.840919) -- (3.57875,0.841965) -- (3.60806,0.841965) --
 (3.60806,0.836559) -- (3.63736,0.836559) -- (3.63736,0.831414) -- (3.66667,0.831414) -- (3.66667,0.825659) -- (3.69597,0.825659) -- (3.69597,0.820252) -- (3.72527,0.820252) -- (3.72527,0.814061) -- (3.75458,0.814061) -- (3.75458,0.806126) --
 (3.78388,0.806126) -- (3.78388,0.809178) -- (3.81319,0.809178) -- (3.81319,0.806911) -- (3.84249,0.806911) -- (3.84249,0.800458) -- (3.87179,0.800458) -- (3.87179,0.796359) -- (3.9011,0.796359) -- (3.9011,0.79479) -- (3.9304,0.79479) --
 (3.9304,0.789296) -- (3.95971,0.789296) -- (3.95971,0.786767) -- (3.98901,0.786767) -- (3.98901,0.780227) -- (4.01831,0.780227) -- (4.01831,0.774995) -- (4.04762,0.774995) -- (4.04762,0.778745) -- (4.07692,0.778745) -- (4.07692,0.778135) --
 (4.10623,0.778135) -- (4.10623,0.774647) -- (4.13553,0.774647) -- (4.13553,0.770722) -- (4.16483,0.770722) -- (4.16483,0.764444) -- (4.19414,0.764444) -- (4.19414,0.768281) -- (4.22344,0.768281) -- (4.22344,0.759212) -- (4.25275,0.759212) --
 (4.25275,0.759299) -- (4.28205,0.759299) -- (4.28205,0.7518) -- (4.31136,0.7518) -- (4.31136,0.753108) -- (4.34066,0.753108) -- (4.34066,0.753195) -- (4.36996,0.753195) -- (4.36996,0.753893) -- (4.39927,0.753893) -- (4.39927,0.747265) --
 (4.42857,0.747265) -- (4.42857,0.745783) -- (4.45788,0.745783) -- (4.45788,0.7409) -- (4.48718,0.7409) -- (4.48718,0.742993) -- (4.51648,0.742993) -- (4.51648,0.736889) -- (4.54579,0.736889) -- (4.54579,0.733052) -- (4.57509,0.733052) --
 (4.57509,0.734447) -- (4.6044,0.734447) -- (4.6044,0.730436) -- (4.6337,0.730436) -- (4.6337,0.726599) -- (4.663,0.726599) -- (4.663,0.729825) -- (4.69231,0.729825) -- (4.69231,0.730087) -- (4.72161,0.730087) -- (4.72161,0.733313) --
 (4.75092,0.733313) -- (4.75092,0.731133) -- (4.78022,0.731133) -- (4.78022,0.722849) -- (4.80952,0.722849) -- (4.80952,0.723373) -- (4.83883,0.723373) -- (4.83883,0.722937) -- (4.86813,0.722937) -- (4.86813,0.720756) -- (4.89744,0.720756) --
 (4.89744,0.720233) -- (4.92674,0.720233) -- (4.92674,0.719361) -- (4.95604,0.719361) -- (4.95604,0.717879) -- (4.98535,0.717879) -- (4.98535,0.716571) -- (5.01465,0.716571) -- (5.01465,0.713868) -- (5.04396,0.713868) -- (5.04396,0.713693) --
 (5.07326,0.713693) -- (5.07326,0.711252) -- (5.10256,0.711252) -- (5.10256,0.715612) -- (5.13187,0.715612) -- (5.13187,0.710205) -- (5.16117,0.710205) -- (5.16117,0.710118) -- (5.19048,0.710118) -- (5.19048,0.71099) -- (5.21978,0.71099) --
 (5.21978,0.709856) -- (5.24908,0.709856) -- (5.24908,0.706368) -- (5.27839,0.706368) -- (5.27839,0.708897) -- (5.30769,0.708897) -- (5.30769,0.708287) -- (5.337,0.708287) -- (5.337,0.705845) -- (5.3663,0.705845) -- (5.3663,0.708984) --
 (5.3956,0.708984) -- (5.3956,0.705409) -- (5.42491,0.705409) -- (5.42491,0.707066) -- (5.45421,0.707066) -- (5.45421,0.702968) -- (5.48352,0.702968) -- (5.48352,0.704014) -- (5.51282,0.704014) -- (5.51282,0.704537) -- (5.54212,0.704537) --
 (5.54212,0.702619) -- (5.57143,0.702619) -- (5.57143,0.704276) -- (5.60073,0.704276) -- (5.60073,0.703927) -- (5.63004,0.703927) -- (5.63004,0.700526) -- (5.65934,0.700526) -- (5.65934,0.700788) -- (5.68864,0.700788) -- (5.68864,0.699828) --
 (5.71795,0.699828) -- (5.71795,0.699392) -- (5.74725,0.699392) -- (5.74725,0.699131) -- (5.77656,0.699131) -- (5.77656,0.697735) -- (5.80586,0.697735) -- (5.80586,0.697299) -- (5.83517,0.697299) -- (5.83517,0.698956) -- (5.86447,0.698956) --
 (5.86447,0.698171) -- (5.89377,0.698171) -- (5.89377,0.696689) -- (5.92308,0.696689) -- (5.92308,0.696863) -- (5.95238,0.696863) -- (5.95238,0.69573) -- (5.98169,0.69573) -- (5.98169,0.696951) -- (6.01099,0.696951) -- (6.01099,0.695555) --
 (6.04029,0.695555) -- (6.04029,0.695294) -- (6.0696,0.695294) -- (6.0696,0.694771) -- (6.0989,0.694771) -- (6.0989,0.695207) -- (6.12821,0.695207) -- (6.12821,0.693288) -- (6.15751,0.693288) -- (6.15751,0.691719) -- (6.18681,0.691719) --
 (6.18681,0.694771) -- (6.21612,0.694771) -- (6.21612,0.693986) -- (6.24542,0.693986) -- (6.24542,0.693986) -- (6.27473,0.693986) -- (6.27473,0.693637) -- (6.30403,0.693637) -- (6.30403,0.69198) -- (6.33333,0.69198) -- (6.33333,0.691631) --
 (6.36264,0.691631) -- (6.36264,0.691195) -- (6.39194,0.691195) -- (6.39194,0.692591) -- (6.42125,0.692591) -- (6.42125,0.691021) -- (6.45055,0.691021) -- (6.45055,0.691195) -- (6.47985,0.691195) -- (6.47985,0.692678) -- (6.50916,0.692678) --
 (6.50916,0.691195) -- (6.53846,0.691195) -- (6.53846,0.691893) -- (6.56777,0.691893) -- (6.56777,0.689713) -- (6.59707,0.689713) -- (6.59707,0.691457) -- (6.62637,0.691457) -- (6.62637,0.690847) -- (6.65568,0.690847) -- (6.65568,0.690062) --
 (6.68498,0.690062) -- (6.68498,0.689626) -- (6.71429,0.689626) -- (6.71429,0.691719) -- (6.74359,0.691719) -- (6.74359,0.690672) -- (6.77289,0.690672) -- (6.77289,0.687882) -- (6.8022,0.687882) -- (6.8022,0.688754) -- (6.8315,0.688754) --
 (6.8315,0.690323) -- (6.86081,0.690323) -- (6.86081,0.690585) -- (6.89011,0.690585) -- (6.89011,0.688841) -- (6.91941,0.688841) -- (6.91941,0.688231) -- (6.94872,0.688231) -- (6.94872,0.688056) -- (6.97802,0.688056) -- (6.97802,0.689364) --
 (7.00733,0.689364) -- (7.00733,0.688667) -- (7.03663,0.688667) -- (7.03663,0.688056) -- (7.06593,0.688056) -- (7.06593,0.688841) -- (7.09524,0.688841) -- (7.09524,0.688754) -- (7.12454,0.688754) -- (7.12454,0.688231) -- (7.15385,0.688231) --
 (7.15385,0.687271) -- (7.18315,0.687271) -- (7.18315,0.687707) -- (7.21245,0.687707) -- (7.21245,0.686312) -- (7.24176,0.686312) -- (7.24176,0.687882) -- (7.27106,0.687882) -- (7.27106,0.686835) -- (7.30037,0.686835) -- (7.30037,0.688231) --
 (7.32967,0.688231) -- (7.32967,0.685353) -- (7.35897,0.685353) -- (7.35897,0.687184) -- (7.38828,0.687184) -- (7.38828,0.686661) -- (7.41758,0.686661) -- (7.41758,0.687969) -- (7.44689,0.687969) -- (7.44689,0.686225) -- (7.47619,0.686225) --
 (7.47619,0.68544) -- (7.50549,0.68544) -- (7.50549,0.686051) -- (7.5348,0.686051) -- (7.5348,0.685004) -- (7.5641,0.685004) -- (7.5641,0.685353) -- (7.59341,0.685353) -- (7.59341,0.685876) -- (7.62271,0.685876) -- (7.62271,0.686399) --
 (7.65201,0.686399) -- (7.65201,0.684568) -- (7.68132,0.684568) -- (7.68132,0.684917) -- (7.71062,0.684917) -- (7.71062,0.685004) -- (7.73993,0.685004) -- (7.73993,0.685527) -- (7.76923,0.685527) -- (7.76923,0.68544) -- (7.79853,0.68544) --
 (7.79853,0.684132) -- (7.82784,0.684132) -- (7.82784,0.685527) -- (7.85714,0.685527) -- (7.85714,0.684655) -- (7.88645,0.684655) -- (7.88645,0.684743) -- (7.91575,0.684743) -- (7.91575,0.684655) -- (7.94506,0.684655) -- (7.94506,0.685179) --
 (7.97436,0.685179) -- (7.97436,0.683783) -- (8.00366,0.683783) -- (8.00366,0.685615) -- (8.03297,0.685615) -- (8.03297,0.684481) -- (8.06227,0.684481) -- (8.06227,0.684219) -- (8.09157,0.684219) -- (8.09157,0.684307) -- (8.12088,0.684307) --
 (8.12088,0.68483) -- (8.15018,0.68483) -- (8.15018,0.684132) -- (8.17949,0.684132) -- (8.17949,0.683609) -- (8.20879,0.683609) -- (8.20879,0.683958) -- (8.2381,0.683958) -- (8.2381,0.683783) -- (8.2674,0.683783) -- (8.2674,0.685527) --
 (8.2967,0.685527) -- (8.2967,0.685353) -- (8.32601,0.685353) -- (8.32601,0.684394) -- (8.35531,0.684394) -- (8.35531,0.683435) -- (8.38461,0.683435) -- (8.38461,0.683958) -- (8.41392,0.683958) -- (8.41392,0.683871) -- (8.44322,0.683871) --
 (8.44322,0.683435) -- (8.47253,0.683435) -- (8.47253,0.682911) -- (8.50183,0.682911) -- (8.50183,0.682999) -- (8.53114,0.682999) -- (8.53114,0.682824) -- (8.56044,0.682824) -- (8.56044,0.684481) -- (8.58974,0.684481) -- (8.58974,0.684307) --
 (8.61905,0.684307) -- (8.61905,0.684307) -- (8.64835,0.684307) -- (8.64835,0.68326) -- (8.67766,0.68326) -- (8.67766,0.683871) -- (8.70696,0.683871) -- (8.70696,0.683696) -- (8.73626,0.683696) -- (8.73626,0.682737) -- (8.76557,0.682737) --
 (8.76557,0.683871) -- (8.79487,0.683871) -- (8.79487,0.682911) -- (8.82418,0.682911) -- (8.82418,0.682824) -- (8.85348,0.682824) -- (8.85348,0.683173) -- (8.88278,0.683173) -- (8.88278,0.682999) -- (8.91209,0.682999) -- (8.91209,0.682563) --
 (8.94139,0.682563) -- (8.94139,0.683435) -- (8.9707,0.683435) -- (8.9707,0.681865) -- (9,0.681865) -- (9,0.68326) -- (9,0.68326);
\definecolor{c}{rgb}{0,0,0};
\draw [c] (1,0.679598) -- (9,0.679598);
\draw [c] (1,0.842701) -- (1,0.679598);
\draw [c] (1.2664,0.761149) -- (1.2664,0.679598);
\draw [c] (1.5328,0.761149) -- (1.5328,0.679598);
\draw [c] (1.7992,0.761149) -- (1.7992,0.679598);
\draw [c] (2.0656,0.761149) -- (2.0656,0.679598);
\draw [c] (2.332,0.842701) -- (2.332,0.679598);
\draw [c] (2.5984,0.761149) -- (2.5984,0.679598);
\draw [c] (2.8648,0.761149) -- (2.8648,0.679598);
\draw [c] (3.1312,0.761149) -- (3.1312,0.679598);
\draw [c] (3.3976,0.761149) -- (3.3976,0.679598);
\draw [c] (3.664,0.842701) -- (3.664,0.679598);
\draw [c] (3.9304,0.761149) -- (3.9304,0.679598);
\draw [c] (4.1968,0.761149) -- (4.1968,0.679598);
\draw [c] (4.4632,0.761149) -- (4.4632,0.679598);
\draw [c] (4.7296,0.761149) -- (4.7296,0.679598);
\draw [c] (4.996,0.842701) -- (4.996,0.679598);
\draw [c] (5.2624,0.761149) -- (5.2624,0.679598);
\draw [c] (5.5288,0.761149) -- (5.5288,0.679598);
\draw [c] (5.7952,0.761149) -- (5.7952,0.679598);
\draw [c] (6.0616,0.761149) -- (6.0616,0.679598);
\draw [c] (6.32801,0.842701) -- (6.32801,0.679598);
\draw [c] (6.59441,0.761149) -- (6.59441,0.679598);
\draw [c] (6.86081,0.761149) -- (6.86081,0.679598);
\draw [c] (7.12721,0.761149) -- (7.12721,0.679598);
\draw [c] (7.39361,0.761149) -- (7.39361,0.679598);
\draw [c] (7.66001,0.842701) -- (7.66001,0.679598);
\draw [c] (7.92641,0.761149) -- (7.92641,0.679598);
\draw [c] (8.19281,0.761149) -- (8.19281,0.679598);
\draw [c] (8.45921,0.761149) -- (8.45921,0.679598);
\draw [c] (8.72561,0.761149) -- (8.72561,0.679598);
\draw [c] (8.99201,0.842701) -- (8.99201,0.679598);
\draw [c] (8.99201,0.842701) -- (8.99201,0.679598);
\draw [c] (1,0.679598) -- (1,6.11638);
\draw [c] (1.24,0.679598) -- (1,0.679598);
\draw [c] (1.12,0.859573) -- (1,0.859573);
\draw [c] (1.12,1.03955) -- (1,1.03955);
\draw [c] (1.12,1.21952) -- (1,1.21952);
\draw [c] (1.24,1.3995) -- (1,1.3995);
\draw [c] (1.12,1.57947) -- (1,1.57947);
\draw [c] (1.12,1.75945) -- (1,1.75945);
\draw [c] (1.12,1.93942) -- (1,1.93942);
\draw [c] (1.24,2.1194) -- (1,2.1194);
\draw [c] (1.12,2.29937) -- (1,2.29937);
\draw [c] (1.12,2.47935) -- (1,2.47935);
\draw [c] (1.12,2.65933) -- (1,2.65933);
\draw [c] (1.24,2.8393) -- (1,2.8393);
\draw [c] (1.12,3.01928) -- (1,3.01928);
\draw [c] (1.12,3.19925) -- (1,3.19925);
\draw [c] (1.12,3.37923) -- (1,3.37923);
\draw [c] (1.24,3.5592) -- (1,3.5592);
\draw [c] (1.12,3.73918) -- (1,3.73918);
\draw [c] (1.12,3.91915) -- (1,3.91915);
\draw [c] (1.12,4.09913) -- (1,4.09913);
\draw [c] (1.24,4.2791) -- (1,4.2791);
\draw [c] (1.12,4.45908) -- (1,4.45908);
\draw [c] (1.12,4.63905) -- (1,4.63905);
\draw [c] (1.12,4.81903) -- (1,4.81903);
\draw [c] (1.24,4.999) -- (1,4.999);
\draw [c] (1.12,5.17898) -- (1,5.17898);
\draw [c] (1.12,5.35895) -- (1,5.35895);
\draw [c] (1.12,5.53893) -- (1,5.53893);
\draw [c] (1.24,5.7189) -- (1,5.7189);
\draw [c] (1.24,5.7189) -- (1,5.7189);
\draw [c] (1.12,5.89888) -- (1,5.89888);
\draw [c] (1.12,6.07885) -- (1,6.07885);
\definecolor{c}{rgb}{1,1,1};
\draw [color=c, fill=c] (5,4.5533) rectangle (8.8,5.98046);
\definecolor{c}{rgb}{0,0,0};
% \draw [c] (5,4.5533) -- (8.8,4.5533);
% \draw [c] (8.8,4.5533) -- (8.8,5.98046);
% \draw [c] (8.8,5.98046) -- (5,5.98046);
% \draw [c] (5,5.98046) -- (5,4.5533);
\draw [anchor=base west] (5.95,5.72179) node[color=c, rotate=0]{Leading tight};
\definecolor{c}{rgb}{1,1,1};
\draw [c, fill=c] (5.1425,5.67719) -- (5.8075,5.67719) -- (5.8075,5.92694) -- (5.1425,5.92694);
\colorlet{c}{natgreen}
\draw [c] (5.1425,5.80207) -- (5.8075,5.80207);
\definecolor{c}{rgb}{0,0,0};
\draw [anchor=base west] (5.95,5.365) node[color=c, rotate=0]{Leading non--tight};
\definecolor{c}{rgb}{1,1,1};
\draw [c, fill=c] (5.1425,5.3204) -- (5.8075,5.3204) -- (5.8075,5.57015) -- (5.1425,5.57015);
\colorlet{c}{natgreen!50}
\draw [c] (5.1425,5.44528) -- (5.8075,5.44528);
\definecolor{c}{rgb}{0,0,0};
\draw [anchor=base west] (5.95,5.00821) node[color=c, rotate=0]{Subleading tight};
\definecolor{c}{rgb}{1,1,1};
\draw [c, fill=c] (5.1425,4.96361) -- (5.8075,4.96361) -- (5.8075,5.21336) -- (5.1425,5.21336);
\colorlet{c}{natcomp}
\draw [c] (5.1425,5.08849) -- (5.8075,5.08849);
\definecolor{c}{rgb}{0,0,0};
\draw [anchor=base west] (5.95,4.65142) node[color=c, rotate=0]{Subleading non--tight};
\definecolor{c}{rgb}{1,1,1};
\draw [c, fill=c] (5.1425,4.60682) -- (5.8075,4.60682) -- (5.8075,4.85658) -- (5.1425,4.85658);
\colorlet{c}{natcomp!50}
\draw [c] (5.1425,4.7317) -- (5.8075,4.7317);
\definecolor{c}{rgb}{0,0,0};
\end{tikzpicture}

\end{infilsf}
\end{minipage}\hfill\begin{minipage}[b]{.3\textwidth}
\caption{The transverse isolation energy $E_T^{\text{isol}}$ for the tight and non--tight photon selection of the leading photons, and for the set of subleading photons with partners in the `A' sample. The non--tight samples have been scaled so that the D region contains the same number of events as the B region. For both sets of samples, the shapes of the distributions match up [quantify?] after the scaling, which supports the assumption that the shape of the background distribution is the same in the signal region as well.
\label{etiso}}
\end{minipage}
\end{figure}

Performing this process for each bin in $M_{\gamma\gamma}$, we obtain the distribution of background events shown in figure~\ref{mggbck}. Note, though, that the plots of $E_T^{\text{isol}}$ above showed the combined distribution for all $M_{\gamma\gamma}$ bins combined.

\begin{figure}[htp]
\begin{minipage}[b]{.69\textwidth}
\begin{infilsf} \tiny 
\begin{tikzpicture}[x=.1\textwidth,y=.1\textwidth]
\pgfdeclareplotmark{cross} {
\pgfpathmoveto{\pgfpoint{-0.3\pgfplotmarksize}{\pgfplotmarksize}}
\pgfpathlineto{\pgfpoint{+0.3\pgfplotmarksize}{\pgfplotmarksize}}
\pgfpathlineto{\pgfpoint{+0.3\pgfplotmarksize}{0.3\pgfplotmarksize}}
\pgfpathlineto{\pgfpoint{+1\pgfplotmarksize}{0.3\pgfplotmarksize}}
\pgfpathlineto{\pgfpoint{+1\pgfplotmarksize}{-0.3\pgfplotmarksize}}
\pgfpathlineto{\pgfpoint{+0.3\pgfplotmarksize}{-0.3\pgfplotmarksize}}
\pgfpathlineto{\pgfpoint{+0.3\pgfplotmarksize}{-1.\pgfplotmarksize}}
\pgfpathlineto{\pgfpoint{-0.3\pgfplotmarksize}{-1.\pgfplotmarksize}}
\pgfpathlineto{\pgfpoint{-0.3\pgfplotmarksize}{-0.3\pgfplotmarksize}}
\pgfpathlineto{\pgfpoint{-1.\pgfplotmarksize}{-0.3\pgfplotmarksize}}
\pgfpathlineto{\pgfpoint{-1.\pgfplotmarksize}{0.3\pgfplotmarksize}}
\pgfpathlineto{\pgfpoint{-0.3\pgfplotmarksize}{0.3\pgfplotmarksize}}
\pgfpathclose
\pgfusepathqstroke
}
\pgfdeclareplotmark{cross*} {
\pgfpathmoveto{\pgfpoint{-0.3\pgfplotmarksize}{\pgfplotmarksize}}
\pgfpathlineto{\pgfpoint{+0.3\pgfplotmarksize}{\pgfplotmarksize}}
\pgfpathlineto{\pgfpoint{+0.3\pgfplotmarksize}{0.3\pgfplotmarksize}}
\pgfpathlineto{\pgfpoint{+1\pgfplotmarksize}{0.3\pgfplotmarksize}}
\pgfpathlineto{\pgfpoint{+1\pgfplotmarksize}{-0.3\pgfplotmarksize}}
\pgfpathlineto{\pgfpoint{+0.3\pgfplotmarksize}{-0.3\pgfplotmarksize}}
\pgfpathlineto{\pgfpoint{+0.3\pgfplotmarksize}{-1.\pgfplotmarksize}}
\pgfpathlineto{\pgfpoint{-0.3\pgfplotmarksize}{-1.\pgfplotmarksize}}
\pgfpathlineto{\pgfpoint{-0.3\pgfplotmarksize}{-0.3\pgfplotmarksize}}
\pgfpathlineto{\pgfpoint{-1.\pgfplotmarksize}{-0.3\pgfplotmarksize}}
\pgfpathlineto{\pgfpoint{-1.\pgfplotmarksize}{0.3\pgfplotmarksize}}
\pgfpathlineto{\pgfpoint{-0.3\pgfplotmarksize}{0.3\pgfplotmarksize}}
\pgfpathclose
\pgfusepathqfillstroke
}
\pgfdeclareplotmark{newstar} {
\pgfpathmoveto{\pgfqpoint{0pt}{\pgfplotmarksize}}
\pgfpathlineto{\pgfqpointpolar{44}{0.5\pgfplotmarksize}}
\pgfpathlineto{\pgfqpointpolar{18}{\pgfplotmarksize}}
\pgfpathlineto{\pgfqpointpolar{-20}{0.5\pgfplotmarksize}}
\pgfpathlineto{\pgfqpointpolar{-54}{\pgfplotmarksize}}
\pgfpathlineto{\pgfqpointpolar{-90}{0.5\pgfplotmarksize}}
\pgfpathlineto{\pgfqpointpolar{234}{\pgfplotmarksize}}
\pgfpathlineto{\pgfqpointpolar{198}{0.5\pgfplotmarksize}}
\pgfpathlineto{\pgfqpointpolar{162}{\pgfplotmarksize}}
\pgfpathlineto{\pgfqpointpolar{134}{0.5\pgfplotmarksize}}
\pgfpathclose
\pgfusepathqstroke
}
\pgfdeclareplotmark{newstar*} {
\pgfpathmoveto{\pgfqpoint{0pt}{\pgfplotmarksize}}
\pgfpathlineto{\pgfqpointpolar{44}{0.5\pgfplotmarksize}}
\pgfpathlineto{\pgfqpointpolar{18}{\pgfplotmarksize}}
\pgfpathlineto{\pgfqpointpolar{-20}{0.5\pgfplotmarksize}}
\pgfpathlineto{\pgfqpointpolar{-54}{\pgfplotmarksize}}
\pgfpathlineto{\pgfqpointpolar{-90}{0.5\pgfplotmarksize}}
\pgfpathlineto{\pgfqpointpolar{234}{\pgfplotmarksize}}
\pgfpathlineto{\pgfqpointpolar{198}{0.5\pgfplotmarksize}}
\pgfpathlineto{\pgfqpointpolar{162}{\pgfplotmarksize}}
\pgfpathlineto{\pgfqpointpolar{134}{0.5\pgfplotmarksize}}
\pgfpathclose
\pgfusepathqfillstroke
}
\definecolor{c}{rgb}{1,1,1};
\draw [color=c, fill=c] (0.998563,0.689655) rectangle (9.00144,6.12069);
\definecolor{c}{rgb}{0,0,0};
\draw [c] (0.998563,0.689655) -- (0.998563,6.12069) -- (9.00144,6.12069) -- (9.00144,0.689655) -- (0.998563,0.689655);
\definecolor{c}{rgb}{1,1,1};
\draw [color=c, fill=c] (0.998563,0.689655) rectangle (9.00144,6.12069);
\definecolor{c}{rgb}{0,0,0};
\draw [c] (0.998563,0.689655) -- (0.998563,6.12069) -- (9.00144,6.12069) -- (9.00144,0.689655) -- (0.998563,0.689655);
\colorlet{c}{natgreen}
\draw [c] (0.998563,0.702137) -- (1.07859,0.702137) -- (1.07859,0.739584) -- (1.15862,0.739584) -- (1.15862,0.792485) -- (1.23865,0.792485) -- (1.23865,0.937517) -- (1.31868,0.937517) -- (1.31868,0.950594) -- (1.39871,0.950594) -- (1.39871,0.976747)
 -- (1.47874,0.976747) -- (1.47874,1.00171) -- (1.55876,1.00171) -- (1.55876,1.01122) -- (1.63879,1.01122) -- (1.63879,1.16993) -- (1.71882,1.16993) -- (1.71882,1.30545) -- (1.79885,1.30545) -- (1.79885,2.61846) -- (1.87888,2.61846) --
 (1.87888,4.84031) -- (1.95891,4.84031) -- (1.95891,5.86207) -- (2.03894,5.86207) -- (2.03894,5.8169) -- (2.11897,5.8169) -- (2.11897,5.33187) -- (2.19899,5.33187) -- (2.19899,4.74401) -- (2.27902,4.74401) -- (2.27902,4.26018) -- (2.35905,4.26018) --
 (2.35905,3.65746) -- (2.43908,3.65746) -- (2.43908,3.21405) -- (2.51911,3.21405) -- (2.51911,2.77301) -- (2.59914,2.77301) -- (2.59914,2.53584) -- (2.67917,2.53584) -- (2.67917,2.23508) -- (2.7592,2.23508) -- (2.7592,2.01515) -- (2.83922,2.01515) --
 (2.83922,1.85288) -- (2.91925,1.85288) -- (2.91925,1.68883) -- (2.99928,1.68883) -- (2.99928,1.58838) -- (3.07931,1.58838) -- (3.07931,1.44691) -- (3.15934,1.44691) -- (3.15934,1.31555) -- (3.23937,1.31555) -- (3.23937,1.22758) -- (3.3194,1.22758)
 -- (3.3194,1.18538) -- (3.39943,1.18538) -- (3.39943,1.13367) -- (3.47945,1.13367) -- (3.47945,1.04213) -- (3.55948,1.04213) -- (3.55948,1.03024) -- (3.63951,1.03024) -- (3.63951,1.00052) -- (3.71954,1.00052) -- (3.71954,0.935734) --
 (3.79957,0.935734) -- (3.79957,0.938112) -- (3.8796,0.938112) -- (3.8796,0.903637) -- (3.95963,0.903637) -- (3.95963,0.879861) -- (4.03966,0.879861) -- (4.03966,0.865001) -- (4.11968,0.865001) -- (4.11968,0.863813) -- (4.19971,0.863813) --
 (4.19971,0.843009) -- (4.27974,0.843009) -- (4.27974,0.80794) -- (4.35977,0.80794) -- (4.35977,0.807345) -- (4.4398,0.807345) -- (4.4398,0.798429) -- (4.51983,0.798429) -- (4.51983,0.777031) -- (4.59986,0.777031) -- (4.59986,0.775842) --
 (4.67988,0.775842) -- (4.67988,0.754444) -- (4.75991,0.754444) -- (4.75991,0.77287) -- (4.83994,0.77287) -- (4.83994,0.75385) -- (4.91997,0.75385) -- (4.91997,0.747311) -- (5,0.747311) -- (5,0.751472) -- (5.08003,0.751472) -- (5.08003,0.734829) --
 (5.16006,0.734829) -- (5.16006,0.73899) -- (5.24009,0.73899) -- (5.24009,0.734829) -- (5.32012,0.734829) -- (5.32012,0.727102) -- (5.40014,0.727102) -- (5.40014,0.723536) -- (5.48017,0.723536) -- (5.48017,0.725913) -- (5.5602,0.725913) --
 (5.5602,0.721158) -- (5.64023,0.721158) -- (5.64023,0.722941) -- (5.72026,0.722941) -- (5.72026,0.71878) -- (5.80029,0.71878) -- (5.80029,0.70927) -- (5.88032,0.70927) -- (5.88032,0.704515) -- (5.96034,0.704515) -- (5.96034,0.713431) --
 (6.04037,0.713431) -- (6.04037,0.707487) -- (6.1204,0.707487) -- (6.1204,0.702732) -- (6.20043,0.702732) -- (6.20043,0.702732) -- (6.28046,0.702732) -- (6.28046,0.703921) -- (6.36049,0.703921) -- (6.36049,0.700949) -- (6.44052,0.700949) --
 (6.44052,0.69976) -- (6.52055,0.69976) -- (6.52055,0.710459) -- (6.60057,0.710459) -- (6.60057,0.700354) -- (6.6806,0.700354) -- (6.6806,0.69976) -- (6.76063,0.69976) -- (6.76063,0.698571) -- (6.84066,0.698571) -- (6.84066,0.697977) --
 (6.92069,0.697977) -- (6.92069,0.697382) -- (7.00072,0.697382) -- (7.00072,0.696788) -- (7.08075,0.696788) -- (7.08075,0.696194) -- (7.16078,0.696194) -- (7.16078,0.69441) -- (7.2408,0.69441) -- (7.2408,0.696194) -- (7.32083,0.696194) --
 (7.32083,0.696788) -- (7.40086,0.696788) -- (7.40086,0.691438) -- (7.48089,0.691438) -- (7.48089,0.696194) -- (7.56092,0.696194) -- (7.56092,0.695005) -- (7.64095,0.695005) -- (7.64095,0.69441) -- (7.72098,0.69441) -- (7.72098,0.693222) --
 (7.80101,0.693222) -- (7.80101,0.693816) -- (7.88103,0.693816) -- (7.88103,0.695005) -- (7.96106,0.695005) -- (7.96106,0.693222) -- (8.04109,0.693222) -- (8.04109,0.692033) -- (8.12112,0.692033) -- (8.12112,0.692627) -- (8.20115,0.692627) --
 (8.20115,0.692627) -- (8.28118,0.692627) -- (8.28118,0.692627) -- (8.36121,0.692627) -- (8.36121,0.691438) -- (8.44124,0.691438) -- (8.44124,0.692627) -- (8.52126,0.692627) -- (8.52126,0.692627) -- (8.60129,0.692627) -- (8.60129,0.691438) --
 (8.68132,0.691438) -- (8.68132,0.690844) -- (8.76135,0.690844) -- (8.76135,0.692627) -- (8.84138,0.692627) -- (8.84138,0.693222) -- (8.92141,0.693222) -- (8.92141,0.691438) -- (9.00144,0.691438);
\definecolor{c}{rgb}{0,0,0};
\draw [c] (0.998563,0.689655) -- (9.00144,0.689655);
\draw [anchor= east] (9.00144,0.30908) node[color=c, rotate=0]{$M_{\gamma\gamma} [GeV]$};
\draw [c] (0.998563,0.852817) -- (0.998563,0.689655);
\draw [c] (1.15862,0.771236) -- (1.15862,0.689655);
\draw [c] (1.31868,0.771236) -- (1.31868,0.689655);
\draw [c] (1.47874,0.771236) -- (1.47874,0.689655);
\draw [c] (1.63879,0.771236) -- (1.63879,0.689655);
\draw [c] (1.79885,0.852817) -- (1.79885,0.689655);
\draw [c] (1.95891,0.771236) -- (1.95891,0.689655);
\draw [c] (2.11897,0.771236) -- (2.11897,0.689655);
\draw [c] (2.27902,0.771236) -- (2.27902,0.689655);
\draw [c] (2.43908,0.771236) -- (2.43908,0.689655);
\draw [c] (2.59914,0.852817) -- (2.59914,0.689655);
\draw [c] (2.7592,0.771236) -- (2.7592,0.689655);
\draw [c] (2.91925,0.771236) -- (2.91925,0.689655);
\draw [c] (3.07931,0.771236) -- (3.07931,0.689655);
\draw [c] (3.23937,0.771236) -- (3.23937,0.689655);
\draw [c] (3.39943,0.852817) -- (3.39943,0.689655);
\draw [c] (3.55948,0.771236) -- (3.55948,0.689655);
\draw [c] (3.71954,0.771236) -- (3.71954,0.689655);
\draw [c] (3.8796,0.771236) -- (3.8796,0.689655);
\draw [c] (4.03966,0.771236) -- (4.03966,0.689655);
\draw [c] (4.19971,0.852817) -- (4.19971,0.689655);
\draw [c] (4.35977,0.771236) -- (4.35977,0.689655);
\draw [c] (4.51983,0.771236) -- (4.51983,0.689655);
\draw [c] (4.67988,0.771236) -- (4.67988,0.689655);
\draw [c] (4.83994,0.771236) -- (4.83994,0.689655);
\draw [c] (5,0.852817) -- (5,0.689655);
\draw [c] (5.16006,0.771236) -- (5.16006,0.689655);
\draw [c] (5.32012,0.771236) -- (5.32012,0.689655);
\draw [c] (5.48017,0.771236) -- (5.48017,0.689655);
\draw [c] (5.64023,0.771236) -- (5.64023,0.689655);
\draw [c] (5.80029,0.852817) -- (5.80029,0.689655);
\draw [c] (5.96034,0.771236) -- (5.96034,0.689655);
\draw [c] (6.1204,0.771236) -- (6.1204,0.689655);
\draw [c] (6.28046,0.771236) -- (6.28046,0.689655);
\draw [c] (6.44052,0.771236) -- (6.44052,0.689655);
\draw [c] (6.60057,0.852817) -- (6.60057,0.689655);
\draw [c] (6.76063,0.771236) -- (6.76063,0.689655);
\draw [c] (6.92069,0.771236) -- (6.92069,0.689655);
\draw [c] (7.08075,0.771236) -- (7.08075,0.689655);
\draw [c] (7.2408,0.771236) -- (7.2408,0.689655);
\draw [c] (7.40086,0.852817) -- (7.40086,0.689655);
\draw [c] (7.56092,0.771236) -- (7.56092,0.689655);
\draw [c] (7.72098,0.771236) -- (7.72098,0.689655);
\draw [c] (7.88103,0.771236) -- (7.88103,0.689655);
\draw [c] (8.04109,0.771236) -- (8.04109,0.689655);
\draw [c] (8.20115,0.852817) -- (8.20115,0.689655);
\draw [c] (8.36121,0.771236) -- (8.36121,0.689655);
\draw [c] (8.52126,0.771236) -- (8.52126,0.689655);
\draw [c] (8.68132,0.771236) -- (8.68132,0.689655);
\draw [c] (8.84138,0.771236) -- (8.84138,0.689655);
\draw [c] (9.00144,0.852817) -- (9.00144,0.689655);
\draw [anchor=base] (0.998563,0.465388) node[color=c, rotate=0]{0};
\draw [anchor=base] (1.79885,0.465388) node[color=c, rotate=0]{100};
\draw [anchor=base] (2.59914,0.465388) node[color=c, rotate=0]{200};
\draw [anchor=base] (3.39943,0.465388) node[color=c, rotate=0]{300};
\draw [anchor=base] (4.19971,0.465388) node[color=c, rotate=0]{400};
\draw [anchor=base] (5,0.465388) node[color=c, rotate=0]{500};
\draw [anchor=base] (5.80029,0.465388) node[color=c, rotate=0]{600};
\draw [anchor=base] (6.60057,0.465388) node[color=c, rotate=0]{700};
\draw [anchor=base] (7.40086,0.465388) node[color=c, rotate=0]{800};
\draw [anchor=base] (8.20115,0.465388) node[color=c, rotate=0]{900};
\draw [anchor=base] (9.00144,0.465388) node[color=c, rotate=0]{1000};
\draw [c] (0.998563,0.689655) -- (0.998563,6.12069);
\draw [anchor= east] (0.102563,6.12069) node[color=c, rotate=90]{Number of events};
\draw [c] (1.23831,0.689655) -- (0.998563,0.689655);
\draw [c] (1.11844,0.808534) -- (0.998563,0.808534);
\draw [c] (1.11844,0.927413) -- (0.998563,0.927413);
\draw [c] (1.11844,1.04629) -- (0.998563,1.04629);
\draw [c] (1.11844,1.16517) -- (0.998563,1.16517);
\draw [c] (1.23831,1.28405) -- (0.998563,1.28405);
\draw [c] (1.11844,1.40293) -- (0.998563,1.40293);
\draw [c] (1.11844,1.52181) -- (0.998563,1.52181);
\draw [c] (1.11844,1.64069) -- (0.998563,1.64069);
\draw [c] (1.11844,1.75956) -- (0.998563,1.75956);
\draw [c] (1.23831,1.87844) -- (0.998563,1.87844);
\draw [c] (1.11844,1.99732) -- (0.998563,1.99732);
\draw [c] (1.11844,2.1162) -- (0.998563,2.1162);
\draw [c] (1.11844,2.23508) -- (0.998563,2.23508);
\draw [c] (1.11844,2.35396) -- (0.998563,2.35396);
\draw [c] (1.23831,2.47284) -- (0.998563,2.47284);
\draw [c] (1.11844,2.59172) -- (0.998563,2.59172);
\draw [c] (1.11844,2.71059) -- (0.998563,2.71059);
\draw [c] (1.11844,2.82947) -- (0.998563,2.82947);
\draw [c] (1.11844,2.94835) -- (0.998563,2.94835);
\draw [c] (1.23831,3.06723) -- (0.998563,3.06723);
\draw [c] (1.11844,3.18611) -- (0.998563,3.18611);
\draw [c] (1.11844,3.30499) -- (0.998563,3.30499);
\draw [c] (1.11844,3.42387) -- (0.998563,3.42387);
\draw [c] (1.11844,3.54274) -- (0.998563,3.54274);
\draw [c] (1.23831,3.66162) -- (0.998563,3.66162);
\draw [c] (1.11844,3.7805) -- (0.998563,3.7805);
\draw [c] (1.11844,3.89938) -- (0.998563,3.89938);
\draw [c] (1.11844,4.01826) -- (0.998563,4.01826);
\draw [c] (1.11844,4.13714) -- (0.998563,4.13714);
\draw [c] (1.23831,4.25602) -- (0.998563,4.25602);
\draw [c] (1.11844,4.3749) -- (0.998563,4.3749);
\draw [c] (1.11844,4.49377) -- (0.998563,4.49377);
\draw [c] (1.11844,4.61265) -- (0.998563,4.61265);
\draw [c] (1.11844,4.73153) -- (0.998563,4.73153);
\draw [c] (1.23831,4.85041) -- (0.998563,4.85041);
\draw [c] (1.11844,4.96929) -- (0.998563,4.96929);
\draw [c] (1.11844,5.08817) -- (0.998563,5.08817);
\draw [c] (1.11844,5.20705) -- (0.998563,5.20705);
\draw [c] (1.11844,5.32593) -- (0.998563,5.32593);
\draw [c] (1.23831,5.4448) -- (0.998563,5.4448);
\draw [c] (1.11844,5.56368) -- (0.998563,5.56368);
\draw [c] (1.11844,5.68256) -- (0.998563,5.68256);
\draw [c] (1.11844,5.80144) -- (0.998563,5.80144);
\draw [c] (1.11844,5.92032) -- (0.998563,5.92032);
\draw [c] (1.23831,6.0392) -- (0.998563,6.0392);
\draw [c] (1.23831,6.0392) -- (0.998563,6.0392);
\draw [anchor= east] (0.948563,0.689655) node[color=c, rotate=0]{0};
\draw [anchor= east] (0.948563,1.28405) node[color=c, rotate=0]{1000};
\draw [anchor= east] (0.948563,1.87844) node[color=c, rotate=0]{2000};
\draw [anchor= east] (0.948563,2.47284) node[color=c, rotate=0]{3000};
\draw [anchor= east] (0.948563,3.06723) node[color=c, rotate=0]{4000};
\draw [anchor= east] (0.948563,3.66162) node[color=c, rotate=0]{5000};
\draw [anchor= east] (0.948563,4.25602) node[color=c, rotate=0]{6000};
\draw [anchor= east] (0.948563,4.85041) node[color=c, rotate=0]{7000};
\draw [anchor= east] (0.948563,5.4448) node[color=c, rotate=0]{8000};
\draw [anchor= east] (0.948563,6.0392) node[color=c, rotate=0]{9000};
\colorlet{c}{natcomp!50}
\draw [c, fill=c] (9.00144,0.689655) -- (0.998563,0.689655) -- (0.998563,0.701848) -- (1.07859,0.701848) -- (1.07859,0.698637) -- (1.15862,0.698637) -- (1.15862,0.788295) -- (1.23865,0.788295) -- (1.23865,0.871174) -- (1.31868,0.871174) --
 (1.31868,0.873346) -- (1.39871,0.873346) -- (1.39871,0.864878) -- (1.47874,0.864878) -- (1.47874,0.868318) -- (1.55876,0.868318) -- (1.55876,0.852822) -- (1.63879,0.852822) -- (1.63879,0.924654) -- (1.71882,0.924654) -- (1.71882,0.952279) --
 (1.79885,0.952279) -- (1.79885,1.70751) -- (1.87888,1.70751) -- (1.87888,2.9749) -- (1.95891,2.9749) -- (1.95891,3.53843) -- (2.03894,3.53843) -- (2.03894,3.69432) -- (2.11897,3.69432) -- (2.11897,3.40972) -- (2.19899,3.40972) -- (2.19899,3.00464)
 -- (2.27902,3.00464) -- (2.27902,2.76975) -- (2.35905,2.76975) -- (2.35905,2.39812) -- (2.43908,2.39812) -- (2.43908,2.2075) -- (2.51911,2.2075) -- (2.51911,1.94288) -- (2.59914,1.94288) -- (2.59914,1.79038) -- (2.67917,1.79038) -- (2.67917,1.6174)
 -- (2.7592,1.6174) -- (2.7592,1.47017) -- (2.83922,1.47017) -- (2.83922,1.36563) -- (2.91925,1.36563) -- (2.91925,1.29531) -- (2.99928,1.29531) -- (2.99928,1.20918) -- (3.07931,1.20918) -- (3.07931,1.13649) -- (3.15934,1.13649) -- (3.15934,1.05766)
 -- (3.23937,1.05766) -- (3.23937,1.01256) -- (3.3194,1.01256) -- (3.3194,1.00057) -- (3.39943,1.00057) -- (3.39943,0.960322) -- (3.47945,0.960322) -- (3.47945,0.911478) -- (3.55948,0.911478) -- (3.55948,0.899538) -- (3.63951,0.899538) --
 (3.63951,0.871288) -- (3.71954,0.871288) -- (3.71954,0.85599) -- (3.79957,0.85599) -- (3.79957,0.828789) -- (3.8796,0.828789) -- (3.8796,0.827299) -- (3.95963,0.827299) -- (3.95963,0.807137) -- (4.03966,0.807137) -- (4.03966,0.787592) --
 (4.11968,0.787592) -- (4.11968,0.79521) -- (4.19971,0.79521) -- (4.19971,0.780932) -- (4.27974,0.780932) -- (4.27974,0.758032) -- (4.35977,0.758032) -- (4.35977,0.767466) -- (4.4398,0.767466) -- (4.4398,0.756108) -- (4.51983,0.756108) --
 (4.51983,0.747922) -- (4.59986,0.747922) -- (4.59986,0.747466) -- (4.67988,0.747466) -- (4.67988,0.730954) -- (4.75991,0.730954) -- (4.75991,0.738646) -- (4.83994,0.738646) -- (4.83994,0.729191) -- (4.91997,0.729191) -- (4.91997,0.725896) --
 (5,0.725896) -- (5,0.730316) -- (5.08003,0.730316) -- (5.08003,0.719) -- (5.16006,0.719) -- (5.16006,0.719739) -- (5.24009,0.719739) -- (5.24009,0.720047) -- (5.32012,0.720047) -- (5.32012,0.715823) -- (5.40014,0.715823) -- (5.40014,0.714127) --
 (5.48017,0.714127) -- (5.48017,0.712854) -- (5.5602,0.712854) -- (5.5602,0.703855) -- (5.64023,0.703855) -- (5.64023,0.71375) -- (5.72026,0.71375) -- (5.72026,0.709567) -- (5.80029,0.709567) -- (5.80029,0.707108) -- (5.88032,0.707108) --
 (5.88032,0.702234) -- (5.96034,0.702234) -- (5.96034,0.70632) -- (6.04037,0.70632) -- (6.04037,0.701145) -- (6.1204,0.701145) -- (6.1204,0.698775) -- (6.20043,0.698775) -- (6.20043,0.698325) -- (6.28046,0.698325) -- (6.28046,0.701989) --
 (6.36049,0.701989) -- (6.36049,0.700949) -- (6.44052,0.700949) -- (6.44052,0.697507) -- (6.52055,0.697507) -- (6.52055,0.699112) -- (6.60057,0.699112) -- (6.60057,0.699585) -- (6.6806,0.699585) -- (6.6806,0.696208) -- (6.76063,0.696208) --
 (6.76063,0.696329) -- (6.84066,0.696329) -- (6.84066,0.69503) -- (6.92069,0.69503) -- (6.92069,0.695371) -- (7.00072,0.695371) -- (7.00072,0.693388) -- (7.08075,0.693388) -- (7.08075,0.695641) -- (7.16078,0.695641) -- (7.16078,0.693823) --
 (7.2408,0.693823) -- (7.2408,0.693231) -- (7.32083,0.693231) -- (7.32083,0.693549) -- (7.40086,0.693549) -- (7.40086,0.691353) -- (7.48089,0.691353) -- (7.48089,0.691054) -- (7.56092,0.691054) -- (7.56092,0.696028) -- (7.64095,0.696028) --
 (7.64095,0.692722) -- (7.72098,0.692722) -- (7.72098,0.692357) -- (7.80101,0.692357) -- (7.80101,0.692679) -- (7.88103,0.692679) -- (7.88103,0.692043) -- (7.96106,0.692043) -- (7.96106,0.691862) -- (8.04109,0.691862) -- (8.04109,0.69249) --
 (8.12112,0.69249) -- (8.12112,0.691756) -- (8.20115,0.691756) -- (8.20115,0.692125) -- (8.28118,0.692125) -- (8.28118,0.690481) -- (8.36121,0.690481) -- (8.36121,0.691073) -- (8.44124,0.691073) -- (8.44124,0.692675) -- (8.52126,0.692675) --
 (8.52126,0.690114) -- (8.60129,0.690114) -- (8.60129,0.692907) -- (8.68132,0.692907) -- (8.68132,0.691637) -- (8.76135,0.691637) -- (8.76135,0.691536) -- (8.84138,0.691536) -- (8.84138,0.691591) -- (8.92141,0.691591) -- (8.92141,0.691947) --
 (9.00144,0.691947) -- (9.00144,0.69009) -- (9.00144,0.69009);
\colorlet{c}{natcomp}
\draw [c] (0.998563,0.701848) -- (1.07859,0.701848) -- (1.07859,0.698637) -- (1.15862,0.698637) -- (1.15862,0.788295) -- (1.23865,0.788295) -- (1.23865,0.871174) -- (1.31868,0.871174) -- (1.31868,0.873346) -- (1.39871,0.873346) -- (1.39871,0.864878)
 -- (1.47874,0.864878) -- (1.47874,0.868318) -- (1.55876,0.868318) -- (1.55876,0.852822) -- (1.63879,0.852822) -- (1.63879,0.924654) -- (1.71882,0.924654) -- (1.71882,0.952279) -- (1.79885,0.952279) -- (1.79885,1.70751) -- (1.87888,1.70751) --
 (1.87888,2.9749) -- (1.95891,2.9749) -- (1.95891,3.53843) -- (2.03894,3.53843) -- (2.03894,3.69432) -- (2.11897,3.69432) -- (2.11897,3.40972) -- (2.19899,3.40972) -- (2.19899,3.00464) -- (2.27902,3.00464) -- (2.27902,2.76975) -- (2.35905,2.76975) --
 (2.35905,2.39812) -- (2.43908,2.39812) -- (2.43908,2.2075) -- (2.51911,2.2075) -- (2.51911,1.94288) -- (2.59914,1.94288) -- (2.59914,1.79038) -- (2.67917,1.79038) -- (2.67917,1.6174) -- (2.7592,1.6174) -- (2.7592,1.47017) -- (2.83922,1.47017) --
 (2.83922,1.36563) -- (2.91925,1.36563) -- (2.91925,1.29531) -- (2.99928,1.29531) -- (2.99928,1.20918) -- (3.07931,1.20918) -- (3.07931,1.13649) -- (3.15934,1.13649) -- (3.15934,1.05766) -- (3.23937,1.05766) -- (3.23937,1.01256) -- (3.3194,1.01256)
 -- (3.3194,1.00057) -- (3.39943,1.00057) -- (3.39943,0.960322) -- (3.47945,0.960322) -- (3.47945,0.911478) -- (3.55948,0.911478) -- (3.55948,0.899538) -- (3.63951,0.899538) -- (3.63951,0.871288) -- (3.71954,0.871288) -- (3.71954,0.85599) --
 (3.79957,0.85599) -- (3.79957,0.828789) -- (3.8796,0.828789) -- (3.8796,0.827299) -- (3.95963,0.827299) -- (3.95963,0.807137) -- (4.03966,0.807137) -- (4.03966,0.787592) -- (4.11968,0.787592) -- (4.11968,0.79521) -- (4.19971,0.79521) --
 (4.19971,0.780932) -- (4.27974,0.780932) -- (4.27974,0.758032) -- (4.35977,0.758032) -- (4.35977,0.767466) -- (4.4398,0.767466) -- (4.4398,0.756108) -- (4.51983,0.756108) -- (4.51983,0.747922) -- (4.59986,0.747922) -- (4.59986,0.747466) --
 (4.67988,0.747466) -- (4.67988,0.730954) -- (4.75991,0.730954) -- (4.75991,0.738646) -- (4.83994,0.738646) -- (4.83994,0.729191) -- (4.91997,0.729191) -- (4.91997,0.725896) -- (5,0.725896) -- (5,0.730316) -- (5.08003,0.730316) -- (5.08003,0.719) --
 (5.16006,0.719) -- (5.16006,0.719739) -- (5.24009,0.719739) -- (5.24009,0.720047) -- (5.32012,0.720047) -- (5.32012,0.715823) -- (5.40014,0.715823) -- (5.40014,0.714127) -- (5.48017,0.714127) -- (5.48017,0.712854) -- (5.5602,0.712854) --
 (5.5602,0.703855) -- (5.64023,0.703855) -- (5.64023,0.71375) -- (5.72026,0.71375) -- (5.72026,0.709567) -- (5.80029,0.709567) -- (5.80029,0.707108) -- (5.88032,0.707108) -- (5.88032,0.702234) -- (5.96034,0.702234) -- (5.96034,0.70632) --
 (6.04037,0.70632) -- (6.04037,0.701145) -- (6.1204,0.701145) -- (6.1204,0.698775) -- (6.20043,0.698775) -- (6.20043,0.698325) -- (6.28046,0.698325) -- (6.28046,0.701989) -- (6.36049,0.701989) -- (6.36049,0.700949) -- (6.44052,0.700949) --
 (6.44052,0.697507) -- (6.52055,0.697507) -- (6.52055,0.699112) -- (6.60057,0.699112) -- (6.60057,0.699585) -- (6.6806,0.699585) -- (6.6806,0.696208) -- (6.76063,0.696208) -- (6.76063,0.696329) -- (6.84066,0.696329) -- (6.84066,0.69503) --
 (6.92069,0.69503) -- (6.92069,0.695371) -- (7.00072,0.695371) -- (7.00072,0.693388) -- (7.08075,0.693388) -- (7.08075,0.695641) -- (7.16078,0.695641) -- (7.16078,0.693823) -- (7.2408,0.693823) -- (7.2408,0.693231) -- (7.32083,0.693231) --
 (7.32083,0.693549) -- (7.40086,0.693549) -- (7.40086,0.691353) -- (7.48089,0.691353) -- (7.48089,0.691054) -- (7.56092,0.691054) -- (7.56092,0.696028) -- (7.64095,0.696028) -- (7.64095,0.692722) -- (7.72098,0.692722) -- (7.72098,0.692357) --
 (7.80101,0.692357) -- (7.80101,0.692679) -- (7.88103,0.692679) -- (7.88103,0.692043) -- (7.96106,0.692043) -- (7.96106,0.691862) -- (8.04109,0.691862) -- (8.04109,0.69249) -- (8.12112,0.69249) -- (8.12112,0.691756) -- (8.20115,0.691756) --
 (8.20115,0.692125) -- (8.28118,0.692125) -- (8.28118,0.690481) -- (8.36121,0.690481) -- (8.36121,0.691073) -- (8.44124,0.691073) -- (8.44124,0.692675) -- (8.52126,0.692675) -- (8.52126,0.689655) -- (8.60129,0.689655) -- (8.60129,0.692907) --
 (8.68132,0.692907) -- (8.68132,0.691637) -- (8.76135,0.691637) -- (8.76135,0.691536) -- (8.84138,0.691536) -- (8.84138,0.691591) -- (8.92141,0.691591) -- (8.92141,0.691947) -- (9.00144,0.691947) -- (9.00144,0.689655) -- (9.00144,0.689655);
\definecolor{c}{rgb}{1,1,1};
\draw [color=c, fill=c] (0,0) rectangle (0,0);
\definecolor{c}{rgb}{0,0,0};
\draw [c] (0,0) -- (0,0);
\draw [c] (0,0) -- (0,0);
\draw [c] (0,0) -- (0,0);
\draw [c] (0,0) -- (0,0);
\colorlet{c}{natgreen}
\draw [c] (0,0) -- (0,0);
\colorlet{c}{natcomp!50}
\draw [c, fill=c] (0,0) -- (0,0) -- (0,0) -- (0,0);
\colorlet{c}{natcomp}
\draw [c] (0,0) -- (0,0);
\definecolor{c}{rgb}{1,1,1};
\draw [color=c, fill=c] (0,0) rectangle (0,0);
\definecolor{c}{rgb}{0,0,0};
\draw [c] (0,0) -- (0,0);
\draw [c] (0,0) -- (0,0);
\draw [c] (0,0) -- (0,0);
\draw [c] (0,0) -- (0,0);
\colorlet{c}{natgreen}
\draw [c] (0,0) -- (0,0);
\colorlet{c}{natcomp!50}
\draw [c, fill=c] (0,0) -- (0,0) -- (0,0) -- (0,0);
\colorlet{c}{natcomp}
\draw [c] (0,0) -- (0,0);
\definecolor{c}{rgb}{1,1,1};
\draw [color=c, fill=c] (5,4.5533) rectangle (8.8,5.98046);
\draw [c] (5,4.5533) -- (8.8,4.5533);
\draw [c] (8.8,4.5533) -- (8.8,5.98046);
\draw [c] (8.8,5.98046) -- (5,5.98046);
\draw [c] (5,5.98046) -- (5,4.5533);
\definecolor{c}{rgb}{0,0,0};
\draw [anchor= west] (5.95,5.62367) node[color=c, rotate=0]{Total signal};
\colorlet{c}{natgreen}
\draw [c] (5.1425,5.62367) -- (5.8075,5.62367);
\definecolor{c}{rgb}{0,0,0};
\draw [anchor= west] (5.95,4.91009) node[color=c, rotate=0]{Estimated background};
\colorlet{c}{natcomp!50}
\draw [c, fill=c] (5.1425,4.66034) -- (5.8075,4.66034) -- (5.8075,5.15985) -- (5.1425,5.15985);
\colorlet{c}{natcomp}
\draw [c] (5.1425,4.91009) -- (5.8075,4.91009);
\definecolor{c}{rgb}{0,0,0};
\draw [c] (0.998563,0.689655) -- (9.00144,0.689655);
\draw [c] (0.998563,0.852817) -- (0.998563,0.689655);
\draw [c] (1.15862,0.771236) -- (1.15862,0.689655);
\draw [c] (1.31868,0.771236) -- (1.31868,0.689655);
\draw [c] (1.47874,0.771236) -- (1.47874,0.689655);
\draw [c] (1.63879,0.771236) -- (1.63879,0.689655);
\draw [c] (1.79885,0.852817) -- (1.79885,0.689655);
\draw [c] (1.95891,0.771236) -- (1.95891,0.689655);
\draw [c] (2.11897,0.771236) -- (2.11897,0.689655);
\draw [c] (2.27902,0.771236) -- (2.27902,0.689655);
\draw [c] (2.43908,0.771236) -- (2.43908,0.689655);
\draw [c] (2.59914,0.852817) -- (2.59914,0.689655);
\draw [c] (2.7592,0.771236) -- (2.7592,0.689655);
\draw [c] (2.91925,0.771236) -- (2.91925,0.689655);
\draw [c] (3.07931,0.771236) -- (3.07931,0.689655);
\draw [c] (3.23937,0.771236) -- (3.23937,0.689655);
\draw [c] (3.39943,0.852817) -- (3.39943,0.689655);
\draw [c] (3.55948,0.771236) -- (3.55948,0.689655);
\draw [c] (3.71954,0.771236) -- (3.71954,0.689655);
\draw [c] (3.8796,0.771236) -- (3.8796,0.689655);
\draw [c] (4.03966,0.771236) -- (4.03966,0.689655);
\draw [c] (4.19971,0.852817) -- (4.19971,0.689655);
\draw [c] (4.35977,0.771236) -- (4.35977,0.689655);
\draw [c] (4.51983,0.771236) -- (4.51983,0.689655);
\draw [c] (4.67988,0.771236) -- (4.67988,0.689655);
\draw [c] (4.83994,0.771236) -- (4.83994,0.689655);
\draw [c] (5,0.852817) -- (5,0.689655);
\draw [c] (5.16006,0.771236) -- (5.16006,0.689655);
\draw [c] (5.32012,0.771236) -- (5.32012,0.689655);
\draw [c] (5.48017,0.771236) -- (5.48017,0.689655);
\draw [c] (5.64023,0.771236) -- (5.64023,0.689655);
\draw [c] (5.80029,0.852817) -- (5.80029,0.689655);
\draw [c] (5.96034,0.771236) -- (5.96034,0.689655);
\draw [c] (6.1204,0.771236) -- (6.1204,0.689655);
\draw [c] (6.28046,0.771236) -- (6.28046,0.689655);
\draw [c] (6.44052,0.771236) -- (6.44052,0.689655);
\draw [c] (6.60057,0.852817) -- (6.60057,0.689655);
\draw [c] (6.76063,0.771236) -- (6.76063,0.689655);
\draw [c] (6.92069,0.771236) -- (6.92069,0.689655);
\draw [c] (7.08075,0.771236) -- (7.08075,0.689655);
\draw [c] (7.2408,0.771236) -- (7.2408,0.689655);
\draw [c] (7.40086,0.852817) -- (7.40086,0.689655);
\draw [c] (7.56092,0.771236) -- (7.56092,0.689655);
\draw [c] (7.72098,0.771236) -- (7.72098,0.689655);
\draw [c] (7.88103,0.771236) -- (7.88103,0.689655);
\draw [c] (8.04109,0.771236) -- (8.04109,0.689655);
\draw [c] (8.20115,0.852817) -- (8.20115,0.689655);
\draw [c] (8.36121,0.771236) -- (8.36121,0.689655);
\draw [c] (8.52126,0.771236) -- (8.52126,0.689655);
\draw [c] (8.68132,0.771236) -- (8.68132,0.689655);
\draw [c] (8.84138,0.771236) -- (8.84138,0.689655);
\draw [c] (9.00144,0.852817) -- (9.00144,0.689655);
\draw [c] (0.998563,0.689655) -- (0.998563,6.12069);
\draw [c] (1.23831,0.689655) -- (0.998563,0.689655);
\draw [c] (1.11844,0.808534) -- (0.998563,0.808534);
\draw [c] (1.11844,0.927413) -- (0.998563,0.927413);
\draw [c] (1.11844,1.04629) -- (0.998563,1.04629);
\draw [c] (1.11844,1.16517) -- (0.998563,1.16517);
\draw [c] (1.23831,1.28405) -- (0.998563,1.28405);
\draw [c] (1.11844,1.40293) -- (0.998563,1.40293);
\draw [c] (1.11844,1.52181) -- (0.998563,1.52181);
\draw [c] (1.11844,1.64069) -- (0.998563,1.64069);
\draw [c] (1.11844,1.75956) -- (0.998563,1.75956);
\draw [c] (1.23831,1.87844) -- (0.998563,1.87844);
\draw [c] (1.11844,1.99732) -- (0.998563,1.99732);
\draw [c] (1.11844,2.1162) -- (0.998563,2.1162);
\draw [c] (1.11844,2.23508) -- (0.998563,2.23508);
\draw [c] (1.11844,2.35396) -- (0.998563,2.35396);
\draw [c] (1.23831,2.47284) -- (0.998563,2.47284);
\draw [c] (1.11844,2.59172) -- (0.998563,2.59172);
\draw [c] (1.11844,2.71059) -- (0.998563,2.71059);
\draw [c] (1.11844,2.82947) -- (0.998563,2.82947);
\draw [c] (1.11844,2.94835) -- (0.998563,2.94835);
\draw [c] (1.23831,3.06723) -- (0.998563,3.06723);
\draw [c] (1.11844,3.18611) -- (0.998563,3.18611);
\draw [c] (1.11844,3.30499) -- (0.998563,3.30499);
\draw [c] (1.11844,3.42387) -- (0.998563,3.42387);
\draw [c] (1.11844,3.54274) -- (0.998563,3.54274);
\draw [c] (1.23831,3.66162) -- (0.998563,3.66162);
\draw [c] (1.11844,3.7805) -- (0.998563,3.7805);
\draw [c] (1.11844,3.89938) -- (0.998563,3.89938);
\draw [c] (1.11844,4.01826) -- (0.998563,4.01826);
\draw [c] (1.11844,4.13714) -- (0.998563,4.13714);
\draw [c] (1.23831,4.25602) -- (0.998563,4.25602);
\draw [c] (1.11844,4.3749) -- (0.998563,4.3749);
\draw [c] (1.11844,4.49377) -- (0.998563,4.49377);
\draw [c] (1.11844,4.61265) -- (0.998563,4.61265);
\draw [c] (1.11844,4.73153) -- (0.998563,4.73153);
\draw [c] (1.23831,4.85041) -- (0.998563,4.85041);
\draw [c] (1.11844,4.96929) -- (0.998563,4.96929);
\draw [c] (1.11844,5.08817) -- (0.998563,5.08817);
\draw [c] (1.11844,5.20705) -- (0.998563,5.20705);
\draw [c] (1.11844,5.32593) -- (0.998563,5.32593);
\draw [c] (1.23831,5.4448) -- (0.998563,5.4448);
\draw [c] (1.11844,5.56368) -- (0.998563,5.56368);
\draw [c] (1.11844,5.68256) -- (0.998563,5.68256);
\draw [c] (1.11844,5.80144) -- (0.998563,5.80144);
\draw [c] (1.11844,5.92032) -- (0.998563,5.92032);
\draw [c] (1.23831,6.0392) -- (0.998563,6.0392);
\draw [c] (1.23831,6.0392) -- (0.998563,6.0392);
\end{tikzpicture}

\end{infilsf}
\end{minipage}\hfill\begin{minipage}[b]{.3\textwidth}
\caption{The distribution of background events in $M_{\gamma\gamma}$ estimated from data with the ABCD method, along with the distribution of diphoton events given by the data.
\label{mggbck}}
\end{minipage}
\end{figure}

\section{Comparison to SM MC}
[Herfra er alting work in progress]

The \atlas{} collaboration has produced a variety of Monte Carlo event sets that describe processes which might produce events that pass the selection cuts. Among a larger selection of candidate sets, the four [three? two!? one\textinterrobang\textinterrobang] used to produce figure~\ref{mclist} were found to yield the greatest contribution after the selection criteria used for the data were applied to them\footnote{[Something, something, list of full names, maybe appendix on contributions from sets?]}. The greatest contribution comes from the MC set that describes the Standard Model $pp\,\rightarrow\,\gamma\gamma$ process. The remaining MC sets describe processes with $\gamma$--jet or single--$\gamma$ final states, and  $Z\,\rightarrow\,ee$ decays as the most significant among processes with $ee$ final states. These processes might produce false positives either by emitting hard photons through bremsstrahlung or by false identification of electrons as converted photons.

\begin{figure}[htp]
\begin{minipage}[b]{.69\textwidth}
\begin{infilsf} \tiny
\begin{tikzpicture}[x=.1\textwidth,y=.1\textwidth]
\pgfdeclareplotmark{cross} {
\pgfpathmoveto{\pgfpoint{-0.3\pgfplotmarksize}{\pgfplotmarksize}}
\pgfpathlineto{\pgfpoint{+0.3\pgfplotmarksize}{\pgfplotmarksize}}
\pgfpathlineto{\pgfpoint{+0.3\pgfplotmarksize}{0.3\pgfplotmarksize}}
\pgfpathlineto{\pgfpoint{+1\pgfplotmarksize}{0.3\pgfplotmarksize}}
\pgfpathlineto{\pgfpoint{+1\pgfplotmarksize}{-0.3\pgfplotmarksize}}
\pgfpathlineto{\pgfpoint{+0.3\pgfplotmarksize}{-0.3\pgfplotmarksize}}
\pgfpathlineto{\pgfpoint{+0.3\pgfplotmarksize}{-1.\pgfplotmarksize}}
\pgfpathlineto{\pgfpoint{-0.3\pgfplotmarksize}{-1.\pgfplotmarksize}}
\pgfpathlineto{\pgfpoint{-0.3\pgfplotmarksize}{-0.3\pgfplotmarksize}}
\pgfpathlineto{\pgfpoint{-1.\pgfplotmarksize}{-0.3\pgfplotmarksize}}
\pgfpathlineto{\pgfpoint{-1.\pgfplotmarksize}{0.3\pgfplotmarksize}}
\pgfpathlineto{\pgfpoint{-0.3\pgfplotmarksize}{0.3\pgfplotmarksize}}
\pgfpathclose
\pgfusepathqstroke
}
\pgfdeclareplotmark{cross*} {
\pgfpathmoveto{\pgfpoint{-0.3\pgfplotmarksize}{\pgfplotmarksize}}
\pgfpathlineto{\pgfpoint{+0.3\pgfplotmarksize}{\pgfplotmarksize}}
\pgfpathlineto{\pgfpoint{+0.3\pgfplotmarksize}{0.3\pgfplotmarksize}}
\pgfpathlineto{\pgfpoint{+1\pgfplotmarksize}{0.3\pgfplotmarksize}}
\pgfpathlineto{\pgfpoint{+1\pgfplotmarksize}{-0.3\pgfplotmarksize}}
\pgfpathlineto{\pgfpoint{+0.3\pgfplotmarksize}{-0.3\pgfplotmarksize}}
\pgfpathlineto{\pgfpoint{+0.3\pgfplotmarksize}{-1.\pgfplotmarksize}}
\pgfpathlineto{\pgfpoint{-0.3\pgfplotmarksize}{-1.\pgfplotmarksize}}
\pgfpathlineto{\pgfpoint{-0.3\pgfplotmarksize}{-0.3\pgfplotmarksize}}
\pgfpathlineto{\pgfpoint{-1.\pgfplotmarksize}{-0.3\pgfplotmarksize}}
\pgfpathlineto{\pgfpoint{-1.\pgfplotmarksize}{0.3\pgfplotmarksize}}
\pgfpathlineto{\pgfpoint{-0.3\pgfplotmarksize}{0.3\pgfplotmarksize}}
\pgfpathclose
\pgfusepathqfillstroke
}
\pgfdeclareplotmark{newstar} {
\pgfpathmoveto{\pgfqpoint{0pt}{\pgfplotmarksize}}
\pgfpathlineto{\pgfqpointpolar{44}{0.5\pgfplotmarksize}}
\pgfpathlineto{\pgfqpointpolar{18}{\pgfplotmarksize}}
\pgfpathlineto{\pgfqpointpolar{-20}{0.5\pgfplotmarksize}}
\pgfpathlineto{\pgfqpointpolar{-54}{\pgfplotmarksize}}
\pgfpathlineto{\pgfqpointpolar{-90}{0.5\pgfplotmarksize}}
\pgfpathlineto{\pgfqpointpolar{234}{\pgfplotmarksize}}
\pgfpathlineto{\pgfqpointpolar{198}{0.5\pgfplotmarksize}}
\pgfpathlineto{\pgfqpointpolar{162}{\pgfplotmarksize}}
\pgfpathlineto{\pgfqpointpolar{134}{0.5\pgfplotmarksize}}
\pgfpathclose
\pgfusepathqstroke
}
\pgfdeclareplotmark{newstar*} {
\pgfpathmoveto{\pgfqpoint{0pt}{\pgfplotmarksize}}
\pgfpathlineto{\pgfqpointpolar{44}{0.5\pgfplotmarksize}}
\pgfpathlineto{\pgfqpointpolar{18}{\pgfplotmarksize}}
\pgfpathlineto{\pgfqpointpolar{-20}{0.5\pgfplotmarksize}}
\pgfpathlineto{\pgfqpointpolar{-54}{\pgfplotmarksize}}
\pgfpathlineto{\pgfqpointpolar{-90}{0.5\pgfplotmarksize}}
\pgfpathlineto{\pgfqpointpolar{234}{\pgfplotmarksize}}
\pgfpathlineto{\pgfqpointpolar{198}{0.5\pgfplotmarksize}}
\pgfpathlineto{\pgfqpointpolar{162}{\pgfplotmarksize}}
\pgfpathlineto{\pgfqpointpolar{134}{0.5\pgfplotmarksize}}
\pgfpathclose
\pgfusepathqfillstroke
}
\definecolor{c}{rgb}{1,1,1};
% \draw [color=c, fill=c] (0,0) rectangle (10,6.79598);
% \draw [color=c, fill=c] (1,0.679598) rectangle (9.8,6.66006);
\definecolor{c}{rgb}{0,0,0};
\draw [c,line width=0.9] (1,0.679598) -- (1,6.66006) -- (9.8,6.66006) -- (9.8,0.679598) -- (1,0.679598);
\colorlet{c}{kugray};
\draw [c,line width=0.9] (1.22,0.69155) -- (1.22,0.725222);
\draw [c,line width=0.9] (1.22,0.725222) -- (1.22,0.758894);
\draw [c,line width=0.9] (1.176,0.725222) -- (1.22,0.725222);
\draw [c,line width=0.9] (1.22,0.725222) -- (1.264,0.725222);
\definecolor{c}{rgb}{0,0,0};
\foreach \P in {(1.22,0.725222)}{\draw[mark options={color=c,fill=c},mark size=2.402402pt,mark=*,mark size=1pt] plot coordinates {\P};}
\colorlet{c}{kugray};
\draw [c,line width=0.9] (1.308,0.891522) -- (1.308,0.987411);
\draw [c,line width=0.9] (1.308,0.987411) -- (1.308,1.0833);
\draw [c,line width=0.9] (1.264,0.987411) -- (1.308,0.987411);
\draw [c,line width=0.9] (1.308,0.987411) -- (1.352,0.987411);
\definecolor{c}{rgb}{0,0,0};
\foreach \P in {(1.308,0.987411)}{\draw[mark options={color=c,fill=c},mark size=2.402402pt,mark=*,mark size=1pt] plot coordinates {\P};}
\colorlet{c}{kugray};
\draw [c,line width=0.9] (1.396,0.851697) -- (1.396,0.92657);
\draw [c,line width=0.9] (1.396,0.92657) -- (1.396,1.00144);
\draw [c,line width=0.9] (1.352,0.92657) -- (1.396,0.92657);
\draw [c,line width=0.9] (1.396,0.92657) -- (1.44,0.92657);
\definecolor{c}{rgb}{0,0,0};
\foreach \P in {(1.396,0.92657)}{\draw[mark options={color=c,fill=c},mark size=2.402402pt,mark=*,mark size=1pt] plot coordinates {\P};}
\colorlet{c}{kugray};
\draw [c,line width=0.9] (1.484,0.76652) -- (1.484,0.826614);
\draw [c,line width=0.9] (1.484,0.826614) -- (1.484,0.886709);
\draw [c,line width=0.9] (1.44,0.826614) -- (1.484,0.826614);
\draw [c,line width=0.9] (1.484,0.826614) -- (1.528,0.826614);
\definecolor{c}{rgb}{0,0,0};
\foreach \P in {(1.484,0.826614)}{\draw[mark options={color=c,fill=c},mark size=2.402402pt,mark=*,mark size=1pt] plot coordinates {\P};}
\colorlet{c}{kugray};
\draw [c,line width=0.9] (1.572,0.805142) -- (1.572,0.867335);
\draw [c,line width=0.9] (1.572,0.867335) -- (1.572,0.929529);
\draw [c,line width=0.9] (1.528,0.867335) -- (1.572,0.867335);
\draw [c,line width=0.9] (1.572,0.867335) -- (1.616,0.867335);
\definecolor{c}{rgb}{0,0,0};
\foreach \P in {(1.572,0.867335)}{\draw[mark options={color=c,fill=c},mark size=2.402402pt,mark=*,mark size=1pt] plot coordinates {\P};}
\colorlet{c}{kugray};
\draw [c,line width=0.9] (1.66,0.933718) -- (1.66,1.0243);
\draw [c,line width=0.9] (1.66,1.0243) -- (1.66,1.11487);
\draw [c,line width=0.9] (1.616,1.0243) -- (1.66,1.0243);
\draw [c,line width=0.9] (1.66,1.0243) -- (1.704,1.0243);
\definecolor{c}{rgb}{0,0,0};
\foreach \P in {(1.66,1.0243)}{\draw[mark options={color=c,fill=c},mark size=2.402402pt,mark=*,mark size=1pt] plot coordinates {\P};}
\colorlet{c}{kugray};
\draw [c,line width=0.9] (1.748,0.968817) -- (1.748,1.04905);
\draw [c,line width=0.9] (1.748,1.04905) -- (1.748,1.12928);
\draw [c,line width=0.9] (1.704,1.04905) -- (1.748,1.04905);
\draw [c,line width=0.9] (1.748,1.04905) -- (1.792,1.04905);
\definecolor{c}{rgb}{0,0,0};
\foreach \P in {(1.748,1.04905)}{\draw[mark options={color=c,fill=c},mark size=2.402402pt,mark=*,mark size=1pt] plot coordinates {\P};}
\colorlet{c}{kugray};
\draw [c,line width=0.9] (1.836,1.04632) -- (1.836,1.13214);
\draw [c,line width=0.9] (1.836,1.13214) -- (1.836,1.21796);
\draw [c,line width=0.9] (1.792,1.13214) -- (1.836,1.13214);
\draw [c,line width=0.9] (1.836,1.13214) -- (1.88,1.13214);
\definecolor{c}{rgb}{0,0,0};
\foreach \P in {(1.836,1.13214)}{\draw[mark options={color=c,fill=c},mark size=2.402402pt,mark=*,mark size=1pt] plot coordinates {\P};}
\colorlet{c}{kugray};
\draw [c,line width=0.9] (1.924,2.32224) -- (1.924,2.42303);
\draw [c,line width=0.9] (1.924,2.42303) -- (1.924,2.52381);
\draw [c,line width=0.9] (1.88,2.42303) -- (1.924,2.42303);
\draw [c,line width=0.9] (1.924,2.42303) -- (1.968,2.42303);
\definecolor{c}{rgb}{0,0,0};
\foreach \P in {(1.924,2.42303)}{\draw[mark options={color=c,fill=c},mark size=2.402402pt,mark=*,mark size=1pt] plot coordinates {\P};}
\colorlet{c}{kugray};
\draw [c,line width=0.9] (2.012,4.1446) -- (2.012,4.27979);
\draw [c,line width=0.9] (2.012,4.27979) -- (2.012,4.41498);
\draw [c,line width=0.9] (1.968,4.27979) -- (2.012,4.27979);
\draw [c,line width=0.9] (2.012,4.27979) -- (2.056,4.27979);
\definecolor{c}{rgb}{0,0,0};
\foreach \P in {(2.012,4.27979)}{\draw[mark options={color=c,fill=c},mark size=2.402402pt,mark=*,mark size=1pt] plot coordinates {\P};}
\colorlet{c}{kugray};
\draw [c,line width=0.9] (2.1,5.18788) -- (2.1,5.38242);
\draw [c,line width=0.9] (2.1,5.38242) -- (2.1,5.57696);
\draw [c,line width=0.9] (2.056,5.38242) -- (2.1,5.38242);
\draw [c,line width=0.9] (2.1,5.38242) -- (2.144,5.38242);
\definecolor{c}{rgb}{0,0,0};
\foreach \P in {(2.1,5.38242)}{\draw[mark options={color=c,fill=c},mark size=2.402402pt,mark=*,mark size=1pt] plot coordinates {\P};}
\colorlet{c}{kugray};
\draw [c,line width=0.9] (2.188,5.21472) -- (2.188,5.42535);
\draw [c,line width=0.9] (2.188,5.42535) -- (2.188,5.63598);
\draw [c,line width=0.9] (2.144,5.42535) -- (2.188,5.42535);
\draw [c,line width=0.9] (2.188,5.42535) -- (2.232,5.42535);
\definecolor{c}{rgb}{0,0,0};
\foreach \P in {(2.188,5.42535)}{\draw[mark options={color=c,fill=c},mark size=2.402402pt,mark=*,mark size=1pt] plot coordinates {\P};}
\colorlet{c}{kugray};
\draw [c,line width=0.9] (2.276,5.82473) -- (2.276,6.1);
\draw [c,line width=0.9] (2.276,6.1) -- (2.276,6.37527);
\draw [c,line width=0.9] (2.232,6.1) -- (2.276,6.1);
\draw [c,line width=0.9] (2.276,6.1) -- (2.32,6.1);
\definecolor{c}{rgb}{0,0,0};
\foreach \P in {(2.276,6.1)}{\draw[mark options={color=c,fill=c},mark size=2.402402pt,mark=*,mark size=1pt] plot coordinates {\P};}
\colorlet{c}{kugray};
\draw [c,line width=0.9] (2.364,4.79152) -- (2.364,5.02074);
\draw [c,line width=0.9] (2.364,5.02074) -- (2.364,5.24995);
\draw [c,line width=0.9] (2.32,5.02074) -- (2.364,5.02074);
\draw [c,line width=0.9] (2.364,5.02074) -- (2.408,5.02074);
\definecolor{c}{rgb}{0,0,0};
\foreach \P in {(2.364,5.02074)}{\draw[mark options={color=c,fill=c},mark size=2.402402pt,mark=*,mark size=1pt] plot coordinates {\P};}
\colorlet{c}{kugray};
\draw [c,line width=0.9] (2.452,4.43791) -- (2.452,4.67658);
\draw [c,line width=0.9] (2.452,4.67658) -- (2.452,4.91525);
\draw [c,line width=0.9] (2.408,4.67658) -- (2.452,4.67658);
\draw [c,line width=0.9] (2.452,4.67658) -- (2.496,4.67658);
\definecolor{c}{rgb}{0,0,0};
\foreach \P in {(2.452,4.67658)}{\draw[mark options={color=c,fill=c},mark size=2.402402pt,mark=*,mark size=1pt] plot coordinates {\P};}
\colorlet{c}{kugray};
\draw [c,line width=0.9] (2.54,3.88875) -- (2.54,4.1051);
\draw [c,line width=0.9] (2.54,4.1051) -- (2.54,4.32144);
\draw [c,line width=0.9] (2.496,4.1051) -- (2.54,4.1051);
\draw [c,line width=0.9] (2.54,4.1051) -- (2.584,4.1051);
\definecolor{c}{rgb}{0,0,0};
\foreach \P in {(2.54,4.1051)}{\draw[mark options={color=c,fill=c},mark size=2.402402pt,mark=*,mark size=1pt] plot coordinates {\P};}
\colorlet{c}{kugray};
\draw [c,line width=0.9] (2.628,3.22791) -- (2.628,3.41227);
\draw [c,line width=0.9] (2.628,3.41227) -- (2.628,3.59664);
\draw [c,line width=0.9] (2.584,3.41227) -- (2.628,3.41227);
\draw [c,line width=0.9] (2.628,3.41227) -- (2.672,3.41227);
\definecolor{c}{rgb}{0,0,0};
\foreach \P in {(2.628,3.41227)}{\draw[mark options={color=c,fill=c},mark size=2.402402pt,mark=*,mark size=1pt] plot coordinates {\P};}
\colorlet{c}{kugray};
\draw [c,line width=0.9] (2.716,2.82361) -- (2.716,2.98148);
\draw [c,line width=0.9] (2.716,2.98148) -- (2.716,3.13934);
\draw [c,line width=0.9] (2.672,2.98148) -- (2.716,2.98148);
\draw [c,line width=0.9] (2.716,2.98148) -- (2.76,2.98148);
\definecolor{c}{rgb}{0,0,0};
\foreach \P in {(2.716,2.98148)}{\draw[mark options={color=c,fill=c},mark size=2.402402pt,mark=*,mark size=1pt] plot coordinates {\P};}
\colorlet{c}{kugray};
\draw [c,line width=0.9] (2.804,2.61139) -- (2.804,2.78244);
\draw [c,line width=0.9] (2.804,2.78244) -- (2.804,2.9535);
\draw [c,line width=0.9] (2.76,2.78244) -- (2.804,2.78244);
\draw [c,line width=0.9] (2.804,2.78244) -- (2.848,2.78244);
\definecolor{c}{rgb}{0,0,0};
\foreach \P in {(2.804,2.78244)}{\draw[mark options={color=c,fill=c},mark size=2.402402pt,mark=*,mark size=1pt] plot coordinates {\P};}
\colorlet{c}{kugray};
\draw [c,line width=0.9] (2.892,2.43748) -- (2.892,2.60718);
\draw [c,line width=0.9] (2.892,2.60718) -- (2.892,2.77688);
\draw [c,line width=0.9] (2.848,2.60718) -- (2.892,2.60718);
\draw [c,line width=0.9] (2.892,2.60718) -- (2.936,2.60718);
\definecolor{c}{rgb}{0,0,0};
\foreach \P in {(2.892,2.60718)}{\draw[mark options={color=c,fill=c},mark size=2.402402pt,mark=*,mark size=1pt] plot coordinates {\P};}
\colorlet{c}{kugray};
\draw [c,line width=0.9] (2.98,1.91521) -- (2.98,2.046);
\draw [c,line width=0.9] (2.98,2.046) -- (2.98,2.17679);
\draw [c,line width=0.9] (2.936,2.046) -- (2.98,2.046);
\draw [c,line width=0.9] (2.98,2.046) -- (3.024,2.046);
\definecolor{c}{rgb}{0,0,0};
\foreach \P in {(2.98,2.046)}{\draw[mark options={color=c,fill=c},mark size=2.402402pt,mark=*,mark size=1pt] plot coordinates {\P};}
\colorlet{c}{kugray};
\draw [c,line width=0.9] (3.068,1.81618) -- (3.068,1.94136);
\draw [c,line width=0.9] (3.068,1.94136) -- (3.068,2.06654);
\draw [c,line width=0.9] (3.024,1.94136) -- (3.068,1.94136);
\draw [c,line width=0.9] (3.068,1.94136) -- (3.112,1.94136);
\definecolor{c}{rgb}{0,0,0};
\foreach \P in {(3.068,1.94136)}{\draw[mark options={color=c,fill=c},mark size=2.402402pt,mark=*,mark size=1pt] plot coordinates {\P};}
\colorlet{c}{kugray};
\draw [c,line width=0.9] (3.156,1.47213) -- (3.156,1.57105);
\draw [c,line width=0.9] (3.156,1.57105) -- (3.156,1.66996);
\draw [c,line width=0.9] (3.112,1.57105) -- (3.156,1.57105);
\draw [c,line width=0.9] (3.156,1.57105) -- (3.2,1.57105);
\definecolor{c}{rgb}{0,0,0};
\foreach \P in {(3.156,1.57105)}{\draw[mark options={color=c,fill=c},mark size=2.402402pt,mark=*,mark size=1pt] plot coordinates {\P};}
\colorlet{c}{kugray};
\draw [c,line width=0.9] (3.244,1.52436) -- (3.244,1.64249);
\draw [c,line width=0.9] (3.244,1.64249) -- (3.244,1.76062);
\draw [c,line width=0.9] (3.2,1.64249) -- (3.244,1.64249);
\draw [c,line width=0.9] (3.244,1.64249) -- (3.288,1.64249);
\definecolor{c}{rgb}{0,0,0};
\foreach \P in {(3.244,1.64249)}{\draw[mark options={color=c,fill=c},mark size=2.402402pt,mark=*,mark size=1pt] plot coordinates {\P};}
\colorlet{c}{kugray};
\draw [c,line width=0.9] (3.332,1.39019) -- (3.332,1.48945);
\draw [c,line width=0.9] (3.332,1.48945) -- (3.332,1.58872);
\draw [c,line width=0.9] (3.288,1.48945) -- (3.332,1.48945);
\draw [c,line width=0.9] (3.332,1.48945) -- (3.376,1.48945);
\definecolor{c}{rgb}{0,0,0};
\foreach \P in {(3.332,1.48945)}{\draw[mark options={color=c,fill=c},mark size=2.402402pt,mark=*,mark size=1pt] plot coordinates {\P};}
\colorlet{c}{kugray};
\draw [c,line width=0.9] (3.42,1.26448) -- (3.42,1.34887);
\draw [c,line width=0.9] (3.42,1.34887) -- (3.42,1.43326);
\draw [c,line width=0.9] (3.376,1.34887) -- (3.42,1.34887);
\draw [c,line width=0.9] (3.42,1.34887) -- (3.464,1.34887);
\definecolor{c}{rgb}{0,0,0};
\foreach \P in {(3.42,1.34887)}{\draw[mark options={color=c,fill=c},mark size=2.402402pt,mark=*,mark size=1pt] plot coordinates {\P};}
\colorlet{c}{kugray};
\draw [c,line width=0.9] (3.508,1.16295) -- (3.508,1.23709);
\draw [c,line width=0.9] (3.508,1.23709) -- (3.508,1.31124);
\draw [c,line width=0.9] (3.464,1.23709) -- (3.508,1.23709);
\draw [c,line width=0.9] (3.508,1.23709) -- (3.552,1.23709);
\definecolor{c}{rgb}{0,0,0};
\foreach \P in {(3.508,1.23709)}{\draw[mark options={color=c,fill=c},mark size=2.402402pt,mark=*,mark size=1pt] plot coordinates {\P};}
\colorlet{c}{kugray};
\draw [c,line width=0.9] (3.596,1.1455) -- (3.596,1.2477);
\draw [c,line width=0.9] (3.596,1.2477) -- (3.596,1.3499);
\draw [c,line width=0.9] (3.552,1.2477) -- (3.596,1.2477);
\draw [c,line width=0.9] (3.596,1.2477) -- (3.64,1.2477);
\definecolor{c}{rgb}{0,0,0};
\foreach \P in {(3.596,1.2477)}{\draw[mark options={color=c,fill=c},mark size=2.402402pt,mark=*,mark size=1pt] plot coordinates {\P};}
\colorlet{c}{kugray};
\draw [c,line width=0.9] (3.684,1.17095) -- (3.684,1.26753);
\draw [c,line width=0.9] (3.684,1.26753) -- (3.684,1.36411);
\draw [c,line width=0.9] (3.64,1.26753) -- (3.684,1.26753);
\draw [c,line width=0.9] (3.684,1.26753) -- (3.728,1.26753);
\definecolor{c}{rgb}{0,0,0};
\foreach \P in {(3.684,1.26753)}{\draw[mark options={color=c,fill=c},mark size=2.402402pt,mark=*,mark size=1pt] plot coordinates {\P};}
\colorlet{c}{kugray};
\draw [c,line width=0.9] (3.772,1.00902) -- (3.772,1.08017);
\draw [c,line width=0.9] (3.772,1.08017) -- (3.772,1.15133);
\draw [c,line width=0.9] (3.728,1.08017) -- (3.772,1.08017);
\draw [c,line width=0.9] (3.772,1.08017) -- (3.816,1.08017);
\definecolor{c}{rgb}{0,0,0};
\foreach \P in {(3.772,1.08017)}{\draw[mark options={color=c,fill=c},mark size=2.402402pt,mark=*,mark size=1pt] plot coordinates {\P};}
\colorlet{c}{kugray};
\draw [c,line width=0.9] (3.86,0.96861) -- (3.86,1.03792);
\draw [c,line width=0.9] (3.86,1.03792) -- (3.86,1.10723);
\draw [c,line width=0.9] (3.816,1.03792) -- (3.86,1.03792);
\draw [c,line width=0.9] (3.86,1.03792) -- (3.904,1.03792);
\definecolor{c}{rgb}{0,0,0};
\foreach \P in {(3.86,1.03792)}{\draw[mark options={color=c,fill=c},mark size=2.402402pt,mark=*,mark size=1pt] plot coordinates {\P};}
\colorlet{c}{kugray};
\draw [c,line width=0.9] (3.948,0.895759) -- (3.948,0.959071);
\draw [c,line width=0.9] (3.948,0.959071) -- (3.948,1.02238);
\draw [c,line width=0.9] (3.904,0.959071) -- (3.948,0.959071);
\draw [c,line width=0.9] (3.948,0.959071) -- (3.992,0.959071);
\definecolor{c}{rgb}{0,0,0};
\foreach \P in {(3.948,0.959071)}{\draw[mark options={color=c,fill=c},mark size=2.402402pt,mark=*,mark size=1pt] plot coordinates {\P};}
\colorlet{c}{kugray};
\draw [c,line width=0.9] (4.036,0.875785) -- (4.036,0.927766);
\draw [c,line width=0.9] (4.036,0.927766) -- (4.036,0.979746);
\draw [c,line width=0.9] (3.992,0.927766) -- (4.036,0.927766);
\draw [c,line width=0.9] (4.036,0.927766) -- (4.08,0.927766);
\definecolor{c}{rgb}{0,0,0};
\foreach \P in {(4.036,0.927766)}{\draw[mark options={color=c,fill=c},mark size=2.402402pt,mark=*,mark size=1pt] plot coordinates {\P};}
\colorlet{c}{kugray};
\draw [c,line width=0.9] (4.124,0.85319) -- (4.124,0.904832);
\draw [c,line width=0.9] (4.124,0.904832) -- (4.124,0.956474);
\draw [c,line width=0.9] (4.08,0.904832) -- (4.124,0.904832);
\draw [c,line width=0.9] (4.124,0.904832) -- (4.168,0.904832);
\definecolor{c}{rgb}{0,0,0};
\foreach \P in {(4.124,0.904832)}{\draw[mark options={color=c,fill=c},mark size=2.402402pt,mark=*,mark size=1pt] plot coordinates {\P};}
\colorlet{c}{kugray};
\draw [c,line width=0.9] (4.212,0.833545) -- (4.212,0.872968);
\draw [c,line width=0.9] (4.212,0.872968) -- (4.212,0.91239);
\draw [c,line width=0.9] (4.168,0.872968) -- (4.212,0.872968);
\draw [c,line width=0.9] (4.212,0.872968) -- (4.256,0.872968);
\definecolor{c}{rgb}{0,0,0};
\foreach \P in {(4.212,0.872968)}{\draw[mark options={color=c,fill=c},mark size=2.402402pt,mark=*,mark size=1pt] plot coordinates {\P};}
\colorlet{c}{kugray};
\draw [c,line width=0.9] (4.3,0.904478) -- (4.3,0.983047);
\draw [c,line width=0.9] (4.3,0.983047) -- (4.3,1.06162);
\draw [c,line width=0.9] (4.256,0.983047) -- (4.3,0.983047);
\draw [c,line width=0.9] (4.3,0.983047) -- (4.344,0.983047);
\definecolor{c}{rgb}{0,0,0};
\foreach \P in {(4.3,0.983047)}{\draw[mark options={color=c,fill=c},mark size=2.402402pt,mark=*,mark size=1pt] plot coordinates {\P};}
\colorlet{c}{kugray};
\draw [c,line width=0.9] (4.388,0.766651) -- (4.388,0.778978);
\draw [c,line width=0.9] (4.388,0.778978) -- (4.388,0.791304);
\draw [c,line width=0.9] (4.344,0.778978) -- (4.388,0.778978);
\draw [c,line width=0.9] (4.388,0.778978) -- (4.432,0.778978);
\definecolor{c}{rgb}{0,0,0};
\foreach \P in {(4.388,0.778978)}{\draw[mark options={color=c,fill=c},mark size=2.402402pt,mark=*,mark size=1pt] plot coordinates {\P};}
\colorlet{c}{kugray};
\draw [c,line width=0.9] (4.476,0.763784) -- (4.476,0.77592);
\draw [c,line width=0.9] (4.476,0.77592) -- (4.476,0.788055);
\draw [c,line width=0.9] (4.432,0.77592) -- (4.476,0.77592);
\draw [c,line width=0.9] (4.476,0.77592) -- (4.52,0.77592);
\definecolor{c}{rgb}{0,0,0};
\foreach \P in {(4.476,0.77592)}{\draw[mark options={color=c,fill=c},mark size=2.402402pt,mark=*,mark size=1pt] plot coordinates {\P};}
\colorlet{c}{kugray};
\draw [c,line width=0.9] (4.564,0.799068) -- (4.564,0.837803);
\draw [c,line width=0.9] (4.564,0.837803) -- (4.564,0.876537);
\draw [c,line width=0.9] (4.52,0.837803) -- (4.564,0.837803);
\draw [c,line width=0.9] (4.564,0.837803) -- (4.608,0.837803);
\definecolor{c}{rgb}{0,0,0};
\foreach \P in {(4.564,0.837803)}{\draw[mark options={color=c,fill=c},mark size=2.402402pt,mark=*,mark size=1pt] plot coordinates {\P};}
\colorlet{c}{kugray};
\draw [c,line width=0.9] (4.652,0.749851) -- (4.652,0.78485);
\draw [c,line width=0.9] (4.652,0.78485) -- (4.652,0.81985);
\draw [c,line width=0.9] (4.608,0.78485) -- (4.652,0.78485);
\draw [c,line width=0.9] (4.652,0.78485) -- (4.696,0.78485);
\definecolor{c}{rgb}{0,0,0};
\foreach \P in {(4.652,0.78485)}{\draw[mark options={color=c,fill=c},mark size=2.402402pt,mark=*,mark size=1pt] plot coordinates {\P};}
\colorlet{c}{kugray};
\draw [c,line width=0.9] (4.74,0.743812) -- (4.74,0.754515);
\draw [c,line width=0.9] (4.74,0.754515) -- (4.74,0.765217);
\draw [c,line width=0.9] (4.696,0.754515) -- (4.74,0.754515);
\draw [c,line width=0.9] (4.74,0.754515) -- (4.784,0.754515);
\definecolor{c}{rgb}{0,0,0};
\foreach \P in {(4.74,0.754515)}{\draw[mark options={color=c,fill=c},mark size=2.402402pt,mark=*,mark size=1pt] plot coordinates {\P};}
\colorlet{c}{kugray};
\draw [c,line width=0.9] (4.828,0.785728) -- (4.828,0.829039);
\draw [c,line width=0.9] (4.828,0.829039) -- (4.828,0.87235);
\draw [c,line width=0.9] (4.784,0.829039) -- (4.828,0.829039);
\draw [c,line width=0.9] (4.828,0.829039) -- (4.872,0.829039);
\definecolor{c}{rgb}{0,0,0};
\foreach \P in {(4.828,0.829039)}{\draw[mark options={color=c,fill=c},mark size=2.402402pt,mark=*,mark size=1pt] plot coordinates {\P};}
\colorlet{c}{kugray};
\draw [c,line width=0.9] (4.916,0.752579) -- (4.916,0.792904);
\draw [c,line width=0.9] (4.916,0.792904) -- (4.916,0.83323);
\draw [c,line width=0.9] (4.872,0.792904) -- (4.916,0.792904);
\draw [c,line width=0.9] (4.916,0.792904) -- (4.96,0.792904);
\definecolor{c}{rgb}{0,0,0};
\foreach \P in {(4.916,0.792904)}{\draw[mark options={color=c,fill=c},mark size=2.402402pt,mark=*,mark size=1pt] plot coordinates {\P};}
\colorlet{c}{kugray};
\draw [c,line width=0.9] (5.004,0.743631) -- (5.004,0.781233);
\draw [c,line width=0.9] (5.004,0.781233) -- (5.004,0.818834);
\draw [c,line width=0.9] (4.96,0.781233) -- (5.004,0.781233);
\draw [c,line width=0.9] (5.004,0.781233) -- (5.048,0.781233);
\definecolor{c}{rgb}{0,0,0};
\foreach \P in {(5.004,0.781233)}{\draw[mark options={color=c,fill=c},mark size=2.402402pt,mark=*,mark size=1pt] plot coordinates {\P};}
\colorlet{c}{kugray};
\draw [c,line width=0.9] (5.092,0.725926) -- (5.092,0.760387);
\draw [c,line width=0.9] (5.092,0.760387) -- (5.092,0.794849);
\draw [c,line width=0.9] (5.048,0.760387) -- (5.092,0.760387);
\draw [c,line width=0.9] (5.092,0.760387) -- (5.136,0.760387);
\definecolor{c}{rgb}{0,0,0};
\foreach \P in {(5.092,0.760387)}{\draw[mark options={color=c,fill=c},mark size=2.402402pt,mark=*,mark size=1pt] plot coordinates {\P};}
\colorlet{c}{kugray};
\draw [c,line width=0.9] (5.18,0.721269) -- (5.18,0.730052);
\draw [c,line width=0.9] (5.18,0.730052) -- (5.18,0.738835);
\draw [c,line width=0.9] (5.136,0.730052) -- (5.18,0.730052);
\draw [c,line width=0.9] (5.18,0.730052) -- (5.224,0.730052);
\definecolor{c}{rgb}{0,0,0};
\foreach \P in {(5.18,0.730052)}{\draw[mark options={color=c,fill=c},mark size=2.402402pt,mark=*,mark size=1pt] plot coordinates {\P};}
\colorlet{c}{kugray};
\draw [c,line width=0.9] (5.268,0.703339) -- (5.268,0.710176);
\draw [c,line width=0.9] (5.268,0.710176) -- (5.268,0.717014);
\draw [c,line width=0.9] (5.224,0.710176) -- (5.268,0.710176);
\draw [c,line width=0.9] (5.268,0.710176) -- (5.312,0.710176);
\definecolor{c}{rgb}{0,0,0};
\foreach \P in {(5.268,0.710176)}{\draw[mark options={color=c,fill=c},mark size=2.402402pt,mark=*,mark size=1pt] plot coordinates {\P};}
\colorlet{c}{kugray};
\draw [c,line width=0.9] (5.356,0.717711) -- (5.356,0.739635);
\draw [c,line width=0.9] (5.356,0.739635) -- (5.356,0.76156);
\draw [c,line width=0.9] (5.312,0.739635) -- (5.356,0.739635);
\draw [c,line width=0.9] (5.356,0.739635) -- (5.4,0.739635);
\definecolor{c}{rgb}{0,0,0};
\foreach \P in {(5.356,0.739635)}{\draw[mark options={color=c,fill=c},mark size=2.402402pt,mark=*,mark size=1pt] plot coordinates {\P};}
\colorlet{c}{kugray};
\draw [c,line width=0.9] (5.444,0.722936) -- (5.444,0.75733);
\draw [c,line width=0.9] (5.444,0.75733) -- (5.444,0.791723);
\draw [c,line width=0.9] (5.4,0.75733) -- (5.444,0.75733);
\draw [c,line width=0.9] (5.444,0.75733) -- (5.488,0.75733);
\definecolor{c}{rgb}{0,0,0};
\foreach \P in {(5.444,0.75733)}{\draw[mark options={color=c,fill=c},mark size=2.402402pt,mark=*,mark size=1pt] plot coordinates {\P};}
\colorlet{c}{kugray};
\draw [c,line width=0.9] (5.532,0.701983) -- (5.532,0.708647);
\draw [c,line width=0.9] (5.532,0.708647) -- (5.532,0.715312);
\draw [c,line width=0.9] (5.488,0.708647) -- (5.532,0.708647);
\draw [c,line width=0.9] (5.532,0.708647) -- (5.576,0.708647);
\definecolor{c}{rgb}{0,0,0};
\foreach \P in {(5.532,0.708647)}{\draw[mark options={color=c,fill=c},mark size=2.402402pt,mark=*,mark size=1pt] plot coordinates {\P};}
\colorlet{c}{kugray};
\draw [c,line width=0.9] (5.62,0.707532) -- (5.62,0.715501);
\draw [c,line width=0.9] (5.62,0.715501) -- (5.62,0.723469);
\draw [c,line width=0.9] (5.576,0.715501) -- (5.62,0.715501);
\draw [c,line width=0.9] (5.62,0.715501) -- (5.664,0.715501);
\definecolor{c}{rgb}{0,0,0};
\foreach \P in {(5.62,0.715501)}{\draw[mark options={color=c,fill=c},mark size=2.402402pt,mark=*,mark size=1pt] plot coordinates {\P};}
\colorlet{c}{kugray};
\draw [c,line width=0.9] (5.708,0.708802) -- (5.708,0.716292);
\draw [c,line width=0.9] (5.708,0.716292) -- (5.708,0.723782);
\draw [c,line width=0.9] (5.664,0.716292) -- (5.708,0.716292);
\draw [c,line width=0.9] (5.708,0.716292) -- (5.752,0.716292);
\definecolor{c}{rgb}{0,0,0};
\foreach \P in {(5.708,0.716292)}{\draw[mark options={color=c,fill=c},mark size=2.402402pt,mark=*,mark size=1pt] plot coordinates {\P};}
\colorlet{c}{kugray};
\draw [c,line width=0.9] (5.796,0.699022) -- (5.796,0.732867);
\draw [c,line width=0.9] (5.796,0.732867) -- (5.796,0.766712);
\draw [c,line width=0.9] (5.752,0.732867) -- (5.796,0.732867);
\draw [c,line width=0.9] (5.796,0.732867) -- (5.84,0.732867);
\definecolor{c}{rgb}{0,0,0};
\foreach \P in {(5.796,0.732867)}{\draw[mark options={color=c,fill=c},mark size=2.402402pt,mark=*,mark size=1pt] plot coordinates {\P};}
\colorlet{c}{kugray};
\draw [c,line width=0.9] (5.884,0.69266) -- (5.884,0.708088);
\draw [c,line width=0.9] (5.884,0.708088) -- (5.884,0.723515);
\draw [c,line width=0.9] (5.84,0.708088) -- (5.884,0.708088);
\draw [c,line width=0.9] (5.884,0.708088) -- (5.928,0.708088);
\definecolor{c}{rgb}{0,0,0};
\foreach \P in {(5.884,0.708088)}{\draw[mark options={color=c,fill=c},mark size=2.402402pt,mark=*,mark size=1pt] plot coordinates {\P};}
\colorlet{c}{kugray};
\draw [c,line width=0.9] (5.972,0.703505) -- (5.972,0.737454);
\draw [c,line width=0.9] (5.972,0.737454) -- (5.972,0.771402);
\draw [c,line width=0.9] (5.928,0.737454) -- (5.972,0.737454);
\draw [c,line width=0.9] (5.972,0.737454) -- (6.016,0.737454);
\definecolor{c}{rgb}{0,0,0};
\foreach \P in {(5.972,0.737454)}{\draw[mark options={color=c,fill=c},mark size=2.402402pt,mark=*,mark size=1pt] plot coordinates {\P};}
\colorlet{c}{kugray};
\draw [c,line width=0.9] (6.06,0.686255) -- (6.06,0.6903);
\draw [c,line width=0.9] (6.06,0.6903) -- (6.06,0.694345);
\draw [c,line width=0.9] (6.016,0.6903) -- (6.06,0.6903);
\draw [c,line width=0.9] (6.06,0.6903) -- (6.104,0.6903);
\definecolor{c}{rgb}{0,0,0};
\foreach \P in {(6.06,0.6903)}{\draw[mark options={color=c,fill=c},mark size=2.402402pt,mark=*,mark size=1pt] plot coordinates {\P};}
\colorlet{c}{kugray};
\draw [c,line width=0.9] (6.148,0.685026) -- (6.148,0.688771);
\draw [c,line width=0.9] (6.148,0.688771) -- (6.148,0.692516);
\draw [c,line width=0.9] (6.104,0.688771) -- (6.148,0.688771);
\draw [c,line width=0.9] (6.148,0.688771) -- (6.192,0.688771);
\definecolor{c}{rgb}{0,0,0};
\foreach \P in {(6.148,0.688771)}{\draw[mark options={color=c,fill=c},mark size=2.402402pt,mark=*,mark size=1pt] plot coordinates {\P};}
\colorlet{c}{kugray};
\draw [c,line width=0.9] (6.236,0.686255) -- (6.236,0.6903);
\draw [c,line width=0.9] (6.236,0.6903) -- (6.236,0.694345);
\draw [c,line width=0.9] (6.192,0.6903) -- (6.236,0.6903);
\draw [c,line width=0.9] (6.236,0.6903) -- (6.28,0.6903);
\definecolor{c}{rgb}{0,0,0};
\foreach \P in {(6.236,0.6903)}{\draw[mark options={color=c,fill=c},mark size=2.402402pt,mark=*,mark size=1pt] plot coordinates {\P};}
\colorlet{c}{kugray};
\draw [c,line width=0.9] (6.324,0.687505) -- (6.324,0.691829);
\draw [c,line width=0.9] (6.324,0.691829) -- (6.324,0.696154);
\draw [c,line width=0.9] (6.28,0.691829) -- (6.324,0.691829);
\draw [c,line width=0.9] (6.324,0.691829) -- (6.368,0.691829);
\definecolor{c}{rgb}{0,0,0};
\foreach \P in {(6.324,0.691829)}{\draw[mark options={color=c,fill=c},mark size=2.402402pt,mark=*,mark size=1pt] plot coordinates {\P};}
\colorlet{c}{kugray};
\draw [c,line width=0.9] (6.412,0.683824) -- (6.412,0.687242);
\draw [c,line width=0.9] (6.412,0.687242) -- (6.412,0.690661);
\draw [c,line width=0.9] (6.368,0.687242) -- (6.412,0.687242);
\draw [c,line width=0.9] (6.412,0.687242) -- (6.456,0.687242);
\definecolor{c}{rgb}{0,0,0};
\foreach \P in {(6.412,0.687242)}{\draw[mark options={color=c,fill=c},mark size=2.402402pt,mark=*,mark size=1pt] plot coordinates {\P};}
\colorlet{c}{kugray};
\draw [c,line width=0.9] (6.5,0.701702) -- (6.5,0.738423);
\draw [c,line width=0.9] (6.5,0.738423) -- (6.5,0.775144);
\draw [c,line width=0.9] (6.456,0.738423) -- (6.5,0.738423);
\draw [c,line width=0.9] (6.5,0.738423) -- (6.544,0.738423);
\definecolor{c}{rgb}{0,0,0};
\foreach \P in {(6.5,0.738423)}{\draw[mark options={color=c,fill=c},mark size=2.402402pt,mark=*,mark size=1pt] plot coordinates {\P};}
\colorlet{c}{kugray};
\draw [c,line width=0.9] (6.588,0.680493) -- (6.588,0.682656);
\draw [c,line width=0.9] (6.588,0.682656) -- (6.588,0.684818);
\draw [c,line width=0.9] (6.544,0.682656) -- (6.588,0.682656);
\draw [c,line width=0.9] (6.588,0.682656) -- (6.632,0.682656);
\definecolor{c}{rgb}{0,0,0};
\foreach \P in {(6.588,0.682656)}{\draw[mark options={color=c,fill=c},mark size=2.402402pt,mark=*,mark size=1pt] plot coordinates {\P};}
\colorlet{c}{kugray};
\draw [c,line width=0.9] (6.676,0.688771) -- (6.676,0.693358);
\draw [c,line width=0.9] (6.676,0.693358) -- (6.676,0.697945);
\draw [c,line width=0.9] (6.632,0.693358) -- (6.676,0.693358);
\draw [c,line width=0.9] (6.676,0.693358) -- (6.72,0.693358);
\definecolor{c}{rgb}{0,0,0};
\foreach \P in {(6.676,0.693358)}{\draw[mark options={color=c,fill=c},mark size=2.402402pt,mark=*,mark size=1pt] plot coordinates {\P};}
\colorlet{c}{kugray};
\draw [c,line width=0.9] (6.764,0.682656) -- (6.764,0.685713);
\draw [c,line width=0.9] (6.764,0.685713) -- (6.764,0.688771);
\draw [c,line width=0.9] (6.72,0.685713) -- (6.764,0.685713);
\draw [c,line width=0.9] (6.764,0.685713) -- (6.808,0.685713);
\definecolor{c}{rgb}{0,0,0};
\foreach \P in {(6.764,0.685713)}{\draw[mark options={color=c,fill=c},mark size=2.402402pt,mark=*,mark size=1pt] plot coordinates {\P};}
\colorlet{c}{kugray};
\draw [c,line width=0.9] (6.852,0.682656) -- (6.852,0.685713);
\draw [c,line width=0.9] (6.852,0.685713) -- (6.852,0.688771);
\draw [c,line width=0.9] (6.808,0.685713) -- (6.852,0.685713);
\draw [c,line width=0.9] (6.852,0.685713) -- (6.896,0.685713);
\definecolor{c}{rgb}{0,0,0};
\foreach \P in {(6.852,0.685713)}{\draw[mark options={color=c,fill=c},mark size=2.402402pt,mark=*,mark size=1pt] plot coordinates {\P};}
\colorlet{c}{kugray};
\draw [c,line width=0.9] (6.94,0.683824) -- (6.94,0.687242);
\draw [c,line width=0.9] (6.94,0.687242) -- (6.94,0.690661);
\draw [c,line width=0.9] (6.896,0.687242) -- (6.94,0.687242);
\draw [c,line width=0.9] (6.94,0.687242) -- (6.984,0.687242);
\definecolor{c}{rgb}{0,0,0};
\foreach \P in {(6.94,0.687242)}{\draw[mark options={color=c,fill=c},mark size=2.402402pt,mark=*,mark size=1pt] plot coordinates {\P};}
\colorlet{c}{kugray};
\draw [c,line width=0.9] (7.028,0.685026) -- (7.028,0.688771);
\draw [c,line width=0.9] (7.028,0.688771) -- (7.028,0.692516);
\draw [c,line width=0.9] (6.984,0.688771) -- (7.028,0.688771);
\draw [c,line width=0.9] (7.028,0.688771) -- (7.072,0.688771);
\definecolor{c}{rgb}{0,0,0};
\foreach \P in {(7.028,0.688771)}{\draw[mark options={color=c,fill=c},mark size=2.402402pt,mark=*,mark size=1pt] plot coordinates {\P};}
\colorlet{c}{kugray};
\draw [c,line width=0.9] (7.116,0.688771) -- (7.116,0.693358);
\draw [c,line width=0.9] (7.116,0.693358) -- (7.116,0.697945);
\draw [c,line width=0.9] (7.072,0.693358) -- (7.116,0.693358);
\draw [c,line width=0.9] (7.116,0.693358) -- (7.16,0.693358);
\definecolor{c}{rgb}{0,0,0};
\foreach \P in {(7.116,0.693358)}{\draw[mark options={color=c,fill=c},mark size=2.402402pt,mark=*,mark size=1pt] plot coordinates {\P};}
\colorlet{c}{kugray};
\draw [c,line width=0.9] (7.204,0.681536) -- (7.204,0.684184);
\draw [c,line width=0.9] (7.204,0.684184) -- (7.204,0.686833);
\draw [c,line width=0.9] (7.16,0.684184) -- (7.204,0.684184);
\draw [c,line width=0.9] (7.204,0.684184) -- (7.248,0.684184);
\definecolor{c}{rgb}{0,0,0};
\foreach \P in {(7.204,0.684184)}{\draw[mark options={color=c,fill=c},mark size=2.402402pt,mark=*,mark size=1pt] plot coordinates {\P};}
\colorlet{c}{kugray};
\draw [c,line width=0.9] (7.292,0.685026) -- (7.292,0.688771);
\draw [c,line width=0.9] (7.292,0.688771) -- (7.292,0.692516);
\draw [c,line width=0.9] (7.248,0.688771) -- (7.292,0.688771);
\draw [c,line width=0.9] (7.292,0.688771) -- (7.336,0.688771);
\definecolor{c}{rgb}{0,0,0};
\foreach \P in {(7.292,0.688771)}{\draw[mark options={color=c,fill=c},mark size=2.402402pt,mark=*,mark size=1pt] plot coordinates {\P};}
\colorlet{c}{kugray};
\draw [c,line width=0.9] (7.38,0.682656) -- (7.38,0.685713);
\draw [c,line width=0.9] (7.38,0.685713) -- (7.38,0.688771);
\draw [c,line width=0.9] (7.336,0.685713) -- (7.38,0.685713);
\draw [c,line width=0.9] (7.38,0.685713) -- (7.424,0.685713);
\definecolor{c}{rgb}{0,0,0};
\foreach \P in {(7.38,0.685713)}{\draw[mark options={color=c,fill=c},mark size=2.402402pt,mark=*,mark size=1pt] plot coordinates {\P};}
\colorlet{c}{kugray};
\draw [c,line width=0.9] (7.468,0.680493) -- (7.468,0.682656);
\draw [c,line width=0.9] (7.468,0.682656) -- (7.468,0.684818);
\draw [c,line width=0.9] (7.424,0.682656) -- (7.468,0.682656);
\draw [c,line width=0.9] (7.468,0.682656) -- (7.512,0.682656);
\definecolor{c}{rgb}{0,0,0};
\foreach \P in {(7.468,0.682656)}{\draw[mark options={color=c,fill=c},mark size=2.402402pt,mark=*,mark size=1pt] plot coordinates {\P};}
\colorlet{c}{kugray};
\draw [c,line width=0.9] (7.556,0.679598) -- (7.556,0.681127);
\draw [c,line width=0.9] (7.556,0.681127) -- (7.556,0.682656);
\draw [c,line width=0.9] (7.512,0.681127) -- (7.556,0.681127);
\draw [c,line width=0.9] (7.556,0.681127) -- (7.6,0.681127);
\definecolor{c}{rgb}{0,0,0};
\foreach \P in {(7.556,0.681127)}{\draw[mark options={color=c,fill=c},mark size=2.402402pt,mark=*,mark size=1pt] plot coordinates {\P};}
\colorlet{c}{kugray};
\draw [c,line width=0.9] (7.644,0.681536) -- (7.644,0.684184);
\draw [c,line width=0.9] (7.644,0.684184) -- (7.644,0.686833);
\draw [c,line width=0.9] (7.6,0.684184) -- (7.644,0.684184);
\draw [c,line width=0.9] (7.644,0.684184) -- (7.688,0.684184);
\definecolor{c}{rgb}{0,0,0};
\foreach \P in {(7.644,0.684184)}{\draw[mark options={color=c,fill=c},mark size=2.402402pt,mark=*,mark size=1pt] plot coordinates {\P};}
\colorlet{c}{kugray};
\draw [c,line width=0.9] (7.732,0.679598) -- (7.732,0.681127);
\draw [c,line width=0.9] (7.732,0.681127) -- (7.732,0.682656);
\draw [c,line width=0.9] (7.688,0.681127) -- (7.732,0.681127);
\draw [c,line width=0.9] (7.732,0.681127) -- (7.776,0.681127);
\definecolor{c}{rgb}{0,0,0};
\foreach \P in {(7.732,0.681127)}{\draw[mark options={color=c,fill=c},mark size=2.402402pt,mark=*,mark size=1pt] plot coordinates {\P};}
\colorlet{c}{kugray};
\draw [c,line width=0.9] (7.82,0.679598) -- (7.82,0.681127);
\draw [c,line width=0.9] (7.82,0.681127) -- (7.82,0.682656);
\draw [c,line width=0.9] (7.776,0.681127) -- (7.82,0.681127);
\draw [c,line width=0.9] (7.82,0.681127) -- (7.864,0.681127);
\definecolor{c}{rgb}{0,0,0};
\foreach \P in {(7.82,0.681127)}{\draw[mark options={color=c,fill=c},mark size=2.402402pt,mark=*,mark size=1pt] plot coordinates {\P};}
\colorlet{c}{kugray};
\draw [c,line width=0.9] (7.908,0.682498) -- (7.908,0.697385);
\draw [c,line width=0.9] (7.908,0.697385) -- (7.908,0.712273);
\draw [c,line width=0.9] (7.864,0.697385) -- (7.908,0.697385);
\draw [c,line width=0.9] (7.908,0.697385) -- (7.952,0.697385);
\definecolor{c}{rgb}{0,0,0};
\foreach \P in {(7.908,0.697385)}{\draw[mark options={color=c,fill=c},mark size=2.402402pt,mark=*,mark size=1pt] plot coordinates {\P};}
\colorlet{c}{kugray};
\draw [c,line width=0.9] (7.996,0.679598) -- (7.996,0.681127);
\draw [c,line width=0.9] (7.996,0.681127) -- (7.996,0.682656);
\draw [c,line width=0.9] (7.952,0.681127) -- (7.996,0.681127);
\draw [c,line width=0.9] (7.996,0.681127) -- (8.04,0.681127);
\definecolor{c}{rgb}{0,0,0};
\foreach \P in {(7.996,0.681127)}{\draw[mark options={color=c,fill=c},mark size=2.402402pt,mark=*,mark size=1pt] plot coordinates {\P};}
\colorlet{c}{kugray};
\draw [c,line width=0.9] (8.084,0.679598) -- (8.084,0.681127);
\draw [c,line width=0.9] (8.084,0.681127) -- (8.084,0.682656);
\draw [c,line width=0.9] (8.04,0.681127) -- (8.084,0.681127);
\draw [c,line width=0.9] (8.084,0.681127) -- (8.128,0.681127);
\definecolor{c}{rgb}{0,0,0};
\foreach \P in {(8.084,0.681127)}{\draw[mark options={color=c,fill=c},mark size=2.402402pt,mark=*,mark size=1pt] plot coordinates {\P};}
\colorlet{c}{kugray};
\draw [c,line width=0.9] (8.172,0.685026) -- (8.172,0.688771);
\draw [c,line width=0.9] (8.172,0.688771) -- (8.172,0.692516);
\draw [c,line width=0.9] (8.128,0.688771) -- (8.172,0.688771);
\draw [c,line width=0.9] (8.172,0.688771) -- (8.216,0.688771);
\definecolor{c}{rgb}{0,0,0};
\foreach \P in {(8.172,0.688771)}{\draw[mark options={color=c,fill=c},mark size=2.402402pt,mark=*,mark size=1pt] plot coordinates {\P};}
\colorlet{c}{kugray};
\draw [c,line width=0.9] (8.26,0.680493) -- (8.26,0.682656);
\draw [c,line width=0.9] (8.26,0.682656) -- (8.26,0.684818);
\draw [c,line width=0.9] (8.216,0.682656) -- (8.26,0.682656);
\draw [c,line width=0.9] (8.26,0.682656) -- (8.304,0.682656);
\definecolor{c}{rgb}{0,0,0};
\foreach \P in {(8.26,0.682656)}{\draw[mark options={color=c,fill=c},mark size=2.402402pt,mark=*,mark size=1pt] plot coordinates {\P};}
\colorlet{c}{kugray};
\draw [c,line width=0.9] (8.612,0.679598) -- (8.612,0.681127);
\draw [c,line width=0.9] (8.612,0.681127) -- (8.612,0.682656);
\draw [c,line width=0.9] (8.568,0.681127) -- (8.612,0.681127);
\draw [c,line width=0.9] (8.612,0.681127) -- (8.656,0.681127);
\definecolor{c}{rgb}{0,0,0};
\foreach \P in {(8.612,0.681127)}{\draw[mark options={color=c,fill=c},mark size=2.402402pt,mark=*,mark size=1pt] plot coordinates {\P};}
\colorlet{c}{kugray};
\draw [c,line width=0.9] (8.7,0.68408) -- (8.7,0.717578);
\draw [c,line width=0.9] (8.7,0.717578) -- (8.7,0.751076);
\draw [c,line width=0.9] (8.656,0.717578) -- (8.7,0.717578);
\draw [c,line width=0.9] (8.7,0.717578) -- (8.744,0.717578);
\definecolor{c}{rgb}{0,0,0};
\foreach \P in {(8.7,0.717578)}{\draw[mark options={color=c,fill=c},mark size=2.402402pt,mark=*,mark size=1pt] plot coordinates {\P};}
\colorlet{c}{kugray};
\draw [c,line width=0.9] (8.788,0.679598) -- (8.788,0.681127);
\draw [c,line width=0.9] (8.788,0.681127) -- (8.788,0.682656);
\draw [c,line width=0.9] (8.744,0.681127) -- (8.788,0.681127);
\draw [c,line width=0.9] (8.788,0.681127) -- (8.832,0.681127);
\definecolor{c}{rgb}{0,0,0};
\foreach \P in {(8.788,0.681127)}{\draw[mark options={color=c,fill=c},mark size=2.402402pt,mark=*,mark size=1pt] plot coordinates {\P};}
\colorlet{c}{kugray};
\draw [c,line width=0.9] (8.876,0.679598) -- (8.876,0.681127);
\draw [c,line width=0.9] (8.876,0.681127) -- (8.876,0.682656);
\draw [c,line width=0.9] (8.832,0.681127) -- (8.876,0.681127);
\draw [c,line width=0.9] (8.876,0.681127) -- (8.92,0.681127);
\definecolor{c}{rgb}{0,0,0};
\foreach \P in {(8.876,0.681127)}{\draw[mark options={color=c,fill=c},mark size=2.402402pt,mark=*,mark size=1pt] plot coordinates {\P};}
\colorlet{c}{kugray};
\draw [c,line width=0.9] (9.14,0.679598) -- (9.14,0.681127);
\draw [c,line width=0.9] (9.14,0.681127) -- (9.14,0.682656);
\draw [c,line width=0.9] (9.096,0.681127) -- (9.14,0.681127);
\draw [c,line width=0.9] (9.14,0.681127) -- (9.184,0.681127);
\definecolor{c}{rgb}{0,0,0};
\foreach \P in {(9.14,0.681127)}{\draw[mark options={color=c,fill=c},mark size=2.402402pt,mark=*,mark size=1pt] plot coordinates {\P};}
\colorlet{c}{kugray};
\draw [c,line width=0.9] (9.228,0.682498) -- (9.228,0.697385);
\draw [c,line width=0.9] (9.228,0.697385) -- (9.228,0.712273);
\draw [c,line width=0.9] (9.184,0.697385) -- (9.228,0.697385);
\draw [c,line width=0.9] (9.228,0.697385) -- (9.272,0.697385);
\definecolor{c}{rgb}{0,0,0};
\foreach \P in {(9.228,0.697385)}{\draw[mark options={color=c,fill=c},mark size=2.402402pt,mark=*,mark size=1pt] plot coordinates {\P};}
\colorlet{c}{kugray};
\draw [c,line width=0.9] (9.316,0.682656) -- (9.316,0.685713);
\draw [c,line width=0.9] (9.316,0.685713) -- (9.316,0.688771);
\draw [c,line width=0.9] (9.272,0.685713) -- (9.316,0.685713);
\draw [c,line width=0.9] (9.316,0.685713) -- (9.36,0.685713);
\definecolor{c}{rgb}{0,0,0};
\foreach \P in {(9.316,0.685713)}{\draw[mark options={color=c,fill=c},mark size=2.402402pt,mark=*,mark size=1pt] plot coordinates {\P};}
\colorlet{c}{kugray};
\draw [c,line width=0.9] (9.58,0.679598) -- (9.58,0.681127);
\draw [c,line width=0.9] (9.58,0.681127) -- (9.58,0.682656);
\draw [c,line width=0.9] (9.536,0.681127) -- (9.58,0.681127);
\draw [c,line width=0.9] (9.58,0.681127) -- (9.624,0.681127);
\definecolor{c}{rgb}{0,0,0};
\foreach \P in {(9.58,0.681127)}{\draw[mark options={color=c,fill=c},mark size=2.402402pt,mark=*,mark size=1pt] plot coordinates {\P};}
\colorlet{c}{kugray};
\draw [c,line width=0.9] (9.756,0.679598) -- (9.756,0.681127);
\draw [c,line width=0.9] (9.756,0.681127) -- (9.756,0.682656);
\draw [c,line width=0.9] (9.712,0.681127) -- (9.756,0.681127);
\draw [c,line width=0.9] (9.756,0.681127) -- (9.8,0.681127);
\definecolor{c}{rgb}{0,0,0};
\foreach \P in {(9.756,0.681127)}{\draw[mark options={color=c,fill=c},mark size=2.402402pt,mark=*,mark size=1pt] plot coordinates {\P};}
\draw [c,line width=0.9] (1,0.679598) -- (9.8,0.679598);
\draw [anchor= east] (9.8,0.299023) node[color=c, rotate=0]{$M_{\gamma\gamma} [GeV]$};
\draw [c,line width=0.9] (1,0.859012) -- (1,0.679598);
\draw [c,line width=0.9] (1.176,0.769305) -- (1.176,0.679598);
\draw [c,line width=0.9] (1.352,0.769305) -- (1.352,0.679598);
\draw [c,line width=0.9] (1.528,0.769305) -- (1.528,0.679598);
\draw [c,line width=0.9] (1.704,0.769305) -- (1.704,0.679598);
\draw [c,line width=0.9] (1.88,0.859012) -- (1.88,0.679598);
\draw [c,line width=0.9] (2.056,0.769305) -- (2.056,0.679598);
\draw [c,line width=0.9] (2.232,0.769305) -- (2.232,0.679598);
\draw [c,line width=0.9] (2.408,0.769305) -- (2.408,0.679598);
\draw [c,line width=0.9] (2.584,0.769305) -- (2.584,0.679598);
\draw [c,line width=0.9] (2.76,0.859012) -- (2.76,0.679598);
\draw [c,line width=0.9] (2.936,0.769305) -- (2.936,0.679598);
\draw [c,line width=0.9] (3.112,0.769305) -- (3.112,0.679598);
\draw [c,line width=0.9] (3.288,0.769305) -- (3.288,0.679598);
\draw [c,line width=0.9] (3.464,0.769305) -- (3.464,0.679598);
\draw [c,line width=0.9] (3.64,0.859012) -- (3.64,0.679598);
\draw [c,line width=0.9] (3.816,0.769305) -- (3.816,0.679598);
\draw [c,line width=0.9] (3.992,0.769305) -- (3.992,0.679598);
\draw [c,line width=0.9] (4.168,0.769305) -- (4.168,0.679598);
\draw [c,line width=0.9] (4.344,0.769305) -- (4.344,0.679598);
\draw [c,line width=0.9] (4.52,0.859012) -- (4.52,0.679598);
\draw [c,line width=0.9] (4.696,0.769305) -- (4.696,0.679598);
\draw [c,line width=0.9] (4.872,0.769305) -- (4.872,0.679598);
\draw [c,line width=0.9] (5.048,0.769305) -- (5.048,0.679598);
\draw [c,line width=0.9] (5.224,0.769305) -- (5.224,0.679598);
\draw [c,line width=0.9] (5.4,0.859012) -- (5.4,0.679598);
\draw [c,line width=0.9] (5.576,0.769305) -- (5.576,0.679598);
\draw [c,line width=0.9] (5.752,0.769305) -- (5.752,0.679598);
\draw [c,line width=0.9] (5.928,0.769305) -- (5.928,0.679598);
\draw [c,line width=0.9] (6.104,0.769305) -- (6.104,0.679598);
\draw [c,line width=0.9] (6.28,0.859012) -- (6.28,0.679598);
\draw [c,line width=0.9] (6.456,0.769305) -- (6.456,0.679598);
\draw [c,line width=0.9] (6.632,0.769305) -- (6.632,0.679598);
\draw [c,line width=0.9] (6.808,0.769305) -- (6.808,0.679598);
\draw [c,line width=0.9] (6.984,0.769305) -- (6.984,0.679598);
\draw [c,line width=0.9] (7.16,0.859012) -- (7.16,0.679598);
\draw [c,line width=0.9] (7.336,0.769305) -- (7.336,0.679598);
\draw [c,line width=0.9] (7.512,0.769305) -- (7.512,0.679598);
\draw [c,line width=0.9] (7.688,0.769305) -- (7.688,0.679598);
\draw [c,line width=0.9] (7.864,0.769305) -- (7.864,0.679598);
\draw [c,line width=0.9] (8.04,0.859012) -- (8.04,0.679598);
\draw [c,line width=0.9] (8.216,0.769305) -- (8.216,0.679598);
\draw [c,line width=0.9] (8.392,0.769305) -- (8.392,0.679598);
\draw [c,line width=0.9] (8.568,0.769305) -- (8.568,0.679598);
\draw [c,line width=0.9] (8.744,0.769305) -- (8.744,0.679598);
\draw [c,line width=0.9] (8.92,0.859012) -- (8.92,0.679598);
\draw [c,line width=0.9] (9.096,0.769305) -- (9.096,0.679598);
\draw [c,line width=0.9] (9.272,0.769305) -- (9.272,0.679598);
\draw [c,line width=0.9] (9.448,0.769305) -- (9.448,0.679598);
\draw [c,line width=0.9] (9.624,0.769305) -- (9.624,0.679598);
\draw [c,line width=0.9] (9.8,0.859012) -- (9.8,0.679598);
\draw [anchor=base] (1,0.45533) node[color=c, rotate=0]{0};
\draw [anchor=base] (1.88,0.45533) node[color=c, rotate=0]{100};
\draw [anchor=base] (2.76,0.45533) node[color=c, rotate=0]{200};
\draw [anchor=base] (3.64,0.45533) node[color=c, rotate=0]{300};
\draw [anchor=base] (4.52,0.45533) node[color=c, rotate=0]{400};
\draw [anchor=base] (5.4,0.45533) node[color=c, rotate=0]{500};
\draw [anchor=base] (6.28,0.45533) node[color=c, rotate=0]{600};
\draw [anchor=base] (7.16,0.45533) node[color=c, rotate=0]{700};
\draw [anchor=base] (8.04,0.45533) node[color=c, rotate=0]{800};
\draw [anchor=base] (8.92,0.45533) node[color=c, rotate=0]{900};
\draw [anchor=base] (9.8,0.45533) node[color=c, rotate=0]{1000};
\draw [c,line width=0.9] (1,0.679598) -- (1,6.66006);
\draw [anchor= east] (0.272,6.66006) node[color=c, rotate=90]{$\ events (normalised)$};
\draw [c,line width=0.9] (1.264,0.679598) -- (1,0.679598);
\draw [c,line width=0.9] (1.132,0.866269) -- (1,0.866269);
\draw [c,line width=0.9] (1.132,1.05294) -- (1,1.05294);
\draw [c,line width=0.9] (1.132,1.23961) -- (1,1.23961);
\draw [c,line width=0.9] (1.132,1.42628) -- (1,1.42628);
\draw [c,line width=0.9] (1.264,1.61295) -- (1,1.61295);
\draw [c,line width=0.9] (1.132,1.79963) -- (1,1.79963);
\draw [c,line width=0.9] (1.132,1.9863) -- (1,1.9863);
\draw [c,line width=0.9] (1.132,2.17297) -- (1,2.17297);
\draw [c,line width=0.9] (1.132,2.35964) -- (1,2.35964);
\draw [c,line width=0.9] (1.264,2.54631) -- (1,2.54631);
\draw [c,line width=0.9] (1.132,2.73298) -- (1,2.73298);
\draw [c,line width=0.9] (1.132,2.91965) -- (1,2.91965);
\draw [c,line width=0.9] (1.132,3.10632) -- (1,3.10632);
\draw [c,line width=0.9] (1.132,3.293) -- (1,3.293);
\draw [c,line width=0.9] (1.264,3.47967) -- (1,3.47967);
\draw [c,line width=0.9] (1.132,3.66634) -- (1,3.66634);
\draw [c,line width=0.9] (1.132,3.85301) -- (1,3.85301);
\draw [c,line width=0.9] (1.132,4.03968) -- (1,4.03968);
\draw [c,line width=0.9] (1.132,4.22635) -- (1,4.22635);
\draw [c,line width=0.9] (1.264,4.41302) -- (1,4.41302);
\draw [c,line width=0.9] (1.132,4.5997) -- (1,4.5997);
\draw [c,line width=0.9] (1.132,4.78637) -- (1,4.78637);
\draw [c,line width=0.9] (1.132,4.97304) -- (1,4.97304);
\draw [c,line width=0.9] (1.132,5.15971) -- (1,5.15971);
\draw [c,line width=0.9] (1.264,5.34638) -- (1,5.34638);
\draw [c,line width=0.9] (1.132,5.53305) -- (1,5.53305);
\draw [c,line width=0.9] (1.132,5.71972) -- (1,5.71972);
\draw [c,line width=0.9] (1.132,5.90639) -- (1,5.90639);
\draw [c,line width=0.9] (1.132,6.09307) -- (1,6.09307);
\draw [c,line width=0.9] (1.264,6.27974) -- (1,6.27974);
\draw [c,line width=0.9] (1.264,6.27974) -- (1,6.27974);
\draw [c,line width=0.9] (1.132,6.46641) -- (1,6.46641);
\draw [c,line width=0.9] (1.132,6.65308) -- (1,6.65308);
\draw [anchor= east] (0.95,0.679598) node[color=c, rotate=0]{0};
\draw [anchor= east] (0.95,1.61295) node[color=c, rotate=0]{1000};
\draw [anchor= east] (0.95,2.54631) node[color=c, rotate=0]{2000};
\draw [anchor= east] (0.95,3.47967) node[color=c, rotate=0]{3000};
\draw [anchor= east] (0.95,4.41302) node[color=c, rotate=0]{4000};
\draw [anchor= east] (0.95,5.34638) node[color=c, rotate=0]{5000};
\draw [anchor= east] (0.95,6.27974) node[color=c, rotate=0]{6000};
\colorlet{c}{natgreen};
\draw [c,line width=0.9] (1.22,0.687505) -- (1.22,0.691829);
\draw [c,line width=0.9] (1.22,0.691829) -- (1.22,0.696154);
\draw [c,line width=0.9] (1.176,0.691829) -- (1.22,0.691829);
\draw [c,line width=0.9] (1.22,0.691829) -- (1.264,0.691829);
\definecolor{c}{rgb}{0,0,0};
\foreach \P in {(1.22,0.691829)}{\draw[mark options={color=c,fill=c},mark size=2.402402pt,mark=*,mark size=1pt] plot coordinates {\P};}
\colorlet{c}{natgreen};
\draw [c,line width=0.9] (1.308,0.692648) -- (1.308,0.697945);
\draw [c,line width=0.9] (1.308,0.697945) -- (1.308,0.703241);
\draw [c,line width=0.9] (1.264,0.697945) -- (1.308,0.697945);
\draw [c,line width=0.9] (1.308,0.697945) -- (1.352,0.697945);
\definecolor{c}{rgb}{0,0,0};
\foreach \P in {(1.308,0.697945)}{\draw[mark options={color=c,fill=c},mark size=2.402402pt,mark=*,mark size=1pt] plot coordinates {\P};}
\colorlet{c}{natgreen};
\draw [c,line width=0.9] (1.396,0.711554) -- (1.396,0.71935);
\draw [c,line width=0.9] (1.396,0.71935) -- (1.396,0.727146);
\draw [c,line width=0.9] (1.352,0.71935) -- (1.396,0.71935);
\draw [c,line width=0.9] (1.396,0.71935) -- (1.44,0.71935);
\definecolor{c}{rgb}{0,0,0};
\foreach \P in {(1.396,0.71935)}{\draw[mark options={color=c,fill=c},mark size=2.402402pt,mark=*,mark size=1pt] plot coordinates {\P};}
\colorlet{c}{natgreen};
\draw [c,line width=0.9] (1.484,0.704699) -- (1.484,0.711705);
\draw [c,line width=0.9] (1.484,0.711705) -- (1.484,0.718711);
\draw [c,line width=0.9] (1.44,0.711705) -- (1.484,0.711705);
\draw [c,line width=0.9] (1.484,0.711705) -- (1.528,0.711705);
\definecolor{c}{rgb}{0,0,0};
\foreach \P in {(1.484,0.711705)}{\draw[mark options={color=c,fill=c},mark size=2.402402pt,mark=*,mark size=1pt] plot coordinates {\P};}
\colorlet{c}{natgreen};
\draw [c,line width=0.9] (1.572,0.728272) -- (1.572,0.737697);
\draw [c,line width=0.9] (1.572,0.737697) -- (1.572,0.747122);
\draw [c,line width=0.9] (1.528,0.737697) -- (1.572,0.737697);
\draw [c,line width=0.9] (1.572,0.737697) -- (1.616,0.737697);
\definecolor{c}{rgb}{0,0,0};
\foreach \P in {(1.572,0.737697)}{\draw[mark options={color=c,fill=c},mark size=2.402402pt,mark=*,mark size=1pt] plot coordinates {\P};}
\colorlet{c}{natgreen};
\draw [c,line width=0.9] (1.66,0.7495) -- (1.66,0.76063);
\draw [c,line width=0.9] (1.66,0.76063) -- (1.66,0.771761);
\draw [c,line width=0.9] (1.616,0.76063) -- (1.66,0.76063);
\draw [c,line width=0.9] (1.66,0.76063) -- (1.704,0.76063);
\definecolor{c}{rgb}{0,0,0};
\foreach \P in {(1.66,0.76063)}{\draw[mark options={color=c,fill=c},mark size=2.402402pt,mark=*,mark size=1pt] plot coordinates {\P};}
\colorlet{c}{natgreen};
\draw [c,line width=0.9] (1.748,0.768085) -- (1.748,0.780506);
\draw [c,line width=0.9] (1.748,0.780506) -- (1.748,0.792927);
\draw [c,line width=0.9] (1.704,0.780506) -- (1.748,0.780506);
\draw [c,line width=0.9] (1.748,0.780506) -- (1.792,0.780506);
\definecolor{c}{rgb}{0,0,0};
\foreach \P in {(1.748,0.780506)}{\draw[mark options={color=c,fill=c},mark size=2.402402pt,mark=*,mark size=1pt] plot coordinates {\P};}
\colorlet{c}{natgreen};
\draw [c,line width=0.9] (1.836,0.858002) -- (1.836,0.8753);
\draw [c,line width=0.9] (1.836,0.8753) -- (1.836,0.892597);
\draw [c,line width=0.9] (1.792,0.8753) -- (1.836,0.8753);
\draw [c,line width=0.9] (1.836,0.8753) -- (1.88,0.8753);
\definecolor{c}{rgb}{0,0,0};
\foreach \P in {(1.836,0.8753)}{\draw[mark options={color=c,fill=c},mark size=2.402402pt,mark=*,mark size=1pt] plot coordinates {\P};}
\colorlet{c}{natgreen};
\draw [c,line width=0.9] (1.924,2.12255) -- (1.924,2.17029);
\draw [c,line width=0.9] (1.924,2.17029) -- (1.924,2.21804);
\draw [c,line width=0.9] (1.88,2.17029) -- (1.924,2.17029);
\draw [c,line width=0.9] (1.924,2.17029) -- (1.968,2.17029);
\definecolor{c}{rgb}{0,0,0};
\foreach \P in {(1.924,2.17029)}{\draw[mark options={color=c,fill=c},mark size=2.402402pt,mark=*,mark size=1pt] plot coordinates {\P};}
\colorlet{c}{natgreen};
\draw [c,line width=0.9] (2.012,3.72803) -- (2.012,3.79707);
\draw [c,line width=0.9] (2.012,3.79707) -- (2.012,3.8661);
\draw [c,line width=0.9] (1.968,3.79707) -- (2.012,3.79707);
\draw [c,line width=0.9] (2.012,3.79707) -- (2.056,3.79707);
\definecolor{c}{rgb}{0,0,0};
\foreach \P in {(2.012,3.79707)}{\draw[mark options={color=c,fill=c},mark size=2.402402pt,mark=*,mark size=1pt] plot coordinates {\P};}
\colorlet{c}{natgreen};
\draw [c,line width=0.9] (2.1,4.27562) -- (2.1,4.35053);
\draw [c,line width=0.9] (2.1,4.35053) -- (2.1,4.42545);
\draw [c,line width=0.9] (2.056,4.35053) -- (2.1,4.35053);
\draw [c,line width=0.9] (2.1,4.35053) -- (2.144,4.35053);
\definecolor{c}{rgb}{0,0,0};
\foreach \P in {(2.1,4.35053)}{\draw[mark options={color=c,fill=c},mark size=2.402402pt,mark=*,mark size=1pt] plot coordinates {\P};}
\colorlet{c}{natgreen};
\draw [c,line width=0.9] (2.188,4.08043) -- (2.188,4.1533);
\draw [c,line width=0.9] (2.188,4.1533) -- (2.188,4.22618);
\draw [c,line width=0.9] (2.144,4.1533) -- (2.188,4.1533);
\draw [c,line width=0.9] (2.188,4.1533) -- (2.232,4.1533);
\definecolor{c}{rgb}{0,0,0};
\foreach \P in {(2.188,4.1533)}{\draw[mark options={color=c,fill=c},mark size=2.402402pt,mark=*,mark size=1pt] plot coordinates {\P};}
\colorlet{c}{natgreen};
\draw [c,line width=0.9] (2.276,3.68872) -- (2.276,3.75731);
\draw [c,line width=0.9] (2.276,3.75731) -- (2.276,3.82591);
\draw [c,line width=0.9] (2.232,3.75731) -- (2.276,3.75731);
\draw [c,line width=0.9] (2.276,3.75731) -- (2.32,3.75731);
\definecolor{c}{rgb}{0,0,0};
\foreach \P in {(2.276,3.75731)}{\draw[mark options={color=c,fill=c},mark size=2.402402pt,mark=*,mark size=1pt] plot coordinates {\P};}
\colorlet{c}{natgreen};
\draw [c,line width=0.9] (2.364,3.3547) -- (2.364,3.41942);
\draw [c,line width=0.9] (2.364,3.41942) -- (2.364,3.48414);
\draw [c,line width=0.9] (2.32,3.41942) -- (2.364,3.41942);
\draw [c,line width=0.9] (2.364,3.41942) -- (2.408,3.41942);
\definecolor{c}{rgb}{0,0,0};
\foreach \P in {(2.364,3.41942)}{\draw[mark options={color=c,fill=c},mark size=2.402402pt,mark=*,mark size=1pt] plot coordinates {\P};}
\colorlet{c}{natgreen};
\draw [c,line width=0.9] (2.452,2.87151) -- (2.452,2.93017);
\draw [c,line width=0.9] (2.452,2.93017) -- (2.452,2.98883);
\draw [c,line width=0.9] (2.408,2.93017) -- (2.452,2.93017);
\draw [c,line width=0.9] (2.452,2.93017) -- (2.496,2.93017);
\definecolor{c}{rgb}{0,0,0};
\foreach \P in {(2.452,2.93017)}{\draw[mark options={color=c,fill=c},mark size=2.402402pt,mark=*,mark size=1pt] plot coordinates {\P};}
\colorlet{c}{natgreen};
\draw [c,line width=0.9] (2.54,2.62564) -- (2.54,2.68095);
\draw [c,line width=0.9] (2.54,2.68095) -- (2.54,2.73627);
\draw [c,line width=0.9] (2.496,2.68095) -- (2.54,2.68095);
\draw [c,line width=0.9] (2.54,2.68095) -- (2.584,2.68095);
\definecolor{c}{rgb}{0,0,0};
\foreach \P in {(2.54,2.68095)}{\draw[mark options={color=c,fill=c},mark size=2.402402pt,mark=*,mark size=1pt] plot coordinates {\P};}
\colorlet{c}{natgreen};
\draw [c,line width=0.9] (2.628,2.32727) -- (2.628,2.37823);
\draw [c,line width=0.9] (2.628,2.37823) -- (2.628,2.42919);
\draw [c,line width=0.9] (2.584,2.37823) -- (2.628,2.37823);
\draw [c,line width=0.9] (2.628,2.37823) -- (2.672,2.37823);
\definecolor{c}{rgb}{0,0,0};
\foreach \P in {(2.628,2.37823)}{\draw[mark options={color=c,fill=c},mark size=2.402402pt,mark=*,mark size=1pt] plot coordinates {\P};}
\colorlet{c}{natgreen};
\draw [c,line width=0.9] (2.716,2.11202) -- (2.716,2.15959);
\draw [c,line width=0.9] (2.716,2.15959) -- (2.716,2.20716);
\draw [c,line width=0.9] (2.672,2.15959) -- (2.716,2.15959);
\draw [c,line width=0.9] (2.716,2.15959) -- (2.76,2.15959);
\definecolor{c}{rgb}{0,0,0};
\foreach \P in {(2.716,2.15959)}{\draw[mark options={color=c,fill=c},mark size=2.402402pt,mark=*,mark size=1pt] plot coordinates {\P};}
\colorlet{c}{natgreen};
\draw [c,line width=0.9] (2.804,1.84597) -- (2.804,1.88897);
\draw [c,line width=0.9] (2.804,1.88897) -- (2.804,1.93197);
\draw [c,line width=0.9] (2.76,1.88897) -- (2.804,1.88897);
\draw [c,line width=0.9] (2.804,1.88897) -- (2.848,1.88897);
\definecolor{c}{rgb}{0,0,0};
\foreach \P in {(2.804,1.88897)}{\draw[mark options={color=c,fill=c},mark size=2.402402pt,mark=*,mark size=1pt] plot coordinates {\P};}
\colorlet{c}{natgreen};
\draw [c,line width=0.9] (2.892,1.68839) -- (2.892,1.72844);
\draw [c,line width=0.9] (2.892,1.72844) -- (2.892,1.76848);
\draw [c,line width=0.9] (2.848,1.72844) -- (2.892,1.72844);
\draw [c,line width=0.9] (2.892,1.72844) -- (2.936,1.72844);
\definecolor{c}{rgb}{0,0,0};
\foreach \P in {(2.892,1.72844)}{\draw[mark options={color=c,fill=c},mark size=2.402402pt,mark=*,mark size=1pt] plot coordinates {\P};}
\colorlet{c}{natgreen};
\draw [c,line width=0.9] (2.98,1.48016) -- (2.98,1.51592);
\draw [c,line width=0.9] (2.98,1.51592) -- (2.98,1.55168);
\draw [c,line width=0.9] (2.936,1.51592) -- (2.98,1.51592);
\draw [c,line width=0.9] (2.98,1.51592) -- (3.024,1.51592);
\definecolor{c}{rgb}{0,0,0};
\foreach \P in {(2.98,1.51592)}{\draw[mark options={color=c,fill=c},mark size=2.402402pt,mark=*,mark size=1pt] plot coordinates {\P};}
\colorlet{c}{natgreen};
\draw [c,line width=0.9] (3.068,1.43229) -- (3.068,1.46699);
\draw [c,line width=0.9] (3.068,1.46699) -- (3.068,1.50169);
\draw [c,line width=0.9] (3.024,1.46699) -- (3.068,1.46699);
\draw [c,line width=0.9] (3.068,1.46699) -- (3.112,1.46699);
\definecolor{c}{rgb}{0,0,0};
\foreach \P in {(3.068,1.46699)}{\draw[mark options={color=c,fill=c},mark size=2.402402pt,mark=*,mark size=1pt] plot coordinates {\P};}
\colorlet{c}{natgreen};
\draw [c,line width=0.9] (3.156,1.23376) -- (3.156,1.26365);
\draw [c,line width=0.9] (3.156,1.26365) -- (3.156,1.29353);
\draw [c,line width=0.9] (3.112,1.26365) -- (3.156,1.26365);
\draw [c,line width=0.9] (3.156,1.26365) -- (3.2,1.26365);
\definecolor{c}{rgb}{0,0,0};
\foreach \P in {(3.156,1.26365)}{\draw[mark options={color=c,fill=c},mark size=2.402402pt,mark=*,mark size=1pt] plot coordinates {\P};}
\colorlet{c}{natgreen};
\draw [c,line width=0.9] (3.244,1.1876) -- (3.244,1.21625);
\draw [c,line width=0.9] (3.244,1.21625) -- (3.244,1.24489);
\draw [c,line width=0.9] (3.2,1.21625) -- (3.244,1.21625);
\draw [c,line width=0.9] (3.244,1.21625) -- (3.288,1.21625);
\definecolor{c}{rgb}{0,0,0};
\foreach \P in {(3.244,1.21625)}{\draw[mark options={color=c,fill=c},mark size=2.402402pt,mark=*,mark size=1pt] plot coordinates {\P};}
\colorlet{c}{natgreen};
\draw [c,line width=0.9] (3.332,1.14002) -- (3.332,1.16732);
\draw [c,line width=0.9] (3.332,1.16732) -- (3.332,1.19463);
\draw [c,line width=0.9] (3.288,1.16732) -- (3.332,1.16732);
\draw [c,line width=0.9] (3.332,1.16732) -- (3.376,1.16732);
\definecolor{c}{rgb}{0,0,0};
\foreach \P in {(3.332,1.16732)}{\draw[mark options={color=c,fill=c},mark size=2.402402pt,mark=*,mark size=1pt] plot coordinates {\P};}
\colorlet{c}{natgreen};
\draw [c,line width=0.9] (3.42,1.09695) -- (3.42,1.12298);
\draw [c,line width=0.9] (3.42,1.12298) -- (3.42,1.14902);
\draw [c,line width=0.9] (3.376,1.12298) -- (3.42,1.12298);
\draw [c,line width=0.9] (3.42,1.12298) -- (3.464,1.12298);
\definecolor{c}{rgb}{0,0,0};
\foreach \P in {(3.42,1.12298)}{\draw[mark options={color=c,fill=c},mark size=2.402402pt,mark=*,mark size=1pt] plot coordinates {\P};}
\colorlet{c}{natgreen};
\draw [c,line width=0.9] (3.508,1.0495) -- (3.508,1.07406);
\draw [c,line width=0.9] (3.508,1.07406) -- (3.508,1.09862);
\draw [c,line width=0.9] (3.464,1.07406) -- (3.508,1.07406);
\draw [c,line width=0.9] (3.508,1.07406) -- (3.552,1.07406);
\definecolor{c}{rgb}{0,0,0};
\foreach \P in {(3.508,1.07406)}{\draw[mark options={color=c,fill=c},mark size=2.402402pt,mark=*,mark size=1pt] plot coordinates {\P};}
\colorlet{c}{natgreen};
\draw [c,line width=0.9] (3.596,0.926933) -- (3.596,0.947159);
\draw [c,line width=0.9] (3.596,0.947159) -- (3.596,0.967384);
\draw [c,line width=0.9] (3.552,0.947159) -- (3.596,0.947159);
\draw [c,line width=0.9] (3.596,0.947159) -- (3.64,0.947159);
\definecolor{c}{rgb}{0,0,0};
\foreach \P in {(3.596,0.947159)}{\draw[mark options={color=c,fill=c},mark size=2.402402pt,mark=*,mark size=1pt] plot coordinates {\P};}
\colorlet{c}{natgreen};
\draw [c,line width=0.9] (3.684,0.935763) -- (3.684,0.956332);
\draw [c,line width=0.9] (3.684,0.956332) -- (3.684,0.976902);
\draw [c,line width=0.9] (3.64,0.956332) -- (3.684,0.956332);
\draw [c,line width=0.9] (3.684,0.956332) -- (3.728,0.956332);
\definecolor{c}{rgb}{0,0,0};
\foreach \P in {(3.684,0.956332)}{\draw[mark options={color=c,fill=c},mark size=2.402402pt,mark=*,mark size=1pt] plot coordinates {\P};}
\colorlet{c}{natgreen};
\draw [c,line width=0.9] (3.772,0.91223) -- (3.772,0.93187);
\draw [c,line width=0.9] (3.772,0.93187) -- (3.772,0.951509);
\draw [c,line width=0.9] (3.728,0.93187) -- (3.772,0.93187);
\draw [c,line width=0.9] (3.772,0.93187) -- (3.816,0.93187);
\definecolor{c}{rgb}{0,0,0};
\foreach \P in {(3.772,0.93187)}{\draw[mark options={color=c,fill=c},mark size=2.402402pt,mark=*,mark size=1pt] plot coordinates {\P};}
\colorlet{c}{natgreen};
\draw [c,line width=0.9] (3.86,0.885812) -- (3.86,0.904349);
\draw [c,line width=0.9] (3.86,0.904349) -- (3.86,0.922886);
\draw [c,line width=0.9] (3.816,0.904349) -- (3.86,0.904349);
\draw [c,line width=0.9] (3.86,0.904349) -- (3.904,0.904349);
\definecolor{c}{rgb}{0,0,0};
\foreach \P in {(3.86,0.904349)}{\draw[mark options={color=c,fill=c},mark size=2.402402pt,mark=*,mark size=1pt] plot coordinates {\P};}
\colorlet{c}{natgreen};
\draw [c,line width=0.9] (3.948,0.814296) -- (3.948,0.829432);
\draw [c,line width=0.9] (3.948,0.829432) -- (3.948,0.844567);
\draw [c,line width=0.9] (3.904,0.829432) -- (3.948,0.829432);
\draw [c,line width=0.9] (3.948,0.829432) -- (3.992,0.829432);
\definecolor{c}{rgb}{0,0,0};
\foreach \P in {(3.948,0.829432)}{\draw[mark options={color=c,fill=c},mark size=2.402402pt,mark=*,mark size=1pt] plot coordinates {\P};}
\colorlet{c}{natgreen};
\draw [c,line width=0.9] (4.036,0.830288) -- (4.036,0.84625);
\draw [c,line width=0.9] (4.036,0.84625) -- (4.036,0.862212);
\draw [c,line width=0.9] (3.992,0.84625) -- (4.036,0.84625);
\draw [c,line width=0.9] (4.036,0.84625) -- (4.08,0.84625);
\definecolor{c}{rgb}{0,0,0};
\foreach \P in {(4.036,0.84625)}{\draw[mark options={color=c,fill=c},mark size=2.402402pt,mark=*,mark size=1pt] plot coordinates {\P};}
\colorlet{c}{natgreen};
\draw [c,line width=0.9] (4.124,0.808493) -- (4.124,0.823316);
\draw [c,line width=0.9] (4.124,0.823316) -- (4.124,0.83814);
\draw [c,line width=0.9] (4.08,0.823316) -- (4.124,0.823316);
\draw [c,line width=0.9] (4.124,0.823316) -- (4.168,0.823316);
\definecolor{c}{rgb}{0,0,0};
\foreach \P in {(4.124,0.823316)}{\draw[mark options={color=c,fill=c},mark size=2.402402pt,mark=*,mark size=1pt] plot coordinates {\P};}
\colorlet{c}{natgreen};
\draw [c,line width=0.9] (4.212,0.809943) -- (4.212,0.824845);
\draw [c,line width=0.9] (4.212,0.824845) -- (4.212,0.839747);
\draw [c,line width=0.9] (4.168,0.824845) -- (4.212,0.824845);
\draw [c,line width=0.9] (4.212,0.824845) -- (4.256,0.824845);
\definecolor{c}{rgb}{0,0,0};
\foreach \P in {(4.212,0.824845)}{\draw[mark options={color=c,fill=c},mark size=2.402402pt,mark=*,mark size=1pt] plot coordinates {\P};}
\colorlet{c}{natgreen};
\draw [c,line width=0.9] (4.3,0.77383) -- (4.3,0.786622);
\draw [c,line width=0.9] (4.3,0.786622) -- (4.3,0.799414);
\draw [c,line width=0.9] (4.256,0.786622) -- (4.3,0.786622);
\draw [c,line width=0.9] (4.3,0.786622) -- (4.344,0.786622);
\definecolor{c}{rgb}{0,0,0};
\foreach \P in {(4.3,0.786622)}{\draw[mark options={color=c,fill=c},mark size=2.402402pt,mark=*,mark size=1pt] plot coordinates {\P};}
\colorlet{c}{natgreen};
\draw [c,line width=0.9] (4.388,0.766651) -- (4.388,0.778978);
\draw [c,line width=0.9] (4.388,0.778978) -- (4.388,0.791304);
\draw [c,line width=0.9] (4.344,0.778978) -- (4.388,0.778978);
\draw [c,line width=0.9] (4.388,0.778978) -- (4.432,0.778978);
\definecolor{c}{rgb}{0,0,0};
\foreach \P in {(4.388,0.778978)}{\draw[mark options={color=c,fill=c},mark size=2.402402pt,mark=*,mark size=1pt] plot coordinates {\P};}
\colorlet{c}{natgreen};
\draw [c,line width=0.9] (4.476,0.763784) -- (4.476,0.77592);
\draw [c,line width=0.9] (4.476,0.77592) -- (4.476,0.788055);
\draw [c,line width=0.9] (4.432,0.77592) -- (4.476,0.77592);
\draw [c,line width=0.9] (4.476,0.77592) -- (4.52,0.77592);
\definecolor{c}{rgb}{0,0,0};
\foreach \P in {(4.476,0.77592)}{\draw[mark options={color=c,fill=c},mark size=2.402402pt,mark=*,mark size=1pt] plot coordinates {\P};}
\colorlet{c}{natgreen};
\draw [c,line width=0.9] (4.564,0.776707) -- (4.564,0.78968);
\draw [c,line width=0.9] (4.564,0.78968) -- (4.564,0.802653);
\draw [c,line width=0.9] (4.52,0.78968) -- (4.564,0.78968);
\draw [c,line width=0.9] (4.564,0.78968) -- (4.608,0.78968);
\definecolor{c}{rgb}{0,0,0};
\foreach \P in {(4.564,0.78968)}{\draw[mark options={color=c,fill=c},mark size=2.402402pt,mark=*,mark size=1pt] plot coordinates {\P};}
\colorlet{c}{natgreen};
\draw [c,line width=0.9] (4.652,0.740975) -- (4.652,0.751457);
\draw [c,line width=0.9] (4.652,0.751457) -- (4.652,0.761939);
\draw [c,line width=0.9] (4.608,0.751457) -- (4.652,0.751457);
\draw [c,line width=0.9] (4.652,0.751457) -- (4.696,0.751457);
\definecolor{c}{rgb}{0,0,0};
\foreach \P in {(4.652,0.751457)}{\draw[mark options={color=c,fill=c},mark size=2.402402pt,mark=*,mark size=1pt] plot coordinates {\P};}
\colorlet{c}{natgreen};
\draw [c,line width=0.9] (4.74,0.743812) -- (4.74,0.754515);
\draw [c,line width=0.9] (4.74,0.754515) -- (4.74,0.765217);
\draw [c,line width=0.9] (4.696,0.754515) -- (4.74,0.754515);
\draw [c,line width=0.9] (4.74,0.754515) -- (4.784,0.754515);
\definecolor{c}{rgb}{0,0,0};
\foreach \P in {(4.74,0.754515)}{\draw[mark options={color=c,fill=c},mark size=2.402402pt,mark=*,mark size=1pt] plot coordinates {\P};}
\colorlet{c}{natgreen};
\draw [c,line width=0.9] (4.828,0.740975) -- (4.828,0.751457);
\draw [c,line width=0.9] (4.828,0.751457) -- (4.828,0.761939);
\draw [c,line width=0.9] (4.784,0.751457) -- (4.828,0.751457);
\draw [c,line width=0.9] (4.828,0.751457) -- (4.872,0.751457);
\definecolor{c}{rgb}{0,0,0};
\foreach \P in {(4.828,0.751457)}{\draw[mark options={color=c,fill=c},mark size=2.402402pt,mark=*,mark size=1pt] plot coordinates {\P};}
\colorlet{c}{natgreen};
\draw [c,line width=0.9] (4.916,0.721269) -- (4.916,0.730052);
\draw [c,line width=0.9] (4.916,0.730052) -- (4.916,0.738835);
\draw [c,line width=0.9] (4.872,0.730052) -- (4.916,0.730052);
\draw [c,line width=0.9] (4.916,0.730052) -- (4.96,0.730052);
\definecolor{c}{rgb}{0,0,0};
\foreach \P in {(4.916,0.730052)}{\draw[mark options={color=c,fill=c},mark size=2.402402pt,mark=*,mark size=1pt] plot coordinates {\P};}
\colorlet{c}{natgreen};
\draw [c,line width=0.9] (5.004,0.724065) -- (5.004,0.73311);
\draw [c,line width=0.9] (5.004,0.73311) -- (5.004,0.742155);
\draw [c,line width=0.9] (4.96,0.73311) -- (5.004,0.73311);
\draw [c,line width=0.9] (5.004,0.73311) -- (5.048,0.73311);
\definecolor{c}{rgb}{0,0,0};
\foreach \P in {(5.004,0.73311)}{\draw[mark options={color=c,fill=c},mark size=2.402402pt,mark=*,mark size=1pt] plot coordinates {\P};}
\colorlet{c}{natgreen};
\draw [c,line width=0.9] (5.092,0.718482) -- (5.092,0.726994);
\draw [c,line width=0.9] (5.092,0.726994) -- (5.092,0.735507);
\draw [c,line width=0.9] (5.048,0.726994) -- (5.092,0.726994);
\draw [c,line width=0.9] (5.092,0.726994) -- (5.136,0.726994);
\definecolor{c}{rgb}{0,0,0};
\foreach \P in {(5.092,0.726994)}{\draw[mark options={color=c,fill=c},mark size=2.402402pt,mark=*,mark size=1pt] plot coordinates {\P};}
\colorlet{c}{natgreen};
\draw [c,line width=0.9] (5.18,0.721269) -- (5.18,0.730052);
\draw [c,line width=0.9] (5.18,0.730052) -- (5.18,0.738835);
\draw [c,line width=0.9] (5.136,0.730052) -- (5.18,0.730052);
\draw [c,line width=0.9] (5.18,0.730052) -- (5.224,0.730052);
\definecolor{c}{rgb}{0,0,0};
\foreach \P in {(5.18,0.730052)}{\draw[mark options={color=c,fill=c},mark size=2.402402pt,mark=*,mark size=1pt] plot coordinates {\P};}
\colorlet{c}{natgreen};
\draw [c,line width=0.9] (5.268,0.703339) -- (5.268,0.710176);
\draw [c,line width=0.9] (5.268,0.710176) -- (5.268,0.717014);
\draw [c,line width=0.9] (5.224,0.710176) -- (5.268,0.710176);
\draw [c,line width=0.9] (5.268,0.710176) -- (5.312,0.710176);
\definecolor{c}{rgb}{0,0,0};
\foreach \P in {(5.268,0.710176)}{\draw[mark options={color=c,fill=c},mark size=2.402402pt,mark=*,mark size=1pt] plot coordinates {\P};}
\colorlet{c}{natgreen};
\draw [c,line width=0.9] (5.356,0.703339) -- (5.356,0.710176);
\draw [c,line width=0.9] (5.356,0.710176) -- (5.356,0.717014);
\draw [c,line width=0.9] (5.312,0.710176) -- (5.356,0.710176);
\draw [c,line width=0.9] (5.356,0.710176) -- (5.4,0.710176);
\definecolor{c}{rgb}{0,0,0};
\foreach \P in {(5.356,0.710176)}{\draw[mark options={color=c,fill=c},mark size=2.402402pt,mark=*,mark size=1pt] plot coordinates {\P};}
\colorlet{c}{natgreen};
\draw [c,line width=0.9] (5.444,0.715703) -- (5.444,0.723936);
\draw [c,line width=0.9] (5.444,0.723936) -- (5.444,0.73217);
\draw [c,line width=0.9] (5.4,0.723936) -- (5.444,0.723936);
\draw [c,line width=0.9] (5.444,0.723936) -- (5.488,0.723936);
\definecolor{c}{rgb}{0,0,0};
\foreach \P in {(5.444,0.723936)}{\draw[mark options={color=c,fill=c},mark size=2.402402pt,mark=*,mark size=1pt] plot coordinates {\P};}
\colorlet{c}{natgreen};
\draw [c,line width=0.9] (5.532,0.701983) -- (5.532,0.708647);
\draw [c,line width=0.9] (5.532,0.708647) -- (5.532,0.715312);
\draw [c,line width=0.9] (5.488,0.708647) -- (5.532,0.708647);
\draw [c,line width=0.9] (5.532,0.708647) -- (5.576,0.708647);
\definecolor{c}{rgb}{0,0,0};
\foreach \P in {(5.532,0.708647)}{\draw[mark options={color=c,fill=c},mark size=2.402402pt,mark=*,mark size=1pt] plot coordinates {\P};}
\colorlet{c}{natgreen};
\draw [c,line width=0.9] (5.62,0.704699) -- (5.62,0.711705);
\draw [c,line width=0.9] (5.62,0.711705) -- (5.62,0.718711);
\draw [c,line width=0.9] (5.576,0.711705) -- (5.62,0.711705);
\draw [c,line width=0.9] (5.62,0.711705) -- (5.664,0.711705);
\definecolor{c}{rgb}{0,0,0};
\foreach \P in {(5.62,0.711705)}{\draw[mark options={color=c,fill=c},mark size=2.402402pt,mark=*,mark size=1pt] plot coordinates {\P};}
\colorlet{c}{natgreen};
\draw [c,line width=0.9] (5.708,0.708802) -- (5.708,0.716292);
\draw [c,line width=0.9] (5.708,0.716292) -- (5.708,0.723782);
\draw [c,line width=0.9] (5.664,0.716292) -- (5.708,0.716292);
\draw [c,line width=0.9] (5.708,0.716292) -- (5.752,0.716292);
\definecolor{c}{rgb}{0,0,0};
\foreach \P in {(5.708,0.716292)}{\draw[mark options={color=c,fill=c},mark size=2.402402pt,mark=*,mark size=1pt] plot coordinates {\P};}
\colorlet{c}{natgreen};
\draw [c,line width=0.9] (5.796,0.693961) -- (5.796,0.699474);
\draw [c,line width=0.9] (5.796,0.699474) -- (5.796,0.704986);
\draw [c,line width=0.9] (5.752,0.699474) -- (5.796,0.699474);
\draw [c,line width=0.9] (5.796,0.699474) -- (5.84,0.699474);
\definecolor{c}{rgb}{0,0,0};
\foreach \P in {(5.796,0.699474)}{\draw[mark options={color=c,fill=c},mark size=2.402402pt,mark=*,mark size=1pt] plot coordinates {\P};}
\colorlet{c}{natgreen};
\draw [c,line width=0.9] (5.884,0.688771) -- (5.884,0.693358);
\draw [c,line width=0.9] (5.884,0.693358) -- (5.884,0.697945);
\draw [c,line width=0.9] (5.84,0.693358) -- (5.884,0.693358);
\draw [c,line width=0.9] (5.884,0.693358) -- (5.928,0.693358);
\definecolor{c}{rgb}{0,0,0};
\foreach \P in {(5.884,0.693358)}{\draw[mark options={color=c,fill=c},mark size=2.402402pt,mark=*,mark size=1pt] plot coordinates {\P};}
\colorlet{c}{natgreen};
\draw [c,line width=0.9] (5.972,0.697945) -- (5.972,0.70406);
\draw [c,line width=0.9] (5.972,0.70406) -- (5.972,0.710176);
\draw [c,line width=0.9] (5.928,0.70406) -- (5.972,0.70406);
\draw [c,line width=0.9] (5.972,0.70406) -- (6.016,0.70406);
\definecolor{c}{rgb}{0,0,0};
\foreach \P in {(5.972,0.70406)}{\draw[mark options={color=c,fill=c},mark size=2.402402pt,mark=*,mark size=1pt] plot coordinates {\P};}
\colorlet{c}{natgreen};
\draw [c,line width=0.9] (6.06,0.686255) -- (6.06,0.6903);
\draw [c,line width=0.9] (6.06,0.6903) -- (6.06,0.694345);
\draw [c,line width=0.9] (6.016,0.6903) -- (6.06,0.6903);
\draw [c,line width=0.9] (6.06,0.6903) -- (6.104,0.6903);
\definecolor{c}{rgb}{0,0,0};
\foreach \P in {(6.06,0.6903)}{\draw[mark options={color=c,fill=c},mark size=2.402402pt,mark=*,mark size=1pt] plot coordinates {\P};}
\colorlet{c}{natgreen};
\draw [c,line width=0.9] (6.148,0.685026) -- (6.148,0.688771);
\draw [c,line width=0.9] (6.148,0.688771) -- (6.148,0.692516);
\draw [c,line width=0.9] (6.104,0.688771) -- (6.148,0.688771);
\draw [c,line width=0.9] (6.148,0.688771) -- (6.192,0.688771);
\definecolor{c}{rgb}{0,0,0};
\foreach \P in {(6.148,0.688771)}{\draw[mark options={color=c,fill=c},mark size=2.402402pt,mark=*,mark size=1pt] plot coordinates {\P};}
\colorlet{c}{natgreen};
\draw [c,line width=0.9] (6.236,0.686255) -- (6.236,0.6903);
\draw [c,line width=0.9] (6.236,0.6903) -- (6.236,0.694345);
\draw [c,line width=0.9] (6.192,0.6903) -- (6.236,0.6903);
\draw [c,line width=0.9] (6.236,0.6903) -- (6.28,0.6903);
\definecolor{c}{rgb}{0,0,0};
\foreach \P in {(6.236,0.6903)}{\draw[mark options={color=c,fill=c},mark size=2.402402pt,mark=*,mark size=1pt] plot coordinates {\P};}
\colorlet{c}{natgreen};
\draw [c,line width=0.9] (6.324,0.687505) -- (6.324,0.691829);
\draw [c,line width=0.9] (6.324,0.691829) -- (6.324,0.696154);
\draw [c,line width=0.9] (6.28,0.691829) -- (6.324,0.691829);
\draw [c,line width=0.9] (6.324,0.691829) -- (6.368,0.691829);
\definecolor{c}{rgb}{0,0,0};
\foreach \P in {(6.324,0.691829)}{\draw[mark options={color=c,fill=c},mark size=2.402402pt,mark=*,mark size=1pt] plot coordinates {\P};}
\colorlet{c}{natgreen};
\draw [c,line width=0.9] (6.412,0.683824) -- (6.412,0.687242);
\draw [c,line width=0.9] (6.412,0.687242) -- (6.412,0.690661);
\draw [c,line width=0.9] (6.368,0.687242) -- (6.412,0.687242);
\draw [c,line width=0.9] (6.412,0.687242) -- (6.456,0.687242);
\definecolor{c}{rgb}{0,0,0};
\foreach \P in {(6.412,0.687242)}{\draw[mark options={color=c,fill=c},mark size=2.402402pt,mark=*,mark size=1pt] plot coordinates {\P};}
\colorlet{c}{natgreen};
\draw [c,line width=0.9] (6.5,0.686255) -- (6.5,0.6903);
\draw [c,line width=0.9] (6.5,0.6903) -- (6.5,0.694345);
\draw [c,line width=0.9] (6.456,0.6903) -- (6.5,0.6903);
\draw [c,line width=0.9] (6.5,0.6903) -- (6.544,0.6903);
\definecolor{c}{rgb}{0,0,0};
\foreach \P in {(6.5,0.6903)}{\draw[mark options={color=c,fill=c},mark size=2.402402pt,mark=*,mark size=1pt] plot coordinates {\P};}
\colorlet{c}{natgreen};
\draw [c,line width=0.9] (6.588,0.680493) -- (6.588,0.682656);
\draw [c,line width=0.9] (6.588,0.682656) -- (6.588,0.684818);
\draw [c,line width=0.9] (6.544,0.682656) -- (6.588,0.682656);
\draw [c,line width=0.9] (6.588,0.682656) -- (6.632,0.682656);
\definecolor{c}{rgb}{0,0,0};
\foreach \P in {(6.588,0.682656)}{\draw[mark options={color=c,fill=c},mark size=2.402402pt,mark=*,mark size=1pt] plot coordinates {\P};}
\colorlet{c}{natgreen};
\draw [c,line width=0.9] (6.676,0.688771) -- (6.676,0.693358);
\draw [c,line width=0.9] (6.676,0.693358) -- (6.676,0.697945);
\draw [c,line width=0.9] (6.632,0.693358) -- (6.676,0.693358);
\draw [c,line width=0.9] (6.676,0.693358) -- (6.72,0.693358);
\definecolor{c}{rgb}{0,0,0};
\foreach \P in {(6.676,0.693358)}{\draw[mark options={color=c,fill=c},mark size=2.402402pt,mark=*,mark size=1pt] plot coordinates {\P};}
\colorlet{c}{natgreen};
\draw [c,line width=0.9] (6.764,0.682656) -- (6.764,0.685713);
\draw [c,line width=0.9] (6.764,0.685713) -- (6.764,0.688771);
\draw [c,line width=0.9] (6.72,0.685713) -- (6.764,0.685713);
\draw [c,line width=0.9] (6.764,0.685713) -- (6.808,0.685713);
\definecolor{c}{rgb}{0,0,0};
\foreach \P in {(6.764,0.685713)}{\draw[mark options={color=c,fill=c},mark size=2.402402pt,mark=*,mark size=1pt] plot coordinates {\P};}
\colorlet{c}{natgreen};
\draw [c,line width=0.9] (6.852,0.682656) -- (6.852,0.685713);
\draw [c,line width=0.9] (6.852,0.685713) -- (6.852,0.688771);
\draw [c,line width=0.9] (6.808,0.685713) -- (6.852,0.685713);
\draw [c,line width=0.9] (6.852,0.685713) -- (6.896,0.685713);
\definecolor{c}{rgb}{0,0,0};
\foreach \P in {(6.852,0.685713)}{\draw[mark options={color=c,fill=c},mark size=2.402402pt,mark=*,mark size=1pt] plot coordinates {\P};}
\colorlet{c}{natgreen};
\draw [c,line width=0.9] (6.94,0.683824) -- (6.94,0.687242);
\draw [c,line width=0.9] (6.94,0.687242) -- (6.94,0.690661);
\draw [c,line width=0.9] (6.896,0.687242) -- (6.94,0.687242);
\draw [c,line width=0.9] (6.94,0.687242) -- (6.984,0.687242);
\definecolor{c}{rgb}{0,0,0};
\foreach \P in {(6.94,0.687242)}{\draw[mark options={color=c,fill=c},mark size=2.402402pt,mark=*,mark size=1pt] plot coordinates {\P};}
\colorlet{c}{natgreen};
\draw [c,line width=0.9] (7.028,0.685026) -- (7.028,0.688771);
\draw [c,line width=0.9] (7.028,0.688771) -- (7.028,0.692516);
\draw [c,line width=0.9] (6.984,0.688771) -- (7.028,0.688771);
\draw [c,line width=0.9] (7.028,0.688771) -- (7.072,0.688771);
\definecolor{c}{rgb}{0,0,0};
\foreach \P in {(7.028,0.688771)}{\draw[mark options={color=c,fill=c},mark size=2.402402pt,mark=*,mark size=1pt] plot coordinates {\P};}
\colorlet{c}{natgreen};
\draw [c,line width=0.9] (7.116,0.688771) -- (7.116,0.693358);
\draw [c,line width=0.9] (7.116,0.693358) -- (7.116,0.697945);
\draw [c,line width=0.9] (7.072,0.693358) -- (7.116,0.693358);
\draw [c,line width=0.9] (7.116,0.693358) -- (7.16,0.693358);
\definecolor{c}{rgb}{0,0,0};
\foreach \P in {(7.116,0.693358)}{\draw[mark options={color=c,fill=c},mark size=2.402402pt,mark=*,mark size=1pt] plot coordinates {\P};}
\colorlet{c}{natgreen};
\draw [c,line width=0.9] (7.204,0.681536) -- (7.204,0.684184);
\draw [c,line width=0.9] (7.204,0.684184) -- (7.204,0.686833);
\draw [c,line width=0.9] (7.16,0.684184) -- (7.204,0.684184);
\draw [c,line width=0.9] (7.204,0.684184) -- (7.248,0.684184);
\definecolor{c}{rgb}{0,0,0};
\foreach \P in {(7.204,0.684184)}{\draw[mark options={color=c,fill=c},mark size=2.402402pt,mark=*,mark size=1pt] plot coordinates {\P};}
\colorlet{c}{natgreen};
\draw [c,line width=0.9] (7.292,0.685026) -- (7.292,0.688771);
\draw [c,line width=0.9] (7.292,0.688771) -- (7.292,0.692516);
\draw [c,line width=0.9] (7.248,0.688771) -- (7.292,0.688771);
\draw [c,line width=0.9] (7.292,0.688771) -- (7.336,0.688771);
\definecolor{c}{rgb}{0,0,0};
\foreach \P in {(7.292,0.688771)}{\draw[mark options={color=c,fill=c},mark size=2.402402pt,mark=*,mark size=1pt] plot coordinates {\P};}
\colorlet{c}{natgreen};
\draw [c,line width=0.9] (7.38,0.682656) -- (7.38,0.685713);
\draw [c,line width=0.9] (7.38,0.685713) -- (7.38,0.688771);
\draw [c,line width=0.9] (7.336,0.685713) -- (7.38,0.685713);
\draw [c,line width=0.9] (7.38,0.685713) -- (7.424,0.685713);
\definecolor{c}{rgb}{0,0,0};
\foreach \P in {(7.38,0.685713)}{\draw[mark options={color=c,fill=c},mark size=2.402402pt,mark=*,mark size=1pt] plot coordinates {\P};}
\colorlet{c}{natgreen};
\draw [c,line width=0.9] (7.468,0.680493) -- (7.468,0.682656);
\draw [c,line width=0.9] (7.468,0.682656) -- (7.468,0.684818);
\draw [c,line width=0.9] (7.424,0.682656) -- (7.468,0.682656);
\draw [c,line width=0.9] (7.468,0.682656) -- (7.512,0.682656);
\definecolor{c}{rgb}{0,0,0};
\foreach \P in {(7.468,0.682656)}{\draw[mark options={color=c,fill=c},mark size=2.402402pt,mark=*,mark size=1pt] plot coordinates {\P};}
\colorlet{c}{natgreen};
\draw [c,line width=0.9] (7.556,0.679598) -- (7.556,0.681127);
\draw [c,line width=0.9] (7.556,0.681127) -- (7.556,0.682656);
\draw [c,line width=0.9] (7.512,0.681127) -- (7.556,0.681127);
\draw [c,line width=0.9] (7.556,0.681127) -- (7.6,0.681127);
\definecolor{c}{rgb}{0,0,0};
\foreach \P in {(7.556,0.681127)}{\draw[mark options={color=c,fill=c},mark size=2.402402pt,mark=*,mark size=1pt] plot coordinates {\P};}
\colorlet{c}{natgreen};
\draw [c,line width=0.9] (7.644,0.681536) -- (7.644,0.684184);
\draw [c,line width=0.9] (7.644,0.684184) -- (7.644,0.686833);
\draw [c,line width=0.9] (7.6,0.684184) -- (7.644,0.684184);
\draw [c,line width=0.9] (7.644,0.684184) -- (7.688,0.684184);
\definecolor{c}{rgb}{0,0,0};
\foreach \P in {(7.644,0.684184)}{\draw[mark options={color=c,fill=c},mark size=2.402402pt,mark=*,mark size=1pt] plot coordinates {\P};}
\colorlet{c}{natgreen};
\draw [c,line width=0.9] (7.732,0.679598) -- (7.732,0.681127);
\draw [c,line width=0.9] (7.732,0.681127) -- (7.732,0.682656);
\draw [c,line width=0.9] (7.688,0.681127) -- (7.732,0.681127);
\draw [c,line width=0.9] (7.732,0.681127) -- (7.776,0.681127);
\definecolor{c}{rgb}{0,0,0};
\foreach \P in {(7.732,0.681127)}{\draw[mark options={color=c,fill=c},mark size=2.402402pt,mark=*,mark size=1pt] plot coordinates {\P};}
\colorlet{c}{natgreen};
\draw [c,line width=0.9] (7.82,0.679598) -- (7.82,0.681127);
\draw [c,line width=0.9] (7.82,0.681127) -- (7.82,0.682656);
\draw [c,line width=0.9] (7.776,0.681127) -- (7.82,0.681127);
\draw [c,line width=0.9] (7.82,0.681127) -- (7.864,0.681127);
\definecolor{c}{rgb}{0,0,0};
\foreach \P in {(7.82,0.681127)}{\draw[mark options={color=c,fill=c},mark size=2.402402pt,mark=*,mark size=1pt] plot coordinates {\P};}
\colorlet{c}{natgreen};
\draw [c,line width=0.9] (7.908,0.680493) -- (7.908,0.682656);
\draw [c,line width=0.9] (7.908,0.682656) -- (7.908,0.684818);
\draw [c,line width=0.9] (7.864,0.682656) -- (7.908,0.682656);
\draw [c,line width=0.9] (7.908,0.682656) -- (7.952,0.682656);
\definecolor{c}{rgb}{0,0,0};
\foreach \P in {(7.908,0.682656)}{\draw[mark options={color=c,fill=c},mark size=2.402402pt,mark=*,mark size=1pt] plot coordinates {\P};}
\colorlet{c}{natgreen};
\draw [c,line width=0.9] (7.996,0.679598) -- (7.996,0.681127);
\draw [c,line width=0.9] (7.996,0.681127) -- (7.996,0.682656);
\draw [c,line width=0.9] (7.952,0.681127) -- (7.996,0.681127);
\draw [c,line width=0.9] (7.996,0.681127) -- (8.04,0.681127);
\definecolor{c}{rgb}{0,0,0};
\foreach \P in {(7.996,0.681127)}{\draw[mark options={color=c,fill=c},mark size=2.402402pt,mark=*,mark size=1pt] plot coordinates {\P};}
\colorlet{c}{natgreen};
\draw [c,line width=0.9] (8.084,0.679598) -- (8.084,0.681127);
\draw [c,line width=0.9] (8.084,0.681127) -- (8.084,0.682656);
\draw [c,line width=0.9] (8.04,0.681127) -- (8.084,0.681127);
\draw [c,line width=0.9] (8.084,0.681127) -- (8.128,0.681127);
\definecolor{c}{rgb}{0,0,0};
\foreach \P in {(8.084,0.681127)}{\draw[mark options={color=c,fill=c},mark size=2.402402pt,mark=*,mark size=1pt] plot coordinates {\P};}
\colorlet{c}{natgreen};
\draw [c,line width=0.9] (8.172,0.685026) -- (8.172,0.688771);
\draw [c,line width=0.9] (8.172,0.688771) -- (8.172,0.692516);
\draw [c,line width=0.9] (8.128,0.688771) -- (8.172,0.688771);
\draw [c,line width=0.9] (8.172,0.688771) -- (8.216,0.688771);
\definecolor{c}{rgb}{0,0,0};
\foreach \P in {(8.172,0.688771)}{\draw[mark options={color=c,fill=c},mark size=2.402402pt,mark=*,mark size=1pt] plot coordinates {\P};}
\colorlet{c}{natgreen};
\draw [c,line width=0.9] (8.26,0.680493) -- (8.26,0.682656);
\draw [c,line width=0.9] (8.26,0.682656) -- (8.26,0.684818);
\draw [c,line width=0.9] (8.216,0.682656) -- (8.26,0.682656);
\draw [c,line width=0.9] (8.26,0.682656) -- (8.304,0.682656);
\definecolor{c}{rgb}{0,0,0};
\foreach \P in {(8.26,0.682656)}{\draw[mark options={color=c,fill=c},mark size=2.402402pt,mark=*,mark size=1pt] plot coordinates {\P};}
\colorlet{c}{natgreen};
\draw [c,line width=0.9] (8.612,0.679598) -- (8.612,0.681127);
\draw [c,line width=0.9] (8.612,0.681127) -- (8.612,0.682656);
\draw [c,line width=0.9] (8.568,0.681127) -- (8.612,0.681127);
\draw [c,line width=0.9] (8.612,0.681127) -- (8.656,0.681127);
\definecolor{c}{rgb}{0,0,0};
\foreach \P in {(8.612,0.681127)}{\draw[mark options={color=c,fill=c},mark size=2.402402pt,mark=*,mark size=1pt] plot coordinates {\P};}
\colorlet{c}{natgreen};
\draw [c,line width=0.9] (8.7,0.681536) -- (8.7,0.684184);
\draw [c,line width=0.9] (8.7,0.684184) -- (8.7,0.686833);
\draw [c,line width=0.9] (8.656,0.684184) -- (8.7,0.684184);
\draw [c,line width=0.9] (8.7,0.684184) -- (8.744,0.684184);
\definecolor{c}{rgb}{0,0,0};
\foreach \P in {(8.7,0.684184)}{\draw[mark options={color=c,fill=c},mark size=2.402402pt,mark=*,mark size=1pt] plot coordinates {\P};}
\colorlet{c}{natgreen};
\draw [c,line width=0.9] (8.788,0.679598) -- (8.788,0.681127);
\draw [c,line width=0.9] (8.788,0.681127) -- (8.788,0.682656);
\draw [c,line width=0.9] (8.744,0.681127) -- (8.788,0.681127);
\draw [c,line width=0.9] (8.788,0.681127) -- (8.832,0.681127);
\definecolor{c}{rgb}{0,0,0};
\foreach \P in {(8.788,0.681127)}{\draw[mark options={color=c,fill=c},mark size=2.402402pt,mark=*,mark size=1pt] plot coordinates {\P};}
\colorlet{c}{natgreen};
\draw [c,line width=0.9] (8.876,0.679598) -- (8.876,0.681127);
\draw [c,line width=0.9] (8.876,0.681127) -- (8.876,0.682656);
\draw [c,line width=0.9] (8.832,0.681127) -- (8.876,0.681127);
\draw [c,line width=0.9] (8.876,0.681127) -- (8.92,0.681127);
\definecolor{c}{rgb}{0,0,0};
\foreach \P in {(8.876,0.681127)}{\draw[mark options={color=c,fill=c},mark size=2.402402pt,mark=*,mark size=1pt] plot coordinates {\P};}
\colorlet{c}{natgreen};
\draw [c,line width=0.9] (9.14,0.679598) -- (9.14,0.681127);
\draw [c,line width=0.9] (9.14,0.681127) -- (9.14,0.682656);
\draw [c,line width=0.9] (9.096,0.681127) -- (9.14,0.681127);
\draw [c,line width=0.9] (9.14,0.681127) -- (9.184,0.681127);
\definecolor{c}{rgb}{0,0,0};
\foreach \P in {(9.14,0.681127)}{\draw[mark options={color=c,fill=c},mark size=2.402402pt,mark=*,mark size=1pt] plot coordinates {\P};}
\colorlet{c}{natgreen};
\draw [c,line width=0.9] (9.228,0.680493) -- (9.228,0.682656);
\draw [c,line width=0.9] (9.228,0.682656) -- (9.228,0.684818);
\draw [c,line width=0.9] (9.184,0.682656) -- (9.228,0.682656);
\draw [c,line width=0.9] (9.228,0.682656) -- (9.272,0.682656);
\definecolor{c}{rgb}{0,0,0};
\foreach \P in {(9.228,0.682656)}{\draw[mark options={color=c,fill=c},mark size=2.402402pt,mark=*,mark size=1pt] plot coordinates {\P};}
\colorlet{c}{natgreen};
\draw [c,line width=0.9] (9.316,0.682656) -- (9.316,0.685713);
\draw [c,line width=0.9] (9.316,0.685713) -- (9.316,0.688771);
\draw [c,line width=0.9] (9.272,0.685713) -- (9.316,0.685713);
\draw [c,line width=0.9] (9.316,0.685713) -- (9.36,0.685713);
\definecolor{c}{rgb}{0,0,0};
\foreach \P in {(9.316,0.685713)}{\draw[mark options={color=c,fill=c},mark size=2.402402pt,mark=*,mark size=1pt] plot coordinates {\P};}
\colorlet{c}{natgreen};
\draw [c,line width=0.9] (9.58,0.679598) -- (9.58,0.681127);
\draw [c,line width=0.9] (9.58,0.681127) -- (9.58,0.682656);
\draw [c,line width=0.9] (9.536,0.681127) -- (9.58,0.681127);
\draw [c,line width=0.9] (9.58,0.681127) -- (9.624,0.681127);
\definecolor{c}{rgb}{0,0,0};
\foreach \P in {(9.58,0.681127)}{\draw[mark options={color=c,fill=c},mark size=2.402402pt,mark=*,mark size=1pt] plot coordinates {\P};}
\colorlet{c}{natgreen};
\draw [c,line width=0.9] (9.756,0.679598) -- (9.756,0.681127);
\draw [c,line width=0.9] (9.756,0.681127) -- (9.756,0.682656);
\draw [c,line width=0.9] (9.712,0.681127) -- (9.756,0.681127);
\draw [c,line width=0.9] (9.756,0.681127) -- (9.8,0.681127);
\definecolor{c}{rgb}{0,0,0};
\foreach \P in {(9.756,0.681127)}{\draw[mark options={color=c,fill=c},mark size=2.402402pt,mark=*,mark size=1pt] plot coordinates {\P};}
\colorlet{c}{natgreen!80};
\draw [c,line width=0.9] (1.22,0.679598) -- (1.22,0.712991);
\draw [c,line width=0.9] (1.22,0.712991) -- (1.22,0.746384);
\draw [c,line width=0.9] (1.176,0.712991) -- (1.22,0.712991);
\draw [c,line width=0.9] (1.22,0.712991) -- (1.264,0.712991);
\definecolor{c}{rgb}{0,0,0};
\foreach \P in {(1.22,0.712991)}{\draw[mark options={color=c,fill=c},mark size=2.402402pt,mark=*,mark size=1pt] plot coordinates {\P};}
\colorlet{c}{natgreen!80};
\draw [c,line width=0.9] (1.308,0.852293) -- (1.308,0.946743);
\draw [c,line width=0.9] (1.308,0.946743) -- (1.308,1.04119);
\draw [c,line width=0.9] (1.264,0.946743) -- (1.308,0.946743);
\draw [c,line width=0.9] (1.308,0.946743) -- (1.352,0.946743);
\definecolor{c}{rgb}{0,0,0};
\foreach \P in {(1.308,0.946743)}{\draw[mark options={color=c,fill=c},mark size=2.402402pt,mark=*,mark size=1pt] plot coordinates {\P};}
\colorlet{c}{natgreen!80};
\draw [c,line width=0.9] (1.396,0.746384) -- (1.396,0.81317);
\draw [c,line width=0.9] (1.396,0.81317) -- (1.396,0.879957);
\draw [c,line width=0.9] (1.352,0.81317) -- (1.396,0.81317);
\draw [c,line width=0.9] (1.396,0.81317) -- (1.44,0.81317);
\definecolor{c}{rgb}{0,0,0};
\foreach \P in {(1.396,0.81317)}{\draw[mark options={color=c,fill=c},mark size=2.402402pt,mark=*,mark size=1pt] plot coordinates {\P};}
\colorlet{c}{natgreen!80};
\draw [c,line width=0.9] (1.484,0.721939) -- (1.484,0.779777);
\draw [c,line width=0.9] (1.484,0.779777) -- (1.484,0.837616);
\draw [c,line width=0.9] (1.44,0.779777) -- (1.484,0.779777);
\draw [c,line width=0.9] (1.484,0.779777) -- (1.528,0.779777);
\definecolor{c}{rgb}{0,0,0};
\foreach \P in {(1.484,0.779777)}{\draw[mark options={color=c,fill=c},mark size=2.402402pt,mark=*,mark size=1pt] plot coordinates {\P};}
\colorlet{c}{natgreen!80};
\draw [c,line width=0.9] (1.572,0.721939) -- (1.572,0.779777);
\draw [c,line width=0.9] (1.572,0.779777) -- (1.572,0.837616);
\draw [c,line width=0.9] (1.528,0.779777) -- (1.572,0.779777);
\draw [c,line width=0.9] (1.572,0.779777) -- (1.616,0.779777);
\definecolor{c}{rgb}{0,0,0};
\foreach \P in {(1.572,0.779777)}{\draw[mark options={color=c,fill=c},mark size=2.402402pt,mark=*,mark size=1pt] plot coordinates {\P};}
\colorlet{c}{natgreen!80};
\draw [c,line width=0.9] (1.66,0.825) -- (1.66,0.91335);
\draw [c,line width=0.9] (1.66,0.91335) -- (1.66,1.0017);
\draw [c,line width=0.9] (1.616,0.91335) -- (1.66,0.91335);
\draw [c,line width=0.9] (1.66,0.91335) -- (1.704,0.91335);
\definecolor{c}{rgb}{0,0,0};
\foreach \P in {(1.66,0.91335)}{\draw[mark options={color=c,fill=c},mark size=2.402402pt,mark=*,mark size=1pt] plot coordinates {\P};}
\colorlet{c}{natgreen!80};
\draw [c,line width=0.9] (1.748,0.771894) -- (1.748,0.846564);
\draw [c,line width=0.9] (1.748,0.846564) -- (1.748,0.921233);
\draw [c,line width=0.9] (1.704,0.846564) -- (1.748,0.846564);
\draw [c,line width=0.9] (1.748,0.846564) -- (1.792,0.846564);
\definecolor{c}{rgb}{0,0,0};
\foreach \P in {(1.748,0.846564)}{\draw[mark options={color=c,fill=c},mark size=2.402402pt,mark=*,mark size=1pt] plot coordinates {\P};}
\colorlet{c}{natgreen!80};
\draw [c,line width=0.9] (1.836,0.79816) -- (1.836,0.879957);
\draw [c,line width=0.9] (1.836,0.879957) -- (1.836,0.961753);
\draw [c,line width=0.9] (1.792,0.879957) -- (1.836,0.879957);
\draw [c,line width=0.9] (1.836,0.879957) -- (1.88,0.879957);
\definecolor{c}{rgb}{0,0,0};
\foreach \P in {(1.836,0.879957)}{\draw[mark options={color=c,fill=c},mark size=2.402402pt,mark=*,mark size=1pt] plot coordinates {\P};}
\colorlet{c}{natgreen!80};
\draw [c,line width=0.9] (1.924,0.825) -- (1.924,0.91335);
\draw [c,line width=0.9] (1.924,0.91335) -- (1.924,1.0017);
\draw [c,line width=0.9] (1.88,0.91335) -- (1.924,0.91335);
\draw [c,line width=0.9] (1.924,0.91335) -- (1.968,0.91335);
\definecolor{c}{rgb}{0,0,0};
\foreach \P in {(1.924,0.91335)}{\draw[mark options={color=c,fill=c},mark size=2.402402pt,mark=*,mark size=1pt] plot coordinates {\P};}
\colorlet{c}{natgreen!80};
\draw [c,line width=0.9] (2.012,0.93617) -- (2.012,1.04692);
\draw [c,line width=0.9] (2.012,1.04692) -- (2.012,1.15768);
\draw [c,line width=0.9] (1.968,1.04692) -- (2.012,1.04692);
\draw [c,line width=0.9] (2.012,1.04692) -- (2.056,1.04692);
\definecolor{c}{rgb}{0,0,0};
\foreach \P in {(2.012,1.04692)}{\draw[mark options={color=c,fill=c},mark size=2.402402pt,mark=*,mark size=1pt] plot coordinates {\P};}
\colorlet{c}{natgreen!80};
\draw [c,line width=0.9] (2.1,1.43791) -- (2.1,1.61461);
\draw [c,line width=0.9] (2.1,1.61461) -- (2.1,1.79131);
\draw [c,line width=0.9] (2.056,1.61461) -- (2.1,1.61461);
\draw [c,line width=0.9] (2.1,1.61461) -- (2.144,1.61461);
\definecolor{c}{rgb}{0,0,0};
\foreach \P in {(2.1,1.61461)}{\draw[mark options={color=c,fill=c},mark size=2.402402pt,mark=*,mark size=1pt] plot coordinates {\P};}
\colorlet{c}{natgreen!80};
\draw [c,line width=0.9] (2.188,1.58974) -- (2.188,1.78157);
\draw [c,line width=0.9] (2.188,1.78157) -- (2.188,1.9734);
\draw [c,line width=0.9] (2.144,1.78157) -- (2.188,1.78157);
\draw [c,line width=0.9] (2.188,1.78157) -- (2.232,1.78157);
\definecolor{c}{rgb}{0,0,0};
\foreach \P in {(2.188,1.78157)}{\draw[mark options={color=c,fill=c},mark size=2.402402pt,mark=*,mark size=1pt] plot coordinates {\P};}
\colorlet{c}{natgreen!80};
\draw [c,line width=0.9] (2.276,2.3933) -- (2.276,2.64979);
\draw [c,line width=0.9] (2.276,2.64979) -- (2.276,2.90629);
\draw [c,line width=0.9] (2.232,2.64979) -- (2.276,2.64979);
\draw [c,line width=0.9] (2.276,2.64979) -- (2.32,2.64979);
\definecolor{c}{rgb}{0,0,0};
\foreach \P in {(2.276,2.64979)}{\draw[mark options={color=c,fill=c},mark size=2.402402pt,mark=*,mark size=1pt] plot coordinates {\P};}
\colorlet{c}{natgreen!80};
\draw [c,line width=0.9] (2.364,1.80413) -- (2.364,2.01532);
\draw [c,line width=0.9] (2.364,2.01532) -- (2.364,2.22652);
\draw [c,line width=0.9] (2.32,2.01532) -- (2.364,2.01532);
\draw [c,line width=0.9] (2.364,2.01532) -- (2.408,2.01532);
\definecolor{c}{rgb}{0,0,0};
\foreach \P in {(2.364,2.01532)}{\draw[mark options={color=c,fill=c},mark size=2.402402pt,mark=*,mark size=1pt] plot coordinates {\P};}
\colorlet{c}{natgreen!80};
\draw [c,line width=0.9] (2.452,1.95828) -- (2.452,2.18229);
\draw [c,line width=0.9] (2.452,2.18229) -- (2.452,2.4063);
\draw [c,line width=0.9] (2.408,2.18229) -- (2.452,2.18229);
\draw [c,line width=0.9] (2.452,2.18229) -- (2.496,2.18229);
\definecolor{c}{rgb}{0,0,0};
\foreach \P in {(2.452,2.18229)}{\draw[mark options={color=c,fill=c},mark size=2.402402pt,mark=*,mark size=1pt] plot coordinates {\P};}
\colorlet{c}{natgreen!80};
\draw [c,line width=0.9] (2.54,1.71202) -- (2.54,1.91514);
\draw [c,line width=0.9] (2.54,1.91514) -- (2.54,2.11827);
\draw [c,line width=0.9] (2.496,1.91514) -- (2.54,1.91514);
\draw [c,line width=0.9] (2.54,1.91514) -- (2.584,1.91514);
\definecolor{c}{rgb}{0,0,0};
\foreach \P in {(2.54,1.91514)}{\draw[mark options={color=c,fill=c},mark size=2.402402pt,mark=*,mark size=1pt] plot coordinates {\P};}
\colorlet{c}{natgreen!80};
\draw [c,line width=0.9] (2.628,1.37755) -- (2.628,1.54782);
\draw [c,line width=0.9] (2.628,1.54782) -- (2.628,1.71809);
\draw [c,line width=0.9] (2.584,1.54782) -- (2.628,1.54782);
\draw [c,line width=0.9] (2.628,1.54782) -- (2.672,1.54782);
\definecolor{c}{rgb}{0,0,0};
\foreach \P in {(2.628,1.54782)}{\draw[mark options={color=c,fill=c},mark size=2.402402pt,mark=*,mark size=1pt] plot coordinates {\P};}
\colorlet{c}{natgreen!80};
\draw [c,line width=0.9] (2.716,1.1096) -- (2.716,1.24728);
\draw [c,line width=0.9] (2.716,1.24728) -- (2.716,1.38497);
\draw [c,line width=0.9] (2.672,1.24728) -- (2.716,1.24728);
\draw [c,line width=0.9] (2.716,1.24728) -- (2.76,1.24728);
\definecolor{c}{rgb}{0,0,0};
\foreach \P in {(2.716,1.24728)}{\draw[mark options={color=c,fill=c},mark size=2.402402pt,mark=*,mark size=1pt] plot coordinates {\P};}
\colorlet{c}{natgreen!80};
\draw [c,line width=0.9] (2.804,1.28749) -- (2.804,1.44764);
\draw [c,line width=0.9] (2.804,1.44764) -- (2.804,1.60779);
\draw [c,line width=0.9] (2.76,1.44764) -- (2.804,1.44764);
\draw [c,line width=0.9] (2.804,1.44764) -- (2.848,1.44764);
\definecolor{c}{rgb}{0,0,0};
\foreach \P in {(2.804,1.44764)}{\draw[mark options={color=c,fill=c},mark size=2.402402pt,mark=*,mark size=1pt] plot coordinates {\P};}
\colorlet{c}{natgreen!80};
\draw [c,line width=0.9] (2.892,1.28749) -- (2.892,1.44764);
\draw [c,line width=0.9] (2.892,1.44764) -- (2.892,1.60779);
\draw [c,line width=0.9] (2.848,1.44764) -- (2.892,1.44764);
\draw [c,line width=0.9] (2.892,1.44764) -- (2.936,1.44764);
\definecolor{c}{rgb}{0,0,0};
\foreach \P in {(2.892,1.44764)}{\draw[mark options={color=c,fill=c},mark size=2.402402pt,mark=*,mark size=1pt] plot coordinates {\P};}
\colorlet{c}{natgreen!80};
\draw [c,line width=0.9] (2.98,0.993308) -- (2.98,1.11371);
\draw [c,line width=0.9] (2.98,1.11371) -- (2.98,1.23411);
\draw [c,line width=0.9] (2.936,1.11371) -- (2.98,1.11371);
\draw [c,line width=0.9] (2.98,1.11371) -- (3.024,1.11371);
\definecolor{c}{rgb}{0,0,0};
\foreach \P in {(2.98,1.11371)}{\draw[mark options={color=c,fill=c},mark size=2.402402pt,mark=*,mark size=1pt] plot coordinates {\P};}
\colorlet{c}{natgreen!80};
\draw [c,line width=0.9] (3.068,0.964638) -- (3.068,1.08032);
\draw [c,line width=0.9] (3.068,1.08032) -- (3.068,1.19599);
\draw [c,line width=0.9] (3.024,1.08032) -- (3.068,1.08032);
\draw [c,line width=0.9] (3.068,1.08032) -- (3.112,1.08032);
\definecolor{c}{rgb}{0,0,0};
\foreach \P in {(3.068,1.08032)}{\draw[mark options={color=c,fill=c},mark size=2.402402pt,mark=*,mark size=1pt] plot coordinates {\P};}
\colorlet{c}{natgreen!80};
\draw [c,line width=0.9] (3.156,0.825) -- (3.156,0.91335);
\draw [c,line width=0.9] (3.156,0.91335) -- (3.156,1.0017);
\draw [c,line width=0.9] (3.112,0.91335) -- (3.156,0.91335);
\draw [c,line width=0.9] (3.156,0.91335) -- (3.2,0.91335);
\definecolor{c}{rgb}{0,0,0};
\foreach \P in {(3.156,0.91335)}{\draw[mark options={color=c,fill=c},mark size=2.402402pt,mark=*,mark size=1pt] plot coordinates {\P};}
\colorlet{c}{natgreen!80};
\draw [c,line width=0.9] (3.244,0.93617) -- (3.244,1.04692);
\draw [c,line width=0.9] (3.244,1.04692) -- (3.244,1.15768);
\draw [c,line width=0.9] (3.2,1.04692) -- (3.244,1.04692);
\draw [c,line width=0.9] (3.244,1.04692) -- (3.288,1.04692);
\definecolor{c}{rgb}{0,0,0};
\foreach \P in {(3.244,1.04692)}{\draw[mark options={color=c,fill=c},mark size=2.402402pt,mark=*,mark size=1pt] plot coordinates {\P};}
\colorlet{c}{natgreen!80};
\draw [c,line width=0.9] (3.332,0.825) -- (3.332,0.91335);
\draw [c,line width=0.9] (3.332,0.91335) -- (3.332,1.0017);
\draw [c,line width=0.9] (3.288,0.91335) -- (3.332,0.91335);
\draw [c,line width=0.9] (3.332,0.91335) -- (3.376,0.91335);
\definecolor{c}{rgb}{0,0,0};
\foreach \P in {(3.332,0.91335)}{\draw[mark options={color=c,fill=c},mark size=2.402402pt,mark=*,mark size=1pt] plot coordinates {\P};}
\colorlet{c}{natgreen!80};
\draw [c,line width=0.9] (3.42,0.771894) -- (3.42,0.846564);
\draw [c,line width=0.9] (3.42,0.846564) -- (3.42,0.921233);
\draw [c,line width=0.9] (3.376,0.846564) -- (3.42,0.846564);
\draw [c,line width=0.9] (3.42,0.846564) -- (3.464,0.846564);
\definecolor{c}{rgb}{0,0,0};
\foreach \P in {(3.42,0.846564)}{\draw[mark options={color=c,fill=c},mark size=2.402402pt,mark=*,mark size=1pt] plot coordinates {\P};}
\colorlet{c}{natgreen!80};
\draw [c,line width=0.9] (3.508,0.746384) -- (3.508,0.81317);
\draw [c,line width=0.9] (3.508,0.81317) -- (3.508,0.879957);
\draw [c,line width=0.9] (3.464,0.81317) -- (3.508,0.81317);
\draw [c,line width=0.9] (3.508,0.81317) -- (3.552,0.81317);
\definecolor{c}{rgb}{0,0,0};
\foreach \P in {(3.508,0.81317)}{\draw[mark options={color=c,fill=c},mark size=2.402402pt,mark=*,mark size=1pt] plot coordinates {\P};}
\colorlet{c}{natgreen!80};
\draw [c,line width=0.9] (3.596,0.879957) -- (3.596,0.980136);
\draw [c,line width=0.9] (3.596,0.980136) -- (3.596,1.08032);
\draw [c,line width=0.9] (3.552,0.980136) -- (3.596,0.980136);
\draw [c,line width=0.9] (3.596,0.980136) -- (3.64,0.980136);
\definecolor{c}{rgb}{0,0,0};
\foreach \P in {(3.596,0.980136)}{\draw[mark options={color=c,fill=c},mark size=2.402402pt,mark=*,mark size=1pt] plot coordinates {\P};}
\colorlet{c}{natgreen!80};
\draw [c,line width=0.9] (3.684,0.825) -- (3.684,0.91335);
\draw [c,line width=0.9] (3.684,0.91335) -- (3.684,1.0017);
\draw [c,line width=0.9] (3.64,0.91335) -- (3.684,0.91335);
\draw [c,line width=0.9] (3.684,0.91335) -- (3.728,0.91335);
\definecolor{c}{rgb}{0,0,0};
\foreach \P in {(3.684,0.91335)}{\draw[mark options={color=c,fill=c},mark size=2.402402pt,mark=*,mark size=1pt] plot coordinates {\P};}
\colorlet{c}{natgreen!80};
\draw [c,line width=0.9] (3.772,0.746384) -- (3.772,0.81317);
\draw [c,line width=0.9] (3.772,0.81317) -- (3.772,0.879957);
\draw [c,line width=0.9] (3.728,0.81317) -- (3.772,0.81317);
\draw [c,line width=0.9] (3.772,0.81317) -- (3.816,0.81317);
\definecolor{c}{rgb}{0,0,0};
\foreach \P in {(3.772,0.81317)}{\draw[mark options={color=c,fill=c},mark size=2.402402pt,mark=*,mark size=1pt] plot coordinates {\P};}
\colorlet{c}{natgreen!80};
\draw [c,line width=0.9] (3.86,0.746384) -- (3.86,0.81317);
\draw [c,line width=0.9] (3.86,0.81317) -- (3.86,0.879957);
\draw [c,line width=0.9] (3.816,0.81317) -- (3.86,0.81317);
\draw [c,line width=0.9] (3.86,0.81317) -- (3.904,0.81317);
\definecolor{c}{rgb}{0,0,0};
\foreach \P in {(3.86,0.81317)}{\draw[mark options={color=c,fill=c},mark size=2.402402pt,mark=*,mark size=1pt] plot coordinates {\P};}
\colorlet{c}{natgreen!80};
\draw [c,line width=0.9] (3.948,0.721939) -- (3.948,0.779777);
\draw [c,line width=0.9] (3.948,0.779777) -- (3.948,0.837616);
\draw [c,line width=0.9] (3.904,0.779777) -- (3.948,0.779777);
\draw [c,line width=0.9] (3.948,0.779777) -- (3.992,0.779777);
\definecolor{c}{rgb}{0,0,0};
\foreach \P in {(3.948,0.779777)}{\draw[mark options={color=c,fill=c},mark size=2.402402pt,mark=*,mark size=1pt] plot coordinates {\P};}
\colorlet{c}{natgreen!80};
\draw [c,line width=0.9] (4.036,0.699159) -- (4.036,0.746384);
\draw [c,line width=0.9] (4.036,0.746384) -- (4.036,0.793609);
\draw [c,line width=0.9] (3.992,0.746384) -- (4.036,0.746384);
\draw [c,line width=0.9] (4.036,0.746384) -- (4.08,0.746384);
\definecolor{c}{rgb}{0,0,0};
\foreach \P in {(4.036,0.746384)}{\draw[mark options={color=c,fill=c},mark size=2.402402pt,mark=*,mark size=1pt] plot coordinates {\P};}
\colorlet{c}{natgreen!80};
\draw [c,line width=0.9] (4.124,0.699159) -- (4.124,0.746384);
\draw [c,line width=0.9] (4.124,0.746384) -- (4.124,0.793609);
\draw [c,line width=0.9] (4.08,0.746384) -- (4.124,0.746384);
\draw [c,line width=0.9] (4.124,0.746384) -- (4.168,0.746384);
\definecolor{c}{rgb}{0,0,0};
\foreach \P in {(4.124,0.746384)}{\draw[mark options={color=c,fill=c},mark size=2.402402pt,mark=*,mark size=1pt] plot coordinates {\P};}
\colorlet{c}{natgreen!80};
\draw [c,line width=0.9] (4.212,0.679598) -- (4.212,0.712991);
\draw [c,line width=0.9] (4.212,0.712991) -- (4.212,0.746384);
\draw [c,line width=0.9] (4.168,0.712991) -- (4.212,0.712991);
\draw [c,line width=0.9] (4.212,0.712991) -- (4.256,0.712991);
\definecolor{c}{rgb}{0,0,0};
\foreach \P in {(4.212,0.712991)}{\draw[mark options={color=c,fill=c},mark size=2.402402pt,mark=*,mark size=1pt] plot coordinates {\P};}
\colorlet{c}{natgreen!80};
\draw [c,line width=0.9] (4.3,0.771894) -- (4.3,0.846564);
\draw [c,line width=0.9] (4.3,0.846564) -- (4.3,0.921233);
\draw [c,line width=0.9] (4.256,0.846564) -- (4.3,0.846564);
\draw [c,line width=0.9] (4.3,0.846564) -- (4.344,0.846564);
\definecolor{c}{rgb}{0,0,0};
\foreach \P in {(4.3,0.846564)}{\draw[mark options={color=c,fill=c},mark size=2.402402pt,mark=*,mark size=1pt] plot coordinates {\P};}
\colorlet{c}{natgreen!80};
\draw [c,line width=0.9] (4.564,0.679598) -- (4.564,0.712991);
\draw [c,line width=0.9] (4.564,0.712991) -- (4.564,0.746384);
\draw [c,line width=0.9] (4.52,0.712991) -- (4.564,0.712991);
\draw [c,line width=0.9] (4.564,0.712991) -- (4.608,0.712991);
\definecolor{c}{rgb}{0,0,0};
\foreach \P in {(4.564,0.712991)}{\draw[mark options={color=c,fill=c},mark size=2.402402pt,mark=*,mark size=1pt] plot coordinates {\P};}
\colorlet{c}{natgreen!80};
\draw [c,line width=0.9] (4.652,0.679598) -- (4.652,0.712991);
\draw [c,line width=0.9] (4.652,0.712991) -- (4.652,0.746384);
\draw [c,line width=0.9] (4.608,0.712991) -- (4.652,0.712991);
\draw [c,line width=0.9] (4.652,0.712991) -- (4.696,0.712991);
\definecolor{c}{rgb}{0,0,0};
\foreach \P in {(4.652,0.712991)}{\draw[mark options={color=c,fill=c},mark size=2.402402pt,mark=*,mark size=1pt] plot coordinates {\P};}
\colorlet{c}{natgreen!80};
\draw [c,line width=0.9] (4.828,0.679598) -- (4.828,0.712991);
\draw [c,line width=0.9] (4.828,0.712991) -- (4.828,0.746384);
\draw [c,line width=0.9] (4.784,0.712991) -- (4.828,0.712991);
\draw [c,line width=0.9] (4.828,0.712991) -- (4.872,0.712991);
\definecolor{c}{rgb}{0,0,0};
\foreach \P in {(4.828,0.712991)}{\draw[mark options={color=c,fill=c},mark size=2.402402pt,mark=*,mark size=1pt] plot coordinates {\P};}
\colorlet{c}{natgreen!80};
\draw [c,line width=0.9] (4.916,0.679598) -- (4.916,0.712991);
\draw [c,line width=0.9] (4.916,0.712991) -- (4.916,0.746384);
\draw [c,line width=0.9] (4.872,0.712991) -- (4.916,0.712991);
\draw [c,line width=0.9] (4.916,0.712991) -- (4.96,0.712991);
\definecolor{c}{rgb}{0,0,0};
\foreach \P in {(4.916,0.712991)}{\draw[mark options={color=c,fill=c},mark size=2.402402pt,mark=*,mark size=1pt] plot coordinates {\P};}
\colorlet{c}{natgreen!80};
\draw [c,line width=0.9] (5.004,0.679598) -- (5.004,0.712991);
\draw [c,line width=0.9] (5.004,0.712991) -- (5.004,0.746384);
\draw [c,line width=0.9] (4.96,0.712991) -- (5.004,0.712991);
\draw [c,line width=0.9] (5.004,0.712991) -- (5.048,0.712991);
\definecolor{c}{rgb}{0,0,0};
\foreach \P in {(5.004,0.712991)}{\draw[mark options={color=c,fill=c},mark size=2.402402pt,mark=*,mark size=1pt] plot coordinates {\P};}
\colorlet{c}{natgreen!80};
\draw [c,line width=0.9] (5.092,0.679598) -- (5.092,0.712991);
\draw [c,line width=0.9] (5.092,0.712991) -- (5.092,0.746384);
\draw [c,line width=0.9] (5.048,0.712991) -- (5.092,0.712991);
\draw [c,line width=0.9] (5.092,0.712991) -- (5.136,0.712991);
\definecolor{c}{rgb}{0,0,0};
\foreach \P in {(5.092,0.712991)}{\draw[mark options={color=c,fill=c},mark size=2.402402pt,mark=*,mark size=1pt] plot coordinates {\P};}
\colorlet{c}{natgreen!80};
\draw [c,line width=0.9] (5.444,0.679598) -- (5.444,0.712991);
\draw [c,line width=0.9] (5.444,0.712991) -- (5.444,0.746384);
\draw [c,line width=0.9] (5.4,0.712991) -- (5.444,0.712991);
\draw [c,line width=0.9] (5.444,0.712991) -- (5.488,0.712991);
\definecolor{c}{rgb}{0,0,0};
\foreach \P in {(5.444,0.712991)}{\draw[mark options={color=c,fill=c},mark size=2.402402pt,mark=*,mark size=1pt] plot coordinates {\P};}
\colorlet{c}{natgreen!80};
\draw [c,line width=0.9] (5.796,0.679598) -- (5.796,0.712991);
\draw [c,line width=0.9] (5.796,0.712991) -- (5.796,0.746384);
\draw [c,line width=0.9] (5.752,0.712991) -- (5.796,0.712991);
\draw [c,line width=0.9] (5.796,0.712991) -- (5.84,0.712991);
\definecolor{c}{rgb}{0,0,0};
\foreach \P in {(5.796,0.712991)}{\draw[mark options={color=c,fill=c},mark size=2.402402pt,mark=*,mark size=1pt] plot coordinates {\P};}
\colorlet{c}{natgreen!80};
\draw [c,line width=0.9] (5.972,0.679598) -- (5.972,0.712991);
\draw [c,line width=0.9] (5.972,0.712991) -- (5.972,0.746384);
\draw [c,line width=0.9] (5.928,0.712991) -- (5.972,0.712991);
\draw [c,line width=0.9] (5.972,0.712991) -- (6.016,0.712991);
\definecolor{c}{rgb}{0,0,0};
\foreach \P in {(5.972,0.712991)}{\draw[mark options={color=c,fill=c},mark size=2.402402pt,mark=*,mark size=1pt] plot coordinates {\P};}
\colorlet{c}{natgreen!80};
\draw [c,line width=0.9] (6.5,0.679598) -- (6.5,0.712991);
\draw [c,line width=0.9] (6.5,0.712991) -- (6.5,0.746384);
\draw [c,line width=0.9] (6.456,0.712991) -- (6.5,0.712991);
\draw [c,line width=0.9] (6.5,0.712991) -- (6.544,0.712991);
\definecolor{c}{rgb}{0,0,0};
\foreach \P in {(6.5,0.712991)}{\draw[mark options={color=c,fill=c},mark size=2.402402pt,mark=*,mark size=1pt] plot coordinates {\P};}
\colorlet{c}{natgreen!80};
\draw [c,line width=0.9] (8.7,0.679598) -- (8.7,0.712991);
\draw [c,line width=0.9] (8.7,0.712991) -- (8.7,0.746384);
\draw [c,line width=0.9] (8.656,0.712991) -- (8.7,0.712991);
\draw [c,line width=0.9] (8.7,0.712991) -- (8.744,0.712991);
\definecolor{c}{rgb}{0,0,0};
\foreach \P in {(8.7,0.712991)}{\draw[mark options={color=c,fill=c},mark size=2.402402pt,mark=*,mark size=1pt] plot coordinates {\P};}
\colorlet{c}{natgreen!60};
\draw [c,line width=0.9] (1.308,0.679598) -- (1.308,0.694327);
\draw [c,line width=0.9] (1.308,0.694327) -- (1.308,0.709057);
\draw [c,line width=0.9] (1.264,0.694327) -- (1.308,0.694327);
\draw [c,line width=0.9] (1.308,0.694327) -- (1.352,0.694327);
\definecolor{c}{rgb}{0,0,0};
\foreach \P in {(1.308,0.694327)}{\draw[mark options={color=c,fill=c},mark size=2.402402pt,mark=*,mark size=1pt] plot coordinates {\P};}
\colorlet{c}{natgreen!60};
\draw [c,line width=0.9] (1.396,0.720309) -- (1.396,0.753246);
\draw [c,line width=0.9] (1.396,0.753246) -- (1.396,0.786182);
\draw [c,line width=0.9] (1.352,0.753246) -- (1.396,0.753246);
\draw [c,line width=0.9] (1.396,0.753246) -- (1.44,0.753246);
\definecolor{c}{rgb}{0,0,0};
\foreach \P in {(1.396,0.753246)}{\draw[mark options={color=c,fill=c},mark size=2.402402pt,mark=*,mark size=1pt] plot coordinates {\P};}
\colorlet{c}{natgreen!60};
\draw [c,line width=0.9] (1.484,0.679598) -- (1.484,0.694327);
\draw [c,line width=0.9] (1.484,0.694327) -- (1.484,0.709057);
\draw [c,line width=0.9] (1.44,0.694327) -- (1.484,0.694327);
\draw [c,line width=0.9] (1.484,0.694327) -- (1.528,0.694327);
\definecolor{c}{rgb}{0,0,0};
\foreach \P in {(1.484,0.694327)}{\draw[mark options={color=c,fill=c},mark size=2.402402pt,mark=*,mark size=1pt] plot coordinates {\P};}
\colorlet{c}{natgreen!60};
\draw [c,line width=0.9] (1.572,0.688226) -- (1.572,0.709057);
\draw [c,line width=0.9] (1.572,0.709057) -- (1.572,0.729888);
\draw [c,line width=0.9] (1.528,0.709057) -- (1.572,0.709057);
\draw [c,line width=0.9] (1.572,0.709057) -- (1.616,0.709057);
\definecolor{c}{rgb}{0,0,0};
\foreach \P in {(1.572,0.709057)}{\draw[mark options={color=c,fill=c},mark size=2.402402pt,mark=*,mark size=1pt] plot coordinates {\P};}
\colorlet{c}{natgreen!60};
\draw [c,line width=0.9] (1.66,0.679598) -- (1.66,0.694327);
\draw [c,line width=0.9] (1.66,0.694327) -- (1.66,0.709057);
\draw [c,line width=0.9] (1.616,0.694327) -- (1.66,0.694327);
\draw [c,line width=0.9] (1.66,0.694327) -- (1.704,0.694327);
\definecolor{c}{rgb}{0,0,0};
\foreach \P in {(1.66,0.694327)}{\draw[mark options={color=c,fill=c},mark size=2.402402pt,mark=*,mark size=1pt] plot coordinates {\P};}
\colorlet{c}{natgreen!60};
\draw [c,line width=0.9] (1.748,0.688226) -- (1.748,0.709057);
\draw [c,line width=0.9] (1.748,0.709057) -- (1.748,0.729888);
\draw [c,line width=0.9] (1.704,0.709057) -- (1.748,0.709057);
\draw [c,line width=0.9] (1.748,0.709057) -- (1.792,0.709057);
\definecolor{c}{rgb}{0,0,0};
\foreach \P in {(1.748,0.709057)}{\draw[mark options={color=c,fill=c},mark size=2.402402pt,mark=*,mark size=1pt] plot coordinates {\P};}
\colorlet{c}{natgreen!60};
\draw [c,line width=0.9] (1.836,0.679598) -- (1.836,0.694327);
\draw [c,line width=0.9] (1.836,0.694327) -- (1.836,0.709057);
\draw [c,line width=0.9] (1.792,0.694327) -- (1.836,0.694327);
\draw [c,line width=0.9] (1.836,0.694327) -- (1.88,0.694327);
\definecolor{c}{rgb}{0,0,0};
\foreach \P in {(1.836,0.694327)}{\draw[mark options={color=c,fill=c},mark size=2.402402pt,mark=*,mark size=1pt] plot coordinates {\P};}
\colorlet{c}{natgreen!60};
\draw [c,line width=0.9] (2.012,0.720309) -- (2.012,0.753246);
\draw [c,line width=0.9] (2.012,0.753246) -- (2.012,0.786182);
\draw [c,line width=0.9] (1.968,0.753246) -- (2.012,0.753246);
\draw [c,line width=0.9] (2.012,0.753246) -- (2.056,0.753246);
\definecolor{c}{rgb}{0,0,0};
\foreach \P in {(2.012,0.753246)}{\draw[mark options={color=c,fill=c},mark size=2.402402pt,mark=*,mark size=1pt] plot coordinates {\P};}
\colorlet{c}{natgreen!60};
\draw [c,line width=0.9] (2.1,0.709057) -- (2.1,0.738516);
\draw [c,line width=0.9] (2.1,0.738516) -- (2.1,0.767975);
\draw [c,line width=0.9] (2.056,0.738516) -- (2.1,0.738516);
\draw [c,line width=0.9] (2.1,0.738516) -- (2.144,0.738516);
\definecolor{c}{rgb}{0,0,0};
\foreach \P in {(2.1,0.738516)}{\draw[mark options={color=c,fill=c},mark size=2.402402pt,mark=*,mark size=1pt] plot coordinates {\P};}
\colorlet{c}{natgreen!60};
\draw [c,line width=0.9] (2.188,0.780315) -- (2.188,0.826894);
\draw [c,line width=0.9] (2.188,0.826894) -- (2.188,0.873473);
\draw [c,line width=0.9] (2.144,0.826894) -- (2.188,0.826894);
\draw [c,line width=0.9] (2.188,0.826894) -- (2.232,0.826894);
\definecolor{c}{rgb}{0,0,0};
\foreach \P in {(2.188,0.826894)}{\draw[mark options={color=c,fill=c},mark size=2.402402pt,mark=*,mark size=1pt] plot coordinates {\P};}
\colorlet{c}{natgreen!60};
\draw [c,line width=0.9] (2.276,0.960948) -- (2.276,1.03311);
\draw [c,line width=0.9] (2.276,1.03311) -- (2.276,1.10527);
\draw [c,line width=0.9] (2.232,1.03311) -- (2.276,1.03311);
\draw [c,line width=0.9] (2.276,1.03311) -- (2.32,1.03311);
\definecolor{c}{rgb}{0,0,0};
\foreach \P in {(2.276,1.03311)}{\draw[mark options={color=c,fill=c},mark size=2.402402pt,mark=*,mark size=1pt] plot coordinates {\P};}
\colorlet{c}{natgreen!60};
\draw [c,line width=0.9] (2.364,0.869269) -- (2.364,0.930001);
\draw [c,line width=0.9] (2.364,0.930001) -- (2.364,0.990733);
\draw [c,line width=0.9] (2.32,0.930001) -- (2.364,0.930001);
\draw [c,line width=0.9] (2.364,0.930001) -- (2.408,0.930001);
\definecolor{c}{rgb}{0,0,0};
\foreach \P in {(2.364,0.930001)}{\draw[mark options={color=c,fill=c},mark size=2.402402pt,mark=*,mark size=1pt] plot coordinates {\P};}
\colorlet{c}{natgreen!60};
\draw [c,line width=0.9] (2.452,0.843494) -- (2.452,0.900542);
\draw [c,line width=0.9] (2.452,0.900542) -- (2.452,0.957589);
\draw [c,line width=0.9] (2.408,0.900542) -- (2.452,0.900542);
\draw [c,line width=0.9] (2.452,0.900542) -- (2.496,0.900542);
\definecolor{c}{rgb}{0,0,0};
\foreach \P in {(2.452,0.900542)}{\draw[mark options={color=c,fill=c},mark size=2.402402pt,mark=*,mark size=1pt] plot coordinates {\P};}
\colorlet{c}{natgreen!60};
\draw [c,line width=0.9] (2.54,0.792771) -- (2.54,0.841623);
\draw [c,line width=0.9] (2.54,0.841623) -- (2.54,0.890476);
\draw [c,line width=0.9] (2.496,0.841623) -- (2.54,0.841623);
\draw [c,line width=0.9] (2.54,0.841623) -- (2.584,0.841623);
\definecolor{c}{rgb}{0,0,0};
\foreach \P in {(2.54,0.841623)}{\draw[mark options={color=c,fill=c},mark size=2.402402pt,mark=*,mark size=1pt] plot coordinates {\P};}
\colorlet{c}{natgreen!60};
\draw [c,line width=0.9] (2.628,0.792771) -- (2.628,0.841623);
\draw [c,line width=0.9] (2.628,0.841623) -- (2.628,0.890476);
\draw [c,line width=0.9] (2.584,0.841623) -- (2.628,0.841623);
\draw [c,line width=0.9] (2.628,0.841623) -- (2.672,0.841623);
\definecolor{c}{rgb}{0,0,0};
\foreach \P in {(2.628,0.841623)}{\draw[mark options={color=c,fill=c},mark size=2.402402pt,mark=*,mark size=1pt] plot coordinates {\P};}
\colorlet{c}{natgreen!60};
\draw [c,line width=0.9] (2.716,0.869269) -- (2.716,0.930001);
\draw [c,line width=0.9] (2.716,0.930001) -- (2.716,0.990733);
\draw [c,line width=0.9] (2.672,0.930001) -- (2.716,0.930001);
\draw [c,line width=0.9] (2.716,0.930001) -- (2.76,0.930001);
\definecolor{c}{rgb}{0,0,0};
\foreach \P in {(2.716,0.930001)}{\draw[mark options={color=c,fill=c},mark size=2.402402pt,mark=*,mark size=1pt] plot coordinates {\P};}
\colorlet{c}{natgreen!60};
\draw [c,line width=0.9] (2.804,0.755773) -- (2.804,0.797435);
\draw [c,line width=0.9] (2.804,0.797435) -- (2.804,0.839096);
\draw [c,line width=0.9] (2.76,0.797435) -- (2.804,0.797435);
\draw [c,line width=0.9] (2.804,0.797435) -- (2.848,0.797435);
\definecolor{c}{rgb}{0,0,0};
\foreach \P in {(2.804,0.797435)}{\draw[mark options={color=c,fill=c},mark size=2.402402pt,mark=*,mark size=1pt] plot coordinates {\P};}
\colorlet{c}{natgreen!60};
\draw [c,line width=0.9] (2.892,0.743734) -- (2.892,0.782705);
\draw [c,line width=0.9] (2.892,0.782705) -- (2.892,0.821676);
\draw [c,line width=0.9] (2.848,0.782705) -- (2.892,0.782705);
\draw [c,line width=0.9] (2.892,0.782705) -- (2.936,0.782705);
\definecolor{c}{rgb}{0,0,0};
\foreach \P in {(2.892,0.782705)}{\draw[mark options={color=c,fill=c},mark size=2.402402pt,mark=*,mark size=1pt] plot coordinates {\P};}
\colorlet{c}{natgreen!60};
\draw [c,line width=0.9] (2.98,0.731895) -- (2.98,0.767975);
\draw [c,line width=0.9] (2.98,0.767975) -- (2.98,0.804055);
\draw [c,line width=0.9] (2.936,0.767975) -- (2.98,0.767975);
\draw [c,line width=0.9] (2.98,0.767975) -- (3.024,0.767975);
\definecolor{c}{rgb}{0,0,0};
\foreach \P in {(2.98,0.767975)}{\draw[mark options={color=c,fill=c},mark size=2.402402pt,mark=*,mark size=1pt] plot coordinates {\P};}
\colorlet{c}{natgreen!60};
\draw [c,line width=0.9] (3.068,0.720309) -- (3.068,0.753246);
\draw [c,line width=0.9] (3.068,0.753246) -- (3.068,0.786182);
\draw [c,line width=0.9] (3.024,0.753246) -- (3.068,0.753246);
\draw [c,line width=0.9] (3.068,0.753246) -- (3.112,0.753246);
\definecolor{c}{rgb}{0,0,0};
\foreach \P in {(3.068,0.753246)}{\draw[mark options={color=c,fill=c},mark size=2.402402pt,mark=*,mark size=1pt] plot coordinates {\P};}
\colorlet{c}{natgreen!60};
\draw [c,line width=0.9] (3.156,0.720309) -- (3.156,0.753246);
\draw [c,line width=0.9] (3.156,0.753246) -- (3.156,0.786182);
\draw [c,line width=0.9] (3.112,0.753246) -- (3.156,0.753246);
\draw [c,line width=0.9] (3.156,0.753246) -- (3.2,0.753246);
\definecolor{c}{rgb}{0,0,0};
\foreach \P in {(3.156,0.753246)}{\draw[mark options={color=c,fill=c},mark size=2.402402pt,mark=*,mark size=1pt] plot coordinates {\P};}
\colorlet{c}{natgreen!60};
\draw [c,line width=0.9] (3.244,0.709057) -- (3.244,0.738516);
\draw [c,line width=0.9] (3.244,0.738516) -- (3.244,0.767975);
\draw [c,line width=0.9] (3.2,0.738516) -- (3.244,0.738516);
\draw [c,line width=0.9] (3.244,0.738516) -- (3.288,0.738516);
\definecolor{c}{rgb}{0,0,0};
\foreach \P in {(3.244,0.738516)}{\draw[mark options={color=c,fill=c},mark size=2.402402pt,mark=*,mark size=1pt] plot coordinates {\P};}
\colorlet{c}{natgreen!60};
\draw [c,line width=0.9] (3.332,0.731895) -- (3.332,0.767975);
\draw [c,line width=0.9] (3.332,0.767975) -- (3.332,0.804055);
\draw [c,line width=0.9] (3.288,0.767975) -- (3.332,0.767975);
\draw [c,line width=0.9] (3.332,0.767975) -- (3.376,0.767975);
\definecolor{c}{rgb}{0,0,0};
\foreach \P in {(3.332,0.767975)}{\draw[mark options={color=c,fill=c},mark size=2.402402pt,mark=*,mark size=1pt] plot coordinates {\P};}
\colorlet{c}{natgreen!60};
\draw [c,line width=0.9] (3.42,0.709057) -- (3.42,0.738516);
\draw [c,line width=0.9] (3.42,0.738516) -- (3.42,0.767975);
\draw [c,line width=0.9] (3.376,0.738516) -- (3.42,0.738516);
\draw [c,line width=0.9] (3.42,0.738516) -- (3.464,0.738516);
\definecolor{c}{rgb}{0,0,0};
\foreach \P in {(3.42,0.738516)}{\draw[mark options={color=c,fill=c},mark size=2.402402pt,mark=*,mark size=1pt] plot coordinates {\P};}
\colorlet{c}{natgreen!60};
\draw [c,line width=0.9] (3.508,0.688226) -- (3.508,0.709057);
\draw [c,line width=0.9] (3.508,0.709057) -- (3.508,0.729888);
\draw [c,line width=0.9] (3.464,0.709057) -- (3.508,0.709057);
\draw [c,line width=0.9] (3.508,0.709057) -- (3.552,0.709057);
\definecolor{c}{rgb}{0,0,0};
\foreach \P in {(3.508,0.709057)}{\draw[mark options={color=c,fill=c},mark size=2.402402pt,mark=*,mark size=1pt] plot coordinates {\P};}
\colorlet{c}{natgreen!60};
\draw [c,line width=0.9] (3.684,0.720309) -- (3.684,0.753246);
\draw [c,line width=0.9] (3.684,0.753246) -- (3.684,0.786182);
\draw [c,line width=0.9] (3.64,0.753246) -- (3.684,0.753246);
\draw [c,line width=0.9] (3.684,0.753246) -- (3.728,0.753246);
\definecolor{c}{rgb}{0,0,0};
\foreach \P in {(3.684,0.753246)}{\draw[mark options={color=c,fill=c},mark size=2.402402pt,mark=*,mark size=1pt] plot coordinates {\P};}
\colorlet{c}{natgreen!60};
\draw [c,line width=0.9] (3.772,0.679598) -- (3.772,0.694327);
\draw [c,line width=0.9] (3.772,0.694327) -- (3.772,0.709057);
\draw [c,line width=0.9] (3.728,0.694327) -- (3.772,0.694327);
\draw [c,line width=0.9] (3.772,0.694327) -- (3.816,0.694327);
\definecolor{c}{rgb}{0,0,0};
\foreach \P in {(3.772,0.694327)}{\draw[mark options={color=c,fill=c},mark size=2.402402pt,mark=*,mark size=1pt] plot coordinates {\P};}
\colorlet{c}{natgreen!60};
\draw [c,line width=0.9] (3.948,0.688226) -- (3.948,0.709057);
\draw [c,line width=0.9] (3.948,0.709057) -- (3.948,0.729888);
\draw [c,line width=0.9] (3.904,0.709057) -- (3.948,0.709057);
\draw [c,line width=0.9] (3.948,0.709057) -- (3.992,0.709057);
\definecolor{c}{rgb}{0,0,0};
\foreach \P in {(3.948,0.709057)}{\draw[mark options={color=c,fill=c},mark size=2.402402pt,mark=*,mark size=1pt] plot coordinates {\P};}
\colorlet{c}{natgreen!60};
\draw [c,line width=0.9] (4.036,0.679598) -- (4.036,0.694327);
\draw [c,line width=0.9] (4.036,0.694327) -- (4.036,0.709057);
\draw [c,line width=0.9] (3.992,0.694327) -- (4.036,0.694327);
\draw [c,line width=0.9] (4.036,0.694327) -- (4.08,0.694327);
\definecolor{c}{rgb}{0,0,0};
\foreach \P in {(4.036,0.694327)}{\draw[mark options={color=c,fill=c},mark size=2.402402pt,mark=*,mark size=1pt] plot coordinates {\P};}
\colorlet{c}{natgreen!60};
\draw [c,line width=0.9] (4.124,0.679598) -- (4.124,0.694327);
\draw [c,line width=0.9] (4.124,0.694327) -- (4.124,0.709057);
\draw [c,line width=0.9] (4.08,0.694327) -- (4.124,0.694327);
\draw [c,line width=0.9] (4.124,0.694327) -- (4.168,0.694327);
\definecolor{c}{rgb}{0,0,0};
\foreach \P in {(4.124,0.694327)}{\draw[mark options={color=c,fill=c},mark size=2.402402pt,mark=*,mark size=1pt] plot coordinates {\P};}
\colorlet{c}{natgreen!60};
\draw [c,line width=0.9] (4.212,0.679598) -- (4.212,0.694327);
\draw [c,line width=0.9] (4.212,0.694327) -- (4.212,0.709057);
\draw [c,line width=0.9] (4.168,0.694327) -- (4.212,0.694327);
\draw [c,line width=0.9] (4.212,0.694327) -- (4.256,0.694327);
\definecolor{c}{rgb}{0,0,0};
\foreach \P in {(4.212,0.694327)}{\draw[mark options={color=c,fill=c},mark size=2.402402pt,mark=*,mark size=1pt] plot coordinates {\P};}
\colorlet{c}{natgreen!60};
\draw [c,line width=0.9] (4.3,0.688226) -- (4.3,0.709057);
\draw [c,line width=0.9] (4.3,0.709057) -- (4.3,0.729888);
\draw [c,line width=0.9] (4.256,0.709057) -- (4.3,0.709057);
\draw [c,line width=0.9] (4.3,0.709057) -- (4.344,0.709057);
\definecolor{c}{rgb}{0,0,0};
\foreach \P in {(4.3,0.709057)}{\draw[mark options={color=c,fill=c},mark size=2.402402pt,mark=*,mark size=1pt] plot coordinates {\P};}
\colorlet{c}{natgreen!60};
\draw [c,line width=0.9] (4.564,0.679598) -- (4.564,0.694327);
\draw [c,line width=0.9] (4.564,0.694327) -- (4.564,0.709057);
\draw [c,line width=0.9] (4.52,0.694327) -- (4.564,0.694327);
\draw [c,line width=0.9] (4.564,0.694327) -- (4.608,0.694327);
\definecolor{c}{rgb}{0,0,0};
\foreach \P in {(4.564,0.694327)}{\draw[mark options={color=c,fill=c},mark size=2.402402pt,mark=*,mark size=1pt] plot coordinates {\P};}
\colorlet{c}{natgreen!60};
\draw [c,line width=0.9] (4.828,0.698274) -- (4.828,0.723787);
\draw [c,line width=0.9] (4.828,0.723787) -- (4.828,0.749299);
\draw [c,line width=0.9] (4.784,0.723787) -- (4.828,0.723787);
\draw [c,line width=0.9] (4.828,0.723787) -- (4.872,0.723787);
\definecolor{c}{rgb}{0,0,0};
\foreach \P in {(4.828,0.723787)}{\draw[mark options={color=c,fill=c},mark size=2.402402pt,mark=*,mark size=1pt] plot coordinates {\P};}
\colorlet{c}{natgreen!60};
\draw [c,line width=0.9] (4.916,0.688226) -- (4.916,0.709057);
\draw [c,line width=0.9] (4.916,0.709057) -- (4.916,0.729888);
\draw [c,line width=0.9] (4.872,0.709057) -- (4.916,0.709057);
\draw [c,line width=0.9] (4.916,0.709057) -- (4.96,0.709057);
\definecolor{c}{rgb}{0,0,0};
\foreach \P in {(4.916,0.709057)}{\draw[mark options={color=c,fill=c},mark size=2.402402pt,mark=*,mark size=1pt] plot coordinates {\P};}
\colorlet{c}{natgreen!60};
\draw [c,line width=0.9] (5.004,0.679598) -- (5.004,0.694327);
\draw [c,line width=0.9] (5.004,0.694327) -- (5.004,0.709057);
\draw [c,line width=0.9] (4.96,0.694327) -- (5.004,0.694327);
\draw [c,line width=0.9] (5.004,0.694327) -- (5.048,0.694327);
\definecolor{c}{rgb}{0,0,0};
\foreach \P in {(5.004,0.694327)}{\draw[mark options={color=c,fill=c},mark size=2.402402pt,mark=*,mark size=1pt] plot coordinates {\P};}
\colorlet{c}{natgreen!60};
\draw [c,line width=0.9] (5.356,0.688226) -- (5.356,0.709057);
\draw [c,line width=0.9] (5.356,0.709057) -- (5.356,0.729888);
\draw [c,line width=0.9] (5.312,0.709057) -- (5.356,0.709057);
\draw [c,line width=0.9] (5.356,0.709057) -- (5.4,0.709057);
\definecolor{c}{rgb}{0,0,0};
\foreach \P in {(5.356,0.709057)}{\draw[mark options={color=c,fill=c},mark size=2.402402pt,mark=*,mark size=1pt] plot coordinates {\P};}
\colorlet{c}{natgreen!60};
\draw [c,line width=0.9] (5.884,0.679598) -- (5.884,0.694327);
\draw [c,line width=0.9] (5.884,0.694327) -- (5.884,0.709057);
\draw [c,line width=0.9] (5.84,0.694327) -- (5.884,0.694327);
\draw [c,line width=0.9] (5.884,0.694327) -- (5.928,0.694327);
\definecolor{c}{rgb}{0,0,0};
\foreach \P in {(5.884,0.694327)}{\draw[mark options={color=c,fill=c},mark size=2.402402pt,mark=*,mark size=1pt] plot coordinates {\P};}
\colorlet{c}{natgreen!60};
\draw [c,line width=0.9] (6.5,0.679598) -- (6.5,0.694327);
\draw [c,line width=0.9] (6.5,0.694327) -- (6.5,0.709057);
\draw [c,line width=0.9] (6.456,0.694327) -- (6.5,0.694327);
\draw [c,line width=0.9] (6.5,0.694327) -- (6.544,0.694327);
\definecolor{c}{rgb}{0,0,0};
\foreach \P in {(6.5,0.694327)}{\draw[mark options={color=c,fill=c},mark size=2.402402pt,mark=*,mark size=1pt] plot coordinates {\P};}
\colorlet{c}{natgreen!60};
\draw [c,line width=0.9] (7.908,0.679598) -- (7.908,0.694327);
\draw [c,line width=0.9] (7.908,0.694327) -- (7.908,0.709057);
\draw [c,line width=0.9] (7.864,0.694327) -- (7.908,0.694327);
\draw [c,line width=0.9] (7.908,0.694327) -- (7.952,0.694327);
\definecolor{c}{rgb}{0,0,0};
\foreach \P in {(7.908,0.694327)}{\draw[mark options={color=c,fill=c},mark size=2.402402pt,mark=*,mark size=1pt] plot coordinates {\P};}
\colorlet{c}{natgreen!60};
\draw [c,line width=0.9] (9.228,0.679598) -- (9.228,0.694327);
\draw [c,line width=0.9] (9.228,0.694327) -- (9.228,0.709057);
\draw [c,line width=0.9] (9.184,0.694327) -- (9.228,0.694327);
\draw [c,line width=0.9] (9.228,0.694327) -- (9.272,0.694327);
\definecolor{c}{rgb}{0,0,0};
\foreach \P in {(9.228,0.694327)}{\draw[mark options={color=c,fill=c},mark size=2.402402pt,mark=*,mark size=1pt] plot coordinates {\P};}
\colorlet{c}{natgreen!40};
\draw [c,line width=0.9] (1.308,0.681821) -- (1.308,0.687189);
\draw [c,line width=0.9] (1.308,0.687189) -- (1.308,0.692557);
\draw [c,line width=0.9] (1.264,0.687189) -- (1.308,0.687189);
\draw [c,line width=0.9] (1.308,0.687189) -- (1.352,0.687189);
\definecolor{c}{rgb}{0,0,0};
\foreach \P in {(1.308,0.687189)}{\draw[mark options={color=c,fill=c},mark size=2.402402pt,mark=*,mark size=1pt] plot coordinates {\P};}
\colorlet{c}{natgreen!40};
\draw [c,line width=0.9] (1.66,0.687189) -- (1.66,0.694781);
\draw [c,line width=0.9] (1.66,0.694781) -- (1.66,0.702372);
\draw [c,line width=0.9] (1.616,0.694781) -- (1.66,0.694781);
\draw [c,line width=0.9] (1.66,0.694781) -- (1.704,0.694781);
\definecolor{c}{rgb}{0,0,0};
\foreach \P in {(1.66,0.694781)}{\draw[mark options={color=c,fill=c},mark size=2.402402pt,mark=*,mark size=1pt] plot coordinates {\P};}
\colorlet{c}{natgreen!40};
\draw [c,line width=0.9] (1.748,0.735171) -- (1.748,0.751717);
\draw [c,line width=0.9] (1.748,0.751717) -- (1.748,0.768262);
\draw [c,line width=0.9] (1.704,0.751717) -- (1.748,0.751717);
\draw [c,line width=0.9] (1.748,0.751717) -- (1.792,0.751717);
\definecolor{c}{rgb}{0,0,0};
\foreach \P in {(1.748,0.751717)}{\draw[mark options={color=c,fill=c},mark size=2.402402pt,mark=*,mark size=1pt] plot coordinates {\P};}
\colorlet{c}{natgreen!40};
\draw [c,line width=0.9] (1.836,0.708762) -- (1.836,0.721351);
\draw [c,line width=0.9] (1.836,0.721351) -- (1.836,0.73394);
\draw [c,line width=0.9] (1.792,0.721351) -- (1.836,0.721351);
\draw [c,line width=0.9] (1.836,0.721351) -- (1.88,0.721351);
\definecolor{c}{rgb}{0,0,0};
\foreach \P in {(1.836,0.721351)}{\draw[mark options={color=c,fill=c},mark size=2.402402pt,mark=*,mark size=1pt] plot coordinates {\P};}
\colorlet{c}{natgreen!40};
\draw [c,line width=0.9] (1.924,0.690089) -- (1.924,0.698576);
\draw [c,line width=0.9] (1.924,0.698576) -- (1.924,0.707064);
\draw [c,line width=0.9] (1.88,0.698576) -- (1.924,0.698576);
\draw [c,line width=0.9] (1.924,0.698576) -- (1.968,0.698576);
\definecolor{c}{rgb}{0,0,0};
\foreach \P in {(1.924,0.698576)}{\draw[mark options={color=c,fill=c},mark size=2.402402pt,mark=*,mark size=1pt] plot coordinates {\P};}
\colorlet{c}{natgreen!40};
\draw [c,line width=0.9] (2.012,0.708762) -- (2.012,0.721351);
\draw [c,line width=0.9] (2.012,0.721351) -- (2.012,0.73394);
\draw [c,line width=0.9] (1.968,0.721351) -- (2.012,0.721351);
\draw [c,line width=0.9] (2.012,0.721351) -- (2.056,0.721351);
\definecolor{c}{rgb}{0,0,0};
\foreach \P in {(2.012,0.721351)}{\draw[mark options={color=c,fill=c},mark size=2.402402pt,mark=*,mark size=1pt] plot coordinates {\P};}
\colorlet{c}{natgreen!40};
\draw [c,line width=0.9] (2.1,0.705552) -- (2.1,0.717555);
\draw [c,line width=0.9] (2.1,0.717555) -- (2.1,0.729558);
\draw [c,line width=0.9] (2.056,0.717555) -- (2.1,0.717555);
\draw [c,line width=0.9] (2.1,0.717555) -- (2.144,0.717555);
\definecolor{c}{rgb}{0,0,0};
\foreach \P in {(2.1,0.717555)}{\draw[mark options={color=c,fill=c},mark size=2.402402pt,mark=*,mark size=1pt] plot coordinates {\P};}
\colorlet{c}{natgreen!40};
\draw [c,line width=0.9] (2.188,0.693075) -- (2.188,0.702372);
\draw [c,line width=0.9] (2.188,0.702372) -- (2.188,0.71167);
\draw [c,line width=0.9] (2.144,0.702372) -- (2.188,0.702372);
\draw [c,line width=0.9] (2.188,0.702372) -- (2.232,0.702372);
\definecolor{c}{rgb}{0,0,0};
\foreach \P in {(2.188,0.702372)}{\draw[mark options={color=c,fill=c},mark size=2.402402pt,mark=*,mark size=1pt] plot coordinates {\P};}
\colorlet{c}{natgreen!40};
\draw [c,line width=0.9] (2.276,0.690089) -- (2.276,0.698576);
\draw [c,line width=0.9] (2.276,0.698576) -- (2.276,0.707064);
\draw [c,line width=0.9] (2.232,0.698576) -- (2.276,0.698576);
\draw [c,line width=0.9] (2.276,0.698576) -- (2.32,0.698576);
\definecolor{c}{rgb}{0,0,0};
\foreach \P in {(2.276,0.698576)}{\draw[mark options={color=c,fill=c},mark size=2.402402pt,mark=*,mark size=1pt] plot coordinates {\P};}
\colorlet{c}{natgreen!40};
\draw [c,line width=0.9] (2.364,0.687189) -- (2.364,0.694781);
\draw [c,line width=0.9] (2.364,0.694781) -- (2.364,0.702372);
\draw [c,line width=0.9] (2.32,0.694781) -- (2.364,0.694781);
\draw [c,line width=0.9] (2.364,0.694781) -- (2.408,0.694781);
\definecolor{c}{rgb}{0,0,0};
\foreach \P in {(2.364,0.694781)}{\draw[mark options={color=c,fill=c},mark size=2.402402pt,mark=*,mark size=1pt] plot coordinates {\P};}
\colorlet{c}{natgreen!40};
\draw [c,line width=0.9] (2.452,0.693075) -- (2.452,0.702372);
\draw [c,line width=0.9] (2.452,0.702372) -- (2.452,0.71167);
\draw [c,line width=0.9] (2.408,0.702372) -- (2.452,0.702372);
\draw [c,line width=0.9] (2.452,0.702372) -- (2.496,0.702372);
\definecolor{c}{rgb}{0,0,0};
\foreach \P in {(2.452,0.702372)}{\draw[mark options={color=c,fill=c},mark size=2.402402pt,mark=*,mark size=1pt] plot coordinates {\P};}
\colorlet{c}{natgreen!40};
\draw [c,line width=0.9] (2.54,0.696125) -- (2.54,0.706168);
\draw [c,line width=0.9] (2.54,0.706168) -- (2.54,0.71621);
\draw [c,line width=0.9] (2.496,0.706168) -- (2.54,0.706168);
\draw [c,line width=0.9] (2.54,0.706168) -- (2.584,0.706168);
\definecolor{c}{rgb}{0,0,0};
\foreach \P in {(2.54,0.706168)}{\draw[mark options={color=c,fill=c},mark size=2.402402pt,mark=*,mark size=1pt] plot coordinates {\P};}
\colorlet{c}{natgreen!40};
\draw [c,line width=0.9] (2.628,0.679598) -- (2.628,0.683393);
\draw [c,line width=0.9] (2.628,0.683393) -- (2.628,0.687189);
\draw [c,line width=0.9] (2.584,0.683393) -- (2.628,0.683393);
\draw [c,line width=0.9] (2.628,0.683393) -- (2.672,0.683393);
\definecolor{c}{rgb}{0,0,0};
\foreach \P in {(2.628,0.683393)}{\draw[mark options={color=c,fill=c},mark size=2.402402pt,mark=*,mark size=1pt] plot coordinates {\P};}
\colorlet{c}{natgreen!40};
\draw [c,line width=0.9] (2.716,0.679598) -- (2.716,0.683393);
\draw [c,line width=0.9] (2.716,0.683393) -- (2.716,0.687189);
\draw [c,line width=0.9] (2.672,0.683393) -- (2.716,0.683393);
\draw [c,line width=0.9] (2.716,0.683393) -- (2.76,0.683393);
\definecolor{c}{rgb}{0,0,0};
\foreach \P in {(2.716,0.683393)}{\draw[mark options={color=c,fill=c},mark size=2.402402pt,mark=*,mark size=1pt] plot coordinates {\P};}
\colorlet{c}{natgreen!40};
\draw [c,line width=0.9] (2.804,0.681821) -- (2.804,0.687189);
\draw [c,line width=0.9] (2.804,0.687189) -- (2.804,0.692557);
\draw [c,line width=0.9] (2.76,0.687189) -- (2.804,0.687189);
\draw [c,line width=0.9] (2.804,0.687189) -- (2.848,0.687189);
\definecolor{c}{rgb}{0,0,0};
\foreach \P in {(2.804,0.687189)}{\draw[mark options={color=c,fill=c},mark size=2.402402pt,mark=*,mark size=1pt] plot coordinates {\P};}
\colorlet{c}{natgreen!40};
\draw [c,line width=0.9] (2.892,0.681821) -- (2.892,0.687189);
\draw [c,line width=0.9] (2.892,0.687189) -- (2.892,0.692557);
\draw [c,line width=0.9] (2.848,0.687189) -- (2.892,0.687189);
\draw [c,line width=0.9] (2.892,0.687189) -- (2.936,0.687189);
\definecolor{c}{rgb}{0,0,0};
\foreach \P in {(2.892,0.687189)}{\draw[mark options={color=c,fill=c},mark size=2.402402pt,mark=*,mark size=1pt] plot coordinates {\P};}
\colorlet{c}{natgreen!40};
\draw [c,line width=0.9] (2.98,0.681821) -- (2.98,0.687189);
\draw [c,line width=0.9] (2.98,0.687189) -- (2.98,0.692557);
\draw [c,line width=0.9] (2.936,0.687189) -- (2.98,0.687189);
\draw [c,line width=0.9] (2.98,0.687189) -- (3.024,0.687189);
\definecolor{c}{rgb}{0,0,0};
\foreach \P in {(2.98,0.687189)}{\draw[mark options={color=c,fill=c},mark size=2.402402pt,mark=*,mark size=1pt] plot coordinates {\P};}
\colorlet{c}{natgreen!40};
\draw [c,line width=0.9] (3.684,0.679598) -- (3.684,0.683393);
\draw [c,line width=0.9] (3.684,0.683393) -- (3.684,0.687189);
\draw [c,line width=0.9] (3.64,0.683393) -- (3.684,0.683393);
\draw [c,line width=0.9] (3.684,0.683393) -- (3.728,0.683393);
\definecolor{c}{rgb}{0,0,0};
\foreach \P in {(3.684,0.683393)}{\draw[mark options={color=c,fill=c},mark size=2.402402pt,mark=*,mark size=1pt] plot coordinates {\P};}
\colorlet{c}{natgreen!40};
\draw [c,line width=0.9] (5.62,0.679598) -- (5.62,0.683393);
\draw [c,line width=0.9] (5.62,0.683393) -- (5.62,0.687189);
\draw [c,line width=0.9] (5.576,0.683393) -- (5.62,0.683393);
\draw [c,line width=0.9] (5.62,0.683393) -- (5.664,0.683393);
\definecolor{c}{rgb}{0,0,0};
\foreach \P in {(5.62,0.683393)}{\draw[mark options={color=c,fill=c},mark size=2.402402pt,mark=*,mark size=1pt] plot coordinates {\P};}
\definecolor{c}{rgb}{1,1,1};
\draw [color=c, fill=c] (5,4.75718) rectangle (9.7,6.5921);
\draw [c,line width=0.9] (5,4.75718) -- (9.7,4.75718);
\draw [c,line width=0.9] (9.7,4.75718) -- (9.7,6.5921);
\draw [c,line width=0.9] (9.7,6.5921) -- (5,6.5921);
\draw [c,line width=0.9] (5,6.5921) -- (5,4.75718);
\definecolor{c}{rgb}{0,0,0};
\draw [anchor= west] (6.175,6.40861) node[color=c, rotate=0]{Total};
\colorlet{c}{kugray};
\draw [c,line width=0.9] (5.17625,6.40861) -- (5.99875,6.40861);
\definecolor{c}{rgb}{0,0,0};
\draw [anchor= west] (6.175,6.04162) node[color=c, rotate=0]{Pythia8\_AU2CTEQ6L1\_gammagamma\_2DP20};
\colorlet{c}{natgreen};
\draw [c,line width=0.9] (5.17625,6.04162) -- (5.99875,6.04162);
\definecolor{c}{rgb}{0,0,0};
\draw [anchor= west] (6.175,5.67464) node[color=c, rotate=0]{Pythia8\_AU2CTEQ6L1\_gammajet\_DP70};
\colorlet{c}{natgreen!80};
\draw [c,line width=0.9] (5.17625,5.67464) -- (5.99875,5.67464);
\definecolor{c}{rgb}{0,0,0};
\draw [anchor= west] (6.175,5.30766) node[color=c, rotate=0]{Sherpa\_CT10\_SinglePhotonPt70};
\colorlet{c}{natgreen!60};
\draw [c,line width=0.9] (5.17625,5.30766) -- (5.99875,5.30766);
\definecolor{c}{rgb}{0,0,0};
\draw [anchor= west] (6.175,4.94068) node[color=c, rotate=0]{PowhegPythia8\_AU2CT10\_Zee};
\colorlet{c}{natgreen!40};
\draw [c,line width=0.9] (5.17625,4.94068) -- (5.99875,4.94068);
\definecolor{c}{rgb}{0,0,0};
\draw [c,line width=0.9] (1,0.679598) -- (9.8,0.679598);
\draw [c,line width=0.9] (1,0.859012) -- (1,0.679598);
\draw [c,line width=0.9] (1.176,0.769305) -- (1.176,0.679598);
\draw [c,line width=0.9] (1.352,0.769305) -- (1.352,0.679598);
\draw [c,line width=0.9] (1.528,0.769305) -- (1.528,0.679598);
\draw [c,line width=0.9] (1.704,0.769305) -- (1.704,0.679598);
\draw [c,line width=0.9] (1.88,0.859012) -- (1.88,0.679598);
\draw [c,line width=0.9] (2.056,0.769305) -- (2.056,0.679598);
\draw [c,line width=0.9] (2.232,0.769305) -- (2.232,0.679598);
\draw [c,line width=0.9] (2.408,0.769305) -- (2.408,0.679598);
\draw [c,line width=0.9] (2.584,0.769305) -- (2.584,0.679598);
\draw [c,line width=0.9] (2.76,0.859012) -- (2.76,0.679598);
\draw [c,line width=0.9] (2.936,0.769305) -- (2.936,0.679598);
\draw [c,line width=0.9] (3.112,0.769305) -- (3.112,0.679598);
\draw [c,line width=0.9] (3.288,0.769305) -- (3.288,0.679598);
\draw [c,line width=0.9] (3.464,0.769305) -- (3.464,0.679598);
\draw [c,line width=0.9] (3.64,0.859012) -- (3.64,0.679598);
\draw [c,line width=0.9] (3.816,0.769305) -- (3.816,0.679598);
\draw [c,line width=0.9] (3.992,0.769305) -- (3.992,0.679598);
\draw [c,line width=0.9] (4.168,0.769305) -- (4.168,0.679598);
\draw [c,line width=0.9] (4.344,0.769305) -- (4.344,0.679598);
\draw [c,line width=0.9] (4.52,0.859012) -- (4.52,0.679598);
\draw [c,line width=0.9] (4.696,0.769305) -- (4.696,0.679598);
\draw [c,line width=0.9] (4.872,0.769305) -- (4.872,0.679598);
\draw [c,line width=0.9] (5.048,0.769305) -- (5.048,0.679598);
\draw [c,line width=0.9] (5.224,0.769305) -- (5.224,0.679598);
\draw [c,line width=0.9] (5.4,0.859012) -- (5.4,0.679598);
\draw [c,line width=0.9] (5.576,0.769305) -- (5.576,0.679598);
\draw [c,line width=0.9] (5.752,0.769305) -- (5.752,0.679598);
\draw [c,line width=0.9] (5.928,0.769305) -- (5.928,0.679598);
\draw [c,line width=0.9] (6.104,0.769305) -- (6.104,0.679598);
\draw [c,line width=0.9] (6.28,0.859012) -- (6.28,0.679598);
\draw [c,line width=0.9] (6.456,0.769305) -- (6.456,0.679598);
\draw [c,line width=0.9] (6.632,0.769305) -- (6.632,0.679598);
\draw [c,line width=0.9] (6.808,0.769305) -- (6.808,0.679598);
\draw [c,line width=0.9] (6.984,0.769305) -- (6.984,0.679598);
\draw [c,line width=0.9] (7.16,0.859012) -- (7.16,0.679598);
\draw [c,line width=0.9] (7.336,0.769305) -- (7.336,0.679598);
\draw [c,line width=0.9] (7.512,0.769305) -- (7.512,0.679598);
\draw [c,line width=0.9] (7.688,0.769305) -- (7.688,0.679598);
\draw [c,line width=0.9] (7.864,0.769305) -- (7.864,0.679598);
\draw [c,line width=0.9] (8.04,0.859012) -- (8.04,0.679598);
\draw [c,line width=0.9] (8.216,0.769305) -- (8.216,0.679598);
\draw [c,line width=0.9] (8.392,0.769305) -- (8.392,0.679598);
\draw [c,line width=0.9] (8.568,0.769305) -- (8.568,0.679598);
\draw [c,line width=0.9] (8.744,0.769305) -- (8.744,0.679598);
\draw [c,line width=0.9] (8.92,0.859012) -- (8.92,0.679598);
\draw [c,line width=0.9] (9.096,0.769305) -- (9.096,0.679598);
\draw [c,line width=0.9] (9.272,0.769305) -- (9.272,0.679598);
\draw [c,line width=0.9] (9.448,0.769305) -- (9.448,0.679598);
\draw [c,line width=0.9] (9.624,0.769305) -- (9.624,0.679598);
\draw [c,line width=0.9] (9.8,0.859012) -- (9.8,0.679598);
\draw [c,line width=0.9] (1,0.679598) -- (1,6.66006);
\draw [c,line width=0.9] (1.264,0.679598) -- (1,0.679598);
\draw [c,line width=0.9] (1.132,0.866269) -- (1,0.866269);
\draw [c,line width=0.9] (1.132,1.05294) -- (1,1.05294);
\draw [c,line width=0.9] (1.132,1.23961) -- (1,1.23961);
\draw [c,line width=0.9] (1.132,1.42628) -- (1,1.42628);
\draw [c,line width=0.9] (1.264,1.61295) -- (1,1.61295);
\draw [c,line width=0.9] (1.132,1.79963) -- (1,1.79963);
\draw [c,line width=0.9] (1.132,1.9863) -- (1,1.9863);
\draw [c,line width=0.9] (1.132,2.17297) -- (1,2.17297);
\draw [c,line width=0.9] (1.132,2.35964) -- (1,2.35964);
\draw [c,line width=0.9] (1.264,2.54631) -- (1,2.54631);
\draw [c,line width=0.9] (1.132,2.73298) -- (1,2.73298);
\draw [c,line width=0.9] (1.132,2.91965) -- (1,2.91965);
\draw [c,line width=0.9] (1.132,3.10632) -- (1,3.10632);
\draw [c,line width=0.9] (1.132,3.293) -- (1,3.293);
\draw [c,line width=0.9] (1.264,3.47967) -- (1,3.47967);
\draw [c,line width=0.9] (1.132,3.66634) -- (1,3.66634);
\draw [c,line width=0.9] (1.132,3.85301) -- (1,3.85301);
\draw [c,line width=0.9] (1.132,4.03968) -- (1,4.03968);
\draw [c,line width=0.9] (1.132,4.22635) -- (1,4.22635);
\draw [c,line width=0.9] (1.264,4.41302) -- (1,4.41302);
\draw [c,line width=0.9] (1.132,4.5997) -- (1,4.5997);
\draw [c,line width=0.9] (1.132,4.78637) -- (1,4.78637);
\draw [c,line width=0.9] (1.132,4.97304) -- (1,4.97304);
\draw [c,line width=0.9] (1.132,5.15971) -- (1,5.15971);
\draw [c,line width=0.9] (1.264,5.34638) -- (1,5.34638);
\draw [c,line width=0.9] (1.132,5.53305) -- (1,5.53305);
\draw [c,line width=0.9] (1.132,5.71972) -- (1,5.71972);
\draw [c,line width=0.9] (1.132,5.90639) -- (1,5.90639);
\draw [c,line width=0.9] (1.132,6.09307) -- (1,6.09307);
\draw [c,line width=0.9] (1.264,6.27974) -- (1,6.27974);
\draw [c,line width=0.9] (1.264,6.27974) -- (1,6.27974);
\draw [c,line width=0.9] (1.132,6.46641) -- (1,6.46641);
\draw [c,line width=0.9] (1.132,6.65308) -- (1,6.65308);
\end{tikzpicture}

\end{infilsf}
\end{minipage}
\begin{minipage}[b]{.3\textwidth}
\caption{The invariant mass distribution of diphoton events from the \atlas{} Monte Carlo event sets that pass all selection criteria, along with the contribution from each included Monte Carlo set. All distributions have been normalised to the luminosity of the data sample.}
\label{mclist}
\end{minipage}
\end{figure}


a set of monte carlo events that simulate the diphoton events that the standard model predicts will be produced in \atlas{}. These samples are chosen from a larger selection of MC samples as the only ones that provide a non--negligible contribution.

...which describes the data quite well.

Subjecting those samples to the same background estimation process as the data sample underwent above, we obtain a prediction for the background events in the signal region, which are not removed by the data driven method. Combining this prediction with the estimated background, we obtain a distribution which represents the estimated difference between the data sample and the pure Monte Carlo sample also found in chapter~\ref{ch.mc}. Figure~\ref{tobck} compares these distributions.

\begin{figure}[htp]
\begin{minipage}[b]{.69\textwidth}
\centering (a very impressive plot)
\end{minipage}\hfill\begin{minipage}[b]{.3\textwidth}
\caption{A very informative caption.
\label{tobck}}
\end{minipage}
\end{figure}




