\documentclass[a4paper,11pt,openany]{memoir}
\usepackage{amsmath}
%\usepackage[sc,osf]{mathpazo}
%\usepackage[garamond]{mathdesign}
\usepackage{fontspec}
\usepackage[math-style=TeX]{unicode-math}
\usepackage{xunicode}
\usepackage{xltxtra}
\usepackage{polyglossia}
\defaultfontfeatures{Mapping=tex-text}
\setmainfont[   Path=fonts/xits/,
                BoldFont={xits-bold.otf},
                ItalicFont={xits-italic.otf},
                BoldItalicFont={xits-bolditalic.otf},
                SmallCapsFont={../MasticSC-Regular.otf}
            ]{xits-regular.otf}
\setsansfont[   Path=fonts/frutiger/,
                Scale=MatchLowercase,
                BoldFont={FrutigerLTStd-Bold.otf},
                ItalicFont={FrutigerLTStd-Italic.otf},
                BoldItalicFont={FrutigerLTStd-BoldItalic.otf}
            ]{FrutigerLTStd-Roman.otf}
            
\setmathfont{xits-math.otf}
\setmathfont[version=roman,range=\mathcal,Path=fonts/]{latinmodern-math.otf}
%\setmathfont[range=\mathcal,Scale=MatchUppercase]{Lynda Cursive}
%\setmathfont[range=\mathup]{Garamond Premier Pro}
%\setmathfont[range=\mathit]{Garamond Premier Pro}
\setdefaultlanguage{danish}
\setotherlanguage[variant=british]{english}

\usepackage[usenames,dvipsnames,svgnames,table]{xcolor}
\usepackage{graphicx,epic,eepic}
\usepackage{tkz-graph,tkz-euclide}
\usetikzlibrary{calc,intersections,shapes.geometric,decorations.pathmorphing}
\usetikzlibrary{snakes,patterns,plotmarks,decorations.text}
\renewcommand*{\VertexSmallMinSize}{4pt}
\usepackage{tabulary}
\usepackage{pdfpages}
\usepackage{wrapfig}
\usepackage{pifont}
\usepackage{multirow}
\usepackage{colortbl}
\usepackage[normalem]{ulem}

%\renewcommand{\thefootnote}{\fnsymbol{footnote}}

\renewcommand\bibname{References}
\renewcommand{\(}{\begin{equation}}
\renewcommand{\)}{\end{equation}}
\newcommand{\vet}[1]{\underline{#1}}
\newcommand{\mtx}[1]{\underline{\underline{#1}}}
\newcommand{\ono}[1]{\frac{1}{#1}}
\newcommand{\half}{\frac{1}{2}}
\newcommand{\thehead}{}
\newcommand{\head}[1]{\renewcommand{\thehead}{#1}}
\newcommand{\theohead}{}
\newcommand{\ohead}[1]{\renewcommand{\theohead}{#1}}
\newcommand{\thetnote}{}
\newcommand{\tnote}[1]{\renewcommand{\thetnote}{#1}}
\newcommand{\bra}[1]{\left\langle #1 \right|}
\newcommand{\ket}[1]{\left| #1 \right\rangle}
\newcommand{\braket}[2]{\left\langle#1\middle|#2\right\rangle}
\newcommand{\obraket}[3]{\left\langle#1\middle|#2\middle|#3\right\rangle}
\newcommand{\emf}{\mathcal{E}}
\newcommand{\di}{\text{d}}
\newcommand{\atlas}{\textsc{atlas}}
\newcommand{\Atlas}{\textsc{Atlas}}
\newcommand\numberthis{\stepcounter{equation}\tag{\theequation}}

\newenvironment{infilsf}{
    \begin{sffamily}
    \setmathfont[range=\mathup/{num},
                 Scale=MatchLowercase,
                 Path=fonts/frutiger/,
                 ]{FrutigerLTStd-Roman.otf}
    \setmathfont[range=\mathit/{latin,Latin},
                 Scale=MatchLowercase,
                 Path=fonts/frutiger/,
                    ]{FrutigerLTStd-Italic.otf}
    \setmathfont[range=\mathit/{greek,Greek},
                 Scale=MatchLowercase,
                 Path=fonts/dejavu/,
                    ]{DejaVuSans-Oblique.ttf}
    \setmathfont[range=\mathup/{greek,Greek},
                 Scale=MatchLowercase,
                 Path=fonts/dejavu/,
                    ]{DejaVuSans.ttf}
%    \setmathfont[range=\text,
%                 Scale=MatchLowercase,
%                 Path=fonts/frutiger/,
%                    ]{FrutigerLTStd-Roman.otf}
}{
    \setmathfont{xits-math.otf}
    \setmathfont[range=\mathcal,Path=fonts/]{latinmodern-math.otf}
    \end{sffamily}
}

\newenvironment{new}{\color{Blue}}{}

\definecolor{kugray}{RGB}{102,102,102}
\definecolor{kugray1}{RGB}{52,52,52}
\definecolor{natgreen}{RGB}{70,116,60}
\definecolor{natgreen1}{HTML}{63875B}
\definecolor{natgreen2}{HTML}{859B81}
\definecolor{natcomp}{RGB}{134,69,76}
\definecolor{natcomp1}{HTML}{9D696E}
\definecolor{natcomp2}{HTML}{B39598}
\definecolor{natlcomp}{HTML}{5d323d}
\definecolor{natrcomp}{HTML}{52325d}
\definecolor{natyellow}{HTML}{8B6448}
\definecolor{natblue}{HTML}{524F81}
\definecolor{natscatb}{HTML}{3B8178}
\definecolor{natscatg}{HTML}{418B5C}
\definecolor{natscaty}{HTML}{7C8B41}
\definecolor{natscatr}{HTML}{81783C}

\makepagestyle{fancy}

\makepsmarks{fancy}{%
\nouppercaseheads
\createmark{chapter}{left}{nonumber}{}{}
\createmark{section}{right}{shownumber}{}{ \space}
\createplainmark{toc}{both}{\contentsname}
\createplainmark{lof}{both}{\listfigurename}
\createplainmark{lot}{both}{\listtablename}
\createplainmark{bib}{both}{\bibname}
\createplainmark{index}{both}{\indexname}
\createplainmark{glossary}{both}{\glossaryname}}

\makeoddhead{fancy}
   {}{}{}
\makeoddfoot{fancy}{\makebox[0pt][r]{\raisebox{15pt}[20pt]{\textcolor{natgreen}{\rule{1.1\spinemargin}{1pt}}}\makebox[0pt][l]{\raisebox{15pt}[20pt]{\textcolor{natgreen}{\rule{\paperwidth}{1pt}}}}}\textcolor{kugray}{\textsf{\rightmark}}}{}{\textcolor{kugray}{\textsf{\thepage}}}
\makeevenfoot{fancy}{\textcolor{kugray}{\textsf{\thepage}}}{}{\textcolor{kugray}{\textsf{\leftmark}}\makebox[0pt][l]{\raisebox{15pt}{\textcolor{natgreen}{\rule{1.1\spinemargin}{1pt}}}}\makebox[0pt][r]{\raisebox{15pt}{\textcolor{natgreen}{\rule{\paperwidth}{1pt}}}}}
\setlength{\footskip}{40pt}
\pagestyle{fancy}
\aliaspagestyle{chapter}{fancy}

\captionnamefont{\sffamily\color{natgreen}\bfseries} \captiontitlefont{\footnotesize} \captionstyle{\\}
\renewcommand*{\printchaptername}{}
\renewcommand*{\chapternamenum}{}
\renewcommand*{\afterchapternum}{}
\renewcommand{\chapnumfont}{\chaptitlefont\sffamily\HUGE}
\renewcommand{\printchapternum}{\chapnumfont \colorbox{natgreen}{\textcolor{white}{\hspace{.2em}\thechapter\hspace{.2em}}}\hspace{1em}}
\setsecheadstyle{\large\bfseries}
\setsubsecheadstyle{\bfseries}
\setsubsubsecheadstyle{}
\setsecnumformat{\textsf{\color{natgreen}\csname the#1\endcsname\quad}}
\maxsecnumdepth{subsubsection}
\renewcommand{\labelenumi}{\sffamily\bfseries\color{natgreen}\theenumi.}
\renewcommand{\labelitemi}{\color{natgreen}\ding{110}}
\renewcommand{\labelitemii}{\color{natgreen}\textbullet}
\setcounter{tocdepth}{2}

\setlength{\arrayrulewidth}{2pt}

\newsubfloat{figure}

\begin{hyphenrules}{danish}
\hyphenation{be-stem-mes}
\hyphenation{rest-klas-se-sæt-ning}
\end{hyphenrules}
\begin{hyphenrules}{english}
\hyphenation{Ham-il-ton}
\hyphenation{ATLAS}
\hyphenation{Atlas}
\hyphenation{atlas}
\hyphenation{CERN}
\hyphenation{i-den-ti-cal}
\hyphenation{pro-vid-ed}
\hyphenation{Calc-HEP}
\hyphenation{par-ticles}
\hyphenation{brems-strahl-ung}
\end{hyphenrules}
\usepackage[pdfusetitle]{hyperref}
\urlstyle{sf}


\begin{document}
\begin{english}
\chapter{Introduction}

Since the late 1960s, our---at times evolving---understanding of the properties and interactions of the fundamental particles has been summarised by the Standard Model. An overview of a selection of these properties and interactions is given in fig.~\ref{SMsum}.

\begin{figure}[hbt]
\begin{minipage}[b]{.73\textwidth}
\begin{infilsf}\begin{sffamily}\begin{scriptsize}
\pgfdeclarelayer{back}
\pgfsetlayers{back,main}
\begin{tikzpicture}[yscale=1.8,xscale=0.35]

\tikzstyle{quark}=[font=\footnotesize,white,circle,draw=white,fill=natgreen,inner sep=0pt,minimum size=12pt]
\tikzstyle{gauge}=[font=\footnotesize,white,circle,draw=white,fill=natcomp,inner sep=0pt,minimum size=12pt]
\tikzstyle{higgs}=[font=\footnotesize,white,circle,draw=white,fill=natyellow,inner sep=0pt,minimum size=12pt]
\tikzstyle{lepton}=[font=\footnotesize,white,circle,draw=white,fill=natblue,inner sep=0pt,minimum size=12pt]
\tikzstyle{quarko}=[fill=white,circle,draw=natgreen,inner sep=0pt,minimum size=16pt]
\tikzstyle{gaugeo}=[fill=white,circle,draw=natcomp,inner sep=0pt,minimum size=16pt]
\tikzstyle{higgso}=[fill=white,circle,draw=natyellow,inner sep=0pt,minimum size=16pt]
\tikzstyle{leptono}=[fill=white,circle,draw=natblue,inner sep=0pt,minimum size=16pt]
\tikzstyle{inter}=[kugray!50, ultra thick,cap=rect]
\tikzstyle{under}=[line width=3pt,white]

\draw (-3.5,1.8) -- (-3.5,-1.5) -- (-1.7,-1.5) [snake=zigzag] -- (-.3,-1.5) [snake=none] --  (21,-1.5) -- (21,1.8) -- cycle;
\draw (-3.5,0) node[left] {0} -- +(8pt,0);
\draw (-3.5,1) node[left] {1} -- +(8pt,0);
\draw (-3.5,-1) node[left] {-1} -- +(8pt,0);
\draw (-2,-1.5) -- +(0,2.5pt) +(0,-.9em) node[above] {0};
\foreach \x in {0,1}
\draw (\x*2.303,-1.5) -- +(0,2.5pt) 
    +(0.693,0) -- +(0.693,1.5pt) +(1.1,0) -- +(1.1,1.5pt) 
    +(1.39,0) -- +(1.39,1.5pt) +(1.61,0) -- +(1.61,1.5pt) +(1.79,0) -- +(1.79,1.5pt)
    +(1.95,0) -- +(1.95,1.5pt) +(2.08,0) -- +(2.08,1.5pt) +(2.2,0) -- +(2.2,1.5pt);
\foreach \x in {2,3,...,8}
\draw (\x*2.303,-1.5) -- +(0,2.5pt) +(0,-.9em) node[above] {$10^{\x}$}
    +(0.693,0) -- +(0.693,1.5pt) +(1.1,0) -- +(1.1,1.5pt) 
    +(1.39,0) -- +(1.39,1.5pt) +(1.61,0) -- +(1.61,1.5pt) +(1.79,0) -- +(1.79,1.5pt)
    +(1.95,0) -- +(1.95,1.5pt) +(2.08,0) -- +(2.08,1.5pt) +(2.2,0) -- +(2.2,1.5pt);
\draw (0,-1.5) -- +(0,2.5pt) +(0,-.9em) node[above] {1};
\draw (2.303,-1.5) -- +(0,2.5pt) +(0,-.9em) node[above] {10};
\draw (9*2.303,-1.5) -- +(0,2.5pt) +(0,-.9em) node[above] {$10^9$};
\node at (-4.8,1.7) [rotate=90,left] {Electromagnetic charge [e]};
\node at (21,-1.8) [left] {Mass [eV/c$^2$]};
\draw (7.74,0.667) node [quarko] (u) {} node {\tikz {\fill[natgreen] (0,0) -- ++(0,8pt) arc (90:-90:8pt) -- cycle; \draw +(180:8pt);}} node [quark] {u};
\draw (8.48,-0.333) node [quarko] (d) {} node {\tikz {\fill[natgreen] (0,0) -- ++(0,8pt) arc (90:-90:8pt) -- cycle; \draw +(180:8pt);}} node [quark] {d};
\draw (14.06,0.667) node [quarko] (s) {} node {\tikz {\fill[natgreen] (0,0) -- ++(0,8pt) arc (90:-90:8pt) -- cycle; \draw +(180:8pt);}} node [quark] {s};
\draw (11.46,-0.333) node [quarko] (c) {} node {\tikz {\fill[natgreen] (0,0) -- ++(0,8pt) arc (90:-90:8pt) -- cycle; \draw +(180:8pt);}} node [quark] {c};
\draw (18.97,0.667) node [quarko] (t) {} node {\tikz {\fill[natgreen] (0,0) -- ++(0,8pt) arc (90:-90:8pt) -- cycle; \draw +(180:8pt);}} node [quark] {t};
\draw (15.25,-0.333) node [quarko] (b) {} node {\tikz {\fill[natgreen] (0,0) -- ++(0,8pt) arc (90:-90:8pt) -- cycle; \draw +(180:8pt);}} node [quark] {b};
\draw (-2,0) ++(100:6.5pt) ++(-4.47pt,0) node (gamma) [gaugeo] {} node {\tikz {\shade[top color=white,bottom color=natcomp] (0,0)  +(-80:12pt) -- +(-45:8pt) arc (-45:-115:8pt) -- cycle; \filldraw[natcomp] (0,0) circle (8pt); \draw +(100:12pt);}} node [gauge] {$\gamma$};
\draw (-2,0) ++(-80:6.5pt) ++(4.47pt,0) node [gaugeo] (g) {} node {\tikz {\shade[top color=natcomp,bottom color=white] +(100:12pt) -- +(135:8pt) arc (135:65:8pt) -- cycle; \filldraw[natcomp] (0,0) circle (8pt); \draw +(-80:12pt);}} node [gauge] {g};
\draw (-2,0) node {\tikz \fill[natcomp] circle (1pt);};
\draw (18.33,0) ++(100:6.5pt) ++(-4.47pt,0) node [gaugeo] (Z) {} node {\tikz {\shade[top color=white,bottom color=natcomp] (0,0)  +(-80:12pt) -- +(-45:8pt) arc (-45:-115:8pt) -- cycle; \filldraw[natcomp] (0,0) circle (8pt); \draw +(100:12pt);}} node [gauge] {Z};
\draw (18.20,1) node [gaugeo] (W+) {} node {\tikz {\filldraw[natcomp] (0,0) circle (8pt);}} node [gauge] {\tiny W$^+$};
\draw (18.20,-1) node [gaugeo] (W-) {} node {\tikz {\filldraw[natcomp] (0,0) circle (8pt);}} node [gauge] {\tiny W$^-$};
\draw (18.65,0)  ++(-80:6.5pt) ++(4.47pt,0) node [higgso] (H) {} node {\tikz {\shade[top color=natyellow,bottom color=white] +(100:12pt) -- +(135:8pt) arc (135:65:8pt) -- cycle; \draw +(-80:12pt); \draw[natyellow] (0,0) circle (8pt);}} node [higgs] {H};
\draw (18.33,0) node {\tikz \fill[natcomp] circle (1pt);};
\draw (18.65,0) node {\tikz \fill[natyellow] circle (1pt);};
\draw (0.79,0) node [leptono] (ve) {} node {\tikz {\fill[natblue] (0,0) -- ++(0,8pt) arc (90:-90:8pt) -- cycle; \draw +(180:8pt);}} node [lepton] {$\nu_{\text e}$};
\draw (5.14,0) node [leptono] (vmu) {} node {\tikz {\fill[natblue] (0,0) -- ++(0,8pt) arc (90:-90:8pt) -- cycle; \draw +(180:8pt);}} node [lepton] {$\nu_\mu$};
\draw (9.65,0) node [leptono] (vtau) {} node {\tikz {\fill[natblue] (0,0) -- ++(0,8pt) arc (90:-90:8pt) -- cycle; \draw +(180:8pt);}} node [lepton] {$\nu_\tau$};
\draw (6.24,-1) node [leptono] (e) {} node {\tikz {\fill[natblue] (0,0) -- ++(0,8pt) arc (90:-90:8pt) -- cycle; \draw +(180:8pt);}} node [lepton] {e};
\draw (11.57,-1) node [leptono] (mu) {} node {\tikz {\fill[natblue] (0,0) -- ++(0,8pt) arc (90:-90:8pt) -- cycle; \draw +(180:8pt);}} node [lepton] {$\mu$};
\draw (14.39,-1) node [leptono] (tau) {} node {\tikz {\fill[natblue] (0,0) -- ++(0,8pt) arc (90:-90:8pt) -- cycle; \draw +(180:8pt);}} node [lepton] {$\tau$};

\begin{pgfonlayer}{back}
\node at (12,1) [inner sep=0] (uqs) {};
\node at (20.7,.6) [inner sep=0] (qgnode) {};
\draw [inter] (u) to[out=0,in=270,max distance=7pt]  (uqs);
\draw [inter] (s) to[out=170,in=270,max distance=4pt] (uqs);
\draw [inter] (t) to[out=190,in=270,max distance=25pt] (uqs);
\draw [inter] (d) to[out=5,in=270,max distance=30pt] (uqs);
\draw [inter] (c) to[out=60,in=270,max distance=10pt] (uqs);
\draw [inter] (b) to[out=170,in=270,max distance=20pt] (uqs)
                  to[out=90,in=40,max distance=17pt] (g);
\draw (g) node[below right] {\tikz {\draw [inter] (0,0) to[in=-40,out=-90,min distance=20pt] (0,0);}};
\draw [inter] (uqs) to[out=90,in=30,max distance=15pt] (gamma);
\draw [inter] (uqs) to[out=90,in=150,max distance=7pt] (W+);
\draw [inter] (uqs) .. controls (12,1.4) and (20.7,1.7) .. (qgnode.south)
                    to[out=270,in=10,max distance=7pt] (Z);
\draw [inter] (qgnode) to[out=270,in=20,max distance=10pt] (H);
\draw [inter] (qgnode) to[out=270,in=20,max distance=30pt] (W-);

\draw [under] (gamma) -- ++(15,0) node [inner sep=0] (gamw) {};
\draw [under] (gamw.west) to[out=0,in=90,max distance=30pt] (W-);
\draw [under] (gamw.west) to[out=0,in=250,max distance=20pt] (W+);
\draw [inter] (gamma) -- (gamw.east);
\draw [inter] (gamw.west) to[out=0,in=250,max distance=20pt] (W+);
\draw [inter] (gamw.west) to[out=0,in=90,max distance=30pt] (W-);

\node at (17,-.3) [inner sep=0] (wz-) {};
\node at (17,.6) [inner sep=0] (wz+) {};
\draw [under] (wz-) -- (wz+) to[out=270,in=160,max distance=10pt] (Z);
\draw [under] (H) to[out=30,in=-5,min distance=30pt] (W+);
\draw [inter] (W-) to[out=160,in=270,max distance=10pt] (wz-)
           to (wz+.north) to[out=90,in=200,max distance=10pt] (W+);
\node [above right] at (W+) {\tikz {
    \draw[under] (0,0) to[out=40,in=90,min distance=20pt] (0,0);
    \draw[inter] (0,0) to[out=40,in=90,min distance=20pt] (0,0);}};
\draw [inter] (wz-) to[out=90,in=200,max distance=10pt] (Z);
\draw [inter] (wz+) to[out=270,in=160,max distance=10pt] (Z);
\draw [inter] (W-) -- (H);
\draw (W-) node[below right] {\tikz {\draw [inter] (0,0) to[in=-40,out=-90,min distance=20pt] (0,0);}};
\draw [inter] (H) to[out=50,in=-10,max distance=5pt] (Z);
\draw [inter] (H) to[out=30,in=-5,min distance=30pt] (W+);

\draw (tau) ++(1.5,.25) node [inner sep=0] (lepga) {};
\draw [inter] (tau) to[out=70,in=180,max distance=3pt] (lepga)
                    to[out=0,in=270,max distance=10pt] (wz-);
\draw [inter] (mu) to[out=70,in=180,max distance=3pt] ++(1.5,.25) -- (lepga);
\draw [inter] (e) to[out=70,in=180,max distance=3pt] ++(1.5,.25)
              node [inner sep=0] (ega) {} -- (lepga);
\draw [inter] (lepga) to[out=0,in=152,max distance=10pt] (W-);

\draw (vmu) ++(-1.5,-.25) node [inner sep=0] (neul) {};
\draw [inter] (vmu) to[out=250,in=0,max distance=3pt] (neul);
\draw [inter] (vtau) to[out=185,in=0,min distance = 30pt] (neul);
\draw [inter] (ve) -- ++(0,-.5) node [inner sep=0] (neud) {};
\draw [inter] (neul.east) to[out=180,in=90,max distance=7pt] (neud);
\draw [inter] (neud.north) to [out=270,in=180,max distance=10pt] (ega);

\draw (g) ++(-1,0) node [inner sep=0] (gamlep) {};
\draw [inter] (e) to [out=250,in=0,max distance=3pt] ++(-1.5,-.25)
              node (lepneu) [inner sep=0] {};
\draw [inter] (mu) to[out=250,in=0,max distance=3pt] ++(-1.5,-.25) -- (lepneu);
\draw [inter] (tau) to[out=250,in=0,max distance=3pt] ++(-1.5,-.25) -- (lepneu);
\draw [inter] (lepneu.east) to[out=180,in=270,max distance=20pt] (neud.north);
\draw [inter] (lepneu) to[out=180,in=270,max distance=30pt] (gamlep);
\draw [inter] (gamma) to[out=230,in=90] (gamlep.south);
\end{pgfonlayer}

\draw[inter,cap=round] (-2,1) -- (0,1) node[right,kugray] {Interactions};

\draw (-1,1.4) node [quarko] (n) {} node {\tikz {\fill[natgreen] (0,0) -- ++(0,8pt) arc (90:-90:8pt) -- cycle; \draw +(180:8pt);}} node [quark] {n};
\node at (n) {\tikz {\draw [white,postaction={decorate,decoration={text along path,text align=center,text={|\tiny|Spin}}}] ++(270:10pt) arc (270:90:10pt); \draw (-1,0) (1,0);}};
\begin{scope}
\clip (n.center) -- ++(2,0) -- ++(0,.2) -- ++(-2,0) -- cycle;
\draw (n) node [gaugeo] {} node {\tikz {\filldraw[natcomp] (0,0) circle (8pt);}} node [gauge] {n};
\end{scope}
\begin{scope}
\clip (n.center) -- ++(2,0) -- ++(0,-.2) -- ++(-2,0) -- cycle;
\draw (n) node [leptono] {} node {\tikz {\fill[natblue] (0,0) -- ++(0,8pt) arc (90:-90:8pt) -- cycle; \draw +(180:8pt);}} node [lepton] {n};
\end{scope}
\begin{scope}
\clip (n.center) -- ++(-2,0) -- ++(0,-.2) -- ++(2,0) -- cycle;
\draw (n) node [higgso] {} node {\tikz {\shade[top color=natyellow,bottom color=white] +(100:12pt) -- +(135:8pt) arc (135:65:8pt) -- cycle; \draw +(-80:12pt); \draw[natyellow] (0,0) circle (8pt);}} node [higgs] {n};
\end{scope}

\draw (n) ++(160:20pt) node [right,natgreen] {\tiny Quarks};
\draw (n) ++(-160:20pt) node [right,natyellow] {\tiny Higgs boson};
\draw (n) ++(0:20pt) node [above right,natcomp] {\tiny Gauge bosons};
\draw (n) ++(0:20pt) node [below right,natblue] {\tiny Leptons};

\end{tikzpicture}
\end{scriptsize}\end{sffamily}\end{infilsf}
\end{minipage}
\hfill\begin{minipage}[b]{.25\textwidth}
\caption{An overview of the particles of the Standard Model. The particles are arranged by mass and charge. Colour indicates particle type, the filling of the border indicates the spin of particles and lines are drawn between those particles that the Standard Model describes interactions between. The currently known maximum bounds on neutrino masses have been used to place the neutrinos in the mass direction. Table values from \cite{wikism}.\label{SMsum}}
\end{minipage}
\end{figure}

In its current form, the Standard Model makes no attempt to explain any physics beyond this.\footnote{The overview in figure~\ref{SMsum} includes massive neutrinos, which have been found experimentally, even though the Standard Model does not at present include them. There are, however, several proposed methods of extending the SM to do so.} The most obvious missing element is gravity, which continues to resist grand unification, along with a number of phenomena from cosmology, such as dark matter and dark energy, may or may not be explained by particles or forces that a complete theory of fundamental particles and forces can be expected to include.

Within these limits, the Standard Model has been remarkably successful, withstanding decades of experimental tests, correctly predicting the existence and properties of a number of particles.\footnote{Most recently, the existence of the Higgs boson was confirmed experimentally. At the time of writing, confirmation of its predicted properties is still a work in progress.} Those successes not withstanding, there are some issues within the Standard Model.

As it is formulated, the SM depends on at least 19 numerical constants,\footnote{Not counting any additional constants needed to account for neutrino masses.} the value of which must be determined experimentally, since the model offers no insight into the origin of or relations between these constants. Worse still, as formulated in the SM, higher order corrections will tend to increase the Higgs mass, with no constraint save the Planck energy. This is one example of the hierarchy problem. This implies either that some unknown physics exist between the Higgs mass scale and the Planck scale to constrain the Higgs mass, or the bare mass of the Higgs boson is very finely tuned to cancel the higher order contributions.

In the first case, we will obviously want to search for evidence of the unknown mechanism. In the latter case, we might expect there to be some underlying mechanism that ensures that the bare Higgs mass, and possibly the other free parameters of the SM, have the proper value. Again, we will want to search for physics outside the SM, as a clue to what that underlying mechanism is.

There is also the possibility that neither of those mechanisms exist, since the Standard Model, strictly speaking, does not require them. In that a search for new physics that discovers nothing is still a valuable, if less illuminating, result.

In this thesis, we shall approach the task of searching for physics beyond the Standard Model by introducing to it an extension via the effective Lagrangian approach. Specifically, we will introduce a $q\bar q\rightarrow\gamma\gamma$ point interaction, and then examine the effect of the new interaction on the distributions of certain observable quantities by performing pseudoexperiments. We can then, finally, compare the results of the pseudoexperiments to the results of actual collision experiments performed at CERN's Large Hadron Collider.

\renewcommand{\bibname}{References}
\bibliographystyle{plainurl}
\bibliography{cite}


\end{english}
\end{document}