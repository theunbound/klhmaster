\section{Feynman diagrams}
When studying individual processes described by a theory, Feynman diagrams are a useful tool. So useful, in fact, that much of the software developed to simulate processes is formulated in terms of Feynman diagrams. To see how they work, consider the simple model described by the Lagrangian \cite{srednicki}\footnote{Ignoring counterterms.}
\[\mathcal L= \half\partial^\mu\phi\partial_\mu\phi-\half m^2\phi^2-\frac{\lambda}{4!}\phi^4=\phi[\partial^2-m^2]\phi-\frac{\lambda}{4!}\phi^4\]
This is an example of a $\phi^4$ theory. The first terms in this Lagrangian, which involves two $\phi$s, describes a field propagating into another field, and the other one, which involves 4 $\phi$s, describes and interaction between four fields.

The goal will be to calculate the transition amplitude for a state $\ket{\phi_a}$ going to some other state $\ket{\phi_A}$. One procedure for doing so, which is inspired by \cite{wiki.feydiag}, is to start by writing the state in terms of an integral over all momentum states $\ket{k}$:
\[\ket{\phi}=\int\frac{d^4k}{(2\pi)^4}\tilde\phi(k)\ket{k},\]
where $k$ is a four-momentum. Using eq.~\eqref{e.Dphi}, the transition amplitude can be expressed as
\(A\propto\int\frac{d^4k_A}{(2\pi)^4}\frac{d^4k_a}{(2\pi)^4}\int\mathcal D\phi\,\phi(k_A)\phi(k_a)e^{iS}.\label{transa}\)

To express $S$ in terms of momenta, we go back to the Lagrangian and express $\phi$ in terms of its Fourier modes:
\[\phi(x)=\int\frac{d^4k}{(2\pi)^4}e^{ikx}\tilde\phi(k)\]
The Lagrangian is now
\(\begin{aligned}\mathcal L=&-\int\frac{d^4k}{(2\pi)^4}\frac{d^4k\prime}{(2\pi)^4} e^{i(k+k\prime)x}\tilde\phi(k)[kk\prime +m^2]\tilde\phi(k\prime)\\
&-\frac{\lambda}{4!}\int\frac{d^4p_1\,d^4p_2\,d^4p_3\,d^4p_4}{(2\pi)^{16}}e^{i(p_1+p_2+p_3+p_4)x}\tilde\phi(p_1)\tilde\phi(p_2)\tilde\phi(p_3)\tilde\phi(p_4).
\end{aligned}\)
Inserting this into eq.~\eqref{e.S}, it becomes clear that $x$ only appears as a phase factor, which means that integrating over $x$ only produces delta functions:
\(\begin{aligned}
S=&\int\frac{d^4k}{(2\pi)^4}\tilde\phi(-k)(k^2-m^2)\tilde\phi(k)\\
&-\frac{\lambda}{4!}\int\frac{d^4p_1\,d^4p_2\,d^4p_3\,d^4p_4}{(2\pi)^{16}}\tilde\phi(p_1)\tilde\phi(p_2)\tilde\phi(p_3)\tilde\phi(p_4)\delta(p_1+p_2+p_3+p_4),
\end{aligned}\)
where, in the first term, the delta function identified $k\prime=-k$. The first term of the action describes the free part of the theory, and the second term describes the interacting part, so we will call them $S_F$ and $S_I$, respectively. Using this expression in place of $S$, we can Taylor expand in $\lambda$:
\(e^{iS}=e^{i(S_F+S_I)}=e^{iS_F}\left(1-iS_I+\frac{(-iS_I)^2}{2}+\frac{(-iS_I)^3}{3!}+\frac{(-iS_I)^4}{4!}+\cdots\right)\)
This assumes that $\lambda$ is small enough to make the interaction merely a pertubation of the theory.


Inserting this back into eq.~\eqref{transa}, we get an expression for the transition amplitude expanded in powers of $\lambda$. If we call these terms $A_n$, so that $A\propto\sum_{n=0}^\infty A_n$, the first term of the expansion is
\(A_0=\int\frac{d^4k_A}{(2\pi)^4}\frac{d^4k_a}{(2\pi)^4}\underbrace{\int\mathcal D\phi\,\phi(k_A)\phi(k_a)e^{iS_F}}_{D_0}.\)
Looking more closely at the part labelled $D_0$ above, we can expand it to find that
\(D_0=\int\mathcal D\phi\,\phi(k_A)\phi(k_a)e^{\int\frac{d^4k}{(2\pi)^4}\phi(k)[k^2-m^2]\phi(-k)},\)
which is a Gaussian (functional) integral \cite{armbjorn}:
\[\int d^nx\,x^k\cdots x^{2N}e^{-\half x^iA_{ij}x^j}.\label{gausint}\]
As such, the integral has the following solution, provided that the participating momenta are identical:
\(D_0=\half\frac{\delta^4(k_a-k_A)}{k^2-m^2}\int\mathcal D\phi\,e^{iS_F},\)
where the delta function is introduced to ensure that the momenta are identical, as required. Introducing this delta function is equivalent to imposing momentum conservation. The remaining integral over the free action corresponds to the vacuum $0\rightarrow0$ process in free theory, and is a constant with respect to $\phi$. This constant can be interpreted as the vacuum energy content of all space, which we will nevertheless simply divide out:
\(A\propto\int\frac{d^4k_A}{(2\pi)^4}\frac{d^4k_a}{(2\pi)^4}\,\frac{\int\mathcal D\phi\,\phi(k_A)\phi(k_a)e^{iS}}{\int\mathcal D\phi\,e^{iS_F}}\)
With this, we find that 
\(D_0\Rightarrow\frac{\delta^4(k_a-k_A)}{k^2-m^2},\)
which is the propagator in momentum space.

Were we to carry out the momentum integrations over $D_0$, there would evidently be a singularity at $k^2=m^2$. This singularity can be avoided by slightly modifying the integration path. There are several ways of doing this, including Feynman's prescription, which yields the expression $1/(k^2-m^2+i\epsilon)$, the Feynman propagator in momentum space. 

The second term in the expansion is
\(A_1=\int\frac{d^4k_A}{(2\pi)^4}\frac{d^4k_a}{(2\pi)^4}\left(\prod_{n=1}^4\frac{d^4p_n}{(2\pi)^4}\right)\,D_1,\)
where\footnote{Here, $\phi^n$ is shorthand for a product of $n$ $\tilde\phi$ functions of separate momenta.}
\(D_1=-\frac{i\lambda}{4!}\delta^4(p_1+p_2+p_3+p_4)\frac{\int\mathcal D\phi\,\phi^6e^{iS_F}}{\int\mathcal D\phi\,e^{iS_F}}.\) 
Solving the Gaussian integral tell us that $D_1$ is equal to a sum of terms of the form 
\(-\ono{3!2^3}\frac{i\lambda}{4!}\delta^4(p_1+p_2+p_3+p_4)\frac{\delta^4(k_a-p_1)}{{k_a}^2-m^2}\frac{\delta^4(p_2-p_3)}{{p_2}^2-m^2}\frac{\delta^4(p_4-k_A)}{{k_A}^2-m^2},\label{1t1f}\)
where the momenta are paired in all possible combinations. There are $6!$ different combinations, but they fall into only two topologically inequivalent groups.

To see the topological classification of these terms, consider illustrating the flow of momentum given by terms of the form of eq.~\eqref{1t1f} as connected lines.

Indeed, consider this illustration
\(\text{
\begin{footnotesize}
\begin{tikzpicture}[baseline=1.5em]
\draw (1,.2) node[left]{$a$} -- ++(.2,0) ++(1.6,0) -- ++(.2,0) node[right]{$A$ \normalsize,} 
      ++(-1,.5) -- +(45:.2) +(0,0) -- +(315:.2) +(0,0) -- +(225:.2) +(0,0) -- +(135:.2); 
     % ++(0,0) to[in=225,out=135,min distance=15mm,looseness=8] ++(0,0);
\end{tikzpicture}
\end{footnotesize}
}\label{unconn}\)
which lays out the external momenta $k_a$ and $k_A$ as line ends labelled $a$ and $A$. The line ends associated with the four $p$--momenta of the four--point interaction, which must conserve momentum internally, are connected from the outset in what we shall call a vertex. What remains is to connect these momenta in pairs as required by the last three $\delta$--functions.

Connecting the external momentum $a$ to one of the internal momenta leaves no option but to connect the other external momentum to another one of the internal momenta, which leaves only one option for connecting the remaining two internal momenta, resulting in this diagram
\(\text{
\begin{footnotesize}
\begin{tikzpicture}[baseline=1.5em]
\draw (1,.2) node[left]{$a$} -- ++(.2,0) node(a) {} ++(1.6,0) node(A) {} -- ++(.2,0) node[right]{$A$ \normalsize.} 
      ++(-1,.5) -- +(45:.2) node(p4) {} +(0,0) -- +(315:.2) node(p3) {} +(0,0) -- +(225:.2) node(p2) {} +(0,0) -- +(135:.2) node(p1) {}; 
\draw[natgreen] (a) to[out=0,in=225] (p2) (A) to[out=180,in=315] (p3) (p4) to[in=135,out=45,looseness=4] (p1);
\end{tikzpicture}
\end{footnotesize}
}\label{tadpole}\)
Note that this diagram can be used to represent the described process regardless of which of the internal momenta are selected at each step, since we have neglected labelling the lines that emerge from the vertex. This is a result of the topological equivalence of the diagrams that illustrate these $4!$ terms.

Making a choice distinct from the above, we can connect the external momentum $a$ to $A$, leaving the internal momenta to be connected to one another in any order, once again leaving $4!$ topologically equivalent options, and the diagram
\(\text{
\begin{footnotesize}
\begin{tikzpicture}[baseline=1.5em]
\draw (1,.2) node[left]{$a$} -- ++(.2,0) node(a) {} ++(1.6,0) node(A) {} -- ++(.2,0) node[right]{$A$ \normalsize.} 
      ++(-1,.5) -- +(45:.2) node(p4) {} +(0,0) -- +(315:.2) node(p3) {} +(0,0) -- +(225:.2) node(p2) {} +(0,0) -- +(135:.2) node(p1) {}; 
\draw[natgreen] (a) -- (A)  (p3) to[in=45,out=315,looseness=4] (p4) (p1) to[out=135,in=225,looseness=4] (p2);
\end{tikzpicture}
\end{footnotesize}
}\label{vacbub}\)

Looking for more options, we might wish to begin with the $A$ external line rather then $a$. However, if we connect the $A$ external to any of the external momenta emanating from the vertex, we will inevitably end up with a diagram topologically equivalent to diagram~\eqref{tadpole}, only with the lines laid down in a different order. Indeed, the option of laying down the lines in any order gives us an additional $3!$ terms covered by diagram~\eqref{tadpole} and \eqref{vacbub}, which both have three lines in them. In the case of diagram~\eqref{vacbub} though, we do count a factor $2$ too many, since the exchange of lines connected to the vertex were already covered by the $4!$ ways we connected those lines to the vertex.

Attempting to connect $A$ to $a$ leads us to diagrams equivalent to diagram~\eqref{vacbub}, only with the direction of the line between $a$ and $A$ reversed. In general, reversing the direction of the lines in our diagrams gives us a factor $2^3$ more diagrams for both types, although once again, we are including diagrams that were already counted in the $4!$ ways of connecting the vertex. In diagram~\eqref{tadpole} we overcount by a factor $2$ and in diagram~\eqref{vacbub} by a factor $2^2$.

This covers every possibility for connecting diagram~\eqref{unconn}, leaving us with
\[4!\,3!\,2^3\,\left(\ono2+\ono{2^3}\right)=4!\,6\,\,5=6!\]
distinct diagrams, which is also the number of distinct orderings of the momenta in eq.~\eqref{1t1f}, all of which can be drawn to look like either diagram~\eqref{tadpole} or \eqref{vacbub}. For the present purposes, we shall note that diagram~\eqref{tadpole} and \eqref{vacbub} are topologically inequivalent because they cannot be rearranged to look like one another without disconnecting a line from a vertex,\footnote{In the parlance of the topic, two topologies are inequivalent when there does not exist a continuous map that takes one into the other. However, to properly define all the terns in the previous sentence, we would need to venture somewhat beyond the scope of this thesis.}. In this context the ingoing and outgoing state labels are attached to the external lines. This becomes important when working with more in- and outgoing states and/or more vertices.

Looking back on the $D_0$ term, we note that it fits into the scheme of diagrams with this, quite straightforward, diagram
\(\text{
\begin{footnotesize}
\begin{tikzpicture}[baseline=.5em]
\draw (1,.2) node[left]{$a$} -- ++(.2,0) node(a) {}
    ++(1.6,0) node(A) {} -- ++(.2,0) node[right]{$A$ \normalsize.};
\draw[natgreen] (a) -- (A);
\end{tikzpicture}
\end{footnotesize}
}\)
In fact, the $D_0$ term also carries the factor of $1/k^2-m^2$ that we see connected to the $\delta$--functions in eq.~\eqref{1t1f}. It seems natural, then, to associate the factor of $1/k^2-m^2$ with the lines in the diagrams above, thus having identified lines in our diagrams with the momentum space propagator. While we're at it, wa also associate the factor of $i\lambda$ with the vertex. This allows us to claim that the above diagram is completely equivalent to the $D_0$ term, and that the $D_1$ term can be expressed as
\(D_1=\ono2\left(\text{
\begin{footnotesize}
\begin{tikzpicture}[baseline=1.5em]
\draw (1,.2) node[left]{$a$} -- ++(.2,0) node(a) {} ++(1.6,0) node(A) {} -- ++(.2,0) node[right]{$A$} 
      ++(-1,.5) -- +(45:.2) node(p4) {} +(0,0) -- +(315:.2) node(p3) {} +(0,0) -- +(225:.2) node(p2) {} +(0,0) -- +(135:.2) node(p1) {}; 
\draw[natgreen] (a) to[out=0,in=225] (p2) (A) to[out=180,in=315] (p3) (p4) to[in=135,out=45,looseness=4] (p1);
\end{tikzpicture}
\end{footnotesize}
}\right)+\ono{2^3}\left(\text{
\begin{footnotesize}
\tikz[baseline=1.5em]{
\draw (1,.2) node[left]{$a$} -- ++(.2,0) node(a) {} ++(1.6,0) node(A) {} -- ++(.2,0) node[right]{$A$} 
      ++(-1,.5) -- +(45:.2) node(p4) {} +(0,0) -- +(315:.2) node(p3) {} +(0,0) -- +(225:.2) node(p2) {} +(0,0) -- +(135:.2) node(p1) {}; 
\draw[natgreen] (a) -- (A)  (p3) to[in=45,out=315,looseness=4] (p4) (p1) to[out=135,in=225,looseness=4] (p2);}
\end{footnotesize}
}\right),\)
where we have allowed the factors of $2^3$, $3!$ and $4!$ in eq.~\eqref{1t1f} to cancel with the number of topologically equivalent terms covered by each diagram. The remaining factors on each diagram are their symmetry factors, the number of ways in which the same diagram can be reached by separate transformations. This is also the number of by which we overcounted, when working out the number of terms that each diagram represents above. The symmetry factors are inherent to the diagrams, not a product of the process by which they were constructed.

It turns out, then, that these diagrams contain the same information as the equations we started out with. These diagrams are due to Richard Feynman, and as such are called Feynman diagrams.

Looking at eq.~\eqref{1t1f}, at least one of the internal $p$ momenta can not be fixed to the external $k$ momenta, which leaves this term proportional to a diverging integral, associated with the looping line in diagram~\eqref{tadpole}. There are established methods for renormalising these divergent terms, however for the present purposes, we note simply that for any process, there will among the lowest order terms that describe it be a tree level\footnote{In graph theory, a tree is a connected, loop-free graph.} diagram, which is then the leading order diagram for that process. Because we are working in a perturbative regime by assumption, loop-level diagrams of that process can be considered part of the higher order corrections to that leading order term.

Knowing how to construct the Feynman diagrams that describe the terms of the Lagrangian to some order, and knowing how to translate those Feynman diagrams, the possibility presents itself that we can derive the Lagrangian by constructing the proper diagrams, rather than going through the derivation above.

To do so, we set down the following Feynman rules, the rules for constructing the proper set of Feynman diagrams:

\begin{figure}[htb]
\hfill
\begin{tikzpicture}
\draw (-3,1) -- (-1,-1) (-3,-1) -- (-1,1);
\node[right] at (-1,0) {$=-i\lambda$};
\end{tikzpicture}
\hspace{5ex}
\begin{tikzpicture}
\node[right] at (-1,0) {$=\dfrac{1}{k^2-m^2}$};
\draw (-3,0) -- (-1,0);
\node at (0,-1) {};
\node at (0,1) {};
\end{tikzpicture}
\hfill \phantom{d}
\caption{The building blocks for Feynman diagrams in $\phi^4$ theory. Once constructed, find the momentum of each propagator by imposing momentum conservation at each vertex. Any momentum that cannot be related to one of the external momenta is integrated over.
\label{phi4rules}}
\end{figure}

\begin{enumerate}
\item Construct all topologically inequivalent diagrams in which the ingoing and outgoing states for the process in question and the proper number of vertices for the present order in $\lambda$ are connected by propagator lines.
\item Impose momentum conservation at all vertices and across all propagators. As was already noted once, this is equivalent to introducing delta functions over the momenta.
\item Determine the symmetry factor for each diagram.
\item Construct for each diagram its value by taking the product of the values for each element of the diagram from figure~\ref{phi4rules}. Integrate over all momenta that have not been related to external momenta with measure $d^4p/(2\pi)^4$. Divide by the symmetry factor associated with the diagram.
\end{enumerate}

\begin{figure}[htp]
\begin{footnotesize}\begin{center}
\begin{tikzpicture}
\draw (-4,1) node[left]{$a$} -- (-2,1) node[right]{$A$};
\draw (-4,-1) node[left]{$b$} -- (-2,-1) node[right]{$B$};
\draw (0,0) node{{\normalsize $+$}};
\draw (2,1) node[left]{$a$} -- (4,-1) node[right]{$B$};
\draw[line width=5pt,white] (2,-1) -- (4,1) ;
\draw (2,-1) node[left]{$b$} -- (4,1) node[right]{$A$};
\end{tikzpicture}
\end{center}\end{footnotesize}
\caption{The Feynman diagrams associated with the first term in the expansion of the $2\rightarrow2$ transition amplitude. Once again, connected states are required to have the same momentum, however with more particles going into and coming out of the process, there are more than one way of connecting the ingoing and outgoing states.
\label{efeydig1}}
\end{figure}


For example, the $2\rightarrow2$ transition amplitude to zeroth order in $\lambda$ is given by the Feynman diagrams in figure~\ref{efeydig1}.

The value of these diagrams, using the rules, is
\[\frac{\delta^4(k_1-k_A)}{{k_1}^2-m^2}\frac{\delta^4(k_2-k_B)}{{k_2}^2-m^2}+\frac{\delta^4(k_1-k_B)}{{k_1}^2-m^2}\frac{\delta^4(k_2-k_A)}{{k_2}^2-m^2},\]
which is indeed what we would get from solving the Gaussian integral. 

\begin{figure}[htp]
\begin{footnotesize}
\begin{minipage}{.09\textwidth}
\normalsize \hfill
\end{minipage}
\begin{minipage}{.9\textwidth}
\begin{center}
\begin{tikzpicture}[scale=.75]
\draw (-6.5,0) to [in=225,out=315,min distance=25mm,looseness=8] (-6.5,0) to [in=135,out=45,min distance=25mm,looseness=8] (-6.5,0);
\draw (-5,0) node{\normalsize$\times$};
\draw (-4,0) node{$\left(\text{\tikz[scale=.75] \draw (0,1) (0,-1);}\right.$};
\draw (-3,1) node[left]{$a$} -- (-1,1) node[right]{$A$};
\draw (-3,-1) node[left]{$b$} -- (-1,-1) node[right]{$B$};
\draw (0,0) node{{\normalsize $+$}};
\draw (1,1) node[left]{$a$} -- (3,-1) node[right]{$B$};
\draw[line width=5pt,white] (1,-1) -- (3,1) ;
\draw (1,-1) node[left]{$b$} -- (3,1) node[right]{$A$};
\draw (4,0) node{$\left)\text{\tikz[scale=.75] \draw (0,1) (0,-1);}\right.$};
\draw (5.5,0);
\end{tikzpicture}
\end{center}
\end{minipage}

\vspace{.5em}

\begin{minipage}{.09\textwidth}
\normalsize $+$
\end{minipage}
\begin{minipage}{.9\textwidth}
\begin{center}
\begin{tikzpicture}[scale=.75]
\draw (1,1) node[left]{$a$} -- ++(2,0) node[right]{$A$}
      ++(-2,-2) node[left]{$b$} -- 
      ++(1,0) to[in=45,out=135,min distance=15mm,looseness=8] ++(0,0) --
      ++(1,0) node[right]{$B$};
\draw (4,0) node{\normalsize $+$};
\draw (5,1) node[left]{$a$} -- 
      ++(1,0) to[in=225,out=315,min distance=15mm,looseness=8] ++(0,0) --
      ++(1,0) node[right]{$A$}
      ++(-2,-2) node[left]{$b$} -- 
      ++(2,0) node[right]{$B$};
\draw (8,0) node{\normalsize $+$};
\draw (9,1) node(p1)[left]{$a$} -- ++(2,-2) node[right]{$B$};
\draw[line width=5pt,white] (p1.east) ++(0,-2) -- ++(2,2);
\draw (p1.east) ++(0,-2) node[left]{$b$} --
      +(.5,.5) to[in=180,out=90,min distance=15mm,looseness=8] +(0.5,0.5) --
      ++(1.5,1.5) node[right]{$A$};
\draw (12,0) node{\normalsize $+$};
\draw (13,1) node(p2)[left]{$a$} -- 
      +(.5,-.5) to[in=180,out=270,min distance=15mm,looseness=8] +(0.5,-0.5) --
      ++(1.5,-1.5) node[right]{$B$};
\draw[line width=5pt,white] (p2.east) ++(0,-2) -- ++(2,2);
\draw (p2.east) ++(0,-2) node[left]{$b$} --
      ++(2,2) node[right]{$A$};
\end{tikzpicture}
\end{center}
\end{minipage}

\vspace{.5em}

\begin{minipage}{.09\textwidth}
\normalsize $+$
\end{minipage}
\begin{minipage}{.9\textwidth}
\begin{center}
\begin{tikzpicture}[scale=.75]
\draw (1,1) node[left]{$a$} -- (3,-1) node[right]{$B$};
\draw (1,-1) node[left]{$b$} -- (3,1) node[right]{$A$};
\end{tikzpicture}
\end{center}
\end{minipage}

\end{footnotesize}
\caption{The Feynman diagrams associated with the second term in the expansion of the $2\rightarrow2$ transition amplitude. The joining of four momenta by the last delta function is illustrated by having four lines meet in a point.
\label{efeydig2}}
\end{figure}

At the next order in $\lambda$, we can build the diagrams in fig.~\ref{efeydig2}.

Of these diagrams, those in the first two lines are simply those of fig.~\ref{efeydig1} with loops added, and can be considered the one-loop, or next-to leading order, part of those processes.
 The last diagram is the only one that we have not seen before, and it introduces a new feature. In all the diagrams we have examined so far, momentum conservation has required that one of the final states be exactly identical to one of the initial states. Not so in the final diagram, where the delta function at the vertex only requires that the sum of momenta, here defined so that positive momenta flow toward the vertex, is zero. In $S$-matrix notation, where the transition matrix $S$, which transits an initial state into a final state, can be written as
\[S=1+iT,\]
this last diagram is the first part of the non-trivial $T$-matrix.

As for the disconnected vacuum bubble seen in the top row, and in diagram~\eqref{vacbub}, note that the diagram(s) that the vacuum bubble multiplies are the diagrams for the process from the preceeding orders. At higher orders in the expansion, we will find it as a repeated pattern that the diagrams from the previous orders reoccur, multiplied with various combinations of vacuum bubble diagrams. Combining the vacuum bubble contributions on any one diagram at all orders, we find that they can be written as the exponential of the sum of all possible vacuum bubbles \cite{sred:freediagexp}. The same result can be reached by writing
\(\int\mathcal D\phi\,e^{i(S_F+S_I)},\)
the expression for the $0\rightarrow0$ process to all orders. Since this is another constant, we can divide it out like we did with the vacuum normalisation, making the expression for the transition amplitude now
\(A\propto\int\frac{d^4k_A}{(2\pi)^4}\frac{d^4k_a}{(2\pi)^4}\frac{\int\mathcal D\phi\,\phi(k_A)\phi(k_a)e^{iS}}{\int\mathcal D\phi\,e^{iS}}.\)

With that, we find that the non-trivial part of the expression, the $T$-matrix from above, can be found by taking only the connected diagrams---the diagrams in which it is possible to go from any one part to any other along connected lines---into account. Using just this process in the transition amplitude, we can calculate the probability of the system going to some final state specifically through the process described by this diagram. That probability will depend on $\lambda$, which means that if we have a way of seeing how frequently the interaction occurs, we have a way of measuring $\lambda$. If we imagine a collider experiment with this type of physics, the fact that the interacting process is the only one that allows momentum exchange---to this order---means that this is the only process that allows imaginary particles to leave the imaginary beam axis, making this process rather easy to distinguish from the others.

This simple model is, unsurprisingly, far too simple to describe reality. To begin to do so requires the development of the full Standard Model, which is the subject of entire textbooks \cite{srednicki}, and we will not go into further detail here.

\begin{figure}[htb]
\parbox[t]{.45\textwidth}{\begin{center}\begin{footnotesize}\begin{tikzpicture} [>=triangle 45]
\draw[>-] (-1,1) -- (0,1);
\draw[->] (0,1) -- (0,0);
\draw[<-] (-1,-1) -- (0,-1) -- (0,0);
\draw (-2,1) node[left] {$q$} -- (-1,1);
\draw (-2,-1)  node[left] {$\bar q$} -- (-1,-1);
\draw[snake=coil,segment aspect=0] (0,1) -- (2,1) node[right] {$\gamma$};
\draw[snake=coil,segment aspect=0] (0,-1) -- (2,-1) node[right] {$\gamma$}; 
\end{tikzpicture}
\end{footnotesize}\end{center}
\subcaption{SM contribution at tree level. \label{lofeyn}}}\hfill
\parbox[t]{.52\textwidth}{\begin{center}\begin{footnotesize}
\begin{tikzpicture} [>=triangle 45]
\draw[>-] (1,1) node[below]{$q$} -- (2,1);
\draw[decorate, decoration={coil,amplitude=2pt, segment length=2.68pt}] 
    (-2,1) node[left]{$g$} -- (0,1) ;
\draw[decorate, decoration={coil,amplitude=2pt, segment length=2.68pt}] 
    (-2,-1) node[left]{$g$} -- (0,-1); 
\draw[<-] (1,-1) node[above]{$\bar q$} -- (2,-1);
\draw (0,1) -- (1,1);
\draw (0,-1) -- (1,-1);
\draw[-<] (0,1) -- (0,0);
\draw (0,0) -- (0,-1);
\draw[->] (2,1) -- (2,0);
\draw (2,0) -- (2,-1);
\draw[decorate, decoration={snake}] (2,1) -- (4,1) node[right]{$\gamma$};
\draw[decorate, decoration={snake}] (2,-1) -- (4,-1) node[right]{$\gamma$};
\end{tikzpicture}
\end{footnotesize}\end{center}
\subcaption{SM contribution at loop level. $g$s mark gluons.\label{boxdiag}}}\hfill
\caption{ Feynman diagrams for the two leading Standard Model processes that produce a $\gamma\gamma$ final state.\label{smfeyn}}
\end{figure}

Even so, the process of developing Feynman diagrams to represent terms in the Lagrangian that describe individual processes is also applicable to the full Standard Model. This requires that the Feynman rules are extended by introducing several types of fields, which can be represented in Feynman diagrams by some new styles of lines (dashed, wavy, curled, etc.). Charge is introduced by adding a direction to the lines associated with charged particles---since reversing the charge of a particle is equivalent to reversing the time direction. These new fields interact in several new types of vertices, weighted by three coupling constants: the electromagnetic coupling $\alpha_\text{QED}$, the weak coupling constant $\alpha_W$ and the strong coupling constant $\alpha_s$. With this individual Standard Model processes can also be represented by Feynman diagrams. The processes that produce the preponderance of two-photon final states with the two diagrams are shown in figure~\ref{smfeyn}.

We can get a feel for the relative strength of these two diagrams by turning to two sets of simulated collisions available from the \atlas{} collaboration.
%\footnote{The internal ATLAS names are \verbatim{mc12_8TeV.129180.Pythia8_AU2CTEQ6L1_gammagamma_2DP20.merge.NTUP_PHOTON.e1199_s1479_s1470_r3542_r3549_p1344} and \verbatim{mc12_8TeV.146800.Pythia8_AU2CTEQ6L1_GamGamBox_pT35pT20.merge.NTUP_PHOTON.e1222_s1469_s1470_r3542_r3549_p1344}.}
Plotted in figure~\ref{boxpart} are the distribution of the invariant masses of photon pairs, defined as \cite{marshaw}
\begin{align*} 
M_{\gamma\gamma}&=\sqrt{(E_1+E_2)^2-|\mathbf p_1+\mathbf p_2|^2},
\intertext{which, in the case of massless particles, can be rewritten as}
&=\sqrt{2p_1p_2(1-\cos\theta)}. \numberthis\label{sinvmass}
\end{align*}

\begin{figure}[htp]
\begin{minipage}[b]{.69\textwidth}
\begin{sffamily}
\pgfdeclareplotmark{cross} {
\pgfpathmoveto{\pgfpoint{-0.3\pgfplotmarksize}{\pgfplotmarksize}}
\pgfpathlineto{\pgfpoint{+0.3\pgfplotmarksize}{\pgfplotmarksize}}
\pgfpathlineto{\pgfpoint{+0.3\pgfplotmarksize}{0.3\pgfplotmarksize}}
\pgfpathlineto{\pgfpoint{+1\pgfplotmarksize}{0.3\pgfplotmarksize}}
\pgfpathlineto{\pgfpoint{+1\pgfplotmarksize}{-0.3\pgfplotmarksize}}
\pgfpathlineto{\pgfpoint{+0.3\pgfplotmarksize}{-0.3\pgfplotmarksize}}
\pgfpathlineto{\pgfpoint{+0.3\pgfplotmarksize}{-1.\pgfplotmarksize}}
\pgfpathlineto{\pgfpoint{-0.3\pgfplotmarksize}{-1.\pgfplotmarksize}}
\pgfpathlineto{\pgfpoint{-0.3\pgfplotmarksize}{-0.3\pgfplotmarksize}}
\pgfpathlineto{\pgfpoint{-1.\pgfplotmarksize}{-0.3\pgfplotmarksize}}
\pgfpathlineto{\pgfpoint{-1.\pgfplotmarksize}{0.3\pgfplotmarksize}}
\pgfpathlineto{\pgfpoint{-0.3\pgfplotmarksize}{0.3\pgfplotmarksize}}
\pgfpathclose
\pgfusepathqstroke
}
\pgfdeclareplotmark{cross*} {
\pgfpathmoveto{\pgfpoint{-0.3\pgfplotmarksize}{\pgfplotmarksize}}
\pgfpathlineto{\pgfpoint{+0.3\pgfplotmarksize}{\pgfplotmarksize}}
\pgfpathlineto{\pgfpoint{+0.3\pgfplotmarksize}{0.3\pgfplotmarksize}}
\pgfpathlineto{\pgfpoint{+1\pgfplotmarksize}{0.3\pgfplotmarksize}}
\pgfpathlineto{\pgfpoint{+1\pgfplotmarksize}{-0.3\pgfplotmarksize}}
\pgfpathlineto{\pgfpoint{+0.3\pgfplotmarksize}{-0.3\pgfplotmarksize}}
\pgfpathlineto{\pgfpoint{+0.3\pgfplotmarksize}{-1.\pgfplotmarksize}}
\pgfpathlineto{\pgfpoint{-0.3\pgfplotmarksize}{-1.\pgfplotmarksize}}
\pgfpathlineto{\pgfpoint{-0.3\pgfplotmarksize}{-0.3\pgfplotmarksize}}
\pgfpathlineto{\pgfpoint{-1.\pgfplotmarksize}{-0.3\pgfplotmarksize}}
\pgfpathlineto{\pgfpoint{-1.\pgfplotmarksize}{0.3\pgfplotmarksize}}
\pgfpathlineto{\pgfpoint{-0.3\pgfplotmarksize}{0.3\pgfplotmarksize}}
\pgfpathclose
\pgfusepathqfillstroke
}
\pgfdeclareplotmark{newstar} {
\pgfpathmoveto{\pgfqpoint{0pt}{\pgfplotmarksize}}
\pgfpathlineto{\pgfqpointpolar{44}{0.5\pgfplotmarksize}}
\pgfpathlineto{\pgfqpointpolar{18}{\pgfplotmarksize}}
\pgfpathlineto{\pgfqpointpolar{-20}{0.5\pgfplotmarksize}}
\pgfpathlineto{\pgfqpointpolar{-54}{\pgfplotmarksize}}
\pgfpathlineto{\pgfqpointpolar{-90}{0.5\pgfplotmarksize}}
\pgfpathlineto{\pgfqpointpolar{234}{\pgfplotmarksize}}
\pgfpathlineto{\pgfqpointpolar{198}{0.5\pgfplotmarksize}}
\pgfpathlineto{\pgfqpointpolar{162}{\pgfplotmarksize}}
\pgfpathlineto{\pgfqpointpolar{134}{0.5\pgfplotmarksize}}
\pgfpathclose
\pgfusepathqstroke
}
\pgfdeclareplotmark{newstar*} {
\pgfpathmoveto{\pgfqpoint{0pt}{\pgfplotmarksize}}
\pgfpathlineto{\pgfqpointpolar{44}{0.5\pgfplotmarksize}}
\pgfpathlineto{\pgfqpointpolar{18}{\pgfplotmarksize}}
\pgfpathlineto{\pgfqpointpolar{-20}{0.5\pgfplotmarksize}}
\pgfpathlineto{\pgfqpointpolar{-54}{\pgfplotmarksize}}
\pgfpathlineto{\pgfqpointpolar{-90}{0.5\pgfplotmarksize}}
\pgfpathlineto{\pgfqpointpolar{234}{\pgfplotmarksize}}
\pgfpathlineto{\pgfqpointpolar{198}{0.5\pgfplotmarksize}}
\pgfpathlineto{\pgfqpointpolar{162}{\pgfplotmarksize}}
\pgfpathlineto{\pgfqpointpolar{134}{0.5\pgfplotmarksize}}
\pgfpathclose
\pgfusepathqfillstroke
}\begin{tiny}
\begin{tikzpicture}[x=.1\textwidth,y=.1\textwidth]
%\definecolor{c}{rgb}{1,1,1};
%\draw [color=c, fill=c] (0,0) rectangle (10,5.92133);
%\draw [color=c, fill=c] (1,0.592133) rectangle (9,5.32919);
\definecolor{c}{rgb}{0,0,0};
\draw [c] (1,0.592133) -- (1,5.32919) -- (9,5.32919) -- (9,0.592133) -- (1,0.592133);
\definecolor{c}{rgb}{1,1,1};
\draw [color=c, fill=c] (1,0.592133) rectangle (9,5.32919);
\definecolor{c}{rgb}{0,0,0};
\draw [c] (1,0.592133) -- (1,5.32919) -- (9,5.32919) -- (9,0.592133) -- (1,0.592133);
\definecolor{c}{named}{natgreen};
\draw [c] (1,0.592133) -- (1.032,0.592133) -- (1.032,0.592133) -- (1.064,0.592133) -- (1.064,0.592133) -- (1.096,0.592133) -- (1.096,0.592133) -- (1.128,0.592133) -- (1.128,0.592133) -- (1.16,0.592133) -- (1.16,0.596703) -- (1.192,0.596703) --
 (1.192,0.610416) -- (1.224,0.610416) -- (1.224,0.614987) -- (1.256,0.614987) -- (1.256,0.614987) -- (1.288,0.614987) -- (1.288,0.614987) -- (1.32,0.614987) -- (1.32,0.624129) -- (1.352,0.624129) -- (1.352,0.665267) -- (1.384,0.665267) --
 (1.384,0.624129) -- (1.416,0.624129) -- (1.416,0.6287) -- (1.448,0.6287) -- (1.448,0.646983) -- (1.48,0.646983) -- (1.48,0.656125) -- (1.512,0.656125) -- (1.512,0.642413) -- (1.544,0.642413) -- (1.544,0.710976) -- (1.576,0.710976) --
 (1.576,0.688122) -- (1.608,0.688122) -- (1.608,0.67898) -- (1.64,0.67898) -- (1.64,0.651554) -- (1.672,0.651554) -- (1.672,0.724689) -- (1.704,0.724689) -- (1.704,0.747543) -- (1.736,0.747543) -- (1.736,0.752114) -- (1.768,0.752114) --
 (1.768,0.953234) -- (1.8,0.953234) -- (1.8,1.44232) -- (1.832,1.44232) -- (1.832,2.71303) -- (1.864,2.71303) -- (1.864,3.70035) -- (1.896,3.70035) -- (1.896,4.31742) -- (1.928,4.31742) -- (1.928,4.74709) -- (1.96,4.74709) -- (1.96,4.7928) --
 (1.992,4.7928) -- (1.992,5.10362) -- (2.024,5.10362) -- (2.024,4.94364) -- (2.056,4.94364) -- (2.056,4.71966) -- (2.088,4.71966) -- (2.088,4.59625) -- (2.12,4.59625) -- (2.12,4.44541) -- (2.152,4.44541) -- (2.152,4.29457) -- (2.184,4.29457) --
 (2.184,4.00203) -- (2.216,4.00203) -- (2.216,3.93804) -- (2.248,3.93804) -- (2.248,3.77348) -- (2.28,3.77348) -- (2.28,3.27068) -- (2.312,3.27068) -- (2.312,3.42152) -- (2.344,3.42152) -- (2.344,3.18841) -- (2.376,3.18841) -- (2.376,2.94158) --
 (2.408,2.94158) -- (2.408,2.94615) -- (2.44,2.94615) -- (2.44,2.76331) -- (2.472,2.76331) -- (2.472,2.47535) -- (2.504,2.47535) -- (2.504,2.50734) -- (2.536,2.50734) -- (2.536,2.26508) -- (2.568,2.26508) -- (2.568,2.39307) -- (2.6,2.39307) --
 (2.6,2.03197) -- (2.632,2.03197) -- (2.632,2.07768) -- (2.664,2.07768) -- (2.664,1.82628) -- (2.696,1.82628) -- (2.696,1.87656) -- (2.728,1.87656) -- (2.728,1.90398) -- (2.76,1.90398) -- (2.76,1.64801) -- (2.792,1.64801) -- (2.792,1.59316) --
 (2.824,1.59316) -- (2.824,1.59773) -- (2.856,1.59773) -- (2.856,1.49717) -- (2.888,1.49717) -- (2.888,1.43775) -- (2.92,1.43775) -- (2.92,1.32348) -- (2.952,1.32348) -- (2.952,1.30977) -- (2.984,1.30977) -- (2.984,1.29148) -- (3.016,1.29148) --
 (3.016,1.25491) -- (3.048,1.25491) -- (3.048,1.16807) -- (3.08,1.16807) -- (3.08,1.14521) -- (3.112,1.14521) -- (3.112,1.17264) -- (3.144,1.17264) -- (3.144,1.1315) -- (3.176,1.1315) -- (3.176,1.18635) -- (3.208,1.18635) -- (3.208,1.08122) --
 (3.24,1.08122) -- (3.24,1.10865) -- (3.272,1.10865) -- (3.272,1.07665) -- (3.304,1.07665) -- (3.304,0.994372) -- (3.336,0.994372) -- (3.336,0.957805) -- (3.368,0.957805) -- (3.368,0.848103) -- (3.4,0.848103) -- (3.4,0.957805) -- (3.432,0.957805) --
 (3.432,0.838962) -- (3.464,0.838962) -- (3.464,0.934951) -- (3.496,0.934951) -- (3.496,0.870958) -- (3.528,0.870958) -- (3.528,0.870958) -- (3.56,0.870958) -- (3.56,0.884671) -- (3.592,0.884671) -- (3.592,0.884671) -- (3.624,0.884671) --
 (3.624,0.806965) -- (3.656,0.806965) -- (3.656,0.761256) -- (3.688,0.761256) -- (3.688,0.797823) -- (3.72,0.797823) -- (3.72,0.774969) -- (3.752,0.774969) -- (3.752,0.820678) -- (3.784,0.820678) -- (3.784,0.738402) -- (3.816,0.738402) --
 (3.816,0.793252) -- (3.848,0.793252) -- (3.848,0.765827) -- (3.88,0.765827) -- (3.88,0.765827) -- (3.912,0.765827) -- (3.912,0.733831) -- (3.944,0.733831) -- (3.944,0.770398) -- (3.976,0.770398) -- (3.976,0.710976) -- (4.008,0.710976) --
 (4.008,0.720118) -- (4.04,0.720118) -- (4.04,0.701834) -- (4.072,0.701834) -- (4.072,0.715547) -- (4.104,0.715547) -- (4.104,0.742972) -- (4.136,0.742972) -- (4.136,0.724689) -- (4.168,0.724689) -- (4.168,0.710976) -- (4.2,0.710976) --
 (4.2,0.706405) -- (4.232,0.706405) -- (4.232,0.710976) -- (4.264,0.710976) -- (4.264,0.701834) -- (4.296,0.701834) -- (4.296,0.697263) -- (4.328,0.697263) -- (4.328,0.674409) -- (4.36,0.674409) -- (4.36,0.669838) -- (4.392,0.669838) --
 (4.392,0.674409) -- (4.424,0.674409) -- (4.424,0.669838) -- (4.456,0.669838) -- (4.456,0.701834) -- (4.488,0.701834) -- (4.488,0.665267) -- (4.52,0.665267) -- (4.52,0.633271) -- (4.552,0.633271) -- (4.552,0.665267) -- (4.584,0.665267) --
 (4.584,0.633271) -- (4.616,0.633271) -- (4.616,0.688122) -- (4.648,0.688122) -- (4.648,0.669838) -- (4.68,0.669838) -- (4.68,0.637842) -- (4.712,0.637842) -- (4.712,0.633271) -- (4.744,0.633271) -- (4.744,0.656125) -- (4.776,0.656125) --
 (4.776,0.656125) -- (4.808,0.656125) -- (4.808,0.6287) -- (4.84,0.6287) -- (4.84,0.642413) -- (4.872,0.642413) -- (4.872,0.633271) -- (4.904,0.633271) -- (4.904,0.624129) -- (4.936,0.624129) -- (4.936,0.619558) -- (4.968,0.619558) --
 (4.968,0.633271) -- (5,0.633271) -- (5,0.646983) -- (5.032,0.646983) -- (5.032,0.6287) -- (5.064,0.6287) -- (5.064,0.665267) -- (5.096,0.665267) -- (5.096,0.624129) -- (5.128,0.624129) -- (5.128,0.614987) -- (5.16,0.614987) -- (5.16,0.651554) --
 (5.192,0.651554) -- (5.192,0.624129) -- (5.224,0.624129) -- (5.224,0.624129) -- (5.256,0.624129) -- (5.256,0.642413) -- (5.288,0.642413) -- (5.288,0.624129) -- (5.32,0.624129) -- (5.32,0.614987) -- (5.352,0.614987) -- (5.352,0.605845) --
 (5.384,0.605845) -- (5.384,0.614987) -- (5.416,0.614987) -- (5.416,0.601274) -- (5.448,0.601274) -- (5.448,0.610416) -- (5.48,0.610416) -- (5.48,0.605845) -- (5.512,0.605845) -- (5.512,0.633271) -- (5.544,0.633271) -- (5.544,0.610416) --
 (5.576,0.610416) -- (5.576,0.610416) -- (5.608,0.610416) -- (5.608,0.610416) -- (5.64,0.610416) -- (5.64,0.605845) -- (5.672,0.605845) -- (5.672,0.610416) -- (5.704,0.610416) -- (5.704,0.610416) -- (5.736,0.610416) -- (5.736,0.592133) --
 (5.768,0.592133) -- (5.768,0.601274) -- (5.8,0.601274) -- (5.8,0.596703) -- (5.832,0.596703) -- (5.832,0.610416) -- (5.864,0.610416) -- (5.864,0.605845) -- (5.896,0.605845) -- (5.896,0.596703) -- (5.928,0.596703) -- (5.928,0.610416) --
 (5.96,0.610416) -- (5.96,0.601274) -- (5.992,0.601274) -- (5.992,0.596703) -- (6.024,0.596703) -- (6.024,0.605845) -- (6.056,0.605845) -- (6.056,0.614987) -- (6.088,0.614987) -- (6.088,0.601274) -- (6.12,0.601274) -- (6.12,0.610416) --
 (6.152,0.610416) -- (6.152,0.601274) -- (6.184,0.601274) -- (6.184,0.596703) -- (6.216,0.596703) -- (6.216,0.605845) -- (6.248,0.605845) -- (6.248,0.605845) -- (6.28,0.605845) -- (6.28,0.596703) -- (6.312,0.596703) -- (6.312,0.601274) --
 (6.344,0.601274) -- (6.344,0.596703) -- (6.376,0.596703) -- (6.376,0.605845) -- (6.408,0.605845) -- (6.408,0.592133) -- (6.44,0.592133) -- (6.44,0.596703) -- (6.472,0.596703) -- (6.472,0.601274) -- (6.504,0.601274) -- (6.504,0.610416) --
 (6.536,0.610416) -- (6.536,0.605845) -- (6.568,0.605845) -- (6.568,0.601274) -- (6.6,0.601274) -- (6.6,0.601274) -- (6.632,0.601274) -- (6.632,0.610416) -- (6.664,0.610416) -- (6.664,0.596703) -- (6.696,0.596703) -- (6.696,0.601274) --
 (6.728,0.601274) -- (6.728,0.596703) -- (6.76,0.596703) -- (6.76,0.592133) -- (6.792,0.592133) -- (6.792,0.601274) -- (6.824,0.601274) -- (6.824,0.596703) -- (6.856,0.596703) -- (6.856,0.601274) -- (6.888,0.601274) -- (6.888,0.596703) --
 (6.92,0.596703) -- (6.92,0.605845) -- (6.952,0.605845) -- (6.952,0.596703) -- (6.984,0.596703) -- (6.984,0.596703) -- (7.016,0.596703) -- (7.016,0.592133) -- (7.048,0.592133) -- (7.048,0.592133) -- (7.08,0.592133) -- (7.08,0.596703) --
 (7.112,0.596703) -- (7.112,0.592133) -- (7.144,0.592133) -- (7.144,0.592133) -- (7.176,0.592133) -- (7.176,0.601274) -- (7.208,0.601274) -- (7.208,0.592133) -- (7.24,0.592133) -- (7.24,0.592133) -- (7.272,0.592133) -- (7.272,0.596703) --
 (7.304,0.596703) -- (7.304,0.596703) -- (7.336,0.596703) -- (7.336,0.596703) -- (7.368,0.596703) -- (7.368,0.601274) -- (7.4,0.601274) -- (7.4,0.605845) -- (7.432,0.605845) -- (7.432,0.596703) -- (7.464,0.596703) -- (7.464,0.592133) --
 (7.496,0.592133) -- (7.496,0.601274) -- (7.528,0.601274) -- (7.528,0.596703) -- (7.56,0.596703) -- (7.56,0.596703) -- (7.592,0.596703) -- (7.592,0.592133) -- (7.624,0.592133) -- (7.624,0.592133) -- (7.656,0.592133) -- (7.656,0.596703) --
 (7.688,0.596703) -- (7.688,0.592133) -- (7.72,0.592133) -- (7.72,0.592133) -- (7.752,0.592133) -- (7.752,0.592133) -- (7.784,0.592133) -- (7.784,0.592133) -- (7.816,0.592133) -- (7.816,0.592133) -- (7.848,0.592133) -- (7.848,0.596703) --
 (7.88,0.596703) -- (7.88,0.592133) -- (7.912,0.592133) -- (7.912,0.592133) -- (7.944,0.592133) -- (7.944,0.592133) -- (7.976,0.592133) -- (7.976,0.596703) -- (8.008,0.596703) -- (8.008,0.601274) -- (8.04,0.601274) -- (8.04,0.592133) --
 (8.072,0.592133) -- (8.072,0.592133) -- (8.104,0.592133) -- (8.104,0.592133) -- (8.136,0.592133) -- (8.136,0.596703) -- (8.168,0.596703) -- (8.168,0.592133) -- (8.2,0.592133) -- (8.2,0.596703) -- (8.232,0.596703) -- (8.232,0.592133) --
 (8.264,0.592133) -- (8.264,0.596703) -- (8.296,0.596703) -- (8.296,0.592133) -- (8.328,0.592133) -- (8.328,0.592133) -- (8.36,0.592133) -- (8.36,0.592133) -- (8.392,0.592133) -- (8.392,0.605845) -- (8.424,0.605845) -- (8.424,0.592133) --
 (8.456,0.592133) -- (8.456,0.592133) -- (8.488,0.592133) -- (8.488,0.601274) -- (8.52,0.601274) -- (8.52,0.592133) -- (8.552,0.592133) -- (8.552,0.592133) -- (8.584,0.592133) -- (8.584,0.596703) -- (8.616,0.596703) -- (8.616,0.592133) --
 (8.648,0.592133) -- (8.648,0.596703) -- (8.68,0.596703) -- (8.68,0.592133) -- (8.712,0.592133) -- (8.712,0.592133) -- (8.744,0.592133) -- (8.744,0.596703) -- (8.776,0.596703) -- (8.776,0.592133) -- (8.808,0.592133) -- (8.808,0.592133) --
 (8.84,0.592133) -- (8.84,0.592133) -- (8.872,0.592133) -- (8.872,0.592133) -- (8.904,0.592133) -- (8.904,0.592133) -- (8.936,0.592133) -- (8.936,0.592133) -- (8.968,0.592133) -- (8.968,0.596703) -- (9,0.596703);
\definecolor{c}{named}{natcomp};
\draw [c, fill=c!30] (1.704,0.592133) -- (1.704,0.60144) -- (1.736,0.60144) -- (1.736,0.612609) -- (1.768,0.612609) -- (1.768,0.636808) -- (1.8,0.636808) -- (1.8,0.795035) -- (1.832,0.795035) -- (1.832,1.13755) -- (1.864,1.13755) -- (1.864,1.32742) --
 (1.896,1.32742) -- (1.896,1.4205) -- (1.928,1.4205) -- (1.928,1.55266) -- (1.96,1.55266) -- (1.96,1.56383) -- (1.992,1.56383) -- (1.992,1.52474) -- (2.024,1.52474) -- (2.024,1.4782) -- (2.056,1.4782) -- (2.056,1.46517) -- (2.088,1.46517) --
 (2.088,1.43725) -- (2.12,1.43725) -- (2.12,1.34417) -- (2.152,1.34417) -- (2.152,1.31811) -- (2.184,1.31811) -- (2.184,1.22876) -- (2.216,1.22876) -- (2.216,1.2511) -- (2.248,1.2511) -- (2.248,1.17106) -- (2.28,1.17106) -- (2.28,1.11149) --
 (2.312,1.11149) -- (2.312,1.03889) -- (2.344,1.03889) -- (2.344,0.996076) -- (2.376,0.996076) -- (2.376,0.96443) -- (2.408,0.96443) -- (2.408,0.930924) -- (2.44,0.930924) -- (2.44,0.932785) -- (2.472,0.932785) -- (2.472,0.899278) -- (2.504,0.899278)
 -- (2.504,0.841572) -- (2.536,0.841572) -- (2.536,0.817373) -- (2.568,0.817373) -- (2.568,0.806204) -- (2.6,0.806204) -- (2.6,0.791312) -- (2.632,0.791312) -- (2.632,0.752221) -- (2.664,0.752221) -- (2.664,0.737329) -- (2.696,0.737329) --
 (2.696,0.759667) -- (2.728,0.759667) -- (2.728,0.70196) -- (2.76,0.70196) -- (2.76,0.728021) -- (2.792,0.728021) -- (2.792,0.698237) -- (2.824,0.698237) -- (2.824,0.705683) -- (2.856,0.705683) -- (2.856,0.694514) -- (2.888,0.694514) --
 (2.888,0.711268) -- (2.92,0.711268) -- (2.92,0.679623) -- (2.952,0.679623) -- (2.952,0.687069) -- (2.984,0.687069) -- (2.984,0.657285) -- (3.016,0.657285) -- (3.016,0.662869) -- (3.048,0.662869) -- (3.048,0.657285) -- (3.08,0.657285) --
 (3.08,0.662869) -- (3.112,0.662869) -- (3.112,0.642393) -- (3.144,0.642393) -- (3.144,0.657285) -- (3.176,0.657285) -- (3.176,0.633085) -- (3.208,0.633085) -- (3.208,0.634947) -- (3.24,0.634947) -- (3.24,0.63867) -- (3.272,0.63867) --
 (3.272,0.640531) -- (3.304,0.640531) -- (3.304,0.61447) -- (3.336,0.61447) -- (3.336,0.623778) -- (3.368,0.623778) -- (3.368,0.625639) -- (3.4,0.625639) -- (3.4,0.612609) -- (3.432,0.612609) -- (3.432,0.625639) -- (3.464,0.625639) --
 (3.464,0.618193) -- (3.496,0.618193) -- (3.496,0.608886) -- (3.528,0.608886) -- (3.528,0.621916) -- (3.56,0.621916) -- (3.56,0.620055) -- (3.592,0.620055) -- (3.592,0.61447) -- (3.624,0.61447) -- (3.624,0.605163) -- (3.656,0.605163) --
 (3.656,0.610747) -- (3.688,0.610747) -- (3.688,0.607024) -- (3.72,0.607024) -- (3.72,0.60144) -- (3.752,0.60144) -- (3.752,0.597717) -- (3.784,0.597717) -- (3.784,0.612609) -- (3.816,0.612609) -- (3.816,0.605163) -- (3.848,0.605163) --
 (3.848,0.612609) -- (3.88,0.612609) -- (3.88,0.603301) -- (3.912,0.603301) -- (3.912,0.597717) -- (3.944,0.597717) -- (3.944,0.60144) -- (3.976,0.60144) -- (3.976,0.60144) -- (4.008,0.60144) -- (4.008,0.599578) -- (4.04,0.599578) -- (4.04,0.595856)
 -- (4.072,0.595856) -- (4.072,0.599578) -- (4.104,0.599578) -- (4.104,0.595856) -- (4.136,0.595856) -- (4.136,0.603301) -- (4.168,0.603301) -- (4.168,0.599578) -- (4.2,0.599578) -- (4.2,0.599578) -- (4.232,0.599578) -- (4.232,0.599578) --
 (4.264,0.599578) -- (4.264,0.595856) -- (4.296,0.595856) -- (4.296,0.593994) -- (4.328,0.593994) -- (4.328,0.593994) -- (4.36,0.593994) -- (4.36,0.597717) -- (4.392,0.597717) -- (4.392,0.60144) -- (4.424,0.60144) -- (4.424,0.595856) --
 (4.456,0.595856) -- (4.456,0.592133) -- (4.488,0.592133) -- (4.488,0.593994) -- (4.52,0.593994) -- (4.52,0.595856) -- (4.552,0.595856) -- (4.552,0.597717) -- (4.584,0.597717) -- (4.584,0.592133) -- (4.616,0.592133) -- (4.616,0.592133) --
 (4.648,0.592133) -- (4.648,0.592133) -- (4.68,0.592133) -- (4.68,0.593994) -- (4.712,0.593994) -- (4.712,0.60144) -- (4.744,0.60144) -- (4.744,0.593994) -- (4.776,0.593994) -- (4.776,0.597717) -- (4.808,0.597717) -- (4.808,0.592133) --
 (4.84,0.592133) -- (4.84,0.593994) -- (4.872,0.593994) -- (4.872,0.593994) -- (4.904,0.593994) -- (4.904,0.593994) -- (4.936,0.593994) -- (4.936,0.595856) -- (4.968,0.595856) -- (4.968,0.595856) -- (5,0.595856) -- (5,0.592133) -- (5.032,0.592133) --
 (5.032,0.595856) -- (5.064,0.595856) -- (5.064,0.592133) -- (5.096,0.592133) -- (5.096,0.592133) -- (5.128,0.592133) -- (5.128,0.593994) -- (5.16,0.593994) -- (5.16,0.592133) -- (5.192,0.592133) -- (5.192,0.593994) -- (5.224,0.593994) --
 (5.224,0.597717) -- (5.256,0.597717) -- (5.256,0.592133) -- (5.288,0.592133) -- (5.288,0.593994) -- (5.32,0.593994) -- (5.32,0.592133) -- (5.352,0.592133) -- (5.352,0.592133) -- (5.384,0.592133) -- (5.384,0.592133) -- (5.416,0.592133) --
 (5.416,0.592133) -- (5.448,0.592133) -- (5.448,0.592133) -- (5.48,0.592133) -- (5.48,0.592133) -- (5.512,0.592133) -- (5.512,0.593994) -- (5.544,0.593994) -- (5.544,0.593994) -- (5.576,0.593994) -- (5.576,0.593994) -- (5.608,0.593994) --
 (5.608,0.592133) -- (5.64,0.592133) -- (5.64,0.592133) -- (5.672,0.592133) -- (5.672,0.593994) -- (5.704,0.593994) -- (5.704,0.595856) -- (5.736,0.595856) -- (5.736,0.592133) -- (5.768,0.592133) -- (5.768,0.592133) -- (5.8,0.592133) --
 (5.8,0.593994) -- (5.832,0.593994) -- (5.832,0.592133) -- (5.864,0.592133) -- (5.864,0.592133) -- (5.896,0.592133) -- (5.896,0.592133) -- (5.928,0.592133) -- (5.928,0.592133) -- (5.96,0.592133) -- (5.96,0.592133) -- (5.992,0.592133) --
 (5.992,0.592133) -- (6.024,0.592133) -- (6.024,0.592133) -- (6.056,0.592133) -- (6.056,0.592133) -- (6.088,0.592133) -- (6.088,0.593994) -- (6.12,0.593994) -- (6.12,0.595856) -- (6.152,0.595856) -- (6.152,0.593994) -- (6.184,0.593994) --
 (6.184,0.592133) -- (6.216,0.592133) -- (6.216,0.593994) -- (6.248,0.593994) -- (6.248,0.592133) -- (6.28,0.592133) -- (6.28,0.592133) -- (6.312,0.592133) -- (6.312,0.592133) -- (6.344,0.592133) -- (6.344,0.592133) -- (6.376,0.592133) --
 (6.376,0.592133) -- (6.408,0.592133) -- (6.408,0.593994) -- (6.44,0.593994) -- (6.44,0.592133) -- (6.472,0.592133) -- (6.472,0.593994) -- (6.504,0.593994) -- (6.504,0.592133) -- (6.536,0.592133) -- (6.536,0.592133) -- (6.568,0.592133) --
 (6.568,0.592133) -- (6.6,0.592133) -- (6.6,0.593994) -- (6.632,0.593994) -- (6.632,0.592133) -- (6.664,0.592133) -- (6.664,0.592133) -- (6.696,0.592133) -- (6.696,0.592133) -- (6.728,0.592133) -- (6.728,0.592133) -- (6.76,0.592133) --
 (6.76,0.592133) -- (6.792,0.592133) -- (6.792,0.593994) -- (6.824,0.593994) -- (6.824,0.592133) -- (6.856,0.592133) -- (6.856,0.592133) -- (6.888,0.592133) -- (6.888,0.592133) -- (6.92,0.592133) -- (6.92,0.592133) -- (6.952,0.592133) --
 (6.952,0.592133) -- (6.984,0.592133) -- (6.984,0.592133) -- (7.016,0.592133) -- (7.016,0.592133) -- (7.048,0.592133) -- (7.048,0.592133) -- (7.08,0.592133) -- (7.08,0.592133) -- (7.112,0.592133) -- (7.112,0.592133) -- (7.144,0.592133) --
 (7.144,0.592133) -- (7.176,0.592133) -- (7.176,0.592133) -- (7.208,0.592133) -- (7.208,0.592133) -- (7.24,0.592133) -- (7.24,0.592133) -- (7.272,0.592133) -- (7.272,0.592133) -- (7.304,0.592133) -- (7.304,0.592133) -- (7.336,0.592133) --
 (7.336,0.592133) -- (7.368,0.592133) -- (7.368,0.592133) -- (7.4,0.592133) -- (7.4,0.592133) -- (7.432,0.592133) -- (7.432,0.592133) -- (7.464,0.592133) -- (7.464,0.592133) -- (7.496,0.592133) -- (7.496,0.592133) -- (7.528,0.592133) --
 (7.528,0.592133) -- (7.56,0.592133) -- (7.56,0.592133) -- (7.592,0.592133) -- (7.592,0.592133) -- (7.624,0.592133) -- (7.624,0.592133) -- (7.656,0.592133) -- (7.656,0.592133) -- (7.688,0.592133) -- (7.688,0.592133) -- (7.72,0.592133) --
 (7.72,0.592133) -- (7.752,0.592133) -- (7.752,0.592133) -- (7.784,0.592133) -- (7.784,0.592133) -- (7.816,0.592133) -- (7.816,0.592133) -- (7.848,0.592133) -- (7.848,0.592133) -- (7.88,0.592133) -- (7.88,0.592133) -- (7.912,0.592133) --
 (7.912,0.592133) -- (7.944,0.592133) -- (7.944,0.592133) -- (7.976,0.592133) -- (7.976,0.592133) -- (8.008,0.592133) -- (8.008,0.592133) -- (8.04,0.592133) -- (8.04,0.592133) -- (8.072,0.592133) -- (8.072,0.592133) -- (8.104,0.592133) --
 (8.104,0.592133) -- (8.136,0.592133) -- (8.136,0.592133) -- (8.168,0.592133) -- (8.168,0.592133) -- (8.2,0.592133) -- (8.2,0.592133) -- (8.232,0.592133) -- (8.232,0.592133) -- (8.264,0.592133) -- (8.264,0.592133) -- (8.296,0.592133) --
 (8.296,0.592133) -- (8.328,0.592133) -- (8.328,0.592133) -- (8.36,0.592133) -- (8.36,0.592133) -- (8.392,0.592133) -- (8.392,0.592133) -- (8.424,0.592133) -- (8.424,0.592133) -- (8.456,0.592133) -- (8.456,0.592133) -- (8.488,0.592133) --
 (8.488,0.592133) -- (8.52,0.592133) -- (8.52,0.592133) -- (8.552,0.592133) -- (8.552,0.592133) -- (8.584,0.592133) -- (8.584,0.592133) -- (8.616,0.592133) -- (8.616,0.592133) -- (8.648,0.592133) -- (8.648,0.592133) -- (8.68,0.592133) --
 (8.68,0.592133) -- (8.712,0.592133) -- (8.712,0.592133) -- (8.744,0.592133) -- (8.744,0.592133) -- (8.776,0.592133) -- (8.776,0.592133) -- (8.808,0.592133) -- (8.808,0.592133) -- (8.84,0.592133) -- (8.84,0.592133) -- (8.872,0.592133) --
 (8.872,0.592133) -- (8.904,0.592133) -- (8.904,0.592133) -- (8.936,0.592133) -- (8.936,0.592133) -- (8.968,0.592133) -- (8.968,0.592133) -- (9,0.592133) -- (9,0.592133);
\draw [c] (1.704,0.592133) -- (1.704,0.60144) -- (1.736,0.60144) -- (1.736,0.612609) -- (1.768,0.612609) -- (1.768,0.636808) -- (1.8,0.636808) -- (1.8,0.795035) -- (1.832,0.795035) -- (1.832,1.13755) -- (1.864,1.13755) -- (1.864,1.32742) --
 (1.896,1.32742) -- (1.896,1.4205) -- (1.928,1.4205) -- (1.928,1.55266) -- (1.96,1.55266) -- (1.96,1.56383) -- (1.992,1.56383) -- (1.992,1.52474) -- (2.024,1.52474) -- (2.024,1.4782) -- (2.056,1.4782) -- (2.056,1.46517) -- (2.088,1.46517) --
 (2.088,1.43725) -- (2.12,1.43725) -- (2.12,1.34417) -- (2.152,1.34417) -- (2.152,1.31811) -- (2.184,1.31811) -- (2.184,1.22876) -- (2.216,1.22876) -- (2.216,1.2511) -- (2.248,1.2511) -- (2.248,1.17106) -- (2.28,1.17106) -- (2.28,1.11149) --
 (2.312,1.11149) -- (2.312,1.03889) -- (2.344,1.03889) -- (2.344,0.996076) -- (2.376,0.996076) -- (2.376,0.96443) -- (2.408,0.96443) -- (2.408,0.930924) -- (2.44,0.930924) -- (2.44,0.932785) -- (2.472,0.932785) -- (2.472,0.899278) -- (2.504,0.899278)
 -- (2.504,0.841572) -- (2.536,0.841572) -- (2.536,0.817373) -- (2.568,0.817373) -- (2.568,0.806204) -- (2.6,0.806204) -- (2.6,0.791312) -- (2.632,0.791312) -- (2.632,0.752221) -- (2.664,0.752221) -- (2.664,0.737329) -- (2.696,0.737329) --
 (2.696,0.759667) -- (2.728,0.759667) -- (2.728,0.70196) -- (2.76,0.70196) -- (2.76,0.728021) -- (2.792,0.728021) -- (2.792,0.698237) -- (2.824,0.698237) -- (2.824,0.705683) -- (2.856,0.705683) -- (2.856,0.694514) -- (2.888,0.694514) --
 (2.888,0.711268) -- (2.92,0.711268) -- (2.92,0.679623) -- (2.952,0.679623) -- (2.952,0.687069) -- (2.984,0.687069) -- (2.984,0.657285) -- (3.016,0.657285) -- (3.016,0.662869) -- (3.048,0.662869) -- (3.048,0.657285) -- (3.08,0.657285) --
 (3.08,0.662869) -- (3.112,0.662869) -- (3.112,0.642393) -- (3.144,0.642393) -- (3.144,0.657285) -- (3.176,0.657285) -- (3.176,0.633085) -- (3.208,0.633085) -- (3.208,0.634947) -- (3.24,0.634947) -- (3.24,0.63867) -- (3.272,0.63867) --
 (3.272,0.640531) -- (3.304,0.640531) -- (3.304,0.61447) -- (3.336,0.61447) -- (3.336,0.623778) -- (3.368,0.623778) -- (3.368,0.625639) -- (3.4,0.625639) -- (3.4,0.612609) -- (3.432,0.612609) -- (3.432,0.625639) -- (3.464,0.625639) --
 (3.464,0.618193) -- (3.496,0.618193) -- (3.496,0.608886) -- (3.528,0.608886) -- (3.528,0.621916) -- (3.56,0.621916) -- (3.56,0.620055) -- (3.592,0.620055) -- (3.592,0.61447) -- (3.624,0.61447) -- (3.624,0.605163) -- (3.656,0.605163) --
 (3.656,0.610747) -- (3.688,0.610747) -- (3.688,0.607024) -- (3.72,0.607024) -- (3.72,0.60144) -- (3.752,0.60144) -- (3.752,0.597717) -- (3.784,0.597717) -- (3.784,0.612609) -- (3.816,0.612609) -- (3.816,0.605163) -- (3.848,0.605163) --
 (3.848,0.612609) -- (3.88,0.612609) -- (3.88,0.603301) -- (3.912,0.603301) -- (3.912,0.597717) -- (3.944,0.597717) -- (3.944,0.60144) -- (3.976,0.60144) -- (3.976,0.60144) -- (4.008,0.60144) -- (4.008,0.599578) -- (4.04,0.599578) -- (4.04,0.595856)
 -- (4.072,0.595856) -- (4.072,0.599578) -- (4.104,0.599578) -- (4.104,0.595856) -- (4.136,0.595856) -- (4.136,0.603301) -- (4.168,0.603301) -- (4.168,0.599578) -- (4.2,0.599578) -- (4.2,0.599578) -- (4.232,0.599578) -- (4.232,0.599578) --
 (4.264,0.599578) -- (4.264,0.595856) -- (4.296,0.595856) -- (4.296,0.593994) -- (4.328,0.593994) -- (4.328,0.593994) -- (4.36,0.593994) -- (4.36,0.597717) -- (4.392,0.597717) -- (4.392,0.60144) -- (4.424,0.60144) -- (4.424,0.595856) --
 (4.456,0.595856) -- (4.456,0.592133) -- (4.488,0.592133) -- (4.488,0.593994) -- (4.52,0.593994) -- (4.52,0.595856) -- (4.552,0.595856) -- (4.552,0.597717) -- (4.584,0.597717) -- (4.584,0.592133) -- (4.616,0.592133) -- (4.616,0.592133) --
 (4.648,0.592133) -- (4.648,0.592133) -- (4.68,0.592133) -- (4.68,0.593994) -- (4.712,0.593994) -- (4.712,0.60144) -- (4.744,0.60144) -- (4.744,0.593994) -- (4.776,0.593994) -- (4.776,0.597717) -- (4.808,0.597717) -- (4.808,0.592133) --
 (4.84,0.592133) -- (4.84,0.593994) -- (4.872,0.593994) -- (4.872,0.593994) -- (4.904,0.593994) -- (4.904,0.593994) -- (4.936,0.593994) -- (4.936,0.595856) -- (4.968,0.595856) -- (4.968,0.595856) -- (5,0.595856) -- (5,0.592133) -- (5.032,0.592133) --
 (5.032,0.595856) -- (5.064,0.595856) -- (5.064,0.592133) -- (5.096,0.592133) -- (5.096,0.592133) -- (5.128,0.592133) -- (5.128,0.593994) -- (5.16,0.593994) -- (5.16,0.592133) -- (5.192,0.592133) -- (5.192,0.593994) -- (5.224,0.593994) --
 (5.224,0.597717) -- (5.256,0.597717) -- (5.256,0.592133) -- (5.288,0.592133) -- (5.288,0.593994) -- (5.32,0.593994) -- (5.32,0.592133) -- (5.352,0.592133) -- (5.352,0.592133) -- (5.384,0.592133) -- (5.384,0.592133) -- (5.416,0.592133) --
 (5.416,0.592133) -- (5.448,0.592133) -- (5.448,0.592133) -- (5.48,0.592133) -- (5.48,0.592133) -- (5.512,0.592133) -- (5.512,0.593994) -- (5.544,0.593994) -- (5.544,0.593994) -- (5.576,0.593994) -- (5.576,0.593994) -- (5.608,0.593994) --
 (5.608,0.592133) -- (5.64,0.592133) -- (5.64,0.592133) -- (5.672,0.592133) -- (5.672,0.593994) -- (5.704,0.593994) -- (5.704,0.595856) -- (5.736,0.595856) -- (5.736,0.592133) -- (5.768,0.592133) -- (5.768,0.592133) -- (5.8,0.592133) --
 (5.8,0.593994) -- (5.832,0.593994) -- (5.832,0.592133) -- (5.864,0.592133) -- (5.864,0.592133) -- (5.896,0.592133) -- (5.896,0.592133) -- (5.928,0.592133) -- (5.928,0.592133) -- (5.96,0.592133) -- (5.96,0.592133) -- (5.992,0.592133) --
 (5.992,0.592133) -- (6.024,0.592133) -- (6.024,0.592133) -- (6.056,0.592133) -- (6.056,0.592133) -- (6.088,0.592133) -- (6.088,0.593994) -- (6.12,0.593994) -- (6.12,0.595856) -- (6.152,0.595856) -- (6.152,0.593994) -- (6.184,0.593994) --
 (6.184,0.592133) -- (6.216,0.592133) -- (6.216,0.593994) -- (6.248,0.593994) -- (6.248,0.592133) -- (6.28,0.592133) -- (6.28,0.592133) -- (6.312,0.592133) -- (6.312,0.592133) -- (6.344,0.592133) -- (6.344,0.592133) -- (6.376,0.592133) --
 (6.376,0.592133) -- (6.408,0.592133) -- (6.408,0.593994) -- (6.44,0.593994) -- (6.44,0.592133) -- (6.472,0.592133) -- (6.472,0.593994) -- (6.504,0.593994) -- (6.504,0.592133) -- (6.536,0.592133) -- (6.536,0.592133) -- (6.568,0.592133) --
 (6.568,0.592133) -- (6.6,0.592133) -- (6.6,0.593994) -- (6.632,0.593994) -- (6.632,0.592133) -- (6.664,0.592133) -- (6.664,0.592133) -- (6.696,0.592133) -- (6.696,0.592133) -- (6.728,0.592133) -- (6.728,0.592133) -- (6.76,0.592133) --
 (6.76,0.592133) -- (6.792,0.592133) -- (6.792,0.593994) -- (6.824,0.593994) -- (6.824,0.592133) -- (6.856,0.592133) -- (6.856,0.592133) -- (6.888,0.592133) -- (6.888,0.592133) -- (6.92,0.592133) -- (6.92,0.592133) -- (6.952,0.592133) --
 (6.952,0.592133) -- (6.984,0.592133) -- (6.984,0.592133) -- (7.016,0.592133) -- (7.016,0.592133) -- (7.048,0.592133) -- (7.048,0.592133) -- (7.08,0.592133) -- (7.08,0.592133) -- (7.112,0.592133) -- (7.112,0.592133) -- (7.144,0.592133) --
 (7.144,0.592133) -- (7.176,0.592133) -- (7.176,0.592133) -- (7.208,0.592133) -- (7.208,0.592133) -- (7.24,0.592133) -- (7.24,0.592133) -- (7.272,0.592133) -- (7.272,0.592133) -- (7.304,0.592133) -- (7.304,0.592133) -- (7.336,0.592133) --
 (7.336,0.592133) -- (7.368,0.592133) -- (7.368,0.592133) -- (7.4,0.592133) -- (7.4,0.592133) -- (7.432,0.592133) -- (7.432,0.592133) -- (7.464,0.592133) -- (7.464,0.592133) -- (7.496,0.592133) -- (7.496,0.592133) -- (7.528,0.592133) --
 (7.528,0.592133) -- (7.56,0.592133) -- (7.56,0.592133) -- (7.592,0.592133) -- (7.592,0.592133) -- (7.624,0.592133) -- (7.624,0.592133) -- (7.656,0.592133) -- (7.656,0.592133) -- (7.688,0.592133) -- (7.688,0.592133) -- (7.72,0.592133) --
 (7.72,0.592133) -- (7.752,0.592133) -- (7.752,0.592133) -- (7.784,0.592133) -- (7.784,0.592133) -- (7.816,0.592133) -- (7.816,0.592133) -- (7.848,0.592133) -- (7.848,0.592133) -- (7.88,0.592133) -- (7.88,0.592133) -- (7.912,0.592133) --
 (7.912,0.592133) -- (7.944,0.592133) -- (7.944,0.592133) -- (7.976,0.592133) -- (7.976,0.592133) -- (8.008,0.592133) -- (8.008,0.592133) -- (8.04,0.592133) -- (8.04,0.592133) -- (8.072,0.592133) -- (8.072,0.592133) -- (8.104,0.592133) --
 (8.104,0.592133) -- (8.136,0.592133) -- (8.136,0.592133) -- (8.168,0.592133) -- (8.168,0.592133) -- (8.2,0.592133) -- (8.2,0.592133) -- (8.232,0.592133) -- (8.232,0.592133) -- (8.264,0.592133) -- (8.264,0.592133) -- (8.296,0.592133) --
 (8.296,0.592133) -- (8.328,0.592133) -- (8.328,0.592133) -- (8.36,0.592133) -- (8.36,0.592133) -- (8.392,0.592133) -- (8.392,0.592133) -- (8.424,0.592133) -- (8.424,0.592133) -- (8.456,0.592133) -- (8.456,0.592133) -- (8.488,0.592133) --
 (8.488,0.592133) -- (8.52,0.592133) -- (8.52,0.592133) -- (8.552,0.592133) -- (8.552,0.592133) -- (8.584,0.592133) -- (8.584,0.592133) -- (8.616,0.592133) -- (8.616,0.592133) -- (8.648,0.592133) -- (8.648,0.592133) -- (8.68,0.592133) --
 (8.68,0.592133) -- (8.712,0.592133) -- (8.712,0.592133) -- (8.744,0.592133) -- (8.744,0.592133) -- (8.776,0.592133) -- (8.776,0.592133) -- (8.808,0.592133) -- (8.808,0.592133) -- (8.84,0.592133) -- (8.84,0.592133) -- (8.872,0.592133) --
 (8.872,0.592133) -- (8.904,0.592133) -- (8.904,0.592133) -- (8.936,0.592133) -- (8.936,0.592133) -- (8.968,0.592133) -- (8.968,0.592133) -- (9,0.592133);
\definecolor{c}{rgb}{0,0,0};
\draw [c] (1,0.592133) -- (9,0.592133);
\draw [c] (1,0.734244) -- (1,0.592133);
\draw [c] (1.16,0.663188) -- (1.16,0.592133);
\draw [c] (1.32,0.663188) -- (1.32,0.592133);
\draw [c] (1.48,0.663188) -- (1.48,0.592133);
\draw [c] (1.64,0.663188) -- (1.64,0.592133);
\draw [c] (1.8,0.734244) -- (1.8,0.592133);
\draw [c] (1.96,0.663188) -- (1.96,0.592133);
\draw [c] (2.12,0.663188) -- (2.12,0.592133);
\draw [c] (2.28,0.663188) -- (2.28,0.592133);
\draw [c] (2.44,0.663188) -- (2.44,0.592133);
\draw [c] (2.6,0.734244) -- (2.6,0.592133);
\draw [c] (2.76,0.663188) -- (2.76,0.592133);
\draw [c] (2.92,0.663188) -- (2.92,0.592133);
\draw [c] (3.08,0.663188) -- (3.08,0.592133);
\draw [c] (3.24,0.663188) -- (3.24,0.592133);
\draw [c] (3.4,0.734244) -- (3.4,0.592133);
\draw [c] (3.56,0.663188) -- (3.56,0.592133);
\draw [c] (3.72,0.663188) -- (3.72,0.592133);
\draw [c] (3.88,0.663188) -- (3.88,0.592133);
\draw [c] (4.04,0.663188) -- (4.04,0.592133);
\draw [c] (4.2,0.734244) -- (4.2,0.592133);
\draw [c] (4.36,0.663188) -- (4.36,0.592133);
\draw [c] (4.52,0.663188) -- (4.52,0.592133);
\draw [c] (4.68,0.663188) -- (4.68,0.592133);
\draw [c] (4.84,0.663188) -- (4.84,0.592133);
\draw [c] (5,0.734244) -- (5,0.592133);
\draw [c] (5.16,0.663188) -- (5.16,0.592133);
\draw [c] (5.32,0.663188) -- (5.32,0.592133);
\draw [c] (5.48,0.663188) -- (5.48,0.592133);
\draw [c] (5.64,0.663188) -- (5.64,0.592133);
\draw [c] (5.8,0.734244) -- (5.8,0.592133);
\draw [c] (5.96,0.663188) -- (5.96,0.592133);
\draw [c] (6.12,0.663188) -- (6.12,0.592133);
\draw [c] (6.28,0.663188) -- (6.28,0.592133);
\draw [c] (6.44,0.663188) -- (6.44,0.592133);
\draw [c] (6.6,0.734244) -- (6.6,0.592133);
\draw [c] (6.76,0.663188) -- (6.76,0.592133);
\draw [c] (6.92,0.663188) -- (6.92,0.592133);
\draw [c] (7.08,0.663188) -- (7.08,0.592133);
\draw [c] (7.24,0.663188) -- (7.24,0.592133);
\draw [c] (7.4,0.734244) -- (7.4,0.592133);
\draw [c] (7.56,0.663188) -- (7.56,0.592133);
\draw [c] (7.72,0.663188) -- (7.72,0.592133);
\draw [c] (7.88,0.663188) -- (7.88,0.592133);
\draw [c] (8.04,0.663188) -- (8.04,0.592133);
\draw [c] (8.2,0.734244) -- (8.2,0.592133);
\draw [c] (8.36,0.663188) -- (8.36,0.592133);
\draw [c] (8.52,0.663188) -- (8.52,0.592133);
\draw [c] (8.68,0.663188) -- (8.68,0.592133);
\draw [c] (8.84,0.663188) -- (8.84,0.592133);
\draw [c] (9,0.734244) -- (9,0.592133);
\draw [anchor=base] (1,0.396729) node[]{0};
\draw [anchor=base] (1.8,0.396729) node[]{100};
\draw [anchor=base] (2.6,0.396729) node[]{200};
\draw [anchor=base] (3.4,0.396729) node[]{300};
\draw [anchor=base] (4.2,0.396729) node[]{400};
\draw [anchor=base] (5,0.396729) node[]{500};
\draw [anchor=base] (5.8,0.396729) node[]{600};
\draw [anchor=base] (6.6,0.396729) node[]{700};
\draw [anchor=base] (7.4,0.396729) node[]{800};
\draw [anchor=base] (8.2,0.396729) node[]{900};
\draw [anchor=base] (9,0.396729) node[]{1000};
\draw [anchor=base] (9,0.15) node[left] {\textit{M$_{\gamma\gamma}$} [GeV]};
\draw [c] (1,0.592133) -- (1,5.32919);
\draw [c] (1.24,0.592133) -- (1,0.592133);
\draw [c] (1.12,0.732749) -- (1,0.732749);
\draw [c] (1.12,0.873365) -- (1,0.873365);
\draw [c] (1.12,1.01398) -- (1,1.01398);
\draw [c] (1.24,1.1546) -- (1,1.1546);
\draw [c] (1.12,1.29521) -- (1,1.29521);
\draw [c] (1.12,1.43583) -- (1,1.43583);
\draw [c] (1.12,1.57644) -- (1,1.57644);
\draw [c] (1.24,1.71706) -- (1,1.71706);
\draw [c] (1.12,1.85768) -- (1,1.85768);
\draw [c] (1.12,1.99829) -- (1,1.99829);
\draw [c] (1.12,2.13891) -- (1,2.13891);
\draw [c] (1.24,2.27953) -- (1,2.27953);
\draw [c] (1.12,2.42014) -- (1,2.42014);
\draw [c] (1.12,2.56076) -- (1,2.56076);
\draw [c] (1.12,2.70137) -- (1,2.70137);
\draw [c] (1.24,2.84199) -- (1,2.84199);
\draw [c] (1.12,2.9826) -- (1,2.9826);
\draw [c] (1.12,3.12322) -- (1,3.12322);
\draw [c] (1.12,3.26384) -- (1,3.26384);
\draw [c] (1.24,3.40445) -- (1,3.40445);
\draw [c] (1.12,3.54507) -- (1,3.54507);
\draw [c] (1.12,3.68569) -- (1,3.68569);
\draw [c] (1.12,3.8263) -- (1,3.8263);
\draw [c] (1.24,3.96692) -- (1,3.96692);
\draw [c] (1.12,4.10753) -- (1,4.10753);
\draw [c] (1.12,4.24815) -- (1,4.24815);
\draw [c] (1.12,4.38877) -- (1,4.38877);
\draw [c] (1.24,4.52938) -- (1,4.52938);
\draw [c] (1.12,4.67) -- (1,4.67);
\draw [c] (1.12,4.81061) -- (1,4.81061);
\draw [c] (1.12,4.95123) -- (1,4.95123);
\draw [c] (1.24,5.09185) -- (1,5.09185);
\draw [c] (1.24,5.09185) -- (1,5.09185);
\draw [c] (1.12,5.23246) -- (1,5.23246);
\draw [anchor= east] (0.95,0.592133) node[]{0};
\draw [anchor= east] (0.95,1.1546) node[]{200};
\draw [anchor= east] (0.95,1.71706) node[]{400};
\draw [anchor= east] (0.95,2.27953) node[]{600};
\draw [anchor= east] (0.95,2.84199) node[]{800};
\draw [anchor= east] (0.95,3.40445) node[]{1000};
\draw [anchor= east] (0.95,3.96692) node[]{1200};
\draw [anchor= east] (0.95,4.52938) node[]{1400};
\draw [anchor= east] (0.95,5.09185) node[]{1600};
\draw (0.15,5.2) node[left,rotate=90] {Events / 4 GeV};
\draw [anchor=base west] (5.95,4.76001) node[]{Total contribution};
\definecolor{c}{rgb}{1,1,1};
\draw [c, fill=c] (5.1425,4.68229) -- (5.8075,4.68229) -- (5.8075,5.11751) -- (5.1425,5.11751);
\definecolor{c}{named}{natgreen};
\draw [c] (5.1425,4.8999) -- (5.8075,4.8999);
\definecolor{c}{rgb}{0,0,0};

\draw [anchor=base west] (5.95,4.13827) node[]{Box contribution};
\definecolor{c}{named}{natcomp};
\draw [c!30, fill=c!30] (5.1425,4.06055) -- (5.8075,4.06055) -- (5.8075,4.49577) -- (5.1425,4.49577);
\draw [c] (5.1425,4.27816) -- (5.8075,4.27816);
\definecolor{c}{rgb}{0,0,0};

\end{tikzpicture}
\end{tiny}
\end{sffamily}
\end{minipage}
\begin{minipage}[b]{.3\textwidth}
\caption{The distribution of invariant masses of Standard Model diphoton events as predicted by simulation. The box contribution gives just those events produced by the box diagram in fig.~\ref{boxdiag}. These are ATLAS datasets produced with pythia8 \cite{pythia}, and normalised to the luminosity of the data sample. \label{boxpart}}
\end{minipage}
\end{figure}
