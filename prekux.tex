\documentclass[a4paper,11pt,openany]{memoir}
\usepackage{amsmath}
%\usepackage[sc,osf]{mathpazo}
%\usepackage[garamond]{mathdesign}
\usepackage{fontspec}
\usepackage[math-style=TeX]{unicode-math}
\usepackage{xunicode}
\usepackage{xltxtra}
\usepackage{polyglossia}
\defaultfontfeatures{Mapping=tex-text}
\setmainfont[   Path=fonts/xits/,
                BoldFont={xits-bold.otf},
                ItalicFont={xits-italic.otf},
                BoldItalicFont={xits-bolditalic.otf},
                SmallCapsFont={../MasticSC-Regular.otf}
            ]{xits-regular.otf}
\setsansfont[   Path=fonts/frutiger/,
                Scale=MatchLowercase,
                BoldFont={FrutigerLTStd-Bold.otf},
                ItalicFont={FrutigerLTStd-Italic.otf},
                BoldItalicFont={FrutigerLTStd-BoldItalic.otf}
            ]{FrutigerLTStd-Roman.otf}
            
\setmathfont{xits-math.otf}
\setmathfont[version=roman,range=\mathcal,Path=fonts/]{latinmodern-math.otf}
%\setmathfont[range=\mathcal,Scale=MatchUppercase]{Lynda Cursive}
%\setmathfont[range=\mathup]{Garamond Premier Pro}
%\setmathfont[range=\mathit]{Garamond Premier Pro}
\setdefaultlanguage{danish}
\setotherlanguage[variant=british]{english}

\usepackage[usenames,dvipsnames,svgnames,table]{xcolor}
\usepackage{graphicx,epic,eepic}
\usepackage{tkz-graph,tkz-euclide}
\usetikzlibrary{calc,intersections,shapes.geometric,decorations.pathmorphing}
\usetikzlibrary{snakes,patterns,plotmarks,decorations.text}
\renewcommand*{\VertexSmallMinSize}{4pt}
\usepackage{tabulary}
\usepackage{pdfpages}
\usepackage{wrapfig}
\usepackage{pifont}
\usepackage{multirow}
\usepackage{colortbl}

%\renewcommand{\thefootnote}{\fnsymbol{footnote}}

\renewcommand\bibname{References}
\renewcommand{\(}{\begin{equation}}
\renewcommand{\)}{\end{equation}}
\newcommand{\vet}[1]{\underline{#1}}
\newcommand{\mtx}[1]{\underline{\underline{#1}}}
\newcommand{\ono}[1]{\frac{1}{#1}}
\newcommand{\half}{\frac{1}{2}}
\newcommand{\thehead}{}
\newcommand{\head}[1]{\renewcommand{\thehead}{#1}}
\newcommand{\theohead}{}
\newcommand{\ohead}[1]{\renewcommand{\theohead}{#1}}
\newcommand{\thetnote}{}
\newcommand{\tnote}[1]{\renewcommand{\thetnote}{#1}}
\newcommand{\bra}[1]{\left\langle #1 \right|}
\newcommand{\ket}[1]{\left| #1 \right\rangle}
\newcommand{\braket}[2]{\left\langle#1\middle|#2\right\rangle}
\newcommand{\obraket}[3]{\left\langle#1\middle|#2\middle|#3\right\rangle}
\newcommand{\emf}{\mathcal{E}}
\newcommand{\di}{\text{d}}
\newcommand{\atlas}{\textsc{atlas}}
\newcommand{\Atlas}{\textsc{Atlas}}

\newenvironment{infilsf}{
    \begin{sffamily}
    \setmathfont[range=\mathup/{num},
                 Scale=MatchLowercase,
                 Path=fonts/frutiger/,
                 ]{FrutigerLTStd-Roman.otf}
    \setmathfont[range=\mathit/{latin,Latin},
                 Scale=MatchLowercase,
                 Path=fonts/frutiger/,
                    ]{FrutigerLTStd-Italic.otf}
    \setmathfont[range=\mathit/{greek,Greek},
                 Scale=MatchLowercase,
                 Path=fonts/dejavu/,
                    ]{DejaVuSans-Oblique.ttf}
    \setmathfont[range=\mathup/{greek,Greek},
                 Scale=MatchLowercase,
                 Path=fonts/dejavu/,
                    ]{DejaVuSans.ttf}
%    \setmathfont[range=\text,
%                 Scale=MatchLowercase,
%                 Path=fonts/frutiger/,
%                    ]{FrutigerLTStd-Roman.otf}
}{
    \setmathfont{xits-math.otf}
    \setmathfont[range=\mathcal,Path=fonts/]{latinmodern-math.otf}
    \end{sffamily}
}

\newenvironment{new}{\color{Blue}}{}

\definecolor{kugray}{RGB}{102,102,102}
\definecolor{kugray1}{RGB}{52,52,52}
\definecolor{natgreen}{RGB}{70,116,60}
\definecolor{natgreen1}{HTML}{63875B}
\definecolor{natgreen2}{HTML}{859B81}
\definecolor{natcomp}{HTML}{86454C}
\definecolor{natcomp1}{HTML}{9D696E}
\definecolor{natcomp2}{HTML}{B39598}
\definecolor{natlcomp}{HTML}{5d323d}
\definecolor{natrcomp}{HTML}{52325d}
\definecolor{natyellow}{HTML}{8B6448}
\definecolor{natblue}{HTML}{524F81}
\definecolor{natscatb}{HTML}{3B8178}
\definecolor{natscatg}{HTML}{418B5C}
\definecolor{natscaty}{HTML}{7C8B41}
\definecolor{natscatr}{HTML}{81783C}

\makepagestyle{fancy}

\makepsmarks{fancy}{%
\nouppercaseheads
\createmark{chapter}{left}{nonumber}{}{}
\createmark{section}{right}{shownumber}{}{ \space}
\createplainmark{toc}{both}{\contentsname}
\createplainmark{lof}{both}{\listfigurename}
\createplainmark{lot}{both}{\listtablename}
\createplainmark{bib}{both}{\bibname}
\createplainmark{index}{both}{\indexname}
\createplainmark{glossary}{both}{\glossaryname}}

\makeoddhead{fancy}
   {}{}{}
\makeoddfoot{fancy}{\makebox[0pt][r]{\raisebox{15pt}[20pt]{\textcolor{natgreen}{\rule{1.1\spinemargin}{1pt}}}\makebox[0pt][l]{\raisebox{15pt}[20pt]{\textcolor{natgreen}{\rule{\paperwidth}{1pt}}}}}\textcolor{kugray}{\textsf{\rightmark}}}{}{\textcolor{kugray}{\textsf{\thepage}}}
\makeevenfoot{fancy}{\textcolor{kugray}{\textsf{\thepage}}}{}{\textcolor{kugray}{\textsf{\leftmark}}\makebox[0pt][l]{\raisebox{15pt}{\textcolor{natgreen}{\rule{1.1\spinemargin}{1pt}}}}\makebox[0pt][r]{\raisebox{15pt}{\textcolor{natgreen}{\rule{\paperwidth}{1pt}}}}}
\setlength{\footskip}{40pt}
\pagestyle{fancy}
\aliaspagestyle{chapter}{fancy}

\captionnamefont{\sffamily\color{natgreen}\bfseries} \captiontitlefont{\footnotesize} \captionstyle{\\}
\renewcommand*{\printchaptername}{}
\renewcommand*{\chapternamenum}{}
\renewcommand*{\afterchapternum}{}
\renewcommand{\chapnumfont}{\chaptitlefont\sffamily\HUGE}
\renewcommand{\printchapternum}{\chapnumfont \colorbox{natgreen}{\textcolor{white}{\hspace{.2em}\thechapter\hspace{.2em}}}\hspace{1em}}
\setsecheadstyle{\large\bfseries}
\setsubsecheadstyle{\bfseries}
\setsubsubsecheadstyle{}
\setsecnumformat{\textsf{\color{natgreen}\csname the#1\endcsname\quad}}
\maxsecnumdepth{subsubsection}
\renewcommand{\labelenumi}{\sffamily\bfseries\color{natgreen}\theenumi.}
\renewcommand{\labelitemi}{\color{natgreen}\ding{110}}
\renewcommand{\labelitemii}{\color{natgreen}\textbullet}
\setcounter{tocdepth}{2}

\setlength{\arrayrulewidth}{2pt}

\newsubfloat{figure}

\begin{hyphenrules}{danish}
\hyphenation{be-stem-mes}
\hyphenation{rest-klas-se-sæt-ning}
\end{hyphenrules}
\begin{hyphenrules}{english}
\hyphenation{Ham-il-ton}
\hyphenation{ATLAS}
\hyphenation{Atlas}
\hyphenation{atlas}
\hyphenation{CERN}
\hyphenation{i-den-ti-cal}
\hyphenation{pro-vid-ed}
\hyphenation{Calc-HEP}
\hyphenation{par-ticles}
\hyphenation{brems-strahl-ung}
\end{hyphenrules}
\usepackage[pdfusetitle]{hyperref}
\urlstyle{sf}
