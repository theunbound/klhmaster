\chapter{Analysis}\label{ch.an}

Recalling from chapter \ref{ch.theory} that we may express the number of predicted events in a given bin in the $M_{\gamma\gamma}$ as a second order polynomial in $\Lambda^4$, our aim here will be, first, to determine, for each bin, the coefficients of this polynomial. Second, using a maximum likelihood fit, to determine a most likely value of $\Lambda$, and a confidence interval on that value.

\section{Corrections to Monte Carlo samples}

In chapters~\ref{ch.mc} and~\ref{ch.ex}, we produced, among others, a sample of Monte Carlo events that gives the SM prediction for the distribution of events. Since we discovered above, that the \atlas{} $\gamma\gamma$ Monte Carlo sample gives a distribution of events that matches data well, we will compare these two samples to asses how well our SM prediction matches data. In doing so, we encounter a few problems.

First, it appears that the procedure which should correct for pileup in the detector simulation procedure has not functioned as intended. During detector simulation, the events produced for this thesis are assigned only a very limited range of values for the number of interactions per bunch crossing. Ordinarily, the pileup correction procedure expects a distribution of events per bunch crossing that covers the same range as found in data, and simply reweights each event, so that the shape of the distribution in MC is altered to resemble the one found in data. Since the distribution of numbers of interaction per bunch crossing available, the distribution found in data can not be recovered in MC.

Since this appears to be a technical glitch, we will substitute a reweighting in the number of reconstructed primary vertices---vertices considered to originate directly from interactions between protons---in each event. The distribution of numbers of primary vertices in the present MC data set is compared to the one found in the \atlas{} MC set in figure~\ref{pvnnone}.

\begin{figure}[htp]
\begin{minipage}[b]{.69\textwidth}
\begin{infilsf} \tiny
\begin{tikzpicture}[x=.092\textwidth,y=.092\textwidth]
\pgfdeclareplotmark{cross} {
\pgfpathmoveto{\pgfpoint{-0.3\pgfplotmarksize}{\pgfplotmarksize}}
\pgfpathlineto{\pgfpoint{+0.3\pgfplotmarksize}{\pgfplotmarksize}}
\pgfpathlineto{\pgfpoint{+0.3\pgfplotmarksize}{0.3\pgfplotmarksize}}
\pgfpathlineto{\pgfpoint{+1\pgfplotmarksize}{0.3\pgfplotmarksize}}
\pgfpathlineto{\pgfpoint{+1\pgfplotmarksize}{-0.3\pgfplotmarksize}}
\pgfpathlineto{\pgfpoint{+0.3\pgfplotmarksize}{-0.3\pgfplotmarksize}}
\pgfpathlineto{\pgfpoint{+0.3\pgfplotmarksize}{-1.\pgfplotmarksize}}
\pgfpathlineto{\pgfpoint{-0.3\pgfplotmarksize}{-1.\pgfplotmarksize}}
\pgfpathlineto{\pgfpoint{-0.3\pgfplotmarksize}{-0.3\pgfplotmarksize}}
\pgfpathlineto{\pgfpoint{-1.\pgfplotmarksize}{-0.3\pgfplotmarksize}}
\pgfpathlineto{\pgfpoint{-1.\pgfplotmarksize}{0.3\pgfplotmarksize}}
\pgfpathlineto{\pgfpoint{-0.3\pgfplotmarksize}{0.3\pgfplotmarksize}}
\pgfpathclose
\pgfusepathqstroke
}
\pgfdeclareplotmark{cross*} {
\pgfpathmoveto{\pgfpoint{-0.3\pgfplotmarksize}{\pgfplotmarksize}}
\pgfpathlineto{\pgfpoint{+0.3\pgfplotmarksize}{\pgfplotmarksize}}
\pgfpathlineto{\pgfpoint{+0.3\pgfplotmarksize}{0.3\pgfplotmarksize}}
\pgfpathlineto{\pgfpoint{+1\pgfplotmarksize}{0.3\pgfplotmarksize}}
\pgfpathlineto{\pgfpoint{+1\pgfplotmarksize}{-0.3\pgfplotmarksize}}
\pgfpathlineto{\pgfpoint{+0.3\pgfplotmarksize}{-0.3\pgfplotmarksize}}
\pgfpathlineto{\pgfpoint{+0.3\pgfplotmarksize}{-1.\pgfplotmarksize}}
\pgfpathlineto{\pgfpoint{-0.3\pgfplotmarksize}{-1.\pgfplotmarksize}}
\pgfpathlineto{\pgfpoint{-0.3\pgfplotmarksize}{-0.3\pgfplotmarksize}}
\pgfpathlineto{\pgfpoint{-1.\pgfplotmarksize}{-0.3\pgfplotmarksize}}
\pgfpathlineto{\pgfpoint{-1.\pgfplotmarksize}{0.3\pgfplotmarksize}}
\pgfpathlineto{\pgfpoint{-0.3\pgfplotmarksize}{0.3\pgfplotmarksize}}
\pgfpathclose
\pgfusepathqfillstroke
}
\pgfdeclareplotmark{newstar} {
\pgfpathmoveto{\pgfqpoint{0pt}{\pgfplotmarksize}}
\pgfpathlineto{\pgfqpointpolar{44}{0.5\pgfplotmarksize}}
\pgfpathlineto{\pgfqpointpolar{18}{\pgfplotmarksize}}
\pgfpathlineto{\pgfqpointpolar{-20}{0.5\pgfplotmarksize}}
\pgfpathlineto{\pgfqpointpolar{-54}{\pgfplotmarksize}}
\pgfpathlineto{\pgfqpointpolar{-90}{0.5\pgfplotmarksize}}
\pgfpathlineto{\pgfqpointpolar{234}{\pgfplotmarksize}}
\pgfpathlineto{\pgfqpointpolar{198}{0.5\pgfplotmarksize}}
\pgfpathlineto{\pgfqpointpolar{162}{\pgfplotmarksize}}
\pgfpathlineto{\pgfqpointpolar{134}{0.5\pgfplotmarksize}}
\pgfpathclose
\pgfusepathqstroke
}
\pgfdeclareplotmark{newstar*} {
\pgfpathmoveto{\pgfqpoint{0pt}{\pgfplotmarksize}}
\pgfpathlineto{\pgfqpointpolar{44}{0.5\pgfplotmarksize}}
\pgfpathlineto{\pgfqpointpolar{18}{\pgfplotmarksize}}
\pgfpathlineto{\pgfqpointpolar{-20}{0.5\pgfplotmarksize}}
\pgfpathlineto{\pgfqpointpolar{-54}{\pgfplotmarksize}}
\pgfpathlineto{\pgfqpointpolar{-90}{0.5\pgfplotmarksize}}
\pgfpathlineto{\pgfqpointpolar{234}{\pgfplotmarksize}}
\pgfpathlineto{\pgfqpointpolar{198}{0.5\pgfplotmarksize}}
\pgfpathlineto{\pgfqpointpolar{162}{\pgfplotmarksize}}
\pgfpathlineto{\pgfqpointpolar{134}{0.5\pgfplotmarksize}}
\pgfpathclose
\pgfusepathqfillstroke
}
\definecolor{c}{rgb}{1,1,1};
\draw [color=c, fill=c] (0,0) rectangle (10,6.80516);
\draw [color=c, fill=c] (1,0.680516) rectangle (9.95,6.73711);
\definecolor{c}{rgb}{0,0,0};
\draw [c] (1,0.680516) -- (1,6.73711) -- (9.95,6.73711) -- (9.95,0.680516) -- (1,0.680516);
\definecolor{c}{rgb}{1,1,1};
\draw [color=c, fill=c] (1,0.680516) rectangle (9.95,6.73711);
\definecolor{c}{rgb}{0,0,0};
\draw [c] (1,0.680516) -- (1,6.73711) -- (9.95,6.73711) -- (9.95,0.680516) -- (1,0.680516);
\colorlet{c}{natcomp!70};
\draw [c] (2.01048,0.686784) -- (2.01048,0.695341);
\draw [c] (2.01048,0.695341) -- (2.01048,0.703897);
\draw [c] (1.86613,0.695341) -- (2.01048,0.695341);
\draw [c] (2.01048,0.695341) -- (2.15484,0.695341);
\definecolor{c}{rgb}{0,0,0};
\colorlet{c}{natcomp!70};
\draw [c] (2.29919,0.680526) -- (2.29919,0.680532);
\draw [c] (2.29919,0.680532) -- (2.29919,0.680538);
\draw [c] (2.15484,0.680532) -- (2.29919,0.680532);
\draw [c] (2.29919,0.680532) -- (2.44355,0.680532);
\definecolor{c}{rgb}{0,0,0};
\colorlet{c}{natcomp!70};
\draw [c] (2.5879,0.752927) -- (2.5879,0.77446);
\draw [c] (2.5879,0.77446) -- (2.5879,0.795993);
\draw [c] (2.44355,0.77446) -- (2.5879,0.77446);
\draw [c] (2.5879,0.77446) -- (2.73226,0.77446);
\definecolor{c}{rgb}{0,0,0};
\colorlet{c}{natcomp!70};
\draw [c] (2.87661,0.943483) -- (2.87661,0.982067);
\draw [c] (2.87661,0.982067) -- (2.87661,1.02065);
\draw [c] (2.73226,0.982067) -- (2.87661,0.982067);
\draw [c] (2.87661,0.982067) -- (3.02097,0.982067);
\definecolor{c}{rgb}{0,0,0};
\colorlet{c}{natcomp!70};
\draw [c] (3.16532,1.29053) -- (3.16532,1.34793);
\draw [c] (3.16532,1.34793) -- (3.16532,1.40533);
\draw [c] (3.02097,1.34793) -- (3.16532,1.34793);
\draw [c] (3.16532,1.34793) -- (3.30968,1.34793);
\definecolor{c}{rgb}{0,0,0};
\colorlet{c}{natcomp!70};
\draw [c] (3.45403,1.93426) -- (3.45403,2.01543);
\draw [c] (3.45403,2.01543) -- (3.45403,2.09661);
\draw [c] (3.30968,2.01543) -- (3.45403,2.01543);
\draw [c] (3.45403,2.01543) -- (3.59839,2.01543);
\definecolor{c}{rgb}{0,0,0};
\colorlet{c}{natcomp!70};
\draw [c] (3.74274,2.71861) -- (3.74274,2.82141);
\draw [c] (3.74274,2.82141) -- (3.74274,2.9242);
\draw [c] (3.59839,2.82141) -- (3.74274,2.82141);
\draw [c] (3.74274,2.82141) -- (3.8871,2.82141);
\definecolor{c}{rgb}{0,0,0};
\colorlet{c}{natcomp!70};
\draw [c] (4.03145,4.11719) -- (4.03145,4.24993);
\draw [c] (4.03145,4.24993) -- (4.03145,4.38267);
\draw [c] (3.8871,4.24993) -- (4.03145,4.24993);
\draw [c] (4.03145,4.24993) -- (4.17581,4.24993);
\definecolor{c}{rgb}{0,0,0};
\colorlet{c}{natcomp!70};
\draw [c] (4.32016,4.63729) -- (4.32016,4.77952);
\draw [c] (4.32016,4.77952) -- (4.32016,4.92176);
\draw [c] (4.17581,4.77952) -- (4.32016,4.77952);
\draw [c] (4.32016,4.77952) -- (4.46452,4.77952);
\definecolor{c}{rgb}{0,0,0};
\colorlet{c}{natcomp!70};
\draw [c] (4.60887,5.6146) -- (4.60887,5.77315);
\draw [c] (4.60887,5.77315) -- (4.60887,5.93169);
\draw [c] (4.46452,5.77315) -- (4.60887,5.77315);
\draw [c] (4.60887,5.77315) -- (4.75323,5.77315);
\definecolor{c}{rgb}{0,0,0};
\colorlet{c}{natcomp!70};
\draw [c] (4.89758,6.11613) -- (4.89758,6.28241);
\draw [c] (4.89758,6.28241) -- (4.89758,6.4487);
\draw [c] (4.75323,6.28241) -- (4.89758,6.28241);
\draw [c] (4.89758,6.28241) -- (5.04194,6.28241);
\definecolor{c}{rgb}{0,0,0};
\colorlet{c}{natcomp!70};
\draw [c] (5.18629,5.576) -- (5.18629,5.73393);
\draw [c] (5.18629,5.73393) -- (5.18629,5.89186);
\draw [c] (5.04194,5.73393) -- (5.18629,5.73393);
\draw [c] (5.18629,5.73393) -- (5.33065,5.73393);
\definecolor{c}{rgb}{0,0,0};
\colorlet{c}{natcomp!70};
\draw [c] (5.475,5.47333) -- (5.475,5.62963);
\draw [c] (5.475,5.62963) -- (5.475,5.78593);
\draw [c] (5.33065,5.62963) -- (5.475,5.62963);
\draw [c] (5.475,5.62963) -- (5.61935,5.62963);
\definecolor{c}{rgb}{0,0,0};
\colorlet{c}{natcomp!70};
\draw [c] (5.76371,4.43808) -- (5.76371,4.57675);
\draw [c] (5.76371,4.57675) -- (5.76371,4.71543);
\draw [c] (5.61935,4.57675) -- (5.76371,4.57675);
\draw [c] (5.76371,4.57675) -- (5.90806,4.57675);
\definecolor{c}{rgb}{0,0,0};
\colorlet{c}{natcomp!70};
\draw [c] (6.05242,3.40517) -- (6.05242,3.52363);
\draw [c] (6.05242,3.52363) -- (6.05242,3.64209);
\draw [c] (5.90806,3.52363) -- (6.05242,3.52363);
\draw [c] (6.05242,3.52363) -- (6.19677,3.52363);
\definecolor{c}{rgb}{0,0,0};
\colorlet{c}{natcomp!70};
\draw [c] (6.34113,2.64159) -- (6.34113,2.74247);
\draw [c] (6.34113,2.74247) -- (6.34113,2.84335);
\draw [c] (6.19677,2.74247) -- (6.34113,2.74247);
\draw [c] (6.34113,2.74247) -- (6.48548,2.74247);
\definecolor{c}{rgb}{0,0,0};
\colorlet{c}{natcomp!70};
\draw [c] (6.62984,2.05919) -- (6.62984,2.14419);
\draw [c] (6.62984,2.14419) -- (6.62984,2.22918);
\draw [c] (6.48548,2.14419) -- (6.62984,2.14419);
\draw [c] (6.62984,2.14419) -- (6.77419,2.14419);
\definecolor{c}{rgb}{0,0,0};
\colorlet{c}{natcomp!70};
\draw [c] (6.91855,1.74752) -- (6.91855,1.8226);
\draw [c] (6.91855,1.8226) -- (6.91855,1.89768);
\draw [c] (6.77419,1.8226) -- (6.91855,1.8226);
\draw [c] (6.91855,1.8226) -- (7.0629,1.8226);
\definecolor{c}{rgb}{0,0,0};
\colorlet{c}{natcomp!70};
\draw [c] (7.20726,1.16326) -- (7.20726,1.2146);
\draw [c] (7.20726,1.2146) -- (7.20726,1.26594);
\draw [c] (7.0629,1.2146) -- (7.20726,1.2146);
\draw [c] (7.20726,1.2146) -- (7.35161,1.2146);
\definecolor{c}{rgb}{0,0,0};
\colorlet{c}{natcomp!70};
\draw [c] (7.49597,0.925145) -- (7.49597,0.962442);
\draw [c] (7.49597,0.962442) -- (7.49597,0.999739);
\draw [c] (7.35161,0.962442) -- (7.49597,0.962442);
\draw [c] (7.49597,0.962442) -- (7.64032,0.962442);
\definecolor{c}{rgb}{0,0,0};
\colorlet{c}{natcomp!70};
\draw [c] (7.78468,0.847054) -- (7.78468,0.878298);
\draw [c] (7.78468,0.878298) -- (7.78468,0.909542);
\draw [c] (7.64032,0.878298) -- (7.78468,0.878298);
\draw [c] (7.78468,0.878298) -- (7.92903,0.878298);
\definecolor{c}{rgb}{0,0,0};
\colorlet{c}{natcomp!70};
\draw [c] (8.07339,0.7399) -- (8.07339,0.759661);
\draw [c] (8.07339,0.759661) -- (8.07339,0.779421);
\draw [c] (7.92903,0.759661) -- (8.07339,0.759661);
\draw [c] (8.07339,0.759661) -- (8.21774,0.759661);
\definecolor{c}{rgb}{0,0,0};
\colorlet{c}{natcomp!70};
\draw [c] (8.3621,0.710206) -- (8.3621,0.725026);
\draw [c] (8.3621,0.725026) -- (8.3621,0.739846);
\draw [c] (8.21774,0.725026) -- (8.3621,0.725026);
\draw [c] (8.3621,0.725026) -- (8.50645,0.725026);
\definecolor{c}{rgb}{0,0,0};
\colorlet{c}{natcomp!70};
\draw [c] (8.65081,0.686798) -- (8.65081,0.695355);
\draw [c] (8.65081,0.695355) -- (8.65081,0.703911);
\draw [c] (8.50645,0.695355) -- (8.65081,0.695355);
\draw [c] (8.65081,0.695355) -- (8.79516,0.695355);
\definecolor{c}{rgb}{0,0,0};
\colorlet{c}{natcomp!70};
\draw [c] (8.93952,0.683424) -- (8.93952,0.69041);
\draw [c] (8.93952,0.69041) -- (8.93952,0.697396);
\draw [c] (8.79516,0.69041) -- (8.93952,0.69041);
\draw [c] (8.93952,0.69041) -- (9.08387,0.69041);
\definecolor{c}{rgb}{0,0,0};
\colorlet{c}{natcomp!70};
\draw [c] (9.22823,0.690398) -- (9.22823,0.700279);
\draw [c] (9.22823,0.700279) -- (9.22823,0.710159);
\draw [c] (9.08387,0.700279) -- (9.22823,0.700279);
\draw [c] (9.22823,0.700279) -- (9.37258,0.700279);
\definecolor{c}{rgb}{0,0,0};
\colorlet{c}{natcomp!70};
\draw [c] (9.80564,0.680517) -- (9.80564,0.68052);
\draw [c] (9.80564,0.68052) -- (9.80564,0.680524);
\draw [c] (9.66129,0.68052) -- (9.80564,0.68052);
\draw [c] (9.80564,0.68052) -- (9.95,0.68052);
\definecolor{c}{rgb}{0,0,0};
\draw [c] (1,0.680516) -- (9.95,0.680516);
\draw [anchor= east] (9.95,0.108883) node[color=c, rotate=0]{Number of primary vertices};
\draw [c] (1.14435,0.863234) -- (1.14435,0.680516);
\draw [c] (1.43306,0.771875) -- (1.43306,0.680516);
\draw [c] (1.72177,0.771875) -- (1.72177,0.680516);
\draw [c] (2.01048,0.771875) -- (2.01048,0.680516);
\draw [c] (2.29919,0.771875) -- (2.29919,0.680516);
\draw [c] (2.5879,0.863234) -- (2.5879,0.680516);
\draw [c] (2.87661,0.771875) -- (2.87661,0.680516);
\draw [c] (3.16532,0.771875) -- (3.16532,0.680516);
\draw [c] (3.45403,0.771875) -- (3.45403,0.680516);
\draw [c] (3.74274,0.771875) -- (3.74274,0.680516);
\draw [c] (4.03145,0.863234) -- (4.03145,0.680516);
\draw [c] (4.32016,0.771875) -- (4.32016,0.680516);
\draw [c] (4.60887,0.771875) -- (4.60887,0.680516);
\draw [c] (4.89758,0.771875) -- (4.89758,0.680516);
\draw [c] (5.18629,0.771875) -- (5.18629,0.680516);
\draw [c] (5.475,0.863234) -- (5.475,0.680516);
\draw [c] (5.76371,0.771875) -- (5.76371,0.680516);
\draw [c] (6.05242,0.771875) -- (6.05242,0.680516);
\draw [c] (6.34113,0.771875) -- (6.34113,0.680516);
\draw [c] (6.62984,0.771875) -- (6.62984,0.680516);
\draw [c] (6.91855,0.863234) -- (6.91855,0.680516);
\draw [c] (7.20726,0.771875) -- (7.20726,0.680516);
\draw [c] (7.49597,0.771875) -- (7.49597,0.680516);
\draw [c] (7.78468,0.771875) -- (7.78468,0.680516);
\draw [c] (8.07339,0.771875) -- (8.07339,0.680516);
\draw [c] (8.3621,0.863234) -- (8.3621,0.680516);
\draw [c] (8.65081,0.771875) -- (8.65081,0.680516);
\draw [c] (8.93952,0.771875) -- (8.93952,0.680516);
\draw [c] (9.22823,0.771875) -- (9.22823,0.680516);
\draw [c] (9.51694,0.771875) -- (9.51694,0.680516);
\draw [c] (9.80564,0.863234) -- (9.80564,0.680516);
\draw [c] (1.14435,0.863234) -- (1.14435,0.680516);
\draw [c] (9.80564,0.863234) -- (9.80564,0.680516);
\draw [anchor=base] (1.14435,0.353868) node[color=c, rotate=0]{0};
\draw [anchor=base] (2.5879,0.353868) node[color=c, rotate=0]{5};
\draw [anchor=base] (4.03145,0.353868) node[color=c, rotate=0]{10};
\draw [anchor=base] (5.475,0.353868) node[color=c, rotate=0]{15};
\draw [anchor=base] (6.91855,0.353868) node[color=c, rotate=0]{20};
\draw [anchor=base] (8.3621,0.353868) node[color=c, rotate=0]{25};
\draw [anchor=base] (9.80564,0.353868) node[color=c, rotate=0]{30};
\draw [c] (1,0.680516) -- (1,6.73711);
\draw [anchor= east] (-0.12,6.73711) node[color=c, rotate=90]{Normalised number of events};
\draw [c] (1.267,0.680516) -- (1,0.680516);
\draw [c] (1.1335,0.878524) -- (1,0.878524);
\draw [c] (1.1335,1.07653) -- (1,1.07653);
\draw [c] (1.1335,1.27454) -- (1,1.27454);
\draw [c] (1.1335,1.47255) -- (1,1.47255);
\draw [c] (1.267,1.67056) -- (1,1.67056);
\draw [c] (1.1335,1.86857) -- (1,1.86857);
\draw [c] (1.1335,2.06657) -- (1,2.06657);
\draw [c] (1.1335,2.26458) -- (1,2.26458);
\draw [c] (1.1335,2.46259) -- (1,2.46259);
\draw [c] (1.267,2.6606) -- (1,2.6606);
\draw [c] (1.1335,2.85861) -- (1,2.85861);
\draw [c] (1.1335,3.05662) -- (1,3.05662);
\draw [c] (1.1335,3.25462) -- (1,3.25462);
\draw [c] (1.1335,3.45263) -- (1,3.45263);
\draw [c] (1.267,3.65064) -- (1,3.65064);
\draw [c] (1.1335,3.84865) -- (1,3.84865);
\draw [c] (1.1335,4.04666) -- (1,4.04666);
\draw [c] (1.1335,4.24467) -- (1,4.24467);
\draw [c] (1.1335,4.44267) -- (1,4.44267);
\draw [c] (1.267,4.64068) -- (1,4.64068);
\draw [c] (1.1335,4.83869) -- (1,4.83869);
\draw [c] (1.1335,5.0367) -- (1,5.0367);
\draw [c] (1.1335,5.23471) -- (1,5.23471);
\draw [c] (1.1335,5.43272) -- (1,5.43272);
\draw [c] (1.267,5.63072) -- (1,5.63072);
\draw [c] (1.1335,5.82873) -- (1,5.82873);
\draw [c] (1.1335,6.02674) -- (1,6.02674);
\draw [c] (1.1335,6.22475) -- (1,6.22475);
\draw [c] (1.1335,6.42276) -- (1,6.42276);
\draw [c] (1.267,6.62077) -- (1,6.62077);
\draw [c] (1.267,6.62077) -- (1,6.62077);
\draw [anchor= east] (0.95,0.680516) node[color=c, rotate=0]{0};
\draw [anchor= east] (0.95,1.67056) node[color=c, rotate=0]{500};
\draw [anchor= east] (0.95,2.6606) node[color=c, rotate=0]{1000};
\draw [anchor= east] (0.95,3.65064) node[color=c, rotate=0]{1500};
\draw [anchor= east] (0.95,4.64068) node[color=c, rotate=0]{2000};
\draw [anchor= east] (0.95,5.63072) node[color=c, rotate=0]{2500};
\draw [anchor= east] (0.95,6.62077) node[color=c, rotate=0]{3000};
\colorlet{c}{natgreen};
\draw [c] (1.72177,0.680523) -- (1.72177,0.681242);
\draw [c] (1.72177,0.681242) -- (1.72177,0.68196);
\draw [c] (1.57742,0.681242) -- (1.72177,0.681242);
\draw [c] (1.72177,0.681242) -- (1.86613,0.681242);
\definecolor{c}{rgb}{0,0,0};
\colorlet{c}{natgreen};
\draw [c] (2.01048,0.689073) -- (2.01048,0.696007);
\draw [c] (2.01048,0.696007) -- (2.01048,0.702941);
\draw [c] (1.86613,0.696007) -- (2.01048,0.696007);
\draw [c] (2.01048,0.696007) -- (2.15484,0.696007);
\definecolor{c}{rgb}{0,0,0};
\colorlet{c}{natgreen};
\draw [c] (2.29919,0.716299) -- (2.29919,0.731686);
\draw [c] (2.29919,0.731686) -- (2.29919,0.747074);
\draw [c] (2.15484,0.731686) -- (2.29919,0.731686);
\draw [c] (2.29919,0.731686) -- (2.44355,0.731686);
\definecolor{c}{rgb}{0,0,0};
\colorlet{c}{natgreen};
\draw [c] (2.5879,0.871864) -- (2.5879,0.902068);
\draw [c] (2.5879,0.902068) -- (2.5879,0.932273);
\draw [c] (2.44355,0.902068) -- (2.5879,0.902068);
\draw [c] (2.5879,0.902068) -- (2.73226,0.902068);
\definecolor{c}{rgb}{0,0,0};
\colorlet{c}{natgreen};
\draw [c] (2.87661,1.14594) -- (2.87661,1.19239);
\draw [c] (2.87661,1.19239) -- (2.87661,1.23883);
\draw [c] (2.73226,1.19239) -- (2.87661,1.19239);
\draw [c] (2.87661,1.19239) -- (3.02097,1.19239);
\definecolor{c}{rgb}{0,0,0};
\colorlet{c}{natgreen};
\draw [c] (3.16532,1.64654) -- (3.16532,1.715);
\draw [c] (3.16532,1.715) -- (3.16532,1.78347);
\draw [c] (3.02097,1.715) -- (3.16532,1.715);
\draw [c] (3.16532,1.715) -- (3.30968,1.715);
\definecolor{c}{rgb}{0,0,0};
\colorlet{c}{natgreen};
\draw [c] (3.45403,2.17071) -- (3.45403,2.25603);
\draw [c] (3.45403,2.25603) -- (3.45403,2.34135);
\draw [c] (3.30968,2.25603) -- (3.45403,2.25603);
\draw [c] (3.45403,2.25603) -- (3.59839,2.25603);
\definecolor{c}{rgb}{0,0,0};
\colorlet{c}{natgreen};
\draw [c] (3.74274,3.01706) -- (3.74274,3.12356);
\draw [c] (3.74274,3.12356) -- (3.74274,3.23006);
\draw [c] (3.59839,3.12356) -- (3.74274,3.12356);
\draw [c] (3.74274,3.12356) -- (3.8871,3.12356);
\definecolor{c}{rgb}{0,0,0};
\colorlet{c}{natgreen};
\draw [c] (4.03145,3.62689) -- (4.03145,3.74661);
\draw [c] (4.03145,3.74661) -- (4.03145,3.86633);
\draw [c] (3.8871,3.74661) -- (4.03145,3.74661);
\draw [c] (4.03145,3.74661) -- (4.17581,3.74661);
\definecolor{c}{rgb}{0,0,0};
\colorlet{c}{natgreen};
\draw [c] (4.32016,4.27462) -- (4.32016,4.40545);
\draw [c] (4.32016,4.40545) -- (4.32016,4.53628);
\draw [c] (4.17581,4.40545) -- (4.32016,4.40545);
\draw [c] (4.32016,4.40545) -- (4.46452,4.40545);
\definecolor{c}{rgb}{0,0,0};
\colorlet{c}{natgreen};
\draw [c] (4.60887,4.55969) -- (4.60887,4.69362);
\draw [c] (4.60887,4.69362) -- (4.60887,4.82755);
\draw [c] (4.46452,4.69362) -- (4.60887,4.69362);
\draw [c] (4.60887,4.69362) -- (4.75323,4.69362);
\definecolor{c}{rgb}{0,0,0};
\colorlet{c}{natgreen};
\draw [c] (4.89758,4.53815) -- (4.89758,4.66935);
\draw [c] (4.89758,4.66935) -- (4.89758,4.80056);
\draw [c] (4.75323,4.66935) -- (4.89758,4.66935);
\draw [c] (4.89758,4.66935) -- (5.04194,4.66935);
\definecolor{c}{rgb}{0,0,0};
\colorlet{c}{natgreen};
\draw [c] (5.18629,4.99176) -- (5.18629,5.12865);
\draw [c] (5.18629,5.12865) -- (5.18629,5.26553);
\draw [c] (5.04194,5.12865) -- (5.18629,5.12865);
\draw [c] (5.18629,5.12865) -- (5.33065,5.12865);
\definecolor{c}{rgb}{0,0,0};
\colorlet{c}{natgreen};
\draw [c] (5.475,4.29737) -- (5.475,4.41993);
\draw [c] (5.475,4.41993) -- (5.475,4.54248);
\draw [c] (5.33065,4.41993) -- (5.475,4.41993);
\draw [c] (5.475,4.41993) -- (5.61935,4.41993);
\definecolor{c}{rgb}{0,0,0};
\colorlet{c}{natgreen};
\draw [c] (5.76371,4.12649) -- (5.76371,4.24387);
\draw [c] (5.76371,4.24387) -- (5.76371,4.36126);
\draw [c] (5.61935,4.24387) -- (5.76371,4.24387);
\draw [c] (5.76371,4.24387) -- (5.90806,4.24387);
\definecolor{c}{rgb}{0,0,0};
\colorlet{c}{natgreen};
\draw [c] (6.05242,3.79458) -- (6.05242,3.90367);
\draw [c] (6.05242,3.90367) -- (6.05242,4.01276);
\draw [c] (5.90806,3.90367) -- (6.05242,3.90367);
\draw [c] (6.05242,3.90367) -- (6.19677,3.90367);
\definecolor{c}{rgb}{0,0,0};
\colorlet{c}{natgreen};
\draw [c] (6.34113,3.46262) -- (6.34113,3.56305);
\draw [c] (6.34113,3.56305) -- (6.34113,3.66348);
\draw [c] (6.19677,3.56305) -- (6.34113,3.56305);
\draw [c] (6.34113,3.56305) -- (6.48548,3.56305);
\definecolor{c}{rgb}{0,0,0};
\colorlet{c}{natgreen};
\draw [c] (6.62984,2.62966) -- (6.62984,2.71154);
\draw [c] (6.62984,2.71154) -- (6.62984,2.79342);
\draw [c] (6.48548,2.71154) -- (6.62984,2.71154);
\draw [c] (6.62984,2.71154) -- (6.77419,2.71154);
\definecolor{c}{rgb}{0,0,0};
\colorlet{c}{natgreen};
\draw [c] (6.91855,2.28926) -- (6.91855,2.36189);
\draw [c] (6.91855,2.36189) -- (6.91855,2.43452);
\draw [c] (6.77419,2.36189) -- (6.91855,2.36189);
\draw [c] (6.91855,2.36189) -- (7.0629,2.36189);
\definecolor{c}{rgb}{0,0,0};
\colorlet{c}{natgreen};
\draw [c] (7.20726,1.66134) -- (7.20726,1.71656);
\draw [c] (7.20726,1.71656) -- (7.20726,1.77178);
\draw [c] (7.0629,1.71656) -- (7.20726,1.71656);
\draw [c] (7.20726,1.71656) -- (7.35161,1.71656);
\definecolor{c}{rgb}{0,0,0};
\colorlet{c}{natgreen};
\draw [c] (7.49597,1.50108) -- (7.49597,1.55148);
\draw [c] (7.49597,1.55148) -- (7.49597,1.60188);
\draw [c] (7.35161,1.55148) -- (7.49597,1.55148);
\draw [c] (7.49597,1.55148) -- (7.64032,1.55148);
\definecolor{c}{rgb}{0,0,0};
\colorlet{c}{natgreen};
\draw [c] (7.78468,1.21206) -- (7.78468,1.25203);
\draw [c] (7.78468,1.25203) -- (7.78468,1.292);
\draw [c] (7.64032,1.25203) -- (7.78468,1.25203);
\draw [c] (7.78468,1.25203) -- (7.92903,1.25203);
\definecolor{c}{rgb}{0,0,0};
\colorlet{c}{natgreen};
\draw [c] (8.07339,1.01304) -- (8.07339,1.04329);
\draw [c] (8.07339,1.04329) -- (8.07339,1.07354);
\draw [c] (7.92903,1.04329) -- (8.07339,1.04329);
\draw [c] (8.07339,1.04329) -- (8.21774,1.04329);
\definecolor{c}{rgb}{0,0,0};
\colorlet{c}{natgreen};
\draw [c] (8.3621,0.847918) -- (8.3621,0.869254);
\draw [c] (8.3621,0.869254) -- (8.3621,0.890591);
\draw [c] (8.21774,0.869254) -- (8.3621,0.869254);
\draw [c] (8.3621,0.869254) -- (8.50645,0.869254);
\definecolor{c}{rgb}{0,0,0};
\colorlet{c}{natgreen};
\draw [c] (8.65081,0.787574) -- (8.65081,0.80412);
\draw [c] (8.65081,0.80412) -- (8.65081,0.820665);
\draw [c] (8.50645,0.80412) -- (8.65081,0.80412);
\draw [c] (8.65081,0.80412) -- (8.79516,0.80412);
\definecolor{c}{rgb}{0,0,0};
\colorlet{c}{natgreen};
\draw [c] (8.93952,0.739471) -- (8.93952,0.751966);
\draw [c] (8.93952,0.751966) -- (8.93952,0.76446);
\draw [c] (8.79516,0.751966) -- (8.93952,0.751966);
\draw [c] (8.93952,0.751966) -- (9.08387,0.751966);
\definecolor{c}{rgb}{0,0,0};
\colorlet{c}{natgreen};
\draw [c] (9.22823,0.713252) -- (9.22823,0.722645);
\draw [c] (9.22823,0.722645) -- (9.22823,0.732039);
\draw [c] (9.08387,0.722645) -- (9.22823,0.722645);
\draw [c] (9.22823,0.722645) -- (9.37258,0.722645);
\definecolor{c}{rgb}{0,0,0};
\colorlet{c}{natgreen};
\draw [c] (9.51694,0.695598) -- (9.51694,0.701995);
\draw [c] (9.51694,0.701995) -- (9.51694,0.708392);
\draw [c] (9.37258,0.701995) -- (9.51694,0.701995);
\draw [c] (9.51694,0.701995) -- (9.66129,0.701995);
\definecolor{c}{rgb}{0,0,0};
\colorlet{c}{natgreen};
\draw [c] (9.80564,0.682123) -- (9.80564,0.68405);
\draw [c] (9.80564,0.68405) -- (9.80564,0.685978);
\draw [c] (9.66129,0.68405) -- (9.80564,0.68405);
\draw [c] (9.80564,0.68405) -- (9.95,0.68405);
\definecolor{c}{rgb}{0,0,0};
\draw [anchor=base west] (6.96633,6.17962) node[color=c, rotate=0]{ATLAS MC};
\colorlet{c}{natgreen};
\draw [c] (6.13521,6.27149) -- (6.81966,6.27149);
\draw [c] (6.47744,6.149) -- (6.47744,6.39398);
\definecolor{c}{rgb}{0,0,0};
\draw [anchor=base west] (6.96633,5.77131) node[color=c, rotate=0]{Our MC};
\colorlet{c}{natcomp!70};
\draw [c] (6.13521,5.86318) -- (6.81966,5.86318);
\draw [c] (6.47744,5.74069) -- (6.47744,5.98567);
\end{tikzpicture}

\end{infilsf}
\end{minipage}
\begin{minipage}[b]{.3\textwidth}
\caption{The distribution of the number of reconstructed primary vertices in the \atlas{} $\gamma\gamma$ MC set and in the CalcHEP MC set produced for this thesis, normalised to the same number of events.}\label{pvnnone}
\end{minipage}
\end{figure}


The second issue we encounter is in the distribution of $E_T^\text{iso}$, which is much broader in the CalcHEP MC set than in the \atlas{} one, as illustrated in figure~\ref{etpv}. Assuming that the distribution in the CalcHEP sample is just the one in the \atlas{} sample, but broadened and shifted slightly, we develop a mapping function to reverse that effect: 
\(E_{T,\text{mapped}}^\text{iso}=\frac{E_T^\text{iso}}{4.574}-0.020\text{ [GeV].}\)
The result of applying this function is shown in fig.~\ref{etmap}.

\begin{figure}[htp]
\begin{minipage}[b]{.49\textwidth}
\begin{infilsf} \tiny 
\begin{tikzpicture}[x=.092\textwidth,y=.092\textwidth]
\pgfdeclareplotmark{cross} {
\pgfpathmoveto{\pgfpoint{-0.3\pgfplotmarksize}{\pgfplotmarksize}}
\pgfpathlineto{\pgfpoint{+0.3\pgfplotmarksize}{\pgfplotmarksize}}
\pgfpathlineto{\pgfpoint{+0.3\pgfplotmarksize}{0.3\pgfplotmarksize}}
\pgfpathlineto{\pgfpoint{+1\pgfplotmarksize}{0.3\pgfplotmarksize}}
\pgfpathlineto{\pgfpoint{+1\pgfplotmarksize}{-0.3\pgfplotmarksize}}
\pgfpathlineto{\pgfpoint{+0.3\pgfplotmarksize}{-0.3\pgfplotmarksize}}
\pgfpathlineto{\pgfpoint{+0.3\pgfplotmarksize}{-1.\pgfplotmarksize}}
\pgfpathlineto{\pgfpoint{-0.3\pgfplotmarksize}{-1.\pgfplotmarksize}}
\pgfpathlineto{\pgfpoint{-0.3\pgfplotmarksize}{-0.3\pgfplotmarksize}}
\pgfpathlineto{\pgfpoint{-1.\pgfplotmarksize}{-0.3\pgfplotmarksize}}
\pgfpathlineto{\pgfpoint{-1.\pgfplotmarksize}{0.3\pgfplotmarksize}}
\pgfpathlineto{\pgfpoint{-0.3\pgfplotmarksize}{0.3\pgfplotmarksize}}
\pgfpathclose
\pgfusepathqstroke
}
\pgfdeclareplotmark{cross*} {
\pgfpathmoveto{\pgfpoint{-0.3\pgfplotmarksize}{\pgfplotmarksize}}
\pgfpathlineto{\pgfpoint{+0.3\pgfplotmarksize}{\pgfplotmarksize}}
\pgfpathlineto{\pgfpoint{+0.3\pgfplotmarksize}{0.3\pgfplotmarksize}}
\pgfpathlineto{\pgfpoint{+1\pgfplotmarksize}{0.3\pgfplotmarksize}}
\pgfpathlineto{\pgfpoint{+1\pgfplotmarksize}{-0.3\pgfplotmarksize}}
\pgfpathlineto{\pgfpoint{+0.3\pgfplotmarksize}{-0.3\pgfplotmarksize}}
\pgfpathlineto{\pgfpoint{+0.3\pgfplotmarksize}{-1.\pgfplotmarksize}}
\pgfpathlineto{\pgfpoint{-0.3\pgfplotmarksize}{-1.\pgfplotmarksize}}
\pgfpathlineto{\pgfpoint{-0.3\pgfplotmarksize}{-0.3\pgfplotmarksize}}
\pgfpathlineto{\pgfpoint{-1.\pgfplotmarksize}{-0.3\pgfplotmarksize}}
\pgfpathlineto{\pgfpoint{-1.\pgfplotmarksize}{0.3\pgfplotmarksize}}
\pgfpathlineto{\pgfpoint{-0.3\pgfplotmarksize}{0.3\pgfplotmarksize}}
\pgfpathclose
\pgfusepathqfillstroke
}
\pgfdeclareplotmark{newstar} {
\pgfpathmoveto{\pgfqpoint{0pt}{\pgfplotmarksize}}
\pgfpathlineto{\pgfqpointpolar{44}{0.5\pgfplotmarksize}}
\pgfpathlineto{\pgfqpointpolar{18}{\pgfplotmarksize}}
\pgfpathlineto{\pgfqpointpolar{-20}{0.5\pgfplotmarksize}}
\pgfpathlineto{\pgfqpointpolar{-54}{\pgfplotmarksize}}
\pgfpathlineto{\pgfqpointpolar{-90}{0.5\pgfplotmarksize}}
\pgfpathlineto{\pgfqpointpolar{234}{\pgfplotmarksize}}
\pgfpathlineto{\pgfqpointpolar{198}{0.5\pgfplotmarksize}}
\pgfpathlineto{\pgfqpointpolar{162}{\pgfplotmarksize}}
\pgfpathlineto{\pgfqpointpolar{134}{0.5\pgfplotmarksize}}
\pgfpathclose
\pgfusepathqstroke
}
\pgfdeclareplotmark{newstar*} {
\pgfpathmoveto{\pgfqpoint{0pt}{\pgfplotmarksize}}
\pgfpathlineto{\pgfqpointpolar{44}{0.5\pgfplotmarksize}}
\pgfpathlineto{\pgfqpointpolar{18}{\pgfplotmarksize}}
\pgfpathlineto{\pgfqpointpolar{-20}{0.5\pgfplotmarksize}}
\pgfpathlineto{\pgfqpointpolar{-54}{\pgfplotmarksize}}
\pgfpathlineto{\pgfqpointpolar{-90}{0.5\pgfplotmarksize}}
\pgfpathlineto{\pgfqpointpolar{234}{\pgfplotmarksize}}
\pgfpathlineto{\pgfqpointpolar{198}{0.5\pgfplotmarksize}}
\pgfpathlineto{\pgfqpointpolar{162}{\pgfplotmarksize}}
\pgfpathlineto{\pgfqpointpolar{134}{0.5\pgfplotmarksize}}
\pgfpathclose
\pgfusepathqfillstroke
}
\definecolor{c}{rgb}{1,1,1};
\draw [color=c, fill=c] (0,0) rectangle (10,6.80516);
\draw [color=c, fill=c] (1,0.680516) rectangle (9.95,6.73711);
\definecolor{c}{rgb}{0,0,0};
\draw [c] (1,0.680516) -- (1,6.73711) -- (9.95,6.73711) -- (9.95,0.680516) -- (1,0.680516);
\definecolor{c}{rgb}{1,1,1};
\draw [color=c, fill=c] (1,0.680516) rectangle (9.95,6.73711);
\definecolor{c}{rgb}{0,0,0};
\draw [c] (1,0.680516) -- (1,6.73711) -- (9.95,6.73711) -- (9.95,0.680516) -- (1,0.680516);
\colorlet{c}{natgreen};
\draw [c] (1.59395,0.686894) -- (1.59395,0.686894);
\draw [c] (1.59395,0.686894) -- (1.59395,0.686894);
\draw [c] (1.58582,0.686894) -- (1.59395,0.686894);
\draw [c] (1.59395,0.686894) -- (1.60209,0.686894);
\definecolor{c}{rgb}{0,0,0};
\colorlet{c}{natgreen};
\draw [c] (1.96823,0.686894) -- (1.96823,0.686894);
\draw [c] (1.96823,0.686894) -- (1.96823,0.686894);
\draw [c] (1.96009,0.686894) -- (1.96823,0.686894);
\draw [c] (1.96823,0.686894) -- (1.97636,0.686894);
\definecolor{c}{rgb}{0,0,0};
\colorlet{c}{natgreen};
\draw [c] (2.24486,0.686894) -- (2.24486,0.686894);
\draw [c] (2.24486,0.686894) -- (2.24486,0.686894);
\draw [c] (2.23673,0.686894) -- (2.24486,0.686894);
\draw [c] (2.24486,0.686894) -- (2.253,0.686894);
\definecolor{c}{rgb}{0,0,0};
\colorlet{c}{natgreen};
\draw [c] (2.94459,0.692107) -- (2.94459,0.702737);
\draw [c] (2.94459,0.702737) -- (2.94459,0.713368);
\draw [c] (2.93645,0.702737) -- (2.94459,0.702737);
\draw [c] (2.94459,0.702737) -- (2.95273,0.702737);
\definecolor{c}{rgb}{0,0,0};
\colorlet{c}{natgreen};
\draw [c] (2.96086,0.691743) -- (2.96086,0.701841);
\draw [c] (2.96086,0.701841) -- (2.96086,0.711939);
\draw [c] (2.95273,0.701841) -- (2.96086,0.701841);
\draw [c] (2.96086,0.701841) -- (2.969,0.701841);
\definecolor{c}{rgb}{0,0,0};
\colorlet{c}{natgreen};
\draw [c] (3.02595,0.68941) -- (3.02595,0.69722);
\draw [c] (3.02595,0.69722) -- (3.02595,0.70503);
\draw [c] (3.01782,0.69722) -- (3.02595,0.69722);
\draw [c] (3.02595,0.69722) -- (3.03409,0.69722);
\definecolor{c}{rgb}{0,0,0};
\colorlet{c}{natgreen};
\draw [c] (3.0585,0.70998) -- (3.0585,0.729499);
\draw [c] (3.0585,0.729499) -- (3.0585,0.749018);
\draw [c] (3.05036,0.729499) -- (3.0585,0.729499);
\draw [c] (3.0585,0.729499) -- (3.06664,0.729499);
\definecolor{c}{rgb}{0,0,0};
\colorlet{c}{natgreen};
\draw [c] (3.07477,0.73884) -- (3.07477,0.766271);
\draw [c] (3.07477,0.766271) -- (3.07477,0.793702);
\draw [c] (3.06664,0.766271) -- (3.07477,0.766271);
\draw [c] (3.07477,0.766271) -- (3.08291,0.766271);
\definecolor{c}{rgb}{0,0,0};
\colorlet{c}{natgreen};
\draw [c] (3.09105,0.70686) -- (3.09105,0.727728);
\draw [c] (3.09105,0.727728) -- (3.09105,0.748596);
\draw [c] (3.08291,0.727728) -- (3.09105,0.727728);
\draw [c] (3.09105,0.727728) -- (3.09918,0.727728);
\definecolor{c}{rgb}{0,0,0};
\colorlet{c}{natgreen};
\draw [c] (3.10732,0.74735) -- (3.10732,0.773852);
\draw [c] (3.10732,0.773852) -- (3.10732,0.800355);
\draw [c] (3.09918,0.773852) -- (3.10732,0.773852);
\draw [c] (3.10732,0.773852) -- (3.11545,0.773852);
\definecolor{c}{rgb}{0,0,0};
\colorlet{c}{natgreen};
\draw [c] (3.12359,0.721184) -- (3.12359,0.744621);
\draw [c] (3.12359,0.744621) -- (3.12359,0.768058);
\draw [c] (3.11545,0.744621) -- (3.12359,0.744621);
\draw [c] (3.12359,0.744621) -- (3.13173,0.744621);
\definecolor{c}{rgb}{0,0,0};
\colorlet{c}{natgreen};
\draw [c] (3.13986,0.758599) -- (3.13986,0.790979);
\draw [c] (3.13986,0.790979) -- (3.13986,0.82336);
\draw [c] (3.13173,0.790979) -- (3.13986,0.790979);
\draw [c] (3.13986,0.790979) -- (3.148,0.790979);
\definecolor{c}{rgb}{0,0,0};
\colorlet{c}{natgreen};
\draw [c] (3.15614,0.770727) -- (3.15614,0.801482);
\draw [c] (3.15614,0.801482) -- (3.15614,0.832236);
\draw [c] (3.148,0.801482) -- (3.15614,0.801482);
\draw [c] (3.15614,0.801482) -- (3.16427,0.801482);
\definecolor{c}{rgb}{0,0,0};
\colorlet{c}{natgreen};
\draw [c] (3.17241,0.793587) -- (3.17241,0.827995);
\draw [c] (3.17241,0.827995) -- (3.17241,0.862404);
\draw [c] (3.16427,0.827995) -- (3.17241,0.827995);
\draw [c] (3.17241,0.827995) -- (3.18055,0.827995);
\definecolor{c}{rgb}{0,0,0};
\colorlet{c}{natgreen};
\draw [c] (3.18868,0.815002) -- (3.18868,0.854342);
\draw [c] (3.18868,0.854342) -- (3.18868,0.893683);
\draw [c] (3.18055,0.854342) -- (3.18868,0.854342);
\draw [c] (3.18868,0.854342) -- (3.19682,0.854342);
\definecolor{c}{rgb}{0,0,0};
\colorlet{c}{natgreen};
\draw [c] (3.20495,0.920998) -- (3.20495,0.973836);
\draw [c] (3.20495,0.973836) -- (3.20495,1.02667);
\draw [c] (3.19682,0.973836) -- (3.20495,0.973836);
\draw [c] (3.20495,0.973836) -- (3.21309,0.973836);
\definecolor{c}{rgb}{0,0,0};
\colorlet{c}{natgreen};
\draw [c] (3.22123,0.915327) -- (3.22123,0.966615);
\draw [c] (3.22123,0.966615) -- (3.22123,1.0179);
\draw [c] (3.21309,0.966615) -- (3.22123,0.966615);
\draw [c] (3.22123,0.966615) -- (3.22936,0.966615);
\definecolor{c}{rgb}{0,0,0};
\colorlet{c}{natgreen};
\draw [c] (3.2375,1.09644) -- (3.2375,1.16438);
\draw [c] (3.2375,1.16438) -- (3.2375,1.23231);
\draw [c] (3.22936,1.16438) -- (3.2375,1.16438);
\draw [c] (3.2375,1.16438) -- (3.24564,1.16438);
\definecolor{c}{rgb}{0,0,0};
\colorlet{c}{natgreen};
\draw [c] (3.25377,1.19172) -- (3.25377,1.26618);
\draw [c] (3.25377,1.26618) -- (3.25377,1.34065);
\draw [c] (3.24564,1.26618) -- (3.25377,1.26618);
\draw [c] (3.25377,1.26618) -- (3.26191,1.26618);
\definecolor{c}{rgb}{0,0,0};
\colorlet{c}{natgreen};
\draw [c] (3.27005,1.3208) -- (3.27005,1.40739);
\draw [c] (3.27005,1.40739) -- (3.27005,1.49398);
\draw [c] (3.26191,1.40739) -- (3.27005,1.40739);
\draw [c] (3.27005,1.40739) -- (3.27818,1.40739);
\definecolor{c}{rgb}{0,0,0};
\colorlet{c}{natgreen};
\draw [c] (3.28632,1.49265) -- (3.28632,1.58839);
\draw [c] (3.28632,1.58839) -- (3.28632,1.68413);
\draw [c] (3.27818,1.58839) -- (3.28632,1.58839);
\draw [c] (3.28632,1.58839) -- (3.29445,1.58839);
\definecolor{c}{rgb}{0,0,0};
\colorlet{c}{natgreen};
\draw [c] (3.30259,1.75408) -- (3.30259,1.865);
\draw [c] (3.30259,1.865) -- (3.30259,1.97593);
\draw [c] (3.29445,1.865) -- (3.30259,1.865);
\draw [c] (3.30259,1.865) -- (3.31073,1.865);
\definecolor{c}{rgb}{0,0,0};
\colorlet{c}{natgreen};
\draw [c] (3.31886,1.97279) -- (3.31886,2.09474);
\draw [c] (3.31886,2.09474) -- (3.31886,2.21669);
\draw [c] (3.31073,2.09474) -- (3.31886,2.09474);
\draw [c] (3.31886,2.09474) -- (3.327,2.09474);
\definecolor{c}{rgb}{0,0,0};
\colorlet{c}{natgreen};
\draw [c] (3.33514,2.31441) -- (3.33514,2.45017);
\draw [c] (3.33514,2.45017) -- (3.33514,2.58593);
\draw [c] (3.327,2.45017) -- (3.33514,2.45017);
\draw [c] (3.33514,2.45017) -- (3.34327,2.45017);
\definecolor{c}{rgb}{0,0,0};
\colorlet{c}{natgreen};
\draw [c] (3.35141,2.64591) -- (3.35141,2.7972);
\draw [c] (3.35141,2.7972) -- (3.35141,2.9485);
\draw [c] (3.34327,2.7972) -- (3.35141,2.7972);
\draw [c] (3.35141,2.7972) -- (3.35955,2.7972);
\definecolor{c}{rgb}{0,0,0};
\colorlet{c}{natgreen};
\draw [c] (3.36768,3.07353) -- (3.36768,3.24084);
\draw [c] (3.36768,3.24084) -- (3.36768,3.40814);
\draw [c] (3.35955,3.24084) -- (3.36768,3.24084);
\draw [c] (3.36768,3.24084) -- (3.37582,3.24084);
\definecolor{c}{rgb}{0,0,0};
\colorlet{c}{natgreen};
\draw [c] (3.38395,3.50243) -- (3.38395,3.68251);
\draw [c] (3.38395,3.68251) -- (3.38395,3.8626);
\draw [c] (3.37582,3.68251) -- (3.38395,3.68251);
\draw [c] (3.38395,3.68251) -- (3.39209,3.68251);
\definecolor{c}{rgb}{0,0,0};
\colorlet{c}{natgreen};
\draw [c] (3.40023,4.04994) -- (3.40023,4.24798);
\draw [c] (3.40023,4.24798) -- (3.40023,4.44601);
\draw [c] (3.39209,4.24798) -- (3.40023,4.24798);
\draw [c] (3.40023,4.24798) -- (3.40836,4.24798);
\definecolor{c}{rgb}{0,0,0};
\colorlet{c}{natgreen};
\draw [c] (3.4165,4.32972) -- (3.4165,4.53659);
\draw [c] (3.4165,4.53659) -- (3.4165,4.74346);
\draw [c] (3.40836,4.53659) -- (3.4165,4.53659);
\draw [c] (3.4165,4.53659) -- (3.42464,4.53659);
\definecolor{c}{rgb}{0,0,0};
\colorlet{c}{natgreen};
\draw [c] (3.43277,5.58375) -- (3.43277,5.82833);
\draw [c] (3.43277,5.82833) -- (3.43277,6.0729);
\draw [c] (3.42464,5.82833) -- (3.43277,5.82833);
\draw [c] (3.43277,5.82833) -- (3.44091,5.82833);
\definecolor{c}{rgb}{0,0,0};
\colorlet{c}{natgreen};
\draw [c] (3.44905,5.85444) -- (3.44905,6.10271);
\draw [c] (3.44905,6.10271) -- (3.44905,6.35098);
\draw [c] (3.44091,6.10271) -- (3.44905,6.10271);
\draw [c] (3.44905,6.10271) -- (3.45718,6.10271);
\definecolor{c}{rgb}{0,0,0};
\colorlet{c}{natgreen};
\draw [c] (3.46532,5.99755) -- (3.46532,6.2519);
\draw [c] (3.46532,6.2519) -- (3.46532,6.50624);
\draw [c] (3.45718,6.2519) -- (3.46532,6.2519);
\draw [c] (3.46532,6.2519) -- (3.47345,6.2519);
\definecolor{c}{rgb}{0,0,0};
\colorlet{c}{natgreen};
\draw [c] (3.48159,5.75888) -- (3.48159,6.00699);
\draw [c] (3.48159,6.00699) -- (3.48159,6.2551);
\draw [c] (3.47345,6.00699) -- (3.48159,6.00699);
\draw [c] (3.48159,6.00699) -- (3.48973,6.00699);
\definecolor{c}{rgb}{0,0,0};
\colorlet{c}{natgreen};
\draw [c] (3.49786,5.99961) -- (3.49786,6.25327);
\draw [c] (3.49786,6.25327) -- (3.49786,6.50693);
\draw [c] (3.48973,6.25327) -- (3.49786,6.25327);
\draw [c] (3.49786,6.25327) -- (3.506,6.25327);
\definecolor{c}{rgb}{0,0,0};
\colorlet{c}{natgreen};
\draw [c] (3.51414,6.07844) -- (3.51414,6.33275);
\draw [c] (3.51414,6.33275) -- (3.51414,6.58705);
\draw [c] (3.506,6.33275) -- (3.51414,6.33275);
\draw [c] (3.51414,6.33275) -- (3.52227,6.33275);
\definecolor{c}{rgb}{0,0,0};
\colorlet{c}{natgreen};
\draw [c] (3.53041,5.86819) -- (3.53041,6.1164);
\draw [c] (3.53041,6.1164) -- (3.53041,6.36461);
\draw [c] (3.52227,6.1164) -- (3.53041,6.1164);
\draw [c] (3.53041,6.1164) -- (3.53855,6.1164);
\definecolor{c}{rgb}{0,0,0};
\colorlet{c}{natgreen};
\draw [c] (3.54668,5.42558) -- (3.54668,5.66604);
\draw [c] (3.54668,5.66604) -- (3.54668,5.90649);
\draw [c] (3.53855,5.66604) -- (3.54668,5.66604);
\draw [c] (3.54668,5.66604) -- (3.55482,5.66604);
\definecolor{c}{rgb}{0,0,0};
\colorlet{c}{natgreen};
\draw [c] (3.56295,5.15763) -- (3.56295,5.3894);
\draw [c] (3.56295,5.3894) -- (3.56295,5.62118);
\draw [c] (3.55482,5.3894) -- (3.56295,5.3894);
\draw [c] (3.56295,5.3894) -- (3.57109,5.3894);
\definecolor{c}{rgb}{0,0,0};
\colorlet{c}{natgreen};
\draw [c] (3.57923,4.48033) -- (3.57923,4.69298);
\draw [c] (3.57923,4.69298) -- (3.57923,4.90562);
\draw [c] (3.57109,4.69298) -- (3.57923,4.69298);
\draw [c] (3.57923,4.69298) -- (3.58736,4.69298);
\definecolor{c}{rgb}{0,0,0};
\colorlet{c}{natgreen};
\draw [c] (3.5955,4.00128) -- (3.5955,4.20052);
\draw [c] (3.5955,4.20052) -- (3.5955,4.39975);
\draw [c] (3.58736,4.20052) -- (3.5955,4.20052);
\draw [c] (3.5955,4.20052) -- (3.60364,4.20052);
\definecolor{c}{rgb}{0,0,0};
\colorlet{c}{natgreen};
\draw [c] (3.61177,4.5575) -- (3.61177,4.7732);
\draw [c] (3.61177,4.7732) -- (3.61177,4.98891);
\draw [c] (3.60364,4.7732) -- (3.61177,4.7732);
\draw [c] (3.61177,4.7732) -- (3.61991,4.7732);
\definecolor{c}{rgb}{0,0,0};
\colorlet{c}{natgreen};
\draw [c] (3.62805,3.69671) -- (3.62805,3.88706);
\draw [c] (3.62805,3.88706) -- (3.62805,4.07741);
\draw [c] (3.61991,3.88706) -- (3.62805,3.88706);
\draw [c] (3.62805,3.88706) -- (3.63618,3.88706);
\definecolor{c}{rgb}{0,0,0};
\colorlet{c}{natgreen};
\draw [c] (3.64432,3.2541) -- (3.64432,3.43157);
\draw [c] (3.64432,3.43157) -- (3.64432,3.60905);
\draw [c] (3.63618,3.43157) -- (3.64432,3.43157);
\draw [c] (3.64432,3.43157) -- (3.65245,3.43157);
\definecolor{c}{rgb}{0,0,0};
\colorlet{c}{natgreen};
\draw [c] (3.66059,3.29397) -- (3.66059,3.46949);
\draw [c] (3.66059,3.46949) -- (3.66059,3.64502);
\draw [c] (3.65245,3.46949) -- (3.66059,3.46949);
\draw [c] (3.66059,3.46949) -- (3.66873,3.46949);
\definecolor{c}{rgb}{0,0,0};
\colorlet{c}{natgreen};
\draw [c] (3.67686,2.88985) -- (3.67686,3.05307);
\draw [c] (3.67686,3.05307) -- (3.67686,3.21628);
\draw [c] (3.66873,3.05307) -- (3.67686,3.05307);
\draw [c] (3.67686,3.05307) -- (3.685,3.05307);
\definecolor{c}{rgb}{0,0,0};
\colorlet{c}{natgreen};
\draw [c] (3.69314,2.89355) -- (3.69314,3.05681);
\draw [c] (3.69314,3.05681) -- (3.69314,3.22008);
\draw [c] (3.685,3.05681) -- (3.69314,3.05681);
\draw [c] (3.69314,3.05681) -- (3.70127,3.05681);
\definecolor{c}{rgb}{0,0,0};
\colorlet{c}{natgreen};
\draw [c] (3.70941,2.51251) -- (3.70941,2.66097);
\draw [c] (3.70941,2.66097) -- (3.70941,2.80943);
\draw [c] (3.70127,2.66097) -- (3.70941,2.66097);
\draw [c] (3.70941,2.66097) -- (3.71755,2.66097);
\definecolor{c}{rgb}{0,0,0};
\colorlet{c}{natgreen};
\draw [c] (3.72568,2.57195) -- (3.72568,2.72513);
\draw [c] (3.72568,2.72513) -- (3.72568,2.87831);
\draw [c] (3.71755,2.72513) -- (3.72568,2.72513);
\draw [c] (3.72568,2.72513) -- (3.73382,2.72513);
\definecolor{c}{rgb}{0,0,0};
\colorlet{c}{natgreen};
\draw [c] (3.74195,2.17915) -- (3.74195,2.31346);
\draw [c] (3.74195,2.31346) -- (3.74195,2.44777);
\draw [c] (3.73382,2.31346) -- (3.74195,2.31346);
\draw [c] (3.74195,2.31346) -- (3.75009,2.31346);
\definecolor{c}{rgb}{0,0,0};
\colorlet{c}{natgreen};
\draw [c] (3.75823,2.19392) -- (3.75823,2.32903);
\draw [c] (3.75823,2.32903) -- (3.75823,2.46413);
\draw [c] (3.75009,2.32903) -- (3.75823,2.32903);
\draw [c] (3.75823,2.32903) -- (3.76636,2.32903);
\definecolor{c}{rgb}{0,0,0};
\colorlet{c}{natgreen};
\draw [c] (3.7745,1.81437) -- (3.7745,1.92945);
\draw [c] (3.7745,1.92945) -- (3.7745,2.04452);
\draw [c] (3.76636,1.92945) -- (3.7745,1.92945);
\draw [c] (3.7745,1.92945) -- (3.78264,1.92945);
\definecolor{c}{rgb}{0,0,0};
\colorlet{c}{natgreen};
\draw [c] (3.79077,1.94943) -- (3.79077,2.07455);
\draw [c] (3.79077,2.07455) -- (3.79077,2.19966);
\draw [c] (3.78264,2.07455) -- (3.79077,2.07455);
\draw [c] (3.79077,2.07455) -- (3.79891,2.07455);
\definecolor{c}{rgb}{0,0,0};
\colorlet{c}{natgreen};
\draw [c] (3.80705,1.89796) -- (3.80705,2.01752);
\draw [c] (3.80705,2.01752) -- (3.80705,2.13708);
\draw [c] (3.79891,2.01752) -- (3.80705,2.01752);
\draw [c] (3.80705,2.01752) -- (3.81518,2.01752);
\definecolor{c}{rgb}{0,0,0};
\colorlet{c}{natgreen};
\draw [c] (3.82332,1.63297) -- (3.82332,1.7425);
\draw [c] (3.82332,1.7425) -- (3.82332,1.85203);
\draw [c] (3.81518,1.7425) -- (3.82332,1.7425);
\draw [c] (3.82332,1.7425) -- (3.83145,1.7425);
\definecolor{c}{rgb}{0,0,0};
\colorlet{c}{natgreen};
\draw [c] (3.83959,1.88026) -- (3.83959,1.99932);
\draw [c] (3.83959,1.99932) -- (3.83959,2.11838);
\draw [c] (3.83145,1.99932) -- (3.83959,1.99932);
\draw [c] (3.83959,1.99932) -- (3.84773,1.99932);
\definecolor{c}{rgb}{0,0,0};
\colorlet{c}{natgreen};
\draw [c] (3.85586,1.55762) -- (3.85586,1.66135);
\draw [c] (3.85586,1.66135) -- (3.85586,1.76508);
\draw [c] (3.84773,1.66135) -- (3.85586,1.66135);
\draw [c] (3.85586,1.66135) -- (3.864,1.66135);
\definecolor{c}{rgb}{0,0,0};
\colorlet{c}{natgreen};
\draw [c] (3.87214,1.37707) -- (3.87214,1.46506);
\draw [c] (3.87214,1.46506) -- (3.87214,1.55305);
\draw [c] (3.864,1.46506) -- (3.87214,1.46506);
\draw [c] (3.87214,1.46506) -- (3.88027,1.46506);
\definecolor{c}{rgb}{0,0,0};
\colorlet{c}{natgreen};
\draw [c] (3.88841,1.38074) -- (3.88841,1.47519);
\draw [c] (3.88841,1.47519) -- (3.88841,1.56964);
\draw [c] (3.88027,1.47519) -- (3.88841,1.47519);
\draw [c] (3.88841,1.47519) -- (3.89655,1.47519);
\definecolor{c}{rgb}{0,0,0};
\colorlet{c}{natgreen};
\draw [c] (3.90468,1.34633) -- (3.90468,1.43759);
\draw [c] (3.90468,1.43759) -- (3.90468,1.52885);
\draw [c] (3.89655,1.43759) -- (3.90468,1.43759);
\draw [c] (3.90468,1.43759) -- (3.91282,1.43759);
\definecolor{c}{rgb}{0,0,0};
\colorlet{c}{natgreen};
\draw [c] (3.92095,1.20184) -- (3.92095,1.28192);
\draw [c] (3.92095,1.28192) -- (3.92095,1.362);
\draw [c] (3.91282,1.28192) -- (3.92095,1.28192);
\draw [c] (3.92095,1.28192) -- (3.92909,1.28192);
\definecolor{c}{rgb}{0,0,0};
\colorlet{c}{natgreen};
\draw [c] (3.93723,1.14683) -- (3.93723,1.22409);
\draw [c] (3.93723,1.22409) -- (3.93723,1.30136);
\draw [c] (3.92909,1.22409) -- (3.93723,1.22409);
\draw [c] (3.93723,1.22409) -- (3.94536,1.22409);
\definecolor{c}{rgb}{0,0,0};
\colorlet{c}{natgreen};
\draw [c] (3.9535,1.09219) -- (3.9535,1.16486);
\draw [c] (3.9535,1.16486) -- (3.9535,1.23754);
\draw [c] (3.94536,1.16486) -- (3.9535,1.16486);
\draw [c] (3.9535,1.16486) -- (3.96164,1.16486);
\definecolor{c}{rgb}{0,0,0};
\colorlet{c}{natgreen};
\draw [c] (3.96977,1.03834) -- (3.96977,1.10508);
\draw [c] (3.96977,1.10508) -- (3.96977,1.17181);
\draw [c] (3.96164,1.10508) -- (3.96977,1.10508);
\draw [c] (3.96977,1.10508) -- (3.97791,1.10508);
\definecolor{c}{rgb}{0,0,0};
\colorlet{c}{natgreen};
\draw [c] (3.98605,1.09536) -- (3.98605,1.16626);
\draw [c] (3.98605,1.16626) -- (3.98605,1.23716);
\draw [c] (3.97791,1.16626) -- (3.98605,1.16626);
\draw [c] (3.98605,1.16626) -- (3.99418,1.16626);
\definecolor{c}{rgb}{0,0,0};
\colorlet{c}{natgreen};
\draw [c] (4.00232,0.977898) -- (4.00232,1.03829);
\draw [c] (4.00232,1.03829) -- (4.00232,1.09867);
\draw [c] (3.99418,1.03829) -- (4.00232,1.03829);
\draw [c] (4.00232,1.03829) -- (4.01045,1.03829);
\definecolor{c}{rgb}{0,0,0};
\colorlet{c}{natgreen};
\draw [c] (4.01859,1.13448) -- (4.01859,1.21092);
\draw [c] (4.01859,1.21092) -- (4.01859,1.28736);
\draw [c] (4.01045,1.21092) -- (4.01859,1.21092);
\draw [c] (4.01859,1.21092) -- (4.02673,1.21092);
\definecolor{c}{rgb}{0,0,0};
\colorlet{c}{natgreen};
\draw [c] (4.03486,0.932526) -- (4.03486,0.989231);
\draw [c] (4.03486,0.989231) -- (4.03486,1.04594);
\draw [c] (4.02673,0.989231) -- (4.03486,0.989231);
\draw [c] (4.03486,0.989231) -- (4.043,0.989231);
\definecolor{c}{rgb}{0,0,0};
\colorlet{c}{natgreen};
\draw [c] (4.05114,0.909038) -- (4.05114,0.964041);
\draw [c] (4.05114,0.964041) -- (4.05114,1.01904);
\draw [c] (4.043,0.964041) -- (4.05114,0.964041);
\draw [c] (4.05114,0.964041) -- (4.05927,0.964041);
\definecolor{c}{rgb}{0,0,0};
\colorlet{c}{natgreen};
\draw [c] (4.06741,0.905278) -- (4.06741,0.955802);
\draw [c] (4.06741,0.955802) -- (4.06741,1.00633);
\draw [c] (4.05927,0.955802) -- (4.06741,0.955802);
\draw [c] (4.06741,0.955802) -- (4.07555,0.955802);
\definecolor{c}{rgb}{0,0,0};
\colorlet{c}{natgreen};
\draw [c] (4.08368,0.910484) -- (4.08368,0.965967);
\draw [c] (4.08368,0.965967) -- (4.08368,1.02145);
\draw [c] (4.07555,0.965967) -- (4.08368,0.965967);
\draw [c] (4.08368,0.965967) -- (4.09182,0.965967);
\definecolor{c}{rgb}{0,0,0};
\colorlet{c}{natgreen};
\draw [c] (4.09995,0.84669) -- (4.09995,0.893411);
\draw [c] (4.09995,0.893411) -- (4.09995,0.940132);
\draw [c] (4.09182,0.893411) -- (4.09995,0.893411);
\draw [c] (4.09995,0.893411) -- (4.10809,0.893411);
\definecolor{c}{rgb}{0,0,0};
\colorlet{c}{natgreen};
\draw [c] (4.11623,0.816785) -- (4.11623,0.859365);
\draw [c] (4.11623,0.859365) -- (4.11623,0.901946);
\draw [c] (4.10809,0.859365) -- (4.11623,0.859365);
\draw [c] (4.11623,0.859365) -- (4.12436,0.859365);
\definecolor{c}{rgb}{0,0,0};
\colorlet{c}{natgreen};
\draw [c] (4.1325,0.820052) -- (4.1325,0.86177);
\draw [c] (4.1325,0.86177) -- (4.1325,0.903487);
\draw [c] (4.12436,0.86177) -- (4.1325,0.86177);
\draw [c] (4.1325,0.86177) -- (4.14064,0.86177);
\definecolor{c}{rgb}{0,0,0};
\colorlet{c}{natgreen};
\draw [c] (4.14877,0.830401) -- (4.14877,0.871992);
\draw [c] (4.14877,0.871992) -- (4.14877,0.913582);
\draw [c] (4.14064,0.871992) -- (4.14877,0.871992);
\draw [c] (4.14877,0.871992) -- (4.15691,0.871992);
\definecolor{c}{rgb}{0,0,0};
\colorlet{c}{natgreen};
\draw [c] (4.16505,0.79867) -- (4.16505,0.837525);
\draw [c] (4.16505,0.837525) -- (4.16505,0.87638);
\draw [c] (4.15691,0.837525) -- (4.16505,0.837525);
\draw [c] (4.16505,0.837525) -- (4.17318,0.837525);
\definecolor{c}{rgb}{0,0,0};
\colorlet{c}{natgreen};
\draw [c] (4.18132,0.784983) -- (4.18132,0.822264);
\draw [c] (4.18132,0.822264) -- (4.18132,0.859546);
\draw [c] (4.17318,0.822264) -- (4.18132,0.822264);
\draw [c] (4.18132,0.822264) -- (4.18945,0.822264);
\definecolor{c}{rgb}{0,0,0};
\colorlet{c}{natgreen};
\draw [c] (4.19759,0.794485) -- (4.19759,0.831406);
\draw [c] (4.19759,0.831406) -- (4.19759,0.868327);
\draw [c] (4.18945,0.831406) -- (4.19759,0.831406);
\draw [c] (4.19759,0.831406) -- (4.20573,0.831406);
\definecolor{c}{rgb}{0,0,0};
\colorlet{c}{natgreen};
\draw [c] (4.21386,0.787833) -- (4.21386,0.826205);
\draw [c] (4.21386,0.826205) -- (4.21386,0.864577);
\draw [c] (4.20573,0.826205) -- (4.21386,0.826205);
\draw [c] (4.21386,0.826205) -- (4.222,0.826205);
\definecolor{c}{rgb}{0,0,0};
\colorlet{c}{natgreen};
\draw [c] (4.23014,0.779584) -- (4.23014,0.816318);
\draw [c] (4.23014,0.816318) -- (4.23014,0.853052);
\draw [c] (4.222,0.816318) -- (4.23014,0.816318);
\draw [c] (4.23014,0.816318) -- (4.23827,0.816318);
\definecolor{c}{rgb}{0,0,0};
\colorlet{c}{natgreen};
\draw [c] (4.24641,0.773341) -- (4.24641,0.809215);
\draw [c] (4.24641,0.809215) -- (4.24641,0.845089);
\draw [c] (4.23827,0.809215) -- (4.24641,0.809215);
\draw [c] (4.24641,0.809215) -- (4.25455,0.809215);
\definecolor{c}{rgb}{0,0,0};
\colorlet{c}{natgreen};
\draw [c] (4.26268,0.754109) -- (4.26268,0.789001);
\draw [c] (4.26268,0.789001) -- (4.26268,0.823894);
\draw [c] (4.25455,0.789001) -- (4.26268,0.789001);
\draw [c] (4.26268,0.789001) -- (4.27082,0.789001);
\definecolor{c}{rgb}{0,0,0};
\colorlet{c}{natgreen};
\draw [c] (4.27895,0.761187) -- (4.27895,0.795352);
\draw [c] (4.27895,0.795352) -- (4.27895,0.829516);
\draw [c] (4.27082,0.795352) -- (4.27895,0.795352);
\draw [c] (4.27895,0.795352) -- (4.28709,0.795352);
\definecolor{c}{rgb}{0,0,0};
\colorlet{c}{natgreen};
\draw [c] (4.29523,0.703056) -- (4.29523,0.723502);
\draw [c] (4.29523,0.723502) -- (4.29523,0.743949);
\draw [c] (4.28709,0.723502) -- (4.29523,0.723502);
\draw [c] (4.29523,0.723502) -- (4.30336,0.723502);
\definecolor{c}{rgb}{0,0,0};
\colorlet{c}{natgreen};
\draw [c] (4.3115,0.731532) -- (4.3115,0.761049);
\draw [c] (4.3115,0.761049) -- (4.3115,0.790567);
\draw [c] (4.30336,0.761049) -- (4.3115,0.761049);
\draw [c] (4.3115,0.761049) -- (4.31964,0.761049);
\definecolor{c}{rgb}{0,0,0};
\colorlet{c}{natgreen};
\draw [c] (4.32777,0.745916) -- (4.32777,0.773891);
\draw [c] (4.32777,0.773891) -- (4.32777,0.801865);
\draw [c] (4.31964,0.773891) -- (4.32777,0.773891);
\draw [c] (4.32777,0.773891) -- (4.33591,0.773891);
\definecolor{c}{rgb}{0,0,0};
\colorlet{c}{natgreen};
\draw [c] (4.34405,0.743554) -- (4.34405,0.773152);
\draw [c] (4.34405,0.773152) -- (4.34405,0.80275);
\draw [c] (4.33591,0.773152) -- (4.34405,0.773152);
\draw [c] (4.34405,0.773152) -- (4.35218,0.773152);
\definecolor{c}{rgb}{0,0,0};
\colorlet{c}{natgreen};
\draw [c] (4.36032,0.692936) -- (4.36032,0.707758);
\draw [c] (4.36032,0.707758) -- (4.36032,0.72258);
\draw [c] (4.35218,0.707758) -- (4.36032,0.707758);
\draw [c] (4.36032,0.707758) -- (4.36845,0.707758);
\definecolor{c}{rgb}{0,0,0};
\colorlet{c}{natgreen};
\draw [c] (4.37659,0.757202) -- (4.37659,0.792096);
\draw [c] (4.37659,0.792096) -- (4.37659,0.826991);
\draw [c] (4.36845,0.792096) -- (4.37659,0.792096);
\draw [c] (4.37659,0.792096) -- (4.38473,0.792096);
\definecolor{c}{rgb}{0,0,0};
\colorlet{c}{natgreen};
\draw [c] (4.39286,0.702945) -- (4.39286,0.719518);
\draw [c] (4.39286,0.719518) -- (4.39286,0.73609);
\draw [c] (4.38473,0.719518) -- (4.39286,0.719518);
\draw [c] (4.39286,0.719518) -- (4.401,0.719518);
\definecolor{c}{rgb}{0,0,0};
\colorlet{c}{natgreen};
\draw [c] (4.40914,0.723316) -- (4.40914,0.749374);
\draw [c] (4.40914,0.749374) -- (4.40914,0.775431);
\draw [c] (4.401,0.749374) -- (4.40914,0.749374);
\draw [c] (4.40914,0.749374) -- (4.41727,0.749374);
\definecolor{c}{rgb}{0,0,0};
\colorlet{c}{natgreen};
\draw [c] (4.42541,0.70816) -- (4.42541,0.729558);
\draw [c] (4.42541,0.729558) -- (4.42541,0.750956);
\draw [c] (4.41727,0.729558) -- (4.42541,0.729558);
\draw [c] (4.42541,0.729558) -- (4.43355,0.729558);
\definecolor{c}{rgb}{0,0,0};
\colorlet{c}{natgreen};
\draw [c] (4.44168,0.731388) -- (4.44168,0.758337);
\draw [c] (4.44168,0.758337) -- (4.44168,0.785287);
\draw [c] (4.43355,0.758337) -- (4.44168,0.758337);
\draw [c] (4.44168,0.758337) -- (4.44982,0.758337);
\definecolor{c}{rgb}{0,0,0};
\colorlet{c}{natgreen};
\draw [c] (4.45795,0.715949) -- (4.45795,0.737013);
\draw [c] (4.45795,0.737013) -- (4.45795,0.758078);
\draw [c] (4.44982,0.737013) -- (4.45795,0.737013);
\draw [c] (4.45795,0.737013) -- (4.46609,0.737013);
\definecolor{c}{rgb}{0,0,0};
\colorlet{c}{natgreen};
\draw [c] (4.47423,0.708102) -- (4.47423,0.72723);
\draw [c] (4.47423,0.72723) -- (4.47423,0.746359);
\draw [c] (4.46609,0.72723) -- (4.47423,0.72723);
\draw [c] (4.47423,0.72723) -- (4.48236,0.72723);
\definecolor{c}{rgb}{0,0,0};
\colorlet{c}{natgreen};
\draw [c] (4.4905,0.695838) -- (4.4905,0.713323);
\draw [c] (4.4905,0.713323) -- (4.4905,0.730808);
\draw [c] (4.48236,0.713323) -- (4.4905,0.713323);
\draw [c] (4.4905,0.713323) -- (4.49864,0.713323);
\definecolor{c}{rgb}{0,0,0};
\colorlet{c}{natgreen};
\draw [c] (4.50677,0.70065) -- (4.50677,0.71825);
\draw [c] (4.50677,0.71825) -- (4.50677,0.735851);
\draw [c] (4.49864,0.71825) -- (4.50677,0.71825);
\draw [c] (4.50677,0.71825) -- (4.51491,0.71825);
\definecolor{c}{rgb}{0,0,0};
\colorlet{c}{natgreen};
\draw [c] (4.52305,0.692273) -- (4.52305,0.70583);
\draw [c] (4.52305,0.70583) -- (4.52305,0.719387);
\draw [c] (4.51491,0.70583) -- (4.52305,0.70583);
\draw [c] (4.52305,0.70583) -- (4.53118,0.70583);
\definecolor{c}{rgb}{0,0,0};
\colorlet{c}{natgreen};
\draw [c] (4.53932,0.708612) -- (4.53932,0.730843);
\draw [c] (4.53932,0.730843) -- (4.53932,0.753075);
\draw [c] (4.53118,0.730843) -- (4.53932,0.730843);
\draw [c] (4.53932,0.730843) -- (4.54745,0.730843);
\definecolor{c}{rgb}{0,0,0};
\colorlet{c}{natgreen};
\draw [c] (4.55559,0.699596) -- (4.55559,0.720953);
\draw [c] (4.55559,0.720953) -- (4.55559,0.74231);
\draw [c] (4.54745,0.720953) -- (4.55559,0.720953);
\draw [c] (4.55559,0.720953) -- (4.56373,0.720953);
\definecolor{c}{rgb}{0,0,0};
\colorlet{c}{natgreen};
\draw [c] (4.57186,0.709471) -- (4.57186,0.731051);
\draw [c] (4.57186,0.731051) -- (4.57186,0.75263);
\draw [c] (4.56373,0.731051) -- (4.57186,0.731051);
\draw [c] (4.57186,0.731051) -- (4.58,0.731051);
\definecolor{c}{rgb}{0,0,0};
\colorlet{c}{natgreen};
\draw [c] (4.58814,0.69136) -- (4.58814,0.704002);
\draw [c] (4.58814,0.704002) -- (4.58814,0.716643);
\draw [c] (4.58,0.704002) -- (4.58814,0.704002);
\draw [c] (4.58814,0.704002) -- (4.59627,0.704002);
\definecolor{c}{rgb}{0,0,0};
\colorlet{c}{natgreen};
\draw [c] (4.60441,0.688612) -- (4.60441,0.700212);
\draw [c] (4.60441,0.700212) -- (4.60441,0.711812);
\draw [c] (4.59627,0.700212) -- (4.60441,0.700212);
\draw [c] (4.60441,0.700212) -- (4.61255,0.700212);
\definecolor{c}{rgb}{0,0,0};
\colorlet{c}{natgreen};
\draw [c] (4.62068,0.691463) -- (4.62068,0.70493);
\draw [c] (4.62068,0.70493) -- (4.62068,0.718398);
\draw [c] (4.61255,0.70493) -- (4.62068,0.70493);
\draw [c] (4.62068,0.70493) -- (4.62882,0.70493);
\definecolor{c}{rgb}{0,0,0};
\colorlet{c}{natgreen};
\draw [c] (4.65323,0.690572) -- (4.65323,0.702891);
\draw [c] (4.65323,0.702891) -- (4.65323,0.71521);
\draw [c] (4.64509,0.702891) -- (4.65323,0.702891);
\draw [c] (4.65323,0.702891) -- (4.66136,0.702891);
\definecolor{c}{rgb}{0,0,0};
\colorlet{c}{natgreen};
\draw [c] (4.68577,0.686894) -- (4.68577,0.69532);
\draw [c] (4.68577,0.69532) -- (4.68577,0.703746);
\draw [c] (4.67764,0.69532) -- (4.68577,0.69532);
\draw [c] (4.68577,0.69532) -- (4.69391,0.69532);
\definecolor{c}{rgb}{0,0,0};
\colorlet{c}{natgreen};
\draw [c] (4.70205,0.702308) -- (4.70205,0.723366);
\draw [c] (4.70205,0.723366) -- (4.70205,0.744423);
\draw [c] (4.69391,0.723366) -- (4.70205,0.723366);
\draw [c] (4.70205,0.723366) -- (4.71018,0.723366);
\definecolor{c}{rgb}{0,0,0};
\colorlet{c}{natgreen};
\draw [c] (4.73459,0.686894) -- (4.73459,0.687013);
\draw [c] (4.73459,0.687013) -- (4.73459,0.687133);
\draw [c] (4.72645,0.687013) -- (4.73459,0.687013);
\draw [c] (4.73459,0.687013) -- (4.74273,0.687013);
\definecolor{c}{rgb}{0,0,0};
\colorlet{c}{natgreen};
\draw [c] (4.75086,0.686894) -- (4.75086,0.699741);
\draw [c] (4.75086,0.699741) -- (4.75086,0.712588);
\draw [c] (4.74273,0.699741) -- (4.75086,0.699741);
\draw [c] (4.75086,0.699741) -- (4.759,0.699741);
\definecolor{c}{rgb}{0,0,0};
\colorlet{c}{natgreen};
\draw [c] (4.76714,0.689559) -- (4.76714,0.695992);
\draw [c] (4.76714,0.695992) -- (4.76714,0.702426);
\draw [c] (4.759,0.695992) -- (4.76714,0.695992);
\draw [c] (4.76714,0.695992) -- (4.77527,0.695992);
\definecolor{c}{rgb}{0,0,0};
\colorlet{c}{natgreen};
\draw [c] (4.78341,0.688608) -- (4.78341,0.699913);
\draw [c] (4.78341,0.699913) -- (4.78341,0.711218);
\draw [c] (4.77527,0.699913) -- (4.78341,0.699913);
\draw [c] (4.78341,0.699913) -- (4.79155,0.699913);
\definecolor{c}{rgb}{0,0,0};
\colorlet{c}{natgreen};
\draw [c] (4.79968,0.697235) -- (4.79968,0.712428);
\draw [c] (4.79968,0.712428) -- (4.79968,0.72762);
\draw [c] (4.79155,0.712428) -- (4.79968,0.712428);
\draw [c] (4.79968,0.712428) -- (4.80782,0.712428);
\definecolor{c}{rgb}{0,0,0};
\colorlet{c}{natgreen};
\draw [c] (4.83223,0.686894) -- (4.83223,0.691443);
\draw [c] (4.83223,0.691443) -- (4.83223,0.695992);
\draw [c] (4.82409,0.691443) -- (4.83223,0.691443);
\draw [c] (4.83223,0.691443) -- (4.84036,0.691443);
\definecolor{c}{rgb}{0,0,0};
\colorlet{c}{natgreen};
\draw [c] (4.8485,0.686894) -- (4.8485,0.692853);
\draw [c] (4.8485,0.692853) -- (4.8485,0.698811);
\draw [c] (4.84036,0.692853) -- (4.8485,0.692853);
\draw [c] (4.8485,0.692853) -- (4.85664,0.692853);
\definecolor{c}{rgb}{0,0,0};
\colorlet{c}{natgreen};
\draw [c] (4.88105,0.691479) -- (4.88105,0.70509);
\draw [c] (4.88105,0.70509) -- (4.88105,0.718702);
\draw [c] (4.87291,0.70509) -- (4.88105,0.70509);
\draw [c] (4.88105,0.70509) -- (4.88918,0.70509);
\definecolor{c}{rgb}{0,0,0};
\colorlet{c}{natgreen};
\draw [c] (4.89732,0.697187) -- (4.89732,0.714402);
\draw [c] (4.89732,0.714402) -- (4.89732,0.731616);
\draw [c] (4.88918,0.714402) -- (4.89732,0.714402);
\draw [c] (4.89732,0.714402) -- (4.90545,0.714402);
\definecolor{c}{rgb}{0,0,0};
\colorlet{c}{natgreen};
\draw [c] (4.94614,0.701168) -- (4.94614,0.722014);
\draw [c] (4.94614,0.722014) -- (4.94614,0.74286);
\draw [c] (4.938,0.722014) -- (4.94614,0.722014);
\draw [c] (4.94614,0.722014) -- (4.95427,0.722014);
\definecolor{c}{rgb}{0,0,0};
\colorlet{c}{natgreen};
\draw [c] (4.96241,0.686894) -- (4.96241,0.70053);
\draw [c] (4.96241,0.70053) -- (4.96241,0.714166);
\draw [c] (4.95427,0.70053) -- (4.96241,0.70053);
\draw [c] (4.96241,0.70053) -- (4.97055,0.70053);
\definecolor{c}{rgb}{0,0,0};
\colorlet{c}{natgreen};
\draw [c] (5.04377,0.686894) -- (5.04377,0.688764);
\draw [c] (5.04377,0.688764) -- (5.04377,0.690634);
\draw [c] (5.03564,0.688764) -- (5.04377,0.688764);
\draw [c] (5.04377,0.688764) -- (5.05191,0.688764);
\definecolor{c}{rgb}{0,0,0};
\colorlet{c}{natgreen};
\draw [c] (5.06005,0.686894) -- (5.06005,0.698043);
\draw [c] (5.06005,0.698043) -- (5.06005,0.709192);
\draw [c] (5.05191,0.698043) -- (5.06005,0.698043);
\draw [c] (5.06005,0.698043) -- (5.06818,0.698043);
\definecolor{c}{rgb}{0,0,0};
\colorlet{c}{natgreen};
\draw [c] (5.10886,0.686894) -- (5.10886,0.697253);
\draw [c] (5.10886,0.697253) -- (5.10886,0.707613);
\draw [c] (5.10073,0.697253) -- (5.10886,0.697253);
\draw [c] (5.10886,0.697253) -- (5.117,0.697253);
\definecolor{c}{rgb}{0,0,0};
\colorlet{c}{natgreen};
\draw [c] (5.15768,0.686894) -- (5.15768,0.699132);
\draw [c] (5.15768,0.699132) -- (5.15768,0.71137);
\draw [c] (5.14955,0.699132) -- (5.15768,0.699132);
\draw [c] (5.15768,0.699132) -- (5.16582,0.699132);
\definecolor{c}{rgb}{0,0,0};
\colorlet{c}{natgreen};
\draw [c] (5.19023,0.686894) -- (5.19023,0.687595);
\draw [c] (5.19023,0.687595) -- (5.19023,0.688296);
\draw [c] (5.18209,0.687595) -- (5.19023,0.687595);
\draw [c] (5.19023,0.687595) -- (5.19836,0.687595);
\definecolor{c}{rgb}{0,0,0};
\colorlet{c}{natgreen};
\draw [c] (5.30414,0.686894) -- (5.30414,0.692853);
\draw [c] (5.30414,0.692853) -- (5.30414,0.698811);
\draw [c] (5.296,0.692853) -- (5.30414,0.692853);
\draw [c] (5.30414,0.692853) -- (5.31227,0.692853);
\definecolor{c}{rgb}{0,0,0};
\colorlet{c}{natgreen};
\draw [c] (5.33668,0.686894) -- (5.33668,0.694017);
\draw [c] (5.33668,0.694017) -- (5.33668,0.70114);
\draw [c] (5.32855,0.694017) -- (5.33668,0.694017);
\draw [c] (5.33668,0.694017) -- (5.34482,0.694017);
\definecolor{c}{rgb}{0,0,0};
\colorlet{c}{natgreen};
\draw [c] (5.36923,0.686894) -- (5.36923,0.69065);
\draw [c] (5.36923,0.69065) -- (5.36923,0.694407);
\draw [c] (5.36109,0.69065) -- (5.36923,0.69065);
\draw [c] (5.36923,0.69065) -- (5.37736,0.69065);
\definecolor{c}{rgb}{0,0,0};
\colorlet{c}{natgreen};
\draw [c] (5.3855,0.686894) -- (5.3855,0.687595);
\draw [c] (5.3855,0.687595) -- (5.3855,0.688296);
\draw [c] (5.37736,0.687595) -- (5.3855,0.687595);
\draw [c] (5.3855,0.687595) -- (5.39364,0.687595);
\definecolor{c}{rgb}{0,0,0};
\colorlet{c}{natgreen};
\draw [c] (5.40177,0.686894) -- (5.40177,0.694017);
\draw [c] (5.40177,0.694017) -- (5.40177,0.70114);
\draw [c] (5.39364,0.694017) -- (5.40177,0.694017);
\draw [c] (5.40177,0.694017) -- (5.40991,0.694017);
\definecolor{c}{rgb}{0,0,0};
\colorlet{c}{natgreen};
\draw [c] (5.41805,0.686894) -- (5.41805,0.694017);
\draw [c] (5.41805,0.694017) -- (5.41805,0.70114);
\draw [c] (5.40991,0.694017) -- (5.41805,0.694017);
\draw [c] (5.41805,0.694017) -- (5.42618,0.694017);
\definecolor{c}{rgb}{0,0,0};
\colorlet{c}{natgreen};
\draw [c] (5.43432,0.686894) -- (5.43432,0.696318);
\draw [c] (5.43432,0.696318) -- (5.43432,0.705742);
\draw [c] (5.42618,0.696318) -- (5.43432,0.696318);
\draw [c] (5.43432,0.696318) -- (5.44245,0.696318);
\definecolor{c}{rgb}{0,0,0};
\colorlet{c}{natgreen};
\draw [c] (5.45059,0.686894) -- (5.45059,0.691443);
\draw [c] (5.45059,0.691443) -- (5.45059,0.695992);
\draw [c] (5.44245,0.691443) -- (5.45059,0.691443);
\draw [c] (5.45059,0.691443) -- (5.45873,0.691443);
\definecolor{c}{rgb}{0,0,0};
\colorlet{c}{natgreen};
\draw [c] (5.48314,0.686894) -- (5.48314,0.698972);
\draw [c] (5.48314,0.698972) -- (5.48314,0.711049);
\draw [c] (5.475,0.698972) -- (5.48314,0.698972);
\draw [c] (5.48314,0.698972) -- (5.49127,0.698972);
\definecolor{c}{rgb}{0,0,0};
\colorlet{c}{natgreen};
\draw [c] (5.71095,0.686894) -- (5.71095,0.70053);
\draw [c] (5.71095,0.70053) -- (5.71095,0.714166);
\draw [c] (5.70282,0.70053) -- (5.71095,0.70053);
\draw [c] (5.71095,0.70053) -- (5.71909,0.70053);
\definecolor{c}{rgb}{0,0,0};
\colorlet{c}{natgreen};
\draw [c] (5.7435,0.686894) -- (5.7435,0.698334);
\draw [c] (5.7435,0.698334) -- (5.7435,0.709774);
\draw [c] (5.73536,0.698334) -- (5.7435,0.698334);
\draw [c] (5.7435,0.698334) -- (5.75164,0.698334);
\definecolor{c}{rgb}{0,0,0};
\colorlet{c}{natgreen};
\draw [c] (5.79232,0.686894) -- (5.79232,0.69532);
\draw [c] (5.79232,0.69532) -- (5.79232,0.703746);
\draw [c] (5.78418,0.69532) -- (5.79232,0.69532);
\draw [c] (5.79232,0.69532) -- (5.80045,0.69532);
\definecolor{c}{rgb}{0,0,0};
\colorlet{c}{natgreen};
\draw [c] (5.80859,0.686894) -- (5.80859,0.694017);
\draw [c] (5.80859,0.694017) -- (5.80859,0.70114);
\draw [c] (5.80045,0.694017) -- (5.80859,0.694017);
\draw [c] (5.80859,0.694017) -- (5.81673,0.694017);
\definecolor{c}{rgb}{0,0,0};
\colorlet{c}{natgreen};
\draw [c] (5.93877,0.686894) -- (5.93877,0.697253);
\draw [c] (5.93877,0.697253) -- (5.93877,0.707613);
\draw [c] (5.93064,0.697253) -- (5.93877,0.697253);
\draw [c] (5.93877,0.697253) -- (5.94691,0.697253);
\definecolor{c}{rgb}{0,0,0};
\colorlet{c}{natgreen};
\draw [c] (6.39441,0.686894) -- (6.39441,0.698489);
\draw [c] (6.39441,0.698489) -- (6.39441,0.710083);
\draw [c] (6.38627,0.698489) -- (6.39441,0.698489);
\draw [c] (6.39441,0.698489) -- (6.40255,0.698489);
\definecolor{c}{rgb}{0,0,0};
\colorlet{c}{natgreen};
\draw [c] (6.57341,0.686894) -- (6.57341,0.698606);
\draw [c] (6.57341,0.698606) -- (6.57341,0.710318);
\draw [c] (6.56527,0.698606) -- (6.57341,0.698606);
\draw [c] (6.57341,0.698606) -- (6.58155,0.698606);
\definecolor{c}{rgb}{0,0,0};
\colorlet{c}{natgreen};
\draw [c] (7.74505,0.686894) -- (7.74505,0.690097);
\draw [c] (7.74505,0.690097) -- (7.74505,0.6933);
\draw [c] (7.73691,0.690097) -- (7.74505,0.690097);
\draw [c] (7.74505,0.690097) -- (7.75318,0.690097);
\definecolor{c}{rgb}{0,0,0};
\draw [c] (1,0.680516) -- (9.95,0.680516);
\draw [anchor= east] (9.95,0.108883) node[color=c, rotate=0]{$E_{T}^{iso} \text{ [GeV]}$};
\draw [c] (1,0.863234) -- (1,0.680516);
\draw [c] (1.25571,0.771875) -- (1.25571,0.680516);
\draw [c] (1.51143,0.771875) -- (1.51143,0.680516);
\draw [c] (1.76714,0.771875) -- (1.76714,0.680516);
\draw [c] (2.02286,0.771875) -- (2.02286,0.680516);
\draw [c] (2.27857,0.863234) -- (2.27857,0.680516);
\draw [c] (2.53429,0.771875) -- (2.53429,0.680516);
\draw [c] (2.79,0.771875) -- (2.79,0.680516);
\draw [c] (3.04571,0.771875) -- (3.04571,0.680516);
\draw [c] (3.30143,0.771875) -- (3.30143,0.680516);
\draw [c] (3.55714,0.863234) -- (3.55714,0.680516);
\draw [c] (3.81286,0.771875) -- (3.81286,0.680516);
\draw [c] (4.06857,0.771875) -- (4.06857,0.680516);
\draw [c] (4.32429,0.771875) -- (4.32429,0.680516);
\draw [c] (4.58,0.771875) -- (4.58,0.680516);
\draw [c] (4.83571,0.863234) -- (4.83571,0.680516);
\draw [c] (5.09143,0.771875) -- (5.09143,0.680516);
\draw [c] (5.34714,0.771875) -- (5.34714,0.680516);
\draw [c] (5.60286,0.771875) -- (5.60286,0.680516);
\draw [c] (5.85857,0.771875) -- (5.85857,0.680516);
\draw [c] (6.11429,0.863234) -- (6.11429,0.680516);
\draw [c] (6.37,0.771875) -- (6.37,0.680516);
\draw [c] (6.62571,0.771875) -- (6.62571,0.680516);
\draw [c] (6.88143,0.771875) -- (6.88143,0.680516);
\draw [c] (7.13714,0.771875) -- (7.13714,0.680516);
\draw [c] (7.39286,0.863234) -- (7.39286,0.680516);
\draw [c] (7.64857,0.771875) -- (7.64857,0.680516);
\draw [c] (7.90429,0.771875) -- (7.90429,0.680516);
\draw [c] (8.16,0.771875) -- (8.16,0.680516);
\draw [c] (8.41571,0.771875) -- (8.41571,0.680516);
\draw [c] (8.67143,0.863234) -- (8.67143,0.680516);
\draw [c] (8.92714,0.771875) -- (8.92714,0.680516);
\draw [c] (9.18286,0.771875) -- (9.18286,0.680516);
\draw [c] (9.43857,0.771875) -- (9.43857,0.680516);
\draw [c] (9.69429,0.771875) -- (9.69429,0.680516);
\draw [c] (9.95,0.863234) -- (9.95,0.680516);
\draw [c] (9.95,0.863234) -- (9.95,0.680516);
\draw [anchor=base] (1,0.353868) node[color=c, rotate=0]{-20};
\draw [anchor=base] (2.27857,0.353868) node[color=c, rotate=0]{-10};
\draw [anchor=base] (3.55714,0.353868) node[color=c, rotate=0]{0};
\draw [anchor=base] (4.83571,0.353868) node[color=c, rotate=0]{10};
\draw [anchor=base] (6.11429,0.353868) node[color=c, rotate=0]{20};
\draw [anchor=base] (7.39286,0.353868) node[color=c, rotate=0]{30};
\draw [anchor=base] (8.67143,0.353868) node[color=c, rotate=0]{40};
\draw [anchor=base] (9.95,0.353868) node[color=c, rotate=0]{50};
\draw [c] (1,0.680516) -- (1,6.73711);
\draw [anchor= east] (-0.12,6.73711) node[color=c, rotate=90]{Normalised number of events};
\draw [c] (1.267,0.686894) -- (1,0.686894);
\draw [c] (1.1335,0.960568) -- (1,0.960568);
\draw [c] (1.1335,1.23424) -- (1,1.23424);
\draw [c] (1.1335,1.50792) -- (1,1.50792);
\draw [c] (1.267,1.78159) -- (1,1.78159);
\draw [c] (1.1335,2.05527) -- (1,2.05527);
\draw [c] (1.1335,2.32894) -- (1,2.32894);
\draw [c] (1.1335,2.60262) -- (1,2.60262);
\draw [c] (1.267,2.87629) -- (1,2.87629);
\draw [c] (1.1335,3.14996) -- (1,3.14996);
\draw [c] (1.1335,3.42364) -- (1,3.42364);
\draw [c] (1.1335,3.69731) -- (1,3.69731);
\draw [c] (1.267,3.97099) -- (1,3.97099);
\draw [c] (1.1335,4.24466) -- (1,4.24466);
\draw [c] (1.1335,4.51834) -- (1,4.51834);
\draw [c] (1.1335,4.79201) -- (1,4.79201);
\draw [c] (1.267,5.06569) -- (1,5.06569);
\draw [c] (1.1335,5.33936) -- (1,5.33936);
\draw [c] (1.1335,5.61304) -- (1,5.61304);
\draw [c] (1.1335,5.88671) -- (1,5.88671);
\draw [c] (1.267,6.16039) -- (1,6.16039);
\draw [c] (1.267,0.686894) -- (1,0.686894);
\draw [c] (1.267,6.16039) -- (1,6.16039);
\draw [c] (1.1335,6.43406) -- (1,6.43406);
\draw [c] (1.1335,6.70773) -- (1,6.70773);
\draw [anchor= east] (0.95,0.686894) node[color=c, rotate=0]{0};
\draw [anchor= east] (0.95,1.78159) node[color=c, rotate=0]{200};
\draw [anchor= east] (0.95,2.87629) node[color=c, rotate=0]{400};
\draw [anchor= east] (0.95,3.97099) node[color=c, rotate=0]{600};
\draw [anchor= east] (0.95,5.06569) node[color=c, rotate=0]{800};
\draw [anchor= east] (0.95,6.16039) node[color=c, rotate=0]{1000};
\colorlet{c}{natcomp!70};
\draw [c] (1.04068,0.686894) -- (1.04068,0.6869);
\draw [c] (1.04068,0.6869) -- (1.04068,0.686907);
\draw [c] (1.03255,0.6869) -- (1.04068,0.6869);
\draw [c] (1.04068,0.6869) -- (1.04882,0.6869);
\definecolor{c}{rgb}{0,0,0};
\colorlet{c}{natcomp!70};
\draw [c] (1.10577,0.686894) -- (1.10577,0.705403);
\draw [c] (1.10577,0.705403) -- (1.10577,0.723913);
\draw [c] (1.09764,0.705403) -- (1.10577,0.705403);
\draw [c] (1.10577,0.705403) -- (1.11391,0.705403);
\definecolor{c}{rgb}{0,0,0};
\colorlet{c}{natcomp!70};
\draw [c] (1.12205,0.6869) -- (1.12205,0.708668);
\draw [c] (1.12205,0.708668) -- (1.12205,0.730436);
\draw [c] (1.11391,0.708668) -- (1.12205,0.708668);
\draw [c] (1.12205,0.708668) -- (1.13018,0.708668);
\definecolor{c}{rgb}{0,0,0};
\colorlet{c}{natcomp!70};
\draw [c] (1.13832,0.686894) -- (1.13832,0.686899);
\draw [c] (1.13832,0.686899) -- (1.13832,0.686904);
\draw [c] (1.13018,0.686899) -- (1.13832,0.686899);
\draw [c] (1.13832,0.686899) -- (1.14645,0.686899);
\definecolor{c}{rgb}{0,0,0};
\colorlet{c}{natcomp!70};
\draw [c] (1.15459,0.686907) -- (1.15459,0.695046);
\draw [c] (1.15459,0.695046) -- (1.15459,0.703185);
\draw [c] (1.14645,0.695046) -- (1.15459,0.695046);
\draw [c] (1.15459,0.695046) -- (1.16273,0.695046);
\definecolor{c}{rgb}{0,0,0};
\colorlet{c}{natcomp!70};
\draw [c] (1.17086,0.686894) -- (1.17086,0.686904);
\draw [c] (1.17086,0.686904) -- (1.17086,0.686914);
\draw [c] (1.16273,0.686904) -- (1.17086,0.686904);
\draw [c] (1.17086,0.686904) -- (1.179,0.686904);
\definecolor{c}{rgb}{0,0,0};
\colorlet{c}{natcomp!70};
\draw [c] (1.18714,0.686897) -- (1.18714,0.686904);
\draw [c] (1.18714,0.686904) -- (1.18714,0.686911);
\draw [c] (1.179,0.686904) -- (1.18714,0.686904);
\draw [c] (1.18714,0.686904) -- (1.19527,0.686904);
\definecolor{c}{rgb}{0,0,0};
\colorlet{c}{natcomp!70};
\draw [c] (1.20341,0.686906) -- (1.20341,0.705415);
\draw [c] (1.20341,0.705415) -- (1.20341,0.723925);
\draw [c] (1.19527,0.705415) -- (1.20341,0.705415);
\draw [c] (1.20341,0.705415) -- (1.21155,0.705415);
\definecolor{c}{rgb}{0,0,0};
\colorlet{c}{natcomp!70};
\draw [c] (1.21968,0.686898) -- (1.21968,0.68691);
\draw [c] (1.21968,0.68691) -- (1.21968,0.686923);
\draw [c] (1.21155,0.68691) -- (1.21968,0.68691);
\draw [c] (1.21968,0.68691) -- (1.22782,0.68691);
\definecolor{c}{rgb}{0,0,0};
\colorlet{c}{natcomp!70};
\draw [c] (1.25223,0.695733) -- (1.25223,0.723286);
\draw [c] (1.25223,0.723286) -- (1.25223,0.75084);
\draw [c] (1.24409,0.723286) -- (1.25223,0.723286);
\draw [c] (1.25223,0.723286) -- (1.26036,0.723286);
\definecolor{c}{rgb}{0,0,0};
\colorlet{c}{natcomp!70};
\draw [c] (1.2685,0.686901) -- (1.2685,0.686911);
\draw [c] (1.2685,0.686911) -- (1.2685,0.686922);
\draw [c] (1.26036,0.686911) -- (1.2685,0.686911);
\draw [c] (1.2685,0.686911) -- (1.27664,0.686911);
\definecolor{c}{rgb}{0,0,0};
\colorlet{c}{natcomp!70};
\draw [c] (1.28477,0.686898) -- (1.28477,0.686915);
\draw [c] (1.28477,0.686915) -- (1.28477,0.686932);
\draw [c] (1.27664,0.686915) -- (1.28477,0.686915);
\draw [c] (1.28477,0.686915) -- (1.29291,0.686915);
\definecolor{c}{rgb}{0,0,0};
\colorlet{c}{natcomp!70};
\draw [c] (1.31732,0.694427) -- (1.31732,0.721188);
\draw [c] (1.31732,0.721188) -- (1.31732,0.747948);
\draw [c] (1.30918,0.721188) -- (1.31732,0.721188);
\draw [c] (1.31732,0.721188) -- (1.32545,0.721188);
\definecolor{c}{rgb}{0,0,0};
\colorlet{c}{natcomp!70};
\draw [c] (1.34986,0.686899) -- (1.34986,0.69462);
\draw [c] (1.34986,0.69462) -- (1.34986,0.702342);
\draw [c] (1.34173,0.69462) -- (1.34986,0.69462);
\draw [c] (1.34986,0.69462) -- (1.358,0.69462);
\definecolor{c}{rgb}{0,0,0};
\colorlet{c}{natcomp!70};
\draw [c] (1.36614,0.686894) -- (1.36614,0.6869);
\draw [c] (1.36614,0.6869) -- (1.36614,0.686907);
\draw [c] (1.358,0.6869) -- (1.36614,0.6869);
\draw [c] (1.36614,0.6869) -- (1.37427,0.6869);
\definecolor{c}{rgb}{0,0,0};
\colorlet{c}{natcomp!70};
\draw [c] (1.38241,0.68691) -- (1.38241,0.71207);
\draw [c] (1.38241,0.71207) -- (1.38241,0.73723);
\draw [c] (1.37427,0.71207) -- (1.38241,0.71207);
\draw [c] (1.38241,0.71207) -- (1.39055,0.71207);
\definecolor{c}{rgb}{0,0,0};
\colorlet{c}{natcomp!70};
\draw [c] (1.39868,0.696781) -- (1.39868,0.722664);
\draw [c] (1.39868,0.722664) -- (1.39868,0.748546);
\draw [c] (1.39055,0.722664) -- (1.39868,0.722664);
\draw [c] (1.39868,0.722664) -- (1.40682,0.722664);
\definecolor{c}{rgb}{0,0,0};
\colorlet{c}{natcomp!70};
\draw [c] (1.41495,0.686894) -- (1.41495,0.697305);
\draw [c] (1.41495,0.697305) -- (1.41495,0.707716);
\draw [c] (1.40682,0.697305) -- (1.41495,0.697305);
\draw [c] (1.41495,0.697305) -- (1.42309,0.697305);
\definecolor{c}{rgb}{0,0,0};
\colorlet{c}{natcomp!70};
\draw [c] (1.43123,0.6869) -- (1.43123,0.700903);
\draw [c] (1.43123,0.700903) -- (1.43123,0.714905);
\draw [c] (1.42309,0.700903) -- (1.43123,0.700903);
\draw [c] (1.43123,0.700903) -- (1.43936,0.700903);
\definecolor{c}{rgb}{0,0,0};
\colorlet{c}{natcomp!70};
\draw [c] (1.4475,0.686909) -- (1.4475,0.705418);
\draw [c] (1.4475,0.705418) -- (1.4475,0.723928);
\draw [c] (1.43936,0.705418) -- (1.4475,0.705418);
\draw [c] (1.4475,0.705418) -- (1.45564,0.705418);
\definecolor{c}{rgb}{0,0,0};
\colorlet{c}{natcomp!70};
\draw [c] (1.46377,0.686915) -- (1.46377,0.705424);
\draw [c] (1.46377,0.705424) -- (1.46377,0.723933);
\draw [c] (1.45564,0.705424) -- (1.46377,0.705424);
\draw [c] (1.46377,0.705424) -- (1.47191,0.705424);
\definecolor{c}{rgb}{0,0,0};
\colorlet{c}{natcomp!70};
\draw [c] (1.48005,0.6869) -- (1.48005,0.686916);
\draw [c] (1.48005,0.686916) -- (1.48005,0.686932);
\draw [c] (1.47191,0.686916) -- (1.48005,0.686916);
\draw [c] (1.48005,0.686916) -- (1.48818,0.686916);
\definecolor{c}{rgb}{0,0,0};
\colorlet{c}{natcomp!70};
\draw [c] (1.49632,0.686904) -- (1.49632,0.700906);
\draw [c] (1.49632,0.700906) -- (1.49632,0.714908);
\draw [c] (1.48818,0.700906) -- (1.49632,0.700906);
\draw [c] (1.49632,0.700906) -- (1.50445,0.700906);
\definecolor{c}{rgb}{0,0,0};
\colorlet{c}{natcomp!70};
\draw [c] (1.51259,0.686912) -- (1.51259,0.698144);
\draw [c] (1.51259,0.698144) -- (1.51259,0.709377);
\draw [c] (1.50445,0.698144) -- (1.51259,0.698144);
\draw [c] (1.51259,0.698144) -- (1.52073,0.698144);
\definecolor{c}{rgb}{0,0,0};
\colorlet{c}{natcomp!70};
\draw [c] (1.52886,0.686905) -- (1.52886,0.697316);
\draw [c] (1.52886,0.697316) -- (1.52886,0.707727);
\draw [c] (1.52073,0.697316) -- (1.52886,0.697316);
\draw [c] (1.52886,0.697316) -- (1.537,0.697316);
\definecolor{c}{rgb}{0,0,0};
\colorlet{c}{natcomp!70};
\draw [c] (1.54514,0.68691) -- (1.54514,0.686925);
\draw [c] (1.54514,0.686925) -- (1.54514,0.68694);
\draw [c] (1.537,0.686925) -- (1.54514,0.686925);
\draw [c] (1.54514,0.686925) -- (1.55327,0.686925);
\definecolor{c}{rgb}{0,0,0};
\colorlet{c}{natcomp!70};
\draw [c] (1.56141,0.686899) -- (1.56141,0.686912);
\draw [c] (1.56141,0.686912) -- (1.56141,0.686926);
\draw [c] (1.55327,0.686912) -- (1.56141,0.686912);
\draw [c] (1.56141,0.686912) -- (1.56955,0.686912);
\definecolor{c}{rgb}{0,0,0};
\colorlet{c}{natcomp!70};
\draw [c] (1.57768,0.686904) -- (1.57768,0.705413);
\draw [c] (1.57768,0.705413) -- (1.57768,0.723923);
\draw [c] (1.56955,0.705413) -- (1.57768,0.705413);
\draw [c] (1.57768,0.705413) -- (1.58582,0.705413);
\definecolor{c}{rgb}{0,0,0};
\colorlet{c}{natcomp!70};
\draw [c] (1.59395,0.686923) -- (1.59395,0.696817);
\draw [c] (1.59395,0.696817) -- (1.59395,0.706711);
\draw [c] (1.58582,0.696817) -- (1.59395,0.696817);
\draw [c] (1.59395,0.696817) -- (1.60209,0.696817);
\definecolor{c}{rgb}{0,0,0};
\colorlet{c}{natcomp!70};
\draw [c] (1.61023,0.686911) -- (1.61023,0.719865);
\draw [c] (1.61023,0.719865) -- (1.61023,0.752819);
\draw [c] (1.60209,0.719865) -- (1.61023,0.719865);
\draw [c] (1.61023,0.719865) -- (1.61836,0.719865);
\definecolor{c}{rgb}{0,0,0};
\colorlet{c}{natcomp!70};
\draw [c] (1.6265,0.692256) -- (1.6265,0.705147);
\draw [c] (1.6265,0.705147) -- (1.6265,0.718038);
\draw [c] (1.61836,0.705147) -- (1.6265,0.705147);
\draw [c] (1.6265,0.705147) -- (1.63464,0.705147);
\definecolor{c}{rgb}{0,0,0};
\colorlet{c}{natcomp!70};
\draw [c] (1.64277,0.686912) -- (1.64277,0.686932);
\draw [c] (1.64277,0.686932) -- (1.64277,0.686952);
\draw [c] (1.63464,0.686932) -- (1.64277,0.686932);
\draw [c] (1.64277,0.686932) -- (1.65091,0.686932);
\definecolor{c}{rgb}{0,0,0};
\colorlet{c}{natcomp!70};
\draw [c] (1.65905,0.692192) -- (1.65905,0.704906);
\draw [c] (1.65905,0.704906) -- (1.65905,0.717619);
\draw [c] (1.65091,0.704906) -- (1.65905,0.704906);
\draw [c] (1.65905,0.704906) -- (1.66718,0.704906);
\definecolor{c}{rgb}{0,0,0};
\colorlet{c}{natcomp!70};
\draw [c] (1.67532,0.700693) -- (1.67532,0.723056);
\draw [c] (1.67532,0.723056) -- (1.67532,0.74542);
\draw [c] (1.66718,0.723056) -- (1.67532,0.723056);
\draw [c] (1.67532,0.723056) -- (1.68345,0.723056);
\definecolor{c}{rgb}{0,0,0};
\colorlet{c}{natcomp!70};
\draw [c] (1.69159,0.686942) -- (1.69159,0.708221);
\draw [c] (1.69159,0.708221) -- (1.69159,0.729499);
\draw [c] (1.68345,0.708221) -- (1.69159,0.708221);
\draw [c] (1.69159,0.708221) -- (1.69973,0.708221);
\definecolor{c}{rgb}{0,0,0};
\colorlet{c}{natcomp!70};
\draw [c] (1.70786,0.705683) -- (1.70786,0.724842);
\draw [c] (1.70786,0.724842) -- (1.70786,0.744);
\draw [c] (1.69973,0.724842) -- (1.70786,0.724842);
\draw [c] (1.70786,0.724842) -- (1.716,0.724842);
\definecolor{c}{rgb}{0,0,0};
\colorlet{c}{natcomp!70};
\draw [c] (1.72414,0.686939) -- (1.72414,0.700941);
\draw [c] (1.72414,0.700941) -- (1.72414,0.714943);
\draw [c] (1.716,0.700941) -- (1.72414,0.700941);
\draw [c] (1.72414,0.700941) -- (1.73227,0.700941);
\definecolor{c}{rgb}{0,0,0};
\colorlet{c}{natcomp!70};
\draw [c] (1.74041,0.686936) -- (1.74041,0.708215);
\draw [c] (1.74041,0.708215) -- (1.74041,0.729493);
\draw [c] (1.73227,0.708215) -- (1.74041,0.708215);
\draw [c] (1.74041,0.708215) -- (1.74855,0.708215);
\definecolor{c}{rgb}{0,0,0};
\colorlet{c}{natcomp!70};
\draw [c] (1.75668,0.686903) -- (1.75668,0.696018);
\draw [c] (1.75668,0.696018) -- (1.75668,0.705133);
\draw [c] (1.74855,0.696018) -- (1.75668,0.696018);
\draw [c] (1.75668,0.696018) -- (1.76482,0.696018);
\definecolor{c}{rgb}{0,0,0};
\colorlet{c}{natcomp!70};
\draw [c] (1.77295,0.714667) -- (1.77295,0.738024);
\draw [c] (1.77295,0.738024) -- (1.77295,0.761381);
\draw [c] (1.76482,0.738024) -- (1.77295,0.738024);
\draw [c] (1.77295,0.738024) -- (1.78109,0.738024);
\definecolor{c}{rgb}{0,0,0};
\colorlet{c}{natcomp!70};
\draw [c] (1.78923,0.713535) -- (1.78923,0.735176);
\draw [c] (1.78923,0.735176) -- (1.78923,0.756817);
\draw [c] (1.78109,0.735176) -- (1.78923,0.735176);
\draw [c] (1.78923,0.735176) -- (1.79736,0.735176);
\definecolor{c}{rgb}{0,0,0};
\colorlet{c}{natcomp!70};
\draw [c] (1.8055,0.706706) -- (1.8055,0.727668);
\draw [c] (1.8055,0.727668) -- (1.8055,0.748629);
\draw [c] (1.79736,0.727668) -- (1.8055,0.727668);
\draw [c] (1.8055,0.727668) -- (1.81364,0.727668);
\definecolor{c}{rgb}{0,0,0};
\colorlet{c}{natcomp!70};
\draw [c] (1.82177,0.71253) -- (1.82177,0.7417);
\draw [c] (1.82177,0.7417) -- (1.82177,0.77087);
\draw [c] (1.81364,0.7417) -- (1.82177,0.7417);
\draw [c] (1.82177,0.7417) -- (1.82991,0.7417);
\definecolor{c}{rgb}{0,0,0};
\colorlet{c}{natcomp!70};
\draw [c] (1.83805,0.686939) -- (1.83805,0.698172);
\draw [c] (1.83805,0.698172) -- (1.83805,0.709404);
\draw [c] (1.82991,0.698172) -- (1.83805,0.698172);
\draw [c] (1.83805,0.698172) -- (1.84618,0.698172);
\definecolor{c}{rgb}{0,0,0};
\colorlet{c}{natcomp!70};
\draw [c] (1.85432,0.693926) -- (1.85432,0.714558);
\draw [c] (1.85432,0.714558) -- (1.85432,0.73519);
\draw [c] (1.84618,0.714558) -- (1.85432,0.714558);
\draw [c] (1.85432,0.714558) -- (1.86245,0.714558);
\definecolor{c}{rgb}{0,0,0};
\colorlet{c}{natcomp!70};
\draw [c] (1.87059,0.702794) -- (1.87059,0.726598);
\draw [c] (1.87059,0.726598) -- (1.87059,0.750403);
\draw [c] (1.86245,0.726598) -- (1.87059,0.726598);
\draw [c] (1.87059,0.726598) -- (1.87873,0.726598);
\definecolor{c}{rgb}{0,0,0};
\colorlet{c}{natcomp!70};
\draw [c] (1.88686,0.723513) -- (1.88686,0.749566);
\draw [c] (1.88686,0.749566) -- (1.88686,0.775619);
\draw [c] (1.87873,0.749566) -- (1.88686,0.749566);
\draw [c] (1.88686,0.749566) -- (1.895,0.749566);
\definecolor{c}{rgb}{0,0,0};
\colorlet{c}{natcomp!70};
\draw [c] (1.90314,0.703651) -- (1.90314,0.728586);
\draw [c] (1.90314,0.728586) -- (1.90314,0.753521);
\draw [c] (1.895,0.728586) -- (1.90314,0.728586);
\draw [c] (1.90314,0.728586) -- (1.91127,0.728586);
\definecolor{c}{rgb}{0,0,0};
\colorlet{c}{natcomp!70};
\draw [c] (1.91941,0.735552) -- (1.91941,0.762435);
\draw [c] (1.91941,0.762435) -- (1.91941,0.789317);
\draw [c] (1.91127,0.762435) -- (1.91941,0.762435);
\draw [c] (1.91941,0.762435) -- (1.92755,0.762435);
\definecolor{c}{rgb}{0,0,0};
\colorlet{c}{natcomp!70};
\draw [c] (1.93568,0.705159) -- (1.93568,0.723496);
\draw [c] (1.93568,0.723496) -- (1.93568,0.741834);
\draw [c] (1.92755,0.723496) -- (1.93568,0.723496);
\draw [c] (1.93568,0.723496) -- (1.94382,0.723496);
\definecolor{c}{rgb}{0,0,0};
\colorlet{c}{natcomp!70};
\draw [c] (1.95195,0.728073) -- (1.95195,0.759278);
\draw [c] (1.95195,0.759278) -- (1.95195,0.790483);
\draw [c] (1.94382,0.759278) -- (1.95195,0.759278);
\draw [c] (1.95195,0.759278) -- (1.96009,0.759278);
\definecolor{c}{rgb}{0,0,0};
\colorlet{c}{natcomp!70};
\draw [c] (1.96823,0.721308) -- (1.96823,0.745166);
\draw [c] (1.96823,0.745166) -- (1.96823,0.769024);
\draw [c] (1.96009,0.745166) -- (1.96823,0.745166);
\draw [c] (1.96823,0.745166) -- (1.97636,0.745166);
\definecolor{c}{rgb}{0,0,0};
\colorlet{c}{natcomp!70};
\draw [c] (1.9845,0.717672) -- (1.9845,0.74667);
\draw [c] (1.9845,0.74667) -- (1.9845,0.775667);
\draw [c] (1.97636,0.74667) -- (1.9845,0.74667);
\draw [c] (1.9845,0.74667) -- (1.99264,0.74667);
\definecolor{c}{rgb}{0,0,0};
\colorlet{c}{natcomp!70};
\draw [c] (2.00077,0.754545) -- (2.00077,0.791474);
\draw [c] (2.00077,0.791474) -- (2.00077,0.828402);
\draw [c] (1.99264,0.791474) -- (2.00077,0.791474);
\draw [c] (2.00077,0.791474) -- (2.00891,0.791474);
\definecolor{c}{rgb}{0,0,0};
\colorlet{c}{natcomp!70};
\draw [c] (2.01705,0.812339) -- (2.01705,0.861065);
\draw [c] (2.01705,0.861065) -- (2.01705,0.909792);
\draw [c] (2.00891,0.861065) -- (2.01705,0.861065);
\draw [c] (2.01705,0.861065) -- (2.02518,0.861065);
\definecolor{c}{rgb}{0,0,0};
\colorlet{c}{natcomp!70};
\draw [c] (2.03332,0.756859) -- (2.03332,0.788069);
\draw [c] (2.03332,0.788069) -- (2.03332,0.819279);
\draw [c] (2.02518,0.788069) -- (2.03332,0.788069);
\draw [c] (2.03332,0.788069) -- (2.04145,0.788069);
\definecolor{c}{rgb}{0,0,0};
\colorlet{c}{natcomp!70};
\draw [c] (2.04959,0.733143) -- (2.04959,0.772578);
\draw [c] (2.04959,0.772578) -- (2.04959,0.812014);
\draw [c] (2.04145,0.772578) -- (2.04959,0.772578);
\draw [c] (2.04959,0.772578) -- (2.05773,0.772578);
\definecolor{c}{rgb}{0,0,0};
\colorlet{c}{natcomp!70};
\draw [c] (2.06586,0.734846) -- (2.06586,0.767565);
\draw [c] (2.06586,0.767565) -- (2.06586,0.800284);
\draw [c] (2.05773,0.767565) -- (2.06586,0.767565);
\draw [c] (2.06586,0.767565) -- (2.074,0.767565);
\definecolor{c}{rgb}{0,0,0};
\colorlet{c}{natcomp!70};
\draw [c] (2.08214,0.789195) -- (2.08214,0.827149);
\draw [c] (2.08214,0.827149) -- (2.08214,0.865104);
\draw [c] (2.074,0.827149) -- (2.08214,0.827149);
\draw [c] (2.08214,0.827149) -- (2.09027,0.827149);
\definecolor{c}{rgb}{0,0,0};
\colorlet{c}{natcomp!70};
\draw [c] (2.09841,0.788521) -- (2.09841,0.828396);
\draw [c] (2.09841,0.828396) -- (2.09841,0.868271);
\draw [c] (2.09027,0.828396) -- (2.09841,0.828396);
\draw [c] (2.09841,0.828396) -- (2.10655,0.828396);
\definecolor{c}{rgb}{0,0,0};
\colorlet{c}{natcomp!70};
\draw [c] (2.11468,0.759161) -- (2.11468,0.799535);
\draw [c] (2.11468,0.799535) -- (2.11468,0.839908);
\draw [c] (2.10655,0.799535) -- (2.11468,0.799535);
\draw [c] (2.11468,0.799535) -- (2.12282,0.799535);
\definecolor{c}{rgb}{0,0,0};
\colorlet{c}{natcomp!70};
\draw [c] (2.13095,0.7679) -- (2.13095,0.805702);
\draw [c] (2.13095,0.805702) -- (2.13095,0.843503);
\draw [c] (2.12282,0.805702) -- (2.13095,0.805702);
\draw [c] (2.13095,0.805702) -- (2.13909,0.805702);
\definecolor{c}{rgb}{0,0,0};
\colorlet{c}{natcomp!70};
\draw [c] (2.14723,0.746911) -- (2.14723,0.783535);
\draw [c] (2.14723,0.783535) -- (2.14723,0.820159);
\draw [c] (2.13909,0.783535) -- (2.14723,0.783535);
\draw [c] (2.14723,0.783535) -- (2.15536,0.783535);
\definecolor{c}{rgb}{0,0,0};
\colorlet{c}{natcomp!70};
\draw [c] (2.1635,0.772802) -- (2.1635,0.808728);
\draw [c] (2.1635,0.808728) -- (2.1635,0.844654);
\draw [c] (2.15536,0.808728) -- (2.1635,0.808728);
\draw [c] (2.1635,0.808728) -- (2.17164,0.808728);
\definecolor{c}{rgb}{0,0,0};
\colorlet{c}{natcomp!70};
\draw [c] (2.17977,0.810552) -- (2.17977,0.849296);
\draw [c] (2.17977,0.849296) -- (2.17977,0.888039);
\draw [c] (2.17164,0.849296) -- (2.17977,0.849296);
\draw [c] (2.17977,0.849296) -- (2.18791,0.849296);
\definecolor{c}{rgb}{0,0,0};
\colorlet{c}{natcomp!70};
\draw [c] (2.19605,0.778252) -- (2.19605,0.813555);
\draw [c] (2.19605,0.813555) -- (2.19605,0.848858);
\draw [c] (2.18791,0.813555) -- (2.19605,0.813555);
\draw [c] (2.19605,0.813555) -- (2.20418,0.813555);
\definecolor{c}{rgb}{0,0,0};
\colorlet{c}{natcomp!70};
\draw [c] (2.21232,0.789468) -- (2.21232,0.827907);
\draw [c] (2.21232,0.827907) -- (2.21232,0.866346);
\draw [c] (2.20418,0.827907) -- (2.21232,0.827907);
\draw [c] (2.21232,0.827907) -- (2.22045,0.827907);
\definecolor{c}{rgb}{0,0,0};
\colorlet{c}{natcomp!70};
\draw [c] (2.22859,0.815059) -- (2.22859,0.859924);
\draw [c] (2.22859,0.859924) -- (2.22859,0.904788);
\draw [c] (2.22045,0.859924) -- (2.22859,0.859924);
\draw [c] (2.22859,0.859924) -- (2.23673,0.859924);
\definecolor{c}{rgb}{0,0,0};
\colorlet{c}{natcomp!70};
\draw [c] (2.24486,0.802412) -- (2.24486,0.839797);
\draw [c] (2.24486,0.839797) -- (2.24486,0.877183);
\draw [c] (2.23673,0.839797) -- (2.24486,0.839797);
\draw [c] (2.24486,0.839797) -- (2.253,0.839797);
\definecolor{c}{rgb}{0,0,0};
\colorlet{c}{natcomp!70};
\draw [c] (2.26114,0.840272) -- (2.26114,0.885266);
\draw [c] (2.26114,0.885266) -- (2.26114,0.93026);
\draw [c] (2.253,0.885266) -- (2.26114,0.885266);
\draw [c] (2.26114,0.885266) -- (2.26927,0.885266);
\definecolor{c}{rgb}{0,0,0};
\colorlet{c}{natcomp!70};
\draw [c] (2.27741,0.805072) -- (2.27741,0.848826);
\draw [c] (2.27741,0.848826) -- (2.27741,0.892579);
\draw [c] (2.26927,0.848826) -- (2.27741,0.848826);
\draw [c] (2.27741,0.848826) -- (2.28555,0.848826);
\definecolor{c}{rgb}{0,0,0};
\colorlet{c}{natcomp!70};
\draw [c] (2.29368,0.791504) -- (2.29368,0.826715);
\draw [c] (2.29368,0.826715) -- (2.29368,0.861925);
\draw [c] (2.28555,0.826715) -- (2.29368,0.826715);
\draw [c] (2.29368,0.826715) -- (2.30182,0.826715);
\definecolor{c}{rgb}{0,0,0};
\colorlet{c}{natcomp!70};
\draw [c] (2.30995,0.856209) -- (2.30995,0.906613);
\draw [c] (2.30995,0.906613) -- (2.30995,0.957017);
\draw [c] (2.30182,0.906613) -- (2.30995,0.906613);
\draw [c] (2.30995,0.906613) -- (2.31809,0.906613);
\definecolor{c}{rgb}{0,0,0};
\colorlet{c}{natcomp!70};
\draw [c] (2.32623,0.828277) -- (2.32623,0.869435);
\draw [c] (2.32623,0.869435) -- (2.32623,0.910592);
\draw [c] (2.31809,0.869435) -- (2.32623,0.869435);
\draw [c] (2.32623,0.869435) -- (2.33436,0.869435);
\definecolor{c}{rgb}{0,0,0};
\colorlet{c}{natcomp!70};
\draw [c] (2.3425,0.88625) -- (2.3425,0.939801);
\draw [c] (2.3425,0.939801) -- (2.3425,0.993352);
\draw [c] (2.33436,0.939801) -- (2.3425,0.939801);
\draw [c] (2.3425,0.939801) -- (2.35064,0.939801);
\definecolor{c}{rgb}{0,0,0};
\colorlet{c}{natcomp!70};
\draw [c] (2.35877,0.916275) -- (2.35877,0.969867);
\draw [c] (2.35877,0.969867) -- (2.35877,1.02346);
\draw [c] (2.35064,0.969867) -- (2.35877,0.969867);
\draw [c] (2.35877,0.969867) -- (2.36691,0.969867);
\definecolor{c}{rgb}{0,0,0};
\colorlet{c}{natcomp!70};
\draw [c] (2.37505,0.907296) -- (2.37505,0.962405);
\draw [c] (2.37505,0.962405) -- (2.37505,1.01751);
\draw [c] (2.36691,0.962405) -- (2.37505,0.962405);
\draw [c] (2.37505,0.962405) -- (2.38318,0.962405);
\definecolor{c}{rgb}{0,0,0};
\colorlet{c}{natcomp!70};
\draw [c] (2.39132,0.827878) -- (2.39132,0.86913);
\draw [c] (2.39132,0.86913) -- (2.39132,0.910383);
\draw [c] (2.38318,0.86913) -- (2.39132,0.86913);
\draw [c] (2.39132,0.86913) -- (2.39945,0.86913);
\definecolor{c}{rgb}{0,0,0};
\colorlet{c}{natcomp!70};
\draw [c] (2.40759,0.857178) -- (2.40759,0.910918);
\draw [c] (2.40759,0.910918) -- (2.40759,0.964658);
\draw [c] (2.39945,0.910918) -- (2.40759,0.910918);
\draw [c] (2.40759,0.910918) -- (2.41573,0.910918);
\definecolor{c}{rgb}{0,0,0};
\colorlet{c}{natcomp!70};
\draw [c] (2.42386,0.919688) -- (2.42386,0.974446);
\draw [c] (2.42386,0.974446) -- (2.42386,1.0292);
\draw [c] (2.41573,0.974446) -- (2.42386,0.974446);
\draw [c] (2.42386,0.974446) -- (2.432,0.974446);
\definecolor{c}{rgb}{0,0,0};
\colorlet{c}{natcomp!70};
\draw [c] (2.44014,0.898282) -- (2.44014,0.948641);
\draw [c] (2.44014,0.948641) -- (2.44014,0.999);
\draw [c] (2.432,0.948641) -- (2.44014,0.948641);
\draw [c] (2.44014,0.948641) -- (2.44827,0.948641);
\definecolor{c}{rgb}{0,0,0};
\colorlet{c}{natcomp!70};
\draw [c] (2.45641,0.910429) -- (2.45641,0.968628);
\draw [c] (2.45641,0.968628) -- (2.45641,1.02683);
\draw [c] (2.44827,0.968628) -- (2.45641,0.968628);
\draw [c] (2.45641,0.968628) -- (2.46455,0.968628);
\definecolor{c}{rgb}{0,0,0};
\colorlet{c}{natcomp!70};
\draw [c] (2.47268,0.884339) -- (2.47268,0.931852);
\draw [c] (2.47268,0.931852) -- (2.47268,0.979366);
\draw [c] (2.46455,0.931852) -- (2.47268,0.931852);
\draw [c] (2.47268,0.931852) -- (2.48082,0.931852);
\definecolor{c}{rgb}{0,0,0};
\colorlet{c}{natcomp!70};
\draw [c] (2.48895,0.969757) -- (2.48895,1.0318);
\draw [c] (2.48895,1.0318) -- (2.48895,1.09385);
\draw [c] (2.48082,1.0318) -- (2.48895,1.0318);
\draw [c] (2.48895,1.0318) -- (2.49709,1.0318);
\definecolor{c}{rgb}{0,0,0};
\colorlet{c}{natcomp!70};
\draw [c] (2.50523,0.971435) -- (2.50523,1.03092);
\draw [c] (2.50523,1.03092) -- (2.50523,1.0904);
\draw [c] (2.49709,1.03092) -- (2.50523,1.03092);
\draw [c] (2.50523,1.03092) -- (2.51336,1.03092);
\definecolor{c}{rgb}{0,0,0};
\colorlet{c}{natcomp!70};
\draw [c] (2.5215,0.990909) -- (2.5215,1.05577);
\draw [c] (2.5215,1.05577) -- (2.5215,1.12064);
\draw [c] (2.51336,1.05577) -- (2.5215,1.05577);
\draw [c] (2.5215,1.05577) -- (2.52964,1.05577);
\definecolor{c}{rgb}{0,0,0};
\colorlet{c}{natcomp!70};
\draw [c] (2.53777,1.00227) -- (2.53777,1.06689);
\draw [c] (2.53777,1.06689) -- (2.53777,1.13151);
\draw [c] (2.52964,1.06689) -- (2.53777,1.06689);
\draw [c] (2.53777,1.06689) -- (2.54591,1.06689);
\definecolor{c}{rgb}{0,0,0};
\colorlet{c}{natcomp!70};
\draw [c] (2.55405,1.04669) -- (2.55405,1.11011);
\draw [c] (2.55405,1.11011) -- (2.55405,1.17353);
\draw [c] (2.54591,1.11011) -- (2.55405,1.11011);
\draw [c] (2.55405,1.11011) -- (2.56218,1.11011);
\definecolor{c}{rgb}{0,0,0};
\colorlet{c}{natcomp!70};
\draw [c] (2.57032,0.995602) -- (2.57032,1.06059);
\draw [c] (2.57032,1.06059) -- (2.57032,1.12558);
\draw [c] (2.56218,1.06059) -- (2.57032,1.06059);
\draw [c] (2.57032,1.06059) -- (2.57845,1.06059);
\definecolor{c}{rgb}{0,0,0};
\colorlet{c}{natcomp!70};
\draw [c] (2.58659,0.99475) -- (2.58659,1.05464);
\draw [c] (2.58659,1.05464) -- (2.58659,1.11453);
\draw [c] (2.57845,1.05464) -- (2.58659,1.05464);
\draw [c] (2.58659,1.05464) -- (2.59473,1.05464);
\definecolor{c}{rgb}{0,0,0};
\colorlet{c}{natcomp!70};
\draw [c] (2.60286,1.07289) -- (2.60286,1.14199);
\draw [c] (2.60286,1.14199) -- (2.60286,1.21109);
\draw [c] (2.59473,1.14199) -- (2.60286,1.14199);
\draw [c] (2.60286,1.14199) -- (2.611,1.14199);
\definecolor{c}{rgb}{0,0,0};
\colorlet{c}{natcomp!70};
\draw [c] (2.61914,1.13773) -- (2.61914,1.21239);
\draw [c] (2.61914,1.21239) -- (2.61914,1.28705);
\draw [c] (2.611,1.21239) -- (2.61914,1.21239);
\draw [c] (2.61914,1.21239) -- (2.62727,1.21239);
\definecolor{c}{rgb}{0,0,0};
\colorlet{c}{natcomp!70};
\draw [c] (2.63541,0.996344) -- (2.63541,1.05637);
\draw [c] (2.63541,1.05637) -- (2.63541,1.1164);
\draw [c] (2.62727,1.05637) -- (2.63541,1.05637);
\draw [c] (2.63541,1.05637) -- (2.64355,1.05637);
\definecolor{c}{rgb}{0,0,0};
\colorlet{c}{natcomp!70};
\draw [c] (2.65168,0.988957) -- (2.65168,1.0492);
\draw [c] (2.65168,1.0492) -- (2.65168,1.10944);
\draw [c] (2.64355,1.0492) -- (2.65168,1.0492);
\draw [c] (2.65168,1.0492) -- (2.65982,1.0492);
\definecolor{c}{rgb}{0,0,0};
\colorlet{c}{natcomp!70};
\draw [c] (2.66795,1.1185) -- (2.66795,1.18712);
\draw [c] (2.66795,1.18712) -- (2.66795,1.25574);
\draw [c] (2.65982,1.18712) -- (2.66795,1.18712);
\draw [c] (2.66795,1.18712) -- (2.67609,1.18712);
\definecolor{c}{rgb}{0,0,0};
\colorlet{c}{natcomp!70};
\draw [c] (2.68423,1.11432) -- (2.68423,1.19207);
\draw [c] (2.68423,1.19207) -- (2.68423,1.26981);
\draw [c] (2.67609,1.19207) -- (2.68423,1.19207);
\draw [c] (2.68423,1.19207) -- (2.69236,1.19207);
\definecolor{c}{rgb}{0,0,0};
\colorlet{c}{natcomp!70};
\draw [c] (2.7005,1.21466) -- (2.7005,1.29003);
\draw [c] (2.7005,1.29003) -- (2.7005,1.3654);
\draw [c] (2.69236,1.29003) -- (2.7005,1.29003);
\draw [c] (2.7005,1.29003) -- (2.70864,1.29003);
\definecolor{c}{rgb}{0,0,0};
\colorlet{c}{natcomp!70};
\draw [c] (2.71677,1.10429) -- (2.71677,1.17543);
\draw [c] (2.71677,1.17543) -- (2.71677,1.24658);
\draw [c] (2.70864,1.17543) -- (2.71677,1.17543);
\draw [c] (2.71677,1.17543) -- (2.72491,1.17543);
\definecolor{c}{rgb}{0,0,0};
\colorlet{c}{natcomp!70};
\draw [c] (2.73305,1.12661) -- (2.73305,1.2009);
\draw [c] (2.73305,1.2009) -- (2.73305,1.27519);
\draw [c] (2.72491,1.2009) -- (2.73305,1.2009);
\draw [c] (2.73305,1.2009) -- (2.74118,1.2009);
\definecolor{c}{rgb}{0,0,0};
\colorlet{c}{natcomp!70};
\draw [c] (2.74932,1.06158) -- (2.74932,1.12477);
\draw [c] (2.74932,1.12477) -- (2.74932,1.18796);
\draw [c] (2.74118,1.12477) -- (2.74932,1.12477);
\draw [c] (2.74932,1.12477) -- (2.75745,1.12477);
\definecolor{c}{rgb}{0,0,0};
\colorlet{c}{natcomp!70};
\draw [c] (2.76559,1.18504) -- (2.76559,1.26123);
\draw [c] (2.76559,1.26123) -- (2.76559,1.33741);
\draw [c] (2.75745,1.26123) -- (2.76559,1.26123);
\draw [c] (2.76559,1.26123) -- (2.77373,1.26123);
\definecolor{c}{rgb}{0,0,0};
\colorlet{c}{natcomp!70};
\draw [c] (2.78186,1.23072) -- (2.78186,1.31228);
\draw [c] (2.78186,1.31228) -- (2.78186,1.39384);
\draw [c] (2.77373,1.31228) -- (2.78186,1.31228);
\draw [c] (2.78186,1.31228) -- (2.79,1.31228);
\definecolor{c}{rgb}{0,0,0};
\colorlet{c}{natcomp!70};
\draw [c] (2.79814,1.28934) -- (2.79814,1.37187);
\draw [c] (2.79814,1.37187) -- (2.79814,1.45439);
\draw [c] (2.79,1.37187) -- (2.79814,1.37187);
\draw [c] (2.79814,1.37187) -- (2.80627,1.37187);
\definecolor{c}{rgb}{0,0,0};
\colorlet{c}{natcomp!70};
\draw [c] (2.81441,1.32982) -- (2.81441,1.41625);
\draw [c] (2.81441,1.41625) -- (2.81441,1.50267);
\draw [c] (2.80627,1.41625) -- (2.81441,1.41625);
\draw [c] (2.81441,1.41625) -- (2.82255,1.41625);
\definecolor{c}{rgb}{0,0,0};
\colorlet{c}{natcomp!70};
\draw [c] (2.83068,1.30347) -- (2.83068,1.38681);
\draw [c] (2.83068,1.38681) -- (2.83068,1.47015);
\draw [c] (2.82255,1.38681) -- (2.83068,1.38681);
\draw [c] (2.83068,1.38681) -- (2.83882,1.38681);
\definecolor{c}{rgb}{0,0,0};
\colorlet{c}{natcomp!70};
\draw [c] (2.84695,1.23426) -- (2.84695,1.31216);
\draw [c] (2.84695,1.31216) -- (2.84695,1.39006);
\draw [c] (2.83882,1.31216) -- (2.84695,1.31216);
\draw [c] (2.84695,1.31216) -- (2.85509,1.31216);
\definecolor{c}{rgb}{0,0,0};
\colorlet{c}{natcomp!70};
\draw [c] (2.86323,1.36782) -- (2.86323,1.45635);
\draw [c] (2.86323,1.45635) -- (2.86323,1.54489);
\draw [c] (2.85509,1.45635) -- (2.86323,1.45635);
\draw [c] (2.86323,1.45635) -- (2.87136,1.45635);
\definecolor{c}{rgb}{0,0,0};
\colorlet{c}{natcomp!70};
\draw [c] (2.8795,1.31293) -- (2.8795,1.3965);
\draw [c] (2.8795,1.3965) -- (2.8795,1.48008);
\draw [c] (2.87136,1.3965) -- (2.8795,1.3965);
\draw [c] (2.8795,1.3965) -- (2.88764,1.3965);
\definecolor{c}{rgb}{0,0,0};
\colorlet{c}{natcomp!70};
\draw [c] (2.89577,1.42357) -- (2.89577,1.51525);
\draw [c] (2.89577,1.51525) -- (2.89577,1.60692);
\draw [c] (2.88764,1.51525) -- (2.89577,1.51525);
\draw [c] (2.89577,1.51525) -- (2.90391,1.51525);
\definecolor{c}{rgb}{0,0,0};
\colorlet{c}{natcomp!70};
\draw [c] (2.91205,1.43079) -- (2.91205,1.52154);
\draw [c] (2.91205,1.52154) -- (2.91205,1.61228);
\draw [c] (2.90391,1.52154) -- (2.91205,1.52154);
\draw [c] (2.91205,1.52154) -- (2.92018,1.52154);
\definecolor{c}{rgb}{0,0,0};
\colorlet{c}{natcomp!70};
\draw [c] (2.92832,1.34088) -- (2.92832,1.42855);
\draw [c] (2.92832,1.42855) -- (2.92832,1.51622);
\draw [c] (2.92018,1.42855) -- (2.92832,1.42855);
\draw [c] (2.92832,1.42855) -- (2.93645,1.42855);
\definecolor{c}{rgb}{0,0,0};
\colorlet{c}{natcomp!70};
\draw [c] (2.94459,1.41547) -- (2.94459,1.50915);
\draw [c] (2.94459,1.50915) -- (2.94459,1.60282);
\draw [c] (2.93645,1.50915) -- (2.94459,1.50915);
\draw [c] (2.94459,1.50915) -- (2.95273,1.50915);
\definecolor{c}{rgb}{0,0,0};
\colorlet{c}{natcomp!70};
\draw [c] (2.96086,1.58909) -- (2.96086,1.69352);
\draw [c] (2.96086,1.69352) -- (2.96086,1.79794);
\draw [c] (2.95273,1.69352) -- (2.96086,1.69352);
\draw [c] (2.96086,1.69352) -- (2.969,1.69352);
\definecolor{c}{rgb}{0,0,0};
\colorlet{c}{natcomp!70};
\draw [c] (2.97714,1.33395) -- (2.97714,1.41959);
\draw [c] (2.97714,1.41959) -- (2.97714,1.50522);
\draw [c] (2.969,1.41959) -- (2.97714,1.41959);
\draw [c] (2.97714,1.41959) -- (2.98527,1.41959);
\definecolor{c}{rgb}{0,0,0};
\colorlet{c}{natcomp!70};
\draw [c] (2.99341,1.59874) -- (2.99341,1.69978);
\draw [c] (2.99341,1.69978) -- (2.99341,1.80083);
\draw [c] (2.98527,1.69978) -- (2.99341,1.69978);
\draw [c] (2.99341,1.69978) -- (3.00155,1.69978);
\definecolor{c}{rgb}{0,0,0};
\colorlet{c}{natcomp!70};
\draw [c] (3.00968,1.52556) -- (3.00968,1.61936);
\draw [c] (3.00968,1.61936) -- (3.00968,1.71316);
\draw [c] (3.00155,1.61936) -- (3.00968,1.61936);
\draw [c] (3.00968,1.61936) -- (3.01782,1.61936);
\definecolor{c}{rgb}{0,0,0};
\colorlet{c}{natcomp!70};
\draw [c] (3.02595,1.6018) -- (3.02595,1.70136);
\draw [c] (3.02595,1.70136) -- (3.02595,1.80091);
\draw [c] (3.01782,1.70136) -- (3.02595,1.70136);
\draw [c] (3.02595,1.70136) -- (3.03409,1.70136);
\definecolor{c}{rgb}{0,0,0};
\colorlet{c}{natcomp!70};
\draw [c] (3.04223,1.58016) -- (3.04223,1.68072);
\draw [c] (3.04223,1.68072) -- (3.04223,1.78129);
\draw [c] (3.03409,1.68072) -- (3.04223,1.68072);
\draw [c] (3.04223,1.68072) -- (3.05036,1.68072);
\definecolor{c}{rgb}{0,0,0};
\colorlet{c}{natcomp!70};
\draw [c] (3.0585,1.72343) -- (3.0585,1.82892);
\draw [c] (3.0585,1.82892) -- (3.0585,1.93442);
\draw [c] (3.05036,1.82892) -- (3.0585,1.82892);
\draw [c] (3.0585,1.82892) -- (3.06664,1.82892);
\definecolor{c}{rgb}{0,0,0};
\colorlet{c}{natcomp!70};
\draw [c] (3.07477,1.80323) -- (3.07477,1.92123);
\draw [c] (3.07477,1.92123) -- (3.07477,2.03923);
\draw [c] (3.06664,1.92123) -- (3.07477,1.92123);
\draw [c] (3.07477,1.92123) -- (3.08291,1.92123);
\definecolor{c}{rgb}{0,0,0};
\colorlet{c}{natcomp!70};
\draw [c] (3.09105,1.72774) -- (3.09105,1.8397);
\draw [c] (3.09105,1.8397) -- (3.09105,1.95166);
\draw [c] (3.08291,1.8397) -- (3.09105,1.8397);
\draw [c] (3.09105,1.8397) -- (3.09918,1.8397);
\definecolor{c}{rgb}{0,0,0};
\colorlet{c}{natcomp!70};
\draw [c] (3.10732,1.5984) -- (3.10732,1.70303);
\draw [c] (3.10732,1.70303) -- (3.10732,1.80766);
\draw [c] (3.09918,1.70303) -- (3.10732,1.70303);
\draw [c] (3.10732,1.70303) -- (3.11545,1.70303);
\definecolor{c}{rgb}{0,0,0};
\colorlet{c}{natcomp!70};
\draw [c] (3.12359,1.72127) -- (3.12359,1.83007);
\draw [c] (3.12359,1.83007) -- (3.12359,1.93888);
\draw [c] (3.11545,1.83007) -- (3.12359,1.83007);
\draw [c] (3.12359,1.83007) -- (3.13173,1.83007);
\definecolor{c}{rgb}{0,0,0};
\colorlet{c}{natcomp!70};
\draw [c] (3.13986,1.64464) -- (3.13986,1.74758);
\draw [c] (3.13986,1.74758) -- (3.13986,1.85053);
\draw [c] (3.13173,1.74758) -- (3.13986,1.74758);
\draw [c] (3.13986,1.74758) -- (3.148,1.74758);
\definecolor{c}{rgb}{0,0,0};
\colorlet{c}{natcomp!70};
\draw [c] (3.15614,1.69808) -- (3.15614,1.80701);
\draw [c] (3.15614,1.80701) -- (3.15614,1.91594);
\draw [c] (3.148,1.80701) -- (3.15614,1.80701);
\draw [c] (3.15614,1.80701) -- (3.16427,1.80701);
\definecolor{c}{rgb}{0,0,0};
\colorlet{c}{natcomp!70};
\draw [c] (3.17241,1.85831) -- (3.17241,1.98447);
\draw [c] (3.17241,1.98447) -- (3.17241,2.11062);
\draw [c] (3.16427,1.98447) -- (3.17241,1.98447);
\draw [c] (3.17241,1.98447) -- (3.18055,1.98447);
\definecolor{c}{rgb}{0,0,0};
\colorlet{c}{natcomp!70};
\draw [c] (3.18868,1.75745) -- (3.18868,1.87281);
\draw [c] (3.18868,1.87281) -- (3.18868,1.98818);
\draw [c] (3.18055,1.87281) -- (3.18868,1.87281);
\draw [c] (3.18868,1.87281) -- (3.19682,1.87281);
\definecolor{c}{rgb}{0,0,0};
\colorlet{c}{natcomp!70};
\draw [c] (3.20495,1.88058) -- (3.20495,2.00281);
\draw [c] (3.20495,2.00281) -- (3.20495,2.12505);
\draw [c] (3.19682,2.00281) -- (3.20495,2.00281);
\draw [c] (3.20495,2.00281) -- (3.21309,2.00281);
\definecolor{c}{rgb}{0,0,0};
\colorlet{c}{natcomp!70};
\draw [c] (3.22123,1.80611) -- (3.22123,1.92753);
\draw [c] (3.22123,1.92753) -- (3.22123,2.04895);
\draw [c] (3.21309,1.92753) -- (3.22123,1.92753);
\draw [c] (3.22123,1.92753) -- (3.22936,1.92753);
\definecolor{c}{rgb}{0,0,0};
\colorlet{c}{natcomp!70};
\draw [c] (3.2375,1.91345) -- (3.2375,2.03616);
\draw [c] (3.2375,2.03616) -- (3.2375,2.15888);
\draw [c] (3.22936,2.03616) -- (3.2375,2.03616);
\draw [c] (3.2375,2.03616) -- (3.24564,2.03616);
\definecolor{c}{rgb}{0,0,0};
\colorlet{c}{natcomp!70};
\draw [c] (3.25377,1.96375) -- (3.25377,2.09043);
\draw [c] (3.25377,2.09043) -- (3.25377,2.21712);
\draw [c] (3.24564,2.09043) -- (3.25377,2.09043);
\draw [c] (3.25377,2.09043) -- (3.26191,2.09043);
\definecolor{c}{rgb}{0,0,0};
\colorlet{c}{natcomp!70};
\draw [c] (3.27005,1.92288) -- (3.27005,2.04725);
\draw [c] (3.27005,2.04725) -- (3.27005,2.17162);
\draw [c] (3.26191,2.04725) -- (3.27005,2.04725);
\draw [c] (3.27005,2.04725) -- (3.27818,2.04725);
\definecolor{c}{rgb}{0,0,0};
\colorlet{c}{natcomp!70};
\draw [c] (3.28632,1.84932) -- (3.28632,1.97407);
\draw [c] (3.28632,1.97407) -- (3.28632,2.09882);
\draw [c] (3.27818,1.97407) -- (3.28632,1.97407);
\draw [c] (3.28632,1.97407) -- (3.29445,1.97407);
\definecolor{c}{rgb}{0,0,0};
\colorlet{c}{natcomp!70};
\draw [c] (3.30259,1.90077) -- (3.30259,2.02007);
\draw [c] (3.30259,2.02007) -- (3.30259,2.13937);
\draw [c] (3.29445,2.02007) -- (3.30259,2.02007);
\draw [c] (3.30259,2.02007) -- (3.31073,2.02007);
\definecolor{c}{rgb}{0,0,0};
\colorlet{c}{natcomp!70};
\draw [c] (3.31886,1.83797) -- (3.31886,1.96045);
\draw [c] (3.31886,1.96045) -- (3.31886,2.08293);
\draw [c] (3.31073,1.96045) -- (3.31886,1.96045);
\draw [c] (3.31886,1.96045) -- (3.327,1.96045);
\definecolor{c}{rgb}{0,0,0};
\colorlet{c}{natcomp!70};
\draw [c] (3.33514,1.88986) -- (3.33514,2.01298);
\draw [c] (3.33514,2.01298) -- (3.33514,2.1361);
\draw [c] (3.327,2.01298) -- (3.33514,2.01298);
\draw [c] (3.33514,2.01298) -- (3.34327,2.01298);
\definecolor{c}{rgb}{0,0,0};
\colorlet{c}{natcomp!70};
\draw [c] (3.35141,1.97995) -- (3.35141,2.10928);
\draw [c] (3.35141,2.10928) -- (3.35141,2.2386);
\draw [c] (3.34327,2.10928) -- (3.35141,2.10928);
\draw [c] (3.35141,2.10928) -- (3.35955,2.10928);
\definecolor{c}{rgb}{0,0,0};
\colorlet{c}{natcomp!70};
\draw [c] (3.36768,1.69146) -- (3.36768,1.80413);
\draw [c] (3.36768,1.80413) -- (3.36768,1.9168);
\draw [c] (3.35955,1.80413) -- (3.36768,1.80413);
\draw [c] (3.36768,1.80413) -- (3.37582,1.80413);
\definecolor{c}{rgb}{0,0,0};
\colorlet{c}{natcomp!70};
\draw [c] (3.38395,1.76713) -- (3.38395,1.8867);
\draw [c] (3.38395,1.8867) -- (3.38395,2.00627);
\draw [c] (3.37582,1.8867) -- (3.38395,1.8867);
\draw [c] (3.38395,1.8867) -- (3.39209,1.8867);
\definecolor{c}{rgb}{0,0,0};
\colorlet{c}{natcomp!70};
\draw [c] (3.40023,1.73703) -- (3.40023,1.85405);
\draw [c] (3.40023,1.85405) -- (3.40023,1.97106);
\draw [c] (3.39209,1.85405) -- (3.40023,1.85405);
\draw [c] (3.40023,1.85405) -- (3.40836,1.85405);
\definecolor{c}{rgb}{0,0,0};
\colorlet{c}{natcomp!70};
\draw [c] (3.4165,1.75427) -- (3.4165,1.87418);
\draw [c] (3.4165,1.87418) -- (3.4165,1.99409);
\draw [c] (3.40836,1.87418) -- (3.4165,1.87418);
\draw [c] (3.4165,1.87418) -- (3.42464,1.87418);
\definecolor{c}{rgb}{0,0,0};
\colorlet{c}{natcomp!70};
\draw [c] (3.43277,1.59071) -- (3.43277,1.70168);
\draw [c] (3.43277,1.70168) -- (3.43277,1.81264);
\draw [c] (3.42464,1.70168) -- (3.43277,1.70168);
\draw [c] (3.43277,1.70168) -- (3.44091,1.70168);
\definecolor{c}{rgb}{0,0,0};
\colorlet{c}{natcomp!70};
\draw [c] (3.44905,2.0116) -- (3.44905,2.1488);
\draw [c] (3.44905,2.1488) -- (3.44905,2.286);
\draw [c] (3.44091,2.1488) -- (3.44905,2.1488);
\draw [c] (3.44905,2.1488) -- (3.45718,2.1488);
\definecolor{c}{rgb}{0,0,0};
\colorlet{c}{natcomp!70};
\draw [c] (3.46532,1.58379) -- (3.46532,1.69731);
\draw [c] (3.46532,1.69731) -- (3.46532,1.81082);
\draw [c] (3.45718,1.69731) -- (3.46532,1.69731);
\draw [c] (3.46532,1.69731) -- (3.47345,1.69731);
\definecolor{c}{rgb}{0,0,0};
\colorlet{c}{natcomp!70};
\draw [c] (3.48159,1.69179) -- (3.48159,1.8072);
\draw [c] (3.48159,1.8072) -- (3.48159,1.92261);
\draw [c] (3.47345,1.8072) -- (3.48159,1.8072);
\draw [c] (3.48159,1.8072) -- (3.48973,1.8072);
\definecolor{c}{rgb}{0,0,0};
\colorlet{c}{natcomp!70};
\draw [c] (3.49786,1.66795) -- (3.49786,1.7778);
\draw [c] (3.49786,1.7778) -- (3.49786,1.88766);
\draw [c] (3.48973,1.7778) -- (3.49786,1.7778);
\draw [c] (3.49786,1.7778) -- (3.506,1.7778);
\definecolor{c}{rgb}{0,0,0};
\colorlet{c}{natcomp!70};
\draw [c] (3.51414,1.54378) -- (3.51414,1.65414);
\draw [c] (3.51414,1.65414) -- (3.51414,1.76451);
\draw [c] (3.506,1.65414) -- (3.51414,1.65414);
\draw [c] (3.51414,1.65414) -- (3.52227,1.65414);
\definecolor{c}{rgb}{0,0,0};
\colorlet{c}{natcomp!70};
\draw [c] (3.53041,1.63429) -- (3.53041,1.75469);
\draw [c] (3.53041,1.75469) -- (3.53041,1.87509);
\draw [c] (3.52227,1.75469) -- (3.53041,1.75469);
\draw [c] (3.53041,1.75469) -- (3.53855,1.75469);
\definecolor{c}{rgb}{0,0,0};
\colorlet{c}{natcomp!70};
\draw [c] (3.54668,1.49617) -- (3.54668,1.60081);
\draw [c] (3.54668,1.60081) -- (3.54668,1.70545);
\draw [c] (3.53855,1.60081) -- (3.54668,1.60081);
\draw [c] (3.54668,1.60081) -- (3.55482,1.60081);
\definecolor{c}{rgb}{0,0,0};
\colorlet{c}{natcomp!70};
\draw [c] (3.56295,1.58314) -- (3.56295,1.69229);
\draw [c] (3.56295,1.69229) -- (3.56295,1.80144);
\draw [c] (3.55482,1.69229) -- (3.56295,1.69229);
\draw [c] (3.56295,1.69229) -- (3.57109,1.69229);
\definecolor{c}{rgb}{0,0,0};
\colorlet{c}{natcomp!70};
\draw [c] (3.57923,1.46417) -- (3.57923,1.56462);
\draw [c] (3.57923,1.56462) -- (3.57923,1.66507);
\draw [c] (3.57109,1.56462) -- (3.57923,1.56462);
\draw [c] (3.57923,1.56462) -- (3.58736,1.56462);
\definecolor{c}{rgb}{0,0,0};
\colorlet{c}{natcomp!70};
\draw [c] (3.5955,1.7506) -- (3.5955,1.87231);
\draw [c] (3.5955,1.87231) -- (3.5955,1.99402);
\draw [c] (3.58736,1.87231) -- (3.5955,1.87231);
\draw [c] (3.5955,1.87231) -- (3.60364,1.87231);
\definecolor{c}{rgb}{0,0,0};
\colorlet{c}{natcomp!70};
\draw [c] (3.61177,1.53631) -- (3.61177,1.63577);
\draw [c] (3.61177,1.63577) -- (3.61177,1.73524);
\draw [c] (3.60364,1.63577) -- (3.61177,1.63577);
\draw [c] (3.61177,1.63577) -- (3.61991,1.63577);
\definecolor{c}{rgb}{0,0,0};
\colorlet{c}{natcomp!70};
\draw [c] (3.62805,1.60831) -- (3.62805,1.71397);
\draw [c] (3.62805,1.71397) -- (3.62805,1.81962);
\draw [c] (3.61991,1.71397) -- (3.62805,1.71397);
\draw [c] (3.62805,1.71397) -- (3.63618,1.71397);
\definecolor{c}{rgb}{0,0,0};
\colorlet{c}{natcomp!70};
\draw [c] (3.64432,1.41401) -- (3.64432,1.5139);
\draw [c] (3.64432,1.5139) -- (3.64432,1.61378);
\draw [c] (3.63618,1.5139) -- (3.64432,1.5139);
\draw [c] (3.64432,1.5139) -- (3.65245,1.5139);
\definecolor{c}{rgb}{0,0,0};
\colorlet{c}{natcomp!70};
\draw [c] (3.66059,1.47867) -- (3.66059,1.57869);
\draw [c] (3.66059,1.57869) -- (3.66059,1.67871);
\draw [c] (3.65245,1.57869) -- (3.66059,1.57869);
\draw [c] (3.66059,1.57869) -- (3.66873,1.57869);
\definecolor{c}{rgb}{0,0,0};
\colorlet{c}{natcomp!70};
\draw [c] (3.67686,1.43725) -- (3.67686,1.53155);
\draw [c] (3.67686,1.53155) -- (3.67686,1.62584);
\draw [c] (3.66873,1.53155) -- (3.67686,1.53155);
\draw [c] (3.67686,1.53155) -- (3.685,1.53155);
\definecolor{c}{rgb}{0,0,0};
\colorlet{c}{natcomp!70};
\draw [c] (3.69314,1.44595) -- (3.69314,1.54196);
\draw [c] (3.69314,1.54196) -- (3.69314,1.63798);
\draw [c] (3.685,1.54196) -- (3.69314,1.54196);
\draw [c] (3.69314,1.54196) -- (3.70127,1.54196);
\definecolor{c}{rgb}{0,0,0};
\colorlet{c}{natcomp!70};
\draw [c] (3.70941,1.31331) -- (3.70941,1.4008);
\draw [c] (3.70941,1.4008) -- (3.70941,1.48828);
\draw [c] (3.70127,1.4008) -- (3.70941,1.4008);
\draw [c] (3.70941,1.4008) -- (3.71755,1.4008);
\definecolor{c}{rgb}{0,0,0};
\colorlet{c}{natcomp!70};
\draw [c] (3.72568,1.38718) -- (3.72568,1.47882);
\draw [c] (3.72568,1.47882) -- (3.72568,1.57046);
\draw [c] (3.71755,1.47882) -- (3.72568,1.47882);
\draw [c] (3.72568,1.47882) -- (3.73382,1.47882);
\definecolor{c}{rgb}{0,0,0};
\colorlet{c}{natcomp!70};
\draw [c] (3.74195,1.30543) -- (3.74195,1.39264);
\draw [c] (3.74195,1.39264) -- (3.74195,1.47984);
\draw [c] (3.73382,1.39264) -- (3.74195,1.39264);
\draw [c] (3.74195,1.39264) -- (3.75009,1.39264);
\definecolor{c}{rgb}{0,0,0};
\colorlet{c}{natcomp!70};
\draw [c] (3.75823,1.33609) -- (3.75823,1.42593);
\draw [c] (3.75823,1.42593) -- (3.75823,1.51577);
\draw [c] (3.75009,1.42593) -- (3.75823,1.42593);
\draw [c] (3.75823,1.42593) -- (3.76636,1.42593);
\definecolor{c}{rgb}{0,0,0};
\colorlet{c}{natcomp!70};
\draw [c] (3.7745,1.48113) -- (3.7745,1.57802);
\draw [c] (3.7745,1.57802) -- (3.7745,1.67491);
\draw [c] (3.76636,1.57802) -- (3.7745,1.57802);
\draw [c] (3.7745,1.57802) -- (3.78264,1.57802);
\definecolor{c}{rgb}{0,0,0};
\colorlet{c}{natcomp!70};
\draw [c] (3.79077,1.36985) -- (3.79077,1.4665);
\draw [c] (3.79077,1.4665) -- (3.79077,1.56316);
\draw [c] (3.78264,1.4665) -- (3.79077,1.4665);
\draw [c] (3.79077,1.4665) -- (3.79891,1.4665);
\definecolor{c}{rgb}{0,0,0};
\colorlet{c}{natcomp!70};
\draw [c] (3.80705,1.3613) -- (3.80705,1.45155);
\draw [c] (3.80705,1.45155) -- (3.80705,1.5418);
\draw [c] (3.79891,1.45155) -- (3.80705,1.45155);
\draw [c] (3.80705,1.45155) -- (3.81518,1.45155);
\definecolor{c}{rgb}{0,0,0};
\colorlet{c}{natcomp!70};
\draw [c] (3.82332,1.34548) -- (3.82332,1.43349);
\draw [c] (3.82332,1.43349) -- (3.82332,1.5215);
\draw [c] (3.81518,1.43349) -- (3.82332,1.43349);
\draw [c] (3.82332,1.43349) -- (3.83145,1.43349);
\definecolor{c}{rgb}{0,0,0};
\colorlet{c}{natcomp!70};
\draw [c] (3.83959,1.34648) -- (3.83959,1.44227);
\draw [c] (3.83959,1.44227) -- (3.83959,1.53807);
\draw [c] (3.83145,1.44227) -- (3.83959,1.44227);
\draw [c] (3.83959,1.44227) -- (3.84773,1.44227);
\definecolor{c}{rgb}{0,0,0};
\colorlet{c}{natcomp!70};
\draw [c] (3.85586,1.26603) -- (3.85586,1.34858);
\draw [c] (3.85586,1.34858) -- (3.85586,1.43114);
\draw [c] (3.84773,1.34858) -- (3.85586,1.34858);
\draw [c] (3.85586,1.34858) -- (3.864,1.34858);
\definecolor{c}{rgb}{0,0,0};
\colorlet{c}{natcomp!70};
\draw [c] (3.87214,1.19945) -- (3.87214,1.28167);
\draw [c] (3.87214,1.28167) -- (3.87214,1.3639);
\draw [c] (3.864,1.28167) -- (3.87214,1.28167);
\draw [c] (3.87214,1.28167) -- (3.88027,1.28167);
\definecolor{c}{rgb}{0,0,0};
\colorlet{c}{natcomp!70};
\draw [c] (3.88841,1.37139) -- (3.88841,1.46647);
\draw [c] (3.88841,1.46647) -- (3.88841,1.56155);
\draw [c] (3.88027,1.46647) -- (3.88841,1.46647);
\draw [c] (3.88841,1.46647) -- (3.89655,1.46647);
\definecolor{c}{rgb}{0,0,0};
\colorlet{c}{natcomp!70};
\draw [c] (3.90468,1.0781) -- (3.90468,1.14498);
\draw [c] (3.90468,1.14498) -- (3.90468,1.21187);
\draw [c] (3.89655,1.14498) -- (3.90468,1.14498);
\draw [c] (3.90468,1.14498) -- (3.91282,1.14498);
\definecolor{c}{rgb}{0,0,0};
\colorlet{c}{natcomp!70};
\draw [c] (3.92095,1.12409) -- (3.92095,1.19694);
\draw [c] (3.92095,1.19694) -- (3.92095,1.26979);
\draw [c] (3.91282,1.19694) -- (3.92095,1.19694);
\draw [c] (3.92095,1.19694) -- (3.92909,1.19694);
\definecolor{c}{rgb}{0,0,0};
\colorlet{c}{natcomp!70};
\draw [c] (3.93723,1.21279) -- (3.93723,1.29838);
\draw [c] (3.93723,1.29838) -- (3.93723,1.38397);
\draw [c] (3.92909,1.29838) -- (3.93723,1.29838);
\draw [c] (3.93723,1.29838) -- (3.94536,1.29838);
\definecolor{c}{rgb}{0,0,0};
\colorlet{c}{natcomp!70};
\draw [c] (3.9535,1.18838) -- (3.9535,1.26484);
\draw [c] (3.9535,1.26484) -- (3.9535,1.3413);
\draw [c] (3.94536,1.26484) -- (3.9535,1.26484);
\draw [c] (3.9535,1.26484) -- (3.96164,1.26484);
\definecolor{c}{rgb}{0,0,0};
\colorlet{c}{natcomp!70};
\draw [c] (3.96977,1.13244) -- (3.96977,1.20537);
\draw [c] (3.96977,1.20537) -- (3.96977,1.2783);
\draw [c] (3.96164,1.20537) -- (3.96977,1.20537);
\draw [c] (3.96977,1.20537) -- (3.97791,1.20537);
\definecolor{c}{rgb}{0,0,0};
\colorlet{c}{natcomp!70};
\draw [c] (3.98605,1.18644) -- (3.98605,1.26695);
\draw [c] (3.98605,1.26695) -- (3.98605,1.34746);
\draw [c] (3.97791,1.26695) -- (3.98605,1.26695);
\draw [c] (3.98605,1.26695) -- (3.99418,1.26695);
\definecolor{c}{rgb}{0,0,0};
\colorlet{c}{natcomp!70};
\draw [c] (4.00232,1.18905) -- (4.00232,1.27326);
\draw [c] (4.00232,1.27326) -- (4.00232,1.35748);
\draw [c] (3.99418,1.27326) -- (4.00232,1.27326);
\draw [c] (4.00232,1.27326) -- (4.01045,1.27326);
\definecolor{c}{rgb}{0,0,0};
\colorlet{c}{natcomp!70};
\draw [c] (4.01859,1.21515) -- (4.01859,1.29654);
\draw [c] (4.01859,1.29654) -- (4.01859,1.37792);
\draw [c] (4.01045,1.29654) -- (4.01859,1.29654);
\draw [c] (4.01859,1.29654) -- (4.02673,1.29654);
\definecolor{c}{rgb}{0,0,0};
\colorlet{c}{natcomp!70};
\draw [c] (4.03486,1.0567) -- (4.03486,1.12126);
\draw [c] (4.03486,1.12126) -- (4.03486,1.18582);
\draw [c] (4.02673,1.12126) -- (4.03486,1.12126);
\draw [c] (4.03486,1.12126) -- (4.043,1.12126);
\definecolor{c}{rgb}{0,0,0};
\colorlet{c}{natcomp!70};
\draw [c] (4.05114,1.0637) -- (4.05114,1.13565);
\draw [c] (4.05114,1.13565) -- (4.05114,1.2076);
\draw [c] (4.043,1.13565) -- (4.05114,1.13565);
\draw [c] (4.05114,1.13565) -- (4.05927,1.13565);
\definecolor{c}{rgb}{0,0,0};
\colorlet{c}{natcomp!70};
\draw [c] (4.06741,1.12543) -- (4.06741,1.19947);
\draw [c] (4.06741,1.19947) -- (4.06741,1.27351);
\draw [c] (4.05927,1.19947) -- (4.06741,1.19947);
\draw [c] (4.06741,1.19947) -- (4.07555,1.19947);
\definecolor{c}{rgb}{0,0,0};
\colorlet{c}{natcomp!70};
\draw [c] (4.08368,1.11089) -- (4.08368,1.18647);
\draw [c] (4.08368,1.18647) -- (4.08368,1.26206);
\draw [c] (4.07555,1.18647) -- (4.08368,1.18647);
\draw [c] (4.08368,1.18647) -- (4.09182,1.18647);
\definecolor{c}{rgb}{0,0,0};
\colorlet{c}{natcomp!70};
\draw [c] (4.09995,1.09755) -- (4.09995,1.16945);
\draw [c] (4.09995,1.16945) -- (4.09995,1.24135);
\draw [c] (4.09182,1.16945) -- (4.09995,1.16945);
\draw [c] (4.09995,1.16945) -- (4.10809,1.16945);
\definecolor{c}{rgb}{0,0,0};
\colorlet{c}{natcomp!70};
\draw [c] (4.11623,1.07487) -- (4.11623,1.14657);
\draw [c] (4.11623,1.14657) -- (4.11623,1.21826);
\draw [c] (4.10809,1.14657) -- (4.11623,1.14657);
\draw [c] (4.11623,1.14657) -- (4.12436,1.14657);
\definecolor{c}{rgb}{0,0,0};
\colorlet{c}{natcomp!70};
\draw [c] (4.1325,1.03624) -- (4.1325,1.10277);
\draw [c] (4.1325,1.10277) -- (4.1325,1.16931);
\draw [c] (4.12436,1.10277) -- (4.1325,1.10277);
\draw [c] (4.1325,1.10277) -- (4.14064,1.10277);
\definecolor{c}{rgb}{0,0,0};
\colorlet{c}{natcomp!70};
\draw [c] (4.14877,0.952458) -- (4.14877,1.00761);
\draw [c] (4.14877,1.00761) -- (4.14877,1.06276);
\draw [c] (4.14064,1.00761) -- (4.14877,1.00761);
\draw [c] (4.14877,1.00761) -- (4.15691,1.00761);
\definecolor{c}{rgb}{0,0,0};
\colorlet{c}{natcomp!70};
\draw [c] (4.16505,1.04998) -- (4.16505,1.12012);
\draw [c] (4.16505,1.12012) -- (4.16505,1.19027);
\draw [c] (4.15691,1.12012) -- (4.16505,1.12012);
\draw [c] (4.16505,1.12012) -- (4.17318,1.12012);
\definecolor{c}{rgb}{0,0,0};
\colorlet{c}{natcomp!70};
\draw [c] (4.18132,1.08985) -- (4.18132,1.16335);
\draw [c] (4.18132,1.16335) -- (4.18132,1.23686);
\draw [c] (4.17318,1.16335) -- (4.18132,1.16335);
\draw [c] (4.18132,1.16335) -- (4.18945,1.16335);
\definecolor{c}{rgb}{0,0,0};
\colorlet{c}{natcomp!70};
\draw [c] (4.19759,1.06558) -- (4.19759,1.13256);
\draw [c] (4.19759,1.13256) -- (4.19759,1.19953);
\draw [c] (4.18945,1.13256) -- (4.19759,1.13256);
\draw [c] (4.19759,1.13256) -- (4.20573,1.13256);
\definecolor{c}{rgb}{0,0,0};
\colorlet{c}{natcomp!70};
\draw [c] (4.21386,0.99441) -- (4.21386,1.05948);
\draw [c] (4.21386,1.05948) -- (4.21386,1.12455);
\draw [c] (4.20573,1.05948) -- (4.21386,1.05948);
\draw [c] (4.21386,1.05948) -- (4.222,1.05948);
\definecolor{c}{rgb}{0,0,0};
\colorlet{c}{natcomp!70};
\draw [c] (4.23014,1.15782) -- (4.23014,1.23917);
\draw [c] (4.23014,1.23917) -- (4.23014,1.32052);
\draw [c] (4.222,1.23917) -- (4.23014,1.23917);
\draw [c] (4.23014,1.23917) -- (4.23827,1.23917);
\definecolor{c}{rgb}{0,0,0};
\colorlet{c}{natcomp!70};
\draw [c] (4.24641,0.983316) -- (4.24641,1.04763);
\draw [c] (4.24641,1.04763) -- (4.24641,1.11194);
\draw [c] (4.23827,1.04763) -- (4.24641,1.04763);
\draw [c] (4.24641,1.04763) -- (4.25455,1.04763);
\definecolor{c}{rgb}{0,0,0};
\colorlet{c}{natcomp!70};
\draw [c] (4.26268,1.03462) -- (4.26268,1.10754);
\draw [c] (4.26268,1.10754) -- (4.26268,1.18046);
\draw [c] (4.25455,1.10754) -- (4.26268,1.10754);
\draw [c] (4.26268,1.10754) -- (4.27082,1.10754);
\definecolor{c}{rgb}{0,0,0};
\colorlet{c}{natcomp!70};
\draw [c] (4.27895,0.980957) -- (4.27895,1.04168);
\draw [c] (4.27895,1.04168) -- (4.27895,1.1024);
\draw [c] (4.27082,1.04168) -- (4.27895,1.04168);
\draw [c] (4.27895,1.04168) -- (4.28709,1.04168);
\definecolor{c}{rgb}{0,0,0};
\colorlet{c}{natcomp!70};
\draw [c] (4.29523,0.98657) -- (4.29523,1.04335);
\draw [c] (4.29523,1.04335) -- (4.29523,1.10013);
\draw [c] (4.28709,1.04335) -- (4.29523,1.04335);
\draw [c] (4.29523,1.04335) -- (4.30336,1.04335);
\definecolor{c}{rgb}{0,0,0};
\colorlet{c}{natcomp!70};
\draw [c] (4.3115,0.987687) -- (4.3115,1.0472);
\draw [c] (4.3115,1.0472) -- (4.3115,1.10672);
\draw [c] (4.30336,1.0472) -- (4.3115,1.0472);
\draw [c] (4.3115,1.0472) -- (4.31964,1.0472);
\definecolor{c}{rgb}{0,0,0};
\colorlet{c}{natcomp!70};
\draw [c] (4.32777,1.01895) -- (4.32777,1.08324);
\draw [c] (4.32777,1.08324) -- (4.32777,1.14754);
\draw [c] (4.31964,1.08324) -- (4.32777,1.08324);
\draw [c] (4.32777,1.08324) -- (4.33591,1.08324);
\definecolor{c}{rgb}{0,0,0};
\colorlet{c}{natcomp!70};
\draw [c] (4.34405,0.968473) -- (4.34405,1.02884);
\draw [c] (4.34405,1.02884) -- (4.34405,1.08921);
\draw [c] (4.33591,1.02884) -- (4.34405,1.02884);
\draw [c] (4.34405,1.02884) -- (4.35218,1.02884);
\definecolor{c}{rgb}{0,0,0};
\colorlet{c}{natcomp!70};
\draw [c] (4.36032,0.974112) -- (4.36032,1.03613);
\draw [c] (4.36032,1.03613) -- (4.36032,1.09814);
\draw [c] (4.35218,1.03613) -- (4.36032,1.03613);
\draw [c] (4.36032,1.03613) -- (4.36845,1.03613);
\definecolor{c}{rgb}{0,0,0};
\colorlet{c}{natcomp!70};
\draw [c] (4.37659,0.980809) -- (4.37659,1.04153);
\draw [c] (4.37659,1.04153) -- (4.37659,1.10224);
\draw [c] (4.36845,1.04153) -- (4.37659,1.04153);
\draw [c] (4.37659,1.04153) -- (4.38473,1.04153);
\definecolor{c}{rgb}{0,0,0};
\colorlet{c}{natcomp!70};
\draw [c] (4.39286,0.971029) -- (4.39286,1.02981);
\draw [c] (4.39286,1.02981) -- (4.39286,1.08859);
\draw [c] (4.38473,1.02981) -- (4.39286,1.02981);
\draw [c] (4.39286,1.02981) -- (4.401,1.02981);
\definecolor{c}{rgb}{0,0,0};
\colorlet{c}{natcomp!70};
\draw [c] (4.40914,0.95256) -- (4.40914,1.00793);
\draw [c] (4.40914,1.00793) -- (4.40914,1.06331);
\draw [c] (4.401,1.00793) -- (4.40914,1.00793);
\draw [c] (4.40914,1.00793) -- (4.41727,1.00793);
\definecolor{c}{rgb}{0,0,0};
\colorlet{c}{natcomp!70};
\draw [c] (4.42541,1.04433) -- (4.42541,1.1097);
\draw [c] (4.42541,1.1097) -- (4.42541,1.17507);
\draw [c] (4.41727,1.1097) -- (4.42541,1.1097);
\draw [c] (4.42541,1.1097) -- (4.43355,1.1097);
\definecolor{c}{rgb}{0,0,0};
\colorlet{c}{natcomp!70};
\draw [c] (4.44168,0.921556) -- (4.44168,0.976194);
\draw [c] (4.44168,0.976194) -- (4.44168,1.03083);
\draw [c] (4.43355,0.976194) -- (4.44168,0.976194);
\draw [c] (4.44168,0.976194) -- (4.44982,0.976194);
\definecolor{c}{rgb}{0,0,0};
\colorlet{c}{natcomp!70};
\draw [c] (4.45795,0.972593) -- (4.45795,1.04013);
\draw [c] (4.45795,1.04013) -- (4.45795,1.10767);
\draw [c] (4.44982,1.04013) -- (4.45795,1.04013);
\draw [c] (4.45795,1.04013) -- (4.46609,1.04013);
\definecolor{c}{rgb}{0,0,0};
\colorlet{c}{natcomp!70};
\draw [c] (4.47423,0.946413) -- (4.47423,0.999826);
\draw [c] (4.47423,0.999826) -- (4.47423,1.05324);
\draw [c] (4.46609,0.999826) -- (4.47423,0.999826);
\draw [c] (4.47423,0.999826) -- (4.48236,0.999826);
\definecolor{c}{rgb}{0,0,0};
\colorlet{c}{natcomp!70};
\draw [c] (4.4905,0.928871) -- (4.4905,0.983045);
\draw [c] (4.4905,0.983045) -- (4.4905,1.03722);
\draw [c] (4.48236,0.983045) -- (4.4905,0.983045);
\draw [c] (4.4905,0.983045) -- (4.49864,0.983045);
\definecolor{c}{rgb}{0,0,0};
\colorlet{c}{natcomp!70};
\draw [c] (4.50677,0.92616) -- (4.50677,0.979906);
\draw [c] (4.50677,0.979906) -- (4.50677,1.03365);
\draw [c] (4.49864,0.979906) -- (4.50677,0.979906);
\draw [c] (4.50677,0.979906) -- (4.51491,0.979906);
\definecolor{c}{rgb}{0,0,0};
\colorlet{c}{natcomp!70};
\draw [c] (4.52305,0.866944) -- (4.52305,0.914198);
\draw [c] (4.52305,0.914198) -- (4.52305,0.961452);
\draw [c] (4.51491,0.914198) -- (4.52305,0.914198);
\draw [c] (4.52305,0.914198) -- (4.53118,0.914198);
\definecolor{c}{rgb}{0,0,0};
\colorlet{c}{natcomp!70};
\draw [c] (4.53932,0.956535) -- (4.53932,1.01368);
\draw [c] (4.53932,1.01368) -- (4.53932,1.07083);
\draw [c] (4.53118,1.01368) -- (4.53932,1.01368);
\draw [c] (4.53932,1.01368) -- (4.54745,1.01368);
\definecolor{c}{rgb}{0,0,0};
\colorlet{c}{natcomp!70};
\draw [c] (4.55559,1.00276) -- (4.55559,1.06462);
\draw [c] (4.55559,1.06462) -- (4.55559,1.12648);
\draw [c] (4.54745,1.06462) -- (4.55559,1.06462);
\draw [c] (4.55559,1.06462) -- (4.56373,1.06462);
\definecolor{c}{rgb}{0,0,0};
\colorlet{c}{natcomp!70};
\draw [c] (4.57186,0.919644) -- (4.57186,0.985824);
\draw [c] (4.57186,0.985824) -- (4.57186,1.052);
\draw [c] (4.56373,0.985824) -- (4.57186,0.985824);
\draw [c] (4.57186,0.985824) -- (4.58,0.985824);
\definecolor{c}{rgb}{0,0,0};
\colorlet{c}{natcomp!70};
\draw [c] (4.58814,0.882513) -- (4.58814,0.934061);
\draw [c] (4.58814,0.934061) -- (4.58814,0.985609);
\draw [c] (4.58,0.934061) -- (4.58814,0.934061);
\draw [c] (4.58814,0.934061) -- (4.59627,0.934061);
\definecolor{c}{rgb}{0,0,0};
\colorlet{c}{natcomp!70};
\draw [c] (4.60441,0.912457) -- (4.60441,0.967236);
\draw [c] (4.60441,0.967236) -- (4.60441,1.02202);
\draw [c] (4.59627,0.967236) -- (4.60441,0.967236);
\draw [c] (4.60441,0.967236) -- (4.61255,0.967236);
\definecolor{c}{rgb}{0,0,0};
\colorlet{c}{natcomp!70};
\draw [c] (4.62068,0.914155) -- (4.62068,0.969166);
\draw [c] (4.62068,0.969166) -- (4.62068,1.02418);
\draw [c] (4.61255,0.969166) -- (4.62068,0.969166);
\draw [c] (4.62068,0.969166) -- (4.62882,0.969166);
\definecolor{c}{rgb}{0,0,0};
\colorlet{c}{natcomp!70};
\draw [c] (4.63695,0.879878) -- (4.63695,0.938137);
\draw [c] (4.63695,0.938137) -- (4.63695,0.996396);
\draw [c] (4.62882,0.938137) -- (4.63695,0.938137);
\draw [c] (4.63695,0.938137) -- (4.64509,0.938137);
\definecolor{c}{rgb}{0,0,0};
\colorlet{c}{natcomp!70};
\draw [c] (4.65323,0.899783) -- (4.65323,0.956467);
\draw [c] (4.65323,0.956467) -- (4.65323,1.01315);
\draw [c] (4.64509,0.956467) -- (4.65323,0.956467);
\draw [c] (4.65323,0.956467) -- (4.66136,0.956467);
\definecolor{c}{rgb}{0,0,0};
\colorlet{c}{natcomp!70};
\draw [c] (4.6695,0.884772) -- (4.6695,0.936176);
\draw [c] (4.6695,0.936176) -- (4.6695,0.987579);
\draw [c] (4.66136,0.936176) -- (4.6695,0.936176);
\draw [c] (4.6695,0.936176) -- (4.67764,0.936176);
\definecolor{c}{rgb}{0,0,0};
\colorlet{c}{natcomp!70};
\draw [c] (4.68577,0.985752) -- (4.68577,1.05675);
\draw [c] (4.68577,1.05675) -- (4.68577,1.12775);
\draw [c] (4.67764,1.05675) -- (4.68577,1.05675);
\draw [c] (4.68577,1.05675) -- (4.69391,1.05675);
\definecolor{c}{rgb}{0,0,0};
\colorlet{c}{natcomp!70};
\draw [c] (4.70205,0.888065) -- (4.70205,0.939341);
\draw [c] (4.70205,0.939341) -- (4.70205,0.990618);
\draw [c] (4.69391,0.939341) -- (4.70205,0.939341);
\draw [c] (4.70205,0.939341) -- (4.71018,0.939341);
\definecolor{c}{rgb}{0,0,0};
\colorlet{c}{natcomp!70};
\draw [c] (4.71832,0.950968) -- (4.71832,1.01404);
\draw [c] (4.71832,1.01404) -- (4.71832,1.0771);
\draw [c] (4.71018,1.01404) -- (4.71832,1.01404);
\draw [c] (4.71832,1.01404) -- (4.72645,1.01404);
\definecolor{c}{rgb}{0,0,0};
\colorlet{c}{natcomp!70};
\draw [c] (4.73459,0.890061) -- (4.73459,0.949548);
\draw [c] (4.73459,0.949548) -- (4.73459,1.00904);
\draw [c] (4.72645,0.949548) -- (4.73459,0.949548);
\draw [c] (4.73459,0.949548) -- (4.74273,0.949548);
\definecolor{c}{rgb}{0,0,0};
\colorlet{c}{natcomp!70};
\draw [c] (4.75086,0.903141) -- (4.75086,0.959014);
\draw [c] (4.75086,0.959014) -- (4.75086,1.01489);
\draw [c] (4.74273,0.959014) -- (4.75086,0.959014);
\draw [c] (4.75086,0.959014) -- (4.759,0.959014);
\definecolor{c}{rgb}{0,0,0};
\colorlet{c}{natcomp!70};
\draw [c] (4.76714,0.878125) -- (4.76714,0.929474);
\draw [c] (4.76714,0.929474) -- (4.76714,0.980823);
\draw [c] (4.759,0.929474) -- (4.76714,0.929474);
\draw [c] (4.76714,0.929474) -- (4.77527,0.929474);
\definecolor{c}{rgb}{0,0,0};
\colorlet{c}{natcomp!70};
\draw [c] (4.78341,0.831227) -- (4.78341,0.880248);
\draw [c] (4.78341,0.880248) -- (4.78341,0.929269);
\draw [c] (4.77527,0.880248) -- (4.78341,0.880248);
\draw [c] (4.78341,0.880248) -- (4.79155,0.880248);
\definecolor{c}{rgb}{0,0,0};
\colorlet{c}{natcomp!70};
\draw [c] (4.79968,0.864439) -- (4.79968,0.917743);
\draw [c] (4.79968,0.917743) -- (4.79968,0.971047);
\draw [c] (4.79155,0.917743) -- (4.79968,0.917743);
\draw [c] (4.79968,0.917743) -- (4.80782,0.917743);
\definecolor{c}{rgb}{0,0,0};
\colorlet{c}{natcomp!70};
\draw [c] (4.81595,0.852991) -- (4.81595,0.902588);
\draw [c] (4.81595,0.902588) -- (4.81595,0.952185);
\draw [c] (4.80782,0.902588) -- (4.81595,0.902588);
\draw [c] (4.81595,0.902588) -- (4.82409,0.902588);
\definecolor{c}{rgb}{0,0,0};
\colorlet{c}{natcomp!70};
\draw [c] (4.83223,0.838268) -- (4.83223,0.879525);
\draw [c] (4.83223,0.879525) -- (4.83223,0.920782);
\draw [c] (4.82409,0.879525) -- (4.83223,0.879525);
\draw [c] (4.83223,0.879525) -- (4.84036,0.879525);
\definecolor{c}{rgb}{0,0,0};
\colorlet{c}{natcomp!70};
\draw [c] (4.8485,0.849755) -- (4.8485,0.901917);
\draw [c] (4.8485,0.901917) -- (4.8485,0.954078);
\draw [c] (4.84036,0.901917) -- (4.8485,0.901917);
\draw [c] (4.8485,0.901917) -- (4.85664,0.901917);
\definecolor{c}{rgb}{0,0,0};
\colorlet{c}{natcomp!70};
\draw [c] (4.86477,0.88973) -- (4.86477,0.942964);
\draw [c] (4.86477,0.942964) -- (4.86477,0.996198);
\draw [c] (4.85664,0.942964) -- (4.86477,0.942964);
\draw [c] (4.86477,0.942964) -- (4.87291,0.942964);
\definecolor{c}{rgb}{0,0,0};
\colorlet{c}{natcomp!70};
\draw [c] (4.88105,0.852361) -- (4.88105,0.898715);
\draw [c] (4.88105,0.898715) -- (4.88105,0.945068);
\draw [c] (4.87291,0.898715) -- (4.88105,0.898715);
\draw [c] (4.88105,0.898715) -- (4.88918,0.898715);
\definecolor{c}{rgb}{0,0,0};
\colorlet{c}{natcomp!70};
\draw [c] (4.89732,0.889338) -- (4.89732,0.94623);
\draw [c] (4.89732,0.94623) -- (4.89732,1.00312);
\draw [c] (4.88918,0.94623) -- (4.89732,0.94623);
\draw [c] (4.89732,0.94623) -- (4.90545,0.94623);
\definecolor{c}{rgb}{0,0,0};
\colorlet{c}{natcomp!70};
\draw [c] (4.91359,0.818285) -- (4.91359,0.857852);
\draw [c] (4.91359,0.857852) -- (4.91359,0.897419);
\draw [c] (4.90545,0.857852) -- (4.91359,0.857852);
\draw [c] (4.91359,0.857852) -- (4.92173,0.857852);
\definecolor{c}{rgb}{0,0,0};
\colorlet{c}{natcomp!70};
\draw [c] (4.92986,0.839442) -- (4.92986,0.884558);
\draw [c] (4.92986,0.884558) -- (4.92986,0.929674);
\draw [c] (4.92173,0.884558) -- (4.92986,0.884558);
\draw [c] (4.92986,0.884558) -- (4.938,0.884558);
\definecolor{c}{rgb}{0,0,0};
\colorlet{c}{natcomp!70};
\draw [c] (4.94614,0.928982) -- (4.94614,0.983751);
\draw [c] (4.94614,0.983751) -- (4.94614,1.03852);
\draw [c] (4.938,0.983751) -- (4.94614,0.983751);
\draw [c] (4.94614,0.983751) -- (4.95427,0.983751);
\definecolor{c}{rgb}{0,0,0};
\colorlet{c}{natcomp!70};
\draw [c] (4.96241,0.756623) -- (4.96241,0.793018);
\draw [c] (4.96241,0.793018) -- (4.96241,0.829412);
\draw [c] (4.95427,0.793018) -- (4.96241,0.793018);
\draw [c] (4.96241,0.793018) -- (4.97055,0.793018);
\definecolor{c}{rgb}{0,0,0};
\colorlet{c}{natcomp!70};
\draw [c] (4.97868,0.863301) -- (4.97868,0.913408);
\draw [c] (4.97868,0.913408) -- (4.97868,0.963516);
\draw [c] (4.97055,0.913408) -- (4.97868,0.913408);
\draw [c] (4.97868,0.913408) -- (4.98682,0.913408);
\definecolor{c}{rgb}{0,0,0};
\colorlet{c}{natcomp!70};
\draw [c] (4.99495,0.820068) -- (4.99495,0.859989);
\draw [c] (4.99495,0.859989) -- (4.99495,0.89991);
\draw [c] (4.98682,0.859989) -- (4.99495,0.859989);
\draw [c] (4.99495,0.859989) -- (5.00309,0.859989);
\definecolor{c}{rgb}{0,0,0};
\colorlet{c}{natcomp!70};
\draw [c] (5.01123,0.781416) -- (5.01123,0.816202);
\draw [c] (5.01123,0.816202) -- (5.01123,0.850987);
\draw [c] (5.00309,0.816202) -- (5.01123,0.816202);
\draw [c] (5.01123,0.816202) -- (5.01936,0.816202);
\definecolor{c}{rgb}{0,0,0};
\colorlet{c}{natcomp!70};
\draw [c] (5.0275,0.790764) -- (5.0275,0.827697);
\draw [c] (5.0275,0.827697) -- (5.0275,0.86463);
\draw [c] (5.01936,0.827697) -- (5.0275,0.827697);
\draw [c] (5.0275,0.827697) -- (5.03564,0.827697);
\definecolor{c}{rgb}{0,0,0};
\colorlet{c}{natcomp!70};
\draw [c] (5.04377,0.750942) -- (5.04377,0.784172);
\draw [c] (5.04377,0.784172) -- (5.04377,0.817401);
\draw [c] (5.03564,0.784172) -- (5.04377,0.784172);
\draw [c] (5.04377,0.784172) -- (5.05191,0.784172);
\definecolor{c}{rgb}{0,0,0};
\colorlet{c}{natcomp!70};
\draw [c] (5.06005,0.881823) -- (5.06005,0.934976);
\draw [c] (5.06005,0.934976) -- (5.06005,0.988128);
\draw [c] (5.05191,0.934976) -- (5.06005,0.934976);
\draw [c] (5.06005,0.934976) -- (5.06818,0.934976);
\definecolor{c}{rgb}{0,0,0};
\colorlet{c}{natcomp!70};
\draw [c] (5.07632,0.879455) -- (5.07632,0.936439);
\draw [c] (5.07632,0.936439) -- (5.07632,0.993423);
\draw [c] (5.06818,0.936439) -- (5.07632,0.936439);
\draw [c] (5.07632,0.936439) -- (5.08445,0.936439);
\definecolor{c}{rgb}{0,0,0};
\colorlet{c}{natcomp!70};
\draw [c] (5.09259,0.765327) -- (5.09259,0.799544);
\draw [c] (5.09259,0.799544) -- (5.09259,0.833762);
\draw [c] (5.08445,0.799544) -- (5.09259,0.799544);
\draw [c] (5.09259,0.799544) -- (5.10073,0.799544);
\definecolor{c}{rgb}{0,0,0};
\colorlet{c}{natcomp!70};
\draw [c] (5.10886,0.816846) -- (5.10886,0.854773);
\draw [c] (5.10886,0.854773) -- (5.10886,0.8927);
\draw [c] (5.10073,0.854773) -- (5.10886,0.854773);
\draw [c] (5.10886,0.854773) -- (5.117,0.854773);
\definecolor{c}{rgb}{0,0,0};
\colorlet{c}{natcomp!70};
\draw [c] (5.12514,0.878289) -- (5.12514,0.930852);
\draw [c] (5.12514,0.930852) -- (5.12514,0.983415);
\draw [c] (5.117,0.930852) -- (5.12514,0.930852);
\draw [c] (5.12514,0.930852) -- (5.13327,0.930852);
\definecolor{c}{rgb}{0,0,0};
\colorlet{c}{natcomp!70};
\draw [c] (5.14141,0.886117) -- (5.14141,0.93627);
\draw [c] (5.14141,0.93627) -- (5.14141,0.986423);
\draw [c] (5.13327,0.93627) -- (5.14141,0.93627);
\draw [c] (5.14141,0.93627) -- (5.14955,0.93627);
\definecolor{c}{rgb}{0,0,0};
\colorlet{c}{natcomp!70};
\draw [c] (5.15768,0.756901) -- (5.15768,0.7883);
\draw [c] (5.15768,0.7883) -- (5.15768,0.819698);
\draw [c] (5.14955,0.7883) -- (5.15768,0.7883);
\draw [c] (5.15768,0.7883) -- (5.16582,0.7883);
\definecolor{c}{rgb}{0,0,0};
\colorlet{c}{natcomp!70};
\draw [c] (5.17395,0.850388) -- (5.17395,0.903041);
\draw [c] (5.17395,0.903041) -- (5.17395,0.955693);
\draw [c] (5.16582,0.903041) -- (5.17395,0.903041);
\draw [c] (5.17395,0.903041) -- (5.18209,0.903041);
\definecolor{c}{rgb}{0,0,0};
\colorlet{c}{natcomp!70};
\draw [c] (5.19023,0.837397) -- (5.19023,0.880272);
\draw [c] (5.19023,0.880272) -- (5.19023,0.923147);
\draw [c] (5.18209,0.880272) -- (5.19023,0.880272);
\draw [c] (5.19023,0.880272) -- (5.19836,0.880272);
\definecolor{c}{rgb}{0,0,0};
\colorlet{c}{natcomp!70};
\draw [c] (5.2065,0.782891) -- (5.2065,0.819807);
\draw [c] (5.2065,0.819807) -- (5.2065,0.856723);
\draw [c] (5.19836,0.819807) -- (5.2065,0.819807);
\draw [c] (5.2065,0.819807) -- (5.21464,0.819807);
\definecolor{c}{rgb}{0,0,0};
\colorlet{c}{natcomp!70};
\draw [c] (5.22277,0.877003) -- (5.22277,0.925132);
\draw [c] (5.22277,0.925132) -- (5.22277,0.973261);
\draw [c] (5.21464,0.925132) -- (5.22277,0.925132);
\draw [c] (5.22277,0.925132) -- (5.23091,0.925132);
\definecolor{c}{rgb}{0,0,0};
\colorlet{c}{natcomp!70};
\draw [c] (5.23905,0.832507) -- (5.23905,0.877913);
\draw [c] (5.23905,0.877913) -- (5.23905,0.923319);
\draw [c] (5.23091,0.877913) -- (5.23905,0.877913);
\draw [c] (5.23905,0.877913) -- (5.24718,0.877913);
\definecolor{c}{rgb}{0,0,0};
\colorlet{c}{natcomp!70};
\draw [c] (5.25532,0.809327) -- (5.25532,0.853797);
\draw [c] (5.25532,0.853797) -- (5.25532,0.898268);
\draw [c] (5.24718,0.853797) -- (5.25532,0.853797);
\draw [c] (5.25532,0.853797) -- (5.26345,0.853797);
\definecolor{c}{rgb}{0,0,0};
\colorlet{c}{natcomp!70};
\draw [c] (5.27159,0.838939) -- (5.27159,0.884592);
\draw [c] (5.27159,0.884592) -- (5.27159,0.930245);
\draw [c] (5.26345,0.884592) -- (5.27159,0.884592);
\draw [c] (5.27159,0.884592) -- (5.27973,0.884592);
\definecolor{c}{rgb}{0,0,0};
\colorlet{c}{natcomp!70};
\draw [c] (5.28786,0.774997) -- (5.28786,0.815419);
\draw [c] (5.28786,0.815419) -- (5.28786,0.85584);
\draw [c] (5.27973,0.815419) -- (5.28786,0.815419);
\draw [c] (5.28786,0.815419) -- (5.296,0.815419);
\definecolor{c}{rgb}{0,0,0};
\colorlet{c}{natcomp!70};
\draw [c] (5.30414,0.802019) -- (5.30414,0.840859);
\draw [c] (5.30414,0.840859) -- (5.30414,0.879699);
\draw [c] (5.296,0.840859) -- (5.30414,0.840859);
\draw [c] (5.30414,0.840859) -- (5.31227,0.840859);
\definecolor{c}{rgb}{0,0,0};
\colorlet{c}{natcomp!70};
\draw [c] (5.32041,0.734849) -- (5.32041,0.765883);
\draw [c] (5.32041,0.765883) -- (5.32041,0.796917);
\draw [c] (5.31227,0.765883) -- (5.32041,0.765883);
\draw [c] (5.32041,0.765883) -- (5.32855,0.765883);
\definecolor{c}{rgb}{0,0,0};
\colorlet{c}{natcomp!70};
\draw [c] (5.33668,0.76708) -- (5.33668,0.802514);
\draw [c] (5.33668,0.802514) -- (5.33668,0.837949);
\draw [c] (5.32855,0.802514) -- (5.33668,0.802514);
\draw [c] (5.33668,0.802514) -- (5.34482,0.802514);
\definecolor{c}{rgb}{0,0,0};
\colorlet{c}{natcomp!70};
\draw [c] (5.35295,0.820219) -- (5.35295,0.864005);
\draw [c] (5.35295,0.864005) -- (5.35295,0.907791);
\draw [c] (5.34482,0.864005) -- (5.35295,0.864005);
\draw [c] (5.35295,0.864005) -- (5.36109,0.864005);
\definecolor{c}{rgb}{0,0,0};
\colorlet{c}{natcomp!70};
\draw [c] (5.36923,0.792286) -- (5.36923,0.832453);
\draw [c] (5.36923,0.832453) -- (5.36923,0.87262);
\draw [c] (5.36109,0.832453) -- (5.36923,0.832453);
\draw [c] (5.36923,0.832453) -- (5.37736,0.832453);
\definecolor{c}{rgb}{0,0,0};
\colorlet{c}{natcomp!70};
\draw [c] (5.3855,0.779486) -- (5.3855,0.817789);
\draw [c] (5.3855,0.817789) -- (5.3855,0.856092);
\draw [c] (5.37736,0.817789) -- (5.3855,0.817789);
\draw [c] (5.3855,0.817789) -- (5.39364,0.817789);
\definecolor{c}{rgb}{0,0,0};
\colorlet{c}{natcomp!70};
\draw [c] (5.40177,0.758942) -- (5.40177,0.79033);
\draw [c] (5.40177,0.79033) -- (5.40177,0.821719);
\draw [c] (5.39364,0.79033) -- (5.40177,0.79033);
\draw [c] (5.40177,0.79033) -- (5.40991,0.79033);
\definecolor{c}{rgb}{0,0,0};
\colorlet{c}{natcomp!70};
\draw [c] (5.41805,0.828335) -- (5.41805,0.872313);
\draw [c] (5.41805,0.872313) -- (5.41805,0.916292);
\draw [c] (5.40991,0.872313) -- (5.41805,0.872313);
\draw [c] (5.41805,0.872313) -- (5.42618,0.872313);
\definecolor{c}{rgb}{0,0,0};
\colorlet{c}{natcomp!70};
\draw [c] (5.43432,0.796227) -- (5.43432,0.833442);
\draw [c] (5.43432,0.833442) -- (5.43432,0.870657);
\draw [c] (5.42618,0.833442) -- (5.43432,0.833442);
\draw [c] (5.43432,0.833442) -- (5.44245,0.833442);
\definecolor{c}{rgb}{0,0,0};
\colorlet{c}{natcomp!70};
\draw [c] (5.45059,0.824412) -- (5.45059,0.86803);
\draw [c] (5.45059,0.86803) -- (5.45059,0.911649);
\draw [c] (5.44245,0.86803) -- (5.45059,0.86803);
\draw [c] (5.45059,0.86803) -- (5.45873,0.86803);
\definecolor{c}{rgb}{0,0,0};
\colorlet{c}{natcomp!70};
\draw [c] (5.46686,0.749011) -- (5.46686,0.778106);
\draw [c] (5.46686,0.778106) -- (5.46686,0.807201);
\draw [c] (5.45873,0.778106) -- (5.46686,0.778106);
\draw [c] (5.46686,0.778106) -- (5.475,0.778106);
\definecolor{c}{rgb}{0,0,0};
\colorlet{c}{natcomp!70};
\draw [c] (5.48314,0.717158) -- (5.48314,0.738269);
\draw [c] (5.48314,0.738269) -- (5.48314,0.75938);
\draw [c] (5.475,0.738269) -- (5.48314,0.738269);
\draw [c] (5.48314,0.738269) -- (5.49127,0.738269);
\definecolor{c}{rgb}{0,0,0};
\colorlet{c}{natcomp!70};
\draw [c] (5.49941,0.760794) -- (5.49941,0.791025);
\draw [c] (5.49941,0.791025) -- (5.49941,0.821255);
\draw [c] (5.49127,0.791025) -- (5.49941,0.791025);
\draw [c] (5.49941,0.791025) -- (5.50755,0.791025);
\definecolor{c}{rgb}{0,0,0};
\colorlet{c}{natcomp!70};
\draw [c] (5.51568,0.780683) -- (5.51568,0.819473);
\draw [c] (5.51568,0.819473) -- (5.51568,0.858263);
\draw [c] (5.50755,0.819473) -- (5.51568,0.819473);
\draw [c] (5.51568,0.819473) -- (5.52382,0.819473);
\definecolor{c}{rgb}{0,0,0};
\colorlet{c}{natcomp!70};
\draw [c] (5.53195,0.767697) -- (5.53195,0.81179);
\draw [c] (5.53195,0.81179) -- (5.53195,0.855883);
\draw [c] (5.52382,0.81179) -- (5.53195,0.81179);
\draw [c] (5.53195,0.81179) -- (5.54009,0.81179);
\definecolor{c}{rgb}{0,0,0};
\colorlet{c}{natcomp!70};
\draw [c] (5.54823,0.789662) -- (5.54823,0.82657);
\draw [c] (5.54823,0.82657) -- (5.54823,0.863477);
\draw [c] (5.54009,0.82657) -- (5.54823,0.82657);
\draw [c] (5.54823,0.82657) -- (5.55636,0.82657);
\definecolor{c}{rgb}{0,0,0};
\colorlet{c}{natcomp!70};
\draw [c] (5.5645,0.836001) -- (5.5645,0.881788);
\draw [c] (5.5645,0.881788) -- (5.5645,0.927576);
\draw [c] (5.55636,0.881788) -- (5.5645,0.881788);
\draw [c] (5.5645,0.881788) -- (5.57264,0.881788);
\definecolor{c}{rgb}{0,0,0};
\colorlet{c}{natcomp!70};
\draw [c] (5.58077,0.77978) -- (5.58077,0.81503);
\draw [c] (5.58077,0.81503) -- (5.58077,0.85028);
\draw [c] (5.57264,0.81503) -- (5.58077,0.81503);
\draw [c] (5.58077,0.81503) -- (5.58891,0.81503);
\definecolor{c}{rgb}{0,0,0};
\colorlet{c}{natcomp!70};
\draw [c] (5.59705,0.765328) -- (5.59705,0.798276);
\draw [c] (5.59705,0.798276) -- (5.59705,0.831225);
\draw [c] (5.58891,0.798276) -- (5.59705,0.798276);
\draw [c] (5.59705,0.798276) -- (5.60518,0.798276);
\definecolor{c}{rgb}{0,0,0};
\colorlet{c}{natcomp!70};
\draw [c] (5.61332,0.713101) -- (5.61332,0.735885);
\draw [c] (5.61332,0.735885) -- (5.61332,0.758668);
\draw [c] (5.60518,0.735885) -- (5.61332,0.735885);
\draw [c] (5.61332,0.735885) -- (5.62145,0.735885);
\definecolor{c}{rgb}{0,0,0};
\colorlet{c}{natcomp!70};
\draw [c] (5.62959,0.77156) -- (5.62959,0.806931);
\draw [c] (5.62959,0.806931) -- (5.62959,0.842301);
\draw [c] (5.62145,0.806931) -- (5.62959,0.806931);
\draw [c] (5.62959,0.806931) -- (5.63773,0.806931);
\definecolor{c}{rgb}{0,0,0};
\colorlet{c}{natcomp!70};
\draw [c] (5.64586,0.767598) -- (5.64586,0.801089);
\draw [c] (5.64586,0.801089) -- (5.64586,0.83458);
\draw [c] (5.63773,0.801089) -- (5.64586,0.801089);
\draw [c] (5.64586,0.801089) -- (5.654,0.801089);
\definecolor{c}{rgb}{0,0,0};
\colorlet{c}{natcomp!70};
\draw [c] (5.66214,0.812558) -- (5.66214,0.852249);
\draw [c] (5.66214,0.852249) -- (5.66214,0.891941);
\draw [c] (5.654,0.852249) -- (5.66214,0.852249);
\draw [c] (5.66214,0.852249) -- (5.67027,0.852249);
\definecolor{c}{rgb}{0,0,0};
\colorlet{c}{natcomp!70};
\draw [c] (5.67841,0.750819) -- (5.67841,0.790319);
\draw [c] (5.67841,0.790319) -- (5.67841,0.829819);
\draw [c] (5.67027,0.790319) -- (5.67841,0.790319);
\draw [c] (5.67841,0.790319) -- (5.68655,0.790319);
\definecolor{c}{rgb}{0,0,0};
\colorlet{c}{natcomp!70};
\draw [c] (5.69468,0.763809) -- (5.69468,0.800161);
\draw [c] (5.69468,0.800161) -- (5.69468,0.836514);
\draw [c] (5.68655,0.800161) -- (5.69468,0.800161);
\draw [c] (5.69468,0.800161) -- (5.70282,0.800161);
\definecolor{c}{rgb}{0,0,0};
\colorlet{c}{natcomp!70};
\draw [c] (5.71095,0.773509) -- (5.71095,0.810299);
\draw [c] (5.71095,0.810299) -- (5.71095,0.84709);
\draw [c] (5.70282,0.810299) -- (5.71095,0.810299);
\draw [c] (5.71095,0.810299) -- (5.71909,0.810299);
\definecolor{c}{rgb}{0,0,0};
\colorlet{c}{natcomp!70};
\draw [c] (5.72723,0.781311) -- (5.72723,0.815949);
\draw [c] (5.72723,0.815949) -- (5.72723,0.850586);
\draw [c] (5.71909,0.815949) -- (5.72723,0.815949);
\draw [c] (5.72723,0.815949) -- (5.73536,0.815949);
\definecolor{c}{rgb}{0,0,0};
\colorlet{c}{natcomp!70};
\draw [c] (5.7435,0.749041) -- (5.7435,0.783543);
\draw [c] (5.7435,0.783543) -- (5.7435,0.818044);
\draw [c] (5.73536,0.783543) -- (5.7435,0.783543);
\draw [c] (5.7435,0.783543) -- (5.75164,0.783543);
\definecolor{c}{rgb}{0,0,0};
\colorlet{c}{natcomp!70};
\draw [c] (5.75977,0.734525) -- (5.75977,0.760913);
\draw [c] (5.75977,0.760913) -- (5.75977,0.787301);
\draw [c] (5.75164,0.760913) -- (5.75977,0.760913);
\draw [c] (5.75977,0.760913) -- (5.76791,0.760913);
\definecolor{c}{rgb}{0,0,0};
\colorlet{c}{natcomp!70};
\draw [c] (5.77605,0.757329) -- (5.77605,0.795295);
\draw [c] (5.77605,0.795295) -- (5.77605,0.833262);
\draw [c] (5.76791,0.795295) -- (5.77605,0.795295);
\draw [c] (5.77605,0.795295) -- (5.78418,0.795295);
\definecolor{c}{rgb}{0,0,0};
\colorlet{c}{natcomp!70};
\draw [c] (5.79232,0.778456) -- (5.79232,0.824701);
\draw [c] (5.79232,0.824701) -- (5.79232,0.870946);
\draw [c] (5.78418,0.824701) -- (5.79232,0.824701);
\draw [c] (5.79232,0.824701) -- (5.80045,0.824701);
\definecolor{c}{rgb}{0,0,0};
\colorlet{c}{natcomp!70};
\draw [c] (5.80859,0.791499) -- (5.80859,0.828233);
\draw [c] (5.80859,0.828233) -- (5.80859,0.864968);
\draw [c] (5.80045,0.828233) -- (5.80859,0.828233);
\draw [c] (5.80859,0.828233) -- (5.81673,0.828233);
\definecolor{c}{rgb}{0,0,0};
\colorlet{c}{natcomp!70};
\draw [c] (5.82486,0.778777) -- (5.82486,0.81849);
\draw [c] (5.82486,0.81849) -- (5.82486,0.858202);
\draw [c] (5.81673,0.81849) -- (5.82486,0.81849);
\draw [c] (5.82486,0.81849) -- (5.833,0.81849);
\definecolor{c}{rgb}{0,0,0};
\colorlet{c}{natcomp!70};
\draw [c] (5.84114,0.766161) -- (5.84114,0.815256);
\draw [c] (5.84114,0.815256) -- (5.84114,0.864352);
\draw [c] (5.833,0.815256) -- (5.84114,0.815256);
\draw [c] (5.84114,0.815256) -- (5.84927,0.815256);
\definecolor{c}{rgb}{0,0,0};
\colorlet{c}{natcomp!70};
\draw [c] (5.85741,0.739717) -- (5.85741,0.766483);
\draw [c] (5.85741,0.766483) -- (5.85741,0.79325);
\draw [c] (5.84927,0.766483) -- (5.85741,0.766483);
\draw [c] (5.85741,0.766483) -- (5.86555,0.766483);
\definecolor{c}{rgb}{0,0,0};
\colorlet{c}{natcomp!70};
\draw [c] (5.87368,0.767167) -- (5.87368,0.798411);
\draw [c] (5.87368,0.798411) -- (5.87368,0.829655);
\draw [c] (5.86555,0.798411) -- (5.87368,0.798411);
\draw [c] (5.87368,0.798411) -- (5.88182,0.798411);
\definecolor{c}{rgb}{0,0,0};
\colorlet{c}{natcomp!70};
\draw [c] (5.88995,0.7537) -- (5.88995,0.78274);
\draw [c] (5.88995,0.78274) -- (5.88995,0.811781);
\draw [c] (5.88182,0.78274) -- (5.88995,0.78274);
\draw [c] (5.88995,0.78274) -- (5.89809,0.78274);
\definecolor{c}{rgb}{0,0,0};
\colorlet{c}{natcomp!70};
\draw [c] (5.90623,0.787736) -- (5.90623,0.823567);
\draw [c] (5.90623,0.823567) -- (5.90623,0.859398);
\draw [c] (5.89809,0.823567) -- (5.90623,0.823567);
\draw [c] (5.90623,0.823567) -- (5.91436,0.823567);
\definecolor{c}{rgb}{0,0,0};
\colorlet{c}{natcomp!70};
\draw [c] (5.9225,0.7657) -- (5.9225,0.797966);
\draw [c] (5.9225,0.797966) -- (5.9225,0.830233);
\draw [c] (5.91436,0.797966) -- (5.9225,0.797966);
\draw [c] (5.9225,0.797966) -- (5.93064,0.797966);
\definecolor{c}{rgb}{0,0,0};
\colorlet{c}{natcomp!70};
\draw [c] (5.93877,0.776906) -- (5.93877,0.812086);
\draw [c] (5.93877,0.812086) -- (5.93877,0.847266);
\draw [c] (5.93064,0.812086) -- (5.93877,0.812086);
\draw [c] (5.93877,0.812086) -- (5.94691,0.812086);
\definecolor{c}{rgb}{0,0,0};
\colorlet{c}{natcomp!70};
\draw [c] (5.95505,0.727034) -- (5.95505,0.751779);
\draw [c] (5.95505,0.751779) -- (5.95505,0.776524);
\draw [c] (5.94691,0.751779) -- (5.95505,0.751779);
\draw [c] (5.95505,0.751779) -- (5.96318,0.751779);
\definecolor{c}{rgb}{0,0,0};
\colorlet{c}{natcomp!70};
\draw [c] (5.97132,0.718677) -- (5.97132,0.7484);
\draw [c] (5.97132,0.7484) -- (5.97132,0.778123);
\draw [c] (5.96318,0.7484) -- (5.97132,0.7484);
\draw [c] (5.97132,0.7484) -- (5.97945,0.7484);
\definecolor{c}{rgb}{0,0,0};
\colorlet{c}{natcomp!70};
\draw [c] (5.98759,0.723722) -- (5.98759,0.746507);
\draw [c] (5.98759,0.746507) -- (5.98759,0.769292);
\draw [c] (5.97945,0.746507) -- (5.98759,0.746507);
\draw [c] (5.98759,0.746507) -- (5.99573,0.746507);
\definecolor{c}{rgb}{0,0,0};
\colorlet{c}{natcomp!70};
\draw [c] (6.00386,0.756977) -- (6.00386,0.792774);
\draw [c] (6.00386,0.792774) -- (6.00386,0.828571);
\draw [c] (5.99573,0.792774) -- (6.00386,0.792774);
\draw [c] (6.00386,0.792774) -- (6.012,0.792774);
\definecolor{c}{rgb}{0,0,0};
\colorlet{c}{natcomp!70};
\draw [c] (6.02014,0.72654) -- (6.02014,0.750922);
\draw [c] (6.02014,0.750922) -- (6.02014,0.775304);
\draw [c] (6.012,0.750922) -- (6.02014,0.750922);
\draw [c] (6.02014,0.750922) -- (6.02827,0.750922);
\definecolor{c}{rgb}{0,0,0};
\colorlet{c}{natcomp!70};
\draw [c] (6.03641,0.765876) -- (6.03641,0.798169);
\draw [c] (6.03641,0.798169) -- (6.03641,0.830462);
\draw [c] (6.02827,0.798169) -- (6.03641,0.798169);
\draw [c] (6.03641,0.798169) -- (6.04455,0.798169);
\definecolor{c}{rgb}{0,0,0};
\colorlet{c}{natcomp!70};
\draw [c] (6.05268,0.723692) -- (6.05268,0.763257);
\draw [c] (6.05268,0.763257) -- (6.05268,0.802821);
\draw [c] (6.04455,0.763257) -- (6.05268,0.763257);
\draw [c] (6.05268,0.763257) -- (6.06082,0.763257);
\definecolor{c}{rgb}{0,0,0};
\colorlet{c}{natcomp!70};
\draw [c] (6.06895,0.735947) -- (6.06895,0.776286);
\draw [c] (6.06895,0.776286) -- (6.06895,0.816626);
\draw [c] (6.06082,0.776286) -- (6.06895,0.776286);
\draw [c] (6.06895,0.776286) -- (6.07709,0.776286);
\definecolor{c}{rgb}{0,0,0};
\colorlet{c}{natcomp!70};
\draw [c] (6.08523,0.768199) -- (6.08523,0.8023);
\draw [c] (6.08523,0.8023) -- (6.08523,0.836401);
\draw [c] (6.07709,0.8023) -- (6.08523,0.8023);
\draw [c] (6.08523,0.8023) -- (6.09336,0.8023);
\definecolor{c}{rgb}{0,0,0};
\colorlet{c}{natcomp!70};
\draw [c] (6.1015,0.710277) -- (6.1015,0.729228);
\draw [c] (6.1015,0.729228) -- (6.1015,0.748179);
\draw [c] (6.09336,0.729228) -- (6.1015,0.729228);
\draw [c] (6.1015,0.729228) -- (6.10964,0.729228);
\definecolor{c}{rgb}{0,0,0};
\colorlet{c}{natcomp!70};
\draw [c] (6.11777,0.740692) -- (6.11777,0.767798);
\draw [c] (6.11777,0.767798) -- (6.11777,0.794904);
\draw [c] (6.10964,0.767798) -- (6.11777,0.767798);
\draw [c] (6.11777,0.767798) -- (6.12591,0.767798);
\definecolor{c}{rgb}{0,0,0};
\colorlet{c}{natcomp!70};
\draw [c] (6.13405,0.720528) -- (6.13405,0.743887);
\draw [c] (6.13405,0.743887) -- (6.13405,0.767247);
\draw [c] (6.12591,0.743887) -- (6.13405,0.743887);
\draw [c] (6.13405,0.743887) -- (6.14218,0.743887);
\definecolor{c}{rgb}{0,0,0};
\colorlet{c}{natcomp!70};
\draw [c] (6.15032,0.69748) -- (6.15032,0.712307);
\draw [c] (6.15032,0.712307) -- (6.15032,0.727135);
\draw [c] (6.14218,0.712307) -- (6.15032,0.712307);
\draw [c] (6.15032,0.712307) -- (6.15845,0.712307);
\definecolor{c}{rgb}{0,0,0};
\colorlet{c}{natcomp!70};
\draw [c] (6.16659,0.709648) -- (6.16659,0.728192);
\draw [c] (6.16659,0.728192) -- (6.16659,0.746736);
\draw [c] (6.15845,0.728192) -- (6.16659,0.728192);
\draw [c] (6.16659,0.728192) -- (6.17473,0.728192);
\definecolor{c}{rgb}{0,0,0};
\colorlet{c}{natcomp!70};
\draw [c] (6.18286,0.7346) -- (6.18286,0.766423);
\draw [c] (6.18286,0.766423) -- (6.18286,0.798247);
\draw [c] (6.17473,0.766423) -- (6.18286,0.766423);
\draw [c] (6.18286,0.766423) -- (6.191,0.766423);
\definecolor{c}{rgb}{0,0,0};
\colorlet{c}{natcomp!70};
\draw [c] (6.19914,0.725682) -- (6.19914,0.753499);
\draw [c] (6.19914,0.753499) -- (6.19914,0.781317);
\draw [c] (6.191,0.753499) -- (6.19914,0.753499);
\draw [c] (6.19914,0.753499) -- (6.20727,0.753499);
\definecolor{c}{rgb}{0,0,0};
\colorlet{c}{natcomp!70};
\draw [c] (6.21541,0.726686) -- (6.21541,0.759284);
\draw [c] (6.21541,0.759284) -- (6.21541,0.791882);
\draw [c] (6.20727,0.759284) -- (6.21541,0.759284);
\draw [c] (6.21541,0.759284) -- (6.22355,0.759284);
\definecolor{c}{rgb}{0,0,0};
\colorlet{c}{natcomp!70};
\draw [c] (6.23168,0.742176) -- (6.23168,0.777624);
\draw [c] (6.23168,0.777624) -- (6.23168,0.813072);
\draw [c] (6.22355,0.777624) -- (6.23168,0.777624);
\draw [c] (6.23168,0.777624) -- (6.23982,0.777624);
\definecolor{c}{rgb}{0,0,0};
\colorlet{c}{natcomp!70};
\draw [c] (6.24795,0.776333) -- (6.24795,0.816985);
\draw [c] (6.24795,0.816985) -- (6.24795,0.857637);
\draw [c] (6.23982,0.816985) -- (6.24795,0.816985);
\draw [c] (6.24795,0.816985) -- (6.25609,0.816985);
\definecolor{c}{rgb}{0,0,0};
\colorlet{c}{natcomp!70};
\draw [c] (6.26423,0.716593) -- (6.26423,0.737387);
\draw [c] (6.26423,0.737387) -- (6.26423,0.758182);
\draw [c] (6.25609,0.737387) -- (6.26423,0.737387);
\draw [c] (6.26423,0.737387) -- (6.27236,0.737387);
\definecolor{c}{rgb}{0,0,0};
\colorlet{c}{natcomp!70};
\draw [c] (6.2805,0.73157) -- (6.2805,0.759414);
\draw [c] (6.2805,0.759414) -- (6.2805,0.787258);
\draw [c] (6.27236,0.759414) -- (6.2805,0.759414);
\draw [c] (6.2805,0.759414) -- (6.28864,0.759414);
\definecolor{c}{rgb}{0,0,0};
\colorlet{c}{natcomp!70};
\draw [c] (6.29677,0.765117) -- (6.29677,0.800819);
\draw [c] (6.29677,0.800819) -- (6.29677,0.83652);
\draw [c] (6.28864,0.800819) -- (6.29677,0.800819);
\draw [c] (6.29677,0.800819) -- (6.30491,0.800819);
\definecolor{c}{rgb}{0,0,0};
\colorlet{c}{natcomp!70};
\draw [c] (6.31305,0.697057) -- (6.31305,0.711255);
\draw [c] (6.31305,0.711255) -- (6.31305,0.725454);
\draw [c] (6.30491,0.711255) -- (6.31305,0.711255);
\draw [c] (6.31305,0.711255) -- (6.32118,0.711255);
\definecolor{c}{rgb}{0,0,0};
\colorlet{c}{natcomp!70};
\draw [c] (6.32932,0.699766) -- (6.32932,0.721515);
\draw [c] (6.32932,0.721515) -- (6.32932,0.743263);
\draw [c] (6.32118,0.721515) -- (6.32932,0.721515);
\draw [c] (6.32932,0.721515) -- (6.33745,0.721515);
\definecolor{c}{rgb}{0,0,0};
\colorlet{c}{natcomp!70};
\draw [c] (6.34559,0.75193) -- (6.34559,0.784287);
\draw [c] (6.34559,0.784287) -- (6.34559,0.816645);
\draw [c] (6.33745,0.784287) -- (6.34559,0.784287);
\draw [c] (6.34559,0.784287) -- (6.35373,0.784287);
\definecolor{c}{rgb}{0,0,0};
\colorlet{c}{natcomp!70};
\draw [c] (6.36186,0.751842) -- (6.36186,0.788755);
\draw [c] (6.36186,0.788755) -- (6.36186,0.825668);
\draw [c] (6.35373,0.788755) -- (6.36186,0.788755);
\draw [c] (6.36186,0.788755) -- (6.37,0.788755);
\definecolor{c}{rgb}{0,0,0};
\colorlet{c}{natcomp!70};
\draw [c] (6.37814,0.740937) -- (6.37814,0.773723);
\draw [c] (6.37814,0.773723) -- (6.37814,0.806508);
\draw [c] (6.37,0.773723) -- (6.37814,0.773723);
\draw [c] (6.37814,0.773723) -- (6.38627,0.773723);
\definecolor{c}{rgb}{0,0,0};
\colorlet{c}{natcomp!70};
\draw [c] (6.39441,0.698759) -- (6.39441,0.715275);
\draw [c] (6.39441,0.715275) -- (6.39441,0.73179);
\draw [c] (6.38627,0.715275) -- (6.39441,0.715275);
\draw [c] (6.39441,0.715275) -- (6.40255,0.715275);
\definecolor{c}{rgb}{0,0,0};
\colorlet{c}{natcomp!70};
\draw [c] (6.41068,0.710571) -- (6.41068,0.730263);
\draw [c] (6.41068,0.730263) -- (6.41068,0.749955);
\draw [c] (6.40255,0.730263) -- (6.41068,0.730263);
\draw [c] (6.41068,0.730263) -- (6.41882,0.730263);
\definecolor{c}{rgb}{0,0,0};
\colorlet{c}{natcomp!70};
\draw [c] (6.42695,0.733223) -- (6.42695,0.762826);
\draw [c] (6.42695,0.762826) -- (6.42695,0.792429);
\draw [c] (6.41882,0.762826) -- (6.42695,0.762826);
\draw [c] (6.42695,0.762826) -- (6.43509,0.762826);
\definecolor{c}{rgb}{0,0,0};
\colorlet{c}{natcomp!70};
\draw [c] (6.44323,0.722776) -- (6.44323,0.747622);
\draw [c] (6.44323,0.747622) -- (6.44323,0.772467);
\draw [c] (6.43509,0.747622) -- (6.44323,0.747622);
\draw [c] (6.44323,0.747622) -- (6.45136,0.747622);
\definecolor{c}{rgb}{0,0,0};
\colorlet{c}{natcomp!70};
\draw [c] (6.4595,0.739345) -- (6.4595,0.766635);
\draw [c] (6.4595,0.766635) -- (6.4595,0.793925);
\draw [c] (6.45136,0.766635) -- (6.4595,0.766635);
\draw [c] (6.4595,0.766635) -- (6.46764,0.766635);
\definecolor{c}{rgb}{0,0,0};
\colorlet{c}{natcomp!70};
\draw [c] (6.47577,0.710389) -- (6.47577,0.729721);
\draw [c] (6.47577,0.729721) -- (6.47577,0.749052);
\draw [c] (6.46764,0.729721) -- (6.47577,0.729721);
\draw [c] (6.47577,0.729721) -- (6.48391,0.729721);
\definecolor{c}{rgb}{0,0,0};
\colorlet{c}{natcomp!70};
\draw [c] (6.49205,0.741601) -- (6.49205,0.774287);
\draw [c] (6.49205,0.774287) -- (6.49205,0.806974);
\draw [c] (6.48391,0.774287) -- (6.49205,0.774287);
\draw [c] (6.49205,0.774287) -- (6.50018,0.774287);
\definecolor{c}{rgb}{0,0,0};
\colorlet{c}{natcomp!70};
\draw [c] (6.50832,0.704581) -- (6.50832,0.722973);
\draw [c] (6.50832,0.722973) -- (6.50832,0.741364);
\draw [c] (6.50018,0.722973) -- (6.50832,0.722973);
\draw [c] (6.50832,0.722973) -- (6.51645,0.722973);
\definecolor{c}{rgb}{0,0,0};
\colorlet{c}{natcomp!70};
\draw [c] (6.52459,0.698445) -- (6.52459,0.714356);
\draw [c] (6.52459,0.714356) -- (6.52459,0.730267);
\draw [c] (6.51645,0.714356) -- (6.52459,0.714356);
\draw [c] (6.52459,0.714356) -- (6.53273,0.714356);
\definecolor{c}{rgb}{0,0,0};
\colorlet{c}{natcomp!70};
\draw [c] (6.54086,0.713893) -- (6.54086,0.736042);
\draw [c] (6.54086,0.736042) -- (6.54086,0.758192);
\draw [c] (6.53273,0.736042) -- (6.54086,0.736042);
\draw [c] (6.54086,0.736042) -- (6.549,0.736042);
\definecolor{c}{rgb}{0,0,0};
\colorlet{c}{natcomp!70};
\draw [c] (6.55714,0.705456) -- (6.55714,0.724441);
\draw [c] (6.55714,0.724441) -- (6.55714,0.743426);
\draw [c] (6.549,0.724441) -- (6.55714,0.724441);
\draw [c] (6.55714,0.724441) -- (6.56527,0.724441);
\definecolor{c}{rgb}{0,0,0};
\colorlet{c}{natcomp!70};
\draw [c] (6.57341,0.708858) -- (6.57341,0.726682);
\draw [c] (6.57341,0.726682) -- (6.57341,0.744506);
\draw [c] (6.56527,0.726682) -- (6.57341,0.726682);
\draw [c] (6.57341,0.726682) -- (6.58155,0.726682);
\definecolor{c}{rgb}{0,0,0};
\colorlet{c}{natcomp!70};
\draw [c] (6.58968,0.744164) -- (6.58968,0.773304);
\draw [c] (6.58968,0.773304) -- (6.58968,0.802443);
\draw [c] (6.58155,0.773304) -- (6.58968,0.773304);
\draw [c] (6.58968,0.773304) -- (6.59782,0.773304);
\definecolor{c}{rgb}{0,0,0};
\colorlet{c}{natcomp!70};
\draw [c] (6.60595,0.728187) -- (6.60595,0.754684);
\draw [c] (6.60595,0.754684) -- (6.60595,0.781181);
\draw [c] (6.59782,0.754684) -- (6.60595,0.754684);
\draw [c] (6.60595,0.754684) -- (6.61409,0.754684);
\definecolor{c}{rgb}{0,0,0};
\colorlet{c}{natcomp!70};
\draw [c] (6.62223,0.716607) -- (6.62223,0.741593);
\draw [c] (6.62223,0.741593) -- (6.62223,0.766578);
\draw [c] (6.61409,0.741593) -- (6.62223,0.741593);
\draw [c] (6.62223,0.741593) -- (6.63036,0.741593);
\definecolor{c}{rgb}{0,0,0};
\colorlet{c}{natcomp!70};
\draw [c] (6.6385,0.699804) -- (6.6385,0.717549);
\draw [c] (6.6385,0.717549) -- (6.6385,0.735295);
\draw [c] (6.63036,0.717549) -- (6.6385,0.717549);
\draw [c] (6.6385,0.717549) -- (6.64664,0.717549);
\definecolor{c}{rgb}{0,0,0};
\colorlet{c}{natcomp!70};
\draw [c] (6.65477,0.753668) -- (6.65477,0.789033);
\draw [c] (6.65477,0.789033) -- (6.65477,0.824398);
\draw [c] (6.64664,0.789033) -- (6.65477,0.789033);
\draw [c] (6.65477,0.789033) -- (6.66291,0.789033);
\definecolor{c}{rgb}{0,0,0};
\colorlet{c}{natcomp!70};
\draw [c] (6.67105,0.708234) -- (6.67105,0.734765);
\draw [c] (6.67105,0.734765) -- (6.67105,0.761297);
\draw [c] (6.66291,0.734765) -- (6.67105,0.734765);
\draw [c] (6.67105,0.734765) -- (6.67918,0.734765);
\definecolor{c}{rgb}{0,0,0};
\colorlet{c}{natcomp!70};
\draw [c] (6.68732,0.713447) -- (6.68732,0.738205);
\draw [c] (6.68732,0.738205) -- (6.68732,0.762963);
\draw [c] (6.67918,0.738205) -- (6.68732,0.738205);
\draw [c] (6.68732,0.738205) -- (6.69545,0.738205);
\definecolor{c}{rgb}{0,0,0};
\colorlet{c}{natcomp!70};
\draw [c] (6.70359,0.733101) -- (6.70359,0.759424);
\draw [c] (6.70359,0.759424) -- (6.70359,0.785747);
\draw [c] (6.69545,0.759424) -- (6.70359,0.759424);
\draw [c] (6.70359,0.759424) -- (6.71173,0.759424);
\definecolor{c}{rgb}{0,0,0};
\colorlet{c}{natcomp!70};
\draw [c] (6.71986,0.703929) -- (6.71986,0.721255);
\draw [c] (6.71986,0.721255) -- (6.71986,0.738582);
\draw [c] (6.71173,0.721255) -- (6.71986,0.721255);
\draw [c] (6.71986,0.721255) -- (6.728,0.721255);
\definecolor{c}{rgb}{0,0,0};
\colorlet{c}{natcomp!70};
\draw [c] (6.73614,0.704397) -- (6.73614,0.721875);
\draw [c] (6.73614,0.721875) -- (6.73614,0.739353);
\draw [c] (6.728,0.721875) -- (6.73614,0.721875);
\draw [c] (6.73614,0.721875) -- (6.74427,0.721875);
\definecolor{c}{rgb}{0,0,0};
\colorlet{c}{natcomp!70};
\draw [c] (6.75241,0.710501) -- (6.75241,0.729699);
\draw [c] (6.75241,0.729699) -- (6.75241,0.748896);
\draw [c] (6.74427,0.729699) -- (6.75241,0.729699);
\draw [c] (6.75241,0.729699) -- (6.76055,0.729699);
\definecolor{c}{rgb}{0,0,0};
\colorlet{c}{natcomp!70};
\draw [c] (6.76868,0.711589) -- (6.76868,0.731751);
\draw [c] (6.76868,0.731751) -- (6.76868,0.751914);
\draw [c] (6.76055,0.731751) -- (6.76868,0.731751);
\draw [c] (6.76868,0.731751) -- (6.77682,0.731751);
\definecolor{c}{rgb}{0,0,0};
\colorlet{c}{natcomp!70};
\draw [c] (6.78495,0.718017) -- (6.78495,0.745245);
\draw [c] (6.78495,0.745245) -- (6.78495,0.772473);
\draw [c] (6.77682,0.745245) -- (6.78495,0.745245);
\draw [c] (6.78495,0.745245) -- (6.79309,0.745245);
\definecolor{c}{rgb}{0,0,0};
\colorlet{c}{natcomp!70};
\draw [c] (6.80123,0.697887) -- (6.80123,0.713515);
\draw [c] (6.80123,0.713515) -- (6.80123,0.729144);
\draw [c] (6.79309,0.713515) -- (6.80123,0.713515);
\draw [c] (6.80123,0.713515) -- (6.80936,0.713515);
\definecolor{c}{rgb}{0,0,0};
\colorlet{c}{natcomp!70};
\draw [c] (6.8175,0.712023) -- (6.8175,0.732788);
\draw [c] (6.8175,0.732788) -- (6.8175,0.753553);
\draw [c] (6.80936,0.732788) -- (6.8175,0.732788);
\draw [c] (6.8175,0.732788) -- (6.82564,0.732788);
\definecolor{c}{rgb}{0,0,0};
\colorlet{c}{natcomp!70};
\draw [c] (6.83377,0.714232) -- (6.83377,0.736446);
\draw [c] (6.83377,0.736446) -- (6.83377,0.758661);
\draw [c] (6.82564,0.736446) -- (6.83377,0.736446);
\draw [c] (6.83377,0.736446) -- (6.84191,0.736446);
\definecolor{c}{rgb}{0,0,0};
\colorlet{c}{natcomp!70};
\draw [c] (6.85005,0.709462) -- (6.85005,0.739161);
\draw [c] (6.85005,0.739161) -- (6.85005,0.768861);
\draw [c] (6.84191,0.739161) -- (6.85005,0.739161);
\draw [c] (6.85005,0.739161) -- (6.85818,0.739161);
\definecolor{c}{rgb}{0,0,0};
\colorlet{c}{natcomp!70};
\draw [c] (6.86632,0.69326) -- (6.86632,0.708575);
\draw [c] (6.86632,0.708575) -- (6.86632,0.72389);
\draw [c] (6.85818,0.708575) -- (6.86632,0.708575);
\draw [c] (6.86632,0.708575) -- (6.87445,0.708575);
\definecolor{c}{rgb}{0,0,0};
\colorlet{c}{natcomp!70};
\draw [c] (6.88259,0.703644) -- (6.88259,0.720612);
\draw [c] (6.88259,0.720612) -- (6.88259,0.73758);
\draw [c] (6.87445,0.720612) -- (6.88259,0.720612);
\draw [c] (6.88259,0.720612) -- (6.89073,0.720612);
\definecolor{c}{rgb}{0,0,0};
\colorlet{c}{natcomp!70};
\draw [c] (6.89886,0.692271) -- (6.89886,0.705485);
\draw [c] (6.89886,0.705485) -- (6.89886,0.7187);
\draw [c] (6.89073,0.705485) -- (6.89886,0.705485);
\draw [c] (6.89886,0.705485) -- (6.907,0.705485);
\definecolor{c}{rgb}{0,0,0};
\colorlet{c}{natcomp!70};
\draw [c] (6.91514,0.692587) -- (6.91514,0.706325);
\draw [c] (6.91514,0.706325) -- (6.91514,0.720063);
\draw [c] (6.907,0.706325) -- (6.91514,0.706325);
\draw [c] (6.91514,0.706325) -- (6.92327,0.706325);
\definecolor{c}{rgb}{0,0,0};
\colorlet{c}{natcomp!70};
\draw [c] (6.93141,0.686927) -- (6.93141,0.695065);
\draw [c] (6.93141,0.695065) -- (6.93141,0.703204);
\draw [c] (6.92327,0.695065) -- (6.93141,0.695065);
\draw [c] (6.93141,0.695065) -- (6.93955,0.695065);
\definecolor{c}{rgb}{0,0,0};
\colorlet{c}{natcomp!70};
\draw [c] (6.94768,0.704333) -- (6.94768,0.7221);
\draw [c] (6.94768,0.7221) -- (6.94768,0.739867);
\draw [c] (6.93955,0.7221) -- (6.94768,0.7221);
\draw [c] (6.94768,0.7221) -- (6.95582,0.7221);
\definecolor{c}{rgb}{0,0,0};
\colorlet{c}{natcomp!70};
\draw [c] (6.96395,0.686909) -- (6.96395,0.68692);
\draw [c] (6.96395,0.68692) -- (6.96395,0.686931);
\draw [c] (6.95582,0.68692) -- (6.96395,0.68692);
\draw [c] (6.96395,0.68692) -- (6.97209,0.68692);
\definecolor{c}{rgb}{0,0,0};
\colorlet{c}{natcomp!70};
\draw [c] (6.98023,0.718409) -- (6.98023,0.740294);
\draw [c] (6.98023,0.740294) -- (6.98023,0.762179);
\draw [c] (6.97209,0.740294) -- (6.98023,0.740294);
\draw [c] (6.98023,0.740294) -- (6.98836,0.740294);
\definecolor{c}{rgb}{0,0,0};
\colorlet{c}{natcomp!70};
\draw [c] (6.9965,0.686917) -- (6.9965,0.68693);
\draw [c] (6.9965,0.68693) -- (6.9965,0.686943);
\draw [c] (6.98836,0.68693) -- (6.9965,0.68693);
\draw [c] (6.9965,0.68693) -- (7.00464,0.68693);
\definecolor{c}{rgb}{0,0,0};
\colorlet{c}{natcomp!70};
\draw [c] (7.01277,0.713906) -- (7.01277,0.736963);
\draw [c] (7.01277,0.736963) -- (7.01277,0.760019);
\draw [c] (7.00464,0.736963) -- (7.01277,0.736963);
\draw [c] (7.01277,0.736963) -- (7.02091,0.736963);
\definecolor{c}{rgb}{0,0,0};
\colorlet{c}{natcomp!70};
\draw [c] (7.02905,0.703869) -- (7.02905,0.721431);
\draw [c] (7.02905,0.721431) -- (7.02905,0.738992);
\draw [c] (7.02091,0.721431) -- (7.02905,0.721431);
\draw [c] (7.02905,0.721431) -- (7.03718,0.721431);
\definecolor{c}{rgb}{0,0,0};
\colorlet{c}{natcomp!70};
\draw [c] (7.04532,0.705655) -- (7.04532,0.724877);
\draw [c] (7.04532,0.724877) -- (7.04532,0.7441);
\draw [c] (7.03718,0.724877) -- (7.04532,0.724877);
\draw [c] (7.04532,0.724877) -- (7.05345,0.724877);
\definecolor{c}{rgb}{0,0,0};
\colorlet{c}{natcomp!70};
\draw [c] (7.06159,0.703801) -- (7.06159,0.72083);
\draw [c] (7.06159,0.72083) -- (7.06159,0.73786);
\draw [c] (7.05345,0.72083) -- (7.06159,0.72083);
\draw [c] (7.06159,0.72083) -- (7.06973,0.72083);
\definecolor{c}{rgb}{0,0,0};
\colorlet{c}{natcomp!70};
\draw [c] (7.07786,0.719124) -- (7.07786,0.741674);
\draw [c] (7.07786,0.741674) -- (7.07786,0.764224);
\draw [c] (7.06973,0.741674) -- (7.07786,0.741674);
\draw [c] (7.07786,0.741674) -- (7.086,0.741674);
\definecolor{c}{rgb}{0,0,0};
\colorlet{c}{natcomp!70};
\draw [c] (7.09414,0.709945) -- (7.09414,0.738016);
\draw [c] (7.09414,0.738016) -- (7.09414,0.766087);
\draw [c] (7.086,0.738016) -- (7.09414,0.738016);
\draw [c] (7.09414,0.738016) -- (7.10227,0.738016);
\definecolor{c}{rgb}{0,0,0};
\colorlet{c}{natcomp!70};
\draw [c] (7.11041,0.691149) -- (7.11041,0.701332);
\draw [c] (7.11041,0.701332) -- (7.11041,0.711516);
\draw [c] (7.10227,0.701332) -- (7.11041,0.701332);
\draw [c] (7.11041,0.701332) -- (7.11855,0.701332);
\definecolor{c}{rgb}{0,0,0};
\colorlet{c}{natcomp!70};
\draw [c] (7.12668,0.706897) -- (7.12668,0.728052);
\draw [c] (7.12668,0.728052) -- (7.12668,0.749206);
\draw [c] (7.11855,0.728052) -- (7.12668,0.728052);
\draw [c] (7.12668,0.728052) -- (7.13482,0.728052);
\definecolor{c}{rgb}{0,0,0};
\colorlet{c}{natcomp!70};
\draw [c] (7.14295,0.691954) -- (7.14295,0.704175);
\draw [c] (7.14295,0.704175) -- (7.14295,0.716395);
\draw [c] (7.13482,0.704175) -- (7.14295,0.704175);
\draw [c] (7.14295,0.704175) -- (7.15109,0.704175);
\definecolor{c}{rgb}{0,0,0};
\colorlet{c}{natcomp!70};
\draw [c] (7.15923,0.686944) -- (7.15923,0.695807);
\draw [c] (7.15923,0.695807) -- (7.15923,0.704669);
\draw [c] (7.15109,0.695807) -- (7.15923,0.695807);
\draw [c] (7.15923,0.695807) -- (7.16736,0.695807);
\definecolor{c}{rgb}{0,0,0};
\colorlet{c}{natcomp!70};
\draw [c] (7.1755,0.713402) -- (7.1755,0.735241);
\draw [c] (7.1755,0.735241) -- (7.1755,0.75708);
\draw [c] (7.16736,0.735241) -- (7.1755,0.735241);
\draw [c] (7.1755,0.735241) -- (7.18364,0.735241);
\definecolor{c}{rgb}{0,0,0};
\colorlet{c}{natcomp!70};
\draw [c] (7.19177,0.70731) -- (7.19177,0.72784);
\draw [c] (7.19177,0.72784) -- (7.19177,0.74837);
\draw [c] (7.18364,0.72784) -- (7.19177,0.72784);
\draw [c] (7.19177,0.72784) -- (7.19991,0.72784);
\definecolor{c}{rgb}{0,0,0};
\colorlet{c}{natcomp!70};
\draw [c] (7.20805,0.694873) -- (7.20805,0.730967);
\draw [c] (7.20805,0.730967) -- (7.20805,0.767061);
\draw [c] (7.19991,0.730967) -- (7.20805,0.730967);
\draw [c] (7.20805,0.730967) -- (7.21618,0.730967);
\definecolor{c}{rgb}{0,0,0};
\colorlet{c}{natcomp!70};
\draw [c] (7.22432,0.692201) -- (7.22432,0.704986);
\draw [c] (7.22432,0.704986) -- (7.22432,0.71777);
\draw [c] (7.21618,0.704986) -- (7.22432,0.704986);
\draw [c] (7.22432,0.704986) -- (7.23245,0.704986);
\definecolor{c}{rgb}{0,0,0};
\colorlet{c}{natcomp!70};
\draw [c] (7.24059,0.694145) -- (7.24059,0.712713);
\draw [c] (7.24059,0.712713) -- (7.24059,0.731281);
\draw [c] (7.23245,0.712713) -- (7.24059,0.712713);
\draw [c] (7.24059,0.712713) -- (7.24873,0.712713);
\definecolor{c}{rgb}{0,0,0};
\colorlet{c}{natcomp!70};
\draw [c] (7.25686,0.686911) -- (7.25686,0.696805);
\draw [c] (7.25686,0.696805) -- (7.25686,0.706699);
\draw [c] (7.24873,0.696805) -- (7.25686,0.696805);
\draw [c] (7.25686,0.696805) -- (7.265,0.696805);
\definecolor{c}{rgb}{0,0,0};
\colorlet{c}{natcomp!70};
\draw [c] (7.27314,0.686939) -- (7.27314,0.698171);
\draw [c] (7.27314,0.698171) -- (7.27314,0.709404);
\draw [c] (7.265,0.698171) -- (7.27314,0.698171);
\draw [c] (7.27314,0.698171) -- (7.28127,0.698171);
\definecolor{c}{rgb}{0,0,0};
\colorlet{c}{natcomp!70};
\draw [c] (7.28941,0.705817) -- (7.28941,0.724823);
\draw [c] (7.28941,0.724823) -- (7.28941,0.743829);
\draw [c] (7.28127,0.724823) -- (7.28941,0.724823);
\draw [c] (7.28941,0.724823) -- (7.29755,0.724823);
\definecolor{c}{rgb}{0,0,0};
\colorlet{c}{natcomp!70};
\draw [c] (7.30568,0.704012) -- (7.30568,0.721633);
\draw [c] (7.30568,0.721633) -- (7.30568,0.739254);
\draw [c] (7.29755,0.721633) -- (7.30568,0.721633);
\draw [c] (7.30568,0.721633) -- (7.31382,0.721633);
\definecolor{c}{rgb}{0,0,0};
\colorlet{c}{natcomp!70};
\draw [c] (7.32195,0.714809) -- (7.32195,0.753099);
\draw [c] (7.32195,0.753099) -- (7.32195,0.791389);
\draw [c] (7.31382,0.753099) -- (7.32195,0.753099);
\draw [c] (7.32195,0.753099) -- (7.33009,0.753099);
\definecolor{c}{rgb}{0,0,0};
\colorlet{c}{natcomp!70};
\draw [c] (7.33823,0.697923) -- (7.33823,0.713018);
\draw [c] (7.33823,0.713018) -- (7.33823,0.728113);
\draw [c] (7.33009,0.713018) -- (7.33823,0.713018);
\draw [c] (7.33823,0.713018) -- (7.34636,0.713018);
\definecolor{c}{rgb}{0,0,0};
\colorlet{c}{natcomp!70};
\draw [c] (7.3545,0.698691) -- (7.3545,0.715314);
\draw [c] (7.3545,0.715314) -- (7.3545,0.731936);
\draw [c] (7.34636,0.715314) -- (7.3545,0.715314);
\draw [c] (7.3545,0.715314) -- (7.36264,0.715314);
\definecolor{c}{rgb}{0,0,0};
\colorlet{c}{natcomp!70};
\draw [c] (7.37077,0.717852) -- (7.37077,0.745539);
\draw [c] (7.37077,0.745539) -- (7.37077,0.773225);
\draw [c] (7.36264,0.745539) -- (7.37077,0.745539);
\draw [c] (7.37077,0.745539) -- (7.37891,0.745539);
\definecolor{c}{rgb}{0,0,0};
\colorlet{c}{natcomp!70};
\draw [c] (7.38705,0.701315) -- (7.38705,0.726433);
\draw [c] (7.38705,0.726433) -- (7.38705,0.751551);
\draw [c] (7.37891,0.726433) -- (7.38705,0.726433);
\draw [c] (7.38705,0.726433) -- (7.39518,0.726433);
\definecolor{c}{rgb}{0,0,0};
\colorlet{c}{natcomp!70};
\draw [c] (7.40332,0.719987) -- (7.40332,0.743126);
\draw [c] (7.40332,0.743126) -- (7.40332,0.766266);
\draw [c] (7.39518,0.743126) -- (7.40332,0.743126);
\draw [c] (7.40332,0.743126) -- (7.41145,0.743126);
\definecolor{c}{rgb}{0,0,0};
\colorlet{c}{natcomp!70};
\draw [c] (7.41959,0.691933) -- (7.41959,0.704153);
\draw [c] (7.41959,0.704153) -- (7.41959,0.716373);
\draw [c] (7.41145,0.704153) -- (7.41959,0.704153);
\draw [c] (7.41959,0.704153) -- (7.42773,0.704153);
\definecolor{c}{rgb}{0,0,0};
\colorlet{c}{natcomp!70};
\draw [c] (7.45214,0.686902) -- (7.45214,0.695866);
\draw [c] (7.45214,0.695866) -- (7.45214,0.704829);
\draw [c] (7.444,0.695866) -- (7.45214,0.695866);
\draw [c] (7.45214,0.695866) -- (7.46027,0.695866);
\definecolor{c}{rgb}{0,0,0};
\colorlet{c}{natcomp!70};
\draw [c] (7.46841,0.686921) -- (7.46841,0.695885);
\draw [c] (7.46841,0.695885) -- (7.46841,0.704848);
\draw [c] (7.46027,0.695885) -- (7.46841,0.695885);
\draw [c] (7.46841,0.695885) -- (7.47655,0.695885);
\definecolor{c}{rgb}{0,0,0};
\colorlet{c}{natcomp!70};
\draw [c] (7.48468,0.686897) -- (7.48468,0.686905);
\draw [c] (7.48468,0.686905) -- (7.48468,0.686914);
\draw [c] (7.47655,0.686905) -- (7.48468,0.686905);
\draw [c] (7.48468,0.686905) -- (7.49282,0.686905);
\definecolor{c}{rgb}{0,0,0};
\colorlet{c}{natcomp!70};
\draw [c] (7.50095,0.686894) -- (7.50095,0.686899);
\draw [c] (7.50095,0.686899) -- (7.50095,0.686904);
\draw [c] (7.49282,0.686899) -- (7.50095,0.686899);
\draw [c] (7.50095,0.686899) -- (7.50909,0.686899);
\definecolor{c}{rgb}{0,0,0};
\colorlet{c}{natcomp!70};
\draw [c] (7.51723,0.702942) -- (7.51723,0.719147);
\draw [c] (7.51723,0.719147) -- (7.51723,0.735351);
\draw [c] (7.50909,0.719147) -- (7.51723,0.719147);
\draw [c] (7.51723,0.719147) -- (7.52536,0.719147);
\definecolor{c}{rgb}{0,0,0};
\colorlet{c}{natcomp!70};
\draw [c] (7.5335,0.691582) -- (7.5335,0.703001);
\draw [c] (7.5335,0.703001) -- (7.5335,0.71442);
\draw [c] (7.52536,0.703001) -- (7.5335,0.703001);
\draw [c] (7.5335,0.703001) -- (7.54164,0.703001);
\definecolor{c}{rgb}{0,0,0};
\colorlet{c}{natcomp!70};
\draw [c] (7.54977,0.699339) -- (7.54977,0.718609);
\draw [c] (7.54977,0.718609) -- (7.54977,0.737879);
\draw [c] (7.54164,0.718609) -- (7.54977,0.718609);
\draw [c] (7.54977,0.718609) -- (7.55791,0.718609);
\definecolor{c}{rgb}{0,0,0};
\colorlet{c}{natcomp!70};
\draw [c] (7.56605,0.705399) -- (7.56605,0.728064);
\draw [c] (7.56605,0.728064) -- (7.56605,0.750729);
\draw [c] (7.55791,0.728064) -- (7.56605,0.728064);
\draw [c] (7.56605,0.728064) -- (7.57418,0.728064);
\definecolor{c}{rgb}{0,0,0};
\colorlet{c}{natcomp!70};
\draw [c] (7.58232,0.699404) -- (7.58232,0.723597);
\draw [c] (7.58232,0.723597) -- (7.58232,0.74779);
\draw [c] (7.57418,0.723597) -- (7.58232,0.723597);
\draw [c] (7.58232,0.723597) -- (7.59045,0.723597);
\definecolor{c}{rgb}{0,0,0};
\colorlet{c}{natcomp!70};
\draw [c] (7.59859,0.686912) -- (7.59859,0.694113);
\draw [c] (7.59859,0.694113) -- (7.59859,0.701313);
\draw [c] (7.59045,0.694113) -- (7.59859,0.694113);
\draw [c] (7.59859,0.694113) -- (7.60673,0.694113);
\definecolor{c}{rgb}{0,0,0};
\colorlet{c}{natcomp!70};
\draw [c] (7.61486,0.686896) -- (7.61486,0.686902);
\draw [c] (7.61486,0.686902) -- (7.61486,0.686909);
\draw [c] (7.60673,0.686902) -- (7.61486,0.686902);
\draw [c] (7.61486,0.686902) -- (7.623,0.686902);
\definecolor{c}{rgb}{0,0,0};
\colorlet{c}{natcomp!70};
\draw [c] (7.63114,0.721485) -- (7.63114,0.74621);
\draw [c] (7.63114,0.74621) -- (7.63114,0.770934);
\draw [c] (7.623,0.74621) -- (7.63114,0.74621);
\draw [c] (7.63114,0.74621) -- (7.63927,0.74621);
\definecolor{c}{rgb}{0,0,0};
\colorlet{c}{natcomp!70};
\draw [c] (7.64741,0.686934) -- (7.64741,0.695073);
\draw [c] (7.64741,0.695073) -- (7.64741,0.703212);
\draw [c] (7.63927,0.695073) -- (7.64741,0.695073);
\draw [c] (7.64741,0.695073) -- (7.65555,0.695073);
\definecolor{c}{rgb}{0,0,0};
\colorlet{c}{natcomp!70};
\draw [c] (7.66368,0.69193) -- (7.66368,0.704888);
\draw [c] (7.66368,0.704888) -- (7.66368,0.717847);
\draw [c] (7.65555,0.704888) -- (7.66368,0.704888);
\draw [c] (7.66368,0.704888) -- (7.67182,0.704888);
\definecolor{c}{rgb}{0,0,0};
\colorlet{c}{natcomp!70};
\draw [c] (7.67995,0.693227) -- (7.67995,0.708478);
\draw [c] (7.67995,0.708478) -- (7.67995,0.723729);
\draw [c] (7.67182,0.708478) -- (7.67995,0.708478);
\draw [c] (7.67995,0.708478) -- (7.68809,0.708478);
\definecolor{c}{rgb}{0,0,0};
\colorlet{c}{natcomp!70};
\draw [c] (7.69623,0.705019) -- (7.69623,0.723124);
\draw [c] (7.69623,0.723124) -- (7.69623,0.74123);
\draw [c] (7.68809,0.723124) -- (7.69623,0.723124);
\draw [c] (7.69623,0.723124) -- (7.70436,0.723124);
\definecolor{c}{rgb}{0,0,0};
\colorlet{c}{natcomp!70};
\draw [c] (7.7125,0.686901) -- (7.7125,0.686914);
\draw [c] (7.7125,0.686914) -- (7.7125,0.686926);
\draw [c] (7.70436,0.686914) -- (7.7125,0.686914);
\draw [c] (7.7125,0.686914) -- (7.72064,0.686914);
\definecolor{c}{rgb}{0,0,0};
\colorlet{c}{natcomp!70};
\draw [c] (7.72877,0.686911) -- (7.72877,0.697705);
\draw [c] (7.72877,0.697705) -- (7.72877,0.7085);
\draw [c] (7.72064,0.697705) -- (7.72877,0.697705);
\draw [c] (7.72877,0.697705) -- (7.73691,0.697705);
\definecolor{c}{rgb}{0,0,0};
\colorlet{c}{natcomp!70};
\draw [c] (7.74505,0.697486) -- (7.74505,0.712097);
\draw [c] (7.74505,0.712097) -- (7.74505,0.726708);
\draw [c] (7.73691,0.712097) -- (7.74505,0.712097);
\draw [c] (7.74505,0.712097) -- (7.75318,0.712097);
\definecolor{c}{rgb}{0,0,0};
\colorlet{c}{natcomp!70};
\draw [c] (7.76132,0.692362) -- (7.76132,0.708107);
\draw [c] (7.76132,0.708107) -- (7.76132,0.723852);
\draw [c] (7.75318,0.708107) -- (7.76132,0.708107);
\draw [c] (7.76132,0.708107) -- (7.76945,0.708107);
\definecolor{c}{rgb}{0,0,0};
\colorlet{c}{natcomp!70};
\draw [c] (7.77759,0.699709) -- (7.77759,0.717404);
\draw [c] (7.77759,0.717404) -- (7.77759,0.735098);
\draw [c] (7.76945,0.717404) -- (7.77759,0.717404);
\draw [c] (7.77759,0.717404) -- (7.78573,0.717404);
\definecolor{c}{rgb}{0,0,0};
\colorlet{c}{natcomp!70};
\draw [c] (7.79386,0.733221) -- (7.79386,0.759074);
\draw [c] (7.79386,0.759074) -- (7.79386,0.784928);
\draw [c] (7.78573,0.759074) -- (7.79386,0.759074);
\draw [c] (7.79386,0.759074) -- (7.802,0.759074);
\definecolor{c}{rgb}{0,0,0};
\colorlet{c}{natcomp!70};
\draw [c] (7.81014,0.697974) -- (7.81014,0.713528);
\draw [c] (7.81014,0.713528) -- (7.81014,0.729082);
\draw [c] (7.802,0.713528) -- (7.81014,0.713528);
\draw [c] (7.81014,0.713528) -- (7.81827,0.713528);
\definecolor{c}{rgb}{0,0,0};
\colorlet{c}{natcomp!70};
\draw [c] (7.82641,0.692082) -- (7.82641,0.705044);
\draw [c] (7.82641,0.705044) -- (7.82641,0.718006);
\draw [c] (7.81827,0.705044) -- (7.82641,0.705044);
\draw [c] (7.82641,0.705044) -- (7.83455,0.705044);
\definecolor{c}{rgb}{0,0,0};
\colorlet{c}{natcomp!70};
\draw [c] (7.84268,0.69139) -- (7.84268,0.702257);
\draw [c] (7.84268,0.702257) -- (7.84268,0.713124);
\draw [c] (7.83455,0.702257) -- (7.84268,0.702257);
\draw [c] (7.84268,0.702257) -- (7.85082,0.702257);
\definecolor{c}{rgb}{0,0,0};
\colorlet{c}{natcomp!70};
\draw [c] (7.85895,0.68692) -- (7.85895,0.696815);
\draw [c] (7.85895,0.696815) -- (7.85895,0.706709);
\draw [c] (7.85082,0.696815) -- (7.85895,0.696815);
\draw [c] (7.85895,0.696815) -- (7.86709,0.696815);
\definecolor{c}{rgb}{0,0,0};
\colorlet{c}{natcomp!70};
\draw [c] (7.87523,0.699254) -- (7.87523,0.718923);
\draw [c] (7.87523,0.718923) -- (7.87523,0.738592);
\draw [c] (7.86709,0.718923) -- (7.87523,0.718923);
\draw [c] (7.87523,0.718923) -- (7.88336,0.718923);
\definecolor{c}{rgb}{0,0,0};
\colorlet{c}{natcomp!70};
\draw [c] (7.8915,0.711611) -- (7.8915,0.732071);
\draw [c] (7.8915,0.732071) -- (7.8915,0.752532);
\draw [c] (7.88336,0.732071) -- (7.8915,0.732071);
\draw [c] (7.8915,0.732071) -- (7.89964,0.732071);
\definecolor{c}{rgb}{0,0,0};
\colorlet{c}{natcomp!70};
\draw [c] (7.90777,0.715085) -- (7.90777,0.738825);
\draw [c] (7.90777,0.738825) -- (7.90777,0.762565);
\draw [c] (7.89964,0.738825) -- (7.90777,0.738825);
\draw [c] (7.90777,0.738825) -- (7.91591,0.738825);
\definecolor{c}{rgb}{0,0,0};
\colorlet{c}{natcomp!70};
\draw [c] (7.92405,0.686922) -- (7.92405,0.696038);
\draw [c] (7.92405,0.696038) -- (7.92405,0.705153);
\draw [c] (7.91591,0.696038) -- (7.92405,0.696038);
\draw [c] (7.92405,0.696038) -- (7.93218,0.696038);
\definecolor{c}{rgb}{0,0,0};
\colorlet{c}{natcomp!70};
\draw [c] (7.94032,0.686907) -- (7.94032,0.696802);
\draw [c] (7.94032,0.696802) -- (7.94032,0.706696);
\draw [c] (7.93218,0.696802) -- (7.94032,0.696802);
\draw [c] (7.94032,0.696802) -- (7.94845,0.696802);
\definecolor{c}{rgb}{0,0,0};
\colorlet{c}{natcomp!70};
\draw [c] (7.95659,0.691941) -- (7.95659,0.704161);
\draw [c] (7.95659,0.704161) -- (7.95659,0.716381);
\draw [c] (7.94845,0.704161) -- (7.95659,0.704161);
\draw [c] (7.95659,0.704161) -- (7.96473,0.704161);
\definecolor{c}{rgb}{0,0,0};
\colorlet{c}{natcomp!70};
\draw [c] (7.97286,0.686912) -- (7.97286,0.69505);
\draw [c] (7.97286,0.69505) -- (7.97286,0.703189);
\draw [c] (7.96473,0.69505) -- (7.97286,0.69505);
\draw [c] (7.97286,0.69505) -- (7.981,0.69505);
\definecolor{c}{rgb}{0,0,0};
\colorlet{c}{natcomp!70};
\draw [c] (7.98914,0.693662) -- (7.98914,0.710807);
\draw [c] (7.98914,0.710807) -- (7.98914,0.727952);
\draw [c] (7.981,0.710807) -- (7.98914,0.710807);
\draw [c] (7.98914,0.710807) -- (7.99727,0.710807);
\definecolor{c}{rgb}{0,0,0};
\colorlet{c}{natcomp!70};
\draw [c] (8.00541,0.686922) -- (8.00541,0.695785);
\draw [c] (8.00541,0.695785) -- (8.00541,0.704647);
\draw [c] (7.99727,0.695785) -- (8.00541,0.695785);
\draw [c] (8.00541,0.695785) -- (8.01355,0.695785);
\definecolor{c}{rgb}{0,0,0};
\colorlet{c}{natcomp!70};
\draw [c] (8.02168,0.691552) -- (8.02168,0.702771);
\draw [c] (8.02168,0.702771) -- (8.02168,0.71399);
\draw [c] (8.01355,0.702771) -- (8.02168,0.702771);
\draw [c] (8.02168,0.702771) -- (8.02982,0.702771);
\definecolor{c}{rgb}{0,0,0};
\colorlet{c}{natcomp!70};
\draw [c] (8.03795,0.686908) -- (8.03795,0.70091);
\draw [c] (8.03795,0.70091) -- (8.03795,0.714913);
\draw [c] (8.02982,0.70091) -- (8.03795,0.70091);
\draw [c] (8.03795,0.70091) -- (8.04609,0.70091);
\definecolor{c}{rgb}{0,0,0};
\colorlet{c}{natcomp!70};
\draw [c] (8.05423,0.686902) -- (8.05423,0.695765);
\draw [c] (8.05423,0.695765) -- (8.05423,0.704627);
\draw [c] (8.04609,0.695765) -- (8.05423,0.695765);
\draw [c] (8.05423,0.695765) -- (8.06236,0.695765);
\definecolor{c}{rgb}{0,0,0};
\colorlet{c}{natcomp!70};
\draw [c] (8.0705,0.686913) -- (8.0705,0.694635);
\draw [c] (8.0705,0.694635) -- (8.0705,0.702356);
\draw [c] (8.06236,0.694635) -- (8.0705,0.694635);
\draw [c] (8.0705,0.694635) -- (8.07864,0.694635);
\definecolor{c}{rgb}{0,0,0};
\colorlet{c}{natcomp!70};
\draw [c] (8.08677,0.691886) -- (8.08677,0.703918);
\draw [c] (8.08677,0.703918) -- (8.08677,0.715951);
\draw [c] (8.07864,0.703918) -- (8.08677,0.703918);
\draw [c] (8.08677,0.703918) -- (8.09491,0.703918);
\definecolor{c}{rgb}{0,0,0};
\colorlet{c}{natcomp!70};
\draw [c] (8.10305,0.699036) -- (8.10305,0.715859);
\draw [c] (8.10305,0.715859) -- (8.10305,0.732682);
\draw [c] (8.09491,0.715859) -- (8.10305,0.715859);
\draw [c] (8.10305,0.715859) -- (8.11118,0.715859);
\definecolor{c}{rgb}{0,0,0};
\colorlet{c}{natcomp!70};
\draw [c] (8.11932,0.70093) -- (8.11932,0.723385);
\draw [c] (8.11932,0.723385) -- (8.11932,0.74584);
\draw [c] (8.11118,0.723385) -- (8.11932,0.723385);
\draw [c] (8.11932,0.723385) -- (8.12745,0.723385);
\definecolor{c}{rgb}{0,0,0};
\colorlet{c}{natcomp!70};
\draw [c] (8.13559,0.691991) -- (8.13559,0.704542);
\draw [c] (8.13559,0.704542) -- (8.13559,0.717092);
\draw [c] (8.12745,0.704542) -- (8.13559,0.704542);
\draw [c] (8.13559,0.704542) -- (8.14373,0.704542);
\definecolor{c}{rgb}{0,0,0};
\colorlet{c}{natcomp!70};
\draw [c] (8.15186,0.704381) -- (8.15186,0.722348);
\draw [c] (8.15186,0.722348) -- (8.15186,0.740315);
\draw [c] (8.14373,0.722348) -- (8.15186,0.722348);
\draw [c] (8.15186,0.722348) -- (8.16,0.722348);
\definecolor{c}{rgb}{0,0,0};
\colorlet{c}{natcomp!70};
\draw [c] (8.16814,0.692514) -- (8.16814,0.706186);
\draw [c] (8.16814,0.706186) -- (8.16814,0.719858);
\draw [c] (8.16,0.706186) -- (8.16814,0.706186);
\draw [c] (8.16814,0.706186) -- (8.17627,0.706186);
\definecolor{c}{rgb}{0,0,0};
\colorlet{c}{natcomp!70};
\draw [c] (8.18441,0.686908) -- (8.18441,0.697319);
\draw [c] (8.18441,0.697319) -- (8.18441,0.70773);
\draw [c] (8.17627,0.697319) -- (8.18441,0.697319);
\draw [c] (8.18441,0.697319) -- (8.19255,0.697319);
\definecolor{c}{rgb}{0,0,0};
\colorlet{c}{natcomp!70};
\draw [c] (8.20068,0.686908) -- (8.20068,0.708675);
\draw [c] (8.20068,0.708675) -- (8.20068,0.730443);
\draw [c] (8.19255,0.708675) -- (8.20068,0.708675);
\draw [c] (8.20068,0.708675) -- (8.20882,0.708675);
\definecolor{c}{rgb}{0,0,0};
\colorlet{c}{natcomp!70};
\draw [c] (8.21695,0.691366) -- (8.21695,0.702233);
\draw [c] (8.21695,0.702233) -- (8.21695,0.7131);
\draw [c] (8.20882,0.702233) -- (8.21695,0.702233);
\draw [c] (8.21695,0.702233) -- (8.22509,0.702233);
\definecolor{c}{rgb}{0,0,0};
\colorlet{c}{natcomp!70};
\draw [c] (8.23323,0.700598) -- (8.23323,0.719335);
\draw [c] (8.23323,0.719335) -- (8.23323,0.738073);
\draw [c] (8.22509,0.719335) -- (8.23323,0.719335);
\draw [c] (8.23323,0.719335) -- (8.24136,0.719335);
\definecolor{c}{rgb}{0,0,0};
\colorlet{c}{natcomp!70};
\draw [c] (8.2495,0.698239) -- (8.2495,0.713998);
\draw [c] (8.2495,0.713998) -- (8.2495,0.729758);
\draw [c] (8.24136,0.713998) -- (8.2495,0.713998);
\draw [c] (8.2495,0.713998) -- (8.25764,0.713998);
\definecolor{c}{rgb}{0,0,0};
\colorlet{c}{natcomp!70};
\draw [c] (8.26577,0.704207) -- (8.26577,0.72189);
\draw [c] (8.26577,0.72189) -- (8.26577,0.739573);
\draw [c] (8.25764,0.72189) -- (8.26577,0.72189);
\draw [c] (8.26577,0.72189) -- (8.27391,0.72189);
\definecolor{c}{rgb}{0,0,0};
\colorlet{c}{natcomp!70};
\draw [c] (8.28205,0.699406) -- (8.28205,0.719236);
\draw [c] (8.28205,0.719236) -- (8.28205,0.739065);
\draw [c] (8.27391,0.719236) -- (8.28205,0.719236);
\draw [c] (8.28205,0.719236) -- (8.29018,0.719236);
\definecolor{c}{rgb}{0,0,0};
\colorlet{c}{natcomp!70};
\draw [c] (8.29832,0.686897) -- (8.29832,0.686904);
\draw [c] (8.29832,0.686904) -- (8.29832,0.686911);
\draw [c] (8.29018,0.686904) -- (8.29832,0.686904);
\draw [c] (8.29832,0.686904) -- (8.30645,0.686904);
\definecolor{c}{rgb}{0,0,0};
\colorlet{c}{natcomp!70};
\draw [c] (8.31459,0.692156) -- (8.31459,0.704967);
\draw [c] (8.31459,0.704967) -- (8.31459,0.717779);
\draw [c] (8.30645,0.704967) -- (8.31459,0.704967);
\draw [c] (8.31459,0.704967) -- (8.32273,0.704967);
\definecolor{c}{rgb}{0,0,0};
\colorlet{c}{natcomp!70};
\draw [c] (8.33086,0.686894) -- (8.33086,0.686898);
\draw [c] (8.33086,0.686898) -- (8.33086,0.686902);
\draw [c] (8.32273,0.686898) -- (8.33086,0.686898);
\draw [c] (8.33086,0.686898) -- (8.339,0.686898);
\definecolor{c}{rgb}{0,0,0};
\colorlet{c}{natcomp!70};
\draw [c] (8.34714,0.686913) -- (8.34714,0.696808);
\draw [c] (8.34714,0.696808) -- (8.34714,0.706702);
\draw [c] (8.339,0.696808) -- (8.34714,0.696808);
\draw [c] (8.34714,0.696808) -- (8.35527,0.696808);
\definecolor{c}{rgb}{0,0,0};
\colorlet{c}{natcomp!70};
\draw [c] (8.36341,0.6869) -- (8.36341,0.686908);
\draw [c] (8.36341,0.686908) -- (8.36341,0.686916);
\draw [c] (8.35527,0.686908) -- (8.36341,0.686908);
\draw [c] (8.36341,0.686908) -- (8.37155,0.686908);
\definecolor{c}{rgb}{0,0,0};
\colorlet{c}{natcomp!70};
\draw [c] (8.37968,0.709537) -- (8.37968,0.733883);
\draw [c] (8.37968,0.733883) -- (8.37968,0.758228);
\draw [c] (8.37155,0.733883) -- (8.37968,0.733883);
\draw [c] (8.37968,0.733883) -- (8.38782,0.733883);
\definecolor{c}{rgb}{0,0,0};
\colorlet{c}{natcomp!70};
\draw [c] (8.39595,0.686897) -- (8.39595,0.696792);
\draw [c] (8.39595,0.696792) -- (8.39595,0.706686);
\draw [c] (8.38782,0.696792) -- (8.39595,0.696792);
\draw [c] (8.39595,0.696792) -- (8.40409,0.696792);
\definecolor{c}{rgb}{0,0,0};
\colorlet{c}{natcomp!70};
\draw [c] (8.41223,0.686907) -- (8.41223,0.696023);
\draw [c] (8.41223,0.696023) -- (8.41223,0.705138);
\draw [c] (8.40409,0.696023) -- (8.41223,0.696023);
\draw [c] (8.41223,0.696023) -- (8.42036,0.696023);
\definecolor{c}{rgb}{0,0,0};
\colorlet{c}{natcomp!70};
\draw [c] (8.4285,0.691882) -- (8.4285,0.703914);
\draw [c] (8.4285,0.703914) -- (8.4285,0.715947);
\draw [c] (8.42036,0.703914) -- (8.4285,0.703914);
\draw [c] (8.4285,0.703914) -- (8.43664,0.703914);
\definecolor{c}{rgb}{0,0,0};
\colorlet{c}{natcomp!70};
\draw [c] (8.44477,0.686908) -- (8.44477,0.69814);
\draw [c] (8.44477,0.69814) -- (8.44477,0.709372);
\draw [c] (8.43664,0.69814) -- (8.44477,0.69814);
\draw [c] (8.44477,0.69814) -- (8.45291,0.69814);
\definecolor{c}{rgb}{0,0,0};
\colorlet{c}{natcomp!70};
\draw [c] (8.46105,0.686896) -- (8.46105,0.686903);
\draw [c] (8.46105,0.686903) -- (8.46105,0.686909);
\draw [c] (8.45291,0.686903) -- (8.46105,0.686903);
\draw [c] (8.46105,0.686903) -- (8.46918,0.686903);
\definecolor{c}{rgb}{0,0,0};
\colorlet{c}{natcomp!70};
\draw [c] (8.47732,0.686902) -- (8.47732,0.686911);
\draw [c] (8.47732,0.686911) -- (8.47732,0.686919);
\draw [c] (8.46918,0.686911) -- (8.47732,0.686911);
\draw [c] (8.47732,0.686911) -- (8.48545,0.686911);
\definecolor{c}{rgb}{0,0,0};
\colorlet{c}{natcomp!70};
\draw [c] (8.49359,0.686915) -- (8.49359,0.712075);
\draw [c] (8.49359,0.712075) -- (8.49359,0.737235);
\draw [c] (8.48545,0.712075) -- (8.49359,0.712075);
\draw [c] (8.49359,0.712075) -- (8.50173,0.712075);
\definecolor{c}{rgb}{0,0,0};
\colorlet{c}{natcomp!70};
\draw [c] (8.50986,0.686899) -- (8.50986,0.686907);
\draw [c] (8.50986,0.686907) -- (8.50986,0.686915);
\draw [c] (8.50173,0.686907) -- (8.50986,0.686907);
\draw [c] (8.50986,0.686907) -- (8.518,0.686907);
\definecolor{c}{rgb}{0,0,0};
\colorlet{c}{natcomp!70};
\draw [c] (8.52614,0.686912) -- (8.52614,0.694113);
\draw [c] (8.52614,0.694113) -- (8.52614,0.701314);
\draw [c] (8.518,0.694113) -- (8.52614,0.694113);
\draw [c] (8.52614,0.694113) -- (8.53427,0.694113);
\definecolor{c}{rgb}{0,0,0};
\colorlet{c}{natcomp!70};
\draw [c] (8.54241,0.70362) -- (8.54241,0.720773);
\draw [c] (8.54241,0.720773) -- (8.54241,0.737927);
\draw [c] (8.53427,0.720773) -- (8.54241,0.720773);
\draw [c] (8.54241,0.720773) -- (8.55055,0.720773);
\definecolor{c}{rgb}{0,0,0};
\colorlet{c}{natcomp!70};
\draw [c] (8.55868,0.686902) -- (8.55868,0.697697);
\draw [c] (8.55868,0.697697) -- (8.55868,0.708492);
\draw [c] (8.55055,0.697697) -- (8.55868,0.697697);
\draw [c] (8.55868,0.697697) -- (8.56682,0.697697);
\definecolor{c}{rgb}{0,0,0};
\colorlet{c}{natcomp!70};
\draw [c] (8.57495,0.692719) -- (8.57495,0.70709);
\draw [c] (8.57495,0.70709) -- (8.57495,0.721461);
\draw [c] (8.56682,0.70709) -- (8.57495,0.70709);
\draw [c] (8.57495,0.70709) -- (8.58309,0.70709);
\definecolor{c}{rgb}{0,0,0};
\colorlet{c}{natcomp!70};
\draw [c] (8.59123,0.686902) -- (8.59123,0.686909);
\draw [c] (8.59123,0.686909) -- (8.59123,0.686917);
\draw [c] (8.58309,0.686909) -- (8.59123,0.686909);
\draw [c] (8.59123,0.686909) -- (8.59936,0.686909);
\definecolor{c}{rgb}{0,0,0};
\colorlet{c}{natcomp!70};
\draw [c] (8.6075,0.686894) -- (8.6075,0.686897);
\draw [c] (8.6075,0.686897) -- (8.6075,0.686901);
\draw [c] (8.59936,0.686897) -- (8.6075,0.686897);
\draw [c] (8.6075,0.686897) -- (8.61564,0.686897);
\definecolor{c}{rgb}{0,0,0};
\colorlet{c}{natcomp!70};
\draw [c] (8.62377,0.686903) -- (8.62377,0.686912);
\draw [c] (8.62377,0.686912) -- (8.62377,0.686922);
\draw [c] (8.61564,0.686912) -- (8.62377,0.686912);
\draw [c] (8.62377,0.686912) -- (8.63191,0.686912);
\definecolor{c}{rgb}{0,0,0};
\colorlet{c}{natcomp!70};
\draw [c] (8.64005,0.706296) -- (8.64005,0.725795);
\draw [c] (8.64005,0.725795) -- (8.64005,0.745294);
\draw [c] (8.63191,0.725795) -- (8.64005,0.725795);
\draw [c] (8.64005,0.725795) -- (8.64818,0.725795);
\definecolor{c}{rgb}{0,0,0};
\colorlet{c}{natcomp!70};
\draw [c] (8.65632,0.686902) -- (8.65632,0.697676);
\draw [c] (8.65632,0.697676) -- (8.65632,0.708449);
\draw [c] (8.64818,0.697676) -- (8.65632,0.697676);
\draw [c] (8.65632,0.697676) -- (8.66445,0.697676);
\definecolor{c}{rgb}{0,0,0};
\colorlet{c}{natcomp!70};
\draw [c] (8.67259,0.697669) -- (8.67259,0.712756);
\draw [c] (8.67259,0.712756) -- (8.67259,0.727844);
\draw [c] (8.66445,0.712756) -- (8.67259,0.712756);
\draw [c] (8.67259,0.712756) -- (8.68073,0.712756);
\definecolor{c}{rgb}{0,0,0};
\colorlet{c}{natcomp!70};
\draw [c] (8.68886,0.686917) -- (8.68886,0.695056);
\draw [c] (8.68886,0.695056) -- (8.68886,0.703195);
\draw [c] (8.68073,0.695056) -- (8.68886,0.695056);
\draw [c] (8.68886,0.695056) -- (8.697,0.695056);
\definecolor{c}{rgb}{0,0,0};
\colorlet{c}{natcomp!70};
\draw [c] (8.70514,0.686894) -- (8.70514,0.686898);
\draw [c] (8.70514,0.686898) -- (8.70514,0.686902);
\draw [c] (8.697,0.686898) -- (8.70514,0.686898);
\draw [c] (8.70514,0.686898) -- (8.71327,0.686898);
\definecolor{c}{rgb}{0,0,0};
\colorlet{c}{natcomp!70};
\draw [c] (8.72141,0.686896) -- (8.72141,0.686903);
\draw [c] (8.72141,0.686903) -- (8.72141,0.686909);
\draw [c] (8.71327,0.686903) -- (8.72141,0.686903);
\draw [c] (8.72141,0.686903) -- (8.72955,0.686903);
\definecolor{c}{rgb}{0,0,0};
\colorlet{c}{natcomp!70};
\draw [c] (8.73768,0.6869) -- (8.73768,0.686909);
\draw [c] (8.73768,0.686909) -- (8.73768,0.686917);
\draw [c] (8.72955,0.686909) -- (8.73768,0.686909);
\draw [c] (8.73768,0.686909) -- (8.74582,0.686909);
\definecolor{c}{rgb}{0,0,0};
\colorlet{c}{natcomp!70};
\draw [c] (8.75395,0.686898) -- (8.75395,0.697671);
\draw [c] (8.75395,0.697671) -- (8.75395,0.708445);
\draw [c] (8.74582,0.697671) -- (8.75395,0.697671);
\draw [c] (8.75395,0.697671) -- (8.76209,0.697671);
\definecolor{c}{rgb}{0,0,0};
\colorlet{c}{natcomp!70};
\draw [c] (8.77023,0.686896) -- (8.77023,0.686902);
\draw [c] (8.77023,0.686902) -- (8.77023,0.686908);
\draw [c] (8.76209,0.686902) -- (8.77023,0.686902);
\draw [c] (8.77023,0.686902) -- (8.77836,0.686902);
\definecolor{c}{rgb}{0,0,0};
\colorlet{c}{natcomp!70};
\draw [c] (8.80277,0.686894) -- (8.80277,0.697305);
\draw [c] (8.80277,0.697305) -- (8.80277,0.707716);
\draw [c] (8.79464,0.697305) -- (8.80277,0.697305);
\draw [c] (8.80277,0.697305) -- (8.81091,0.697305);
\definecolor{c}{rgb}{0,0,0};
\colorlet{c}{natcomp!70};
\draw [c] (8.81905,0.686894) -- (8.81905,0.686899);
\draw [c] (8.81905,0.686899) -- (8.81905,0.686904);
\draw [c] (8.81091,0.686899) -- (8.81905,0.686899);
\draw [c] (8.81905,0.686899) -- (8.82718,0.686899);
\definecolor{c}{rgb}{0,0,0};
\colorlet{c}{natcomp!70};
\draw [c] (8.83532,0.686894) -- (8.83532,0.686899);
\draw [c] (8.83532,0.686899) -- (8.83532,0.686904);
\draw [c] (8.82718,0.686899) -- (8.83532,0.686899);
\draw [c] (8.83532,0.686899) -- (8.84345,0.686899);
\definecolor{c}{rgb}{0,0,0};
\colorlet{c}{natcomp!70};
\draw [c] (8.85159,0.686907) -- (8.85159,0.696022);
\draw [c] (8.85159,0.696022) -- (8.85159,0.705137);
\draw [c] (8.84345,0.696022) -- (8.85159,0.696022);
\draw [c] (8.85159,0.696022) -- (8.85973,0.696022);
\definecolor{c}{rgb}{0,0,0};
\colorlet{c}{natcomp!70};
\draw [c] (8.86786,0.686894) -- (8.86786,0.686897);
\draw [c] (8.86786,0.686897) -- (8.86786,0.686901);
\draw [c] (8.85973,0.686897) -- (8.86786,0.686897);
\draw [c] (8.86786,0.686897) -- (8.876,0.686897);
\definecolor{c}{rgb}{0,0,0};
\colorlet{c}{natcomp!70};
\draw [c] (8.88414,0.700826) -- (8.88414,0.720637);
\draw [c] (8.88414,0.720637) -- (8.88414,0.740448);
\draw [c] (8.876,0.720637) -- (8.88414,0.720637);
\draw [c] (8.88414,0.720637) -- (8.89227,0.720637);
\definecolor{c}{rgb}{0,0,0};
\colorlet{c}{natcomp!70};
\draw [c] (8.90041,0.686894) -- (8.90041,0.686897);
\draw [c] (8.90041,0.686897) -- (8.90041,0.686901);
\draw [c] (8.89227,0.686897) -- (8.90041,0.686897);
\draw [c] (8.90041,0.686897) -- (8.90855,0.686897);
\definecolor{c}{rgb}{0,0,0};
\colorlet{c}{natcomp!70};
\draw [c] (8.91668,0.722516) -- (8.91668,0.755184);
\draw [c] (8.91668,0.755184) -- (8.91668,0.787852);
\draw [c] (8.90855,0.755184) -- (8.91668,0.755184);
\draw [c] (8.91668,0.755184) -- (8.92482,0.755184);
\definecolor{c}{rgb}{0,0,0};
\colorlet{c}{natcomp!70};
\draw [c] (8.93295,0.6869) -- (8.93295,0.694622);
\draw [c] (8.93295,0.694622) -- (8.93295,0.702344);
\draw [c] (8.92482,0.694622) -- (8.93295,0.694622);
\draw [c] (8.93295,0.694622) -- (8.94109,0.694622);
\definecolor{c}{rgb}{0,0,0};
\colorlet{c}{natcomp!70};
\draw [c] (8.94923,0.686894) -- (8.94923,0.686898);
\draw [c] (8.94923,0.686898) -- (8.94923,0.686902);
\draw [c] (8.94109,0.686898) -- (8.94923,0.686898);
\draw [c] (8.94923,0.686898) -- (8.95736,0.686898);
\definecolor{c}{rgb}{0,0,0};
\colorlet{c}{natcomp!70};
\draw [c] (8.9655,0.686894) -- (8.9655,0.686898);
\draw [c] (8.9655,0.686898) -- (8.9655,0.686902);
\draw [c] (8.95736,0.686898) -- (8.9655,0.686898);
\draw [c] (8.9655,0.686898) -- (8.97364,0.686898);
\definecolor{c}{rgb}{0,0,0};
\colorlet{c}{natcomp!70};
\draw [c] (8.98177,0.686896) -- (8.98177,0.686902);
\draw [c] (8.98177,0.686902) -- (8.98177,0.686907);
\draw [c] (8.97364,0.686902) -- (8.98177,0.686902);
\draw [c] (8.98177,0.686902) -- (8.98991,0.686902);
\definecolor{c}{rgb}{0,0,0};
\colorlet{c}{natcomp!70};
\draw [c] (8.99805,0.686904) -- (8.99805,0.686922);
\draw [c] (8.99805,0.686922) -- (8.99805,0.686939);
\draw [c] (8.98991,0.686922) -- (8.99805,0.686922);
\draw [c] (8.99805,0.686922) -- (9.00618,0.686922);
\definecolor{c}{rgb}{0,0,0};
\colorlet{c}{natcomp!70};
\draw [c] (9.01432,0.686897) -- (9.01432,0.686903);
\draw [c] (9.01432,0.686903) -- (9.01432,0.68691);
\draw [c] (9.00618,0.686903) -- (9.01432,0.686903);
\draw [c] (9.01432,0.686903) -- (9.02245,0.686903);
\definecolor{c}{rgb}{0,0,0};
\colorlet{c}{natcomp!70};
\draw [c] (9.04686,0.708993) -- (9.04686,0.731658);
\draw [c] (9.04686,0.731658) -- (9.04686,0.754323);
\draw [c] (9.03873,0.731658) -- (9.04686,0.731658);
\draw [c] (9.04686,0.731658) -- (9.055,0.731658);
\definecolor{c}{rgb}{0,0,0};
\colorlet{c}{natcomp!70};
\draw [c] (9.07941,0.692085) -- (9.07941,0.704618);
\draw [c] (9.07941,0.704618) -- (9.07941,0.717151);
\draw [c] (9.07127,0.704618) -- (9.07941,0.704618);
\draw [c] (9.07941,0.704618) -- (9.08755,0.704618);
\definecolor{c}{rgb}{0,0,0};
\colorlet{c}{natcomp!70};
\draw [c] (9.09568,0.686897) -- (9.09568,0.695861);
\draw [c] (9.09568,0.695861) -- (9.09568,0.704825);
\draw [c] (9.08755,0.695861) -- (9.09568,0.695861);
\draw [c] (9.09568,0.695861) -- (9.10382,0.695861);
\definecolor{c}{rgb}{0,0,0};
\colorlet{c}{natcomp!70};
\draw [c] (9.12823,0.686894) -- (9.12823,0.686897);
\draw [c] (9.12823,0.686897) -- (9.12823,0.686901);
\draw [c] (9.12009,0.686897) -- (9.12823,0.686897);
\draw [c] (9.12823,0.686897) -- (9.13636,0.686897);
\definecolor{c}{rgb}{0,0,0};
\colorlet{c}{natcomp!70};
\draw [c] (9.1445,0.686899) -- (9.1445,0.696014);
\draw [c] (9.1445,0.696014) -- (9.1445,0.70513);
\draw [c] (9.13636,0.696014) -- (9.1445,0.696014);
\draw [c] (9.1445,0.696014) -- (9.15264,0.696014);
\definecolor{c}{rgb}{0,0,0};
\colorlet{c}{natcomp!70};
\draw [c] (9.16077,0.686897) -- (9.16077,0.686903);
\draw [c] (9.16077,0.686903) -- (9.16077,0.68691);
\draw [c] (9.15264,0.686903) -- (9.16077,0.686903);
\draw [c] (9.16077,0.686903) -- (9.16891,0.686903);
\definecolor{c}{rgb}{0,0,0};
\colorlet{c}{natcomp!70};
\draw [c] (9.17705,0.686894) -- (9.17705,0.705403);
\draw [c] (9.17705,0.705403) -- (9.17705,0.723913);
\draw [c] (9.16891,0.705403) -- (9.17705,0.705403);
\draw [c] (9.17705,0.705403) -- (9.18518,0.705403);
\definecolor{c}{rgb}{0,0,0};
\colorlet{c}{natcomp!70};
\draw [c] (9.19332,0.686897) -- (9.19332,0.686906);
\draw [c] (9.19332,0.686906) -- (9.19332,0.686914);
\draw [c] (9.18518,0.686906) -- (9.19332,0.686906);
\draw [c] (9.19332,0.686906) -- (9.20145,0.686906);
\definecolor{c}{rgb}{0,0,0};
\colorlet{c}{natcomp!70};
\draw [c] (9.20959,0.686906) -- (9.20959,0.694628);
\draw [c] (9.20959,0.694628) -- (9.20959,0.70235);
\draw [c] (9.20145,0.694628) -- (9.20959,0.694628);
\draw [c] (9.20959,0.694628) -- (9.21773,0.694628);
\definecolor{c}{rgb}{0,0,0};
\colorlet{c}{natcomp!70};
\draw [c] (9.22586,0.686898) -- (9.22586,0.697671);
\draw [c] (9.22586,0.697671) -- (9.22586,0.708444);
\draw [c] (9.21773,0.697671) -- (9.22586,0.697671);
\draw [c] (9.22586,0.697671) -- (9.234,0.697671);
\definecolor{c}{rgb}{0,0,0};
\colorlet{c}{natcomp!70};
\draw [c] (9.24214,0.697919) -- (9.24214,0.713014);
\draw [c] (9.24214,0.713014) -- (9.24214,0.72811);
\draw [c] (9.234,0.713014) -- (9.24214,0.713014);
\draw [c] (9.24214,0.713014) -- (9.25027,0.713014);
\definecolor{c}{rgb}{0,0,0};
\colorlet{c}{natcomp!70};
\draw [c] (9.25841,0.686896) -- (9.25841,0.686902);
\draw [c] (9.25841,0.686902) -- (9.25841,0.686908);
\draw [c] (9.25027,0.686902) -- (9.25841,0.686902);
\draw [c] (9.25841,0.686902) -- (9.26655,0.686902);
\definecolor{c}{rgb}{0,0,0};
\colorlet{c}{natcomp!70};
\draw [c] (9.27468,0.686897) -- (9.27468,0.686903);
\draw [c] (9.27468,0.686903) -- (9.27468,0.68691);
\draw [c] (9.26655,0.686903) -- (9.27468,0.686903);
\draw [c] (9.27468,0.686903) -- (9.28282,0.686903);
\definecolor{c}{rgb}{0,0,0};
\colorlet{c}{natcomp!70};
\draw [c] (9.30723,0.691421) -- (9.30723,0.702341);
\draw [c] (9.30723,0.702341) -- (9.30723,0.713261);
\draw [c] (9.29909,0.702341) -- (9.30723,0.702341);
\draw [c] (9.30723,0.702341) -- (9.31536,0.702341);
\definecolor{c}{rgb}{0,0,0};
\colorlet{c}{natcomp!70};
\draw [c] (9.33977,0.686902) -- (9.33977,0.69504);
\draw [c] (9.33977,0.69504) -- (9.33977,0.703179);
\draw [c] (9.33164,0.69504) -- (9.33977,0.69504);
\draw [c] (9.33977,0.69504) -- (9.34791,0.69504);
\definecolor{c}{rgb}{0,0,0};
\colorlet{c}{natcomp!70};
\draw [c] (9.35605,0.693227) -- (9.35605,0.708478);
\draw [c] (9.35605,0.708478) -- (9.35605,0.723729);
\draw [c] (9.34791,0.708478) -- (9.35605,0.708478);
\draw [c] (9.35605,0.708478) -- (9.36418,0.708478);
\definecolor{c}{rgb}{0,0,0};
\colorlet{c}{natcomp!70};
\draw [c] (9.37232,0.686896) -- (9.37232,0.686902);
\draw [c] (9.37232,0.686902) -- (9.37232,0.686908);
\draw [c] (9.36418,0.686902) -- (9.37232,0.686902);
\draw [c] (9.37232,0.686902) -- (9.38046,0.686902);
\definecolor{c}{rgb}{0,0,0};
\colorlet{c}{natcomp!70};
\draw [c] (9.38859,0.686897) -- (9.38859,0.696013);
\draw [c] (9.38859,0.696013) -- (9.38859,0.705128);
\draw [c] (9.38046,0.696013) -- (9.38859,0.696013);
\draw [c] (9.38859,0.696013) -- (9.39673,0.696013);
\definecolor{c}{rgb}{0,0,0};
\colorlet{c}{natcomp!70};
\draw [c] (9.43741,0.686902) -- (9.43741,0.697312);
\draw [c] (9.43741,0.697312) -- (9.43741,0.707723);
\draw [c] (9.42927,0.697312) -- (9.43741,0.697312);
\draw [c] (9.43741,0.697312) -- (9.44555,0.697312);
\definecolor{c}{rgb}{0,0,0};
\colorlet{c}{natcomp!70};
\draw [c] (9.45368,0.686897) -- (9.45368,0.686903);
\draw [c] (9.45368,0.686903) -- (9.45368,0.68691);
\draw [c] (9.44555,0.686903) -- (9.45368,0.686903);
\draw [c] (9.45368,0.686903) -- (9.46182,0.686903);
\definecolor{c}{rgb}{0,0,0};
\colorlet{c}{natcomp!70};
\draw [c] (9.46995,0.692681) -- (9.46995,0.706988);
\draw [c] (9.46995,0.706988) -- (9.46995,0.721296);
\draw [c] (9.46182,0.706988) -- (9.46995,0.706988);
\draw [c] (9.46995,0.706988) -- (9.47809,0.706988);
\definecolor{c}{rgb}{0,0,0};
\colorlet{c}{natcomp!70};
\draw [c] (9.48623,0.686894) -- (9.48623,0.686903);
\draw [c] (9.48623,0.686903) -- (9.48623,0.686911);
\draw [c] (9.47809,0.686903) -- (9.48623,0.686903);
\draw [c] (9.48623,0.686903) -- (9.49436,0.686903);
\definecolor{c}{rgb}{0,0,0};
\colorlet{c}{natcomp!70};
\draw [c] (9.5025,0.686898) -- (9.5025,0.696793);
\draw [c] (9.5025,0.696793) -- (9.5025,0.706687);
\draw [c] (9.49436,0.696793) -- (9.5025,0.696793);
\draw [c] (9.5025,0.696793) -- (9.51064,0.696793);
\definecolor{c}{rgb}{0,0,0};
\colorlet{c}{natcomp!70};
\draw [c] (9.51877,0.686894) -- (9.51877,0.697689);
\draw [c] (9.51877,0.697689) -- (9.51877,0.708483);
\draw [c] (9.51064,0.697689) -- (9.51877,0.697689);
\draw [c] (9.51877,0.697689) -- (9.52691,0.697689);
\definecolor{c}{rgb}{0,0,0};
\colorlet{c}{natcomp!70};
\draw [c] (9.53505,0.686896) -- (9.53505,0.686901);
\draw [c] (9.53505,0.686901) -- (9.53505,0.686906);
\draw [c] (9.52691,0.686901) -- (9.53505,0.686901);
\draw [c] (9.53505,0.686901) -- (9.54318,0.686901);
\definecolor{c}{rgb}{0,0,0};
\colorlet{c}{natcomp!70};
\draw [c] (9.55132,0.686896) -- (9.55132,0.686902);
\draw [c] (9.55132,0.686902) -- (9.55132,0.686908);
\draw [c] (9.54318,0.686902) -- (9.55132,0.686902);
\draw [c] (9.55132,0.686902) -- (9.55945,0.686902);
\definecolor{c}{rgb}{0,0,0};
\colorlet{c}{natcomp!70};
\draw [c] (9.56759,0.686894) -- (9.56759,0.696788);
\draw [c] (9.56759,0.696788) -- (9.56759,0.706683);
\draw [c] (9.55945,0.696788) -- (9.56759,0.696788);
\draw [c] (9.56759,0.696788) -- (9.57573,0.696788);
\definecolor{c}{rgb}{0,0,0};
\colorlet{c}{natcomp!70};
\draw [c] (9.58386,0.686894) -- (9.58386,0.686898);
\draw [c] (9.58386,0.686898) -- (9.58386,0.686902);
\draw [c] (9.57573,0.686898) -- (9.58386,0.686898);
\draw [c] (9.58386,0.686898) -- (9.592,0.686898);
\definecolor{c}{rgb}{0,0,0};
\colorlet{c}{natcomp!70};
\draw [c] (9.60014,0.686899) -- (9.60014,0.695761);
\draw [c] (9.60014,0.695761) -- (9.60014,0.704623);
\draw [c] (9.592,0.695761) -- (9.60014,0.695761);
\draw [c] (9.60014,0.695761) -- (9.60827,0.695761);
\definecolor{c}{rgb}{0,0,0};
\colorlet{c}{natcomp!70};
\draw [c] (9.61641,0.691959) -- (9.61641,0.70451);
\draw [c] (9.61641,0.70451) -- (9.61641,0.717061);
\draw [c] (9.60827,0.70451) -- (9.61641,0.70451);
\draw [c] (9.61641,0.70451) -- (9.62455,0.70451);
\definecolor{c}{rgb}{0,0,0};
\colorlet{c}{natcomp!70};
\draw [c] (9.64895,0.686899) -- (9.64895,0.686907);
\draw [c] (9.64895,0.686907) -- (9.64895,0.686914);
\draw [c] (9.64082,0.686907) -- (9.64895,0.686907);
\draw [c] (9.64895,0.686907) -- (9.65709,0.686907);
\definecolor{c}{rgb}{0,0,0};
\colorlet{c}{natcomp!70};
\draw [c] (9.66523,0.686899) -- (9.66523,0.686907);
\draw [c] (9.66523,0.686907) -- (9.66523,0.686914);
\draw [c] (9.65709,0.686907) -- (9.66523,0.686907);
\draw [c] (9.66523,0.686907) -- (9.67336,0.686907);
\definecolor{c}{rgb}{0,0,0};
\colorlet{c}{natcomp!70};
\draw [c] (9.6815,0.686894) -- (9.6815,0.686897);
\draw [c] (9.6815,0.686897) -- (9.6815,0.686901);
\draw [c] (9.67336,0.686897) -- (9.6815,0.686897);
\draw [c] (9.6815,0.686897) -- (9.68964,0.686897);
\definecolor{c}{rgb}{0,0,0};
\colorlet{c}{natcomp!70};
\draw [c] (9.71405,0.692495) -- (9.71405,0.706167);
\draw [c] (9.71405,0.706167) -- (9.71405,0.719839);
\draw [c] (9.70591,0.706167) -- (9.71405,0.706167);
\draw [c] (9.71405,0.706167) -- (9.72218,0.706167);
\definecolor{c}{rgb}{0,0,0};
\colorlet{c}{natcomp!70};
\draw [c] (9.74659,0.686894) -- (9.74659,0.686897);
\draw [c] (9.74659,0.686897) -- (9.74659,0.686901);
\draw [c] (9.73845,0.686897) -- (9.74659,0.686897);
\draw [c] (9.74659,0.686897) -- (9.75473,0.686897);
\definecolor{c}{rgb}{0,0,0};
\colorlet{c}{natcomp!70};
\draw [c] (9.76286,0.686896) -- (9.76286,0.686901);
\draw [c] (9.76286,0.686901) -- (9.76286,0.686906);
\draw [c] (9.75473,0.686901) -- (9.76286,0.686901);
\draw [c] (9.76286,0.686901) -- (9.771,0.686901);
\definecolor{c}{rgb}{0,0,0};
\colorlet{c}{natcomp!70};
\draw [c] (9.77914,0.686896) -- (9.77914,0.686902);
\draw [c] (9.77914,0.686902) -- (9.77914,0.686908);
\draw [c] (9.771,0.686902) -- (9.77914,0.686902);
\draw [c] (9.77914,0.686902) -- (9.78727,0.686902);
\definecolor{c}{rgb}{0,0,0};
\colorlet{c}{natcomp!70};
\draw [c] (9.79541,0.686898) -- (9.79541,0.696014);
\draw [c] (9.79541,0.696014) -- (9.79541,0.705129);
\draw [c] (9.78727,0.696014) -- (9.79541,0.696014);
\draw [c] (9.79541,0.696014) -- (9.80354,0.696014);
\definecolor{c}{rgb}{0,0,0};
\colorlet{c}{natcomp!70};
\draw [c] (9.81168,0.686894) -- (9.81168,0.686899);
\draw [c] (9.81168,0.686899) -- (9.81168,0.686903);
\draw [c] (9.80354,0.686899) -- (9.81168,0.686899);
\draw [c] (9.81168,0.686899) -- (9.81982,0.686899);
\definecolor{c}{rgb}{0,0,0};
\colorlet{c}{natcomp!70};
\draw [c] (9.82795,0.692367) -- (9.82795,0.70565);
\draw [c] (9.82795,0.70565) -- (9.82795,0.718933);
\draw [c] (9.81982,0.70565) -- (9.82795,0.70565);
\draw [c] (9.82795,0.70565) -- (9.83609,0.70565);
\definecolor{c}{rgb}{0,0,0};
\colorlet{c}{natcomp!70};
\draw [c] (9.84423,0.686894) -- (9.84423,0.686898);
\draw [c] (9.84423,0.686898) -- (9.84423,0.686902);
\draw [c] (9.83609,0.686898) -- (9.84423,0.686898);
\draw [c] (9.84423,0.686898) -- (9.85236,0.686898);
\definecolor{c}{rgb}{0,0,0};
\colorlet{c}{natcomp!70};
\draw [c] (9.8605,0.686894) -- (9.8605,0.686901);
\draw [c] (9.8605,0.686901) -- (9.8605,0.686908);
\draw [c] (9.85236,0.686901) -- (9.8605,0.686901);
\draw [c] (9.8605,0.686901) -- (9.86864,0.686901);
\definecolor{c}{rgb}{0,0,0};
\colorlet{c}{natcomp!70};
\draw [c] (9.87677,0.686894) -- (9.87677,0.686899);
\draw [c] (9.87677,0.686899) -- (9.87677,0.686903);
\draw [c] (9.86864,0.686899) -- (9.87677,0.686899);
\draw [c] (9.87677,0.686899) -- (9.88491,0.686899);
\definecolor{c}{rgb}{0,0,0};
\colorlet{c}{natcomp!70};
\draw [c] (9.89305,0.691538) -- (9.89305,0.702957);
\draw [c] (9.89305,0.702957) -- (9.89305,0.714375);
\draw [c] (9.88491,0.702957) -- (9.89305,0.702957);
\draw [c] (9.89305,0.702957) -- (9.90118,0.702957);
\definecolor{c}{rgb}{0,0,0};
\colorlet{c}{natcomp!70};
\draw [c] (9.90932,0.692367) -- (9.90932,0.70565);
\draw [c] (9.90932,0.70565) -- (9.90932,0.718933);
\draw [c] (9.90118,0.70565) -- (9.90932,0.70565);
\draw [c] (9.90932,0.70565) -- (9.91745,0.70565);
\definecolor{c}{rgb}{0,0,0};
\colorlet{c}{natcomp!70};
\draw [c] (9.94186,0.686899) -- (9.94186,0.686907);
\draw [c] (9.94186,0.686907) -- (9.94186,0.686915);
\draw [c] (9.93373,0.686907) -- (9.94186,0.686907);
\draw [c] (9.94186,0.686907) -- (9.95,0.686907);
\definecolor{c}{rgb}{0,0,0};
\colorlet{c}{natcomp!70};
\draw [c] (1.04068,0.686894) -- (1.04068,0.6869);
\draw [c] (1.04068,0.6869) -- (1.04068,0.686907);
\draw [c] (1.03255,0.6869) -- (1.04068,0.6869);
\draw [c] (1.04068,0.6869) -- (1.04882,0.6869);
\definecolor{c}{rgb}{0,0,0};
\colorlet{c}{natcomp!70};
\draw [c] (1.10577,0.686894) -- (1.10577,0.705403);
\draw [c] (1.10577,0.705403) -- (1.10577,0.723913);
\draw [c] (1.09764,0.705403) -- (1.10577,0.705403);
\draw [c] (1.10577,0.705403) -- (1.11391,0.705403);
\definecolor{c}{rgb}{0,0,0};
\colorlet{c}{natcomp!70};
\draw [c] (1.12205,0.6869) -- (1.12205,0.708668);
\draw [c] (1.12205,0.708668) -- (1.12205,0.730436);
\draw [c] (1.11391,0.708668) -- (1.12205,0.708668);
\draw [c] (1.12205,0.708668) -- (1.13018,0.708668);
\definecolor{c}{rgb}{0,0,0};
\colorlet{c}{natcomp!70};
\draw [c] (1.13832,0.686894) -- (1.13832,0.686899);
\draw [c] (1.13832,0.686899) -- (1.13832,0.686904);
\draw [c] (1.13018,0.686899) -- (1.13832,0.686899);
\draw [c] (1.13832,0.686899) -- (1.14645,0.686899);
\definecolor{c}{rgb}{0,0,0};
\colorlet{c}{natcomp!70};
\draw [c] (1.15459,0.686907) -- (1.15459,0.695046);
\draw [c] (1.15459,0.695046) -- (1.15459,0.703185);
\draw [c] (1.14645,0.695046) -- (1.15459,0.695046);
\draw [c] (1.15459,0.695046) -- (1.16273,0.695046);
\definecolor{c}{rgb}{0,0,0};
\colorlet{c}{natcomp!70};
\draw [c] (1.17086,0.686894) -- (1.17086,0.686904);
\draw [c] (1.17086,0.686904) -- (1.17086,0.686914);
\draw [c] (1.16273,0.686904) -- (1.17086,0.686904);
\draw [c] (1.17086,0.686904) -- (1.179,0.686904);
\definecolor{c}{rgb}{0,0,0};
\colorlet{c}{natcomp!70};
\draw [c] (1.18714,0.686897) -- (1.18714,0.686904);
\draw [c] (1.18714,0.686904) -- (1.18714,0.686911);
\draw [c] (1.179,0.686904) -- (1.18714,0.686904);
\draw [c] (1.18714,0.686904) -- (1.19527,0.686904);
\definecolor{c}{rgb}{0,0,0};
\colorlet{c}{natcomp!70};
\draw [c] (1.20341,0.686906) -- (1.20341,0.705415);
\draw [c] (1.20341,0.705415) -- (1.20341,0.723925);
\draw [c] (1.19527,0.705415) -- (1.20341,0.705415);
\draw [c] (1.20341,0.705415) -- (1.21155,0.705415);
\definecolor{c}{rgb}{0,0,0};
\colorlet{c}{natcomp!70};
\draw [c] (1.21968,0.686898) -- (1.21968,0.68691);
\draw [c] (1.21968,0.68691) -- (1.21968,0.686923);
\draw [c] (1.21155,0.68691) -- (1.21968,0.68691);
\draw [c] (1.21968,0.68691) -- (1.22782,0.68691);
\definecolor{c}{rgb}{0,0,0};
\colorlet{c}{natcomp!70};
\draw [c] (1.25223,0.695733) -- (1.25223,0.723286);
\draw [c] (1.25223,0.723286) -- (1.25223,0.75084);
\draw [c] (1.24409,0.723286) -- (1.25223,0.723286);
\draw [c] (1.25223,0.723286) -- (1.26036,0.723286);
\definecolor{c}{rgb}{0,0,0};
\colorlet{c}{natcomp!70};
\draw [c] (1.2685,0.686901) -- (1.2685,0.686911);
\draw [c] (1.2685,0.686911) -- (1.2685,0.686922);
\draw [c] (1.26036,0.686911) -- (1.2685,0.686911);
\draw [c] (1.2685,0.686911) -- (1.27664,0.686911);
\definecolor{c}{rgb}{0,0,0};
\colorlet{c}{natcomp!70};
\draw [c] (1.28477,0.686898) -- (1.28477,0.686915);
\draw [c] (1.28477,0.686915) -- (1.28477,0.686932);
\draw [c] (1.27664,0.686915) -- (1.28477,0.686915);
\draw [c] (1.28477,0.686915) -- (1.29291,0.686915);
\definecolor{c}{rgb}{0,0,0};
\colorlet{c}{natcomp!70};
\draw [c] (1.31732,0.694427) -- (1.31732,0.721188);
\draw [c] (1.31732,0.721188) -- (1.31732,0.747948);
\draw [c] (1.30918,0.721188) -- (1.31732,0.721188);
\draw [c] (1.31732,0.721188) -- (1.32545,0.721188);
\definecolor{c}{rgb}{0,0,0};
\colorlet{c}{natcomp!70};
\draw [c] (1.34986,0.686899) -- (1.34986,0.69462);
\draw [c] (1.34986,0.69462) -- (1.34986,0.702342);
\draw [c] (1.34173,0.69462) -- (1.34986,0.69462);
\draw [c] (1.34986,0.69462) -- (1.358,0.69462);
\definecolor{c}{rgb}{0,0,0};
\colorlet{c}{natcomp!70};
\draw [c] (1.36614,0.686894) -- (1.36614,0.6869);
\draw [c] (1.36614,0.6869) -- (1.36614,0.686907);
\draw [c] (1.358,0.6869) -- (1.36614,0.6869);
\draw [c] (1.36614,0.6869) -- (1.37427,0.6869);
\definecolor{c}{rgb}{0,0,0};
\colorlet{c}{natcomp!70};
\draw [c] (1.38241,0.68691) -- (1.38241,0.71207);
\draw [c] (1.38241,0.71207) -- (1.38241,0.73723);
\draw [c] (1.37427,0.71207) -- (1.38241,0.71207);
\draw [c] (1.38241,0.71207) -- (1.39055,0.71207);
\definecolor{c}{rgb}{0,0,0};
\colorlet{c}{natcomp!70};
\draw [c] (1.39868,0.696781) -- (1.39868,0.722664);
\draw [c] (1.39868,0.722664) -- (1.39868,0.748546);
\draw [c] (1.39055,0.722664) -- (1.39868,0.722664);
\draw [c] (1.39868,0.722664) -- (1.40682,0.722664);
\definecolor{c}{rgb}{0,0,0};
\colorlet{c}{natcomp!70};
\draw [c] (1.41495,0.686894) -- (1.41495,0.697305);
\draw [c] (1.41495,0.697305) -- (1.41495,0.707716);
\draw [c] (1.40682,0.697305) -- (1.41495,0.697305);
\draw [c] (1.41495,0.697305) -- (1.42309,0.697305);
\definecolor{c}{rgb}{0,0,0};
\colorlet{c}{natcomp!70};
\draw [c] (1.43123,0.6869) -- (1.43123,0.700903);
\draw [c] (1.43123,0.700903) -- (1.43123,0.714905);
\draw [c] (1.42309,0.700903) -- (1.43123,0.700903);
\draw [c] (1.43123,0.700903) -- (1.43936,0.700903);
\definecolor{c}{rgb}{0,0,0};
\colorlet{c}{natcomp!70};
\draw [c] (1.4475,0.686909) -- (1.4475,0.705418);
\draw [c] (1.4475,0.705418) -- (1.4475,0.723928);
\draw [c] (1.43936,0.705418) -- (1.4475,0.705418);
\draw [c] (1.4475,0.705418) -- (1.45564,0.705418);
\definecolor{c}{rgb}{0,0,0};
\colorlet{c}{natcomp!70};
\draw [c] (1.46377,0.686915) -- (1.46377,0.705424);
\draw [c] (1.46377,0.705424) -- (1.46377,0.723933);
\draw [c] (1.45564,0.705424) -- (1.46377,0.705424);
\draw [c] (1.46377,0.705424) -- (1.47191,0.705424);
\definecolor{c}{rgb}{0,0,0};
\colorlet{c}{natcomp!70};
\draw [c] (1.48005,0.6869) -- (1.48005,0.686916);
\draw [c] (1.48005,0.686916) -- (1.48005,0.686932);
\draw [c] (1.47191,0.686916) -- (1.48005,0.686916);
\draw [c] (1.48005,0.686916) -- (1.48818,0.686916);
\definecolor{c}{rgb}{0,0,0};
\colorlet{c}{natcomp!70};
\draw [c] (1.49632,0.686904) -- (1.49632,0.700906);
\draw [c] (1.49632,0.700906) -- (1.49632,0.714908);
\draw [c] (1.48818,0.700906) -- (1.49632,0.700906);
\draw [c] (1.49632,0.700906) -- (1.50445,0.700906);
\definecolor{c}{rgb}{0,0,0};
\colorlet{c}{natcomp!70};
\draw [c] (1.51259,0.686912) -- (1.51259,0.698144);
\draw [c] (1.51259,0.698144) -- (1.51259,0.709377);
\draw [c] (1.50445,0.698144) -- (1.51259,0.698144);
\draw [c] (1.51259,0.698144) -- (1.52073,0.698144);
\definecolor{c}{rgb}{0,0,0};
\colorlet{c}{natcomp!70};
\draw [c] (1.52886,0.686905) -- (1.52886,0.697316);
\draw [c] (1.52886,0.697316) -- (1.52886,0.707727);
\draw [c] (1.52073,0.697316) -- (1.52886,0.697316);
\draw [c] (1.52886,0.697316) -- (1.537,0.697316);
\definecolor{c}{rgb}{0,0,0};
\colorlet{c}{natcomp!70};
\draw [c] (1.54514,0.68691) -- (1.54514,0.686925);
\draw [c] (1.54514,0.686925) -- (1.54514,0.68694);
\draw [c] (1.537,0.686925) -- (1.54514,0.686925);
\draw [c] (1.54514,0.686925) -- (1.55327,0.686925);
\definecolor{c}{rgb}{0,0,0};
\colorlet{c}{natcomp!70};
\draw [c] (1.56141,0.686899) -- (1.56141,0.686912);
\draw [c] (1.56141,0.686912) -- (1.56141,0.686926);
\draw [c] (1.55327,0.686912) -- (1.56141,0.686912);
\draw [c] (1.56141,0.686912) -- (1.56955,0.686912);
\definecolor{c}{rgb}{0,0,0};
\colorlet{c}{natcomp!70};
\draw [c] (1.57768,0.686904) -- (1.57768,0.705413);
\draw [c] (1.57768,0.705413) -- (1.57768,0.723923);
\draw [c] (1.56955,0.705413) -- (1.57768,0.705413);
\draw [c] (1.57768,0.705413) -- (1.58582,0.705413);
\definecolor{c}{rgb}{0,0,0};
\colorlet{c}{natcomp!70};
\draw [c] (1.59395,0.686923) -- (1.59395,0.696817);
\draw [c] (1.59395,0.696817) -- (1.59395,0.706711);
\draw [c] (1.58582,0.696817) -- (1.59395,0.696817);
\draw [c] (1.59395,0.696817) -- (1.60209,0.696817);
\definecolor{c}{rgb}{0,0,0};
\colorlet{c}{natcomp!70};
\draw [c] (1.61023,0.686911) -- (1.61023,0.719865);
\draw [c] (1.61023,0.719865) -- (1.61023,0.752819);
\draw [c] (1.60209,0.719865) -- (1.61023,0.719865);
\draw [c] (1.61023,0.719865) -- (1.61836,0.719865);
\definecolor{c}{rgb}{0,0,0};
\colorlet{c}{natcomp!70};
\draw [c] (1.6265,0.692256) -- (1.6265,0.705147);
\draw [c] (1.6265,0.705147) -- (1.6265,0.718038);
\draw [c] (1.61836,0.705147) -- (1.6265,0.705147);
\draw [c] (1.6265,0.705147) -- (1.63464,0.705147);
\definecolor{c}{rgb}{0,0,0};
\colorlet{c}{natcomp!70};
\draw [c] (1.64277,0.686912) -- (1.64277,0.686932);
\draw [c] (1.64277,0.686932) -- (1.64277,0.686952);
\draw [c] (1.63464,0.686932) -- (1.64277,0.686932);
\draw [c] (1.64277,0.686932) -- (1.65091,0.686932);
\definecolor{c}{rgb}{0,0,0};
\colorlet{c}{natcomp!70};
\draw [c] (1.65905,0.692192) -- (1.65905,0.704906);
\draw [c] (1.65905,0.704906) -- (1.65905,0.717619);
\draw [c] (1.65091,0.704906) -- (1.65905,0.704906);
\draw [c] (1.65905,0.704906) -- (1.66718,0.704906);
\definecolor{c}{rgb}{0,0,0};
\colorlet{c}{natcomp!70};
\draw [c] (1.67532,0.700693) -- (1.67532,0.723056);
\draw [c] (1.67532,0.723056) -- (1.67532,0.74542);
\draw [c] (1.66718,0.723056) -- (1.67532,0.723056);
\draw [c] (1.67532,0.723056) -- (1.68345,0.723056);
\definecolor{c}{rgb}{0,0,0};
\colorlet{c}{natcomp!70};
\draw [c] (1.69159,0.686942) -- (1.69159,0.708221);
\draw [c] (1.69159,0.708221) -- (1.69159,0.729499);
\draw [c] (1.68345,0.708221) -- (1.69159,0.708221);
\draw [c] (1.69159,0.708221) -- (1.69973,0.708221);
\definecolor{c}{rgb}{0,0,0};
\colorlet{c}{natcomp!70};
\draw [c] (1.70786,0.705683) -- (1.70786,0.724842);
\draw [c] (1.70786,0.724842) -- (1.70786,0.744);
\draw [c] (1.69973,0.724842) -- (1.70786,0.724842);
\draw [c] (1.70786,0.724842) -- (1.716,0.724842);
\definecolor{c}{rgb}{0,0,0};
\colorlet{c}{natcomp!70};
\draw [c] (1.72414,0.686939) -- (1.72414,0.700941);
\draw [c] (1.72414,0.700941) -- (1.72414,0.714943);
\draw [c] (1.716,0.700941) -- (1.72414,0.700941);
\draw [c] (1.72414,0.700941) -- (1.73227,0.700941);
\definecolor{c}{rgb}{0,0,0};
\colorlet{c}{natcomp!70};
\draw [c] (1.74041,0.686936) -- (1.74041,0.708215);
\draw [c] (1.74041,0.708215) -- (1.74041,0.729493);
\draw [c] (1.73227,0.708215) -- (1.74041,0.708215);
\draw [c] (1.74041,0.708215) -- (1.74855,0.708215);
\definecolor{c}{rgb}{0,0,0};
\colorlet{c}{natcomp!70};
\draw [c] (1.75668,0.686903) -- (1.75668,0.696018);
\draw [c] (1.75668,0.696018) -- (1.75668,0.705133);
\draw [c] (1.74855,0.696018) -- (1.75668,0.696018);
\draw [c] (1.75668,0.696018) -- (1.76482,0.696018);
\definecolor{c}{rgb}{0,0,0};
\colorlet{c}{natcomp!70};
\draw [c] (1.77295,0.714667) -- (1.77295,0.738024);
\draw [c] (1.77295,0.738024) -- (1.77295,0.761381);
\draw [c] (1.76482,0.738024) -- (1.77295,0.738024);
\draw [c] (1.77295,0.738024) -- (1.78109,0.738024);
\definecolor{c}{rgb}{0,0,0};
\colorlet{c}{natcomp!70};
\draw [c] (1.78923,0.713535) -- (1.78923,0.735176);
\draw [c] (1.78923,0.735176) -- (1.78923,0.756817);
\draw [c] (1.78109,0.735176) -- (1.78923,0.735176);
\draw [c] (1.78923,0.735176) -- (1.79736,0.735176);
\definecolor{c}{rgb}{0,0,0};
\colorlet{c}{natcomp!70};
\draw [c] (1.8055,0.706706) -- (1.8055,0.727668);
\draw [c] (1.8055,0.727668) -- (1.8055,0.748629);
\draw [c] (1.79736,0.727668) -- (1.8055,0.727668);
\draw [c] (1.8055,0.727668) -- (1.81364,0.727668);
\definecolor{c}{rgb}{0,0,0};
\colorlet{c}{natcomp!70};
\draw [c] (1.82177,0.71253) -- (1.82177,0.7417);
\draw [c] (1.82177,0.7417) -- (1.82177,0.77087);
\draw [c] (1.81364,0.7417) -- (1.82177,0.7417);
\draw [c] (1.82177,0.7417) -- (1.82991,0.7417);
\definecolor{c}{rgb}{0,0,0};
\colorlet{c}{natcomp!70};
\draw [c] (1.83805,0.686939) -- (1.83805,0.698172);
\draw [c] (1.83805,0.698172) -- (1.83805,0.709404);
\draw [c] (1.82991,0.698172) -- (1.83805,0.698172);
\draw [c] (1.83805,0.698172) -- (1.84618,0.698172);
\definecolor{c}{rgb}{0,0,0};
\colorlet{c}{natcomp!70};
\draw [c] (1.85432,0.693926) -- (1.85432,0.714558);
\draw [c] (1.85432,0.714558) -- (1.85432,0.73519);
\draw [c] (1.84618,0.714558) -- (1.85432,0.714558);
\draw [c] (1.85432,0.714558) -- (1.86245,0.714558);
\definecolor{c}{rgb}{0,0,0};
\colorlet{c}{natcomp!70};
\draw [c] (1.87059,0.702794) -- (1.87059,0.726598);
\draw [c] (1.87059,0.726598) -- (1.87059,0.750403);
\draw [c] (1.86245,0.726598) -- (1.87059,0.726598);
\draw [c] (1.87059,0.726598) -- (1.87873,0.726598);
\definecolor{c}{rgb}{0,0,0};
\colorlet{c}{natcomp!70};
\draw [c] (1.88686,0.723513) -- (1.88686,0.749566);
\draw [c] (1.88686,0.749566) -- (1.88686,0.775619);
\draw [c] (1.87873,0.749566) -- (1.88686,0.749566);
\draw [c] (1.88686,0.749566) -- (1.895,0.749566);
\definecolor{c}{rgb}{0,0,0};
\colorlet{c}{natcomp!70};
\draw [c] (1.90314,0.703651) -- (1.90314,0.728586);
\draw [c] (1.90314,0.728586) -- (1.90314,0.753521);
\draw [c] (1.895,0.728586) -- (1.90314,0.728586);
\draw [c] (1.90314,0.728586) -- (1.91127,0.728586);
\definecolor{c}{rgb}{0,0,0};
\colorlet{c}{natcomp!70};
\draw [c] (1.91941,0.735552) -- (1.91941,0.762435);
\draw [c] (1.91941,0.762435) -- (1.91941,0.789317);
\draw [c] (1.91127,0.762435) -- (1.91941,0.762435);
\draw [c] (1.91941,0.762435) -- (1.92755,0.762435);
\definecolor{c}{rgb}{0,0,0};
\colorlet{c}{natcomp!70};
\draw [c] (1.93568,0.705159) -- (1.93568,0.723496);
\draw [c] (1.93568,0.723496) -- (1.93568,0.741834);
\draw [c] (1.92755,0.723496) -- (1.93568,0.723496);
\draw [c] (1.93568,0.723496) -- (1.94382,0.723496);
\definecolor{c}{rgb}{0,0,0};
\colorlet{c}{natcomp!70};
\draw [c] (1.95195,0.728073) -- (1.95195,0.759278);
\draw [c] (1.95195,0.759278) -- (1.95195,0.790483);
\draw [c] (1.94382,0.759278) -- (1.95195,0.759278);
\draw [c] (1.95195,0.759278) -- (1.96009,0.759278);
\definecolor{c}{rgb}{0,0,0};
\colorlet{c}{natcomp!70};
\draw [c] (1.96823,0.721308) -- (1.96823,0.745166);
\draw [c] (1.96823,0.745166) -- (1.96823,0.769024);
\draw [c] (1.96009,0.745166) -- (1.96823,0.745166);
\draw [c] (1.96823,0.745166) -- (1.97636,0.745166);
\definecolor{c}{rgb}{0,0,0};
\colorlet{c}{natcomp!70};
\draw [c] (1.9845,0.717672) -- (1.9845,0.74667);
\draw [c] (1.9845,0.74667) -- (1.9845,0.775667);
\draw [c] (1.97636,0.74667) -- (1.9845,0.74667);
\draw [c] (1.9845,0.74667) -- (1.99264,0.74667);
\definecolor{c}{rgb}{0,0,0};
\colorlet{c}{natcomp!70};
\draw [c] (2.00077,0.754545) -- (2.00077,0.791474);
\draw [c] (2.00077,0.791474) -- (2.00077,0.828402);
\draw [c] (1.99264,0.791474) -- (2.00077,0.791474);
\draw [c] (2.00077,0.791474) -- (2.00891,0.791474);
\definecolor{c}{rgb}{0,0,0};
\colorlet{c}{natcomp!70};
\draw [c] (2.01705,0.812339) -- (2.01705,0.861065);
\draw [c] (2.01705,0.861065) -- (2.01705,0.909792);
\draw [c] (2.00891,0.861065) -- (2.01705,0.861065);
\draw [c] (2.01705,0.861065) -- (2.02518,0.861065);
\definecolor{c}{rgb}{0,0,0};
\colorlet{c}{natcomp!70};
\draw [c] (2.03332,0.756859) -- (2.03332,0.788069);
\draw [c] (2.03332,0.788069) -- (2.03332,0.819279);
\draw [c] (2.02518,0.788069) -- (2.03332,0.788069);
\draw [c] (2.03332,0.788069) -- (2.04145,0.788069);
\definecolor{c}{rgb}{0,0,0};
\colorlet{c}{natcomp!70};
\draw [c] (2.04959,0.733143) -- (2.04959,0.772578);
\draw [c] (2.04959,0.772578) -- (2.04959,0.812014);
\draw [c] (2.04145,0.772578) -- (2.04959,0.772578);
\draw [c] (2.04959,0.772578) -- (2.05773,0.772578);
\definecolor{c}{rgb}{0,0,0};
\colorlet{c}{natcomp!70};
\draw [c] (2.06586,0.734846) -- (2.06586,0.767565);
\draw [c] (2.06586,0.767565) -- (2.06586,0.800284);
\draw [c] (2.05773,0.767565) -- (2.06586,0.767565);
\draw [c] (2.06586,0.767565) -- (2.074,0.767565);
\definecolor{c}{rgb}{0,0,0};
\colorlet{c}{natcomp!70};
\draw [c] (2.08214,0.789195) -- (2.08214,0.827149);
\draw [c] (2.08214,0.827149) -- (2.08214,0.865104);
\draw [c] (2.074,0.827149) -- (2.08214,0.827149);
\draw [c] (2.08214,0.827149) -- (2.09027,0.827149);
\definecolor{c}{rgb}{0,0,0};
\colorlet{c}{natcomp!70};
\draw [c] (2.09841,0.788521) -- (2.09841,0.828396);
\draw [c] (2.09841,0.828396) -- (2.09841,0.868271);
\draw [c] (2.09027,0.828396) -- (2.09841,0.828396);
\draw [c] (2.09841,0.828396) -- (2.10655,0.828396);
\definecolor{c}{rgb}{0,0,0};
\colorlet{c}{natcomp!70};
\draw [c] (2.11468,0.759161) -- (2.11468,0.799535);
\draw [c] (2.11468,0.799535) -- (2.11468,0.839908);
\draw [c] (2.10655,0.799535) -- (2.11468,0.799535);
\draw [c] (2.11468,0.799535) -- (2.12282,0.799535);
\definecolor{c}{rgb}{0,0,0};
\colorlet{c}{natcomp!70};
\draw [c] (2.13095,0.7679) -- (2.13095,0.805702);
\draw [c] (2.13095,0.805702) -- (2.13095,0.843503);
\draw [c] (2.12282,0.805702) -- (2.13095,0.805702);
\draw [c] (2.13095,0.805702) -- (2.13909,0.805702);
\definecolor{c}{rgb}{0,0,0};
\colorlet{c}{natcomp!70};
\draw [c] (2.14723,0.746911) -- (2.14723,0.783535);
\draw [c] (2.14723,0.783535) -- (2.14723,0.820159);
\draw [c] (2.13909,0.783535) -- (2.14723,0.783535);
\draw [c] (2.14723,0.783535) -- (2.15536,0.783535);
\definecolor{c}{rgb}{0,0,0};
\colorlet{c}{natcomp!70};
\draw [c] (2.1635,0.772802) -- (2.1635,0.808728);
\draw [c] (2.1635,0.808728) -- (2.1635,0.844654);
\draw [c] (2.15536,0.808728) -- (2.1635,0.808728);
\draw [c] (2.1635,0.808728) -- (2.17164,0.808728);
\definecolor{c}{rgb}{0,0,0};
\colorlet{c}{natcomp!70};
\draw [c] (2.17977,0.810552) -- (2.17977,0.849296);
\draw [c] (2.17977,0.849296) -- (2.17977,0.888039);
\draw [c] (2.17164,0.849296) -- (2.17977,0.849296);
\draw [c] (2.17977,0.849296) -- (2.18791,0.849296);
\definecolor{c}{rgb}{0,0,0};
\colorlet{c}{natcomp!70};
\draw [c] (2.19605,0.778252) -- (2.19605,0.813555);
\draw [c] (2.19605,0.813555) -- (2.19605,0.848858);
\draw [c] (2.18791,0.813555) -- (2.19605,0.813555);
\draw [c] (2.19605,0.813555) -- (2.20418,0.813555);
\definecolor{c}{rgb}{0,0,0};
\colorlet{c}{natcomp!70};
\draw [c] (2.21232,0.789468) -- (2.21232,0.827907);
\draw [c] (2.21232,0.827907) -- (2.21232,0.866346);
\draw [c] (2.20418,0.827907) -- (2.21232,0.827907);
\draw [c] (2.21232,0.827907) -- (2.22045,0.827907);
\definecolor{c}{rgb}{0,0,0};
\colorlet{c}{natcomp!70};
\draw [c] (2.22859,0.815059) -- (2.22859,0.859924);
\draw [c] (2.22859,0.859924) -- (2.22859,0.904788);
\draw [c] (2.22045,0.859924) -- (2.22859,0.859924);
\draw [c] (2.22859,0.859924) -- (2.23673,0.859924);
\definecolor{c}{rgb}{0,0,0};
\colorlet{c}{natcomp!70};
\draw [c] (2.24486,0.802412) -- (2.24486,0.839797);
\draw [c] (2.24486,0.839797) -- (2.24486,0.877183);
\draw [c] (2.23673,0.839797) -- (2.24486,0.839797);
\draw [c] (2.24486,0.839797) -- (2.253,0.839797);
\definecolor{c}{rgb}{0,0,0};
\colorlet{c}{natcomp!70};
\draw [c] (2.26114,0.840272) -- (2.26114,0.885266);
\draw [c] (2.26114,0.885266) -- (2.26114,0.93026);
\draw [c] (2.253,0.885266) -- (2.26114,0.885266);
\draw [c] (2.26114,0.885266) -- (2.26927,0.885266);
\definecolor{c}{rgb}{0,0,0};
\colorlet{c}{natcomp!70};
\draw [c] (2.27741,0.805072) -- (2.27741,0.848826);
\draw [c] (2.27741,0.848826) -- (2.27741,0.892579);
\draw [c] (2.26927,0.848826) -- (2.27741,0.848826);
\draw [c] (2.27741,0.848826) -- (2.28555,0.848826);
\definecolor{c}{rgb}{0,0,0};
\colorlet{c}{natcomp!70};
\draw [c] (2.29368,0.791504) -- (2.29368,0.826715);
\draw [c] (2.29368,0.826715) -- (2.29368,0.861925);
\draw [c] (2.28555,0.826715) -- (2.29368,0.826715);
\draw [c] (2.29368,0.826715) -- (2.30182,0.826715);
\definecolor{c}{rgb}{0,0,0};
\colorlet{c}{natcomp!70};
\draw [c] (2.30995,0.856209) -- (2.30995,0.906613);
\draw [c] (2.30995,0.906613) -- (2.30995,0.957017);
\draw [c] (2.30182,0.906613) -- (2.30995,0.906613);
\draw [c] (2.30995,0.906613) -- (2.31809,0.906613);
\definecolor{c}{rgb}{0,0,0};
\colorlet{c}{natcomp!70};
\draw [c] (2.32623,0.828277) -- (2.32623,0.869435);
\draw [c] (2.32623,0.869435) -- (2.32623,0.910592);
\draw [c] (2.31809,0.869435) -- (2.32623,0.869435);
\draw [c] (2.32623,0.869435) -- (2.33436,0.869435);
\definecolor{c}{rgb}{0,0,0};
\colorlet{c}{natcomp!70};
\draw [c] (2.3425,0.88625) -- (2.3425,0.939801);
\draw [c] (2.3425,0.939801) -- (2.3425,0.993352);
\draw [c] (2.33436,0.939801) -- (2.3425,0.939801);
\draw [c] (2.3425,0.939801) -- (2.35064,0.939801);
\definecolor{c}{rgb}{0,0,0};
\colorlet{c}{natcomp!70};
\draw [c] (2.35877,0.916275) -- (2.35877,0.969867);
\draw [c] (2.35877,0.969867) -- (2.35877,1.02346);
\draw [c] (2.35064,0.969867) -- (2.35877,0.969867);
\draw [c] (2.35877,0.969867) -- (2.36691,0.969867);
\definecolor{c}{rgb}{0,0,0};
\colorlet{c}{natcomp!70};
\draw [c] (2.37505,0.907296) -- (2.37505,0.962405);
\draw [c] (2.37505,0.962405) -- (2.37505,1.01751);
\draw [c] (2.36691,0.962405) -- (2.37505,0.962405);
\draw [c] (2.37505,0.962405) -- (2.38318,0.962405);
\definecolor{c}{rgb}{0,0,0};
\colorlet{c}{natcomp!70};
\draw [c] (2.39132,0.827878) -- (2.39132,0.86913);
\draw [c] (2.39132,0.86913) -- (2.39132,0.910383);
\draw [c] (2.38318,0.86913) -- (2.39132,0.86913);
\draw [c] (2.39132,0.86913) -- (2.39945,0.86913);
\definecolor{c}{rgb}{0,0,0};
\colorlet{c}{natcomp!70};
\draw [c] (2.40759,0.857178) -- (2.40759,0.910918);
\draw [c] (2.40759,0.910918) -- (2.40759,0.964658);
\draw [c] (2.39945,0.910918) -- (2.40759,0.910918);
\draw [c] (2.40759,0.910918) -- (2.41573,0.910918);
\definecolor{c}{rgb}{0,0,0};
\colorlet{c}{natcomp!70};
\draw [c] (2.42386,0.919688) -- (2.42386,0.974446);
\draw [c] (2.42386,0.974446) -- (2.42386,1.0292);
\draw [c] (2.41573,0.974446) -- (2.42386,0.974446);
\draw [c] (2.42386,0.974446) -- (2.432,0.974446);
\definecolor{c}{rgb}{0,0,0};
\colorlet{c}{natcomp!70};
\draw [c] (2.44014,0.898282) -- (2.44014,0.948641);
\draw [c] (2.44014,0.948641) -- (2.44014,0.999);
\draw [c] (2.432,0.948641) -- (2.44014,0.948641);
\draw [c] (2.44014,0.948641) -- (2.44827,0.948641);
\definecolor{c}{rgb}{0,0,0};
\colorlet{c}{natcomp!70};
\draw [c] (2.45641,0.910429) -- (2.45641,0.968628);
\draw [c] (2.45641,0.968628) -- (2.45641,1.02683);
\draw [c] (2.44827,0.968628) -- (2.45641,0.968628);
\draw [c] (2.45641,0.968628) -- (2.46455,0.968628);
\definecolor{c}{rgb}{0,0,0};
\colorlet{c}{natcomp!70};
\draw [c] (2.47268,0.884339) -- (2.47268,0.931852);
\draw [c] (2.47268,0.931852) -- (2.47268,0.979366);
\draw [c] (2.46455,0.931852) -- (2.47268,0.931852);
\draw [c] (2.47268,0.931852) -- (2.48082,0.931852);
\definecolor{c}{rgb}{0,0,0};
\colorlet{c}{natcomp!70};
\draw [c] (2.48895,0.969757) -- (2.48895,1.0318);
\draw [c] (2.48895,1.0318) -- (2.48895,1.09385);
\draw [c] (2.48082,1.0318) -- (2.48895,1.0318);
\draw [c] (2.48895,1.0318) -- (2.49709,1.0318);
\definecolor{c}{rgb}{0,0,0};
\colorlet{c}{natcomp!70};
\draw [c] (2.50523,0.971435) -- (2.50523,1.03092);
\draw [c] (2.50523,1.03092) -- (2.50523,1.0904);
\draw [c] (2.49709,1.03092) -- (2.50523,1.03092);
\draw [c] (2.50523,1.03092) -- (2.51336,1.03092);
\definecolor{c}{rgb}{0,0,0};
\colorlet{c}{natcomp!70};
\draw [c] (2.5215,0.990909) -- (2.5215,1.05577);
\draw [c] (2.5215,1.05577) -- (2.5215,1.12064);
\draw [c] (2.51336,1.05577) -- (2.5215,1.05577);
\draw [c] (2.5215,1.05577) -- (2.52964,1.05577);
\definecolor{c}{rgb}{0,0,0};
\colorlet{c}{natcomp!70};
\draw [c] (2.53777,1.00227) -- (2.53777,1.06689);
\draw [c] (2.53777,1.06689) -- (2.53777,1.13151);
\draw [c] (2.52964,1.06689) -- (2.53777,1.06689);
\draw [c] (2.53777,1.06689) -- (2.54591,1.06689);
\definecolor{c}{rgb}{0,0,0};
\colorlet{c}{natcomp!70};
\draw [c] (2.55405,1.04669) -- (2.55405,1.11011);
\draw [c] (2.55405,1.11011) -- (2.55405,1.17353);
\draw [c] (2.54591,1.11011) -- (2.55405,1.11011);
\draw [c] (2.55405,1.11011) -- (2.56218,1.11011);
\definecolor{c}{rgb}{0,0,0};
\colorlet{c}{natcomp!70};
\draw [c] (2.57032,0.995602) -- (2.57032,1.06059);
\draw [c] (2.57032,1.06059) -- (2.57032,1.12558);
\draw [c] (2.56218,1.06059) -- (2.57032,1.06059);
\draw [c] (2.57032,1.06059) -- (2.57845,1.06059);
\definecolor{c}{rgb}{0,0,0};
\colorlet{c}{natcomp!70};
\draw [c] (2.58659,0.99475) -- (2.58659,1.05464);
\draw [c] (2.58659,1.05464) -- (2.58659,1.11453);
\draw [c] (2.57845,1.05464) -- (2.58659,1.05464);
\draw [c] (2.58659,1.05464) -- (2.59473,1.05464);
\definecolor{c}{rgb}{0,0,0};
\colorlet{c}{natcomp!70};
\draw [c] (2.60286,1.07289) -- (2.60286,1.14199);
\draw [c] (2.60286,1.14199) -- (2.60286,1.21109);
\draw [c] (2.59473,1.14199) -- (2.60286,1.14199);
\draw [c] (2.60286,1.14199) -- (2.611,1.14199);
\definecolor{c}{rgb}{0,0,0};
\colorlet{c}{natcomp!70};
\draw [c] (2.61914,1.13773) -- (2.61914,1.21239);
\draw [c] (2.61914,1.21239) -- (2.61914,1.28705);
\draw [c] (2.611,1.21239) -- (2.61914,1.21239);
\draw [c] (2.61914,1.21239) -- (2.62727,1.21239);
\definecolor{c}{rgb}{0,0,0};
\colorlet{c}{natcomp!70};
\draw [c] (2.63541,0.996344) -- (2.63541,1.05637);
\draw [c] (2.63541,1.05637) -- (2.63541,1.1164);
\draw [c] (2.62727,1.05637) -- (2.63541,1.05637);
\draw [c] (2.63541,1.05637) -- (2.64355,1.05637);
\definecolor{c}{rgb}{0,0,0};
\colorlet{c}{natcomp!70};
\draw [c] (2.65168,0.988957) -- (2.65168,1.0492);
\draw [c] (2.65168,1.0492) -- (2.65168,1.10944);
\draw [c] (2.64355,1.0492) -- (2.65168,1.0492);
\draw [c] (2.65168,1.0492) -- (2.65982,1.0492);
\definecolor{c}{rgb}{0,0,0};
\colorlet{c}{natcomp!70};
\draw [c] (2.66795,1.1185) -- (2.66795,1.18712);
\draw [c] (2.66795,1.18712) -- (2.66795,1.25574);
\draw [c] (2.65982,1.18712) -- (2.66795,1.18712);
\draw [c] (2.66795,1.18712) -- (2.67609,1.18712);
\definecolor{c}{rgb}{0,0,0};
\colorlet{c}{natcomp!70};
\draw [c] (2.68423,1.11432) -- (2.68423,1.19207);
\draw [c] (2.68423,1.19207) -- (2.68423,1.26981);
\draw [c] (2.67609,1.19207) -- (2.68423,1.19207);
\draw [c] (2.68423,1.19207) -- (2.69236,1.19207);
\definecolor{c}{rgb}{0,0,0};
\colorlet{c}{natcomp!70};
\draw [c] (2.7005,1.21466) -- (2.7005,1.29003);
\draw [c] (2.7005,1.29003) -- (2.7005,1.3654);
\draw [c] (2.69236,1.29003) -- (2.7005,1.29003);
\draw [c] (2.7005,1.29003) -- (2.70864,1.29003);
\definecolor{c}{rgb}{0,0,0};
\colorlet{c}{natcomp!70};
\draw [c] (2.71677,1.10429) -- (2.71677,1.17543);
\draw [c] (2.71677,1.17543) -- (2.71677,1.24658);
\draw [c] (2.70864,1.17543) -- (2.71677,1.17543);
\draw [c] (2.71677,1.17543) -- (2.72491,1.17543);
\definecolor{c}{rgb}{0,0,0};
\colorlet{c}{natcomp!70};
\draw [c] (2.73305,1.12661) -- (2.73305,1.2009);
\draw [c] (2.73305,1.2009) -- (2.73305,1.27519);
\draw [c] (2.72491,1.2009) -- (2.73305,1.2009);
\draw [c] (2.73305,1.2009) -- (2.74118,1.2009);
\definecolor{c}{rgb}{0,0,0};
\colorlet{c}{natcomp!70};
\draw [c] (2.74932,1.06158) -- (2.74932,1.12477);
\draw [c] (2.74932,1.12477) -- (2.74932,1.18796);
\draw [c] (2.74118,1.12477) -- (2.74932,1.12477);
\draw [c] (2.74932,1.12477) -- (2.75745,1.12477);
\definecolor{c}{rgb}{0,0,0};
\colorlet{c}{natcomp!70};
\draw [c] (2.76559,1.18504) -- (2.76559,1.26123);
\draw [c] (2.76559,1.26123) -- (2.76559,1.33741);
\draw [c] (2.75745,1.26123) -- (2.76559,1.26123);
\draw [c] (2.76559,1.26123) -- (2.77373,1.26123);
\definecolor{c}{rgb}{0,0,0};
\colorlet{c}{natcomp!70};
\draw [c] (2.78186,1.23072) -- (2.78186,1.31228);
\draw [c] (2.78186,1.31228) -- (2.78186,1.39384);
\draw [c] (2.77373,1.31228) -- (2.78186,1.31228);
\draw [c] (2.78186,1.31228) -- (2.79,1.31228);
\definecolor{c}{rgb}{0,0,0};
\colorlet{c}{natcomp!70};
\draw [c] (2.79814,1.28934) -- (2.79814,1.37187);
\draw [c] (2.79814,1.37187) -- (2.79814,1.45439);
\draw [c] (2.79,1.37187) -- (2.79814,1.37187);
\draw [c] (2.79814,1.37187) -- (2.80627,1.37187);
\definecolor{c}{rgb}{0,0,0};
\colorlet{c}{natcomp!70};
\draw [c] (2.81441,1.32982) -- (2.81441,1.41625);
\draw [c] (2.81441,1.41625) -- (2.81441,1.50267);
\draw [c] (2.80627,1.41625) -- (2.81441,1.41625);
\draw [c] (2.81441,1.41625) -- (2.82255,1.41625);
\definecolor{c}{rgb}{0,0,0};
\colorlet{c}{natcomp!70};
\draw [c] (2.83068,1.30347) -- (2.83068,1.38681);
\draw [c] (2.83068,1.38681) -- (2.83068,1.47015);
\draw [c] (2.82255,1.38681) -- (2.83068,1.38681);
\draw [c] (2.83068,1.38681) -- (2.83882,1.38681);
\definecolor{c}{rgb}{0,0,0};
\colorlet{c}{natcomp!70};
\draw [c] (2.84695,1.23426) -- (2.84695,1.31216);
\draw [c] (2.84695,1.31216) -- (2.84695,1.39006);
\draw [c] (2.83882,1.31216) -- (2.84695,1.31216);
\draw [c] (2.84695,1.31216) -- (2.85509,1.31216);
\definecolor{c}{rgb}{0,0,0};
\colorlet{c}{natcomp!70};
\draw [c] (2.86323,1.36782) -- (2.86323,1.45635);
\draw [c] (2.86323,1.45635) -- (2.86323,1.54489);
\draw [c] (2.85509,1.45635) -- (2.86323,1.45635);
\draw [c] (2.86323,1.45635) -- (2.87136,1.45635);
\definecolor{c}{rgb}{0,0,0};
\colorlet{c}{natcomp!70};
\draw [c] (2.8795,1.31293) -- (2.8795,1.3965);
\draw [c] (2.8795,1.3965) -- (2.8795,1.48008);
\draw [c] (2.87136,1.3965) -- (2.8795,1.3965);
\draw [c] (2.8795,1.3965) -- (2.88764,1.3965);
\definecolor{c}{rgb}{0,0,0};
\colorlet{c}{natcomp!70};
\draw [c] (2.89577,1.42357) -- (2.89577,1.51525);
\draw [c] (2.89577,1.51525) -- (2.89577,1.60692);
\draw [c] (2.88764,1.51525) -- (2.89577,1.51525);
\draw [c] (2.89577,1.51525) -- (2.90391,1.51525);
\definecolor{c}{rgb}{0,0,0};
\colorlet{c}{natcomp!70};
\draw [c] (2.91205,1.43079) -- (2.91205,1.52154);
\draw [c] (2.91205,1.52154) -- (2.91205,1.61228);
\draw [c] (2.90391,1.52154) -- (2.91205,1.52154);
\draw [c] (2.91205,1.52154) -- (2.92018,1.52154);
\definecolor{c}{rgb}{0,0,0};
\colorlet{c}{natcomp!70};
\draw [c] (2.92832,1.34088) -- (2.92832,1.42855);
\draw [c] (2.92832,1.42855) -- (2.92832,1.51622);
\draw [c] (2.92018,1.42855) -- (2.92832,1.42855);
\draw [c] (2.92832,1.42855) -- (2.93645,1.42855);
\definecolor{c}{rgb}{0,0,0};
\colorlet{c}{natcomp!70};
\draw [c] (2.94459,1.41547) -- (2.94459,1.50915);
\draw [c] (2.94459,1.50915) -- (2.94459,1.60282);
\draw [c] (2.93645,1.50915) -- (2.94459,1.50915);
\draw [c] (2.94459,1.50915) -- (2.95273,1.50915);
\definecolor{c}{rgb}{0,0,0};
\colorlet{c}{natcomp!70};
\draw [c] (2.96086,1.58909) -- (2.96086,1.69352);
\draw [c] (2.96086,1.69352) -- (2.96086,1.79794);
\draw [c] (2.95273,1.69352) -- (2.96086,1.69352);
\draw [c] (2.96086,1.69352) -- (2.969,1.69352);
\definecolor{c}{rgb}{0,0,0};
\colorlet{c}{natcomp!70};
\draw [c] (2.97714,1.33395) -- (2.97714,1.41959);
\draw [c] (2.97714,1.41959) -- (2.97714,1.50522);
\draw [c] (2.969,1.41959) -- (2.97714,1.41959);
\draw [c] (2.97714,1.41959) -- (2.98527,1.41959);
\definecolor{c}{rgb}{0,0,0};
\colorlet{c}{natcomp!70};
\draw [c] (2.99341,1.59874) -- (2.99341,1.69978);
\draw [c] (2.99341,1.69978) -- (2.99341,1.80083);
\draw [c] (2.98527,1.69978) -- (2.99341,1.69978);
\draw [c] (2.99341,1.69978) -- (3.00155,1.69978);
\definecolor{c}{rgb}{0,0,0};
\colorlet{c}{natcomp!70};
\draw [c] (3.00968,1.52556) -- (3.00968,1.61936);
\draw [c] (3.00968,1.61936) -- (3.00968,1.71316);
\draw [c] (3.00155,1.61936) -- (3.00968,1.61936);
\draw [c] (3.00968,1.61936) -- (3.01782,1.61936);
\definecolor{c}{rgb}{0,0,0};
\colorlet{c}{natcomp!70};
\draw [c] (3.02595,1.6018) -- (3.02595,1.70136);
\draw [c] (3.02595,1.70136) -- (3.02595,1.80091);
\draw [c] (3.01782,1.70136) -- (3.02595,1.70136);
\draw [c] (3.02595,1.70136) -- (3.03409,1.70136);
\definecolor{c}{rgb}{0,0,0};
\colorlet{c}{natcomp!70};
\draw [c] (3.04223,1.58016) -- (3.04223,1.68072);
\draw [c] (3.04223,1.68072) -- (3.04223,1.78129);
\draw [c] (3.03409,1.68072) -- (3.04223,1.68072);
\draw [c] (3.04223,1.68072) -- (3.05036,1.68072);
\definecolor{c}{rgb}{0,0,0};
\colorlet{c}{natcomp!70};
\draw [c] (3.0585,1.72343) -- (3.0585,1.82892);
\draw [c] (3.0585,1.82892) -- (3.0585,1.93442);
\draw [c] (3.05036,1.82892) -- (3.0585,1.82892);
\draw [c] (3.0585,1.82892) -- (3.06664,1.82892);
\definecolor{c}{rgb}{0,0,0};
\colorlet{c}{natcomp!70};
\draw [c] (3.07477,1.80323) -- (3.07477,1.92123);
\draw [c] (3.07477,1.92123) -- (3.07477,2.03923);
\draw [c] (3.06664,1.92123) -- (3.07477,1.92123);
\draw [c] (3.07477,1.92123) -- (3.08291,1.92123);
\definecolor{c}{rgb}{0,0,0};
\colorlet{c}{natcomp!70};
\draw [c] (3.09105,1.72774) -- (3.09105,1.8397);
\draw [c] (3.09105,1.8397) -- (3.09105,1.95166);
\draw [c] (3.08291,1.8397) -- (3.09105,1.8397);
\draw [c] (3.09105,1.8397) -- (3.09918,1.8397);
\definecolor{c}{rgb}{0,0,0};
\colorlet{c}{natcomp!70};
\draw [c] (3.10732,1.5984) -- (3.10732,1.70303);
\draw [c] (3.10732,1.70303) -- (3.10732,1.80766);
\draw [c] (3.09918,1.70303) -- (3.10732,1.70303);
\draw [c] (3.10732,1.70303) -- (3.11545,1.70303);
\definecolor{c}{rgb}{0,0,0};
\colorlet{c}{natcomp!70};
\draw [c] (3.12359,1.72127) -- (3.12359,1.83007);
\draw [c] (3.12359,1.83007) -- (3.12359,1.93888);
\draw [c] (3.11545,1.83007) -- (3.12359,1.83007);
\draw [c] (3.12359,1.83007) -- (3.13173,1.83007);
\definecolor{c}{rgb}{0,0,0};
\colorlet{c}{natcomp!70};
\draw [c] (3.13986,1.64464) -- (3.13986,1.74758);
\draw [c] (3.13986,1.74758) -- (3.13986,1.85053);
\draw [c] (3.13173,1.74758) -- (3.13986,1.74758);
\draw [c] (3.13986,1.74758) -- (3.148,1.74758);
\definecolor{c}{rgb}{0,0,0};
\colorlet{c}{natcomp!70};
\draw [c] (3.15614,1.69808) -- (3.15614,1.80701);
\draw [c] (3.15614,1.80701) -- (3.15614,1.91594);
\draw [c] (3.148,1.80701) -- (3.15614,1.80701);
\draw [c] (3.15614,1.80701) -- (3.16427,1.80701);
\definecolor{c}{rgb}{0,0,0};
\colorlet{c}{natcomp!70};
\draw [c] (3.17241,1.85831) -- (3.17241,1.98447);
\draw [c] (3.17241,1.98447) -- (3.17241,2.11062);
\draw [c] (3.16427,1.98447) -- (3.17241,1.98447);
\draw [c] (3.17241,1.98447) -- (3.18055,1.98447);
\definecolor{c}{rgb}{0,0,0};
\colorlet{c}{natcomp!70};
\draw [c] (3.18868,1.75745) -- (3.18868,1.87281);
\draw [c] (3.18868,1.87281) -- (3.18868,1.98818);
\draw [c] (3.18055,1.87281) -- (3.18868,1.87281);
\draw [c] (3.18868,1.87281) -- (3.19682,1.87281);
\definecolor{c}{rgb}{0,0,0};
\colorlet{c}{natcomp!70};
\draw [c] (3.20495,1.88058) -- (3.20495,2.00281);
\draw [c] (3.20495,2.00281) -- (3.20495,2.12505);
\draw [c] (3.19682,2.00281) -- (3.20495,2.00281);
\draw [c] (3.20495,2.00281) -- (3.21309,2.00281);
\definecolor{c}{rgb}{0,0,0};
\colorlet{c}{natcomp!70};
\draw [c] (3.22123,1.80611) -- (3.22123,1.92753);
\draw [c] (3.22123,1.92753) -- (3.22123,2.04895);
\draw [c] (3.21309,1.92753) -- (3.22123,1.92753);
\draw [c] (3.22123,1.92753) -- (3.22936,1.92753);
\definecolor{c}{rgb}{0,0,0};
\colorlet{c}{natcomp!70};
\draw [c] (3.2375,1.91345) -- (3.2375,2.03616);
\draw [c] (3.2375,2.03616) -- (3.2375,2.15888);
\draw [c] (3.22936,2.03616) -- (3.2375,2.03616);
\draw [c] (3.2375,2.03616) -- (3.24564,2.03616);
\definecolor{c}{rgb}{0,0,0};
\colorlet{c}{natcomp!70};
\draw [c] (3.25377,1.96375) -- (3.25377,2.09043);
\draw [c] (3.25377,2.09043) -- (3.25377,2.21712);
\draw [c] (3.24564,2.09043) -- (3.25377,2.09043);
\draw [c] (3.25377,2.09043) -- (3.26191,2.09043);
\definecolor{c}{rgb}{0,0,0};
\colorlet{c}{natcomp!70};
\draw [c] (3.27005,1.92288) -- (3.27005,2.04725);
\draw [c] (3.27005,2.04725) -- (3.27005,2.17162);
\draw [c] (3.26191,2.04725) -- (3.27005,2.04725);
\draw [c] (3.27005,2.04725) -- (3.27818,2.04725);
\definecolor{c}{rgb}{0,0,0};
\colorlet{c}{natcomp!70};
\draw [c] (3.28632,1.84932) -- (3.28632,1.97407);
\draw [c] (3.28632,1.97407) -- (3.28632,2.09882);
\draw [c] (3.27818,1.97407) -- (3.28632,1.97407);
\draw [c] (3.28632,1.97407) -- (3.29445,1.97407);
\definecolor{c}{rgb}{0,0,0};
\colorlet{c}{natcomp!70};
\draw [c] (3.30259,1.90077) -- (3.30259,2.02007);
\draw [c] (3.30259,2.02007) -- (3.30259,2.13937);
\draw [c] (3.29445,2.02007) -- (3.30259,2.02007);
\draw [c] (3.30259,2.02007) -- (3.31073,2.02007);
\definecolor{c}{rgb}{0,0,0};
\colorlet{c}{natcomp!70};
\draw [c] (3.31886,1.83797) -- (3.31886,1.96045);
\draw [c] (3.31886,1.96045) -- (3.31886,2.08293);
\draw [c] (3.31073,1.96045) -- (3.31886,1.96045);
\draw [c] (3.31886,1.96045) -- (3.327,1.96045);
\definecolor{c}{rgb}{0,0,0};
\colorlet{c}{natcomp!70};
\draw [c] (3.33514,1.88986) -- (3.33514,2.01298);
\draw [c] (3.33514,2.01298) -- (3.33514,2.1361);
\draw [c] (3.327,2.01298) -- (3.33514,2.01298);
\draw [c] (3.33514,2.01298) -- (3.34327,2.01298);
\definecolor{c}{rgb}{0,0,0};
\colorlet{c}{natcomp!70};
\draw [c] (3.35141,1.97995) -- (3.35141,2.10928);
\draw [c] (3.35141,2.10928) -- (3.35141,2.2386);
\draw [c] (3.34327,2.10928) -- (3.35141,2.10928);
\draw [c] (3.35141,2.10928) -- (3.35955,2.10928);
\definecolor{c}{rgb}{0,0,0};
\colorlet{c}{natcomp!70};
\draw [c] (3.36768,1.69146) -- (3.36768,1.80413);
\draw [c] (3.36768,1.80413) -- (3.36768,1.9168);
\draw [c] (3.35955,1.80413) -- (3.36768,1.80413);
\draw [c] (3.36768,1.80413) -- (3.37582,1.80413);
\definecolor{c}{rgb}{0,0,0};
\colorlet{c}{natcomp!70};
\draw [c] (3.38395,1.76713) -- (3.38395,1.8867);
\draw [c] (3.38395,1.8867) -- (3.38395,2.00627);
\draw [c] (3.37582,1.8867) -- (3.38395,1.8867);
\draw [c] (3.38395,1.8867) -- (3.39209,1.8867);
\definecolor{c}{rgb}{0,0,0};
\colorlet{c}{natcomp!70};
\draw [c] (3.40023,1.73703) -- (3.40023,1.85405);
\draw [c] (3.40023,1.85405) -- (3.40023,1.97106);
\draw [c] (3.39209,1.85405) -- (3.40023,1.85405);
\draw [c] (3.40023,1.85405) -- (3.40836,1.85405);
\definecolor{c}{rgb}{0,0,0};
\colorlet{c}{natcomp!70};
\draw [c] (3.4165,1.75427) -- (3.4165,1.87418);
\draw [c] (3.4165,1.87418) -- (3.4165,1.99409);
\draw [c] (3.40836,1.87418) -- (3.4165,1.87418);
\draw [c] (3.4165,1.87418) -- (3.42464,1.87418);
\definecolor{c}{rgb}{0,0,0};
\colorlet{c}{natcomp!70};
\draw [c] (3.43277,1.59071) -- (3.43277,1.70168);
\draw [c] (3.43277,1.70168) -- (3.43277,1.81264);
\draw [c] (3.42464,1.70168) -- (3.43277,1.70168);
\draw [c] (3.43277,1.70168) -- (3.44091,1.70168);
\definecolor{c}{rgb}{0,0,0};
\colorlet{c}{natcomp!70};
\draw [c] (3.44905,2.0116) -- (3.44905,2.1488);
\draw [c] (3.44905,2.1488) -- (3.44905,2.286);
\draw [c] (3.44091,2.1488) -- (3.44905,2.1488);
\draw [c] (3.44905,2.1488) -- (3.45718,2.1488);
\definecolor{c}{rgb}{0,0,0};
\colorlet{c}{natcomp!70};
\draw [c] (3.46532,1.58379) -- (3.46532,1.69731);
\draw [c] (3.46532,1.69731) -- (3.46532,1.81082);
\draw [c] (3.45718,1.69731) -- (3.46532,1.69731);
\draw [c] (3.46532,1.69731) -- (3.47345,1.69731);
\definecolor{c}{rgb}{0,0,0};
\colorlet{c}{natcomp!70};
\draw [c] (3.48159,1.69179) -- (3.48159,1.8072);
\draw [c] (3.48159,1.8072) -- (3.48159,1.92261);
\draw [c] (3.47345,1.8072) -- (3.48159,1.8072);
\draw [c] (3.48159,1.8072) -- (3.48973,1.8072);
\definecolor{c}{rgb}{0,0,0};
\colorlet{c}{natcomp!70};
\draw [c] (3.49786,1.66795) -- (3.49786,1.7778);
\draw [c] (3.49786,1.7778) -- (3.49786,1.88766);
\draw [c] (3.48973,1.7778) -- (3.49786,1.7778);
\draw [c] (3.49786,1.7778) -- (3.506,1.7778);
\definecolor{c}{rgb}{0,0,0};
\colorlet{c}{natcomp!70};
\draw [c] (3.51414,1.54378) -- (3.51414,1.65414);
\draw [c] (3.51414,1.65414) -- (3.51414,1.76451);
\draw [c] (3.506,1.65414) -- (3.51414,1.65414);
\draw [c] (3.51414,1.65414) -- (3.52227,1.65414);
\definecolor{c}{rgb}{0,0,0};
\colorlet{c}{natcomp!70};
\draw [c] (3.53041,1.63429) -- (3.53041,1.75469);
\draw [c] (3.53041,1.75469) -- (3.53041,1.87509);
\draw [c] (3.52227,1.75469) -- (3.53041,1.75469);
\draw [c] (3.53041,1.75469) -- (3.53855,1.75469);
\definecolor{c}{rgb}{0,0,0};
\colorlet{c}{natcomp!70};
\draw [c] (3.54668,1.49617) -- (3.54668,1.60081);
\draw [c] (3.54668,1.60081) -- (3.54668,1.70545);
\draw [c] (3.53855,1.60081) -- (3.54668,1.60081);
\draw [c] (3.54668,1.60081) -- (3.55482,1.60081);
\definecolor{c}{rgb}{0,0,0};
\colorlet{c}{natcomp!70};
\draw [c] (3.56295,1.58314) -- (3.56295,1.69229);
\draw [c] (3.56295,1.69229) -- (3.56295,1.80144);
\draw [c] (3.55482,1.69229) -- (3.56295,1.69229);
\draw [c] (3.56295,1.69229) -- (3.57109,1.69229);
\definecolor{c}{rgb}{0,0,0};
\colorlet{c}{natcomp!70};
\draw [c] (3.57923,1.46417) -- (3.57923,1.56462);
\draw [c] (3.57923,1.56462) -- (3.57923,1.66507);
\draw [c] (3.57109,1.56462) -- (3.57923,1.56462);
\draw [c] (3.57923,1.56462) -- (3.58736,1.56462);
\definecolor{c}{rgb}{0,0,0};
\colorlet{c}{natcomp!70};
\draw [c] (3.5955,1.7506) -- (3.5955,1.87231);
\draw [c] (3.5955,1.87231) -- (3.5955,1.99402);
\draw [c] (3.58736,1.87231) -- (3.5955,1.87231);
\draw [c] (3.5955,1.87231) -- (3.60364,1.87231);
\definecolor{c}{rgb}{0,0,0};
\colorlet{c}{natcomp!70};
\draw [c] (3.61177,1.53631) -- (3.61177,1.63577);
\draw [c] (3.61177,1.63577) -- (3.61177,1.73524);
\draw [c] (3.60364,1.63577) -- (3.61177,1.63577);
\draw [c] (3.61177,1.63577) -- (3.61991,1.63577);
\definecolor{c}{rgb}{0,0,0};
\colorlet{c}{natcomp!70};
\draw [c] (3.62805,1.60831) -- (3.62805,1.71397);
\draw [c] (3.62805,1.71397) -- (3.62805,1.81962);
\draw [c] (3.61991,1.71397) -- (3.62805,1.71397);
\draw [c] (3.62805,1.71397) -- (3.63618,1.71397);
\definecolor{c}{rgb}{0,0,0};
\colorlet{c}{natcomp!70};
\draw [c] (3.64432,1.41401) -- (3.64432,1.5139);
\draw [c] (3.64432,1.5139) -- (3.64432,1.61378);
\draw [c] (3.63618,1.5139) -- (3.64432,1.5139);
\draw [c] (3.64432,1.5139) -- (3.65245,1.5139);
\definecolor{c}{rgb}{0,0,0};
\colorlet{c}{natcomp!70};
\draw [c] (3.66059,1.47867) -- (3.66059,1.57869);
\draw [c] (3.66059,1.57869) -- (3.66059,1.67871);
\draw [c] (3.65245,1.57869) -- (3.66059,1.57869);
\draw [c] (3.66059,1.57869) -- (3.66873,1.57869);
\definecolor{c}{rgb}{0,0,0};
\colorlet{c}{natcomp!70};
\draw [c] (3.67686,1.43725) -- (3.67686,1.53155);
\draw [c] (3.67686,1.53155) -- (3.67686,1.62584);
\draw [c] (3.66873,1.53155) -- (3.67686,1.53155);
\draw [c] (3.67686,1.53155) -- (3.685,1.53155);
\definecolor{c}{rgb}{0,0,0};
\colorlet{c}{natcomp!70};
\draw [c] (3.69314,1.44595) -- (3.69314,1.54196);
\draw [c] (3.69314,1.54196) -- (3.69314,1.63798);
\draw [c] (3.685,1.54196) -- (3.69314,1.54196);
\draw [c] (3.69314,1.54196) -- (3.70127,1.54196);
\definecolor{c}{rgb}{0,0,0};
\colorlet{c}{natcomp!70};
\draw [c] (3.70941,1.31331) -- (3.70941,1.4008);
\draw [c] (3.70941,1.4008) -- (3.70941,1.48828);
\draw [c] (3.70127,1.4008) -- (3.70941,1.4008);
\draw [c] (3.70941,1.4008) -- (3.71755,1.4008);
\definecolor{c}{rgb}{0,0,0};
\colorlet{c}{natcomp!70};
\draw [c] (3.72568,1.38718) -- (3.72568,1.47882);
\draw [c] (3.72568,1.47882) -- (3.72568,1.57046);
\draw [c] (3.71755,1.47882) -- (3.72568,1.47882);
\draw [c] (3.72568,1.47882) -- (3.73382,1.47882);
\definecolor{c}{rgb}{0,0,0};
\colorlet{c}{natcomp!70};
\draw [c] (3.74195,1.30543) -- (3.74195,1.39264);
\draw [c] (3.74195,1.39264) -- (3.74195,1.47984);
\draw [c] (3.73382,1.39264) -- (3.74195,1.39264);
\draw [c] (3.74195,1.39264) -- (3.75009,1.39264);
\definecolor{c}{rgb}{0,0,0};
\colorlet{c}{natcomp!70};
\draw [c] (3.75823,1.33609) -- (3.75823,1.42593);
\draw [c] (3.75823,1.42593) -- (3.75823,1.51577);
\draw [c] (3.75009,1.42593) -- (3.75823,1.42593);
\draw [c] (3.75823,1.42593) -- (3.76636,1.42593);
\definecolor{c}{rgb}{0,0,0};
\colorlet{c}{natcomp!70};
\draw [c] (3.7745,1.48113) -- (3.7745,1.57802);
\draw [c] (3.7745,1.57802) -- (3.7745,1.67491);
\draw [c] (3.76636,1.57802) -- (3.7745,1.57802);
\draw [c] (3.7745,1.57802) -- (3.78264,1.57802);
\definecolor{c}{rgb}{0,0,0};
\colorlet{c}{natcomp!70};
\draw [c] (3.79077,1.36985) -- (3.79077,1.4665);
\draw [c] (3.79077,1.4665) -- (3.79077,1.56316);
\draw [c] (3.78264,1.4665) -- (3.79077,1.4665);
\draw [c] (3.79077,1.4665) -- (3.79891,1.4665);
\definecolor{c}{rgb}{0,0,0};
\colorlet{c}{natcomp!70};
\draw [c] (3.80705,1.3613) -- (3.80705,1.45155);
\draw [c] (3.80705,1.45155) -- (3.80705,1.5418);
\draw [c] (3.79891,1.45155) -- (3.80705,1.45155);
\draw [c] (3.80705,1.45155) -- (3.81518,1.45155);
\definecolor{c}{rgb}{0,0,0};
\colorlet{c}{natcomp!70};
\draw [c] (3.82332,1.34548) -- (3.82332,1.43349);
\draw [c] (3.82332,1.43349) -- (3.82332,1.5215);
\draw [c] (3.81518,1.43349) -- (3.82332,1.43349);
\draw [c] (3.82332,1.43349) -- (3.83145,1.43349);
\definecolor{c}{rgb}{0,0,0};
\colorlet{c}{natcomp!70};
\draw [c] (3.83959,1.34648) -- (3.83959,1.44227);
\draw [c] (3.83959,1.44227) -- (3.83959,1.53807);
\draw [c] (3.83145,1.44227) -- (3.83959,1.44227);
\draw [c] (3.83959,1.44227) -- (3.84773,1.44227);
\definecolor{c}{rgb}{0,0,0};
\colorlet{c}{natcomp!70};
\draw [c] (3.85586,1.26603) -- (3.85586,1.34858);
\draw [c] (3.85586,1.34858) -- (3.85586,1.43114);
\draw [c] (3.84773,1.34858) -- (3.85586,1.34858);
\draw [c] (3.85586,1.34858) -- (3.864,1.34858);
\definecolor{c}{rgb}{0,0,0};
\colorlet{c}{natcomp!70};
\draw [c] (3.87214,1.19945) -- (3.87214,1.28167);
\draw [c] (3.87214,1.28167) -- (3.87214,1.3639);
\draw [c] (3.864,1.28167) -- (3.87214,1.28167);
\draw [c] (3.87214,1.28167) -- (3.88027,1.28167);
\definecolor{c}{rgb}{0,0,0};
\colorlet{c}{natcomp!70};
\draw [c] (3.88841,1.37139) -- (3.88841,1.46647);
\draw [c] (3.88841,1.46647) -- (3.88841,1.56155);
\draw [c] (3.88027,1.46647) -- (3.88841,1.46647);
\draw [c] (3.88841,1.46647) -- (3.89655,1.46647);
\definecolor{c}{rgb}{0,0,0};
\colorlet{c}{natcomp!70};
\draw [c] (3.90468,1.0781) -- (3.90468,1.14498);
\draw [c] (3.90468,1.14498) -- (3.90468,1.21187);
\draw [c] (3.89655,1.14498) -- (3.90468,1.14498);
\draw [c] (3.90468,1.14498) -- (3.91282,1.14498);
\definecolor{c}{rgb}{0,0,0};
\colorlet{c}{natcomp!70};
\draw [c] (3.92095,1.12409) -- (3.92095,1.19694);
\draw [c] (3.92095,1.19694) -- (3.92095,1.26979);
\draw [c] (3.91282,1.19694) -- (3.92095,1.19694);
\draw [c] (3.92095,1.19694) -- (3.92909,1.19694);
\definecolor{c}{rgb}{0,0,0};
\colorlet{c}{natcomp!70};
\draw [c] (3.93723,1.21279) -- (3.93723,1.29838);
\draw [c] (3.93723,1.29838) -- (3.93723,1.38397);
\draw [c] (3.92909,1.29838) -- (3.93723,1.29838);
\draw [c] (3.93723,1.29838) -- (3.94536,1.29838);
\definecolor{c}{rgb}{0,0,0};
\colorlet{c}{natcomp!70};
\draw [c] (3.9535,1.18838) -- (3.9535,1.26484);
\draw [c] (3.9535,1.26484) -- (3.9535,1.3413);
\draw [c] (3.94536,1.26484) -- (3.9535,1.26484);
\draw [c] (3.9535,1.26484) -- (3.96164,1.26484);
\definecolor{c}{rgb}{0,0,0};
\colorlet{c}{natcomp!70};
\draw [c] (3.96977,1.13244) -- (3.96977,1.20537);
\draw [c] (3.96977,1.20537) -- (3.96977,1.2783);
\draw [c] (3.96164,1.20537) -- (3.96977,1.20537);
\draw [c] (3.96977,1.20537) -- (3.97791,1.20537);
\definecolor{c}{rgb}{0,0,0};
\colorlet{c}{natcomp!70};
\draw [c] (3.98605,1.18644) -- (3.98605,1.26695);
\draw [c] (3.98605,1.26695) -- (3.98605,1.34746);
\draw [c] (3.97791,1.26695) -- (3.98605,1.26695);
\draw [c] (3.98605,1.26695) -- (3.99418,1.26695);
\definecolor{c}{rgb}{0,0,0};
\colorlet{c}{natcomp!70};
\draw [c] (4.00232,1.18905) -- (4.00232,1.27326);
\draw [c] (4.00232,1.27326) -- (4.00232,1.35748);
\draw [c] (3.99418,1.27326) -- (4.00232,1.27326);
\draw [c] (4.00232,1.27326) -- (4.01045,1.27326);
\definecolor{c}{rgb}{0,0,0};
\colorlet{c}{natcomp!70};
\draw [c] (4.01859,1.21515) -- (4.01859,1.29654);
\draw [c] (4.01859,1.29654) -- (4.01859,1.37792);
\draw [c] (4.01045,1.29654) -- (4.01859,1.29654);
\draw [c] (4.01859,1.29654) -- (4.02673,1.29654);
\definecolor{c}{rgb}{0,0,0};
\colorlet{c}{natcomp!70};
\draw [c] (4.03486,1.0567) -- (4.03486,1.12126);
\draw [c] (4.03486,1.12126) -- (4.03486,1.18582);
\draw [c] (4.02673,1.12126) -- (4.03486,1.12126);
\draw [c] (4.03486,1.12126) -- (4.043,1.12126);
\definecolor{c}{rgb}{0,0,0};
\colorlet{c}{natcomp!70};
\draw [c] (4.05114,1.0637) -- (4.05114,1.13565);
\draw [c] (4.05114,1.13565) -- (4.05114,1.2076);
\draw [c] (4.043,1.13565) -- (4.05114,1.13565);
\draw [c] (4.05114,1.13565) -- (4.05927,1.13565);
\definecolor{c}{rgb}{0,0,0};
\colorlet{c}{natcomp!70};
\draw [c] (4.06741,1.12543) -- (4.06741,1.19947);
\draw [c] (4.06741,1.19947) -- (4.06741,1.27351);
\draw [c] (4.05927,1.19947) -- (4.06741,1.19947);
\draw [c] (4.06741,1.19947) -- (4.07555,1.19947);
\definecolor{c}{rgb}{0,0,0};
\colorlet{c}{natcomp!70};
\draw [c] (4.08368,1.11089) -- (4.08368,1.18647);
\draw [c] (4.08368,1.18647) -- (4.08368,1.26206);
\draw [c] (4.07555,1.18647) -- (4.08368,1.18647);
\draw [c] (4.08368,1.18647) -- (4.09182,1.18647);
\definecolor{c}{rgb}{0,0,0};
\colorlet{c}{natcomp!70};
\draw [c] (4.09995,1.09755) -- (4.09995,1.16945);
\draw [c] (4.09995,1.16945) -- (4.09995,1.24135);
\draw [c] (4.09182,1.16945) -- (4.09995,1.16945);
\draw [c] (4.09995,1.16945) -- (4.10809,1.16945);
\definecolor{c}{rgb}{0,0,0};
\colorlet{c}{natcomp!70};
\draw [c] (4.11623,1.07487) -- (4.11623,1.14657);
\draw [c] (4.11623,1.14657) -- (4.11623,1.21826);
\draw [c] (4.10809,1.14657) -- (4.11623,1.14657);
\draw [c] (4.11623,1.14657) -- (4.12436,1.14657);
\definecolor{c}{rgb}{0,0,0};
\colorlet{c}{natcomp!70};
\draw [c] (4.1325,1.03624) -- (4.1325,1.10277);
\draw [c] (4.1325,1.10277) -- (4.1325,1.16931);
\draw [c] (4.12436,1.10277) -- (4.1325,1.10277);
\draw [c] (4.1325,1.10277) -- (4.14064,1.10277);
\definecolor{c}{rgb}{0,0,0};
\colorlet{c}{natcomp!70};
\draw [c] (4.14877,0.952458) -- (4.14877,1.00761);
\draw [c] (4.14877,1.00761) -- (4.14877,1.06276);
\draw [c] (4.14064,1.00761) -- (4.14877,1.00761);
\draw [c] (4.14877,1.00761) -- (4.15691,1.00761);
\definecolor{c}{rgb}{0,0,0};
\colorlet{c}{natcomp!70};
\draw [c] (4.16505,1.04998) -- (4.16505,1.12012);
\draw [c] (4.16505,1.12012) -- (4.16505,1.19027);
\draw [c] (4.15691,1.12012) -- (4.16505,1.12012);
\draw [c] (4.16505,1.12012) -- (4.17318,1.12012);
\definecolor{c}{rgb}{0,0,0};
\colorlet{c}{natcomp!70};
\draw [c] (4.18132,1.08985) -- (4.18132,1.16335);
\draw [c] (4.18132,1.16335) -- (4.18132,1.23686);
\draw [c] (4.17318,1.16335) -- (4.18132,1.16335);
\draw [c] (4.18132,1.16335) -- (4.18945,1.16335);
\definecolor{c}{rgb}{0,0,0};
\colorlet{c}{natcomp!70};
\draw [c] (4.19759,1.06558) -- (4.19759,1.13256);
\draw [c] (4.19759,1.13256) -- (4.19759,1.19953);
\draw [c] (4.18945,1.13256) -- (4.19759,1.13256);
\draw [c] (4.19759,1.13256) -- (4.20573,1.13256);
\definecolor{c}{rgb}{0,0,0};
\colorlet{c}{natcomp!70};
\draw [c] (4.21386,0.99441) -- (4.21386,1.05948);
\draw [c] (4.21386,1.05948) -- (4.21386,1.12455);
\draw [c] (4.20573,1.05948) -- (4.21386,1.05948);
\draw [c] (4.21386,1.05948) -- (4.222,1.05948);
\definecolor{c}{rgb}{0,0,0};
\colorlet{c}{natcomp!70};
\draw [c] (4.23014,1.15782) -- (4.23014,1.23917);
\draw [c] (4.23014,1.23917) -- (4.23014,1.32052);
\draw [c] (4.222,1.23917) -- (4.23014,1.23917);
\draw [c] (4.23014,1.23917) -- (4.23827,1.23917);
\definecolor{c}{rgb}{0,0,0};
\colorlet{c}{natcomp!70};
\draw [c] (4.24641,0.983316) -- (4.24641,1.04763);
\draw [c] (4.24641,1.04763) -- (4.24641,1.11194);
\draw [c] (4.23827,1.04763) -- (4.24641,1.04763);
\draw [c] (4.24641,1.04763) -- (4.25455,1.04763);
\definecolor{c}{rgb}{0,0,0};
\colorlet{c}{natcomp!70};
\draw [c] (4.26268,1.03462) -- (4.26268,1.10754);
\draw [c] (4.26268,1.10754) -- (4.26268,1.18046);
\draw [c] (4.25455,1.10754) -- (4.26268,1.10754);
\draw [c] (4.26268,1.10754) -- (4.27082,1.10754);
\definecolor{c}{rgb}{0,0,0};
\colorlet{c}{natcomp!70};
\draw [c] (4.27895,0.980957) -- (4.27895,1.04168);
\draw [c] (4.27895,1.04168) -- (4.27895,1.1024);
\draw [c] (4.27082,1.04168) -- (4.27895,1.04168);
\draw [c] (4.27895,1.04168) -- (4.28709,1.04168);
\definecolor{c}{rgb}{0,0,0};
\colorlet{c}{natcomp!70};
\draw [c] (4.29523,0.98657) -- (4.29523,1.04335);
\draw [c] (4.29523,1.04335) -- (4.29523,1.10013);
\draw [c] (4.28709,1.04335) -- (4.29523,1.04335);
\draw [c] (4.29523,1.04335) -- (4.30336,1.04335);
\definecolor{c}{rgb}{0,0,0};
\colorlet{c}{natcomp!70};
\draw [c] (4.3115,0.987687) -- (4.3115,1.0472);
\draw [c] (4.3115,1.0472) -- (4.3115,1.10672);
\draw [c] (4.30336,1.0472) -- (4.3115,1.0472);
\draw [c] (4.3115,1.0472) -- (4.31964,1.0472);
\definecolor{c}{rgb}{0,0,0};
\colorlet{c}{natcomp!70};
\draw [c] (4.32777,1.01895) -- (4.32777,1.08324);
\draw [c] (4.32777,1.08324) -- (4.32777,1.14754);
\draw [c] (4.31964,1.08324) -- (4.32777,1.08324);
\draw [c] (4.32777,1.08324) -- (4.33591,1.08324);
\definecolor{c}{rgb}{0,0,0};
\colorlet{c}{natcomp!70};
\draw [c] (4.34405,0.968473) -- (4.34405,1.02884);
\draw [c] (4.34405,1.02884) -- (4.34405,1.08921);
\draw [c] (4.33591,1.02884) -- (4.34405,1.02884);
\draw [c] (4.34405,1.02884) -- (4.35218,1.02884);
\definecolor{c}{rgb}{0,0,0};
\colorlet{c}{natcomp!70};
\draw [c] (4.36032,0.974112) -- (4.36032,1.03613);
\draw [c] (4.36032,1.03613) -- (4.36032,1.09814);
\draw [c] (4.35218,1.03613) -- (4.36032,1.03613);
\draw [c] (4.36032,1.03613) -- (4.36845,1.03613);
\definecolor{c}{rgb}{0,0,0};
\colorlet{c}{natcomp!70};
\draw [c] (4.37659,0.980809) -- (4.37659,1.04153);
\draw [c] (4.37659,1.04153) -- (4.37659,1.10224);
\draw [c] (4.36845,1.04153) -- (4.37659,1.04153);
\draw [c] (4.37659,1.04153) -- (4.38473,1.04153);
\definecolor{c}{rgb}{0,0,0};
\colorlet{c}{natcomp!70};
\draw [c] (4.39286,0.971029) -- (4.39286,1.02981);
\draw [c] (4.39286,1.02981) -- (4.39286,1.08859);
\draw [c] (4.38473,1.02981) -- (4.39286,1.02981);
\draw [c] (4.39286,1.02981) -- (4.401,1.02981);
\definecolor{c}{rgb}{0,0,0};
\colorlet{c}{natcomp!70};
\draw [c] (4.40914,0.95256) -- (4.40914,1.00793);
\draw [c] (4.40914,1.00793) -- (4.40914,1.06331);
\draw [c] (4.401,1.00793) -- (4.40914,1.00793);
\draw [c] (4.40914,1.00793) -- (4.41727,1.00793);
\definecolor{c}{rgb}{0,0,0};
\colorlet{c}{natcomp!70};
\draw [c] (4.42541,1.04433) -- (4.42541,1.1097);
\draw [c] (4.42541,1.1097) -- (4.42541,1.17507);
\draw [c] (4.41727,1.1097) -- (4.42541,1.1097);
\draw [c] (4.42541,1.1097) -- (4.43355,1.1097);
\definecolor{c}{rgb}{0,0,0};
\colorlet{c}{natcomp!70};
\draw [c] (4.44168,0.921556) -- (4.44168,0.976194);
\draw [c] (4.44168,0.976194) -- (4.44168,1.03083);
\draw [c] (4.43355,0.976194) -- (4.44168,0.976194);
\draw [c] (4.44168,0.976194) -- (4.44982,0.976194);
\definecolor{c}{rgb}{0,0,0};
\colorlet{c}{natcomp!70};
\draw [c] (4.45795,0.972593) -- (4.45795,1.04013);
\draw [c] (4.45795,1.04013) -- (4.45795,1.10767);
\draw [c] (4.44982,1.04013) -- (4.45795,1.04013);
\draw [c] (4.45795,1.04013) -- (4.46609,1.04013);
\definecolor{c}{rgb}{0,0,0};
\colorlet{c}{natcomp!70};
\draw [c] (4.47423,0.946413) -- (4.47423,0.999826);
\draw [c] (4.47423,0.999826) -- (4.47423,1.05324);
\draw [c] (4.46609,0.999826) -- (4.47423,0.999826);
\draw [c] (4.47423,0.999826) -- (4.48236,0.999826);
\definecolor{c}{rgb}{0,0,0};
\colorlet{c}{natcomp!70};
\draw [c] (4.4905,0.928871) -- (4.4905,0.983045);
\draw [c] (4.4905,0.983045) -- (4.4905,1.03722);
\draw [c] (4.48236,0.983045) -- (4.4905,0.983045);
\draw [c] (4.4905,0.983045) -- (4.49864,0.983045);
\definecolor{c}{rgb}{0,0,0};
\colorlet{c}{natcomp!70};
\draw [c] (4.50677,0.92616) -- (4.50677,0.979906);
\draw [c] (4.50677,0.979906) -- (4.50677,1.03365);
\draw [c] (4.49864,0.979906) -- (4.50677,0.979906);
\draw [c] (4.50677,0.979906) -- (4.51491,0.979906);
\definecolor{c}{rgb}{0,0,0};
\colorlet{c}{natcomp!70};
\draw [c] (4.52305,0.866944) -- (4.52305,0.914198);
\draw [c] (4.52305,0.914198) -- (4.52305,0.961452);
\draw [c] (4.51491,0.914198) -- (4.52305,0.914198);
\draw [c] (4.52305,0.914198) -- (4.53118,0.914198);
\definecolor{c}{rgb}{0,0,0};
\colorlet{c}{natcomp!70};
\draw [c] (4.53932,0.956535) -- (4.53932,1.01368);
\draw [c] (4.53932,1.01368) -- (4.53932,1.07083);
\draw [c] (4.53118,1.01368) -- (4.53932,1.01368);
\draw [c] (4.53932,1.01368) -- (4.54745,1.01368);
\definecolor{c}{rgb}{0,0,0};
\colorlet{c}{natcomp!70};
\draw [c] (4.55559,1.00276) -- (4.55559,1.06462);
\draw [c] (4.55559,1.06462) -- (4.55559,1.12648);
\draw [c] (4.54745,1.06462) -- (4.55559,1.06462);
\draw [c] (4.55559,1.06462) -- (4.56373,1.06462);
\definecolor{c}{rgb}{0,0,0};
\colorlet{c}{natcomp!70};
\draw [c] (4.57186,0.919644) -- (4.57186,0.985824);
\draw [c] (4.57186,0.985824) -- (4.57186,1.052);
\draw [c] (4.56373,0.985824) -- (4.57186,0.985824);
\draw [c] (4.57186,0.985824) -- (4.58,0.985824);
\definecolor{c}{rgb}{0,0,0};
\colorlet{c}{natcomp!70};
\draw [c] (4.58814,0.882513) -- (4.58814,0.934061);
\draw [c] (4.58814,0.934061) -- (4.58814,0.985609);
\draw [c] (4.58,0.934061) -- (4.58814,0.934061);
\draw [c] (4.58814,0.934061) -- (4.59627,0.934061);
\definecolor{c}{rgb}{0,0,0};
\colorlet{c}{natcomp!70};
\draw [c] (4.60441,0.912457) -- (4.60441,0.967236);
\draw [c] (4.60441,0.967236) -- (4.60441,1.02202);
\draw [c] (4.59627,0.967236) -- (4.60441,0.967236);
\draw [c] (4.60441,0.967236) -- (4.61255,0.967236);
\definecolor{c}{rgb}{0,0,0};
\colorlet{c}{natcomp!70};
\draw [c] (4.62068,0.914155) -- (4.62068,0.969166);
\draw [c] (4.62068,0.969166) -- (4.62068,1.02418);
\draw [c] (4.61255,0.969166) -- (4.62068,0.969166);
\draw [c] (4.62068,0.969166) -- (4.62882,0.969166);
\definecolor{c}{rgb}{0,0,0};
\colorlet{c}{natcomp!70};
\draw [c] (4.63695,0.879878) -- (4.63695,0.938137);
\draw [c] (4.63695,0.938137) -- (4.63695,0.996396);
\draw [c] (4.62882,0.938137) -- (4.63695,0.938137);
\draw [c] (4.63695,0.938137) -- (4.64509,0.938137);
\definecolor{c}{rgb}{0,0,0};
\colorlet{c}{natcomp!70};
\draw [c] (4.65323,0.899783) -- (4.65323,0.956467);
\draw [c] (4.65323,0.956467) -- (4.65323,1.01315);
\draw [c] (4.64509,0.956467) -- (4.65323,0.956467);
\draw [c] (4.65323,0.956467) -- (4.66136,0.956467);
\definecolor{c}{rgb}{0,0,0};
\colorlet{c}{natcomp!70};
\draw [c] (4.6695,0.884772) -- (4.6695,0.936176);
\draw [c] (4.6695,0.936176) -- (4.6695,0.987579);
\draw [c] (4.66136,0.936176) -- (4.6695,0.936176);
\draw [c] (4.6695,0.936176) -- (4.67764,0.936176);
\definecolor{c}{rgb}{0,0,0};
\colorlet{c}{natcomp!70};
\draw [c] (4.68577,0.985752) -- (4.68577,1.05675);
\draw [c] (4.68577,1.05675) -- (4.68577,1.12775);
\draw [c] (4.67764,1.05675) -- (4.68577,1.05675);
\draw [c] (4.68577,1.05675) -- (4.69391,1.05675);
\definecolor{c}{rgb}{0,0,0};
\colorlet{c}{natcomp!70};
\draw [c] (4.70205,0.888065) -- (4.70205,0.939341);
\draw [c] (4.70205,0.939341) -- (4.70205,0.990618);
\draw [c] (4.69391,0.939341) -- (4.70205,0.939341);
\draw [c] (4.70205,0.939341) -- (4.71018,0.939341);
\definecolor{c}{rgb}{0,0,0};
\colorlet{c}{natcomp!70};
\draw [c] (4.71832,0.950968) -- (4.71832,1.01404);
\draw [c] (4.71832,1.01404) -- (4.71832,1.0771);
\draw [c] (4.71018,1.01404) -- (4.71832,1.01404);
\draw [c] (4.71832,1.01404) -- (4.72645,1.01404);
\definecolor{c}{rgb}{0,0,0};
\colorlet{c}{natcomp!70};
\draw [c] (4.73459,0.890061) -- (4.73459,0.949548);
\draw [c] (4.73459,0.949548) -- (4.73459,1.00904);
\draw [c] (4.72645,0.949548) -- (4.73459,0.949548);
\draw [c] (4.73459,0.949548) -- (4.74273,0.949548);
\definecolor{c}{rgb}{0,0,0};
\colorlet{c}{natcomp!70};
\draw [c] (4.75086,0.903141) -- (4.75086,0.959014);
\draw [c] (4.75086,0.959014) -- (4.75086,1.01489);
\draw [c] (4.74273,0.959014) -- (4.75086,0.959014);
\draw [c] (4.75086,0.959014) -- (4.759,0.959014);
\definecolor{c}{rgb}{0,0,0};
\colorlet{c}{natcomp!70};
\draw [c] (4.76714,0.878125) -- (4.76714,0.929474);
\draw [c] (4.76714,0.929474) -- (4.76714,0.980823);
\draw [c] (4.759,0.929474) -- (4.76714,0.929474);
\draw [c] (4.76714,0.929474) -- (4.77527,0.929474);
\definecolor{c}{rgb}{0,0,0};
\colorlet{c}{natcomp!70};
\draw [c] (4.78341,0.831227) -- (4.78341,0.880248);
\draw [c] (4.78341,0.880248) -- (4.78341,0.929269);
\draw [c] (4.77527,0.880248) -- (4.78341,0.880248);
\draw [c] (4.78341,0.880248) -- (4.79155,0.880248);
\definecolor{c}{rgb}{0,0,0};
\colorlet{c}{natcomp!70};
\draw [c] (4.79968,0.864439) -- (4.79968,0.917743);
\draw [c] (4.79968,0.917743) -- (4.79968,0.971047);
\draw [c] (4.79155,0.917743) -- (4.79968,0.917743);
\draw [c] (4.79968,0.917743) -- (4.80782,0.917743);
\definecolor{c}{rgb}{0,0,0};
\colorlet{c}{natcomp!70};
\draw [c] (4.81595,0.852991) -- (4.81595,0.902588);
\draw [c] (4.81595,0.902588) -- (4.81595,0.952185);
\draw [c] (4.80782,0.902588) -- (4.81595,0.902588);
\draw [c] (4.81595,0.902588) -- (4.82409,0.902588);
\definecolor{c}{rgb}{0,0,0};
\colorlet{c}{natcomp!70};
\draw [c] (4.83223,0.838268) -- (4.83223,0.879525);
\draw [c] (4.83223,0.879525) -- (4.83223,0.920782);
\draw [c] (4.82409,0.879525) -- (4.83223,0.879525);
\draw [c] (4.83223,0.879525) -- (4.84036,0.879525);
\definecolor{c}{rgb}{0,0,0};
\colorlet{c}{natcomp!70};
\draw [c] (4.8485,0.849755) -- (4.8485,0.901917);
\draw [c] (4.8485,0.901917) -- (4.8485,0.954078);
\draw [c] (4.84036,0.901917) -- (4.8485,0.901917);
\draw [c] (4.8485,0.901917) -- (4.85664,0.901917);
\definecolor{c}{rgb}{0,0,0};
\colorlet{c}{natcomp!70};
\draw [c] (4.86477,0.88973) -- (4.86477,0.942964);
\draw [c] (4.86477,0.942964) -- (4.86477,0.996198);
\draw [c] (4.85664,0.942964) -- (4.86477,0.942964);
\draw [c] (4.86477,0.942964) -- (4.87291,0.942964);
\definecolor{c}{rgb}{0,0,0};
\colorlet{c}{natcomp!70};
\draw [c] (4.88105,0.852361) -- (4.88105,0.898715);
\draw [c] (4.88105,0.898715) -- (4.88105,0.945068);
\draw [c] (4.87291,0.898715) -- (4.88105,0.898715);
\draw [c] (4.88105,0.898715) -- (4.88918,0.898715);
\definecolor{c}{rgb}{0,0,0};
\colorlet{c}{natcomp!70};
\draw [c] (4.89732,0.889338) -- (4.89732,0.94623);
\draw [c] (4.89732,0.94623) -- (4.89732,1.00312);
\draw [c] (4.88918,0.94623) -- (4.89732,0.94623);
\draw [c] (4.89732,0.94623) -- (4.90545,0.94623);
\definecolor{c}{rgb}{0,0,0};
\colorlet{c}{natcomp!70};
\draw [c] (4.91359,0.818285) -- (4.91359,0.857852);
\draw [c] (4.91359,0.857852) -- (4.91359,0.897419);
\draw [c] (4.90545,0.857852) -- (4.91359,0.857852);
\draw [c] (4.91359,0.857852) -- (4.92173,0.857852);
\definecolor{c}{rgb}{0,0,0};
\colorlet{c}{natcomp!70};
\draw [c] (4.92986,0.839442) -- (4.92986,0.884558);
\draw [c] (4.92986,0.884558) -- (4.92986,0.929674);
\draw [c] (4.92173,0.884558) -- (4.92986,0.884558);
\draw [c] (4.92986,0.884558) -- (4.938,0.884558);
\definecolor{c}{rgb}{0,0,0};
\colorlet{c}{natcomp!70};
\draw [c] (4.94614,0.928982) -- (4.94614,0.983751);
\draw [c] (4.94614,0.983751) -- (4.94614,1.03852);
\draw [c] (4.938,0.983751) -- (4.94614,0.983751);
\draw [c] (4.94614,0.983751) -- (4.95427,0.983751);
\definecolor{c}{rgb}{0,0,0};
\colorlet{c}{natcomp!70};
\draw [c] (4.96241,0.756623) -- (4.96241,0.793018);
\draw [c] (4.96241,0.793018) -- (4.96241,0.829412);
\draw [c] (4.95427,0.793018) -- (4.96241,0.793018);
\draw [c] (4.96241,0.793018) -- (4.97055,0.793018);
\definecolor{c}{rgb}{0,0,0};
\colorlet{c}{natcomp!70};
\draw [c] (4.97868,0.863301) -- (4.97868,0.913408);
\draw [c] (4.97868,0.913408) -- (4.97868,0.963516);
\draw [c] (4.97055,0.913408) -- (4.97868,0.913408);
\draw [c] (4.97868,0.913408) -- (4.98682,0.913408);
\definecolor{c}{rgb}{0,0,0};
\colorlet{c}{natcomp!70};
\draw [c] (4.99495,0.820068) -- (4.99495,0.859989);
\draw [c] (4.99495,0.859989) -- (4.99495,0.89991);
\draw [c] (4.98682,0.859989) -- (4.99495,0.859989);
\draw [c] (4.99495,0.859989) -- (5.00309,0.859989);
\definecolor{c}{rgb}{0,0,0};
\colorlet{c}{natcomp!70};
\draw [c] (5.01123,0.781416) -- (5.01123,0.816202);
\draw [c] (5.01123,0.816202) -- (5.01123,0.850987);
\draw [c] (5.00309,0.816202) -- (5.01123,0.816202);
\draw [c] (5.01123,0.816202) -- (5.01936,0.816202);
\definecolor{c}{rgb}{0,0,0};
\colorlet{c}{natcomp!70};
\draw [c] (5.0275,0.790764) -- (5.0275,0.827697);
\draw [c] (5.0275,0.827697) -- (5.0275,0.86463);
\draw [c] (5.01936,0.827697) -- (5.0275,0.827697);
\draw [c] (5.0275,0.827697) -- (5.03564,0.827697);
\definecolor{c}{rgb}{0,0,0};
\colorlet{c}{natcomp!70};
\draw [c] (5.04377,0.750942) -- (5.04377,0.784172);
\draw [c] (5.04377,0.784172) -- (5.04377,0.817401);
\draw [c] (5.03564,0.784172) -- (5.04377,0.784172);
\draw [c] (5.04377,0.784172) -- (5.05191,0.784172);
\definecolor{c}{rgb}{0,0,0};
\colorlet{c}{natcomp!70};
\draw [c] (5.06005,0.881823) -- (5.06005,0.934976);
\draw [c] (5.06005,0.934976) -- (5.06005,0.988128);
\draw [c] (5.05191,0.934976) -- (5.06005,0.934976);
\draw [c] (5.06005,0.934976) -- (5.06818,0.934976);
\definecolor{c}{rgb}{0,0,0};
\colorlet{c}{natcomp!70};
\draw [c] (5.07632,0.879455) -- (5.07632,0.936439);
\draw [c] (5.07632,0.936439) -- (5.07632,0.993423);
\draw [c] (5.06818,0.936439) -- (5.07632,0.936439);
\draw [c] (5.07632,0.936439) -- (5.08445,0.936439);
\definecolor{c}{rgb}{0,0,0};
\colorlet{c}{natcomp!70};
\draw [c] (5.09259,0.765327) -- (5.09259,0.799544);
\draw [c] (5.09259,0.799544) -- (5.09259,0.833762);
\draw [c] (5.08445,0.799544) -- (5.09259,0.799544);
\draw [c] (5.09259,0.799544) -- (5.10073,0.799544);
\definecolor{c}{rgb}{0,0,0};
\colorlet{c}{natcomp!70};
\draw [c] (5.10886,0.816846) -- (5.10886,0.854773);
\draw [c] (5.10886,0.854773) -- (5.10886,0.8927);
\draw [c] (5.10073,0.854773) -- (5.10886,0.854773);
\draw [c] (5.10886,0.854773) -- (5.117,0.854773);
\definecolor{c}{rgb}{0,0,0};
\colorlet{c}{natcomp!70};
\draw [c] (5.12514,0.878289) -- (5.12514,0.930852);
\draw [c] (5.12514,0.930852) -- (5.12514,0.983415);
\draw [c] (5.117,0.930852) -- (5.12514,0.930852);
\draw [c] (5.12514,0.930852) -- (5.13327,0.930852);
\definecolor{c}{rgb}{0,0,0};
\colorlet{c}{natcomp!70};
\draw [c] (5.14141,0.886117) -- (5.14141,0.93627);
\draw [c] (5.14141,0.93627) -- (5.14141,0.986423);
\draw [c] (5.13327,0.93627) -- (5.14141,0.93627);
\draw [c] (5.14141,0.93627) -- (5.14955,0.93627);
\definecolor{c}{rgb}{0,0,0};
\colorlet{c}{natcomp!70};
\draw [c] (5.15768,0.756901) -- (5.15768,0.7883);
\draw [c] (5.15768,0.7883) -- (5.15768,0.819698);
\draw [c] (5.14955,0.7883) -- (5.15768,0.7883);
\draw [c] (5.15768,0.7883) -- (5.16582,0.7883);
\definecolor{c}{rgb}{0,0,0};
\colorlet{c}{natcomp!70};
\draw [c] (5.17395,0.850388) -- (5.17395,0.903041);
\draw [c] (5.17395,0.903041) -- (5.17395,0.955693);
\draw [c] (5.16582,0.903041) -- (5.17395,0.903041);
\draw [c] (5.17395,0.903041) -- (5.18209,0.903041);
\definecolor{c}{rgb}{0,0,0};
\colorlet{c}{natcomp!70};
\draw [c] (5.19023,0.837397) -- (5.19023,0.880272);
\draw [c] (5.19023,0.880272) -- (5.19023,0.923147);
\draw [c] (5.18209,0.880272) -- (5.19023,0.880272);
\draw [c] (5.19023,0.880272) -- (5.19836,0.880272);
\definecolor{c}{rgb}{0,0,0};
\colorlet{c}{natcomp!70};
\draw [c] (5.2065,0.782891) -- (5.2065,0.819807);
\draw [c] (5.2065,0.819807) -- (5.2065,0.856723);
\draw [c] (5.19836,0.819807) -- (5.2065,0.819807);
\draw [c] (5.2065,0.819807) -- (5.21464,0.819807);
\definecolor{c}{rgb}{0,0,0};
\colorlet{c}{natcomp!70};
\draw [c] (5.22277,0.877003) -- (5.22277,0.925132);
\draw [c] (5.22277,0.925132) -- (5.22277,0.973261);
\draw [c] (5.21464,0.925132) -- (5.22277,0.925132);
\draw [c] (5.22277,0.925132) -- (5.23091,0.925132);
\definecolor{c}{rgb}{0,0,0};
\colorlet{c}{natcomp!70};
\draw [c] (5.23905,0.832507) -- (5.23905,0.877913);
\draw [c] (5.23905,0.877913) -- (5.23905,0.923319);
\draw [c] (5.23091,0.877913) -- (5.23905,0.877913);
\draw [c] (5.23905,0.877913) -- (5.24718,0.877913);
\definecolor{c}{rgb}{0,0,0};
\colorlet{c}{natcomp!70};
\draw [c] (5.25532,0.809327) -- (5.25532,0.853797);
\draw [c] (5.25532,0.853797) -- (5.25532,0.898268);
\draw [c] (5.24718,0.853797) -- (5.25532,0.853797);
\draw [c] (5.25532,0.853797) -- (5.26345,0.853797);
\definecolor{c}{rgb}{0,0,0};
\colorlet{c}{natcomp!70};
\draw [c] (5.27159,0.838939) -- (5.27159,0.884592);
\draw [c] (5.27159,0.884592) -- (5.27159,0.930245);
\draw [c] (5.26345,0.884592) -- (5.27159,0.884592);
\draw [c] (5.27159,0.884592) -- (5.27973,0.884592);
\definecolor{c}{rgb}{0,0,0};
\colorlet{c}{natcomp!70};
\draw [c] (5.28786,0.774997) -- (5.28786,0.815419);
\draw [c] (5.28786,0.815419) -- (5.28786,0.85584);
\draw [c] (5.27973,0.815419) -- (5.28786,0.815419);
\draw [c] (5.28786,0.815419) -- (5.296,0.815419);
\definecolor{c}{rgb}{0,0,0};
\colorlet{c}{natcomp!70};
\draw [c] (5.30414,0.802019) -- (5.30414,0.840859);
\draw [c] (5.30414,0.840859) -- (5.30414,0.879699);
\draw [c] (5.296,0.840859) -- (5.30414,0.840859);
\draw [c] (5.30414,0.840859) -- (5.31227,0.840859);
\definecolor{c}{rgb}{0,0,0};
\colorlet{c}{natcomp!70};
\draw [c] (5.32041,0.734849) -- (5.32041,0.765883);
\draw [c] (5.32041,0.765883) -- (5.32041,0.796917);
\draw [c] (5.31227,0.765883) -- (5.32041,0.765883);
\draw [c] (5.32041,0.765883) -- (5.32855,0.765883);
\definecolor{c}{rgb}{0,0,0};
\colorlet{c}{natcomp!70};
\draw [c] (5.33668,0.76708) -- (5.33668,0.802514);
\draw [c] (5.33668,0.802514) -- (5.33668,0.837949);
\draw [c] (5.32855,0.802514) -- (5.33668,0.802514);
\draw [c] (5.33668,0.802514) -- (5.34482,0.802514);
\definecolor{c}{rgb}{0,0,0};
\colorlet{c}{natcomp!70};
\draw [c] (5.35295,0.820219) -- (5.35295,0.864005);
\draw [c] (5.35295,0.864005) -- (5.35295,0.907791);
\draw [c] (5.34482,0.864005) -- (5.35295,0.864005);
\draw [c] (5.35295,0.864005) -- (5.36109,0.864005);
\definecolor{c}{rgb}{0,0,0};
\colorlet{c}{natcomp!70};
\draw [c] (5.36923,0.792286) -- (5.36923,0.832453);
\draw [c] (5.36923,0.832453) -- (5.36923,0.87262);
\draw [c] (5.36109,0.832453) -- (5.36923,0.832453);
\draw [c] (5.36923,0.832453) -- (5.37736,0.832453);
\definecolor{c}{rgb}{0,0,0};
\colorlet{c}{natcomp!70};
\draw [c] (5.3855,0.779486) -- (5.3855,0.817789);
\draw [c] (5.3855,0.817789) -- (5.3855,0.856092);
\draw [c] (5.37736,0.817789) -- (5.3855,0.817789);
\draw [c] (5.3855,0.817789) -- (5.39364,0.817789);
\definecolor{c}{rgb}{0,0,0};
\colorlet{c}{natcomp!70};
\draw [c] (5.40177,0.758942) -- (5.40177,0.79033);
\draw [c] (5.40177,0.79033) -- (5.40177,0.821719);
\draw [c] (5.39364,0.79033) -- (5.40177,0.79033);
\draw [c] (5.40177,0.79033) -- (5.40991,0.79033);
\definecolor{c}{rgb}{0,0,0};
\colorlet{c}{natcomp!70};
\draw [c] (5.41805,0.828335) -- (5.41805,0.872313);
\draw [c] (5.41805,0.872313) -- (5.41805,0.916292);
\draw [c] (5.40991,0.872313) -- (5.41805,0.872313);
\draw [c] (5.41805,0.872313) -- (5.42618,0.872313);
\definecolor{c}{rgb}{0,0,0};
\colorlet{c}{natcomp!70};
\draw [c] (5.43432,0.796227) -- (5.43432,0.833442);
\draw [c] (5.43432,0.833442) -- (5.43432,0.870657);
\draw [c] (5.42618,0.833442) -- (5.43432,0.833442);
\draw [c] (5.43432,0.833442) -- (5.44245,0.833442);
\definecolor{c}{rgb}{0,0,0};
\colorlet{c}{natcomp!70};
\draw [c] (5.45059,0.824412) -- (5.45059,0.86803);
\draw [c] (5.45059,0.86803) -- (5.45059,0.911649);
\draw [c] (5.44245,0.86803) -- (5.45059,0.86803);
\draw [c] (5.45059,0.86803) -- (5.45873,0.86803);
\definecolor{c}{rgb}{0,0,0};
\colorlet{c}{natcomp!70};
\draw [c] (5.46686,0.749011) -- (5.46686,0.778106);
\draw [c] (5.46686,0.778106) -- (5.46686,0.807201);
\draw [c] (5.45873,0.778106) -- (5.46686,0.778106);
\draw [c] (5.46686,0.778106) -- (5.475,0.778106);
\definecolor{c}{rgb}{0,0,0};
\colorlet{c}{natcomp!70};
\draw [c] (5.48314,0.717158) -- (5.48314,0.738269);
\draw [c] (5.48314,0.738269) -- (5.48314,0.75938);
\draw [c] (5.475,0.738269) -- (5.48314,0.738269);
\draw [c] (5.48314,0.738269) -- (5.49127,0.738269);
\definecolor{c}{rgb}{0,0,0};
\colorlet{c}{natcomp!70};
\draw [c] (5.49941,0.760794) -- (5.49941,0.791025);
\draw [c] (5.49941,0.791025) -- (5.49941,0.821255);
\draw [c] (5.49127,0.791025) -- (5.49941,0.791025);
\draw [c] (5.49941,0.791025) -- (5.50755,0.791025);
\definecolor{c}{rgb}{0,0,0};
\colorlet{c}{natcomp!70};
\draw [c] (5.51568,0.780683) -- (5.51568,0.819473);
\draw [c] (5.51568,0.819473) -- (5.51568,0.858263);
\draw [c] (5.50755,0.819473) -- (5.51568,0.819473);
\draw [c] (5.51568,0.819473) -- (5.52382,0.819473);
\definecolor{c}{rgb}{0,0,0};
\colorlet{c}{natcomp!70};
\draw [c] (5.53195,0.767697) -- (5.53195,0.81179);
\draw [c] (5.53195,0.81179) -- (5.53195,0.855883);
\draw [c] (5.52382,0.81179) -- (5.53195,0.81179);
\draw [c] (5.53195,0.81179) -- (5.54009,0.81179);
\definecolor{c}{rgb}{0,0,0};
\colorlet{c}{natcomp!70};
\draw [c] (5.54823,0.789662) -- (5.54823,0.82657);
\draw [c] (5.54823,0.82657) -- (5.54823,0.863477);
\draw [c] (5.54009,0.82657) -- (5.54823,0.82657);
\draw [c] (5.54823,0.82657) -- (5.55636,0.82657);
\definecolor{c}{rgb}{0,0,0};
\colorlet{c}{natcomp!70};
\draw [c] (5.5645,0.836001) -- (5.5645,0.881788);
\draw [c] (5.5645,0.881788) -- (5.5645,0.927576);
\draw [c] (5.55636,0.881788) -- (5.5645,0.881788);
\draw [c] (5.5645,0.881788) -- (5.57264,0.881788);
\definecolor{c}{rgb}{0,0,0};
\colorlet{c}{natcomp!70};
\draw [c] (5.58077,0.77978) -- (5.58077,0.81503);
\draw [c] (5.58077,0.81503) -- (5.58077,0.85028);
\draw [c] (5.57264,0.81503) -- (5.58077,0.81503);
\draw [c] (5.58077,0.81503) -- (5.58891,0.81503);
\definecolor{c}{rgb}{0,0,0};
\colorlet{c}{natcomp!70};
\draw [c] (5.59705,0.765328) -- (5.59705,0.798276);
\draw [c] (5.59705,0.798276) -- (5.59705,0.831225);
\draw [c] (5.58891,0.798276) -- (5.59705,0.798276);
\draw [c] (5.59705,0.798276) -- (5.60518,0.798276);
\definecolor{c}{rgb}{0,0,0};
\colorlet{c}{natcomp!70};
\draw [c] (5.61332,0.713101) -- (5.61332,0.735885);
\draw [c] (5.61332,0.735885) -- (5.61332,0.758668);
\draw [c] (5.60518,0.735885) -- (5.61332,0.735885);
\draw [c] (5.61332,0.735885) -- (5.62145,0.735885);
\definecolor{c}{rgb}{0,0,0};
\colorlet{c}{natcomp!70};
\draw [c] (5.62959,0.77156) -- (5.62959,0.806931);
\draw [c] (5.62959,0.806931) -- (5.62959,0.842301);
\draw [c] (5.62145,0.806931) -- (5.62959,0.806931);
\draw [c] (5.62959,0.806931) -- (5.63773,0.806931);
\definecolor{c}{rgb}{0,0,0};
\colorlet{c}{natcomp!70};
\draw [c] (5.64586,0.767598) -- (5.64586,0.801089);
\draw [c] (5.64586,0.801089) -- (5.64586,0.83458);
\draw [c] (5.63773,0.801089) -- (5.64586,0.801089);
\draw [c] (5.64586,0.801089) -- (5.654,0.801089);
\definecolor{c}{rgb}{0,0,0};
\colorlet{c}{natcomp!70};
\draw [c] (5.66214,0.812558) -- (5.66214,0.852249);
\draw [c] (5.66214,0.852249) -- (5.66214,0.891941);
\draw [c] (5.654,0.852249) -- (5.66214,0.852249);
\draw [c] (5.66214,0.852249) -- (5.67027,0.852249);
\definecolor{c}{rgb}{0,0,0};
\colorlet{c}{natcomp!70};
\draw [c] (5.67841,0.750819) -- (5.67841,0.790319);
\draw [c] (5.67841,0.790319) -- (5.67841,0.829819);
\draw [c] (5.67027,0.790319) -- (5.67841,0.790319);
\draw [c] (5.67841,0.790319) -- (5.68655,0.790319);
\definecolor{c}{rgb}{0,0,0};
\colorlet{c}{natcomp!70};
\draw [c] (5.69468,0.763809) -- (5.69468,0.800161);
\draw [c] (5.69468,0.800161) -- (5.69468,0.836514);
\draw [c] (5.68655,0.800161) -- (5.69468,0.800161);
\draw [c] (5.69468,0.800161) -- (5.70282,0.800161);
\definecolor{c}{rgb}{0,0,0};
\colorlet{c}{natcomp!70};
\draw [c] (5.71095,0.773509) -- (5.71095,0.810299);
\draw [c] (5.71095,0.810299) -- (5.71095,0.84709);
\draw [c] (5.70282,0.810299) -- (5.71095,0.810299);
\draw [c] (5.71095,0.810299) -- (5.71909,0.810299);
\definecolor{c}{rgb}{0,0,0};
\colorlet{c}{natcomp!70};
\draw [c] (5.72723,0.781311) -- (5.72723,0.815949);
\draw [c] (5.72723,0.815949) -- (5.72723,0.850586);
\draw [c] (5.71909,0.815949) -- (5.72723,0.815949);
\draw [c] (5.72723,0.815949) -- (5.73536,0.815949);
\definecolor{c}{rgb}{0,0,0};
\colorlet{c}{natcomp!70};
\draw [c] (5.7435,0.749041) -- (5.7435,0.783543);
\draw [c] (5.7435,0.783543) -- (5.7435,0.818044);
\draw [c] (5.73536,0.783543) -- (5.7435,0.783543);
\draw [c] (5.7435,0.783543) -- (5.75164,0.783543);
\definecolor{c}{rgb}{0,0,0};
\colorlet{c}{natcomp!70};
\draw [c] (5.75977,0.734525) -- (5.75977,0.760913);
\draw [c] (5.75977,0.760913) -- (5.75977,0.787301);
\draw [c] (5.75164,0.760913) -- (5.75977,0.760913);
\draw [c] (5.75977,0.760913) -- (5.76791,0.760913);
\definecolor{c}{rgb}{0,0,0};
\colorlet{c}{natcomp!70};
\draw [c] (5.77605,0.757329) -- (5.77605,0.795295);
\draw [c] (5.77605,0.795295) -- (5.77605,0.833262);
\draw [c] (5.76791,0.795295) -- (5.77605,0.795295);
\draw [c] (5.77605,0.795295) -- (5.78418,0.795295);
\definecolor{c}{rgb}{0,0,0};
\colorlet{c}{natcomp!70};
\draw [c] (5.79232,0.778456) -- (5.79232,0.824701);
\draw [c] (5.79232,0.824701) -- (5.79232,0.870946);
\draw [c] (5.78418,0.824701) -- (5.79232,0.824701);
\draw [c] (5.79232,0.824701) -- (5.80045,0.824701);
\definecolor{c}{rgb}{0,0,0};
\colorlet{c}{natcomp!70};
\draw [c] (5.80859,0.791499) -- (5.80859,0.828233);
\draw [c] (5.80859,0.828233) -- (5.80859,0.864968);
\draw [c] (5.80045,0.828233) -- (5.80859,0.828233);
\draw [c] (5.80859,0.828233) -- (5.81673,0.828233);
\definecolor{c}{rgb}{0,0,0};
\colorlet{c}{natcomp!70};
\draw [c] (5.82486,0.778777) -- (5.82486,0.81849);
\draw [c] (5.82486,0.81849) -- (5.82486,0.858202);
\draw [c] (5.81673,0.81849) -- (5.82486,0.81849);
\draw [c] (5.82486,0.81849) -- (5.833,0.81849);
\definecolor{c}{rgb}{0,0,0};
\colorlet{c}{natcomp!70};
\draw [c] (5.84114,0.766161) -- (5.84114,0.815256);
\draw [c] (5.84114,0.815256) -- (5.84114,0.864352);
\draw [c] (5.833,0.815256) -- (5.84114,0.815256);
\draw [c] (5.84114,0.815256) -- (5.84927,0.815256);
\definecolor{c}{rgb}{0,0,0};
\colorlet{c}{natcomp!70};
\draw [c] (5.85741,0.739717) -- (5.85741,0.766483);
\draw [c] (5.85741,0.766483) -- (5.85741,0.79325);
\draw [c] (5.84927,0.766483) -- (5.85741,0.766483);
\draw [c] (5.85741,0.766483) -- (5.86555,0.766483);
\definecolor{c}{rgb}{0,0,0};
\colorlet{c}{natcomp!70};
\draw [c] (5.87368,0.767167) -- (5.87368,0.798411);
\draw [c] (5.87368,0.798411) -- (5.87368,0.829655);
\draw [c] (5.86555,0.798411) -- (5.87368,0.798411);
\draw [c] (5.87368,0.798411) -- (5.88182,0.798411);
\definecolor{c}{rgb}{0,0,0};
\colorlet{c}{natcomp!70};
\draw [c] (5.88995,0.7537) -- (5.88995,0.78274);
\draw [c] (5.88995,0.78274) -- (5.88995,0.811781);
\draw [c] (5.88182,0.78274) -- (5.88995,0.78274);
\draw [c] (5.88995,0.78274) -- (5.89809,0.78274);
\definecolor{c}{rgb}{0,0,0};
\colorlet{c}{natcomp!70};
\draw [c] (5.90623,0.787736) -- (5.90623,0.823567);
\draw [c] (5.90623,0.823567) -- (5.90623,0.859398);
\draw [c] (5.89809,0.823567) -- (5.90623,0.823567);
\draw [c] (5.90623,0.823567) -- (5.91436,0.823567);
\definecolor{c}{rgb}{0,0,0};
\colorlet{c}{natcomp!70};
\draw [c] (5.9225,0.7657) -- (5.9225,0.797966);
\draw [c] (5.9225,0.797966) -- (5.9225,0.830233);
\draw [c] (5.91436,0.797966) -- (5.9225,0.797966);
\draw [c] (5.9225,0.797966) -- (5.93064,0.797966);
\definecolor{c}{rgb}{0,0,0};
\colorlet{c}{natcomp!70};
\draw [c] (5.93877,0.776906) -- (5.93877,0.812086);
\draw [c] (5.93877,0.812086) -- (5.93877,0.847266);
\draw [c] (5.93064,0.812086) -- (5.93877,0.812086);
\draw [c] (5.93877,0.812086) -- (5.94691,0.812086);
\definecolor{c}{rgb}{0,0,0};
\colorlet{c}{natcomp!70};
\draw [c] (5.95505,0.727034) -- (5.95505,0.751779);
\draw [c] (5.95505,0.751779) -- (5.95505,0.776524);
\draw [c] (5.94691,0.751779) -- (5.95505,0.751779);
\draw [c] (5.95505,0.751779) -- (5.96318,0.751779);
\definecolor{c}{rgb}{0,0,0};
\colorlet{c}{natcomp!70};
\draw [c] (5.97132,0.718677) -- (5.97132,0.7484);
\draw [c] (5.97132,0.7484) -- (5.97132,0.778123);
\draw [c] (5.96318,0.7484) -- (5.97132,0.7484);
\draw [c] (5.97132,0.7484) -- (5.97945,0.7484);
\definecolor{c}{rgb}{0,0,0};
\colorlet{c}{natcomp!70};
\draw [c] (5.98759,0.723722) -- (5.98759,0.746507);
\draw [c] (5.98759,0.746507) -- (5.98759,0.769292);
\draw [c] (5.97945,0.746507) -- (5.98759,0.746507);
\draw [c] (5.98759,0.746507) -- (5.99573,0.746507);
\definecolor{c}{rgb}{0,0,0};
\colorlet{c}{natcomp!70};
\draw [c] (6.00386,0.756977) -- (6.00386,0.792774);
\draw [c] (6.00386,0.792774) -- (6.00386,0.828571);
\draw [c] (5.99573,0.792774) -- (6.00386,0.792774);
\draw [c] (6.00386,0.792774) -- (6.012,0.792774);
\definecolor{c}{rgb}{0,0,0};
\colorlet{c}{natcomp!70};
\draw [c] (6.02014,0.72654) -- (6.02014,0.750922);
\draw [c] (6.02014,0.750922) -- (6.02014,0.775304);
\draw [c] (6.012,0.750922) -- (6.02014,0.750922);
\draw [c] (6.02014,0.750922) -- (6.02827,0.750922);
\definecolor{c}{rgb}{0,0,0};
\colorlet{c}{natcomp!70};
\draw [c] (6.03641,0.765876) -- (6.03641,0.798169);
\draw [c] (6.03641,0.798169) -- (6.03641,0.830462);
\draw [c] (6.02827,0.798169) -- (6.03641,0.798169);
\draw [c] (6.03641,0.798169) -- (6.04455,0.798169);
\definecolor{c}{rgb}{0,0,0};
\colorlet{c}{natcomp!70};
\draw [c] (6.05268,0.723692) -- (6.05268,0.763257);
\draw [c] (6.05268,0.763257) -- (6.05268,0.802821);
\draw [c] (6.04455,0.763257) -- (6.05268,0.763257);
\draw [c] (6.05268,0.763257) -- (6.06082,0.763257);
\definecolor{c}{rgb}{0,0,0};
\colorlet{c}{natcomp!70};
\draw [c] (6.06895,0.735947) -- (6.06895,0.776286);
\draw [c] (6.06895,0.776286) -- (6.06895,0.816626);
\draw [c] (6.06082,0.776286) -- (6.06895,0.776286);
\draw [c] (6.06895,0.776286) -- (6.07709,0.776286);
\definecolor{c}{rgb}{0,0,0};
\colorlet{c}{natcomp!70};
\draw [c] (6.08523,0.768199) -- (6.08523,0.8023);
\draw [c] (6.08523,0.8023) -- (6.08523,0.836401);
\draw [c] (6.07709,0.8023) -- (6.08523,0.8023);
\draw [c] (6.08523,0.8023) -- (6.09336,0.8023);
\definecolor{c}{rgb}{0,0,0};
\colorlet{c}{natcomp!70};
\draw [c] (6.1015,0.710277) -- (6.1015,0.729228);
\draw [c] (6.1015,0.729228) -- (6.1015,0.748179);
\draw [c] (6.09336,0.729228) -- (6.1015,0.729228);
\draw [c] (6.1015,0.729228) -- (6.10964,0.729228);
\definecolor{c}{rgb}{0,0,0};
\colorlet{c}{natcomp!70};
\draw [c] (6.11777,0.740692) -- (6.11777,0.767798);
\draw [c] (6.11777,0.767798) -- (6.11777,0.794904);
\draw [c] (6.10964,0.767798) -- (6.11777,0.767798);
\draw [c] (6.11777,0.767798) -- (6.12591,0.767798);
\definecolor{c}{rgb}{0,0,0};
\colorlet{c}{natcomp!70};
\draw [c] (6.13405,0.720528) -- (6.13405,0.743887);
\draw [c] (6.13405,0.743887) -- (6.13405,0.767247);
\draw [c] (6.12591,0.743887) -- (6.13405,0.743887);
\draw [c] (6.13405,0.743887) -- (6.14218,0.743887);
\definecolor{c}{rgb}{0,0,0};
\colorlet{c}{natcomp!70};
\draw [c] (6.15032,0.69748) -- (6.15032,0.712307);
\draw [c] (6.15032,0.712307) -- (6.15032,0.727135);
\draw [c] (6.14218,0.712307) -- (6.15032,0.712307);
\draw [c] (6.15032,0.712307) -- (6.15845,0.712307);
\definecolor{c}{rgb}{0,0,0};
\colorlet{c}{natcomp!70};
\draw [c] (6.16659,0.709648) -- (6.16659,0.728192);
\draw [c] (6.16659,0.728192) -- (6.16659,0.746736);
\draw [c] (6.15845,0.728192) -- (6.16659,0.728192);
\draw [c] (6.16659,0.728192) -- (6.17473,0.728192);
\definecolor{c}{rgb}{0,0,0};
\colorlet{c}{natcomp!70};
\draw [c] (6.18286,0.7346) -- (6.18286,0.766423);
\draw [c] (6.18286,0.766423) -- (6.18286,0.798247);
\draw [c] (6.17473,0.766423) -- (6.18286,0.766423);
\draw [c] (6.18286,0.766423) -- (6.191,0.766423);
\definecolor{c}{rgb}{0,0,0};
\colorlet{c}{natcomp!70};
\draw [c] (6.19914,0.725682) -- (6.19914,0.753499);
\draw [c] (6.19914,0.753499) -- (6.19914,0.781317);
\draw [c] (6.191,0.753499) -- (6.19914,0.753499);
\draw [c] (6.19914,0.753499) -- (6.20727,0.753499);
\definecolor{c}{rgb}{0,0,0};
\colorlet{c}{natcomp!70};
\draw [c] (6.21541,0.726686) -- (6.21541,0.759284);
\draw [c] (6.21541,0.759284) -- (6.21541,0.791882);
\draw [c] (6.20727,0.759284) -- (6.21541,0.759284);
\draw [c] (6.21541,0.759284) -- (6.22355,0.759284);
\definecolor{c}{rgb}{0,0,0};
\colorlet{c}{natcomp!70};
\draw [c] (6.23168,0.742176) -- (6.23168,0.777624);
\draw [c] (6.23168,0.777624) -- (6.23168,0.813072);
\draw [c] (6.22355,0.777624) -- (6.23168,0.777624);
\draw [c] (6.23168,0.777624) -- (6.23982,0.777624);
\definecolor{c}{rgb}{0,0,0};
\colorlet{c}{natcomp!70};
\draw [c] (6.24795,0.776333) -- (6.24795,0.816985);
\draw [c] (6.24795,0.816985) -- (6.24795,0.857637);
\draw [c] (6.23982,0.816985) -- (6.24795,0.816985);
\draw [c] (6.24795,0.816985) -- (6.25609,0.816985);
\definecolor{c}{rgb}{0,0,0};
\colorlet{c}{natcomp!70};
\draw [c] (6.26423,0.716593) -- (6.26423,0.737387);
\draw [c] (6.26423,0.737387) -- (6.26423,0.758182);
\draw [c] (6.25609,0.737387) -- (6.26423,0.737387);
\draw [c] (6.26423,0.737387) -- (6.27236,0.737387);
\definecolor{c}{rgb}{0,0,0};
\colorlet{c}{natcomp!70};
\draw [c] (6.2805,0.73157) -- (6.2805,0.759414);
\draw [c] (6.2805,0.759414) -- (6.2805,0.787258);
\draw [c] (6.27236,0.759414) -- (6.2805,0.759414);
\draw [c] (6.2805,0.759414) -- (6.28864,0.759414);
\definecolor{c}{rgb}{0,0,0};
\colorlet{c}{natcomp!70};
\draw [c] (6.29677,0.765117) -- (6.29677,0.800819);
\draw [c] (6.29677,0.800819) -- (6.29677,0.83652);
\draw [c] (6.28864,0.800819) -- (6.29677,0.800819);
\draw [c] (6.29677,0.800819) -- (6.30491,0.800819);
\definecolor{c}{rgb}{0,0,0};
\colorlet{c}{natcomp!70};
\draw [c] (6.31305,0.697057) -- (6.31305,0.711255);
\draw [c] (6.31305,0.711255) -- (6.31305,0.725454);
\draw [c] (6.30491,0.711255) -- (6.31305,0.711255);
\draw [c] (6.31305,0.711255) -- (6.32118,0.711255);
\definecolor{c}{rgb}{0,0,0};
\colorlet{c}{natcomp!70};
\draw [c] (6.32932,0.699766) -- (6.32932,0.721515);
\draw [c] (6.32932,0.721515) -- (6.32932,0.743263);
\draw [c] (6.32118,0.721515) -- (6.32932,0.721515);
\draw [c] (6.32932,0.721515) -- (6.33745,0.721515);
\definecolor{c}{rgb}{0,0,0};
\colorlet{c}{natcomp!70};
\draw [c] (6.34559,0.75193) -- (6.34559,0.784287);
\draw [c] (6.34559,0.784287) -- (6.34559,0.816645);
\draw [c] (6.33745,0.784287) -- (6.34559,0.784287);
\draw [c] (6.34559,0.784287) -- (6.35373,0.784287);
\definecolor{c}{rgb}{0,0,0};
\colorlet{c}{natcomp!70};
\draw [c] (6.36186,0.751842) -- (6.36186,0.788755);
\draw [c] (6.36186,0.788755) -- (6.36186,0.825668);
\draw [c] (6.35373,0.788755) -- (6.36186,0.788755);
\draw [c] (6.36186,0.788755) -- (6.37,0.788755);
\definecolor{c}{rgb}{0,0,0};
\colorlet{c}{natcomp!70};
\draw [c] (6.37814,0.740937) -- (6.37814,0.773723);
\draw [c] (6.37814,0.773723) -- (6.37814,0.806508);
\draw [c] (6.37,0.773723) -- (6.37814,0.773723);
\draw [c] (6.37814,0.773723) -- (6.38627,0.773723);
\definecolor{c}{rgb}{0,0,0};
\colorlet{c}{natcomp!70};
\draw [c] (6.39441,0.698759) -- (6.39441,0.715275);
\draw [c] (6.39441,0.715275) -- (6.39441,0.73179);
\draw [c] (6.38627,0.715275) -- (6.39441,0.715275);
\draw [c] (6.39441,0.715275) -- (6.40255,0.715275);
\definecolor{c}{rgb}{0,0,0};
\colorlet{c}{natcomp!70};
\draw [c] (6.41068,0.710571) -- (6.41068,0.730263);
\draw [c] (6.41068,0.730263) -- (6.41068,0.749955);
\draw [c] (6.40255,0.730263) -- (6.41068,0.730263);
\draw [c] (6.41068,0.730263) -- (6.41882,0.730263);
\definecolor{c}{rgb}{0,0,0};
\colorlet{c}{natcomp!70};
\draw [c] (6.42695,0.733223) -- (6.42695,0.762826);
\draw [c] (6.42695,0.762826) -- (6.42695,0.792429);
\draw [c] (6.41882,0.762826) -- (6.42695,0.762826);
\draw [c] (6.42695,0.762826) -- (6.43509,0.762826);
\definecolor{c}{rgb}{0,0,0};
\colorlet{c}{natcomp!70};
\draw [c] (6.44323,0.722776) -- (6.44323,0.747622);
\draw [c] (6.44323,0.747622) -- (6.44323,0.772467);
\draw [c] (6.43509,0.747622) -- (6.44323,0.747622);
\draw [c] (6.44323,0.747622) -- (6.45136,0.747622);
\definecolor{c}{rgb}{0,0,0};
\colorlet{c}{natcomp!70};
\draw [c] (6.4595,0.739345) -- (6.4595,0.766635);
\draw [c] (6.4595,0.766635) -- (6.4595,0.793925);
\draw [c] (6.45136,0.766635) -- (6.4595,0.766635);
\draw [c] (6.4595,0.766635) -- (6.46764,0.766635);
\definecolor{c}{rgb}{0,0,0};
\colorlet{c}{natcomp!70};
\draw [c] (6.47577,0.710389) -- (6.47577,0.729721);
\draw [c] (6.47577,0.729721) -- (6.47577,0.749052);
\draw [c] (6.46764,0.729721) -- (6.47577,0.729721);
\draw [c] (6.47577,0.729721) -- (6.48391,0.729721);
\definecolor{c}{rgb}{0,0,0};
\colorlet{c}{natcomp!70};
\draw [c] (6.49205,0.741601) -- (6.49205,0.774287);
\draw [c] (6.49205,0.774287) -- (6.49205,0.806974);
\draw [c] (6.48391,0.774287) -- (6.49205,0.774287);
\draw [c] (6.49205,0.774287) -- (6.50018,0.774287);
\definecolor{c}{rgb}{0,0,0};
\colorlet{c}{natcomp!70};
\draw [c] (6.50832,0.704581) -- (6.50832,0.722973);
\draw [c] (6.50832,0.722973) -- (6.50832,0.741364);
\draw [c] (6.50018,0.722973) -- (6.50832,0.722973);
\draw [c] (6.50832,0.722973) -- (6.51645,0.722973);
\definecolor{c}{rgb}{0,0,0};
\colorlet{c}{natcomp!70};
\draw [c] (6.52459,0.698445) -- (6.52459,0.714356);
\draw [c] (6.52459,0.714356) -- (6.52459,0.730267);
\draw [c] (6.51645,0.714356) -- (6.52459,0.714356);
\draw [c] (6.52459,0.714356) -- (6.53273,0.714356);
\definecolor{c}{rgb}{0,0,0};
\colorlet{c}{natcomp!70};
\draw [c] (6.54086,0.713893) -- (6.54086,0.736042);
\draw [c] (6.54086,0.736042) -- (6.54086,0.758192);
\draw [c] (6.53273,0.736042) -- (6.54086,0.736042);
\draw [c] (6.54086,0.736042) -- (6.549,0.736042);
\definecolor{c}{rgb}{0,0,0};
\colorlet{c}{natcomp!70};
\draw [c] (6.55714,0.705456) -- (6.55714,0.724441);
\draw [c] (6.55714,0.724441) -- (6.55714,0.743426);
\draw [c] (6.549,0.724441) -- (6.55714,0.724441);
\draw [c] (6.55714,0.724441) -- (6.56527,0.724441);
\definecolor{c}{rgb}{0,0,0};
\colorlet{c}{natcomp!70};
\draw [c] (6.57341,0.708858) -- (6.57341,0.726682);
\draw [c] (6.57341,0.726682) -- (6.57341,0.744506);
\draw [c] (6.56527,0.726682) -- (6.57341,0.726682);
\draw [c] (6.57341,0.726682) -- (6.58155,0.726682);
\definecolor{c}{rgb}{0,0,0};
\colorlet{c}{natcomp!70};
\draw [c] (6.58968,0.744164) -- (6.58968,0.773304);
\draw [c] (6.58968,0.773304) -- (6.58968,0.802443);
\draw [c] (6.58155,0.773304) -- (6.58968,0.773304);
\draw [c] (6.58968,0.773304) -- (6.59782,0.773304);
\definecolor{c}{rgb}{0,0,0};
\colorlet{c}{natcomp!70};
\draw [c] (6.60595,0.728187) -- (6.60595,0.754684);
\draw [c] (6.60595,0.754684) -- (6.60595,0.781181);
\draw [c] (6.59782,0.754684) -- (6.60595,0.754684);
\draw [c] (6.60595,0.754684) -- (6.61409,0.754684);
\definecolor{c}{rgb}{0,0,0};
\colorlet{c}{natcomp!70};
\draw [c] (6.62223,0.716607) -- (6.62223,0.741593);
\draw [c] (6.62223,0.741593) -- (6.62223,0.766578);
\draw [c] (6.61409,0.741593) -- (6.62223,0.741593);
\draw [c] (6.62223,0.741593) -- (6.63036,0.741593);
\definecolor{c}{rgb}{0,0,0};
\colorlet{c}{natcomp!70};
\draw [c] (6.6385,0.699804) -- (6.6385,0.717549);
\draw [c] (6.6385,0.717549) -- (6.6385,0.735295);
\draw [c] (6.63036,0.717549) -- (6.6385,0.717549);
\draw [c] (6.6385,0.717549) -- (6.64664,0.717549);
\definecolor{c}{rgb}{0,0,0};
\colorlet{c}{natcomp!70};
\draw [c] (6.65477,0.753668) -- (6.65477,0.789033);
\draw [c] (6.65477,0.789033) -- (6.65477,0.824398);
\draw [c] (6.64664,0.789033) -- (6.65477,0.789033);
\draw [c] (6.65477,0.789033) -- (6.66291,0.789033);
\definecolor{c}{rgb}{0,0,0};
\colorlet{c}{natcomp!70};
\draw [c] (6.67105,0.708234) -- (6.67105,0.734765);
\draw [c] (6.67105,0.734765) -- (6.67105,0.761297);
\draw [c] (6.66291,0.734765) -- (6.67105,0.734765);
\draw [c] (6.67105,0.734765) -- (6.67918,0.734765);
\definecolor{c}{rgb}{0,0,0};
\colorlet{c}{natcomp!70};
\draw [c] (6.68732,0.713447) -- (6.68732,0.738205);
\draw [c] (6.68732,0.738205) -- (6.68732,0.762963);
\draw [c] (6.67918,0.738205) -- (6.68732,0.738205);
\draw [c] (6.68732,0.738205) -- (6.69545,0.738205);
\definecolor{c}{rgb}{0,0,0};
\colorlet{c}{natcomp!70};
\draw [c] (6.70359,0.733101) -- (6.70359,0.759424);
\draw [c] (6.70359,0.759424) -- (6.70359,0.785747);
\draw [c] (6.69545,0.759424) -- (6.70359,0.759424);
\draw [c] (6.70359,0.759424) -- (6.71173,0.759424);
\definecolor{c}{rgb}{0,0,0};
\colorlet{c}{natcomp!70};
\draw [c] (6.71986,0.703929) -- (6.71986,0.721255);
\draw [c] (6.71986,0.721255) -- (6.71986,0.738582);
\draw [c] (6.71173,0.721255) -- (6.71986,0.721255);
\draw [c] (6.71986,0.721255) -- (6.728,0.721255);
\definecolor{c}{rgb}{0,0,0};
\colorlet{c}{natcomp!70};
\draw [c] (6.73614,0.704397) -- (6.73614,0.721875);
\draw [c] (6.73614,0.721875) -- (6.73614,0.739353);
\draw [c] (6.728,0.721875) -- (6.73614,0.721875);
\draw [c] (6.73614,0.721875) -- (6.74427,0.721875);
\definecolor{c}{rgb}{0,0,0};
\colorlet{c}{natcomp!70};
\draw [c] (6.75241,0.710501) -- (6.75241,0.729699);
\draw [c] (6.75241,0.729699) -- (6.75241,0.748896);
\draw [c] (6.74427,0.729699) -- (6.75241,0.729699);
\draw [c] (6.75241,0.729699) -- (6.76055,0.729699);
\definecolor{c}{rgb}{0,0,0};
\colorlet{c}{natcomp!70};
\draw [c] (6.76868,0.711589) -- (6.76868,0.731751);
\draw [c] (6.76868,0.731751) -- (6.76868,0.751914);
\draw [c] (6.76055,0.731751) -- (6.76868,0.731751);
\draw [c] (6.76868,0.731751) -- (6.77682,0.731751);
\definecolor{c}{rgb}{0,0,0};
\colorlet{c}{natcomp!70};
\draw [c] (6.78495,0.718017) -- (6.78495,0.745245);
\draw [c] (6.78495,0.745245) -- (6.78495,0.772473);
\draw [c] (6.77682,0.745245) -- (6.78495,0.745245);
\draw [c] (6.78495,0.745245) -- (6.79309,0.745245);
\definecolor{c}{rgb}{0,0,0};
\colorlet{c}{natcomp!70};
\draw [c] (6.80123,0.697887) -- (6.80123,0.713515);
\draw [c] (6.80123,0.713515) -- (6.80123,0.729144);
\draw [c] (6.79309,0.713515) -- (6.80123,0.713515);
\draw [c] (6.80123,0.713515) -- (6.80936,0.713515);
\definecolor{c}{rgb}{0,0,0};
\colorlet{c}{natcomp!70};
\draw [c] (6.8175,0.712023) -- (6.8175,0.732788);
\draw [c] (6.8175,0.732788) -- (6.8175,0.753553);
\draw [c] (6.80936,0.732788) -- (6.8175,0.732788);
\draw [c] (6.8175,0.732788) -- (6.82564,0.732788);
\definecolor{c}{rgb}{0,0,0};
\colorlet{c}{natcomp!70};
\draw [c] (6.83377,0.714232) -- (6.83377,0.736446);
\draw [c] (6.83377,0.736446) -- (6.83377,0.758661);
\draw [c] (6.82564,0.736446) -- (6.83377,0.736446);
\draw [c] (6.83377,0.736446) -- (6.84191,0.736446);
\definecolor{c}{rgb}{0,0,0};
\colorlet{c}{natcomp!70};
\draw [c] (6.85005,0.709462) -- (6.85005,0.739161);
\draw [c] (6.85005,0.739161) -- (6.85005,0.768861);
\draw [c] (6.84191,0.739161) -- (6.85005,0.739161);
\draw [c] (6.85005,0.739161) -- (6.85818,0.739161);
\definecolor{c}{rgb}{0,0,0};
\colorlet{c}{natcomp!70};
\draw [c] (6.86632,0.69326) -- (6.86632,0.708575);
\draw [c] (6.86632,0.708575) -- (6.86632,0.72389);
\draw [c] (6.85818,0.708575) -- (6.86632,0.708575);
\draw [c] (6.86632,0.708575) -- (6.87445,0.708575);
\definecolor{c}{rgb}{0,0,0};
\colorlet{c}{natcomp!70};
\draw [c] (6.88259,0.703644) -- (6.88259,0.720612);
\draw [c] (6.88259,0.720612) -- (6.88259,0.73758);
\draw [c] (6.87445,0.720612) -- (6.88259,0.720612);
\draw [c] (6.88259,0.720612) -- (6.89073,0.720612);
\definecolor{c}{rgb}{0,0,0};
\colorlet{c}{natcomp!70};
\draw [c] (6.89886,0.692271) -- (6.89886,0.705485);
\draw [c] (6.89886,0.705485) -- (6.89886,0.7187);
\draw [c] (6.89073,0.705485) -- (6.89886,0.705485);
\draw [c] (6.89886,0.705485) -- (6.907,0.705485);
\definecolor{c}{rgb}{0,0,0};
\colorlet{c}{natcomp!70};
\draw [c] (6.91514,0.692587) -- (6.91514,0.706325);
\draw [c] (6.91514,0.706325) -- (6.91514,0.720063);
\draw [c] (6.907,0.706325) -- (6.91514,0.706325);
\draw [c] (6.91514,0.706325) -- (6.92327,0.706325);
\definecolor{c}{rgb}{0,0,0};
\colorlet{c}{natcomp!70};
\draw [c] (6.93141,0.686927) -- (6.93141,0.695065);
\draw [c] (6.93141,0.695065) -- (6.93141,0.703204);
\draw [c] (6.92327,0.695065) -- (6.93141,0.695065);
\draw [c] (6.93141,0.695065) -- (6.93955,0.695065);
\definecolor{c}{rgb}{0,0,0};
\colorlet{c}{natcomp!70};
\draw [c] (6.94768,0.704333) -- (6.94768,0.7221);
\draw [c] (6.94768,0.7221) -- (6.94768,0.739867);
\draw [c] (6.93955,0.7221) -- (6.94768,0.7221);
\draw [c] (6.94768,0.7221) -- (6.95582,0.7221);
\definecolor{c}{rgb}{0,0,0};
\colorlet{c}{natcomp!70};
\draw [c] (6.96395,0.686909) -- (6.96395,0.68692);
\draw [c] (6.96395,0.68692) -- (6.96395,0.686931);
\draw [c] (6.95582,0.68692) -- (6.96395,0.68692);
\draw [c] (6.96395,0.68692) -- (6.97209,0.68692);
\definecolor{c}{rgb}{0,0,0};
\colorlet{c}{natcomp!70};
\draw [c] (6.98023,0.718409) -- (6.98023,0.740294);
\draw [c] (6.98023,0.740294) -- (6.98023,0.762179);
\draw [c] (6.97209,0.740294) -- (6.98023,0.740294);
\draw [c] (6.98023,0.740294) -- (6.98836,0.740294);
\definecolor{c}{rgb}{0,0,0};
\colorlet{c}{natcomp!70};
\draw [c] (6.9965,0.686917) -- (6.9965,0.68693);
\draw [c] (6.9965,0.68693) -- (6.9965,0.686943);
\draw [c] (6.98836,0.68693) -- (6.9965,0.68693);
\draw [c] (6.9965,0.68693) -- (7.00464,0.68693);
\definecolor{c}{rgb}{0,0,0};
\colorlet{c}{natcomp!70};
\draw [c] (7.01277,0.713906) -- (7.01277,0.736963);
\draw [c] (7.01277,0.736963) -- (7.01277,0.760019);
\draw [c] (7.00464,0.736963) -- (7.01277,0.736963);
\draw [c] (7.01277,0.736963) -- (7.02091,0.736963);
\definecolor{c}{rgb}{0,0,0};
\colorlet{c}{natcomp!70};
\draw [c] (7.02905,0.703869) -- (7.02905,0.721431);
\draw [c] (7.02905,0.721431) -- (7.02905,0.738992);
\draw [c] (7.02091,0.721431) -- (7.02905,0.721431);
\draw [c] (7.02905,0.721431) -- (7.03718,0.721431);
\definecolor{c}{rgb}{0,0,0};
\colorlet{c}{natcomp!70};
\draw [c] (7.04532,0.705655) -- (7.04532,0.724877);
\draw [c] (7.04532,0.724877) -- (7.04532,0.7441);
\draw [c] (7.03718,0.724877) -- (7.04532,0.724877);
\draw [c] (7.04532,0.724877) -- (7.05345,0.724877);
\definecolor{c}{rgb}{0,0,0};
\colorlet{c}{natcomp!70};
\draw [c] (7.06159,0.703801) -- (7.06159,0.72083);
\draw [c] (7.06159,0.72083) -- (7.06159,0.73786);
\draw [c] (7.05345,0.72083) -- (7.06159,0.72083);
\draw [c] (7.06159,0.72083) -- (7.06973,0.72083);
\definecolor{c}{rgb}{0,0,0};
\colorlet{c}{natcomp!70};
\draw [c] (7.07786,0.719124) -- (7.07786,0.741674);
\draw [c] (7.07786,0.741674) -- (7.07786,0.764224);
\draw [c] (7.06973,0.741674) -- (7.07786,0.741674);
\draw [c] (7.07786,0.741674) -- (7.086,0.741674);
\definecolor{c}{rgb}{0,0,0};
\colorlet{c}{natcomp!70};
\draw [c] (7.09414,0.709945) -- (7.09414,0.738016);
\draw [c] (7.09414,0.738016) -- (7.09414,0.766087);
\draw [c] (7.086,0.738016) -- (7.09414,0.738016);
\draw [c] (7.09414,0.738016) -- (7.10227,0.738016);
\definecolor{c}{rgb}{0,0,0};
\colorlet{c}{natcomp!70};
\draw [c] (7.11041,0.691149) -- (7.11041,0.701332);
\draw [c] (7.11041,0.701332) -- (7.11041,0.711516);
\draw [c] (7.10227,0.701332) -- (7.11041,0.701332);
\draw [c] (7.11041,0.701332) -- (7.11855,0.701332);
\definecolor{c}{rgb}{0,0,0};
\colorlet{c}{natcomp!70};
\draw [c] (7.12668,0.706897) -- (7.12668,0.728052);
\draw [c] (7.12668,0.728052) -- (7.12668,0.749206);
\draw [c] (7.11855,0.728052) -- (7.12668,0.728052);
\draw [c] (7.12668,0.728052) -- (7.13482,0.728052);
\definecolor{c}{rgb}{0,0,0};
\colorlet{c}{natcomp!70};
\draw [c] (7.14295,0.691954) -- (7.14295,0.704175);
\draw [c] (7.14295,0.704175) -- (7.14295,0.716395);
\draw [c] (7.13482,0.704175) -- (7.14295,0.704175);
\draw [c] (7.14295,0.704175) -- (7.15109,0.704175);
\definecolor{c}{rgb}{0,0,0};
\colorlet{c}{natcomp!70};
\draw [c] (7.15923,0.686944) -- (7.15923,0.695807);
\draw [c] (7.15923,0.695807) -- (7.15923,0.704669);
\draw [c] (7.15109,0.695807) -- (7.15923,0.695807);
\draw [c] (7.15923,0.695807) -- (7.16736,0.695807);
\definecolor{c}{rgb}{0,0,0};
\colorlet{c}{natcomp!70};
\draw [c] (7.1755,0.713402) -- (7.1755,0.735241);
\draw [c] (7.1755,0.735241) -- (7.1755,0.75708);
\draw [c] (7.16736,0.735241) -- (7.1755,0.735241);
\draw [c] (7.1755,0.735241) -- (7.18364,0.735241);
\definecolor{c}{rgb}{0,0,0};
\colorlet{c}{natcomp!70};
\draw [c] (7.19177,0.70731) -- (7.19177,0.72784);
\draw [c] (7.19177,0.72784) -- (7.19177,0.74837);
\draw [c] (7.18364,0.72784) -- (7.19177,0.72784);
\draw [c] (7.19177,0.72784) -- (7.19991,0.72784);
\definecolor{c}{rgb}{0,0,0};
\colorlet{c}{natcomp!70};
\draw [c] (7.20805,0.694873) -- (7.20805,0.730967);
\draw [c] (7.20805,0.730967) -- (7.20805,0.767061);
\draw [c] (7.19991,0.730967) -- (7.20805,0.730967);
\draw [c] (7.20805,0.730967) -- (7.21618,0.730967);
\definecolor{c}{rgb}{0,0,0};
\colorlet{c}{natcomp!70};
\draw [c] (7.22432,0.692201) -- (7.22432,0.704986);
\draw [c] (7.22432,0.704986) -- (7.22432,0.71777);
\draw [c] (7.21618,0.704986) -- (7.22432,0.704986);
\draw [c] (7.22432,0.704986) -- (7.23245,0.704986);
\definecolor{c}{rgb}{0,0,0};
\colorlet{c}{natcomp!70};
\draw [c] (7.24059,0.694145) -- (7.24059,0.712713);
\draw [c] (7.24059,0.712713) -- (7.24059,0.731281);
\draw [c] (7.23245,0.712713) -- (7.24059,0.712713);
\draw [c] (7.24059,0.712713) -- (7.24873,0.712713);
\definecolor{c}{rgb}{0,0,0};
\colorlet{c}{natcomp!70};
\draw [c] (7.25686,0.686911) -- (7.25686,0.696805);
\draw [c] (7.25686,0.696805) -- (7.25686,0.706699);
\draw [c] (7.24873,0.696805) -- (7.25686,0.696805);
\draw [c] (7.25686,0.696805) -- (7.265,0.696805);
\definecolor{c}{rgb}{0,0,0};
\colorlet{c}{natcomp!70};
\draw [c] (7.27314,0.686939) -- (7.27314,0.698171);
\draw [c] (7.27314,0.698171) -- (7.27314,0.709404);
\draw [c] (7.265,0.698171) -- (7.27314,0.698171);
\draw [c] (7.27314,0.698171) -- (7.28127,0.698171);
\definecolor{c}{rgb}{0,0,0};
\colorlet{c}{natcomp!70};
\draw [c] (7.28941,0.705817) -- (7.28941,0.724823);
\draw [c] (7.28941,0.724823) -- (7.28941,0.743829);
\draw [c] (7.28127,0.724823) -- (7.28941,0.724823);
\draw [c] (7.28941,0.724823) -- (7.29755,0.724823);
\definecolor{c}{rgb}{0,0,0};
\colorlet{c}{natcomp!70};
\draw [c] (7.30568,0.704012) -- (7.30568,0.721633);
\draw [c] (7.30568,0.721633) -- (7.30568,0.739254);
\draw [c] (7.29755,0.721633) -- (7.30568,0.721633);
\draw [c] (7.30568,0.721633) -- (7.31382,0.721633);
\definecolor{c}{rgb}{0,0,0};
\colorlet{c}{natcomp!70};
\draw [c] (7.32195,0.714809) -- (7.32195,0.753099);
\draw [c] (7.32195,0.753099) -- (7.32195,0.791389);
\draw [c] (7.31382,0.753099) -- (7.32195,0.753099);
\draw [c] (7.32195,0.753099) -- (7.33009,0.753099);
\definecolor{c}{rgb}{0,0,0};
\colorlet{c}{natcomp!70};
\draw [c] (7.33823,0.697923) -- (7.33823,0.713018);
\draw [c] (7.33823,0.713018) -- (7.33823,0.728113);
\draw [c] (7.33009,0.713018) -- (7.33823,0.713018);
\draw [c] (7.33823,0.713018) -- (7.34636,0.713018);
\definecolor{c}{rgb}{0,0,0};
\colorlet{c}{natcomp!70};
\draw [c] (7.3545,0.698691) -- (7.3545,0.715314);
\draw [c] (7.3545,0.715314) -- (7.3545,0.731936);
\draw [c] (7.34636,0.715314) -- (7.3545,0.715314);
\draw [c] (7.3545,0.715314) -- (7.36264,0.715314);
\definecolor{c}{rgb}{0,0,0};
\colorlet{c}{natcomp!70};
\draw [c] (7.37077,0.717852) -- (7.37077,0.745539);
\draw [c] (7.37077,0.745539) -- (7.37077,0.773225);
\draw [c] (7.36264,0.745539) -- (7.37077,0.745539);
\draw [c] (7.37077,0.745539) -- (7.37891,0.745539);
\definecolor{c}{rgb}{0,0,0};
\colorlet{c}{natcomp!70};
\draw [c] (7.38705,0.701315) -- (7.38705,0.726433);
\draw [c] (7.38705,0.726433) -- (7.38705,0.751551);
\draw [c] (7.37891,0.726433) -- (7.38705,0.726433);
\draw [c] (7.38705,0.726433) -- (7.39518,0.726433);
\definecolor{c}{rgb}{0,0,0};
\colorlet{c}{natcomp!70};
\draw [c] (7.40332,0.719987) -- (7.40332,0.743126);
\draw [c] (7.40332,0.743126) -- (7.40332,0.766266);
\draw [c] (7.39518,0.743126) -- (7.40332,0.743126);
\draw [c] (7.40332,0.743126) -- (7.41145,0.743126);
\definecolor{c}{rgb}{0,0,0};
\colorlet{c}{natcomp!70};
\draw [c] (7.41959,0.691933) -- (7.41959,0.704153);
\draw [c] (7.41959,0.704153) -- (7.41959,0.716373);
\draw [c] (7.41145,0.704153) -- (7.41959,0.704153);
\draw [c] (7.41959,0.704153) -- (7.42773,0.704153);
\definecolor{c}{rgb}{0,0,0};
\colorlet{c}{natcomp!70};
\draw [c] (7.45214,0.686902) -- (7.45214,0.695866);
\draw [c] (7.45214,0.695866) -- (7.45214,0.704829);
\draw [c] (7.444,0.695866) -- (7.45214,0.695866);
\draw [c] (7.45214,0.695866) -- (7.46027,0.695866);
\definecolor{c}{rgb}{0,0,0};
\colorlet{c}{natcomp!70};
\draw [c] (7.46841,0.686921) -- (7.46841,0.695885);
\draw [c] (7.46841,0.695885) -- (7.46841,0.704848);
\draw [c] (7.46027,0.695885) -- (7.46841,0.695885);
\draw [c] (7.46841,0.695885) -- (7.47655,0.695885);
\definecolor{c}{rgb}{0,0,0};
\colorlet{c}{natcomp!70};
\draw [c] (7.48468,0.686897) -- (7.48468,0.686905);
\draw [c] (7.48468,0.686905) -- (7.48468,0.686914);
\draw [c] (7.47655,0.686905) -- (7.48468,0.686905);
\draw [c] (7.48468,0.686905) -- (7.49282,0.686905);
\definecolor{c}{rgb}{0,0,0};
\colorlet{c}{natcomp!70};
\draw [c] (7.50095,0.686894) -- (7.50095,0.686899);
\draw [c] (7.50095,0.686899) -- (7.50095,0.686904);
\draw [c] (7.49282,0.686899) -- (7.50095,0.686899);
\draw [c] (7.50095,0.686899) -- (7.50909,0.686899);
\definecolor{c}{rgb}{0,0,0};
\colorlet{c}{natcomp!70};
\draw [c] (7.51723,0.702942) -- (7.51723,0.719147);
\draw [c] (7.51723,0.719147) -- (7.51723,0.735351);
\draw [c] (7.50909,0.719147) -- (7.51723,0.719147);
\draw [c] (7.51723,0.719147) -- (7.52536,0.719147);
\definecolor{c}{rgb}{0,0,0};
\colorlet{c}{natcomp!70};
\draw [c] (7.5335,0.691582) -- (7.5335,0.703001);
\draw [c] (7.5335,0.703001) -- (7.5335,0.71442);
\draw [c] (7.52536,0.703001) -- (7.5335,0.703001);
\draw [c] (7.5335,0.703001) -- (7.54164,0.703001);
\definecolor{c}{rgb}{0,0,0};
\colorlet{c}{natcomp!70};
\draw [c] (7.54977,0.699339) -- (7.54977,0.718609);
\draw [c] (7.54977,0.718609) -- (7.54977,0.737879);
\draw [c] (7.54164,0.718609) -- (7.54977,0.718609);
\draw [c] (7.54977,0.718609) -- (7.55791,0.718609);
\definecolor{c}{rgb}{0,0,0};
\colorlet{c}{natcomp!70};
\draw [c] (7.56605,0.705399) -- (7.56605,0.728064);
\draw [c] (7.56605,0.728064) -- (7.56605,0.750729);
\draw [c] (7.55791,0.728064) -- (7.56605,0.728064);
\draw [c] (7.56605,0.728064) -- (7.57418,0.728064);
\definecolor{c}{rgb}{0,0,0};
\colorlet{c}{natcomp!70};
\draw [c] (7.58232,0.699404) -- (7.58232,0.723597);
\draw [c] (7.58232,0.723597) -- (7.58232,0.74779);
\draw [c] (7.57418,0.723597) -- (7.58232,0.723597);
\draw [c] (7.58232,0.723597) -- (7.59045,0.723597);
\definecolor{c}{rgb}{0,0,0};
\colorlet{c}{natcomp!70};
\draw [c] (7.59859,0.686912) -- (7.59859,0.694113);
\draw [c] (7.59859,0.694113) -- (7.59859,0.701313);
\draw [c] (7.59045,0.694113) -- (7.59859,0.694113);
\draw [c] (7.59859,0.694113) -- (7.60673,0.694113);
\definecolor{c}{rgb}{0,0,0};
\colorlet{c}{natcomp!70};
\draw [c] (7.61486,0.686896) -- (7.61486,0.686902);
\draw [c] (7.61486,0.686902) -- (7.61486,0.686909);
\draw [c] (7.60673,0.686902) -- (7.61486,0.686902);
\draw [c] (7.61486,0.686902) -- (7.623,0.686902);
\definecolor{c}{rgb}{0,0,0};
\colorlet{c}{natcomp!70};
\draw [c] (7.63114,0.721485) -- (7.63114,0.74621);
\draw [c] (7.63114,0.74621) -- (7.63114,0.770934);
\draw [c] (7.623,0.74621) -- (7.63114,0.74621);
\draw [c] (7.63114,0.74621) -- (7.63927,0.74621);
\definecolor{c}{rgb}{0,0,0};
\colorlet{c}{natcomp!70};
\draw [c] (7.64741,0.686934) -- (7.64741,0.695073);
\draw [c] (7.64741,0.695073) -- (7.64741,0.703212);
\draw [c] (7.63927,0.695073) -- (7.64741,0.695073);
\draw [c] (7.64741,0.695073) -- (7.65555,0.695073);
\definecolor{c}{rgb}{0,0,0};
\colorlet{c}{natcomp!70};
\draw [c] (7.66368,0.69193) -- (7.66368,0.704888);
\draw [c] (7.66368,0.704888) -- (7.66368,0.717847);
\draw [c] (7.65555,0.704888) -- (7.66368,0.704888);
\draw [c] (7.66368,0.704888) -- (7.67182,0.704888);
\definecolor{c}{rgb}{0,0,0};
\colorlet{c}{natcomp!70};
\draw [c] (7.67995,0.693227) -- (7.67995,0.708478);
\draw [c] (7.67995,0.708478) -- (7.67995,0.723729);
\draw [c] (7.67182,0.708478) -- (7.67995,0.708478);
\draw [c] (7.67995,0.708478) -- (7.68809,0.708478);
\definecolor{c}{rgb}{0,0,0};
\colorlet{c}{natcomp!70};
\draw [c] (7.69623,0.705019) -- (7.69623,0.723124);
\draw [c] (7.69623,0.723124) -- (7.69623,0.74123);
\draw [c] (7.68809,0.723124) -- (7.69623,0.723124);
\draw [c] (7.69623,0.723124) -- (7.70436,0.723124);
\definecolor{c}{rgb}{0,0,0};
\colorlet{c}{natcomp!70};
\draw [c] (7.7125,0.686901) -- (7.7125,0.686914);
\draw [c] (7.7125,0.686914) -- (7.7125,0.686926);
\draw [c] (7.70436,0.686914) -- (7.7125,0.686914);
\draw [c] (7.7125,0.686914) -- (7.72064,0.686914);
\definecolor{c}{rgb}{0,0,0};
\colorlet{c}{natcomp!70};
\draw [c] (7.72877,0.686911) -- (7.72877,0.697705);
\draw [c] (7.72877,0.697705) -- (7.72877,0.7085);
\draw [c] (7.72064,0.697705) -- (7.72877,0.697705);
\draw [c] (7.72877,0.697705) -- (7.73691,0.697705);
\definecolor{c}{rgb}{0,0,0};
\colorlet{c}{natcomp!70};
\draw [c] (7.74505,0.697486) -- (7.74505,0.712097);
\draw [c] (7.74505,0.712097) -- (7.74505,0.726708);
\draw [c] (7.73691,0.712097) -- (7.74505,0.712097);
\draw [c] (7.74505,0.712097) -- (7.75318,0.712097);
\definecolor{c}{rgb}{0,0,0};
\colorlet{c}{natcomp!70};
\draw [c] (7.76132,0.692362) -- (7.76132,0.708107);
\draw [c] (7.76132,0.708107) -- (7.76132,0.723852);
\draw [c] (7.75318,0.708107) -- (7.76132,0.708107);
\draw [c] (7.76132,0.708107) -- (7.76945,0.708107);
\definecolor{c}{rgb}{0,0,0};
\colorlet{c}{natcomp!70};
\draw [c] (7.77759,0.699709) -- (7.77759,0.717404);
\draw [c] (7.77759,0.717404) -- (7.77759,0.735098);
\draw [c] (7.76945,0.717404) -- (7.77759,0.717404);
\draw [c] (7.77759,0.717404) -- (7.78573,0.717404);
\definecolor{c}{rgb}{0,0,0};
\colorlet{c}{natcomp!70};
\draw [c] (7.79386,0.733221) -- (7.79386,0.759074);
\draw [c] (7.79386,0.759074) -- (7.79386,0.784928);
\draw [c] (7.78573,0.759074) -- (7.79386,0.759074);
\draw [c] (7.79386,0.759074) -- (7.802,0.759074);
\definecolor{c}{rgb}{0,0,0};
\colorlet{c}{natcomp!70};
\draw [c] (7.81014,0.697974) -- (7.81014,0.713528);
\draw [c] (7.81014,0.713528) -- (7.81014,0.729082);
\draw [c] (7.802,0.713528) -- (7.81014,0.713528);
\draw [c] (7.81014,0.713528) -- (7.81827,0.713528);
\definecolor{c}{rgb}{0,0,0};
\colorlet{c}{natcomp!70};
\draw [c] (7.82641,0.692082) -- (7.82641,0.705044);
\draw [c] (7.82641,0.705044) -- (7.82641,0.718006);
\draw [c] (7.81827,0.705044) -- (7.82641,0.705044);
\draw [c] (7.82641,0.705044) -- (7.83455,0.705044);
\definecolor{c}{rgb}{0,0,0};
\colorlet{c}{natcomp!70};
\draw [c] (7.84268,0.69139) -- (7.84268,0.702257);
\draw [c] (7.84268,0.702257) -- (7.84268,0.713124);
\draw [c] (7.83455,0.702257) -- (7.84268,0.702257);
\draw [c] (7.84268,0.702257) -- (7.85082,0.702257);
\definecolor{c}{rgb}{0,0,0};
\colorlet{c}{natcomp!70};
\draw [c] (7.85895,0.68692) -- (7.85895,0.696815);
\draw [c] (7.85895,0.696815) -- (7.85895,0.706709);
\draw [c] (7.85082,0.696815) -- (7.85895,0.696815);
\draw [c] (7.85895,0.696815) -- (7.86709,0.696815);
\definecolor{c}{rgb}{0,0,0};
\colorlet{c}{natcomp!70};
\draw [c] (7.87523,0.699254) -- (7.87523,0.718923);
\draw [c] (7.87523,0.718923) -- (7.87523,0.738592);
\draw [c] (7.86709,0.718923) -- (7.87523,0.718923);
\draw [c] (7.87523,0.718923) -- (7.88336,0.718923);
\definecolor{c}{rgb}{0,0,0};
\colorlet{c}{natcomp!70};
\draw [c] (7.8915,0.711611) -- (7.8915,0.732071);
\draw [c] (7.8915,0.732071) -- (7.8915,0.752532);
\draw [c] (7.88336,0.732071) -- (7.8915,0.732071);
\draw [c] (7.8915,0.732071) -- (7.89964,0.732071);
\definecolor{c}{rgb}{0,0,0};
\colorlet{c}{natcomp!70};
\draw [c] (7.90777,0.715085) -- (7.90777,0.738825);
\draw [c] (7.90777,0.738825) -- (7.90777,0.762565);
\draw [c] (7.89964,0.738825) -- (7.90777,0.738825);
\draw [c] (7.90777,0.738825) -- (7.91591,0.738825);
\definecolor{c}{rgb}{0,0,0};
\colorlet{c}{natcomp!70};
\draw [c] (7.92405,0.686922) -- (7.92405,0.696038);
\draw [c] (7.92405,0.696038) -- (7.92405,0.705153);
\draw [c] (7.91591,0.696038) -- (7.92405,0.696038);
\draw [c] (7.92405,0.696038) -- (7.93218,0.696038);
\definecolor{c}{rgb}{0,0,0};
\colorlet{c}{natcomp!70};
\draw [c] (7.94032,0.686907) -- (7.94032,0.696802);
\draw [c] (7.94032,0.696802) -- (7.94032,0.706696);
\draw [c] (7.93218,0.696802) -- (7.94032,0.696802);
\draw [c] (7.94032,0.696802) -- (7.94845,0.696802);
\definecolor{c}{rgb}{0,0,0};
\colorlet{c}{natcomp!70};
\draw [c] (7.95659,0.691941) -- (7.95659,0.704161);
\draw [c] (7.95659,0.704161) -- (7.95659,0.716381);
\draw [c] (7.94845,0.704161) -- (7.95659,0.704161);
\draw [c] (7.95659,0.704161) -- (7.96473,0.704161);
\definecolor{c}{rgb}{0,0,0};
\colorlet{c}{natcomp!70};
\draw [c] (7.97286,0.686912) -- (7.97286,0.69505);
\draw [c] (7.97286,0.69505) -- (7.97286,0.703189);
\draw [c] (7.96473,0.69505) -- (7.97286,0.69505);
\draw [c] (7.97286,0.69505) -- (7.981,0.69505);
\definecolor{c}{rgb}{0,0,0};
\colorlet{c}{natcomp!70};
\draw [c] (7.98914,0.693662) -- (7.98914,0.710807);
\draw [c] (7.98914,0.710807) -- (7.98914,0.727952);
\draw [c] (7.981,0.710807) -- (7.98914,0.710807);
\draw [c] (7.98914,0.710807) -- (7.99727,0.710807);
\definecolor{c}{rgb}{0,0,0};
\colorlet{c}{natcomp!70};
\draw [c] (8.00541,0.686922) -- (8.00541,0.695785);
\draw [c] (8.00541,0.695785) -- (8.00541,0.704647);
\draw [c] (7.99727,0.695785) -- (8.00541,0.695785);
\draw [c] (8.00541,0.695785) -- (8.01355,0.695785);
\definecolor{c}{rgb}{0,0,0};
\colorlet{c}{natcomp!70};
\draw [c] (8.02168,0.691552) -- (8.02168,0.702771);
\draw [c] (8.02168,0.702771) -- (8.02168,0.71399);
\draw [c] (8.01355,0.702771) -- (8.02168,0.702771);
\draw [c] (8.02168,0.702771) -- (8.02982,0.702771);
\definecolor{c}{rgb}{0,0,0};
\colorlet{c}{natcomp!70};
\draw [c] (8.03795,0.686908) -- (8.03795,0.70091);
\draw [c] (8.03795,0.70091) -- (8.03795,0.714913);
\draw [c] (8.02982,0.70091) -- (8.03795,0.70091);
\draw [c] (8.03795,0.70091) -- (8.04609,0.70091);
\definecolor{c}{rgb}{0,0,0};
\colorlet{c}{natcomp!70};
\draw [c] (8.05423,0.686902) -- (8.05423,0.695765);
\draw [c] (8.05423,0.695765) -- (8.05423,0.704627);
\draw [c] (8.04609,0.695765) -- (8.05423,0.695765);
\draw [c] (8.05423,0.695765) -- (8.06236,0.695765);
\definecolor{c}{rgb}{0,0,0};
\colorlet{c}{natcomp!70};
\draw [c] (8.0705,0.686913) -- (8.0705,0.694635);
\draw [c] (8.0705,0.694635) -- (8.0705,0.702356);
\draw [c] (8.06236,0.694635) -- (8.0705,0.694635);
\draw [c] (8.0705,0.694635) -- (8.07864,0.694635);
\definecolor{c}{rgb}{0,0,0};
\colorlet{c}{natcomp!70};
\draw [c] (8.08677,0.691886) -- (8.08677,0.703918);
\draw [c] (8.08677,0.703918) -- (8.08677,0.715951);
\draw [c] (8.07864,0.703918) -- (8.08677,0.703918);
\draw [c] (8.08677,0.703918) -- (8.09491,0.703918);
\definecolor{c}{rgb}{0,0,0};
\colorlet{c}{natcomp!70};
\draw [c] (8.10305,0.699036) -- (8.10305,0.715859);
\draw [c] (8.10305,0.715859) -- (8.10305,0.732682);
\draw [c] (8.09491,0.715859) -- (8.10305,0.715859);
\draw [c] (8.10305,0.715859) -- (8.11118,0.715859);
\definecolor{c}{rgb}{0,0,0};
\colorlet{c}{natcomp!70};
\draw [c] (8.11932,0.70093) -- (8.11932,0.723385);
\draw [c] (8.11932,0.723385) -- (8.11932,0.74584);
\draw [c] (8.11118,0.723385) -- (8.11932,0.723385);
\draw [c] (8.11932,0.723385) -- (8.12745,0.723385);
\definecolor{c}{rgb}{0,0,0};
\colorlet{c}{natcomp!70};
\draw [c] (8.13559,0.691991) -- (8.13559,0.704542);
\draw [c] (8.13559,0.704542) -- (8.13559,0.717092);
\draw [c] (8.12745,0.704542) -- (8.13559,0.704542);
\draw [c] (8.13559,0.704542) -- (8.14373,0.704542);
\definecolor{c}{rgb}{0,0,0};
\colorlet{c}{natcomp!70};
\draw [c] (8.15186,0.704381) -- (8.15186,0.722348);
\draw [c] (8.15186,0.722348) -- (8.15186,0.740315);
\draw [c] (8.14373,0.722348) -- (8.15186,0.722348);
\draw [c] (8.15186,0.722348) -- (8.16,0.722348);
\definecolor{c}{rgb}{0,0,0};
\colorlet{c}{natcomp!70};
\draw [c] (8.16814,0.692514) -- (8.16814,0.706186);
\draw [c] (8.16814,0.706186) -- (8.16814,0.719858);
\draw [c] (8.16,0.706186) -- (8.16814,0.706186);
\draw [c] (8.16814,0.706186) -- (8.17627,0.706186);
\definecolor{c}{rgb}{0,0,0};
\colorlet{c}{natcomp!70};
\draw [c] (8.18441,0.686908) -- (8.18441,0.697319);
\draw [c] (8.18441,0.697319) -- (8.18441,0.70773);
\draw [c] (8.17627,0.697319) -- (8.18441,0.697319);
\draw [c] (8.18441,0.697319) -- (8.19255,0.697319);
\definecolor{c}{rgb}{0,0,0};
\colorlet{c}{natcomp!70};
\draw [c] (8.20068,0.686908) -- (8.20068,0.708675);
\draw [c] (8.20068,0.708675) -- (8.20068,0.730443);
\draw [c] (8.19255,0.708675) -- (8.20068,0.708675);
\draw [c] (8.20068,0.708675) -- (8.20882,0.708675);
\definecolor{c}{rgb}{0,0,0};
\colorlet{c}{natcomp!70};
\draw [c] (8.21695,0.691366) -- (8.21695,0.702233);
\draw [c] (8.21695,0.702233) -- (8.21695,0.7131);
\draw [c] (8.20882,0.702233) -- (8.21695,0.702233);
\draw [c] (8.21695,0.702233) -- (8.22509,0.702233);
\definecolor{c}{rgb}{0,0,0};
\colorlet{c}{natcomp!70};
\draw [c] (8.23323,0.700598) -- (8.23323,0.719335);
\draw [c] (8.23323,0.719335) -- (8.23323,0.738073);
\draw [c] (8.22509,0.719335) -- (8.23323,0.719335);
\draw [c] (8.23323,0.719335) -- (8.24136,0.719335);
\definecolor{c}{rgb}{0,0,0};
\colorlet{c}{natcomp!70};
\draw [c] (8.2495,0.698239) -- (8.2495,0.713998);
\draw [c] (8.2495,0.713998) -- (8.2495,0.729758);
\draw [c] (8.24136,0.713998) -- (8.2495,0.713998);
\draw [c] (8.2495,0.713998) -- (8.25764,0.713998);
\definecolor{c}{rgb}{0,0,0};
\colorlet{c}{natcomp!70};
\draw [c] (8.26577,0.704207) -- (8.26577,0.72189);
\draw [c] (8.26577,0.72189) -- (8.26577,0.739573);
\draw [c] (8.25764,0.72189) -- (8.26577,0.72189);
\draw [c] (8.26577,0.72189) -- (8.27391,0.72189);
\definecolor{c}{rgb}{0,0,0};
\colorlet{c}{natcomp!70};
\draw [c] (8.28205,0.699406) -- (8.28205,0.719236);
\draw [c] (8.28205,0.719236) -- (8.28205,0.739065);
\draw [c] (8.27391,0.719236) -- (8.28205,0.719236);
\draw [c] (8.28205,0.719236) -- (8.29018,0.719236);
\definecolor{c}{rgb}{0,0,0};
\colorlet{c}{natcomp!70};
\draw [c] (8.29832,0.686897) -- (8.29832,0.686904);
\draw [c] (8.29832,0.686904) -- (8.29832,0.686911);
\draw [c] (8.29018,0.686904) -- (8.29832,0.686904);
\draw [c] (8.29832,0.686904) -- (8.30645,0.686904);
\definecolor{c}{rgb}{0,0,0};
\colorlet{c}{natcomp!70};
\draw [c] (8.31459,0.692156) -- (8.31459,0.704967);
\draw [c] (8.31459,0.704967) -- (8.31459,0.717779);
\draw [c] (8.30645,0.704967) -- (8.31459,0.704967);
\draw [c] (8.31459,0.704967) -- (8.32273,0.704967);
\definecolor{c}{rgb}{0,0,0};
\colorlet{c}{natcomp!70};
\draw [c] (8.33086,0.686894) -- (8.33086,0.686898);
\draw [c] (8.33086,0.686898) -- (8.33086,0.686902);
\draw [c] (8.32273,0.686898) -- (8.33086,0.686898);
\draw [c] (8.33086,0.686898) -- (8.339,0.686898);
\definecolor{c}{rgb}{0,0,0};
\colorlet{c}{natcomp!70};
\draw [c] (8.34714,0.686913) -- (8.34714,0.696808);
\draw [c] (8.34714,0.696808) -- (8.34714,0.706702);
\draw [c] (8.339,0.696808) -- (8.34714,0.696808);
\draw [c] (8.34714,0.696808) -- (8.35527,0.696808);
\definecolor{c}{rgb}{0,0,0};
\colorlet{c}{natcomp!70};
\draw [c] (8.36341,0.6869) -- (8.36341,0.686908);
\draw [c] (8.36341,0.686908) -- (8.36341,0.686916);
\draw [c] (8.35527,0.686908) -- (8.36341,0.686908);
\draw [c] (8.36341,0.686908) -- (8.37155,0.686908);
\definecolor{c}{rgb}{0,0,0};
\colorlet{c}{natcomp!70};
\draw [c] (8.37968,0.709537) -- (8.37968,0.733883);
\draw [c] (8.37968,0.733883) -- (8.37968,0.758228);
\draw [c] (8.37155,0.733883) -- (8.37968,0.733883);
\draw [c] (8.37968,0.733883) -- (8.38782,0.733883);
\definecolor{c}{rgb}{0,0,0};
\colorlet{c}{natcomp!70};
\draw [c] (8.39595,0.686897) -- (8.39595,0.696792);
\draw [c] (8.39595,0.696792) -- (8.39595,0.706686);
\draw [c] (8.38782,0.696792) -- (8.39595,0.696792);
\draw [c] (8.39595,0.696792) -- (8.40409,0.696792);
\definecolor{c}{rgb}{0,0,0};
\colorlet{c}{natcomp!70};
\draw [c] (8.41223,0.686907) -- (8.41223,0.696023);
\draw [c] (8.41223,0.696023) -- (8.41223,0.705138);
\draw [c] (8.40409,0.696023) -- (8.41223,0.696023);
\draw [c] (8.41223,0.696023) -- (8.42036,0.696023);
\definecolor{c}{rgb}{0,0,0};
\colorlet{c}{natcomp!70};
\draw [c] (8.4285,0.691882) -- (8.4285,0.703914);
\draw [c] (8.4285,0.703914) -- (8.4285,0.715947);
\draw [c] (8.42036,0.703914) -- (8.4285,0.703914);
\draw [c] (8.4285,0.703914) -- (8.43664,0.703914);
\definecolor{c}{rgb}{0,0,0};
\colorlet{c}{natcomp!70};
\draw [c] (8.44477,0.686908) -- (8.44477,0.69814);
\draw [c] (8.44477,0.69814) -- (8.44477,0.709372);
\draw [c] (8.43664,0.69814) -- (8.44477,0.69814);
\draw [c] (8.44477,0.69814) -- (8.45291,0.69814);
\definecolor{c}{rgb}{0,0,0};
\colorlet{c}{natcomp!70};
\draw [c] (8.46105,0.686896) -- (8.46105,0.686903);
\draw [c] (8.46105,0.686903) -- (8.46105,0.686909);
\draw [c] (8.45291,0.686903) -- (8.46105,0.686903);
\draw [c] (8.46105,0.686903) -- (8.46918,0.686903);
\definecolor{c}{rgb}{0,0,0};
\colorlet{c}{natcomp!70};
\draw [c] (8.47732,0.686902) -- (8.47732,0.686911);
\draw [c] (8.47732,0.686911) -- (8.47732,0.686919);
\draw [c] (8.46918,0.686911) -- (8.47732,0.686911);
\draw [c] (8.47732,0.686911) -- (8.48545,0.686911);
\definecolor{c}{rgb}{0,0,0};
\colorlet{c}{natcomp!70};
\draw [c] (8.49359,0.686915) -- (8.49359,0.712075);
\draw [c] (8.49359,0.712075) -- (8.49359,0.737235);
\draw [c] (8.48545,0.712075) -- (8.49359,0.712075);
\draw [c] (8.49359,0.712075) -- (8.50173,0.712075);
\definecolor{c}{rgb}{0,0,0};
\colorlet{c}{natcomp!70};
\draw [c] (8.50986,0.686899) -- (8.50986,0.686907);
\draw [c] (8.50986,0.686907) -- (8.50986,0.686915);
\draw [c] (8.50173,0.686907) -- (8.50986,0.686907);
\draw [c] (8.50986,0.686907) -- (8.518,0.686907);
\definecolor{c}{rgb}{0,0,0};
\colorlet{c}{natcomp!70};
\draw [c] (8.52614,0.686912) -- (8.52614,0.694113);
\draw [c] (8.52614,0.694113) -- (8.52614,0.701314);
\draw [c] (8.518,0.694113) -- (8.52614,0.694113);
\draw [c] (8.52614,0.694113) -- (8.53427,0.694113);
\definecolor{c}{rgb}{0,0,0};
\colorlet{c}{natcomp!70};
\draw [c] (8.54241,0.70362) -- (8.54241,0.720773);
\draw [c] (8.54241,0.720773) -- (8.54241,0.737927);
\draw [c] (8.53427,0.720773) -- (8.54241,0.720773);
\draw [c] (8.54241,0.720773) -- (8.55055,0.720773);
\definecolor{c}{rgb}{0,0,0};
\colorlet{c}{natcomp!70};
\draw [c] (8.55868,0.686902) -- (8.55868,0.697697);
\draw [c] (8.55868,0.697697) -- (8.55868,0.708492);
\draw [c] (8.55055,0.697697) -- (8.55868,0.697697);
\draw [c] (8.55868,0.697697) -- (8.56682,0.697697);
\definecolor{c}{rgb}{0,0,0};
\colorlet{c}{natcomp!70};
\draw [c] (8.57495,0.692719) -- (8.57495,0.70709);
\draw [c] (8.57495,0.70709) -- (8.57495,0.721461);
\draw [c] (8.56682,0.70709) -- (8.57495,0.70709);
\draw [c] (8.57495,0.70709) -- (8.58309,0.70709);
\definecolor{c}{rgb}{0,0,0};
\colorlet{c}{natcomp!70};
\draw [c] (8.59123,0.686902) -- (8.59123,0.686909);
\draw [c] (8.59123,0.686909) -- (8.59123,0.686917);
\draw [c] (8.58309,0.686909) -- (8.59123,0.686909);
\draw [c] (8.59123,0.686909) -- (8.59936,0.686909);
\definecolor{c}{rgb}{0,0,0};
\colorlet{c}{natcomp!70};
\draw [c] (8.6075,0.686894) -- (8.6075,0.686897);
\draw [c] (8.6075,0.686897) -- (8.6075,0.686901);
\draw [c] (8.59936,0.686897) -- (8.6075,0.686897);
\draw [c] (8.6075,0.686897) -- (8.61564,0.686897);
\definecolor{c}{rgb}{0,0,0};
\colorlet{c}{natcomp!70};
\draw [c] (8.62377,0.686903) -- (8.62377,0.686912);
\draw [c] (8.62377,0.686912) -- (8.62377,0.686922);
\draw [c] (8.61564,0.686912) -- (8.62377,0.686912);
\draw [c] (8.62377,0.686912) -- (8.63191,0.686912);
\definecolor{c}{rgb}{0,0,0};
\colorlet{c}{natcomp!70};
\draw [c] (8.64005,0.706296) -- (8.64005,0.725795);
\draw [c] (8.64005,0.725795) -- (8.64005,0.745294);
\draw [c] (8.63191,0.725795) -- (8.64005,0.725795);
\draw [c] (8.64005,0.725795) -- (8.64818,0.725795);
\definecolor{c}{rgb}{0,0,0};
\colorlet{c}{natcomp!70};
\draw [c] (8.65632,0.686902) -- (8.65632,0.697676);
\draw [c] (8.65632,0.697676) -- (8.65632,0.708449);
\draw [c] (8.64818,0.697676) -- (8.65632,0.697676);
\draw [c] (8.65632,0.697676) -- (8.66445,0.697676);
\definecolor{c}{rgb}{0,0,0};
\colorlet{c}{natcomp!70};
\draw [c] (8.67259,0.697669) -- (8.67259,0.712756);
\draw [c] (8.67259,0.712756) -- (8.67259,0.727844);
\draw [c] (8.66445,0.712756) -- (8.67259,0.712756);
\draw [c] (8.67259,0.712756) -- (8.68073,0.712756);
\definecolor{c}{rgb}{0,0,0};
\colorlet{c}{natcomp!70};
\draw [c] (8.68886,0.686917) -- (8.68886,0.695056);
\draw [c] (8.68886,0.695056) -- (8.68886,0.703195);
\draw [c] (8.68073,0.695056) -- (8.68886,0.695056);
\draw [c] (8.68886,0.695056) -- (8.697,0.695056);
\definecolor{c}{rgb}{0,0,0};
\colorlet{c}{natcomp!70};
\draw [c] (8.70514,0.686894) -- (8.70514,0.686898);
\draw [c] (8.70514,0.686898) -- (8.70514,0.686902);
\draw [c] (8.697,0.686898) -- (8.70514,0.686898);
\draw [c] (8.70514,0.686898) -- (8.71327,0.686898);
\definecolor{c}{rgb}{0,0,0};
\colorlet{c}{natcomp!70};
\draw [c] (8.72141,0.686896) -- (8.72141,0.686903);
\draw [c] (8.72141,0.686903) -- (8.72141,0.686909);
\draw [c] (8.71327,0.686903) -- (8.72141,0.686903);
\draw [c] (8.72141,0.686903) -- (8.72955,0.686903);
\definecolor{c}{rgb}{0,0,0};
\colorlet{c}{natcomp!70};
\draw [c] (8.73768,0.6869) -- (8.73768,0.686909);
\draw [c] (8.73768,0.686909) -- (8.73768,0.686917);
\draw [c] (8.72955,0.686909) -- (8.73768,0.686909);
\draw [c] (8.73768,0.686909) -- (8.74582,0.686909);
\definecolor{c}{rgb}{0,0,0};
\colorlet{c}{natcomp!70};
\draw [c] (8.75395,0.686898) -- (8.75395,0.697671);
\draw [c] (8.75395,0.697671) -- (8.75395,0.708445);
\draw [c] (8.74582,0.697671) -- (8.75395,0.697671);
\draw [c] (8.75395,0.697671) -- (8.76209,0.697671);
\definecolor{c}{rgb}{0,0,0};
\colorlet{c}{natcomp!70};
\draw [c] (8.77023,0.686896) -- (8.77023,0.686902);
\draw [c] (8.77023,0.686902) -- (8.77023,0.686908);
\draw [c] (8.76209,0.686902) -- (8.77023,0.686902);
\draw [c] (8.77023,0.686902) -- (8.77836,0.686902);
\definecolor{c}{rgb}{0,0,0};
\colorlet{c}{natcomp!70};
\draw [c] (8.80277,0.686894) -- (8.80277,0.697305);
\draw [c] (8.80277,0.697305) -- (8.80277,0.707716);
\draw [c] (8.79464,0.697305) -- (8.80277,0.697305);
\draw [c] (8.80277,0.697305) -- (8.81091,0.697305);
\definecolor{c}{rgb}{0,0,0};
\colorlet{c}{natcomp!70};
\draw [c] (8.81905,0.686894) -- (8.81905,0.686899);
\draw [c] (8.81905,0.686899) -- (8.81905,0.686904);
\draw [c] (8.81091,0.686899) -- (8.81905,0.686899);
\draw [c] (8.81905,0.686899) -- (8.82718,0.686899);
\definecolor{c}{rgb}{0,0,0};
\colorlet{c}{natcomp!70};
\draw [c] (8.83532,0.686894) -- (8.83532,0.686899);
\draw [c] (8.83532,0.686899) -- (8.83532,0.686904);
\draw [c] (8.82718,0.686899) -- (8.83532,0.686899);
\draw [c] (8.83532,0.686899) -- (8.84345,0.686899);
\definecolor{c}{rgb}{0,0,0};
\colorlet{c}{natcomp!70};
\draw [c] (8.85159,0.686907) -- (8.85159,0.696022);
\draw [c] (8.85159,0.696022) -- (8.85159,0.705137);
\draw [c] (8.84345,0.696022) -- (8.85159,0.696022);
\draw [c] (8.85159,0.696022) -- (8.85973,0.696022);
\definecolor{c}{rgb}{0,0,0};
\colorlet{c}{natcomp!70};
\draw [c] (8.86786,0.686894) -- (8.86786,0.686897);
\draw [c] (8.86786,0.686897) -- (8.86786,0.686901);
\draw [c] (8.85973,0.686897) -- (8.86786,0.686897);
\draw [c] (8.86786,0.686897) -- (8.876,0.686897);
\definecolor{c}{rgb}{0,0,0};
\colorlet{c}{natcomp!70};
\draw [c] (8.88414,0.700826) -- (8.88414,0.720637);
\draw [c] (8.88414,0.720637) -- (8.88414,0.740448);
\draw [c] (8.876,0.720637) -- (8.88414,0.720637);
\draw [c] (8.88414,0.720637) -- (8.89227,0.720637);
\definecolor{c}{rgb}{0,0,0};
\colorlet{c}{natcomp!70};
\draw [c] (8.90041,0.686894) -- (8.90041,0.686897);
\draw [c] (8.90041,0.686897) -- (8.90041,0.686901);
\draw [c] (8.89227,0.686897) -- (8.90041,0.686897);
\draw [c] (8.90041,0.686897) -- (8.90855,0.686897);
\definecolor{c}{rgb}{0,0,0};
\colorlet{c}{natcomp!70};
\draw [c] (8.91668,0.722516) -- (8.91668,0.755184);
\draw [c] (8.91668,0.755184) -- (8.91668,0.787852);
\draw [c] (8.90855,0.755184) -- (8.91668,0.755184);
\draw [c] (8.91668,0.755184) -- (8.92482,0.755184);
\definecolor{c}{rgb}{0,0,0};
\colorlet{c}{natcomp!70};
\draw [c] (8.93295,0.6869) -- (8.93295,0.694622);
\draw [c] (8.93295,0.694622) -- (8.93295,0.702344);
\draw [c] (8.92482,0.694622) -- (8.93295,0.694622);
\draw [c] (8.93295,0.694622) -- (8.94109,0.694622);
\definecolor{c}{rgb}{0,0,0};
\colorlet{c}{natcomp!70};
\draw [c] (8.94923,0.686894) -- (8.94923,0.686898);
\draw [c] (8.94923,0.686898) -- (8.94923,0.686902);
\draw [c] (8.94109,0.686898) -- (8.94923,0.686898);
\draw [c] (8.94923,0.686898) -- (8.95736,0.686898);
\definecolor{c}{rgb}{0,0,0};
\colorlet{c}{natcomp!70};
\draw [c] (8.9655,0.686894) -- (8.9655,0.686898);
\draw [c] (8.9655,0.686898) -- (8.9655,0.686902);
\draw [c] (8.95736,0.686898) -- (8.9655,0.686898);
\draw [c] (8.9655,0.686898) -- (8.97364,0.686898);
\definecolor{c}{rgb}{0,0,0};
\colorlet{c}{natcomp!70};
\draw [c] (8.98177,0.686896) -- (8.98177,0.686902);
\draw [c] (8.98177,0.686902) -- (8.98177,0.686907);
\draw [c] (8.97364,0.686902) -- (8.98177,0.686902);
\draw [c] (8.98177,0.686902) -- (8.98991,0.686902);
\definecolor{c}{rgb}{0,0,0};
\colorlet{c}{natcomp!70};
\draw [c] (8.99805,0.686904) -- (8.99805,0.686922);
\draw [c] (8.99805,0.686922) -- (8.99805,0.686939);
\draw [c] (8.98991,0.686922) -- (8.99805,0.686922);
\draw [c] (8.99805,0.686922) -- (9.00618,0.686922);
\definecolor{c}{rgb}{0,0,0};
\colorlet{c}{natcomp!70};
\draw [c] (9.01432,0.686897) -- (9.01432,0.686903);
\draw [c] (9.01432,0.686903) -- (9.01432,0.68691);
\draw [c] (9.00618,0.686903) -- (9.01432,0.686903);
\draw [c] (9.01432,0.686903) -- (9.02245,0.686903);
\definecolor{c}{rgb}{0,0,0};
\colorlet{c}{natcomp!70};
\draw [c] (9.04686,0.708993) -- (9.04686,0.731658);
\draw [c] (9.04686,0.731658) -- (9.04686,0.754323);
\draw [c] (9.03873,0.731658) -- (9.04686,0.731658);
\draw [c] (9.04686,0.731658) -- (9.055,0.731658);
\definecolor{c}{rgb}{0,0,0};
\colorlet{c}{natcomp!70};
\draw [c] (9.07941,0.692085) -- (9.07941,0.704618);
\draw [c] (9.07941,0.704618) -- (9.07941,0.717151);
\draw [c] (9.07127,0.704618) -- (9.07941,0.704618);
\draw [c] (9.07941,0.704618) -- (9.08755,0.704618);
\definecolor{c}{rgb}{0,0,0};
\colorlet{c}{natcomp!70};
\draw [c] (9.09568,0.686897) -- (9.09568,0.695861);
\draw [c] (9.09568,0.695861) -- (9.09568,0.704825);
\draw [c] (9.08755,0.695861) -- (9.09568,0.695861);
\draw [c] (9.09568,0.695861) -- (9.10382,0.695861);
\definecolor{c}{rgb}{0,0,0};
\colorlet{c}{natcomp!70};
\draw [c] (9.12823,0.686894) -- (9.12823,0.686897);
\draw [c] (9.12823,0.686897) -- (9.12823,0.686901);
\draw [c] (9.12009,0.686897) -- (9.12823,0.686897);
\draw [c] (9.12823,0.686897) -- (9.13636,0.686897);
\definecolor{c}{rgb}{0,0,0};
\colorlet{c}{natcomp!70};
\draw [c] (9.1445,0.686899) -- (9.1445,0.696014);
\draw [c] (9.1445,0.696014) -- (9.1445,0.70513);
\draw [c] (9.13636,0.696014) -- (9.1445,0.696014);
\draw [c] (9.1445,0.696014) -- (9.15264,0.696014);
\definecolor{c}{rgb}{0,0,0};
\colorlet{c}{natcomp!70};
\draw [c] (9.16077,0.686897) -- (9.16077,0.686903);
\draw [c] (9.16077,0.686903) -- (9.16077,0.68691);
\draw [c] (9.15264,0.686903) -- (9.16077,0.686903);
\draw [c] (9.16077,0.686903) -- (9.16891,0.686903);
\definecolor{c}{rgb}{0,0,0};
\colorlet{c}{natcomp!70};
\draw [c] (9.17705,0.686894) -- (9.17705,0.705403);
\draw [c] (9.17705,0.705403) -- (9.17705,0.723913);
\draw [c] (9.16891,0.705403) -- (9.17705,0.705403);
\draw [c] (9.17705,0.705403) -- (9.18518,0.705403);
\definecolor{c}{rgb}{0,0,0};
\colorlet{c}{natcomp!70};
\draw [c] (9.19332,0.686897) -- (9.19332,0.686906);
\draw [c] (9.19332,0.686906) -- (9.19332,0.686914);
\draw [c] (9.18518,0.686906) -- (9.19332,0.686906);
\draw [c] (9.19332,0.686906) -- (9.20145,0.686906);
\definecolor{c}{rgb}{0,0,0};
\colorlet{c}{natcomp!70};
\draw [c] (9.20959,0.686906) -- (9.20959,0.694628);
\draw [c] (9.20959,0.694628) -- (9.20959,0.70235);
\draw [c] (9.20145,0.694628) -- (9.20959,0.694628);
\draw [c] (9.20959,0.694628) -- (9.21773,0.694628);
\definecolor{c}{rgb}{0,0,0};
\colorlet{c}{natcomp!70};
\draw [c] (9.22586,0.686898) -- (9.22586,0.697671);
\draw [c] (9.22586,0.697671) -- (9.22586,0.708444);
\draw [c] (9.21773,0.697671) -- (9.22586,0.697671);
\draw [c] (9.22586,0.697671) -- (9.234,0.697671);
\definecolor{c}{rgb}{0,0,0};
\colorlet{c}{natcomp!70};
\draw [c] (9.24214,0.697919) -- (9.24214,0.713014);
\draw [c] (9.24214,0.713014) -- (9.24214,0.72811);
\draw [c] (9.234,0.713014) -- (9.24214,0.713014);
\draw [c] (9.24214,0.713014) -- (9.25027,0.713014);
\definecolor{c}{rgb}{0,0,0};
\colorlet{c}{natcomp!70};
\draw [c] (9.25841,0.686896) -- (9.25841,0.686902);
\draw [c] (9.25841,0.686902) -- (9.25841,0.686908);
\draw [c] (9.25027,0.686902) -- (9.25841,0.686902);
\draw [c] (9.25841,0.686902) -- (9.26655,0.686902);
\definecolor{c}{rgb}{0,0,0};
\colorlet{c}{natcomp!70};
\draw [c] (9.27468,0.686897) -- (9.27468,0.686903);
\draw [c] (9.27468,0.686903) -- (9.27468,0.68691);
\draw [c] (9.26655,0.686903) -- (9.27468,0.686903);
\draw [c] (9.27468,0.686903) -- (9.28282,0.686903);
\definecolor{c}{rgb}{0,0,0};
\colorlet{c}{natcomp!70};
\draw [c] (9.30723,0.691421) -- (9.30723,0.702341);
\draw [c] (9.30723,0.702341) -- (9.30723,0.713261);
\draw [c] (9.29909,0.702341) -- (9.30723,0.702341);
\draw [c] (9.30723,0.702341) -- (9.31536,0.702341);
\definecolor{c}{rgb}{0,0,0};
\colorlet{c}{natcomp!70};
\draw [c] (9.33977,0.686902) -- (9.33977,0.69504);
\draw [c] (9.33977,0.69504) -- (9.33977,0.703179);
\draw [c] (9.33164,0.69504) -- (9.33977,0.69504);
\draw [c] (9.33977,0.69504) -- (9.34791,0.69504);
\definecolor{c}{rgb}{0,0,0};
\colorlet{c}{natcomp!70};
\draw [c] (9.35605,0.693227) -- (9.35605,0.708478);
\draw [c] (9.35605,0.708478) -- (9.35605,0.723729);
\draw [c] (9.34791,0.708478) -- (9.35605,0.708478);
\draw [c] (9.35605,0.708478) -- (9.36418,0.708478);
\definecolor{c}{rgb}{0,0,0};
\colorlet{c}{natcomp!70};
\draw [c] (9.37232,0.686896) -- (9.37232,0.686902);
\draw [c] (9.37232,0.686902) -- (9.37232,0.686908);
\draw [c] (9.36418,0.686902) -- (9.37232,0.686902);
\draw [c] (9.37232,0.686902) -- (9.38046,0.686902);
\definecolor{c}{rgb}{0,0,0};
\colorlet{c}{natcomp!70};
\draw [c] (9.38859,0.686897) -- (9.38859,0.696013);
\draw [c] (9.38859,0.696013) -- (9.38859,0.705128);
\draw [c] (9.38046,0.696013) -- (9.38859,0.696013);
\draw [c] (9.38859,0.696013) -- (9.39673,0.696013);
\definecolor{c}{rgb}{0,0,0};
\colorlet{c}{natcomp!70};
\draw [c] (9.43741,0.686902) -- (9.43741,0.697312);
\draw [c] (9.43741,0.697312) -- (9.43741,0.707723);
\draw [c] (9.42927,0.697312) -- (9.43741,0.697312);
\draw [c] (9.43741,0.697312) -- (9.44555,0.697312);
\definecolor{c}{rgb}{0,0,0};
\colorlet{c}{natcomp!70};
\draw [c] (9.45368,0.686897) -- (9.45368,0.686903);
\draw [c] (9.45368,0.686903) -- (9.45368,0.68691);
\draw [c] (9.44555,0.686903) -- (9.45368,0.686903);
\draw [c] (9.45368,0.686903) -- (9.46182,0.686903);
\definecolor{c}{rgb}{0,0,0};
\colorlet{c}{natcomp!70};
\draw [c] (9.46995,0.692681) -- (9.46995,0.706988);
\draw [c] (9.46995,0.706988) -- (9.46995,0.721296);
\draw [c] (9.46182,0.706988) -- (9.46995,0.706988);
\draw [c] (9.46995,0.706988) -- (9.47809,0.706988);
\definecolor{c}{rgb}{0,0,0};
\colorlet{c}{natcomp!70};
\draw [c] (9.48623,0.686894) -- (9.48623,0.686903);
\draw [c] (9.48623,0.686903) -- (9.48623,0.686911);
\draw [c] (9.47809,0.686903) -- (9.48623,0.686903);
\draw [c] (9.48623,0.686903) -- (9.49436,0.686903);
\definecolor{c}{rgb}{0,0,0};
\colorlet{c}{natcomp!70};
\draw [c] (9.5025,0.686898) -- (9.5025,0.696793);
\draw [c] (9.5025,0.696793) -- (9.5025,0.706687);
\draw [c] (9.49436,0.696793) -- (9.5025,0.696793);
\draw [c] (9.5025,0.696793) -- (9.51064,0.696793);
\definecolor{c}{rgb}{0,0,0};
\colorlet{c}{natcomp!70};
\draw [c] (9.51877,0.686894) -- (9.51877,0.697689);
\draw [c] (9.51877,0.697689) -- (9.51877,0.708483);
\draw [c] (9.51064,0.697689) -- (9.51877,0.697689);
\draw [c] (9.51877,0.697689) -- (9.52691,0.697689);
\definecolor{c}{rgb}{0,0,0};
\colorlet{c}{natcomp!70};
\draw [c] (9.53505,0.686896) -- (9.53505,0.686901);
\draw [c] (9.53505,0.686901) -- (9.53505,0.686906);
\draw [c] (9.52691,0.686901) -- (9.53505,0.686901);
\draw [c] (9.53505,0.686901) -- (9.54318,0.686901);
\definecolor{c}{rgb}{0,0,0};
\colorlet{c}{natcomp!70};
\draw [c] (9.55132,0.686896) -- (9.55132,0.686902);
\draw [c] (9.55132,0.686902) -- (9.55132,0.686908);
\draw [c] (9.54318,0.686902) -- (9.55132,0.686902);
\draw [c] (9.55132,0.686902) -- (9.55945,0.686902);
\definecolor{c}{rgb}{0,0,0};
\colorlet{c}{natcomp!70};
\draw [c] (9.56759,0.686894) -- (9.56759,0.696788);
\draw [c] (9.56759,0.696788) -- (9.56759,0.706683);
\draw [c] (9.55945,0.696788) -- (9.56759,0.696788);
\draw [c] (9.56759,0.696788) -- (9.57573,0.696788);
\definecolor{c}{rgb}{0,0,0};
\colorlet{c}{natcomp!70};
\draw [c] (9.58386,0.686894) -- (9.58386,0.686898);
\draw [c] (9.58386,0.686898) -- (9.58386,0.686902);
\draw [c] (9.57573,0.686898) -- (9.58386,0.686898);
\draw [c] (9.58386,0.686898) -- (9.592,0.686898);
\definecolor{c}{rgb}{0,0,0};
\colorlet{c}{natcomp!70};
\draw [c] (9.60014,0.686899) -- (9.60014,0.695761);
\draw [c] (9.60014,0.695761) -- (9.60014,0.704623);
\draw [c] (9.592,0.695761) -- (9.60014,0.695761);
\draw [c] (9.60014,0.695761) -- (9.60827,0.695761);
\definecolor{c}{rgb}{0,0,0};
\colorlet{c}{natcomp!70};
\draw [c] (9.61641,0.691959) -- (9.61641,0.70451);
\draw [c] (9.61641,0.70451) -- (9.61641,0.717061);
\draw [c] (9.60827,0.70451) -- (9.61641,0.70451);
\draw [c] (9.61641,0.70451) -- (9.62455,0.70451);
\definecolor{c}{rgb}{0,0,0};
\colorlet{c}{natcomp!70};
\draw [c] (9.64895,0.686899) -- (9.64895,0.686907);
\draw [c] (9.64895,0.686907) -- (9.64895,0.686914);
\draw [c] (9.64082,0.686907) -- (9.64895,0.686907);
\draw [c] (9.64895,0.686907) -- (9.65709,0.686907);
\definecolor{c}{rgb}{0,0,0};
\colorlet{c}{natcomp!70};
\draw [c] (9.66523,0.686899) -- (9.66523,0.686907);
\draw [c] (9.66523,0.686907) -- (9.66523,0.686914);
\draw [c] (9.65709,0.686907) -- (9.66523,0.686907);
\draw [c] (9.66523,0.686907) -- (9.67336,0.686907);
\definecolor{c}{rgb}{0,0,0};
\colorlet{c}{natcomp!70};
\draw [c] (9.6815,0.686894) -- (9.6815,0.686897);
\draw [c] (9.6815,0.686897) -- (9.6815,0.686901);
\draw [c] (9.67336,0.686897) -- (9.6815,0.686897);
\draw [c] (9.6815,0.686897) -- (9.68964,0.686897);
\definecolor{c}{rgb}{0,0,0};
\colorlet{c}{natcomp!70};
\draw [c] (9.71405,0.692495) -- (9.71405,0.706167);
\draw [c] (9.71405,0.706167) -- (9.71405,0.719839);
\draw [c] (9.70591,0.706167) -- (9.71405,0.706167);
\draw [c] (9.71405,0.706167) -- (9.72218,0.706167);
\definecolor{c}{rgb}{0,0,0};
\colorlet{c}{natcomp!70};
\draw [c] (9.74659,0.686894) -- (9.74659,0.686897);
\draw [c] (9.74659,0.686897) -- (9.74659,0.686901);
\draw [c] (9.73845,0.686897) -- (9.74659,0.686897);
\draw [c] (9.74659,0.686897) -- (9.75473,0.686897);
\definecolor{c}{rgb}{0,0,0};
\colorlet{c}{natcomp!70};
\draw [c] (9.76286,0.686896) -- (9.76286,0.686901);
\draw [c] (9.76286,0.686901) -- (9.76286,0.686906);
\draw [c] (9.75473,0.686901) -- (9.76286,0.686901);
\draw [c] (9.76286,0.686901) -- (9.771,0.686901);
\definecolor{c}{rgb}{0,0,0};
\colorlet{c}{natcomp!70};
\draw [c] (9.77914,0.686896) -- (9.77914,0.686902);
\draw [c] (9.77914,0.686902) -- (9.77914,0.686908);
\draw [c] (9.771,0.686902) -- (9.77914,0.686902);
\draw [c] (9.77914,0.686902) -- (9.78727,0.686902);
\definecolor{c}{rgb}{0,0,0};
\colorlet{c}{natcomp!70};
\draw [c] (9.79541,0.686898) -- (9.79541,0.696014);
\draw [c] (9.79541,0.696014) -- (9.79541,0.705129);
\draw [c] (9.78727,0.696014) -- (9.79541,0.696014);
\draw [c] (9.79541,0.696014) -- (9.80354,0.696014);
\definecolor{c}{rgb}{0,0,0};
\colorlet{c}{natcomp!70};
\draw [c] (9.81168,0.686894) -- (9.81168,0.686899);
\draw [c] (9.81168,0.686899) -- (9.81168,0.686903);
\draw [c] (9.80354,0.686899) -- (9.81168,0.686899);
\draw [c] (9.81168,0.686899) -- (9.81982,0.686899);
\definecolor{c}{rgb}{0,0,0};
\colorlet{c}{natcomp!70};
\draw [c] (9.82795,0.692367) -- (9.82795,0.70565);
\draw [c] (9.82795,0.70565) -- (9.82795,0.718933);
\draw [c] (9.81982,0.70565) -- (9.82795,0.70565);
\draw [c] (9.82795,0.70565) -- (9.83609,0.70565);
\definecolor{c}{rgb}{0,0,0};
\colorlet{c}{natcomp!70};
\draw [c] (9.84423,0.686894) -- (9.84423,0.686898);
\draw [c] (9.84423,0.686898) -- (9.84423,0.686902);
\draw [c] (9.83609,0.686898) -- (9.84423,0.686898);
\draw [c] (9.84423,0.686898) -- (9.85236,0.686898);
\definecolor{c}{rgb}{0,0,0};
\colorlet{c}{natcomp!70};
\draw [c] (9.8605,0.686894) -- (9.8605,0.686901);
\draw [c] (9.8605,0.686901) -- (9.8605,0.686908);
\draw [c] (9.85236,0.686901) -- (9.8605,0.686901);
\draw [c] (9.8605,0.686901) -- (9.86864,0.686901);
\definecolor{c}{rgb}{0,0,0};
\colorlet{c}{natcomp!70};
\draw [c] (9.87677,0.686894) -- (9.87677,0.686899);
\draw [c] (9.87677,0.686899) -- (9.87677,0.686903);
\draw [c] (9.86864,0.686899) -- (9.87677,0.686899);
\draw [c] (9.87677,0.686899) -- (9.88491,0.686899);
\definecolor{c}{rgb}{0,0,0};
\colorlet{c}{natcomp!70};
\draw [c] (9.89305,0.691538) -- (9.89305,0.702957);
\draw [c] (9.89305,0.702957) -- (9.89305,0.714375);
\draw [c] (9.88491,0.702957) -- (9.89305,0.702957);
\draw [c] (9.89305,0.702957) -- (9.90118,0.702957);
\definecolor{c}{rgb}{0,0,0};
\colorlet{c}{natcomp!70};
\draw [c] (9.90932,0.692367) -- (9.90932,0.70565);
\draw [c] (9.90932,0.70565) -- (9.90932,0.718933);
\draw [c] (9.90118,0.70565) -- (9.90932,0.70565);
\draw [c] (9.90932,0.70565) -- (9.91745,0.70565);
\definecolor{c}{rgb}{0,0,0};
\colorlet{c}{natcomp!70};
\draw [c] (9.94186,0.686899) -- (9.94186,0.686907);
\draw [c] (9.94186,0.686907) -- (9.94186,0.686915);
\draw [c] (9.93373,0.686907) -- (9.94186,0.686907);
\draw [c] (9.94186,0.686907) -- (9.95,0.686907);
\definecolor{c}{rgb}{0,0,0};
\draw [anchor=base west] (6.62249,6.01827) node[color=c, rotate=0]{ALTAS MC};
\colorlet{c}{natgreen};
\draw [c] (5.87661,6.12464) -- (6.49087,6.12464);
\draw [c] (6.18374,5.98281) -- (6.18374,6.26648);
\definecolor{c}{rgb}{0,0,0};
\draw [anchor=base west] (6.62249,5.54549) node[color=c, rotate=0]{CalcHEP MC};
\colorlet{c}{natcomp!70};
\draw [c] (5.87661,5.65186) -- (6.49087,5.65186);
\draw [c] (6.18374,5.51003) -- (6.18374,5.7937);
\end{tikzpicture}

\end{infilsf}
\end{minipage}
\hfill
\begin{minipage}[b]{.49\textwidth}
\begin{infilsf} \tiny 
\begin{tikzpicture}[x=.092\textwidth,y=.092\textwidth]
\pgfdeclareplotmark{cross} {
\pgfpathmoveto{\pgfpoint{-0.3\pgfplotmarksize}{\pgfplotmarksize}}
\pgfpathlineto{\pgfpoint{+0.3\pgfplotmarksize}{\pgfplotmarksize}}
\pgfpathlineto{\pgfpoint{+0.3\pgfplotmarksize}{0.3\pgfplotmarksize}}
\pgfpathlineto{\pgfpoint{+1\pgfplotmarksize}{0.3\pgfplotmarksize}}
\pgfpathlineto{\pgfpoint{+1\pgfplotmarksize}{-0.3\pgfplotmarksize}}
\pgfpathlineto{\pgfpoint{+0.3\pgfplotmarksize}{-0.3\pgfplotmarksize}}
\pgfpathlineto{\pgfpoint{+0.3\pgfplotmarksize}{-1.\pgfplotmarksize}}
\pgfpathlineto{\pgfpoint{-0.3\pgfplotmarksize}{-1.\pgfplotmarksize}}
\pgfpathlineto{\pgfpoint{-0.3\pgfplotmarksize}{-0.3\pgfplotmarksize}}
\pgfpathlineto{\pgfpoint{-1.\pgfplotmarksize}{-0.3\pgfplotmarksize}}
\pgfpathlineto{\pgfpoint{-1.\pgfplotmarksize}{0.3\pgfplotmarksize}}
\pgfpathlineto{\pgfpoint{-0.3\pgfplotmarksize}{0.3\pgfplotmarksize}}
\pgfpathclose
\pgfusepathqstroke
}
\pgfdeclareplotmark{cross*} {
\pgfpathmoveto{\pgfpoint{-0.3\pgfplotmarksize}{\pgfplotmarksize}}
\pgfpathlineto{\pgfpoint{+0.3\pgfplotmarksize}{\pgfplotmarksize}}
\pgfpathlineto{\pgfpoint{+0.3\pgfplotmarksize}{0.3\pgfplotmarksize}}
\pgfpathlineto{\pgfpoint{+1\pgfplotmarksize}{0.3\pgfplotmarksize}}
\pgfpathlineto{\pgfpoint{+1\pgfplotmarksize}{-0.3\pgfplotmarksize}}
\pgfpathlineto{\pgfpoint{+0.3\pgfplotmarksize}{-0.3\pgfplotmarksize}}
\pgfpathlineto{\pgfpoint{+0.3\pgfplotmarksize}{-1.\pgfplotmarksize}}
\pgfpathlineto{\pgfpoint{-0.3\pgfplotmarksize}{-1.\pgfplotmarksize}}
\pgfpathlineto{\pgfpoint{-0.3\pgfplotmarksize}{-0.3\pgfplotmarksize}}
\pgfpathlineto{\pgfpoint{-1.\pgfplotmarksize}{-0.3\pgfplotmarksize}}
\pgfpathlineto{\pgfpoint{-1.\pgfplotmarksize}{0.3\pgfplotmarksize}}
\pgfpathlineto{\pgfpoint{-0.3\pgfplotmarksize}{0.3\pgfplotmarksize}}
\pgfpathclose
\pgfusepathqfillstroke
}
\pgfdeclareplotmark{newstar} {
\pgfpathmoveto{\pgfqpoint{0pt}{\pgfplotmarksize}}
\pgfpathlineto{\pgfqpointpolar{44}{0.5\pgfplotmarksize}}
\pgfpathlineto{\pgfqpointpolar{18}{\pgfplotmarksize}}
\pgfpathlineto{\pgfqpointpolar{-20}{0.5\pgfplotmarksize}}
\pgfpathlineto{\pgfqpointpolar{-54}{\pgfplotmarksize}}
\pgfpathlineto{\pgfqpointpolar{-90}{0.5\pgfplotmarksize}}
\pgfpathlineto{\pgfqpointpolar{234}{\pgfplotmarksize}}
\pgfpathlineto{\pgfqpointpolar{198}{0.5\pgfplotmarksize}}
\pgfpathlineto{\pgfqpointpolar{162}{\pgfplotmarksize}}
\pgfpathlineto{\pgfqpointpolar{134}{0.5\pgfplotmarksize}}
\pgfpathclose
\pgfusepathqstroke
}
\pgfdeclareplotmark{newstar*} {
\pgfpathmoveto{\pgfqpoint{0pt}{\pgfplotmarksize}}
\pgfpathlineto{\pgfqpointpolar{44}{0.5\pgfplotmarksize}}
\pgfpathlineto{\pgfqpointpolar{18}{\pgfplotmarksize}}
\pgfpathlineto{\pgfqpointpolar{-20}{0.5\pgfplotmarksize}}
\pgfpathlineto{\pgfqpointpolar{-54}{\pgfplotmarksize}}
\pgfpathlineto{\pgfqpointpolar{-90}{0.5\pgfplotmarksize}}
\pgfpathlineto{\pgfqpointpolar{234}{\pgfplotmarksize}}
\pgfpathlineto{\pgfqpointpolar{198}{0.5\pgfplotmarksize}}
\pgfpathlineto{\pgfqpointpolar{162}{\pgfplotmarksize}}
\pgfpathlineto{\pgfqpointpolar{134}{0.5\pgfplotmarksize}}
\pgfpathclose
\pgfusepathqfillstroke
}
\definecolor{c}{rgb}{1,1,1};
\draw [color=c, fill=c] (0,0) rectangle (10,6.80516);
\draw [color=c, fill=c] (1,0.680516) rectangle (9.95,6.73711);
\definecolor{c}{rgb}{0,0,0};
\draw [c] (1,0.680516) -- (1,6.73711) -- (9.95,6.73711) -- (9.95,0.680516) -- (1,0.680516);
\definecolor{c}{rgb}{1,1,1};
\draw [color=c, fill=c] (1,0.680516) rectangle (9.95,6.73711);
\definecolor{c}{rgb}{0,0,0};
\draw [c] (1,0.680516) -- (1,6.73711) -- (9.95,6.73711) -- (9.95,0.680516) -- (1,0.680516);
\colorlet{c}{natgreen};
\draw [c] (1.09022,0.740799) -- (1.09022,0.753375);
\draw [c] (1.09022,0.753375) -- (1.09022,0.76595);
\draw [c] (1.07218,0.753375) -- (1.09022,0.753375);
\draw [c] (1.09022,0.753375) -- (1.10827,0.753375);
\definecolor{c}{rgb}{0,0,0};
\colorlet{c}{natgreen};
\draw [c] (1.12631,0.740369) -- (1.12631,0.752314);
\draw [c] (1.12631,0.752314) -- (1.12631,0.76426);
\draw [c] (1.10827,0.752314) -- (1.12631,0.752314);
\draw [c] (1.12631,0.752314) -- (1.14435,0.752314);
\definecolor{c}{rgb}{0,0,0};
\colorlet{c}{natgreen};
\draw [c] (1.30675,0.737608) -- (1.30675,0.746848);
\draw [c] (1.30675,0.746848) -- (1.30675,0.756087);
\draw [c] (1.28871,0.746848) -- (1.30675,0.746848);
\draw [c] (1.30675,0.746848) -- (1.3248,0.746848);
\definecolor{c}{rgb}{0,0,0};
\colorlet{c}{natgreen};
\draw [c] (1.37893,0.755182) -- (1.37893,0.777154);
\draw [c] (1.37893,0.777154) -- (1.37893,0.799127);
\draw [c] (1.36089,0.777154) -- (1.37893,0.777154);
\draw [c] (1.37893,0.777154) -- (1.39698,0.777154);
\definecolor{c}{rgb}{0,0,0};
\colorlet{c}{natgreen};
\draw [c] (1.41502,0.755947) -- (1.41502,0.779138);
\draw [c] (1.41502,0.779138) -- (1.41502,0.802329);
\draw [c] (1.39698,0.779138) -- (1.41502,0.779138);
\draw [c] (1.41502,0.779138) -- (1.43306,0.779138);
\definecolor{c}{rgb}{0,0,0};
\colorlet{c}{natgreen};
\draw [c] (1.45111,0.768124) -- (1.45111,0.791905);
\draw [c] (1.45111,0.791905) -- (1.45111,0.815686);
\draw [c] (1.43306,0.791905) -- (1.45111,0.791905);
\draw [c] (1.45111,0.791905) -- (1.46915,0.791905);
\definecolor{c}{rgb}{0,0,0};
\colorlet{c}{natgreen};
\draw [c] (1.4872,0.758252) -- (1.4872,0.782938);
\draw [c] (1.4872,0.782938) -- (1.4872,0.807624);
\draw [c] (1.46915,0.782938) -- (1.4872,0.782938);
\draw [c] (1.4872,0.782938) -- (1.50524,0.782938);
\definecolor{c}{rgb}{0,0,0};
\colorlet{c}{natgreen};
\draw [c] (1.52329,0.78146) -- (1.52329,0.808304);
\draw [c] (1.52329,0.808304) -- (1.52329,0.835148);
\draw [c] (1.50524,0.808304) -- (1.52329,0.808304);
\draw [c] (1.52329,0.808304) -- (1.54133,0.808304);
\definecolor{c}{rgb}{0,0,0};
\colorlet{c}{natgreen};
\draw [c] (1.55938,0.77573) -- (1.55938,0.802259);
\draw [c] (1.55938,0.802259) -- (1.55938,0.828788);
\draw [c] (1.54133,0.802259) -- (1.55938,0.802259);
\draw [c] (1.55938,0.802259) -- (1.57742,0.802259);
\definecolor{c}{rgb}{0,0,0};
\colorlet{c}{natgreen};
\draw [c] (1.59546,0.761209) -- (1.59546,0.784068);
\draw [c] (1.59546,0.784068) -- (1.59546,0.806926);
\draw [c] (1.57742,0.784068) -- (1.59546,0.784068);
\draw [c] (1.59546,0.784068) -- (1.61351,0.784068);
\definecolor{c}{rgb}{0,0,0};
\colorlet{c}{natgreen};
\draw [c] (1.63155,0.802522) -- (1.63155,0.838187);
\draw [c] (1.63155,0.838187) -- (1.63155,0.873852);
\draw [c] (1.61351,0.838187) -- (1.63155,0.838187);
\draw [c] (1.63155,0.838187) -- (1.6496,0.838187);
\definecolor{c}{rgb}{0,0,0};
\colorlet{c}{natgreen};
\draw [c] (1.66764,0.829492) -- (1.66764,0.865601);
\draw [c] (1.66764,0.865601) -- (1.66764,0.901709);
\draw [c] (1.6496,0.865601) -- (1.66764,0.865601);
\draw [c] (1.66764,0.865601) -- (1.68569,0.865601);
\definecolor{c}{rgb}{0,0,0};
\colorlet{c}{natgreen};
\draw [c] (1.70373,0.80146) -- (1.70373,0.833741);
\draw [c] (1.70373,0.833741) -- (1.70373,0.866023);
\draw [c] (1.68569,0.833741) -- (1.70373,0.833741);
\draw [c] (1.70373,0.833741) -- (1.72177,0.833741);
\definecolor{c}{rgb}{0,0,0};
\colorlet{c}{natgreen};
\draw [c] (1.73982,0.803179) -- (1.73982,0.835265);
\draw [c] (1.73982,0.835265) -- (1.73982,0.867351);
\draw [c] (1.72177,0.835265) -- (1.73982,0.835265);
\draw [c] (1.73982,0.835265) -- (1.75786,0.835265);
\definecolor{c}{rgb}{0,0,0};
\colorlet{c}{natgreen};
\draw [c] (1.77591,0.898175) -- (1.77591,0.946579);
\draw [c] (1.77591,0.946579) -- (1.77591,0.994983);
\draw [c] (1.75786,0.946579) -- (1.77591,0.946579);
\draw [c] (1.77591,0.946579) -- (1.79395,0.946579);
\definecolor{c}{rgb}{0,0,0};
\colorlet{c}{natgreen};
\draw [c] (1.812,0.953061) -- (1.812,1.00914);
\draw [c] (1.812,1.00914) -- (1.812,1.06522);
\draw [c] (1.79395,1.00914) -- (1.812,1.00914);
\draw [c] (1.812,1.00914) -- (1.83004,1.00914);
\definecolor{c}{rgb}{0,0,0};
\colorlet{c}{natgreen};
\draw [c] (1.84808,0.939044) -- (1.84808,0.99126);
\draw [c] (1.84808,0.99126) -- (1.84808,1.04348);
\draw [c] (1.83004,0.99126) -- (1.84808,0.99126);
\draw [c] (1.84808,0.99126) -- (1.86613,0.99126);
\definecolor{c}{rgb}{0,0,0};
\colorlet{c}{natgreen};
\draw [c] (1.88417,1.08459) -- (1.88417,1.15614);
\draw [c] (1.88417,1.15614) -- (1.88417,1.22769);
\draw [c] (1.86613,1.15614) -- (1.88417,1.15614);
\draw [c] (1.88417,1.15614) -- (1.90222,1.15614);
\definecolor{c}{rgb}{0,0,0};
\colorlet{c}{natgreen};
\draw [c] (1.92026,1.07599) -- (1.92026,1.14389);
\draw [c] (1.92026,1.14389) -- (1.92026,1.21179);
\draw [c] (1.90222,1.14389) -- (1.92026,1.14389);
\draw [c] (1.92026,1.14389) -- (1.93831,1.14389);
\definecolor{c}{rgb}{0,0,0};
\colorlet{c}{natgreen};
\draw [c] (1.95635,1.20977) -- (1.95635,1.28799);
\draw [c] (1.95635,1.28799) -- (1.95635,1.36621);
\draw [c] (1.93831,1.28799) -- (1.95635,1.28799);
\draw [c] (1.95635,1.28799) -- (1.9744,1.28799);
\definecolor{c}{rgb}{0,0,0};
\colorlet{c}{natgreen};
\draw [c] (1.99244,1.34676) -- (1.99244,1.43945);
\draw [c] (1.99244,1.43945) -- (1.99244,1.53215);
\draw [c] (1.9744,1.43945) -- (1.99244,1.43945);
\draw [c] (1.99244,1.43945) -- (2.01048,1.43945);
\definecolor{c}{rgb}{0,0,0};
\colorlet{c}{natgreen};
\draw [c] (2.02853,1.38683) -- (2.02853,1.48215);
\draw [c] (2.02853,1.48215) -- (2.02853,1.57747);
\draw [c] (2.01048,1.48215) -- (2.02853,1.48215);
\draw [c] (2.02853,1.48215) -- (2.04657,1.48215);
\definecolor{c}{rgb}{0,0,0};
\colorlet{c}{natgreen};
\draw [c] (2.06462,1.59757) -- (2.06462,1.70779);
\draw [c] (2.06462,1.70779) -- (2.06462,1.81801);
\draw [c] (2.04657,1.70779) -- (2.06462,1.70779);
\draw [c] (2.06462,1.70779) -- (2.08266,1.70779);
\definecolor{c}{rgb}{0,0,0};
\colorlet{c}{natgreen};
\draw [c] (2.10071,1.84589) -- (2.10071,1.96852);
\draw [c] (2.10071,1.96852) -- (2.10071,2.09115);
\draw [c] (2.08266,1.96852) -- (2.10071,1.96852);
\draw [c] (2.10071,1.96852) -- (2.11875,1.96852);
\definecolor{c}{rgb}{0,0,0};
\colorlet{c}{natgreen};
\draw [c] (2.13679,1.88974) -- (2.13679,2.01623);
\draw [c] (2.13679,2.01623) -- (2.13679,2.14272);
\draw [c] (2.11875,2.01623) -- (2.13679,2.01623);
\draw [c] (2.13679,2.01623) -- (2.15484,2.01623);
\definecolor{c}{rgb}{0,0,0};
\colorlet{c}{natgreen};
\draw [c] (2.17288,2.27127) -- (2.17288,2.41567);
\draw [c] (2.17288,2.41567) -- (2.17288,2.56008);
\draw [c] (2.15484,2.41567) -- (2.17288,2.41567);
\draw [c] (2.17288,2.41567) -- (2.19093,2.41567);
\definecolor{c}{rgb}{0,0,0};
\colorlet{c}{natgreen};
\draw [c] (2.20897,2.40762) -- (2.20897,2.55977);
\draw [c] (2.20897,2.55977) -- (2.20897,2.71191);
\draw [c] (2.19093,2.55977) -- (2.20897,2.55977);
\draw [c] (2.20897,2.55977) -- (2.22702,2.55977);
\definecolor{c}{rgb}{0,0,0};
\colorlet{c}{natgreen};
\draw [c] (2.24506,2.6453) -- (2.24506,2.80823);
\draw [c] (2.24506,2.80823) -- (2.24506,2.97115);
\draw [c] (2.22702,2.80823) -- (2.24506,2.80823);
\draw [c] (2.24506,2.80823) -- (2.2631,2.80823);
\definecolor{c}{rgb}{0,0,0};
\colorlet{c}{natgreen};
\draw [c] (2.28115,3.10578) -- (2.28115,3.28792);
\draw [c] (2.28115,3.28792) -- (2.28115,3.47006);
\draw [c] (2.2631,3.28792) -- (2.28115,3.28792);
\draw [c] (2.28115,3.28792) -- (2.29919,3.28792);
\definecolor{c}{rgb}{0,0,0};
\colorlet{c}{natgreen};
\draw [c] (2.31724,3.25624) -- (2.31724,3.44303);
\draw [c] (2.31724,3.44303) -- (2.31724,3.62982);
\draw [c] (2.29919,3.44303) -- (2.31724,3.44303);
\draw [c] (2.31724,3.44303) -- (2.33528,3.44303);
\definecolor{c}{rgb}{0,0,0};
\colorlet{c}{natgreen};
\draw [c] (2.35333,3.66962) -- (2.35333,3.87126);
\draw [c] (2.35333,3.87126) -- (2.35333,4.07289);
\draw [c] (2.33528,3.87126) -- (2.35333,3.87126);
\draw [c] (2.35333,3.87126) -- (2.37137,3.87126);
\definecolor{c}{rgb}{0,0,0};
\colorlet{c}{natgreen};
\draw [c] (2.38942,4.0976) -- (2.38942,4.3147);
\draw [c] (2.38942,4.3147) -- (2.38942,4.5318);
\draw [c] (2.37137,4.3147) -- (2.38942,4.3147);
\draw [c] (2.38942,4.3147) -- (2.40746,4.3147);
\definecolor{c}{rgb}{0,0,0};
\colorlet{c}{natgreen};
\draw [c] (2.4255,4.48404) -- (2.4255,4.71294);
\draw [c] (2.4255,4.71294) -- (2.4255,4.94183);
\draw [c] (2.40746,4.71294) -- (2.4255,4.71294);
\draw [c] (2.4255,4.71294) -- (2.44355,4.71294);
\definecolor{c}{rgb}{0,0,0};
\colorlet{c}{natgreen};
\draw [c] (2.46159,5.32319) -- (2.46159,5.58255);
\draw [c] (2.46159,5.58255) -- (2.46159,5.84191);
\draw [c] (2.44355,5.58255) -- (2.46159,5.58255);
\draw [c] (2.46159,5.58255) -- (2.47964,5.58255);
\definecolor{c}{rgb}{0,0,0};
\colorlet{c}{natgreen};
\draw [c] (2.49768,5.48167) -- (2.49768,5.74099);
\draw [c] (2.49768,5.74099) -- (2.49768,6.00031);
\draw [c] (2.47964,5.74099) -- (2.49768,5.74099);
\draw [c] (2.49768,5.74099) -- (2.51573,5.74099);
\definecolor{c}{rgb}{0,0,0};
\colorlet{c}{natgreen};
\draw [c] (2.53377,5.79817) -- (2.53377,6.06855);
\draw [c] (2.53377,6.06855) -- (2.53377,6.33893);
\draw [c] (2.51573,6.06855) -- (2.53377,6.06855);
\draw [c] (2.53377,6.06855) -- (2.55181,6.06855);
\definecolor{c}{rgb}{0,0,0};
\colorlet{c}{natgreen};
\draw [c] (2.56986,5.60846) -- (2.56986,5.87462);
\draw [c] (2.56986,5.87462) -- (2.56986,6.14078);
\draw [c] (2.55181,5.87462) -- (2.56986,5.87462);
\draw [c] (2.56986,5.87462) -- (2.5879,5.87462);
\definecolor{c}{rgb}{0,0,0};
\colorlet{c}{natgreen};
\draw [c] (2.60595,5.15422) -- (2.60595,5.4071);
\draw [c] (2.60595,5.4071) -- (2.60595,5.65997);
\draw [c] (2.5879,5.4071) -- (2.60595,5.4071);
\draw [c] (2.60595,5.4071) -- (2.62399,5.4071);
\definecolor{c}{rgb}{0,0,0};
\colorlet{c}{natgreen};
\draw [c] (2.64204,5.79256) -- (2.64204,6.06206);
\draw [c] (2.64204,6.06206) -- (2.64204,6.33155);
\draw [c] (2.62399,6.06206) -- (2.64204,6.06206);
\draw [c] (2.64204,6.06206) -- (2.66008,6.06206);
\definecolor{c}{rgb}{0,0,0};
\colorlet{c}{natgreen};
\draw [c] (2.67813,5.8431) -- (2.67813,6.11415);
\draw [c] (2.67813,6.11415) -- (2.67813,6.38519);
\draw [c] (2.66008,6.11415) -- (2.67813,6.11415);
\draw [c] (2.67813,6.11415) -- (2.69617,6.11415);
\definecolor{c}{rgb}{0,0,0};
\colorlet{c}{natgreen};
\draw [c] (2.71421,5.63737) -- (2.71421,5.90035);
\draw [c] (2.71421,5.90035) -- (2.71421,6.16332);
\draw [c] (2.69617,5.90035) -- (2.71421,5.90035);
\draw [c] (2.71421,5.90035) -- (2.73226,5.90035);
\definecolor{c}{rgb}{0,0,0};
\colorlet{c}{natgreen};
\draw [c] (2.7503,5.25438) -- (2.7503,5.50765);
\draw [c] (2.7503,5.50765) -- (2.7503,5.76092);
\draw [c] (2.73226,5.50765) -- (2.7503,5.50765);
\draw [c] (2.7503,5.50765) -- (2.76835,5.50765);
\definecolor{c}{rgb}{0,0,0};
\colorlet{c}{natgreen};
\draw [c] (2.78639,5.03639) -- (2.78639,5.2877);
\draw [c] (2.78639,5.2877) -- (2.78639,5.53901);
\draw [c] (2.76835,5.2877) -- (2.78639,5.2877);
\draw [c] (2.78639,5.2877) -- (2.80444,5.2877);
\definecolor{c}{rgb}{0,0,0};
\colorlet{c}{natgreen};
\draw [c] (2.82248,4.89291) -- (2.82248,5.13628);
\draw [c] (2.82248,5.13628) -- (2.82248,5.37966);
\draw [c] (2.80444,5.13628) -- (2.82248,5.13628);
\draw [c] (2.82248,5.13628) -- (2.84052,5.13628);
\definecolor{c}{rgb}{0,0,0};
\colorlet{c}{natgreen};
\draw [c] (2.85857,4.3619) -- (2.85857,4.58945);
\draw [c] (2.85857,4.58945) -- (2.85857,4.81699);
\draw [c] (2.84052,4.58945) -- (2.85857,4.58945);
\draw [c] (2.85857,4.58945) -- (2.87661,4.58945);
\definecolor{c}{rgb}{0,0,0};
\colorlet{c}{natgreen};
\draw [c] (2.89466,3.86365) -- (2.89466,4.07372);
\draw [c] (2.89466,4.07372) -- (2.89466,4.28379);
\draw [c] (2.87661,4.07372) -- (2.89466,4.07372);
\draw [c] (2.89466,4.07372) -- (2.9127,4.07372);
\definecolor{c}{rgb}{0,0,0};
\colorlet{c}{natgreen};
\draw [c] (2.93075,3.79834) -- (2.93075,4.0075);
\draw [c] (2.93075,4.0075) -- (2.93075,4.21665);
\draw [c] (2.9127,4.0075) -- (2.93075,4.0075);
\draw [c] (2.93075,4.0075) -- (2.94879,4.0075);
\definecolor{c}{rgb}{0,0,0};
\colorlet{c}{natgreen};
\draw [c] (2.96683,4.24582) -- (2.96683,4.47087);
\draw [c] (2.96683,4.47087) -- (2.96683,4.69591);
\draw [c] (2.94879,4.47087) -- (2.96683,4.47087);
\draw [c] (2.96683,4.47087) -- (2.98488,4.47087);
\definecolor{c}{rgb}{0,0,0};
\colorlet{c}{natgreen};
\draw [c] (3.00292,3.50606) -- (3.00292,3.70501);
\draw [c] (3.00292,3.70501) -- (3.00292,3.90397);
\draw [c] (2.98488,3.70501) -- (3.00292,3.70501);
\draw [c] (3.00292,3.70501) -- (3.02097,3.70501);
\definecolor{c}{rgb}{0,0,0};
\colorlet{c}{natgreen};
\draw [c] (3.03901,3.34697) -- (3.03901,3.54204);
\draw [c] (3.03901,3.54204) -- (3.03901,3.73711);
\draw [c] (3.02097,3.54204) -- (3.03901,3.54204);
\draw [c] (3.03901,3.54204) -- (3.05706,3.54204);
\definecolor{c}{rgb}{0,0,0};
\colorlet{c}{natgreen};
\draw [c] (3.0751,3.09791) -- (3.0751,3.28161);
\draw [c] (3.0751,3.28161) -- (3.0751,3.46531);
\draw [c] (3.05706,3.28161) -- (3.0751,3.28161);
\draw [c] (3.0751,3.28161) -- (3.09315,3.28161);
\definecolor{c}{rgb}{0,0,0};
\colorlet{c}{natgreen};
\draw [c] (3.11119,2.9437) -- (3.11119,3.12193);
\draw [c] (3.11119,3.12193) -- (3.11119,3.30016);
\draw [c] (3.09315,3.12193) -- (3.11119,3.12193);
\draw [c] (3.11119,3.12193) -- (3.12923,3.12193);
\definecolor{c}{rgb}{0,0,0};
\colorlet{c}{natgreen};
\draw [c] (3.14728,2.78324) -- (3.14728,2.95447);
\draw [c] (3.14728,2.95447) -- (3.14728,3.1257);
\draw [c] (3.12923,2.95447) -- (3.14728,2.95447);
\draw [c] (3.14728,2.95447) -- (3.16532,2.95447);
\definecolor{c}{rgb}{0,0,0};
\colorlet{c}{natgreen};
\draw [c] (3.18337,2.77691) -- (3.18337,2.94856);
\draw [c] (3.18337,2.94856) -- (3.18337,3.1202);
\draw [c] (3.16532,2.94856) -- (3.18337,2.94856);
\draw [c] (3.18337,2.94856) -- (3.20141,2.94856);
\definecolor{c}{rgb}{0,0,0};
\colorlet{c}{natgreen};
\draw [c] (3.21946,2.48398) -- (3.21946,2.64232);
\draw [c] (3.21946,2.64232) -- (3.21946,2.80067);
\draw [c] (3.20141,2.64232) -- (3.21946,2.64232);
\draw [c] (3.21946,2.64232) -- (3.2375,2.64232);
\definecolor{c}{rgb}{0,0,0};
\colorlet{c}{natgreen};
\draw [c] (3.25554,2.42662) -- (3.25554,2.58401);
\draw [c] (3.25554,2.58401) -- (3.25554,2.7414);
\draw [c] (3.2375,2.58401) -- (3.25554,2.58401);
\draw [c] (3.25554,2.58401) -- (3.27359,2.58401);
\definecolor{c}{rgb}{0,0,0};
\colorlet{c}{natgreen};
\draw [c] (3.29163,2.45462) -- (3.29163,2.61483);
\draw [c] (3.29163,2.61483) -- (3.29163,2.77503);
\draw [c] (3.27359,2.61483) -- (3.29163,2.61483);
\draw [c] (3.29163,2.61483) -- (3.30968,2.61483);
\definecolor{c}{rgb}{0,0,0};
\colorlet{c}{natgreen};
\draw [c] (3.32772,2.09209) -- (3.32772,2.23211);
\draw [c] (3.32772,2.23211) -- (3.32772,2.37213);
\draw [c] (3.30968,2.23211) -- (3.32772,2.23211);
\draw [c] (3.32772,2.23211) -- (3.34577,2.23211);
\definecolor{c}{rgb}{0,0,0};
\colorlet{c}{natgreen};
\draw [c] (3.36381,2.05646) -- (3.36381,2.19661);
\draw [c] (3.36381,2.19661) -- (3.36381,2.33676);
\draw [c] (3.34577,2.19661) -- (3.36381,2.19661);
\draw [c] (3.36381,2.19661) -- (3.38185,2.19661);
\definecolor{c}{rgb}{0,0,0};
\colorlet{c}{natgreen};
\draw [c] (3.3999,1.90968) -- (3.3999,2.0376);
\draw [c] (3.3999,2.0376) -- (3.3999,2.16552);
\draw [c] (3.38185,2.0376) -- (3.3999,2.0376);
\draw [c] (3.3999,2.0376) -- (3.41794,2.0376);
\definecolor{c}{rgb}{0,0,0};
\colorlet{c}{natgreen};
\draw [c] (3.43599,1.8398) -- (3.43599,1.96403);
\draw [c] (3.43599,1.96403) -- (3.43599,2.08826);
\draw [c] (3.41794,1.96403) -- (3.43599,1.96403);
\draw [c] (3.43599,1.96403) -- (3.45403,1.96403);
\definecolor{c}{rgb}{0,0,0};
\colorlet{c}{natgreen};
\draw [c] (3.47208,1.89732) -- (3.47208,2.02992);
\draw [c] (3.47208,2.02992) -- (3.47208,2.16253);
\draw [c] (3.45403,2.02992) -- (3.47208,2.02992);
\draw [c] (3.47208,2.02992) -- (3.49012,2.02992);
\definecolor{c}{rgb}{0,0,0};
\colorlet{c}{natgreen};
\draw [c] (3.50817,1.76818) -- (3.50817,1.88907);
\draw [c] (3.50817,1.88907) -- (3.50817,2.00995);
\draw [c] (3.49012,1.88907) -- (3.50817,1.88907);
\draw [c] (3.50817,1.88907) -- (3.52621,1.88907);
\definecolor{c}{rgb}{0,0,0};
\colorlet{c}{natgreen};
\draw [c] (3.54425,1.65461) -- (3.54425,1.77173);
\draw [c] (3.54425,1.77173) -- (3.54425,1.88885);
\draw [c] (3.52621,1.77173) -- (3.54425,1.77173);
\draw [c] (3.54425,1.77173) -- (3.5623,1.77173);
\definecolor{c}{rgb}{0,0,0};
\colorlet{c}{natgreen};
\draw [c] (3.58034,1.83018) -- (3.58034,1.9561);
\draw [c] (3.58034,1.9561) -- (3.58034,2.08202);
\draw [c] (3.5623,1.9561) -- (3.58034,1.9561);
\draw [c] (3.58034,1.9561) -- (3.59839,1.9561);
\definecolor{c}{rgb}{0,0,0};
\colorlet{c}{natgreen};
\draw [c] (3.61643,1.73218) -- (3.61643,1.85168);
\draw [c] (3.61643,1.85168) -- (3.61643,1.97118);
\draw [c] (3.59839,1.85168) -- (3.61643,1.85168);
\draw [c] (3.61643,1.85168) -- (3.63448,1.85168);
\definecolor{c}{rgb}{0,0,0};
\colorlet{c}{natgreen};
\draw [c] (3.65252,1.5139) -- (3.65252,1.62264);
\draw [c] (3.65252,1.62264) -- (3.65252,1.73139);
\draw [c] (3.63448,1.62264) -- (3.65252,1.62264);
\draw [c] (3.65252,1.62264) -- (3.67056,1.62264);
\definecolor{c}{rgb}{0,0,0};
\colorlet{c}{natgreen};
\draw [c] (3.68861,1.3318) -- (3.68861,1.42091);
\draw [c] (3.68861,1.42091) -- (3.68861,1.51002);
\draw [c] (3.67056,1.42091) -- (3.68861,1.42091);
\draw [c] (3.68861,1.42091) -- (3.70665,1.42091);
\definecolor{c}{rgb}{0,0,0};
\colorlet{c}{natgreen};
\draw [c] (3.7247,1.36919) -- (3.7247,1.46681);
\draw [c] (3.7247,1.46681) -- (3.7247,1.56444);
\draw [c] (3.70665,1.46681) -- (3.7247,1.46681);
\draw [c] (3.7247,1.46681) -- (3.74274,1.46681);
\definecolor{c}{rgb}{0,0,0};
\colorlet{c}{natgreen};
\draw [c] (3.76079,1.42997) -- (3.76079,1.53404);
\draw [c] (3.76079,1.53404) -- (3.76079,1.63811);
\draw [c] (3.74274,1.53404) -- (3.76079,1.53404);
\draw [c] (3.76079,1.53404) -- (3.77883,1.53404);
\definecolor{c}{rgb}{0,0,0};
\colorlet{c}{natgreen};
\draw [c] (3.79688,1.27256) -- (3.79688,1.36142);
\draw [c] (3.79688,1.36142) -- (3.79688,1.45027);
\draw [c] (3.77883,1.36142) -- (3.79688,1.36142);
\draw [c] (3.79688,1.36142) -- (3.81492,1.36142);
\definecolor{c}{rgb}{0,0,0};
\colorlet{c}{natgreen};
\draw [c] (3.83296,1.21638) -- (3.83296,1.30191);
\draw [c] (3.83296,1.30191) -- (3.83296,1.38745);
\draw [c] (3.81492,1.30191) -- (3.83296,1.30191);
\draw [c] (3.83296,1.30191) -- (3.85101,1.30191);
\definecolor{c}{rgb}{0,0,0};
\colorlet{c}{natgreen};
\draw [c] (3.86905,1.12283) -- (3.86905,1.20232);
\draw [c] (3.86905,1.20232) -- (3.86905,1.28182);
\draw [c] (3.85101,1.20232) -- (3.86905,1.20232);
\draw [c] (3.86905,1.20232) -- (3.8871,1.20232);
\definecolor{c}{rgb}{0,0,0};
\colorlet{c}{natgreen};
\draw [c] (3.90514,1.13958) -- (3.90514,1.21936);
\draw [c] (3.90514,1.21936) -- (3.90514,1.29914);
\draw [c] (3.8871,1.21936) -- (3.90514,1.21936);
\draw [c] (3.90514,1.21936) -- (3.92319,1.21936);
\definecolor{c}{rgb}{0,0,0};
\colorlet{c}{natgreen};
\draw [c] (3.94123,1.05205) -- (3.94123,1.1216);
\draw [c] (3.94123,1.1216) -- (3.94123,1.19116);
\draw [c] (3.92319,1.1216) -- (3.94123,1.1216);
\draw [c] (3.94123,1.1216) -- (3.95927,1.1216);
\definecolor{c}{rgb}{0,0,0};
\colorlet{c}{natgreen};
\draw [c] (3.97732,1.07434) -- (3.97732,1.14602);
\draw [c] (3.97732,1.14602) -- (3.97732,1.2177);
\draw [c] (3.95927,1.14602) -- (3.97732,1.14602);
\draw [c] (3.97732,1.14602) -- (3.99536,1.14602);
\definecolor{c}{rgb}{0,0,0};
\colorlet{c}{natgreen};
\draw [c] (4.01341,1.1326) -- (4.01341,1.20974);
\draw [c] (4.01341,1.20974) -- (4.01341,1.28687);
\draw [c] (3.99536,1.20974) -- (4.01341,1.20974);
\draw [c] (4.01341,1.20974) -- (4.03145,1.20974);
\definecolor{c}{rgb}{0,0,0};
\colorlet{c}{natgreen};
\draw [c] (4.0495,0.939214) -- (4.0495,0.995054);
\draw [c] (4.0495,0.995054) -- (4.0495,1.05089);
\draw [c] (4.03145,0.995054) -- (4.0495,0.995054);
\draw [c] (4.0495,0.995054) -- (4.06754,0.995054);
\definecolor{c}{rgb}{0,0,0};
\colorlet{c}{natgreen};
\draw [c] (4.08558,1.20029) -- (4.08558,1.28407);
\draw [c] (4.08558,1.28407) -- (4.08558,1.36784);
\draw [c] (4.06754,1.28407) -- (4.08558,1.28407);
\draw [c] (4.08558,1.28407) -- (4.10363,1.28407);
\definecolor{c}{rgb}{0,0,0};
\colorlet{c}{natgreen};
\draw [c] (4.12167,1.0113) -- (4.12167,1.07931);
\draw [c] (4.12167,1.07931) -- (4.12167,1.14733);
\draw [c] (4.10363,1.07931) -- (4.12167,1.07931);
\draw [c] (4.12167,1.07931) -- (4.13972,1.07931);
\definecolor{c}{rgb}{0,0,0};
\colorlet{c}{natgreen};
\draw [c] (4.15776,0.981008) -- (4.15776,1.04309);
\draw [c] (4.15776,1.04309) -- (4.15776,1.10517);
\draw [c] (4.13972,1.04309) -- (4.15776,1.04309);
\draw [c] (4.15776,1.04309) -- (4.17581,1.04309);
\definecolor{c}{rgb}{0,0,0};
\colorlet{c}{natgreen};
\draw [c] (4.19385,0.886966) -- (4.19385,0.937799);
\draw [c] (4.19385,0.937799) -- (4.19385,0.988632);
\draw [c] (4.17581,0.937799) -- (4.19385,0.937799);
\draw [c] (4.19385,0.937799) -- (4.21189,0.937799);
\definecolor{c}{rgb}{0,0,0};
\colorlet{c}{natgreen};
\draw [c] (4.22994,0.996004) -- (4.22994,1.0585);
\draw [c] (4.22994,1.0585) -- (4.22994,1.121);
\draw [c] (4.21189,1.0585) -- (4.22994,1.0585);
\draw [c] (4.22994,1.0585) -- (4.24798,1.0585);
\definecolor{c}{rgb}{0,0,0};
\colorlet{c}{natgreen};
\draw [c] (4.26603,0.916623) -- (4.26603,0.96946);
\draw [c] (4.26603,0.96946) -- (4.26603,1.0223);
\draw [c] (4.24798,0.96946) -- (4.26603,0.96946);
\draw [c] (4.26603,0.96946) -- (4.28407,0.96946);
\definecolor{c}{rgb}{0,0,0};
\colorlet{c}{natgreen};
\draw [c] (4.30212,0.950391) -- (4.30212,1.01253);
\draw [c] (4.30212,1.01253) -- (4.30212,1.07467);
\draw [c] (4.28407,1.01253) -- (4.30212,1.01253);
\draw [c] (4.30212,1.01253) -- (4.32016,1.01253);
\definecolor{c}{rgb}{0,0,0};
\colorlet{c}{natgreen};
\draw [c] (4.33821,0.870271) -- (4.33821,0.915271);
\draw [c] (4.33821,0.915271) -- (4.33821,0.960271);
\draw [c] (4.32016,0.915271) -- (4.33821,0.915271);
\draw [c] (4.33821,0.915271) -- (4.35625,0.915271);
\definecolor{c}{rgb}{0,0,0};
\colorlet{c}{natgreen};
\draw [c] (4.37429,0.832558) -- (4.37429,0.87354);
\draw [c] (4.37429,0.87354) -- (4.37429,0.914522);
\draw [c] (4.35625,0.87354) -- (4.37429,0.87354);
\draw [c] (4.37429,0.87354) -- (4.39234,0.87354);
\definecolor{c}{rgb}{0,0,0};
\colorlet{c}{natgreen};
\draw [c] (4.41038,0.865532) -- (4.41038,0.913695);
\draw [c] (4.41038,0.913695) -- (4.41038,0.961857);
\draw [c] (4.39234,0.913695) -- (4.41038,0.913695);
\draw [c] (4.41038,0.913695) -- (4.42843,0.913695);
\definecolor{c}{rgb}{0,0,0};
\colorlet{c}{natgreen};
\draw [c] (4.44647,0.891549) -- (4.44647,0.93864);
\draw [c] (4.44647,0.93864) -- (4.44647,0.985731);
\draw [c] (4.42843,0.93864) -- (4.44647,0.93864);
\draw [c] (4.44647,0.93864) -- (4.46452,0.93864);
\definecolor{c}{rgb}{0,0,0};
\colorlet{c}{natgreen};
\draw [c] (4.48256,0.808217) -- (4.48256,0.84252);
\draw [c] (4.48256,0.84252) -- (4.48256,0.876822);
\draw [c] (4.46452,0.84252) -- (4.48256,0.84252);
\draw [c] (4.48256,0.84252) -- (4.5006,0.84252);
\definecolor{c}{rgb}{0,0,0};
\colorlet{c}{natgreen};
\draw [c] (4.51865,0.844653) -- (4.51865,0.886753);
\draw [c] (4.51865,0.886753) -- (4.51865,0.928852);
\draw [c] (4.5006,0.886753) -- (4.51865,0.886753);
\draw [c] (4.51865,0.886753) -- (4.53669,0.886753);
\definecolor{c}{rgb}{0,0,0};
\colorlet{c}{natgreen};
\draw [c] (4.55474,0.817059) -- (4.55474,0.854363);
\draw [c] (4.55474,0.854363) -- (4.55474,0.891668);
\draw [c] (4.53669,0.854363) -- (4.55474,0.854363);
\draw [c] (4.55474,0.854363) -- (4.57278,0.854363);
\definecolor{c}{rgb}{0,0,0};
\colorlet{c}{natgreen};
\draw [c] (4.59083,0.798305) -- (4.59083,0.832897);
\draw [c] (4.59083,0.832897) -- (4.59083,0.86749);
\draw [c] (4.57278,0.832897) -- (4.59083,0.832897);
\draw [c] (4.59083,0.832897) -- (4.60887,0.832897);
\definecolor{c}{rgb}{0,0,0};
\colorlet{c}{natgreen};
\draw [c] (4.62692,0.850203) -- (4.62692,0.892336);
\draw [c] (4.62692,0.892336) -- (4.62692,0.934469);
\draw [c] (4.60887,0.892336) -- (4.62692,0.892336);
\draw [c] (4.62692,0.892336) -- (4.64496,0.892336);
\definecolor{c}{rgb}{0,0,0};
\colorlet{c}{natgreen};
\draw [c] (4.663,0.837422) -- (4.663,0.880898);
\draw [c] (4.663,0.880898) -- (4.663,0.924374);
\draw [c] (4.64496,0.880898) -- (4.663,0.880898);
\draw [c] (4.663,0.880898) -- (4.68105,0.880898);
\definecolor{c}{rgb}{0,0,0};
\colorlet{c}{natgreen};
\draw [c] (4.69909,0.79697) -- (4.69909,0.828716);
\draw [c] (4.69909,0.828716) -- (4.69909,0.860462);
\draw [c] (4.68105,0.828716) -- (4.69909,0.828716);
\draw [c] (4.69909,0.828716) -- (4.71714,0.828716);
\definecolor{c}{rgb}{0,0,0};
\colorlet{c}{natgreen};
\draw [c] (4.73518,0.840524) -- (4.73518,0.884851);
\draw [c] (4.73518,0.884851) -- (4.73518,0.929178);
\draw [c] (4.71714,0.884851) -- (4.73518,0.884851);
\draw [c] (4.73518,0.884851) -- (4.75323,0.884851);
\definecolor{c}{rgb}{0,0,0};
\colorlet{c}{natgreen};
\draw [c] (4.77127,0.780186) -- (4.77127,0.813231);
\draw [c] (4.77127,0.813231) -- (4.77127,0.846275);
\draw [c] (4.75323,0.813231) -- (4.77127,0.813231);
\draw [c] (4.77127,0.813231) -- (4.78931,0.813231);
\definecolor{c}{rgb}{0,0,0};
\colorlet{c}{natgreen};
\draw [c] (4.80736,0.813031) -- (4.80736,0.853508);
\draw [c] (4.80736,0.853508) -- (4.80736,0.893985);
\draw [c] (4.78931,0.853508) -- (4.80736,0.853508);
\draw [c] (4.80736,0.853508) -- (4.8254,0.853508);
\definecolor{c}{rgb}{0,0,0};
\colorlet{c}{natgreen};
\draw [c] (4.84345,0.793976) -- (4.84345,0.828606);
\draw [c] (4.84345,0.828606) -- (4.84345,0.863236);
\draw [c] (4.8254,0.828606) -- (4.84345,0.828606);
\draw [c] (4.84345,0.828606) -- (4.86149,0.828606);
\definecolor{c}{rgb}{0,0,0};
\colorlet{c}{natgreen};
\draw [c] (4.87954,0.758384) -- (4.87954,0.784644);
\draw [c] (4.87954,0.784644) -- (4.87954,0.810903);
\draw [c] (4.86149,0.784644) -- (4.87954,0.784644);
\draw [c] (4.87954,0.784644) -- (4.89758,0.784644);
\definecolor{c}{rgb}{0,0,0};
\colorlet{c}{natgreen};
\draw [c] (4.91563,0.758837) -- (4.91563,0.787174);
\draw [c] (4.91563,0.787174) -- (4.91563,0.815512);
\draw [c] (4.89758,0.787174) -- (4.91563,0.787174);
\draw [c] (4.91563,0.787174) -- (4.93367,0.787174);
\definecolor{c}{rgb}{0,0,0};
\colorlet{c}{natgreen};
\draw [c] (4.95171,0.761736) -- (4.95171,0.785406);
\draw [c] (4.95171,0.785406) -- (4.95171,0.809077);
\draw [c] (4.93367,0.785406) -- (4.95171,0.785406);
\draw [c] (4.95171,0.785406) -- (4.96976,0.785406);
\definecolor{c}{rgb}{0,0,0};
\colorlet{c}{natgreen};
\draw [c] (4.9878,0.800978) -- (4.9878,0.832609);
\draw [c] (4.9878,0.832609) -- (4.9878,0.86424);
\draw [c] (4.96976,0.832609) -- (4.9878,0.832609);
\draw [c] (4.9878,0.832609) -- (5.00585,0.832609);
\definecolor{c}{rgb}{0,0,0};
\colorlet{c}{natgreen};
\draw [c] (5.02389,0.775133) -- (5.02389,0.80439);
\draw [c] (5.02389,0.80439) -- (5.02389,0.833647);
\draw [c] (5.00585,0.80439) -- (5.02389,0.80439);
\draw [c] (5.02389,0.80439) -- (5.04194,0.80439);
\definecolor{c}{rgb}{0,0,0};
\colorlet{c}{natgreen};
\draw [c] (5.05998,0.742255) -- (5.05998,0.760704);
\draw [c] (5.05998,0.760704) -- (5.05998,0.779153);
\draw [c] (5.04194,0.760704) -- (5.05998,0.760704);
\draw [c] (5.05998,0.760704) -- (5.07802,0.760704);
\definecolor{c}{rgb}{0,0,0};
\colorlet{c}{natgreen};
\draw [c] (5.09607,0.751018) -- (5.09607,0.773695);
\draw [c] (5.09607,0.773695) -- (5.09607,0.796372);
\draw [c] (5.07802,0.773695) -- (5.09607,0.773695);
\draw [c] (5.09607,0.773695) -- (5.11411,0.773695);
\definecolor{c}{rgb}{0,0,0};
\colorlet{c}{natgreen};
\draw [c] (5.13216,0.812429) -- (5.13216,0.851758);
\draw [c] (5.13216,0.851758) -- (5.13216,0.891088);
\draw [c] (5.11411,0.851758) -- (5.13216,0.851758);
\draw [c] (5.13216,0.851758) -- (5.1502,0.851758);
\definecolor{c}{rgb}{0,0,0};
\colorlet{c}{natgreen};
\draw [c] (5.16825,0.747876) -- (5.16825,0.76617);
\draw [c] (5.16825,0.76617) -- (5.16825,0.784464);
\draw [c] (5.1502,0.76617) -- (5.16825,0.76617);
\draw [c] (5.16825,0.76617) -- (5.18629,0.76617);
\definecolor{c}{rgb}{0,0,0};
\colorlet{c}{natgreen};
\draw [c] (5.20433,0.757096) -- (5.20433,0.780973);
\draw [c] (5.20433,0.780973) -- (5.20433,0.804849);
\draw [c] (5.18629,0.780973) -- (5.20433,0.780973);
\draw [c] (5.20433,0.780973) -- (5.22238,0.780973);
\definecolor{c}{rgb}{0,0,0};
\colorlet{c}{natgreen};
\draw [c] (5.24042,0.780723) -- (5.24042,0.812674);
\draw [c] (5.24042,0.812674) -- (5.24042,0.844625);
\draw [c] (5.22238,0.812674) -- (5.24042,0.812674);
\draw [c] (5.24042,0.812674) -- (5.25847,0.812674);
\definecolor{c}{rgb}{0,0,0};
\colorlet{c}{natgreen};
\draw [c] (5.27651,0.746262) -- (5.27651,0.76547);
\draw [c] (5.27651,0.76547) -- (5.27651,0.784677);
\draw [c] (5.25847,0.76547) -- (5.27651,0.76547);
\draw [c] (5.27651,0.76547) -- (5.29456,0.76547);
\definecolor{c}{rgb}{0,0,0};
\colorlet{c}{natgreen};
\draw [c] (5.3126,0.77607) -- (5.3126,0.803785);
\draw [c] (5.3126,0.803785) -- (5.3126,0.8315);
\draw [c] (5.29456,0.803785) -- (5.3126,0.803785);
\draw [c] (5.3126,0.803785) -- (5.33065,0.803785);
\definecolor{c}{rgb}{0,0,0};
\colorlet{c}{natgreen};
\draw [c] (5.34869,0.75608) -- (5.34869,0.778446);
\draw [c] (5.34869,0.778446) -- (5.34869,0.800813);
\draw [c] (5.33065,0.778446) -- (5.34869,0.778446);
\draw [c] (5.34869,0.778446) -- (5.36673,0.778446);
\definecolor{c}{rgb}{0,0,0};
\colorlet{c}{natgreen};
\draw [c] (5.38478,0.752922) -- (5.38478,0.773923);
\draw [c] (5.38478,0.773923) -- (5.38478,0.794924);
\draw [c] (5.36673,0.773923) -- (5.38478,0.773923);
\draw [c] (5.38478,0.773923) -- (5.40282,0.773923);
\definecolor{c}{rgb}{0,0,0};
\colorlet{c}{natgreen};
\draw [c] (5.42087,0.742951) -- (5.42087,0.760468);
\draw [c] (5.42087,0.760468) -- (5.42087,0.777986);
\draw [c] (5.40282,0.760468) -- (5.42087,0.760468);
\draw [c] (5.42087,0.760468) -- (5.43891,0.760468);
\definecolor{c}{rgb}{0,0,0};
\colorlet{c}{natgreen};
\draw [c] (5.45696,0.742899) -- (5.45696,0.762868);
\draw [c] (5.45696,0.762868) -- (5.45696,0.782838);
\draw [c] (5.43891,0.762868) -- (5.45696,0.762868);
\draw [c] (5.45696,0.762868) -- (5.475,0.762868);
\definecolor{c}{rgb}{0,0,0};
\colorlet{c}{natgreen};
\draw [c] (5.49304,0.750905) -- (5.49304,0.771726);
\draw [c] (5.49304,0.771726) -- (5.49304,0.792547);
\draw [c] (5.475,0.771726) -- (5.49304,0.771726);
\draw [c] (5.49304,0.771726) -- (5.51109,0.771726);
\definecolor{c}{rgb}{0,0,0};
\colorlet{c}{natgreen};
\draw [c] (5.52913,0.735553) -- (5.52913,0.742652);
\draw [c] (5.52913,0.742652) -- (5.52913,0.749751);
\draw [c] (5.51109,0.742652) -- (5.52913,0.742652);
\draw [c] (5.52913,0.742652) -- (5.54718,0.742652);
\definecolor{c}{rgb}{0,0,0};
\colorlet{c}{natgreen};
\draw [c] (5.56522,0.752318) -- (5.56522,0.776656);
\draw [c] (5.56522,0.776656) -- (5.56522,0.800993);
\draw [c] (5.54718,0.776656) -- (5.56522,0.776656);
\draw [c] (5.56522,0.776656) -- (5.58327,0.776656);
\definecolor{c}{rgb}{0,0,0};
\colorlet{c}{natgreen};
\draw [c] (5.60131,0.752888) -- (5.60131,0.779509);
\draw [c] (5.60131,0.779509) -- (5.60131,0.80613);
\draw [c] (5.58327,0.779509) -- (5.60131,0.779509);
\draw [c] (5.60131,0.779509) -- (5.61935,0.779509);
\definecolor{c}{rgb}{0,0,0};
\colorlet{c}{natgreen};
\draw [c] (5.6374,0.754683) -- (5.6374,0.778493);
\draw [c] (5.6374,0.778493) -- (5.6374,0.802304);
\draw [c] (5.61935,0.778493) -- (5.6374,0.778493);
\draw [c] (5.6374,0.778493) -- (5.65544,0.778493);
\definecolor{c}{rgb}{0,0,0};
\colorlet{c}{natgreen};
\draw [c] (5.67349,0.737725) -- (5.67349,0.748389);
\draw [c] (5.67349,0.748389) -- (5.67349,0.759052);
\draw [c] (5.65544,0.748389) -- (5.67349,0.748389);
\draw [c] (5.67349,0.748389) -- (5.69153,0.748389);
\definecolor{c}{rgb}{0,0,0};
\colorlet{c}{natgreen};
\draw [c] (5.70958,0.734632) -- (5.70958,0.741681);
\draw [c] (5.70958,0.741681) -- (5.70958,0.74873);
\draw [c] (5.69153,0.741681) -- (5.70958,0.741681);
\draw [c] (5.70958,0.741681) -- (5.72762,0.741681);
\definecolor{c}{rgb}{0,0,0};
\colorlet{c}{natgreen};
\draw [c] (5.74567,0.744543) -- (5.74567,0.763576);
\draw [c] (5.74567,0.763576) -- (5.74567,0.782609);
\draw [c] (5.72762,0.763576) -- (5.74567,0.763576);
\draw [c] (5.74567,0.763576) -- (5.76371,0.763576);
\definecolor{c}{rgb}{0,0,0};
\colorlet{c}{natgreen};
\draw [c] (5.81784,0.740037) -- (5.81784,0.755969);
\draw [c] (5.81784,0.755969) -- (5.81784,0.771901);
\draw [c] (5.7998,0.755969) -- (5.81784,0.755969);
\draw [c] (5.81784,0.755969) -- (5.83589,0.755969);
\definecolor{c}{rgb}{0,0,0};
\colorlet{c}{natgreen};
\draw [c] (5.89002,0.738984) -- (5.89002,0.753556);
\draw [c] (5.89002,0.753556) -- (5.89002,0.768129);
\draw [c] (5.87198,0.753556) -- (5.89002,0.753556);
\draw [c] (5.89002,0.753556) -- (5.90806,0.753556);
\definecolor{c}{rgb}{0,0,0};
\colorlet{c}{natgreen};
\draw [c] (5.99829,0.734632) -- (5.99829,0.7446);
\draw [c] (5.99829,0.7446) -- (5.99829,0.754568);
\draw [c] (5.98024,0.7446) -- (5.99829,0.7446);
\draw [c] (5.99829,0.7446) -- (6.01633,0.7446);
\definecolor{c}{rgb}{0,0,0};
\colorlet{c}{natgreen};
\draw [c] (6.03438,0.752867) -- (6.03438,0.777778);
\draw [c] (6.03438,0.777778) -- (6.03438,0.802688);
\draw [c] (6.01633,0.777778) -- (6.03438,0.777778);
\draw [c] (6.03438,0.777778) -- (6.05242,0.777778);
\definecolor{c}{rgb}{0,0,0};
\colorlet{c}{natgreen};
\draw [c] (6.14264,0.734633) -- (6.14264,0.734774);
\draw [c] (6.14264,0.734774) -- (6.14264,0.734916);
\draw [c] (6.1246,0.734774) -- (6.14264,0.734774);
\draw [c] (6.14264,0.734774) -- (6.16069,0.734774);
\definecolor{c}{rgb}{0,0,0};
\colorlet{c}{natgreen};
\draw [c] (6.17873,0.734632) -- (6.17873,0.74983);
\draw [c] (6.17873,0.74983) -- (6.17873,0.765027);
\draw [c] (6.16069,0.74983) -- (6.17873,0.74983);
\draw [c] (6.17873,0.74983) -- (6.19677,0.74983);
\definecolor{c}{rgb}{0,0,0};
\colorlet{c}{natgreen};
\draw [c] (6.21482,0.737785) -- (6.21482,0.745395);
\draw [c] (6.21482,0.745395) -- (6.21482,0.753006);
\draw [c] (6.19677,0.745395) -- (6.21482,0.745395);
\draw [c] (6.21482,0.745395) -- (6.23286,0.745395);
\definecolor{c}{rgb}{0,0,0};
\colorlet{c}{natgreen};
\draw [c] (6.25091,0.73666) -- (6.25091,0.750033);
\draw [c] (6.25091,0.750033) -- (6.25091,0.763407);
\draw [c] (6.23286,0.750033) -- (6.25091,0.750033);
\draw [c] (6.25091,0.750033) -- (6.26895,0.750033);
\definecolor{c}{rgb}{0,0,0};
\colorlet{c}{natgreen};
\draw [c] (6.287,0.734632) -- (6.287,0.747821);
\draw [c] (6.287,0.747821) -- (6.287,0.76101);
\draw [c] (6.26895,0.747821) -- (6.287,0.747821);
\draw [c] (6.287,0.747821) -- (6.30504,0.747821);
\definecolor{c}{rgb}{0,0,0};
\colorlet{c}{natgreen};
\draw [c] (6.32308,0.739441) -- (6.32308,0.751649);
\draw [c] (6.32308,0.751649) -- (6.32308,0.763857);
\draw [c] (6.30504,0.751649) -- (6.32308,0.751649);
\draw [c] (6.32308,0.751649) -- (6.34113,0.751649);
\definecolor{c}{rgb}{0,0,0};
\colorlet{c}{natgreen};
\draw [c] (6.39526,0.734632) -- (6.39526,0.740014);
\draw [c] (6.39526,0.740014) -- (6.39526,0.745395);
\draw [c] (6.37722,0.740014) -- (6.39526,0.740014);
\draw [c] (6.39526,0.740014) -- (6.41331,0.740014);
\definecolor{c}{rgb}{0,0,0};
\colorlet{c}{natgreen};
\draw [c] (6.43135,0.734632) -- (6.43135,0.741681);
\draw [c] (6.43135,0.741681) -- (6.43135,0.74873);
\draw [c] (6.41331,0.741681) -- (6.43135,0.741681);
\draw [c] (6.43135,0.741681) -- (6.4494,0.741681);
\definecolor{c}{rgb}{0,0,0};
\colorlet{c}{natgreen};
\draw [c] (6.53962,0.740056) -- (6.53962,0.756158);
\draw [c] (6.53962,0.756158) -- (6.53962,0.77226);
\draw [c] (6.52157,0.756158) -- (6.53962,0.756158);
\draw [c] (6.53962,0.756158) -- (6.55766,0.756158);
\definecolor{c}{rgb}{0,0,0};
\colorlet{c}{natgreen};
\draw [c] (6.57571,0.746809) -- (6.57571,0.767173);
\draw [c] (6.57571,0.767173) -- (6.57571,0.787537);
\draw [c] (6.55766,0.767173) -- (6.57571,0.767173);
\draw [c] (6.57571,0.767173) -- (6.59375,0.767173);
\definecolor{c}{rgb}{0,0,0};
\colorlet{c}{natgreen};
\draw [c] (6.72006,0.74203) -- (6.72006,0.762645);
\draw [c] (6.72006,0.762645) -- (6.72006,0.78326);
\draw [c] (6.70202,0.762645) -- (6.72006,0.762645);
\draw [c] (6.72006,0.762645) -- (6.7381,0.762645);
\definecolor{c}{rgb}{0,0,0};
\colorlet{c}{natgreen};
\draw [c] (6.75615,0.74324) -- (6.75615,0.764297);
\draw [c] (6.75615,0.764297) -- (6.75615,0.785353);
\draw [c] (6.7381,0.764297) -- (6.75615,0.764297);
\draw [c] (6.75615,0.764297) -- (6.77419,0.764297);
\definecolor{c}{rgb}{0,0,0};
\colorlet{c}{natgreen};
\draw [c] (7.00877,0.734632) -- (7.00877,0.736845);
\draw [c] (7.00877,0.736845) -- (7.00877,0.739057);
\draw [c] (6.99073,0.736845) -- (7.00877,0.736845);
\draw [c] (7.00877,0.736845) -- (7.02681,0.736845);
\definecolor{c}{rgb}{0,0,0};
\colorlet{c}{natgreen};
\draw [c] (7.08095,0.734632) -- (7.08095,0.747821);
\draw [c] (7.08095,0.747821) -- (7.08095,0.76101);
\draw [c] (7.0629,0.747821) -- (7.08095,0.747821);
\draw [c] (7.08095,0.747821) -- (7.09899,0.747821);
\definecolor{c}{rgb}{0,0,0};
\colorlet{c}{natgreen};
\draw [c] (7.18921,0.734632) -- (7.18921,0.746887);
\draw [c] (7.18921,0.746887) -- (7.18921,0.759142);
\draw [c] (7.17117,0.746887) -- (7.18921,0.746887);
\draw [c] (7.18921,0.746887) -- (7.20726,0.746887);
\definecolor{c}{rgb}{0,0,0};
\colorlet{c}{natgreen};
\draw [c] (7.33357,0.734632) -- (7.33357,0.749109);
\draw [c] (7.33357,0.749109) -- (7.33357,0.763586);
\draw [c] (7.31552,0.749109) -- (7.33357,0.749109);
\draw [c] (7.33357,0.749109) -- (7.35161,0.749109);
\definecolor{c}{rgb}{0,0,0};
\colorlet{c}{natgreen};
\draw [c] (7.44183,0.734632) -- (7.44183,0.735461);
\draw [c] (7.44183,0.735461) -- (7.44183,0.736291);
\draw [c] (7.42379,0.735461) -- (7.44183,0.735461);
\draw [c] (7.44183,0.735461) -- (7.45988,0.735461);
\definecolor{c}{rgb}{0,0,0};
\colorlet{c}{natgreen};
\draw [c] (7.73054,0.734632) -- (7.73054,0.741681);
\draw [c] (7.73054,0.741681) -- (7.73054,0.74873);
\draw [c] (7.7125,0.741681) -- (7.73054,0.741681);
\draw [c] (7.73054,0.741681) -- (7.74859,0.741681);
\definecolor{c}{rgb}{0,0,0};
\colorlet{c}{natgreen};
\draw [c] (7.83881,0.734632) -- (7.83881,0.743059);
\draw [c] (7.83881,0.743059) -- (7.83881,0.751485);
\draw [c] (7.82077,0.743059) -- (7.83881,0.743059);
\draw [c] (7.83881,0.743059) -- (7.85685,0.743059);
\definecolor{c}{rgb}{0,0,0};
\colorlet{c}{natgreen};
\draw [c] (7.91099,0.734632) -- (7.91099,0.739076);
\draw [c] (7.91099,0.739076) -- (7.91099,0.74352);
\draw [c] (7.89294,0.739076) -- (7.91099,0.739076);
\draw [c] (7.91099,0.739076) -- (7.92903,0.739076);
\definecolor{c}{rgb}{0,0,0};
\colorlet{c}{natgreen};
\draw [c] (7.94708,0.734632) -- (7.94708,0.735461);
\draw [c] (7.94708,0.735461) -- (7.94708,0.736291);
\draw [c] (7.92903,0.735461) -- (7.94708,0.735461);
\draw [c] (7.94708,0.735461) -- (7.96512,0.735461);
\definecolor{c}{rgb}{0,0,0};
\colorlet{c}{natgreen};
\draw [c] (8.01925,0.734632) -- (8.01925,0.743059);
\draw [c] (8.01925,0.743059) -- (8.01925,0.751485);
\draw [c] (8.00121,0.743059) -- (8.01925,0.743059);
\draw [c] (8.01925,0.743059) -- (8.0373,0.743059);
\definecolor{c}{rgb}{0,0,0};
\colorlet{c}{natgreen};
\draw [c] (8.05534,0.734632) -- (8.05534,0.743059);
\draw [c] (8.05534,0.743059) -- (8.05534,0.751485);
\draw [c] (8.0373,0.743059) -- (8.05534,0.743059);
\draw [c] (8.05534,0.743059) -- (8.07339,0.743059);
\definecolor{c}{rgb}{0,0,0};
\colorlet{c}{natgreen};
\draw [c] (8.09143,0.734632) -- (8.09143,0.745781);
\draw [c] (8.09143,0.745781) -- (8.09143,0.756929);
\draw [c] (8.07339,0.745781) -- (8.09143,0.745781);
\draw [c] (8.09143,0.745781) -- (8.10948,0.745781);
\definecolor{c}{rgb}{0,0,0};
\colorlet{c}{natgreen};
\draw [c] (8.12752,0.734632) -- (8.12752,0.740014);
\draw [c] (8.12752,0.740014) -- (8.12752,0.745395);
\draw [c] (8.10948,0.740014) -- (8.12752,0.740014);
\draw [c] (8.12752,0.740014) -- (8.14556,0.740014);
\definecolor{c}{rgb}{0,0,0};
\colorlet{c}{natgreen};
\draw [c] (8.23579,0.734632) -- (8.23579,0.74892);
\draw [c] (8.23579,0.74892) -- (8.23579,0.763207);
\draw [c] (8.21774,0.74892) -- (8.23579,0.74892);
\draw [c] (8.23579,0.74892) -- (8.25383,0.74892);
\definecolor{c}{rgb}{0,0,0};
\colorlet{c}{natgreen};
\draw [c] (8.88538,0.734632) -- (8.88538,0.750764);
\draw [c] (8.88538,0.750764) -- (8.88538,0.766895);
\draw [c] (8.86734,0.750764) -- (8.88538,0.750764);
\draw [c] (8.88538,0.750764) -- (8.90343,0.750764);
\definecolor{c}{rgb}{0,0,0};
\colorlet{c}{natgreen};
\draw [c] (8.95756,0.734632) -- (8.95756,0.748166);
\draw [c] (8.95756,0.748166) -- (8.95756,0.761699);
\draw [c] (8.93952,0.748166) -- (8.95756,0.748166);
\draw [c] (8.95756,0.748166) -- (8.97561,0.748166);
\definecolor{c}{rgb}{0,0,0};
\colorlet{c}{natgreen};
\draw [c] (9.10192,0.734632) -- (9.10192,0.7446);
\draw [c] (9.10192,0.7446) -- (9.10192,0.754568);
\draw [c] (9.08387,0.7446) -- (9.10192,0.7446);
\draw [c] (9.10192,0.7446) -- (9.11996,0.7446);
\definecolor{c}{rgb}{0,0,0};
\colorlet{c}{natgreen};
\draw [c] (9.17409,0.734632) -- (9.17409,0.743059);
\draw [c] (9.17409,0.743059) -- (9.17409,0.751485);
\draw [c] (9.15605,0.743059) -- (9.17409,0.743059);
\draw [c] (9.17409,0.743059) -- (9.19214,0.743059);
\definecolor{c}{rgb}{0,0,0};
\colorlet{c}{natgreen};
\draw [c] (9.53498,0.734632) -- (9.53498,0.746887);
\draw [c] (9.53498,0.746887) -- (9.53498,0.759142);
\draw [c] (9.51694,0.746887) -- (9.53498,0.746887);
\draw [c] (9.53498,0.746887) -- (9.55302,0.746887);
\definecolor{c}{rgb}{0,0,0};
\draw [c] (1,0.680516) -- (9.95,0.680516);
\draw [anchor= east] (9.95,-0.0816619) node[color=c, rotate=0]{$E_{T}^{\text{iso}}\text{ [GeV]}$};
\draw [c] (1,0.863234) -- (1,0.680516);
\draw [c] (1.36089,0.771875) -- (1.36089,0.680516);
\draw [c] (1.72177,0.771875) -- (1.72177,0.680516);
\draw [c] (2.08266,0.771875) -- (2.08266,0.680516);
\draw [c] (2.44355,0.771875) -- (2.44355,0.680516);
\draw [c] (2.80444,0.863234) -- (2.80444,0.680516);
\draw [c] (3.16532,0.771875) -- (3.16532,0.680516);
\draw [c] (3.52621,0.771875) -- (3.52621,0.680516);
\draw [c] (3.8871,0.771875) -- (3.8871,0.680516);
\draw [c] (4.24798,0.771875) -- (4.24798,0.680516);
\draw [c] (4.60887,0.863234) -- (4.60887,0.680516);
\draw [c] (4.96976,0.771875) -- (4.96976,0.680516);
\draw [c] (5.33065,0.771875) -- (5.33065,0.680516);
\draw [c] (5.69153,0.771875) -- (5.69153,0.680516);
\draw [c] (6.05242,0.771875) -- (6.05242,0.680516);
\draw [c] (6.41331,0.863234) -- (6.41331,0.680516);
\draw [c] (6.77419,0.771875) -- (6.77419,0.680516);
\draw [c] (7.13508,0.771875) -- (7.13508,0.680516);
\draw [c] (7.49597,0.771875) -- (7.49597,0.680516);
\draw [c] (7.85685,0.771875) -- (7.85685,0.680516);
\draw [c] (8.21774,0.863234) -- (8.21774,0.680516);
\draw [c] (8.21774,0.863234) -- (8.21774,0.680516);
\draw [c] (8.57863,0.771875) -- (8.57863,0.680516);
\draw [c] (8.93952,0.771875) -- (8.93952,0.680516);
\draw [c] (9.3004,0.771875) -- (9.3004,0.680516);
\draw [c] (9.66129,0.771875) -- (9.66129,0.680516);
\draw [anchor=base] (1,0.285817) node[color=c, rotate=0]{-5};
\draw [anchor=base] (2.80444,0.285817) node[color=c, rotate=0]{0};
\draw [anchor=base] (4.60887,0.285817) node[color=c, rotate=0]{5};
\draw [anchor=base] (6.41331,0.285817) node[color=c, rotate=0]{10};
\draw [anchor=base] (8.21774,0.285817) node[color=c, rotate=0]{15};
\draw [c] (1,0.680516) -- (1,6.73711);
\draw [anchor= east] (-0.4,6.73711) node[color=c, rotate=90]{Normalised number of events};
\draw [c] (1.267,0.734632) -- (1,0.734632);
\draw [c] (1.1335,0.864132) -- (1,0.864132);
\draw [c] (1.1335,0.993631) -- (1,0.993631);
\draw [c] (1.1335,1.12313) -- (1,1.12313);
\draw [c] (1.1335,1.25263) -- (1,1.25263);
\draw [c] (1.267,1.38213) -- (1,1.38213);
\draw [c] (1.1335,1.51163) -- (1,1.51163);
\draw [c] (1.1335,1.64113) -- (1,1.64113);
\draw [c] (1.1335,1.77063) -- (1,1.77063);
\draw [c] (1.1335,1.90013) -- (1,1.90013);
\draw [c] (1.267,2.02963) -- (1,2.02963);
\draw [c] (1.1335,2.15913) -- (1,2.15913);
\draw [c] (1.1335,2.28863) -- (1,2.28863);
\draw [c] (1.1335,2.41812) -- (1,2.41812);
\draw [c] (1.1335,2.54762) -- (1,2.54762);
\draw [c] (1.267,2.67712) -- (1,2.67712);
\draw [c] (1.1335,2.80662) -- (1,2.80662);
\draw [c] (1.1335,2.93612) -- (1,2.93612);
\draw [c] (1.1335,3.06562) -- (1,3.06562);
\draw [c] (1.1335,3.19512) -- (1,3.19512);
\draw [c] (1.267,3.32462) -- (1,3.32462);
\draw [c] (1.1335,3.45412) -- (1,3.45412);
\draw [c] (1.1335,3.58362) -- (1,3.58362);
\draw [c] (1.1335,3.71312) -- (1,3.71312);
\draw [c] (1.1335,3.84262) -- (1,3.84262);
\draw [c] (1.267,3.97212) -- (1,3.97212);
\draw [c] (1.1335,4.10162) -- (1,4.10162);
\draw [c] (1.1335,4.23112) -- (1,4.23112);
\draw [c] (1.1335,4.36062) -- (1,4.36062);
\draw [c] (1.1335,4.49012) -- (1,4.49012);
\draw [c] (1.267,4.61962) -- (1,4.61962);
\draw [c] (1.1335,4.74911) -- (1,4.74911);
\draw [c] (1.1335,4.87861) -- (1,4.87861);
\draw [c] (1.1335,5.00811) -- (1,5.00811);
\draw [c] (1.1335,5.13761) -- (1,5.13761);
\draw [c] (1.267,5.26711) -- (1,5.26711);
\draw [c] (1.1335,5.39661) -- (1,5.39661);
\draw [c] (1.1335,5.52611) -- (1,5.52611);
\draw [c] (1.1335,5.65561) -- (1,5.65561);
\draw [c] (1.1335,5.78511) -- (1,5.78511);
\draw [c] (1.267,5.91461) -- (1,5.91461);
\draw [c] (1.1335,6.04411) -- (1,6.04411);
\draw [c] (1.1335,6.17361) -- (1,6.17361);
\draw [c] (1.1335,6.30311) -- (1,6.30311);
\draw [c] (1.1335,6.43261) -- (1,6.43261);
\draw [c] (1.267,6.56211) -- (1,6.56211);
\draw [c] (1.267,0.734632) -- (1,0.734632);
\draw [c] (1.267,6.56211) -- (1,6.56211);
\draw [c] (1.1335,6.69161) -- (1,6.69161);
\draw [anchor= east] (0.95,0.734632) node[color=c, rotate=0]{0};
\draw [anchor= east] (0.95,1.38213) node[color=c, rotate=0]{100};
\draw [anchor= east] (0.95,2.02963) node[color=c, rotate=0]{200};
\draw [anchor= east] (0.95,2.67712) node[color=c, rotate=0]{300};
\draw [anchor= east] (0.95,3.32462) node[color=c, rotate=0]{400};
\draw [anchor= east] (0.95,3.97212) node[color=c, rotate=0]{500};
\draw [anchor= east] (0.95,4.61962) node[color=c, rotate=0]{600};
\draw [anchor= east] (0.95,5.26711) node[color=c, rotate=0]{700};
\draw [anchor= east] (0.95,5.91461) node[color=c, rotate=0]{800};
\draw [anchor= east] (0.95,6.56211) node[color=c, rotate=0]{900};
\colorlet{c}{natcomp!70};
\draw [c] (1.01804,0.734643) -- (1.01804,0.756364);
\draw [c] (1.01804,0.756364) -- (1.01804,0.778085);
\draw [c] (1,0.756364) -- (1.01804,0.756364);
\draw [c] (1.01804,0.756364) -- (1.03609,0.756364);
\definecolor{c}{rgb}{0,0,0};
\colorlet{c}{natcomp!70};
\draw [c] (1.05413,0.734632) -- (1.05413,0.734644);
\draw [c] (1.05413,0.734644) -- (1.05413,0.734656);
\draw [c] (1.03609,0.734644) -- (1.05413,0.734644);
\draw [c] (1.05413,0.734644) -- (1.07218,0.734644);
\definecolor{c}{rgb}{0,0,0};
\colorlet{c}{natcomp!70};
\draw [c] (1.19849,0.734643) -- (1.19849,0.764169);
\draw [c] (1.19849,0.764169) -- (1.19849,0.793695);
\draw [c] (1.18044,0.764169) -- (1.19849,0.764169);
\draw [c] (1.19849,0.764169) -- (1.21653,0.764169);
\definecolor{c}{rgb}{0,0,0};
\colorlet{c}{natcomp!70};
\draw [c] (1.23458,0.734636) -- (1.23458,0.734645);
\draw [c] (1.23458,0.734645) -- (1.23458,0.734654);
\draw [c] (1.21653,0.734645) -- (1.23458,0.734645);
\draw [c] (1.23458,0.734645) -- (1.25262,0.734645);
\definecolor{c}{rgb}{0,0,0};
\colorlet{c}{natcomp!70};
\draw [c] (1.27067,0.73464) -- (1.27067,0.756361);
\draw [c] (1.27067,0.756361) -- (1.27067,0.778083);
\draw [c] (1.25262,0.756361) -- (1.27067,0.756361);
\draw [c] (1.27067,0.756361) -- (1.28871,0.756361);
\definecolor{c}{rgb}{0,0,0};
\colorlet{c}{natcomp!70};
\draw [c] (1.30675,0.74249) -- (1.30675,0.769762);
\draw [c] (1.30675,0.769762) -- (1.30675,0.797034);
\draw [c] (1.28871,0.769762) -- (1.30675,0.769762);
\draw [c] (1.30675,0.769762) -- (1.3248,0.769762);
\definecolor{c}{rgb}{0,0,0};
\colorlet{c}{natcomp!70};
\draw [c] (1.34284,0.734678) -- (1.34284,0.756399);
\draw [c] (1.34284,0.756399) -- (1.34284,0.77812);
\draw [c] (1.3248,0.756399) -- (1.34284,0.756399);
\draw [c] (1.34284,0.756399) -- (1.36089,0.756399);
\definecolor{c}{rgb}{0,0,0};
\colorlet{c}{natcomp!70};
\draw [c] (1.37893,0.74505) -- (1.37893,0.777385);
\draw [c] (1.37893,0.777385) -- (1.37893,0.80972);
\draw [c] (1.36089,0.777385) -- (1.37893,0.777385);
\draw [c] (1.37893,0.777385) -- (1.39698,0.777385);
\definecolor{c}{rgb}{0,0,0};
\colorlet{c}{natcomp!70};
\draw [c] (1.41502,0.751247) -- (1.41502,0.783932);
\draw [c] (1.41502,0.783932) -- (1.41502,0.816618);
\draw [c] (1.39698,0.783932) -- (1.41502,0.783932);
\draw [c] (1.41502,0.783932) -- (1.43306,0.783932);
\definecolor{c}{rgb}{0,0,0};
\colorlet{c}{natcomp!70};
\draw [c] (1.45111,0.774293) -- (1.45111,0.81838);
\draw [c] (1.45111,0.81838) -- (1.45111,0.862466);
\draw [c] (1.43306,0.81838) -- (1.45111,0.81838);
\draw [c] (1.45111,0.81838) -- (1.46915,0.81838);
\definecolor{c}{rgb}{0,0,0};
\colorlet{c}{natcomp!70};
\draw [c] (1.4872,0.759731) -- (1.4872,0.794569);
\draw [c] (1.4872,0.794569) -- (1.4872,0.829406);
\draw [c] (1.46915,0.794569) -- (1.4872,0.794569);
\draw [c] (1.4872,0.794569) -- (1.50524,0.794569);
\definecolor{c}{rgb}{0,0,0};
\colorlet{c}{natcomp!70};
\draw [c] (1.52329,0.752172) -- (1.52329,0.776524);
\draw [c] (1.52329,0.776524) -- (1.52329,0.800876);
\draw [c] (1.50524,0.776524) -- (1.52329,0.776524);
\draw [c] (1.52329,0.776524) -- (1.54133,0.776524);
\definecolor{c}{rgb}{0,0,0};
\colorlet{c}{natcomp!70};
\draw [c] (1.55938,0.734713) -- (1.55938,0.756435);
\draw [c] (1.55938,0.756435) -- (1.55938,0.778156);
\draw [c] (1.54133,0.756435) -- (1.55938,0.756435);
\draw [c] (1.55938,0.756435) -- (1.57742,0.756435);
\definecolor{c}{rgb}{0,0,0};
\colorlet{c}{natcomp!70};
\draw [c] (1.59546,0.763306) -- (1.59546,0.806425);
\draw [c] (1.59546,0.806425) -- (1.59546,0.849545);
\draw [c] (1.57742,0.806425) -- (1.59546,0.806425);
\draw [c] (1.59546,0.806425) -- (1.61351,0.806425);
\definecolor{c}{rgb}{0,0,0};
\colorlet{c}{natcomp!70};
\draw [c] (1.63155,0.784078) -- (1.63155,0.823256);
\draw [c] (1.63155,0.823256) -- (1.63155,0.862434);
\draw [c] (1.61351,0.823256) -- (1.63155,0.823256);
\draw [c] (1.63155,0.823256) -- (1.6496,0.823256);
\definecolor{c}{rgb}{0,0,0};
\colorlet{c}{natcomp!70};
\draw [c] (1.66764,0.800855) -- (1.66764,0.840914);
\draw [c] (1.66764,0.840914) -- (1.66764,0.880972);
\draw [c] (1.6496,0.840914) -- (1.66764,0.840914);
\draw [c] (1.66764,0.840914) -- (1.68569,0.840914);
\definecolor{c}{rgb}{0,0,0};
\colorlet{c}{natcomp!70};
\draw [c] (1.70373,0.898383) -- (1.70373,0.953898);
\draw [c] (1.70373,0.953898) -- (1.70373,1.00941);
\draw [c] (1.68569,0.953898) -- (1.70373,0.953898);
\draw [c] (1.70373,0.953898) -- (1.72177,0.953898);
\definecolor{c}{rgb}{0,0,0};
\colorlet{c}{natcomp!70};
\draw [c] (1.73982,0.798717) -- (1.73982,0.84012);
\draw [c] (1.73982,0.84012) -- (1.73982,0.881522);
\draw [c] (1.72177,0.84012) -- (1.73982,0.84012);
\draw [c] (1.73982,0.84012) -- (1.75786,0.84012);
\definecolor{c}{rgb}{0,0,0};
\colorlet{c}{natcomp!70};
\draw [c] (1.77591,0.920053) -- (1.77591,0.975511);
\draw [c] (1.77591,0.975511) -- (1.77591,1.03097);
\draw [c] (1.75786,0.975511) -- (1.77591,0.975511);
\draw [c] (1.77591,0.975511) -- (1.79395,0.975511);
\definecolor{c}{rgb}{0,0,0};
\colorlet{c}{natcomp!70};
\draw [c] (1.812,0.936516) -- (1.812,0.998493);
\draw [c] (1.812,0.998493) -- (1.812,1.06047);
\draw [c] (1.79395,0.998493) -- (1.812,0.998493);
\draw [c] (1.812,0.998493) -- (1.83004,0.998493);
\definecolor{c}{rgb}{0,0,0};
\colorlet{c}{natcomp!70};
\draw [c] (1.84808,1.18919) -- (1.84808,1.28414);
\draw [c] (1.84808,1.28414) -- (1.84808,1.3791);
\draw [c] (1.83004,1.28414) -- (1.84808,1.28414);
\draw [c] (1.84808,1.28414) -- (1.86613,1.28414);
\definecolor{c}{rgb}{0,0,0};
\colorlet{c}{natcomp!70};
\draw [c] (1.88417,1.12118) -- (1.88417,1.20085);
\draw [c] (1.88417,1.20085) -- (1.88417,1.28052);
\draw [c] (1.86613,1.20085) -- (1.88417,1.20085);
\draw [c] (1.88417,1.20085) -- (1.90222,1.20085);
\definecolor{c}{rgb}{0,0,0};
\colorlet{c}{natcomp!70};
\draw [c] (1.92026,1.14624) -- (1.92026,1.22811);
\draw [c] (1.92026,1.22811) -- (1.92026,1.30998);
\draw [c] (1.90222,1.22811) -- (1.92026,1.22811);
\draw [c] (1.92026,1.22811) -- (1.93831,1.22811);
\definecolor{c}{rgb}{0,0,0};
\colorlet{c}{natcomp!70};
\draw [c] (1.95635,1.30269) -- (1.95635,1.39258);
\draw [c] (1.95635,1.39258) -- (1.95635,1.48248);
\draw [c] (1.93831,1.39258) -- (1.95635,1.39258);
\draw [c] (1.95635,1.39258) -- (1.9744,1.39258);
\definecolor{c}{rgb}{0,0,0};
\colorlet{c}{natcomp!70};
\draw [c] (1.99244,1.33385) -- (1.99244,1.42513);
\draw [c] (1.99244,1.42513) -- (1.99244,1.51641);
\draw [c] (1.9744,1.42513) -- (1.99244,1.42513);
\draw [c] (1.99244,1.42513) -- (2.01048,1.42513);
\definecolor{c}{rgb}{0,0,0};
\colorlet{c}{natcomp!70};
\draw [c] (2.02853,1.50746) -- (2.02853,1.6117);
\draw [c] (2.02853,1.6117) -- (2.02853,1.71594);
\draw [c] (2.01048,1.6117) -- (2.02853,1.6117);
\draw [c] (2.02853,1.6117) -- (2.04657,1.6117);
\definecolor{c}{rgb}{0,0,0};
\colorlet{c}{natcomp!70};
\draw [c] (2.06462,1.66914) -- (2.06462,1.7831);
\draw [c] (2.06462,1.7831) -- (2.06462,1.89706);
\draw [c] (2.04657,1.7831) -- (2.06462,1.7831);
\draw [c] (2.06462,1.7831) -- (2.08266,1.7831);
\definecolor{c}{rgb}{0,0,0};
\colorlet{c}{natcomp!70};
\draw [c] (2.10071,1.71336) -- (2.10071,1.83223);
\draw [c] (2.10071,1.83223) -- (2.10071,1.9511);
\draw [c] (2.08266,1.83223) -- (2.10071,1.83223);
\draw [c] (2.10071,1.83223) -- (2.11875,1.83223);
\definecolor{c}{rgb}{0,0,0};
\colorlet{c}{natcomp!70};
\draw [c] (2.13679,2.01604) -- (2.13679,2.14978);
\draw [c] (2.13679,2.14978) -- (2.13679,2.28352);
\draw [c] (2.11875,2.14978) -- (2.13679,2.14978);
\draw [c] (2.13679,2.14978) -- (2.15484,2.14978);
\definecolor{c}{rgb}{0,0,0};
\colorlet{c}{natcomp!70};
\draw [c] (2.17288,2.19175) -- (2.17288,2.33083);
\draw [c] (2.17288,2.33083) -- (2.17288,2.46991);
\draw [c] (2.15484,2.33083) -- (2.17288,2.33083);
\draw [c] (2.17288,2.33083) -- (2.19093,2.33083);
\definecolor{c}{rgb}{0,0,0};
\colorlet{c}{natcomp!70};
\draw [c] (2.20897,2.44823) -- (2.20897,2.5985);
\draw [c] (2.20897,2.5985) -- (2.20897,2.74878);
\draw [c] (2.19093,2.5985) -- (2.20897,2.5985);
\draw [c] (2.20897,2.5985) -- (2.22702,2.5985);
\definecolor{c}{rgb}{0,0,0};
\colorlet{c}{natcomp!70};
\draw [c] (2.24506,2.63778) -- (2.24506,2.79551);
\draw [c] (2.24506,2.79551) -- (2.24506,2.95324);
\draw [c] (2.22702,2.79551) -- (2.24506,2.79551);
\draw [c] (2.24506,2.79551) -- (2.2631,2.79551);
\definecolor{c}{rgb}{0,0,0};
\colorlet{c}{natcomp!70};
\draw [c] (2.28115,2.66197) -- (2.28115,2.81917);
\draw [c] (2.28115,2.81917) -- (2.28115,2.97637);
\draw [c] (2.2631,2.81917) -- (2.28115,2.81917);
\draw [c] (2.28115,2.81917) -- (2.29919,2.81917);
\definecolor{c}{rgb}{0,0,0};
\colorlet{c}{natcomp!70};
\draw [c] (2.31724,3.3542) -- (2.31724,3.53769);
\draw [c] (2.31724,3.53769) -- (2.31724,3.72117);
\draw [c] (2.29919,3.53769) -- (2.31724,3.53769);
\draw [c] (2.31724,3.53769) -- (2.33528,3.53769);
\definecolor{c}{rgb}{0,0,0};
\colorlet{c}{natcomp!70};
\draw [c] (2.35333,3.50262) -- (2.35333,3.68753);
\draw [c] (2.35333,3.68753) -- (2.35333,3.87244);
\draw [c] (2.33528,3.68753) -- (2.35333,3.68753);
\draw [c] (2.35333,3.68753) -- (2.37137,3.68753);
\definecolor{c}{rgb}{0,0,0};
\colorlet{c}{natcomp!70};
\draw [c] (2.38942,3.94866) -- (2.38942,4.15279);
\draw [c] (2.38942,4.15279) -- (2.38942,4.35691);
\draw [c] (2.37137,4.15279) -- (2.38942,4.15279);
\draw [c] (2.38942,4.15279) -- (2.40746,4.15279);
\definecolor{c}{rgb}{0,0,0};
\colorlet{c}{natcomp!70};
\draw [c] (2.4255,4.34127) -- (2.4255,4.55624);
\draw [c] (2.4255,4.55624) -- (2.4255,4.77121);
\draw [c] (2.40746,4.55624) -- (2.4255,4.55624);
\draw [c] (2.4255,4.55624) -- (2.44355,4.55624);
\definecolor{c}{rgb}{0,0,0};
\colorlet{c}{natcomp!70};
\draw [c] (2.46159,4.60059) -- (2.46159,4.81922);
\draw [c] (2.46159,4.81922) -- (2.46159,5.03785);
\draw [c] (2.44355,4.81922) -- (2.46159,4.81922);
\draw [c] (2.46159,4.81922) -- (2.47964,4.81922);
\definecolor{c}{rgb}{0,0,0};
\colorlet{c}{natcomp!70};
\draw [c] (2.49768,5.39548) -- (2.49768,5.64371);
\draw [c] (2.49768,5.64371) -- (2.49768,5.89194);
\draw [c] (2.47964,5.64371) -- (2.49768,5.64371);
\draw [c] (2.49768,5.64371) -- (2.51573,5.64371);
\definecolor{c}{rgb}{0,0,0};
\colorlet{c}{natcomp!70};
\draw [c] (2.53377,5.06867) -- (2.53377,5.3066);
\draw [c] (2.53377,5.3066) -- (2.53377,5.54453);
\draw [c] (2.51573,5.3066) -- (2.53377,5.3066);
\draw [c] (2.53377,5.3066) -- (2.55181,5.3066);
\definecolor{c}{rgb}{0,0,0};
\colorlet{c}{natcomp!70};
\draw [c] (2.56986,5.88276) -- (2.56986,6.15509);
\draw [c] (2.56986,6.15509) -- (2.56986,6.42741);
\draw [c] (2.55181,6.15509) -- (2.56986,6.15509);
\draw [c] (2.56986,6.15509) -- (2.5879,6.15509);
\definecolor{c}{rgb}{0,0,0};
\colorlet{c}{natcomp!70};
\draw [c] (2.60595,6.14439) -- (2.60595,6.41923);
\draw [c] (2.60595,6.41923) -- (2.60595,6.69406);
\draw [c] (2.5879,6.41923) -- (2.60595,6.41923);
\draw [c] (2.60595,6.41923) -- (2.62399,6.41923);
\definecolor{c}{rgb}{0,0,0};
\colorlet{c}{natcomp!70};
\draw [c] (2.64204,6.13672) -- (2.64204,6.41349);
\draw [c] (2.64204,6.41349) -- (2.64204,6.69026);
\draw [c] (2.62399,6.41349) -- (2.64204,6.41349);
\draw [c] (2.64204,6.41349) -- (2.66008,6.41349);
\definecolor{c}{rgb}{0,0,0};
\colorlet{c}{natcomp!70};
\draw [c] (2.67813,5.66473) -- (2.67813,5.93277);
\draw [c] (2.67813,5.93277) -- (2.67813,6.20081);
\draw [c] (2.66008,5.93277) -- (2.67813,5.93277);
\draw [c] (2.67813,5.93277) -- (2.69617,5.93277);
\definecolor{c}{rgb}{0,0,0};
\colorlet{c}{natcomp!70};
\draw [c] (2.71421,5.38953) -- (2.71421,5.65592);
\draw [c] (2.71421,5.65592) -- (2.71421,5.92231);
\draw [c] (2.69617,5.65592) -- (2.71421,5.65592);
\draw [c] (2.71421,5.65592) -- (2.73226,5.65592);
\definecolor{c}{rgb}{0,0,0};
\colorlet{c}{natcomp!70};
\draw [c] (2.7503,4.96565) -- (2.7503,5.21733);
\draw [c] (2.7503,5.21733) -- (2.7503,5.46902);
\draw [c] (2.73226,5.21733) -- (2.7503,5.21733);
\draw [c] (2.7503,5.21733) -- (2.76835,5.21733);
\definecolor{c}{rgb}{0,0,0};
\colorlet{c}{natcomp!70};
\draw [c] (2.78639,4.72809) -- (2.78639,4.97622);
\draw [c] (2.78639,4.97622) -- (2.78639,5.22435);
\draw [c] (2.76835,4.97622) -- (2.78639,4.97622);
\draw [c] (2.78639,4.97622) -- (2.80444,4.97622);
\definecolor{c}{rgb}{0,0,0};
\colorlet{c}{natcomp!70};
\draw [c] (2.82248,4.72571) -- (2.82248,4.96387);
\draw [c] (2.82248,4.96387) -- (2.82248,5.20204);
\draw [c] (2.80444,4.96387) -- (2.82248,4.96387);
\draw [c] (2.82248,4.96387) -- (2.84052,4.96387);
\definecolor{c}{rgb}{0,0,0};
\colorlet{c}{natcomp!70};
\draw [c] (2.85857,4.18523) -- (2.85857,4.403);
\draw [c] (2.85857,4.403) -- (2.85857,4.62076);
\draw [c] (2.84052,4.403) -- (2.85857,4.403);
\draw [c] (2.85857,4.403) -- (2.87661,4.403);
\definecolor{c}{rgb}{0,0,0};
\colorlet{c}{natcomp!70};
\draw [c] (2.89466,3.65899) -- (2.89466,3.85657);
\draw [c] (2.89466,3.85657) -- (2.89466,4.05415);
\draw [c] (2.87661,3.85657) -- (2.89466,3.85657);
\draw [c] (2.89466,3.85657) -- (2.9127,3.85657);
\definecolor{c}{rgb}{0,0,0};
\colorlet{c}{natcomp!70};
\draw [c] (2.93075,3.93162) -- (2.93075,4.14189);
\draw [c] (2.93075,4.14189) -- (2.93075,4.35217);
\draw [c] (2.9127,4.14189) -- (2.93075,4.14189);
\draw [c] (2.93075,4.14189) -- (2.94879,4.14189);
\definecolor{c}{rgb}{0,0,0};
\colorlet{c}{natcomp!70};
\draw [c] (2.96683,3.4435) -- (2.96683,3.63714);
\draw [c] (2.96683,3.63714) -- (2.96683,3.83078);
\draw [c] (2.94879,3.63714) -- (2.96683,3.63714);
\draw [c] (2.96683,3.63714) -- (2.98488,3.63714);
\definecolor{c}{rgb}{0,0,0};
\colorlet{c}{natcomp!70};
\draw [c] (3.00292,3.01529) -- (3.00292,3.1925);
\draw [c] (3.00292,3.1925) -- (3.00292,3.36972);
\draw [c] (2.98488,3.1925) -- (3.00292,3.1925);
\draw [c] (3.00292,3.1925) -- (3.02097,3.1925);
\definecolor{c}{rgb}{0,0,0};
\colorlet{c}{natcomp!70};
\draw [c] (3.03901,3.01986) -- (3.03901,3.19798);
\draw [c] (3.03901,3.19798) -- (3.03901,3.3761);
\draw [c] (3.02097,3.19798) -- (3.03901,3.19798);
\draw [c] (3.03901,3.19798) -- (3.05706,3.19798);
\definecolor{c}{rgb}{0,0,0};
\colorlet{c}{natcomp!70};
\draw [c] (3.0751,2.88122) -- (3.0751,3.05473);
\draw [c] (3.0751,3.05473) -- (3.0751,3.22824);
\draw [c] (3.05706,3.05473) -- (3.0751,3.05473);
\draw [c] (3.0751,3.05473) -- (3.09315,3.05473);
\definecolor{c}{rgb}{0,0,0};
\colorlet{c}{natcomp!70};
\draw [c] (3.11119,2.58374) -- (3.11119,2.74401);
\draw [c] (3.11119,2.74401) -- (3.11119,2.90429);
\draw [c] (3.09315,2.74401) -- (3.11119,2.74401);
\draw [c] (3.11119,2.74401) -- (3.12923,2.74401);
\definecolor{c}{rgb}{0,0,0};
\colorlet{c}{natcomp!70};
\draw [c] (3.14728,2.30093) -- (3.14728,2.44861);
\draw [c] (3.14728,2.44861) -- (3.14728,2.59629);
\draw [c] (3.12923,2.44861) -- (3.14728,2.44861);
\draw [c] (3.14728,2.44861) -- (3.16532,2.44861);
\definecolor{c}{rgb}{0,0,0};
\colorlet{c}{natcomp!70};
\draw [c] (3.18337,2.3983) -- (3.18337,2.55001);
\draw [c] (3.18337,2.55001) -- (3.18337,2.70172);
\draw [c] (3.16532,2.55001) -- (3.18337,2.55001);
\draw [c] (3.18337,2.55001) -- (3.20141,2.55001);
\definecolor{c}{rgb}{0,0,0};
\colorlet{c}{natcomp!70};
\draw [c] (3.21946,2.43359) -- (3.21946,2.5937);
\draw [c] (3.21946,2.5937) -- (3.21946,2.75381);
\draw [c] (3.20141,2.5937) -- (3.21946,2.5937);
\draw [c] (3.21946,2.5937) -- (3.2375,2.5937);
\definecolor{c}{rgb}{0,0,0};
\colorlet{c}{natcomp!70};
\draw [c] (3.25554,2.18533) -- (3.25554,2.32113);
\draw [c] (3.25554,2.32113) -- (3.25554,2.45694);
\draw [c] (3.2375,2.32113) -- (3.25554,2.32113);
\draw [c] (3.25554,2.32113) -- (3.27359,2.32113);
\definecolor{c}{rgb}{0,0,0};
\colorlet{c}{natcomp!70};
\draw [c] (3.29163,2.0191) -- (3.29163,2.15263);
\draw [c] (3.29163,2.15263) -- (3.29163,2.28615);
\draw [c] (3.27359,2.15263) -- (3.29163,2.15263);
\draw [c] (3.29163,2.15263) -- (3.30968,2.15263);
\definecolor{c}{rgb}{0,0,0};
\colorlet{c}{natcomp!70};
\draw [c] (3.32772,2.11944) -- (3.32772,2.256);
\draw [c] (3.32772,2.256) -- (3.32772,2.39256);
\draw [c] (3.30968,2.256) -- (3.32772,2.256);
\draw [c] (3.32772,2.256) -- (3.34577,2.256);
\definecolor{c}{rgb}{0,0,0};
\colorlet{c}{natcomp!70};
\draw [c] (3.36381,1.88428) -- (3.36381,2.00683);
\draw [c] (3.36381,2.00683) -- (3.36381,2.12938);
\draw [c] (3.34577,2.00683) -- (3.36381,2.00683);
\draw [c] (3.36381,2.00683) -- (3.38185,2.00683);
\definecolor{c}{rgb}{0,0,0};
\colorlet{c}{natcomp!70};
\draw [c] (3.3999,1.92813) -- (3.3999,2.05525);
\draw [c] (3.3999,2.05525) -- (3.3999,2.18238);
\draw [c] (3.38185,2.05525) -- (3.3999,2.05525);
\draw [c] (3.3999,2.05525) -- (3.41794,2.05525);
\definecolor{c}{rgb}{0,0,0};
\colorlet{c}{natcomp!70};
\draw [c] (3.43599,1.74122) -- (3.43599,1.86638);
\draw [c] (3.43599,1.86638) -- (3.43599,1.99155);
\draw [c] (3.41794,1.86638) -- (3.43599,1.86638);
\draw [c] (3.43599,1.86638) -- (3.45403,1.86638);
\definecolor{c}{rgb}{0,0,0};
\colorlet{c}{natcomp!70};
\draw [c] (3.47208,1.81489) -- (3.47208,1.94716);
\draw [c] (3.47208,1.94716) -- (3.47208,2.07944);
\draw [c] (3.45403,1.94716) -- (3.47208,1.94716);
\draw [c] (3.47208,1.94716) -- (3.49012,1.94716);
\definecolor{c}{rgb}{0,0,0};
\colorlet{c}{natcomp!70};
\draw [c] (3.50817,1.85856) -- (3.50817,1.99328);
\draw [c] (3.50817,1.99328) -- (3.50817,2.128);
\draw [c] (3.49012,1.99328) -- (3.50817,1.99328);
\draw [c] (3.50817,1.99328) -- (3.52621,1.99328);
\definecolor{c}{rgb}{0,0,0};
\colorlet{c}{natcomp!70};
\draw [c] (3.54425,1.51474) -- (3.54425,1.62467);
\draw [c] (3.54425,1.62467) -- (3.54425,1.7346);
\draw [c] (3.52621,1.62467) -- (3.54425,1.62467);
\draw [c] (3.54425,1.62467) -- (3.5623,1.62467);
\definecolor{c}{rgb}{0,0,0};
\colorlet{c}{natcomp!70};
\draw [c] (3.58034,1.58337) -- (3.58034,1.69582);
\draw [c] (3.58034,1.69582) -- (3.58034,1.80827);
\draw [c] (3.5623,1.69582) -- (3.58034,1.69582);
\draw [c] (3.58034,1.69582) -- (3.59839,1.69582);
\definecolor{c}{rgb}{0,0,0};
\colorlet{c}{natcomp!70};
\draw [c] (3.61643,1.54794) -- (3.61643,1.65528);
\draw [c] (3.61643,1.65528) -- (3.61643,1.76262);
\draw [c] (3.59839,1.65528) -- (3.61643,1.65528);
\draw [c] (3.61643,1.65528) -- (3.63448,1.65528);
\definecolor{c}{rgb}{0,0,0};
\colorlet{c}{natcomp!70};
\draw [c] (3.65252,1.42522) -- (3.65252,1.52375);
\draw [c] (3.65252,1.52375) -- (3.65252,1.62228);
\draw [c] (3.63448,1.52375) -- (3.65252,1.52375);
\draw [c] (3.65252,1.52375) -- (3.67056,1.52375);
\definecolor{c}{rgb}{0,0,0};
\colorlet{c}{natcomp!70};
\draw [c] (3.68861,1.31417) -- (3.68861,1.40383);
\draw [c] (3.68861,1.40383) -- (3.68861,1.4935);
\draw [c] (3.67056,1.40383) -- (3.68861,1.40383);
\draw [c] (3.68861,1.40383) -- (3.70665,1.40383);
\definecolor{c}{rgb}{0,0,0};
\colorlet{c}{natcomp!70};
\draw [c] (3.7247,1.39879) -- (3.7247,1.50121);
\draw [c] (3.7247,1.50121) -- (3.7247,1.60364);
\draw [c] (3.70665,1.50121) -- (3.7247,1.50121);
\draw [c] (3.7247,1.50121) -- (3.74274,1.50121);
\definecolor{c}{rgb}{0,0,0};
\colorlet{c}{natcomp!70};
\draw [c] (3.76079,1.47938) -- (3.76079,1.57989);
\draw [c] (3.76079,1.57989) -- (3.76079,1.68041);
\draw [c] (3.74274,1.57989) -- (3.76079,1.57989);
\draw [c] (3.76079,1.57989) -- (3.77883,1.57989);
\definecolor{c}{rgb}{0,0,0};
\colorlet{c}{natcomp!70};
\draw [c] (3.79688,1.37064) -- (3.79688,1.46625);
\draw [c] (3.79688,1.46625) -- (3.79688,1.56186);
\draw [c] (3.77883,1.46625) -- (3.79688,1.46625);
\draw [c] (3.79688,1.46625) -- (3.81492,1.46625);
\definecolor{c}{rgb}{0,0,0};
\colorlet{c}{natcomp!70};
\draw [c] (3.83296,1.48496) -- (3.83296,1.58921);
\draw [c] (3.83296,1.58921) -- (3.83296,1.69345);
\draw [c] (3.81492,1.58921) -- (3.83296,1.58921);
\draw [c] (3.83296,1.58921) -- (3.85101,1.58921);
\definecolor{c}{rgb}{0,0,0};
\colorlet{c}{natcomp!70};
\draw [c] (3.86905,1.16297) -- (3.86905,1.24409);
\draw [c] (3.86905,1.24409) -- (3.86905,1.3252);
\draw [c] (3.85101,1.24409) -- (3.86905,1.24409);
\draw [c] (3.86905,1.24409) -- (3.8871,1.24409);
\definecolor{c}{rgb}{0,0,0};
\colorlet{c}{natcomp!70};
\draw [c] (3.90514,1.26804) -- (3.90514,1.35812);
\draw [c] (3.90514,1.35812) -- (3.90514,1.4482);
\draw [c] (3.8871,1.35812) -- (3.90514,1.35812);
\draw [c] (3.90514,1.35812) -- (3.92319,1.35812);
\definecolor{c}{rgb}{0,0,0};
\colorlet{c}{natcomp!70};
\draw [c] (3.94123,1.22039) -- (3.94123,1.30135);
\draw [c] (3.94123,1.30135) -- (3.94123,1.38232);
\draw [c] (3.92319,1.30135) -- (3.94123,1.30135);
\draw [c] (3.94123,1.30135) -- (3.95927,1.30135);
\definecolor{c}{rgb}{0,0,0};
\colorlet{c}{natcomp!70};
\draw [c] (3.97732,1.12845) -- (3.97732,1.20076);
\draw [c] (3.97732,1.20076) -- (3.97732,1.27306);
\draw [c] (3.95927,1.20076) -- (3.97732,1.20076);
\draw [c] (3.97732,1.20076) -- (3.99536,1.20076);
\definecolor{c}{rgb}{0,0,0};
\colorlet{c}{natcomp!70};
\draw [c] (4.01341,1.29099) -- (4.01341,1.38407);
\draw [c] (4.01341,1.38407) -- (4.01341,1.47715);
\draw [c] (3.99536,1.38407) -- (4.01341,1.38407);
\draw [c] (4.01341,1.38407) -- (4.03145,1.38407);
\definecolor{c}{rgb}{0,0,0};
\colorlet{c}{natcomp!70};
\draw [c] (4.0495,1.08142) -- (4.0495,1.15343);
\draw [c] (4.0495,1.15343) -- (4.0495,1.22545);
\draw [c] (4.03145,1.15343) -- (4.0495,1.15343);
\draw [c] (4.0495,1.15343) -- (4.06754,1.15343);
\definecolor{c}{rgb}{0,0,0};
\colorlet{c}{natcomp!70};
\draw [c] (4.08558,1.18677) -- (4.08558,1.27028);
\draw [c] (4.08558,1.27028) -- (4.08558,1.35379);
\draw [c] (4.06754,1.27028) -- (4.08558,1.27028);
\draw [c] (4.08558,1.27028) -- (4.10363,1.27028);
\definecolor{c}{rgb}{0,0,0};
\colorlet{c}{natcomp!70};
\draw [c] (4.12167,1.14251) -- (4.12167,1.21997);
\draw [c] (4.12167,1.21997) -- (4.12167,1.29744);
\draw [c] (4.10363,1.21997) -- (4.12167,1.21997);
\draw [c] (4.12167,1.21997) -- (4.13972,1.21997);
\definecolor{c}{rgb}{0,0,0};
\colorlet{c}{natcomp!70};
\draw [c] (4.15776,1.11761) -- (4.15776,1.20294);
\draw [c] (4.15776,1.20294) -- (4.15776,1.28827);
\draw [c] (4.13972,1.20294) -- (4.15776,1.20294);
\draw [c] (4.15776,1.20294) -- (4.17581,1.20294);
\definecolor{c}{rgb}{0,0,0};
\colorlet{c}{natcomp!70};
\draw [c] (4.19385,1.17105) -- (4.19385,1.25955);
\draw [c] (4.19385,1.25955) -- (4.19385,1.34804);
\draw [c] (4.17581,1.25955) -- (4.19385,1.25955);
\draw [c] (4.19385,1.25955) -- (4.21189,1.25955);
\definecolor{c}{rgb}{0,0,0};
\colorlet{c}{natcomp!70};
\draw [c] (4.22994,1.10263) -- (4.22994,1.17065);
\draw [c] (4.22994,1.17065) -- (4.22994,1.23867);
\draw [c] (4.21189,1.17065) -- (4.22994,1.17065);
\draw [c] (4.22994,1.17065) -- (4.24798,1.17065);
\definecolor{c}{rgb}{0,0,0};
\colorlet{c}{natcomp!70};
\draw [c] (4.26603,1.06985) -- (4.26603,1.1406);
\draw [c] (4.26603,1.1406) -- (4.26603,1.21135);
\draw [c] (4.24798,1.1406) -- (4.26603,1.1406);
\draw [c] (4.26603,1.1406) -- (4.28407,1.1406);
\definecolor{c}{rgb}{0,0,0};
\colorlet{c}{natcomp!70};
\draw [c] (4.30212,1.04355) -- (4.30212,1.1102);
\draw [c] (4.30212,1.1102) -- (4.30212,1.17684);
\draw [c] (4.28407,1.1102) -- (4.30212,1.1102);
\draw [c] (4.30212,1.1102) -- (4.32016,1.1102);
\definecolor{c}{rgb}{0,0,0};
\colorlet{c}{natcomp!70};
\draw [c] (4.33821,1.01865) -- (4.33821,1.09854);
\draw [c] (4.33821,1.09854) -- (4.33821,1.17843);
\draw [c] (4.32016,1.09854) -- (4.33821,1.09854);
\draw [c] (4.33821,1.09854) -- (4.35625,1.09854);
\definecolor{c}{rgb}{0,0,0};
\colorlet{c}{natcomp!70};
\draw [c] (4.37429,0.906879) -- (4.37429,0.955253);
\draw [c] (4.37429,0.955253) -- (4.37429,1.00363);
\draw [c] (4.35625,0.955253) -- (4.37429,0.955253);
\draw [c] (4.37429,0.955253) -- (4.39234,0.955253);
\definecolor{c}{rgb}{0,0,0};
\colorlet{c}{natcomp!70};
\draw [c] (4.41038,0.926954) -- (4.41038,0.983752);
\draw [c] (4.41038,0.983752) -- (4.41038,1.04055);
\draw [c] (4.39234,0.983752) -- (4.41038,0.983752);
\draw [c] (4.41038,0.983752) -- (4.42843,0.983752);
\definecolor{c}{rgb}{0,0,0};
\colorlet{c}{natcomp!70};
\draw [c] (4.44647,1.03078) -- (4.44647,1.10645);
\draw [c] (4.44647,1.10645) -- (4.44647,1.18211);
\draw [c] (4.42843,1.10645) -- (4.44647,1.10645);
\draw [c] (4.44647,1.10645) -- (4.46452,1.10645);
\definecolor{c}{rgb}{0,0,0};
\colorlet{c}{natcomp!70};
\draw [c] (4.48256,0.96438) -- (4.48256,1.02395);
\draw [c] (4.48256,1.02395) -- (4.48256,1.08353);
\draw [c] (4.46452,1.02395) -- (4.48256,1.02395);
\draw [c] (4.48256,1.02395) -- (4.5006,1.02395);
\definecolor{c}{rgb}{0,0,0};
\colorlet{c}{natcomp!70};
\draw [c] (4.51865,1.01953) -- (4.51865,1.09213);
\draw [c] (4.51865,1.09213) -- (4.51865,1.16473);
\draw [c] (4.5006,1.09213) -- (4.51865,1.09213);
\draw [c] (4.51865,1.09213) -- (4.53669,1.09213);
\definecolor{c}{rgb}{0,0,0};
\colorlet{c}{natcomp!70};
\draw [c] (4.55474,0.89496) -- (4.55474,0.945912);
\draw [c] (4.55474,0.945912) -- (4.55474,0.996863);
\draw [c] (4.53669,0.945912) -- (4.55474,0.945912);
\draw [c] (4.55474,0.945912) -- (4.57278,0.945912);
\definecolor{c}{rgb}{0,0,0};
\colorlet{c}{natcomp!70};
\draw [c] (4.59083,0.986726) -- (4.59083,1.04818);
\draw [c] (4.59083,1.04818) -- (4.59083,1.10964);
\draw [c] (4.57278,1.04818) -- (4.59083,1.04818);
\draw [c] (4.59083,1.04818) -- (4.60887,1.04818);
\definecolor{c}{rgb}{0,0,0};
\colorlet{c}{natcomp!70};
\draw [c] (4.62692,0.828778) -- (4.62692,0.867519);
\draw [c] (4.62692,0.867519) -- (4.62692,0.90626);
\draw [c] (4.60887,0.867519) -- (4.62692,0.867519);
\draw [c] (4.62692,0.867519) -- (4.64496,0.867519);
\definecolor{c}{rgb}{0,0,0};
\colorlet{c}{natcomp!70};
\draw [c] (4.663,0.929865) -- (4.663,0.982585);
\draw [c] (4.663,0.982585) -- (4.663,1.03531);
\draw [c] (4.64496,0.982585) -- (4.663,0.982585);
\draw [c] (4.663,0.982585) -- (4.68105,0.982585);
\definecolor{c}{rgb}{0,0,0};
\colorlet{c}{natcomp!70};
\draw [c] (4.69909,0.937887) -- (4.69909,1.0001);
\draw [c] (4.69909,1.0001) -- (4.69909,1.06232);
\draw [c] (4.68105,1.0001) -- (4.69909,1.0001);
\draw [c] (4.69909,1.0001) -- (4.71714,1.0001);
\definecolor{c}{rgb}{0,0,0};
\colorlet{c}{natcomp!70};
\draw [c] (4.73518,0.910001) -- (4.73518,0.961309);
\draw [c] (4.73518,0.961309) -- (4.73518,1.01262);
\draw [c] (4.71714,0.961309) -- (4.73518,0.961309);
\draw [c] (4.73518,0.961309) -- (4.75323,0.961309);
\definecolor{c}{rgb}{0,0,0};
\colorlet{c}{natcomp!70};
\draw [c] (4.77127,0.868019) -- (4.77127,0.914493);
\draw [c] (4.77127,0.914493) -- (4.77127,0.960967);
\draw [c] (4.75323,0.914493) -- (4.77127,0.914493);
\draw [c] (4.77127,0.914493) -- (4.78931,0.914493);
\definecolor{c}{rgb}{0,0,0};
\colorlet{c}{natcomp!70};
\draw [c] (4.80736,0.878272) -- (4.80736,0.930742);
\draw [c] (4.80736,0.930742) -- (4.80736,0.983212);
\draw [c] (4.78931,0.930742) -- (4.80736,0.930742);
\draw [c] (4.80736,0.930742) -- (4.8254,0.930742);
\definecolor{c}{rgb}{0,0,0};
\colorlet{c}{natcomp!70};
\draw [c] (4.84345,0.809312) -- (4.84345,0.84424);
\draw [c] (4.84345,0.84424) -- (4.84345,0.879168);
\draw [c] (4.8254,0.84424) -- (4.84345,0.84424);
\draw [c] (4.84345,0.84424) -- (4.86149,0.84424);
\definecolor{c}{rgb}{0,0,0};
\colorlet{c}{natcomp!70};
\draw [c] (4.87954,0.762677) -- (4.87954,0.78561);
\draw [c] (4.87954,0.78561) -- (4.87954,0.808544);
\draw [c] (4.86149,0.78561) -- (4.87954,0.78561);
\draw [c] (4.87954,0.78561) -- (4.89758,0.78561);
\definecolor{c}{rgb}{0,0,0};
\colorlet{c}{natcomp!70};
\draw [c] (4.91563,0.835344) -- (4.91563,0.874984);
\draw [c] (4.91563,0.874984) -- (4.91563,0.914624);
\draw [c] (4.89758,0.874984) -- (4.91563,0.874984);
\draw [c] (4.91563,0.874984) -- (4.93367,0.874984);
\definecolor{c}{rgb}{0,0,0};
\colorlet{c}{natcomp!70};
\draw [c] (4.95171,0.87042) -- (4.95171,0.914521);
\draw [c] (4.95171,0.914521) -- (4.95171,0.958622);
\draw [c] (4.93367,0.914521) -- (4.95171,0.914521);
\draw [c] (4.95171,0.914521) -- (4.96976,0.914521);
\definecolor{c}{rgb}{0,0,0};
\colorlet{c}{natcomp!70};
\draw [c] (4.9878,0.826334) -- (4.9878,0.870592);
\draw [c] (4.9878,0.870592) -- (4.9878,0.91485);
\draw [c] (4.96976,0.870592) -- (4.9878,0.870592);
\draw [c] (4.9878,0.870592) -- (5.00585,0.870592);
\definecolor{c}{rgb}{0,0,0};
\colorlet{c}{natcomp!70};
\draw [c] (5.02389,0.822381) -- (5.02389,0.86059);
\draw [c] (5.02389,0.86059) -- (5.02389,0.898798);
\draw [c] (5.00585,0.86059) -- (5.02389,0.86059);
\draw [c] (5.02389,0.86059) -- (5.04194,0.86059);
\definecolor{c}{rgb}{0,0,0};
\colorlet{c}{natcomp!70};
\draw [c] (5.05998,0.788662) -- (5.05998,0.838806);
\draw [c] (5.05998,0.838806) -- (5.05998,0.88895);
\draw [c] (5.04194,0.838806) -- (5.05998,0.838806);
\draw [c] (5.05998,0.838806) -- (5.07802,0.838806);
\definecolor{c}{rgb}{0,0,0};
\colorlet{c}{natcomp!70};
\draw [c] (5.09607,0.81529) -- (5.09607,0.850679);
\draw [c] (5.09607,0.850679) -- (5.09607,0.886068);
\draw [c] (5.07802,0.850679) -- (5.09607,0.850679);
\draw [c] (5.09607,0.850679) -- (5.11411,0.850679);
\definecolor{c}{rgb}{0,0,0};
\colorlet{c}{natcomp!70};
\draw [c] (5.13216,0.877018) -- (5.13216,0.93804);
\draw [c] (5.13216,0.93804) -- (5.13216,0.999062);
\draw [c] (5.11411,0.93804) -- (5.13216,0.93804);
\draw [c] (5.13216,0.93804) -- (5.1502,0.93804);
\definecolor{c}{rgb}{0,0,0};
\colorlet{c}{natcomp!70};
\draw [c] (5.16825,0.815589) -- (5.16825,0.856866);
\draw [c] (5.16825,0.856866) -- (5.16825,0.898142);
\draw [c] (5.1502,0.856866) -- (5.16825,0.856866);
\draw [c] (5.16825,0.856866) -- (5.18629,0.856866);
\definecolor{c}{rgb}{0,0,0};
\colorlet{c}{natcomp!70};
\draw [c] (5.20433,0.740854) -- (5.20433,0.755731);
\draw [c] (5.20433,0.755731) -- (5.20433,0.770607);
\draw [c] (5.18629,0.755731) -- (5.20433,0.755731);
\draw [c] (5.20433,0.755731) -- (5.22238,0.755731);
\definecolor{c}{rgb}{0,0,0};
\colorlet{c}{natcomp!70};
\draw [c] (5.24042,0.812639) -- (5.24042,0.85358);
\draw [c] (5.24042,0.85358) -- (5.24042,0.89452);
\draw [c] (5.22238,0.85358) -- (5.24042,0.85358);
\draw [c] (5.24042,0.85358) -- (5.25847,0.85358);
\definecolor{c}{rgb}{0,0,0};
\colorlet{c}{natcomp!70};
\draw [c] (5.27651,0.803111) -- (5.27651,0.844052);
\draw [c] (5.27651,0.844052) -- (5.27651,0.884993);
\draw [c] (5.25847,0.844052) -- (5.27651,0.844052);
\draw [c] (5.27651,0.844052) -- (5.29456,0.844052);
\definecolor{c}{rgb}{0,0,0};
\colorlet{c}{natcomp!70};
\draw [c] (5.3126,0.800844) -- (5.3126,0.834967);
\draw [c] (5.3126,0.834967) -- (5.3126,0.869089);
\draw [c] (5.29456,0.834967) -- (5.3126,0.834967);
\draw [c] (5.3126,0.834967) -- (5.33065,0.834967);
\definecolor{c}{rgb}{0,0,0};
\colorlet{c}{natcomp!70};
\draw [c] (5.34869,0.784644) -- (5.34869,0.815177);
\draw [c] (5.34869,0.815177) -- (5.34869,0.845709);
\draw [c] (5.33065,0.815177) -- (5.34869,0.815177);
\draw [c] (5.34869,0.815177) -- (5.36673,0.815177);
\definecolor{c}{rgb}{0,0,0};
\colorlet{c}{natcomp!70};
\draw [c] (5.38478,0.802543) -- (5.38478,0.837564);
\draw [c] (5.38478,0.837564) -- (5.38478,0.872585);
\draw [c] (5.36673,0.837564) -- (5.38478,0.837564);
\draw [c] (5.38478,0.837564) -- (5.40282,0.837564);
\definecolor{c}{rgb}{0,0,0};
\colorlet{c}{natcomp!70};
\draw [c] (5.42087,0.838708) -- (5.42087,0.877281);
\draw [c] (5.42087,0.877281) -- (5.42087,0.915853);
\draw [c] (5.40282,0.877281) -- (5.42087,0.877281);
\draw [c] (5.42087,0.877281) -- (5.43891,0.877281);
\definecolor{c}{rgb}{0,0,0};
\colorlet{c}{natcomp!70};
\draw [c] (5.45696,0.828845) -- (5.45696,0.868725);
\draw [c] (5.45696,0.868725) -- (5.45696,0.908606);
\draw [c] (5.43891,0.868725) -- (5.45696,0.868725);
\draw [c] (5.45696,0.868725) -- (5.475,0.868725);
\definecolor{c}{rgb}{0,0,0};
\colorlet{c}{natcomp!70};
\draw [c] (5.49304,0.776612) -- (5.49304,0.806357);
\draw [c] (5.49304,0.806357) -- (5.49304,0.836102);
\draw [c] (5.475,0.806357) -- (5.49304,0.806357);
\draw [c] (5.49304,0.806357) -- (5.51109,0.806357);
\definecolor{c}{rgb}{0,0,0};
\colorlet{c}{natcomp!70};
\draw [c] (5.52913,0.765895) -- (5.52913,0.792266);
\draw [c] (5.52913,0.792266) -- (5.52913,0.818637);
\draw [c] (5.51109,0.792266) -- (5.52913,0.792266);
\draw [c] (5.52913,0.792266) -- (5.54718,0.792266);
\definecolor{c}{rgb}{0,0,0};
\colorlet{c}{natcomp!70};
\draw [c] (5.56522,0.772359) -- (5.56522,0.799594);
\draw [c] (5.56522,0.799594) -- (5.56522,0.82683);
\draw [c] (5.54718,0.799594) -- (5.56522,0.799594);
\draw [c] (5.56522,0.799594) -- (5.58327,0.799594);
\definecolor{c}{rgb}{0,0,0};
\colorlet{c}{natcomp!70};
\draw [c] (5.60131,0.796115) -- (5.60131,0.832186);
\draw [c] (5.60131,0.832186) -- (5.60131,0.868257);
\draw [c] (5.58327,0.832186) -- (5.60131,0.832186);
\draw [c] (5.60131,0.832186) -- (5.61935,0.832186);
\definecolor{c}{rgb}{0,0,0};
\colorlet{c}{natcomp!70};
\draw [c] (5.6374,0.781953) -- (5.6374,0.811129);
\draw [c] (5.6374,0.811129) -- (5.6374,0.840305);
\draw [c] (5.61935,0.811129) -- (5.6374,0.811129);
\draw [c] (5.6374,0.811129) -- (5.65544,0.811129);
\definecolor{c}{rgb}{0,0,0};
\colorlet{c}{natcomp!70};
\draw [c] (5.67349,0.81513) -- (5.67349,0.856508);
\draw [c] (5.67349,0.856508) -- (5.67349,0.897886);
\draw [c] (5.65544,0.856508) -- (5.67349,0.856508);
\draw [c] (5.67349,0.856508) -- (5.69153,0.856508);
\definecolor{c}{rgb}{0,0,0};
\colorlet{c}{natcomp!70};
\draw [c] (5.70958,0.792858) -- (5.70958,0.826425);
\draw [c] (5.70958,0.826425) -- (5.70958,0.859993);
\draw [c] (5.69153,0.826425) -- (5.70958,0.826425);
\draw [c] (5.70958,0.826425) -- (5.72762,0.826425);
\definecolor{c}{rgb}{0,0,0};
\colorlet{c}{natcomp!70};
\draw [c] (5.74567,0.749354) -- (5.74567,0.771177);
\draw [c] (5.74567,0.771177) -- (5.74567,0.793);
\draw [c] (5.72762,0.771177) -- (5.74567,0.771177);
\draw [c] (5.74567,0.771177) -- (5.76371,0.771177);
\definecolor{c}{rgb}{0,0,0};
\colorlet{c}{natcomp!70};
\draw [c] (5.78175,0.776801) -- (5.78175,0.80719);
\draw [c] (5.78175,0.80719) -- (5.78175,0.837579);
\draw [c] (5.76371,0.80719) -- (5.78175,0.80719);
\draw [c] (5.78175,0.80719) -- (5.7998,0.80719);
\definecolor{c}{rgb}{0,0,0};
\colorlet{c}{natcomp!70};
\draw [c] (5.81784,0.745067) -- (5.81784,0.777402);
\draw [c] (5.81784,0.777402) -- (5.81784,0.809737);
\draw [c] (5.7998,0.777402) -- (5.81784,0.777402);
\draw [c] (5.81784,0.777402) -- (5.83589,0.777402);
\definecolor{c}{rgb}{0,0,0};
\colorlet{c}{natcomp!70};
\draw [c] (5.85393,0.761061) -- (5.85393,0.782893);
\draw [c] (5.85393,0.782893) -- (5.85393,0.804725);
\draw [c] (5.83589,0.782893) -- (5.85393,0.782893);
\draw [c] (5.85393,0.782893) -- (5.87198,0.782893);
\definecolor{c}{rgb}{0,0,0};
\colorlet{c}{natcomp!70};
\draw [c] (5.89002,0.749941) -- (5.89002,0.771033);
\draw [c] (5.89002,0.771033) -- (5.89002,0.792126);
\draw [c] (5.87198,0.771033) -- (5.89002,0.771033);
\draw [c] (5.89002,0.771033) -- (5.90806,0.771033);
\definecolor{c}{rgb}{0,0,0};
\colorlet{c}{natcomp!70};
\draw [c] (5.92611,0.77436) -- (5.92611,0.80203);
\draw [c] (5.92611,0.80203) -- (5.92611,0.829699);
\draw [c] (5.90806,0.80203) -- (5.92611,0.80203);
\draw [c] (5.92611,0.80203) -- (5.94415,0.80203);
\definecolor{c}{rgb}{0,0,0};
\colorlet{c}{natcomp!70};
\draw [c] (5.9622,0.74754) -- (5.9622,0.765501);
\draw [c] (5.9622,0.765501) -- (5.9622,0.783462);
\draw [c] (5.94415,0.765501) -- (5.9622,0.765501);
\draw [c] (5.9622,0.765501) -- (5.98024,0.765501);
\definecolor{c}{rgb}{0,0,0};
\colorlet{c}{natcomp!70};
\draw [c] (5.99829,0.734669) -- (5.99829,0.747312);
\draw [c] (5.99829,0.747312) -- (5.99829,0.759955);
\draw [c] (5.98024,0.747312) -- (5.99829,0.747312);
\draw [c] (5.99829,0.747312) -- (6.01633,0.747312);
\definecolor{c}{rgb}{0,0,0};
\colorlet{c}{natcomp!70};
\draw [c] (6.03438,0.734644) -- (6.03438,0.746862);
\draw [c] (6.03438,0.746862) -- (6.03438,0.759079);
\draw [c] (6.01633,0.746862) -- (6.03438,0.746862);
\draw [c] (6.03438,0.746862) -- (6.05242,0.746862);
\definecolor{c}{rgb}{0,0,0};
\colorlet{c}{natcomp!70};
\draw [c] (6.07046,0.75936) -- (6.07046,0.784951);
\draw [c] (6.07046,0.784951) -- (6.07046,0.810543);
\draw [c] (6.05242,0.784951) -- (6.07046,0.784951);
\draw [c] (6.07046,0.784951) -- (6.08851,0.784951);
\definecolor{c}{rgb}{0,0,0};
\colorlet{c}{natcomp!70};
\draw [c] (6.10655,0.784459) -- (6.10655,0.823852);
\draw [c] (6.10655,0.823852) -- (6.10655,0.863245);
\draw [c] (6.08851,0.823852) -- (6.10655,0.823852);
\draw [c] (6.10655,0.823852) -- (6.1246,0.823852);
\definecolor{c}{rgb}{0,0,0};
\colorlet{c}{natcomp!70};
\draw [c] (6.14264,0.734662) -- (6.14264,0.734685);
\draw [c] (6.14264,0.734685) -- (6.14264,0.734709);
\draw [c] (6.1246,0.734685) -- (6.14264,0.734685);
\draw [c] (6.14264,0.734685) -- (6.16069,0.734685);
\definecolor{c}{rgb}{0,0,0};
\colorlet{c}{natcomp!70};
\draw [c] (6.17873,0.769006) -- (6.17873,0.797565);
\draw [c] (6.17873,0.797565) -- (6.17873,0.826124);
\draw [c] (6.16069,0.797565) -- (6.17873,0.797565);
\draw [c] (6.17873,0.797565) -- (6.19677,0.797565);
\definecolor{c}{rgb}{0,0,0};
\colorlet{c}{natcomp!70};
\draw [c] (6.21482,0.740767) -- (6.21482,0.75556);
\draw [c] (6.21482,0.75556) -- (6.21482,0.770352);
\draw [c] (6.19677,0.75556) -- (6.21482,0.75556);
\draw [c] (6.21482,0.75556) -- (6.23286,0.75556);
\definecolor{c}{rgb}{0,0,0};
\colorlet{c}{natcomp!70};
\draw [c] (6.25091,0.742869) -- (6.25091,0.767082);
\draw [c] (6.25091,0.767082) -- (6.25091,0.791294);
\draw [c] (6.23286,0.767082) -- (6.25091,0.767082);
\draw [c] (6.25091,0.767082) -- (6.26895,0.767082);
\definecolor{c}{rgb}{0,0,0};
\colorlet{c}{natcomp!70};
\draw [c] (6.287,0.763439) -- (6.287,0.787014);
\draw [c] (6.287,0.787014) -- (6.287,0.810589);
\draw [c] (6.26895,0.787014) -- (6.287,0.787014);
\draw [c] (6.287,0.787014) -- (6.30504,0.787014);
\definecolor{c}{rgb}{0,0,0};
\colorlet{c}{natcomp!70};
\draw [c] (6.32308,0.739956) -- (6.32308,0.752771);
\draw [c] (6.32308,0.752771) -- (6.32308,0.765586);
\draw [c] (6.30504,0.752771) -- (6.32308,0.752771);
\draw [c] (6.32308,0.752771) -- (6.34113,0.752771);
\definecolor{c}{rgb}{0,0,0};
\colorlet{c}{natcomp!70};
\draw [c] (6.35917,0.749245) -- (6.35917,0.769531);
\draw [c] (6.35917,0.769531) -- (6.35917,0.789818);
\draw [c] (6.34113,0.769531) -- (6.35917,0.769531);
\draw [c] (6.35917,0.769531) -- (6.37722,0.769531);
\definecolor{c}{rgb}{0,0,0};
\colorlet{c}{natcomp!70};
\draw [c] (6.39526,0.734642) -- (6.39526,0.745339);
\draw [c] (6.39526,0.745339) -- (6.39526,0.756036);
\draw [c] (6.37722,0.745339) -- (6.39526,0.745339);
\draw [c] (6.39526,0.745339) -- (6.41331,0.745339);
\definecolor{c}{rgb}{0,0,0};
\colorlet{c}{natcomp!70};
\draw [c] (6.43135,0.749697) -- (6.43135,0.770462);
\draw [c] (6.43135,0.770462) -- (6.43135,0.791227);
\draw [c] (6.41331,0.770462) -- (6.43135,0.770462);
\draw [c] (6.43135,0.770462) -- (6.4494,0.770462);
\definecolor{c}{rgb}{0,0,0};
\colorlet{c}{natcomp!70};
\draw [c] (6.46744,0.741741) -- (6.46744,0.758925);
\draw [c] (6.46744,0.758925) -- (6.46744,0.77611);
\draw [c] (6.4494,0.758925) -- (6.46744,0.758925);
\draw [c] (6.46744,0.758925) -- (6.48548,0.758925);
\definecolor{c}{rgb}{0,0,0};
\colorlet{c}{natcomp!70};
\draw [c] (6.50353,0.741076) -- (6.50353,0.756664);
\draw [c] (6.50353,0.756664) -- (6.50353,0.772252);
\draw [c] (6.48548,0.756664) -- (6.50353,0.756664);
\draw [c] (6.50353,0.756664) -- (6.52157,0.756664);
\definecolor{c}{rgb}{0,0,0};
\colorlet{c}{natcomp!70};
\draw [c] (6.53962,0.740597) -- (6.53962,0.755326);
\draw [c] (6.53962,0.755326) -- (6.53962,0.770055);
\draw [c] (6.52157,0.755326) -- (6.53962,0.755326);
\draw [c] (6.53962,0.755326) -- (6.55766,0.755326);
\definecolor{c}{rgb}{0,0,0};
\colorlet{c}{natcomp!70};
\draw [c] (6.57571,0.741224) -- (6.57571,0.757269);
\draw [c] (6.57571,0.757269) -- (6.57571,0.773313);
\draw [c] (6.55766,0.757269) -- (6.57571,0.757269);
\draw [c] (6.57571,0.757269) -- (6.59375,0.757269);
\definecolor{c}{rgb}{0,0,0};
\colorlet{c}{natcomp!70};
\draw [c] (6.61179,0.734641) -- (6.61179,0.73465);
\draw [c] (6.61179,0.73465) -- (6.61179,0.734658);
\draw [c] (6.59375,0.73465) -- (6.61179,0.73465);
\draw [c] (6.61179,0.73465) -- (6.62984,0.73465);
\definecolor{c}{rgb}{0,0,0};
\colorlet{c}{natcomp!70};
\draw [c] (6.64788,0.748451) -- (6.64788,0.767356);
\draw [c] (6.64788,0.767356) -- (6.64788,0.786261);
\draw [c] (6.62984,0.767356) -- (6.64788,0.767356);
\draw [c] (6.64788,0.767356) -- (6.66593,0.767356);
\definecolor{c}{rgb}{0,0,0};
\colorlet{c}{natcomp!70};
\draw [c] (6.68397,0.734651) -- (6.68397,0.745051);
\draw [c] (6.68397,0.745051) -- (6.68397,0.755451);
\draw [c] (6.66593,0.745051) -- (6.68397,0.745051);
\draw [c] (6.68397,0.745051) -- (6.70202,0.745051);
\definecolor{c}{rgb}{0,0,0};
\colorlet{c}{natcomp!70};
\draw [c] (6.72006,0.747383) -- (6.72006,0.765114);
\draw [c] (6.72006,0.765114) -- (6.72006,0.782845);
\draw [c] (6.70202,0.765114) -- (6.72006,0.765114);
\draw [c] (6.72006,0.765114) -- (6.7381,0.765114);
\definecolor{c}{rgb}{0,0,0};
\colorlet{c}{natcomp!70};
\draw [c] (6.75615,0.734642) -- (6.75615,0.744194);
\draw [c] (6.75615,0.744194) -- (6.75615,0.753745);
\draw [c] (6.7381,0.744194) -- (6.75615,0.744194);
\draw [c] (6.75615,0.744194) -- (6.77419,0.744194);
\definecolor{c}{rgb}{0,0,0};
\colorlet{c}{natcomp!70};
\draw [c] (6.79224,0.745364) -- (6.79224,0.760001);
\draw [c] (6.79224,0.760001) -- (6.79224,0.774637);
\draw [c] (6.77419,0.760001) -- (6.79224,0.760001);
\draw [c] (6.79224,0.760001) -- (6.81028,0.760001);
\definecolor{c}{rgb}{0,0,0};
\colorlet{c}{natcomp!70};
\draw [c] (6.82833,0.748794) -- (6.82833,0.768589);
\draw [c] (6.82833,0.768589) -- (6.82833,0.788384);
\draw [c] (6.81028,0.768589) -- (6.82833,0.768589);
\draw [c] (6.82833,0.768589) -- (6.84637,0.768589);
\definecolor{c}{rgb}{0,0,0};
\colorlet{c}{natcomp!70};
\draw [c] (6.86442,0.734632) -- (6.86442,0.744183);
\draw [c] (6.86442,0.744183) -- (6.86442,0.753735);
\draw [c] (6.84637,0.744183) -- (6.86442,0.744183);
\draw [c] (6.86442,0.744183) -- (6.88246,0.744183);
\definecolor{c}{rgb}{0,0,0};
\colorlet{c}{natcomp!70};
\draw [c] (6.9005,0.74204) -- (6.9005,0.761487);
\draw [c] (6.9005,0.761487) -- (6.9005,0.780933);
\draw [c] (6.88246,0.761487) -- (6.9005,0.761487);
\draw [c] (6.9005,0.761487) -- (6.91855,0.761487);
\definecolor{c}{rgb}{0,0,0};
\colorlet{c}{natcomp!70};
\draw [c] (6.93659,0.734641) -- (6.93659,0.745338);
\draw [c] (6.93659,0.745338) -- (6.93659,0.756035);
\draw [c] (6.91855,0.745338) -- (6.93659,0.745338);
\draw [c] (6.93659,0.745338) -- (6.95464,0.745338);
\definecolor{c}{rgb}{0,0,0};
\colorlet{c}{natcomp!70};
\draw [c] (6.97268,0.734636) -- (6.97268,0.734649);
\draw [c] (6.97268,0.734649) -- (6.97268,0.734662);
\draw [c] (6.95464,0.734649) -- (6.97268,0.734649);
\draw [c] (6.97268,0.734649) -- (6.99073,0.734649);
\definecolor{c}{rgb}{0,0,0};
\colorlet{c}{natcomp!70};
\draw [c] (7.00877,0.743265) -- (7.00877,0.76582);
\draw [c] (7.00877,0.76582) -- (7.00877,0.788375);
\draw [c] (6.99073,0.76582) -- (7.00877,0.76582);
\draw [c] (7.00877,0.76582) -- (7.02681,0.76582);
\definecolor{c}{rgb}{0,0,0};
\colorlet{c}{natcomp!70};
\draw [c] (7.04486,0.740152) -- (7.04486,0.753784);
\draw [c] (7.04486,0.753784) -- (7.04486,0.767416);
\draw [c] (7.02681,0.753784) -- (7.04486,0.753784);
\draw [c] (7.04486,0.753784) -- (7.0629,0.753784);
\definecolor{c}{rgb}{0,0,0};
\colorlet{c}{natcomp!70};
\draw [c] (7.08095,0.73465) -- (7.08095,0.746868);
\draw [c] (7.08095,0.746868) -- (7.08095,0.759085);
\draw [c] (7.0629,0.746868) -- (7.08095,0.746868);
\draw [c] (7.08095,0.746868) -- (7.09899,0.746868);
\definecolor{c}{rgb}{0,0,0};
\colorlet{c}{natcomp!70};
\draw [c] (7.11704,0.734641) -- (7.11704,0.744192);
\draw [c] (7.11704,0.744192) -- (7.11704,0.753744);
\draw [c] (7.09899,0.744192) -- (7.11704,0.744192);
\draw [c] (7.11704,0.744192) -- (7.13508,0.744192);
\definecolor{c}{rgb}{0,0,0};
\colorlet{c}{natcomp!70};
\draw [c] (7.15312,0.734653) -- (7.15312,0.746265);
\draw [c] (7.15312,0.746265) -- (7.15312,0.757876);
\draw [c] (7.13508,0.746265) -- (7.15312,0.746265);
\draw [c] (7.15312,0.746265) -- (7.17117,0.746265);
\definecolor{c}{rgb}{0,0,0};
\colorlet{c}{natcomp!70};
\draw [c] (7.18921,0.734635) -- (7.18921,0.734641);
\draw [c] (7.18921,0.734641) -- (7.18921,0.734648);
\draw [c] (7.17117,0.734641) -- (7.18921,0.734641);
\draw [c] (7.18921,0.734641) -- (7.20726,0.734641);
\definecolor{c}{rgb}{0,0,0};
\colorlet{c}{natcomp!70};
\draw [c] (7.2253,0.734641) -- (7.2253,0.760185);
\draw [c] (7.2253,0.760185) -- (7.2253,0.78573);
\draw [c] (7.20726,0.760185) -- (7.2253,0.760185);
\draw [c] (7.2253,0.760185) -- (7.24335,0.760185);
\definecolor{c}{rgb}{0,0,0};
\colorlet{c}{natcomp!70};
\draw [c] (7.26139,0.734632) -- (7.26139,0.734637);
\draw [c] (7.26139,0.734637) -- (7.26139,0.734642);
\draw [c] (7.24335,0.734637) -- (7.26139,0.734637);
\draw [c] (7.26139,0.734637) -- (7.27944,0.734637);
\definecolor{c}{rgb}{0,0,0};
\colorlet{c}{natcomp!70};
\draw [c] (7.29748,0.734642) -- (7.29748,0.745339);
\draw [c] (7.29748,0.745339) -- (7.29748,0.756036);
\draw [c] (7.27944,0.745339) -- (7.29748,0.745339);
\draw [c] (7.29748,0.745339) -- (7.31552,0.745339);
\definecolor{c}{rgb}{0,0,0};
\colorlet{c}{natcomp!70};
\draw [c] (7.33357,0.734637) -- (7.33357,0.745037);
\draw [c] (7.33357,0.745037) -- (7.33357,0.755437);
\draw [c] (7.31552,0.745037) -- (7.33357,0.745037);
\draw [c] (7.33357,0.745037) -- (7.35161,0.745037);
\definecolor{c}{rgb}{0,0,0};
\colorlet{c}{natcomp!70};
\draw [c] (7.36966,0.734641) -- (7.36966,0.746252);
\draw [c] (7.36966,0.746252) -- (7.36966,0.757864);
\draw [c] (7.35161,0.746252) -- (7.36966,0.746252);
\draw [c] (7.36966,0.746252) -- (7.3877,0.746252);
\definecolor{c}{rgb}{0,0,0};
\colorlet{c}{natcomp!70};
\draw [c] (7.40575,0.734656) -- (7.40575,0.743718);
\draw [c] (7.40575,0.743718) -- (7.40575,0.752779);
\draw [c] (7.3877,0.743718) -- (7.40575,0.743718);
\draw [c] (7.40575,0.743718) -- (7.42379,0.743718);
\definecolor{c}{rgb}{0,0,0};
\colorlet{c}{natcomp!70};
\draw [c] (7.44183,0.734632) -- (7.44183,0.734637);
\draw [c] (7.44183,0.734637) -- (7.44183,0.734641);
\draw [c] (7.42379,0.734637) -- (7.44183,0.734637);
\draw [c] (7.44183,0.734637) -- (7.45988,0.734637);
\definecolor{c}{rgb}{0,0,0};
\colorlet{c}{natcomp!70};
\draw [c] (7.58619,0.734637) -- (7.58619,0.73465);
\draw [c] (7.58619,0.73465) -- (7.58619,0.734663);
\draw [c] (7.56815,0.73465) -- (7.58619,0.73465);
\draw [c] (7.58619,0.73465) -- (7.60423,0.73465);
\definecolor{c}{rgb}{0,0,0};
\colorlet{c}{natcomp!70};
\draw [c] (7.62228,0.740087) -- (7.62228,0.753488);
\draw [c] (7.62228,0.753488) -- (7.62228,0.766888);
\draw [c] (7.60423,0.753488) -- (7.62228,0.753488);
\draw [c] (7.62228,0.753488) -- (7.64032,0.753488);
\definecolor{c}{rgb}{0,0,0};
\colorlet{c}{natcomp!70};
\draw [c] (7.73054,0.740987) -- (7.73054,0.756833);
\draw [c] (7.73054,0.756833) -- (7.73054,0.772678);
\draw [c] (7.7125,0.756833) -- (7.73054,0.756833);
\draw [c] (7.73054,0.756833) -- (7.74859,0.756833);
\definecolor{c}{rgb}{0,0,0};
\colorlet{c}{natcomp!70};
\draw [c] (7.76663,0.74081) -- (7.76663,0.755729);
\draw [c] (7.76663,0.755729) -- (7.76663,0.770649);
\draw [c] (7.74859,0.755729) -- (7.76663,0.755729);
\draw [c] (7.76663,0.755729) -- (7.78468,0.755729);
\definecolor{c}{rgb}{0,0,0};
\colorlet{c}{natcomp!70};
\draw [c] (7.80272,0.734636) -- (7.80272,0.734645);
\draw [c] (7.80272,0.734645) -- (7.80272,0.734655);
\draw [c] (7.78468,0.734645) -- (7.80272,0.734645);
\draw [c] (7.80272,0.734645) -- (7.82077,0.734645);
\definecolor{c}{rgb}{0,0,0};
\colorlet{c}{natcomp!70};
\draw [c] (7.83881,0.734638) -- (7.83881,0.734652);
\draw [c] (7.83881,0.734652) -- (7.83881,0.734666);
\draw [c] (7.82077,0.734652) -- (7.83881,0.734652);
\draw [c] (7.83881,0.734652) -- (7.85685,0.734652);
\definecolor{c}{rgb}{0,0,0};
\colorlet{c}{natcomp!70};
\draw [c] (7.8749,0.734638) -- (7.8749,0.744189);
\draw [c] (7.8749,0.744189) -- (7.8749,0.75374);
\draw [c] (7.85685,0.744189) -- (7.8749,0.744189);
\draw [c] (7.8749,0.744189) -- (7.89294,0.744189);
\definecolor{c}{rgb}{0,0,0};
\colorlet{c}{natcomp!70};
\draw [c] (7.91099,0.734635) -- (7.91099,0.734642);
\draw [c] (7.91099,0.734642) -- (7.91099,0.734649);
\draw [c] (7.89294,0.734642) -- (7.91099,0.734642);
\draw [c] (7.91099,0.734642) -- (7.92903,0.734642);
\definecolor{c}{rgb}{0,0,0};
\colorlet{c}{natcomp!70};
\draw [c] (7.94708,0.734632) -- (7.94708,0.746244);
\draw [c] (7.94708,0.746244) -- (7.94708,0.757855);
\draw [c] (7.92903,0.746244) -- (7.94708,0.746244);
\draw [c] (7.94708,0.746244) -- (7.96512,0.746244);
\definecolor{c}{rgb}{0,0,0};
\colorlet{c}{natcomp!70};
\draw [c] (7.98317,0.734657) -- (7.98317,0.745354);
\draw [c] (7.98317,0.745354) -- (7.98317,0.756051);
\draw [c] (7.96512,0.745354) -- (7.98317,0.745354);
\draw [c] (7.98317,0.745354) -- (8.00121,0.745354);
\definecolor{c}{rgb}{0,0,0};
\colorlet{c}{natcomp!70};
\draw [c] (8.01925,0.734632) -- (8.01925,0.734638);
\draw [c] (8.01925,0.734638) -- (8.01925,0.734644);
\draw [c] (8.00121,0.734638) -- (8.01925,0.734638);
\draw [c] (8.01925,0.734638) -- (8.0373,0.734638);
\definecolor{c}{rgb}{0,0,0};
\colorlet{c}{natcomp!70};
\draw [c] (8.05534,0.734632) -- (8.05534,0.734637);
\draw [c] (8.05534,0.734637) -- (8.05534,0.734642);
\draw [c] (8.0373,0.734637) -- (8.05534,0.734637);
\draw [c] (8.05534,0.734637) -- (8.07339,0.734637);
\definecolor{c}{rgb}{0,0,0};
\colorlet{c}{natcomp!70};
\draw [c] (8.09143,0.734632) -- (8.09143,0.7473);
\draw [c] (8.09143,0.7473) -- (8.09143,0.759968);
\draw [c] (8.07339,0.7473) -- (8.09143,0.7473);
\draw [c] (8.09143,0.7473) -- (8.10948,0.7473);
\definecolor{c}{rgb}{0,0,0};
\colorlet{c}{natcomp!70};
\draw [c] (8.16361,0.734632) -- (8.16361,0.743694);
\draw [c] (8.16361,0.743694) -- (8.16361,0.752755);
\draw [c] (8.14556,0.743694) -- (8.16361,0.743694);
\draw [c] (8.16361,0.743694) -- (8.18165,0.743694);
\definecolor{c}{rgb}{0,0,0};
\colorlet{c}{natcomp!70};
\draw [c] (8.23579,0.734632) -- (8.23579,0.734637);
\draw [c] (8.23579,0.734637) -- (8.23579,0.734642);
\draw [c] (8.21774,0.734637) -- (8.23579,0.734637);
\draw [c] (8.23579,0.734637) -- (8.25383,0.734637);
\definecolor{c}{rgb}{0,0,0};
\colorlet{c}{natcomp!70};
\draw [c] (8.27188,0.734637) -- (8.27188,0.745156);
\draw [c] (8.27188,0.745156) -- (8.27188,0.755676);
\draw [c] (8.25383,0.745156) -- (8.27188,0.745156);
\draw [c] (8.27188,0.745156) -- (8.28992,0.745156);
\definecolor{c}{rgb}{0,0,0};
\colorlet{c}{natcomp!70};
\draw [c] (8.34405,0.734651) -- (8.34405,0.747294);
\draw [c] (8.34405,0.747294) -- (8.34405,0.759937);
\draw [c] (8.32601,0.747294) -- (8.34405,0.747294);
\draw [c] (8.34405,0.747294) -- (8.3621,0.747294);
\definecolor{c}{rgb}{0,0,0};
\colorlet{c}{natcomp!70};
\draw [c] (8.45232,0.734632) -- (8.45232,0.734637);
\draw [c] (8.45232,0.734637) -- (8.45232,0.734642);
\draw [c] (8.43427,0.734637) -- (8.45232,0.734637);
\draw [c] (8.45232,0.734637) -- (8.47036,0.734637);
\definecolor{c}{rgb}{0,0,0};
\colorlet{c}{natcomp!70};
\draw [c] (8.48841,0.734637) -- (8.48841,0.744188);
\draw [c] (8.48841,0.744188) -- (8.48841,0.75374);
\draw [c] (8.47036,0.744188) -- (8.48841,0.744188);
\draw [c] (8.48841,0.744188) -- (8.50645,0.744188);
\definecolor{c}{rgb}{0,0,0};
\colorlet{c}{natcomp!70};
\draw [c] (8.74103,0.734632) -- (8.74103,0.734637);
\draw [c] (8.74103,0.734637) -- (8.74103,0.734641);
\draw [c] (8.72298,0.734637) -- (8.74103,0.734637);
\draw [c] (8.74103,0.734637) -- (8.75907,0.734637);
\definecolor{c}{rgb}{0,0,0};
\colorlet{c}{natcomp!70};
\draw [c] (8.77712,0.734632) -- (8.77712,0.734636);
\draw [c] (8.77712,0.734636) -- (8.77712,0.73464);
\draw [c] (8.75907,0.734636) -- (8.77712,0.734636);
\draw [c] (8.77712,0.734636) -- (8.79516,0.734636);
\definecolor{c}{rgb}{0,0,0};
\colorlet{c}{natcomp!70};
\draw [c] (8.81321,0.734632) -- (8.81321,0.734637);
\draw [c] (8.81321,0.734637) -- (8.81321,0.734642);
\draw [c] (8.79516,0.734637) -- (8.81321,0.734637);
\draw [c] (8.81321,0.734637) -- (8.83125,0.734637);
\definecolor{c}{rgb}{0,0,0};
\colorlet{c}{natcomp!70};
\draw [c] (8.88538,0.734632) -- (8.88538,0.743082);
\draw [c] (8.88538,0.743082) -- (8.88538,0.751533);
\draw [c] (8.86734,0.743082) -- (8.88538,0.743082);
\draw [c] (8.88538,0.743082) -- (8.90343,0.743082);
\definecolor{c}{rgb}{0,0,0};
\colorlet{c}{natcomp!70};
\draw [c] (8.95756,0.734632) -- (8.95756,0.734638);
\draw [c] (8.95756,0.734638) -- (8.95756,0.734644);
\draw [c] (8.93952,0.734638) -- (8.95756,0.734638);
\draw [c] (8.95756,0.734638) -- (8.97561,0.734638);
\definecolor{c}{rgb}{0,0,0};
\colorlet{c}{natcomp!70};
\draw [c] (9.10192,0.734632) -- (9.10192,0.734637);
\draw [c] (9.10192,0.734637) -- (9.10192,0.734641);
\draw [c] (9.08387,0.734637) -- (9.10192,0.734637);
\draw [c] (9.10192,0.734637) -- (9.11996,0.734637);
\definecolor{c}{rgb}{0,0,0};
\colorlet{c}{natcomp!70};
\draw [c] (9.21018,0.734632) -- (9.21018,0.753592);
\draw [c] (9.21018,0.753592) -- (9.21018,0.772551);
\draw [c] (9.19214,0.753592) -- (9.21018,0.753592);
\draw [c] (9.21018,0.753592) -- (9.22823,0.753592);
\definecolor{c}{rgb}{0,0,0};
\colorlet{c}{natcomp!70};
\draw [c] (9.28236,0.734632) -- (9.28236,0.745329);
\draw [c] (9.28236,0.745329) -- (9.28236,0.756027);
\draw [c] (9.26431,0.745329) -- (9.28236,0.745329);
\draw [c] (9.28236,0.745329) -- (9.3004,0.745329);
\definecolor{c}{rgb}{0,0,0};
\colorlet{c}{natcomp!70};
\draw [c] (9.31845,0.734632) -- (9.31845,0.73464);
\draw [c] (9.31845,0.73464) -- (9.31845,0.734648);
\draw [c] (9.3004,0.73464) -- (9.31845,0.73464);
\draw [c] (9.31845,0.73464) -- (9.33649,0.73464);
\definecolor{c}{rgb}{0,0,0};
\colorlet{c}{natcomp!70};
\draw [c] (9.4628,0.734632) -- (9.4628,0.746244);
\draw [c] (9.4628,0.746244) -- (9.4628,0.757855);
\draw [c] (9.44476,0.746244) -- (9.4628,0.746244);
\draw [c] (9.4628,0.746244) -- (9.48085,0.746244);
\definecolor{c}{rgb}{0,0,0};
\colorlet{c}{natcomp!70};
\draw [c] (9.49889,0.734632) -- (9.49889,0.744183);
\draw [c] (9.49889,0.744183) -- (9.49889,0.753735);
\draw [c] (9.48085,0.744183) -- (9.49889,0.744183);
\draw [c] (9.49889,0.744183) -- (9.51694,0.744183);
\definecolor{c}{rgb}{0,0,0};
\colorlet{c}{natcomp!70};
\draw [c] (9.57107,0.734632) -- (9.57107,0.745032);
\draw [c] (9.57107,0.745032) -- (9.57107,0.755432);
\draw [c] (9.55302,0.745032) -- (9.57107,0.745032);
\draw [c] (9.57107,0.745032) -- (9.58911,0.745032);
\definecolor{c}{rgb}{0,0,0};
\colorlet{c}{natcomp!70};
\draw [c] (9.67933,0.734632) -- (9.67933,0.734636);
\draw [c] (9.67933,0.734636) -- (9.67933,0.73464);
\draw [c] (9.66129,0.734636) -- (9.67933,0.734636);
\draw [c] (9.67933,0.734636) -- (9.69738,0.734636);
\definecolor{c}{rgb}{0,0,0};
\colorlet{c}{natcomp!70};
\draw [c] (9.7876,0.734632) -- (9.7876,0.734636);
\draw [c] (9.7876,0.734636) -- (9.7876,0.73464);
\draw [c] (9.76956,0.734636) -- (9.7876,0.734636);
\draw [c] (9.7876,0.734636) -- (9.80564,0.734636);
\definecolor{c}{rgb}{0,0,0};
\colorlet{c}{natcomp!70};
\draw [c] (9.82369,0.734632) -- (9.82369,0.734636);
\draw [c] (9.82369,0.734636) -- (9.82369,0.73464);
\draw [c] (9.80564,0.734636) -- (9.82369,0.734636);
\draw [c] (9.82369,0.734636) -- (9.84173,0.734636);
\definecolor{c}{rgb}{0,0,0};
\colorlet{c}{natcomp!70};
\draw [c] (9.93196,0.734632) -- (9.93196,0.746244);
\draw [c] (9.93196,0.746244) -- (9.93196,0.757855);
\draw [c] (9.91391,0.746244) -- (9.93196,0.746244);
\draw [c] (9.93196,0.746244) -- (9.95,0.746244);
\definecolor{c}{rgb}{0,0,0};
\draw [anchor=base west] (6.09599,5.83614) node[color=c, rotate=0]{ATLAS MC};
\colorlet{c}{natgreen};
\draw [c] (5.14004,5.96347) -- (5.92729,5.96347);
\draw [c] (5.53367,5.7937) -- (5.53367,6.13324);
\definecolor{c}{rgb}{0,0,0};
\draw [anchor=base west] (6.09599,5.27024) node[color=c, rotate=0]{CalcHEP MC};
\colorlet{c}{natcomp!70};
\draw [c] (5.14004,5.39756) -- (5.92729,5.39756);
\draw [c] (5.53367,5.22779) -- (5.53367,5.56734);
\end{tikzpicture}

\end{infilsf}
\end{minipage}
\begin{minipage}[b]{.49\textwidth}
\subcaption{Before application of mapping function.\label{etpv}}
\end{minipage}\hfill
\begin{minipage}[b]{.49\textwidth}
\subcaption{After application of mapping function.\label{etmap}}
\end{minipage}
\caption{The distribution of $E_T^\text{iso}$ in the \atlas{} MC set compared with the distribution in the CalcHEP MC set. In \subcaptionref{etmap}, a mapping function which applies a scale and offset to the values for the CalcHEP MC has been applied.}
\end{figure}

These distributions are now very close to being identical. We correct the remaining discrepancy by reweighting the CalcHEP sample.

The weights found above are assigned to the Monte Carlo samples on an event--by--event basis. These weights will of course carry an uncertainty, which can be derived by simple error propagation from the uncertainties of the histograms used to define them. As these weights are assigned event--by--event, we can track the magnitude of uncertainty assigned to each bin of $M_{\gamma\gamma}$ by taking the root square sum of the errors on the weights of each event placed in that bin. This error must then be included in the systematic uncertainties on the analysis.

Finally, the CalcHEP sample only included events produced by the tree level process, whereas the \atlas{} sample also includes the contribution from the box diagram shown in fig.~\ref{hiorder}. So, to meaningfully compare the two, we must know the contribution from the box diagram. This, we glean from another \atlas{} MC sample\footnote{See appendix \ref{ax.ggb}}, which provides a $M_{\gamma\gamma}$ distribution illustrated in fig.~\ref{boxmgg}. As with the estimated background, this distribution has insufficient statistics to accurately represent the shape of the distribution in the interesting region above 1\,000 GeV, forcing us to extrapolate the shape of this distribution as well.

\begin{figure}[htp]
\begin{minipage}[b]{.69\textwidth}
\begin{infilsf} \tiny
\begin{tikzpicture}[x=.092\textwidth,y=.092\textwidth]
\pgfdeclareplotmark{cross} {
\pgfpathmoveto{\pgfpoint{-0.3\pgfplotmarksize}{\pgfplotmarksize}}
\pgfpathlineto{\pgfpoint{+0.3\pgfplotmarksize}{\pgfplotmarksize}}
\pgfpathlineto{\pgfpoint{+0.3\pgfplotmarksize}{0.3\pgfplotmarksize}}
\pgfpathlineto{\pgfpoint{+1\pgfplotmarksize}{0.3\pgfplotmarksize}}
\pgfpathlineto{\pgfpoint{+1\pgfplotmarksize}{-0.3\pgfplotmarksize}}
\pgfpathlineto{\pgfpoint{+0.3\pgfplotmarksize}{-0.3\pgfplotmarksize}}
\pgfpathlineto{\pgfpoint{+0.3\pgfplotmarksize}{-1.\pgfplotmarksize}}
\pgfpathlineto{\pgfpoint{-0.3\pgfplotmarksize}{-1.\pgfplotmarksize}}
\pgfpathlineto{\pgfpoint{-0.3\pgfplotmarksize}{-0.3\pgfplotmarksize}}
\pgfpathlineto{\pgfpoint{-1.\pgfplotmarksize}{-0.3\pgfplotmarksize}}
\pgfpathlineto{\pgfpoint{-1.\pgfplotmarksize}{0.3\pgfplotmarksize}}
\pgfpathlineto{\pgfpoint{-0.3\pgfplotmarksize}{0.3\pgfplotmarksize}}
\pgfpathclose
\pgfusepathqstroke
}
\pgfdeclareplotmark{cross*} {
\pgfpathmoveto{\pgfpoint{-0.3\pgfplotmarksize}{\pgfplotmarksize}}
\pgfpathlineto{\pgfpoint{+0.3\pgfplotmarksize}{\pgfplotmarksize}}
\pgfpathlineto{\pgfpoint{+0.3\pgfplotmarksize}{0.3\pgfplotmarksize}}
\pgfpathlineto{\pgfpoint{+1\pgfplotmarksize}{0.3\pgfplotmarksize}}
\pgfpathlineto{\pgfpoint{+1\pgfplotmarksize}{-0.3\pgfplotmarksize}}
\pgfpathlineto{\pgfpoint{+0.3\pgfplotmarksize}{-0.3\pgfplotmarksize}}
\pgfpathlineto{\pgfpoint{+0.3\pgfplotmarksize}{-1.\pgfplotmarksize}}
\pgfpathlineto{\pgfpoint{-0.3\pgfplotmarksize}{-1.\pgfplotmarksize}}
\pgfpathlineto{\pgfpoint{-0.3\pgfplotmarksize}{-0.3\pgfplotmarksize}}
\pgfpathlineto{\pgfpoint{-1.\pgfplotmarksize}{-0.3\pgfplotmarksize}}
\pgfpathlineto{\pgfpoint{-1.\pgfplotmarksize}{0.3\pgfplotmarksize}}
\pgfpathlineto{\pgfpoint{-0.3\pgfplotmarksize}{0.3\pgfplotmarksize}}
\pgfpathclose
\pgfusepathqfillstroke
}
\pgfdeclareplotmark{newstar} {
\pgfpathmoveto{\pgfqpoint{0pt}{\pgfplotmarksize}}
\pgfpathlineto{\pgfqpointpolar{44}{0.5\pgfplotmarksize}}
\pgfpathlineto{\pgfqpointpolar{18}{\pgfplotmarksize}}
\pgfpathlineto{\pgfqpointpolar{-20}{0.5\pgfplotmarksize}}
\pgfpathlineto{\pgfqpointpolar{-54}{\pgfplotmarksize}}
\pgfpathlineto{\pgfqpointpolar{-90}{0.5\pgfplotmarksize}}
\pgfpathlineto{\pgfqpointpolar{234}{\pgfplotmarksize}}
\pgfpathlineto{\pgfqpointpolar{198}{0.5\pgfplotmarksize}}
\pgfpathlineto{\pgfqpointpolar{162}{\pgfplotmarksize}}
\pgfpathlineto{\pgfqpointpolar{134}{0.5\pgfplotmarksize}}
\pgfpathclose
\pgfusepathqstroke
}
\pgfdeclareplotmark{newstar*} {
\pgfpathmoveto{\pgfqpoint{0pt}{\pgfplotmarksize}}
\pgfpathlineto{\pgfqpointpolar{44}{0.5\pgfplotmarksize}}
\pgfpathlineto{\pgfqpointpolar{18}{\pgfplotmarksize}}
\pgfpathlineto{\pgfqpointpolar{-20}{0.5\pgfplotmarksize}}
\pgfpathlineto{\pgfqpointpolar{-54}{\pgfplotmarksize}}
\pgfpathlineto{\pgfqpointpolar{-90}{0.5\pgfplotmarksize}}
\pgfpathlineto{\pgfqpointpolar{234}{\pgfplotmarksize}}
\pgfpathlineto{\pgfqpointpolar{198}{0.5\pgfplotmarksize}}
\pgfpathlineto{\pgfqpointpolar{162}{\pgfplotmarksize}}
\pgfpathlineto{\pgfqpointpolar{134}{0.5\pgfplotmarksize}}
\pgfpathclose
\pgfusepathqfillstroke
}
\definecolor{c}{rgb}{1,1,1};
\draw [color=c, fill=c] (0,0) rectangle (10,6.80516);
\draw [color=c, fill=c] (1,0.680516) rectangle (9.95,6.73711);
\definecolor{c}{rgb}{0,0,0};
\draw [c] (1,0.680516) -- (1,6.73711) -- (9.95,6.73711) -- (9.95,0.680516) -- (1,0.680516);
\definecolor{c}{rgb}{1,1,1};
\draw [color=c, fill=c] (1,0.680516) rectangle (9.95,6.73711);
\definecolor{c}{rgb}{0,0,0};
\draw [c] (1,0.680516) -- (1,6.73711) -- (9.95,6.73711) -- (9.95,0.680516) -- (1,0.680516);
\colorlet{c}{natgreen};
\draw [c] (1.6646,0.680516) -- (1.6646,2.3049);
\draw [c] (1.6646,2.3049) -- (1.6646,2.69635);
\draw [c] (1.6203,2.3049) -- (1.6646,2.3049);
\draw [c] (1.6646,2.3049) -- (1.70891,2.3049);
\definecolor{c}{rgb}{0,0,0};
\colorlet{c}{natgreen};
\draw [c] (1.75322,2.38132) -- (1.75322,2.95036);
\draw [c] (1.75322,2.95036) -- (1.75322,3.22799);
\draw [c] (1.70891,2.95036) -- (1.75322,2.95036);
\draw [c] (1.75322,2.95036) -- (1.79752,2.95036);
\definecolor{c}{rgb}{0,0,0};
\colorlet{c}{natgreen};
\draw [c] (1.84183,4.31431) -- (1.84183,4.43557);
\draw [c] (1.84183,4.43557) -- (1.84183,4.53534);
\draw [c] (1.79752,4.43557) -- (1.84183,4.43557);
\draw [c] (1.84183,4.43557) -- (1.88614,4.43557);
\definecolor{c}{rgb}{0,0,0};
\colorlet{c}{natgreen};
\draw [c] (1.93045,6.01844) -- (1.93045,6.04598);
\draw [c] (1.93045,6.04598) -- (1.93045,6.07223);
\draw [c] (1.88614,6.04598) -- (1.93045,6.04598);
\draw [c] (1.93045,6.04598) -- (1.97475,6.04598);
\definecolor{c}{rgb}{0,0,0};
\colorlet{c}{natgreen};
\draw [c] (2.01906,6.4381) -- (2.01906,6.45705);
\draw [c] (2.01906,6.45705) -- (2.01906,6.47539);
\draw [c] (1.97475,6.45705) -- (2.01906,6.45705);
\draw [c] (2.01906,6.45705) -- (2.06337,6.45705);
\definecolor{c}{rgb}{0,0,0};
\colorlet{c}{natgreen};
\draw [c] (2.10767,6.48095) -- (2.10767,6.49923);
\draw [c] (2.10767,6.49923) -- (2.10767,6.51694);
\draw [c] (2.06337,6.49923) -- (2.10767,6.49923);
\draw [c] (2.10767,6.49923) -- (2.15198,6.49923);
\definecolor{c}{rgb}{0,0,0};
\colorlet{c}{natgreen};
\draw [c] (2.19629,6.42817) -- (2.19629,6.44727);
\draw [c] (2.19629,6.44727) -- (2.19629,6.46573);
\draw [c] (2.15198,6.44727) -- (2.19629,6.44727);
\draw [c] (2.19629,6.44727) -- (2.24059,6.44727);
\definecolor{c}{rgb}{0,0,0};
\colorlet{c}{natgreen};
\draw [c] (2.2849,6.34203) -- (2.2849,6.36277);
\draw [c] (2.2849,6.36277) -- (2.2849,6.38277);
\draw [c] (2.24059,6.36277) -- (2.2849,6.36277);
\draw [c] (2.2849,6.36277) -- (2.32921,6.36277);
\definecolor{c}{rgb}{0,0,0};
\colorlet{c}{natgreen};
\draw [c] (2.37351,6.22705) -- (2.37351,6.24993);
\draw [c] (2.37351,6.24993) -- (2.37351,6.27192);
\draw [c] (2.32921,6.24993) -- (2.37351,6.24993);
\draw [c] (2.37351,6.24993) -- (2.41782,6.24993);
\definecolor{c}{rgb}{0,0,0};
\colorlet{c}{natgreen};
\draw [c] (2.46213,6.10396) -- (2.46213,6.12959);
\draw [c] (2.46213,6.12959) -- (2.46213,6.15411);
\draw [c] (2.41782,6.12959) -- (2.46213,6.12959);
\draw [c] (2.46213,6.12959) -- (2.50644,6.12959);
\definecolor{c}{rgb}{0,0,0};
\colorlet{c}{natgreen};
\draw [c] (2.55074,5.923) -- (2.55074,5.95292);
\draw [c] (2.55074,5.95292) -- (2.55074,5.98134);
\draw [c] (2.50644,5.95292) -- (2.55074,5.95292);
\draw [c] (2.55074,5.95292) -- (2.59505,5.95292);
\definecolor{c}{rgb}{0,0,0};
\colorlet{c}{natgreen};
\draw [c] (2.63936,5.81438) -- (2.63936,5.84748);
\draw [c] (2.63936,5.84748) -- (2.63936,5.87875);
\draw [c] (2.59505,5.84748) -- (2.63936,5.84748);
\draw [c] (2.63936,5.84748) -- (2.68366,5.84748);
\definecolor{c}{rgb}{0,0,0};
\colorlet{c}{natgreen};
\draw [c] (2.72797,5.6363) -- (2.72797,5.67497);
\draw [c] (2.72797,5.67497) -- (2.72797,5.71115);
\draw [c] (2.68366,5.67497) -- (2.72797,5.67497);
\draw [c] (2.72797,5.67497) -- (2.77228,5.67497);
\definecolor{c}{rgb}{0,0,0};
\colorlet{c}{natgreen};
\draw [c] (2.81658,5.53226) -- (2.81658,5.57507);
\draw [c] (2.81658,5.57507) -- (2.81658,5.61487);
\draw [c] (2.77228,5.57507) -- (2.81658,5.57507);
\draw [c] (2.81658,5.57507) -- (2.86089,5.57507);
\definecolor{c}{rgb}{0,0,0};
\colorlet{c}{natgreen};
\draw [c] (2.9052,5.35892) -- (2.9052,5.4081);
\draw [c] (2.9052,5.4081) -- (2.9052,5.45334);
\draw [c] (2.86089,5.4081) -- (2.9052,5.4081);
\draw [c] (2.9052,5.4081) -- (2.94951,5.4081);
\definecolor{c}{rgb}{0,0,0};
\colorlet{c}{natgreen};
\draw [c] (2.99381,5.24072) -- (2.99381,5.29613);
\draw [c] (2.99381,5.29613) -- (2.99381,5.34658);
\draw [c] (2.94951,5.29613) -- (2.99381,5.29613);
\draw [c] (2.99381,5.29613) -- (3.03812,5.29613);
\definecolor{c}{rgb}{0,0,0};
\colorlet{c}{natgreen};
\draw [c] (3.08243,5.21899) -- (3.08243,5.27515);
\draw [c] (3.08243,5.27515) -- (3.08243,5.32623);
\draw [c] (3.03812,5.27515) -- (3.08243,5.27515);
\draw [c] (3.08243,5.27515) -- (3.12673,5.27515);
\definecolor{c}{rgb}{0,0,0};
\colorlet{c}{natgreen};
\draw [c] (3.17104,5.09339) -- (3.17104,5.1552);
\draw [c] (3.17104,5.1552) -- (3.17104,5.21091);
\draw [c] (3.12673,5.1552) -- (3.17104,5.1552);
\draw [c] (3.17104,5.1552) -- (3.21535,5.1552);
\definecolor{c}{rgb}{0,0,0};
\colorlet{c}{natgreen};
\draw [c] (3.25965,4.96656) -- (3.25965,5.03664);
\draw [c] (3.25965,5.03664) -- (3.25965,5.09898);
\draw [c] (3.21535,5.03664) -- (3.25965,5.03664);
\draw [c] (3.25965,5.03664) -- (3.30396,5.03664);
\definecolor{c}{rgb}{0,0,0};
\colorlet{c}{natgreen};
\draw [c] (3.34827,4.84513) -- (3.34827,4.92243);
\draw [c] (3.34827,4.92243) -- (3.34827,4.99041);
\draw [c] (3.30396,4.92243) -- (3.34827,4.92243);
\draw [c] (3.34827,4.92243) -- (3.39257,4.92243);
\definecolor{c}{rgb}{0,0,0};
\colorlet{c}{natgreen};
\draw [c] (3.43688,4.68676) -- (3.43688,4.7774);
\draw [c] (3.43688,4.7774) -- (3.43688,4.85548);
\draw [c] (3.39257,4.7774) -- (3.43688,4.7774);
\draw [c] (3.43688,4.7774) -- (3.48119,4.7774);
\definecolor{c}{rgb}{0,0,0};
\colorlet{c}{natgreen};
\draw [c] (3.5255,4.69232) -- (3.5255,4.78254);
\draw [c] (3.5255,4.78254) -- (3.5255,4.86032);
\draw [c] (3.48119,4.78254) -- (3.5255,4.78254);
\draw [c] (3.5255,4.78254) -- (3.5698,4.78254);
\definecolor{c}{rgb}{0,0,0};
\colorlet{c}{natgreen};
\draw [c] (3.61411,4.44789) -- (3.61411,4.55535);
\draw [c] (3.61411,4.55535) -- (3.61411,4.64559);
\draw [c] (3.5698,4.55535) -- (3.61411,4.55535);
\draw [c] (3.61411,4.55535) -- (3.65842,4.55535);
\definecolor{c}{rgb}{0,0,0};
\colorlet{c}{natgreen};
\draw [c] (3.70272,4.38517) -- (3.70272,4.50101);
\draw [c] (3.70272,4.50101) -- (3.70272,4.59709);
\draw [c] (3.65842,4.50101) -- (3.70272,4.50101);
\draw [c] (3.70272,4.50101) -- (3.74703,4.50101);
\definecolor{c}{rgb}{0,0,0};
\colorlet{c}{natgreen};
\draw [c] (3.79134,4.43828) -- (3.79134,4.54987);
\draw [c] (3.79134,4.54987) -- (3.79134,4.64301);
\draw [c] (3.74703,4.54987) -- (3.79134,4.54987);
\draw [c] (3.79134,4.54987) -- (3.83564,4.54987);
\definecolor{c}{rgb}{0,0,0};
\colorlet{c}{natgreen};
\draw [c] (3.87995,4.17957) -- (3.87995,4.32023);
\draw [c] (3.87995,4.32023) -- (3.87995,4.43275);
\draw [c] (3.83564,4.32023) -- (3.87995,4.32023);
\draw [c] (3.87995,4.32023) -- (3.92426,4.32023);
\definecolor{c}{rgb}{0,0,0};
\colorlet{c}{natgreen};
\draw [c] (3.96856,3.71331) -- (3.96856,3.92083);
\draw [c] (3.96856,3.92083) -- (3.96856,4.07224);
\draw [c] (3.92426,3.92083) -- (3.96856,3.92083);
\draw [c] (3.96856,3.92083) -- (4.01287,3.92083);
\definecolor{c}{rgb}{0,0,0};
\colorlet{c}{natgreen};
\draw [c] (4.05718,3.73488) -- (4.05718,3.93623);
\draw [c] (4.05718,3.93623) -- (4.05718,4.08436);
\draw [c] (4.01287,3.93623) -- (4.05718,3.93623);
\draw [c] (4.05718,3.93623) -- (4.10149,3.93623);
\definecolor{c}{rgb}{0,0,0};
\colorlet{c}{natgreen};
\draw [c] (4.14579,3.98827) -- (4.14579,4.14568);
\draw [c] (4.14579,4.14568) -- (4.14579,4.26864);
\draw [c] (4.10149,4.14568) -- (4.14579,4.14568);
\draw [c] (4.14579,4.14568) -- (4.1901,4.14568);
\definecolor{c}{rgb}{0,0,0};
\colorlet{c}{natgreen};
\draw [c] (4.23441,3.70996) -- (4.23441,3.92222);
\draw [c] (4.23441,3.92222) -- (4.23441,4.07613);
\draw [c] (4.1901,3.92222) -- (4.23441,3.92222);
\draw [c] (4.23441,3.92222) -- (4.27871,3.92222);
\definecolor{c}{rgb}{0,0,0};
\colorlet{c}{natgreen};
\draw [c] (4.32302,3.33856) -- (4.32302,3.6243);
\draw [c] (4.32302,3.6243) -- (4.32302,3.81313);
\draw [c] (4.27871,3.6243) -- (4.32302,3.6243);
\draw [c] (4.32302,3.6243) -- (4.36733,3.6243);
\definecolor{c}{rgb}{0,0,0};
\colorlet{c}{natgreen};
\draw [c] (4.41163,3.32705) -- (4.41163,3.6277);
\draw [c] (4.41163,3.6277) -- (4.41163,3.82286);
\draw [c] (4.36733,3.6277) -- (4.41163,3.6277);
\draw [c] (4.41163,3.6277) -- (4.45594,3.6277);
\definecolor{c}{rgb}{0,0,0};
\colorlet{c}{natgreen};
\draw [c] (4.50025,3.83238) -- (4.50025,4.02204);
\draw [c] (4.50025,4.02204) -- (4.50025,4.16377);
\draw [c] (4.45594,4.02204) -- (4.50025,4.02204);
\draw [c] (4.50025,4.02204) -- (4.54455,4.02204);
\definecolor{c}{rgb}{0,0,0};
\colorlet{c}{natgreen};
\draw [c] (4.58886,3.27564) -- (4.58886,3.5691);
\draw [c] (4.58886,3.5691) -- (4.58886,3.76124);
\draw [c] (4.54455,3.5691) -- (4.58886,3.5691);
\draw [c] (4.58886,3.5691) -- (4.63317,3.5691);
\definecolor{c}{rgb}{0,0,0};
\colorlet{c}{natgreen};
\draw [c] (4.67748,2.74147) -- (4.67748,3.24921);
\draw [c] (4.67748,3.24921) -- (4.67748,3.51219);
\draw [c] (4.63317,3.24921) -- (4.67748,3.24921);
\draw [c] (4.67748,3.24921) -- (4.72178,3.24921);
\definecolor{c}{rgb}{0,0,0};
\colorlet{c}{natgreen};
\draw [c] (4.76609,3.60758) -- (4.76609,3.85031);
\draw [c] (4.76609,3.85031) -- (4.76609,4.01953);
\draw [c] (4.72178,3.85031) -- (4.76609,3.85031);
\draw [c] (4.76609,3.85031) -- (4.8104,3.85031);
\definecolor{c}{rgb}{0,0,0};
\colorlet{c}{natgreen};
\draw [c] (4.94332,2.30513) -- (4.94332,3.00293);
\draw [c] (4.94332,3.00293) -- (4.94332,3.30569);
\draw [c] (4.89901,3.00293) -- (4.94332,3.00293);
\draw [c] (4.94332,3.00293) -- (4.98762,3.00293);
\definecolor{c}{rgb}{0,0,0};
\colorlet{c}{natgreen};
\draw [c] (5.12054,3.19183) -- (5.12054,3.50284);
\draw [c] (5.12054,3.50284) -- (5.12054,3.70225);
\draw [c] (5.07624,3.50284) -- (5.12054,3.50284);
\draw [c] (5.12054,3.50284) -- (5.16485,3.50284);
\definecolor{c}{rgb}{0,0,0};
\colorlet{c}{natgreen};
\draw [c] (5.20916,2.95994) -- (5.20916,3.3768);
\draw [c] (5.20916,3.3768) -- (5.20916,3.61401);
\draw [c] (5.16485,3.3768) -- (5.20916,3.3768);
\draw [c] (5.20916,3.3768) -- (5.25347,3.3768);
\definecolor{c}{rgb}{0,0,0};
\colorlet{c}{natgreen};
\draw [c] (5.29777,2.67687) -- (5.29777,3.18022);
\draw [c] (5.29777,3.18022) -- (5.29777,3.44206);
\draw [c] (5.25347,3.18022) -- (5.29777,3.18022);
\draw [c] (5.29777,3.18022) -- (5.34208,3.18022);
\definecolor{c}{rgb}{0,0,0};
\colorlet{c}{natgreen};
\draw [c] (5.38639,0.680516) -- (5.38639,2.77962);
\draw [c] (5.38639,2.77962) -- (5.38639,3.17107);
\draw [c] (5.34208,2.77962) -- (5.38639,2.77962);
\draw [c] (5.38639,2.77962) -- (5.43069,2.77962);
\definecolor{c}{rgb}{0,0,0};
\colorlet{c}{natgreen};
\draw [c] (5.475,2.32505) -- (5.475,3.01853);
\draw [c] (5.475,3.01853) -- (5.475,3.32055);
\draw [c] (5.43069,3.01853) -- (5.475,3.01853);
\draw [c] (5.475,3.01853) -- (5.51931,3.01853);
\definecolor{c}{rgb}{0,0,0};
\colorlet{c}{natgreen};
\draw [c] (5.56361,0.680516) -- (5.56361,2.25224);
\draw [c] (5.56361,2.25224) -- (5.56361,2.64369);
\draw [c] (5.51931,2.25224) -- (5.56361,2.25224);
\draw [c] (5.56361,2.25224) -- (5.60792,2.25224);
\definecolor{c}{rgb}{0,0,0};
\colorlet{c}{natgreen};
\draw [c] (5.65223,2.72587) -- (5.65223,3.2156);
\draw [c] (5.65223,3.2156) -- (5.65223,3.47388);
\draw [c] (5.60792,3.2156) -- (5.65223,3.2156);
\draw [c] (5.65223,3.2156) -- (5.69653,3.2156);
\definecolor{c}{rgb}{0,0,0};
\colorlet{c}{natgreen};
\draw [c] (5.74084,2.27391) -- (5.74084,2.96766);
\draw [c] (5.74084,2.96766) -- (5.74084,3.26973);
\draw [c] (5.69653,2.96766) -- (5.74084,2.96766);
\draw [c] (5.74084,2.96766) -- (5.78515,2.96766);
\definecolor{c}{rgb}{0,0,0};
\colorlet{c}{natgreen};
\draw [c] (5.82946,0.680516) -- (5.82946,2.51889);
\draw [c] (5.82946,2.51889) -- (5.82946,2.91034);
\draw [c] (5.78515,2.51889) -- (5.82946,2.51889);
\draw [c] (5.82946,2.51889) -- (5.87376,2.51889);
\definecolor{c}{rgb}{0,0,0};
\colorlet{c}{natgreen};
\draw [c] (6.00668,0.680516) -- (6.00668,2.62777);
\draw [c] (6.00668,2.62777) -- (6.00668,3.01922);
\draw [c] (5.96238,2.62777) -- (6.00668,2.62777);
\draw [c] (6.00668,2.62777) -- (6.05099,2.62777);
\definecolor{c}{rgb}{0,0,0};
\colorlet{c}{natgreen};
\draw [c] (6.0953,0.680516) -- (6.0953,2.62558);
\draw [c] (6.0953,2.62558) -- (6.0953,3.01404);
\draw [c] (6.05099,2.62558) -- (6.0953,2.62558);
\draw [c] (6.0953,2.62558) -- (6.1396,2.62558);
\definecolor{c}{rgb}{0,0,0};
\colorlet{c}{natgreen};
\draw [c] (6.36114,0.680516) -- (6.36114,1.60554);
\draw [c] (6.36114,1.60554) -- (6.36114,1.99699);
\draw [c] (6.31683,1.60554) -- (6.36114,1.60554);
\draw [c] (6.36114,1.60554) -- (6.40545,1.60554);
\definecolor{c}{rgb}{0,0,0};
\colorlet{c}{natgreen};
\draw [c] (6.62698,0.680516) -- (6.62698,2.65744);
\draw [c] (6.62698,2.65744) -- (6.62698,3.04889);
\draw [c] (6.58267,2.65744) -- (6.62698,2.65744);
\draw [c] (6.62698,2.65744) -- (6.67129,2.65744);
\definecolor{c}{rgb}{0,0,0};
\colorlet{c}{natgreen};
\draw [c] (6.71559,2.78195) -- (6.71559,3.27658);
\draw [c] (6.71559,3.27658) -- (6.71559,3.53615);
\draw [c] (6.67129,3.27658) -- (6.71559,3.27658);
\draw [c] (6.71559,3.27658) -- (6.7599,3.27658);
\definecolor{c}{rgb}{0,0,0};
\colorlet{c}{natgreen};
\draw [c] (6.80421,0.680516) -- (6.80421,2.68178);
\draw [c] (6.80421,2.68178) -- (6.80421,3.07323);
\draw [c] (6.7599,2.68178) -- (6.80421,2.68178);
\draw [c] (6.80421,2.68178) -- (6.84852,2.68178);
\definecolor{c}{rgb}{0,0,0};
\colorlet{c}{natgreen};
\draw [c] (7.07005,0.680516) -- (7.07005,2.5647);
\draw [c] (7.07005,2.5647) -- (7.07005,2.95615);
\draw [c] (7.02574,2.5647) -- (7.07005,2.5647);
\draw [c] (7.07005,2.5647) -- (7.11436,2.5647);
\definecolor{c}{rgb}{0,0,0};
\colorlet{c}{natgreen};
\draw [c] (7.4245,0.680516) -- (7.4245,2.60897);
\draw [c] (7.4245,2.60897) -- (7.4245,3.00042);
\draw [c] (7.3802,2.60897) -- (7.4245,2.60897);
\draw [c] (7.4245,2.60897) -- (7.46881,2.60897);
\definecolor{c}{rgb}{0,0,0};
\colorlet{c}{natgreen};
\draw [c] (8.39926,0.680516) -- (8.39926,1.05001);
\draw [c] (8.39926,1.05001) -- (8.39926,1.44146);
\draw [c] (8.35495,1.05001) -- (8.39926,1.05001);
\draw [c] (8.39926,1.05001) -- (8.44356,1.05001);
\definecolor{c}{rgb}{0,0,0};
\draw [c] (1,0.680516) -- (9.95,0.680516);
\draw [anchor= east] (9.95,0.108883) node[color=c, rotate=0]{$M_{\gamma\gamma}\text{ [GeV]}$};
\draw [c] (1,0.863234) -- (1,0.680516);
\draw [c] (1.44307,0.771875) -- (1.44307,0.680516);
\draw [c] (1.88614,0.771875) -- (1.88614,0.680516);
\draw [c] (2.32921,0.771875) -- (2.32921,0.680516);
\draw [c] (2.77228,0.863234) -- (2.77228,0.680516);
\draw [c] (3.21535,0.771875) -- (3.21535,0.680516);
\draw [c] (3.65842,0.771875) -- (3.65842,0.680516);
\draw [c] (4.10149,0.771875) -- (4.10149,0.680516);
\draw [c] (4.54455,0.863234) -- (4.54455,0.680516);
\draw [c] (4.98762,0.771875) -- (4.98762,0.680516);
\draw [c] (5.43069,0.771875) -- (5.43069,0.680516);
\draw [c] (5.87376,0.771875) -- (5.87376,0.680516);
\draw [c] (6.31683,0.863234) -- (6.31683,0.680516);
\draw [c] (6.7599,0.771875) -- (6.7599,0.680516);
\draw [c] (7.20297,0.771875) -- (7.20297,0.680516);
\draw [c] (7.64604,0.771875) -- (7.64604,0.680516);
\draw [c] (8.08911,0.863234) -- (8.08911,0.680516);
\draw [c] (8.53218,0.771875) -- (8.53218,0.680516);
\draw [c] (8.97525,0.771875) -- (8.97525,0.680516);
\draw [c] (9.41832,0.771875) -- (9.41832,0.680516);
\draw [c] (9.86139,0.863234) -- (9.86139,0.680516);
\draw [c] (9.86139,0.863234) -- (9.86139,0.680516);
\draw [anchor=base] (1,0.353868) node[color=c, rotate=0]{0};
\draw [anchor=base] (2.77228,0.353868) node[color=c, rotate=0]{200};
\draw [anchor=base] (4.54455,0.353868) node[color=c, rotate=0]{400};
\draw [anchor=base] (6.31683,0.353868) node[color=c, rotate=0]{600};
\draw [anchor=base] (8.08911,0.353868) node[color=c, rotate=0]{800};
\draw [anchor=base] (9.86139,0.353868) node[color=c, rotate=0]{1000};
\draw [c] (1,0.680516) -- (1,6.73711);
\draw [anchor= east] (-0.12,6.73711) node[color=c, rotate=90]{Number of events};
\draw [c] (1.1335,0.7327) -- (1,0.7327);
\draw [c] (1.1335,0.895167) -- (1,0.895167);
\draw [c] (1.1335,1.02119) -- (1,1.02119);
\draw [c] (1.1335,1.12415) -- (1,1.12415);
\draw [c] (1.1335,1.21121) -- (1,1.21121);
\draw [c] (1.1335,1.28662) -- (1,1.28662);
\draw [c] (1.1335,1.35314) -- (1,1.35314);
\draw [c] (1.267,1.41264) -- (1,1.41264);
\draw [anchor= east] (0.922,1.41264) node[color=c, rotate=0]{$10^{-1}$};
\draw [c] (1.1335,1.80409) -- (1,1.80409);
\draw [c] (1.1335,2.03307) -- (1,2.03307);
\draw [c] (1.1335,2.19554) -- (1,2.19554);
\draw [c] (1.1335,2.32156) -- (1,2.32156);
\draw [c] (1.1335,2.42453) -- (1,2.42453);
\draw [c] (1.1335,2.51158) -- (1,2.51158);
\draw [c] (1.1335,2.58699) -- (1,2.58699);
\draw [c] (1.1335,2.65351) -- (1,2.65351);
\draw [c] (1.267,2.71301) -- (1,2.71301);
\draw [anchor= east] (0.922,2.71301) node[color=c, rotate=0]{1};
\draw [c] (1.1335,3.10446) -- (1,3.10446);
\draw [c] (1.1335,3.33345) -- (1,3.33345);
\draw [c] (1.1335,3.49592) -- (1,3.49592);
\draw [c] (1.1335,3.62193) -- (1,3.62193);
\draw [c] (1.1335,3.7249) -- (1,3.7249);
\draw [c] (1.1335,3.81196) -- (1,3.81196);
\draw [c] (1.1335,3.88737) -- (1,3.88737);
\draw [c] (1.1335,3.95388) -- (1,3.95388);
\draw [c] (1.267,4.01339) -- (1,4.01339);
\draw [anchor= east] (0.922,4.01339) node[color=c, rotate=0]{10};
\draw [c] (1.1335,4.40484) -- (1,4.40484);
\draw [c] (1.1335,4.63382) -- (1,4.63382);
\draw [c] (1.1335,4.79629) -- (1,4.79629);
\draw [c] (1.1335,4.92231) -- (1,4.92231);
\draw [c] (1.1335,5.02527) -- (1,5.02527);
\draw [c] (1.1335,5.11233) -- (1,5.11233);
\draw [c] (1.1335,5.18774) -- (1,5.18774);
\draw [c] (1.1335,5.25426) -- (1,5.25426);
\draw [c] (1.267,5.31376) -- (1,5.31376);
\draw [anchor= east] (0.922,5.31376) node[color=c, rotate=0]{$10^{2}$};
\draw [c] (1.1335,5.70521) -- (1,5.70521);
\draw [c] (1.1335,5.9342) -- (1,5.9342);
\draw [c] (1.1335,6.09666) -- (1,6.09666);
\draw [c] (1.1335,6.22268) -- (1,6.22268);
\draw [c] (1.1335,6.32565) -- (1,6.32565);
\draw [c] (1.1335,6.4127) -- (1,6.4127);
\draw [c] (1.1335,6.48812) -- (1,6.48812);
\draw [c] (1.1335,6.55463) -- (1,6.55463);
\draw [c] (1.267,6.61414) -- (1,6.61414);
\draw [anchor= east] (0.922,6.61414) node[color=c, rotate=0]{$10^{3}$};
\end{tikzpicture}

\end{infilsf}
\end{minipage}
\begin{minipage}[b]{.3\textwidth}
\caption{The distribution of invariant masses in the \atlas{} box diagram data set.}\label{boxmgg}
\end{minipage}
\end{figure}

Adding this to the CalcHEP sample, we get the distribution in fig.~\ref{ggcomp}. As should be evident, there is still a deficit in the CalcHEP sample. The box diagram contribution added to the CalcHEP sample was produced by a different generator than was used to generate the \atlas{} $\gamma\gamma$ sample, which was not necessarily configured with the same parameters as was used for the present sample, or the \altas{} sample could contain effects from interference between the diagrams which simply adding together samples does not capture. In any case, to be comparable to data, both MC sets must be combined with the $\gamma$jet sample. The two combined samples will be consistent within their errors.

\begin{figure}[htp]
\begin{minipage}[b]{.69\textwidth}
\begin{infilsf} \tiny
\begin{tikzpicture}[x=.092\textwidth,y=.092\textwidth]
\pgfdeclareplotmark{cross} {
\pgfpathmoveto{\pgfpoint{-0.3\pgfplotmarksize}{\pgfplotmarksize}}
\pgfpathlineto{\pgfpoint{+0.3\pgfplotmarksize}{\pgfplotmarksize}}
\pgfpathlineto{\pgfpoint{+0.3\pgfplotmarksize}{0.3\pgfplotmarksize}}
\pgfpathlineto{\pgfpoint{+1\pgfplotmarksize}{0.3\pgfplotmarksize}}
\pgfpathlineto{\pgfpoint{+1\pgfplotmarksize}{-0.3\pgfplotmarksize}}
\pgfpathlineto{\pgfpoint{+0.3\pgfplotmarksize}{-0.3\pgfplotmarksize}}
\pgfpathlineto{\pgfpoint{+0.3\pgfplotmarksize}{-1.\pgfplotmarksize}}
\pgfpathlineto{\pgfpoint{-0.3\pgfplotmarksize}{-1.\pgfplotmarksize}}
\pgfpathlineto{\pgfpoint{-0.3\pgfplotmarksize}{-0.3\pgfplotmarksize}}
\pgfpathlineto{\pgfpoint{-1.\pgfplotmarksize}{-0.3\pgfplotmarksize}}
\pgfpathlineto{\pgfpoint{-1.\pgfplotmarksize}{0.3\pgfplotmarksize}}
\pgfpathlineto{\pgfpoint{-0.3\pgfplotmarksize}{0.3\pgfplotmarksize}}
\pgfpathclose
\pgfusepathqstroke
}
\pgfdeclareplotmark{cross*} {
\pgfpathmoveto{\pgfpoint{-0.3\pgfplotmarksize}{\pgfplotmarksize}}
\pgfpathlineto{\pgfpoint{+0.3\pgfplotmarksize}{\pgfplotmarksize}}
\pgfpathlineto{\pgfpoint{+0.3\pgfplotmarksize}{0.3\pgfplotmarksize}}
\pgfpathlineto{\pgfpoint{+1\pgfplotmarksize}{0.3\pgfplotmarksize}}
\pgfpathlineto{\pgfpoint{+1\pgfplotmarksize}{-0.3\pgfplotmarksize}}
\pgfpathlineto{\pgfpoint{+0.3\pgfplotmarksize}{-0.3\pgfplotmarksize}}
\pgfpathlineto{\pgfpoint{+0.3\pgfplotmarksize}{-1.\pgfplotmarksize}}
\pgfpathlineto{\pgfpoint{-0.3\pgfplotmarksize}{-1.\pgfplotmarksize}}
\pgfpathlineto{\pgfpoint{-0.3\pgfplotmarksize}{-0.3\pgfplotmarksize}}
\pgfpathlineto{\pgfpoint{-1.\pgfplotmarksize}{-0.3\pgfplotmarksize}}
\pgfpathlineto{\pgfpoint{-1.\pgfplotmarksize}{0.3\pgfplotmarksize}}
\pgfpathlineto{\pgfpoint{-0.3\pgfplotmarksize}{0.3\pgfplotmarksize}}
\pgfpathclose
\pgfusepathqfillstroke
}
\pgfdeclareplotmark{newstar} {
\pgfpathmoveto{\pgfqpoint{0pt}{\pgfplotmarksize}}
\pgfpathlineto{\pgfqpointpolar{44}{0.5\pgfplotmarksize}}
\pgfpathlineto{\pgfqpointpolar{18}{\pgfplotmarksize}}
\pgfpathlineto{\pgfqpointpolar{-20}{0.5\pgfplotmarksize}}
\pgfpathlineto{\pgfqpointpolar{-54}{\pgfplotmarksize}}
\pgfpathlineto{\pgfqpointpolar{-90}{0.5\pgfplotmarksize}}
\pgfpathlineto{\pgfqpointpolar{234}{\pgfplotmarksize}}
\pgfpathlineto{\pgfqpointpolar{198}{0.5\pgfplotmarksize}}
\pgfpathlineto{\pgfqpointpolar{162}{\pgfplotmarksize}}
\pgfpathlineto{\pgfqpointpolar{134}{0.5\pgfplotmarksize}}
\pgfpathclose
\pgfusepathqstroke
}
\pgfdeclareplotmark{newstar*} {
\pgfpathmoveto{\pgfqpoint{0pt}{\pgfplotmarksize}}
\pgfpathlineto{\pgfqpointpolar{44}{0.5\pgfplotmarksize}}
\pgfpathlineto{\pgfqpointpolar{18}{\pgfplotmarksize}}
\pgfpathlineto{\pgfqpointpolar{-20}{0.5\pgfplotmarksize}}
\pgfpathlineto{\pgfqpointpolar{-54}{\pgfplotmarksize}}
\pgfpathlineto{\pgfqpointpolar{-90}{0.5\pgfplotmarksize}}
\pgfpathlineto{\pgfqpointpolar{234}{\pgfplotmarksize}}
\pgfpathlineto{\pgfqpointpolar{198}{0.5\pgfplotmarksize}}
\pgfpathlineto{\pgfqpointpolar{162}{\pgfplotmarksize}}
\pgfpathlineto{\pgfqpointpolar{134}{0.5\pgfplotmarksize}}
\pgfpathclose
\pgfusepathqfillstroke
}
\definecolor{c}{rgb}{1,1,1};
\draw [color=c, fill=c] (0,0) rectangle (10,6.80516);
\draw [color=c, fill=c] (1,0.680516) rectangle (9.95,6.73711);
\definecolor{c}{rgb}{0,0,0};
\draw [c] (1,0.680516) -- (1,6.73711) -- (9.95,6.73711) -- (9.95,0.680516) -- (1,0.680516);
\definecolor{c}{rgb}{1,1,1};
\draw [color=c, fill=c] (1,0.680516) rectangle (9.95,6.73711);
\definecolor{c}{rgb}{0,0,0};
\draw [c] (1,0.680516) -- (1,6.73711) -- (9.95,6.73711) -- (9.95,0.680516) -- (1,0.680516);
\colorlet{c}{natcomp!70};
\draw [c] (1.34131,1.78352) -- (1.34131,2.51541);
\draw [c] (1.34131,2.51541) -- (1.34131,2.82687);
\draw [c] (1.30339,2.51541) -- (1.34131,2.51541);
\draw [c] (1.34131,2.51541) -- (1.37924,2.51541);
\definecolor{c}{rgb}{0,0,0};
\colorlet{c}{natcomp!70};
\draw [c] (1.41716,1.62511) -- (1.41716,2.33052);
\draw [c] (1.41716,2.33052) -- (1.41716,2.63755);
\draw [c] (1.37924,2.33052) -- (1.41716,2.33052);
\draw [c] (1.41716,2.33052) -- (1.45508,2.33052);
\definecolor{c}{rgb}{0,0,0};
\colorlet{c}{natcomp!70};
\draw [c] (1.56886,1.90599) -- (1.56886,2.61743);
\draw [c] (1.56886,2.61743) -- (1.56886,2.92549);
\draw [c] (1.53093,2.61743) -- (1.56886,2.61743);
\draw [c] (1.56886,2.61743) -- (1.60678,2.61743);
\definecolor{c}{rgb}{0,0,0};
\colorlet{c}{natcomp!70};
\draw [c] (1.6447,2.52941) -- (1.6447,2.8612);
\draw [c] (1.6447,2.8612) -- (1.6447,3.07008);
\draw [c] (1.60678,2.8612) -- (1.6447,2.8612);
\draw [c] (1.6447,2.8612) -- (1.68263,2.8612);
\definecolor{c}{rgb}{0,0,0};
\colorlet{c}{natcomp!70};
\draw [c] (1.72055,3.76948) -- (1.72055,3.89028);
\draw [c] (1.72055,3.89028) -- (1.72055,3.99002);
\draw [c] (1.68263,3.89028) -- (1.72055,3.89028);
\draw [c] (1.72055,3.89028) -- (1.75847,3.89028);
\definecolor{c}{rgb}{0,0,0};
\colorlet{c}{natcomp!70};
\draw [c] (1.7964,5.80305) -- (1.7964,5.82466);
\draw [c] (1.7964,5.82466) -- (1.7964,5.84548);
\draw [c] (1.75847,5.82466) -- (1.7964,5.82466);
\draw [c] (1.7964,5.82466) -- (1.83432,5.82466);
\definecolor{c}{rgb}{0,0,0};
\colorlet{c}{natcomp!70};
\draw [c] (1.87225,6.24998) -- (1.87225,6.26438);
\draw [c] (1.87225,6.26438) -- (1.87225,6.27841);
\draw [c] (1.83432,6.26438) -- (1.87225,6.26438);
\draw [c] (1.87225,6.26438) -- (1.91017,6.26438);
\definecolor{c}{rgb}{0,0,0};
\colorlet{c}{natcomp!70};
\draw [c] (1.94809,6.34388) -- (1.94809,6.35726);
\draw [c] (1.94809,6.35726) -- (1.94809,6.37035);
\draw [c] (1.91017,6.35726) -- (1.94809,6.35726);
\draw [c] (1.94809,6.35726) -- (1.98602,6.35726);
\definecolor{c}{rgb}{0,0,0};
\colorlet{c}{natcomp!70};
\draw [c] (2.02394,6.30879) -- (2.02394,6.32269);
\draw [c] (2.02394,6.32269) -- (2.02394,6.33626);
\draw [c] (1.98602,6.32269) -- (2.02394,6.32269);
\draw [c] (2.02394,6.32269) -- (2.06186,6.32269);
\definecolor{c}{rgb}{0,0,0};
\colorlet{c}{natcomp!70};
\draw [c] (2.09979,6.25019) -- (2.09979,6.26512);
\draw [c] (2.09979,6.26512) -- (2.09979,6.27967);
\draw [c] (2.06186,6.26512) -- (2.09979,6.26512);
\draw [c] (2.09979,6.26512) -- (2.13771,6.26512);
\definecolor{c}{rgb}{0,0,0};
\colorlet{c}{natcomp!70};
\draw [c] (2.17564,6.16207) -- (2.17564,6.17792);
\draw [c] (2.17564,6.17792) -- (2.17564,6.19334);
\draw [c] (2.13771,6.17792) -- (2.17564,6.17792);
\draw [c] (2.17564,6.17792) -- (2.21356,6.17792);
\definecolor{c}{rgb}{0,0,0};
\colorlet{c}{natcomp!70};
\draw [c] (2.25148,6.07193) -- (2.25148,6.08917);
\draw [c] (2.25148,6.08917) -- (2.25148,6.10592);
\draw [c] (2.21356,6.08917) -- (2.25148,6.08917);
\draw [c] (2.25148,6.08917) -- (2.28941,6.08917);
\definecolor{c}{rgb}{0,0,0};
\colorlet{c}{natcomp!70};
\draw [c] (2.32733,5.94806) -- (2.32733,5.96764);
\draw [c] (2.32733,5.96764) -- (2.32733,5.98657);
\draw [c] (2.28941,5.96764) -- (2.32733,5.96764);
\draw [c] (2.32733,5.96764) -- (2.36525,5.96764);
\definecolor{c}{rgb}{0,0,0};
\colorlet{c}{natcomp!70};
\draw [c] (2.40318,5.82449) -- (2.40318,5.846);
\draw [c] (2.40318,5.846) -- (2.40318,5.86674);
\draw [c] (2.36525,5.846) -- (2.40318,5.846);
\draw [c] (2.40318,5.846) -- (2.4411,5.846);
\definecolor{c}{rgb}{0,0,0};
\colorlet{c}{natcomp!70};
\draw [c] (2.47903,5.71169) -- (2.47903,5.73544);
\draw [c] (2.47903,5.73544) -- (2.47903,5.75825);
\draw [c] (2.4411,5.73544) -- (2.47903,5.73544);
\draw [c] (2.47903,5.73544) -- (2.51695,5.73544);
\definecolor{c}{rgb}{0,0,0};
\colorlet{c}{natcomp!70};
\draw [c] (2.55487,5.64532) -- (2.55487,5.67137);
\draw [c] (2.55487,5.67137) -- (2.55487,5.69629);
\draw [c] (2.51695,5.67137) -- (2.55487,5.67137);
\draw [c] (2.55487,5.67137) -- (2.5928,5.67137);
\definecolor{c}{rgb}{0,0,0};
\colorlet{c}{natcomp!70};
\draw [c] (2.63072,5.52352) -- (2.63072,5.55176);
\draw [c] (2.63072,5.55176) -- (2.63072,5.57869);
\draw [c] (2.5928,5.55176) -- (2.63072,5.55176);
\draw [c] (2.63072,5.55176) -- (2.66864,5.55176);
\definecolor{c}{rgb}{0,0,0};
\colorlet{c}{natcomp!70};
\draw [c] (2.70657,5.50217) -- (2.70657,5.53122);
\draw [c] (2.70657,5.53122) -- (2.70657,5.55886);
\draw [c] (2.66864,5.53122) -- (2.70657,5.53122);
\draw [c] (2.70657,5.53122) -- (2.74449,5.53122);
\definecolor{c}{rgb}{0,0,0};
\colorlet{c}{natcomp!70};
\draw [c] (2.78242,5.40445) -- (2.78242,5.43645);
\draw [c] (2.78242,5.43645) -- (2.78242,5.46676);
\draw [c] (2.74449,5.43645) -- (2.78242,5.43645);
\draw [c] (2.78242,5.43645) -- (2.82034,5.43645);
\definecolor{c}{rgb}{0,0,0};
\colorlet{c}{natcomp!70};
\draw [c] (2.85826,5.28652) -- (2.85826,5.32084);
\draw [c] (2.85826,5.32084) -- (2.85826,5.35322);
\draw [c] (2.82034,5.32084) -- (2.85826,5.32084);
\draw [c] (2.85826,5.32084) -- (2.89619,5.32084);
\definecolor{c}{rgb}{0,0,0};
\colorlet{c}{natcomp!70};
\draw [c] (2.93411,5.09026) -- (2.93411,5.13066);
\draw [c] (2.93411,5.13066) -- (2.93411,5.16841);
\draw [c] (2.89619,5.13066) -- (2.93411,5.13066);
\draw [c] (2.93411,5.13066) -- (2.97203,5.13066);
\definecolor{c}{rgb}{0,0,0};
\colorlet{c}{natcomp!70};
\draw [c] (3.00996,5.07889) -- (3.00996,5.12198);
\draw [c] (3.00996,5.12198) -- (3.00996,5.16205);
\draw [c] (2.97203,5.12198) -- (3.00996,5.12198);
\draw [c] (3.00996,5.12198) -- (3.04788,5.12198);
\definecolor{c}{rgb}{0,0,0};
\colorlet{c}{natcomp!70};
\draw [c] (3.08581,4.99109) -- (3.08581,5.03784);
\draw [c] (3.08581,5.03784) -- (3.08581,5.08107);
\draw [c] (3.04788,5.03784) -- (3.08581,5.03784);
\draw [c] (3.08581,5.03784) -- (3.12373,5.03784);
\definecolor{c}{rgb}{0,0,0};
\colorlet{c}{natcomp!70};
\draw [c] (3.16165,4.99427) -- (3.16165,5.04101);
\draw [c] (3.16165,5.04101) -- (3.16165,5.08423);
\draw [c] (3.12373,5.04101) -- (3.16165,5.04101);
\draw [c] (3.16165,5.04101) -- (3.19958,5.04101);
\definecolor{c}{rgb}{0,0,0};
\colorlet{c}{natcomp!70};
\draw [c] (3.2375,4.84683) -- (3.2375,4.89782);
\draw [c] (3.2375,4.89782) -- (3.2375,4.94464);
\draw [c] (3.19958,4.89782) -- (3.2375,4.89782);
\draw [c] (3.2375,4.89782) -- (3.27542,4.89782);
\definecolor{c}{rgb}{0,0,0};
\colorlet{c}{natcomp!70};
\draw [c] (3.31335,4.77703) -- (3.31335,4.8307);
\draw [c] (3.31335,4.8307) -- (3.31335,4.87977);
\draw [c] (3.27542,4.8307) -- (3.31335,4.8307);
\draw [c] (3.31335,4.8307) -- (3.35127,4.8307);
\definecolor{c}{rgb}{0,0,0};
\colorlet{c}{natcomp!70};
\draw [c] (3.38919,4.63709) -- (3.38919,4.69689);
\draw [c] (3.38919,4.69689) -- (3.38919,4.75105);
\draw [c] (3.35127,4.69689) -- (3.38919,4.69689);
\draw [c] (3.38919,4.69689) -- (3.42712,4.69689);
\definecolor{c}{rgb}{0,0,0};
\colorlet{c}{natcomp!70};
\draw [c] (3.46504,4.62899) -- (3.46504,4.69223);
\draw [c] (3.46504,4.69223) -- (3.46504,4.74918);
\draw [c] (3.42712,4.69223) -- (3.46504,4.69223);
\draw [c] (3.46504,4.69223) -- (3.50297,4.69223);
\definecolor{c}{rgb}{0,0,0};
\colorlet{c}{natcomp!70};
\draw [c] (3.54089,4.5342) -- (3.54089,4.60227);
\draw [c] (3.54089,4.60227) -- (3.54089,4.66313);
\draw [c] (3.50297,4.60227) -- (3.54089,4.60227);
\draw [c] (3.54089,4.60227) -- (3.57881,4.60227);
\definecolor{c}{rgb}{0,0,0};
\colorlet{c}{natcomp!70};
\draw [c] (3.61674,4.40663) -- (3.61674,4.47821);
\draw [c] (3.61674,4.47821) -- (3.61674,4.54185);
\draw [c] (3.57881,4.47821) -- (3.61674,4.47821);
\draw [c] (3.61674,4.47821) -- (3.65466,4.47821);
\definecolor{c}{rgb}{0,0,0};
\colorlet{c}{natcomp!70};
\draw [c] (3.69258,4.3837) -- (3.69258,4.46006);
\draw [c] (3.69258,4.46006) -- (3.69258,4.52744);
\draw [c] (3.65466,4.46006) -- (3.69258,4.46006);
\draw [c] (3.69258,4.46006) -- (3.73051,4.46006);
\definecolor{c}{rgb}{0,0,0};
\colorlet{c}{natcomp!70};
\draw [c] (3.76843,4.26121) -- (3.76843,4.35345);
\draw [c] (3.76843,4.35345) -- (3.76843,4.43289);
\draw [c] (3.73051,4.35345) -- (3.76843,4.35345);
\draw [c] (3.76843,4.35345) -- (3.80636,4.35345);
\definecolor{c}{rgb}{0,0,0};
\colorlet{c}{natcomp!70};
\draw [c] (3.84428,4.14632) -- (3.84428,4.24826);
\draw [c] (3.84428,4.24826) -- (3.84428,4.3348);
\draw [c] (3.80636,4.24826) -- (3.84428,4.24826);
\draw [c] (3.84428,4.24826) -- (3.8822,4.24826);
\definecolor{c}{rgb}{0,0,0};
\colorlet{c}{natcomp!70};
\draw [c] (3.92013,4.18657) -- (3.92013,4.28491);
\draw [c] (3.92013,4.28491) -- (3.92013,4.36884);
\draw [c] (3.8822,4.28491) -- (3.92013,4.28491);
\draw [c] (3.92013,4.28491) -- (3.95805,4.28491);
\definecolor{c}{rgb}{0,0,0};
\colorlet{c}{natcomp!70};
\draw [c] (3.99597,4.31132) -- (3.99597,4.39416);
\draw [c] (3.99597,4.39416) -- (3.99597,4.46653);
\draw [c] (3.95805,4.39416) -- (3.99597,4.39416);
\draw [c] (3.99597,4.39416) -- (4.0339,4.39416);
\definecolor{c}{rgb}{0,0,0};
\colorlet{c}{natcomp!70};
\draw [c] (4.07182,3.79936) -- (4.07182,3.92106);
\draw [c] (4.07182,3.92106) -- (4.07182,4.02142);
\draw [c] (4.0339,3.92106) -- (4.07182,3.92106);
\draw [c] (4.07182,3.92106) -- (4.10975,3.92106);
\definecolor{c}{rgb}{0,0,0};
\colorlet{c}{natcomp!70};
\draw [c] (4.14767,4.04423) -- (4.14767,4.15122);
\draw [c] (4.14767,4.15122) -- (4.14767,4.24136);
\draw [c] (4.10975,4.15122) -- (4.14767,4.15122);
\draw [c] (4.14767,4.15122) -- (4.18559,4.15122);
\definecolor{c}{rgb}{0,0,0};
\colorlet{c}{natcomp!70};
\draw [c] (4.22352,4.17213) -- (4.22352,4.27444);
\draw [c] (4.22352,4.27444) -- (4.22352,4.36124);
\draw [c] (4.18559,4.27444) -- (4.22352,4.27444);
\draw [c] (4.22352,4.27444) -- (4.26144,4.27444);
\definecolor{c}{rgb}{0,0,0};
\colorlet{c}{natcomp!70};
\draw [c] (4.29936,3.93105) -- (4.29936,3.95136);
\draw [c] (4.29936,3.95136) -- (4.29936,3.97098);
\draw [c] (4.26144,3.95136) -- (4.29936,3.95136);
\draw [c] (4.29936,3.95136) -- (4.33729,3.95136);
\definecolor{c}{rgb}{0,0,0};
\colorlet{c}{natcomp!70};
\draw [c] (4.37521,3.86494) -- (4.37521,3.89839);
\draw [c] (4.37521,3.89839) -- (4.37521,3.92999);
\draw [c] (4.33729,3.89839) -- (4.37521,3.89839);
\draw [c] (4.37521,3.89839) -- (4.41314,3.89839);
\definecolor{c}{rgb}{0,0,0};
\colorlet{c}{natcomp!70};
\draw [c] (4.45106,3.84202) -- (4.45106,3.86382);
\draw [c] (4.45106,3.86382) -- (4.45106,3.88482);
\draw [c] (4.41314,3.86382) -- (4.45106,3.86382);
\draw [c] (4.45106,3.86382) -- (4.48898,3.86382);
\definecolor{c}{rgb}{0,0,0};
\colorlet{c}{natcomp!70};
\draw [c] (4.52691,3.84293) -- (4.52691,3.8867);
\draw [c] (4.52691,3.8867) -- (4.52691,3.92736);
\draw [c] (4.48898,3.8867) -- (4.52691,3.8867);
\draw [c] (4.52691,3.8867) -- (4.56483,3.8867);
\definecolor{c}{rgb}{0,0,0};
\colorlet{c}{natcomp!70};
\draw [c] (4.60275,3.75875) -- (4.60275,3.80805);
\draw [c] (4.60275,3.80805) -- (4.60275,3.85345);
\draw [c] (4.56483,3.80805) -- (4.60275,3.80805);
\draw [c] (4.60275,3.80805) -- (4.64068,3.80805);
\definecolor{c}{rgb}{0,0,0};
\colorlet{c}{natcomp!70};
\draw [c] (4.6786,3.66225) -- (4.6786,3.71021);
\draw [c] (4.6786,3.71021) -- (4.6786,3.75446);
\draw [c] (4.64068,3.71021) -- (4.6786,3.71021);
\draw [c] (4.6786,3.71021) -- (4.71653,3.71021);
\definecolor{c}{rgb}{0,0,0};
\colorlet{c}{natcomp!70};
\draw [c] (4.75445,3.57524) -- (4.75445,3.62382);
\draw [c] (4.75445,3.62382) -- (4.75445,3.6686);
\draw [c] (4.71653,3.62382) -- (4.75445,3.62382);
\draw [c] (4.75445,3.62382) -- (4.79237,3.62382);
\definecolor{c}{rgb}{0,0,0};
\colorlet{c}{natcomp!70};
\draw [c] (4.8303,3.55668) -- (4.8303,3.60867);
\draw [c] (4.8303,3.60867) -- (4.8303,3.65634);
\draw [c] (4.79237,3.60867) -- (4.8303,3.60867);
\draw [c] (4.8303,3.60867) -- (4.86822,3.60867);
\definecolor{c}{rgb}{0,0,0};
\colorlet{c}{natcomp!70};
\draw [c] (4.90614,3.51853) -- (4.90614,3.55247);
\draw [c] (4.90614,3.55247) -- (4.90614,3.58452);
\draw [c] (4.86822,3.55247) -- (4.90614,3.55247);
\draw [c] (4.90614,3.55247) -- (4.94407,3.55247);
\definecolor{c}{rgb}{0,0,0};
\colorlet{c}{natcomp!70};
\draw [c] (4.98199,3.52238) -- (4.98199,3.58359);
\draw [c] (4.98199,3.58359) -- (4.98199,3.6389);
\draw [c] (4.94407,3.58359) -- (4.98199,3.58359);
\draw [c] (4.98199,3.58359) -- (5.01992,3.58359);
\definecolor{c}{rgb}{0,0,0};
\colorlet{c}{natcomp!70};
\draw [c] (5.05784,3.43016) -- (5.05784,3.48934);
\draw [c] (5.05784,3.48934) -- (5.05784,3.54298);
\draw [c] (5.01992,3.48934) -- (5.05784,3.48934);
\draw [c] (5.05784,3.48934) -- (5.09576,3.48934);
\definecolor{c}{rgb}{0,0,0};
\colorlet{c}{natcomp!70};
\draw [c] (5.13369,3.32771) -- (5.13369,3.37927);
\draw [c] (5.13369,3.37927) -- (5.13369,3.42658);
\draw [c] (5.09576,3.37927) -- (5.13369,3.37927);
\draw [c] (5.13369,3.37927) -- (5.17161,3.37927);
\definecolor{c}{rgb}{0,0,0};
\colorlet{c}{natcomp!70};
\draw [c] (5.20953,3.23557) -- (5.20953,3.27213);
\draw [c] (5.20953,3.27213) -- (5.20953,3.3065);
\draw [c] (5.17161,3.27213) -- (5.20953,3.27213);
\draw [c] (5.20953,3.27213) -- (5.24746,3.27213);
\definecolor{c}{rgb}{0,0,0};
\colorlet{c}{natcomp!70};
\draw [c] (5.28538,3.29755) -- (5.28538,3.35759);
\draw [c] (5.28538,3.35759) -- (5.28538,3.41194);
\draw [c] (5.24746,3.35759) -- (5.28538,3.35759);
\draw [c] (5.28538,3.35759) -- (5.32331,3.35759);
\definecolor{c}{rgb}{0,0,0};
\colorlet{c}{natcomp!70};
\draw [c] (5.36123,3.1904) -- (5.36123,3.25923);
\draw [c] (5.36123,3.25923) -- (5.36123,3.32068);
\draw [c] (5.32331,3.25923) -- (5.36123,3.25923);
\draw [c] (5.36123,3.25923) -- (5.39915,3.25923);
\definecolor{c}{rgb}{0,0,0};
\colorlet{c}{natcomp!70};
\draw [c] (5.43708,3.14764) -- (5.43708,3.18783);
\draw [c] (5.43708,3.18783) -- (5.43708,3.22538);
\draw [c] (5.39915,3.18783) -- (5.43708,3.18783);
\draw [c] (5.43708,3.18783) -- (5.475,3.18783);
\definecolor{c}{rgb}{0,0,0};
\colorlet{c}{natcomp!70};
\draw [c] (5.51292,3.10655) -- (5.51292,3.14909);
\draw [c] (5.51292,3.14909) -- (5.51292,3.18868);
\draw [c] (5.475,3.14909) -- (5.51292,3.14909);
\draw [c] (5.51292,3.14909) -- (5.55085,3.14909);
\definecolor{c}{rgb}{0,0,0};
\colorlet{c}{natcomp!70};
\draw [c] (5.58877,3.12957) -- (5.58877,3.17285);
\draw [c] (5.58877,3.17285) -- (5.58877,3.21309);
\draw [c] (5.55085,3.17285) -- (5.58877,3.17285);
\draw [c] (5.58877,3.17285) -- (5.6267,3.17285);
\definecolor{c}{rgb}{0,0,0};
\colorlet{c}{natcomp!70};
\draw [c] (5.66462,2.98182) -- (5.66462,3.0269);
\draw [c] (5.66462,3.0269) -- (5.66462,3.0687);
\draw [c] (5.6267,3.0269) -- (5.66462,3.0269);
\draw [c] (5.66462,3.0269) -- (5.70254,3.0269);
\definecolor{c}{rgb}{0,0,0};
\colorlet{c}{natcomp!70};
\draw [c] (5.74047,2.95982) -- (5.74047,3.00795);
\draw [c] (5.74047,3.00795) -- (5.74047,3.05235);
\draw [c] (5.70254,3.00795) -- (5.74047,3.00795);
\draw [c] (5.74047,3.00795) -- (5.77839,3.00795);
\definecolor{c}{rgb}{0,0,0};
\colorlet{c}{natcomp!70};
\draw [c] (5.81631,2.9584) -- (5.81631,3.05897);
\draw [c] (5.81631,3.05897) -- (5.81631,3.14452);
\draw [c] (5.77839,3.05897) -- (5.81631,3.05897);
\draw [c] (5.81631,3.05897) -- (5.85424,3.05897);
\definecolor{c}{rgb}{0,0,0};
\colorlet{c}{natcomp!70};
\draw [c] (5.89216,3.10587) -- (5.89216,3.23102);
\draw [c] (5.89216,3.23102) -- (5.89216,3.3337);
\draw [c] (5.85424,3.23102) -- (5.89216,3.23102);
\draw [c] (5.89216,3.23102) -- (5.93008,3.23102);
\definecolor{c}{rgb}{0,0,0};
\colorlet{c}{natcomp!70};
\draw [c] (5.96801,2.82054) -- (5.96801,2.94787);
\draw [c] (5.96801,2.94787) -- (5.96801,3.05201);
\draw [c] (5.93008,2.94787) -- (5.96801,2.94787);
\draw [c] (5.96801,2.94787) -- (6.00593,2.94787);
\definecolor{c}{rgb}{0,0,0};
\colorlet{c}{natcomp!70};
\draw [c] (6.04386,2.72523) -- (6.04386,2.78281);
\draw [c] (6.04386,2.78281) -- (6.04386,2.83514);
\draw [c] (6.00593,2.78281) -- (6.04386,2.78281);
\draw [c] (6.04386,2.78281) -- (6.08178,2.78281);
\definecolor{c}{rgb}{0,0,0};
\colorlet{c}{natcomp!70};
\draw [c] (6.1197,2.78902) -- (6.1197,2.8457);
\draw [c] (6.1197,2.8457) -- (6.1197,2.89729);
\draw [c] (6.08178,2.8457) -- (6.1197,2.8457);
\draw [c] (6.1197,2.8457) -- (6.15763,2.8457);
\definecolor{c}{rgb}{0,0,0};
\colorlet{c}{natcomp!70};
\draw [c] (6.19555,2.76443) -- (6.19555,2.88354);
\draw [c] (6.19555,2.88354) -- (6.19555,2.98212);
\draw [c] (6.15763,2.88354) -- (6.19555,2.88354);
\draw [c] (6.19555,2.88354) -- (6.23347,2.88354);
\definecolor{c}{rgb}{0,0,0};
\colorlet{c}{natcomp!70};
\draw [c] (6.2714,2.71588) -- (6.2714,2.77508);
\draw [c] (6.2714,2.77508) -- (6.2714,2.82873);
\draw [c] (6.23347,2.77508) -- (6.2714,2.77508);
\draw [c] (6.2714,2.77508) -- (6.30932,2.77508);
\definecolor{c}{rgb}{0,0,0};
\colorlet{c}{natcomp!70};
\draw [c] (6.34725,2.55336) -- (6.34725,2.61892);
\draw [c] (6.34725,2.61892) -- (6.34725,2.67775);
\draw [c] (6.30932,2.61892) -- (6.34725,2.61892);
\draw [c] (6.34725,2.61892) -- (6.38517,2.61892);
\definecolor{c}{rgb}{0,0,0};
\colorlet{c}{natcomp!70};
\draw [c] (6.42309,2.54103) -- (6.42309,2.60764);
\draw [c] (6.42309,2.60764) -- (6.42309,2.66732);
\draw [c] (6.38517,2.60764) -- (6.42309,2.60764);
\draw [c] (6.42309,2.60764) -- (6.46102,2.60764);
\definecolor{c}{rgb}{0,0,0};
\colorlet{c}{natcomp!70};
\draw [c] (6.49894,2.57973) -- (6.49894,2.7434);
\draw [c] (6.49894,2.7434) -- (6.49894,2.87061);
\draw [c] (6.46102,2.7434) -- (6.49894,2.7434);
\draw [c] (6.49894,2.7434) -- (6.53686,2.7434);
\definecolor{c}{rgb}{0,0,0};
\colorlet{c}{natcomp!70};
\draw [c] (6.57479,2.28663) -- (6.57479,2.36458);
\draw [c] (6.57479,2.36458) -- (6.57479,2.4332);
\draw [c] (6.53686,2.36458) -- (6.57479,2.36458);
\draw [c] (6.57479,2.36458) -- (6.61271,2.36458);
\definecolor{c}{rgb}{0,0,0};
\colorlet{c}{natcomp!70};
\draw [c] (6.65064,2.40822) -- (6.65064,2.48613);
\draw [c] (6.65064,2.48613) -- (6.65064,2.55472);
\draw [c] (6.61271,2.48613) -- (6.65064,2.48613);
\draw [c] (6.65064,2.48613) -- (6.68856,2.48613);
\definecolor{c}{rgb}{0,0,0};
\colorlet{c}{natcomp!70};
\draw [c] (6.72648,2.47109) -- (6.72648,2.54861);
\draw [c] (6.72648,2.54861) -- (6.72648,2.6169);
\draw [c] (6.68856,2.54861) -- (6.72648,2.54861);
\draw [c] (6.72648,2.54861) -- (6.76441,2.54861);
\definecolor{c}{rgb}{0,0,0};
\colorlet{c}{natcomp!70};
\draw [c] (6.80233,2.34361) -- (6.80233,2.42923);
\draw [c] (6.80233,2.42923) -- (6.80233,2.50371);
\draw [c] (6.76441,2.42923) -- (6.80233,2.42923);
\draw [c] (6.80233,2.42923) -- (6.84025,2.42923);
\definecolor{c}{rgb}{0,0,0};
\colorlet{c}{natcomp!70};
\draw [c] (6.87818,2.39771) -- (6.87818,2.47467);
\draw [c] (6.87818,2.47467) -- (6.87818,2.54252);
\draw [c] (6.84025,2.47467) -- (6.87818,2.47467);
\draw [c] (6.87818,2.47467) -- (6.9161,2.47467);
\definecolor{c}{rgb}{0,0,0};
\colorlet{c}{natcomp!70};
\draw [c] (6.95403,2.13205) -- (6.95403,2.22216);
\draw [c] (6.95403,2.22216) -- (6.95403,2.30002);
\draw [c] (6.9161,2.22216) -- (6.95403,2.22216);
\draw [c] (6.95403,2.22216) -- (6.99195,2.22216);
\definecolor{c}{rgb}{0,0,0};
\colorlet{c}{natcomp!70};
\draw [c] (7.02987,2.25139) -- (7.02987,2.33724);
\draw [c] (7.02987,2.33724) -- (7.02987,2.41191);
\draw [c] (6.99195,2.33724) -- (7.02987,2.33724);
\draw [c] (7.02987,2.33724) -- (7.0678,2.33724);
\definecolor{c}{rgb}{0,0,0};
\colorlet{c}{natcomp!70};
\draw [c] (7.10572,2.32922) -- (7.10572,2.41443);
\draw [c] (7.10572,2.41443) -- (7.10572,2.48861);
\draw [c] (7.0678,2.41443) -- (7.10572,2.41443);
\draw [c] (7.10572,2.41443) -- (7.14364,2.41443);
\definecolor{c}{rgb}{0,0,0};
\colorlet{c}{natcomp!70};
\draw [c] (7.18157,2.12343) -- (7.18157,2.21978);
\draw [c] (7.18157,2.21978) -- (7.18157,2.30226);
\draw [c] (7.14364,2.21978) -- (7.18157,2.21978);
\draw [c] (7.18157,2.21978) -- (7.21949,2.21978);
\definecolor{c}{rgb}{0,0,0};
\colorlet{c}{natcomp!70};
\draw [c] (7.25742,2.03344) -- (7.25742,2.13853);
\draw [c] (7.25742,2.13853) -- (7.25742,2.22732);
\draw [c] (7.21949,2.13853) -- (7.25742,2.13853);
\draw [c] (7.25742,2.13853) -- (7.29534,2.13853);
\definecolor{c}{rgb}{0,0,0};
\colorlet{c}{natcomp!70};
\draw [c] (7.33326,2.02634) -- (7.33326,2.12717);
\draw [c] (7.33326,2.12717) -- (7.33326,2.2129);
\draw [c] (7.29534,2.12717) -- (7.33326,2.12717);
\draw [c] (7.33326,2.12717) -- (7.37119,2.12717);
\definecolor{c}{rgb}{0,0,0};
\colorlet{c}{natcomp!70};
\draw [c] (7.40911,1.83146) -- (7.40911,1.95577);
\draw [c] (7.40911,1.95577) -- (7.40911,2.05788);
\draw [c] (7.37119,1.95577) -- (7.40911,1.95577);
\draw [c] (7.40911,1.95577) -- (7.44703,1.95577);
\definecolor{c}{rgb}{0,0,0};
\colorlet{c}{natcomp!70};
\draw [c] (7.48496,2.08552) -- (7.48496,2.18317);
\draw [c] (7.48496,2.18317) -- (7.48496,2.26659);
\draw [c] (7.44703,2.18317) -- (7.48496,2.18317);
\draw [c] (7.48496,2.18317) -- (7.52288,2.18317);
\definecolor{c}{rgb}{0,0,0};
\colorlet{c}{natcomp!70};
\draw [c] (7.56081,2.1242) -- (7.56081,2.21961);
\draw [c] (7.56081,2.21961) -- (7.56081,2.3014);
\draw [c] (7.52288,2.21961) -- (7.56081,2.21961);
\draw [c] (7.56081,2.21961) -- (7.59873,2.21961);
\definecolor{c}{rgb}{0,0,0};
\colorlet{c}{natcomp!70};
\draw [c] (7.63665,1.9047) -- (7.63665,2.02183);
\draw [c] (7.63665,2.02183) -- (7.63665,2.11905);
\draw [c] (7.59873,2.02183) -- (7.63665,2.02183);
\draw [c] (7.63665,2.02183) -- (7.67458,2.02183);
\definecolor{c}{rgb}{0,0,0};
\colorlet{c}{natcomp!70};
\draw [c] (7.7125,1.94031) -- (7.7125,2.06341);
\draw [c] (7.7125,2.06341) -- (7.7125,2.16472);
\draw [c] (7.67458,2.06341) -- (7.7125,2.06341);
\draw [c] (7.7125,2.06341) -- (7.75042,2.06341);
\definecolor{c}{rgb}{0,0,0};
\colorlet{c}{natcomp!70};
\draw [c] (7.78835,1.72392) -- (7.78835,1.86154);
\draw [c] (7.78835,1.86154) -- (7.78835,1.97246);
\draw [c] (7.75042,1.86154) -- (7.78835,1.86154);
\draw [c] (7.78835,1.86154) -- (7.82627,1.86154);
\definecolor{c}{rgb}{0,0,0};
\colorlet{c}{natcomp!70};
\draw [c] (7.86419,1.73133) -- (7.86419,1.86386);
\draw [c] (7.86419,1.86386) -- (7.86419,1.97145);
\draw [c] (7.82627,1.86386) -- (7.86419,1.86386);
\draw [c] (7.86419,1.86386) -- (7.90212,1.86386);
\definecolor{c}{rgb}{0,0,0};
\colorlet{c}{natcomp!70};
\draw [c] (7.94004,1.51339) -- (7.94004,1.66499);
\draw [c] (7.94004,1.66499) -- (7.94004,1.78479);
\draw [c] (7.90212,1.66499) -- (7.94004,1.66499);
\draw [c] (7.94004,1.66499) -- (7.97797,1.66499);
\definecolor{c}{rgb}{0,0,0};
\colorlet{c}{natcomp!70};
\draw [c] (8.01589,1.80733) -- (8.01589,1.9271);
\draw [c] (8.01589,1.9271) -- (8.01589,2.02614);
\draw [c] (7.97797,1.9271) -- (8.01589,1.9271);
\draw [c] (8.01589,1.9271) -- (8.05381,1.9271);
\definecolor{c}{rgb}{0,0,0};
\colorlet{c}{natcomp!70};
\draw [c] (8.09174,1.74933) -- (8.09174,1.87737);
\draw [c] (8.09174,1.87737) -- (8.09174,1.982);
\draw [c] (8.05381,1.87737) -- (8.09174,1.87737);
\draw [c] (8.09174,1.87737) -- (8.12966,1.87737);
\definecolor{c}{rgb}{0,0,0};
\colorlet{c}{natcomp!70};
\draw [c] (8.16758,1.58389) -- (8.16758,1.7371);
\draw [c] (8.16758,1.7371) -- (8.16758,1.85791);
\draw [c] (8.12966,1.7371) -- (8.16758,1.7371);
\draw [c] (8.16758,1.7371) -- (8.20551,1.7371);
\definecolor{c}{rgb}{0,0,0};
\colorlet{c}{natcomp!70};
\draw [c] (8.24343,1.69622) -- (8.24343,1.83434);
\draw [c] (8.24343,1.83434) -- (8.24343,1.94559);
\draw [c] (8.20551,1.83434) -- (8.24343,1.83434);
\draw [c] (8.24343,1.83434) -- (8.28136,1.83434);
\definecolor{c}{rgb}{0,0,0};
\colorlet{c}{natcomp!70};
\draw [c] (8.31928,1.69369) -- (8.31928,1.82769);
\draw [c] (8.31928,1.82769) -- (8.31928,1.93625);
\draw [c] (8.28136,1.82769) -- (8.31928,1.82769);
\draw [c] (8.31928,1.82769) -- (8.3572,1.82769);
\definecolor{c}{rgb}{0,0,0};
\colorlet{c}{natcomp!70};
\draw [c] (8.39513,1.35148) -- (8.39513,1.52488);
\draw [c] (8.39513,1.52488) -- (8.39513,1.65787);
\draw [c] (8.3572,1.52488) -- (8.39513,1.52488);
\draw [c] (8.39513,1.52488) -- (8.43305,1.52488);
\definecolor{c}{rgb}{0,0,0};
\colorlet{c}{natcomp!70};
\draw [c] (8.47097,1.69673) -- (8.47097,1.81679);
\draw [c] (8.47097,1.81679) -- (8.47097,1.91602);
\draw [c] (8.43305,1.81679) -- (8.47097,1.81679);
\draw [c] (8.47097,1.81679) -- (8.5089,1.81679);
\definecolor{c}{rgb}{0,0,0};
\colorlet{c}{natcomp!70};
\draw [c] (8.54682,1.53076) -- (8.54682,1.65084);
\draw [c] (8.54682,1.65084) -- (8.54682,1.75008);
\draw [c] (8.5089,1.65084) -- (8.54682,1.65084);
\draw [c] (8.54682,1.65084) -- (8.58475,1.65084);
\definecolor{c}{rgb}{0,0,0};
\colorlet{c}{natcomp!70};
\draw [c] (8.62267,1.44173) -- (8.62267,1.5394);
\draw [c] (8.62267,1.5394) -- (8.62267,1.62283);
\draw [c] (8.58475,1.5394) -- (8.62267,1.5394);
\draw [c] (8.62267,1.5394) -- (8.66059,1.5394);
\definecolor{c}{rgb}{0,0,0};
\colorlet{c}{natcomp!70};
\draw [c] (8.69852,1.44392) -- (8.69852,1.46659);
\draw [c] (8.69852,1.46659) -- (8.69852,1.4884);
\draw [c] (8.66059,1.46659) -- (8.69852,1.46659);
\draw [c] (8.69852,1.46659) -- (8.73644,1.46659);
\definecolor{c}{rgb}{0,0,0};
\colorlet{c}{natcomp!70};
\draw [c] (8.77436,1.45513) -- (8.77436,1.47712);
\draw [c] (8.77436,1.47712) -- (8.77436,1.4983);
\draw [c] (8.73644,1.47712) -- (8.77436,1.47712);
\draw [c] (8.77436,1.47712) -- (8.81229,1.47712);
\definecolor{c}{rgb}{0,0,0};
\colorlet{c}{natcomp!70};
\draw [c] (8.85021,1.45819) -- (8.85021,1.48066);
\draw [c] (8.85021,1.48066) -- (8.85021,1.50228);
\draw [c] (8.81229,1.48066) -- (8.85021,1.48066);
\draw [c] (8.85021,1.48066) -- (8.88814,1.48066);
\definecolor{c}{rgb}{0,0,0};
\colorlet{c}{natcomp!70};
\draw [c] (8.92606,1.43318) -- (8.92606,1.45542);
\draw [c] (8.92606,1.45542) -- (8.92606,1.47683);
\draw [c] (8.88814,1.45542) -- (8.92606,1.45542);
\draw [c] (8.92606,1.45542) -- (8.96398,1.45542);
\definecolor{c}{rgb}{0,0,0};
\colorlet{c}{natcomp!70};
\draw [c] (9.00191,1.40407) -- (9.00191,1.42727);
\draw [c] (9.00191,1.42727) -- (9.00191,1.44958);
\draw [c] (8.96398,1.42727) -- (9.00191,1.42727);
\draw [c] (9.00191,1.42727) -- (9.03983,1.42727);
\definecolor{c}{rgb}{0,0,0};
\colorlet{c}{natcomp!70};
\draw [c] (9.07775,1.34916) -- (9.07775,1.37304);
\draw [c] (9.07775,1.37304) -- (9.07775,1.39597);
\draw [c] (9.03983,1.37304) -- (9.07775,1.37304);
\draw [c] (9.07775,1.37304) -- (9.11568,1.37304);
\definecolor{c}{rgb}{0,0,0};
\colorlet{c}{natcomp!70};
\draw [c] (9.1536,1.349) -- (9.1536,1.37351);
\draw [c] (9.1536,1.37351) -- (9.1536,1.39702);
\draw [c] (9.11568,1.37351) -- (9.1536,1.37351);
\draw [c] (9.1536,1.37351) -- (9.19153,1.37351);
\definecolor{c}{rgb}{0,0,0};
\colorlet{c}{natcomp!70};
\draw [c] (9.22945,1.28331) -- (9.22945,1.30947);
\draw [c] (9.22945,1.30947) -- (9.22945,1.3345);
\draw [c] (9.19153,1.30947) -- (9.22945,1.30947);
\draw [c] (9.22945,1.30947) -- (9.26737,1.30947);
\definecolor{c}{rgb}{0,0,0};
\colorlet{c}{natcomp!70};
\draw [c] (9.3053,1.32295) -- (9.3053,1.34851);
\draw [c] (9.3053,1.34851) -- (9.3053,1.37299);
\draw [c] (9.26737,1.34851) -- (9.3053,1.34851);
\draw [c] (9.3053,1.34851) -- (9.34322,1.34851);
\definecolor{c}{rgb}{0,0,0};
\colorlet{c}{natcomp!70};
\draw [c] (9.38114,1.21349) -- (9.38114,1.24116);
\draw [c] (9.38114,1.24116) -- (9.38114,1.26756);
\draw [c] (9.34322,1.24116) -- (9.38114,1.24116);
\draw [c] (9.38114,1.24116) -- (9.41907,1.24116);
\definecolor{c}{rgb}{0,0,0};
\colorlet{c}{natcomp!70};
\draw [c] (9.45699,1.22504) -- (9.45699,1.25181);
\draw [c] (9.45699,1.25181) -- (9.45699,1.27738);
\draw [c] (9.41907,1.25181) -- (9.45699,1.25181);
\draw [c] (9.45699,1.25181) -- (9.49492,1.25181);
\definecolor{c}{rgb}{0,0,0};
\colorlet{c}{natcomp!70};
\draw [c] (9.53284,1.2621) -- (9.53284,1.28829);
\draw [c] (9.53284,1.28829) -- (9.53284,1.31333);
\draw [c] (9.49492,1.28829) -- (9.53284,1.28829);
\draw [c] (9.53284,1.28829) -- (9.57076,1.28829);
\definecolor{c}{rgb}{0,0,0};
\colorlet{c}{natcomp!70};
\draw [c] (9.60869,1.1794) -- (9.60869,1.20761);
\draw [c] (9.60869,1.20761) -- (9.60869,1.23449);
\draw [c] (9.57076,1.20761) -- (9.60869,1.20761);
\draw [c] (9.60869,1.20761) -- (9.64661,1.20761);
\definecolor{c}{rgb}{0,0,0};
\colorlet{c}{natcomp!70};
\draw [c] (9.68453,1.17427) -- (9.68453,1.20235);
\draw [c] (9.68453,1.20235) -- (9.68453,1.22911);
\draw [c] (9.64661,1.20235) -- (9.68453,1.20235);
\draw [c] (9.68453,1.20235) -- (9.72246,1.20235);
\definecolor{c}{rgb}{0,0,0};
\colorlet{c}{natcomp!70};
\draw [c] (9.76038,1.13334) -- (9.76038,1.1631);
\draw [c] (9.76038,1.1631) -- (9.76038,1.19139);
\draw [c] (9.72246,1.1631) -- (9.76038,1.1631);
\draw [c] (9.76038,1.1631) -- (9.79831,1.1631);
\definecolor{c}{rgb}{0,0,0};
\colorlet{c}{natcomp!70};
\draw [c] (9.83623,1.0783) -- (9.83623,1.10911);
\draw [c] (9.83623,1.10911) -- (9.83623,1.13834);
\draw [c] (9.79831,1.10911) -- (9.83623,1.10911);
\draw [c] (9.83623,1.10911) -- (9.87415,1.10911);
\definecolor{c}{rgb}{0,0,0};
\colorlet{c}{natcomp!70};
\draw [c] (9.91208,1.11033) -- (9.91208,1.14063);
\draw [c] (9.91208,1.14063) -- (9.91208,1.16941);
\draw [c] (9.87415,1.14063) -- (9.91208,1.14063);
\draw [c] (9.91208,1.14063) -- (9.95,1.14063);
\definecolor{c}{rgb}{0,0,0};
\draw [c] (1,0.680516) -- (9.95,0.680516);
\draw [anchor= east] (9.95,0.108883) node[color=c, rotate=0]{$M_{\gamma\gamma}\text{ [GeV]}$};
\draw [c] (1,0.863234) -- (1,0.680516);
\draw [c] (1.37924,0.771875) -- (1.37924,0.680516);
\draw [c] (1.75847,0.771875) -- (1.75847,0.680516);
\draw [c] (2.13771,0.771875) -- (2.13771,0.680516);
\draw [c] (2.51695,0.863234) -- (2.51695,0.680516);
\draw [c] (2.89619,0.771875) -- (2.89619,0.680516);
\draw [c] (3.27542,0.771875) -- (3.27542,0.680516);
\draw [c] (3.65466,0.771875) -- (3.65466,0.680516);
\draw [c] (4.0339,0.863234) -- (4.0339,0.680516);
\draw [c] (4.41314,0.771875) -- (4.41314,0.680516);
\draw [c] (4.79237,0.771875) -- (4.79237,0.680516);
\draw [c] (5.17161,0.771875) -- (5.17161,0.680516);
\draw [c] (5.55085,0.863234) -- (5.55085,0.680516);
\draw [c] (5.93008,0.771875) -- (5.93008,0.680516);
\draw [c] (6.30932,0.771875) -- (6.30932,0.680516);
\draw [c] (6.68856,0.771875) -- (6.68856,0.680516);
\draw [c] (7.0678,0.863234) -- (7.0678,0.680516);
\draw [c] (7.44703,0.771875) -- (7.44703,0.680516);
\draw [c] (7.82627,0.771875) -- (7.82627,0.680516);
\draw [c] (8.20551,0.771875) -- (8.20551,0.680516);
\draw [c] (8.58475,0.863234) -- (8.58475,0.680516);
\draw [c] (8.58475,0.863234) -- (8.58475,0.680516);
\draw [c] (8.96398,0.771875) -- (8.96398,0.680516);
\draw [c] (9.34322,0.771875) -- (9.34322,0.680516);
\draw [c] (9.72246,0.771875) -- (9.72246,0.680516);
\draw [anchor=base] (1,0.353868) node[color=c, rotate=0]{0};
\draw [anchor=base] (2.51695,0.353868) node[color=c, rotate=0]{200};
\draw [anchor=base] (4.0339,0.353868) node[color=c, rotate=0]{400};
\draw [anchor=base] (5.55085,0.353868) node[color=c, rotate=0]{600};
\draw [anchor=base] (7.0678,0.353868) node[color=c, rotate=0]{800};
\draw [anchor=base] (8.58475,0.353868) node[color=c, rotate=0]{1000};
\draw [c] (1,0.680516) -- (1,6.73711);
\draw [anchor= east] (-0.12,6.73711) node[color=c, rotate=90]{Number of events};
\draw [c] (1.267,0.688075) -- (1,0.688075);
\draw [anchor= east] (0.922,0.688075) node[color=c, rotate=0]{$10^{-1}$};
\draw [c] (1.1335,1.08586) -- (1,1.08586);
\draw [c] (1.1335,1.31855) -- (1,1.31855);
\draw [c] (1.1335,1.48365) -- (1,1.48365);
\draw [c] (1.1335,1.61171) -- (1,1.61171);
\draw [c] (1.1335,1.71634) -- (1,1.71634);
\draw [c] (1.1335,1.80481) -- (1,1.80481);
\draw [c] (1.1335,1.88144) -- (1,1.88144);
\draw [c] (1.1335,1.94903) -- (1,1.94903);
\draw [c] (1.267,2.0095) -- (1,2.0095);
\draw [anchor= east] (0.922,2.0095) node[color=c, rotate=0]{1};
\draw [c] (1.1335,2.40729) -- (1,2.40729);
\draw [c] (1.1335,2.63998) -- (1,2.63998);
\draw [c] (1.1335,2.80507) -- (1,2.80507);
\draw [c] (1.1335,2.93313) -- (1,2.93313);
\draw [c] (1.1335,3.03776) -- (1,3.03776);
\draw [c] (1.1335,3.12623) -- (1,3.12623);
\draw [c] (1.1335,3.20286) -- (1,3.20286);
\draw [c] (1.1335,3.27046) -- (1,3.27046);
\draw [c] (1.267,3.33092) -- (1,3.33092);
\draw [anchor= east] (0.922,3.33092) node[color=c, rotate=0]{10};
\draw [c] (1.1335,3.72871) -- (1,3.72871);
\draw [c] (1.1335,3.9614) -- (1,3.9614);
\draw [c] (1.1335,4.1265) -- (1,4.1265);
\draw [c] (1.1335,4.25456) -- (1,4.25456);
\draw [c] (1.1335,4.35919) -- (1,4.35919);
\draw [c] (1.1335,4.44765) -- (1,4.44765);
\draw [c] (1.1335,4.52428) -- (1,4.52428);
\draw [c] (1.1335,4.59188) -- (1,4.59188);
\draw [c] (1.267,4.65234) -- (1,4.65234);
\draw [anchor= east] (0.922,4.65234) node[color=c, rotate=0]{$10^{2}$};
\draw [c] (1.1335,5.05013) -- (1,5.05013);
\draw [c] (1.1335,5.28282) -- (1,5.28282);
\draw [c] (1.1335,5.44792) -- (1,5.44792);
\draw [c] (1.1335,5.57598) -- (1,5.57598);
\draw [c] (1.1335,5.68061) -- (1,5.68061);
\draw [c] (1.1335,5.76908) -- (1,5.76908);
\draw [c] (1.1335,5.84571) -- (1,5.84571);
\draw [c] (1.1335,5.9133) -- (1,5.9133);
\draw [c] (1.267,5.97377) -- (1,5.97377);
\draw [anchor= east] (0.922,5.97377) node[color=c, rotate=0]{$10^{3}$};
\draw [c] (1.1335,6.37155) -- (1,6.37155);
\draw [c] (1.1335,6.60425) -- (1,6.60425);
\colorlet{c}{natgreen};
\draw [c] (1.56886,0.680516) -- (1.56886,2.32133);
\draw [c] (1.56886,2.32133) -- (1.56886,2.71912);
\draw [c] (1.53093,2.32133) -- (1.56886,2.32133);
\draw [c] (1.56886,2.32133) -- (1.60678,2.32133);
\definecolor{c}{rgb}{0,0,0};
\colorlet{c}{natgreen};
\draw [c] (1.6447,2.15154) -- (1.6447,2.77477);
\draw [c] (1.6447,2.77477) -- (1.6447,3.06646);
\draw [c] (1.60678,2.77477) -- (1.6447,2.77477);
\draw [c] (1.6447,2.77477) -- (1.68263,2.77477);
\definecolor{c}{rgb}{0,0,0};
\colorlet{c}{natgreen};
\draw [c] (1.72055,4.09914) -- (1.72055,4.23107);
\draw [c] (1.72055,4.23107) -- (1.72055,4.33827);
\draw [c] (1.68263,4.23107) -- (1.72055,4.23107);
\draw [c] (1.72055,4.23107) -- (1.75847,4.23107);
\definecolor{c}{rgb}{0,0,0};
\colorlet{c}{natgreen};
\draw [c] (1.7964,5.88673) -- (1.7964,5.91485);
\draw [c] (1.7964,5.91485) -- (1.7964,5.94166);
\draw [c] (1.75847,5.91485) -- (1.7964,5.91485);
\draw [c] (1.7964,5.91485) -- (1.83432,5.91485);
\definecolor{c}{rgb}{0,0,0};
\colorlet{c}{natgreen};
\draw [c] (1.87225,6.36331) -- (1.87225,6.38193);
\draw [c] (1.87225,6.38193) -- (1.87225,6.39997);
\draw [c] (1.83432,6.38193) -- (1.87225,6.38193);
\draw [c] (1.87225,6.38193) -- (1.91017,6.38193);
\definecolor{c}{rgb}{0,0,0};
\colorlet{c}{natgreen};
\draw [c] (1.94809,6.44614) -- (1.94809,6.46344);
\draw [c] (1.94809,6.46344) -- (1.94809,6.48024);
\draw [c] (1.91017,6.46344) -- (1.94809,6.46344);
\draw [c] (1.94809,6.46344) -- (1.98602,6.46344);
\definecolor{c}{rgb}{0,0,0};
\colorlet{c}{natgreen};
\draw [c] (2.02394,6.37812) -- (2.02394,6.39641);
\draw [c] (2.02394,6.39641) -- (2.02394,6.41414);
\draw [c] (1.98602,6.39641) -- (2.02394,6.39641);
\draw [c] (2.02394,6.39641) -- (2.06186,6.39641);
\definecolor{c}{rgb}{0,0,0};
\colorlet{c}{natgreen};
\draw [c] (2.09979,6.3115) -- (2.09979,6.3308);
\draw [c] (2.09979,6.3308) -- (2.09979,6.34947);
\draw [c] (2.06186,6.3308) -- (2.09979,6.3308);
\draw [c] (2.09979,6.3308) -- (2.13771,6.3308);
\definecolor{c}{rgb}{0,0,0};
\colorlet{c}{natgreen};
\draw [c] (2.17564,6.22934) -- (2.17564,6.25001);
\draw [c] (2.17564,6.25001) -- (2.17564,6.26996);
\draw [c] (2.13771,6.25001) -- (2.17564,6.25001);
\draw [c] (2.17564,6.25001) -- (2.21356,6.25001);
\definecolor{c}{rgb}{0,0,0};
\colorlet{c}{natgreen};
\draw [c] (2.25148,6.12923) -- (2.25148,6.15189);
\draw [c] (2.25148,6.15189) -- (2.25148,6.17369);
\draw [c] (2.21356,6.15189) -- (2.25148,6.15189);
\draw [c] (2.25148,6.15189) -- (2.28941,6.15189);
\definecolor{c}{rgb}{0,0,0};
\colorlet{c}{natgreen};
\draw [c] (2.32733,5.99974) -- (2.32733,6.02512);
\draw [c] (2.32733,6.02512) -- (2.32733,6.04941);
\draw [c] (2.28941,6.02512) -- (2.32733,6.02512);
\draw [c] (2.32733,6.02512) -- (2.36525,6.02512);
\definecolor{c}{rgb}{0,0,0};
\colorlet{c}{natgreen};
\draw [c] (2.40318,5.93984) -- (2.40318,5.96669);
\draw [c] (2.40318,5.96669) -- (2.40318,5.99234);
\draw [c] (2.36525,5.96669) -- (2.40318,5.96669);
\draw [c] (2.40318,5.96669) -- (2.4411,5.96669);
\definecolor{c}{rgb}{0,0,0};
\colorlet{c}{natgreen};
\draw [c] (2.47903,5.86198) -- (2.47903,5.89044);
\draw [c] (2.47903,5.89044) -- (2.47903,5.91755);
\draw [c] (2.4411,5.89044) -- (2.47903,5.89044);
\draw [c] (2.47903,5.89044) -- (2.51695,5.89044);
\definecolor{c}{rgb}{0,0,0};
\colorlet{c}{natgreen};
\draw [c] (2.55487,5.728) -- (2.55487,5.76007);
\draw [c] (2.55487,5.76007) -- (2.55487,5.79044);
\draw [c] (2.51695,5.76007) -- (2.55487,5.76007);
\draw [c] (2.55487,5.76007) -- (2.5928,5.76007);
\definecolor{c}{rgb}{0,0,0};
\colorlet{c}{natgreen};
\draw [c] (2.63072,5.662) -- (2.63072,5.69603);
\draw [c] (2.63072,5.69603) -- (2.63072,5.72815);
\draw [c] (2.5928,5.69603) -- (2.63072,5.69603);
\draw [c] (2.63072,5.69603) -- (2.66864,5.69603);
\definecolor{c}{rgb}{0,0,0};
\colorlet{c}{natgreen};
\draw [c] (2.70657,5.52065) -- (2.70657,5.55932);
\draw [c] (2.70657,5.55932) -- (2.70657,5.59556);
\draw [c] (2.66864,5.55932) -- (2.70657,5.55932);
\draw [c] (2.70657,5.55932) -- (2.74449,5.55932);
\definecolor{c}{rgb}{0,0,0};
\colorlet{c}{natgreen};
\draw [c] (2.78242,5.39535) -- (2.78242,5.43811);
\draw [c] (2.78242,5.43811) -- (2.78242,5.4779);
\draw [c] (2.74449,5.43811) -- (2.78242,5.43811);
\draw [c] (2.78242,5.43811) -- (2.82034,5.43811);
\definecolor{c}{rgb}{0,0,0};
\colorlet{c}{natgreen};
\draw [c] (2.85826,5.25104) -- (2.85826,5.2996);
\draw [c] (2.85826,5.2996) -- (2.85826,5.34438);
\draw [c] (2.82034,5.2996) -- (2.85826,5.2996);
\draw [c] (2.85826,5.2996) -- (2.89619,5.2996);
\definecolor{c}{rgb}{0,0,0};
\colorlet{c}{natgreen};
\draw [c] (2.93411,5.28231) -- (2.93411,5.32982);
\draw [c] (2.93411,5.32982) -- (2.93411,5.37369);
\draw [c] (2.89619,5.32982) -- (2.93411,5.32982);
\draw [c] (2.93411,5.32982) -- (2.97203,5.32982);
\definecolor{c}{rgb}{0,0,0};
\colorlet{c}{natgreen};
\draw [c] (3.00996,5.21897) -- (3.00996,5.2689);
\draw [c] (3.00996,5.2689) -- (3.00996,5.31483);
\draw [c] (2.97203,5.2689) -- (3.00996,5.2689);
\draw [c] (3.00996,5.2689) -- (3.04788,5.2689);
\definecolor{c}{rgb}{0,0,0};
\colorlet{c}{natgreen};
\draw [c] (3.08581,5.21023) -- (3.08581,5.2608);
\draw [c] (3.08581,5.2608) -- (3.08581,5.30728);
\draw [c] (3.04788,5.2608) -- (3.08581,5.2608);
\draw [c] (3.08581,5.2608) -- (3.12373,5.2608);
\definecolor{c}{rgb}{0,0,0};
\colorlet{c}{natgreen};
\draw [c] (3.16165,4.96886) -- (3.16165,5.03107);
\draw [c] (3.16165,5.03107) -- (3.16165,5.08718);
\draw [c] (3.12373,5.03107) -- (3.16165,5.03107);
\draw [c] (3.16165,5.03107) -- (3.19958,5.03107);
\definecolor{c}{rgb}{0,0,0};
\colorlet{c}{natgreen};
\draw [c] (3.2375,4.94124) -- (3.2375,5.00492);
\draw [c] (3.2375,5.00492) -- (3.2375,5.06224);
\draw [c] (3.19958,5.00492) -- (3.2375,5.00492);
\draw [c] (3.2375,5.00492) -- (3.27542,5.00492);
\definecolor{c}{rgb}{0,0,0};
\colorlet{c}{natgreen};
\draw [c] (3.31335,4.69119) -- (3.31335,4.76971);
\draw [c] (3.31335,4.76971) -- (3.31335,4.83877);
\draw [c] (3.27542,4.76971) -- (3.31335,4.76971);
\draw [c] (3.31335,4.76971) -- (3.35127,4.76971);
\definecolor{c}{rgb}{0,0,0};
\colorlet{c}{natgreen};
\draw [c] (3.38919,4.67612) -- (3.38919,4.75548);
\draw [c] (3.38919,4.75548) -- (3.38919,4.82519);
\draw [c] (3.35127,4.75548) -- (3.38919,4.75548);
\draw [c] (3.38919,4.75548) -- (3.42712,4.75548);
\definecolor{c}{rgb}{0,0,0};
\colorlet{c}{natgreen};
\draw [c] (3.46504,4.74061) -- (3.46504,4.81591);
\draw [c] (3.46504,4.81591) -- (3.46504,4.88246);
\draw [c] (3.42712,4.81591) -- (3.46504,4.81591);
\draw [c] (3.46504,4.81591) -- (3.50297,4.81591);
\definecolor{c}{rgb}{0,0,0};
\colorlet{c}{natgreen};
\draw [c] (3.54089,4.54627) -- (3.54089,4.63667);
\draw [c] (3.54089,4.63667) -- (3.54089,4.71475);
\draw [c] (3.50297,4.63667) -- (3.54089,4.63667);
\draw [c] (3.54089,4.63667) -- (3.57881,4.63667);
\definecolor{c}{rgb}{0,0,0};
\colorlet{c}{natgreen};
\draw [c] (3.61674,4.44837) -- (3.61674,4.54601);
\draw [c] (3.61674,4.54601) -- (3.61674,4.62944);
\draw [c] (3.57881,4.54601) -- (3.61674,4.54601);
\draw [c] (3.61674,4.54601) -- (3.65466,4.54601);
\definecolor{c}{rgb}{0,0,0};
\colorlet{c}{natgreen};
\draw [c] (3.69258,4.55513) -- (3.69258,4.64683);
\draw [c] (3.69258,4.64683) -- (3.69258,4.72588);
\draw [c] (3.65466,4.64683) -- (3.69258,4.64683);
\draw [c] (3.69258,4.64683) -- (3.73051,4.64683);
\definecolor{c}{rgb}{0,0,0};
\colorlet{c}{natgreen};
\draw [c] (3.76843,4.48937) -- (3.76843,4.58443);
\draw [c] (3.76843,4.58443) -- (3.76843,4.66596);
\draw [c] (3.73051,4.58443) -- (3.76843,4.58443);
\draw [c] (3.76843,4.58443) -- (3.80636,4.58443);
\definecolor{c}{rgb}{0,0,0};
\colorlet{c}{natgreen};
\draw [c] (3.84428,4.1449) -- (3.84428,4.26971);
\draw [c] (3.84428,4.26971) -- (3.84428,4.37217);
\draw [c] (3.80636,4.26971) -- (3.84428,4.26971);
\draw [c] (3.84428,4.26971) -- (3.8822,4.26971);
\definecolor{c}{rgb}{0,0,0};
\colorlet{c}{natgreen};
\draw [c] (3.92013,4.28662) -- (3.92013,4.40131);
\draw [c] (3.92013,4.40131) -- (3.92013,4.49685);
\draw [c] (3.8822,4.40131) -- (3.92013,4.40131);
\draw [c] (3.92013,4.40131) -- (3.95805,4.40131);
\definecolor{c}{rgb}{0,0,0};
\colorlet{c}{natgreen};
\draw [c] (3.99597,4.2373) -- (3.99597,4.35773);
\draw [c] (3.99597,4.35773) -- (3.99597,4.45722);
\draw [c] (3.95805,4.35773) -- (3.99597,4.35773);
\draw [c] (3.99597,4.35773) -- (4.0339,4.35773);
\definecolor{c}{rgb}{0,0,0};
\colorlet{c}{natgreen};
\draw [c] (4.07182,4.17735) -- (4.07182,4.30431);
\draw [c] (4.07182,4.30431) -- (4.07182,4.40821);
\draw [c] (4.0339,4.30431) -- (4.07182,4.30431);
\draw [c] (4.07182,4.30431) -- (4.10975,4.30431);
\definecolor{c}{rgb}{0,0,0};
\colorlet{c}{natgreen};
\draw [c] (4.14767,3.58812) -- (4.14767,3.79113);
\draw [c] (4.14767,3.79113) -- (4.14767,3.94079);
\draw [c] (4.10975,3.79113) -- (4.14767,3.79113);
\draw [c] (4.14767,3.79113) -- (4.18559,3.79113);
\definecolor{c}{rgb}{0,0,0};
\colorlet{c}{natgreen};
\draw [c] (4.22352,3.94542) -- (4.22352,4.0949);
\draw [c] (4.22352,4.0949) -- (4.22352,4.21338);
\draw [c] (4.18559,4.0949) -- (4.22352,4.0949);
\draw [c] (4.22352,4.0949) -- (4.26144,4.0949);
\definecolor{c}{rgb}{0,0,0};
\colorlet{c}{natgreen};
\draw [c] (4.29936,4.19993) -- (4.29936,4.3255);
\draw [c] (4.29936,4.3255) -- (4.29936,4.42847);
\draw [c] (4.26144,4.3255) -- (4.29936,4.3255);
\draw [c] (4.29936,4.3255) -- (4.33729,4.3255);
\definecolor{c}{rgb}{0,0,0};
\colorlet{c}{natgreen};
\draw [c] (4.37521,3.55265) -- (4.37521,3.76314);
\draw [c] (4.37521,3.76314) -- (4.37521,3.91681);
\draw [c] (4.33729,3.76314) -- (4.37521,3.76314);
\draw [c] (4.37521,3.76314) -- (4.41314,3.76314);
\definecolor{c}{rgb}{0,0,0};
\colorlet{c}{natgreen};
\draw [c] (4.45106,3.92155) -- (4.45106,4.07489);
\draw [c] (4.45106,4.07489) -- (4.45106,4.19578);
\draw [c] (4.41314,4.07489) -- (4.45106,4.07489);
\draw [c] (4.45106,4.07489) -- (4.48898,4.07489);
\definecolor{c}{rgb}{0,0,0};
\colorlet{c}{natgreen};
\draw [c] (4.52691,3.54663) -- (4.52691,3.75691);
\draw [c] (4.52691,3.75691) -- (4.52691,3.91046);
\draw [c] (4.48898,3.75691) -- (4.52691,3.75691);
\draw [c] (4.52691,3.75691) -- (4.56483,3.75691);
\definecolor{c}{rgb}{0,0,0};
\colorlet{c}{natgreen};
\draw [c] (4.60275,3.66877) -- (4.60275,3.8548);
\draw [c] (4.60275,3.8548) -- (4.60275,3.99507);
\draw [c] (4.56483,3.8548) -- (4.60275,3.8548);
\draw [c] (4.60275,3.8548) -- (4.64068,3.8548);
\definecolor{c}{rgb}{0,0,0};
\colorlet{c}{natgreen};
\draw [c] (4.6786,3.38407) -- (4.6786,3.62745);
\draw [c] (4.6786,3.62745) -- (4.6786,3.79781);
\draw [c] (4.64068,3.62745) -- (4.6786,3.62745);
\draw [c] (4.6786,3.62745) -- (4.71653,3.62745);
\definecolor{c}{rgb}{0,0,0};
\colorlet{c}{natgreen};
\draw [c] (4.75445,3.33733) -- (4.75445,3.56304);
\draw [c] (4.75445,3.56304) -- (4.75445,3.72461);
\draw [c] (4.71653,3.56304) -- (4.75445,3.56304);
\draw [c] (4.75445,3.56304) -- (4.79237,3.56304);
\definecolor{c}{rgb}{0,0,0};
\colorlet{c}{natgreen};
\draw [c] (4.8303,3.82832) -- (4.8303,3.99795);
\draw [c] (4.8303,3.99795) -- (4.8303,4.12871);
\draw [c] (4.79237,3.99795) -- (4.8303,3.99795);
\draw [c] (4.8303,3.99795) -- (4.86822,3.99795);
\definecolor{c}{rgb}{0,0,0};
\colorlet{c}{natgreen};
\draw [c] (4.90614,3.33704) -- (4.90614,3.58231);
\draw [c] (4.90614,3.58231) -- (4.90614,3.7536);
\draw [c] (4.86822,3.58231) -- (4.90614,3.58231);
\draw [c] (4.90614,3.58231) -- (4.94407,3.58231);
\definecolor{c}{rgb}{0,0,0};
\colorlet{c}{natgreen};
\draw [c] (4.98199,3.5684) -- (4.98199,3.77461);
\draw [c] (4.98199,3.77461) -- (4.98199,3.92599);
\draw [c] (4.94407,3.77461) -- (4.98199,3.77461);
\draw [c] (4.98199,3.77461) -- (5.01992,3.77461);
\definecolor{c}{rgb}{0,0,0};
\colorlet{c}{natgreen};
\draw [c] (5.05784,3.37991) -- (5.05784,3.63276);
\draw [c] (5.05784,3.63276) -- (5.05784,3.80768);
\draw [c] (5.01992,3.63276) -- (5.05784,3.63276);
\draw [c] (5.05784,3.63276) -- (5.09576,3.63276);
\definecolor{c}{rgb}{0,0,0};
\colorlet{c}{natgreen};
\draw [c] (5.13369,3.09137) -- (5.13369,3.38505);
\draw [c] (5.13369,3.38505) -- (5.13369,3.57837);
\draw [c] (5.09576,3.38505) -- (5.13369,3.38505);
\draw [c] (5.13369,3.38505) -- (5.17161,3.38505);
\definecolor{c}{rgb}{0,0,0};
\colorlet{c}{natgreen};
\draw [c] (5.20953,2.62672) -- (5.20953,3.12268);
\draw [c] (5.20953,3.12268) -- (5.20953,3.38468);
\draw [c] (5.17161,3.12268) -- (5.20953,3.12268);
\draw [c] (5.20953,3.12268) -- (5.24746,3.12268);
\definecolor{c}{rgb}{0,0,0};
\colorlet{c}{natgreen};
\draw [c] (5.28538,3.30387) -- (5.28538,3.56756);
\draw [c] (5.28538,3.56756) -- (5.28538,3.74755);
\draw [c] (5.24746,3.56756) -- (5.28538,3.56756);
\draw [c] (5.28538,3.56756) -- (5.32331,3.56756);
\definecolor{c}{rgb}{0,0,0};
\colorlet{c}{natgreen};
\draw [c] (5.36123,3.06923) -- (5.36123,3.39873);
\draw [c] (5.36123,3.39873) -- (5.36123,3.60673);
\draw [c] (5.32331,3.39873) -- (5.36123,3.39873);
\draw [c] (5.36123,3.39873) -- (5.39915,3.39873);
\definecolor{c}{rgb}{0,0,0};
\colorlet{c}{natgreen};
\draw [c] (5.43708,2.75701) -- (5.43708,3.13388);
\draw [c] (5.43708,3.13388) -- (5.43708,3.35943);
\draw [c] (5.39915,3.13388) -- (5.43708,3.13388);
\draw [c] (5.43708,3.13388) -- (5.475,3.13388);
\definecolor{c}{rgb}{0,0,0};
\colorlet{c}{natgreen};
\draw [c] (5.58877,2.38329) -- (5.58877,2.91028);
\draw [c] (5.58877,2.91028) -- (5.58877,3.18029);
\draw [c] (5.55085,2.91028) -- (5.58877,2.91028);
\draw [c] (5.58877,2.91028) -- (5.6267,2.91028);
\definecolor{c}{rgb}{0,0,0};
\colorlet{c}{natgreen};
\draw [c] (5.66462,2.31794) -- (5.66462,2.87603);
\draw [c] (5.66462,2.87603) -- (5.66462,3.15354);
\draw [c] (5.6267,2.87603) -- (5.66462,2.87603);
\draw [c] (5.66462,2.87603) -- (5.70254,2.87603);
\definecolor{c}{rgb}{0,0,0};
\colorlet{c}{natgreen};
\draw [c] (5.74047,2.64865) -- (5.74047,3.14387);
\draw [c] (5.74047,3.14387) -- (5.74047,3.40568);
\draw [c] (5.70254,3.14387) -- (5.74047,3.14387);
\draw [c] (5.74047,3.14387) -- (5.77839,3.14387);
\definecolor{c}{rgb}{0,0,0};
\colorlet{c}{natgreen};
\draw [c] (5.81631,2.63248) -- (5.81631,3.01946);
\draw [c] (5.81631,3.01946) -- (5.81631,3.2485);
\draw [c] (5.77839,3.01946) -- (5.81631,3.01946);
\draw [c] (5.81631,3.01946) -- (5.85424,3.01946);
\definecolor{c}{rgb}{0,0,0};
\colorlet{c}{natgreen};
\draw [c] (5.89216,3.02506) -- (5.89216,3.35228);
\draw [c] (5.89216,3.35228) -- (5.89216,3.55938);
\draw [c] (5.85424,3.35228) -- (5.89216,3.35228);
\draw [c] (5.89216,3.35228) -- (5.93008,3.35228);
\definecolor{c}{rgb}{0,0,0};
\colorlet{c}{natgreen};
\draw [c] (5.96801,2.96904) -- (5.96801,3.3376);
\draw [c] (5.96801,3.3376) -- (5.96801,3.5602);
\draw [c] (5.93008,3.3376) -- (5.96801,3.3376);
\draw [c] (5.96801,3.3376) -- (6.00593,3.3376);
\definecolor{c}{rgb}{0,0,0};
\colorlet{c}{natgreen};
\draw [c] (6.04386,2.13625) -- (6.04386,2.84097);
\draw [c] (6.04386,2.84097) -- (6.04386,3.14788);
\draw [c] (6.00593,2.84097) -- (6.04386,2.84097);
\draw [c] (6.04386,2.84097) -- (6.08178,2.84097);
\definecolor{c}{rgb}{0,0,0};
\colorlet{c}{natgreen};
\draw [c] (6.1197,0.680516) -- (6.1197,2.38875);
\draw [c] (6.1197,2.38875) -- (6.1197,2.78654);
\draw [c] (6.08178,2.38875) -- (6.1197,2.38875);
\draw [c] (6.1197,2.38875) -- (6.15763,2.38875);
\definecolor{c}{rgb}{0,0,0};
\colorlet{c}{natgreen};
\draw [c] (6.19555,2.54233) -- (6.19555,3.0409);
\draw [c] (6.19555,3.0409) -- (6.19555,3.3036);
\draw [c] (6.15763,3.0409) -- (6.19555,3.0409);
\draw [c] (6.19555,3.0409) -- (6.23347,3.0409);
\definecolor{c}{rgb}{0,0,0};
\colorlet{c}{natgreen};
\draw [c] (6.2714,2.75147) -- (6.2714,3.15831);
\draw [c] (6.2714,3.15831) -- (6.2714,3.39399);
\draw [c] (6.23347,3.15831) -- (6.2714,3.15831);
\draw [c] (6.2714,3.15831) -- (6.30932,3.15831);
\definecolor{c}{rgb}{0,0,0};
\colorlet{c}{natgreen};
\draw [c] (6.34725,1.87883) -- (6.34725,2.64933);
\draw [c] (6.34725,2.64933) -- (6.34725,2.96681);
\draw [c] (6.30932,2.64933) -- (6.34725,2.64933);
\draw [c] (6.34725,2.64933) -- (6.38517,2.64933);
\definecolor{c}{rgb}{0,0,0};
\colorlet{c}{natgreen};
\draw [c] (6.42309,2.61419) -- (6.42309,3.10871);
\draw [c] (6.42309,3.10871) -- (6.42309,3.37033);
\draw [c] (6.38517,3.10871) -- (6.42309,3.10871);
\draw [c] (6.42309,3.10871) -- (6.46102,3.10871);
\definecolor{c}{rgb}{0,0,0};
\colorlet{c}{natgreen};
\draw [c] (6.57479,1.35007) -- (6.57479,2.57712);
\draw [c] (6.57479,2.57712) -- (6.57479,2.94005);
\draw [c] (6.53686,2.57712) -- (6.57479,2.57712);
\draw [c] (6.57479,2.57712) -- (6.61271,2.57712);
\definecolor{c}{rgb}{0,0,0};
\colorlet{c}{natgreen};
\draw [c] (6.65064,2.86211) -- (6.65064,3.23882);
\draw [c] (6.65064,3.23882) -- (6.65064,3.4643);
\draw [c] (6.61271,3.23882) -- (6.65064,3.23882);
\draw [c] (6.65064,3.23882) -- (6.68856,3.23882);
\definecolor{c}{rgb}{0,0,0};
\colorlet{c}{natgreen};
\draw [c] (6.87818,0.680516) -- (6.87818,1.90334);
\draw [c] (6.87818,1.90334) -- (6.87818,2.30113);
\draw [c] (6.84025,1.90334) -- (6.87818,1.90334);
\draw [c] (6.87818,1.90334) -- (6.9161,1.90334);
\definecolor{c}{rgb}{0,0,0};
\colorlet{c}{natgreen};
\draw [c] (7.02987,1.97897) -- (7.02987,2.7321);
\draw [c] (7.02987,2.7321) -- (7.02987,3.04692);
\draw [c] (6.99195,2.7321) -- (7.02987,2.7321);
\draw [c] (7.02987,2.7321) -- (7.0678,2.7321);
\definecolor{c}{rgb}{0,0,0};
\colorlet{c}{natgreen};
\draw [c] (7.18157,0.680516) -- (7.18157,2.58979);
\draw [c] (7.18157,2.58979) -- (7.18157,2.98757);
\draw [c] (7.14364,2.58979) -- (7.18157,2.58979);
\draw [c] (7.18157,2.58979) -- (7.21949,2.58979);
\definecolor{c}{rgb}{0,0,0};
\colorlet{c}{natgreen};
\draw [c] (7.33326,0.680516) -- (7.33326,2.24583);
\draw [c] (7.33326,2.24583) -- (7.33326,2.64362);
\draw [c] (7.29534,2.24583) -- (7.33326,2.24583);
\draw [c] (7.33326,2.24583) -- (7.37119,2.24583);
\definecolor{c}{rgb}{0,0,0};
\colorlet{c}{natgreen};
\draw [c] (7.63665,0.680516) -- (7.63665,2.43257);
\draw [c] (7.63665,2.43257) -- (7.63665,2.83036);
\draw [c] (7.59873,2.43257) -- (7.63665,2.43257);
\draw [c] (7.63665,2.43257) -- (7.67458,2.43257);
\definecolor{c}{rgb}{0,0,0};
\colorlet{c}{natgreen};
\draw [c] (7.78835,0.680516) -- (7.78835,2.25707);
\draw [c] (7.78835,2.25707) -- (7.78835,2.65486);
\draw [c] (7.75042,2.25707) -- (7.78835,2.25707);
\draw [c] (7.78835,2.25707) -- (7.82627,2.25707);
\definecolor{c}{rgb}{0,0,0};
\colorlet{c}{natgreen};
\draw [c] (7.86419,0.680516) -- (7.86419,2.4637);
\draw [c] (7.86419,2.4637) -- (7.86419,2.86148);
\draw [c] (7.82627,2.4637) -- (7.86419,2.4637);
\draw [c] (7.86419,2.4637) -- (7.90212,2.4637);
\definecolor{c}{rgb}{0,0,0};
\colorlet{c}{natgreen};
\draw [c] (8.01589,2.10948) -- (8.01589,2.81591);
\draw [c] (8.01589,2.81591) -- (8.01589,3.12312);
\draw [c] (7.97797,2.81591) -- (8.01589,2.81591);
\draw [c] (8.01589,2.81591) -- (8.05381,2.81591);
\definecolor{c}{rgb}{0,0,0};
\colorlet{c}{natgreen};
\draw [c] (8.09174,2.16055) -- (8.09174,2.86528);
\draw [c] (8.09174,2.86528) -- (8.09174,3.1722);
\draw [c] (8.05381,2.86528) -- (8.09174,2.86528);
\draw [c] (8.09174,2.86528) -- (8.12966,2.86528);
\definecolor{c}{rgb}{0,0,0};
\colorlet{c}{natgreen};
\draw [c] (8.24343,1.97776) -- (8.24343,2.72975);
\draw [c] (8.24343,2.72975) -- (8.24343,3.04439);
\draw [c] (8.20551,2.72975) -- (8.24343,2.72975);
\draw [c] (8.24343,2.72975) -- (8.28136,2.72975);
\definecolor{c}{rgb}{0,0,0};
\colorlet{c}{natgreen};
\draw [c] (8.39513,0.680516) -- (8.39513,2.49913);
\draw [c] (8.39513,2.49913) -- (8.39513,2.89692);
\draw [c] (8.3572,2.49913) -- (8.39513,2.49913);
\draw [c] (8.39513,2.49913) -- (8.43305,2.49913);
\definecolor{c}{rgb}{0,0,0};
\colorlet{c}{natgreen};
\draw [c] (9.60869,0.680516) -- (9.60869,2.58979);
\draw [c] (9.60869,2.58979) -- (9.60869,2.98757);
\draw [c] (9.57076,2.58979) -- (9.60869,2.58979);
\draw [c] (9.60869,2.58979) -- (9.64661,2.58979);
\definecolor{c}{rgb}{0,0,0};
\draw [anchor= west] (6.52221,6.06375) node[color=c, rotate=0]{ATLAS MC};
\colorlet{c}{natgreen};
\draw [c] (5.77633,6.06375) -- (6.39058,6.06375);
\draw [c] (6.08345,5.87679) -- (6.08345,6.25072);
\definecolor{c}{rgb}{0,0,0};
\draw [anchor= west] (6.52221,5.44054) node[color=c, rotate=0]{CalcHEP + box};
\colorlet{c}{natcomp!70};
\draw [c] (5.77633,5.44054) -- (6.39058,5.44054);
\draw [c] (6.08345,5.25358) -- (6.08345,5.62751);
\end{tikzpicture}

\end{infilsf}
\end{minipage}
\begin{minipage}[b]{.3\textwidth}
\caption{Comparing the \atlas{} distribution with the one produced by CalcHEP, combined with a distribution for the box diagram contribution.}\label{ggcomp}
\end{minipage}
\end{figure}

As the estimated data background (see fig.~\ref{atlasmc}) and box diagram (see fig.~\ref{boxmgg}) distributions have the same sort of shape, it is natural to combine them before attempting to extrapolate a shape. The function we fit to this distribution has the expression, suggested by \cite{powlaw}
\(f(x)=\vet{7.9\,10^{-5}}\left(\frac{1-x}{8000}\right)^{\vet{22.5506}}\left(\frac{x}{8000}\right)^{-\left[\vet{7.203553}\;\vet{-0.631809}\ln\frac{x}{8000}\right]},\)
where the four underscored numbers are the fitted parameters. $x$ has been scaled by $8\,000^{-1}$ GeV$^{-1}$ in an attempt to achieve parameter values close to 1. The function is plotted along with the histogram it was fitted to in figure~\ref{bckfit}. This fit, and all further fits, were carried out using \textsc{root}s fitting procedures \cite{root}.

\begin{figure}[htp]
\begin{minipage}[b]{.69\textwidth}
\begin{infilsf} \tiny
\begin{tikzpicture}[x=.092\textwidth,y=.092\textwidth]
\pgfdeclareplotmark{cross} {
\pgfpathmoveto{\pgfpoint{-0.3\pgfplotmarksize}{\pgfplotmarksize}}
\pgfpathlineto{\pgfpoint{+0.3\pgfplotmarksize}{\pgfplotmarksize}}
\pgfpathlineto{\pgfpoint{+0.3\pgfplotmarksize}{0.3\pgfplotmarksize}}
\pgfpathlineto{\pgfpoint{+1\pgfplotmarksize}{0.3\pgfplotmarksize}}
\pgfpathlineto{\pgfpoint{+1\pgfplotmarksize}{-0.3\pgfplotmarksize}}
\pgfpathlineto{\pgfpoint{+0.3\pgfplotmarksize}{-0.3\pgfplotmarksize}}
\pgfpathlineto{\pgfpoint{+0.3\pgfplotmarksize}{-1.\pgfplotmarksize}}
\pgfpathlineto{\pgfpoint{-0.3\pgfplotmarksize}{-1.\pgfplotmarksize}}
\pgfpathlineto{\pgfpoint{-0.3\pgfplotmarksize}{-0.3\pgfplotmarksize}}
\pgfpathlineto{\pgfpoint{-1.\pgfplotmarksize}{-0.3\pgfplotmarksize}}
\pgfpathlineto{\pgfpoint{-1.\pgfplotmarksize}{0.3\pgfplotmarksize}}
\pgfpathlineto{\pgfpoint{-0.3\pgfplotmarksize}{0.3\pgfplotmarksize}}
\pgfpathclose
\pgfusepathqstroke
}
\pgfdeclareplotmark{cross*} {
\pgfpathmoveto{\pgfpoint{-0.3\pgfplotmarksize}{\pgfplotmarksize}}
\pgfpathlineto{\pgfpoint{+0.3\pgfplotmarksize}{\pgfplotmarksize}}
\pgfpathlineto{\pgfpoint{+0.3\pgfplotmarksize}{0.3\pgfplotmarksize}}
\pgfpathlineto{\pgfpoint{+1\pgfplotmarksize}{0.3\pgfplotmarksize}}
\pgfpathlineto{\pgfpoint{+1\pgfplotmarksize}{-0.3\pgfplotmarksize}}
\pgfpathlineto{\pgfpoint{+0.3\pgfplotmarksize}{-0.3\pgfplotmarksize}}
\pgfpathlineto{\pgfpoint{+0.3\pgfplotmarksize}{-1.\pgfplotmarksize}}
\pgfpathlineto{\pgfpoint{-0.3\pgfplotmarksize}{-1.\pgfplotmarksize}}
\pgfpathlineto{\pgfpoint{-0.3\pgfplotmarksize}{-0.3\pgfplotmarksize}}
\pgfpathlineto{\pgfpoint{-1.\pgfplotmarksize}{-0.3\pgfplotmarksize}}
\pgfpathlineto{\pgfpoint{-1.\pgfplotmarksize}{0.3\pgfplotmarksize}}
\pgfpathlineto{\pgfpoint{-0.3\pgfplotmarksize}{0.3\pgfplotmarksize}}
\pgfpathclose
\pgfusepathqfillstroke
}
\pgfdeclareplotmark{newstar} {
\pgfpathmoveto{\pgfqpoint{0pt}{\pgfplotmarksize}}
\pgfpathlineto{\pgfqpointpolar{44}{0.5\pgfplotmarksize}}
\pgfpathlineto{\pgfqpointpolar{18}{\pgfplotmarksize}}
\pgfpathlineto{\pgfqpointpolar{-20}{0.5\pgfplotmarksize}}
\pgfpathlineto{\pgfqpointpolar{-54}{\pgfplotmarksize}}
\pgfpathlineto{\pgfqpointpolar{-90}{0.5\pgfplotmarksize}}
\pgfpathlineto{\pgfqpointpolar{234}{\pgfplotmarksize}}
\pgfpathlineto{\pgfqpointpolar{198}{0.5\pgfplotmarksize}}
\pgfpathlineto{\pgfqpointpolar{162}{\pgfplotmarksize}}
\pgfpathlineto{\pgfqpointpolar{134}{0.5\pgfplotmarksize}}
\pgfpathclose
\pgfusepathqstroke
}
\pgfdeclareplotmark{newstar*} {
\pgfpathmoveto{\pgfqpoint{0pt}{\pgfplotmarksize}}
\pgfpathlineto{\pgfqpointpolar{44}{0.5\pgfplotmarksize}}
\pgfpathlineto{\pgfqpointpolar{18}{\pgfplotmarksize}}
\pgfpathlineto{\pgfqpointpolar{-20}{0.5\pgfplotmarksize}}
\pgfpathlineto{\pgfqpointpolar{-54}{\pgfplotmarksize}}
\pgfpathlineto{\pgfqpointpolar{-90}{0.5\pgfplotmarksize}}
\pgfpathlineto{\pgfqpointpolar{234}{\pgfplotmarksize}}
\pgfpathlineto{\pgfqpointpolar{198}{0.5\pgfplotmarksize}}
\pgfpathlineto{\pgfqpointpolar{162}{\pgfplotmarksize}}
\pgfpathlineto{\pgfqpointpolar{134}{0.5\pgfplotmarksize}}
\pgfpathclose
\pgfusepathqfillstroke
}
\definecolor{c}{rgb}{1,1,1};
\draw [color=c, fill=c] (0,0) rectangle (10,6.80516);
\draw [color=c, fill=c] (1,0.680516) rectangle (9.95,6.73711);
\definecolor{c}{rgb}{0,0,0};
\draw [c] (1,0.680516) -- (1,6.73711) -- (9.95,6.73711) -- (9.95,0.680516) -- (1,0.680516);
\definecolor{c}{rgb}{1,1,1};
\draw [color=c, fill=c] (1,0.680516) rectangle (9.95,6.73711);
\definecolor{c}{rgb}{0,0,0};
\draw [c] (1,0.680516) -- (1,6.73711) -- (9.95,6.73711) -- (9.95,0.680516) -- (1,0.680516);
\colorlet{c}{natgreen};
\draw [c] (1.04431,3.58796) -- (1.04431,3.68717);
\draw [c] (1.04431,3.68717) -- (1.04431,3.77009);
\draw [c] (1,3.68717) -- (1.04431,3.68717);
\draw [c] (1.04431,3.68717) -- (1.08861,3.68717);
\definecolor{c}{rgb}{0,0,0};
\colorlet{c}{natgreen};
\draw [c] (1.13292,4.52557) -- (1.13292,4.56494);
\draw [c] (1.13292,4.56494) -- (1.13292,4.60147);
\draw [c] (1.08861,4.56494) -- (1.13292,4.56494);
\draw [c] (1.13292,4.56494) -- (1.17723,4.56494);
\definecolor{c}{rgb}{0,0,0};
\colorlet{c}{natgreen};
\draw [c] (1.22153,4.55305) -- (1.22153,4.59136);
\draw [c] (1.22153,4.59136) -- (1.22153,4.62698);
\draw [c] (1.17723,4.59136) -- (1.22153,4.59136);
\draw [c] (1.22153,4.59136) -- (1.26584,4.59136);
\definecolor{c}{rgb}{0,0,0};
\colorlet{c}{natgreen};
\draw [c] (1.31015,4.84177) -- (1.31015,4.87058);
\draw [c] (1.31015,4.87058) -- (1.31015,4.89784);
\draw [c] (1.26584,4.87058) -- (1.31015,4.87058);
\draw [c] (1.31015,4.87058) -- (1.35446,4.87058);
\definecolor{c}{rgb}{0,0,0};
\colorlet{c}{natgreen};
\draw [c] (1.39876,4.85866) -- (1.39876,4.887);
\draw [c] (1.39876,4.887) -- (1.39876,4.91384);
\draw [c] (1.35446,4.887) -- (1.39876,4.887);
\draw [c] (1.39876,4.887) -- (1.44307,4.887);
\definecolor{c}{rgb}{0,0,0};
\colorlet{c}{natgreen};
\draw [c] (1.48738,4.90421) -- (1.48738,4.9313);
\draw [c] (1.48738,4.9313) -- (1.48738,4.95702);
\draw [c] (1.44307,4.9313) -- (1.48738,4.9313);
\draw [c] (1.48738,4.9313) -- (1.53168,4.9313);
\definecolor{c}{rgb}{0,0,0};
\colorlet{c}{natgreen};
\draw [c] (1.57599,4.86959) -- (1.57599,4.89763);
\draw [c] (1.57599,4.89763) -- (1.57599,4.92419);
\draw [c] (1.53168,4.89763) -- (1.57599,4.89763);
\draw [c] (1.57599,4.89763) -- (1.6203,4.89763);
\definecolor{c}{rgb}{0,0,0};
\colorlet{c}{natgreen};
\draw [c] (1.6646,4.99569) -- (1.6646,5.02044);
\draw [c] (1.6646,5.02044) -- (1.6646,5.04403);
\draw [c] (1.6203,5.02044) -- (1.6646,5.02044);
\draw [c] (1.6646,5.02044) -- (1.70891,5.02044);
\definecolor{c}{rgb}{0,0,0};
\colorlet{c}{natgreen};
\draw [c] (1.75322,5.16291) -- (1.75322,5.18389);
\draw [c] (1.75322,5.18389) -- (1.75322,5.20403);
\draw [c] (1.70891,5.18389) -- (1.75322,5.18389);
\draw [c] (1.75322,5.18389) -- (1.79752,5.18389);
\definecolor{c}{rgb}{0,0,0};
\colorlet{c}{natgreen};
\draw [c] (1.84183,5.12662) -- (1.84183,5.14828);
\draw [c] (1.84183,5.14828) -- (1.84183,5.16906);
\draw [c] (1.79752,5.14828) -- (1.84183,5.14828);
\draw [c] (1.84183,5.14828) -- (1.88614,5.14828);
\definecolor{c}{rgb}{0,0,0};
\colorlet{c}{natgreen};
\draw [c] (1.93045,5.86287) -- (1.93045,5.87324);
\draw [c] (1.93045,5.87324) -- (1.93045,5.88341);
\draw [c] (1.88614,5.87324) -- (1.93045,5.87324);
\draw [c] (1.93045,5.87324) -- (1.97475,5.87324);
\definecolor{c}{rgb}{0,0,0};
\colorlet{c}{natgreen};
\draw [c] (2.01906,6.27256) -- (2.01906,6.27949);
\draw [c] (2.01906,6.27949) -- (2.01906,6.28633);
\draw [c] (1.97475,6.27949) -- (2.01906,6.27949);
\draw [c] (2.01906,6.27949) -- (2.06337,6.27949);
\definecolor{c}{rgb}{0,0,0};
\colorlet{c}{natgreen};
\draw [c] (2.10767,6.37754) -- (2.10767,6.3838);
\draw [c] (2.10767,6.3838) -- (2.10767,6.38998);
\draw [c] (2.06337,6.3838) -- (2.10767,6.3838);
\draw [c] (2.10767,6.3838) -- (2.15198,6.3838);
\definecolor{c}{rgb}{0,0,0};
\colorlet{c}{natgreen};
\draw [c] (2.19629,6.40128) -- (2.19629,6.4074);
\draw [c] (2.19629,6.4074) -- (2.19629,6.41345);
\draw [c] (2.15198,6.4074) -- (2.19629,6.4074);
\draw [c] (2.19629,6.4074) -- (2.24059,6.4074);
\definecolor{c}{rgb}{0,0,0};
\colorlet{c}{natgreen};
\draw [c] (2.2849,6.36434) -- (2.2849,6.3707);
\draw [c] (2.2849,6.3707) -- (2.2849,6.37697);
\draw [c] (2.24059,6.3707) -- (2.2849,6.3707);
\draw [c] (2.2849,6.3707) -- (2.32921,6.3707);
\definecolor{c}{rgb}{0,0,0};
\colorlet{c}{natgreen};
\draw [c] (2.37351,6.33688) -- (2.37351,6.34342);
\draw [c] (2.37351,6.34342) -- (2.37351,6.34987);
\draw [c] (2.32921,6.34342) -- (2.37351,6.34342);
\draw [c] (2.37351,6.34342) -- (2.41782,6.34342);
\definecolor{c}{rgb}{0,0,0};
\colorlet{c}{natgreen};
\draw [c] (2.46213,6.23385) -- (2.46213,6.24109);
\draw [c] (2.46213,6.24109) -- (2.46213,6.24822);
\draw [c] (2.41782,6.24109) -- (2.46213,6.24109);
\draw [c] (2.46213,6.24109) -- (2.50644,6.24109);
\definecolor{c}{rgb}{0,0,0};
\colorlet{c}{natgreen};
\draw [c] (2.55074,6.16569) -- (2.55074,6.17343);
\draw [c] (2.55074,6.17343) -- (2.55074,6.18107);
\draw [c] (2.50644,6.17343) -- (2.55074,6.17343);
\draw [c] (2.55074,6.17343) -- (2.59505,6.17343);
\definecolor{c}{rgb}{0,0,0};
\colorlet{c}{natgreen};
\draw [c] (2.63936,6.09344) -- (2.63936,6.10177);
\draw [c] (2.63936,6.10177) -- (2.63936,6.10996);
\draw [c] (2.59505,6.10177) -- (2.63936,6.10177);
\draw [c] (2.63936,6.10177) -- (2.68366,6.10177);
\definecolor{c}{rgb}{0,0,0};
\colorlet{c}{natgreen};
\draw [c] (2.72797,6.00711) -- (2.72797,6.01618);
\draw [c] (2.72797,6.01618) -- (2.72797,6.02509);
\draw [c] (2.68366,6.01618) -- (2.72797,6.01618);
\draw [c] (2.72797,6.01618) -- (2.77228,6.01618);
\definecolor{c}{rgb}{0,0,0};
\colorlet{c}{natgreen};
\draw [c] (2.81658,5.93403) -- (2.81658,5.94378);
\draw [c] (2.81658,5.94378) -- (2.81658,5.95336);
\draw [c] (2.77228,5.94378) -- (2.81658,5.94378);
\draw [c] (2.81658,5.94378) -- (2.86089,5.94378);
\definecolor{c}{rgb}{0,0,0};
\colorlet{c}{natgreen};
\draw [c] (2.9052,5.84107) -- (2.9052,5.85176);
\draw [c] (2.9052,5.85176) -- (2.9052,5.86224);
\draw [c] (2.86089,5.85176) -- (2.9052,5.85176);
\draw [c] (2.9052,5.85176) -- (2.94951,5.85176);
\definecolor{c}{rgb}{0,0,0};
\colorlet{c}{natgreen};
\draw [c] (2.99381,5.75037) -- (2.99381,5.76207);
\draw [c] (2.99381,5.76207) -- (2.99381,5.77351);
\draw [c] (2.94951,5.76207) -- (2.99381,5.76207);
\draw [c] (2.99381,5.76207) -- (3.03812,5.76207);
\definecolor{c}{rgb}{0,0,0};
\colorlet{c}{natgreen};
\draw [c] (3.08243,5.71391) -- (3.08243,5.72604);
\draw [c] (3.08243,5.72604) -- (3.08243,5.73789);
\draw [c] (3.03812,5.72604) -- (3.08243,5.72604);
\draw [c] (3.08243,5.72604) -- (3.12673,5.72604);
\definecolor{c}{rgb}{0,0,0};
\colorlet{c}{natgreen};
\draw [c] (3.17104,5.61884) -- (3.17104,5.63216);
\draw [c] (3.17104,5.63216) -- (3.17104,5.64513);
\draw [c] (3.12673,5.63216) -- (3.17104,5.63216);
\draw [c] (3.17104,5.63216) -- (3.21535,5.63216);
\definecolor{c}{rgb}{0,0,0};
\colorlet{c}{natgreen};
\draw [c] (3.25965,5.53981) -- (3.25965,5.55422);
\draw [c] (3.25965,5.55422) -- (3.25965,5.56823);
\draw [c] (3.21535,5.55422) -- (3.25965,5.55422);
\draw [c] (3.25965,5.55422) -- (3.30396,5.55422);
\definecolor{c}{rgb}{0,0,0};
\colorlet{c}{natgreen};
\draw [c] (3.34827,5.4505) -- (3.34827,5.46623);
\draw [c] (3.34827,5.46623) -- (3.34827,5.48148);
\draw [c] (3.30396,5.46623) -- (3.34827,5.46623);
\draw [c] (3.34827,5.46623) -- (3.39257,5.46623);
\definecolor{c}{rgb}{0,0,0};
\colorlet{c}{natgreen};
\draw [c] (3.43688,5.3618) -- (3.43688,5.37899);
\draw [c] (3.43688,5.37899) -- (3.43688,5.39562);
\draw [c] (3.39257,5.37899) -- (3.43688,5.37899);
\draw [c] (3.43688,5.37899) -- (3.48119,5.37899);
\definecolor{c}{rgb}{0,0,0};
\colorlet{c}{natgreen};
\draw [c] (3.5255,5.3192) -- (3.5255,5.33712);
\draw [c] (3.5255,5.33712) -- (3.5255,5.35443);
\draw [c] (3.48119,5.33712) -- (3.5255,5.33712);
\draw [c] (3.5255,5.33712) -- (3.5698,5.33712);
\definecolor{c}{rgb}{0,0,0};
\colorlet{c}{natgreen};
\draw [c] (3.61411,5.29761) -- (3.61411,5.31592);
\draw [c] (3.61411,5.31592) -- (3.61411,5.33358);
\draw [c] (3.5698,5.31592) -- (3.61411,5.31592);
\draw [c] (3.61411,5.31592) -- (3.65842,5.31592);
\definecolor{c}{rgb}{0,0,0};
\colorlet{c}{natgreen};
\draw [c] (3.70272,5.20597) -- (3.70272,5.22601);
\draw [c] (3.70272,5.22601) -- (3.70272,5.2453);
\draw [c] (3.65842,5.22601) -- (3.70272,5.22601);
\draw [c] (3.70272,5.22601) -- (3.74703,5.22601);
\definecolor{c}{rgb}{0,0,0};
\colorlet{c}{natgreen};
\draw [c] (3.79134,5.13236) -- (3.79134,5.15391);
\draw [c] (3.79134,5.15391) -- (3.79134,5.17457);
\draw [c] (3.74703,5.15391) -- (3.79134,5.15391);
\draw [c] (3.79134,5.15391) -- (3.83564,5.15391);
\definecolor{c}{rgb}{0,0,0};
\colorlet{c}{natgreen};
\draw [c] (3.87995,5.04025) -- (3.87995,5.06387);
\draw [c] (3.87995,5.06387) -- (3.87995,5.08644);
\draw [c] (3.83564,5.06387) -- (3.87995,5.06387);
\draw [c] (3.87995,5.06387) -- (3.92426,5.06387);
\definecolor{c}{rgb}{0,0,0};
\colorlet{c}{natgreen};
\draw [c] (3.96856,5.03979) -- (3.96856,5.06345);
\draw [c] (3.96856,5.06345) -- (3.96856,5.08605);
\draw [c] (3.92426,5.06345) -- (3.96856,5.06345);
\draw [c] (3.96856,5.06345) -- (4.01287,5.06345);
\definecolor{c}{rgb}{0,0,0};
\colorlet{c}{natgreen};
\draw [c] (4.05718,4.9159) -- (4.05718,4.94262);
\draw [c] (4.05718,4.94262) -- (4.05718,4.96799);
\draw [c] (4.01287,4.94262) -- (4.05718,4.94262);
\draw [c] (4.05718,4.94262) -- (4.10149,4.94262);
\definecolor{c}{rgb}{0,0,0};
\colorlet{c}{natgreen};
\draw [c] (4.14579,4.77187) -- (4.14579,4.80257);
\draw [c] (4.14579,4.80257) -- (4.14579,4.83152);
\draw [c] (4.10149,4.80257) -- (4.14579,4.80257);
\draw [c] (4.14579,4.80257) -- (4.1901,4.80257);
\definecolor{c}{rgb}{0,0,0};
\colorlet{c}{natgreen};
\draw [c] (4.23441,4.76339) -- (4.23441,4.79444);
\draw [c] (4.23441,4.79444) -- (4.23441,4.8237);
\draw [c] (4.1901,4.79444) -- (4.23441,4.79444);
\draw [c] (4.23441,4.79444) -- (4.27871,4.79444);
\definecolor{c}{rgb}{0,0,0};
\colorlet{c}{natgreen};
\draw [c] (4.32302,4.75203) -- (4.32302,4.78345);
\draw [c] (4.32302,4.78345) -- (4.32302,4.81304);
\draw [c] (4.27871,4.78345) -- (4.32302,4.78345);
\draw [c] (4.32302,4.78345) -- (4.36733,4.78345);
\definecolor{c}{rgb}{0,0,0};
\colorlet{c}{natgreen};
\draw [c] (4.41163,4.63179) -- (4.41163,4.66718);
\draw [c] (4.41163,4.66718) -- (4.41163,4.70026);
\draw [c] (4.36733,4.66718) -- (4.41163,4.66718);
\draw [c] (4.41163,4.66718) -- (4.45594,4.66718);
\definecolor{c}{rgb}{0,0,0};
\colorlet{c}{natgreen};
\draw [c] (4.50025,4.62157) -- (4.50025,4.65723);
\draw [c] (4.50025,4.65723) -- (4.50025,4.69055);
\draw [c] (4.45594,4.65723) -- (4.50025,4.65723);
\draw [c] (4.50025,4.65723) -- (4.54455,4.65723);
\definecolor{c}{rgb}{0,0,0};
\colorlet{c}{natgreen};
\draw [c] (4.58886,4.64707) -- (4.58886,4.6819);
\draw [c] (4.58886,4.6819) -- (4.58886,4.71448);
\draw [c] (4.54455,4.6819) -- (4.58886,4.6819);
\draw [c] (4.58886,4.6819) -- (4.63317,4.6819);
\definecolor{c}{rgb}{0,0,0};
\colorlet{c}{natgreen};
\draw [c] (4.67748,4.39624) -- (4.67748,4.44093);
\draw [c] (4.67748,4.44093) -- (4.67748,4.48199);
\draw [c] (4.63317,4.44093) -- (4.67748,4.44093);
\draw [c] (4.67748,4.44093) -- (4.72178,4.44093);
\definecolor{c}{rgb}{0,0,0};
\colorlet{c}{natgreen};
\draw [c] (4.76609,4.37532) -- (4.76609,4.42087);
\draw [c] (4.76609,4.42087) -- (4.76609,4.46266);
\draw [c] (4.72178,4.42087) -- (4.76609,4.42087);
\draw [c] (4.76609,4.42087) -- (4.8104,4.42087);
\definecolor{c}{rgb}{0,0,0};
\colorlet{c}{natgreen};
\draw [c] (4.8547,4.42175) -- (4.8547,4.46537);
\draw [c] (4.8547,4.46537) -- (4.8547,4.50552);
\draw [c] (4.8104,4.46537) -- (4.8547,4.46537);
\draw [c] (4.8547,4.46537) -- (4.89901,4.46537);
\definecolor{c}{rgb}{0,0,0};
\colorlet{c}{natgreen};
\draw [c] (4.94332,4.31787) -- (4.94332,4.36614);
\draw [c] (4.94332,4.36614) -- (4.94332,4.4102);
\draw [c] (4.89901,4.36614) -- (4.94332,4.36614);
\draw [c] (4.94332,4.36614) -- (4.98762,4.36614);
\definecolor{c}{rgb}{0,0,0};
\colorlet{c}{natgreen};
\draw [c] (5.03193,4.39308) -- (5.03193,4.43794);
\draw [c] (5.03193,4.43794) -- (5.03193,4.47915);
\draw [c] (4.98762,4.43794) -- (5.03193,4.43794);
\draw [c] (5.03193,4.43794) -- (5.07624,4.43794);
\definecolor{c}{rgb}{0,0,0};
\colorlet{c}{natgreen};
\draw [c] (5.12054,4.17587) -- (5.12054,4.23112);
\draw [c] (5.12054,4.23112) -- (5.12054,4.28093);
\draw [c] (5.07624,4.23112) -- (5.12054,4.23112);
\draw [c] (5.12054,4.23112) -- (5.16485,4.23112);
\definecolor{c}{rgb}{0,0,0};
\colorlet{c}{natgreen};
\draw [c] (5.20916,4.1169) -- (5.20916,4.17568);
\draw [c] (5.20916,4.17568) -- (5.20916,4.22834);
\draw [c] (5.16485,4.17568) -- (5.20916,4.17568);
\draw [c] (5.20916,4.17568) -- (5.25347,4.17568);
\definecolor{c}{rgb}{0,0,0};
\colorlet{c}{natgreen};
\draw [c] (5.29777,4.12189) -- (5.29777,4.18035);
\draw [c] (5.29777,4.18035) -- (5.29777,4.23276);
\draw [c] (5.25347,4.18035) -- (5.29777,4.18035);
\draw [c] (5.29777,4.18035) -- (5.34208,4.18035);
\definecolor{c}{rgb}{0,0,0};
\colorlet{c}{natgreen};
\draw [c] (5.38639,4.21247) -- (5.38639,4.26613);
\draw [c] (5.38639,4.26613) -- (5.38639,4.31463);
\draw [c] (5.34208,4.26613) -- (5.38639,4.26613);
\draw [c] (5.38639,4.26613) -- (5.43069,4.26613);
\definecolor{c}{rgb}{0,0,0};
\colorlet{c}{natgreen};
\draw [c] (5.475,4.14696) -- (5.475,4.20408);
\draw [c] (5.475,4.20408) -- (5.475,4.2554);
\draw [c] (5.43069,4.20408) -- (5.475,4.20408);
\draw [c] (5.475,4.20408) -- (5.51931,4.20408);
\definecolor{c}{rgb}{0,0,0};
\colorlet{c}{natgreen};
\draw [c] (5.56361,3.9376) -- (5.56361,4.00776);
\draw [c] (5.56361,4.00776) -- (5.56361,4.06937);
\draw [c] (5.51931,4.00776) -- (5.56361,4.00776);
\draw [c] (5.56361,4.00776) -- (5.60792,4.00776);
\definecolor{c}{rgb}{0,0,0};
\colorlet{c}{natgreen};
\draw [c] (5.65223,3.87671) -- (5.65223,3.95106);
\draw [c] (5.65223,3.95106) -- (5.65223,4.01587);
\draw [c] (5.60792,3.95106) -- (5.65223,3.95106);
\draw [c] (5.65223,3.95106) -- (5.69653,3.95106);
\definecolor{c}{rgb}{0,0,0};
\colorlet{c}{natgreen};
\draw [c] (5.74084,4.07352) -- (5.74084,4.13487);
\draw [c] (5.74084,4.13487) -- (5.74084,4.18959);
\draw [c] (5.69653,4.13487) -- (5.74084,4.13487);
\draw [c] (5.74084,4.13487) -- (5.78515,4.13487);
\definecolor{c}{rgb}{0,0,0};
\colorlet{c}{natgreen};
\draw [c] (5.82946,3.67762) -- (5.82946,3.7682);
\draw [c] (5.82946,3.7682) -- (5.82946,3.84501);
\draw [c] (5.78515,3.7682) -- (5.82946,3.7682);
\draw [c] (5.82946,3.7682) -- (5.87376,3.7682);
\definecolor{c}{rgb}{0,0,0};
\colorlet{c}{natgreen};
\draw [c] (5.91807,3.94968) -- (5.91807,4.01916);
\draw [c] (5.91807,4.01916) -- (5.91807,4.08024);
\draw [c] (5.87376,4.01916) -- (5.91807,4.01916);
\draw [c] (5.91807,4.01916) -- (5.96238,4.01916);
\definecolor{c}{rgb}{0,0,0};
\colorlet{c}{natgreen};
\draw [c] (6.00668,3.75403) -- (6.00668,3.83816);
\draw [c] (6.00668,3.83816) -- (6.00668,3.91028);
\draw [c] (5.96238,3.83816) -- (6.00668,3.83816);
\draw [c] (6.00668,3.83816) -- (6.05099,3.83816);
\definecolor{c}{rgb}{0,0,0};
\colorlet{c}{natgreen};
\draw [c] (6.0953,3.74843) -- (6.0953,3.83301);
\draw [c] (6.0953,3.83301) -- (6.0953,3.90546);
\draw [c] (6.05099,3.83301) -- (6.0953,3.83301);
\draw [c] (6.0953,3.83301) -- (6.1396,3.83301);
\definecolor{c}{rgb}{0,0,0};
\colorlet{c}{natgreen};
\draw [c] (6.18391,3.4963) -- (6.18391,3.60488);
\draw [c] (6.18391,3.60488) -- (6.18391,3.69423);
\draw [c] (6.1396,3.60488) -- (6.18391,3.60488);
\draw [c] (6.18391,3.60488) -- (6.22822,3.60488);
\definecolor{c}{rgb}{0,0,0};
\colorlet{c}{natgreen};
\draw [c] (6.27252,3.50919) -- (6.27252,3.6164);
\draw [c] (6.27252,3.6164) -- (6.27252,3.70483);
\draw [c] (6.22822,3.6164) -- (6.27252,3.6164);
\draw [c] (6.27252,3.6164) -- (6.31683,3.6164);
\definecolor{c}{rgb}{0,0,0};
\colorlet{c}{natgreen};
\draw [c] (6.36114,3.2093) -- (6.36114,3.3527);
\draw [c] (6.36114,3.3527) -- (6.36114,3.46433);
\draw [c] (6.31683,3.3527) -- (6.36114,3.3527);
\draw [c] (6.36114,3.3527) -- (6.40545,3.3527);
\definecolor{c}{rgb}{0,0,0};
\colorlet{c}{natgreen};
\draw [c] (6.44975,3.97014) -- (6.44975,4.03823);
\draw [c] (6.44975,4.03823) -- (6.44975,4.09824);
\draw [c] (6.40545,4.03823) -- (6.44975,4.03823);
\draw [c] (6.44975,4.03823) -- (6.49406,4.03823);
\definecolor{c}{rgb}{0,0,0};
\colorlet{c}{natgreen};
\draw [c] (6.53837,2.88887) -- (6.53837,3.08578);
\draw [c] (6.53837,3.08578) -- (6.53837,3.2272);
\draw [c] (6.49406,3.08578) -- (6.53837,3.08578);
\draw [c] (6.53837,3.08578) -- (6.58267,3.08578);
\definecolor{c}{rgb}{0,0,0};
\colorlet{c}{natgreen};
\draw [c] (6.62698,3.59236) -- (6.62698,3.69101);
\draw [c] (6.62698,3.69101) -- (6.62698,3.77354);
\draw [c] (6.58267,3.69101) -- (6.62698,3.69101);
\draw [c] (6.62698,3.69101) -- (6.67129,3.69101);
\definecolor{c}{rgb}{0,0,0};
\colorlet{c}{natgreen};
\draw [c] (6.71559,3.44265) -- (6.71559,3.55662);
\draw [c] (6.71559,3.55662) -- (6.71559,3.6496);
\draw [c] (6.67129,3.55662) -- (6.71559,3.55662);
\draw [c] (6.71559,3.55662) -- (6.7599,3.55662);
\definecolor{c}{rgb}{0,0,0};
\colorlet{c}{natgreen};
\draw [c] (6.80421,3.38595) -- (6.80421,3.50683);
\draw [c] (6.80421,3.50683) -- (6.80421,3.60434);
\draw [c] (6.7599,3.50683) -- (6.80421,3.50683);
\draw [c] (6.80421,3.50683) -- (6.84852,3.50683);
\definecolor{c}{rgb}{0,0,0};
\colorlet{c}{natgreen};
\draw [c] (6.89282,3.38837) -- (6.89282,3.5091);
\draw [c] (6.89282,3.5091) -- (6.89282,3.60651);
\draw [c] (6.84852,3.5091) -- (6.89282,3.5091);
\draw [c] (6.89282,3.5091) -- (6.93713,3.5091);
\definecolor{c}{rgb}{0,0,0};
\colorlet{c}{natgreen};
\draw [c] (6.98144,3.3475) -- (6.98144,3.47318);
\draw [c] (6.98144,3.47318) -- (6.98144,3.57378);
\draw [c] (6.93713,3.47318) -- (6.98144,3.47318);
\draw [c] (6.98144,3.47318) -- (7.02574,3.47318);
\definecolor{c}{rgb}{0,0,0};
\colorlet{c}{natgreen};
\draw [c] (7.07005,2.39988) -- (7.07005,2.70986);
\draw [c] (7.07005,2.70986) -- (7.07005,2.90074);
\draw [c] (7.02574,2.70986) -- (7.07005,2.70986);
\draw [c] (7.07005,2.70986) -- (7.11436,2.70986);
\definecolor{c}{rgb}{0,0,0};
\colorlet{c}{natgreen};
\draw [c] (7.15866,3.27796) -- (7.15866,3.41252);
\draw [c] (7.15866,3.41252) -- (7.15866,3.51872);
\draw [c] (7.11436,3.41252) -- (7.15866,3.41252);
\draw [c] (7.15866,3.41252) -- (7.20297,3.41252);
\definecolor{c}{rgb}{0,0,0};
\colorlet{c}{natgreen};
\draw [c] (7.24728,3.27734) -- (7.24728,3.41197);
\draw [c] (7.24728,3.41197) -- (7.24728,3.51823);
\draw [c] (7.20297,3.41197) -- (7.24728,3.41197);
\draw [c] (7.24728,3.41197) -- (7.29158,3.41197);
\definecolor{c}{rgb}{0,0,0};
\colorlet{c}{natgreen};
\draw [c] (7.33589,3.29526) -- (7.33589,3.42755);
\draw [c] (7.33589,3.42755) -- (7.33589,3.53234);
\draw [c] (7.29158,3.42755) -- (7.33589,3.42755);
\draw [c] (7.33589,3.42755) -- (7.3802,3.42755);
\definecolor{c}{rgb}{0,0,0};
\colorlet{c}{natgreen};
\draw [c] (7.4245,2.95689) -- (7.4245,3.13997);
\draw [c] (7.4245,3.13997) -- (7.4245,3.27415);
\draw [c] (7.3802,3.13997) -- (7.4245,3.13997);
\draw [c] (7.4245,3.13997) -- (7.46881,3.13997);
\definecolor{c}{rgb}{0,0,0};
\colorlet{c}{natgreen};
\draw [c] (7.51312,2.63192) -- (7.51312,2.88466);
\draw [c] (7.51312,2.88466) -- (7.51312,3.05249);
\draw [c] (7.46881,2.88466) -- (7.51312,2.88466);
\draw [c] (7.51312,2.88466) -- (7.55743,2.88466);
\definecolor{c}{rgb}{0,0,0};
\colorlet{c}{natgreen};
\draw [c] (7.60173,3.10883) -- (7.60173,3.26766);
\draw [c] (7.60173,3.26766) -- (7.60173,3.38838);
\draw [c] (7.55743,3.26766) -- (7.60173,3.26766);
\draw [c] (7.60173,3.26766) -- (7.64604,3.26766);
\definecolor{c}{rgb}{0,0,0};
\colorlet{c}{natgreen};
\draw [c] (7.69035,1.68208) -- (7.69035,2.29607);
\draw [c] (7.69035,2.29607) -- (7.69035,2.56555);
\draw [c] (7.64604,2.29607) -- (7.69035,2.29607);
\draw [c] (7.69035,2.29607) -- (7.73465,2.29607);
\definecolor{c}{rgb}{0,0,0};
\colorlet{c}{natgreen};
\draw [c] (7.77896,3.37021) -- (7.77896,3.49311);
\draw [c] (7.77896,3.49311) -- (7.77896,3.59193);
\draw [c] (7.73465,3.49311) -- (7.77896,3.49311);
\draw [c] (7.77896,3.49311) -- (7.82327,3.49311);
\definecolor{c}{rgb}{0,0,0};
\colorlet{c}{natgreen};
\draw [c] (7.86757,2.70656) -- (7.86757,2.94167);
\draw [c] (7.86757,2.94167) -- (7.86757,3.10162);
\draw [c] (7.82327,2.94167) -- (7.86757,2.94167);
\draw [c] (7.86757,2.94167) -- (7.91188,2.94167);
\definecolor{c}{rgb}{0,0,0};
\colorlet{c}{natgreen};
\draw [c] (7.95619,1.24756) -- (7.95619,2.13237);
\draw [c] (7.95619,2.13237) -- (7.95619,2.43724);
\draw [c] (7.91188,2.13237) -- (7.95619,2.13237);
\draw [c] (7.95619,2.13237) -- (8.00049,2.13237);
\definecolor{c}{rgb}{0,0,0};
\colorlet{c}{natgreen};
\draw [c] (8.0448,2.33128) -- (8.0448,2.66863);
\draw [c] (8.0448,2.66863) -- (8.0448,2.86933);
\draw [c] (8.00049,2.66863) -- (8.0448,2.66863);
\draw [c] (8.0448,2.66863) -- (8.08911,2.66863);
\definecolor{c}{rgb}{0,0,0};
\colorlet{c}{natgreen};
\draw [c] (8.13342,1.23328) -- (8.13342,2.12817);
\draw [c] (8.13342,2.12817) -- (8.13342,2.43399);
\draw [c] (8.08911,2.12817) -- (8.13342,2.12817);
\draw [c] (8.13342,2.12817) -- (8.17772,2.12817);
\definecolor{c}{rgb}{0,0,0};
\colorlet{c}{natgreen};
\draw [c] (8.22203,2.37726) -- (8.22203,2.70014);
\draw [c] (8.22203,2.70014) -- (8.22203,2.89574);
\draw [c] (8.17772,2.70014) -- (8.22203,2.70014);
\draw [c] (8.22203,2.70014) -- (8.26634,2.70014);
\definecolor{c}{rgb}{0,0,0};
\colorlet{c}{natgreen};
\draw [c] (8.31064,2.28945) -- (8.31064,2.64049);
\draw [c] (8.31064,2.64049) -- (8.31064,2.84583);
\draw [c] (8.26634,2.64049) -- (8.31064,2.64049);
\draw [c] (8.31064,2.64049) -- (8.35495,2.64049);
\definecolor{c}{rgb}{0,0,0};
\colorlet{c}{natgreen};
\draw [c] (8.39926,1.52931) -- (8.39926,2.2217);
\draw [c] (8.39926,2.2217) -- (8.39926,2.50372);
\draw [c] (8.35495,2.2217) -- (8.39926,2.2217);
\draw [c] (8.39926,2.2217) -- (8.44356,2.2217);
\definecolor{c}{rgb}{0,0,0};
\colorlet{c}{natgreen};
\draw [c] (8.48787,2.69897) -- (8.48787,2.93582);
\draw [c] (8.48787,2.93582) -- (8.48787,3.09656);
\draw [c] (8.44356,2.93582) -- (8.48787,2.93582);
\draw [c] (8.48787,2.93582) -- (8.53218,2.93582);
\definecolor{c}{rgb}{0,0,0};
\colorlet{c}{natgreen};
\draw [c] (8.6651,1.23778) -- (8.6651,2.12949);
\draw [c] (8.6651,2.12949) -- (8.6651,2.43501);
\draw [c] (8.62079,2.12949) -- (8.6651,2.12949);
\draw [c] (8.6651,2.12949) -- (8.70941,2.12949);
\definecolor{c}{rgb}{0,0,0};
\colorlet{c}{natgreen};
\draw [c] (8.75371,0.680516) -- (8.75371,1.90572);
\draw [c] (8.75371,1.90572) -- (8.75371,2.265);
\draw [c] (8.70941,1.90572) -- (8.75371,1.90572);
\draw [c] (8.75371,1.90572) -- (8.79802,1.90572);
\definecolor{c}{rgb}{0,0,0};
\colorlet{c}{natgreen};
\draw [c] (8.84233,2.55842) -- (8.84233,2.82975);
\draw [c] (8.84233,2.82975) -- (8.84233,3.00548);
\draw [c] (8.79802,2.82975) -- (8.84233,2.82975);
\draw [c] (8.84233,2.82975) -- (8.88663,2.82975);
\definecolor{c}{rgb}{0,0,0};
\colorlet{c}{natgreen};
\draw [c] (8.93094,1.03156) -- (8.93094,2.07613);
\draw [c] (8.93094,2.07613) -- (8.93094,2.39391);
\draw [c] (8.88663,2.07613) -- (8.93094,2.07613);
\draw [c] (8.93094,2.07613) -- (8.97525,2.07613);
\definecolor{c}{rgb}{0,0,0};
\colorlet{c}{natgreen};
\draw [c] (9.01955,2.12561) -- (9.01955,2.53533);
\draw [c] (9.01955,2.53533) -- (9.01955,2.75882);
\draw [c] (8.97525,2.53533) -- (9.01955,2.53533);
\draw [c] (9.01955,2.53533) -- (9.06386,2.53533);
\definecolor{c}{rgb}{0,0,0};
\colorlet{c}{natgreen};
\draw [c] (9.10817,0.680516) -- (9.10817,1.8254);
\draw [c] (9.10817,1.8254) -- (9.10817,2.20548);
\draw [c] (9.06386,1.8254) -- (9.10817,1.8254);
\draw [c] (9.10817,1.8254) -- (9.15248,1.8254);
\definecolor{c}{rgb}{0,0,0};
\colorlet{c}{natgreen};
\draw [c] (9.19678,1.66756) -- (9.19678,2.28945);
\draw [c] (9.19678,2.28945) -- (9.19678,2.56029);
\draw [c] (9.15248,2.28945) -- (9.19678,2.28945);
\draw [c] (9.19678,2.28945) -- (9.24109,2.28945);
\definecolor{c}{rgb}{0,0,0};
\colorlet{c}{natgreen};
\draw [c] (9.46262,0.680516) -- (9.46262,1.93841);
\draw [c] (9.46262,1.93841) -- (9.46262,2.28945);
\draw [c] (9.41832,1.93841) -- (9.46262,1.93841);
\draw [c] (9.46262,1.93841) -- (9.50693,1.93841);
\definecolor{c}{rgb}{0,0,0};
\colorlet{c}{natgreen};
\draw [c] (9.90569,0.680516) -- (9.90569,1.38202);
\draw [c] (9.90569,1.38202) -- (9.90569,1.89102);
\draw [c] (9.86139,1.38202) -- (9.90569,1.38202);
\draw [c] (9.90569,1.38202) -- (9.95,1.38202);
\definecolor{c}{rgb}{0,0,0};
\draw [c] (1,0.680516) -- (9.95,0.680516);
\draw [anchor= east] (9.95,0.108883) node[color=c, rotate=0]{$M_{\gamma\gamma}\text{ [GeV]}$};
\draw [c] (1,0.863234) -- (1,0.680516);
\draw [c] (1.44307,0.771875) -- (1.44307,0.680516);
\draw [c] (1.88614,0.771875) -- (1.88614,0.680516);
\draw [c] (2.32921,0.771875) -- (2.32921,0.680516);
\draw [c] (2.77228,0.863234) -- (2.77228,0.680516);
\draw [c] (3.21535,0.771875) -- (3.21535,0.680516);
\draw [c] (3.65842,0.771875) -- (3.65842,0.680516);
\draw [c] (4.10149,0.771875) -- (4.10149,0.680516);
\draw [c] (4.54455,0.863234) -- (4.54455,0.680516);
\draw [c] (4.98762,0.771875) -- (4.98762,0.680516);
\draw [c] (5.43069,0.771875) -- (5.43069,0.680516);
\draw [c] (5.87376,0.771875) -- (5.87376,0.680516);
\draw [c] (6.31683,0.863234) -- (6.31683,0.680516);
\draw [c] (6.7599,0.771875) -- (6.7599,0.680516);
\draw [c] (7.20297,0.771875) -- (7.20297,0.680516);
\draw [c] (7.64604,0.771875) -- (7.64604,0.680516);
\draw [c] (8.08911,0.863234) -- (8.08911,0.680516);
\draw [c] (8.53218,0.771875) -- (8.53218,0.680516);
\draw [c] (8.97525,0.771875) -- (8.97525,0.680516);
\draw [c] (9.41832,0.771875) -- (9.41832,0.680516);
\draw [c] (9.86139,0.863234) -- (9.86139,0.680516);
\draw [c] (9.86139,0.863234) -- (9.86139,0.680516);
\draw [anchor=base] (1,0.353868) node[color=c, rotate=0]{0};
\draw [anchor=base] (2.77228,0.353868) node[color=c, rotate=0]{200};
\draw [anchor=base] (4.54455,0.353868) node[color=c, rotate=0]{400};
\draw [anchor=base] (6.31683,0.353868) node[color=c, rotate=0]{600};
\draw [anchor=base] (8.08911,0.353868) node[color=c, rotate=0]{800};
\draw [anchor=base] (9.86139,0.353868) node[color=c, rotate=0]{1000};
\draw [c] (1,0.680516) -- (1,6.73711);
\draw [anchor= east] (-0.12,6.73711) node[color=c, rotate=90]{Number of events};
\draw [c] (1.1335,0.718918) -- (1,0.718918);
\draw [c] (1.267,0.772277) -- (1,0.772277);
\draw [anchor= east] (0.922,0.772277) node[color=c, rotate=0]{$10^{-1}$};
\draw [c] (1.1335,1.12332) -- (1,1.12332);
\draw [c] (1.1335,1.32866) -- (1,1.32866);
\draw [c] (1.1335,1.47436) -- (1,1.47436);
\draw [c] (1.1335,1.58737) -- (1,1.58737);
\draw [c] (1.1335,1.6797) -- (1,1.6797);
\draw [c] (1.1335,1.75777) -- (1,1.75777);
\draw [c] (1.1335,1.8254) -- (1,1.8254);
\draw [c] (1.1335,1.88505) -- (1,1.88505);
\draw [c] (1.267,1.93841) -- (1,1.93841);
\draw [anchor= east] (0.922,1.93841) node[color=c, rotate=0]{1};
\draw [c] (1.1335,2.28945) -- (1,2.28945);
\draw [c] (1.1335,2.49479) -- (1,2.49479);
\draw [c] (1.1335,2.64049) -- (1,2.64049);
\draw [c] (1.1335,2.7535) -- (1,2.7535);
\draw [c] (1.1335,2.84583) -- (1,2.84583);
\draw [c] (1.1335,2.9239) -- (1,2.9239);
\draw [c] (1.1335,2.99153) -- (1,2.99153);
\draw [c] (1.1335,3.05118) -- (1,3.05118);
\draw [c] (1.267,3.10454) -- (1,3.10454);
\draw [anchor= east] (0.922,3.10454) node[color=c, rotate=0]{10};
\draw [c] (1.1335,3.45558) -- (1,3.45558);
\draw [c] (1.1335,3.66092) -- (1,3.66092);
\draw [c] (1.1335,3.80662) -- (1,3.80662);
\draw [c] (1.1335,3.91963) -- (1,3.91963);
\draw [c] (1.1335,4.01196) -- (1,4.01196);
\draw [c] (1.1335,4.09003) -- (1,4.09003);
\draw [c] (1.1335,4.15766) -- (1,4.15766);
\draw [c] (1.1335,4.21731) -- (1,4.21731);
\draw [c] (1.267,4.27067) -- (1,4.27067);
\draw [anchor= east] (0.922,4.27067) node[color=c, rotate=0]{$10^{2}$};
\draw [c] (1.1335,4.62171) -- (1,4.62171);
\draw [c] (1.1335,4.82705) -- (1,4.82705);
\draw [c] (1.1335,4.97275) -- (1,4.97275);
\draw [c] (1.1335,5.08576) -- (1,5.08576);
\draw [c] (1.1335,5.1781) -- (1,5.1781);
\draw [c] (1.1335,5.25616) -- (1,5.25616);
\draw [c] (1.1335,5.32379) -- (1,5.32379);
\draw [c] (1.1335,5.38344) -- (1,5.38344);
\draw [c] (1.267,5.4368) -- (1,5.4368);
\draw [anchor= east] (0.922,5.4368) node[color=c, rotate=0]{$10^{3}$};
\draw [c] (1.1335,5.78784) -- (1,5.78784);
\draw [c] (1.1335,5.99319) -- (1,5.99319);
\draw [c] (1.1335,6.13888) -- (1,6.13888);
\draw [c] (1.1335,6.25189) -- (1,6.25189);
\draw [c] (1.1335,6.34423) -- (1,6.34423);
\draw [c] (1.1335,6.42229) -- (1,6.42229);
\draw [c] (1.1335,6.48992) -- (1,6.48992);
\draw [c] (1.1335,6.54957) -- (1,6.54957);
\draw [c] (1.267,6.60293) -- (1,6.60293);
\draw [anchor= east] (0.922,6.60293) node[color=c, rotate=0]{$10^{4}$};
\colorlet{c}{natcomp!70};
\draw [c] (2.36687,6.34851) -- (2.44219,6.27507) -- (2.51751,6.20302) -- (2.59283,6.1323) -- (2.66816,6.06287) -- (2.74348,5.99466) -- (2.8188,5.92763) -- (2.89412,5.86173) -- (2.96944,5.7969) -- (3.04476,5.73311) -- (3.12009,5.6703)
 -- (3.19541,5.60845) -- (3.27073,5.5475) -- (3.34605,5.48743) -- (3.42137,5.42819) -- (3.4967,5.36977) -- (3.57202,5.31211) -- (3.64734,5.25521) -- (3.72266,5.19902) -- (3.79798,5.14352) -- (3.8733,5.08869) -- (3.94863,5.03451) -- (4.02395,4.98094)
 -- (4.09927,4.92798) -- (4.17459,4.87559) -- (4.24991,4.82377) -- (4.32524,4.77249) -- (4.40056,4.72173) -- (4.47588,4.67148) -- (4.5512,4.62172) -- (4.62652,4.57244) -- (4.70184,4.52363) -- (4.77717,4.47526) -- (4.85249,4.42733) --
 (4.92781,4.37982) -- (5.00313,4.33272) -- (5.07845,4.28602) -- (5.15377,4.23971) -- (5.2291,4.19378) -- (5.30442,4.14821) -- (5.37974,4.103) -- (5.45506,4.05814) -- (5.53038,4.01362) -- (5.60571,3.96943) -- (5.68103,3.92556) -- (5.75635,3.88201) --
 (5.83167,3.83876) -- (5.90699,3.79581) -- (5.98231,3.75315) -- (6.05764,3.71077);
\draw [c] (6.05764,3.71077) -- (6.13296,3.66868) -- (6.20828,3.62685) -- (6.2836,3.58529) -- (6.35892,3.54399) -- (6.43425,3.50294) -- (6.50957,3.46213) -- (6.58489,3.42157) -- (6.66021,3.38125) -- (6.73553,3.34115) --
 (6.81085,3.30129) -- (6.88618,3.26164) -- (6.9615,3.22221) -- (7.03682,3.18299) -- (7.11214,3.14398) -- (7.18746,3.10517) -- (7.26278,3.06656) -- (7.33811,3.02815) -- (7.41343,2.98993) -- (7.48875,2.95189) -- (7.56407,2.91404) -- (7.63939,2.87637)
 -- (7.71472,2.83887) -- (7.79004,2.80155) -- (7.86536,2.7644) -- (7.94068,2.72741) -- (8.016,2.69059) -- (8.09132,2.65392) -- (8.16665,2.61742) -- (8.24197,2.58107) -- (8.31729,2.54487) -- (8.39261,2.50882) -- (8.46793,2.47291) -- (8.54325,2.43715)
 -- (8.61858,2.40153) -- (8.6939,2.36605) -- (8.76922,2.3307) -- (8.84454,2.29549) -- (8.91986,2.26041) -- (8.99519,2.22545) -- (9.07051,2.19063) -- (9.14583,2.15593) -- (9.22115,2.12135) -- (9.29647,2.08689) -- (9.37179,2.05255) -- (9.44712,2.01832)
 -- (9.52244,1.98421) -- (9.59776,1.95021) -- (9.67308,1.91633) -- (9.7484,1.88255);
\draw [c] (9.7484,1.88255) -- (9.82373,1.84888);
\draw [c] (2.36687,6.34851) -- (2.44219,6.27507) -- (2.51751,6.20302) -- (2.59283,6.1323) -- (2.66816,6.06287) -- (2.74348,5.99466) -- (2.8188,5.92763) -- (2.89412,5.86173) -- (2.96944,5.7969) -- (3.04476,5.73311) -- (3.12009,5.6703)
 -- (3.19541,5.60845) -- (3.27073,5.5475) -- (3.34605,5.48743) -- (3.42137,5.42819) -- (3.4967,5.36977) -- (3.57202,5.31211) -- (3.64734,5.25521) -- (3.72266,5.19902) -- (3.79798,5.14352) -- (3.8733,5.08869) -- (3.94863,5.03451) -- (4.02395,4.98094)
 -- (4.09927,4.92798) -- (4.17459,4.87559) -- (4.24991,4.82377) -- (4.32524,4.77249) -- (4.40056,4.72173) -- (4.47588,4.67148) -- (4.5512,4.62172) -- (4.62652,4.57244) -- (4.70184,4.52363) -- (4.77717,4.47526) -- (4.85249,4.42733) --
 (4.92781,4.37982) -- (5.00313,4.33272) -- (5.07845,4.28602) -- (5.15377,4.23971) -- (5.2291,4.19378) -- (5.30442,4.14821) -- (5.37974,4.103) -- (5.45506,4.05814) -- (5.53038,4.01362) -- (5.60571,3.96943) -- (5.68103,3.92556) -- (5.75635,3.88201) --
 (5.83167,3.83876) -- (5.90699,3.79581) -- (5.98231,3.75315) -- (6.05764,3.71077);
\draw [c] (6.05764,3.71077) -- (6.13296,3.66868) -- (6.20828,3.62685) -- (6.2836,3.58529) -- (6.35892,3.54399) -- (6.43425,3.50294) -- (6.50957,3.46213) -- (6.58489,3.42157) -- (6.66021,3.38125) -- (6.73553,3.34115) --
 (6.81085,3.30129) -- (6.88618,3.26164) -- (6.9615,3.22221) -- (7.03682,3.18299) -- (7.11214,3.14398) -- (7.18746,3.10517) -- (7.26278,3.06656) -- (7.33811,3.02815) -- (7.41343,2.98993) -- (7.48875,2.95189) -- (7.56407,2.91404) -- (7.63939,2.87637)
 -- (7.71472,2.83887) -- (7.79004,2.80155) -- (7.86536,2.7644) -- (7.94068,2.72741) -- (8.016,2.69059) -- (8.09132,2.65392) -- (8.16665,2.61742) -- (8.24197,2.58107) -- (8.31729,2.54487) -- (8.39261,2.50882) -- (8.46793,2.47291) -- (8.54325,2.43715)
 -- (8.61858,2.40153) -- (8.6939,2.36605) -- (8.76922,2.3307) -- (8.84454,2.29549) -- (8.91986,2.26041) -- (8.99519,2.22545) -- (9.07051,2.19063) -- (9.14583,2.15593) -- (9.22115,2.12135) -- (9.29647,2.08689) -- (9.37179,2.05255) -- (9.44712,2.01832)
 -- (9.52244,1.98421) -- (9.59776,1.95021) -- (9.67308,1.91633) -- (9.7484,1.88255);
\draw [c] (9.7484,1.88255) -- (9.82373,1.84888);
\end{tikzpicture}

\end{infilsf}
\end{minipage}
\begin{minipage}[b]{.3\textwidth}
\caption{Extrapolating the truncated backgrounds by fitting a function. The fit has $\chi^2$ / ndf = 237.2 / 74.}\label{bckfit}
\end{minipage}
\end{figure}

\section{The polynomial coefficients}
Now satisfied with our Monte Carlo samples, we turn toward extracting the coefficients to the polynomials that will allow us to vary $\Lambda$. As is evident from the plots of the distributions in figure~\ref{simfit}, the number of expected events in each bin is subject to relatively large statistical fluctuations. Since the coefficients for each polynomial is determined solely from these three points, the resulting quadratic function may look drastically different from bin to bin. One approach to reducing the statistical fluctuations is to simply combine sets of adjacent bins, rebinning, or to apply some form of smoothing procedure. Here, however, we will attempt to fit a function to each distribution. In addition to smoothing out fluctuations between bins, fitting functions allows us to incorporate some additional knowledge about the relationship between these distributions into our extrapolation of their shape.

Since we have assumed constructive interference between the new interaction and the Standard Model, we know that a distribution which has a contribution from the new interaction, at any strength, must have a number of expected events at least equal to the number of events expected in the Standard Model case at any point. Also, the Standard Model contributes equally in each distribution, thus the shape of each distribution must be the shape of the Standard Model contribution plus an additional contribution. We select for the shape of the Standard Model distribution the function
\(f_\textit{SM}(x)=p_1\left(\frac{1-x}{8000}\right)^{p_2}\left(\frac{x}{8000}\right)^{-\left[p_3+p_4\ln\frac{x}{8000}\right]},\)
the parameters of which must be shared among all three functions. This is the same type of expression as was used to fit the shape of the background distribution in the previous section. The additional contribution from the contact interaction does not follow this shape of distribution.
\begin{edit}
To describe the shape of the additional contribution, we select, among all possible functional dependencies, an expression with a small number of parameters, which once fitted to the distribution, describes the shape well, with a good $\chi^2$.
\end{edit}

We describe the shape of the contribution from the new interaction with functions of the form
\(f_{\Lambda=1.00/0.75}(x)=f_\textit{SM}(x)+p_{5/10}\exp\left[-\half\left(\frac{x-p_{6/11}}{p_{7/12}}\right)^2\right]+\exp\left[p_{8/13}\frac{x}{8000}-p_{9/14}\right],\)
where we have attempted to indicate the parameter numbers belonging to the function describing the contribution at $\Lambda=1.00$ TeV resp. 0.75 TeV with two indexes separated by a slash. This is effectively a Gaussian plus an exponential. We fit the distributions in the range above 150 GeV, which clears the effects of the minimum $p_T$ cuts, and below 3\,000 GeV, where the SM sample runs out of statistics.

The results of fitting these functions simultaneously is shown in figure~\ref{simfit}.


\begin{figure}[hbt]
\begin{infilsf}\tiny
\begin{tikzpicture}[x=.092\textwidth,y=.092\textwidth]
\pgfdeclareplotmark{cross} {
\pgfpathmoveto{\pgfpoint{-0.3\pgfplotmarksize}{\pgfplotmarksize}}
\pgfpathlineto{\pgfpoint{+0.3\pgfplotmarksize}{\pgfplotmarksize}}
\pgfpathlineto{\pgfpoint{+0.3\pgfplotmarksize}{0.3\pgfplotmarksize}}
\pgfpathlineto{\pgfpoint{+1\pgfplotmarksize}{0.3\pgfplotmarksize}}
\pgfpathlineto{\pgfpoint{+1\pgfplotmarksize}{-0.3\pgfplotmarksize}}
\pgfpathlineto{\pgfpoint{+0.3\pgfplotmarksize}{-0.3\pgfplotmarksize}}
\pgfpathlineto{\pgfpoint{+0.3\pgfplotmarksize}{-1.\pgfplotmarksize}}
\pgfpathlineto{\pgfpoint{-0.3\pgfplotmarksize}{-1.\pgfplotmarksize}}
\pgfpathlineto{\pgfpoint{-0.3\pgfplotmarksize}{-0.3\pgfplotmarksize}}
\pgfpathlineto{\pgfpoint{-1.\pgfplotmarksize}{-0.3\pgfplotmarksize}}
\pgfpathlineto{\pgfpoint{-1.\pgfplotmarksize}{0.3\pgfplotmarksize}}
\pgfpathlineto{\pgfpoint{-0.3\pgfplotmarksize}{0.3\pgfplotmarksize}}
\pgfpathclose
\pgfusepathqstroke
}
\pgfdeclareplotmark{cross*} {
\pgfpathmoveto{\pgfpoint{-0.3\pgfplotmarksize}{\pgfplotmarksize}}
\pgfpathlineto{\pgfpoint{+0.3\pgfplotmarksize}{\pgfplotmarksize}}
\pgfpathlineto{\pgfpoint{+0.3\pgfplotmarksize}{0.3\pgfplotmarksize}}
\pgfpathlineto{\pgfpoint{+1\pgfplotmarksize}{0.3\pgfplotmarksize}}
\pgfpathlineto{\pgfpoint{+1\pgfplotmarksize}{-0.3\pgfplotmarksize}}
\pgfpathlineto{\pgfpoint{+0.3\pgfplotmarksize}{-0.3\pgfplotmarksize}}
\pgfpathlineto{\pgfpoint{+0.3\pgfplotmarksize}{-1.\pgfplotmarksize}}
\pgfpathlineto{\pgfpoint{-0.3\pgfplotmarksize}{-1.\pgfplotmarksize}}
\pgfpathlineto{\pgfpoint{-0.3\pgfplotmarksize}{-0.3\pgfplotmarksize}}
\pgfpathlineto{\pgfpoint{-1.\pgfplotmarksize}{-0.3\pgfplotmarksize}}
\pgfpathlineto{\pgfpoint{-1.\pgfplotmarksize}{0.3\pgfplotmarksize}}
\pgfpathlineto{\pgfpoint{-0.3\pgfplotmarksize}{0.3\pgfplotmarksize}}
\pgfpathclose
\pgfusepathqfillstroke
}
\pgfdeclareplotmark{newstar} {
\pgfpathmoveto{\pgfqpoint{0pt}{\pgfplotmarksize}}
\pgfpathlineto{\pgfqpointpolar{44}{0.5\pgfplotmarksize}}
\pgfpathlineto{\pgfqpointpolar{18}{\pgfplotmarksize}}
\pgfpathlineto{\pgfqpointpolar{-20}{0.5\pgfplotmarksize}}
\pgfpathlineto{\pgfqpointpolar{-54}{\pgfplotmarksize}}
\pgfpathlineto{\pgfqpointpolar{-90}{0.5\pgfplotmarksize}}
\pgfpathlineto{\pgfqpointpolar{234}{\pgfplotmarksize}}
\pgfpathlineto{\pgfqpointpolar{198}{0.5\pgfplotmarksize}}
\pgfpathlineto{\pgfqpointpolar{162}{\pgfplotmarksize}}
\pgfpathlineto{\pgfqpointpolar{134}{0.5\pgfplotmarksize}}
\pgfpathclose
\pgfusepathqstroke
}
\pgfdeclareplotmark{newstar*} {
\pgfpathmoveto{\pgfqpoint{0pt}{\pgfplotmarksize}}
\pgfpathlineto{\pgfqpointpolar{44}{0.5\pgfplotmarksize}}
\pgfpathlineto{\pgfqpointpolar{18}{\pgfplotmarksize}}
\pgfpathlineto{\pgfqpointpolar{-20}{0.5\pgfplotmarksize}}
\pgfpathlineto{\pgfqpointpolar{-54}{\pgfplotmarksize}}
\pgfpathlineto{\pgfqpointpolar{-90}{0.5\pgfplotmarksize}}
\pgfpathlineto{\pgfqpointpolar{234}{\pgfplotmarksize}}
\pgfpathlineto{\pgfqpointpolar{198}{0.5\pgfplotmarksize}}
\pgfpathlineto{\pgfqpointpolar{162}{\pgfplotmarksize}}
\pgfpathlineto{\pgfqpointpolar{134}{0.5\pgfplotmarksize}}
\pgfpathclose
\pgfusepathqfillstroke
}
\definecolor{c}{rgb}{1,1,1};
\draw [color=c, fill=c] (0,0) rectangle (10,5.96817);
\draw [color=c, fill=c] (1,0.596817) rectangle (9.95,5.90849);
\definecolor{c}{rgb}{0,0,0};
\draw [c] (1,0.596817) -- (1,5.90849) -- (9.95,5.90849) -- (9.95,0.596817) -- (1,0.596817);
\definecolor{c}{rgb}{1,1,1};
\draw [color=c, fill=c] (1,0.596817) rectangle (9.95,5.90849);
\definecolor{c}{rgb}{0,0,0};
\draw [c] (1,0.596817) -- (1,5.90849) -- (9.95,5.90849) -- (9.95,0.596817) -- (1,0.596817);
\colorlet{c}{kugray};
\draw [c] (1.13336,3.41516) -- (1.13336,3.80777);
\draw [c] (1.13336,3.80777) -- (1.13336,3.97485);
\draw [c] (1.11854,3.80777) -- (1.13336,3.80777);
\draw [c] (1.13336,3.80777) -- (1.14818,3.80777);
\definecolor{c}{rgb}{0,0,0};
\colorlet{c}{kugray};
\draw [c] (1.163,3.33018) -- (1.163,3.70859);
\draw [c] (1.163,3.70859) -- (1.163,3.87329);
\draw [c] (1.14818,3.70859) -- (1.163,3.70859);
\draw [c] (1.163,3.70859) -- (1.17781,3.70859);
\definecolor{c}{rgb}{0,0,0};
\colorlet{c}{kugray};
\draw [c] (1.22227,3.26026) -- (1.22227,3.80576);
\draw [c] (1.22227,3.80576) -- (1.22227,3.99181);
\draw [c] (1.20745,3.80576) -- (1.22227,3.80576);
\draw [c] (1.22227,3.80576) -- (1.23709,3.80576);
\definecolor{c}{rgb}{0,0,0};
\colorlet{c}{kugray};
\draw [c] (1.2519,3.59515) -- (1.2519,3.86294);
\draw [c] (1.2519,3.86294) -- (1.2519,4.00395);
\draw [c] (1.23709,3.86294) -- (1.2519,3.86294);
\draw [c] (1.2519,3.86294) -- (1.26672,3.86294);
\definecolor{c}{rgb}{0,0,0};
\colorlet{c}{kugray};
\draw [c] (1.28154,3.81131) -- (1.28154,4.05468);
\draw [c] (1.28154,4.05468) -- (1.28154,4.18889);
\draw [c] (1.26672,4.05468) -- (1.28154,4.05468);
\draw [c] (1.28154,4.05468) -- (1.29636,4.05468);
\definecolor{c}{rgb}{0,0,0};
\colorlet{c}{kugray};
\draw [c] (1.31118,5.36749) -- (1.31118,5.38512);
\draw [c] (1.31118,5.38512) -- (1.31118,5.40179);
\draw [c] (1.29636,5.38512) -- (1.31118,5.38512);
\draw [c] (1.31118,5.38512) -- (1.32599,5.38512);
\definecolor{c}{rgb}{0,0,0};
\colorlet{c}{kugray};
\draw [c] (1.34081,5.61993) -- (1.34081,5.63126);
\draw [c] (1.34081,5.63126) -- (1.34081,5.6422);
\draw [c] (1.32599,5.63126) -- (1.34081,5.63126);
\draw [c] (1.34081,5.63126) -- (1.35563,5.63126);
\definecolor{c}{rgb}{0,0,0};
\colorlet{c}{kugray};
\draw [c] (1.37045,5.69182) -- (1.37045,5.70194);
\draw [c] (1.37045,5.70194) -- (1.37045,5.71174);
\draw [c] (1.35563,5.70194) -- (1.37045,5.70194);
\draw [c] (1.37045,5.70194) -- (1.38526,5.70194);
\definecolor{c}{rgb}{0,0,0};
\colorlet{c}{kugray};
\draw [c] (1.40008,5.67982) -- (1.40008,5.69024);
\draw [c] (1.40008,5.69024) -- (1.40008,5.70031);
\draw [c] (1.38526,5.69024) -- (1.40008,5.69024);
\draw [c] (1.40008,5.69024) -- (1.4149,5.69024);
\definecolor{c}{rgb}{0,0,0};
\colorlet{c}{kugray};
\draw [c] (1.42972,5.65827) -- (1.42972,5.66928);
\draw [c] (1.42972,5.66928) -- (1.42972,5.67991);
\draw [c] (1.4149,5.66928) -- (1.42972,5.66928);
\draw [c] (1.42972,5.66928) -- (1.44454,5.66928);
\definecolor{c}{rgb}{0,0,0};
\colorlet{c}{kugray};
\draw [c] (1.45935,5.62) -- (1.45935,5.63141);
\draw [c] (1.45935,5.63141) -- (1.45935,5.6424);
\draw [c] (1.44454,5.63141) -- (1.45935,5.63141);
\draw [c] (1.45935,5.63141) -- (1.47417,5.63141);
\definecolor{c}{rgb}{0,0,0};
\colorlet{c}{kugray};
\draw [c] (1.48899,5.58164) -- (1.48899,5.5938);
\draw [c] (1.48899,5.5938) -- (1.48899,5.60549);
\draw [c] (1.47417,5.5938) -- (1.48899,5.5938);
\draw [c] (1.48899,5.5938) -- (1.50381,5.5938);
\definecolor{c}{rgb}{0,0,0};
\colorlet{c}{kugray};
\draw [c] (1.51863,5.53046) -- (1.51863,5.54392);
\draw [c] (1.51863,5.54392) -- (1.51863,5.55682);
\draw [c] (1.50381,5.54392) -- (1.51863,5.54392);
\draw [c] (1.51863,5.54392) -- (1.53344,5.54392);
\definecolor{c}{rgb}{0,0,0};
\colorlet{c}{kugray};
\draw [c] (1.54826,5.46022) -- (1.54826,5.47505);
\draw [c] (1.54826,5.47505) -- (1.54826,5.48921);
\draw [c] (1.53344,5.47505) -- (1.54826,5.47505);
\draw [c] (1.54826,5.47505) -- (1.56308,5.47505);
\definecolor{c}{rgb}{0,0,0};
\colorlet{c}{kugray};
\draw [c] (1.5779,5.41506) -- (1.5779,5.43093);
\draw [c] (1.5779,5.43093) -- (1.5779,5.44602);
\draw [c] (1.56308,5.43093) -- (1.5779,5.43093);
\draw [c] (1.5779,5.43093) -- (1.59272,5.43093);
\definecolor{c}{rgb}{0,0,0};
\colorlet{c}{kugray};
\draw [c] (1.60753,5.38705) -- (1.60753,5.40429);
\draw [c] (1.60753,5.40429) -- (1.60753,5.42061);
\draw [c] (1.59272,5.40429) -- (1.60753,5.40429);
\draw [c] (1.60753,5.40429) -- (1.62235,5.40429);
\definecolor{c}{rgb}{0,0,0};
\colorlet{c}{kugray};
\draw [c] (1.63717,5.33118) -- (1.63717,5.34945);
\draw [c] (1.63717,5.34945) -- (1.63717,5.36669);
\draw [c] (1.62235,5.34945) -- (1.63717,5.34945);
\draw [c] (1.63717,5.34945) -- (1.65199,5.34945);
\definecolor{c}{rgb}{0,0,0};
\colorlet{c}{kugray};
\draw [c] (1.6668,5.33501) -- (1.6668,5.35318);
\draw [c] (1.6668,5.35318) -- (1.6668,5.37033);
\draw [c] (1.65199,5.35318) -- (1.6668,5.35318);
\draw [c] (1.6668,5.35318) -- (1.68162,5.35318);
\definecolor{c}{rgb}{0,0,0};
\colorlet{c}{kugray};
\draw [c] (1.69644,5.27038) -- (1.69644,5.29099);
\draw [c] (1.69644,5.29099) -- (1.69644,5.3103);
\draw [c] (1.68162,5.29099) -- (1.69644,5.29099);
\draw [c] (1.69644,5.29099) -- (1.71126,5.29099);
\definecolor{c}{rgb}{0,0,0};
\colorlet{c}{kugray};
\draw [c] (1.72608,5.20804) -- (1.72608,5.23001);
\draw [c] (1.72608,5.23001) -- (1.72608,5.25052);
\draw [c] (1.71126,5.23001) -- (1.72608,5.23001);
\draw [c] (1.72608,5.23001) -- (1.74089,5.23001);
\definecolor{c}{rgb}{0,0,0};
\colorlet{c}{kugray};
\draw [c] (1.75571,5.08901) -- (1.75571,5.11549);
\draw [c] (1.75571,5.11549) -- (1.75571,5.13987);
\draw [c] (1.74089,5.11549) -- (1.75571,5.11549);
\draw [c] (1.75571,5.11549) -- (1.77053,5.11549);
\definecolor{c}{rgb}{0,0,0};
\colorlet{c}{kugray};
\draw [c] (1.78535,5.10181) -- (1.78535,5.12931);
\draw [c] (1.78535,5.12931) -- (1.78535,5.15455);
\draw [c] (1.77053,5.12931) -- (1.78535,5.12931);
\draw [c] (1.78535,5.12931) -- (1.80017,5.12931);
\definecolor{c}{rgb}{0,0,0};
\colorlet{c}{kugray};
\draw [c] (1.81498,5.06396) -- (1.81498,5.09313);
\draw [c] (1.81498,5.09313) -- (1.81498,5.11978);
\draw [c] (1.80017,5.09313) -- (1.81498,5.09313);
\draw [c] (1.81498,5.09313) -- (1.8298,5.09313);
\definecolor{c}{rgb}{0,0,0};
\colorlet{c}{kugray};
\draw [c] (1.84462,5.06537) -- (1.84462,5.09456);
\draw [c] (1.84462,5.09456) -- (1.84462,5.12123);
\draw [c] (1.8298,5.09456) -- (1.84462,5.09456);
\draw [c] (1.84462,5.09456) -- (1.85944,5.09456);
\definecolor{c}{rgb}{0,0,0};
\colorlet{c}{kugray};
\draw [c] (1.87425,4.99725) -- (1.87425,5.02835);
\draw [c] (1.87425,5.02835) -- (1.87425,5.05659);
\draw [c] (1.85944,5.02835) -- (1.87425,5.02835);
\draw [c] (1.87425,5.02835) -- (1.88907,5.02835);
\definecolor{c}{rgb}{0,0,0};
\colorlet{c}{kugray};
\draw [c] (1.90389,4.95832) -- (1.90389,4.99102);
\draw [c] (1.90389,4.99102) -- (1.90389,5.02058);
\draw [c] (1.88907,4.99102) -- (1.90389,4.99102);
\draw [c] (1.90389,4.99102) -- (1.91871,4.99102);
\definecolor{c}{rgb}{0,0,0};
\colorlet{c}{kugray};
\draw [c] (1.93353,4.85544) -- (1.93353,4.89383);
\draw [c] (1.93353,4.89383) -- (1.93353,4.92796);
\draw [c] (1.91871,4.89383) -- (1.93353,4.89383);
\draw [c] (1.93353,4.89383) -- (1.94834,4.89383);
\definecolor{c}{rgb}{0,0,0};
\colorlet{c}{kugray};
\draw [c] (1.96316,4.88333) -- (1.96316,4.92161);
\draw [c] (1.96316,4.92161) -- (1.96316,4.95565);
\draw [c] (1.94834,4.92161) -- (1.96316,4.92161);
\draw [c] (1.96316,4.92161) -- (1.97798,4.92161);
\definecolor{c}{rgb}{0,0,0};
\colorlet{c}{kugray};
\draw [c] (1.9928,4.85825) -- (1.9928,4.89733);
\draw [c] (1.9928,4.89733) -- (1.9928,4.93201);
\draw [c] (1.97798,4.89733) -- (1.9928,4.89733);
\draw [c] (1.9928,4.89733) -- (2.00762,4.89733);
\definecolor{c}{rgb}{0,0,0};
\colorlet{c}{kugray};
\draw [c] (2.02243,4.78026) -- (2.02243,4.82199);
\draw [c] (2.02243,4.82199) -- (2.02243,4.85874);
\draw [c] (2.00762,4.82199) -- (2.02243,4.82199);
\draw [c] (2.02243,4.82199) -- (2.03725,4.82199);
\definecolor{c}{rgb}{0,0,0};
\colorlet{c}{kugray};
\draw [c] (2.05207,4.74394) -- (2.05207,4.7911);
\draw [c] (2.05207,4.7911) -- (2.05207,4.83199);
\draw [c] (2.03725,4.7911) -- (2.05207,4.7911);
\draw [c] (2.05207,4.7911) -- (2.06689,4.7911);
\definecolor{c}{rgb}{0,0,0};
\colorlet{c}{kugray};
\draw [c] (2.08171,4.69059) -- (2.08171,4.74618);
\draw [c] (2.08171,4.74618) -- (2.08171,4.79324);
\draw [c] (2.06689,4.74618) -- (2.08171,4.74618);
\draw [c] (2.08171,4.74618) -- (2.09652,4.74618);
\definecolor{c}{rgb}{0,0,0};
\colorlet{c}{kugray};
\draw [c] (2.11134,4.64498) -- (2.11134,4.70439);
\draw [c] (2.11134,4.70439) -- (2.11134,4.75417);
\draw [c] (2.09652,4.70439) -- (2.11134,4.70439);
\draw [c] (2.11134,4.70439) -- (2.12616,4.70439);
\definecolor{c}{rgb}{0,0,0};
\colorlet{c}{kugray};
\draw [c] (2.14098,4.66916) -- (2.14098,4.726);
\draw [c] (2.14098,4.726) -- (2.14098,4.77396);
\draw [c] (2.12616,4.726) -- (2.14098,4.726);
\draw [c] (2.14098,4.726) -- (2.15579,4.726);
\definecolor{c}{rgb}{0,0,0};
\colorlet{c}{kugray};
\draw [c] (2.17061,4.71202) -- (2.17061,4.76221);
\draw [c] (2.17061,4.76221) -- (2.17061,4.80536);
\draw [c] (2.15579,4.76221) -- (2.17061,4.76221);
\draw [c] (2.17061,4.76221) -- (2.18543,4.76221);
\definecolor{c}{rgb}{0,0,0};
\colorlet{c}{kugray};
\draw [c] (2.20025,4.43289) -- (2.20025,4.50711);
\draw [c] (2.20025,4.50711) -- (2.20025,4.56686);
\draw [c] (2.18543,4.50711) -- (2.20025,4.50711);
\draw [c] (2.20025,4.50711) -- (2.21507,4.50711);
\definecolor{c}{rgb}{0,0,0};
\colorlet{c}{kugray};
\draw [c] (2.22988,4.60564) -- (2.22988,4.66562);
\draw [c] (2.22988,4.66562) -- (2.22988,4.7158);
\draw [c] (2.21507,4.66562) -- (2.22988,4.66562);
\draw [c] (2.22988,4.66562) -- (2.2447,4.66562);
\definecolor{c}{rgb}{0,0,0};
\colorlet{c}{kugray};
\draw [c] (2.25952,4.64159) -- (2.25952,4.70326);
\draw [c] (2.25952,4.70326) -- (2.25952,4.75461);
\draw [c] (2.2447,4.70326) -- (2.25952,4.70326);
\draw [c] (2.25952,4.70326) -- (2.27434,4.70326);
\definecolor{c}{rgb}{0,0,0};
\colorlet{c}{kugray};
\draw [c] (2.28916,4.56718) -- (2.28916,4.57808);
\draw [c] (2.28916,4.57808) -- (2.28916,4.5886);
\draw [c] (2.27434,4.57808) -- (2.28916,4.57808);
\draw [c] (2.28916,4.57808) -- (2.30397,4.57808);
\definecolor{c}{rgb}{0,0,0};
\colorlet{c}{kugray};
\draw [c] (2.31879,4.51802) -- (2.31879,4.5299);
\draw [c] (2.31879,4.5299) -- (2.31879,4.54134);
\draw [c] (2.30397,4.5299) -- (2.31879,4.5299);
\draw [c] (2.31879,4.5299) -- (2.33361,4.5299);
\definecolor{c}{rgb}{0,0,0};
\colorlet{c}{kugray};
\draw [c] (2.34843,4.51942) -- (2.34843,4.53112);
\draw [c] (2.34843,4.53112) -- (2.34843,4.54238);
\draw [c] (2.33361,4.53112) -- (2.34843,4.53112);
\draw [c] (2.34843,4.53112) -- (2.36325,4.53112);
\definecolor{c}{rgb}{0,0,0};
\colorlet{c}{kugray};
\draw [c] (2.37806,4.47939) -- (2.37806,4.492);
\draw [c] (2.37806,4.492) -- (2.37806,4.50411);
\draw [c] (2.36325,4.492) -- (2.37806,4.492);
\draw [c] (2.37806,4.492) -- (2.39288,4.492);
\definecolor{c}{rgb}{0,0,0};
\colorlet{c}{kugray};
\draw [c] (2.4077,4.44091) -- (2.4077,4.45439);
\draw [c] (2.4077,4.45439) -- (2.4077,4.46731);
\draw [c] (2.39288,4.45439) -- (2.4077,4.45439);
\draw [c] (2.4077,4.45439) -- (2.42252,4.45439);
\definecolor{c}{rgb}{0,0,0};
\colorlet{c}{kugray};
\draw [c] (2.43733,4.39585) -- (2.43733,4.41002);
\draw [c] (2.43733,4.41002) -- (2.43733,4.42357);
\draw [c] (2.42252,4.41002) -- (2.43733,4.41002);
\draw [c] (2.43733,4.41002) -- (2.45215,4.41002);
\definecolor{c}{rgb}{0,0,0};
\colorlet{c}{kugray};
\draw [c] (2.46697,4.36562) -- (2.46697,4.38084);
\draw [c] (2.46697,4.38084) -- (2.46697,4.39535);
\draw [c] (2.45215,4.38084) -- (2.46697,4.38084);
\draw [c] (2.46697,4.38084) -- (2.48179,4.38084);
\definecolor{c}{rgb}{0,0,0};
\colorlet{c}{kugray};
\draw [c] (2.49661,4.34442) -- (2.49661,4.3598);
\draw [c] (2.49661,4.3598) -- (2.49661,4.37444);
\draw [c] (2.48179,4.3598) -- (2.49661,4.3598);
\draw [c] (2.49661,4.3598) -- (2.51142,4.3598);
\definecolor{c}{rgb}{0,0,0};
\colorlet{c}{kugray};
\draw [c] (2.52624,4.33876) -- (2.52624,4.3547);
\draw [c] (2.52624,4.3547) -- (2.52624,4.36985);
\draw [c] (2.51142,4.3547) -- (2.52624,4.3547);
\draw [c] (2.52624,4.3547) -- (2.54106,4.3547);
\definecolor{c}{rgb}{0,0,0};
\colorlet{c}{kugray};
\draw [c] (2.55588,4.31164) -- (2.55588,4.32829);
\draw [c] (2.55588,4.32829) -- (2.55588,4.34409);
\draw [c] (2.54106,4.32829) -- (2.55588,4.32829);
\draw [c] (2.55588,4.32829) -- (2.5707,4.32829);
\definecolor{c}{rgb}{0,0,0};
\colorlet{c}{kugray};
\draw [c] (2.58551,4.27319) -- (2.58551,4.29119);
\draw [c] (2.58551,4.29119) -- (2.58551,4.30819);
\draw [c] (2.5707,4.29119) -- (2.58551,4.29119);
\draw [c] (2.58551,4.29119) -- (2.60033,4.29119);
\definecolor{c}{rgb}{0,0,0};
\colorlet{c}{kugray};
\draw [c] (2.61515,4.23142) -- (2.61515,4.25043);
\draw [c] (2.61515,4.25043) -- (2.61515,4.26834);
\draw [c] (2.60033,4.25043) -- (2.61515,4.25043);
\draw [c] (2.61515,4.25043) -- (2.62997,4.25043);
\definecolor{c}{rgb}{0,0,0};
\colorlet{c}{kugray};
\draw [c] (2.64478,4.1941) -- (2.64478,4.21371);
\draw [c] (2.64478,4.21371) -- (2.64478,4.23215);
\draw [c] (2.62997,4.21371) -- (2.64478,4.21371);
\draw [c] (2.64478,4.21371) -- (2.6596,4.21371);
\definecolor{c}{rgb}{0,0,0};
\colorlet{c}{kugray};
\draw [c] (2.67442,4.21383) -- (2.67442,4.23319);
\draw [c] (2.67442,4.23319) -- (2.67442,4.2514);
\draw [c] (2.6596,4.23319) -- (2.67442,4.23319);
\draw [c] (2.67442,4.23319) -- (2.68924,4.23319);
\definecolor{c}{rgb}{0,0,0};
\colorlet{c}{kugray};
\draw [c] (2.70406,4.15434) -- (2.70406,4.17536);
\draw [c] (2.70406,4.17536) -- (2.70406,4.19503);
\draw [c] (2.68924,4.17536) -- (2.70406,4.17536);
\draw [c] (2.70406,4.17536) -- (2.71887,4.17536);
\definecolor{c}{rgb}{0,0,0};
\colorlet{c}{kugray};
\draw [c] (2.73369,4.14693) -- (2.73369,4.16849);
\draw [c] (2.73369,4.16849) -- (2.73369,4.18863);
\draw [c] (2.71887,4.16849) -- (2.73369,4.16849);
\draw [c] (2.73369,4.16849) -- (2.74851,4.16849);
\definecolor{c}{rgb}{0,0,0};
\colorlet{c}{kugray};
\draw [c] (2.76333,4.12489) -- (2.76333,4.1477);
\draw [c] (2.76333,4.1477) -- (2.76333,4.16894);
\draw [c] (2.74851,4.1477) -- (2.76333,4.1477);
\draw [c] (2.76333,4.1477) -- (2.77815,4.1477);
\definecolor{c}{rgb}{0,0,0};
\colorlet{c}{kugray};
\draw [c] (2.79296,4.13183) -- (2.79296,4.15469);
\draw [c] (2.79296,4.15469) -- (2.79296,4.17597);
\draw [c] (2.77815,4.15469) -- (2.79296,4.15469);
\draw [c] (2.79296,4.15469) -- (2.80778,4.15469);
\definecolor{c}{rgb}{0,0,0};
\colorlet{c}{kugray};
\draw [c] (2.8226,4.05797) -- (2.8226,4.08216);
\draw [c] (2.8226,4.08216) -- (2.8226,4.10458);
\draw [c] (2.80778,4.08216) -- (2.8226,4.08216);
\draw [c] (2.8226,4.08216) -- (2.83742,4.08216);
\definecolor{c}{rgb}{0,0,0};
\colorlet{c}{kugray};
\draw [c] (2.85224,4.04617) -- (2.85224,4.07199);
\draw [c] (2.85224,4.07199) -- (2.85224,4.09581);
\draw [c] (2.83742,4.07199) -- (2.85224,4.07199);
\draw [c] (2.85224,4.07199) -- (2.86705,4.07199);
\definecolor{c}{rgb}{0,0,0};
\colorlet{c}{kugray};
\draw [c] (2.88187,4.02533) -- (2.88187,4.05093);
\draw [c] (2.88187,4.05093) -- (2.88187,4.07457);
\draw [c] (2.86705,4.05093) -- (2.88187,4.05093);
\draw [c] (2.88187,4.05093) -- (2.89669,4.05093);
\definecolor{c}{rgb}{0,0,0};
\colorlet{c}{kugray};
\draw [c] (2.91151,4.0461) -- (2.91151,4.07164);
\draw [c] (2.91151,4.07164) -- (2.91151,4.09522);
\draw [c] (2.89669,4.07164) -- (2.91151,4.07164);
\draw [c] (2.91151,4.07164) -- (2.92632,4.07164);
\definecolor{c}{rgb}{0,0,0};
\colorlet{c}{kugray};
\draw [c] (2.94114,3.94743) -- (2.94114,3.97699);
\draw [c] (2.94114,3.97699) -- (2.94114,4.00396);
\draw [c] (2.92632,3.97699) -- (2.94114,3.97699);
\draw [c] (2.94114,3.97699) -- (2.95596,3.97699);
\definecolor{c}{rgb}{0,0,0};
\colorlet{c}{kugray};
\draw [c] (2.97078,3.92033) -- (2.97078,3.95122);
\draw [c] (2.97078,3.95122) -- (2.97078,3.97929);
\draw [c] (2.95596,3.95122) -- (2.97078,3.95122);
\draw [c] (2.97078,3.95122) -- (2.9856,3.95122);
\definecolor{c}{rgb}{0,0,0};
\colorlet{c}{kugray};
\draw [c] (3.00041,3.95455) -- (3.00041,3.98496);
\draw [c] (3.00041,3.98496) -- (3.00041,4.01263);
\draw [c] (2.9856,3.98496) -- (3.00041,3.98496);
\draw [c] (3.00041,3.98496) -- (3.01523,3.98496);
\definecolor{c}{rgb}{0,0,0};
\colorlet{c}{kugray};
\draw [c] (3.03005,3.91612) -- (3.03005,3.94874);
\draw [c] (3.03005,3.94874) -- (3.03005,3.97824);
\draw [c] (3.01523,3.94874) -- (3.03005,3.94874);
\draw [c] (3.03005,3.94874) -- (3.04487,3.94874);
\definecolor{c}{rgb}{0,0,0};
\colorlet{c}{kugray};
\draw [c] (3.05969,3.91532) -- (3.05969,3.94707);
\draw [c] (3.05969,3.94707) -- (3.05969,3.97585);
\draw [c] (3.04487,3.94707) -- (3.05969,3.94707);
\draw [c] (3.05969,3.94707) -- (3.0745,3.94707);
\definecolor{c}{rgb}{0,0,0};
\colorlet{c}{kugray};
\draw [c] (3.08932,3.82813) -- (3.08932,3.8633);
\draw [c] (3.08932,3.8633) -- (3.08932,3.89486);
\draw [c] (3.0745,3.8633) -- (3.08932,3.8633);
\draw [c] (3.08932,3.8633) -- (3.10414,3.8633);
\definecolor{c}{rgb}{0,0,0};
\colorlet{c}{kugray};
\draw [c] (3.11896,3.82152) -- (3.11896,3.85725);
\draw [c] (3.11896,3.85725) -- (3.11896,3.88926);
\draw [c] (3.10414,3.85725) -- (3.11896,3.85725);
\draw [c] (3.11896,3.85725) -- (3.13377,3.85725);
\definecolor{c}{rgb}{0,0,0};
\colorlet{c}{kugray};
\draw [c] (3.14859,3.81099) -- (3.14859,3.849);
\draw [c] (3.14859,3.849) -- (3.14859,3.88283);
\draw [c] (3.13377,3.849) -- (3.14859,3.849);
\draw [c] (3.14859,3.849) -- (3.16341,3.849);
\definecolor{c}{rgb}{0,0,0};
\colorlet{c}{kugray};
\draw [c] (3.17823,3.68505) -- (3.17823,3.72686);
\draw [c] (3.17823,3.72686) -- (3.17823,3.76367);
\draw [c] (3.16341,3.72686) -- (3.17823,3.72686);
\draw [c] (3.17823,3.72686) -- (3.19305,3.72686);
\definecolor{c}{rgb}{0,0,0};
\colorlet{c}{kugray};
\draw [c] (3.20786,3.75027) -- (3.20786,3.79207);
\draw [c] (3.20786,3.79207) -- (3.20786,3.82886);
\draw [c] (3.19305,3.79207) -- (3.20786,3.79207);
\draw [c] (3.20786,3.79207) -- (3.22268,3.79207);
\definecolor{c}{rgb}{0,0,0};
\colorlet{c}{kugray};
\draw [c] (3.2375,3.78399) -- (3.2375,3.82558);
\draw [c] (3.2375,3.82558) -- (3.2375,3.86222);
\draw [c] (3.22268,3.82558) -- (3.2375,3.82558);
\draw [c] (3.2375,3.82558) -- (3.25232,3.82558);
\definecolor{c}{rgb}{0,0,0};
\colorlet{c}{kugray};
\draw [c] (3.26714,3.71561) -- (3.26714,3.76154);
\draw [c] (3.26714,3.76154) -- (3.26714,3.80149);
\draw [c] (3.25232,3.76154) -- (3.26714,3.76154);
\draw [c] (3.26714,3.76154) -- (3.28195,3.76154);
\definecolor{c}{rgb}{0,0,0};
\colorlet{c}{kugray};
\draw [c] (3.29677,3.74464) -- (3.29677,3.78592);
\draw [c] (3.29677,3.78592) -- (3.29677,3.82231);
\draw [c] (3.28195,3.78592) -- (3.29677,3.78592);
\draw [c] (3.29677,3.78592) -- (3.31159,3.78592);
\definecolor{c}{rgb}{0,0,0};
\colorlet{c}{kugray};
\draw [c] (3.32641,3.60212) -- (3.32641,3.65046);
\draw [c] (3.32641,3.65046) -- (3.32641,3.69223);
\draw [c] (3.31159,3.65046) -- (3.32641,3.65046);
\draw [c] (3.32641,3.65046) -- (3.34123,3.65046);
\definecolor{c}{rgb}{0,0,0};
\colorlet{c}{kugray};
\draw [c] (3.35604,3.66614) -- (3.35604,3.71219);
\draw [c] (3.35604,3.71219) -- (3.35604,3.75225);
\draw [c] (3.34123,3.71219) -- (3.35604,3.71219);
\draw [c] (3.35604,3.71219) -- (3.37086,3.71219);
\definecolor{c}{rgb}{0,0,0};
\colorlet{c}{kugray};
\draw [c] (3.38568,3.70789) -- (3.38568,3.7536);
\draw [c] (3.38568,3.7536) -- (3.38568,3.7934);
\draw [c] (3.37086,3.7536) -- (3.38568,3.7536);
\draw [c] (3.38568,3.7536) -- (3.4005,3.7536);
\definecolor{c}{rgb}{0,0,0};
\colorlet{c}{kugray};
\draw [c] (3.41531,3.59749) -- (3.41531,3.64919);
\draw [c] (3.41531,3.64919) -- (3.41531,3.69343);
\draw [c] (3.4005,3.64919) -- (3.41531,3.64919);
\draw [c] (3.41531,3.64919) -- (3.43013,3.64919);
\definecolor{c}{rgb}{0,0,0};
\colorlet{c}{kugray};
\draw [c] (3.44495,3.54922) -- (3.44495,3.6056);
\draw [c] (3.44495,3.6056) -- (3.44495,3.65323);
\draw [c] (3.43013,3.6056) -- (3.44495,3.6056);
\draw [c] (3.44495,3.6056) -- (3.45977,3.6056);
\definecolor{c}{rgb}{0,0,0};
\colorlet{c}{kugray};
\draw [c] (3.47459,3.5315) -- (3.47459,3.58602);
\draw [c] (3.47459,3.58602) -- (3.47459,3.63231);
\draw [c] (3.45977,3.58602) -- (3.47459,3.58602);
\draw [c] (3.47459,3.58602) -- (3.4894,3.58602);
\definecolor{c}{rgb}{0,0,0};
\colorlet{c}{kugray};
\draw [c] (3.50422,3.44087) -- (3.50422,3.50756);
\draw [c] (3.50422,3.50756) -- (3.50422,3.56234);
\draw [c] (3.4894,3.50756) -- (3.50422,3.50756);
\draw [c] (3.50422,3.50756) -- (3.51904,3.50756);
\definecolor{c}{rgb}{0,0,0};
\colorlet{c}{kugray};
\draw [c] (3.53386,3.57716) -- (3.53386,3.62954);
\draw [c] (3.53386,3.62954) -- (3.53386,3.67429);
\draw [c] (3.51904,3.62954) -- (3.53386,3.62954);
\draw [c] (3.53386,3.62954) -- (3.54868,3.62954);
\definecolor{c}{rgb}{0,0,0};
\colorlet{c}{kugray};
\draw [c] (3.56349,3.59791) -- (3.56349,3.64909);
\draw [c] (3.56349,3.64909) -- (3.56349,3.69297);
\draw [c] (3.54868,3.64909) -- (3.56349,3.64909);
\draw [c] (3.56349,3.64909) -- (3.57831,3.64909);
\definecolor{c}{rgb}{0,0,0};
\colorlet{c}{kugray};
\draw [c] (3.59313,3.48016) -- (3.59313,3.54299);
\draw [c] (3.59313,3.54299) -- (3.59313,3.59515);
\draw [c] (3.57831,3.54299) -- (3.59313,3.54299);
\draw [c] (3.59313,3.54299) -- (3.60795,3.54299);
\definecolor{c}{rgb}{0,0,0};
\colorlet{c}{kugray};
\draw [c] (3.62276,3.49926) -- (3.62276,3.5653);
\draw [c] (3.62276,3.5653) -- (3.62276,3.61965);
\draw [c] (3.60795,3.5653) -- (3.62276,3.5653);
\draw [c] (3.62276,3.5653) -- (3.63758,3.5653);
\definecolor{c}{rgb}{0,0,0};
\colorlet{c}{kugray};
\draw [c] (3.6524,3.38318) -- (3.6524,3.45701);
\draw [c] (3.6524,3.45701) -- (3.6524,3.51651);
\draw [c] (3.63758,3.45701) -- (3.6524,3.45701);
\draw [c] (3.6524,3.45701) -- (3.66722,3.45701);
\definecolor{c}{rgb}{0,0,0};
\colorlet{c}{kugray};
\draw [c] (3.68204,3.38716) -- (3.68204,3.45825);
\draw [c] (3.68204,3.45825) -- (3.68204,3.51597);
\draw [c] (3.66722,3.45825) -- (3.68204,3.45825);
\draw [c] (3.68204,3.45825) -- (3.69685,3.45825);
\definecolor{c}{rgb}{0,0,0};
\colorlet{c}{kugray};
\draw [c] (3.71167,3.27025) -- (3.71167,3.35157);
\draw [c] (3.71167,3.35157) -- (3.71167,3.41584);
\draw [c] (3.69685,3.35157) -- (3.71167,3.35157);
\draw [c] (3.71167,3.35157) -- (3.72649,3.35157);
\definecolor{c}{rgb}{0,0,0};
\colorlet{c}{kugray};
\draw [c] (3.74131,3.42793) -- (3.74131,3.49218);
\draw [c] (3.74131,3.49218) -- (3.74131,3.54531);
\draw [c] (3.72649,3.49218) -- (3.74131,3.49218);
\draw [c] (3.74131,3.49218) -- (3.75613,3.49218);
\definecolor{c}{rgb}{0,0,0};
\colorlet{c}{kugray};
\draw [c] (3.77094,3.39681) -- (3.77094,3.4655);
\draw [c] (3.77094,3.4655) -- (3.77094,3.52163);
\draw [c] (3.75613,3.4655) -- (3.77094,3.4655);
\draw [c] (3.77094,3.4655) -- (3.78576,3.4655);
\definecolor{c}{rgb}{0,0,0};
\colorlet{c}{kugray};
\draw [c] (3.80058,3.30807) -- (3.80058,3.39026);
\draw [c] (3.80058,3.39026) -- (3.80058,3.45506);
\draw [c] (3.78576,3.39026) -- (3.80058,3.39026);
\draw [c] (3.80058,3.39026) -- (3.8154,3.39026);
\definecolor{c}{rgb}{0,0,0};
\colorlet{c}{kugray};
\draw [c] (3.83022,3.36833) -- (3.83022,3.44242);
\draw [c] (3.83022,3.44242) -- (3.83022,3.50209);
\draw [c] (3.8154,3.44242) -- (3.83022,3.44242);
\draw [c] (3.83022,3.44242) -- (3.84503,3.44242);
\definecolor{c}{rgb}{0,0,0};
\colorlet{c}{kugray};
\draw [c] (3.85985,3.36697) -- (3.85985,3.43885);
\draw [c] (3.85985,3.43885) -- (3.85985,3.49708);
\draw [c] (3.84503,3.43885) -- (3.85985,3.43885);
\draw [c] (3.85985,3.43885) -- (3.87467,3.43885);
\definecolor{c}{rgb}{0,0,0};
\colorlet{c}{kugray};
\draw [c] (3.88949,3.18339) -- (3.88949,3.27641);
\draw [c] (3.88949,3.27641) -- (3.88949,3.34775);
\draw [c] (3.87467,3.27641) -- (3.88949,3.27641);
\draw [c] (3.88949,3.27641) -- (3.9043,3.27641);
\definecolor{c}{rgb}{0,0,0};
\colorlet{c}{kugray};
\draw [c] (3.91912,3.36859) -- (3.91912,3.433);
\draw [c] (3.91912,3.433) -- (3.91912,3.48624);
\draw [c] (3.9043,3.433) -- (3.91912,3.433);
\draw [c] (3.91912,3.433) -- (3.93394,3.433);
\definecolor{c}{rgb}{0,0,0};
\colorlet{c}{kugray};
\draw [c] (3.94876,3.27957) -- (3.94876,3.34398);
\draw [c] (3.94876,3.34398) -- (3.94876,3.39722);
\draw [c] (3.93394,3.34398) -- (3.94876,3.34398);
\draw [c] (3.94876,3.34398) -- (3.96358,3.34398);
\definecolor{c}{rgb}{0,0,0};
\colorlet{c}{kugray};
\draw [c] (3.97839,3.23181) -- (3.97839,3.2842);
\draw [c] (3.97839,3.2842) -- (3.97839,3.32896);
\draw [c] (3.96358,3.2842) -- (3.97839,3.2842);
\draw [c] (3.97839,3.2842) -- (3.99321,3.2842);
\definecolor{c}{rgb}{0,0,0};
\colorlet{c}{kugray};
\draw [c] (4.00803,3.23298) -- (4.00803,3.24514);
\draw [c] (4.00803,3.24514) -- (4.00803,3.25684);
\draw [c] (3.99321,3.24514) -- (4.00803,3.24514);
\draw [c] (4.00803,3.24514) -- (4.02285,3.24514);
\definecolor{c}{rgb}{0,0,0};
\colorlet{c}{kugray};
\draw [c] (4.03767,3.23899) -- (4.03767,3.25079);
\draw [c] (4.03767,3.25079) -- (4.03767,3.26215);
\draw [c] (4.02285,3.25079) -- (4.03767,3.25079);
\draw [c] (4.03767,3.25079) -- (4.05248,3.25079);
\definecolor{c}{rgb}{0,0,0};
\colorlet{c}{kugray};
\draw [c] (4.0673,3.24063) -- (4.0673,3.25269);
\draw [c] (4.0673,3.25269) -- (4.0673,3.26429);
\draw [c] (4.05248,3.25269) -- (4.0673,3.25269);
\draw [c] (4.0673,3.25269) -- (4.08212,3.25269);
\definecolor{c}{rgb}{0,0,0};
\colorlet{c}{kugray};
\draw [c] (4.09694,3.22722) -- (4.09694,3.23915);
\draw [c] (4.09694,3.23915) -- (4.09694,3.25063);
\draw [c] (4.08212,3.23915) -- (4.09694,3.23915);
\draw [c] (4.09694,3.23915) -- (4.11175,3.23915);
\definecolor{c}{rgb}{0,0,0};
\colorlet{c}{kugray};
\draw [c] (4.12657,3.2116) -- (4.12657,3.22405);
\draw [c] (4.12657,3.22405) -- (4.12657,3.23601);
\draw [c] (4.11175,3.22405) -- (4.12657,3.22405);
\draw [c] (4.12657,3.22405) -- (4.14139,3.22405);
\definecolor{c}{rgb}{0,0,0};
\colorlet{c}{kugray};
\draw [c] (4.15621,3.18215) -- (4.15621,3.19496);
\draw [c] (4.15621,3.19496) -- (4.15621,3.20726);
\draw [c] (4.14139,3.19496) -- (4.15621,3.19496);
\draw [c] (4.15621,3.19496) -- (4.17103,3.19496);
\definecolor{c}{rgb}{0,0,0};
\colorlet{c}{kugray};
\draw [c] (4.18584,3.18206) -- (4.18584,3.19521);
\draw [c] (4.18584,3.19521) -- (4.18584,3.20782);
\draw [c] (4.17103,3.19521) -- (4.18584,3.19521);
\draw [c] (4.18584,3.19521) -- (4.20066,3.19521);
\definecolor{c}{rgb}{0,0,0};
\colorlet{c}{kugray};
\draw [c] (4.21548,3.14682) -- (4.21548,3.16086);
\draw [c] (4.21548,3.16086) -- (4.21548,3.17428);
\draw [c] (4.20066,3.16086) -- (4.21548,3.16086);
\draw [c] (4.21548,3.16086) -- (4.2303,3.16086);
\definecolor{c}{rgb}{0,0,0};
\colorlet{c}{kugray};
\draw [c] (4.24512,3.16809) -- (4.24512,3.1818);
\draw [c] (4.24512,3.1818) -- (4.24512,3.19493);
\draw [c] (4.2303,3.1818) -- (4.24512,3.1818);
\draw [c] (4.24512,3.1818) -- (4.25993,3.1818);
\definecolor{c}{rgb}{0,0,0};
\colorlet{c}{kugray};
\draw [c] (4.27475,3.10937) -- (4.27475,3.12421);
\draw [c] (4.27475,3.12421) -- (4.27475,3.13837);
\draw [c] (4.25993,3.12421) -- (4.27475,3.12421);
\draw [c] (4.27475,3.12421) -- (4.28957,3.12421);
\definecolor{c}{rgb}{0,0,0};
\colorlet{c}{kugray};
\draw [c] (4.30439,3.11556) -- (4.30439,3.12992);
\draw [c] (4.30439,3.12992) -- (4.30439,3.14364);
\draw [c] (4.28957,3.12992) -- (4.30439,3.12992);
\draw [c] (4.30439,3.12992) -- (4.31921,3.12992);
\definecolor{c}{rgb}{0,0,0};
\colorlet{c}{kugray};
\draw [c] (4.33402,3.13544) -- (4.33402,3.14949);
\draw [c] (4.33402,3.14949) -- (4.33402,3.16293);
\draw [c] (4.31921,3.14949) -- (4.33402,3.14949);
\draw [c] (4.33402,3.14949) -- (4.34884,3.14949);
\definecolor{c}{rgb}{0,0,0};
\colorlet{c}{kugray};
\draw [c] (4.36366,3.09108) -- (4.36366,3.10621);
\draw [c] (4.36366,3.10621) -- (4.36366,3.12063);
\draw [c] (4.34884,3.10621) -- (4.36366,3.10621);
\draw [c] (4.36366,3.10621) -- (4.37848,3.10621);
\definecolor{c}{rgb}{0,0,0};
\colorlet{c}{kugray};
\draw [c] (4.39329,3.08833) -- (4.39329,3.10339);
\draw [c] (4.39329,3.10339) -- (4.39329,3.11775);
\draw [c] (4.37848,3.10339) -- (4.39329,3.10339);
\draw [c] (4.39329,3.10339) -- (4.40811,3.10339);
\definecolor{c}{rgb}{0,0,0};
\colorlet{c}{kugray};
\draw [c] (4.42293,3.06637) -- (4.42293,3.08234);
\draw [c] (4.42293,3.08234) -- (4.42293,3.09751);
\draw [c] (4.40811,3.08234) -- (4.42293,3.08234);
\draw [c] (4.42293,3.08234) -- (4.43775,3.08234);
\definecolor{c}{rgb}{0,0,0};
\colorlet{c}{kugray};
\draw [c] (4.45257,3.03685) -- (4.45257,3.05337);
\draw [c] (4.45257,3.05337) -- (4.45257,3.06905);
\draw [c] (4.43775,3.05337) -- (4.45257,3.05337);
\draw [c] (4.45257,3.05337) -- (4.46738,3.05337);
\definecolor{c}{rgb}{0,0,0};
\colorlet{c}{kugray};
\draw [c] (4.4822,3.05403) -- (4.4822,3.07028);
\draw [c] (4.4822,3.07028) -- (4.4822,3.08572);
\draw [c] (4.46738,3.07028) -- (4.4822,3.07028);
\draw [c] (4.4822,3.07028) -- (4.49702,3.07028);
\definecolor{c}{rgb}{0,0,0};
\colorlet{c}{kugray};
\draw [c] (4.51184,3.02475) -- (4.51184,3.04166);
\draw [c] (4.51184,3.04166) -- (4.51184,3.05768);
\draw [c] (4.49702,3.04166) -- (4.51184,3.04166);
\draw [c] (4.51184,3.04166) -- (4.52666,3.04166);
\definecolor{c}{rgb}{0,0,0};
\colorlet{c}{kugray};
\draw [c] (4.54147,2.97429) -- (4.54147,2.99215);
\draw [c] (4.54147,2.99215) -- (4.54147,3.00903);
\draw [c] (4.52666,2.99215) -- (4.54147,2.99215);
\draw [c] (4.54147,2.99215) -- (4.55629,2.99215);
\definecolor{c}{rgb}{0,0,0};
\colorlet{c}{kugray};
\draw [c] (4.57111,2.97145) -- (4.57111,2.9901);
\draw [c] (4.57111,2.9901) -- (4.57111,3.0077);
\draw [c] (4.55629,2.9901) -- (4.57111,2.9901);
\draw [c] (4.57111,2.9901) -- (4.58593,2.9901);
\definecolor{c}{rgb}{0,0,0};
\colorlet{c}{kugray};
\draw [c] (4.60075,2.99637) -- (4.60075,3.01434);
\draw [c] (4.60075,3.01434) -- (4.60075,3.03131);
\draw [c] (4.58593,3.01434) -- (4.60075,3.01434);
\draw [c] (4.60075,3.01434) -- (4.61556,3.01434);
\definecolor{c}{rgb}{0,0,0};
\colorlet{c}{kugray};
\draw [c] (4.63038,2.94705) -- (4.63038,2.96575);
\draw [c] (4.63038,2.96575) -- (4.63038,2.98338);
\draw [c] (4.61556,2.96575) -- (4.63038,2.96575);
\draw [c] (4.63038,2.96575) -- (4.6452,2.96575);
\definecolor{c}{rgb}{0,0,0};
\colorlet{c}{kugray};
\draw [c] (4.66002,2.94213) -- (4.66002,2.96119);
\draw [c] (4.66002,2.96119) -- (4.66002,2.97915);
\draw [c] (4.6452,2.96119) -- (4.66002,2.96119);
\draw [c] (4.66002,2.96119) -- (4.67483,2.96119);
\definecolor{c}{rgb}{0,0,0};
\colorlet{c}{kugray};
\draw [c] (4.68965,2.92699) -- (4.68965,2.94611);
\draw [c] (4.68965,2.94611) -- (4.68965,2.96412);
\draw [c] (4.67483,2.94611) -- (4.68965,2.94611);
\draw [c] (4.68965,2.94611) -- (4.70447,2.94611);
\definecolor{c}{rgb}{0,0,0};
\colorlet{c}{kugray};
\draw [c] (4.71929,2.90533) -- (4.71929,2.92537);
\draw [c] (4.71929,2.92537) -- (4.71929,2.94418);
\draw [c] (4.70447,2.92537) -- (4.71929,2.92537);
\draw [c] (4.71929,2.92537) -- (4.73411,2.92537);
\definecolor{c}{rgb}{0,0,0};
\colorlet{c}{kugray};
\draw [c] (4.74892,2.89803) -- (4.74892,2.9188);
\draw [c] (4.74892,2.9188) -- (4.74892,2.93826);
\draw [c] (4.73411,2.9188) -- (4.74892,2.9188);
\draw [c] (4.74892,2.9188) -- (4.76374,2.9188);
\definecolor{c}{rgb}{0,0,0};
\colorlet{c}{kugray};
\draw [c] (4.77856,2.88491) -- (4.77856,2.90571);
\draw [c] (4.77856,2.90571) -- (4.77856,2.92519);
\draw [c] (4.76374,2.90571) -- (4.77856,2.90571);
\draw [c] (4.77856,2.90571) -- (4.79338,2.90571);
\definecolor{c}{rgb}{0,0,0};
\colorlet{c}{kugray};
\draw [c] (4.8082,2.83295) -- (4.8082,2.85717);
\draw [c] (4.8082,2.85717) -- (4.8082,2.87962);
\draw [c] (4.79338,2.85717) -- (4.8082,2.85717);
\draw [c] (4.8082,2.85717) -- (4.82301,2.85717);
\definecolor{c}{rgb}{0,0,0};
\colorlet{c}{kugray};
\draw [c] (4.83783,2.84099) -- (4.83783,2.86366);
\draw [c] (4.83783,2.86366) -- (4.83783,2.88478);
\draw [c] (4.82301,2.86366) -- (4.83783,2.86366);
\draw [c] (4.83783,2.86366) -- (4.85265,2.86366);
\definecolor{c}{rgb}{0,0,0};
\colorlet{c}{kugray};
\draw [c] (4.86747,2.84516) -- (4.86747,2.86807);
\draw [c] (4.86747,2.86807) -- (4.86747,2.88939);
\draw [c] (4.85265,2.86807) -- (4.86747,2.86807);
\draw [c] (4.86747,2.86807) -- (4.88228,2.86807);
\definecolor{c}{rgb}{0,0,0};
\colorlet{c}{kugray};
\draw [c] (4.8971,2.82427) -- (4.8971,2.84804);
\draw [c] (4.8971,2.84804) -- (4.8971,2.87011);
\draw [c] (4.88228,2.84804) -- (4.8971,2.84804);
\draw [c] (4.8971,2.84804) -- (4.91192,2.84804);
\definecolor{c}{rgb}{0,0,0};
\colorlet{c}{kugray};
\draw [c] (4.92674,2.7958) -- (4.92674,2.82013);
\draw [c] (4.92674,2.82013) -- (4.92674,2.84268);
\draw [c] (4.91192,2.82013) -- (4.92674,2.82013);
\draw [c] (4.92674,2.82013) -- (4.94156,2.82013);
\definecolor{c}{rgb}{0,0,0};
\colorlet{c}{kugray};
\draw [c] (4.95637,2.72162) -- (4.95637,2.74908);
\draw [c] (4.95637,2.74908) -- (4.95637,2.7743);
\draw [c] (4.94156,2.74908) -- (4.95637,2.74908);
\draw [c] (4.95637,2.74908) -- (4.97119,2.74908);
\definecolor{c}{rgb}{0,0,0};
\colorlet{c}{kugray};
\draw [c] (4.98601,2.74467) -- (4.98601,2.77317);
\draw [c] (4.98601,2.77317) -- (4.98601,2.79924);
\draw [c] (4.97119,2.77317) -- (4.98601,2.77317);
\draw [c] (4.98601,2.77317) -- (5.00083,2.77317);
\definecolor{c}{rgb}{0,0,0};
\colorlet{c}{kugray};
\draw [c] (5.01565,2.72133) -- (5.01565,2.74919);
\draw [c] (5.01565,2.74919) -- (5.01565,2.77473);
\draw [c] (5.00083,2.74919) -- (5.01565,2.74919);
\draw [c] (5.01565,2.74919) -- (5.03046,2.74919);
\definecolor{c}{rgb}{0,0,0};
\colorlet{c}{kugray};
\draw [c] (5.04528,2.68622) -- (5.04528,2.71565);
\draw [c] (5.04528,2.71565) -- (5.04528,2.74251);
\draw [c] (5.03046,2.71565) -- (5.04528,2.71565);
\draw [c] (5.04528,2.71565) -- (5.0601,2.71565);
\definecolor{c}{rgb}{0,0,0};
\colorlet{c}{kugray};
\draw [c] (5.07492,2.69413) -- (5.07492,2.72282);
\draw [c] (5.07492,2.72282) -- (5.07492,2.74907);
\draw [c] (5.0601,2.72282) -- (5.07492,2.72282);
\draw [c] (5.07492,2.72282) -- (5.08974,2.72282);
\definecolor{c}{rgb}{0,0,0};
\colorlet{c}{kugray};
\draw [c] (5.10455,2.68206) -- (5.10455,2.71119);
\draw [c] (5.10455,2.71119) -- (5.10455,2.73781);
\draw [c] (5.08974,2.71119) -- (5.10455,2.71119);
\draw [c] (5.10455,2.71119) -- (5.11937,2.71119);
\definecolor{c}{rgb}{0,0,0};
\colorlet{c}{kugray};
\draw [c] (5.13419,2.68307) -- (5.13419,2.7116);
\draw [c] (5.13419,2.7116) -- (5.13419,2.7377);
\draw [c] (5.11937,2.7116) -- (5.13419,2.7116);
\draw [c] (5.13419,2.7116) -- (5.14901,2.7116);
\definecolor{c}{rgb}{0,0,0};
\colorlet{c}{kugray};
\draw [c] (5.16382,2.72714) -- (5.16382,2.75436);
\draw [c] (5.16382,2.75436) -- (5.16382,2.77936);
\draw [c] (5.14901,2.75436) -- (5.16382,2.75436);
\draw [c] (5.16382,2.75436) -- (5.17864,2.75436);
\definecolor{c}{rgb}{0,0,0};
\colorlet{c}{kugray};
\draw [c] (5.19346,2.64861) -- (5.19346,2.67901);
\draw [c] (5.19346,2.67901) -- (5.19346,2.70667);
\draw [c] (5.17864,2.67901) -- (5.19346,2.67901);
\draw [c] (5.19346,2.67901) -- (5.20828,2.67901);
\definecolor{c}{rgb}{0,0,0};
\colorlet{c}{kugray};
\draw [c] (5.2231,2.688) -- (5.2231,2.71811);
\draw [c] (5.2231,2.71811) -- (5.2231,2.74554);
\draw [c] (5.20828,2.71811) -- (5.2231,2.71811);
\draw [c] (5.2231,2.71811) -- (5.23791,2.71811);
\definecolor{c}{rgb}{0,0,0};
\colorlet{c}{kugray};
\draw [c] (5.25273,2.56862) -- (5.25273,2.60483);
\draw [c] (5.25273,2.60483) -- (5.25273,2.63722);
\draw [c] (5.23791,2.60483) -- (5.25273,2.60483);
\draw [c] (5.25273,2.60483) -- (5.26755,2.60483);
\definecolor{c}{rgb}{0,0,0};
\colorlet{c}{kugray};
\draw [c] (5.28237,2.61362) -- (5.28237,2.64718);
\draw [c] (5.28237,2.64718) -- (5.28237,2.67745);
\draw [c] (5.26755,2.64718) -- (5.28237,2.64718);
\draw [c] (5.28237,2.64718) -- (5.29719,2.64718);
\definecolor{c}{rgb}{0,0,0};
\colorlet{c}{kugray};
\draw [c] (5.312,2.58142) -- (5.312,2.61649);
\draw [c] (5.312,2.61649) -- (5.312,2.64798);
\draw [c] (5.29719,2.61649) -- (5.312,2.61649);
\draw [c] (5.312,2.61649) -- (5.32682,2.61649);
\definecolor{c}{rgb}{0,0,0};
\colorlet{c}{kugray};
\draw [c] (5.34164,2.59936) -- (5.34164,2.63325);
\draw [c] (5.34164,2.63325) -- (5.34164,2.66377);
\draw [c] (5.32682,2.63325) -- (5.34164,2.63325);
\draw [c] (5.34164,2.63325) -- (5.35646,2.63325);
\definecolor{c}{rgb}{0,0,0};
\colorlet{c}{kugray};
\draw [c] (5.37127,2.59301) -- (5.37127,2.62735);
\draw [c] (5.37127,2.62735) -- (5.37127,2.65824);
\draw [c] (5.35646,2.62735) -- (5.37127,2.62735);
\draw [c] (5.37127,2.62735) -- (5.38609,2.62735);
\definecolor{c}{rgb}{0,0,0};
\colorlet{c}{kugray};
\draw [c] (5.40091,2.57979) -- (5.40091,2.61666);
\draw [c] (5.40091,2.61666) -- (5.40091,2.64958);
\draw [c] (5.38609,2.61666) -- (5.40091,2.61666);
\draw [c] (5.40091,2.61666) -- (5.41573,2.61666);
\definecolor{c}{rgb}{0,0,0};
\colorlet{c}{kugray};
\draw [c] (5.43055,2.5888) -- (5.43055,2.62522);
\draw [c] (5.43055,2.62522) -- (5.43055,2.65779);
\draw [c] (5.41573,2.62522) -- (5.43055,2.62522);
\draw [c] (5.43055,2.62522) -- (5.44536,2.62522);
\definecolor{c}{rgb}{0,0,0};
\colorlet{c}{kugray};
\draw [c] (5.46018,2.49892) -- (5.46018,2.53839);
\draw [c] (5.46018,2.53839) -- (5.46018,2.57337);
\draw [c] (5.44536,2.53839) -- (5.46018,2.53839);
\draw [c] (5.46018,2.53839) -- (5.475,2.53839);
\definecolor{c}{rgb}{0,0,0};
\colorlet{c}{kugray};
\draw [c] (5.48982,2.51137) -- (5.48982,2.5487);
\draw [c] (5.48982,2.5487) -- (5.48982,2.58199);
\draw [c] (5.475,2.5487) -- (5.48982,2.5487);
\draw [c] (5.48982,2.5487) -- (5.50464,2.5487);
\definecolor{c}{rgb}{0,0,0};
\colorlet{c}{kugray};
\draw [c] (5.51945,2.5345) -- (5.51945,2.57323);
\draw [c] (5.51945,2.57323) -- (5.51945,2.60762);
\draw [c] (5.50464,2.57323) -- (5.51945,2.57323);
\draw [c] (5.51945,2.57323) -- (5.53427,2.57323);
\definecolor{c}{rgb}{0,0,0};
\colorlet{c}{kugray};
\draw [c] (5.54909,2.49349) -- (5.54909,2.533);
\draw [c] (5.54909,2.533) -- (5.54909,2.56801);
\draw [c] (5.53427,2.533) -- (5.54909,2.533);
\draw [c] (5.54909,2.533) -- (5.56391,2.533);
\definecolor{c}{rgb}{0,0,0};
\colorlet{c}{kugray};
\draw [c] (5.57873,2.46256) -- (5.57873,2.50333);
\draw [c] (5.57873,2.50333) -- (5.57873,2.53932);
\draw [c] (5.56391,2.50333) -- (5.57873,2.50333);
\draw [c] (5.57873,2.50333) -- (5.59354,2.50333);
\definecolor{c}{rgb}{0,0,0};
\colorlet{c}{kugray};
\draw [c] (5.60836,2.50823) -- (5.60836,2.54684);
\draw [c] (5.60836,2.54684) -- (5.60836,2.58115);
\draw [c] (5.59354,2.54684) -- (5.60836,2.54684);
\draw [c] (5.60836,2.54684) -- (5.62318,2.54684);
\definecolor{c}{rgb}{0,0,0};
\colorlet{c}{kugray};
\draw [c] (5.638,2.50569) -- (5.638,2.54708);
\draw [c] (5.638,2.54708) -- (5.638,2.58356);
\draw [c] (5.62318,2.54708) -- (5.638,2.54708);
\draw [c] (5.638,2.54708) -- (5.65281,2.54708);
\definecolor{c}{rgb}{0,0,0};
\colorlet{c}{kugray};
\draw [c] (5.66763,2.48101) -- (5.66763,2.5229);
\draw [c] (5.66763,2.5229) -- (5.66763,2.55976);
\draw [c] (5.65281,2.5229) -- (5.66763,2.5229);
\draw [c] (5.66763,2.5229) -- (5.68245,2.5229);
\definecolor{c}{rgb}{0,0,0};
\colorlet{c}{kugray};
\draw [c] (5.69727,2.3796) -- (5.69727,2.42684);
\draw [c] (5.69727,2.42684) -- (5.69727,2.46778);
\draw [c] (5.68245,2.42684) -- (5.69727,2.42684);
\draw [c] (5.69727,2.42684) -- (5.71209,2.42684);
\definecolor{c}{rgb}{0,0,0};
\colorlet{c}{kugray};
\draw [c] (5.7269,2.30877) -- (5.7269,2.35869);
\draw [c] (5.7269,2.35869) -- (5.7269,2.40163);
\draw [c] (5.71209,2.35869) -- (5.7269,2.35869);
\draw [c] (5.7269,2.35869) -- (5.74172,2.35869);
\definecolor{c}{rgb}{0,0,0};
\colorlet{c}{kugray};
\draw [c] (5.75654,2.43196) -- (5.75654,2.47986);
\draw [c] (5.75654,2.47986) -- (5.75654,2.5213);
\draw [c] (5.74172,2.47986) -- (5.75654,2.47986);
\draw [c] (5.75654,2.47986) -- (5.77136,2.47986);
\definecolor{c}{rgb}{0,0,0};
\colorlet{c}{kugray};
\draw [c] (5.78618,2.43169) -- (5.78618,2.47526);
\draw [c] (5.78618,2.47526) -- (5.78618,2.51342);
\draw [c] (5.77136,2.47526) -- (5.78618,2.47526);
\draw [c] (5.78618,2.47526) -- (5.80099,2.47526);
\definecolor{c}{rgb}{0,0,0};
\colorlet{c}{kugray};
\draw [c] (5.81581,2.39907) -- (5.81581,2.44526);
\draw [c] (5.81581,2.44526) -- (5.81581,2.48541);
\draw [c] (5.80099,2.44526) -- (5.81581,2.44526);
\draw [c] (5.81581,2.44526) -- (5.83063,2.44526);
\definecolor{c}{rgb}{0,0,0};
\colorlet{c}{kugray};
\draw [c] (5.84545,2.43121) -- (5.84545,2.47744);
\draw [c] (5.84545,2.47744) -- (5.84545,2.51762);
\draw [c] (5.83063,2.47744) -- (5.84545,2.47744);
\draw [c] (5.84545,2.47744) -- (5.86026,2.47744);
\definecolor{c}{rgb}{0,0,0};
\colorlet{c}{kugray};
\draw [c] (5.87508,2.29827) -- (5.87508,2.35564);
\draw [c] (5.87508,2.35564) -- (5.87508,2.40397);
\draw [c] (5.86026,2.35564) -- (5.87508,2.35564);
\draw [c] (5.87508,2.35564) -- (5.8899,2.35564);
\definecolor{c}{rgb}{0,0,0};
\colorlet{c}{kugray};
\draw [c] (5.90472,2.35345) -- (5.90472,2.40527);
\draw [c] (5.90472,2.40527) -- (5.90472,2.44961);
\draw [c] (5.8899,2.40527) -- (5.90472,2.40527);
\draw [c] (5.90472,2.40527) -- (5.91954,2.40527);
\definecolor{c}{rgb}{0,0,0};
\colorlet{c}{kugray};
\draw [c] (5.93435,2.28972) -- (5.93435,2.34179);
\draw [c] (5.93435,2.34179) -- (5.93435,2.38632);
\draw [c] (5.91954,2.34179) -- (5.93435,2.34179);
\draw [c] (5.93435,2.34179) -- (5.94917,2.34179);
\definecolor{c}{rgb}{0,0,0};
\colorlet{c}{kugray};
\draw [c] (5.96399,2.28856) -- (5.96399,2.34875);
\draw [c] (5.96399,2.34875) -- (5.96399,2.39907);
\draw [c] (5.94917,2.34875) -- (5.96399,2.34875);
\draw [c] (5.96399,2.34875) -- (5.97881,2.34875);
\definecolor{c}{rgb}{0,0,0};
\colorlet{c}{kugray};
\draw [c] (5.99363,2.34369) -- (5.99363,2.39484);
\draw [c] (5.99363,2.39484) -- (5.99363,2.43869);
\draw [c] (5.97881,2.39484) -- (5.99363,2.39484);
\draw [c] (5.99363,2.39484) -- (6.00844,2.39484);
\definecolor{c}{rgb}{0,0,0};
\colorlet{c}{kugray};
\draw [c] (6.02326,2.22004) -- (6.02326,2.27835);
\draw [c] (6.02326,2.27835) -- (6.02326,2.32735);
\draw [c] (6.00844,2.27835) -- (6.02326,2.27835);
\draw [c] (6.02326,2.27835) -- (6.03808,2.27835);
\definecolor{c}{rgb}{0,0,0};
\colorlet{c}{kugray};
\draw [c] (6.0529,2.24812) -- (6.0529,2.30663);
\draw [c] (6.0529,2.30663) -- (6.0529,2.35577);
\draw [c] (6.03808,2.30663) -- (6.0529,2.30663);
\draw [c] (6.0529,2.30663) -- (6.06772,2.30663);
\definecolor{c}{rgb}{0,0,0};
\colorlet{c}{kugray};
\draw [c] (6.08253,2.21392) -- (6.08253,2.27979);
\draw [c] (6.08253,2.27979) -- (6.08253,2.33402);
\draw [c] (6.06772,2.27979) -- (6.08253,2.27979);
\draw [c] (6.08253,2.27979) -- (6.09735,2.27979);
\definecolor{c}{rgb}{0,0,0};
\colorlet{c}{kugray};
\draw [c] (6.11217,2.28422) -- (6.11217,2.34425);
\draw [c] (6.11217,2.34425) -- (6.11217,2.39446);
\draw [c] (6.09735,2.34425) -- (6.11217,2.34425);
\draw [c] (6.11217,2.34425) -- (6.12699,2.34425);
\definecolor{c}{rgb}{0,0,0};
\colorlet{c}{kugray};
\draw [c] (6.1418,2.13526) -- (6.1418,2.20119);
\draw [c] (6.1418,2.20119) -- (6.1418,2.25547);
\draw [c] (6.12699,2.20119) -- (6.1418,2.20119);
\draw [c] (6.1418,2.20119) -- (6.15662,2.20119);
\definecolor{c}{rgb}{0,0,0};
\colorlet{c}{kugray};
\draw [c] (6.17144,2.21632) -- (6.17144,2.28052);
\draw [c] (6.17144,2.28052) -- (6.17144,2.33362);
\draw [c] (6.15662,2.28052) -- (6.17144,2.28052);
\draw [c] (6.17144,2.28052) -- (6.18626,2.28052);
\definecolor{c}{rgb}{0,0,0};
\colorlet{c}{kugray};
\draw [c] (6.20108,2.25826) -- (6.20108,2.32268);
\draw [c] (6.20108,2.32268) -- (6.20108,2.37592);
\draw [c] (6.18626,2.32268) -- (6.20108,2.32268);
\draw [c] (6.20108,2.32268) -- (6.21589,2.32268);
\definecolor{c}{rgb}{0,0,0};
\colorlet{c}{kugray};
\draw [c] (6.23071,2.21696) -- (6.23071,2.28153);
\draw [c] (6.23071,2.28153) -- (6.23071,2.33487);
\draw [c] (6.21589,2.28153) -- (6.23071,2.28153);
\draw [c] (6.23071,2.28153) -- (6.24553,2.28153);
\definecolor{c}{rgb}{0,0,0};
\colorlet{c}{kugray};
\draw [c] (6.26035,2.11717) -- (6.26035,2.18912);
\draw [c] (6.26035,2.18912) -- (6.26035,2.2474);
\draw [c] (6.24553,2.18912) -- (6.26035,2.18912);
\draw [c] (6.26035,2.18912) -- (6.27517,2.18912);
\definecolor{c}{rgb}{0,0,0};
\colorlet{c}{kugray};
\draw [c] (6.28998,2.21915) -- (6.28998,2.27945);
\draw [c] (6.28998,2.27945) -- (6.28998,2.32985);
\draw [c] (6.27517,2.27945) -- (6.28998,2.27945);
\draw [c] (6.28998,2.27945) -- (6.3048,2.27945);
\definecolor{c}{rgb}{0,0,0};
\colorlet{c}{kugray};
\draw [c] (6.31962,2.15067) -- (6.31962,2.22057);
\draw [c] (6.31962,2.22057) -- (6.31962,2.2775);
\draw [c] (6.3048,2.22057) -- (6.31962,2.22057);
\draw [c] (6.31962,2.22057) -- (6.33444,2.22057);
\definecolor{c}{rgb}{0,0,0};
\colorlet{c}{kugray};
\draw [c] (6.34926,2.08129) -- (6.34926,2.15905);
\draw [c] (6.34926,2.15905) -- (6.34926,2.22107);
\draw [c] (6.33444,2.15905) -- (6.34926,2.15905);
\draw [c] (6.34926,2.15905) -- (6.36407,2.15905);
\definecolor{c}{rgb}{0,0,0};
\colorlet{c}{kugray};
\draw [c] (6.37889,2.26481) -- (6.37889,2.32632);
\draw [c] (6.37889,2.32632) -- (6.37889,2.37756);
\draw [c] (6.36407,2.32632) -- (6.37889,2.32632);
\draw [c] (6.37889,2.32632) -- (6.39371,2.32632);
\definecolor{c}{rgb}{0,0,0};
\colorlet{c}{kugray};
\draw [c] (6.40853,2.10103) -- (6.40853,2.17287);
\draw [c] (6.40853,2.17287) -- (6.40853,2.23108);
\draw [c] (6.39371,2.17287) -- (6.40853,2.17287);
\draw [c] (6.40853,2.17287) -- (6.42334,2.17287);
\definecolor{c}{rgb}{0,0,0};
\colorlet{c}{kugray};
\draw [c] (6.43816,2.01805) -- (6.43816,2.10352);
\draw [c] (6.43816,2.10352) -- (6.43816,2.17034);
\draw [c] (6.42334,2.10352) -- (6.43816,2.10352);
\draw [c] (6.43816,2.10352) -- (6.45298,2.10352);
\definecolor{c}{rgb}{0,0,0};
\colorlet{c}{kugray};
\draw [c] (6.4678,2.13798) -- (6.4678,2.21314);
\draw [c] (6.4678,2.21314) -- (6.4678,2.27351);
\draw [c] (6.45298,2.21314) -- (6.4678,2.21314);
\draw [c] (6.4678,2.21314) -- (6.48262,2.21314);
\definecolor{c}{rgb}{0,0,0};
\colorlet{c}{kugray};
\draw [c] (6.49743,2.0628) -- (6.49743,2.13789);
\draw [c] (6.49743,2.13789) -- (6.49743,2.19821);
\draw [c] (6.48262,2.13789) -- (6.49743,2.13789);
\draw [c] (6.49743,2.13789) -- (6.51225,2.13789);
\definecolor{c}{rgb}{0,0,0};
\colorlet{c}{kugray};
\draw [c] (6.52707,2.03036) -- (6.52707,2.11014);
\draw [c] (6.52707,2.11014) -- (6.52707,2.17345);
\draw [c] (6.51225,2.11014) -- (6.52707,2.11014);
\draw [c] (6.52707,2.11014) -- (6.54189,2.11014);
\definecolor{c}{rgb}{0,0,0};
\colorlet{c}{kugray};
\draw [c] (6.55671,1.95941) -- (6.55671,2.05375);
\draw [c] (6.55671,2.05375) -- (6.55671,2.12586);
\draw [c] (6.54189,2.05375) -- (6.55671,2.05375);
\draw [c] (6.55671,2.05375) -- (6.57152,2.05375);
\definecolor{c}{rgb}{0,0,0};
\colorlet{c}{kugray};
\draw [c] (6.58634,2.0092) -- (6.58634,2.09944);
\draw [c] (6.58634,2.09944) -- (6.58634,2.16914);
\draw [c] (6.57152,2.09944) -- (6.58634,2.09944);
\draw [c] (6.58634,2.09944) -- (6.60116,2.09944);
\definecolor{c}{rgb}{0,0,0};
\colorlet{c}{kugray};
\draw [c] (6.61598,2.06823) -- (6.61598,2.14649);
\draw [c] (6.61598,2.14649) -- (6.61598,2.20883);
\draw [c] (6.60116,2.14649) -- (6.61598,2.14649);
\draw [c] (6.61598,2.14649) -- (6.63079,2.14649);
\definecolor{c}{rgb}{0,0,0};
\colorlet{c}{kugray};
\draw [c] (6.64561,1.91105) -- (6.64561,2.01133);
\draw [c] (6.64561,2.01133) -- (6.64561,2.08685);
\draw [c] (6.63079,2.01133) -- (6.64561,2.01133);
\draw [c] (6.64561,2.01133) -- (6.66043,2.01133);
\definecolor{c}{rgb}{0,0,0};
\colorlet{c}{kugray};
\draw [c] (6.67525,1.90644) -- (6.67525,2.01096);
\draw [c] (6.67525,2.01096) -- (6.67525,2.08884);
\draw [c] (6.66043,2.01096) -- (6.67525,2.01096);
\draw [c] (6.67525,2.01096) -- (6.69007,2.01096);
\definecolor{c}{rgb}{0,0,0};
\colorlet{c}{kugray};
\draw [c] (6.70488,1.98499) -- (6.70488,2.07197);
\draw [c] (6.70488,2.07197) -- (6.70488,2.13971);
\draw [c] (6.69007,2.07197) -- (6.70488,2.07197);
\draw [c] (6.70488,2.07197) -- (6.7197,2.07197);
\definecolor{c}{rgb}{0,0,0};
\colorlet{c}{kugray};
\draw [c] (6.73452,1.98323) -- (6.73452,2.07762);
\draw [c] (6.73452,2.07762) -- (6.73452,2.14976);
\draw [c] (6.7197,2.07762) -- (6.73452,2.07762);
\draw [c] (6.73452,2.07762) -- (6.74934,2.07762);
\definecolor{c}{rgb}{0,0,0};
\colorlet{c}{kugray};
\draw [c] (6.76416,1.8404) -- (6.76416,1.94614);
\draw [c] (6.76416,1.94614) -- (6.76416,2.0247);
\draw [c] (6.74934,1.94614) -- (6.76416,1.94614);
\draw [c] (6.76416,1.94614) -- (6.77897,1.94614);
\definecolor{c}{rgb}{0,0,0};
\colorlet{c}{kugray};
\draw [c] (6.79379,1.95326) -- (6.79379,2.04637);
\draw [c] (6.79379,2.04637) -- (6.79379,2.11775);
\draw [c] (6.77897,2.04637) -- (6.79379,2.04637);
\draw [c] (6.79379,2.04637) -- (6.80861,2.04637);
\definecolor{c}{rgb}{0,0,0};
\colorlet{c}{kugray};
\draw [c] (6.82343,1.84998) -- (6.82343,1.96142);
\draw [c] (6.82343,1.96142) -- (6.82343,2.04307);
\draw [c] (6.80861,1.96142) -- (6.82343,1.96142);
\draw [c] (6.82343,1.96142) -- (6.83824,1.96142);
\definecolor{c}{rgb}{0,0,0};
\colorlet{c}{kugray};
\draw [c] (6.85306,1.97353) -- (6.85306,2.06824);
\draw [c] (6.85306,2.06824) -- (6.85306,2.14058);
\draw [c] (6.83824,2.06824) -- (6.85306,2.06824);
\draw [c] (6.85306,2.06824) -- (6.86788,2.06824);
\definecolor{c}{rgb}{0,0,0};
\colorlet{c}{kugray};
\draw [c] (6.8827,1.94959) -- (6.8827,2.05345);
\draw [c] (6.8827,2.05345) -- (6.8827,2.13097);
\draw [c] (6.86788,2.05345) -- (6.8827,2.05345);
\draw [c] (6.8827,2.05345) -- (6.89752,2.05345);
\definecolor{c}{rgb}{0,0,0};
\colorlet{c}{kugray};
\draw [c] (6.91233,1.64738) -- (6.91233,1.796);
\draw [c] (6.91233,1.796) -- (6.91233,1.8958);
\draw [c] (6.89752,1.796) -- (6.91233,1.796);
\draw [c] (6.91233,1.796) -- (6.92715,1.796);
\definecolor{c}{rgb}{0,0,0};
\colorlet{c}{kugray};
\draw [c] (6.94197,1.81861) -- (6.94197,1.94633);
\draw [c] (6.94197,1.94633) -- (6.94197,2.03632);
\draw [c] (6.92715,1.94633) -- (6.94197,1.94633);
\draw [c] (6.94197,1.94633) -- (6.95679,1.94633);
\definecolor{c}{rgb}{0,0,0};
\colorlet{c}{kugray};
\draw [c] (6.97161,1.45467) -- (6.97161,1.67503);
\draw [c] (6.97161,1.67503) -- (6.97161,1.80215);
\draw [c] (6.95679,1.67503) -- (6.97161,1.67503);
\draw [c] (6.97161,1.67503) -- (6.98642,1.67503);
\definecolor{c}{rgb}{0,0,0};
\colorlet{c}{kugray};
\draw [c] (7.00124,1.92607) -- (7.00124,2.03252);
\draw [c] (7.00124,2.03252) -- (7.00124,2.11147);
\draw [c] (6.98642,2.03252) -- (7.00124,2.03252);
\draw [c] (7.00124,2.03252) -- (7.01606,2.03252);
\definecolor{c}{rgb}{0,0,0};
\colorlet{c}{kugray};
\draw [c] (7.03088,1.70142) -- (7.03088,1.83023);
\draw [c] (7.03088,1.83023) -- (7.03088,1.92076);
\draw [c] (7.01606,1.83023) -- (7.03088,1.83023);
\draw [c] (7.03088,1.83023) -- (7.0457,1.83023);
\definecolor{c}{rgb}{0,0,0};
\colorlet{c}{kugray};
\draw [c] (7.06051,1.69915) -- (7.06051,1.84157);
\draw [c] (7.06051,1.84157) -- (7.06051,1.93856);
\draw [c] (7.0457,1.84157) -- (7.06051,1.84157);
\draw [c] (7.06051,1.84157) -- (7.07533,1.84157);
\definecolor{c}{rgb}{0,0,0};
\colorlet{c}{kugray};
\draw [c] (7.09015,1.30062) -- (7.09015,1.58241);
\draw [c] (7.09015,1.58241) -- (7.09015,1.72703);
\draw [c] (7.07533,1.58241) -- (7.09015,1.58241);
\draw [c] (7.09015,1.58241) -- (7.10497,1.58241);
\definecolor{c}{rgb}{0,0,0};
\colorlet{c}{kugray};
\draw [c] (7.11978,1.6185) -- (7.11978,1.77873);
\draw [c] (7.11978,1.77873) -- (7.11978,1.88358);
\draw [c] (7.10497,1.77873) -- (7.11978,1.77873);
\draw [c] (7.11978,1.77873) -- (7.1346,1.77873);
\definecolor{c}{rgb}{0,0,0};
\colorlet{c}{kugray};
\draw [c] (7.14942,1.83806) -- (7.14942,1.95138);
\draw [c] (7.14942,1.95138) -- (7.14942,2.03403);
\draw [c] (7.1346,1.95138) -- (7.14942,1.95138);
\draw [c] (7.14942,1.95138) -- (7.16424,1.95138);
\definecolor{c}{rgb}{0,0,0};
\colorlet{c}{kugray};
\draw [c] (7.17906,1.68124) -- (7.17906,1.87494);
\draw [c] (7.17906,1.87494) -- (7.17906,1.99292);
\draw [c] (7.16424,1.87494) -- (7.17906,1.87494);
\draw [c] (7.17906,1.87494) -- (7.19387,1.87494);
\definecolor{c}{rgb}{0,0,0};
\colorlet{c}{kugray};
\draw [c] (7.20869,1.63815) -- (7.20869,1.80529);
\draw [c] (7.20869,1.80529) -- (7.20869,1.91302);
\draw [c] (7.19387,1.80529) -- (7.20869,1.80529);
\draw [c] (7.20869,1.80529) -- (7.22351,1.80529);
\definecolor{c}{rgb}{0,0,0};
\colorlet{c}{kugray};
\draw [c] (7.23833,1.534) -- (7.23833,1.72717);
\draw [c] (7.23833,1.72717) -- (7.23833,1.84495);
\draw [c] (7.22351,1.72717) -- (7.23833,1.72717);
\draw [c] (7.23833,1.72717) -- (7.25315,1.72717);
\definecolor{c}{rgb}{0,0,0};
\colorlet{c}{kugray};
\draw [c] (7.26796,1.44669) -- (7.26796,1.66186);
\draw [c] (7.26796,1.66186) -- (7.26796,1.78728);
\draw [c] (7.25315,1.66186) -- (7.26796,1.66186);
\draw [c] (7.26796,1.66186) -- (7.28278,1.66186);
\definecolor{c}{rgb}{0,0,0};
\colorlet{c}{kugray};
\draw [c] (7.2976,1.67008) -- (7.2976,1.80777);
\draw [c] (7.2976,1.80777) -- (7.2976,1.90257);
\draw [c] (7.28278,1.80777) -- (7.2976,1.80777);
\draw [c] (7.2976,1.80777) -- (7.31242,1.80777);
\definecolor{c}{rgb}{0,0,0};
\colorlet{c}{kugray};
\draw [c] (7.32724,1.55019) -- (7.32724,1.78423);
\draw [c] (7.32724,1.78423) -- (7.32724,1.91564);
\draw [c] (7.31242,1.78423) -- (7.32724,1.78423);
\draw [c] (7.32724,1.78423) -- (7.34205,1.78423);
\definecolor{c}{rgb}{0,0,0};
\colorlet{c}{kugray};
\draw [c] (7.35687,1.70986) -- (7.35687,1.88005);
\draw [c] (7.35687,1.88005) -- (7.35687,1.98902);
\draw [c] (7.34205,1.88005) -- (7.35687,1.88005);
\draw [c] (7.35687,1.88005) -- (7.37169,1.88005);
\definecolor{c}{rgb}{0,0,0};
\colorlet{c}{kugray};
\draw [c] (7.38651,1.39569) -- (7.38651,1.61683);
\draw [c] (7.38651,1.61683) -- (7.38651,1.7442);
\draw [c] (7.37169,1.61683) -- (7.38651,1.61683);
\draw [c] (7.38651,1.61683) -- (7.40132,1.61683);
\definecolor{c}{rgb}{0,0,0};
\colorlet{c}{kugray};
\draw [c] (7.41614,1.65215) -- (7.41614,1.82337);
\draw [c] (7.41614,1.82337) -- (7.41614,1.93274);
\draw [c] (7.40132,1.82337) -- (7.41614,1.82337);
\draw [c] (7.41614,1.82337) -- (7.43096,1.82337);
\definecolor{c}{rgb}{0,0,0};
\colorlet{c}{kugray};
\draw [c] (7.44578,1.61682) -- (7.44578,1.76694);
\draw [c] (7.44578,1.76694) -- (7.44578,1.86741);
\draw [c] (7.43096,1.76694) -- (7.44578,1.76694);
\draw [c] (7.44578,1.76694) -- (7.4606,1.76694);
\definecolor{c}{rgb}{0,0,0};
\colorlet{c}{kugray};
\draw [c] (7.47541,0.903192) -- (7.47541,1.30472);
\draw [c] (7.47541,1.30472) -- (7.47541,1.47322);
\draw [c] (7.4606,1.30472) -- (7.47541,1.30472);
\draw [c] (7.47541,1.30472) -- (7.49023,1.30472);
\definecolor{c}{rgb}{0,0,0};
\colorlet{c}{kugray};
\draw [c] (7.50505,1.17257) -- (7.50505,1.55106);
\draw [c] (7.50505,1.55106) -- (7.50505,1.71578);
\draw [c] (7.49023,1.55106) -- (7.50505,1.55106);
\draw [c] (7.50505,1.55106) -- (7.51987,1.55106);
\definecolor{c}{rgb}{0,0,0};
\colorlet{c}{kugray};
\draw [c] (7.53469,1.26039) -- (7.53469,1.5276);
\draw [c] (7.53469,1.5276) -- (7.53469,1.66846);
\draw [c] (7.51987,1.5276) -- (7.53469,1.5276);
\draw [c] (7.53469,1.5276) -- (7.5495,1.5276);
\definecolor{c}{rgb}{0,0,0};
\colorlet{c}{kugray};
\draw [c] (7.56432,1.35089) -- (7.56432,1.65389);
\draw [c] (7.56432,1.65389) -- (7.56432,1.80359);
\draw [c] (7.5495,1.65389) -- (7.56432,1.65389);
\draw [c] (7.56432,1.65389) -- (7.57914,1.65389);
\definecolor{c}{rgb}{0,0,0};
\colorlet{c}{kugray};
\draw [c] (7.59396,1.63685) -- (7.59396,1.81433);
\draw [c] (7.59396,1.81433) -- (7.59396,1.92619);
\draw [c] (7.57914,1.81433) -- (7.59396,1.81433);
\draw [c] (7.59396,1.81433) -- (7.60877,1.81433);
\definecolor{c}{rgb}{0,0,0};
\colorlet{c}{kugray};
\draw [c] (7.62359,1.2207) -- (7.62359,1.48705);
\draw [c] (7.62359,1.48705) -- (7.62359,1.62768);
\draw [c] (7.60877,1.48705) -- (7.62359,1.48705);
\draw [c] (7.62359,1.48705) -- (7.63841,1.48705);
\definecolor{c}{rgb}{0,0,0};
\colorlet{c}{kugray};
\draw [c] (7.65323,1.15339) -- (7.65323,1.44836);
\draw [c] (7.65323,1.44836) -- (7.65323,1.5962);
\draw [c] (7.63841,1.44836) -- (7.65323,1.44836);
\draw [c] (7.65323,1.44836) -- (7.66805,1.44836);
\definecolor{c}{rgb}{0,0,0};
\colorlet{c}{kugray};
\draw [c] (7.68286,0.824113) -- (7.68286,1.32118);
\draw [c] (7.68286,1.32118) -- (7.68286,1.50231);
\draw [c] (7.66805,1.32118) -- (7.68286,1.32118);
\draw [c] (7.68286,1.32118) -- (7.69768,1.32118);
\definecolor{c}{rgb}{0,0,0};
\colorlet{c}{kugray};
\draw [c] (7.7125,1.26126) -- (7.7125,1.5339);
\draw [c] (7.7125,1.5339) -- (7.7125,1.67619);
\draw [c] (7.69768,1.5339) -- (7.7125,1.5339);
\draw [c] (7.7125,1.5339) -- (7.72732,1.5339);
\definecolor{c}{rgb}{0,0,0};
\colorlet{c}{kugray};
\draw [c] (7.74214,1.48045) -- (7.74214,1.72479);
\draw [c] (7.74214,1.72479) -- (7.74214,1.85928);
\draw [c] (7.72732,1.72479) -- (7.74214,1.72479);
\draw [c] (7.74214,1.72479) -- (7.75695,1.72479);
\definecolor{c}{rgb}{0,0,0};
\colorlet{c}{kugray};
\draw [c] (7.77177,1.11205) -- (7.77177,1.40937);
\draw [c] (7.77177,1.40937) -- (7.77177,1.55776);
\draw [c] (7.75695,1.40937) -- (7.77177,1.40937);
\draw [c] (7.77177,1.40937) -- (7.78659,1.40937);
\definecolor{c}{rgb}{0,0,0};
\colorlet{c}{kugray};
\draw [c] (7.80141,1.34931) -- (7.80141,1.63981);
\draw [c] (7.80141,1.63981) -- (7.80141,1.78657);
\draw [c] (7.78659,1.63981) -- (7.80141,1.63981);
\draw [c] (7.80141,1.63981) -- (7.81623,1.63981);
\definecolor{c}{rgb}{0,0,0};
\colorlet{c}{kugray};
\draw [c] (7.83104,1.0292) -- (7.83104,1.41463);
\draw [c] (7.83104,1.41463) -- (7.83104,1.58052);
\draw [c] (7.81623,1.41463) -- (7.83104,1.41463);
\draw [c] (7.83104,1.41463) -- (7.84586,1.41463);
\definecolor{c}{rgb}{0,0,0};
\colorlet{c}{kugray};
\draw [c] (7.86068,0.596817) -- (7.86068,1.16974);
\draw [c] (7.86068,1.16974) -- (7.86068,1.38313);
\draw [c] (7.84586,1.16974) -- (7.86068,1.16974);
\draw [c] (7.86068,1.16974) -- (7.8755,1.16974);
\definecolor{c}{rgb}{0,0,0};
\colorlet{c}{kugray};
\draw [c] (7.89031,1.22506) -- (7.89031,1.49977);
\draw [c] (7.89031,1.49977) -- (7.89031,1.64259);
\draw [c] (7.8755,1.49977) -- (7.89031,1.49977);
\draw [c] (7.89031,1.49977) -- (7.90513,1.49977);
\definecolor{c}{rgb}{0,0,0};
\colorlet{c}{kugray};
\draw [c] (7.91995,1.37554) -- (7.91995,1.58913);
\draw [c] (7.91995,1.58913) -- (7.91995,1.71402);
\draw [c] (7.90513,1.58913) -- (7.91995,1.58913);
\draw [c] (7.91995,1.58913) -- (7.93477,1.58913);
\definecolor{c}{rgb}{0,0,0};
\colorlet{c}{kugray};
\draw [c] (7.94959,0.970556) -- (7.94959,1.35289);
\draw [c] (7.94959,1.35289) -- (7.94959,1.51826);
\draw [c] (7.93477,1.35289) -- (7.94959,1.35289);
\draw [c] (7.94959,1.35289) -- (7.9644,1.35289);
\definecolor{c}{rgb}{0,0,0};
\colorlet{c}{kugray};
\draw [c] (7.97922,1.42663) -- (7.97922,1.62932);
\draw [c] (7.97922,1.62932) -- (7.97922,1.75049);
\draw [c] (7.9644,1.62932) -- (7.97922,1.62932);
\draw [c] (7.97922,1.62932) -- (7.99404,1.62932);
\definecolor{c}{rgb}{0,0,0};
\colorlet{c}{kugray};
\draw [c] (8.00886,1.24387) -- (8.00886,1.51218);
\draw [c] (8.00886,1.51218) -- (8.00886,1.65334);
\draw [c] (7.99404,1.51218) -- (8.00886,1.51218);
\draw [c] (8.00886,1.51218) -- (8.02368,1.51218);
\definecolor{c}{rgb}{0,0,0};
\colorlet{c}{kugray};
\draw [c] (8.03849,1.45206) -- (8.03849,1.64214);
\draw [c] (8.03849,1.64214) -- (8.03849,1.75879);
\draw [c] (8.02368,1.64214) -- (8.03849,1.64214);
\draw [c] (8.03849,1.64214) -- (8.05331,1.64214);
\definecolor{c}{rgb}{0,0,0};
\colorlet{c}{kugray};
\draw [c] (8.09776,1.35157) -- (8.09776,1.5665);
\draw [c] (8.09776,1.5665) -- (8.09776,1.69184);
\draw [c] (8.08295,1.5665) -- (8.09776,1.5665);
\draw [c] (8.09776,1.5665) -- (8.11258,1.5665);
\definecolor{c}{rgb}{0,0,0};
\colorlet{c}{kugray};
\draw [c] (8.1274,1.02943) -- (8.1274,1.41864);
\draw [c] (8.1274,1.41864) -- (8.1274,1.58516);
\draw [c] (8.11258,1.41864) -- (8.1274,1.41864);
\draw [c] (8.1274,1.41864) -- (8.14222,1.41864);
\definecolor{c}{rgb}{0,0,0};
\colorlet{c}{kugray};
\draw [c] (8.18667,1.3217) -- (8.18667,1.60896);
\draw [c] (8.18667,1.60896) -- (8.18667,1.75493);
\draw [c] (8.17185,1.60896) -- (8.18667,1.60896);
\draw [c] (8.18667,1.60896) -- (8.20149,1.60896);
\definecolor{c}{rgb}{0,0,0};
\colorlet{c}{kugray};
\draw [c] (8.21631,1.44991) -- (8.21631,1.72488);
\draw [c] (8.21631,1.72488) -- (8.21631,1.86777);
\draw [c] (8.20149,1.72488) -- (8.21631,1.72488);
\draw [c] (8.21631,1.72488) -- (8.23113,1.72488);
\definecolor{c}{rgb}{0,0,0};
\colorlet{c}{kugray};
\draw [c] (8.24594,0.596817) -- (8.24594,1.18479);
\draw [c] (8.24594,1.18479) -- (8.24594,1.39818);
\draw [c] (8.23113,1.18479) -- (8.24594,1.18479);
\draw [c] (8.24594,1.18479) -- (8.26076,1.18479);
\definecolor{c}{rgb}{0,0,0};
\colorlet{c}{kugray};
\draw [c] (8.27558,1.33422) -- (8.27558,1.70646);
\draw [c] (8.27558,1.70646) -- (8.27558,1.8701);
\draw [c] (8.26076,1.70646) -- (8.27558,1.70646);
\draw [c] (8.27558,1.70646) -- (8.2904,1.70646);
\definecolor{c}{rgb}{0,0,0};
\colorlet{c}{kugray};
\draw [c] (8.30521,1.01154) -- (8.30521,1.39412);
\draw [c] (8.30521,1.39412) -- (8.30521,1.55954);
\draw [c] (8.2904,1.39412) -- (8.30521,1.39412);
\draw [c] (8.30521,1.39412) -- (8.32003,1.39412);
\definecolor{c}{rgb}{0,0,0};
\colorlet{c}{kugray};
\draw [c] (8.33485,1.16538) -- (8.33485,1.451);
\draw [c] (8.33485,1.451) -- (8.33485,1.59657);
\draw [c] (8.32003,1.451) -- (8.33485,1.451);
\draw [c] (8.33485,1.451) -- (8.34967,1.451);
\definecolor{c}{rgb}{0,0,0};
\colorlet{c}{kugray};
\draw [c] (8.36449,0.596817) -- (8.36449,1.27192);
\draw [c] (8.36449,1.27192) -- (8.36449,1.48532);
\draw [c] (8.34967,1.27192) -- (8.36449,1.27192);
\draw [c] (8.36449,1.27192) -- (8.3793,1.27192);
\definecolor{c}{rgb}{0,0,0};
\colorlet{c}{kugray};
\draw [c] (8.39412,0.596817) -- (8.39412,1.1669);
\draw [c] (8.39412,1.1669) -- (8.39412,1.38029);
\draw [c] (8.3793,1.1669) -- (8.39412,1.1669);
\draw [c] (8.39412,1.1669) -- (8.40894,1.1669);
\definecolor{c}{rgb}{0,0,0};
\colorlet{c}{kugray};
\draw [c] (8.45339,1.13633) -- (8.45339,1.56977);
\draw [c] (8.45339,1.56977) -- (8.45339,1.74299);
\draw [c] (8.43858,1.56977) -- (8.45339,1.56977);
\draw [c] (8.45339,1.56977) -- (8.46821,1.56977);
\definecolor{c}{rgb}{0,0,0};
\colorlet{c}{kugray};
\draw [c] (8.48303,0.596817) -- (8.48303,1.22594);
\draw [c] (8.48303,1.22594) -- (8.48303,1.43933);
\draw [c] (8.46821,1.22594) -- (8.48303,1.22594);
\draw [c] (8.48303,1.22594) -- (8.49785,1.22594);
\definecolor{c}{rgb}{0,0,0};
\colorlet{c}{kugray};
\draw [c] (8.5423,0.596817) -- (8.5423,1.08304);
\draw [c] (8.5423,1.08304) -- (8.5423,1.29643);
\draw [c] (8.52748,1.08304) -- (8.5423,1.08304);
\draw [c] (8.5423,1.08304) -- (8.55712,1.08304);
\definecolor{c}{rgb}{0,0,0};
\colorlet{c}{kugray};
\draw [c] (8.57194,0.596817) -- (8.57194,1.19595);
\draw [c] (8.57194,1.19595) -- (8.57194,1.40934);
\draw [c] (8.55712,1.19595) -- (8.57194,1.19595);
\draw [c] (8.57194,1.19595) -- (8.58675,1.19595);
\definecolor{c}{rgb}{0,0,0};
\colorlet{c}{kugray};
\draw [c] (8.66084,0.884632) -- (8.66084,1.27717);
\draw [c] (8.66084,1.27717) -- (8.66084,1.44423);
\draw [c] (8.64603,1.27717) -- (8.66084,1.27717);
\draw [c] (8.66084,1.27717) -- (8.67566,1.27717);
\definecolor{c}{rgb}{0,0,0};
\colorlet{c}{kugray};
\draw [c] (8.69048,0.596817) -- (8.69048,1.18479);
\draw [c] (8.69048,1.18479) -- (8.69048,1.39818);
\draw [c] (8.67566,1.18479) -- (8.69048,1.18479);
\draw [c] (8.69048,1.18479) -- (8.7053,1.18479);
\definecolor{c}{rgb}{0,0,0};
\colorlet{c}{kugray};
\draw [c] (8.72012,0.596817) -- (8.72012,1.21188);
\draw [c] (8.72012,1.21188) -- (8.72012,1.42527);
\draw [c] (8.7053,1.21188) -- (8.72012,1.21188);
\draw [c] (8.72012,1.21188) -- (8.73493,1.21188);
\definecolor{c}{rgb}{0,0,0};
\colorlet{c}{kugray};
\draw [c] (8.74975,0.596817) -- (8.74975,1.16598);
\draw [c] (8.74975,1.16598) -- (8.74975,1.37937);
\draw [c] (8.73493,1.16598) -- (8.74975,1.16598);
\draw [c] (8.74975,1.16598) -- (8.76457,1.16598);
\definecolor{c}{rgb}{0,0,0};
\colorlet{c}{kugray};
\draw [c] (8.77939,0.895648) -- (8.77939,1.31039);
\draw [c] (8.77939,1.31039) -- (8.77939,1.48091);
\draw [c] (8.76457,1.31039) -- (8.77939,1.31039);
\draw [c] (8.77939,1.31039) -- (8.79421,1.31039);
\definecolor{c}{rgb}{0,0,0};
\colorlet{c}{kugray};
\draw [c] (8.80902,1.29252) -- (8.80902,1.5583);
\draw [c] (8.80902,1.5583) -- (8.80902,1.69878);
\draw [c] (8.79421,1.5583) -- (8.80902,1.5583);
\draw [c] (8.80902,1.5583) -- (8.82384,1.5583);
\definecolor{c}{rgb}{0,0,0};
\colorlet{c}{kugray};
\draw [c] (8.83866,0.596817) -- (8.83866,1.24072);
\draw [c] (8.83866,1.24072) -- (8.83866,1.45411);
\draw [c] (8.82384,1.24072) -- (8.83866,1.24072);
\draw [c] (8.83866,1.24072) -- (8.85348,1.24072);
\definecolor{c}{rgb}{0,0,0};
\colorlet{c}{kugray};
\draw [c] (8.86829,0.596817) -- (8.86829,0.976443);
\draw [c] (8.86829,0.976443) -- (8.86829,1.18983);
\draw [c] (8.85348,0.976443) -- (8.86829,0.976443);
\draw [c] (8.86829,0.976443) -- (8.88311,0.976443);
\definecolor{c}{rgb}{0,0,0};
\colorlet{c}{kugray};
\draw [c] (8.92757,0.596817) -- (8.92757,1.19654);
\draw [c] (8.92757,1.19654) -- (8.92757,1.40993);
\draw [c] (8.91275,1.19654) -- (8.92757,1.19654);
\draw [c] (8.92757,1.19654) -- (8.94238,1.19654);
\definecolor{c}{rgb}{0,0,0};
\colorlet{c}{kugray};
\draw [c] (9.01647,0.596817) -- (9.01647,1.23526);
\draw [c] (9.01647,1.23526) -- (9.01647,1.44865);
\draw [c] (9.00166,1.23526) -- (9.01647,1.23526);
\draw [c] (9.01647,1.23526) -- (9.03129,1.23526);
\definecolor{c}{rgb}{0,0,0};
\colorlet{c}{kugray};
\draw [c] (9.07574,0.596817) -- (9.07574,1.10612);
\draw [c] (9.07574,1.10612) -- (9.07574,1.31951);
\draw [c] (9.06093,1.10612) -- (9.07574,1.10612);
\draw [c] (9.07574,1.10612) -- (9.09056,1.10612);
\definecolor{c}{rgb}{0,0,0};
\colorlet{c}{kugray};
\draw [c] (9.10538,0.596817) -- (9.10538,1.18963);
\draw [c] (9.10538,1.18963) -- (9.10538,1.40302);
\draw [c] (9.09056,1.18963) -- (9.10538,1.18963);
\draw [c] (9.10538,1.18963) -- (9.1202,1.18963);
\definecolor{c}{rgb}{0,0,0};
\colorlet{c}{kugray};
\draw [c] (9.13502,0.596817) -- (9.13502,1.14382);
\draw [c] (9.13502,1.14382) -- (9.13502,1.35721);
\draw [c] (9.1202,1.14382) -- (9.13502,1.14382);
\draw [c] (9.13502,1.14382) -- (9.14983,1.14382);
\definecolor{c}{rgb}{0,0,0};
\colorlet{c}{kugray};
\draw [c] (9.22392,1.0928) -- (9.22392,1.47755);
\draw [c] (9.22392,1.47755) -- (9.22392,1.64333);
\draw [c] (9.20911,1.47755) -- (9.22392,1.47755);
\draw [c] (9.22392,1.47755) -- (9.23874,1.47755);
\definecolor{c}{rgb}{0,0,0};
\colorlet{c}{kugray};
\draw [c] (9.2832,0.596817) -- (9.2832,1.19105);
\draw [c] (9.2832,1.19105) -- (9.2832,1.40444);
\draw [c] (9.26838,1.19105) -- (9.2832,1.19105);
\draw [c] (9.2832,1.19105) -- (9.29801,1.19105);
\definecolor{c}{rgb}{0,0,0};
\colorlet{c}{kugray};
\draw [c] (9.34247,0.596817) -- (9.34247,1.0722);
\draw [c] (9.34247,1.0722) -- (9.34247,1.28559);
\draw [c] (9.32765,1.0722) -- (9.34247,1.0722);
\draw [c] (9.34247,1.0722) -- (9.35728,1.0722);
\definecolor{c}{rgb}{0,0,0};
\colorlet{c}{kugray};
\draw [c] (9.54992,0.596817) -- (9.54992,1.17593);
\draw [c] (9.54992,1.17593) -- (9.54992,1.38932);
\draw [c] (9.5351,1.17593) -- (9.54992,1.17593);
\draw [c] (9.54992,1.17593) -- (9.56474,1.17593);
\definecolor{c}{rgb}{0,0,0};
\colorlet{c}{kugray};
\draw [c] (9.60919,0.596817) -- (9.60919,1.45456);
\draw [c] (9.60919,1.45456) -- (9.60919,1.66795);
\draw [c] (9.59437,1.45456) -- (9.60919,1.45456);
\draw [c] (9.60919,1.45456) -- (9.62401,1.45456);
\definecolor{c}{rgb}{0,0,0};
\colorlet{c}{kugray};
\draw [c] (9.63882,0.596817) -- (9.63882,1.11893);
\draw [c] (9.63882,1.11893) -- (9.63882,1.33232);
\draw [c] (9.62401,1.11893) -- (9.63882,1.11893);
\draw [c] (9.63882,1.11893) -- (9.65364,1.11893);
\definecolor{c}{rgb}{0,0,0};
\colorlet{c}{kugray};
\draw [c] (9.72773,0.596817) -- (9.72773,1.59979);
\draw [c] (9.72773,1.59979) -- (9.72773,1.81318);
\draw [c] (9.71291,1.59979) -- (9.72773,1.59979);
\draw [c] (9.72773,1.59979) -- (9.74255,1.59979);
\definecolor{c}{rgb}{0,0,0};
\colorlet{c}{kugray};
\draw [c] (9.787,0.596817) -- (9.787,1.11893);
\draw [c] (9.787,1.11893) -- (9.787,1.33232);
\draw [c] (9.77219,1.11893) -- (9.787,1.11893);
\draw [c] (9.787,1.11893) -- (9.80182,1.11893);
\definecolor{c}{rgb}{0,0,0};
\colorlet{c}{kugray};
\draw [c] (9.90555,0.596817) -- (9.90555,1.17321);
\draw [c] (9.90555,1.17321) -- (9.90555,1.3866);
\draw [c] (9.89073,1.17321) -- (9.90555,1.17321);
\draw [c] (9.90555,1.17321) -- (9.92036,1.17321);
\definecolor{c}{rgb}{0,0,0};
\colorlet{c}{kugray};
\draw [c] (9.93518,0.596817) -- (9.93518,1.17077);
\draw [c] (9.93518,1.17077) -- (9.93518,1.38417);
\draw [c] (9.92036,1.17077) -- (9.93518,1.17077);
\draw [c] (9.93518,1.17077) -- (9.95,1.17077);
\definecolor{c}{rgb}{0,0,0};
\colorlet{c}{natgreen};
\draw [c] (1.51655,5.54533) -- (1.60131,5.40419) -- (1.68607,5.2747) -- (1.77083,5.15478) -- (1.85558,5.04288) -- (1.94034,4.93778) -- (2.0251,4.83855) -- (2.10986,4.74443) -- (2.19462,4.65481) -- (2.27938,4.56919) -- (2.36413,4.48712)
 -- (2.44889,4.40827) -- (2.53365,4.33233) -- (2.61841,4.25902) -- (2.70317,4.18813) -- (2.78793,4.11945) -- (2.87268,4.05281) -- (2.95744,3.98806) -- (3.0422,3.92505) -- (3.12696,3.86367) -- (3.21172,3.80381) -- (3.29648,3.74536) --
 (3.38123,3.68824) -- (3.46599,3.63236) -- (3.55075,3.57765) -- (3.63551,3.52404) -- (3.72027,3.47148) -- (3.80502,3.4199) -- (3.88978,3.36925) -- (3.97454,3.31949) -- (4.0593,3.27056) -- (4.14406,3.22243) -- (4.22882,3.17506) -- (4.31357,3.12841) --
 (4.39833,3.08245) -- (4.48309,3.03714) -- (4.56785,2.99247) -- (4.65261,2.94839) -- (4.73737,2.90489) -- (4.82212,2.86194) -- (4.90688,2.81952) -- (4.99164,2.7776) -- (5.0764,2.73617) -- (5.16116,2.6952) -- (5.24592,2.65468) -- (5.33067,2.6146) --
 (5.41543,2.57492) -- (5.50019,2.53565) -- (5.58495,2.49676) -- (5.66971,2.45824);
\draw [c] (5.66971,2.45824) -- (5.75447,2.42007) -- (5.83922,2.38225) -- (5.92398,2.34476) -- (6.00874,2.30759) -- (6.0935,2.27073) -- (6.17826,2.23416) -- (6.26301,2.19789) -- (6.34777,2.16188) -- (6.43253,2.12615) --
 (6.51729,2.09068) -- (6.60205,2.05545) -- (6.68681,2.02047) -- (6.77156,1.98572) -- (6.85632,1.9512) -- (6.94108,1.91689) -- (7.02584,1.88279) -- (7.1106,1.8489) -- (7.19536,1.81521) -- (7.28011,1.78171) -- (7.36487,1.74839) -- (7.44963,1.71525) --
 (7.53439,1.68228) -- (7.61915,1.64948) -- (7.70391,1.61684) -- (7.78866,1.58435) -- (7.87342,1.55202) -- (7.95818,1.51983) -- (8.04294,1.48778) -- (8.1277,1.45586) -- (8.21246,1.42408) -- (8.29721,1.39242) -- (8.38197,1.36088) -- (8.46673,1.32946)
 -- (8.55149,1.29816) -- (8.63625,1.26696) -- (8.72101,1.23587) -- (8.80576,1.20487) -- (8.89052,1.17398) -- (8.97528,1.14317) -- (9.06004,1.11246) -- (9.1448,1.08183) -- (9.22955,1.05129) -- (9.31431,1.02082) -- (9.39907,0.990431) --
 (9.48383,0.960111) -- (9.56859,0.929859) -- (9.65335,0.899673) -- (9.7381,0.869549) -- (9.82286,0.839483);
\draw [c] (9.82286,0.839483) -- (9.90762,0.809473);
\definecolor{c}{rgb}{0,0,0};
\draw [c] (1,0.596817) -- (9.95,0.596817);
\draw [anchor= east] (9.95,0.0954907) node[color=c, rotate=0]{$M_{\gamma\gamma}\text{ [GeV]}$};
\draw [c] (1.02964,0.757062) -- (1.02964,0.596817);
\draw [c] (1.32599,0.67694) -- (1.32599,0.596817);
\draw [c] (1.62235,0.67694) -- (1.62235,0.596817);
\draw [c] (1.91871,0.67694) -- (1.91871,0.596817);
\draw [c] (2.21507,0.67694) -- (2.21507,0.596817);
\draw [c] (2.51142,0.757062) -- (2.51142,0.596817);
\draw [c] (2.80778,0.67694) -- (2.80778,0.596817);
\draw [c] (3.10414,0.67694) -- (3.10414,0.596817);
\draw [c] (3.4005,0.67694) -- (3.4005,0.596817);
\draw [c] (3.69685,0.67694) -- (3.69685,0.596817);
\draw [c] (3.99321,0.757062) -- (3.99321,0.596817);
\draw [c] (4.28957,0.67694) -- (4.28957,0.596817);
\draw [c] (4.58593,0.67694) -- (4.58593,0.596817);
\draw [c] (4.88228,0.67694) -- (4.88228,0.596817);
\draw [c] (5.17864,0.67694) -- (5.17864,0.596817);
\draw [c] (5.475,0.757062) -- (5.475,0.596817);
\draw [c] (5.77136,0.67694) -- (5.77136,0.596817);
\draw [c] (6.06772,0.67694) -- (6.06772,0.596817);
\draw [c] (6.36407,0.67694) -- (6.36407,0.596817);
\draw [c] (6.66043,0.67694) -- (6.66043,0.596817);
\draw [c] (6.95679,0.757062) -- (6.95679,0.596817);
\draw [c] (7.25315,0.67694) -- (7.25315,0.596817);
\draw [c] (7.5495,0.67694) -- (7.5495,0.596817);
\draw [c] (7.84586,0.67694) -- (7.84586,0.596817);
\draw [c] (8.14222,0.67694) -- (8.14222,0.596817);
\draw [c] (8.43858,0.757062) -- (8.43858,0.596817);
\draw [c] (8.73493,0.67694) -- (8.73493,0.596817);
\draw [c] (9.03129,0.67694) -- (9.03129,0.596817);
\draw [c] (9.32765,0.67694) -- (9.32765,0.596817);
\draw [c] (9.62401,0.67694) -- (9.62401,0.596817);
\draw [c] (9.92036,0.757062) -- (9.92036,0.596817);
\draw [c] (1.02964,0.757062) -- (1.02964,0.596817);
\draw [c] (9.92036,0.757062) -- (9.92036,0.596817);
\draw [anchor=base] (1.02964,0.310345) node[color=c, rotate=0]{0};
\draw [anchor=base] (2.51142,0.310345) node[color=c, rotate=0]{500};
\draw [anchor=base] (3.99321,0.310345) node[color=c, rotate=0]{1000};
\draw [anchor=base] (5.475,0.310345) node[color=c, rotate=0]{1500};
\draw [anchor=base] (6.95679,0.310345) node[color=c, rotate=0]{2000};
\draw [anchor=base] (8.43858,0.310345) node[color=c, rotate=0]{2500};
\draw [anchor=base] (9.92036,0.310345) node[color=c, rotate=0]{3000};
\draw [c] (1,0.596817) -- (1,5.90849);
\draw [anchor= east] (-0.12,5.90849) node[color=c, rotate=90]{Number of events};
\draw [c] (1.1335,0.632213) -- (1,0.632213);
\draw [c] (1.1335,0.668474) -- (1,0.668474);
\draw [c] (1.267,0.70091) -- (1,0.70091);
\draw [anchor= east] (0.922,0.70091) node[color=c, rotate=0]{$10^{-4}$};
\draw [c] (1.1335,0.9143) -- (1,0.9143);
\draw [c] (1.1335,1.03913) -- (1,1.03913);
\draw [c] (1.1335,1.12769) -- (1,1.12769);
\draw [c] (1.1335,1.19639) -- (1,1.19639);
\draw [c] (1.1335,1.25252) -- (1,1.25252);
\draw [c] (1.1335,1.29997) -- (1,1.29997);
\draw [c] (1.1335,1.34108) -- (1,1.34108);
\draw [c] (1.1335,1.37734) -- (1,1.37734);
\draw [c] (1.267,1.40978) -- (1,1.40978);
\draw [anchor= east] (0.922,1.40978) node[color=c, rotate=0]{$10^{-3}$};
\draw [c] (1.1335,1.62317) -- (1,1.62317);
\draw [c] (1.1335,1.74799) -- (1,1.74799);
\draw [c] (1.1335,1.83656) -- (1,1.83656);
\draw [c] (1.1335,1.90525) -- (1,1.90525);
\draw [c] (1.1335,1.96138) -- (1,1.96138);
\draw [c] (1.1335,2.00884) -- (1,2.00884);
\draw [c] (1.1335,2.04995) -- (1,2.04995);
\draw [c] (1.1335,2.08621) -- (1,2.08621);
\draw [c] (1.267,2.11864) -- (1,2.11864);
\draw [anchor= east] (0.922,2.11864) node[color=c, rotate=0]{$10^{-2}$};
\draw [c] (1.1335,2.33203) -- (1,2.33203);
\draw [c] (1.1335,2.45686) -- (1,2.45686);
\draw [c] (1.1335,2.54542) -- (1,2.54542);
\draw [c] (1.1335,2.61412) -- (1,2.61412);
\draw [c] (1.1335,2.67025) -- (1,2.67025);
\draw [c] (1.1335,2.71771) -- (1,2.71771);
\draw [c] (1.1335,2.75882) -- (1,2.75882);
\draw [c] (1.1335,2.79508) -- (1,2.79508);
\draw [c] (1.267,2.82751) -- (1,2.82751);
\draw [anchor= east] (0.922,2.82751) node[color=c, rotate=0]{$10^{-1}$};
\draw [c] (1.1335,3.0409) -- (1,3.0409);
\draw [c] (1.1335,3.16573) -- (1,3.16573);
\draw [c] (1.1335,3.25429) -- (1,3.25429);
\draw [c] (1.1335,3.32299) -- (1,3.32299);
\draw [c] (1.1335,3.37912) -- (1,3.37912);
\draw [c] (1.1335,3.42657) -- (1,3.42657);
\draw [c] (1.1335,3.46768) -- (1,3.46768);
\draw [c] (1.1335,3.50394) -- (1,3.50394);
\draw [c] (1.267,3.53638) -- (1,3.53638);
\draw [anchor= east] (0.922,3.53638) node[color=c, rotate=0]{1};
\draw [c] (1.1335,3.74977) -- (1,3.74977);
\draw [c] (1.1335,3.87459) -- (1,3.87459);
\draw [c] (1.1335,3.96316) -- (1,3.96316);
\draw [c] (1.1335,4.03186) -- (1,4.03186);
\draw [c] (1.1335,4.08799) -- (1,4.08799);
\draw [c] (1.1335,4.13544) -- (1,4.13544);
\draw [c] (1.1335,4.17655) -- (1,4.17655);
\draw [c] (1.1335,4.21281) -- (1,4.21281);
\draw [c] (1.267,4.24525) -- (1,4.24525);
\draw [anchor= east] (0.922,4.24525) node[color=c, rotate=0]{10};
\draw [c] (1.1335,4.45864) -- (1,4.45864);
\draw [c] (1.1335,4.58346) -- (1,4.58346);
\draw [c] (1.1335,4.67203) -- (1,4.67203);
\draw [c] (1.1335,4.74072) -- (1,4.74072);
\draw [c] (1.1335,4.79685) -- (1,4.79685);
\draw [c] (1.1335,4.84431) -- (1,4.84431);
\draw [c] (1.1335,4.88542) -- (1,4.88542);
\draw [c] (1.1335,4.92168) -- (1,4.92168);
\draw [c] (1.267,4.95411) -- (1,4.95411);
\draw [anchor= east] (0.922,4.95411) node[color=c, rotate=0]{$10^{2}$};
\draw [c] (1.1335,5.1675) -- (1,5.1675);
\draw [c] (1.1335,5.29233) -- (1,5.29233);
\draw [c] (1.1335,5.38089) -- (1,5.38089);
\draw [c] (1.1335,5.44959) -- (1,5.44959);
\draw [c] (1.1335,5.50572) -- (1,5.50572);
\draw [c] (1.1335,5.55318) -- (1,5.55318);
\draw [c] (1.1335,5.59428) -- (1,5.59428);
\draw [c] (1.1335,5.63055) -- (1,5.63055);
\draw [c] (1.267,5.66298) -- (1,5.66298);
\draw [anchor= east] (0.922,5.66298) node[color=c, rotate=0]{$10^{3}$};
\draw [c] (1.1335,5.87637) -- (1,5.87637);
\colorlet{c}{natgreen!40};
\draw [c, fill=c] (1.45951,0.596817) -- (1.48904,4.38696) -- (1.51857,5.54721) -- (1.54811,5.49618) -- (1.57764,5.44687) -- (1.60717,5.39917) -- (1.6367,5.35293) -- (1.66623,5.30805) -- (1.69576,5.26443) -- (1.7253,5.22198) -- (1.75483,5.18061) --
 (1.78436,5.14026) -- (1.81389,5.10086) -- (1.84342,5.06235) -- (1.87296,5.02469) -- (1.90249,4.98782) -- (1.93202,4.9517) -- (1.96155,4.91631) -- (1.99108,4.88159) -- (2.02061,4.84752) -- (2.05015,4.81408) -- (2.07968,4.78122) -- (2.10921,4.74894)
 -- (2.13874,4.71719) -- (2.16827,4.68597) -- (2.1978,4.65524) -- (2.22734,4.625) -- (2.25687,4.59522) -- (2.2864,4.56589) -- (2.31593,4.53698) -- (2.34546,4.50849) -- (2.375,4.48039) -- (2.40453,4.45268) -- (2.43406,4.42535) -- (2.46359,4.39838) --
 (2.49312,4.37175) -- (2.52265,4.34546) -- (2.55219,4.3195) -- (2.58172,4.29385) -- (2.61125,4.26852) -- (2.64078,4.24348) -- (2.67031,4.21873) -- (2.69984,4.19426) -- (2.72938,4.17006) -- (2.75891,4.14613) -- (2.78844,4.12245) -- (2.81797,4.09903)
 -- (2.8475,4.07584) -- (2.87704,4.0529) -- (2.90657,4.03018) -- (2.9361,4.00769) -- (2.96563,3.98541) -- (2.99516,3.96335) -- (3.02469,3.94149) -- (3.05423,3.91984) -- (3.08376,3.89838) -- (3.11329,3.87712) -- (3.14282,3.85604) -- (3.17235,3.83514)
 -- (3.20189,3.81442) -- (3.23142,3.79387) -- (3.26095,3.7735) -- (3.29048,3.75328) -- (3.32001,3.73323) -- (3.34954,3.71334) -- (3.37908,3.6936) -- (3.40861,3.67402) -- (3.43814,3.65458) -- (3.46767,3.63528) -- (3.4972,3.61613) -- (3.52673,3.59712)
 -- (3.55627,3.57824) -- (3.5858,3.55949) -- (3.61533,3.54087) -- (3.64486,3.52238) -- (3.67439,3.50402) -- (3.70393,3.48578) -- (3.73346,3.46765) -- (3.76299,3.44965) -- (3.79252,3.43176) -- (3.82205,3.41399) -- (3.85158,3.39632) --
 (3.88112,3.37876) -- (3.91065,3.36132) -- (3.94018,3.34397) -- (3.96971,3.32673) -- (3.99924,3.30959) -- (4.02877,3.29255) -- (4.05831,3.27561) -- (4.08784,3.25877) -- (4.11737,3.24202) -- (4.1469,3.22536) -- (4.17643,3.2088) -- (4.20597,3.19232) --
 (4.2355,3.17594) -- (4.26503,3.15964) -- (4.29456,3.14343) -- (4.32409,3.1273) -- (4.35362,3.11126) -- (4.38316,3.09529) -- (4.41269,3.07941) -- (4.44222,3.06361) -- (4.47175,3.04789) -- (4.50128,3.03225) -- (4.53082,3.01668) -- (4.56035,3.00119) --
 (4.58988,2.98577) -- (4.61941,2.97043) -- (4.64894,2.95515) -- (4.67847,2.93995) -- (4.70801,2.92482) -- (4.73754,2.90977) -- (4.76707,2.89477) -- (4.7966,2.87985) -- (4.82613,2.865) -- (4.85566,2.85021) -- (4.8852,2.83548) -- (4.91473,2.82083) --
 (4.94426,2.80623) -- (4.97379,2.7917) -- (5.00332,2.77723) -- (5.03286,2.76282) -- (5.06239,2.74848) -- (5.09192,2.73419) -- (5.12145,2.71996) -- (5.15098,2.70579) -- (5.18051,2.69168) -- (5.21005,2.67763) -- (5.23958,2.66364) -- (5.26911,2.6497) --
 (5.29864,2.63581) -- (5.32817,2.62198) -- (5.3577,2.60821) -- (5.38724,2.59448) -- (5.41677,2.58081) -- (5.4463,2.5672) -- (5.47583,2.55363) -- (5.50536,2.54011) -- (5.5349,2.52665) -- (5.56443,2.51323) -- (5.59396,2.49987) -- (5.62349,2.48655) --
 (5.65302,2.47328) -- (5.68255,2.46005) -- (5.71209,2.44688) -- (5.74162,2.43375) -- (5.77115,2.42066) -- (5.80068,2.40762) -- (5.83021,2.39462) -- (5.85975,2.38167) -- (5.88928,2.36876) -- (5.91881,2.3559) -- (5.94834,2.34307) -- (5.97787,2.33029)
 -- (6.0074,2.31755) -- (6.03694,2.30485) -- (6.06647,2.29219) -- (6.096,2.27957) -- (6.12553,2.26699) -- (6.15506,2.25444) -- (6.18459,2.24194) -- (6.21413,2.22947) -- (6.24366,2.21704) -- (6.27319,2.20465) -- (6.30272,2.1923) -- (6.33225,2.17998)
 -- (6.36179,2.16769) -- (6.39132,2.15544) -- (6.42085,2.14323) -- (6.45038,2.13105) -- (6.47991,2.1189) -- (6.50944,2.10679) -- (6.53898,2.09471) -- (6.56851,2.08267) -- (6.59804,2.07065) -- (6.62757,2.05867) -- (6.6571,2.04672) -- (6.68663,2.03481)
 -- (6.71617,2.02292) -- (6.7457,2.01106) -- (6.77523,1.99923) -- (6.80476,1.98744) -- (6.83429,1.97567) -- (6.86383,1.96393) -- (6.89336,1.95222) -- (6.92289,1.94054) -- (6.95242,1.92889) -- (6.98195,1.91727) -- (7.01148,1.90567) -- (7.04102,1.8941)
 -- (7.07055,1.88256) -- (7.10008,1.87104) -- (7.12961,1.85955) -- (7.15914,1.84809) -- (7.18867,1.83665) -- (7.21821,1.82524) -- (7.24774,1.81385) -- (7.27727,1.80249) -- (7.3068,1.79115) -- (7.33633,1.77984) -- (7.36587,1.76855) -- (7.3954,1.75728)
 -- (7.42493,1.74604) -- (7.45446,1.73482) -- (7.48399,1.72363) -- (7.51352,1.71245) -- (7.54306,1.7013) -- (7.57259,1.69018) -- (7.60212,1.67907) -- (7.63165,1.66798) -- (7.66118,1.65692) -- (7.69072,1.64588) -- (7.72025,1.63486) --
 (7.74978,1.62386) -- (7.77931,1.61288) -- (7.80884,1.60192) -- (7.83837,1.59098) -- (7.86791,1.58006) -- (7.89744,1.56916) -- (7.92697,1.55828) -- (7.9565,1.54742) -- (7.98603,1.53658) -- (8.01556,1.52576) -- (8.0451,1.51495) -- (8.07463,1.50417) --
 (8.10416,1.4934) -- (8.13369,1.48265) -- (8.16322,1.47191) -- (8.19276,1.4612) -- (8.22229,1.4505) -- (8.25182,1.43982) -- (8.28135,1.42915) -- (8.31088,1.41851) -- (8.34041,1.40788) -- (8.36995,1.39726) -- (8.39948,1.38666) -- (8.42901,1.37608) --
 (8.45854,1.36551) -- (8.48807,1.35496) -- (8.5176,1.34443) -- (8.54714,1.3339) -- (8.57667,1.3234) -- (8.6062,1.31291) -- (8.63573,1.30243) -- (8.66526,1.29197) -- (8.6948,1.28152) -- (8.72433,1.27109) -- (8.75386,1.26067) -- (8.78339,1.25026) --
 (8.81292,1.23987) -- (8.84245,1.22949) -- (8.87199,1.21912) -- (8.90152,1.20877) -- (8.93105,1.19843) -- (8.96058,1.1881) -- (8.99011,1.17779) -- (9.01964,1.16748) -- (9.04918,1.15719) -- (9.07871,1.14692) -- (9.10824,1.13665) -- (9.13777,1.12639)
 -- (9.1673,1.11615) -- (9.19684,1.10592) -- (9.22637,1.0957) -- (9.2559,1.08549) -- (9.28543,1.07529) -- (9.31496,1.06511) -- (9.34449,1.05493) -- (9.37403,1.04476) -- (9.40356,1.03461) -- (9.43309,1.02446) -- (9.46262,1.01433) -- (9.49215,1.0042)
 -- (9.52169,0.994085) -- (9.55122,0.983979) -- (9.58075,0.973882) -- (9.61028,0.963795) -- (9.63981,0.953717) -- (9.66934,0.943648) -- (9.69888,0.933588) -- (9.72841,0.923537) -- (9.75794,0.913495) -- (9.78747,0.90346) -- (9.817,0.893435) --
 (9.84653,0.883417) -- (9.87607,0.873408) -- (9.9056,0.810189) -- (9.93513,0.596817) -- (9.93513,0.596817) -- (9.9056,0.810189) -- (9.87607,0.756916) -- (9.84653,0.768025) -- (9.817,0.779137) -- (9.78747,0.790251) -- (9.75794,0.801367) --
 (9.72841,0.812487) -- (9.69888,0.823609) -- (9.66934,0.834734) -- (9.63981,0.845862) -- (9.61028,0.856994) -- (9.58075,0.868129) -- (9.55122,0.879268) -- (9.52169,0.89041) -- (9.49215,0.901557) -- (9.46262,0.912708) -- (9.43309,0.923863) --
 (9.40356,0.935022) -- (9.37403,0.946186) -- (9.34449,0.957355) -- (9.31496,0.968529) -- (9.28543,0.979708) -- (9.2559,0.990893) -- (9.22637,1.00208) -- (9.19684,1.01328) -- (9.1673,1.02448) -- (9.13777,1.03569) -- (9.10824,1.0469) --
 (9.07871,1.05812) -- (9.04918,1.06935) -- (9.01964,1.08058) -- (8.99011,1.09182) -- (8.96058,1.10306) -- (8.93105,1.11432) -- (8.90152,1.12558) -- (8.87199,1.13685) -- (8.84245,1.14813) -- (8.81292,1.15941) -- (8.78339,1.1707) -- (8.75386,1.182) --
 (8.72433,1.19331) -- (8.6948,1.20463) -- (8.66526,1.21596) -- (8.63573,1.22729) -- (8.6062,1.23863) -- (8.57667,1.24999) -- (8.54714,1.26135) -- (8.5176,1.27272) -- (8.48807,1.28411) -- (8.45854,1.2955) -- (8.42901,1.3069) -- (8.39948,1.31832) --
 (8.36995,1.32974) -- (8.34041,1.34118) -- (8.31088,1.35262) -- (8.28135,1.36408) -- (8.25182,1.37555) -- (8.22229,1.38703) -- (8.19276,1.39852) -- (8.16322,1.41003) -- (8.13369,1.42154) -- (8.10416,1.43307) -- (8.07463,1.44461) -- (8.0451,1.45617)
 -- (8.01556,1.46774) -- (7.98603,1.47932) -- (7.9565,1.49091) -- (7.92697,1.50252) -- (7.89744,1.51414) -- (7.86791,1.52578) -- (7.83837,1.53743) -- (7.80884,1.5491) -- (7.77931,1.56078) -- (7.74978,1.57247) -- (7.72025,1.58418) -- (7.69072,1.59591)
 -- (7.66118,1.60765) -- (7.63165,1.61941) -- (7.60212,1.63118) -- (7.57259,1.64297) -- (7.54306,1.65478) -- (7.51352,1.6666) -- (7.48399,1.67844) -- (7.45446,1.6903) -- (7.42493,1.70217) -- (7.3954,1.71407) -- (7.36587,1.72598) -- (7.33633,1.73791)
 -- (7.3068,1.74986) -- (7.27727,1.76183) -- (7.24774,1.77381) -- (7.21821,1.78582) -- (7.18867,1.79785) -- (7.15914,1.80989) -- (7.12961,1.82196) -- (7.10008,1.83405) -- (7.07055,1.84615) -- (7.04102,1.85828) -- (7.01148,1.87043) --
 (6.98195,1.88261) -- (6.95242,1.8948) -- (6.92289,1.90701) -- (6.89336,1.91925) -- (6.86383,1.93151) -- (6.83429,1.9438) -- (6.80476,1.95611) -- (6.77523,1.96844) -- (6.7457,1.9808) -- (6.71617,1.99318) -- (6.68663,2.00558) -- (6.6571,2.01801) --
 (6.62757,2.03047) -- (6.59804,2.04295) -- (6.56851,2.05545) -- (6.53898,2.06799) -- (6.50944,2.08055) -- (6.47991,2.09313) -- (6.45038,2.10575) -- (6.42085,2.11839) -- (6.39132,2.13106) -- (6.36179,2.14376) -- (6.33225,2.15649) -- (6.30272,2.16925)
 -- (6.27319,2.18203) -- (6.24366,2.19485) -- (6.21413,2.2077) -- (6.18459,2.22057) -- (6.15506,2.23348) -- (6.12553,2.24642) -- (6.096,2.2594) -- (6.06647,2.2724) -- (6.03694,2.28544) -- (6.0074,2.29851) -- (5.97787,2.31161) -- (5.94834,2.32475) --
 (5.91881,2.33793) -- (5.88928,2.35113) -- (5.85975,2.36438) -- (5.83021,2.37766) -- (5.80068,2.39098) -- (5.77115,2.40433) -- (5.74162,2.41772) -- (5.71209,2.43115) -- (5.68255,2.44462) -- (5.65302,2.45812) -- (5.62349,2.47167) -- (5.59396,2.48526)
 -- (5.56443,2.49888) -- (5.5349,2.51255) -- (5.50536,2.52626) -- (5.47583,2.54002) -- (5.4463,2.55381) -- (5.41677,2.56765) -- (5.38724,2.58154) -- (5.3577,2.59546) -- (5.32817,2.60944) -- (5.29864,2.62346) -- (5.26911,2.63753) -- (5.23958,2.65165)
 -- (5.21005,2.66581) -- (5.18051,2.68003) -- (5.15098,2.69429) -- (5.12145,2.70861) -- (5.09192,2.72297) -- (5.06239,2.73739) -- (5.03286,2.75187) -- (5.00332,2.7664) -- (4.97379,2.78098) -- (4.94426,2.79562) -- (4.91473,2.81032) -- (4.8852,2.82507)
 -- (4.85566,2.83989) -- (4.82613,2.85476) -- (4.7966,2.8697) -- (4.76707,2.8847) -- (4.73754,2.89976) -- (4.70801,2.91488) -- (4.67847,2.93008) -- (4.64894,2.94534) -- (4.61941,2.96066) -- (4.58988,2.97606) -- (4.56035,2.99153) -- (4.53082,3.00707)
 -- (4.50128,3.02268) -- (4.47175,3.03837) -- (4.44222,3.05413) -- (4.41269,3.06997) -- (4.38316,3.08589) -- (4.35362,3.10188) -- (4.32409,3.11796) -- (4.29456,3.13412) -- (4.26503,3.15037) -- (4.2355,3.1667) -- (4.20597,3.18312) -- (4.17643,3.19962)
 -- (4.1469,3.21622) -- (4.11737,3.23291) -- (4.08784,3.2497) -- (4.05831,3.26657) -- (4.02877,3.28355) -- (3.99924,3.30062) -- (3.96971,3.3178) -- (3.94018,3.33508) -- (3.91065,3.35246) -- (3.88112,3.36995) -- (3.85158,3.38755) -- (3.82205,3.40525)
 -- (3.79252,3.42308) -- (3.76299,3.44101) -- (3.73346,3.45906) -- (3.70393,3.47723) -- (3.67439,3.49553) -- (3.64486,3.51394) -- (3.61533,3.53249) -- (3.5858,3.55116) -- (3.55627,3.56996) -- (3.52673,3.5889) -- (3.4972,3.60797) -- (3.46767,3.62719)
 -- (3.43814,3.64654) -- (3.40861,3.66604) -- (3.37908,3.68569) -- (3.34954,3.70549) -- (3.32001,3.72545) -- (3.29048,3.74556) -- (3.26095,3.76584) -- (3.23142,3.78628) -- (3.20189,3.80689) -- (3.17235,3.82767) -- (3.14282,3.84863) --
 (3.11329,3.86977) -- (3.08376,3.8911) -- (3.05423,3.91261) -- (3.02469,3.93432) -- (2.99516,3.95623) -- (2.96563,3.97834) -- (2.9361,4.00066) -- (2.90657,4.0232) -- (2.87704,4.04595) -- (2.8475,4.06893) -- (2.81797,4.09214) -- (2.78844,4.11559) --
 (2.75891,4.13928) -- (2.72938,4.16323) -- (2.69984,4.18743) -- (2.67031,4.2119) -- (2.64078,4.23665) -- (2.61125,4.26168) -- (2.58172,4.287) -- (2.55219,4.31262) -- (2.52265,4.33855) -- (2.49312,4.3648) -- (2.46359,4.39139) -- (2.43406,4.41832) --
 (2.40453,4.44561) -- (2.375,4.47326) -- (2.34546,4.5013) -- (2.31593,4.52973) -- (2.2864,4.55858) -- (2.25687,4.58785) -- (2.22734,4.61757) -- (2.1978,4.64776) -- (2.16827,4.67842) -- (2.13874,4.70959) -- (2.10921,4.74129) -- (2.07968,4.77353) --
 (2.05015,4.80635) -- (2.02061,4.83976) -- (1.99108,4.87381) -- (1.96155,4.90851) -- (1.93202,4.9439) -- (1.90249,4.98002) -- (1.87296,5.0169) -- (1.84342,5.05458) -- (1.81389,5.0931) -- (1.78436,5.13251) -- (1.75483,5.17286) -- (1.7253,5.2142) --
 (1.69576,5.25657) -- (1.66623,5.30005) -- (1.6367,5.34468) -- (1.60717,5.39055) -- (1.57764,5.43771) -- (1.54811,5.48626) -- (1.51857,5.5363) -- (1.48904,0.596817) -- (1.45951,0.596817);
\definecolor{c}{rgb}{0,0,0};
\colorlet{c}{natcomp!40};
\draw [c, fill=c] (1.45951,0.596817) -- (1.48904,4.38693) -- (1.51857,5.54742) -- (1.54811,5.49642) -- (1.57764,5.44715) -- (1.60717,5.39949) -- (1.6367,5.3533) -- (1.66623,5.30848) -- (1.69576,5.26492) -- (1.7253,5.22253) -- (1.75483,5.18123) --
 (1.78436,5.14096) -- (1.81389,5.10164) -- (1.84342,5.06323) -- (1.87296,5.02567) -- (1.90249,4.98891) -- (1.93202,4.95292) -- (1.96155,4.91766) -- (1.99108,4.88309) -- (2.02061,4.84917) -- (2.05015,4.8159) -- (2.07968,4.78322) -- (2.10921,4.75113)
 -- (2.13874,4.7196) -- (2.16827,4.6886) -- (2.1978,4.65812) -- (2.22734,4.62814) -- (2.25687,4.59863) -- (2.2864,4.56959) -- (2.31593,4.541) -- (2.34546,4.51285) -- (2.375,4.48512) -- (2.40453,4.45779) -- (2.43406,4.43087) -- (2.46359,4.40432) --
 (2.49312,4.37816) -- (2.52265,4.35236) -- (2.55219,4.32691) -- (2.58172,4.30181) -- (2.61125,4.27705) -- (2.64078,4.25262) -- (2.67031,4.22852) -- (2.69984,4.20473) -- (2.72938,4.18125) -- (2.75891,4.15807) -- (2.78844,4.13519) -- (2.81797,4.1126)
 -- (2.8475,4.0903) -- (2.87704,4.06828) -- (2.90657,4.04653) -- (2.9361,4.02506) -- (2.96563,4.00385) -- (2.99516,3.98291) -- (3.02469,3.96222) -- (3.05423,3.94179) -- (3.08376,3.92162) -- (3.11329,3.90169) -- (3.14282,3.88201) -- (3.17235,3.86257)
 -- (3.20189,3.84337) -- (3.23142,3.82441) -- (3.26095,3.80568) -- (3.29048,3.78719) -- (3.32001,3.76893) -- (3.34954,3.7509) -- (3.37908,3.7331) -- (3.40861,3.71552) -- (3.43814,3.69817) -- (3.46767,3.68103) -- (3.4972,3.66412) -- (3.52673,3.64743)
 -- (3.55627,3.63095) -- (3.5858,3.6147) -- (3.61533,3.59865) -- (3.64486,3.58282) -- (3.67439,3.5672) -- (3.70393,3.5518) -- (3.73346,3.5366) -- (3.76299,3.52161) -- (3.79252,3.50683) -- (3.82205,3.49226) -- (3.85158,3.47789) -- (3.88112,3.46372) --
 (3.91065,3.44975) -- (3.94018,3.43599) -- (3.96971,3.42242) -- (3.99924,3.40905) -- (4.02877,3.39587) -- (4.05831,3.3829) -- (4.08784,3.37011) -- (4.11737,3.35751) -- (4.1469,3.34511) -- (4.17643,3.33289) -- (4.20597,3.32085) -- (4.2355,3.309) --
 (4.26503,3.29733) -- (4.29456,3.28584) -- (4.32409,3.27453) -- (4.35362,3.26339) -- (4.38316,3.25243) -- (4.41269,3.24163) -- (4.44222,3.23101) -- (4.47175,3.22055) -- (4.50128,3.21026) -- (4.53082,3.20013) -- (4.56035,3.19015) -- (4.58988,3.18034)
 -- (4.61941,3.17067) -- (4.64894,3.16116) -- (4.67847,3.1518) -- (4.70801,3.14259) -- (4.73754,3.13352) -- (4.76707,3.12459) -- (4.7966,3.1158) -- (4.82613,3.10714) -- (4.85566,3.09862) -- (4.8852,3.09023) -- (4.91473,3.08197) -- (4.94426,3.07384)
 -- (4.97379,3.06583) -- (5.00332,3.05794) -- (5.03286,3.05016) -- (5.06239,3.04251) -- (5.09192,3.03496) -- (5.12145,3.02753) -- (5.15098,3.02021) -- (5.18051,3.01299) -- (5.21005,3.00587) -- (5.23958,2.99886) -- (5.26911,2.99194) --
 (5.29864,2.98512) -- (5.32817,2.97839) -- (5.3577,2.97176) -- (5.38724,2.96521) -- (5.41677,2.95876) -- (5.4463,2.95238) -- (5.47583,2.94609) -- (5.50536,2.93988) -- (5.5349,2.93375) -- (5.56443,2.9277) -- (5.59396,2.92172) -- (5.62349,2.91582) --
 (5.65302,2.90998) -- (5.68255,2.90422) -- (5.71209,2.89852) -- (5.74162,2.89289) -- (5.77115,2.88732) -- (5.80068,2.88181) -- (5.83021,2.87636) -- (5.85975,2.87098) -- (5.88928,2.86565) -- (5.91881,2.86037) -- (5.94834,2.85515) -- (5.97787,2.84999)
 -- (6.0074,2.84487) -- (6.03694,2.83981) -- (6.06647,2.83479) -- (6.096,2.82982) -- (6.12553,2.8249) -- (6.15506,2.82002) -- (6.18459,2.81519) -- (6.21413,2.8104) -- (6.24366,2.80565) -- (6.27319,2.80094) -- (6.30272,2.79627) -- (6.33225,2.79163) --
 (6.36179,2.78704) -- (6.39132,2.78248) -- (6.42085,2.77796) -- (6.45038,2.77347) -- (6.47991,2.76901) -- (6.50944,2.76459) -- (6.53898,2.7602) -- (6.56851,2.75583) -- (6.59804,2.7515) -- (6.62757,2.7472) -- (6.6571,2.74292) -- (6.68663,2.73868) --
 (6.71617,2.73446) -- (6.7457,2.73026) -- (6.77523,2.72609) -- (6.80476,2.72195) -- (6.83429,2.71783) -- (6.86383,2.71373) -- (6.89336,2.70966) -- (6.92289,2.7056) -- (6.95242,2.70157) -- (6.98195,2.69756) -- (7.01148,2.69357) -- (7.04102,2.6896) --
 (7.07055,2.68565) -- (7.10008,2.68172) -- (7.12961,2.6778) -- (7.15914,2.67391) -- (7.18867,2.67003) -- (7.21821,2.66616) -- (7.24774,2.66231) -- (7.27727,2.65848) -- (7.3068,2.65467) -- (7.33633,2.65086) -- (7.36587,2.64708) -- (7.3954,2.6433) --
 (7.42493,2.63954) -- (7.45446,2.6358) -- (7.48399,2.63206) -- (7.51352,2.62834) -- (7.54306,2.62463) -- (7.57259,2.62093) -- (7.60212,2.61724) -- (7.63165,2.61357) -- (7.66118,2.6099) -- (7.69072,2.60625) -- (7.72025,2.6026) -- (7.74978,2.59897) --
 (7.77931,2.59534) -- (7.80884,2.59173) -- (7.83837,2.58812) -- (7.86791,2.58452) -- (7.89744,2.58093) -- (7.92697,2.57735) -- (7.9565,2.57377) -- (7.98603,2.5702) -- (8.01556,2.56664) -- (8.0451,2.56309) -- (8.07463,2.55954) -- (8.10416,2.556) --
 (8.13369,2.55247) -- (8.16322,2.54894) -- (8.19276,2.54541) -- (8.22229,2.5419) -- (8.25182,2.53838) -- (8.28135,2.53488) -- (8.31088,2.53138) -- (8.34041,2.52788) -- (8.36995,2.52438) -- (8.39948,2.52089) -- (8.42901,2.51741) -- (8.45854,2.51392)
 -- (8.48807,2.51045) -- (8.5176,2.50697) -- (8.54714,2.50349) -- (8.57667,2.50002) -- (8.6062,2.49655) -- (8.63573,2.49309) -- (8.66526,2.48962) -- (8.6948,2.48616) -- (8.72433,2.48269) -- (8.75386,2.47923) -- (8.78339,2.47577) -- (8.81292,2.47231)
 -- (8.84245,2.46884) -- (8.87199,2.46538) -- (8.90152,2.46192) -- (8.93105,2.45845) -- (8.96058,2.45499) -- (8.99011,2.45152) -- (9.01964,2.44805) -- (9.04918,2.44457) -- (9.07871,2.4411) -- (9.10824,2.43762) -- (9.13777,2.43413) -- (9.1673,2.43064)
 -- (9.19684,2.42715) -- (9.22637,2.42365) -- (9.2559,2.42015) -- (9.28543,2.41664) -- (9.31496,2.41313) -- (9.34449,2.4096) -- (9.37403,2.40607) -- (9.40356,2.40253) -- (9.43309,2.39899) -- (9.46262,2.39543) -- (9.49215,2.39187) -- (9.52169,2.3883)
 -- (9.55122,2.38471) -- (9.58075,2.38112) -- (9.61028,2.37752) -- (9.63981,2.3739) -- (9.66934,2.37027) -- (9.69888,2.36663) -- (9.72841,2.36298) -- (9.75794,2.35931) -- (9.78747,2.35564) -- (9.817,2.35194) -- (9.84653,2.34824) -- (9.87607,2.34452)
 -- (9.9056,2.32667) -- (9.93513,0.596817) -- (9.93513,0.596817) -- (9.9056,2.32667) -- (9.87607,2.31564) -- (9.84653,2.3194) -- (9.817,2.32313) -- (9.78747,2.32685) -- (9.75794,2.33056) -- (9.72841,2.33425) -- (9.69888,2.33793) -- (9.66934,2.3416)
 -- (9.63981,2.34526) -- (9.61028,2.34891) -- (9.58075,2.35256) -- (9.55122,2.35619) -- (9.52169,2.35983) -- (9.49215,2.36346) -- (9.46262,2.36708) -- (9.43309,2.3707) -- (9.40356,2.37432) -- (9.37403,2.37794) -- (9.34449,2.38155) --
 (9.31496,2.38516) -- (9.28543,2.38878) -- (9.2559,2.39239) -- (9.22637,2.396) -- (9.19684,2.39962) -- (9.1673,2.40323) -- (9.13777,2.40684) -- (9.10824,2.41046) -- (9.07871,2.41408) -- (9.04918,2.4177) -- (9.01964,2.42132) -- (8.99011,2.42494) --
 (8.96058,2.42857) -- (8.93105,2.4322) -- (8.90152,2.43583) -- (8.87199,2.43946) -- (8.84245,2.4431) -- (8.81292,2.44673) -- (8.78339,2.45038) -- (8.75386,2.45402) -- (8.72433,2.45767) -- (8.6948,2.46132) -- (8.66526,2.46498) -- (8.63573,2.46863) --
 (8.6062,2.4723) -- (8.57667,2.47596) -- (8.54714,2.47963) -- (8.5176,2.4833) -- (8.48807,2.48698) -- (8.45854,2.49066) -- (8.42901,2.49435) -- (8.39948,2.49804) -- (8.36995,2.50173) -- (8.34041,2.50543) -- (8.31088,2.50913) -- (8.28135,2.51284) --
 (8.25182,2.51655) -- (8.22229,2.52027) -- (8.19276,2.52399) -- (8.16322,2.52771) -- (8.13369,2.53145) -- (8.10416,2.53518) -- (8.07463,2.53892) -- (8.0451,2.54267) -- (8.01556,2.54643) -- (7.98603,2.55019) -- (7.9565,2.55395) -- (7.92697,2.55772) --
 (7.89744,2.5615) -- (7.86791,2.56528) -- (7.83837,2.56907) -- (7.80884,2.57287) -- (7.77931,2.57668) -- (7.74978,2.58049) -- (7.72025,2.58431) -- (7.69072,2.58814) -- (7.66118,2.59197) -- (7.63165,2.59581) -- (7.60212,2.59967) -- (7.57259,2.60353)
 -- (7.54306,2.60739) -- (7.51352,2.61127) -- (7.48399,2.61516) -- (7.45446,2.61906) -- (7.42493,2.62296) -- (7.3954,2.62688) -- (7.36587,2.63081) -- (7.33633,2.63475) -- (7.3068,2.6387) -- (7.27727,2.64266) -- (7.24774,2.64664) -- (7.21821,2.65063)
 -- (7.18867,2.65463) -- (7.15914,2.65864) -- (7.12961,2.66267) -- (7.10008,2.66671) -- (7.07055,2.67076) -- (7.04102,2.67483) -- (7.01148,2.67892) -- (6.98195,2.68302) -- (6.95242,2.68714) -- (6.92289,2.69128) -- (6.89336,2.69543) --
 (6.86383,2.6996) -- (6.83429,2.7038) -- (6.80476,2.70801) -- (6.77523,2.71224) -- (6.7457,2.71649) -- (6.71617,2.72077) -- (6.68663,2.72506) -- (6.6571,2.72938) -- (6.62757,2.73373) -- (6.59804,2.7381) -- (6.56851,2.74249) -- (6.53898,2.74691) --
 (6.50944,2.75136) -- (6.47991,2.75584) -- (6.45038,2.76034) -- (6.42085,2.76488) -- (6.39132,2.76945) -- (6.36179,2.77405) -- (6.33225,2.77868) -- (6.30272,2.78335) -- (6.27319,2.78805) -- (6.24366,2.79279) -- (6.21413,2.79757) -- (6.18459,2.80239)
 -- (6.15506,2.80725) -- (6.12553,2.81215) -- (6.096,2.81709) -- (6.06647,2.82208) -- (6.03694,2.82711) -- (6.0074,2.8322) -- (5.97787,2.83733) -- (5.94834,2.84251) -- (5.91881,2.84774) -- (5.88928,2.85303) -- (5.85975,2.85838) -- (5.83021,2.86378)
 -- (5.80068,2.86924) -- (5.77115,2.87476) -- (5.74162,2.88034) -- (5.71209,2.88599) -- (5.68255,2.89171) -- (5.65302,2.89749) -- (5.62349,2.90334) -- (5.59396,2.90927) -- (5.56443,2.91527) -- (5.5349,2.92134) -- (5.50536,2.9275) -- (5.47583,2.93373)
 -- (5.4463,2.94005) -- (5.41677,2.94645) -- (5.38724,2.95294) -- (5.3577,2.95952) -- (5.32817,2.96619) -- (5.29864,2.97296) -- (5.26911,2.97982) -- (5.23958,2.98678) -- (5.21005,2.99385) -- (5.18051,3.00101) -- (5.15098,3.00828) -- (5.12145,3.01567)
 -- (5.09192,3.02316) -- (5.06239,3.03076) -- (5.03286,3.03849) -- (5.00332,3.04633) -- (4.97379,3.05429) -- (4.94426,3.06238) -- (4.91473,3.07059) -- (4.8852,3.07893) -- (4.85566,3.0874) -- (4.82613,3.09601) -- (4.7966,3.10475) -- (4.76707,3.11363)
 -- (4.73754,3.12265) -- (4.70801,3.13182) -- (4.67847,3.14113) -- (4.64894,3.15059) -- (4.61941,3.1602) -- (4.58988,3.16997) -- (4.56035,3.17988) -- (4.53082,3.18996) -- (4.50128,3.2002) -- (4.47175,3.2106) -- (4.44222,3.22116) -- (4.41269,3.23189)
 -- (4.38316,3.24279) -- (4.35362,3.25386) -- (4.32409,3.26511) -- (4.29456,3.27652) -- (4.26503,3.28812) -- (4.2355,3.29989) -- (4.20597,3.31184) -- (4.17643,3.32397) -- (4.1469,3.33629) -- (4.11737,3.34879) -- (4.08784,3.36148) -- (4.05831,3.37436)
 -- (4.02877,3.38743) -- (3.99924,3.40069) -- (3.96971,3.41414) -- (3.94018,3.42779) -- (3.91065,3.44164) -- (3.88112,3.45568) -- (3.85158,3.46992) -- (3.82205,3.48436) -- (3.79252,3.49901) -- (3.76299,3.51386) -- (3.73346,3.52891) --
 (3.70393,3.54417) -- (3.67439,3.55963) -- (3.64486,3.57531) -- (3.61533,3.5912) -- (3.5858,3.60729) -- (3.55627,3.6236) -- (3.52673,3.64013) -- (3.4972,3.65687) -- (3.46767,3.67383) -- (3.43814,3.69101) -- (3.40861,3.70841) -- (3.37908,3.72603) --
 (3.34954,3.74388) -- (3.32001,3.76195) -- (3.29048,3.78025) -- (3.26095,3.79878) -- (3.23142,3.81755) -- (3.20189,3.83654) -- (3.17235,3.85578) -- (3.14282,3.87525) -- (3.11329,3.89496) -- (3.08376,3.91492) -- (3.05423,3.93513) -- (3.02469,3.95559)
 -- (2.99516,3.97629) -- (2.96563,3.99726) -- (2.9361,4.01849) -- (2.90657,4.03998) -- (2.87704,4.06173) -- (2.8475,4.08376) -- (2.81797,4.10607) -- (2.78844,4.12866) -- (2.75891,4.15154) -- (2.72938,4.1747) -- (2.69984,4.19817) -- (2.67031,4.22194)
 -- (2.64078,4.24602) -- (2.61125,4.27042) -- (2.58172,4.29515) -- (2.55219,4.32021) -- (2.52265,4.34561) -- (2.49312,4.37136) -- (2.46359,4.39747) -- (2.43406,4.42396) -- (2.40453,4.45083) -- (2.375,4.47809) -- (2.34546,4.50575) -- (2.31593,4.53384)
 -- (2.2864,4.56236) -- (2.25687,4.59133) -- (2.22734,4.62077) -- (2.1978,4.65069) -- (2.16827,4.68111) -- (2.13874,4.71204) -- (2.10921,4.74352) -- (2.07968,4.77557) -- (2.05015,4.8082) -- (2.02061,4.84144) -- (1.99108,4.87533) -- (1.96155,4.90988)
 -- (1.93202,4.94514) -- (1.90249,4.98113) -- (1.87296,5.0179) -- (1.84342,5.05547) -- (1.81389,5.0939) -- (1.78436,5.13322) -- (1.75483,5.17349) -- (1.7253,5.21475) -- (1.69576,5.25706) -- (1.66623,5.30048) -- (1.6367,5.34506) -- (1.60717,5.39088)
 -- (1.57764,5.438) -- (1.54811,5.48651) -- (1.51857,5.53651) -- (1.48904,0.596817) -- (1.45951,0.596817);
\definecolor{c}{rgb}{0,0,0};
\colorlet{c}{natblue!40};
\draw [c, fill=c] (1.45951,0.596817) -- (1.48904,4.38722) -- (1.51857,5.54801) -- (1.54811,5.4971) -- (1.57764,5.44794) -- (1.60717,5.40039) -- (1.6367,5.35433) -- (1.66623,5.30964) -- (1.69576,5.26623) -- (1.7253,5.22401) -- (1.75483,5.1829) --
 (1.78436,5.14283) -- (1.81389,5.10373) -- (1.84342,5.06555) -- (1.87296,5.02825) -- (1.90249,4.99178) -- (1.93202,4.95609) -- (1.96155,4.92115) -- (1.99108,4.88693) -- (2.02061,4.85341) -- (2.05015,4.82054) -- (2.07968,4.78831) -- (2.10921,4.75669)
 -- (2.13874,4.72567) -- (2.16827,4.69522) -- (2.1978,4.66532) -- (2.22734,4.63596) -- (2.25687,4.60712) -- (2.2864,4.5788) -- (2.31593,4.55096) -- (2.34546,4.52361) -- (2.375,4.49674) -- (2.40453,4.47032) -- (2.43406,4.44436) -- (2.46359,4.41883) --
 (2.49312,4.39374) -- (2.52265,4.36908) -- (2.55219,4.34483) -- (2.58172,4.321) -- (2.61125,4.29757) -- (2.64078,4.27454) -- (2.67031,4.25189) -- (2.69984,4.22964) -- (2.72938,4.20776) -- (2.75891,4.18626) -- (2.78844,4.16512) -- (2.81797,4.14436) --
 (2.8475,4.12395) -- (2.87704,4.10389) -- (2.90657,4.08419) -- (2.9361,4.06484) -- (2.96563,4.04583) -- (2.99516,4.02716) -- (3.02469,4.00883) -- (3.05423,3.99083) -- (3.08376,3.97316) -- (3.11329,3.95582) -- (3.14282,3.93881) -- (3.17235,3.92211) --
 (3.20189,3.90573) -- (3.23142,3.88967) -- (3.26095,3.87392) -- (3.29048,3.85848) -- (3.32001,3.84335) -- (3.34954,3.82852) -- (3.37908,3.81399) -- (3.40861,3.79976) -- (3.43814,3.78582) -- (3.46767,3.77218) -- (3.4972,3.75882) -- (3.52673,3.74575)
 -- (3.55627,3.73297) -- (3.5858,3.72046) -- (3.61533,3.70823) -- (3.64486,3.69627) -- (3.67439,3.68459) -- (3.70393,3.67317) -- (3.73346,3.66201) -- (3.76299,3.65112) -- (3.79252,3.64047) -- (3.82205,3.63009) -- (3.85158,3.61995) --
 (3.88112,3.61005) -- (3.91065,3.6004) -- (3.94018,3.59099) -- (3.96971,3.58181) -- (3.99924,3.57285) -- (4.02877,3.56413) -- (4.05831,3.55563) -- (4.08784,3.54735) -- (4.11737,3.53928) -- (4.1469,3.53142) -- (4.17643,3.52378) -- (4.20597,3.51633) --
 (4.2355,3.50908) -- (4.26503,3.50203) -- (4.29456,3.49518) -- (4.32409,3.48851) -- (4.35362,3.48202) -- (4.38316,3.47572) -- (4.41269,3.46959) -- (4.44222,3.46364) -- (4.47175,3.45785) -- (4.50128,3.45223) -- (4.53082,3.44678) -- (4.56035,3.44148)
 -- (4.58988,3.43634) -- (4.61941,3.43135) -- (4.64894,3.42651) -- (4.67847,3.42181) -- (4.70801,3.41726) -- (4.73754,3.41284) -- (4.76707,3.40856) -- (4.7966,3.40441) -- (4.82613,3.40039) -- (4.85566,3.39649) -- (4.8852,3.39271) -- (4.91473,3.38906)
 -- (4.94426,3.38552) -- (4.97379,3.38209) -- (5.00332,3.37877) -- (5.03286,3.37555) -- (5.06239,3.37244) -- (5.09192,3.36942) -- (5.12145,3.3665) -- (5.15098,3.36368) -- (5.18051,3.36094) -- (5.21005,3.35829) -- (5.23958,3.35573) --
 (5.26911,3.35324) -- (5.29864,3.35084) -- (5.32817,3.34851) -- (5.3577,3.34625) -- (5.38724,3.34406) -- (5.41677,3.34193) -- (5.4463,3.33987) -- (5.47583,3.33787) -- (5.50536,3.33593) -- (5.5349,3.33404) -- (5.56443,3.33221) -- (5.59396,3.33043) --
 (5.62349,3.3287) -- (5.65302,3.32701) -- (5.68255,3.32537) -- (5.71209,3.32377) -- (5.74162,3.32221) -- (5.77115,3.32069) -- (5.80068,3.3192) -- (5.83021,3.31775) -- (5.85975,3.31633) -- (5.88928,3.31494) -- (5.91881,3.31358) -- (5.94834,3.31224) --
 (5.97787,3.31094) -- (6.0074,3.30965) -- (6.03694,3.30838) -- (6.06647,3.30714) -- (6.096,3.30591) -- (6.12553,3.30471) -- (6.15506,3.30351) -- (6.18459,3.30234) -- (6.21413,3.30117) -- (6.24366,3.30002) -- (6.27319,3.29888) -- (6.30272,3.29775) --
 (6.33225,3.29663) -- (6.36179,3.29551) -- (6.39132,3.2944) -- (6.42085,3.2933) -- (6.45038,3.2922) -- (6.47991,3.2911) -- (6.50944,3.29001) -- (6.53898,3.28892) -- (6.56851,3.28783) -- (6.59804,3.28673) -- (6.62757,3.28564) -- (6.6571,3.28455) --
 (6.68663,3.28345) -- (6.71617,3.28235) -- (6.7457,3.28124) -- (6.77523,3.28013) -- (6.80476,3.27902) -- (6.83429,3.2779) -- (6.86383,3.27677) -- (6.89336,3.27563) -- (6.92289,3.27449) -- (6.95242,3.27334) -- (6.98195,3.27217) -- (7.01148,3.271) --
 (7.04102,3.26982) -- (7.07055,3.26863) -- (7.10008,3.26742) -- (7.12961,3.2662) -- (7.15914,3.26497) -- (7.18867,3.26373) -- (7.21821,3.26248) -- (7.24774,3.26121) -- (7.27727,3.25992) -- (7.3068,3.25862) -- (7.33633,3.25731) -- (7.36587,3.25598) --
 (7.3954,3.25464) -- (7.42493,3.25327) -- (7.45446,3.25189) -- (7.48399,3.2505) -- (7.51352,3.24908) -- (7.54306,3.24765) -- (7.57259,3.2462) -- (7.60212,3.24473) -- (7.63165,3.24324) -- (7.66118,3.24173) -- (7.69072,3.2402) -- (7.72025,3.23865) --
 (7.74978,3.23708) -- (7.77931,3.23549) -- (7.80884,3.23388) -- (7.83837,3.23224) -- (7.86791,3.23059) -- (7.89744,3.22891) -- (7.92697,3.2272) -- (7.9565,3.22548) -- (7.98603,3.22373) -- (8.01556,3.22195) -- (8.0451,3.22015) -- (8.07463,3.21833) --
 (8.10416,3.21648) -- (8.13369,3.2146) -- (8.16322,3.2127) -- (8.19276,3.21078) -- (8.22229,3.20882) -- (8.25182,3.20684) -- (8.28135,3.20484) -- (8.31088,3.2028) -- (8.34041,3.20074) -- (8.36995,3.19865) -- (8.39948,3.19653) -- (8.42901,3.19439) --
 (8.45854,3.19221) -- (8.48807,3.19) -- (8.5176,3.18777) -- (8.54714,3.18551) -- (8.57667,3.18321) -- (8.6062,3.18089) -- (8.63573,3.17854) -- (8.66526,3.17615) -- (8.6948,3.17374) -- (8.72433,3.17129) -- (8.75386,3.16881) -- (8.78339,3.1663) --
 (8.81292,3.16376) -- (8.84245,3.16119) -- (8.87199,3.15859) -- (8.90152,3.15595) -- (8.93105,3.15328) -- (8.96058,3.15058) -- (8.99011,3.14785) -- (9.01964,3.14508) -- (9.04918,3.14228) -- (9.07871,3.13945) -- (9.10824,3.13658) -- (9.13777,3.13368)
 -- (9.1673,3.13075) -- (9.19684,3.12779) -- (9.22637,3.12479) -- (9.2559,3.12176) -- (9.28543,3.11869) -- (9.31496,3.11559) -- (9.34449,3.11245) -- (9.37403,3.10928) -- (9.40356,3.10608) -- (9.43309,3.10285) -- (9.46262,3.09957) -- (9.49215,3.09627)
 -- (9.52169,3.09293) -- (9.55122,3.08955) -- (9.58075,3.08615) -- (9.61028,3.0827) -- (9.63981,3.07922) -- (9.66934,3.07571) -- (9.69888,3.07217) -- (9.72841,3.06858) -- (9.75794,3.06497) -- (9.78747,3.06132) -- (9.817,3.05763) -- (9.84653,3.05391)
 -- (9.87607,3.05016) -- (9.9056,3.03778) -- (9.93513,0.596817) -- (9.93513,0.596817) -- (9.9056,3.03778) -- (9.87607,3.03276) -- (9.84653,3.03655) -- (9.817,3.0403) -- (9.78747,3.04402) -- (9.75794,3.04771) -- (9.72841,3.05136) -- (9.69888,3.05498)
 -- (9.66934,3.05857) -- (9.63981,3.06212) -- (9.61028,3.06564) -- (9.58075,3.06912) -- (9.55122,3.07257) -- (9.52169,3.07599) -- (9.49215,3.07937) -- (9.46262,3.08273) -- (9.43309,3.08605) -- (9.40356,3.08933) -- (9.37403,3.09258) --
 (9.34449,3.0958) -- (9.31496,3.09899) -- (9.28543,3.10215) -- (9.2559,3.10527) -- (9.22637,3.10836) -- (9.19684,3.11142) -- (9.1673,3.11445) -- (9.13777,3.11744) -- (9.10824,3.1204) -- (9.07871,3.12333) -- (9.04918,3.12623) -- (9.01964,3.1291) --
 (8.99011,3.13194) -- (8.96058,3.13474) -- (8.93105,3.13751) -- (8.90152,3.14025) -- (8.87199,3.14296) -- (8.84245,3.14564) -- (8.81292,3.14829) -- (8.78339,3.15091) -- (8.75386,3.1535) -- (8.72433,3.15605) -- (8.6948,3.15858) -- (8.66526,3.16107) --
 (8.63573,3.16354) -- (8.6062,3.16597) -- (8.57667,3.16837) -- (8.54714,3.17075) -- (8.5176,3.17309) -- (8.48807,3.1754) -- (8.45854,3.17769) -- (8.42901,3.17994) -- (8.39948,3.18216) -- (8.36995,3.18436) -- (8.34041,3.18653) -- (8.31088,3.18866) --
 (8.28135,3.19077) -- (8.25182,3.19285) -- (8.22229,3.1949) -- (8.19276,3.19692) -- (8.16322,3.19891) -- (8.13369,3.20088) -- (8.10416,3.20281) -- (8.07463,3.20472) -- (8.0451,3.2066) -- (8.01556,3.20845) -- (7.98603,3.21028) -- (7.9565,3.21208) --
 (7.92697,3.21385) -- (7.89744,3.21559) -- (7.86791,3.21731) -- (7.83837,3.219) -- (7.80884,3.22067) -- (7.77931,3.22231) -- (7.74978,3.22392) -- (7.72025,3.22552) -- (7.69072,3.22708) -- (7.66118,3.22862) -- (7.63165,3.23014) -- (7.60212,3.23163) --
 (7.57259,3.2331) -- (7.54306,3.23455) -- (7.51352,3.23598) -- (7.48399,3.23738) -- (7.45446,3.23876) -- (7.42493,3.24012) -- (7.3954,3.24146) -- (7.36587,3.24279) -- (7.33633,3.24409) -- (7.3068,3.24537) -- (7.27727,3.24663) -- (7.24774,3.24788) --
 (7.21821,3.24911) -- (7.18867,3.25032) -- (7.15914,3.25152) -- (7.12961,3.25271) -- (7.10008,3.25387) -- (7.07055,3.25503) -- (7.04102,3.25617) -- (7.01148,3.2573) -- (6.98195,3.25842) -- (6.95242,3.25953) -- (6.92289,3.26062) -- (6.89336,3.26171)
 -- (6.86383,3.26279) -- (6.83429,3.26387) -- (6.80476,3.26493) -- (6.77523,3.266) -- (6.7457,3.26705) -- (6.71617,3.26811) -- (6.68663,3.26916) -- (6.6571,3.2702) -- (6.62757,3.27125) -- (6.59804,3.2723) -- (6.56851,3.27335) -- (6.53898,3.2744) --
 (6.50944,3.27546) -- (6.47991,3.27652) -- (6.45038,3.27758) -- (6.42085,3.27865) -- (6.39132,3.27973) -- (6.36179,3.28082) -- (6.33225,3.28192) -- (6.30272,3.28303) -- (6.27319,3.28415) -- (6.24366,3.28529) -- (6.21413,3.28644) -- (6.18459,3.28761)
 -- (6.15506,3.2888) -- (6.12553,3.29) -- (6.096,3.29123) -- (6.06647,3.29247) -- (6.03694,3.29374) -- (6.0074,3.29504) -- (5.97787,3.29636) -- (5.94834,3.29771) -- (5.91881,3.29909) -- (5.88928,3.3005) -- (5.85975,3.30194) -- (5.83021,3.30341) --
 (5.80068,3.30492) -- (5.77115,3.30647) -- (5.74162,3.30805) -- (5.71209,3.30968) -- (5.68255,3.31134) -- (5.65302,3.31305) -- (5.62349,3.31481) -- (5.59396,3.31661) -- (5.56443,3.31847) -- (5.5349,3.32037) -- (5.50536,3.32232) -- (5.47583,3.32433)
 -- (5.4463,3.3264) -- (5.41677,3.32853) -- (5.38724,3.33071) -- (5.3577,3.33296) -- (5.32817,3.33528) -- (5.29864,3.33766) -- (5.26911,3.34011) -- (5.23958,3.34263) -- (5.21005,3.34523) -- (5.18051,3.3479) -- (5.15098,3.35065) -- (5.12145,3.35349)
 -- (5.09192,3.35641) -- (5.06239,3.35941) -- (5.03286,3.36251) -- (5.00332,3.3657) -- (4.97379,3.36898) -- (4.94426,3.37237) -- (4.91473,3.37585) -- (4.8852,3.37945) -- (4.85566,3.38315) -- (4.82613,3.38697) -- (4.7966,3.3909) -- (4.76707,3.39496)
 -- (4.73754,3.39913) -- (4.70801,3.40344) -- (4.67847,3.40788) -- (4.64894,3.41246) -- (4.61941,3.41718) -- (4.58988,3.42204) -- (4.56035,3.42705) -- (4.53082,3.43222) -- (4.50128,3.43754) -- (4.47175,3.44303) -- (4.44222,3.44869) --
 (4.41269,3.45451) -- (4.38316,3.46052) -- (4.35362,3.4667) -- (4.32409,3.47307) -- (4.29456,3.47963) -- (4.26503,3.48639) -- (4.2355,3.49334) -- (4.20597,3.5005) -- (4.17643,3.50787) -- (4.1469,3.51545) -- (4.11737,3.52325) -- (4.08784,3.53127) --
 (4.05831,3.53951) -- (4.02877,3.54799) -- (3.99924,3.5567) -- (3.96971,3.56565) -- (3.94018,3.57484) -- (3.91065,3.58429) -- (3.88112,3.59398) -- (3.85158,3.60392) -- (3.82205,3.61413) -- (3.79252,3.6246) -- (3.76299,3.63533) -- (3.73346,3.64634) --
 (3.70393,3.65762) -- (3.67439,3.66917) -- (3.64486,3.681) -- (3.61533,3.69312) -- (3.5858,3.70552) -- (3.55627,3.71821) -- (3.52673,3.73119) -- (3.4972,3.74446) -- (3.46767,3.75803) -- (3.43814,3.7719) -- (3.40861,3.78607) -- (3.37908,3.80055) --
 (3.34954,3.81533) -- (3.32001,3.83041) -- (3.29048,3.84581) -- (3.26095,3.86152) -- (3.23142,3.87754) -- (3.20189,3.89388) -- (3.17235,3.91054) -- (3.14282,3.92751) -- (3.11329,3.94481) -- (3.08376,3.96243) -- (3.05423,3.98038) -- (3.02469,3.99865)
 -- (2.99516,4.01726) -- (2.96563,4.0362) -- (2.9361,4.05547) -- (2.90657,4.07508) -- (2.87704,4.09502) -- (2.8475,4.11531) -- (2.81797,4.13595) -- (2.78844,4.15693) -- (2.75891,4.17827) -- (2.72938,4.19996) -- (2.69984,4.22201) -- (2.67031,4.24443)
 -- (2.64078,4.26721) -- (2.61125,4.29037) -- (2.58172,4.3139) -- (2.55219,4.33783) -- (2.52265,4.36214) -- (2.49312,4.38686) -- (2.46359,4.41198) -- (2.43406,4.43752) -- (2.40453,4.46349) -- (2.375,4.4899) -- (2.34546,4.51675) -- (2.31593,4.54407)
 -- (2.2864,4.57186) -- (2.25687,4.60013) -- (2.22734,4.62891) -- (2.1978,4.65821) -- (2.16827,4.68804) -- (2.13874,4.71843) -- (2.10921,4.74939) -- (2.07968,4.78094) -- (2.05015,4.81312) -- (2.02061,4.84593) -- (1.99108,4.87942) -- (1.96155,4.9136)
 -- (1.93202,4.94851) -- (1.90249,4.98418) -- (1.87296,5.02064) -- (1.84342,5.05794) -- (1.81389,5.09611) -- (1.78436,5.1352) -- (1.75483,5.17525) -- (1.7253,5.21632) -- (1.69576,5.25845) -- (1.66623,5.3017) -- (1.6367,5.34613) -- (1.60717,5.39181)
 -- (1.57764,5.43881) -- (1.54811,5.48721) -- (1.51857,5.53712) -- (1.48904,0.596817) -- (1.45951,0.596817);
\colorlet{c}{kugray};
\draw [c] (1.13336,3.41516) -- (1.13336,3.80777);
\draw [c] (1.13336,3.80777) -- (1.13336,3.97485);
\draw [c] (1.11854,3.80777) -- (1.13336,3.80777);
\draw [c] (1.13336,3.80777) -- (1.14818,3.80777);
\definecolor{c}{rgb}{0,0,0};
\colorlet{c}{kugray};
\draw [c] (1.163,3.33018) -- (1.163,3.70859);
\draw [c] (1.163,3.70859) -- (1.163,3.87329);
\draw [c] (1.14818,3.70859) -- (1.163,3.70859);
\draw [c] (1.163,3.70859) -- (1.17781,3.70859);
\definecolor{c}{rgb}{0,0,0};
\colorlet{c}{kugray};
\draw [c] (1.22227,3.26026) -- (1.22227,3.80576);
\draw [c] (1.22227,3.80576) -- (1.22227,3.99181);
\draw [c] (1.20745,3.80576) -- (1.22227,3.80576);
\draw [c] (1.22227,3.80576) -- (1.23709,3.80576);
\definecolor{c}{rgb}{0,0,0};
\colorlet{c}{kugray};
\draw [c] (1.2519,3.59515) -- (1.2519,3.86294);
\draw [c] (1.2519,3.86294) -- (1.2519,4.00395);
\draw [c] (1.23709,3.86294) -- (1.2519,3.86294);
\draw [c] (1.2519,3.86294) -- (1.26672,3.86294);
\definecolor{c}{rgb}{0,0,0};
\colorlet{c}{kugray};
\draw [c] (1.28154,3.81131) -- (1.28154,4.05468);
\draw [c] (1.28154,4.05468) -- (1.28154,4.18889);
\draw [c] (1.26672,4.05468) -- (1.28154,4.05468);
\draw [c] (1.28154,4.05468) -- (1.29636,4.05468);
\definecolor{c}{rgb}{0,0,0};
\colorlet{c}{kugray};
\draw [c] (1.31118,5.36749) -- (1.31118,5.38512);
\draw [c] (1.31118,5.38512) -- (1.31118,5.40179);
\draw [c] (1.29636,5.38512) -- (1.31118,5.38512);
\draw [c] (1.31118,5.38512) -- (1.32599,5.38512);
\definecolor{c}{rgb}{0,0,0};
\colorlet{c}{kugray};
\draw [c] (1.34081,5.61993) -- (1.34081,5.63126);
\draw [c] (1.34081,5.63126) -- (1.34081,5.6422);
\draw [c] (1.32599,5.63126) -- (1.34081,5.63126);
\draw [c] (1.34081,5.63126) -- (1.35563,5.63126);
\definecolor{c}{rgb}{0,0,0};
\colorlet{c}{kugray};
\draw [c] (1.37045,5.69182) -- (1.37045,5.70194);
\draw [c] (1.37045,5.70194) -- (1.37045,5.71174);
\draw [c] (1.35563,5.70194) -- (1.37045,5.70194);
\draw [c] (1.37045,5.70194) -- (1.38526,5.70194);
\definecolor{c}{rgb}{0,0,0};
\colorlet{c}{kugray};
\draw [c] (1.40008,5.67982) -- (1.40008,5.69024);
\draw [c] (1.40008,5.69024) -- (1.40008,5.70031);
\draw [c] (1.38526,5.69024) -- (1.40008,5.69024);
\draw [c] (1.40008,5.69024) -- (1.4149,5.69024);
\definecolor{c}{rgb}{0,0,0};
\colorlet{c}{kugray};
\draw [c] (1.42972,5.65827) -- (1.42972,5.66928);
\draw [c] (1.42972,5.66928) -- (1.42972,5.67991);
\draw [c] (1.4149,5.66928) -- (1.42972,5.66928);
\draw [c] (1.42972,5.66928) -- (1.44454,5.66928);
\definecolor{c}{rgb}{0,0,0};
\colorlet{c}{kugray};
\draw [c] (1.45935,5.62) -- (1.45935,5.63141);
\draw [c] (1.45935,5.63141) -- (1.45935,5.6424);
\draw [c] (1.44454,5.63141) -- (1.45935,5.63141);
\draw [c] (1.45935,5.63141) -- (1.47417,5.63141);
\definecolor{c}{rgb}{0,0,0};
\colorlet{c}{kugray};
\draw [c] (1.48899,5.58164) -- (1.48899,5.5938);
\draw [c] (1.48899,5.5938) -- (1.48899,5.60549);
\draw [c] (1.47417,5.5938) -- (1.48899,5.5938);
\draw [c] (1.48899,5.5938) -- (1.50381,5.5938);
\definecolor{c}{rgb}{0,0,0};
\colorlet{c}{kugray};
\draw [c] (1.51863,5.53046) -- (1.51863,5.54392);
\draw [c] (1.51863,5.54392) -- (1.51863,5.55682);
\draw [c] (1.50381,5.54392) -- (1.51863,5.54392);
\draw [c] (1.51863,5.54392) -- (1.53344,5.54392);
\definecolor{c}{rgb}{0,0,0};
\colorlet{c}{kugray};
\draw [c] (1.54826,5.46022) -- (1.54826,5.47505);
\draw [c] (1.54826,5.47505) -- (1.54826,5.48921);
\draw [c] (1.53344,5.47505) -- (1.54826,5.47505);
\draw [c] (1.54826,5.47505) -- (1.56308,5.47505);
\definecolor{c}{rgb}{0,0,0};
\colorlet{c}{kugray};
\draw [c] (1.5779,5.41506) -- (1.5779,5.43093);
\draw [c] (1.5779,5.43093) -- (1.5779,5.44602);
\draw [c] (1.56308,5.43093) -- (1.5779,5.43093);
\draw [c] (1.5779,5.43093) -- (1.59272,5.43093);
\definecolor{c}{rgb}{0,0,0};
\colorlet{c}{kugray};
\draw [c] (1.60753,5.38705) -- (1.60753,5.40429);
\draw [c] (1.60753,5.40429) -- (1.60753,5.42061);
\draw [c] (1.59272,5.40429) -- (1.60753,5.40429);
\draw [c] (1.60753,5.40429) -- (1.62235,5.40429);
\definecolor{c}{rgb}{0,0,0};
\colorlet{c}{kugray};
\draw [c] (1.63717,5.33118) -- (1.63717,5.34945);
\draw [c] (1.63717,5.34945) -- (1.63717,5.36669);
\draw [c] (1.62235,5.34945) -- (1.63717,5.34945);
\draw [c] (1.63717,5.34945) -- (1.65199,5.34945);
\definecolor{c}{rgb}{0,0,0};
\colorlet{c}{kugray};
\draw [c] (1.6668,5.33501) -- (1.6668,5.35318);
\draw [c] (1.6668,5.35318) -- (1.6668,5.37033);
\draw [c] (1.65199,5.35318) -- (1.6668,5.35318);
\draw [c] (1.6668,5.35318) -- (1.68162,5.35318);
\definecolor{c}{rgb}{0,0,0};
\colorlet{c}{kugray};
\draw [c] (1.69644,5.27038) -- (1.69644,5.29099);
\draw [c] (1.69644,5.29099) -- (1.69644,5.3103);
\draw [c] (1.68162,5.29099) -- (1.69644,5.29099);
\draw [c] (1.69644,5.29099) -- (1.71126,5.29099);
\definecolor{c}{rgb}{0,0,0};
\colorlet{c}{kugray};
\draw [c] (1.72608,5.20804) -- (1.72608,5.23001);
\draw [c] (1.72608,5.23001) -- (1.72608,5.25052);
\draw [c] (1.71126,5.23001) -- (1.72608,5.23001);
\draw [c] (1.72608,5.23001) -- (1.74089,5.23001);
\definecolor{c}{rgb}{0,0,0};
\colorlet{c}{kugray};
\draw [c] (1.75571,5.08901) -- (1.75571,5.11549);
\draw [c] (1.75571,5.11549) -- (1.75571,5.13987);
\draw [c] (1.74089,5.11549) -- (1.75571,5.11549);
\draw [c] (1.75571,5.11549) -- (1.77053,5.11549);
\definecolor{c}{rgb}{0,0,0};
\colorlet{c}{kugray};
\draw [c] (1.78535,5.10181) -- (1.78535,5.12931);
\draw [c] (1.78535,5.12931) -- (1.78535,5.15455);
\draw [c] (1.77053,5.12931) -- (1.78535,5.12931);
\draw [c] (1.78535,5.12931) -- (1.80017,5.12931);
\definecolor{c}{rgb}{0,0,0};
\colorlet{c}{kugray};
\draw [c] (1.81498,5.06396) -- (1.81498,5.09313);
\draw [c] (1.81498,5.09313) -- (1.81498,5.11978);
\draw [c] (1.80017,5.09313) -- (1.81498,5.09313);
\draw [c] (1.81498,5.09313) -- (1.8298,5.09313);
\definecolor{c}{rgb}{0,0,0};
\colorlet{c}{kugray};
\draw [c] (1.84462,5.06537) -- (1.84462,5.09456);
\draw [c] (1.84462,5.09456) -- (1.84462,5.12123);
\draw [c] (1.8298,5.09456) -- (1.84462,5.09456);
\draw [c] (1.84462,5.09456) -- (1.85944,5.09456);
\definecolor{c}{rgb}{0,0,0};
\colorlet{c}{kugray};
\draw [c] (1.87425,4.99725) -- (1.87425,5.02835);
\draw [c] (1.87425,5.02835) -- (1.87425,5.05659);
\draw [c] (1.85944,5.02835) -- (1.87425,5.02835);
\draw [c] (1.87425,5.02835) -- (1.88907,5.02835);
\definecolor{c}{rgb}{0,0,0};
\colorlet{c}{kugray};
\draw [c] (1.90389,4.95832) -- (1.90389,4.99102);
\draw [c] (1.90389,4.99102) -- (1.90389,5.02058);
\draw [c] (1.88907,4.99102) -- (1.90389,4.99102);
\draw [c] (1.90389,4.99102) -- (1.91871,4.99102);
\definecolor{c}{rgb}{0,0,0};
\colorlet{c}{kugray};
\draw [c] (1.93353,4.85544) -- (1.93353,4.89383);
\draw [c] (1.93353,4.89383) -- (1.93353,4.92796);
\draw [c] (1.91871,4.89383) -- (1.93353,4.89383);
\draw [c] (1.93353,4.89383) -- (1.94834,4.89383);
\definecolor{c}{rgb}{0,0,0};
\colorlet{c}{kugray};
\draw [c] (1.96316,4.88333) -- (1.96316,4.92161);
\draw [c] (1.96316,4.92161) -- (1.96316,4.95565);
\draw [c] (1.94834,4.92161) -- (1.96316,4.92161);
\draw [c] (1.96316,4.92161) -- (1.97798,4.92161);
\definecolor{c}{rgb}{0,0,0};
\colorlet{c}{kugray};
\draw [c] (1.9928,4.85825) -- (1.9928,4.89733);
\draw [c] (1.9928,4.89733) -- (1.9928,4.93201);
\draw [c] (1.97798,4.89733) -- (1.9928,4.89733);
\draw [c] (1.9928,4.89733) -- (2.00762,4.89733);
\definecolor{c}{rgb}{0,0,0};
\colorlet{c}{kugray};
\draw [c] (2.02243,4.78026) -- (2.02243,4.82199);
\draw [c] (2.02243,4.82199) -- (2.02243,4.85874);
\draw [c] (2.00762,4.82199) -- (2.02243,4.82199);
\draw [c] (2.02243,4.82199) -- (2.03725,4.82199);
\definecolor{c}{rgb}{0,0,0};
\colorlet{c}{kugray};
\draw [c] (2.05207,4.74394) -- (2.05207,4.7911);
\draw [c] (2.05207,4.7911) -- (2.05207,4.83199);
\draw [c] (2.03725,4.7911) -- (2.05207,4.7911);
\draw [c] (2.05207,4.7911) -- (2.06689,4.7911);
\definecolor{c}{rgb}{0,0,0};
\colorlet{c}{kugray};
\draw [c] (2.08171,4.69059) -- (2.08171,4.74618);
\draw [c] (2.08171,4.74618) -- (2.08171,4.79324);
\draw [c] (2.06689,4.74618) -- (2.08171,4.74618);
\draw [c] (2.08171,4.74618) -- (2.09652,4.74618);
\definecolor{c}{rgb}{0,0,0};
\colorlet{c}{kugray};
\draw [c] (2.11134,4.64498) -- (2.11134,4.70439);
\draw [c] (2.11134,4.70439) -- (2.11134,4.75417);
\draw [c] (2.09652,4.70439) -- (2.11134,4.70439);
\draw [c] (2.11134,4.70439) -- (2.12616,4.70439);
\definecolor{c}{rgb}{0,0,0};
\colorlet{c}{kugray};
\draw [c] (2.14098,4.66916) -- (2.14098,4.726);
\draw [c] (2.14098,4.726) -- (2.14098,4.77396);
\draw [c] (2.12616,4.726) -- (2.14098,4.726);
\draw [c] (2.14098,4.726) -- (2.15579,4.726);
\definecolor{c}{rgb}{0,0,0};
\colorlet{c}{kugray};
\draw [c] (2.17061,4.71202) -- (2.17061,4.76221);
\draw [c] (2.17061,4.76221) -- (2.17061,4.80536);
\draw [c] (2.15579,4.76221) -- (2.17061,4.76221);
\draw [c] (2.17061,4.76221) -- (2.18543,4.76221);
\definecolor{c}{rgb}{0,0,0};
\colorlet{c}{kugray};
\draw [c] (2.20025,4.43289) -- (2.20025,4.50711);
\draw [c] (2.20025,4.50711) -- (2.20025,4.56686);
\draw [c] (2.18543,4.50711) -- (2.20025,4.50711);
\draw [c] (2.20025,4.50711) -- (2.21507,4.50711);
\definecolor{c}{rgb}{0,0,0};
\colorlet{c}{kugray};
\draw [c] (2.22988,4.60564) -- (2.22988,4.66562);
\draw [c] (2.22988,4.66562) -- (2.22988,4.7158);
\draw [c] (2.21507,4.66562) -- (2.22988,4.66562);
\draw [c] (2.22988,4.66562) -- (2.2447,4.66562);
\definecolor{c}{rgb}{0,0,0};
\colorlet{c}{kugray};
\draw [c] (2.25952,4.64159) -- (2.25952,4.70326);
\draw [c] (2.25952,4.70326) -- (2.25952,4.75461);
\draw [c] (2.2447,4.70326) -- (2.25952,4.70326);
\draw [c] (2.25952,4.70326) -- (2.27434,4.70326);
\definecolor{c}{rgb}{0,0,0};
\colorlet{c}{kugray};
\draw [c] (2.28916,4.56718) -- (2.28916,4.57808);
\draw [c] (2.28916,4.57808) -- (2.28916,4.5886);
\draw [c] (2.27434,4.57808) -- (2.28916,4.57808);
\draw [c] (2.28916,4.57808) -- (2.30397,4.57808);
\definecolor{c}{rgb}{0,0,0};
\colorlet{c}{kugray};
\draw [c] (2.31879,4.51802) -- (2.31879,4.5299);
\draw [c] (2.31879,4.5299) -- (2.31879,4.54134);
\draw [c] (2.30397,4.5299) -- (2.31879,4.5299);
\draw [c] (2.31879,4.5299) -- (2.33361,4.5299);
\definecolor{c}{rgb}{0,0,0};
\colorlet{c}{kugray};
\draw [c] (2.34843,4.51942) -- (2.34843,4.53112);
\draw [c] (2.34843,4.53112) -- (2.34843,4.54238);
\draw [c] (2.33361,4.53112) -- (2.34843,4.53112);
\draw [c] (2.34843,4.53112) -- (2.36325,4.53112);
\definecolor{c}{rgb}{0,0,0};
\colorlet{c}{kugray};
\draw [c] (2.37806,4.47939) -- (2.37806,4.492);
\draw [c] (2.37806,4.492) -- (2.37806,4.50411);
\draw [c] (2.36325,4.492) -- (2.37806,4.492);
\draw [c] (2.37806,4.492) -- (2.39288,4.492);
\definecolor{c}{rgb}{0,0,0};
\colorlet{c}{kugray};
\draw [c] (2.4077,4.44091) -- (2.4077,4.45439);
\draw [c] (2.4077,4.45439) -- (2.4077,4.46731);
\draw [c] (2.39288,4.45439) -- (2.4077,4.45439);
\draw [c] (2.4077,4.45439) -- (2.42252,4.45439);
\definecolor{c}{rgb}{0,0,0};
\colorlet{c}{kugray};
\draw [c] (2.43733,4.39585) -- (2.43733,4.41002);
\draw [c] (2.43733,4.41002) -- (2.43733,4.42357);
\draw [c] (2.42252,4.41002) -- (2.43733,4.41002);
\draw [c] (2.43733,4.41002) -- (2.45215,4.41002);
\definecolor{c}{rgb}{0,0,0};
\colorlet{c}{kugray};
\draw [c] (2.46697,4.36562) -- (2.46697,4.38084);
\draw [c] (2.46697,4.38084) -- (2.46697,4.39535);
\draw [c] (2.45215,4.38084) -- (2.46697,4.38084);
\draw [c] (2.46697,4.38084) -- (2.48179,4.38084);
\definecolor{c}{rgb}{0,0,0};
\colorlet{c}{kugray};
\draw [c] (2.49661,4.34442) -- (2.49661,4.3598);
\draw [c] (2.49661,4.3598) -- (2.49661,4.37444);
\draw [c] (2.48179,4.3598) -- (2.49661,4.3598);
\draw [c] (2.49661,4.3598) -- (2.51142,4.3598);
\definecolor{c}{rgb}{0,0,0};
\colorlet{c}{kugray};
\draw [c] (2.52624,4.33876) -- (2.52624,4.3547);
\draw [c] (2.52624,4.3547) -- (2.52624,4.36985);
\draw [c] (2.51142,4.3547) -- (2.52624,4.3547);
\draw [c] (2.52624,4.3547) -- (2.54106,4.3547);
\definecolor{c}{rgb}{0,0,0};
\colorlet{c}{kugray};
\draw [c] (2.55588,4.31164) -- (2.55588,4.32829);
\draw [c] (2.55588,4.32829) -- (2.55588,4.34409);
\draw [c] (2.54106,4.32829) -- (2.55588,4.32829);
\draw [c] (2.55588,4.32829) -- (2.5707,4.32829);
\definecolor{c}{rgb}{0,0,0};
\colorlet{c}{kugray};
\draw [c] (2.58551,4.27319) -- (2.58551,4.29119);
\draw [c] (2.58551,4.29119) -- (2.58551,4.30819);
\draw [c] (2.5707,4.29119) -- (2.58551,4.29119);
\draw [c] (2.58551,4.29119) -- (2.60033,4.29119);
\definecolor{c}{rgb}{0,0,0};
\colorlet{c}{kugray};
\draw [c] (2.61515,4.23142) -- (2.61515,4.25043);
\draw [c] (2.61515,4.25043) -- (2.61515,4.26834);
\draw [c] (2.60033,4.25043) -- (2.61515,4.25043);
\draw [c] (2.61515,4.25043) -- (2.62997,4.25043);
\definecolor{c}{rgb}{0,0,0};
\colorlet{c}{kugray};
\draw [c] (2.64478,4.1941) -- (2.64478,4.21371);
\draw [c] (2.64478,4.21371) -- (2.64478,4.23215);
\draw [c] (2.62997,4.21371) -- (2.64478,4.21371);
\draw [c] (2.64478,4.21371) -- (2.6596,4.21371);
\definecolor{c}{rgb}{0,0,0};
\colorlet{c}{kugray};
\draw [c] (2.67442,4.21383) -- (2.67442,4.23319);
\draw [c] (2.67442,4.23319) -- (2.67442,4.2514);
\draw [c] (2.6596,4.23319) -- (2.67442,4.23319);
\draw [c] (2.67442,4.23319) -- (2.68924,4.23319);
\definecolor{c}{rgb}{0,0,0};
\colorlet{c}{kugray};
\draw [c] (2.70406,4.15434) -- (2.70406,4.17536);
\draw [c] (2.70406,4.17536) -- (2.70406,4.19503);
\draw [c] (2.68924,4.17536) -- (2.70406,4.17536);
\draw [c] (2.70406,4.17536) -- (2.71887,4.17536);
\definecolor{c}{rgb}{0,0,0};
\colorlet{c}{kugray};
\draw [c] (2.73369,4.14693) -- (2.73369,4.16849);
\draw [c] (2.73369,4.16849) -- (2.73369,4.18863);
\draw [c] (2.71887,4.16849) -- (2.73369,4.16849);
\draw [c] (2.73369,4.16849) -- (2.74851,4.16849);
\definecolor{c}{rgb}{0,0,0};
\colorlet{c}{kugray};
\draw [c] (2.76333,4.12489) -- (2.76333,4.1477);
\draw [c] (2.76333,4.1477) -- (2.76333,4.16894);
\draw [c] (2.74851,4.1477) -- (2.76333,4.1477);
\draw [c] (2.76333,4.1477) -- (2.77815,4.1477);
\definecolor{c}{rgb}{0,0,0};
\colorlet{c}{kugray};
\draw [c] (2.79296,4.13183) -- (2.79296,4.15469);
\draw [c] (2.79296,4.15469) -- (2.79296,4.17597);
\draw [c] (2.77815,4.15469) -- (2.79296,4.15469);
\draw [c] (2.79296,4.15469) -- (2.80778,4.15469);
\definecolor{c}{rgb}{0,0,0};
\colorlet{c}{kugray};
\draw [c] (2.8226,4.05797) -- (2.8226,4.08216);
\draw [c] (2.8226,4.08216) -- (2.8226,4.10458);
\draw [c] (2.80778,4.08216) -- (2.8226,4.08216);
\draw [c] (2.8226,4.08216) -- (2.83742,4.08216);
\definecolor{c}{rgb}{0,0,0};
\colorlet{c}{kugray};
\draw [c] (2.85224,4.04617) -- (2.85224,4.07199);
\draw [c] (2.85224,4.07199) -- (2.85224,4.09581);
\draw [c] (2.83742,4.07199) -- (2.85224,4.07199);
\draw [c] (2.85224,4.07199) -- (2.86705,4.07199);
\definecolor{c}{rgb}{0,0,0};
\colorlet{c}{kugray};
\draw [c] (2.88187,4.02533) -- (2.88187,4.05093);
\draw [c] (2.88187,4.05093) -- (2.88187,4.07457);
\draw [c] (2.86705,4.05093) -- (2.88187,4.05093);
\draw [c] (2.88187,4.05093) -- (2.89669,4.05093);
\definecolor{c}{rgb}{0,0,0};
\colorlet{c}{kugray};
\draw [c] (2.91151,4.0461) -- (2.91151,4.07164);
\draw [c] (2.91151,4.07164) -- (2.91151,4.09522);
\draw [c] (2.89669,4.07164) -- (2.91151,4.07164);
\draw [c] (2.91151,4.07164) -- (2.92632,4.07164);
\definecolor{c}{rgb}{0,0,0};
\colorlet{c}{kugray};
\draw [c] (2.94114,3.94743) -- (2.94114,3.97699);
\draw [c] (2.94114,3.97699) -- (2.94114,4.00396);
\draw [c] (2.92632,3.97699) -- (2.94114,3.97699);
\draw [c] (2.94114,3.97699) -- (2.95596,3.97699);
\definecolor{c}{rgb}{0,0,0};
\colorlet{c}{kugray};
\draw [c] (2.97078,3.92033) -- (2.97078,3.95122);
\draw [c] (2.97078,3.95122) -- (2.97078,3.97929);
\draw [c] (2.95596,3.95122) -- (2.97078,3.95122);
\draw [c] (2.97078,3.95122) -- (2.9856,3.95122);
\definecolor{c}{rgb}{0,0,0};
\colorlet{c}{kugray};
\draw [c] (3.00041,3.95455) -- (3.00041,3.98496);
\draw [c] (3.00041,3.98496) -- (3.00041,4.01263);
\draw [c] (2.9856,3.98496) -- (3.00041,3.98496);
\draw [c] (3.00041,3.98496) -- (3.01523,3.98496);
\definecolor{c}{rgb}{0,0,0};
\colorlet{c}{kugray};
\draw [c] (3.03005,3.91612) -- (3.03005,3.94874);
\draw [c] (3.03005,3.94874) -- (3.03005,3.97824);
\draw [c] (3.01523,3.94874) -- (3.03005,3.94874);
\draw [c] (3.03005,3.94874) -- (3.04487,3.94874);
\definecolor{c}{rgb}{0,0,0};
\colorlet{c}{kugray};
\draw [c] (3.05969,3.91532) -- (3.05969,3.94707);
\draw [c] (3.05969,3.94707) -- (3.05969,3.97585);
\draw [c] (3.04487,3.94707) -- (3.05969,3.94707);
\draw [c] (3.05969,3.94707) -- (3.0745,3.94707);
\definecolor{c}{rgb}{0,0,0};
\colorlet{c}{kugray};
\draw [c] (3.08932,3.82813) -- (3.08932,3.8633);
\draw [c] (3.08932,3.8633) -- (3.08932,3.89486);
\draw [c] (3.0745,3.8633) -- (3.08932,3.8633);
\draw [c] (3.08932,3.8633) -- (3.10414,3.8633);
\definecolor{c}{rgb}{0,0,0};
\colorlet{c}{kugray};
\draw [c] (3.11896,3.82152) -- (3.11896,3.85725);
\draw [c] (3.11896,3.85725) -- (3.11896,3.88926);
\draw [c] (3.10414,3.85725) -- (3.11896,3.85725);
\draw [c] (3.11896,3.85725) -- (3.13377,3.85725);
\definecolor{c}{rgb}{0,0,0};
\colorlet{c}{kugray};
\draw [c] (3.14859,3.81099) -- (3.14859,3.849);
\draw [c] (3.14859,3.849) -- (3.14859,3.88283);
\draw [c] (3.13377,3.849) -- (3.14859,3.849);
\draw [c] (3.14859,3.849) -- (3.16341,3.849);
\definecolor{c}{rgb}{0,0,0};
\colorlet{c}{kugray};
\draw [c] (3.17823,3.68505) -- (3.17823,3.72686);
\draw [c] (3.17823,3.72686) -- (3.17823,3.76367);
\draw [c] (3.16341,3.72686) -- (3.17823,3.72686);
\draw [c] (3.17823,3.72686) -- (3.19305,3.72686);
\definecolor{c}{rgb}{0,0,0};
\colorlet{c}{kugray};
\draw [c] (3.20786,3.75027) -- (3.20786,3.79207);
\draw [c] (3.20786,3.79207) -- (3.20786,3.82886);
\draw [c] (3.19305,3.79207) -- (3.20786,3.79207);
\draw [c] (3.20786,3.79207) -- (3.22268,3.79207);
\definecolor{c}{rgb}{0,0,0};
\colorlet{c}{kugray};
\draw [c] (3.2375,3.78399) -- (3.2375,3.82558);
\draw [c] (3.2375,3.82558) -- (3.2375,3.86222);
\draw [c] (3.22268,3.82558) -- (3.2375,3.82558);
\draw [c] (3.2375,3.82558) -- (3.25232,3.82558);
\definecolor{c}{rgb}{0,0,0};
\colorlet{c}{kugray};
\draw [c] (3.26714,3.71561) -- (3.26714,3.76154);
\draw [c] (3.26714,3.76154) -- (3.26714,3.80149);
\draw [c] (3.25232,3.76154) -- (3.26714,3.76154);
\draw [c] (3.26714,3.76154) -- (3.28195,3.76154);
\definecolor{c}{rgb}{0,0,0};
\colorlet{c}{kugray};
\draw [c] (3.29677,3.74464) -- (3.29677,3.78592);
\draw [c] (3.29677,3.78592) -- (3.29677,3.82231);
\draw [c] (3.28195,3.78592) -- (3.29677,3.78592);
\draw [c] (3.29677,3.78592) -- (3.31159,3.78592);
\definecolor{c}{rgb}{0,0,0};
\colorlet{c}{kugray};
\draw [c] (3.32641,3.60212) -- (3.32641,3.65046);
\draw [c] (3.32641,3.65046) -- (3.32641,3.69223);
\draw [c] (3.31159,3.65046) -- (3.32641,3.65046);
\draw [c] (3.32641,3.65046) -- (3.34123,3.65046);
\definecolor{c}{rgb}{0,0,0};
\colorlet{c}{kugray};
\draw [c] (3.35604,3.66614) -- (3.35604,3.71219);
\draw [c] (3.35604,3.71219) -- (3.35604,3.75225);
\draw [c] (3.34123,3.71219) -- (3.35604,3.71219);
\draw [c] (3.35604,3.71219) -- (3.37086,3.71219);
\definecolor{c}{rgb}{0,0,0};
\colorlet{c}{kugray};
\draw [c] (3.38568,3.70789) -- (3.38568,3.7536);
\draw [c] (3.38568,3.7536) -- (3.38568,3.7934);
\draw [c] (3.37086,3.7536) -- (3.38568,3.7536);
\draw [c] (3.38568,3.7536) -- (3.4005,3.7536);
\definecolor{c}{rgb}{0,0,0};
\colorlet{c}{kugray};
\draw [c] (3.41531,3.59749) -- (3.41531,3.64919);
\draw [c] (3.41531,3.64919) -- (3.41531,3.69343);
\draw [c] (3.4005,3.64919) -- (3.41531,3.64919);
\draw [c] (3.41531,3.64919) -- (3.43013,3.64919);
\definecolor{c}{rgb}{0,0,0};
\colorlet{c}{kugray};
\draw [c] (3.44495,3.54922) -- (3.44495,3.6056);
\draw [c] (3.44495,3.6056) -- (3.44495,3.65323);
\draw [c] (3.43013,3.6056) -- (3.44495,3.6056);
\draw [c] (3.44495,3.6056) -- (3.45977,3.6056);
\definecolor{c}{rgb}{0,0,0};
\colorlet{c}{kugray};
\draw [c] (3.47459,3.5315) -- (3.47459,3.58602);
\draw [c] (3.47459,3.58602) -- (3.47459,3.63231);
\draw [c] (3.45977,3.58602) -- (3.47459,3.58602);
\draw [c] (3.47459,3.58602) -- (3.4894,3.58602);
\definecolor{c}{rgb}{0,0,0};
\colorlet{c}{kugray};
\draw [c] (3.50422,3.44087) -- (3.50422,3.50756);
\draw [c] (3.50422,3.50756) -- (3.50422,3.56234);
\draw [c] (3.4894,3.50756) -- (3.50422,3.50756);
\draw [c] (3.50422,3.50756) -- (3.51904,3.50756);
\definecolor{c}{rgb}{0,0,0};
\colorlet{c}{kugray};
\draw [c] (3.53386,3.57716) -- (3.53386,3.62954);
\draw [c] (3.53386,3.62954) -- (3.53386,3.67429);
\draw [c] (3.51904,3.62954) -- (3.53386,3.62954);
\draw [c] (3.53386,3.62954) -- (3.54868,3.62954);
\definecolor{c}{rgb}{0,0,0};
\colorlet{c}{kugray};
\draw [c] (3.56349,3.59791) -- (3.56349,3.64909);
\draw [c] (3.56349,3.64909) -- (3.56349,3.69297);
\draw [c] (3.54868,3.64909) -- (3.56349,3.64909);
\draw [c] (3.56349,3.64909) -- (3.57831,3.64909);
\definecolor{c}{rgb}{0,0,0};
\colorlet{c}{kugray};
\draw [c] (3.59313,3.48016) -- (3.59313,3.54299);
\draw [c] (3.59313,3.54299) -- (3.59313,3.59515);
\draw [c] (3.57831,3.54299) -- (3.59313,3.54299);
\draw [c] (3.59313,3.54299) -- (3.60795,3.54299);
\definecolor{c}{rgb}{0,0,0};
\colorlet{c}{kugray};
\draw [c] (3.62276,3.49926) -- (3.62276,3.5653);
\draw [c] (3.62276,3.5653) -- (3.62276,3.61965);
\draw [c] (3.60795,3.5653) -- (3.62276,3.5653);
\draw [c] (3.62276,3.5653) -- (3.63758,3.5653);
\definecolor{c}{rgb}{0,0,0};
\colorlet{c}{kugray};
\draw [c] (3.6524,3.38318) -- (3.6524,3.45701);
\draw [c] (3.6524,3.45701) -- (3.6524,3.51651);
\draw [c] (3.63758,3.45701) -- (3.6524,3.45701);
\draw [c] (3.6524,3.45701) -- (3.66722,3.45701);
\definecolor{c}{rgb}{0,0,0};
\colorlet{c}{kugray};
\draw [c] (3.68204,3.38716) -- (3.68204,3.45825);
\draw [c] (3.68204,3.45825) -- (3.68204,3.51597);
\draw [c] (3.66722,3.45825) -- (3.68204,3.45825);
\draw [c] (3.68204,3.45825) -- (3.69685,3.45825);
\definecolor{c}{rgb}{0,0,0};
\colorlet{c}{kugray};
\draw [c] (3.71167,3.27025) -- (3.71167,3.35157);
\draw [c] (3.71167,3.35157) -- (3.71167,3.41584);
\draw [c] (3.69685,3.35157) -- (3.71167,3.35157);
\draw [c] (3.71167,3.35157) -- (3.72649,3.35157);
\definecolor{c}{rgb}{0,0,0};
\colorlet{c}{kugray};
\draw [c] (3.74131,3.42793) -- (3.74131,3.49218);
\draw [c] (3.74131,3.49218) -- (3.74131,3.54531);
\draw [c] (3.72649,3.49218) -- (3.74131,3.49218);
\draw [c] (3.74131,3.49218) -- (3.75613,3.49218);
\definecolor{c}{rgb}{0,0,0};
\colorlet{c}{kugray};
\draw [c] (3.77094,3.39681) -- (3.77094,3.4655);
\draw [c] (3.77094,3.4655) -- (3.77094,3.52163);
\draw [c] (3.75613,3.4655) -- (3.77094,3.4655);
\draw [c] (3.77094,3.4655) -- (3.78576,3.4655);
\definecolor{c}{rgb}{0,0,0};
\colorlet{c}{kugray};
\draw [c] (3.80058,3.30807) -- (3.80058,3.39026);
\draw [c] (3.80058,3.39026) -- (3.80058,3.45506);
\draw [c] (3.78576,3.39026) -- (3.80058,3.39026);
\draw [c] (3.80058,3.39026) -- (3.8154,3.39026);
\definecolor{c}{rgb}{0,0,0};
\colorlet{c}{kugray};
\draw [c] (3.83022,3.36833) -- (3.83022,3.44242);
\draw [c] (3.83022,3.44242) -- (3.83022,3.50209);
\draw [c] (3.8154,3.44242) -- (3.83022,3.44242);
\draw [c] (3.83022,3.44242) -- (3.84503,3.44242);
\definecolor{c}{rgb}{0,0,0};
\colorlet{c}{kugray};
\draw [c] (3.85985,3.36697) -- (3.85985,3.43885);
\draw [c] (3.85985,3.43885) -- (3.85985,3.49708);
\draw [c] (3.84503,3.43885) -- (3.85985,3.43885);
\draw [c] (3.85985,3.43885) -- (3.87467,3.43885);
\definecolor{c}{rgb}{0,0,0};
\colorlet{c}{kugray};
\draw [c] (3.88949,3.18339) -- (3.88949,3.27641);
\draw [c] (3.88949,3.27641) -- (3.88949,3.34775);
\draw [c] (3.87467,3.27641) -- (3.88949,3.27641);
\draw [c] (3.88949,3.27641) -- (3.9043,3.27641);
\definecolor{c}{rgb}{0,0,0};
\colorlet{c}{kugray};
\draw [c] (3.91912,3.36859) -- (3.91912,3.433);
\draw [c] (3.91912,3.433) -- (3.91912,3.48624);
\draw [c] (3.9043,3.433) -- (3.91912,3.433);
\draw [c] (3.91912,3.433) -- (3.93394,3.433);
\definecolor{c}{rgb}{0,0,0};
\colorlet{c}{kugray};
\draw [c] (3.94876,3.27957) -- (3.94876,3.34398);
\draw [c] (3.94876,3.34398) -- (3.94876,3.39722);
\draw [c] (3.93394,3.34398) -- (3.94876,3.34398);
\draw [c] (3.94876,3.34398) -- (3.96358,3.34398);
\definecolor{c}{rgb}{0,0,0};
\colorlet{c}{kugray};
\draw [c] (3.97839,3.23181) -- (3.97839,3.2842);
\draw [c] (3.97839,3.2842) -- (3.97839,3.32896);
\draw [c] (3.96358,3.2842) -- (3.97839,3.2842);
\draw [c] (3.97839,3.2842) -- (3.99321,3.2842);
\definecolor{c}{rgb}{0,0,0};
\colorlet{c}{kugray};
\draw [c] (4.00803,3.23298) -- (4.00803,3.24514);
\draw [c] (4.00803,3.24514) -- (4.00803,3.25684);
\draw [c] (3.99321,3.24514) -- (4.00803,3.24514);
\draw [c] (4.00803,3.24514) -- (4.02285,3.24514);
\definecolor{c}{rgb}{0,0,0};
\colorlet{c}{kugray};
\draw [c] (4.03767,3.23899) -- (4.03767,3.25079);
\draw [c] (4.03767,3.25079) -- (4.03767,3.26215);
\draw [c] (4.02285,3.25079) -- (4.03767,3.25079);
\draw [c] (4.03767,3.25079) -- (4.05248,3.25079);
\definecolor{c}{rgb}{0,0,0};
\colorlet{c}{kugray};
\draw [c] (4.0673,3.24063) -- (4.0673,3.25269);
\draw [c] (4.0673,3.25269) -- (4.0673,3.26429);
\draw [c] (4.05248,3.25269) -- (4.0673,3.25269);
\draw [c] (4.0673,3.25269) -- (4.08212,3.25269);
\definecolor{c}{rgb}{0,0,0};
\colorlet{c}{kugray};
\draw [c] (4.09694,3.22722) -- (4.09694,3.23915);
\draw [c] (4.09694,3.23915) -- (4.09694,3.25063);
\draw [c] (4.08212,3.23915) -- (4.09694,3.23915);
\draw [c] (4.09694,3.23915) -- (4.11175,3.23915);
\definecolor{c}{rgb}{0,0,0};
\colorlet{c}{kugray};
\draw [c] (4.12657,3.2116) -- (4.12657,3.22405);
\draw [c] (4.12657,3.22405) -- (4.12657,3.23601);
\draw [c] (4.11175,3.22405) -- (4.12657,3.22405);
\draw [c] (4.12657,3.22405) -- (4.14139,3.22405);
\definecolor{c}{rgb}{0,0,0};
\colorlet{c}{kugray};
\draw [c] (4.15621,3.18215) -- (4.15621,3.19496);
\draw [c] (4.15621,3.19496) -- (4.15621,3.20726);
\draw [c] (4.14139,3.19496) -- (4.15621,3.19496);
\draw [c] (4.15621,3.19496) -- (4.17103,3.19496);
\definecolor{c}{rgb}{0,0,0};
\colorlet{c}{kugray};
\draw [c] (4.18584,3.18206) -- (4.18584,3.19521);
\draw [c] (4.18584,3.19521) -- (4.18584,3.20782);
\draw [c] (4.17103,3.19521) -- (4.18584,3.19521);
\draw [c] (4.18584,3.19521) -- (4.20066,3.19521);
\definecolor{c}{rgb}{0,0,0};
\colorlet{c}{kugray};
\draw [c] (4.21548,3.14682) -- (4.21548,3.16086);
\draw [c] (4.21548,3.16086) -- (4.21548,3.17428);
\draw [c] (4.20066,3.16086) -- (4.21548,3.16086);
\draw [c] (4.21548,3.16086) -- (4.2303,3.16086);
\definecolor{c}{rgb}{0,0,0};
\colorlet{c}{kugray};
\draw [c] (4.24512,3.16809) -- (4.24512,3.1818);
\draw [c] (4.24512,3.1818) -- (4.24512,3.19493);
\draw [c] (4.2303,3.1818) -- (4.24512,3.1818);
\draw [c] (4.24512,3.1818) -- (4.25993,3.1818);
\definecolor{c}{rgb}{0,0,0};
\colorlet{c}{kugray};
\draw [c] (4.27475,3.10937) -- (4.27475,3.12421);
\draw [c] (4.27475,3.12421) -- (4.27475,3.13837);
\draw [c] (4.25993,3.12421) -- (4.27475,3.12421);
\draw [c] (4.27475,3.12421) -- (4.28957,3.12421);
\definecolor{c}{rgb}{0,0,0};
\colorlet{c}{kugray};
\draw [c] (4.30439,3.11556) -- (4.30439,3.12992);
\draw [c] (4.30439,3.12992) -- (4.30439,3.14364);
\draw [c] (4.28957,3.12992) -- (4.30439,3.12992);
\draw [c] (4.30439,3.12992) -- (4.31921,3.12992);
\definecolor{c}{rgb}{0,0,0};
\colorlet{c}{kugray};
\draw [c] (4.33402,3.13544) -- (4.33402,3.14949);
\draw [c] (4.33402,3.14949) -- (4.33402,3.16293);
\draw [c] (4.31921,3.14949) -- (4.33402,3.14949);
\draw [c] (4.33402,3.14949) -- (4.34884,3.14949);
\definecolor{c}{rgb}{0,0,0};
\colorlet{c}{kugray};
\draw [c] (4.36366,3.09108) -- (4.36366,3.10621);
\draw [c] (4.36366,3.10621) -- (4.36366,3.12063);
\draw [c] (4.34884,3.10621) -- (4.36366,3.10621);
\draw [c] (4.36366,3.10621) -- (4.37848,3.10621);
\definecolor{c}{rgb}{0,0,0};
\colorlet{c}{kugray};
\draw [c] (4.39329,3.08833) -- (4.39329,3.10339);
\draw [c] (4.39329,3.10339) -- (4.39329,3.11775);
\draw [c] (4.37848,3.10339) -- (4.39329,3.10339);
\draw [c] (4.39329,3.10339) -- (4.40811,3.10339);
\definecolor{c}{rgb}{0,0,0};
\colorlet{c}{kugray};
\draw [c] (4.42293,3.06637) -- (4.42293,3.08234);
\draw [c] (4.42293,3.08234) -- (4.42293,3.09751);
\draw [c] (4.40811,3.08234) -- (4.42293,3.08234);
\draw [c] (4.42293,3.08234) -- (4.43775,3.08234);
\definecolor{c}{rgb}{0,0,0};
\colorlet{c}{kugray};
\draw [c] (4.45257,3.03685) -- (4.45257,3.05337);
\draw [c] (4.45257,3.05337) -- (4.45257,3.06905);
\draw [c] (4.43775,3.05337) -- (4.45257,3.05337);
\draw [c] (4.45257,3.05337) -- (4.46738,3.05337);
\definecolor{c}{rgb}{0,0,0};
\colorlet{c}{kugray};
\draw [c] (4.4822,3.05403) -- (4.4822,3.07028);
\draw [c] (4.4822,3.07028) -- (4.4822,3.08572);
\draw [c] (4.46738,3.07028) -- (4.4822,3.07028);
\draw [c] (4.4822,3.07028) -- (4.49702,3.07028);
\definecolor{c}{rgb}{0,0,0};
\colorlet{c}{kugray};
\draw [c] (4.51184,3.02475) -- (4.51184,3.04166);
\draw [c] (4.51184,3.04166) -- (4.51184,3.05768);
\draw [c] (4.49702,3.04166) -- (4.51184,3.04166);
\draw [c] (4.51184,3.04166) -- (4.52666,3.04166);
\definecolor{c}{rgb}{0,0,0};
\colorlet{c}{kugray};
\draw [c] (4.54147,2.97429) -- (4.54147,2.99215);
\draw [c] (4.54147,2.99215) -- (4.54147,3.00903);
\draw [c] (4.52666,2.99215) -- (4.54147,2.99215);
\draw [c] (4.54147,2.99215) -- (4.55629,2.99215);
\definecolor{c}{rgb}{0,0,0};
\colorlet{c}{kugray};
\draw [c] (4.57111,2.97145) -- (4.57111,2.9901);
\draw [c] (4.57111,2.9901) -- (4.57111,3.0077);
\draw [c] (4.55629,2.9901) -- (4.57111,2.9901);
\draw [c] (4.57111,2.9901) -- (4.58593,2.9901);
\definecolor{c}{rgb}{0,0,0};
\colorlet{c}{kugray};
\draw [c] (4.60075,2.99637) -- (4.60075,3.01434);
\draw [c] (4.60075,3.01434) -- (4.60075,3.03131);
\draw [c] (4.58593,3.01434) -- (4.60075,3.01434);
\draw [c] (4.60075,3.01434) -- (4.61556,3.01434);
\definecolor{c}{rgb}{0,0,0};
\colorlet{c}{kugray};
\draw [c] (4.63038,2.94705) -- (4.63038,2.96575);
\draw [c] (4.63038,2.96575) -- (4.63038,2.98338);
\draw [c] (4.61556,2.96575) -- (4.63038,2.96575);
\draw [c] (4.63038,2.96575) -- (4.6452,2.96575);
\definecolor{c}{rgb}{0,0,0};
\colorlet{c}{kugray};
\draw [c] (4.66002,2.94213) -- (4.66002,2.96119);
\draw [c] (4.66002,2.96119) -- (4.66002,2.97915);
\draw [c] (4.6452,2.96119) -- (4.66002,2.96119);
\draw [c] (4.66002,2.96119) -- (4.67483,2.96119);
\definecolor{c}{rgb}{0,0,0};
\colorlet{c}{kugray};
\draw [c] (4.68965,2.92699) -- (4.68965,2.94611);
\draw [c] (4.68965,2.94611) -- (4.68965,2.96412);
\draw [c] (4.67483,2.94611) -- (4.68965,2.94611);
\draw [c] (4.68965,2.94611) -- (4.70447,2.94611);
\definecolor{c}{rgb}{0,0,0};
\colorlet{c}{kugray};
\draw [c] (4.71929,2.90533) -- (4.71929,2.92537);
\draw [c] (4.71929,2.92537) -- (4.71929,2.94418);
\draw [c] (4.70447,2.92537) -- (4.71929,2.92537);
\draw [c] (4.71929,2.92537) -- (4.73411,2.92537);
\definecolor{c}{rgb}{0,0,0};
\colorlet{c}{kugray};
\draw [c] (4.74892,2.89803) -- (4.74892,2.9188);
\draw [c] (4.74892,2.9188) -- (4.74892,2.93826);
\draw [c] (4.73411,2.9188) -- (4.74892,2.9188);
\draw [c] (4.74892,2.9188) -- (4.76374,2.9188);
\definecolor{c}{rgb}{0,0,0};
\colorlet{c}{kugray};
\draw [c] (4.77856,2.88491) -- (4.77856,2.90571);
\draw [c] (4.77856,2.90571) -- (4.77856,2.92519);
\draw [c] (4.76374,2.90571) -- (4.77856,2.90571);
\draw [c] (4.77856,2.90571) -- (4.79338,2.90571);
\definecolor{c}{rgb}{0,0,0};
\colorlet{c}{kugray};
\draw [c] (4.8082,2.83295) -- (4.8082,2.85717);
\draw [c] (4.8082,2.85717) -- (4.8082,2.87962);
\draw [c] (4.79338,2.85717) -- (4.8082,2.85717);
\draw [c] (4.8082,2.85717) -- (4.82301,2.85717);
\definecolor{c}{rgb}{0,0,0};
\colorlet{c}{kugray};
\draw [c] (4.83783,2.84099) -- (4.83783,2.86366);
\draw [c] (4.83783,2.86366) -- (4.83783,2.88478);
\draw [c] (4.82301,2.86366) -- (4.83783,2.86366);
\draw [c] (4.83783,2.86366) -- (4.85265,2.86366);
\definecolor{c}{rgb}{0,0,0};
\colorlet{c}{kugray};
\draw [c] (4.86747,2.84516) -- (4.86747,2.86807);
\draw [c] (4.86747,2.86807) -- (4.86747,2.88939);
\draw [c] (4.85265,2.86807) -- (4.86747,2.86807);
\draw [c] (4.86747,2.86807) -- (4.88228,2.86807);
\definecolor{c}{rgb}{0,0,0};
\colorlet{c}{kugray};
\draw [c] (4.8971,2.82427) -- (4.8971,2.84804);
\draw [c] (4.8971,2.84804) -- (4.8971,2.87011);
\draw [c] (4.88228,2.84804) -- (4.8971,2.84804);
\draw [c] (4.8971,2.84804) -- (4.91192,2.84804);
\definecolor{c}{rgb}{0,0,0};
\colorlet{c}{kugray};
\draw [c] (4.92674,2.7958) -- (4.92674,2.82013);
\draw [c] (4.92674,2.82013) -- (4.92674,2.84268);
\draw [c] (4.91192,2.82013) -- (4.92674,2.82013);
\draw [c] (4.92674,2.82013) -- (4.94156,2.82013);
\definecolor{c}{rgb}{0,0,0};
\colorlet{c}{kugray};
\draw [c] (4.95637,2.72162) -- (4.95637,2.74908);
\draw [c] (4.95637,2.74908) -- (4.95637,2.7743);
\draw [c] (4.94156,2.74908) -- (4.95637,2.74908);
\draw [c] (4.95637,2.74908) -- (4.97119,2.74908);
\definecolor{c}{rgb}{0,0,0};
\colorlet{c}{kugray};
\draw [c] (4.98601,2.74467) -- (4.98601,2.77317);
\draw [c] (4.98601,2.77317) -- (4.98601,2.79924);
\draw [c] (4.97119,2.77317) -- (4.98601,2.77317);
\draw [c] (4.98601,2.77317) -- (5.00083,2.77317);
\definecolor{c}{rgb}{0,0,0};
\colorlet{c}{kugray};
\draw [c] (5.01565,2.72133) -- (5.01565,2.74919);
\draw [c] (5.01565,2.74919) -- (5.01565,2.77473);
\draw [c] (5.00083,2.74919) -- (5.01565,2.74919);
\draw [c] (5.01565,2.74919) -- (5.03046,2.74919);
\definecolor{c}{rgb}{0,0,0};
\colorlet{c}{kugray};
\draw [c] (5.04528,2.68622) -- (5.04528,2.71565);
\draw [c] (5.04528,2.71565) -- (5.04528,2.74251);
\draw [c] (5.03046,2.71565) -- (5.04528,2.71565);
\draw [c] (5.04528,2.71565) -- (5.0601,2.71565);
\definecolor{c}{rgb}{0,0,0};
\colorlet{c}{kugray};
\draw [c] (5.07492,2.69413) -- (5.07492,2.72282);
\draw [c] (5.07492,2.72282) -- (5.07492,2.74907);
\draw [c] (5.0601,2.72282) -- (5.07492,2.72282);
\draw [c] (5.07492,2.72282) -- (5.08974,2.72282);
\definecolor{c}{rgb}{0,0,0};
\colorlet{c}{kugray};
\draw [c] (5.10455,2.68206) -- (5.10455,2.71119);
\draw [c] (5.10455,2.71119) -- (5.10455,2.73781);
\draw [c] (5.08974,2.71119) -- (5.10455,2.71119);
\draw [c] (5.10455,2.71119) -- (5.11937,2.71119);
\definecolor{c}{rgb}{0,0,0};
\colorlet{c}{kugray};
\draw [c] (5.13419,2.68307) -- (5.13419,2.7116);
\draw [c] (5.13419,2.7116) -- (5.13419,2.7377);
\draw [c] (5.11937,2.7116) -- (5.13419,2.7116);
\draw [c] (5.13419,2.7116) -- (5.14901,2.7116);
\definecolor{c}{rgb}{0,0,0};
\colorlet{c}{kugray};
\draw [c] (5.16382,2.72714) -- (5.16382,2.75436);
\draw [c] (5.16382,2.75436) -- (5.16382,2.77936);
\draw [c] (5.14901,2.75436) -- (5.16382,2.75436);
\draw [c] (5.16382,2.75436) -- (5.17864,2.75436);
\definecolor{c}{rgb}{0,0,0};
\colorlet{c}{kugray};
\draw [c] (5.19346,2.64861) -- (5.19346,2.67901);
\draw [c] (5.19346,2.67901) -- (5.19346,2.70667);
\draw [c] (5.17864,2.67901) -- (5.19346,2.67901);
\draw [c] (5.19346,2.67901) -- (5.20828,2.67901);
\definecolor{c}{rgb}{0,0,0};
\colorlet{c}{kugray};
\draw [c] (5.2231,2.688) -- (5.2231,2.71811);
\draw [c] (5.2231,2.71811) -- (5.2231,2.74554);
\draw [c] (5.20828,2.71811) -- (5.2231,2.71811);
\draw [c] (5.2231,2.71811) -- (5.23791,2.71811);
\definecolor{c}{rgb}{0,0,0};
\colorlet{c}{kugray};
\draw [c] (5.25273,2.56862) -- (5.25273,2.60483);
\draw [c] (5.25273,2.60483) -- (5.25273,2.63722);
\draw [c] (5.23791,2.60483) -- (5.25273,2.60483);
\draw [c] (5.25273,2.60483) -- (5.26755,2.60483);
\definecolor{c}{rgb}{0,0,0};
\colorlet{c}{kugray};
\draw [c] (5.28237,2.61362) -- (5.28237,2.64718);
\draw [c] (5.28237,2.64718) -- (5.28237,2.67745);
\draw [c] (5.26755,2.64718) -- (5.28237,2.64718);
\draw [c] (5.28237,2.64718) -- (5.29719,2.64718);
\definecolor{c}{rgb}{0,0,0};
\colorlet{c}{kugray};
\draw [c] (5.312,2.58142) -- (5.312,2.61649);
\draw [c] (5.312,2.61649) -- (5.312,2.64798);
\draw [c] (5.29719,2.61649) -- (5.312,2.61649);
\draw [c] (5.312,2.61649) -- (5.32682,2.61649);
\definecolor{c}{rgb}{0,0,0};
\colorlet{c}{kugray};
\draw [c] (5.34164,2.59936) -- (5.34164,2.63325);
\draw [c] (5.34164,2.63325) -- (5.34164,2.66377);
\draw [c] (5.32682,2.63325) -- (5.34164,2.63325);
\draw [c] (5.34164,2.63325) -- (5.35646,2.63325);
\definecolor{c}{rgb}{0,0,0};
\colorlet{c}{kugray};
\draw [c] (5.37127,2.59301) -- (5.37127,2.62735);
\draw [c] (5.37127,2.62735) -- (5.37127,2.65824);
\draw [c] (5.35646,2.62735) -- (5.37127,2.62735);
\draw [c] (5.37127,2.62735) -- (5.38609,2.62735);
\definecolor{c}{rgb}{0,0,0};
\colorlet{c}{kugray};
\draw [c] (5.40091,2.57979) -- (5.40091,2.61666);
\draw [c] (5.40091,2.61666) -- (5.40091,2.64958);
\draw [c] (5.38609,2.61666) -- (5.40091,2.61666);
\draw [c] (5.40091,2.61666) -- (5.41573,2.61666);
\definecolor{c}{rgb}{0,0,0};
\colorlet{c}{kugray};
\draw [c] (5.43055,2.5888) -- (5.43055,2.62522);
\draw [c] (5.43055,2.62522) -- (5.43055,2.65779);
\draw [c] (5.41573,2.62522) -- (5.43055,2.62522);
\draw [c] (5.43055,2.62522) -- (5.44536,2.62522);
\definecolor{c}{rgb}{0,0,0};
\colorlet{c}{kugray};
\draw [c] (5.46018,2.49892) -- (5.46018,2.53839);
\draw [c] (5.46018,2.53839) -- (5.46018,2.57337);
\draw [c] (5.44536,2.53839) -- (5.46018,2.53839);
\draw [c] (5.46018,2.53839) -- (5.475,2.53839);
\definecolor{c}{rgb}{0,0,0};
\colorlet{c}{kugray};
\draw [c] (5.48982,2.51137) -- (5.48982,2.5487);
\draw [c] (5.48982,2.5487) -- (5.48982,2.58199);
\draw [c] (5.475,2.5487) -- (5.48982,2.5487);
\draw [c] (5.48982,2.5487) -- (5.50464,2.5487);
\definecolor{c}{rgb}{0,0,0};
\colorlet{c}{kugray};
\draw [c] (5.51945,2.5345) -- (5.51945,2.57323);
\draw [c] (5.51945,2.57323) -- (5.51945,2.60762);
\draw [c] (5.50464,2.57323) -- (5.51945,2.57323);
\draw [c] (5.51945,2.57323) -- (5.53427,2.57323);
\definecolor{c}{rgb}{0,0,0};
\colorlet{c}{kugray};
\draw [c] (5.54909,2.49349) -- (5.54909,2.533);
\draw [c] (5.54909,2.533) -- (5.54909,2.56801);
\draw [c] (5.53427,2.533) -- (5.54909,2.533);
\draw [c] (5.54909,2.533) -- (5.56391,2.533);
\definecolor{c}{rgb}{0,0,0};
\colorlet{c}{kugray};
\draw [c] (5.57873,2.46256) -- (5.57873,2.50333);
\draw [c] (5.57873,2.50333) -- (5.57873,2.53932);
\draw [c] (5.56391,2.50333) -- (5.57873,2.50333);
\draw [c] (5.57873,2.50333) -- (5.59354,2.50333);
\definecolor{c}{rgb}{0,0,0};
\colorlet{c}{kugray};
\draw [c] (5.60836,2.50823) -- (5.60836,2.54684);
\draw [c] (5.60836,2.54684) -- (5.60836,2.58115);
\draw [c] (5.59354,2.54684) -- (5.60836,2.54684);
\draw [c] (5.60836,2.54684) -- (5.62318,2.54684);
\definecolor{c}{rgb}{0,0,0};
\colorlet{c}{kugray};
\draw [c] (5.638,2.50569) -- (5.638,2.54708);
\draw [c] (5.638,2.54708) -- (5.638,2.58356);
\draw [c] (5.62318,2.54708) -- (5.638,2.54708);
\draw [c] (5.638,2.54708) -- (5.65281,2.54708);
\definecolor{c}{rgb}{0,0,0};
\colorlet{c}{kugray};
\draw [c] (5.66763,2.48101) -- (5.66763,2.5229);
\draw [c] (5.66763,2.5229) -- (5.66763,2.55976);
\draw [c] (5.65281,2.5229) -- (5.66763,2.5229);
\draw [c] (5.66763,2.5229) -- (5.68245,2.5229);
\definecolor{c}{rgb}{0,0,0};
\colorlet{c}{kugray};
\draw [c] (5.69727,2.3796) -- (5.69727,2.42684);
\draw [c] (5.69727,2.42684) -- (5.69727,2.46778);
\draw [c] (5.68245,2.42684) -- (5.69727,2.42684);
\draw [c] (5.69727,2.42684) -- (5.71209,2.42684);
\definecolor{c}{rgb}{0,0,0};
\colorlet{c}{kugray};
\draw [c] (5.7269,2.30877) -- (5.7269,2.35869);
\draw [c] (5.7269,2.35869) -- (5.7269,2.40163);
\draw [c] (5.71209,2.35869) -- (5.7269,2.35869);
\draw [c] (5.7269,2.35869) -- (5.74172,2.35869);
\definecolor{c}{rgb}{0,0,0};
\colorlet{c}{kugray};
\draw [c] (5.75654,2.43196) -- (5.75654,2.47986);
\draw [c] (5.75654,2.47986) -- (5.75654,2.5213);
\draw [c] (5.74172,2.47986) -- (5.75654,2.47986);
\draw [c] (5.75654,2.47986) -- (5.77136,2.47986);
\definecolor{c}{rgb}{0,0,0};
\colorlet{c}{kugray};
\draw [c] (5.78618,2.43169) -- (5.78618,2.47526);
\draw [c] (5.78618,2.47526) -- (5.78618,2.51342);
\draw [c] (5.77136,2.47526) -- (5.78618,2.47526);
\draw [c] (5.78618,2.47526) -- (5.80099,2.47526);
\definecolor{c}{rgb}{0,0,0};
\colorlet{c}{kugray};
\draw [c] (5.81581,2.39907) -- (5.81581,2.44526);
\draw [c] (5.81581,2.44526) -- (5.81581,2.48541);
\draw [c] (5.80099,2.44526) -- (5.81581,2.44526);
\draw [c] (5.81581,2.44526) -- (5.83063,2.44526);
\definecolor{c}{rgb}{0,0,0};
\colorlet{c}{kugray};
\draw [c] (5.84545,2.43121) -- (5.84545,2.47744);
\draw [c] (5.84545,2.47744) -- (5.84545,2.51762);
\draw [c] (5.83063,2.47744) -- (5.84545,2.47744);
\draw [c] (5.84545,2.47744) -- (5.86026,2.47744);
\definecolor{c}{rgb}{0,0,0};
\colorlet{c}{kugray};
\draw [c] (5.87508,2.29827) -- (5.87508,2.35564);
\draw [c] (5.87508,2.35564) -- (5.87508,2.40397);
\draw [c] (5.86026,2.35564) -- (5.87508,2.35564);
\draw [c] (5.87508,2.35564) -- (5.8899,2.35564);
\definecolor{c}{rgb}{0,0,0};
\colorlet{c}{kugray};
\draw [c] (5.90472,2.35345) -- (5.90472,2.40527);
\draw [c] (5.90472,2.40527) -- (5.90472,2.44961);
\draw [c] (5.8899,2.40527) -- (5.90472,2.40527);
\draw [c] (5.90472,2.40527) -- (5.91954,2.40527);
\definecolor{c}{rgb}{0,0,0};
\colorlet{c}{kugray};
\draw [c] (5.93435,2.28972) -- (5.93435,2.34179);
\draw [c] (5.93435,2.34179) -- (5.93435,2.38632);
\draw [c] (5.91954,2.34179) -- (5.93435,2.34179);
\draw [c] (5.93435,2.34179) -- (5.94917,2.34179);
\definecolor{c}{rgb}{0,0,0};
\colorlet{c}{kugray};
\draw [c] (5.96399,2.28856) -- (5.96399,2.34875);
\draw [c] (5.96399,2.34875) -- (5.96399,2.39907);
\draw [c] (5.94917,2.34875) -- (5.96399,2.34875);
\draw [c] (5.96399,2.34875) -- (5.97881,2.34875);
\definecolor{c}{rgb}{0,0,0};
\colorlet{c}{kugray};
\draw [c] (5.99363,2.34369) -- (5.99363,2.39484);
\draw [c] (5.99363,2.39484) -- (5.99363,2.43869);
\draw [c] (5.97881,2.39484) -- (5.99363,2.39484);
\draw [c] (5.99363,2.39484) -- (6.00844,2.39484);
\definecolor{c}{rgb}{0,0,0};
\colorlet{c}{kugray};
\draw [c] (6.02326,2.22004) -- (6.02326,2.27835);
\draw [c] (6.02326,2.27835) -- (6.02326,2.32735);
\draw [c] (6.00844,2.27835) -- (6.02326,2.27835);
\draw [c] (6.02326,2.27835) -- (6.03808,2.27835);
\definecolor{c}{rgb}{0,0,0};
\colorlet{c}{kugray};
\draw [c] (6.0529,2.24812) -- (6.0529,2.30663);
\draw [c] (6.0529,2.30663) -- (6.0529,2.35577);
\draw [c] (6.03808,2.30663) -- (6.0529,2.30663);
\draw [c] (6.0529,2.30663) -- (6.06772,2.30663);
\definecolor{c}{rgb}{0,0,0};
\colorlet{c}{kugray};
\draw [c] (6.08253,2.21392) -- (6.08253,2.27979);
\draw [c] (6.08253,2.27979) -- (6.08253,2.33402);
\draw [c] (6.06772,2.27979) -- (6.08253,2.27979);
\draw [c] (6.08253,2.27979) -- (6.09735,2.27979);
\definecolor{c}{rgb}{0,0,0};
\colorlet{c}{kugray};
\draw [c] (6.11217,2.28422) -- (6.11217,2.34425);
\draw [c] (6.11217,2.34425) -- (6.11217,2.39446);
\draw [c] (6.09735,2.34425) -- (6.11217,2.34425);
\draw [c] (6.11217,2.34425) -- (6.12699,2.34425);
\definecolor{c}{rgb}{0,0,0};
\colorlet{c}{kugray};
\draw [c] (6.1418,2.13526) -- (6.1418,2.20119);
\draw [c] (6.1418,2.20119) -- (6.1418,2.25547);
\draw [c] (6.12699,2.20119) -- (6.1418,2.20119);
\draw [c] (6.1418,2.20119) -- (6.15662,2.20119);
\definecolor{c}{rgb}{0,0,0};
\colorlet{c}{kugray};
\draw [c] (6.17144,2.21632) -- (6.17144,2.28052);
\draw [c] (6.17144,2.28052) -- (6.17144,2.33362);
\draw [c] (6.15662,2.28052) -- (6.17144,2.28052);
\draw [c] (6.17144,2.28052) -- (6.18626,2.28052);
\definecolor{c}{rgb}{0,0,0};
\colorlet{c}{kugray};
\draw [c] (6.20108,2.25826) -- (6.20108,2.32268);
\draw [c] (6.20108,2.32268) -- (6.20108,2.37592);
\draw [c] (6.18626,2.32268) -- (6.20108,2.32268);
\draw [c] (6.20108,2.32268) -- (6.21589,2.32268);
\definecolor{c}{rgb}{0,0,0};
\colorlet{c}{kugray};
\draw [c] (6.23071,2.21696) -- (6.23071,2.28153);
\draw [c] (6.23071,2.28153) -- (6.23071,2.33487);
\draw [c] (6.21589,2.28153) -- (6.23071,2.28153);
\draw [c] (6.23071,2.28153) -- (6.24553,2.28153);
\definecolor{c}{rgb}{0,0,0};
\colorlet{c}{kugray};
\draw [c] (6.26035,2.11717) -- (6.26035,2.18912);
\draw [c] (6.26035,2.18912) -- (6.26035,2.2474);
\draw [c] (6.24553,2.18912) -- (6.26035,2.18912);
\draw [c] (6.26035,2.18912) -- (6.27517,2.18912);
\definecolor{c}{rgb}{0,0,0};
\colorlet{c}{kugray};
\draw [c] (6.28998,2.21915) -- (6.28998,2.27945);
\draw [c] (6.28998,2.27945) -- (6.28998,2.32985);
\draw [c] (6.27517,2.27945) -- (6.28998,2.27945);
\draw [c] (6.28998,2.27945) -- (6.3048,2.27945);
\definecolor{c}{rgb}{0,0,0};
\colorlet{c}{kugray};
\draw [c] (6.31962,2.15067) -- (6.31962,2.22057);
\draw [c] (6.31962,2.22057) -- (6.31962,2.2775);
\draw [c] (6.3048,2.22057) -- (6.31962,2.22057);
\draw [c] (6.31962,2.22057) -- (6.33444,2.22057);
\definecolor{c}{rgb}{0,0,0};
\colorlet{c}{kugray};
\draw [c] (6.34926,2.08129) -- (6.34926,2.15905);
\draw [c] (6.34926,2.15905) -- (6.34926,2.22107);
\draw [c] (6.33444,2.15905) -- (6.34926,2.15905);
\draw [c] (6.34926,2.15905) -- (6.36407,2.15905);
\definecolor{c}{rgb}{0,0,0};
\colorlet{c}{kugray};
\draw [c] (6.37889,2.26481) -- (6.37889,2.32632);
\draw [c] (6.37889,2.32632) -- (6.37889,2.37756);
\draw [c] (6.36407,2.32632) -- (6.37889,2.32632);
\draw [c] (6.37889,2.32632) -- (6.39371,2.32632);
\definecolor{c}{rgb}{0,0,0};
\colorlet{c}{kugray};
\draw [c] (6.40853,2.10103) -- (6.40853,2.17287);
\draw [c] (6.40853,2.17287) -- (6.40853,2.23108);
\draw [c] (6.39371,2.17287) -- (6.40853,2.17287);
\draw [c] (6.40853,2.17287) -- (6.42334,2.17287);
\definecolor{c}{rgb}{0,0,0};
\colorlet{c}{kugray};
\draw [c] (6.43816,2.01805) -- (6.43816,2.10352);
\draw [c] (6.43816,2.10352) -- (6.43816,2.17034);
\draw [c] (6.42334,2.10352) -- (6.43816,2.10352);
\draw [c] (6.43816,2.10352) -- (6.45298,2.10352);
\definecolor{c}{rgb}{0,0,0};
\colorlet{c}{kugray};
\draw [c] (6.4678,2.13798) -- (6.4678,2.21314);
\draw [c] (6.4678,2.21314) -- (6.4678,2.27351);
\draw [c] (6.45298,2.21314) -- (6.4678,2.21314);
\draw [c] (6.4678,2.21314) -- (6.48262,2.21314);
\definecolor{c}{rgb}{0,0,0};
\colorlet{c}{kugray};
\draw [c] (6.49743,2.0628) -- (6.49743,2.13789);
\draw [c] (6.49743,2.13789) -- (6.49743,2.19821);
\draw [c] (6.48262,2.13789) -- (6.49743,2.13789);
\draw [c] (6.49743,2.13789) -- (6.51225,2.13789);
\definecolor{c}{rgb}{0,0,0};
\colorlet{c}{kugray};
\draw [c] (6.52707,2.03036) -- (6.52707,2.11014);
\draw [c] (6.52707,2.11014) -- (6.52707,2.17345);
\draw [c] (6.51225,2.11014) -- (6.52707,2.11014);
\draw [c] (6.52707,2.11014) -- (6.54189,2.11014);
\definecolor{c}{rgb}{0,0,0};
\colorlet{c}{kugray};
\draw [c] (6.55671,1.95941) -- (6.55671,2.05375);
\draw [c] (6.55671,2.05375) -- (6.55671,2.12586);
\draw [c] (6.54189,2.05375) -- (6.55671,2.05375);
\draw [c] (6.55671,2.05375) -- (6.57152,2.05375);
\definecolor{c}{rgb}{0,0,0};
\colorlet{c}{kugray};
\draw [c] (6.58634,2.0092) -- (6.58634,2.09944);
\draw [c] (6.58634,2.09944) -- (6.58634,2.16914);
\draw [c] (6.57152,2.09944) -- (6.58634,2.09944);
\draw [c] (6.58634,2.09944) -- (6.60116,2.09944);
\definecolor{c}{rgb}{0,0,0};
\colorlet{c}{kugray};
\draw [c] (6.61598,2.06823) -- (6.61598,2.14649);
\draw [c] (6.61598,2.14649) -- (6.61598,2.20883);
\draw [c] (6.60116,2.14649) -- (6.61598,2.14649);
\draw [c] (6.61598,2.14649) -- (6.63079,2.14649);
\definecolor{c}{rgb}{0,0,0};
\colorlet{c}{kugray};
\draw [c] (6.64561,1.91105) -- (6.64561,2.01133);
\draw [c] (6.64561,2.01133) -- (6.64561,2.08685);
\draw [c] (6.63079,2.01133) -- (6.64561,2.01133);
\draw [c] (6.64561,2.01133) -- (6.66043,2.01133);
\definecolor{c}{rgb}{0,0,0};
\colorlet{c}{kugray};
\draw [c] (6.67525,1.90644) -- (6.67525,2.01096);
\draw [c] (6.67525,2.01096) -- (6.67525,2.08884);
\draw [c] (6.66043,2.01096) -- (6.67525,2.01096);
\draw [c] (6.67525,2.01096) -- (6.69007,2.01096);
\definecolor{c}{rgb}{0,0,0};
\colorlet{c}{kugray};
\draw [c] (6.70488,1.98499) -- (6.70488,2.07197);
\draw [c] (6.70488,2.07197) -- (6.70488,2.13971);
\draw [c] (6.69007,2.07197) -- (6.70488,2.07197);
\draw [c] (6.70488,2.07197) -- (6.7197,2.07197);
\definecolor{c}{rgb}{0,0,0};
\colorlet{c}{kugray};
\draw [c] (6.73452,1.98323) -- (6.73452,2.07762);
\draw [c] (6.73452,2.07762) -- (6.73452,2.14976);
\draw [c] (6.7197,2.07762) -- (6.73452,2.07762);
\draw [c] (6.73452,2.07762) -- (6.74934,2.07762);
\definecolor{c}{rgb}{0,0,0};
\colorlet{c}{kugray};
\draw [c] (6.76416,1.8404) -- (6.76416,1.94614);
\draw [c] (6.76416,1.94614) -- (6.76416,2.0247);
\draw [c] (6.74934,1.94614) -- (6.76416,1.94614);
\draw [c] (6.76416,1.94614) -- (6.77897,1.94614);
\definecolor{c}{rgb}{0,0,0};
\colorlet{c}{kugray};
\draw [c] (6.79379,1.95326) -- (6.79379,2.04637);
\draw [c] (6.79379,2.04637) -- (6.79379,2.11775);
\draw [c] (6.77897,2.04637) -- (6.79379,2.04637);
\draw [c] (6.79379,2.04637) -- (6.80861,2.04637);
\definecolor{c}{rgb}{0,0,0};
\colorlet{c}{kugray};
\draw [c] (6.82343,1.84998) -- (6.82343,1.96142);
\draw [c] (6.82343,1.96142) -- (6.82343,2.04307);
\draw [c] (6.80861,1.96142) -- (6.82343,1.96142);
\draw [c] (6.82343,1.96142) -- (6.83824,1.96142);
\definecolor{c}{rgb}{0,0,0};
\colorlet{c}{kugray};
\draw [c] (6.85306,1.97353) -- (6.85306,2.06824);
\draw [c] (6.85306,2.06824) -- (6.85306,2.14058);
\draw [c] (6.83824,2.06824) -- (6.85306,2.06824);
\draw [c] (6.85306,2.06824) -- (6.86788,2.06824);
\definecolor{c}{rgb}{0,0,0};
\colorlet{c}{kugray};
\draw [c] (6.8827,1.94959) -- (6.8827,2.05345);
\draw [c] (6.8827,2.05345) -- (6.8827,2.13097);
\draw [c] (6.86788,2.05345) -- (6.8827,2.05345);
\draw [c] (6.8827,2.05345) -- (6.89752,2.05345);
\definecolor{c}{rgb}{0,0,0};
\colorlet{c}{kugray};
\draw [c] (6.91233,1.64738) -- (6.91233,1.796);
\draw [c] (6.91233,1.796) -- (6.91233,1.8958);
\draw [c] (6.89752,1.796) -- (6.91233,1.796);
\draw [c] (6.91233,1.796) -- (6.92715,1.796);
\definecolor{c}{rgb}{0,0,0};
\colorlet{c}{kugray};
\draw [c] (6.94197,1.81861) -- (6.94197,1.94633);
\draw [c] (6.94197,1.94633) -- (6.94197,2.03632);
\draw [c] (6.92715,1.94633) -- (6.94197,1.94633);
\draw [c] (6.94197,1.94633) -- (6.95679,1.94633);
\definecolor{c}{rgb}{0,0,0};
\colorlet{c}{kugray};
\draw [c] (6.97161,1.45467) -- (6.97161,1.67503);
\draw [c] (6.97161,1.67503) -- (6.97161,1.80215);
\draw [c] (6.95679,1.67503) -- (6.97161,1.67503);
\draw [c] (6.97161,1.67503) -- (6.98642,1.67503);
\definecolor{c}{rgb}{0,0,0};
\colorlet{c}{kugray};
\draw [c] (7.00124,1.92607) -- (7.00124,2.03252);
\draw [c] (7.00124,2.03252) -- (7.00124,2.11147);
\draw [c] (6.98642,2.03252) -- (7.00124,2.03252);
\draw [c] (7.00124,2.03252) -- (7.01606,2.03252);
\definecolor{c}{rgb}{0,0,0};
\colorlet{c}{kugray};
\draw [c] (7.03088,1.70142) -- (7.03088,1.83023);
\draw [c] (7.03088,1.83023) -- (7.03088,1.92076);
\draw [c] (7.01606,1.83023) -- (7.03088,1.83023);
\draw [c] (7.03088,1.83023) -- (7.0457,1.83023);
\definecolor{c}{rgb}{0,0,0};
\colorlet{c}{kugray};
\draw [c] (7.06051,1.69915) -- (7.06051,1.84157);
\draw [c] (7.06051,1.84157) -- (7.06051,1.93856);
\draw [c] (7.0457,1.84157) -- (7.06051,1.84157);
\draw [c] (7.06051,1.84157) -- (7.07533,1.84157);
\definecolor{c}{rgb}{0,0,0};
\colorlet{c}{kugray};
\draw [c] (7.09015,1.30062) -- (7.09015,1.58241);
\draw [c] (7.09015,1.58241) -- (7.09015,1.72703);
\draw [c] (7.07533,1.58241) -- (7.09015,1.58241);
\draw [c] (7.09015,1.58241) -- (7.10497,1.58241);
\definecolor{c}{rgb}{0,0,0};
\colorlet{c}{kugray};
\draw [c] (7.11978,1.6185) -- (7.11978,1.77873);
\draw [c] (7.11978,1.77873) -- (7.11978,1.88358);
\draw [c] (7.10497,1.77873) -- (7.11978,1.77873);
\draw [c] (7.11978,1.77873) -- (7.1346,1.77873);
\definecolor{c}{rgb}{0,0,0};
\colorlet{c}{kugray};
\draw [c] (7.14942,1.83806) -- (7.14942,1.95138);
\draw [c] (7.14942,1.95138) -- (7.14942,2.03403);
\draw [c] (7.1346,1.95138) -- (7.14942,1.95138);
\draw [c] (7.14942,1.95138) -- (7.16424,1.95138);
\definecolor{c}{rgb}{0,0,0};
\colorlet{c}{kugray};
\draw [c] (7.17906,1.68124) -- (7.17906,1.87494);
\draw [c] (7.17906,1.87494) -- (7.17906,1.99292);
\draw [c] (7.16424,1.87494) -- (7.17906,1.87494);
\draw [c] (7.17906,1.87494) -- (7.19387,1.87494);
\definecolor{c}{rgb}{0,0,0};
\colorlet{c}{kugray};
\draw [c] (7.20869,1.63815) -- (7.20869,1.80529);
\draw [c] (7.20869,1.80529) -- (7.20869,1.91302);
\draw [c] (7.19387,1.80529) -- (7.20869,1.80529);
\draw [c] (7.20869,1.80529) -- (7.22351,1.80529);
\definecolor{c}{rgb}{0,0,0};
\colorlet{c}{kugray};
\draw [c] (7.23833,1.534) -- (7.23833,1.72717);
\draw [c] (7.23833,1.72717) -- (7.23833,1.84495);
\draw [c] (7.22351,1.72717) -- (7.23833,1.72717);
\draw [c] (7.23833,1.72717) -- (7.25315,1.72717);
\definecolor{c}{rgb}{0,0,0};
\colorlet{c}{kugray};
\draw [c] (7.26796,1.44669) -- (7.26796,1.66186);
\draw [c] (7.26796,1.66186) -- (7.26796,1.78728);
\draw [c] (7.25315,1.66186) -- (7.26796,1.66186);
\draw [c] (7.26796,1.66186) -- (7.28278,1.66186);
\definecolor{c}{rgb}{0,0,0};
\colorlet{c}{kugray};
\draw [c] (7.2976,1.67008) -- (7.2976,1.80777);
\draw [c] (7.2976,1.80777) -- (7.2976,1.90257);
\draw [c] (7.28278,1.80777) -- (7.2976,1.80777);
\draw [c] (7.2976,1.80777) -- (7.31242,1.80777);
\definecolor{c}{rgb}{0,0,0};
\colorlet{c}{kugray};
\draw [c] (7.32724,1.55019) -- (7.32724,1.78423);
\draw [c] (7.32724,1.78423) -- (7.32724,1.91564);
\draw [c] (7.31242,1.78423) -- (7.32724,1.78423);
\draw [c] (7.32724,1.78423) -- (7.34205,1.78423);
\definecolor{c}{rgb}{0,0,0};
\colorlet{c}{kugray};
\draw [c] (7.35687,1.70986) -- (7.35687,1.88005);
\draw [c] (7.35687,1.88005) -- (7.35687,1.98902);
\draw [c] (7.34205,1.88005) -- (7.35687,1.88005);
\draw [c] (7.35687,1.88005) -- (7.37169,1.88005);
\definecolor{c}{rgb}{0,0,0};
\colorlet{c}{kugray};
\draw [c] (7.38651,1.39569) -- (7.38651,1.61683);
\draw [c] (7.38651,1.61683) -- (7.38651,1.7442);
\draw [c] (7.37169,1.61683) -- (7.38651,1.61683);
\draw [c] (7.38651,1.61683) -- (7.40132,1.61683);
\definecolor{c}{rgb}{0,0,0};
\colorlet{c}{kugray};
\draw [c] (7.41614,1.65215) -- (7.41614,1.82337);
\draw [c] (7.41614,1.82337) -- (7.41614,1.93274);
\draw [c] (7.40132,1.82337) -- (7.41614,1.82337);
\draw [c] (7.41614,1.82337) -- (7.43096,1.82337);
\definecolor{c}{rgb}{0,0,0};
\colorlet{c}{kugray};
\draw [c] (7.44578,1.61682) -- (7.44578,1.76694);
\draw [c] (7.44578,1.76694) -- (7.44578,1.86741);
\draw [c] (7.43096,1.76694) -- (7.44578,1.76694);
\draw [c] (7.44578,1.76694) -- (7.4606,1.76694);
\definecolor{c}{rgb}{0,0,0};
\colorlet{c}{kugray};
\draw [c] (7.47541,0.903192) -- (7.47541,1.30472);
\draw [c] (7.47541,1.30472) -- (7.47541,1.47322);
\draw [c] (7.4606,1.30472) -- (7.47541,1.30472);
\draw [c] (7.47541,1.30472) -- (7.49023,1.30472);
\definecolor{c}{rgb}{0,0,0};
\colorlet{c}{kugray};
\draw [c] (7.50505,1.17257) -- (7.50505,1.55106);
\draw [c] (7.50505,1.55106) -- (7.50505,1.71578);
\draw [c] (7.49023,1.55106) -- (7.50505,1.55106);
\draw [c] (7.50505,1.55106) -- (7.51987,1.55106);
\definecolor{c}{rgb}{0,0,0};
\colorlet{c}{kugray};
\draw [c] (7.53469,1.26039) -- (7.53469,1.5276);
\draw [c] (7.53469,1.5276) -- (7.53469,1.66846);
\draw [c] (7.51987,1.5276) -- (7.53469,1.5276);
\draw [c] (7.53469,1.5276) -- (7.5495,1.5276);
\definecolor{c}{rgb}{0,0,0};
\colorlet{c}{kugray};
\draw [c] (7.56432,1.35089) -- (7.56432,1.65389);
\draw [c] (7.56432,1.65389) -- (7.56432,1.80359);
\draw [c] (7.5495,1.65389) -- (7.56432,1.65389);
\draw [c] (7.56432,1.65389) -- (7.57914,1.65389);
\definecolor{c}{rgb}{0,0,0};
\colorlet{c}{kugray};
\draw [c] (7.59396,1.63685) -- (7.59396,1.81433);
\draw [c] (7.59396,1.81433) -- (7.59396,1.92619);
\draw [c] (7.57914,1.81433) -- (7.59396,1.81433);
\draw [c] (7.59396,1.81433) -- (7.60877,1.81433);
\definecolor{c}{rgb}{0,0,0};
\colorlet{c}{kugray};
\draw [c] (7.62359,1.2207) -- (7.62359,1.48705);
\draw [c] (7.62359,1.48705) -- (7.62359,1.62768);
\draw [c] (7.60877,1.48705) -- (7.62359,1.48705);
\draw [c] (7.62359,1.48705) -- (7.63841,1.48705);
\definecolor{c}{rgb}{0,0,0};
\colorlet{c}{kugray};
\draw [c] (7.65323,1.15339) -- (7.65323,1.44836);
\draw [c] (7.65323,1.44836) -- (7.65323,1.5962);
\draw [c] (7.63841,1.44836) -- (7.65323,1.44836);
\draw [c] (7.65323,1.44836) -- (7.66805,1.44836);
\definecolor{c}{rgb}{0,0,0};
\colorlet{c}{kugray};
\draw [c] (7.68286,0.824113) -- (7.68286,1.32118);
\draw [c] (7.68286,1.32118) -- (7.68286,1.50231);
\draw [c] (7.66805,1.32118) -- (7.68286,1.32118);
\draw [c] (7.68286,1.32118) -- (7.69768,1.32118);
\definecolor{c}{rgb}{0,0,0};
\colorlet{c}{kugray};
\draw [c] (7.7125,1.26126) -- (7.7125,1.5339);
\draw [c] (7.7125,1.5339) -- (7.7125,1.67619);
\draw [c] (7.69768,1.5339) -- (7.7125,1.5339);
\draw [c] (7.7125,1.5339) -- (7.72732,1.5339);
\definecolor{c}{rgb}{0,0,0};
\colorlet{c}{kugray};
\draw [c] (7.74214,1.48045) -- (7.74214,1.72479);
\draw [c] (7.74214,1.72479) -- (7.74214,1.85928);
\draw [c] (7.72732,1.72479) -- (7.74214,1.72479);
\draw [c] (7.74214,1.72479) -- (7.75695,1.72479);
\definecolor{c}{rgb}{0,0,0};
\colorlet{c}{kugray};
\draw [c] (7.77177,1.11205) -- (7.77177,1.40937);
\draw [c] (7.77177,1.40937) -- (7.77177,1.55776);
\draw [c] (7.75695,1.40937) -- (7.77177,1.40937);
\draw [c] (7.77177,1.40937) -- (7.78659,1.40937);
\definecolor{c}{rgb}{0,0,0};
\colorlet{c}{kugray};
\draw [c] (7.80141,1.34931) -- (7.80141,1.63981);
\draw [c] (7.80141,1.63981) -- (7.80141,1.78657);
\draw [c] (7.78659,1.63981) -- (7.80141,1.63981);
\draw [c] (7.80141,1.63981) -- (7.81623,1.63981);
\definecolor{c}{rgb}{0,0,0};
\colorlet{c}{kugray};
\draw [c] (7.83104,1.0292) -- (7.83104,1.41463);
\draw [c] (7.83104,1.41463) -- (7.83104,1.58052);
\draw [c] (7.81623,1.41463) -- (7.83104,1.41463);
\draw [c] (7.83104,1.41463) -- (7.84586,1.41463);
\definecolor{c}{rgb}{0,0,0};
\colorlet{c}{kugray};
\draw [c] (7.86068,0.596817) -- (7.86068,1.16974);
\draw [c] (7.86068,1.16974) -- (7.86068,1.38313);
\draw [c] (7.84586,1.16974) -- (7.86068,1.16974);
\draw [c] (7.86068,1.16974) -- (7.8755,1.16974);
\definecolor{c}{rgb}{0,0,0};
\colorlet{c}{kugray};
\draw [c] (7.89031,1.22506) -- (7.89031,1.49977);
\draw [c] (7.89031,1.49977) -- (7.89031,1.64259);
\draw [c] (7.8755,1.49977) -- (7.89031,1.49977);
\draw [c] (7.89031,1.49977) -- (7.90513,1.49977);
\definecolor{c}{rgb}{0,0,0};
\colorlet{c}{kugray};
\draw [c] (7.91995,1.37554) -- (7.91995,1.58913);
\draw [c] (7.91995,1.58913) -- (7.91995,1.71402);
\draw [c] (7.90513,1.58913) -- (7.91995,1.58913);
\draw [c] (7.91995,1.58913) -- (7.93477,1.58913);
\definecolor{c}{rgb}{0,0,0};
\colorlet{c}{kugray};
\draw [c] (7.94959,0.970556) -- (7.94959,1.35289);
\draw [c] (7.94959,1.35289) -- (7.94959,1.51826);
\draw [c] (7.93477,1.35289) -- (7.94959,1.35289);
\draw [c] (7.94959,1.35289) -- (7.9644,1.35289);
\definecolor{c}{rgb}{0,0,0};
\colorlet{c}{kugray};
\draw [c] (7.97922,1.42663) -- (7.97922,1.62932);
\draw [c] (7.97922,1.62932) -- (7.97922,1.75049);
\draw [c] (7.9644,1.62932) -- (7.97922,1.62932);
\draw [c] (7.97922,1.62932) -- (7.99404,1.62932);
\definecolor{c}{rgb}{0,0,0};
\colorlet{c}{kugray};
\draw [c] (8.00886,1.24387) -- (8.00886,1.51218);
\draw [c] (8.00886,1.51218) -- (8.00886,1.65334);
\draw [c] (7.99404,1.51218) -- (8.00886,1.51218);
\draw [c] (8.00886,1.51218) -- (8.02368,1.51218);
\definecolor{c}{rgb}{0,0,0};
\colorlet{c}{kugray};
\draw [c] (8.03849,1.45206) -- (8.03849,1.64214);
\draw [c] (8.03849,1.64214) -- (8.03849,1.75879);
\draw [c] (8.02368,1.64214) -- (8.03849,1.64214);
\draw [c] (8.03849,1.64214) -- (8.05331,1.64214);
\definecolor{c}{rgb}{0,0,0};
\colorlet{c}{kugray};
\draw [c] (8.09776,1.35157) -- (8.09776,1.5665);
\draw [c] (8.09776,1.5665) -- (8.09776,1.69184);
\draw [c] (8.08295,1.5665) -- (8.09776,1.5665);
\draw [c] (8.09776,1.5665) -- (8.11258,1.5665);
\definecolor{c}{rgb}{0,0,0};
\colorlet{c}{kugray};
\draw [c] (8.1274,1.02943) -- (8.1274,1.41864);
\draw [c] (8.1274,1.41864) -- (8.1274,1.58516);
\draw [c] (8.11258,1.41864) -- (8.1274,1.41864);
\draw [c] (8.1274,1.41864) -- (8.14222,1.41864);
\definecolor{c}{rgb}{0,0,0};
\colorlet{c}{kugray};
\draw [c] (8.18667,1.3217) -- (8.18667,1.60896);
\draw [c] (8.18667,1.60896) -- (8.18667,1.75493);
\draw [c] (8.17185,1.60896) -- (8.18667,1.60896);
\draw [c] (8.18667,1.60896) -- (8.20149,1.60896);
\definecolor{c}{rgb}{0,0,0};
\colorlet{c}{kugray};
\draw [c] (8.21631,1.44991) -- (8.21631,1.72488);
\draw [c] (8.21631,1.72488) -- (8.21631,1.86777);
\draw [c] (8.20149,1.72488) -- (8.21631,1.72488);
\draw [c] (8.21631,1.72488) -- (8.23113,1.72488);
\definecolor{c}{rgb}{0,0,0};
\colorlet{c}{kugray};
\draw [c] (8.24594,0.596817) -- (8.24594,1.18479);
\draw [c] (8.24594,1.18479) -- (8.24594,1.39818);
\draw [c] (8.23113,1.18479) -- (8.24594,1.18479);
\draw [c] (8.24594,1.18479) -- (8.26076,1.18479);
\definecolor{c}{rgb}{0,0,0};
\colorlet{c}{kugray};
\draw [c] (8.27558,1.33422) -- (8.27558,1.70646);
\draw [c] (8.27558,1.70646) -- (8.27558,1.8701);
\draw [c] (8.26076,1.70646) -- (8.27558,1.70646);
\draw [c] (8.27558,1.70646) -- (8.2904,1.70646);
\definecolor{c}{rgb}{0,0,0};
\colorlet{c}{kugray};
\draw [c] (8.30521,1.01154) -- (8.30521,1.39412);
\draw [c] (8.30521,1.39412) -- (8.30521,1.55954);
\draw [c] (8.2904,1.39412) -- (8.30521,1.39412);
\draw [c] (8.30521,1.39412) -- (8.32003,1.39412);
\definecolor{c}{rgb}{0,0,0};
\colorlet{c}{kugray};
\draw [c] (8.33485,1.16538) -- (8.33485,1.451);
\draw [c] (8.33485,1.451) -- (8.33485,1.59657);
\draw [c] (8.32003,1.451) -- (8.33485,1.451);
\draw [c] (8.33485,1.451) -- (8.34967,1.451);
\definecolor{c}{rgb}{0,0,0};
\colorlet{c}{kugray};
\draw [c] (8.36449,0.596817) -- (8.36449,1.27192);
\draw [c] (8.36449,1.27192) -- (8.36449,1.48532);
\draw [c] (8.34967,1.27192) -- (8.36449,1.27192);
\draw [c] (8.36449,1.27192) -- (8.3793,1.27192);
\definecolor{c}{rgb}{0,0,0};
\colorlet{c}{kugray};
\draw [c] (8.39412,0.596817) -- (8.39412,1.1669);
\draw [c] (8.39412,1.1669) -- (8.39412,1.38029);
\draw [c] (8.3793,1.1669) -- (8.39412,1.1669);
\draw [c] (8.39412,1.1669) -- (8.40894,1.1669);
\definecolor{c}{rgb}{0,0,0};
\colorlet{c}{kugray};
\draw [c] (8.45339,1.13633) -- (8.45339,1.56977);
\draw [c] (8.45339,1.56977) -- (8.45339,1.74299);
\draw [c] (8.43858,1.56977) -- (8.45339,1.56977);
\draw [c] (8.45339,1.56977) -- (8.46821,1.56977);
\definecolor{c}{rgb}{0,0,0};
\colorlet{c}{kugray};
\draw [c] (8.48303,0.596817) -- (8.48303,1.22594);
\draw [c] (8.48303,1.22594) -- (8.48303,1.43933);
\draw [c] (8.46821,1.22594) -- (8.48303,1.22594);
\draw [c] (8.48303,1.22594) -- (8.49785,1.22594);
\definecolor{c}{rgb}{0,0,0};
\colorlet{c}{kugray};
\draw [c] (8.5423,0.596817) -- (8.5423,1.08304);
\draw [c] (8.5423,1.08304) -- (8.5423,1.29643);
\draw [c] (8.52748,1.08304) -- (8.5423,1.08304);
\draw [c] (8.5423,1.08304) -- (8.55712,1.08304);
\definecolor{c}{rgb}{0,0,0};
\colorlet{c}{kugray};
\draw [c] (8.57194,0.596817) -- (8.57194,1.19595);
\draw [c] (8.57194,1.19595) -- (8.57194,1.40934);
\draw [c] (8.55712,1.19595) -- (8.57194,1.19595);
\draw [c] (8.57194,1.19595) -- (8.58675,1.19595);
\definecolor{c}{rgb}{0,0,0};
\colorlet{c}{kugray};
\draw [c] (8.66084,0.884632) -- (8.66084,1.27717);
\draw [c] (8.66084,1.27717) -- (8.66084,1.44423);
\draw [c] (8.64603,1.27717) -- (8.66084,1.27717);
\draw [c] (8.66084,1.27717) -- (8.67566,1.27717);
\definecolor{c}{rgb}{0,0,0};
\colorlet{c}{kugray};
\draw [c] (8.69048,0.596817) -- (8.69048,1.18479);
\draw [c] (8.69048,1.18479) -- (8.69048,1.39818);
\draw [c] (8.67566,1.18479) -- (8.69048,1.18479);
\draw [c] (8.69048,1.18479) -- (8.7053,1.18479);
\definecolor{c}{rgb}{0,0,0};
\colorlet{c}{kugray};
\draw [c] (8.72012,0.596817) -- (8.72012,1.21188);
\draw [c] (8.72012,1.21188) -- (8.72012,1.42527);
\draw [c] (8.7053,1.21188) -- (8.72012,1.21188);
\draw [c] (8.72012,1.21188) -- (8.73493,1.21188);
\definecolor{c}{rgb}{0,0,0};
\colorlet{c}{kugray};
\draw [c] (8.74975,0.596817) -- (8.74975,1.16598);
\draw [c] (8.74975,1.16598) -- (8.74975,1.37937);
\draw [c] (8.73493,1.16598) -- (8.74975,1.16598);
\draw [c] (8.74975,1.16598) -- (8.76457,1.16598);
\definecolor{c}{rgb}{0,0,0};
\colorlet{c}{kugray};
\draw [c] (8.77939,0.895648) -- (8.77939,1.31039);
\draw [c] (8.77939,1.31039) -- (8.77939,1.48091);
\draw [c] (8.76457,1.31039) -- (8.77939,1.31039);
\draw [c] (8.77939,1.31039) -- (8.79421,1.31039);
\definecolor{c}{rgb}{0,0,0};
\colorlet{c}{kugray};
\draw [c] (8.80902,1.29252) -- (8.80902,1.5583);
\draw [c] (8.80902,1.5583) -- (8.80902,1.69878);
\draw [c] (8.79421,1.5583) -- (8.80902,1.5583);
\draw [c] (8.80902,1.5583) -- (8.82384,1.5583);
\definecolor{c}{rgb}{0,0,0};
\colorlet{c}{kugray};
\draw [c] (8.83866,0.596817) -- (8.83866,1.24072);
\draw [c] (8.83866,1.24072) -- (8.83866,1.45411);
\draw [c] (8.82384,1.24072) -- (8.83866,1.24072);
\draw [c] (8.83866,1.24072) -- (8.85348,1.24072);
\definecolor{c}{rgb}{0,0,0};
\colorlet{c}{kugray};
\draw [c] (8.86829,0.596817) -- (8.86829,0.976443);
\draw [c] (8.86829,0.976443) -- (8.86829,1.18983);
\draw [c] (8.85348,0.976443) -- (8.86829,0.976443);
\draw [c] (8.86829,0.976443) -- (8.88311,0.976443);
\definecolor{c}{rgb}{0,0,0};
\colorlet{c}{kugray};
\draw [c] (8.92757,0.596817) -- (8.92757,1.19654);
\draw [c] (8.92757,1.19654) -- (8.92757,1.40993);
\draw [c] (8.91275,1.19654) -- (8.92757,1.19654);
\draw [c] (8.92757,1.19654) -- (8.94238,1.19654);
\definecolor{c}{rgb}{0,0,0};
\colorlet{c}{kugray};
\draw [c] (9.01647,0.596817) -- (9.01647,1.23526);
\draw [c] (9.01647,1.23526) -- (9.01647,1.44865);
\draw [c] (9.00166,1.23526) -- (9.01647,1.23526);
\draw [c] (9.01647,1.23526) -- (9.03129,1.23526);
\definecolor{c}{rgb}{0,0,0};
\colorlet{c}{kugray};
\draw [c] (9.07574,0.596817) -- (9.07574,1.10612);
\draw [c] (9.07574,1.10612) -- (9.07574,1.31951);
\draw [c] (9.06093,1.10612) -- (9.07574,1.10612);
\draw [c] (9.07574,1.10612) -- (9.09056,1.10612);
\definecolor{c}{rgb}{0,0,0};
\colorlet{c}{kugray};
\draw [c] (9.10538,0.596817) -- (9.10538,1.18963);
\draw [c] (9.10538,1.18963) -- (9.10538,1.40302);
\draw [c] (9.09056,1.18963) -- (9.10538,1.18963);
\draw [c] (9.10538,1.18963) -- (9.1202,1.18963);
\definecolor{c}{rgb}{0,0,0};
\colorlet{c}{kugray};
\draw [c] (9.13502,0.596817) -- (9.13502,1.14382);
\draw [c] (9.13502,1.14382) -- (9.13502,1.35721);
\draw [c] (9.1202,1.14382) -- (9.13502,1.14382);
\draw [c] (9.13502,1.14382) -- (9.14983,1.14382);
\definecolor{c}{rgb}{0,0,0};
\colorlet{c}{kugray};
\draw [c] (9.22392,1.0928) -- (9.22392,1.47755);
\draw [c] (9.22392,1.47755) -- (9.22392,1.64333);
\draw [c] (9.20911,1.47755) -- (9.22392,1.47755);
\draw [c] (9.22392,1.47755) -- (9.23874,1.47755);
\definecolor{c}{rgb}{0,0,0};
\colorlet{c}{kugray};
\draw [c] (9.2832,0.596817) -- (9.2832,1.19105);
\draw [c] (9.2832,1.19105) -- (9.2832,1.40444);
\draw [c] (9.26838,1.19105) -- (9.2832,1.19105);
\draw [c] (9.2832,1.19105) -- (9.29801,1.19105);
\definecolor{c}{rgb}{0,0,0};
\colorlet{c}{kugray};
\draw [c] (9.34247,0.596817) -- (9.34247,1.0722);
\draw [c] (9.34247,1.0722) -- (9.34247,1.28559);
\draw [c] (9.32765,1.0722) -- (9.34247,1.0722);
\draw [c] (9.34247,1.0722) -- (9.35728,1.0722);
\definecolor{c}{rgb}{0,0,0};
\colorlet{c}{kugray};
\draw [c] (9.54992,0.596817) -- (9.54992,1.17593);
\draw [c] (9.54992,1.17593) -- (9.54992,1.38932);
\draw [c] (9.5351,1.17593) -- (9.54992,1.17593);
\draw [c] (9.54992,1.17593) -- (9.56474,1.17593);
\definecolor{c}{rgb}{0,0,0};
\colorlet{c}{kugray};
\draw [c] (9.60919,0.596817) -- (9.60919,1.45456);
\draw [c] (9.60919,1.45456) -- (9.60919,1.66795);
\draw [c] (9.59437,1.45456) -- (9.60919,1.45456);
\draw [c] (9.60919,1.45456) -- (9.62401,1.45456);
\definecolor{c}{rgb}{0,0,0};
\colorlet{c}{kugray};
\draw [c] (9.63882,0.596817) -- (9.63882,1.11893);
\draw [c] (9.63882,1.11893) -- (9.63882,1.33232);
\draw [c] (9.62401,1.11893) -- (9.63882,1.11893);
\draw [c] (9.63882,1.11893) -- (9.65364,1.11893);
\definecolor{c}{rgb}{0,0,0};
\colorlet{c}{kugray};
\draw [c] (9.72773,0.596817) -- (9.72773,1.59979);
\draw [c] (9.72773,1.59979) -- (9.72773,1.81318);
\draw [c] (9.71291,1.59979) -- (9.72773,1.59979);
\draw [c] (9.72773,1.59979) -- (9.74255,1.59979);
\definecolor{c}{rgb}{0,0,0};
\colorlet{c}{kugray};
\draw [c] (9.787,0.596817) -- (9.787,1.11893);
\draw [c] (9.787,1.11893) -- (9.787,1.33232);
\draw [c] (9.77219,1.11893) -- (9.787,1.11893);
\draw [c] (9.787,1.11893) -- (9.80182,1.11893);
\definecolor{c}{rgb}{0,0,0};
\colorlet{c}{kugray};
\draw [c] (9.90555,0.596817) -- (9.90555,1.17321);
\draw [c] (9.90555,1.17321) -- (9.90555,1.3866);
\draw [c] (9.89073,1.17321) -- (9.90555,1.17321);
\draw [c] (9.90555,1.17321) -- (9.92036,1.17321);
\definecolor{c}{rgb}{0,0,0};
\colorlet{c}{kugray};
\draw [c] (9.93518,0.596817) -- (9.93518,1.17077);
\draw [c] (9.93518,1.17077) -- (9.93518,1.38417);
\draw [c] (9.92036,1.17077) -- (9.93518,1.17077);
\draw [c] (9.93518,1.17077) -- (9.95,1.17077);
\definecolor{c}{rgb}{0,0,0};
\colorlet{c}{natgreen};
\draw [c] (1.51655,5.54533) -- (1.60131,5.40419) -- (1.68607,5.2747) -- (1.77083,5.15478) -- (1.85558,5.04288) -- (1.94034,4.93778) -- (2.0251,4.83855) -- (2.10986,4.74443) -- (2.19462,4.65481) -- (2.27938,4.56919) -- (2.36413,4.48712)
 -- (2.44889,4.40827) -- (2.53365,4.33233) -- (2.61841,4.25902) -- (2.70317,4.18813) -- (2.78793,4.11945) -- (2.87268,4.05281) -- (2.95744,3.98806) -- (3.0422,3.92505) -- (3.12696,3.86367) -- (3.21172,3.80381) -- (3.29648,3.74536) --
 (3.38123,3.68824) -- (3.46599,3.63236) -- (3.55075,3.57765) -- (3.63551,3.52404) -- (3.72027,3.47148) -- (3.80502,3.4199) -- (3.88978,3.36925) -- (3.97454,3.31949) -- (4.0593,3.27056) -- (4.14406,3.22243) -- (4.22882,3.17506) -- (4.31357,3.12841) --
 (4.39833,3.08245) -- (4.48309,3.03714) -- (4.56785,2.99247) -- (4.65261,2.94839) -- (4.73737,2.90489) -- (4.82212,2.86194) -- (4.90688,2.81952) -- (4.99164,2.7776) -- (5.0764,2.73617) -- (5.16116,2.6952) -- (5.24592,2.65468) -- (5.33067,2.6146) --
 (5.41543,2.57492) -- (5.50019,2.53565) -- (5.58495,2.49676) -- (5.66971,2.45824);
\draw [c] (5.66971,2.45824) -- (5.75447,2.42007) -- (5.83922,2.38225) -- (5.92398,2.34476) -- (6.00874,2.30759) -- (6.0935,2.27073) -- (6.17826,2.23416) -- (6.26301,2.19789) -- (6.34777,2.16188) -- (6.43253,2.12615) --
 (6.51729,2.09068) -- (6.60205,2.05545) -- (6.68681,2.02047) -- (6.77156,1.98572) -- (6.85632,1.9512) -- (6.94108,1.91689) -- (7.02584,1.88279) -- (7.1106,1.8489) -- (7.19536,1.81521) -- (7.28011,1.78171) -- (7.36487,1.74839) -- (7.44963,1.71525) --
 (7.53439,1.68228) -- (7.61915,1.64948) -- (7.70391,1.61684) -- (7.78866,1.58435) -- (7.87342,1.55202) -- (7.95818,1.51983) -- (8.04294,1.48778) -- (8.1277,1.45586) -- (8.21246,1.42408) -- (8.29721,1.39242) -- (8.38197,1.36088) -- (8.46673,1.32946)
 -- (8.55149,1.29816) -- (8.63625,1.26696) -- (8.72101,1.23587) -- (8.80576,1.20487) -- (8.89052,1.17398) -- (8.97528,1.14317) -- (9.06004,1.11246) -- (9.1448,1.08183) -- (9.22955,1.05129) -- (9.31431,1.02082) -- (9.39907,0.990431) --
 (9.48383,0.960111) -- (9.56859,0.929859) -- (9.65335,0.899673) -- (9.7381,0.869549) -- (9.82286,0.839483);
\draw [c] (9.82286,0.839483) -- (9.90762,0.809473);
\colorlet{c}{kugray};
\draw [c] (1.07409,0.596817) -- (1.07409,1.40897);
\draw [c] (1.07409,1.40897) -- (1.07409,1.62236);
\draw [c] (1.05927,1.40897) -- (1.07409,1.40897);
\draw [c] (1.07409,1.40897) -- (1.08891,1.40897);
\definecolor{c}{rgb}{0,0,0};
\colorlet{c}{kugray};
\draw [c] (1.10373,0.596817) -- (1.10373,3.52791);
\draw [c] (1.10373,3.52791) -- (1.10373,3.7413);
\draw [c] (1.08891,3.52791) -- (1.10373,3.52791);
\draw [c] (1.10373,3.52791) -- (1.11854,3.52791);
\definecolor{c}{rgb}{0,0,0};
\colorlet{c}{kugray};
\draw [c] (1.13336,3.30584) -- (1.13336,3.68425);
\draw [c] (1.13336,3.68425) -- (1.13336,3.84896);
\draw [c] (1.11854,3.68425) -- (1.13336,3.68425);
\draw [c] (1.13336,3.68425) -- (1.14818,3.68425);
\definecolor{c}{rgb}{0,0,0};
\colorlet{c}{kugray};
\draw [c] (1.163,0.596817) -- (1.163,3.53823);
\draw [c] (1.163,3.53823) -- (1.163,3.75162);
\draw [c] (1.14818,3.53823) -- (1.163,3.53823);
\draw [c] (1.163,3.53823) -- (1.17781,3.53823);
\definecolor{c}{rgb}{0,0,0};
\colorlet{c}{kugray};
\draw [c] (1.19263,3.61695) -- (1.19263,3.89008);
\draw [c] (1.19263,3.89008) -- (1.19263,4.03249);
\draw [c] (1.17781,3.89008) -- (1.19263,3.89008);
\draw [c] (1.19263,3.89008) -- (1.20745,3.89008);
\definecolor{c}{rgb}{0,0,0};
\colorlet{c}{kugray};
\draw [c] (1.22227,3.83075) -- (1.22227,4.01677);
\draw [c] (1.22227,4.01677) -- (1.22227,4.13191);
\draw [c] (1.20745,4.01677) -- (1.22227,4.01677);
\draw [c] (1.22227,4.01677) -- (1.23709,4.01677);
\definecolor{c}{rgb}{0,0,0};
\colorlet{c}{kugray};
\draw [c] (1.2519,3.71523) -- (1.2519,3.93196);
\draw [c] (1.2519,3.93196) -- (1.2519,4.05789);
\draw [c] (1.23709,3.93196) -- (1.2519,3.93196);
\draw [c] (1.2519,3.93196) -- (1.26672,3.93196);
\definecolor{c}{rgb}{0,0,0};
\colorlet{c}{kugray};
\draw [c] (1.28154,3.75372) -- (1.28154,3.97484);
\draw [c] (1.28154,3.97484) -- (1.28154,4.1022);
\draw [c] (1.26672,3.97484) -- (1.28154,3.97484);
\draw [c] (1.28154,3.97484) -- (1.29636,3.97484);
\definecolor{c}{rgb}{0,0,0};
\colorlet{c}{kugray};
\draw [c] (1.31118,5.36683) -- (1.31118,5.38443);
\draw [c] (1.31118,5.38443) -- (1.31118,5.40108);
\draw [c] (1.29636,5.38443) -- (1.31118,5.38443);
\draw [c] (1.31118,5.38443) -- (1.32599,5.38443);
\definecolor{c}{rgb}{0,0,0};
\colorlet{c}{kugray};
\draw [c] (1.34081,5.64934) -- (1.34081,5.66037);
\draw [c] (1.34081,5.66037) -- (1.34081,5.67102);
\draw [c] (1.32599,5.66037) -- (1.34081,5.66037);
\draw [c] (1.34081,5.66037) -- (1.35563,5.66037);
\definecolor{c}{rgb}{0,0,0};
\colorlet{c}{kugray};
\draw [c] (1.37045,5.69165) -- (1.37045,5.70189);
\draw [c] (1.37045,5.70189) -- (1.37045,5.7118);
\draw [c] (1.35563,5.70189) -- (1.37045,5.70189);
\draw [c] (1.37045,5.70189) -- (1.38526,5.70189);
\definecolor{c}{rgb}{0,0,0};
\colorlet{c}{kugray};
\draw [c] (1.40008,5.69477) -- (1.40008,5.70516);
\draw [c] (1.40008,5.70516) -- (1.40008,5.71522);
\draw [c] (1.38526,5.70516) -- (1.40008,5.70516);
\draw [c] (1.40008,5.70516) -- (1.4149,5.70516);
\definecolor{c}{rgb}{0,0,0};
\colorlet{c}{kugray};
\draw [c] (1.42972,5.6476) -- (1.42972,5.65871);
\draw [c] (1.42972,5.65871) -- (1.42972,5.66943);
\draw [c] (1.4149,5.65871) -- (1.42972,5.65871);
\draw [c] (1.42972,5.65871) -- (1.44454,5.65871);
\definecolor{c}{rgb}{0,0,0};
\colorlet{c}{kugray};
\draw [c] (1.45935,5.62032) -- (1.45935,5.63176);
\draw [c] (1.45935,5.63176) -- (1.45935,5.6428);
\draw [c] (1.44454,5.63176) -- (1.45935,5.63176);
\draw [c] (1.45935,5.63176) -- (1.47417,5.63176);
\definecolor{c}{rgb}{0,0,0};
\colorlet{c}{kugray};
\draw [c] (1.48899,5.56116) -- (1.48899,5.57382);
\draw [c] (1.48899,5.57382) -- (1.48899,5.58597);
\draw [c] (1.47417,5.57382) -- (1.48899,5.57382);
\draw [c] (1.48899,5.57382) -- (1.50381,5.57382);
\definecolor{c}{rgb}{0,0,0};
\colorlet{c}{kugray};
\draw [c] (1.51863,5.51885) -- (1.51863,5.53274);
\draw [c] (1.51863,5.53274) -- (1.51863,5.54604);
\draw [c] (1.50381,5.53274) -- (1.51863,5.53274);
\draw [c] (1.51863,5.53274) -- (1.53344,5.53274);
\definecolor{c}{rgb}{0,0,0};
\colorlet{c}{kugray};
\draw [c] (1.54826,5.46762) -- (1.54826,5.48249);
\draw [c] (1.54826,5.48249) -- (1.54826,5.49667);
\draw [c] (1.53344,5.48249) -- (1.54826,5.48249);
\draw [c] (1.54826,5.48249) -- (1.56308,5.48249);
\definecolor{c}{rgb}{0,0,0};
\colorlet{c}{kugray};
\draw [c] (1.5779,5.40546) -- (1.5779,5.42176);
\draw [c] (1.5779,5.42176) -- (1.5779,5.43724);
\draw [c] (1.56308,5.42176) -- (1.5779,5.42176);
\draw [c] (1.5779,5.42176) -- (1.59272,5.42176);
\definecolor{c}{rgb}{0,0,0};
\colorlet{c}{kugray};
\draw [c] (1.60753,5.41122) -- (1.60753,5.42791);
\draw [c] (1.60753,5.42791) -- (1.60753,5.44375);
\draw [c] (1.59272,5.42791) -- (1.60753,5.42791);
\draw [c] (1.60753,5.42791) -- (1.62235,5.42791);
\definecolor{c}{rgb}{0,0,0};
\colorlet{c}{kugray};
\draw [c] (1.63717,5.33643) -- (1.63717,5.35499);
\draw [c] (1.63717,5.35499) -- (1.63717,5.37248);
\draw [c] (1.62235,5.35499) -- (1.63717,5.35499);
\draw [c] (1.63717,5.35499) -- (1.65199,5.35499);
\definecolor{c}{rgb}{0,0,0};
\colorlet{c}{kugray};
\draw [c] (1.6668,5.32522) -- (1.6668,5.34431);
\draw [c] (1.6668,5.34431) -- (1.6668,5.36228);
\draw [c] (1.65199,5.34431) -- (1.6668,5.34431);
\draw [c] (1.6668,5.34431) -- (1.68162,5.34431);
\definecolor{c}{rgb}{0,0,0};
\colorlet{c}{kugray};
\draw [c] (1.69644,5.24802) -- (1.69644,5.26862);
\draw [c] (1.69644,5.26862) -- (1.69644,5.28793);
\draw [c] (1.68162,5.26862) -- (1.69644,5.26862);
\draw [c] (1.69644,5.26862) -- (1.71126,5.26862);
\definecolor{c}{rgb}{0,0,0};
\colorlet{c}{kugray};
\draw [c] (1.72608,5.18795) -- (1.72608,5.21178);
\draw [c] (1.72608,5.21178) -- (1.72608,5.23389);
\draw [c] (1.71126,5.21178) -- (1.72608,5.21178);
\draw [c] (1.72608,5.21178) -- (1.74089,5.21178);
\definecolor{c}{rgb}{0,0,0};
\colorlet{c}{kugray};
\draw [c] (1.75571,5.164) -- (1.75571,5.18723);
\draw [c] (1.75571,5.18723) -- (1.75571,5.20882);
\draw [c] (1.74089,5.18723) -- (1.75571,5.18723);
\draw [c] (1.75571,5.18723) -- (1.77053,5.18723);
\definecolor{c}{rgb}{0,0,0};
\colorlet{c}{kugray};
\draw [c] (1.78535,5.12784) -- (1.78535,5.15344);
\draw [c] (1.78535,5.15344) -- (1.78535,5.17707);
\draw [c] (1.77053,5.15344) -- (1.78535,5.15344);
\draw [c] (1.78535,5.15344) -- (1.80017,5.15344);
\definecolor{c}{rgb}{0,0,0};
\colorlet{c}{kugray};
\draw [c] (1.81498,5.07595) -- (1.81498,5.10297);
\draw [c] (1.81498,5.10297) -- (1.81498,5.1278);
\draw [c] (1.80017,5.10297) -- (1.81498,5.10297);
\draw [c] (1.81498,5.10297) -- (1.8298,5.10297);
\definecolor{c}{rgb}{0,0,0};
\colorlet{c}{kugray};
\draw [c] (1.84462,5.05921) -- (1.84462,5.0876);
\draw [c] (1.84462,5.0876) -- (1.84462,5.11359);
\draw [c] (1.8298,5.0876) -- (1.84462,5.0876);
\draw [c] (1.84462,5.0876) -- (1.85944,5.0876);
\definecolor{c}{rgb}{0,0,0};
\colorlet{c}{kugray};
\draw [c] (1.87425,4.98897) -- (1.87425,5.02142);
\draw [c] (1.87425,5.02142) -- (1.87425,5.05078);
\draw [c] (1.85944,5.02142) -- (1.87425,5.02142);
\draw [c] (1.87425,5.02142) -- (1.88907,5.02142);
\definecolor{c}{rgb}{0,0,0};
\colorlet{c}{kugray};
\draw [c] (1.90389,4.87515) -- (1.90389,4.914);
\draw [c] (1.90389,4.914) -- (1.90389,4.94849);
\draw [c] (1.88907,4.914) -- (1.90389,4.914);
\draw [c] (1.90389,4.914) -- (1.91871,4.914);
\definecolor{c}{rgb}{0,0,0};
\colorlet{c}{kugray};
\draw [c] (1.93353,4.89763) -- (1.93353,4.9339);
\draw [c] (1.93353,4.9339) -- (1.93353,4.96634);
\draw [c] (1.91871,4.9339) -- (1.93353,4.9339);
\draw [c] (1.93353,4.9339) -- (1.94834,4.9339);
\definecolor{c}{rgb}{0,0,0};
\colorlet{c}{kugray};
\draw [c] (1.96316,4.88868) -- (1.96316,4.92763);
\draw [c] (1.96316,4.92763) -- (1.96316,4.96221);
\draw [c] (1.94834,4.92763) -- (1.96316,4.92763);
\draw [c] (1.96316,4.92763) -- (1.97798,4.92763);
\definecolor{c}{rgb}{0,0,0};
\colorlet{c}{kugray};
\draw [c] (1.9928,4.78948) -- (1.9928,4.83208);
\draw [c] (1.9928,4.83208) -- (1.9928,4.8695);
\draw [c] (1.97798,4.83208) -- (1.9928,4.83208);
\draw [c] (1.9928,4.83208) -- (2.00762,4.83208);
\definecolor{c}{rgb}{0,0,0};
\colorlet{c}{kugray};
\draw [c] (2.02243,4.71758) -- (2.02243,4.76449);
\draw [c] (2.02243,4.76449) -- (2.02243,4.80519);
\draw [c] (2.00762,4.76449) -- (2.02243,4.76449);
\draw [c] (2.02243,4.76449) -- (2.03725,4.76449);
\definecolor{c}{rgb}{0,0,0};
\colorlet{c}{kugray};
\draw [c] (2.05207,4.70827) -- (2.05207,4.75577);
\draw [c] (2.05207,4.75577) -- (2.05207,4.79692);
\draw [c] (2.03725,4.75577) -- (2.05207,4.75577);
\draw [c] (2.05207,4.75577) -- (2.06689,4.75577);
\definecolor{c}{rgb}{0,0,0};
\colorlet{c}{kugray};
\draw [c] (2.08171,4.68783) -- (2.08171,4.73949);
\draw [c] (2.08171,4.73949) -- (2.08171,4.78371);
\draw [c] (2.06689,4.73949) -- (2.08171,4.73949);
\draw [c] (2.08171,4.73949) -- (2.09652,4.73949);
\definecolor{c}{rgb}{0,0,0};
\colorlet{c}{kugray};
\draw [c] (2.11134,4.66301) -- (2.11134,4.71819);
\draw [c] (2.11134,4.71819) -- (2.11134,4.76496);
\draw [c] (2.09652,4.71819) -- (2.11134,4.71819);
\draw [c] (2.11134,4.71819) -- (2.12616,4.71819);
\definecolor{c}{rgb}{0,0,0};
\colorlet{c}{kugray};
\draw [c] (2.14098,4.66355) -- (2.14098,4.71683);
\draw [c] (2.14098,4.71683) -- (2.14098,4.76224);
\draw [c] (2.12616,4.71683) -- (2.14098,4.71683);
\draw [c] (2.14098,4.71683) -- (2.15579,4.71683);
\definecolor{c}{rgb}{0,0,0};
\colorlet{c}{kugray};
\draw [c] (2.17061,4.50979) -- (2.17061,4.57305);
\draw [c] (2.17061,4.57305) -- (2.17061,4.6255);
\draw [c] (2.15579,4.57305) -- (2.17061,4.57305);
\draw [c] (2.17061,4.57305) -- (2.18543,4.57305);
\definecolor{c}{rgb}{0,0,0};
\colorlet{c}{kugray};
\draw [c] (2.20025,4.5238) -- (2.20025,4.5936);
\draw [c] (2.20025,4.5936) -- (2.20025,4.65046);
\draw [c] (2.18543,4.5936) -- (2.20025,4.5936);
\draw [c] (2.20025,4.5936) -- (2.21507,4.5936);
\definecolor{c}{rgb}{0,0,0};
\colorlet{c}{kugray};
\draw [c] (2.22988,4.49841) -- (2.22988,4.56586);
\draw [c] (2.22988,4.56586) -- (2.22988,4.62114);
\draw [c] (2.21507,4.56586) -- (2.22988,4.56586);
\draw [c] (2.22988,4.56586) -- (2.2447,4.56586);
\definecolor{c}{rgb}{0,0,0};
\colorlet{c}{kugray};
\draw [c] (2.25952,4.51369) -- (2.25952,4.58591);
\draw [c] (2.25952,4.58591) -- (2.25952,4.64436);
\draw [c] (2.2447,4.58591) -- (2.25952,4.58591);
\draw [c] (2.25952,4.58591) -- (2.27434,4.58591);
\definecolor{c}{rgb}{0,0,0};
\colorlet{c}{kugray};
\draw [c] (2.28916,4.55805) -- (2.28916,4.56892);
\draw [c] (2.28916,4.56892) -- (2.28916,4.57942);
\draw [c] (2.27434,4.56892) -- (2.28916,4.56892);
\draw [c] (2.28916,4.56892) -- (2.30397,4.56892);
\definecolor{c}{rgb}{0,0,0};
\colorlet{c}{kugray};
\draw [c] (2.31879,4.50775) -- (2.31879,4.51917);
\draw [c] (2.31879,4.51917) -- (2.31879,4.53019);
\draw [c] (2.30397,4.51917) -- (2.31879,4.51917);
\draw [c] (2.31879,4.51917) -- (2.33361,4.51917);
\definecolor{c}{rgb}{0,0,0};
\colorlet{c}{kugray};
\draw [c] (2.34843,4.49289) -- (2.34843,4.50481);
\draw [c] (2.34843,4.50481) -- (2.34843,4.5163);
\draw [c] (2.33361,4.50481) -- (2.34843,4.50481);
\draw [c] (2.34843,4.50481) -- (2.36325,4.50481);
\definecolor{c}{rgb}{0,0,0};
\colorlet{c}{kugray};
\draw [c] (2.37806,4.46995) -- (2.37806,4.48228);
\draw [c] (2.37806,4.48228) -- (2.37806,4.49413);
\draw [c] (2.36325,4.48228) -- (2.37806,4.48228);
\draw [c] (2.37806,4.48228) -- (2.39288,4.48228);
\definecolor{c}{rgb}{0,0,0};
\colorlet{c}{kugray};
\draw [c] (2.4077,4.4553) -- (2.4077,4.46803);
\draw [c] (2.4077,4.46803) -- (2.4077,4.48025);
\draw [c] (2.39288,4.46803) -- (2.4077,4.46803);
\draw [c] (2.4077,4.46803) -- (2.42252,4.46803);
\definecolor{c}{rgb}{0,0,0};
\colorlet{c}{kugray};
\draw [c] (2.43733,4.4243) -- (2.43733,4.43793);
\draw [c] (2.43733,4.43793) -- (2.43733,4.45097);
\draw [c] (2.42252,4.43793) -- (2.43733,4.43793);
\draw [c] (2.43733,4.43793) -- (2.45215,4.43793);
\definecolor{c}{rgb}{0,0,0};
\colorlet{c}{kugray};
\draw [c] (2.46697,4.40165) -- (2.46697,4.41557);
\draw [c] (2.46697,4.41557) -- (2.46697,4.42889);
\draw [c] (2.45215,4.41557) -- (2.46697,4.41557);
\draw [c] (2.46697,4.41557) -- (2.48179,4.41557);
\definecolor{c}{rgb}{0,0,0};
\colorlet{c}{kugray};
\draw [c] (2.49661,4.3514) -- (2.49661,4.36606);
\draw [c] (2.49661,4.36606) -- (2.49661,4.38005);
\draw [c] (2.48179,4.36606) -- (2.49661,4.36606);
\draw [c] (2.49661,4.36606) -- (2.51142,4.36606);
\definecolor{c}{rgb}{0,0,0};
\colorlet{c}{kugray};
\draw [c] (2.52624,4.33895) -- (2.52624,4.3543);
\draw [c] (2.52624,4.3543) -- (2.52624,4.36892);
\draw [c] (2.51142,4.3543) -- (2.52624,4.3543);
\draw [c] (2.52624,4.3543) -- (2.54106,4.3543);
\definecolor{c}{rgb}{0,0,0};
\colorlet{c}{kugray};
\draw [c] (2.55588,4.30184) -- (2.55588,4.31852);
\draw [c] (2.55588,4.31852) -- (2.55588,4.33435);
\draw [c] (2.54106,4.31852) -- (2.55588,4.31852);
\draw [c] (2.55588,4.31852) -- (2.5707,4.31852);
\definecolor{c}{rgb}{0,0,0};
\colorlet{c}{kugray};
\draw [c] (2.58551,4.29419) -- (2.58551,4.31073);
\draw [c] (2.58551,4.31073) -- (2.58551,4.32643);
\draw [c] (2.5707,4.31073) -- (2.58551,4.31073);
\draw [c] (2.58551,4.31073) -- (2.60033,4.31073);
\definecolor{c}{rgb}{0,0,0};
\colorlet{c}{kugray};
\draw [c] (2.61515,4.23639) -- (2.61515,4.25491);
\draw [c] (2.61515,4.25491) -- (2.61515,4.27238);
\draw [c] (2.60033,4.25491) -- (2.61515,4.25491);
\draw [c] (2.61515,4.25491) -- (2.62997,4.25491);
\definecolor{c}{rgb}{0,0,0};
\colorlet{c}{kugray};
\draw [c] (2.64478,4.2342) -- (2.64478,4.2521);
\draw [c] (2.64478,4.2521) -- (2.64478,4.26903);
\draw [c] (2.62997,4.2521) -- (2.64478,4.2521);
\draw [c] (2.64478,4.2521) -- (2.6596,4.2521);
\definecolor{c}{rgb}{0,0,0};
\colorlet{c}{kugray};
\draw [c] (2.67442,4.19709) -- (2.67442,4.21658);
\draw [c] (2.67442,4.21658) -- (2.67442,4.23491);
\draw [c] (2.6596,4.21658) -- (2.67442,4.21658);
\draw [c] (2.67442,4.21658) -- (2.68924,4.21658);
\definecolor{c}{rgb}{0,0,0};
\colorlet{c}{kugray};
\draw [c] (2.70406,4.15645) -- (2.70406,4.17742);
\draw [c] (2.70406,4.17742) -- (2.70406,4.19706);
\draw [c] (2.68924,4.17742) -- (2.70406,4.17742);
\draw [c] (2.70406,4.17742) -- (2.71887,4.17742);
\definecolor{c}{rgb}{0,0,0};
\colorlet{c}{kugray};
\draw [c] (2.73369,4.10839) -- (2.73369,4.13042);
\draw [c] (2.73369,4.13042) -- (2.73369,4.15097);
\draw [c] (2.71887,4.13042) -- (2.73369,4.13042);
\draw [c] (2.73369,4.13042) -- (2.74851,4.13042);
\definecolor{c}{rgb}{0,0,0};
\colorlet{c}{kugray};
\draw [c] (2.76333,4.13783) -- (2.76333,4.15915);
\draw [c] (2.76333,4.15915) -- (2.76333,4.1791);
\draw [c] (2.74851,4.15915) -- (2.76333,4.15915);
\draw [c] (2.76333,4.15915) -- (2.77815,4.15915);
\definecolor{c}{rgb}{0,0,0};
\colorlet{c}{kugray};
\draw [c] (2.79296,4.09048) -- (2.79296,4.11309);
\draw [c] (2.79296,4.11309) -- (2.79296,4.13416);
\draw [c] (2.77815,4.11309) -- (2.79296,4.11309);
\draw [c] (2.79296,4.11309) -- (2.80778,4.11309);
\definecolor{c}{rgb}{0,0,0};
\colorlet{c}{kugray};
\draw [c] (2.8226,4.09688) -- (2.8226,4.11923);
\draw [c] (2.8226,4.11923) -- (2.8226,4.14007);
\draw [c] (2.80778,4.11923) -- (2.8226,4.11923);
\draw [c] (2.8226,4.11923) -- (2.83742,4.11923);
\definecolor{c}{rgb}{0,0,0};
\colorlet{c}{kugray};
\draw [c] (2.85224,4.00498) -- (2.85224,4.03051);
\draw [c] (2.85224,4.03051) -- (2.85224,4.05409);
\draw [c] (2.83742,4.03051) -- (2.85224,4.03051);
\draw [c] (2.85224,4.03051) -- (2.86705,4.03051);
\definecolor{c}{rgb}{0,0,0};
\colorlet{c}{kugray};
\draw [c] (2.88187,4.03101) -- (2.88187,4.05663);
\draw [c] (2.88187,4.05663) -- (2.88187,4.08028);
\draw [c] (2.86705,4.05663) -- (2.88187,4.05663);
\draw [c] (2.88187,4.05663) -- (2.89669,4.05663);
\definecolor{c}{rgb}{0,0,0};
\colorlet{c}{kugray};
\draw [c] (2.91151,4.01624) -- (2.91151,4.04198);
\draw [c] (2.91151,4.04198) -- (2.91151,4.06572);
\draw [c] (2.89669,4.04198) -- (2.91151,4.04198);
\draw [c] (2.91151,4.04198) -- (2.92632,4.04198);
\definecolor{c}{rgb}{0,0,0};
\colorlet{c}{kugray};
\draw [c] (2.94114,4.00464) -- (2.94114,4.03106);
\draw [c] (2.94114,4.03106) -- (2.94114,4.0554);
\draw [c] (2.92632,4.03106) -- (2.94114,4.03106);
\draw [c] (2.94114,4.03106) -- (2.95596,4.03106);
\definecolor{c}{rgb}{0,0,0};
\colorlet{c}{kugray};
\draw [c] (2.97078,3.992) -- (2.97078,4.01847);
\draw [c] (2.97078,4.01847) -- (2.97078,4.04285);
\draw [c] (2.95596,4.01847) -- (2.97078,4.01847);
\draw [c] (2.97078,4.01847) -- (2.9856,4.01847);
\definecolor{c}{rgb}{0,0,0};
\colorlet{c}{kugray};
\draw [c] (3.00041,3.95995) -- (3.00041,3.9878);
\draw [c] (3.00041,3.9878) -- (3.00041,4.01334);
\draw [c] (2.9856,3.9878) -- (3.00041,3.9878);
\draw [c] (3.00041,3.9878) -- (3.01523,3.9878);
\definecolor{c}{rgb}{0,0,0};
\colorlet{c}{kugray};
\draw [c] (3.03005,3.92381) -- (3.03005,3.95391);
\draw [c] (3.03005,3.95391) -- (3.03005,3.98133);
\draw [c] (3.01523,3.95391) -- (3.03005,3.95391);
\draw [c] (3.03005,3.95391) -- (3.04487,3.95391);
\definecolor{c}{rgb}{0,0,0};
\colorlet{c}{kugray};
\draw [c] (3.05969,3.92723) -- (3.05969,3.9572);
\draw [c] (3.05969,3.9572) -- (3.05969,3.9845);
\draw [c] (3.04487,3.9572) -- (3.05969,3.9572);
\draw [c] (3.05969,3.9572) -- (3.0745,3.9572);
\definecolor{c}{rgb}{0,0,0};
\colorlet{c}{kugray};
\draw [c] (3.08932,3.89686) -- (3.08932,3.92737);
\draw [c] (3.08932,3.92737) -- (3.08932,3.95512);
\draw [c] (3.0745,3.92737) -- (3.08932,3.92737);
\draw [c] (3.08932,3.92737) -- (3.10414,3.92737);
\definecolor{c}{rgb}{0,0,0};
\colorlet{c}{kugray};
\draw [c] (3.11896,3.87075) -- (3.11896,3.9025);
\draw [c] (3.11896,3.9025) -- (3.11896,3.93128);
\draw [c] (3.10414,3.9025) -- (3.11896,3.9025);
\draw [c] (3.11896,3.9025) -- (3.13377,3.9025);
\definecolor{c}{rgb}{0,0,0};
\colorlet{c}{kugray};
\draw [c] (3.14859,3.84537) -- (3.14859,3.88076);
\draw [c] (3.14859,3.88076) -- (3.14859,3.9125);
\draw [c] (3.13377,3.88076) -- (3.14859,3.88076);
\draw [c] (3.14859,3.88076) -- (3.16341,3.88076);
\definecolor{c}{rgb}{0,0,0};
\colorlet{c}{kugray};
\draw [c] (3.17823,3.8829) -- (3.17823,3.91571);
\draw [c] (3.17823,3.91571) -- (3.17823,3.94536);
\draw [c] (3.16341,3.91571) -- (3.17823,3.91571);
\draw [c] (3.17823,3.91571) -- (3.19305,3.91571);
\definecolor{c}{rgb}{0,0,0};
\colorlet{c}{kugray};
\draw [c] (3.20786,3.84639) -- (3.20786,3.8804);
\draw [c] (3.20786,3.8804) -- (3.20786,3.91102);
\draw [c] (3.19305,3.8804) -- (3.20786,3.8804);
\draw [c] (3.20786,3.8804) -- (3.22268,3.8804);
\definecolor{c}{rgb}{0,0,0};
\colorlet{c}{kugray};
\draw [c] (3.2375,3.74035) -- (3.2375,3.78287);
\draw [c] (3.2375,3.78287) -- (3.2375,3.82022);
\draw [c] (3.22268,3.78287) -- (3.2375,3.78287);
\draw [c] (3.2375,3.78287) -- (3.25232,3.78287);
\definecolor{c}{rgb}{0,0,0};
\colorlet{c}{kugray};
\draw [c] (3.26714,3.68918) -- (3.26714,3.73432);
\draw [c] (3.26714,3.73432) -- (3.26714,3.77368);
\draw [c] (3.25232,3.73432) -- (3.26714,3.73432);
\draw [c] (3.26714,3.73432) -- (3.28195,3.73432);
\definecolor{c}{rgb}{0,0,0};
\colorlet{c}{kugray};
\draw [c] (3.29677,3.79443) -- (3.29677,3.83114);
\draw [c] (3.29677,3.83114) -- (3.29677,3.86393);
\draw [c] (3.28195,3.83114) -- (3.29677,3.83114);
\draw [c] (3.29677,3.83114) -- (3.31159,3.83114);
\definecolor{c}{rgb}{0,0,0};
\colorlet{c}{kugray};
\draw [c] (3.32641,3.68348) -- (3.32641,3.72745);
\draw [c] (3.32641,3.72745) -- (3.32641,3.76591);
\draw [c] (3.31159,3.72745) -- (3.32641,3.72745);
\draw [c] (3.32641,3.72745) -- (3.34123,3.72745);
\definecolor{c}{rgb}{0,0,0};
\colorlet{c}{kugray};
\draw [c] (3.35604,3.6893) -- (3.35604,3.73394);
\draw [c] (3.35604,3.73394) -- (3.35604,3.77293);
\draw [c] (3.34123,3.73394) -- (3.35604,3.73394);
\draw [c] (3.35604,3.73394) -- (3.37086,3.73394);
\definecolor{c}{rgb}{0,0,0};
\colorlet{c}{kugray};
\draw [c] (3.38568,3.69206) -- (3.38568,3.73661);
\draw [c] (3.38568,3.73661) -- (3.38568,3.77553);
\draw [c] (3.37086,3.73661) -- (3.38568,3.73661);
\draw [c] (3.38568,3.73661) -- (3.4005,3.73661);
\definecolor{c}{rgb}{0,0,0};
\colorlet{c}{kugray};
\draw [c] (3.41531,3.63598) -- (3.41531,3.68186);
\draw [c] (3.41531,3.68186) -- (3.41531,3.72179);
\draw [c] (3.4005,3.68186) -- (3.41531,3.68186);
\draw [c] (3.41531,3.68186) -- (3.43013,3.68186);
\definecolor{c}{rgb}{0,0,0};
\colorlet{c}{kugray};
\draw [c] (3.44495,3.6432) -- (3.44495,3.68882);
\draw [c] (3.44495,3.68882) -- (3.44495,3.72854);
\draw [c] (3.43013,3.68882) -- (3.44495,3.68882);
\draw [c] (3.44495,3.68882) -- (3.45977,3.68882);
\definecolor{c}{rgb}{0,0,0};
\colorlet{c}{kugray};
\draw [c] (3.47459,3.70732) -- (3.47459,3.7514);
\draw [c] (3.47459,3.7514) -- (3.47459,3.78995);
\draw [c] (3.45977,3.7514) -- (3.47459,3.7514);
\draw [c] (3.47459,3.7514) -- (3.4894,3.7514);
\definecolor{c}{rgb}{0,0,0};
\colorlet{c}{kugray};
\draw [c] (3.50422,3.66603) -- (3.50422,3.71303);
\draw [c] (3.50422,3.71303) -- (3.50422,3.7538);
\draw [c] (3.4894,3.71303) -- (3.50422,3.71303);
\draw [c] (3.50422,3.71303) -- (3.51904,3.71303);
\definecolor{c}{rgb}{0,0,0};
\colorlet{c}{kugray};
\draw [c] (3.53386,3.56775) -- (3.53386,3.61999);
\draw [c] (3.53386,3.61999) -- (3.53386,3.66463);
\draw [c] (3.51904,3.61999) -- (3.53386,3.61999);
\draw [c] (3.53386,3.61999) -- (3.54868,3.61999);
\definecolor{c}{rgb}{0,0,0};
\colorlet{c}{kugray};
\draw [c] (3.56349,3.5261) -- (3.56349,3.58634);
\draw [c] (3.56349,3.58634) -- (3.56349,3.6367);
\draw [c] (3.54868,3.58634) -- (3.56349,3.58634);
\draw [c] (3.56349,3.58634) -- (3.57831,3.58634);
\definecolor{c}{rgb}{0,0,0};
\colorlet{c}{kugray};
\draw [c] (3.59313,3.53612) -- (3.59313,3.59351);
\draw [c] (3.59313,3.59351) -- (3.59313,3.64187);
\draw [c] (3.57831,3.59351) -- (3.59313,3.59351);
\draw [c] (3.59313,3.59351) -- (3.60795,3.59351);
\definecolor{c}{rgb}{0,0,0};
\colorlet{c}{kugray};
\draw [c] (3.62276,3.60477) -- (3.62276,3.65738);
\draw [c] (3.62276,3.65738) -- (3.62276,3.70229);
\draw [c] (3.60795,3.65738) -- (3.62276,3.65738);
\draw [c] (3.62276,3.65738) -- (3.63758,3.65738);
\definecolor{c}{rgb}{0,0,0};
\colorlet{c}{kugray};
\draw [c] (3.6524,3.56534) -- (3.6524,3.6218);
\draw [c] (3.6524,3.6218) -- (3.6524,3.66949);
\draw [c] (3.63758,3.6218) -- (3.6524,3.6218);
\draw [c] (3.6524,3.6218) -- (3.66722,3.6218);
\definecolor{c}{rgb}{0,0,0};
\colorlet{c}{kugray};
\draw [c] (3.68204,3.48401) -- (3.68204,3.54474);
\draw [c] (3.68204,3.54474) -- (3.68204,3.59544);
\draw [c] (3.66722,3.54474) -- (3.68204,3.54474);
\draw [c] (3.68204,3.54474) -- (3.69685,3.54474);
\definecolor{c}{rgb}{0,0,0};
\colorlet{c}{kugray};
\draw [c] (3.71167,3.47019) -- (3.71167,3.52933);
\draw [c] (3.71167,3.52933) -- (3.71167,3.57892);
\draw [c] (3.69685,3.52933) -- (3.71167,3.52933);
\draw [c] (3.71167,3.52933) -- (3.72649,3.52933);
\definecolor{c}{rgb}{0,0,0};
\colorlet{c}{kugray};
\draw [c] (3.74131,3.47175) -- (3.74131,3.53495);
\draw [c] (3.74131,3.53495) -- (3.74131,3.58736);
\draw [c] (3.72649,3.53495) -- (3.74131,3.53495);
\draw [c] (3.74131,3.53495) -- (3.75613,3.53495);
\definecolor{c}{rgb}{0,0,0};
\colorlet{c}{kugray};
\draw [c] (3.77094,3.33813) -- (3.77094,3.41569);
\draw [c] (3.77094,3.41569) -- (3.77094,3.47758);
\draw [c] (3.75613,3.41569) -- (3.77094,3.41569);
\draw [c] (3.77094,3.41569) -- (3.78576,3.41569);
\definecolor{c}{rgb}{0,0,0};
\colorlet{c}{kugray};
\draw [c] (3.80058,3.43206) -- (3.80058,3.5008);
\draw [c] (3.80058,3.5008) -- (3.80058,3.55696);
\draw [c] (3.78576,3.5008) -- (3.80058,3.5008);
\draw [c] (3.80058,3.5008) -- (3.8154,3.5008);
\definecolor{c}{rgb}{0,0,0};
\colorlet{c}{kugray};
\draw [c] (3.83022,3.33768) -- (3.83022,3.41833);
\draw [c] (3.83022,3.41833) -- (3.83022,3.48217);
\draw [c] (3.8154,3.41833) -- (3.83022,3.41833);
\draw [c] (3.83022,3.41833) -- (3.84503,3.41833);
\definecolor{c}{rgb}{0,0,0};
\colorlet{c}{kugray};
\draw [c] (3.85985,3.50616) -- (3.85985,3.56185);
\draw [c] (3.85985,3.56185) -- (3.85985,3.60899);
\draw [c] (3.84503,3.56185) -- (3.85985,3.56185);
\draw [c] (3.85985,3.56185) -- (3.87467,3.56185);
\definecolor{c}{rgb}{0,0,0};
\colorlet{c}{kugray};
\draw [c] (3.88949,3.33314) -- (3.88949,3.40343);
\draw [c] (3.88949,3.40343) -- (3.88949,3.46061);
\draw [c] (3.87467,3.40343) -- (3.88949,3.40343);
\draw [c] (3.88949,3.40343) -- (3.9043,3.40343);
\definecolor{c}{rgb}{0,0,0};
\colorlet{c}{kugray};
\draw [c] (3.91912,3.32482) -- (3.91912,3.38647);
\draw [c] (3.91912,3.38647) -- (3.91912,3.43781);
\draw [c] (3.9043,3.38647) -- (3.91912,3.38647);
\draw [c] (3.91912,3.38647) -- (3.93394,3.38647);
\definecolor{c}{rgb}{0,0,0};
\colorlet{c}{kugray};
\draw [c] (3.94876,3.35526) -- (3.94876,3.41219);
\draw [c] (3.94876,3.41219) -- (3.94876,3.46022);
\draw [c] (3.93394,3.41219) -- (3.94876,3.41219);
\draw [c] (3.94876,3.41219) -- (3.96358,3.41219);
\definecolor{c}{rgb}{0,0,0};
\colorlet{c}{kugray};
\draw [c] (3.97839,3.3937) -- (3.97839,3.43987);
\draw [c] (3.97839,3.43987) -- (3.97839,3.48);
\draw [c] (3.96358,3.43987) -- (3.97839,3.43987);
\draw [c] (3.97839,3.43987) -- (3.99321,3.43987);
\definecolor{c}{rgb}{0,0,0};
\colorlet{c}{kugray};
\draw [c] (4.00803,3.39966) -- (4.00803,3.4339);
\draw [c] (4.00803,3.4339) -- (4.00803,3.4647);
\draw [c] (3.99321,3.4339) -- (4.00803,3.4339);
\draw [c] (4.00803,3.4339) -- (4.02285,3.4339);
\definecolor{c}{rgb}{0,0,0};
\colorlet{c}{kugray};
\draw [c] (4.03767,3.3675) -- (4.03767,3.38466);
\draw [c] (4.03767,3.38466) -- (4.03767,3.4009);
\draw [c] (4.02285,3.38466) -- (4.03767,3.38466);
\draw [c] (4.03767,3.38466) -- (4.05248,3.38466);
\definecolor{c}{rgb}{0,0,0};
\colorlet{c}{kugray};
\draw [c] (4.0673,3.36697) -- (4.0673,3.3818);
\draw [c] (4.0673,3.3818) -- (4.0673,3.39594);
\draw [c] (4.05248,3.3818) -- (4.0673,3.3818);
\draw [c] (4.0673,3.3818) -- (4.08212,3.3818);
\definecolor{c}{rgb}{0,0,0};
\colorlet{c}{kugray};
\draw [c] (4.09694,3.34139) -- (4.09694,3.35685);
\draw [c] (4.09694,3.35685) -- (4.09694,3.37158);
\draw [c] (4.08212,3.35685) -- (4.09694,3.35685);
\draw [c] (4.09694,3.35685) -- (4.11175,3.35685);
\definecolor{c}{rgb}{0,0,0};
\colorlet{c}{kugray};
\draw [c] (4.12657,3.34259) -- (4.12657,3.35775);
\draw [c] (4.12657,3.35775) -- (4.12657,3.37219);
\draw [c] (4.11175,3.35775) -- (4.12657,3.35775);
\draw [c] (4.12657,3.35775) -- (4.14139,3.35775);
\definecolor{c}{rgb}{0,0,0};
\colorlet{c}{kugray};
\draw [c] (4.15621,3.33872) -- (4.15621,3.35421);
\draw [c] (4.15621,3.35421) -- (4.15621,3.36896);
\draw [c] (4.14139,3.35421) -- (4.15621,3.35421);
\draw [c] (4.15621,3.35421) -- (4.17103,3.35421);
\definecolor{c}{rgb}{0,0,0};
\colorlet{c}{kugray};
\draw [c] (4.18584,3.34948) -- (4.18584,3.36456);
\draw [c] (4.18584,3.36456) -- (4.18584,3.37895);
\draw [c] (4.17103,3.36456) -- (4.18584,3.36456);
\draw [c] (4.18584,3.36456) -- (4.20066,3.36456);
\definecolor{c}{rgb}{0,0,0};
\colorlet{c}{kugray};
\draw [c] (4.21548,3.30398) -- (4.21548,3.32047);
\draw [c] (4.21548,3.32047) -- (4.21548,3.33611);
\draw [c] (4.20066,3.32047) -- (4.21548,3.32047);
\draw [c] (4.21548,3.32047) -- (4.2303,3.32047);
\definecolor{c}{rgb}{0,0,0};
\colorlet{c}{kugray};
\draw [c] (4.24512,3.29408) -- (4.24512,3.31043);
\draw [c] (4.24512,3.31043) -- (4.24512,3.32595);
\draw [c] (4.2303,3.31043) -- (4.24512,3.31043);
\draw [c] (4.24512,3.31043) -- (4.25993,3.31043);
\definecolor{c}{rgb}{0,0,0};
\colorlet{c}{kugray};
\draw [c] (4.27475,3.3132) -- (4.27475,3.32917);
\draw [c] (4.27475,3.32917) -- (4.27475,3.34436);
\draw [c] (4.25993,3.32917) -- (4.27475,3.32917);
\draw [c] (4.27475,3.32917) -- (4.28957,3.32917);
\definecolor{c}{rgb}{0,0,0};
\colorlet{c}{kugray};
\draw [c] (4.30439,3.28435) -- (4.30439,3.30161);
\draw [c] (4.30439,3.30161) -- (4.30439,3.31795);
\draw [c] (4.28957,3.30161) -- (4.30439,3.30161);
\draw [c] (4.30439,3.30161) -- (4.31921,3.30161);
\definecolor{c}{rgb}{0,0,0};
\colorlet{c}{kugray};
\draw [c] (4.33402,3.2635) -- (4.33402,3.28087);
\draw [c] (4.33402,3.28087) -- (4.33402,3.29731);
\draw [c] (4.31921,3.28087) -- (4.33402,3.28087);
\draw [c] (4.33402,3.28087) -- (4.34884,3.28087);
\definecolor{c}{rgb}{0,0,0};
\colorlet{c}{kugray};
\draw [c] (4.36366,3.26382) -- (4.36366,3.28126);
\draw [c] (4.36366,3.28126) -- (4.36366,3.29776);
\draw [c] (4.34884,3.28126) -- (4.36366,3.28126);
\draw [c] (4.36366,3.28126) -- (4.37848,3.28126);
\definecolor{c}{rgb}{0,0,0};
\colorlet{c}{kugray};
\draw [c] (4.39329,3.26487) -- (4.39329,3.28278);
\draw [c] (4.39329,3.28278) -- (4.39329,3.29971);
\draw [c] (4.37848,3.28278) -- (4.39329,3.28278);
\draw [c] (4.39329,3.28278) -- (4.40811,3.28278);
\definecolor{c}{rgb}{0,0,0};
\colorlet{c}{kugray};
\draw [c] (4.42293,3.21601) -- (4.42293,3.23471);
\draw [c] (4.42293,3.23471) -- (4.42293,3.25234);
\draw [c] (4.40811,3.23471) -- (4.42293,3.23471);
\draw [c] (4.42293,3.23471) -- (4.43775,3.23471);
\definecolor{c}{rgb}{0,0,0};
\colorlet{c}{kugray};
\draw [c] (4.45257,3.20125) -- (4.45257,3.2207);
\draw [c] (4.45257,3.2207) -- (4.45257,3.239);
\draw [c] (4.43775,3.2207) -- (4.45257,3.2207);
\draw [c] (4.45257,3.2207) -- (4.46738,3.2207);
\definecolor{c}{rgb}{0,0,0};
\colorlet{c}{kugray};
\draw [c] (4.4822,3.21568) -- (4.4822,3.23454);
\draw [c] (4.4822,3.23454) -- (4.4822,3.25231);
\draw [c] (4.46738,3.23454) -- (4.4822,3.23454);
\draw [c] (4.4822,3.23454) -- (4.49702,3.23454);
\definecolor{c}{rgb}{0,0,0};
\colorlet{c}{kugray};
\draw [c] (4.51184,3.21203) -- (4.51184,3.23093);
\draw [c] (4.51184,3.23093) -- (4.51184,3.24874);
\draw [c] (4.49702,3.23093) -- (4.51184,3.23093);
\draw [c] (4.51184,3.23093) -- (4.52666,3.23093);
\definecolor{c}{rgb}{0,0,0};
\colorlet{c}{kugray};
\draw [c] (4.54147,3.1933) -- (4.54147,3.21207);
\draw [c] (4.54147,3.21207) -- (4.54147,3.22977);
\draw [c] (4.52666,3.21207) -- (4.54147,3.21207);
\draw [c] (4.54147,3.21207) -- (4.55629,3.21207);
\definecolor{c}{rgb}{0,0,0};
\colorlet{c}{kugray};
\draw [c] (4.57111,3.16612) -- (4.57111,3.18637);
\draw [c] (4.57111,3.18637) -- (4.57111,3.20537);
\draw [c] (4.55629,3.18637) -- (4.57111,3.18637);
\draw [c] (4.57111,3.18637) -- (4.58593,3.18637);
\definecolor{c}{rgb}{0,0,0};
\colorlet{c}{kugray};
\draw [c] (4.60075,3.144) -- (4.60075,3.1653);
\draw [c] (4.60075,3.1653) -- (4.60075,3.18522);
\draw [c] (4.58593,3.1653) -- (4.60075,3.1653);
\draw [c] (4.60075,3.1653) -- (4.61556,3.1653);
\definecolor{c}{rgb}{0,0,0};
\colorlet{c}{kugray};
\draw [c] (4.63038,3.13842) -- (4.63038,3.15957);
\draw [c] (4.63038,3.15957) -- (4.63038,3.17935);
\draw [c] (4.61556,3.15957) -- (4.63038,3.15957);
\draw [c] (4.63038,3.15957) -- (4.6452,3.15957);
\definecolor{c}{rgb}{0,0,0};
\colorlet{c}{kugray};
\draw [c] (4.66002,3.10825) -- (4.66002,3.12968);
\draw [c] (4.66002,3.12968) -- (4.66002,3.14972);
\draw [c] (4.6452,3.12968) -- (4.66002,3.12968);
\draw [c] (4.66002,3.12968) -- (4.67483,3.12968);
\definecolor{c}{rgb}{0,0,0};
\colorlet{c}{kugray};
\draw [c] (4.68965,3.12507) -- (4.68965,3.14699);
\draw [c] (4.68965,3.14699) -- (4.68965,3.16746);
\draw [c] (4.67483,3.14699) -- (4.68965,3.14699);
\draw [c] (4.68965,3.14699) -- (4.70447,3.14699);
\definecolor{c}{rgb}{0,0,0};
\colorlet{c}{kugray};
\draw [c] (4.71929,3.07502) -- (4.71929,3.09766);
\draw [c] (4.71929,3.09766) -- (4.71929,3.11875);
\draw [c] (4.70447,3.09766) -- (4.71929,3.09766);
\draw [c] (4.71929,3.09766) -- (4.73411,3.09766);
\definecolor{c}{rgb}{0,0,0};
\colorlet{c}{kugray};
\draw [c] (4.74892,3.12089) -- (4.74892,3.1424);
\draw [c] (4.74892,3.1424) -- (4.74892,3.1625);
\draw [c] (4.73411,3.1424) -- (4.74892,3.1424);
\draw [c] (4.74892,3.1424) -- (4.76374,3.1424);
\definecolor{c}{rgb}{0,0,0};
\colorlet{c}{kugray};
\draw [c] (4.77856,3.07673) -- (4.77856,3.09957);
\draw [c] (4.77856,3.09957) -- (4.77856,3.12084);
\draw [c] (4.76374,3.09957) -- (4.77856,3.09957);
\draw [c] (4.77856,3.09957) -- (4.79338,3.09957);
\definecolor{c}{rgb}{0,0,0};
\colorlet{c}{kugray};
\draw [c] (4.8082,3.11216) -- (4.8082,3.13437);
\draw [c] (4.8082,3.13437) -- (4.8082,3.15507);
\draw [c] (4.79338,3.13437) -- (4.8082,3.13437);
\draw [c] (4.8082,3.13437) -- (4.82301,3.13437);
\definecolor{c}{rgb}{0,0,0};
\colorlet{c}{kugray};
\draw [c] (4.83783,3.12021) -- (4.83783,3.14174);
\draw [c] (4.83783,3.14174) -- (4.83783,3.16186);
\draw [c] (4.82301,3.14174) -- (4.83783,3.14174);
\draw [c] (4.83783,3.14174) -- (4.85265,3.14174);
\definecolor{c}{rgb}{0,0,0};
\colorlet{c}{kugray};
\draw [c] (4.86747,3.04958) -- (4.86747,3.07475);
\draw [c] (4.86747,3.07475) -- (4.86747,3.09801);
\draw [c] (4.85265,3.07475) -- (4.86747,3.07475);
\draw [c] (4.86747,3.07475) -- (4.88228,3.07475);
\definecolor{c}{rgb}{0,0,0};
\colorlet{c}{kugray};
\draw [c] (4.8971,3.02902) -- (4.8971,3.05448);
\draw [c] (4.8971,3.05448) -- (4.8971,3.078);
\draw [c] (4.88228,3.05448) -- (4.8971,3.05448);
\draw [c] (4.8971,3.05448) -- (4.91192,3.05448);
\definecolor{c}{rgb}{0,0,0};
\colorlet{c}{kugray};
\draw [c] (4.92674,3.05931) -- (4.92674,3.08346);
\draw [c] (4.92674,3.08346) -- (4.92674,3.10584);
\draw [c] (4.91192,3.08346) -- (4.92674,3.08346);
\draw [c] (4.92674,3.08346) -- (4.94156,3.08346);
\definecolor{c}{rgb}{0,0,0};
\colorlet{c}{kugray};
\draw [c] (4.95637,3.04459) -- (4.95637,3.06898);
\draw [c] (4.95637,3.06898) -- (4.95637,3.09157);
\draw [c] (4.94156,3.06898) -- (4.95637,3.06898);
\draw [c] (4.95637,3.06898) -- (4.97119,3.06898);
\definecolor{c}{rgb}{0,0,0};
\colorlet{c}{kugray};
\draw [c] (4.98601,3.02119) -- (4.98601,3.04833);
\draw [c] (4.98601,3.04833) -- (4.98601,3.07328);
\draw [c] (4.97119,3.04833) -- (4.98601,3.04833);
\draw [c] (4.98601,3.04833) -- (5.00083,3.04833);
\definecolor{c}{rgb}{0,0,0};
\colorlet{c}{kugray};
\draw [c] (5.01565,3.03957) -- (5.01565,3.06478);
\draw [c] (5.01565,3.06478) -- (5.01565,3.08808);
\draw [c] (5.00083,3.06478) -- (5.01565,3.06478);
\draw [c] (5.01565,3.06478) -- (5.03046,3.06478);
\definecolor{c}{rgb}{0,0,0};
\colorlet{c}{kugray};
\draw [c] (5.04528,2.9954) -- (5.04528,3.02095);
\draw [c] (5.04528,3.02095) -- (5.04528,3.04453);
\draw [c] (5.03046,3.02095) -- (5.04528,3.02095);
\draw [c] (5.04528,3.02095) -- (5.0601,3.02095);
\definecolor{c}{rgb}{0,0,0};
\colorlet{c}{kugray};
\draw [c] (5.07492,3.00475) -- (5.07492,3.03087);
\draw [c] (5.07492,3.03087) -- (5.07492,3.05495);
\draw [c] (5.0601,3.03087) -- (5.07492,3.03087);
\draw [c] (5.07492,3.03087) -- (5.08974,3.03087);
\definecolor{c}{rgb}{0,0,0};
\colorlet{c}{kugray};
\draw [c] (5.10455,3.04579) -- (5.10455,3.07035);
\draw [c] (5.10455,3.07035) -- (5.10455,3.0931);
\draw [c] (5.08974,3.07035) -- (5.10455,3.07035);
\draw [c] (5.10455,3.07035) -- (5.11937,3.07035);
\definecolor{c}{rgb}{0,0,0};
\colorlet{c}{kugray};
\draw [c] (5.13419,2.99892) -- (5.13419,3.02565);
\draw [c] (5.13419,3.02565) -- (5.13419,3.05025);
\draw [c] (5.11937,3.02565) -- (5.13419,3.02565);
\draw [c] (5.13419,3.02565) -- (5.14901,3.02565);
\definecolor{c}{rgb}{0,0,0};
\colorlet{c}{kugray};
\draw [c] (5.16382,2.99685) -- (5.16382,3.02306);
\draw [c] (5.16382,3.02306) -- (5.16382,3.0472);
\draw [c] (5.14901,3.02306) -- (5.16382,3.02306);
\draw [c] (5.16382,3.02306) -- (5.17864,3.02306);
\definecolor{c}{rgb}{0,0,0};
\colorlet{c}{kugray};
\draw [c] (5.19346,2.98544) -- (5.19346,3.0123);
\draw [c] (5.19346,3.0123) -- (5.19346,3.03701);
\draw [c] (5.17864,3.0123) -- (5.19346,3.0123);
\draw [c] (5.19346,3.0123) -- (5.20828,3.0123);
\definecolor{c}{rgb}{0,0,0};
\colorlet{c}{kugray};
\draw [c] (5.2231,3.00482) -- (5.2231,3.0317);
\draw [c] (5.2231,3.0317) -- (5.2231,3.05642);
\draw [c] (5.20828,3.0317) -- (5.2231,3.0317);
\draw [c] (5.2231,3.0317) -- (5.23791,3.0317);
\definecolor{c}{rgb}{0,0,0};
\colorlet{c}{kugray};
\draw [c] (5.25273,2.9865) -- (5.25273,3.01355);
\draw [c] (5.25273,3.01355) -- (5.25273,3.03841);
\draw [c] (5.23791,3.01355) -- (5.25273,3.01355);
\draw [c] (5.25273,3.01355) -- (5.26755,3.01355);
\definecolor{c}{rgb}{0,0,0};
\colorlet{c}{kugray};
\draw [c] (5.28237,2.99411) -- (5.28237,3.02124);
\draw [c] (5.28237,3.02124) -- (5.28237,3.04618);
\draw [c] (5.26755,3.02124) -- (5.28237,3.02124);
\draw [c] (5.28237,3.02124) -- (5.29719,3.02124);
\definecolor{c}{rgb}{0,0,0};
\colorlet{c}{kugray};
\draw [c] (5.312,2.94471) -- (5.312,2.97506);
\draw [c] (5.312,2.97506) -- (5.312,3.00268);
\draw [c] (5.29719,2.97506) -- (5.312,2.97506);
\draw [c] (5.312,2.97506) -- (5.32682,2.97506);
\definecolor{c}{rgb}{0,0,0};
\colorlet{c}{kugray};
\draw [c] (5.34164,2.92698) -- (5.34164,2.95685);
\draw [c] (5.34164,2.95685) -- (5.34164,2.98407);
\draw [c] (5.32682,2.95685) -- (5.34164,2.95685);
\draw [c] (5.34164,2.95685) -- (5.35646,2.95685);
\definecolor{c}{rgb}{0,0,0};
\colorlet{c}{kugray};
\draw [c] (5.37127,2.88114) -- (5.37127,2.91303);
\draw [c] (5.37127,2.91303) -- (5.37127,2.94192);
\draw [c] (5.35646,2.91303) -- (5.37127,2.91303);
\draw [c] (5.37127,2.91303) -- (5.38609,2.91303);
\definecolor{c}{rgb}{0,0,0};
\colorlet{c}{kugray};
\draw [c] (5.40091,2.91147) -- (5.40091,2.942);
\draw [c] (5.40091,2.942) -- (5.40091,2.96977);
\draw [c] (5.38609,2.942) -- (5.40091,2.942);
\draw [c] (5.40091,2.942) -- (5.41573,2.942);
\definecolor{c}{rgb}{0,0,0};
\colorlet{c}{kugray};
\draw [c] (5.43055,2.89934) -- (5.43055,2.93053);
\draw [c] (5.43055,2.93053) -- (5.43055,2.95885);
\draw [c] (5.41573,2.93053) -- (5.43055,2.93053);
\draw [c] (5.43055,2.93053) -- (5.44536,2.93053);
\definecolor{c}{rgb}{0,0,0};
\colorlet{c}{kugray};
\draw [c] (5.46018,2.83663) -- (5.46018,2.86962);
\draw [c] (5.46018,2.86962) -- (5.46018,2.89942);
\draw [c] (5.44536,2.86962) -- (5.46018,2.86962);
\draw [c] (5.46018,2.86962) -- (5.475,2.86962);
\definecolor{c}{rgb}{0,0,0};
\colorlet{c}{kugray};
\draw [c] (5.48982,2.89582) -- (5.48982,2.92714);
\draw [c] (5.48982,2.92714) -- (5.48982,2.95557);
\draw [c] (5.475,2.92714) -- (5.48982,2.92714);
\draw [c] (5.48982,2.92714) -- (5.50464,2.92714);
\definecolor{c}{rgb}{0,0,0};
\colorlet{c}{kugray};
\draw [c] (5.51945,2.86857) -- (5.51945,2.90037);
\draw [c] (5.51945,2.90037) -- (5.51945,2.9292);
\draw [c] (5.50464,2.90037) -- (5.51945,2.90037);
\draw [c] (5.51945,2.90037) -- (5.53427,2.90037);
\definecolor{c}{rgb}{0,0,0};
\colorlet{c}{kugray};
\draw [c] (5.54909,2.91849) -- (5.54909,2.94886);
\draw [c] (5.54909,2.94886) -- (5.54909,2.97651);
\draw [c] (5.53427,2.94886) -- (5.54909,2.94886);
\draw [c] (5.54909,2.94886) -- (5.56391,2.94886);
\definecolor{c}{rgb}{0,0,0};
\colorlet{c}{kugray};
\draw [c] (5.57873,2.84325) -- (5.57873,2.87551);
\draw [c] (5.57873,2.87551) -- (5.57873,2.90471);
\draw [c] (5.56391,2.87551) -- (5.57873,2.87551);
\draw [c] (5.57873,2.87551) -- (5.59354,2.87551);
\definecolor{c}{rgb}{0,0,0};
\colorlet{c}{kugray};
\draw [c] (5.60836,2.86623) -- (5.60836,2.89815);
\draw [c] (5.60836,2.89815) -- (5.60836,2.92707);
\draw [c] (5.59354,2.89815) -- (5.60836,2.89815);
\draw [c] (5.60836,2.89815) -- (5.62318,2.89815);
\definecolor{c}{rgb}{0,0,0};
\colorlet{c}{kugray};
\draw [c] (5.638,2.91304) -- (5.638,2.94486);
\draw [c] (5.638,2.94486) -- (5.638,2.97369);
\draw [c] (5.62318,2.94486) -- (5.638,2.94486);
\draw [c] (5.638,2.94486) -- (5.65281,2.94486);
\definecolor{c}{rgb}{0,0,0};
\colorlet{c}{kugray};
\draw [c] (5.66763,2.90882) -- (5.66763,2.93949);
\draw [c] (5.66763,2.93949) -- (5.66763,2.96738);
\draw [c] (5.65281,2.93949) -- (5.66763,2.93949);
\draw [c] (5.66763,2.93949) -- (5.68245,2.93949);
\definecolor{c}{rgb}{0,0,0};
\colorlet{c}{kugray};
\draw [c] (5.69727,2.89105) -- (5.69727,2.92252);
\draw [c] (5.69727,2.92252) -- (5.69727,2.95107);
\draw [c] (5.68245,2.92252) -- (5.69727,2.92252);
\draw [c] (5.69727,2.92252) -- (5.71209,2.92252);
\definecolor{c}{rgb}{0,0,0};
\colorlet{c}{kugray};
\draw [c] (5.7269,2.77815) -- (5.7269,2.81636);
\draw [c] (5.7269,2.81636) -- (5.7269,2.85035);
\draw [c] (5.71209,2.81636) -- (5.7269,2.81636);
\draw [c] (5.7269,2.81636) -- (5.74172,2.81636);
\definecolor{c}{rgb}{0,0,0};
\colorlet{c}{kugray};
\draw [c] (5.75654,2.94321) -- (5.75654,2.97327);
\draw [c] (5.75654,2.97327) -- (5.75654,3.00065);
\draw [c] (5.74172,2.97327) -- (5.75654,2.97327);
\draw [c] (5.75654,2.97327) -- (5.77136,2.97327);
\definecolor{c}{rgb}{0,0,0};
\colorlet{c}{kugray};
\draw [c] (5.78618,2.8141) -- (5.78618,2.84956);
\draw [c] (5.78618,2.84956) -- (5.78618,2.88135);
\draw [c] (5.77136,2.84956) -- (5.78618,2.84956);
\draw [c] (5.78618,2.84956) -- (5.80099,2.84956);
\definecolor{c}{rgb}{0,0,0};
\colorlet{c}{kugray};
\draw [c] (5.81581,2.84857) -- (5.81581,2.88346);
\draw [c] (5.81581,2.88346) -- (5.81581,2.91479);
\draw [c] (5.80099,2.88346) -- (5.81581,2.88346);
\draw [c] (5.81581,2.88346) -- (5.83063,2.88346);
\definecolor{c}{rgb}{0,0,0};
\colorlet{c}{kugray};
\draw [c] (5.84545,2.83643) -- (5.84545,2.87155);
\draw [c] (5.84545,2.87155) -- (5.84545,2.90308);
\draw [c] (5.83063,2.87155) -- (5.84545,2.87155);
\draw [c] (5.84545,2.87155) -- (5.86026,2.87155);
\definecolor{c}{rgb}{0,0,0};
\colorlet{c}{kugray};
\draw [c] (5.87508,2.75529) -- (5.87508,2.79332);
\draw [c] (5.87508,2.79332) -- (5.87508,2.82715);
\draw [c] (5.86026,2.79332) -- (5.87508,2.79332);
\draw [c] (5.87508,2.79332) -- (5.8899,2.79332);
\definecolor{c}{rgb}{0,0,0};
\colorlet{c}{kugray};
\draw [c] (5.90472,2.77226) -- (5.90472,2.81175);
\draw [c] (5.90472,2.81175) -- (5.90472,2.84675);
\draw [c] (5.8899,2.81175) -- (5.90472,2.81175);
\draw [c] (5.90472,2.81175) -- (5.91954,2.81175);
\definecolor{c}{rgb}{0,0,0};
\colorlet{c}{kugray};
\draw [c] (5.93435,2.82058) -- (5.93435,2.85588);
\draw [c] (5.93435,2.85588) -- (5.93435,2.88754);
\draw [c] (5.91954,2.85588) -- (5.93435,2.85588);
\draw [c] (5.93435,2.85588) -- (5.94917,2.85588);
\definecolor{c}{rgb}{0,0,0};
\colorlet{c}{kugray};
\draw [c] (5.96399,2.83477) -- (5.96399,2.87069);
\draw [c] (5.96399,2.87069) -- (5.96399,2.90285);
\draw [c] (5.94917,2.87069) -- (5.96399,2.87069);
\draw [c] (5.96399,2.87069) -- (5.97881,2.87069);
\definecolor{c}{rgb}{0,0,0};
\colorlet{c}{kugray};
\draw [c] (5.99363,2.78914) -- (5.99363,2.8266);
\draw [c] (5.99363,2.8266) -- (5.99363,2.85999);
\draw [c] (5.97881,2.8266) -- (5.99363,2.8266);
\draw [c] (5.99363,2.8266) -- (6.00844,2.8266);
\definecolor{c}{rgb}{0,0,0};
\colorlet{c}{kugray};
\draw [c] (6.02326,2.82633) -- (6.02326,2.86318);
\draw [c] (6.02326,2.86318) -- (6.02326,2.89609);
\draw [c] (6.00844,2.86318) -- (6.02326,2.86318);
\draw [c] (6.02326,2.86318) -- (6.03808,2.86318);
\definecolor{c}{rgb}{0,0,0};
\colorlet{c}{kugray};
\draw [c] (6.0529,2.72961) -- (6.0529,2.7694);
\draw [c] (6.0529,2.7694) -- (6.0529,2.80462);
\draw [c] (6.03808,2.7694) -- (6.0529,2.7694);
\draw [c] (6.0529,2.7694) -- (6.06772,2.7694);
\definecolor{c}{rgb}{0,0,0};
\colorlet{c}{kugray};
\draw [c] (6.08253,2.84391) -- (6.08253,2.87833);
\draw [c] (6.08253,2.87833) -- (6.08253,2.90928);
\draw [c] (6.06772,2.87833) -- (6.08253,2.87833);
\draw [c] (6.08253,2.87833) -- (6.09735,2.87833);
\definecolor{c}{rgb}{0,0,0};
\colorlet{c}{kugray};
\draw [c] (6.11217,2.75513) -- (6.11217,2.79539);
\draw [c] (6.11217,2.79539) -- (6.11217,2.83099);
\draw [c] (6.09735,2.79539) -- (6.11217,2.79539);
\draw [c] (6.11217,2.79539) -- (6.12699,2.79539);
\definecolor{c}{rgb}{0,0,0};
\colorlet{c}{kugray};
\draw [c] (6.1418,2.67424) -- (6.1418,2.71934);
\draw [c] (6.1418,2.71934) -- (6.1418,2.75867);
\draw [c] (6.12699,2.71934) -- (6.1418,2.71934);
\draw [c] (6.1418,2.71934) -- (6.15662,2.71934);
\definecolor{c}{rgb}{0,0,0};
\colorlet{c}{kugray};
\draw [c] (6.17144,2.76048) -- (6.17144,2.79878);
\draw [c] (6.17144,2.79878) -- (6.17144,2.83284);
\draw [c] (6.15662,2.79878) -- (6.17144,2.79878);
\draw [c] (6.17144,2.79878) -- (6.18626,2.79878);
\definecolor{c}{rgb}{0,0,0};
\colorlet{c}{kugray};
\draw [c] (6.20108,2.73144) -- (6.20108,2.77033);
\draw [c] (6.20108,2.77033) -- (6.20108,2.80485);
\draw [c] (6.18626,2.77033) -- (6.20108,2.77033);
\draw [c] (6.20108,2.77033) -- (6.21589,2.77033);
\definecolor{c}{rgb}{0,0,0};
\colorlet{c}{kugray};
\draw [c] (6.23071,2.71313) -- (6.23071,2.75551);
\draw [c] (6.23071,2.75551) -- (6.23071,2.79276);
\draw [c] (6.21589,2.75551) -- (6.23071,2.75551);
\draw [c] (6.23071,2.75551) -- (6.24553,2.75551);
\definecolor{c}{rgb}{0,0,0};
\colorlet{c}{kugray};
\draw [c] (6.26035,2.73793) -- (6.26035,2.77947);
\draw [c] (6.26035,2.77947) -- (6.26035,2.81607);
\draw [c] (6.24553,2.77947) -- (6.26035,2.77947);
\draw [c] (6.26035,2.77947) -- (6.27517,2.77947);
\definecolor{c}{rgb}{0,0,0};
\colorlet{c}{kugray};
\draw [c] (6.28998,2.74513) -- (6.28998,2.7844);
\draw [c] (6.28998,2.7844) -- (6.28998,2.81923);
\draw [c] (6.27517,2.7844) -- (6.28998,2.7844);
\draw [c] (6.28998,2.7844) -- (6.3048,2.7844);
\definecolor{c}{rgb}{0,0,0};
\colorlet{c}{kugray};
\draw [c] (6.31962,2.7653) -- (6.31962,2.80355);
\draw [c] (6.31962,2.80355) -- (6.31962,2.83757);
\draw [c] (6.3048,2.80355) -- (6.31962,2.80355);
\draw [c] (6.31962,2.80355) -- (6.33444,2.80355);
\definecolor{c}{rgb}{0,0,0};
\colorlet{c}{kugray};
\draw [c] (6.34926,2.73889) -- (6.34926,2.78116);
\draw [c] (6.34926,2.78116) -- (6.34926,2.81833);
\draw [c] (6.33444,2.78116) -- (6.34926,2.78116);
\draw [c] (6.34926,2.78116) -- (6.36407,2.78116);
\definecolor{c}{rgb}{0,0,0};
\colorlet{c}{kugray};
\draw [c] (6.37889,2.70662) -- (6.37889,2.74897);
\draw [c] (6.37889,2.74897) -- (6.37889,2.7862);
\draw [c] (6.36407,2.74897) -- (6.37889,2.74897);
\draw [c] (6.37889,2.74897) -- (6.39371,2.74897);
\definecolor{c}{rgb}{0,0,0};
\colorlet{c}{kugray};
\draw [c] (6.40853,2.70883) -- (6.40853,2.74986);
\draw [c] (6.40853,2.74986) -- (6.40853,2.78606);
\draw [c] (6.39371,2.74986) -- (6.40853,2.74986);
\draw [c] (6.40853,2.74986) -- (6.42334,2.74986);
\definecolor{c}{rgb}{0,0,0};
\colorlet{c}{kugray};
\draw [c] (6.43816,2.74777) -- (6.43816,2.79061);
\draw [c] (6.43816,2.79061) -- (6.43816,2.82822);
\draw [c] (6.42334,2.79061) -- (6.43816,2.79061);
\draw [c] (6.43816,2.79061) -- (6.45298,2.79061);
\definecolor{c}{rgb}{0,0,0};
\colorlet{c}{kugray};
\draw [c] (6.4678,2.73848) -- (6.4678,2.78112);
\draw [c] (6.4678,2.78112) -- (6.4678,2.81857);
\draw [c] (6.45298,2.78112) -- (6.4678,2.78112);
\draw [c] (6.4678,2.78112) -- (6.48262,2.78112);
\definecolor{c}{rgb}{0,0,0};
\colorlet{c}{kugray};
\draw [c] (6.49743,2.68882) -- (6.49743,2.73113);
\draw [c] (6.49743,2.73113) -- (6.49743,2.76832);
\draw [c] (6.48262,2.73113) -- (6.49743,2.73113);
\draw [c] (6.49743,2.73113) -- (6.51225,2.73113);
\definecolor{c}{rgb}{0,0,0};
\colorlet{c}{kugray};
\draw [c] (6.52707,2.69846) -- (6.52707,2.74161);
\draw [c] (6.52707,2.74161) -- (6.52707,2.77946);
\draw [c] (6.51225,2.74161) -- (6.52707,2.74161);
\draw [c] (6.52707,2.74161) -- (6.54189,2.74161);
\definecolor{c}{rgb}{0,0,0};
\colorlet{c}{kugray};
\draw [c] (6.55671,2.70662) -- (6.55671,2.74955);
\draw [c] (6.55671,2.74955) -- (6.55671,2.78723);
\draw [c] (6.54189,2.74955) -- (6.55671,2.74955);
\draw [c] (6.55671,2.74955) -- (6.57152,2.74955);
\definecolor{c}{rgb}{0,0,0};
\colorlet{c}{kugray};
\draw [c] (6.58634,2.70574) -- (6.58634,2.75106);
\draw [c] (6.58634,2.75106) -- (6.58634,2.79057);
\draw [c] (6.57152,2.75106) -- (6.58634,2.75106);
\draw [c] (6.58634,2.75106) -- (6.60116,2.75106);
\definecolor{c}{rgb}{0,0,0};
\colorlet{c}{kugray};
\draw [c] (6.61598,2.68906) -- (6.61598,2.73402);
\draw [c] (6.61598,2.73402) -- (6.61598,2.77325);
\draw [c] (6.60116,2.73402) -- (6.61598,2.73402);
\draw [c] (6.61598,2.73402) -- (6.63079,2.73402);
\definecolor{c}{rgb}{0,0,0};
\colorlet{c}{kugray};
\draw [c] (6.64561,2.74536) -- (6.64561,2.78499);
\draw [c] (6.64561,2.78499) -- (6.64561,2.82009);
\draw [c] (6.63079,2.78499) -- (6.64561,2.78499);
\draw [c] (6.64561,2.78499) -- (6.66043,2.78499);
\definecolor{c}{rgb}{0,0,0};
\colorlet{c}{kugray};
\draw [c] (6.67525,2.72189) -- (6.67525,2.76667);
\draw [c] (6.67525,2.76667) -- (6.67525,2.80576);
\draw [c] (6.66043,2.76667) -- (6.67525,2.76667);
\draw [c] (6.67525,2.76667) -- (6.69007,2.76667);
\definecolor{c}{rgb}{0,0,0};
\colorlet{c}{kugray};
\draw [c] (6.70488,2.68289) -- (6.70488,2.72648);
\draw [c] (6.70488,2.72648) -- (6.70488,2.76466);
\draw [c] (6.69007,2.72648) -- (6.70488,2.72648);
\draw [c] (6.70488,2.72648) -- (6.7197,2.72648);
\definecolor{c}{rgb}{0,0,0};
\colorlet{c}{kugray};
\draw [c] (6.73452,2.59638) -- (6.73452,2.64724);
\draw [c] (6.73452,2.64724) -- (6.73452,2.69088);
\draw [c] (6.7197,2.64724) -- (6.73452,2.64724);
\draw [c] (6.73452,2.64724) -- (6.74934,2.64724);
\definecolor{c}{rgb}{0,0,0};
\colorlet{c}{kugray};
\draw [c] (6.76416,2.67667) -- (6.76416,2.72138);
\draw [c] (6.76416,2.72138) -- (6.76416,2.76041);
\draw [c] (6.74934,2.72138) -- (6.76416,2.72138);
\draw [c] (6.76416,2.72138) -- (6.77897,2.72138);
\definecolor{c}{rgb}{0,0,0};
\colorlet{c}{kugray};
\draw [c] (6.79379,2.66082) -- (6.79379,2.70714);
\draw [c] (6.79379,2.70714) -- (6.79379,2.74739);
\draw [c] (6.77897,2.70714) -- (6.79379,2.70714);
\draw [c] (6.79379,2.70714) -- (6.80861,2.70714);
\definecolor{c}{rgb}{0,0,0};
\colorlet{c}{kugray};
\draw [c] (6.82343,2.63019) -- (6.82343,2.67693);
\draw [c] (6.82343,2.67693) -- (6.82343,2.7175);
\draw [c] (6.80861,2.67693) -- (6.82343,2.67693);
\draw [c] (6.82343,2.67693) -- (6.83824,2.67693);
\definecolor{c}{rgb}{0,0,0};
\colorlet{c}{kugray};
\draw [c] (6.85306,2.55239) -- (6.85306,2.60372);
\draw [c] (6.85306,2.60372) -- (6.85306,2.64771);
\draw [c] (6.83824,2.60372) -- (6.85306,2.60372);
\draw [c] (6.85306,2.60372) -- (6.86788,2.60372);
\definecolor{c}{rgb}{0,0,0};
\colorlet{c}{kugray};
\draw [c] (6.8827,2.62294) -- (6.8827,2.67113);
\draw [c] (6.8827,2.67113) -- (6.8827,2.71279);
\draw [c] (6.86788,2.67113) -- (6.8827,2.67113);
\draw [c] (6.8827,2.67113) -- (6.89752,2.67113);
\definecolor{c}{rgb}{0,0,0};
\colorlet{c}{kugray};
\draw [c] (6.91233,2.6117) -- (6.91233,2.65948);
\draw [c] (6.91233,2.65948) -- (6.91233,2.70082);
\draw [c] (6.89752,2.65948) -- (6.91233,2.65948);
\draw [c] (6.91233,2.65948) -- (6.92715,2.65948);
\definecolor{c}{rgb}{0,0,0};
\colorlet{c}{kugray};
\draw [c] (6.94197,2.73243) -- (6.94197,2.77935);
\draw [c] (6.94197,2.77935) -- (6.94197,2.82005);
\draw [c] (6.92715,2.77935) -- (6.94197,2.77935);
\draw [c] (6.94197,2.77935) -- (6.95679,2.77935);
\definecolor{c}{rgb}{0,0,0};
\colorlet{c}{kugray};
\draw [c] (6.97161,2.48019) -- (6.97161,2.53947);
\draw [c] (6.97161,2.53947) -- (6.97161,2.58916);
\draw [c] (6.95679,2.53947) -- (6.97161,2.53947);
\draw [c] (6.97161,2.53947) -- (6.98642,2.53947);
\definecolor{c}{rgb}{0,0,0};
\colorlet{c}{kugray};
\draw [c] (7.00124,2.61681) -- (7.00124,2.66764);
\draw [c] (7.00124,2.66764) -- (7.00124,2.71126);
\draw [c] (6.98642,2.66764) -- (7.00124,2.66764);
\draw [c] (7.00124,2.66764) -- (7.01606,2.66764);
\definecolor{c}{rgb}{0,0,0};
\colorlet{c}{kugray};
\draw [c] (7.03088,2.47861) -- (7.03088,2.53761);
\draw [c] (7.03088,2.53761) -- (7.03088,2.5871);
\draw [c] (7.01606,2.53761) -- (7.03088,2.53761);
\draw [c] (7.03088,2.53761) -- (7.0457,2.53761);
\definecolor{c}{rgb}{0,0,0};
\colorlet{c}{kugray};
\draw [c] (7.06051,2.59337) -- (7.06051,2.64276);
\draw [c] (7.06051,2.64276) -- (7.06051,2.68531);
\draw [c] (7.0457,2.64276) -- (7.06051,2.64276);
\draw [c] (7.06051,2.64276) -- (7.07533,2.64276);
\definecolor{c}{rgb}{0,0,0};
\colorlet{c}{kugray};
\draw [c] (7.09015,2.57461) -- (7.09015,2.62915);
\draw [c] (7.09015,2.62915) -- (7.09015,2.67546);
\draw [c] (7.07533,2.62915) -- (7.09015,2.62915);
\draw [c] (7.09015,2.62915) -- (7.10497,2.62915);
\definecolor{c}{rgb}{0,0,0};
\colorlet{c}{kugray};
\draw [c] (7.11978,2.6974) -- (7.11978,2.74332);
\draw [c] (7.11978,2.74332) -- (7.11978,2.78328);
\draw [c] (7.10497,2.74332) -- (7.11978,2.74332);
\draw [c] (7.11978,2.74332) -- (7.1346,2.74332);
\definecolor{c}{rgb}{0,0,0};
\colorlet{c}{kugray};
\draw [c] (7.14942,2.66302) -- (7.14942,2.70937);
\draw [c] (7.14942,2.70937) -- (7.14942,2.74965);
\draw [c] (7.1346,2.70937) -- (7.14942,2.70937);
\draw [c] (7.14942,2.70937) -- (7.16424,2.70937);
\definecolor{c}{rgb}{0,0,0};
\colorlet{c}{kugray};
\draw [c] (7.17906,2.58815) -- (7.17906,2.64597);
\draw [c] (7.17906,2.64597) -- (7.17906,2.69464);
\draw [c] (7.16424,2.64597) -- (7.17906,2.64597);
\draw [c] (7.17906,2.64597) -- (7.19387,2.64597);
\definecolor{c}{rgb}{0,0,0};
\colorlet{c}{kugray};
\draw [c] (7.20869,2.50752) -- (7.20869,2.5657);
\draw [c] (7.20869,2.5657) -- (7.20869,2.61462);
\draw [c] (7.19387,2.5657) -- (7.20869,2.5657);
\draw [c] (7.20869,2.5657) -- (7.22351,2.5657);
\definecolor{c}{rgb}{0,0,0};
\colorlet{c}{kugray};
\draw [c] (7.23833,2.52123) -- (7.23833,2.57436);
\draw [c] (7.23833,2.57436) -- (7.23833,2.61966);
\draw [c] (7.22351,2.57436) -- (7.23833,2.57436);
\draw [c] (7.23833,2.57436) -- (7.25315,2.57436);
\definecolor{c}{rgb}{0,0,0};
\colorlet{c}{kugray};
\draw [c] (7.26796,2.55873) -- (7.26796,2.6113);
\draw [c] (7.26796,2.6113) -- (7.26796,2.65619);
\draw [c] (7.25315,2.6113) -- (7.26796,2.6113);
\draw [c] (7.26796,2.6113) -- (7.28278,2.6113);
\definecolor{c}{rgb}{0,0,0};
\colorlet{c}{kugray};
\draw [c] (7.2976,2.59094) -- (7.2976,2.64085);
\draw [c] (7.2976,2.64085) -- (7.2976,2.68379);
\draw [c] (7.28278,2.64085) -- (7.2976,2.64085);
\draw [c] (7.2976,2.64085) -- (7.31242,2.64085);
\definecolor{c}{rgb}{0,0,0};
\colorlet{c}{kugray};
\draw [c] (7.32724,2.66777) -- (7.32724,2.71706);
\draw [c] (7.32724,2.71706) -- (7.32724,2.75953);
\draw [c] (7.31242,2.71706) -- (7.32724,2.71706);
\draw [c] (7.32724,2.71706) -- (7.34205,2.71706);
\definecolor{c}{rgb}{0,0,0};
\colorlet{c}{kugray};
\draw [c] (7.35687,2.59037) -- (7.35687,2.6421);
\draw [c] (7.35687,2.6421) -- (7.35687,2.68637);
\draw [c] (7.34205,2.6421) -- (7.35687,2.6421);
\draw [c] (7.35687,2.6421) -- (7.37169,2.6421);
\definecolor{c}{rgb}{0,0,0};
\colorlet{c}{kugray};
\draw [c] (7.38651,2.5739) -- (7.38651,2.62653);
\draw [c] (7.38651,2.62653) -- (7.38651,2.67146);
\draw [c] (7.37169,2.62653) -- (7.38651,2.62653);
\draw [c] (7.38651,2.62653) -- (7.40132,2.62653);
\definecolor{c}{rgb}{0,0,0};
\colorlet{c}{kugray};
\draw [c] (7.41614,2.56099) -- (7.41614,2.61279);
\draw [c] (7.41614,2.61279) -- (7.41614,2.65711);
\draw [c] (7.40132,2.61279) -- (7.41614,2.61279);
\draw [c] (7.41614,2.61279) -- (7.43096,2.61279);
\definecolor{c}{rgb}{0,0,0};
\colorlet{c}{kugray};
\draw [c] (7.44578,2.61977) -- (7.44578,2.67016);
\draw [c] (7.44578,2.67016) -- (7.44578,2.71346);
\draw [c] (7.43096,2.67016) -- (7.44578,2.67016);
\draw [c] (7.44578,2.67016) -- (7.4606,2.67016);
\definecolor{c}{rgb}{0,0,0};
\colorlet{c}{kugray};
\draw [c] (7.47541,2.56963) -- (7.47541,2.62141);
\draw [c] (7.47541,2.62141) -- (7.47541,2.66572);
\draw [c] (7.4606,2.62141) -- (7.47541,2.62141);
\draw [c] (7.47541,2.62141) -- (7.49023,2.62141);
\definecolor{c}{rgb}{0,0,0};
\colorlet{c}{kugray};
\draw [c] (7.50505,2.56848) -- (7.50505,2.62408);
\draw [c] (7.50505,2.62408) -- (7.50505,2.67115);
\draw [c] (7.49023,2.62408) -- (7.50505,2.62408);
\draw [c] (7.50505,2.62408) -- (7.51987,2.62408);
\definecolor{c}{rgb}{0,0,0};
\colorlet{c}{kugray};
\draw [c] (7.53469,2.58512) -- (7.53469,2.63799);
\draw [c] (7.53469,2.63799) -- (7.53469,2.6831);
\draw [c] (7.51987,2.63799) -- (7.53469,2.63799);
\draw [c] (7.53469,2.63799) -- (7.5495,2.63799);
\definecolor{c}{rgb}{0,0,0};
\colorlet{c}{kugray};
\draw [c] (7.56432,2.57531) -- (7.56432,2.63381);
\draw [c] (7.56432,2.63381) -- (7.56432,2.68295);
\draw [c] (7.5495,2.63381) -- (7.56432,2.63381);
\draw [c] (7.56432,2.63381) -- (7.57914,2.63381);
\definecolor{c}{rgb}{0,0,0};
\colorlet{c}{kugray};
\draw [c] (7.59396,2.59086) -- (7.59396,2.64416);
\draw [c] (7.59396,2.64416) -- (7.59396,2.68958);
\draw [c] (7.57914,2.64416) -- (7.59396,2.64416);
\draw [c] (7.59396,2.64416) -- (7.60877,2.64416);
\definecolor{c}{rgb}{0,0,0};
\colorlet{c}{kugray};
\draw [c] (7.62359,2.69067) -- (7.62359,2.73229);
\draw [c] (7.62359,2.73229) -- (7.62359,2.76895);
\draw [c] (7.60877,2.73229) -- (7.62359,2.73229);
\draw [c] (7.62359,2.73229) -- (7.63841,2.73229);
\definecolor{c}{rgb}{0,0,0};
\colorlet{c}{kugray};
\draw [c] (7.65323,2.54373) -- (7.65323,2.60191);
\draw [c] (7.65323,2.60191) -- (7.65323,2.65082);
\draw [c] (7.63841,2.60191) -- (7.65323,2.60191);
\draw [c] (7.65323,2.60191) -- (7.66805,2.60191);
\definecolor{c}{rgb}{0,0,0};
\colorlet{c}{kugray};
\draw [c] (7.68286,2.62652) -- (7.68286,2.67485);
\draw [c] (7.68286,2.67485) -- (7.68286,2.71662);
\draw [c] (7.66805,2.67485) -- (7.68286,2.67485);
\draw [c] (7.68286,2.67485) -- (7.69768,2.67485);
\definecolor{c}{rgb}{0,0,0};
\colorlet{c}{kugray};
\draw [c] (7.7125,2.5223) -- (7.7125,2.58056);
\draw [c] (7.7125,2.58056) -- (7.7125,2.62954);
\draw [c] (7.69768,2.58056) -- (7.7125,2.58056);
\draw [c] (7.7125,2.58056) -- (7.72732,2.58056);
\definecolor{c}{rgb}{0,0,0};
\colorlet{c}{kugray};
\draw [c] (7.74214,2.52932) -- (7.74214,2.58746);
\draw [c] (7.74214,2.58746) -- (7.74214,2.63635);
\draw [c] (7.72732,2.58746) -- (7.74214,2.58746);
\draw [c] (7.74214,2.58746) -- (7.75695,2.58746);
\definecolor{c}{rgb}{0,0,0};
\colorlet{c}{kugray};
\draw [c] (7.77177,2.60875) -- (7.77177,2.65775);
\draw [c] (7.77177,2.65775) -- (7.77177,2.70001);
\draw [c] (7.75695,2.65775) -- (7.77177,2.65775);
\draw [c] (7.77177,2.65775) -- (7.78659,2.65775);
\definecolor{c}{rgb}{0,0,0};
\colorlet{c}{kugray};
\draw [c] (7.80141,2.53381) -- (7.80141,2.58535);
\draw [c] (7.80141,2.58535) -- (7.80141,2.62948);
\draw [c] (7.78659,2.58535) -- (7.80141,2.58535);
\draw [c] (7.80141,2.58535) -- (7.81623,2.58535);
\definecolor{c}{rgb}{0,0,0};
\colorlet{c}{kugray};
\draw [c] (7.83104,2.56688) -- (7.83104,2.62207);
\draw [c] (7.83104,2.62207) -- (7.83104,2.66885);
\draw [c] (7.81623,2.62207) -- (7.83104,2.62207);
\draw [c] (7.83104,2.62207) -- (7.84586,2.62207);
\definecolor{c}{rgb}{0,0,0};
\colorlet{c}{kugray};
\draw [c] (7.86068,2.53374) -- (7.86068,2.58902);
\draw [c] (7.86068,2.58902) -- (7.86068,2.63587);
\draw [c] (7.84586,2.58902) -- (7.86068,2.58902);
\draw [c] (7.86068,2.58902) -- (7.8755,2.58902);
\definecolor{c}{rgb}{0,0,0};
\colorlet{c}{kugray};
\draw [c] (7.89031,2.54177) -- (7.89031,2.59754);
\draw [c] (7.89031,2.59754) -- (7.89031,2.64474);
\draw [c] (7.8755,2.59754) -- (7.89031,2.59754);
\draw [c] (7.89031,2.59754) -- (7.90513,2.59754);
\definecolor{c}{rgb}{0,0,0};
\colorlet{c}{kugray};
\draw [c] (7.91995,2.66944) -- (7.91995,2.71223);
\draw [c] (7.91995,2.71223) -- (7.91995,2.74979);
\draw [c] (7.90513,2.71223) -- (7.91995,2.71223);
\draw [c] (7.91995,2.71223) -- (7.93477,2.71223);
\definecolor{c}{rgb}{0,0,0};
\colorlet{c}{kugray};
\draw [c] (7.94959,2.58273) -- (7.94959,2.63198);
\draw [c] (7.94959,2.63198) -- (7.94959,2.67442);
\draw [c] (7.93477,2.63198) -- (7.94959,2.63198);
\draw [c] (7.94959,2.63198) -- (7.9644,2.63198);
\definecolor{c}{rgb}{0,0,0};
\colorlet{c}{kugray};
\draw [c] (7.97922,2.51141) -- (7.97922,2.56884);
\draw [c] (7.97922,2.56884) -- (7.97922,2.61723);
\draw [c] (7.9644,2.56884) -- (7.97922,2.56884);
\draw [c] (7.97922,2.56884) -- (7.99404,2.56884);
\definecolor{c}{rgb}{0,0,0};
\colorlet{c}{kugray};
\draw [c] (8.00886,2.4778) -- (8.00886,2.5402);
\draw [c] (8.00886,2.5402) -- (8.00886,2.59206);
\draw [c] (7.99404,2.5402) -- (8.00886,2.5402);
\draw [c] (8.00886,2.5402) -- (8.02368,2.5402);
\definecolor{c}{rgb}{0,0,0};
\colorlet{c}{kugray};
\draw [c] (8.03849,2.52697) -- (8.03849,2.58159);
\draw [c] (8.03849,2.58159) -- (8.03849,2.62797);
\draw [c] (8.02368,2.58159) -- (8.03849,2.58159);
\draw [c] (8.03849,2.58159) -- (8.05331,2.58159);
\definecolor{c}{rgb}{0,0,0};
\colorlet{c}{kugray};
\draw [c] (8.06813,2.51298) -- (8.06813,2.57152);
\draw [c] (8.06813,2.57152) -- (8.06813,2.6207);
\draw [c] (8.05331,2.57152) -- (8.06813,2.57152);
\draw [c] (8.06813,2.57152) -- (8.08295,2.57152);
\definecolor{c}{rgb}{0,0,0};
\colorlet{c}{kugray};
\draw [c] (8.09776,2.42785) -- (8.09776,2.49484);
\draw [c] (8.09776,2.49484) -- (8.09776,2.54982);
\draw [c] (8.08295,2.49484) -- (8.09776,2.49484);
\draw [c] (8.09776,2.49484) -- (8.11258,2.49484);
\definecolor{c}{rgb}{0,0,0};
\colorlet{c}{kugray};
\draw [c] (8.1274,2.54013) -- (8.1274,2.59386);
\draw [c] (8.1274,2.59386) -- (8.1274,2.6396);
\draw [c] (8.11258,2.59386) -- (8.1274,2.59386);
\draw [c] (8.1274,2.59386) -- (8.14222,2.59386);
\definecolor{c}{rgb}{0,0,0};
\colorlet{c}{kugray};
\draw [c] (8.15704,2.5211) -- (8.15704,2.5755);
\draw [c] (8.15704,2.5755) -- (8.15704,2.62172);
\draw [c] (8.14222,2.5755) -- (8.15704,2.5755);
\draw [c] (8.15704,2.5755) -- (8.17185,2.5755);
\definecolor{c}{rgb}{0,0,0};
\colorlet{c}{kugray};
\draw [c] (8.18667,2.48828) -- (8.18667,2.54723);
\draw [c] (8.18667,2.54723) -- (8.18667,2.59668);
\draw [c] (8.17185,2.54723) -- (8.18667,2.54723);
\draw [c] (8.18667,2.54723) -- (8.20149,2.54723);
\definecolor{c}{rgb}{0,0,0};
\colorlet{c}{kugray};
\draw [c] (8.21631,2.51647) -- (8.21631,2.57511);
\draw [c] (8.21631,2.57511) -- (8.21631,2.62434);
\draw [c] (8.20149,2.57511) -- (8.21631,2.57511);
\draw [c] (8.21631,2.57511) -- (8.23113,2.57511);
\definecolor{c}{rgb}{0,0,0};
\colorlet{c}{kugray};
\draw [c] (8.24594,2.4989) -- (8.24594,2.55998);
\draw [c] (8.24594,2.55998) -- (8.24594,2.61092);
\draw [c] (8.23113,2.55998) -- (8.24594,2.55998);
\draw [c] (8.24594,2.55998) -- (8.26076,2.55998);
\definecolor{c}{rgb}{0,0,0};
\colorlet{c}{kugray};
\draw [c] (8.27558,2.43272) -- (8.27558,2.4971);
\draw [c] (8.27558,2.4971) -- (8.27558,2.55033);
\draw [c] (8.26076,2.4971) -- (8.27558,2.4971);
\draw [c] (8.27558,2.4971) -- (8.2904,2.4971);
\definecolor{c}{rgb}{0,0,0};
\colorlet{c}{kugray};
\draw [c] (8.30521,2.47691) -- (8.30521,2.53483);
\draw [c] (8.30521,2.53483) -- (8.30521,2.58355);
\draw [c] (8.2904,2.53483) -- (8.30521,2.53483);
\draw [c] (8.30521,2.53483) -- (8.32003,2.53483);
\definecolor{c}{rgb}{0,0,0};
\colorlet{c}{kugray};
\draw [c] (8.33485,2.49351) -- (8.33485,2.55549);
\draw [c] (8.33485,2.55549) -- (8.33485,2.60705);
\draw [c] (8.32003,2.55549) -- (8.33485,2.55549);
\draw [c] (8.33485,2.55549) -- (8.34967,2.55549);
\definecolor{c}{rgb}{0,0,0};
\colorlet{c}{kugray};
\draw [c] (8.36449,2.4196) -- (8.36449,2.48391);
\draw [c] (8.36449,2.48391) -- (8.36449,2.53708);
\draw [c] (8.34967,2.48391) -- (8.36449,2.48391);
\draw [c] (8.36449,2.48391) -- (8.3793,2.48391);
\definecolor{c}{rgb}{0,0,0};
\colorlet{c}{kugray};
\draw [c] (8.39412,2.39842) -- (8.39412,2.46762);
\draw [c] (8.39412,2.46762) -- (8.39412,2.52409);
\draw [c] (8.3793,2.46762) -- (8.39412,2.46762);
\draw [c] (8.39412,2.46762) -- (8.40894,2.46762);
\definecolor{c}{rgb}{0,0,0};
\colorlet{c}{kugray};
\draw [c] (8.42376,2.50236) -- (8.42376,2.56041);
\draw [c] (8.42376,2.56041) -- (8.42376,2.60923);
\draw [c] (8.40894,2.56041) -- (8.42376,2.56041);
\draw [c] (8.42376,2.56041) -- (8.43858,2.56041);
\definecolor{c}{rgb}{0,0,0};
\colorlet{c}{kugray};
\draw [c] (8.45339,2.4599) -- (8.45339,2.5192);
\draw [c] (8.45339,2.5192) -- (8.45339,2.56891);
\draw [c] (8.43858,2.5192) -- (8.45339,2.5192);
\draw [c] (8.45339,2.5192) -- (8.46821,2.5192);
\definecolor{c}{rgb}{0,0,0};
\colorlet{c}{kugray};
\draw [c] (8.48303,2.55597) -- (8.48303,2.61113);
\draw [c] (8.48303,2.61113) -- (8.48303,2.65789);
\draw [c] (8.46821,2.61113) -- (8.48303,2.61113);
\draw [c] (8.48303,2.61113) -- (8.49785,2.61113);
\definecolor{c}{rgb}{0,0,0};
\colorlet{c}{kugray};
\draw [c] (8.51267,2.3936) -- (8.51267,2.46748);
\draw [c] (8.51267,2.46748) -- (8.51267,2.52702);
\draw [c] (8.49785,2.46748) -- (8.51267,2.46748);
\draw [c] (8.51267,2.46748) -- (8.52748,2.46748);
\definecolor{c}{rgb}{0,0,0};
\colorlet{c}{kugray};
\draw [c] (8.5423,2.42909) -- (8.5423,2.50139);
\draw [c] (8.5423,2.50139) -- (8.5423,2.5599);
\draw [c] (8.52748,2.50139) -- (8.5423,2.50139);
\draw [c] (8.5423,2.50139) -- (8.55712,2.50139);
\definecolor{c}{rgb}{0,0,0};
\colorlet{c}{kugray};
\draw [c] (8.57194,2.35401) -- (8.57194,2.42763);
\draw [c] (8.57194,2.42763) -- (8.57194,2.487);
\draw [c] (8.55712,2.42763) -- (8.57194,2.42763);
\draw [c] (8.57194,2.42763) -- (8.58675,2.42763);
\definecolor{c}{rgb}{0,0,0};
\colorlet{c}{kugray};
\draw [c] (8.60157,2.37123) -- (8.60157,2.4445);
\draw [c] (8.60157,2.4445) -- (8.60157,2.50364);
\draw [c] (8.58675,2.4445) -- (8.60157,2.4445);
\draw [c] (8.60157,2.4445) -- (8.61639,2.4445);
\definecolor{c}{rgb}{0,0,0};
\colorlet{c}{kugray};
\draw [c] (8.63121,2.43393) -- (8.63121,2.50355);
\draw [c] (8.63121,2.50355) -- (8.63121,2.5603);
\draw [c] (8.61639,2.50355) -- (8.63121,2.50355);
\draw [c] (8.63121,2.50355) -- (8.64603,2.50355);
\definecolor{c}{rgb}{0,0,0};
\colorlet{c}{kugray};
\draw [c] (8.66084,2.47327) -- (8.66084,2.53303);
\draw [c] (8.66084,2.53303) -- (8.66084,2.58305);
\draw [c] (8.64603,2.53303) -- (8.66084,2.53303);
\draw [c] (8.66084,2.53303) -- (8.67566,2.53303);
\definecolor{c}{rgb}{0,0,0};
\colorlet{c}{kugray};
\draw [c] (8.69048,2.46275) -- (8.69048,2.52243);
\draw [c] (8.69048,2.52243) -- (8.69048,2.5724);
\draw [c] (8.67566,2.52243) -- (8.69048,2.52243);
\draw [c] (8.69048,2.52243) -- (8.7053,2.52243);
\definecolor{c}{rgb}{0,0,0};
\colorlet{c}{kugray};
\draw [c] (8.72012,2.33119) -- (8.72012,2.41226);
\draw [c] (8.72012,2.41226) -- (8.72012,2.47637);
\draw [c] (8.7053,2.41226) -- (8.72012,2.41226);
\draw [c] (8.72012,2.41226) -- (8.73493,2.41226);
\definecolor{c}{rgb}{0,0,0};
\colorlet{c}{kugray};
\draw [c] (8.74975,2.3572) -- (8.74975,2.4275);
\draw [c] (8.74975,2.4275) -- (8.74975,2.48469);
\draw [c] (8.73493,2.4275) -- (8.74975,2.4275);
\draw [c] (8.74975,2.4275) -- (8.76457,2.4275);
\definecolor{c}{rgb}{0,0,0};
\colorlet{c}{kugray};
\draw [c] (8.77939,2.41881) -- (8.77939,2.49045);
\draw [c] (8.77939,2.49045) -- (8.77939,2.54853);
\draw [c] (8.76457,2.49045) -- (8.77939,2.49045);
\draw [c] (8.77939,2.49045) -- (8.79421,2.49045);
\definecolor{c}{rgb}{0,0,0};
\colorlet{c}{kugray};
\draw [c] (8.80902,2.52182) -- (8.80902,2.57585);
\draw [c] (8.80902,2.57585) -- (8.80902,2.62181);
\draw [c] (8.79421,2.57585) -- (8.80902,2.57585);
\draw [c] (8.80902,2.57585) -- (8.82384,2.57585);
\definecolor{c}{rgb}{0,0,0};
\colorlet{c}{kugray};
\draw [c] (8.83866,2.47132) -- (8.83866,2.53141);
\draw [c] (8.83866,2.53141) -- (8.83866,2.58166);
\draw [c] (8.82384,2.53141) -- (8.83866,2.53141);
\draw [c] (8.83866,2.53141) -- (8.85348,2.53141);
\definecolor{c}{rgb}{0,0,0};
\colorlet{c}{kugray};
\draw [c] (8.86829,2.49235) -- (8.86829,2.5502);
\draw [c] (8.86829,2.5502) -- (8.86829,2.59887);
\draw [c] (8.85348,2.5502) -- (8.86829,2.5502);
\draw [c] (8.86829,2.5502) -- (8.88311,2.5502);
\definecolor{c}{rgb}{0,0,0};
\colorlet{c}{kugray};
\draw [c] (8.89793,2.47756) -- (8.89793,2.54006);
\draw [c] (8.89793,2.54006) -- (8.89793,2.59198);
\draw [c] (8.88311,2.54006) -- (8.89793,2.54006);
\draw [c] (8.89793,2.54006) -- (8.91275,2.54006);
\definecolor{c}{rgb}{0,0,0};
\colorlet{c}{kugray};
\draw [c] (8.92757,2.39734) -- (8.92757,2.46961);
\draw [c] (8.92757,2.46961) -- (8.92757,2.52811);
\draw [c] (8.91275,2.46961) -- (8.92757,2.46961);
\draw [c] (8.92757,2.46961) -- (8.94238,2.46961);
\definecolor{c}{rgb}{0,0,0};
\colorlet{c}{kugray};
\draw [c] (8.9572,2.42281) -- (8.9572,2.48693);
\draw [c] (8.9572,2.48693) -- (8.9572,2.53997);
\draw [c] (8.94238,2.48693) -- (8.9572,2.48693);
\draw [c] (8.9572,2.48693) -- (8.97202,2.48693);
\definecolor{c}{rgb}{0,0,0};
\colorlet{c}{kugray};
\draw [c] (8.98684,2.45799) -- (8.98684,2.51852);
\draw [c] (8.98684,2.51852) -- (8.98684,2.56909);
\draw [c] (8.97202,2.51852) -- (8.98684,2.51852);
\draw [c] (8.98684,2.51852) -- (9.00166,2.51852);
\definecolor{c}{rgb}{0,0,0};
\colorlet{c}{kugray};
\draw [c] (9.01647,2.30805) -- (9.01647,2.39137);
\draw [c] (9.01647,2.39137) -- (9.01647,2.45687);
\draw [c] (9.00166,2.39137) -- (9.01647,2.39137);
\draw [c] (9.01647,2.39137) -- (9.03129,2.39137);
\definecolor{c}{rgb}{0,0,0};
\colorlet{c}{kugray};
\draw [c] (9.04611,2.42762) -- (9.04611,2.49931);
\draw [c] (9.04611,2.49931) -- (9.04611,2.55742);
\draw [c] (9.03129,2.49931) -- (9.04611,2.49931);
\draw [c] (9.04611,2.49931) -- (9.06093,2.49931);
\definecolor{c}{rgb}{0,0,0};
\colorlet{c}{kugray};
\draw [c] (9.07574,2.43243) -- (9.07574,2.5203);
\draw [c] (9.07574,2.5203) -- (9.07574,2.58858);
\draw [c] (9.06093,2.5203) -- (9.07574,2.5203);
\draw [c] (9.07574,2.5203) -- (9.09056,2.5203);
\definecolor{c}{rgb}{0,0,0};
\colorlet{c}{kugray};
\draw [c] (9.10538,2.33902) -- (9.10538,2.41416);
\draw [c] (9.10538,2.41416) -- (9.10538,2.47451);
\draw [c] (9.09056,2.41416) -- (9.10538,2.41416);
\draw [c] (9.10538,2.41416) -- (9.1202,2.41416);
\definecolor{c}{rgb}{0,0,0};
\colorlet{c}{kugray};
\draw [c] (9.13502,2.4421) -- (9.13502,2.50747);
\draw [c] (9.13502,2.50747) -- (9.13502,2.56135);
\draw [c] (9.1202,2.50747) -- (9.13502,2.50747);
\draw [c] (9.13502,2.50747) -- (9.14983,2.50747);
\definecolor{c}{rgb}{0,0,0};
\colorlet{c}{kugray};
\draw [c] (9.16465,2.44597) -- (9.16465,2.51695);
\draw [c] (9.16465,2.51695) -- (9.16465,2.5746);
\draw [c] (9.14983,2.51695) -- (9.16465,2.51695);
\draw [c] (9.16465,2.51695) -- (9.17947,2.51695);
\definecolor{c}{rgb}{0,0,0};
\colorlet{c}{kugray};
\draw [c] (9.19429,2.3813) -- (9.19429,2.45249);
\draw [c] (9.19429,2.45249) -- (9.19429,2.51027);
\draw [c] (9.17947,2.45249) -- (9.19429,2.45249);
\draw [c] (9.19429,2.45249) -- (9.20911,2.45249);
\definecolor{c}{rgb}{0,0,0};
\colorlet{c}{kugray};
\draw [c] (9.22392,2.26959) -- (9.22392,2.35173);
\draw [c] (9.22392,2.35173) -- (9.22392,2.4165);
\draw [c] (9.20911,2.35173) -- (9.22392,2.35173);
\draw [c] (9.22392,2.35173) -- (9.23874,2.35173);
\definecolor{c}{rgb}{0,0,0};
\colorlet{c}{kugray};
\draw [c] (9.25356,2.47023) -- (9.25356,2.53233);
\draw [c] (9.25356,2.53233) -- (9.25356,2.58399);
\draw [c] (9.23874,2.53233) -- (9.25356,2.53233);
\draw [c] (9.25356,2.53233) -- (9.26838,2.53233);
\definecolor{c}{rgb}{0,0,0};
\colorlet{c}{kugray};
\draw [c] (9.2832,2.25761) -- (9.2832,2.34259);
\draw [c] (9.2832,2.34259) -- (9.2832,2.40911);
\draw [c] (9.26838,2.34259) -- (9.2832,2.34259);
\draw [c] (9.2832,2.34259) -- (9.29801,2.34259);
\definecolor{c}{rgb}{0,0,0};
\colorlet{c}{kugray};
\draw [c] (9.31283,2.38074) -- (9.31283,2.45146);
\draw [c] (9.31283,2.45146) -- (9.31283,2.50893);
\draw [c] (9.29801,2.45146) -- (9.31283,2.45146);
\draw [c] (9.31283,2.45146) -- (9.32765,2.45146);
\definecolor{c}{rgb}{0,0,0};
\colorlet{c}{kugray};
\draw [c] (9.34247,2.30157) -- (9.34247,2.38622);
\draw [c] (9.34247,2.38622) -- (9.34247,2.45254);
\draw [c] (9.32765,2.38622) -- (9.34247,2.38622);
\draw [c] (9.34247,2.38622) -- (9.35728,2.38622);
\definecolor{c}{rgb}{0,0,0};
\colorlet{c}{kugray};
\draw [c] (9.3721,2.44079) -- (9.3721,2.50347);
\draw [c] (9.3721,2.50347) -- (9.3721,2.55552);
\draw [c] (9.35728,2.50347) -- (9.3721,2.50347);
\draw [c] (9.3721,2.50347) -- (9.38692,2.50347);
\definecolor{c}{rgb}{0,0,0};
\colorlet{c}{kugray};
\draw [c] (9.40174,2.35916) -- (9.40174,2.42967);
\draw [c] (9.40174,2.42967) -- (9.40174,2.487);
\draw [c] (9.38692,2.42967) -- (9.40174,2.42967);
\draw [c] (9.40174,2.42967) -- (9.41656,2.42967);
\definecolor{c}{rgb}{0,0,0};
\colorlet{c}{kugray};
\draw [c] (9.43137,2.34873) -- (9.43137,2.418);
\draw [c] (9.43137,2.418) -- (9.43137,2.4745);
\draw [c] (9.41656,2.418) -- (9.43137,2.418);
\draw [c] (9.43137,2.418) -- (9.44619,2.418);
\definecolor{c}{rgb}{0,0,0};
\colorlet{c}{kugray};
\draw [c] (9.46101,2.48472) -- (9.46101,2.54922);
\draw [c] (9.46101,2.54922) -- (9.46101,2.60252);
\draw [c] (9.44619,2.54922) -- (9.46101,2.54922);
\draw [c] (9.46101,2.54922) -- (9.47583,2.54922);
\definecolor{c}{rgb}{0,0,0};
\colorlet{c}{kugray};
\draw [c] (9.49065,2.28137) -- (9.49065,2.3622);
\draw [c] (9.49065,2.3622) -- (9.49065,2.42615);
\draw [c] (9.47583,2.3622) -- (9.49065,2.3622);
\draw [c] (9.49065,2.3622) -- (9.50546,2.3622);
\definecolor{c}{rgb}{0,0,0};
\colorlet{c}{kugray};
\draw [c] (9.52028,2.44204) -- (9.52028,2.50736);
\draw [c] (9.52028,2.50736) -- (9.52028,2.56121);
\draw [c] (9.50546,2.50736) -- (9.52028,2.50736);
\draw [c] (9.52028,2.50736) -- (9.5351,2.50736);
\definecolor{c}{rgb}{0,0,0};
\colorlet{c}{kugray};
\draw [c] (9.54992,2.16356) -- (9.54992,2.26322);
\draw [c] (9.54992,2.26322) -- (9.54992,2.33839);
\draw [c] (9.5351,2.26322) -- (9.54992,2.26322);
\draw [c] (9.54992,2.26322) -- (9.56474,2.26322);
\definecolor{c}{rgb}{0,0,0};
\colorlet{c}{kugray};
\draw [c] (9.57955,2.38059) -- (9.57955,2.45112);
\draw [c] (9.57955,2.45112) -- (9.57955,2.50847);
\draw [c] (9.56474,2.45112) -- (9.57955,2.45112);
\draw [c] (9.57955,2.45112) -- (9.59437,2.45112);
\definecolor{c}{rgb}{0,0,0};
\colorlet{c}{kugray};
\draw [c] (9.60919,2.39358) -- (9.60919,2.4702);
\draw [c] (9.60919,2.4702) -- (9.60919,2.5315);
\draw [c] (9.59437,2.4702) -- (9.60919,2.4702);
\draw [c] (9.60919,2.4702) -- (9.62401,2.4702);
\definecolor{c}{rgb}{0,0,0};
\colorlet{c}{kugray};
\draw [c] (9.63882,2.215) -- (9.63882,2.30298);
\draw [c] (9.63882,2.30298) -- (9.63882,2.37132);
\draw [c] (9.62401,2.30298) -- (9.63882,2.30298);
\draw [c] (9.63882,2.30298) -- (9.65364,2.30298);
\definecolor{c}{rgb}{0,0,0};
\colorlet{c}{kugray};
\draw [c] (9.66846,2.31943) -- (9.66846,2.39972);
\draw [c] (9.66846,2.39972) -- (9.66846,2.46334);
\draw [c] (9.65364,2.39972) -- (9.66846,2.39972);
\draw [c] (9.66846,2.39972) -- (9.68328,2.39972);
\definecolor{c}{rgb}{0,0,0};
\colorlet{c}{kugray};
\draw [c] (9.6981,2.36691) -- (9.6981,2.44427);
\draw [c] (9.6981,2.44427) -- (9.6981,2.50605);
\draw [c] (9.68328,2.44427) -- (9.6981,2.44427);
\draw [c] (9.6981,2.44427) -- (9.71291,2.44427);
\definecolor{c}{rgb}{0,0,0};
\colorlet{c}{kugray};
\draw [c] (9.72773,2.36518) -- (9.72773,2.44376);
\draw [c] (9.72773,2.44376) -- (9.72773,2.50632);
\draw [c] (9.71291,2.44376) -- (9.72773,2.44376);
\draw [c] (9.72773,2.44376) -- (9.74255,2.44376);
\definecolor{c}{rgb}{0,0,0};
\colorlet{c}{kugray};
\draw [c] (9.75737,2.17203) -- (9.75737,2.267);
\draw [c] (9.75737,2.267) -- (9.75737,2.33947);
\draw [c] (9.74255,2.267) -- (9.75737,2.267);
\draw [c] (9.75737,2.267) -- (9.77219,2.267);
\definecolor{c}{rgb}{0,0,0};
\colorlet{c}{kugray};
\draw [c] (9.787,2.27887) -- (9.787,2.35679);
\draw [c] (9.787,2.35679) -- (9.787,2.41892);
\draw [c] (9.77219,2.35679) -- (9.787,2.35679);
\draw [c] (9.787,2.35679) -- (9.80182,2.35679);
\definecolor{c}{rgb}{0,0,0};
\colorlet{c}{kugray};
\draw [c] (9.81664,2.11839) -- (9.81664,2.23312);
\draw [c] (9.81664,2.23312) -- (9.81664,2.31651);
\draw [c] (9.80182,2.23312) -- (9.81664,2.23312);
\draw [c] (9.81664,2.23312) -- (9.83146,2.23312);
\definecolor{c}{rgb}{0,0,0};
\colorlet{c}{kugray};
\draw [c] (9.84627,2.36042) -- (9.84627,2.42873);
\draw [c] (9.84627,2.42873) -- (9.84627,2.4846);
\draw [c] (9.83146,2.42873) -- (9.84627,2.42873);
\draw [c] (9.84627,2.42873) -- (9.86109,2.42873);
\definecolor{c}{rgb}{0,0,0};
\colorlet{c}{kugray};
\draw [c] (9.87591,2.33174) -- (9.87591,2.40637);
\draw [c] (9.87591,2.40637) -- (9.87591,2.46639);
\draw [c] (9.86109,2.40637) -- (9.87591,2.40637);
\draw [c] (9.87591,2.40637) -- (9.89073,2.40637);
\definecolor{c}{rgb}{0,0,0};
\colorlet{c}{kugray};
\draw [c] (9.90555,2.36337) -- (9.90555,2.43811);
\draw [c] (9.90555,2.43811) -- (9.90555,2.4982);
\draw [c] (9.89073,2.43811) -- (9.90555,2.43811);
\draw [c] (9.90555,2.43811) -- (9.92036,2.43811);
\definecolor{c}{rgb}{0,0,0};
\colorlet{c}{kugray};
\draw [c] (9.93518,2.22165) -- (9.93518,2.328);
\draw [c] (9.93518,2.328) -- (9.93518,2.4069);
\draw [c] (9.92036,2.328) -- (9.93518,2.328);
\draw [c] (9.93518,2.328) -- (9.95,2.328);
\definecolor{c}{rgb}{0,0,0};
\colorlet{c}{natcomp};
\draw [c] (1.51655,5.54553) -- (1.60131,5.4045) -- (1.68607,5.27517) -- (1.77083,5.15545) -- (1.85558,5.0438) -- (1.94034,4.93904) -- (2.0251,4.84024) -- (2.10986,4.74665) -- (2.19462,4.65769) -- (2.27938,4.57286) -- (2.36413,4.49176)
 -- (2.44889,4.41407) -- (2.53365,4.33949) -- (2.61841,4.26781) -- (2.70317,4.19881) -- (2.78793,4.13234) -- (2.87268,4.06825) -- (2.95744,4.00643) -- (3.0422,3.94677) -- (3.12696,3.8892) -- (3.21172,3.83363) -- (3.29648,3.78001) -- (3.38123,3.72829)
 -- (3.46599,3.67842) -- (3.55075,3.63037) -- (3.63551,3.58409) -- (3.72027,3.53955) -- (3.80502,3.49673) -- (3.88978,3.45559) -- (3.97454,3.4161) -- (4.0593,3.37822) -- (4.14406,3.34192) -- (4.22882,3.30715) -- (4.31357,3.27388) -- (4.39833,3.24205)
 -- (4.48309,3.21162) -- (4.56785,3.18254) -- (4.65261,3.15475) -- (4.73737,3.12818) -- (4.82212,3.1028) -- (4.90688,3.07853) -- (4.99164,3.05531) -- (5.0764,3.03308) -- (5.16116,3.01179) -- (5.24592,2.99138) -- (5.33067,2.97179) -- (5.41543,2.95296)
 -- (5.50019,2.93484) -- (5.58495,2.91738) -- (5.66971,2.90053);
\draw [c] (5.66971,2.90053) -- (5.75447,2.88424) -- (5.83922,2.86848) -- (5.92398,2.8532) -- (6.00874,2.83837) -- (6.0935,2.82394) -- (6.17826,2.80989) -- (6.26301,2.79618) -- (6.34777,2.7828) -- (6.43253,2.7697) -- (6.51729,2.75687)
 -- (6.60205,2.74428) -- (6.68681,2.73192) -- (6.77156,2.71977) -- (6.85632,2.7078) -- (6.94108,2.69601) -- (7.02584,2.68437) -- (7.1106,2.67289) -- (7.19536,2.66153) -- (7.28011,2.6503) -- (7.36487,2.63918) -- (7.44963,2.62817) -- (7.53439,2.61724)
 -- (7.61915,2.60641) -- (7.70391,2.59566) -- (7.78866,2.58498) -- (7.87342,2.57436) -- (7.95818,2.56381) -- (8.04294,2.55332) -- (8.1277,2.54287) -- (8.21246,2.53247) -- (8.29721,2.52212) -- (8.38197,2.5118) -- (8.46673,2.50152) -- (8.55149,2.49127)
 -- (8.63625,2.48104) -- (8.72101,2.47084) -- (8.80576,2.46065) -- (8.89052,2.45047) -- (8.97528,2.4403) -- (9.06004,2.43013) -- (9.1448,2.41995) -- (9.22955,2.40976) -- (9.31431,2.39954) -- (9.39907,2.38929) -- (9.48383,2.379) -- (9.56859,2.36866)
 -- (9.65335,2.35824) -- (9.7381,2.34774) -- (9.82286,2.33714);
\draw [c] (9.82286,2.33714) -- (9.90762,2.32641);
\colorlet{c}{kugray};
\draw [c] (1.01482,0.596817) -- (1.01482,3.61122);
\draw [c] (1.01482,3.61122) -- (1.01482,3.82461);
\draw [c] (1,3.61122) -- (1.01482,3.61122);
\draw [c] (1.01482,3.61122) -- (1.02964,3.61122);
\definecolor{c}{rgb}{0,0,0};
\colorlet{c}{kugray};
\draw [c] (1.04445,0.596817) -- (1.04445,1.81958);
\draw [c] (1.04445,1.81958) -- (1.04445,2.03297);
\draw [c] (1.02964,1.81958) -- (1.04445,1.81958);
\draw [c] (1.04445,1.81958) -- (1.05927,1.81958);
\definecolor{c}{rgb}{0,0,0};
\colorlet{c}{kugray};
\draw [c] (1.07409,0.596817) -- (1.07409,3.48891);
\draw [c] (1.07409,3.48891) -- (1.07409,3.7023);
\draw [c] (1.05927,3.48891) -- (1.07409,3.48891);
\draw [c] (1.07409,3.48891) -- (1.08891,3.48891);
\definecolor{c}{rgb}{0,0,0};
\colorlet{c}{kugray};
\draw [c] (1.10373,3.60302) -- (1.10373,3.87778);
\draw [c] (1.10373,3.87778) -- (1.10373,4.02061);
\draw [c] (1.08891,3.87778) -- (1.10373,3.87778);
\draw [c] (1.10373,3.87778) -- (1.11854,3.87778);
\definecolor{c}{rgb}{0,0,0};
\colorlet{c}{kugray};
\draw [c] (1.13336,3.42057) -- (1.13336,3.80196);
\draw [c] (1.13336,3.80196) -- (1.13336,3.96717);
\draw [c] (1.11854,3.80196) -- (1.13336,3.80196);
\draw [c] (1.13336,3.80196) -- (1.14818,3.80196);
\definecolor{c}{rgb}{0,0,0};
\colorlet{c}{kugray};
\draw [c] (1.163,0.596817) -- (1.163,2.01527);
\draw [c] (1.163,2.01527) -- (1.163,2.22866);
\draw [c] (1.14818,2.01527) -- (1.163,2.01527);
\draw [c] (1.163,2.01527) -- (1.17781,2.01527);
\definecolor{c}{rgb}{0,0,0};
\colorlet{c}{kugray};
\draw [c] (1.19263,0.596817) -- (1.19263,3.51401);
\draw [c] (1.19263,3.51401) -- (1.19263,3.7274);
\draw [c] (1.17781,3.51401) -- (1.19263,3.51401);
\draw [c] (1.19263,3.51401) -- (1.20745,3.51401);
\definecolor{c}{rgb}{0,0,0};
\colorlet{c}{kugray};
\draw [c] (1.22227,3.43062) -- (1.22227,3.83166);
\draw [c] (1.22227,3.83166) -- (1.22227,4.00008);
\draw [c] (1.20745,3.83166) -- (1.22227,3.83166);
\draw [c] (1.22227,3.83166) -- (1.23709,3.83166);
\definecolor{c}{rgb}{0,0,0};
\colorlet{c}{kugray};
\draw [c] (1.2519,3.8397) -- (1.2519,4.0877);
\draw [c] (1.2519,4.0877) -- (1.2519,4.22325);
\draw [c] (1.23709,4.0877) -- (1.2519,4.0877);
\draw [c] (1.2519,4.0877) -- (1.26672,4.0877);
\definecolor{c}{rgb}{0,0,0};
\colorlet{c}{kugray};
\draw [c] (1.28154,3.98744) -- (1.28154,4.15251);
\draw [c] (1.28154,4.15251) -- (1.28154,4.25938);
\draw [c] (1.26672,4.15251) -- (1.28154,4.15251);
\draw [c] (1.28154,4.15251) -- (1.29636,4.15251);
\definecolor{c}{rgb}{0,0,0};
\colorlet{c}{kugray};
\draw [c] (1.31118,5.3327) -- (1.31118,5.34989);
\draw [c] (1.31118,5.34989) -- (1.31118,5.36617);
\draw [c] (1.29636,5.34989) -- (1.31118,5.34989);
\draw [c] (1.31118,5.34989) -- (1.32599,5.34989);
\definecolor{c}{rgb}{0,0,0};
\colorlet{c}{kugray};
\draw [c] (1.34081,5.6418) -- (1.34081,5.6522);
\draw [c] (1.34081,5.6522) -- (1.34081,5.66227);
\draw [c] (1.32599,5.6522) -- (1.34081,5.6522);
\draw [c] (1.34081,5.6522) -- (1.35563,5.6522);
\definecolor{c}{rgb}{0,0,0};
\colorlet{c}{kugray};
\draw [c] (1.37045,5.68448) -- (1.37045,5.69437);
\draw [c] (1.37045,5.69437) -- (1.37045,5.70394);
\draw [c] (1.35563,5.69437) -- (1.37045,5.69437);
\draw [c] (1.37045,5.69437) -- (1.38526,5.69437);
\definecolor{c}{rgb}{0,0,0};
\colorlet{c}{kugray};
\draw [c] (1.40008,5.69232) -- (1.40008,5.70203);
\draw [c] (1.40008,5.70203) -- (1.40008,5.71145);
\draw [c] (1.38526,5.70203) -- (1.40008,5.70203);
\draw [c] (1.40008,5.70203) -- (1.4149,5.70203);
\definecolor{c}{rgb}{0,0,0};
\colorlet{c}{kugray};
\draw [c] (1.42972,5.65625) -- (1.42972,5.66643);
\draw [c] (1.42972,5.66643) -- (1.42972,5.67629);
\draw [c] (1.4149,5.66643) -- (1.42972,5.66643);
\draw [c] (1.42972,5.66643) -- (1.44454,5.66643);
\definecolor{c}{rgb}{0,0,0};
\colorlet{c}{kugray};
\draw [c] (1.45935,5.61758) -- (1.45935,5.62829);
\draw [c] (1.45935,5.62829) -- (1.45935,5.63863);
\draw [c] (1.44454,5.62829) -- (1.45935,5.62829);
\draw [c] (1.45935,5.62829) -- (1.47417,5.62829);
\definecolor{c}{rgb}{0,0,0};
\colorlet{c}{kugray};
\draw [c] (1.48899,5.57445) -- (1.48899,5.58624);
\draw [c] (1.48899,5.58624) -- (1.48899,5.5976);
\draw [c] (1.47417,5.58624) -- (1.48899,5.58624);
\draw [c] (1.48899,5.58624) -- (1.50381,5.58624);
\definecolor{c}{rgb}{0,0,0};
\colorlet{c}{kugray};
\draw [c] (1.51863,5.53272) -- (1.51863,5.54514);
\draw [c] (1.51863,5.54514) -- (1.51863,5.55708);
\draw [c] (1.50381,5.54514) -- (1.51863,5.54514);
\draw [c] (1.51863,5.54514) -- (1.53344,5.54514);
\definecolor{c}{rgb}{0,0,0};
\colorlet{c}{kugray};
\draw [c] (1.54826,5.48596) -- (1.54826,5.49969);
\draw [c] (1.54826,5.49969) -- (1.54826,5.51284);
\draw [c] (1.53344,5.49969) -- (1.54826,5.49969);
\draw [c] (1.54826,5.49969) -- (1.56308,5.49969);
\definecolor{c}{rgb}{0,0,0};
\colorlet{c}{kugray};
\draw [c] (1.5779,5.4138) -- (1.5779,5.42885);
\draw [c] (1.5779,5.42885) -- (1.5779,5.44319);
\draw [c] (1.56308,5.42885) -- (1.5779,5.42885);
\draw [c] (1.5779,5.42885) -- (1.59272,5.42885);
\definecolor{c}{rgb}{0,0,0};
\colorlet{c}{kugray};
\draw [c] (1.60753,5.41345) -- (1.60753,5.42901);
\draw [c] (1.60753,5.42901) -- (1.60753,5.44382);
\draw [c] (1.59272,5.42901) -- (1.60753,5.42901);
\draw [c] (1.60753,5.42901) -- (1.62235,5.42901);
\definecolor{c}{rgb}{0,0,0};
\colorlet{c}{kugray};
\draw [c] (1.63717,5.34137) -- (1.63717,5.35821);
\draw [c] (1.63717,5.35821) -- (1.63717,5.37418);
\draw [c] (1.62235,5.35821) -- (1.63717,5.35821);
\draw [c] (1.63717,5.35821) -- (1.65199,5.35821);
\definecolor{c}{rgb}{0,0,0};
\colorlet{c}{kugray};
\draw [c] (1.6668,5.31592) -- (1.6668,5.33358);
\draw [c] (1.6668,5.33358) -- (1.6668,5.35028);
\draw [c] (1.65199,5.33358) -- (1.6668,5.33358);
\draw [c] (1.6668,5.33358) -- (1.68162,5.33358);
\definecolor{c}{rgb}{0,0,0};
\colorlet{c}{kugray};
\draw [c] (1.69644,5.25517) -- (1.69644,5.2753);
\draw [c] (1.69644,5.2753) -- (1.69644,5.2942);
\draw [c] (1.68162,5.2753) -- (1.69644,5.2753);
\draw [c] (1.69644,5.2753) -- (1.71126,5.2753);
\definecolor{c}{rgb}{0,0,0};
\colorlet{c}{kugray};
\draw [c] (1.72608,5.20002) -- (1.72608,5.2216);
\draw [c] (1.72608,5.2216) -- (1.72608,5.24177);
\draw [c] (1.71126,5.2216) -- (1.72608,5.2216);
\draw [c] (1.72608,5.2216) -- (1.74089,5.2216);
\definecolor{c}{rgb}{0,0,0};
\colorlet{c}{kugray};
\draw [c] (1.75571,5.14158) -- (1.75571,5.16518);
\draw [c] (1.75571,5.16518) -- (1.75571,5.18709);
\draw [c] (1.74089,5.16518) -- (1.75571,5.16518);
\draw [c] (1.75571,5.16518) -- (1.77053,5.16518);
\definecolor{c}{rgb}{0,0,0};
\colorlet{c}{kugray};
\draw [c] (1.78535,5.11657) -- (1.78535,5.14067);
\draw [c] (1.78535,5.14067) -- (1.78535,5.16302);
\draw [c] (1.77053,5.14067) -- (1.78535,5.14067);
\draw [c] (1.78535,5.14067) -- (1.80017,5.14067);
\definecolor{c}{rgb}{0,0,0};
\colorlet{c}{kugray};
\draw [c] (1.81498,5.10912) -- (1.81498,5.13446);
\draw [c] (1.81498,5.13446) -- (1.81498,5.15788);
\draw [c] (1.80017,5.13446) -- (1.81498,5.13446);
\draw [c] (1.81498,5.13446) -- (1.8298,5.13446);
\definecolor{c}{rgb}{0,0,0};
\colorlet{c}{kugray};
\draw [c] (1.84462,5.02683) -- (1.84462,5.05542);
\draw [c] (1.84462,5.05542) -- (1.84462,5.08158);
\draw [c] (1.8298,5.05542) -- (1.84462,5.05542);
\draw [c] (1.84462,5.05542) -- (1.85944,5.05542);
\definecolor{c}{rgb}{0,0,0};
\colorlet{c}{kugray};
\draw [c] (1.87425,4.99299) -- (1.87425,5.02669);
\draw [c] (1.87425,5.02669) -- (1.87425,5.05705);
\draw [c] (1.85944,5.02669) -- (1.87425,5.02669);
\draw [c] (1.87425,5.02669) -- (1.88907,5.02669);
\definecolor{c}{rgb}{0,0,0};
\colorlet{c}{kugray};
\draw [c] (1.90389,4.94322) -- (1.90389,4.97626);
\draw [c] (1.90389,4.97626) -- (1.90389,5.00609);
\draw [c] (1.88907,4.97626) -- (1.90389,4.97626);
\draw [c] (1.90389,4.97626) -- (1.91871,4.97626);
\definecolor{c}{rgb}{0,0,0};
\colorlet{c}{kugray};
\draw [c] (1.93353,4.96266) -- (1.93353,4.9937);
\draw [c] (1.93353,4.9937) -- (1.93353,5.02189);
\draw [c] (1.91871,4.9937) -- (1.93353,4.9937);
\draw [c] (1.93353,4.9937) -- (1.94834,4.9937);
\definecolor{c}{rgb}{0,0,0};
\colorlet{c}{kugray};
\draw [c] (1.96316,4.86451) -- (1.96316,4.90192);
\draw [c] (1.96316,4.90192) -- (1.96316,4.93528);
\draw [c] (1.94834,4.90192) -- (1.96316,4.90192);
\draw [c] (1.96316,4.90192) -- (1.97798,4.90192);
\definecolor{c}{rgb}{0,0,0};
\colorlet{c}{kugray};
\draw [c] (1.9928,4.89806) -- (1.9928,4.93394);
\draw [c] (1.9928,4.93394) -- (1.9928,4.96608);
\draw [c] (1.97798,4.93394) -- (1.9928,4.93394);
\draw [c] (1.9928,4.93394) -- (2.00762,4.93394);
\definecolor{c}{rgb}{0,0,0};
\colorlet{c}{kugray};
\draw [c] (2.02243,4.68898) -- (2.02243,4.73645);
\draw [c] (2.02243,4.73645) -- (2.02243,4.77757);
\draw [c] (2.00762,4.73645) -- (2.02243,4.73645);
\draw [c] (2.02243,4.73645) -- (2.03725,4.73645);
\definecolor{c}{rgb}{0,0,0};
\colorlet{c}{kugray};
\draw [c] (2.05207,4.74386) -- (2.05207,4.78796);
\draw [c] (2.05207,4.78796) -- (2.05207,4.82652);
\draw [c] (2.03725,4.78796) -- (2.05207,4.78796);
\draw [c] (2.05207,4.78796) -- (2.06689,4.78796);
\definecolor{c}{rgb}{0,0,0};
\colorlet{c}{kugray};
\draw [c] (2.08171,4.73578) -- (2.08171,4.78026);
\draw [c] (2.08171,4.78026) -- (2.08171,4.81912);
\draw [c] (2.06689,4.78026) -- (2.08171,4.78026);
\draw [c] (2.08171,4.78026) -- (2.09652,4.78026);
\definecolor{c}{rgb}{0,0,0};
\colorlet{c}{kugray};
\draw [c] (2.11134,4.6862) -- (2.11134,4.73798);
\draw [c] (2.11134,4.73798) -- (2.11134,4.7823);
\draw [c] (2.09652,4.73798) -- (2.11134,4.73798);
\draw [c] (2.11134,4.73798) -- (2.12616,4.73798);
\definecolor{c}{rgb}{0,0,0};
\colorlet{c}{kugray};
\draw [c] (2.14098,4.54138) -- (2.14098,4.6098);
\draw [c] (2.14098,4.6098) -- (2.14098,4.66574);
\draw [c] (2.12616,4.6098) -- (2.14098,4.6098);
\draw [c] (2.14098,4.6098) -- (2.15579,4.6098);
\definecolor{c}{rgb}{0,0,0};
\colorlet{c}{kugray};
\draw [c] (2.17061,4.64226) -- (2.17061,4.69994);
\draw [c] (2.17061,4.69994) -- (2.17061,4.74849);
\draw [c] (2.15579,4.69994) -- (2.17061,4.69994);
\draw [c] (2.17061,4.69994) -- (2.18543,4.69994);
\definecolor{c}{rgb}{0,0,0};
\colorlet{c}{kugray};
\draw [c] (2.20025,4.53677) -- (2.20025,4.60067);
\draw [c] (2.20025,4.60067) -- (2.20025,4.65356);
\draw [c] (2.18543,4.60067) -- (2.20025,4.60067);
\draw [c] (2.20025,4.60067) -- (2.21507,4.60067);
\definecolor{c}{rgb}{0,0,0};
\colorlet{c}{kugray};
\draw [c] (2.22988,4.61221) -- (2.22988,4.66583);
\draw [c] (2.22988,4.66583) -- (2.22988,4.71148);
\draw [c] (2.21507,4.66583) -- (2.22988,4.66583);
\draw [c] (2.22988,4.66583) -- (2.2447,4.66583);
\definecolor{c}{rgb}{0,0,0};
\colorlet{c}{kugray};
\draw [c] (2.25952,4.53829) -- (2.25952,4.60341);
\draw [c] (2.25952,4.60341) -- (2.25952,4.65714);
\draw [c] (2.2447,4.60341) -- (2.25952,4.60341);
\draw [c] (2.25952,4.60341) -- (2.27434,4.60341);
\definecolor{c}{rgb}{0,0,0};
\colorlet{c}{kugray};
\draw [c] (2.28916,4.56974) -- (2.28916,4.58026);
\draw [c] (2.28916,4.58026) -- (2.28916,4.59044);
\draw [c] (2.27434,4.58026) -- (2.28916,4.58026);
\draw [c] (2.28916,4.58026) -- (2.30397,4.58026);
\definecolor{c}{rgb}{0,0,0};
\colorlet{c}{kugray};
\draw [c] (2.31879,4.53602) -- (2.31879,4.54723);
\draw [c] (2.31879,4.54723) -- (2.31879,4.55804);
\draw [c] (2.30397,4.54723) -- (2.31879,4.54723);
\draw [c] (2.31879,4.54723) -- (2.33361,4.54723);
\definecolor{c}{rgb}{0,0,0};
\colorlet{c}{kugray};
\draw [c] (2.34843,4.52541) -- (2.34843,4.53687);
\draw [c] (2.34843,4.53687) -- (2.34843,4.54792);
\draw [c] (2.33361,4.53687) -- (2.34843,4.53687);
\draw [c] (2.34843,4.53687) -- (2.36325,4.53687);
\definecolor{c}{rgb}{0,0,0};
\colorlet{c}{kugray};
\draw [c] (2.37806,4.47644) -- (2.37806,4.48842);
\draw [c] (2.37806,4.48842) -- (2.37806,4.49994);
\draw [c] (2.36325,4.48842) -- (2.37806,4.48842);
\draw [c] (2.37806,4.48842) -- (2.39288,4.48842);
\definecolor{c}{rgb}{0,0,0};
\colorlet{c}{kugray};
\draw [c] (2.4077,4.45724) -- (2.4077,4.46981);
\draw [c] (2.4077,4.46981) -- (2.4077,4.48188);
\draw [c] (2.39288,4.46981) -- (2.4077,4.46981);
\draw [c] (2.4077,4.46981) -- (2.42252,4.46981);
\definecolor{c}{rgb}{0,0,0};
\colorlet{c}{kugray};
\draw [c] (2.43733,4.43036) -- (2.43733,4.44342);
\draw [c] (2.43733,4.44342) -- (2.43733,4.45594);
\draw [c] (2.42252,4.44342) -- (2.43733,4.44342);
\draw [c] (2.43733,4.44342) -- (2.45215,4.44342);
\definecolor{c}{rgb}{0,0,0};
\colorlet{c}{kugray};
\draw [c] (2.46697,4.40509) -- (2.46697,4.41861);
\draw [c] (2.46697,4.41861) -- (2.46697,4.43156);
\draw [c] (2.45215,4.41861) -- (2.46697,4.41861);
\draw [c] (2.46697,4.41861) -- (2.48179,4.41861);
\definecolor{c}{rgb}{0,0,0};
\colorlet{c}{kugray};
\draw [c] (2.49661,4.37004) -- (2.49661,4.38467);
\draw [c] (2.49661,4.38467) -- (2.49661,4.39863);
\draw [c] (2.48179,4.38467) -- (2.49661,4.38467);
\draw [c] (2.49661,4.38467) -- (2.51142,4.38467);
\definecolor{c}{rgb}{0,0,0};
\colorlet{c}{kugray};
\draw [c] (2.52624,4.32084) -- (2.52624,4.33659);
\draw [c] (2.52624,4.33659) -- (2.52624,4.35157);
\draw [c] (2.51142,4.33659) -- (2.52624,4.33659);
\draw [c] (2.52624,4.33659) -- (2.54106,4.33659);
\definecolor{c}{rgb}{0,0,0};
\colorlet{c}{kugray};
\draw [c] (2.55588,4.30399) -- (2.55588,4.32008);
\draw [c] (2.55588,4.32008) -- (2.55588,4.33538);
\draw [c] (2.54106,4.32008) -- (2.55588,4.32008);
\draw [c] (2.55588,4.32008) -- (2.5707,4.32008);
\definecolor{c}{rgb}{0,0,0};
\colorlet{c}{kugray};
\draw [c] (2.58551,4.30302) -- (2.58551,4.31916);
\draw [c] (2.58551,4.31916) -- (2.58551,4.3345);
\draw [c] (2.5707,4.31916) -- (2.58551,4.31916);
\draw [c] (2.58551,4.31916) -- (2.60033,4.31916);
\definecolor{c}{rgb}{0,0,0};
\colorlet{c}{kugray};
\draw [c] (2.61515,4.25843) -- (2.61515,4.27581);
\draw [c] (2.61515,4.27581) -- (2.61515,4.29226);
\draw [c] (2.60033,4.27581) -- (2.61515,4.27581);
\draw [c] (2.61515,4.27581) -- (2.62997,4.27581);
\definecolor{c}{rgb}{0,0,0};
\colorlet{c}{kugray};
\draw [c] (2.64478,4.24047) -- (2.64478,4.25814);
\draw [c] (2.64478,4.25814) -- (2.64478,4.27485);
\draw [c] (2.62997,4.25814) -- (2.64478,4.25814);
\draw [c] (2.64478,4.25814) -- (2.6596,4.25814);
\definecolor{c}{rgb}{0,0,0};
\colorlet{c}{kugray};
\draw [c] (2.67442,4.23829) -- (2.67442,4.2564);
\draw [c] (2.67442,4.2564) -- (2.67442,4.27351);
\draw [c] (2.6596,4.2564) -- (2.67442,4.2564);
\draw [c] (2.67442,4.2564) -- (2.68924,4.2564);
\definecolor{c}{rgb}{0,0,0};
\colorlet{c}{kugray};
\draw [c] (2.70406,4.23694) -- (2.70406,4.25493);
\draw [c] (2.70406,4.25493) -- (2.70406,4.27193);
\draw [c] (2.68924,4.25493) -- (2.70406,4.25493);
\draw [c] (2.70406,4.25493) -- (2.71887,4.25493);
\definecolor{c}{rgb}{0,0,0};
\colorlet{c}{kugray};
\draw [c] (2.73369,4.17951) -- (2.73369,4.19937);
\draw [c] (2.73369,4.19937) -- (2.73369,4.21803);
\draw [c] (2.71887,4.19937) -- (2.73369,4.19937);
\draw [c] (2.73369,4.19937) -- (2.74851,4.19937);
\definecolor{c}{rgb}{0,0,0};
\colorlet{c}{kugray};
\draw [c] (2.76333,4.15517) -- (2.76333,4.17566);
\draw [c] (2.76333,4.17566) -- (2.76333,4.19487);
\draw [c] (2.74851,4.17566) -- (2.76333,4.17566);
\draw [c] (2.76333,4.17566) -- (2.77815,4.17566);
\definecolor{c}{rgb}{0,0,0};
\colorlet{c}{kugray};
\draw [c] (2.79296,4.13183) -- (2.79296,4.15328);
\draw [c] (2.79296,4.15328) -- (2.79296,4.17334);
\draw [c] (2.77815,4.15328) -- (2.79296,4.15328);
\draw [c] (2.79296,4.15328) -- (2.80778,4.15328);
\definecolor{c}{rgb}{0,0,0};
\colorlet{c}{kugray};
\draw [c] (2.8226,4.09575) -- (2.8226,4.11831);
\draw [c] (2.8226,4.11831) -- (2.8226,4.13932);
\draw [c] (2.80778,4.11831) -- (2.8226,4.11831);
\draw [c] (2.8226,4.11831) -- (2.83742,4.11831);
\definecolor{c}{rgb}{0,0,0};
\colorlet{c}{kugray};
\draw [c] (2.85224,4.0827) -- (2.85224,4.10635);
\draw [c] (2.85224,4.10635) -- (2.85224,4.12832);
\draw [c] (2.83742,4.10635) -- (2.85224,4.10635);
\draw [c] (2.85224,4.10635) -- (2.86705,4.10635);
\definecolor{c}{rgb}{0,0,0};
\colorlet{c}{kugray};
\draw [c] (2.88187,4.07789) -- (2.88187,4.1009);
\draw [c] (2.88187,4.1009) -- (2.88187,4.12231);
\draw [c] (2.86705,4.1009) -- (2.88187,4.1009);
\draw [c] (2.88187,4.1009) -- (2.89669,4.1009);
\definecolor{c}{rgb}{0,0,0};
\colorlet{c}{kugray};
\draw [c] (2.91151,4.04064) -- (2.91151,4.06493);
\draw [c] (2.91151,4.06493) -- (2.91151,4.08745);
\draw [c] (2.89669,4.06493) -- (2.91151,4.06493);
\draw [c] (2.91151,4.06493) -- (2.92632,4.06493);
\definecolor{c}{rgb}{0,0,0};
\colorlet{c}{kugray};
\draw [c] (2.94114,3.99628) -- (2.94114,4.02251);
\draw [c] (2.94114,4.02251) -- (2.94114,4.04669);
\draw [c] (2.92632,4.02251) -- (2.94114,4.02251);
\draw [c] (2.94114,4.02251) -- (2.95596,4.02251);
\definecolor{c}{rgb}{0,0,0};
\colorlet{c}{kugray};
\draw [c] (2.97078,4.01184) -- (2.97078,4.0375);
\draw [c] (2.97078,4.0375) -- (2.97078,4.06119);
\draw [c] (2.95596,4.0375) -- (2.97078,4.0375);
\draw [c] (2.97078,4.0375) -- (2.9856,4.0375);
\definecolor{c}{rgb}{0,0,0};
\colorlet{c}{kugray};
\draw [c] (3.00041,3.96161) -- (3.00041,3.99102);
\draw [c] (3.00041,3.99102) -- (3.00041,4.01787);
\draw [c] (2.9856,3.99102) -- (3.00041,3.99102);
\draw [c] (3.00041,3.99102) -- (3.01523,3.99102);
\definecolor{c}{rgb}{0,0,0};
\colorlet{c}{kugray};
\draw [c] (3.03005,3.96889) -- (3.03005,3.99707);
\draw [c] (3.03005,3.99707) -- (3.03005,4.02289);
\draw [c] (3.01523,3.99707) -- (3.03005,3.99707);
\draw [c] (3.03005,3.99707) -- (3.04487,3.99707);
\definecolor{c}{rgb}{0,0,0};
\colorlet{c}{kugray};
\draw [c] (3.05969,3.98359) -- (3.05969,4.01138);
\draw [c] (3.05969,4.01138) -- (3.05969,4.03686);
\draw [c] (3.04487,4.01138) -- (3.05969,4.01138);
\draw [c] (3.05969,4.01138) -- (3.0745,4.01138);
\definecolor{c}{rgb}{0,0,0};
\colorlet{c}{kugray};
\draw [c] (3.08932,3.94149) -- (3.08932,3.97026);
\draw [c] (3.08932,3.97026) -- (3.08932,3.99657);
\draw [c] (3.0745,3.97026) -- (3.08932,3.97026);
\draw [c] (3.08932,3.97026) -- (3.10414,3.97026);
\definecolor{c}{rgb}{0,0,0};
\colorlet{c}{kugray};
\draw [c] (3.11896,3.91737) -- (3.11896,3.94858);
\draw [c] (3.11896,3.94858) -- (3.11896,3.97692);
\draw [c] (3.10414,3.94858) -- (3.11896,3.94858);
\draw [c] (3.11896,3.94858) -- (3.13377,3.94858);
\definecolor{c}{rgb}{0,0,0};
\colorlet{c}{kugray};
\draw [c] (3.14859,3.9151) -- (3.14859,3.945);
\draw [c] (3.14859,3.945) -- (3.14859,3.97226);
\draw [c] (3.13377,3.945) -- (3.14859,3.945);
\draw [c] (3.14859,3.945) -- (3.16341,3.945);
\definecolor{c}{rgb}{0,0,0};
\colorlet{c}{kugray};
\draw [c] (3.17823,3.85367) -- (3.17823,3.88787);
\draw [c] (3.17823,3.88787) -- (3.17823,3.91865);
\draw [c] (3.16341,3.88787) -- (3.17823,3.88787);
\draw [c] (3.17823,3.88787) -- (3.19305,3.88787);
\definecolor{c}{rgb}{0,0,0};
\colorlet{c}{kugray};
\draw [c] (3.20786,3.83015) -- (3.20786,3.86574);
\draw [c] (3.20786,3.86574) -- (3.20786,3.89765);
\draw [c] (3.19305,3.86574) -- (3.20786,3.86574);
\draw [c] (3.20786,3.86574) -- (3.22268,3.86574);
\definecolor{c}{rgb}{0,0,0};
\colorlet{c}{kugray};
\draw [c] (3.2375,3.83715) -- (3.2375,3.87247);
\draw [c] (3.2375,3.87247) -- (3.2375,3.90415);
\draw [c] (3.22268,3.87247) -- (3.2375,3.87247);
\draw [c] (3.2375,3.87247) -- (3.25232,3.87247);
\definecolor{c}{rgb}{0,0,0};
\colorlet{c}{kugray};
\draw [c] (3.26714,3.73623) -- (3.26714,3.77648);
\draw [c] (3.26714,3.77648) -- (3.26714,3.81206);
\draw [c] (3.25232,3.77648) -- (3.26714,3.77648);
\draw [c] (3.26714,3.77648) -- (3.28195,3.77648);
\definecolor{c}{rgb}{0,0,0};
\colorlet{c}{kugray};
\draw [c] (3.29677,3.80384) -- (3.29677,3.84266);
\draw [c] (3.29677,3.84266) -- (3.29677,3.87713);
\draw [c] (3.28195,3.84266) -- (3.29677,3.84266);
\draw [c] (3.29677,3.84266) -- (3.31159,3.84266);
\definecolor{c}{rgb}{0,0,0};
\colorlet{c}{kugray};
\draw [c] (3.32641,3.76887) -- (3.32641,3.80715);
\draw [c] (3.32641,3.80715) -- (3.32641,3.84119);
\draw [c] (3.31159,3.80715) -- (3.32641,3.80715);
\draw [c] (3.32641,3.80715) -- (3.34123,3.80715);
\definecolor{c}{rgb}{0,0,0};
\colorlet{c}{kugray};
\draw [c] (3.35604,3.75484) -- (3.35604,3.79665);
\draw [c] (3.35604,3.79665) -- (3.35604,3.83345);
\draw [c] (3.34123,3.79665) -- (3.35604,3.79665);
\draw [c] (3.35604,3.79665) -- (3.37086,3.79665);
\definecolor{c}{rgb}{0,0,0};
\colorlet{c}{kugray};
\draw [c] (3.38568,3.74591) -- (3.38568,3.7847);
\draw [c] (3.38568,3.7847) -- (3.38568,3.81915);
\draw [c] (3.37086,3.7847) -- (3.38568,3.7847);
\draw [c] (3.38568,3.7847) -- (3.4005,3.7847);
\definecolor{c}{rgb}{0,0,0};
\colorlet{c}{kugray};
\draw [c] (3.41531,3.79899) -- (3.41531,3.83877);
\draw [c] (3.41531,3.83877) -- (3.41531,3.874);
\draw [c] (3.4005,3.83877) -- (3.41531,3.83877);
\draw [c] (3.41531,3.83877) -- (3.43013,3.83877);
\definecolor{c}{rgb}{0,0,0};
\colorlet{c}{kugray};
\draw [c] (3.44495,3.68988) -- (3.44495,3.73418);
\draw [c] (3.44495,3.73418) -- (3.44495,3.7729);
\draw [c] (3.43013,3.73418) -- (3.44495,3.73418);
\draw [c] (3.44495,3.73418) -- (3.45977,3.73418);
\definecolor{c}{rgb}{0,0,0};
\colorlet{c}{kugray};
\draw [c] (3.47459,3.63044) -- (3.47459,3.67993);
\draw [c] (3.47459,3.67993) -- (3.47459,3.72255);
\draw [c] (3.45977,3.67993) -- (3.47459,3.67993);
\draw [c] (3.47459,3.67993) -- (3.4894,3.67993);
\definecolor{c}{rgb}{0,0,0};
\colorlet{c}{kugray};
\draw [c] (3.50422,3.78508) -- (3.50422,3.82542);
\draw [c] (3.50422,3.82542) -- (3.50422,3.86107);
\draw [c] (3.4894,3.82542) -- (3.50422,3.82542);
\draw [c] (3.50422,3.82542) -- (3.51904,3.82542);
\definecolor{c}{rgb}{0,0,0};
\colorlet{c}{kugray};
\draw [c] (3.53386,3.72102) -- (3.53386,3.76148);
\draw [c] (3.53386,3.76148) -- (3.53386,3.79724);
\draw [c] (3.51904,3.76148) -- (3.53386,3.76148);
\draw [c] (3.53386,3.76148) -- (3.54868,3.76148);
\definecolor{c}{rgb}{0,0,0};
\colorlet{c}{kugray};
\draw [c] (3.56349,3.66874) -- (3.56349,3.71592);
\draw [c] (3.56349,3.71592) -- (3.56349,3.75683);
\draw [c] (3.54868,3.71592) -- (3.56349,3.71592);
\draw [c] (3.56349,3.71592) -- (3.57831,3.71592);
\definecolor{c}{rgb}{0,0,0};
\colorlet{c}{kugray};
\draw [c] (3.59313,3.62349) -- (3.59313,3.66941);
\draw [c] (3.59313,3.66941) -- (3.59313,3.70935);
\draw [c] (3.57831,3.66941) -- (3.59313,3.66941);
\draw [c] (3.59313,3.66941) -- (3.60795,3.66941);
\definecolor{c}{rgb}{0,0,0};
\colorlet{c}{kugray};
\draw [c] (3.62276,3.72066) -- (3.62276,3.7665);
\draw [c] (3.62276,3.7665) -- (3.62276,3.80639);
\draw [c] (3.60795,3.7665) -- (3.62276,3.7665);
\draw [c] (3.62276,3.7665) -- (3.63758,3.7665);
\definecolor{c}{rgb}{0,0,0};
\colorlet{c}{kugray};
\draw [c] (3.6524,3.60108) -- (3.6524,3.65117);
\draw [c] (3.6524,3.65117) -- (3.6524,3.69424);
\draw [c] (3.63758,3.65117) -- (3.6524,3.65117);
\draw [c] (3.6524,3.65117) -- (3.66722,3.65117);
\definecolor{c}{rgb}{0,0,0};
\colorlet{c}{kugray};
\draw [c] (3.68204,3.59622) -- (3.68204,3.64503);
\draw [c] (3.68204,3.64503) -- (3.68204,3.68715);
\draw [c] (3.66722,3.64503) -- (3.68204,3.64503);
\draw [c] (3.68204,3.64503) -- (3.69685,3.64503);
\definecolor{c}{rgb}{0,0,0};
\colorlet{c}{kugray};
\draw [c] (3.71167,3.61286) -- (3.71167,3.66979);
\draw [c] (3.71167,3.66979) -- (3.71167,3.71782);
\draw [c] (3.69685,3.66979) -- (3.71167,3.66979);
\draw [c] (3.71167,3.66979) -- (3.72649,3.66979);
\definecolor{c}{rgb}{0,0,0};
\colorlet{c}{kugray};
\draw [c] (3.74131,3.57945) -- (3.74131,3.62832);
\draw [c] (3.74131,3.62832) -- (3.74131,3.67048);
\draw [c] (3.72649,3.62832) -- (3.74131,3.62832);
\draw [c] (3.74131,3.62832) -- (3.75613,3.62832);
\definecolor{c}{rgb}{0,0,0};
\colorlet{c}{kugray};
\draw [c] (3.77094,3.5864) -- (3.77094,3.6364);
\draw [c] (3.77094,3.6364) -- (3.77094,3.67939);
\draw [c] (3.75613,3.6364) -- (3.77094,3.6364);
\draw [c] (3.77094,3.6364) -- (3.78576,3.6364);
\definecolor{c}{rgb}{0,0,0};
\colorlet{c}{kugray};
\draw [c] (3.80058,3.58352) -- (3.80058,3.63583);
\draw [c] (3.80058,3.63583) -- (3.80058,3.68052);
\draw [c] (3.78576,3.63583) -- (3.80058,3.63583);
\draw [c] (3.80058,3.63583) -- (3.8154,3.63583);
\definecolor{c}{rgb}{0,0,0};
\colorlet{c}{kugray};
\draw [c] (3.83022,3.54338) -- (3.83022,3.60015);
\draw [c] (3.83022,3.60015) -- (3.83022,3.64806);
\draw [c] (3.8154,3.60015) -- (3.83022,3.60015);
\draw [c] (3.83022,3.60015) -- (3.84503,3.60015);
\definecolor{c}{rgb}{0,0,0};
\colorlet{c}{kugray};
\draw [c] (3.85985,3.54491) -- (3.85985,3.59885);
\draw [c] (3.85985,3.59885) -- (3.85985,3.64473);
\draw [c] (3.84503,3.59885) -- (3.85985,3.59885);
\draw [c] (3.85985,3.59885) -- (3.87467,3.59885);
\definecolor{c}{rgb}{0,0,0};
\colorlet{c}{kugray};
\draw [c] (3.88949,3.51526) -- (3.88949,3.57482);
\draw [c] (3.88949,3.57482) -- (3.88949,3.62471);
\draw [c] (3.87467,3.57482) -- (3.88949,3.57482);
\draw [c] (3.88949,3.57482) -- (3.9043,3.57482);
\definecolor{c}{rgb}{0,0,0};
\colorlet{c}{kugray};
\draw [c] (3.91912,3.60926) -- (3.91912,3.65245);
\draw [c] (3.91912,3.65245) -- (3.91912,3.69032);
\draw [c] (3.9043,3.65245) -- (3.91912,3.65245);
\draw [c] (3.91912,3.65245) -- (3.93394,3.65245);
\definecolor{c}{rgb}{0,0,0};
\colorlet{c}{kugray};
\draw [c] (3.94876,3.62613) -- (3.94876,3.66623);
\draw [c] (3.94876,3.66623) -- (3.94876,3.7017);
\draw [c] (3.93394,3.66623) -- (3.94876,3.66623);
\draw [c] (3.94876,3.66623) -- (3.96358,3.66623);
\definecolor{c}{rgb}{0,0,0};
\colorlet{c}{kugray};
\draw [c] (3.97839,3.57716) -- (3.97839,3.60885);
\draw [c] (3.97839,3.60885) -- (3.97839,3.63758);
\draw [c] (3.96358,3.60885) -- (3.97839,3.60885);
\draw [c] (3.97839,3.60885) -- (3.99321,3.60885);
\definecolor{c}{rgb}{0,0,0};
\colorlet{c}{kugray};
\draw [c] (4.00803,3.51864) -- (4.00803,3.54777);
\draw [c] (4.00803,3.54777) -- (4.00803,3.57438);
\draw [c] (3.99321,3.54777) -- (4.00803,3.54777);
\draw [c] (4.00803,3.54777) -- (4.02285,3.54777);
\definecolor{c}{rgb}{0,0,0};
\colorlet{c}{kugray};
\draw [c] (4.03767,3.54945) -- (4.03767,3.5713);
\draw [c] (4.03767,3.5713) -- (4.03767,3.59171);
\draw [c] (4.02285,3.5713) -- (4.03767,3.5713);
\draw [c] (4.03767,3.5713) -- (4.05248,3.5713);
\definecolor{c}{rgb}{0,0,0};
\colorlet{c}{kugray};
\draw [c] (4.0673,3.52301) -- (4.0673,3.54604);
\draw [c] (4.0673,3.54604) -- (4.0673,3.56746);
\draw [c] (4.05248,3.54604) -- (4.0673,3.54604);
\draw [c] (4.0673,3.54604) -- (4.08212,3.54604);
\definecolor{c}{rgb}{0,0,0};
\colorlet{c}{kugray};
\draw [c] (4.09694,3.48934) -- (4.09694,3.51203);
\draw [c] (4.09694,3.51203) -- (4.09694,3.53317);
\draw [c] (4.08212,3.51203) -- (4.09694,3.51203);
\draw [c] (4.09694,3.51203) -- (4.11175,3.51203);
\definecolor{c}{rgb}{0,0,0};
\colorlet{c}{kugray};
\draw [c] (4.12657,3.50377) -- (4.12657,3.52636);
\draw [c] (4.12657,3.52636) -- (4.12657,3.5474);
\draw [c] (4.11175,3.52636) -- (4.12657,3.52636);
\draw [c] (4.12657,3.52636) -- (4.14139,3.52636);
\definecolor{c}{rgb}{0,0,0};
\colorlet{c}{kugray};
\draw [c] (4.15621,3.51321) -- (4.15621,3.53483);
\draw [c] (4.15621,3.53483) -- (4.15621,3.55503);
\draw [c] (4.14139,3.53483) -- (4.15621,3.53483);
\draw [c] (4.15621,3.53483) -- (4.17103,3.53483);
\definecolor{c}{rgb}{0,0,0};
\colorlet{c}{kugray};
\draw [c] (4.18584,3.5001) -- (4.18584,3.52281);
\draw [c] (4.18584,3.52281) -- (4.18584,3.54395);
\draw [c] (4.17103,3.52281) -- (4.18584,3.52281);
\draw [c] (4.18584,3.52281) -- (4.20066,3.52281);
\definecolor{c}{rgb}{0,0,0};
\colorlet{c}{kugray};
\draw [c] (4.21548,3.46277) -- (4.21548,3.48687);
\draw [c] (4.21548,3.48687) -- (4.21548,3.50921);
\draw [c] (4.20066,3.48687) -- (4.21548,3.48687);
\draw [c] (4.21548,3.48687) -- (4.2303,3.48687);
\definecolor{c}{rgb}{0,0,0};
\colorlet{c}{kugray};
\draw [c] (4.24512,3.50438) -- (4.24512,3.52639);
\draw [c] (4.24512,3.52639) -- (4.24512,3.54694);
\draw [c] (4.2303,3.52639) -- (4.24512,3.52639);
\draw [c] (4.24512,3.52639) -- (4.25993,3.52639);
\definecolor{c}{rgb}{0,0,0};
\colorlet{c}{kugray};
\draw [c] (4.27475,3.47759) -- (4.27475,3.50131);
\draw [c] (4.27475,3.50131) -- (4.27475,3.52334);
\draw [c] (4.25993,3.50131) -- (4.27475,3.50131);
\draw [c] (4.27475,3.50131) -- (4.28957,3.50131);
\definecolor{c}{rgb}{0,0,0};
\colorlet{c}{kugray};
\draw [c] (4.30439,3.4529) -- (4.30439,3.47803);
\draw [c] (4.30439,3.47803) -- (4.30439,3.50126);
\draw [c] (4.28957,3.47803) -- (4.30439,3.47803);
\draw [c] (4.30439,3.47803) -- (4.31921,3.47803);
\definecolor{c}{rgb}{0,0,0};
\colorlet{c}{kugray};
\draw [c] (4.33402,3.41852) -- (4.33402,3.44434);
\draw [c] (4.33402,3.44434) -- (4.33402,3.46816);
\draw [c] (4.31921,3.44434) -- (4.33402,3.44434);
\draw [c] (4.33402,3.44434) -- (4.34884,3.44434);
\definecolor{c}{rgb}{0,0,0};
\colorlet{c}{kugray};
\draw [c] (4.36366,3.41395) -- (4.36366,3.4399);
\draw [c] (4.36366,3.4399) -- (4.36366,3.46384);
\draw [c] (4.34884,3.4399) -- (4.36366,3.4399);
\draw [c] (4.36366,3.4399) -- (4.37848,3.4399);
\definecolor{c}{rgb}{0,0,0};
\colorlet{c}{kugray};
\draw [c] (4.39329,3.46193) -- (4.39329,3.48588);
\draw [c] (4.39329,3.48588) -- (4.39329,3.5081);
\draw [c] (4.37848,3.48588) -- (4.39329,3.48588);
\draw [c] (4.39329,3.48588) -- (4.40811,3.48588);
\definecolor{c}{rgb}{0,0,0};
\colorlet{c}{kugray};
\draw [c] (4.42293,3.43862) -- (4.42293,3.46556);
\draw [c] (4.42293,3.46556) -- (4.42293,3.49033);
\draw [c] (4.40811,3.46556) -- (4.42293,3.46556);
\draw [c] (4.42293,3.46556) -- (4.43775,3.46556);
\definecolor{c}{rgb}{0,0,0};
\colorlet{c}{kugray};
\draw [c] (4.45257,3.40413) -- (4.45257,3.43062);
\draw [c] (4.45257,3.43062) -- (4.45257,3.45502);
\draw [c] (4.43775,3.43062) -- (4.45257,3.43062);
\draw [c] (4.45257,3.43062) -- (4.46738,3.43062);
\definecolor{c}{rgb}{0,0,0};
\colorlet{c}{kugray};
\draw [c] (4.4822,3.45965) -- (4.4822,3.48371);
\draw [c] (4.4822,3.48371) -- (4.4822,3.50602);
\draw [c] (4.46738,3.48371) -- (4.4822,3.48371);
\draw [c] (4.4822,3.48371) -- (4.49702,3.48371);
\definecolor{c}{rgb}{0,0,0};
\colorlet{c}{kugray};
\draw [c] (4.51184,3.43161) -- (4.51184,3.45645);
\draw [c] (4.51184,3.45645) -- (4.51184,3.47944);
\draw [c] (4.49702,3.45645) -- (4.51184,3.45645);
\draw [c] (4.51184,3.45645) -- (4.52666,3.45645);
\definecolor{c}{rgb}{0,0,0};
\colorlet{c}{kugray};
\draw [c] (4.54147,3.41238) -- (4.54147,3.43844);
\draw [c] (4.54147,3.43844) -- (4.54147,3.46247);
\draw [c] (4.52666,3.43844) -- (4.54147,3.43844);
\draw [c] (4.54147,3.43844) -- (4.55629,3.43844);
\definecolor{c}{rgb}{0,0,0};
\colorlet{c}{kugray};
\draw [c] (4.57111,3.40389) -- (4.57111,3.43019);
\draw [c] (4.57111,3.43019) -- (4.57111,3.45442);
\draw [c] (4.55629,3.43019) -- (4.57111,3.43019);
\draw [c] (4.57111,3.43019) -- (4.58593,3.43019);
\definecolor{c}{rgb}{0,0,0};
\colorlet{c}{kugray};
\draw [c] (4.60075,3.40524) -- (4.60075,3.43191);
\draw [c] (4.60075,3.43191) -- (4.60075,3.45646);
\draw [c] (4.58593,3.43191) -- (4.60075,3.43191);
\draw [c] (4.60075,3.43191) -- (4.61556,3.43191);
\definecolor{c}{rgb}{0,0,0};
\colorlet{c}{kugray};
\draw [c] (4.63038,3.40284) -- (4.63038,3.42898);
\draw [c] (4.63038,3.42898) -- (4.63038,3.45308);
\draw [c] (4.61556,3.42898) -- (4.63038,3.42898);
\draw [c] (4.63038,3.42898) -- (4.6452,3.42898);
\definecolor{c}{rgb}{0,0,0};
\colorlet{c}{kugray};
\draw [c] (4.66002,3.40215) -- (4.66002,3.42856);
\draw [c] (4.66002,3.42856) -- (4.66002,3.45289);
\draw [c] (4.6452,3.42856) -- (4.66002,3.42856);
\draw [c] (4.66002,3.42856) -- (4.67483,3.42856);
\definecolor{c}{rgb}{0,0,0};
\colorlet{c}{kugray};
\draw [c] (4.68965,3.33528) -- (4.68965,3.36407);
\draw [c] (4.68965,3.36407) -- (4.68965,3.3904);
\draw [c] (4.67483,3.36407) -- (4.68965,3.36407);
\draw [c] (4.68965,3.36407) -- (4.70447,3.36407);
\definecolor{c}{rgb}{0,0,0};
\colorlet{c}{kugray};
\draw [c] (4.71929,3.39699) -- (4.71929,3.42459);
\draw [c] (4.71929,3.42459) -- (4.71929,3.44992);
\draw [c] (4.70447,3.42459) -- (4.71929,3.42459);
\draw [c] (4.71929,3.42459) -- (4.73411,3.42459);
\definecolor{c}{rgb}{0,0,0};
\colorlet{c}{kugray};
\draw [c] (4.74892,3.37826) -- (4.74892,3.40524);
\draw [c] (4.74892,3.40524) -- (4.74892,3.43005);
\draw [c] (4.73411,3.40524) -- (4.74892,3.40524);
\draw [c] (4.74892,3.40524) -- (4.76374,3.40524);
\definecolor{c}{rgb}{0,0,0};
\colorlet{c}{kugray};
\draw [c] (4.77856,3.42016) -- (4.77856,3.44594);
\draw [c] (4.77856,3.44594) -- (4.77856,3.46973);
\draw [c] (4.76374,3.44594) -- (4.77856,3.44594);
\draw [c] (4.77856,3.44594) -- (4.79338,3.44594);
\definecolor{c}{rgb}{0,0,0};
\colorlet{c}{kugray};
\draw [c] (4.8082,3.38648) -- (4.8082,3.41303);
\draw [c] (4.8082,3.41303) -- (4.8082,3.43746);
\draw [c] (4.79338,3.41303) -- (4.8082,3.41303);
\draw [c] (4.8082,3.41303) -- (4.82301,3.41303);
\definecolor{c}{rgb}{0,0,0};
\colorlet{c}{kugray};
\draw [c] (4.83783,3.39861) -- (4.83783,3.42469);
\draw [c] (4.83783,3.42469) -- (4.83783,3.44873);
\draw [c] (4.82301,3.42469) -- (4.83783,3.42469);
\draw [c] (4.83783,3.42469) -- (4.85265,3.42469);
\definecolor{c}{rgb}{0,0,0};
\colorlet{c}{kugray};
\draw [c] (4.86747,3.30875) -- (4.86747,3.33911);
\draw [c] (4.86747,3.33911) -- (4.86747,3.36675);
\draw [c] (4.85265,3.33911) -- (4.86747,3.33911);
\draw [c] (4.86747,3.33911) -- (4.88228,3.33911);
\definecolor{c}{rgb}{0,0,0};
\colorlet{c}{kugray};
\draw [c] (4.8971,3.35508) -- (4.8971,3.38333);
\draw [c] (4.8971,3.38333) -- (4.8971,3.4092);
\draw [c] (4.88228,3.38333) -- (4.8971,3.38333);
\draw [c] (4.8971,3.38333) -- (4.91192,3.38333);
\definecolor{c}{rgb}{0,0,0};
\colorlet{c}{kugray};
\draw [c] (4.92674,3.30415) -- (4.92674,3.33311);
\draw [c] (4.92674,3.33311) -- (4.92674,3.35958);
\draw [c] (4.91192,3.33311) -- (4.92674,3.33311);
\draw [c] (4.92674,3.33311) -- (4.94156,3.33311);
\definecolor{c}{rgb}{0,0,0};
\colorlet{c}{kugray};
\draw [c] (4.95637,3.33255) -- (4.95637,3.36143);
\draw [c] (4.95637,3.36143) -- (4.95637,3.38782);
\draw [c] (4.94156,3.36143) -- (4.95637,3.36143);
\draw [c] (4.95637,3.36143) -- (4.97119,3.36143);
\definecolor{c}{rgb}{0,0,0};
\colorlet{c}{kugray};
\draw [c] (4.98601,3.37358) -- (4.98601,3.40123);
\draw [c] (4.98601,3.40123) -- (4.98601,3.42661);
\draw [c] (4.97119,3.40123) -- (4.98601,3.40123);
\draw [c] (4.98601,3.40123) -- (5.00083,3.40123);
\definecolor{c}{rgb}{0,0,0};
\colorlet{c}{kugray};
\draw [c] (5.01565,3.35642) -- (5.01565,3.38431);
\draw [c] (5.01565,3.38431) -- (5.01565,3.40989);
\draw [c] (5.00083,3.38431) -- (5.01565,3.38431);
\draw [c] (5.01565,3.38431) -- (5.03046,3.38431);
\definecolor{c}{rgb}{0,0,0};
\colorlet{c}{kugray};
\draw [c] (5.04528,3.31958) -- (5.04528,3.34899);
\draw [c] (5.04528,3.34899) -- (5.04528,3.37583);
\draw [c] (5.03046,3.34899) -- (5.04528,3.34899);
\draw [c] (5.04528,3.34899) -- (5.0601,3.34899);
\definecolor{c}{rgb}{0,0,0};
\colorlet{c}{kugray};
\draw [c] (5.07492,3.3449) -- (5.07492,3.37413);
\draw [c] (5.07492,3.37413) -- (5.07492,3.40082);
\draw [c] (5.0601,3.37413) -- (5.07492,3.37413);
\draw [c] (5.07492,3.37413) -- (5.08974,3.37413);
\definecolor{c}{rgb}{0,0,0};
\colorlet{c}{kugray};
\draw [c] (5.10455,3.35313) -- (5.10455,3.38196);
\draw [c] (5.10455,3.38196) -- (5.10455,3.40832);
\draw [c] (5.08974,3.38196) -- (5.10455,3.38196);
\draw [c] (5.10455,3.38196) -- (5.11937,3.38196);
\definecolor{c}{rgb}{0,0,0};
\colorlet{c}{kugray};
\draw [c] (5.13419,3.35694) -- (5.13419,3.38661);
\draw [c] (5.13419,3.38661) -- (5.13419,3.41367);
\draw [c] (5.11937,3.38661) -- (5.13419,3.38661);
\draw [c] (5.13419,3.38661) -- (5.14901,3.38661);
\definecolor{c}{rgb}{0,0,0};
\colorlet{c}{kugray};
\draw [c] (5.16382,3.34144) -- (5.16382,3.36991);
\draw [c] (5.16382,3.36991) -- (5.16382,3.39596);
\draw [c] (5.14901,3.36991) -- (5.16382,3.36991);
\draw [c] (5.16382,3.36991) -- (5.17864,3.36991);
\definecolor{c}{rgb}{0,0,0};
\colorlet{c}{kugray};
\draw [c] (5.19346,3.37738) -- (5.19346,3.40395);
\draw [c] (5.19346,3.40395) -- (5.19346,3.42841);
\draw [c] (5.17864,3.40395) -- (5.19346,3.40395);
\draw [c] (5.19346,3.40395) -- (5.20828,3.40395);
\definecolor{c}{rgb}{0,0,0};
\colorlet{c}{kugray};
\draw [c] (5.2231,3.33251) -- (5.2231,3.36208);
\draw [c] (5.2231,3.36208) -- (5.2231,3.38905);
\draw [c] (5.20828,3.36208) -- (5.2231,3.36208);
\draw [c] (5.2231,3.36208) -- (5.23791,3.36208);
\definecolor{c}{rgb}{0,0,0};
\colorlet{c}{kugray};
\draw [c] (5.25273,3.32212) -- (5.25273,3.3512);
\draw [c] (5.25273,3.3512) -- (5.25273,3.37777);
\draw [c] (5.23791,3.3512) -- (5.25273,3.3512);
\draw [c] (5.25273,3.3512) -- (5.26755,3.3512);
\definecolor{c}{rgb}{0,0,0};
\colorlet{c}{kugray};
\draw [c] (5.28237,3.37237) -- (5.28237,3.40019);
\draw [c] (5.28237,3.40019) -- (5.28237,3.42569);
\draw [c] (5.26755,3.40019) -- (5.28237,3.40019);
\draw [c] (5.28237,3.40019) -- (5.29719,3.40019);
\definecolor{c}{rgb}{0,0,0};
\colorlet{c}{kugray};
\draw [c] (5.312,3.32807) -- (5.312,3.3584);
\draw [c] (5.312,3.3584) -- (5.312,3.38601);
\draw [c] (5.29719,3.3584) -- (5.312,3.3584);
\draw [c] (5.312,3.3584) -- (5.32682,3.3584);
\definecolor{c}{rgb}{0,0,0};
\colorlet{c}{kugray};
\draw [c] (5.34164,3.30394) -- (5.34164,3.33512);
\draw [c] (5.34164,3.33512) -- (5.34164,3.36344);
\draw [c] (5.32682,3.33512) -- (5.34164,3.33512);
\draw [c] (5.34164,3.33512) -- (5.35646,3.33512);
\definecolor{c}{rgb}{0,0,0};
\colorlet{c}{kugray};
\draw [c] (5.37127,3.27081) -- (5.37127,3.30238);
\draw [c] (5.37127,3.30238) -- (5.37127,3.33101);
\draw [c] (5.35646,3.30238) -- (5.37127,3.30238);
\draw [c] (5.37127,3.30238) -- (5.38609,3.30238);
\definecolor{c}{rgb}{0,0,0};
\colorlet{c}{kugray};
\draw [c] (5.40091,3.34778) -- (5.40091,3.37729);
\draw [c] (5.40091,3.37729) -- (5.40091,3.40421);
\draw [c] (5.38609,3.37729) -- (5.40091,3.37729);
\draw [c] (5.40091,3.37729) -- (5.41573,3.37729);
\definecolor{c}{rgb}{0,0,0};
\colorlet{c}{kugray};
\draw [c] (5.43055,3.31745) -- (5.43055,3.34765);
\draw [c] (5.43055,3.34765) -- (5.43055,3.37514);
\draw [c] (5.41573,3.34765) -- (5.43055,3.34765);
\draw [c] (5.43055,3.34765) -- (5.44536,3.34765);
\definecolor{c}{rgb}{0,0,0};
\colorlet{c}{kugray};
\draw [c] (5.46018,3.27009) -- (5.46018,3.30118);
\draw [c] (5.46018,3.30118) -- (5.46018,3.32942);
\draw [c] (5.44536,3.30118) -- (5.46018,3.30118);
\draw [c] (5.46018,3.30118) -- (5.475,3.30118);
\definecolor{c}{rgb}{0,0,0};
\colorlet{c}{kugray};
\draw [c] (5.48982,3.30023) -- (5.48982,3.32999);
\draw [c] (5.48982,3.32999) -- (5.48982,3.35712);
\draw [c] (5.475,3.32999) -- (5.48982,3.32999);
\draw [c] (5.48982,3.32999) -- (5.50464,3.32999);
\definecolor{c}{rgb}{0,0,0};
\colorlet{c}{kugray};
\draw [c] (5.51945,3.27827) -- (5.51945,3.30971);
\draw [c] (5.51945,3.30971) -- (5.51945,3.33824);
\draw [c] (5.50464,3.30971) -- (5.51945,3.30971);
\draw [c] (5.51945,3.30971) -- (5.53427,3.30971);
\definecolor{c}{rgb}{0,0,0};
\colorlet{c}{kugray};
\draw [c] (5.54909,3.27339) -- (5.54909,3.30456);
\draw [c] (5.54909,3.30456) -- (5.54909,3.33287);
\draw [c] (5.53427,3.30456) -- (5.54909,3.30456);
\draw [c] (5.54909,3.30456) -- (5.56391,3.30456);
\definecolor{c}{rgb}{0,0,0};
\colorlet{c}{kugray};
\draw [c] (5.57873,3.2928) -- (5.57873,3.32215);
\draw [c] (5.57873,3.32215) -- (5.57873,3.34893);
\draw [c] (5.56391,3.32215) -- (5.57873,3.32215);
\draw [c] (5.57873,3.32215) -- (5.59354,3.32215);
\definecolor{c}{rgb}{0,0,0};
\colorlet{c}{kugray};
\draw [c] (5.60836,3.32728) -- (5.60836,3.35666);
\draw [c] (5.60836,3.35666) -- (5.60836,3.38348);
\draw [c] (5.59354,3.35666) -- (5.60836,3.35666);
\draw [c] (5.60836,3.35666) -- (5.62318,3.35666);
\definecolor{c}{rgb}{0,0,0};
\colorlet{c}{kugray};
\draw [c] (5.638,3.35064) -- (5.638,3.3789);
\draw [c] (5.638,3.3789) -- (5.638,3.40479);
\draw [c] (5.62318,3.3789) -- (5.638,3.3789);
\draw [c] (5.638,3.3789) -- (5.65281,3.3789);
\definecolor{c}{rgb}{0,0,0};
\colorlet{c}{kugray};
\draw [c] (5.66763,3.27119) -- (5.66763,3.30362);
\draw [c] (5.66763,3.30362) -- (5.66763,3.33296);
\draw [c] (5.65281,3.30362) -- (5.66763,3.30362);
\draw [c] (5.66763,3.30362) -- (5.68245,3.30362);
\definecolor{c}{rgb}{0,0,0};
\colorlet{c}{kugray};
\draw [c] (5.69727,3.30087) -- (5.69727,3.33324);
\draw [c] (5.69727,3.33324) -- (5.69727,3.36252);
\draw [c] (5.68245,3.33324) -- (5.69727,3.33324);
\draw [c] (5.69727,3.33324) -- (5.71209,3.33324);
\definecolor{c}{rgb}{0,0,0};
\colorlet{c}{kugray};
\draw [c] (5.7269,3.35191) -- (5.7269,3.37987);
\draw [c] (5.7269,3.37987) -- (5.7269,3.4055);
\draw [c] (5.71209,3.37987) -- (5.7269,3.37987);
\draw [c] (5.7269,3.37987) -- (5.74172,3.37987);
\definecolor{c}{rgb}{0,0,0};
\colorlet{c}{kugray};
\draw [c] (5.75654,3.32022) -- (5.75654,3.34998);
\draw [c] (5.75654,3.34998) -- (5.75654,3.37711);
\draw [c] (5.74172,3.34998) -- (5.75654,3.34998);
\draw [c] (5.75654,3.34998) -- (5.77136,3.34998);
\definecolor{c}{rgb}{0,0,0};
\colorlet{c}{kugray};
\draw [c] (5.78618,3.29616) -- (5.78618,3.32751);
\draw [c] (5.78618,3.32751) -- (5.78618,3.35595);
\draw [c] (5.77136,3.32751) -- (5.78618,3.32751);
\draw [c] (5.78618,3.32751) -- (5.80099,3.32751);
\definecolor{c}{rgb}{0,0,0};
\colorlet{c}{kugray};
\draw [c] (5.81581,3.25342) -- (5.81581,3.28535);
\draw [c] (5.81581,3.28535) -- (5.81581,3.31427);
\draw [c] (5.80099,3.28535) -- (5.81581,3.28535);
\draw [c] (5.81581,3.28535) -- (5.83063,3.28535);
\definecolor{c}{rgb}{0,0,0};
\colorlet{c}{kugray};
\draw [c] (5.84545,3.26747) -- (5.84545,3.30131);
\draw [c] (5.84545,3.30131) -- (5.84545,3.3318);
\draw [c] (5.83063,3.30131) -- (5.84545,3.30131);
\draw [c] (5.84545,3.30131) -- (5.86026,3.30131);
\definecolor{c}{rgb}{0,0,0};
\colorlet{c}{kugray};
\draw [c] (5.87508,3.27773) -- (5.87508,3.31106);
\draw [c] (5.87508,3.31106) -- (5.87508,3.34112);
\draw [c] (5.86026,3.31106) -- (5.87508,3.31106);
\draw [c] (5.87508,3.31106) -- (5.8899,3.31106);
\definecolor{c}{rgb}{0,0,0};
\colorlet{c}{kugray};
\draw [c] (5.90472,3.33645) -- (5.90472,3.36537);
\draw [c] (5.90472,3.36537) -- (5.90472,3.3918);
\draw [c] (5.8899,3.36537) -- (5.90472,3.36537);
\draw [c] (5.90472,3.36537) -- (5.91954,3.36537);
\definecolor{c}{rgb}{0,0,0};
\colorlet{c}{kugray};
\draw [c] (5.93435,3.324) -- (5.93435,3.35624);
\draw [c] (5.93435,3.35624) -- (5.93435,3.38541);
\draw [c] (5.91954,3.35624) -- (5.93435,3.35624);
\draw [c] (5.93435,3.35624) -- (5.94917,3.35624);
\definecolor{c}{rgb}{0,0,0};
\colorlet{c}{kugray};
\draw [c] (5.96399,3.26085) -- (5.96399,3.29364);
\draw [c] (5.96399,3.29364) -- (5.96399,3.32327);
\draw [c] (5.94917,3.29364) -- (5.96399,3.29364);
\draw [c] (5.96399,3.29364) -- (5.97881,3.29364);
\definecolor{c}{rgb}{0,0,0};
\colorlet{c}{kugray};
\draw [c] (5.99363,3.26893) -- (5.99363,3.30106);
\draw [c] (5.99363,3.30106) -- (5.99363,3.33016);
\draw [c] (5.97881,3.30106) -- (5.99363,3.30106);
\draw [c] (5.99363,3.30106) -- (6.00844,3.30106);
\definecolor{c}{rgb}{0,0,0};
\colorlet{c}{kugray};
\draw [c] (6.02326,3.32056) -- (6.02326,3.35257);
\draw [c] (6.02326,3.35257) -- (6.02326,3.38156);
\draw [c] (6.00844,3.35257) -- (6.02326,3.35257);
\draw [c] (6.02326,3.35257) -- (6.03808,3.35257);
\definecolor{c}{rgb}{0,0,0};
\colorlet{c}{kugray};
\draw [c] (6.0529,3.27284) -- (6.0529,3.30377);
\draw [c] (6.0529,3.30377) -- (6.0529,3.33188);
\draw [c] (6.03808,3.30377) -- (6.0529,3.30377);
\draw [c] (6.0529,3.30377) -- (6.06772,3.30377);
\definecolor{c}{rgb}{0,0,0};
\colorlet{c}{kugray};
\draw [c] (6.08253,3.28251) -- (6.08253,3.31642);
\draw [c] (6.08253,3.31642) -- (6.08253,3.34697);
\draw [c] (6.06772,3.31642) -- (6.08253,3.31642);
\draw [c] (6.08253,3.31642) -- (6.09735,3.31642);
\definecolor{c}{rgb}{0,0,0};
\colorlet{c}{kugray};
\draw [c] (6.11217,3.31512) -- (6.11217,3.34443);
\draw [c] (6.11217,3.34443) -- (6.11217,3.3712);
\draw [c] (6.09735,3.34443) -- (6.11217,3.34443);
\draw [c] (6.11217,3.34443) -- (6.12699,3.34443);
\definecolor{c}{rgb}{0,0,0};
\colorlet{c}{kugray};
\draw [c] (6.1418,3.24222) -- (6.1418,3.27469);
\draw [c] (6.1418,3.27469) -- (6.1418,3.30406);
\draw [c] (6.12699,3.27469) -- (6.1418,3.27469);
\draw [c] (6.1418,3.27469) -- (6.15662,3.27469);
\definecolor{c}{rgb}{0,0,0};
\colorlet{c}{kugray};
\draw [c] (6.17144,3.19853) -- (6.17144,3.23317);
\draw [c] (6.17144,3.23317) -- (6.17144,3.26431);
\draw [c] (6.15662,3.23317) -- (6.17144,3.23317);
\draw [c] (6.17144,3.23317) -- (6.18626,3.23317);
\definecolor{c}{rgb}{0,0,0};
\colorlet{c}{kugray};
\draw [c] (6.20108,3.26243) -- (6.20108,3.29519);
\draw [c] (6.20108,3.29519) -- (6.20108,3.32479);
\draw [c] (6.18626,3.29519) -- (6.20108,3.29519);
\draw [c] (6.20108,3.29519) -- (6.21589,3.29519);
\definecolor{c}{rgb}{0,0,0};
\colorlet{c}{kugray};
\draw [c] (6.23071,3.26744) -- (6.23071,3.30118);
\draw [c] (6.23071,3.30118) -- (6.23071,3.33159);
\draw [c] (6.21589,3.30118) -- (6.23071,3.30118);
\draw [c] (6.23071,3.30118) -- (6.24553,3.30118);
\definecolor{c}{rgb}{0,0,0};
\colorlet{c}{kugray};
\draw [c] (6.26035,3.29406) -- (6.26035,3.32675);
\draw [c] (6.26035,3.32675) -- (6.26035,3.35629);
\draw [c] (6.24553,3.32675) -- (6.26035,3.32675);
\draw [c] (6.26035,3.32675) -- (6.27517,3.32675);
\definecolor{c}{rgb}{0,0,0};
\colorlet{c}{kugray};
\draw [c] (6.28998,3.23898) -- (6.28998,3.27334);
\draw [c] (6.28998,3.27334) -- (6.28998,3.30426);
\draw [c] (6.27517,3.27334) -- (6.28998,3.27334);
\draw [c] (6.28998,3.27334) -- (6.3048,3.27334);
\definecolor{c}{rgb}{0,0,0};
\colorlet{c}{kugray};
\draw [c] (6.31962,3.22624) -- (6.31962,3.26036);
\draw [c] (6.31962,3.26036) -- (6.31962,3.29107);
\draw [c] (6.3048,3.26036) -- (6.31962,3.26036);
\draw [c] (6.31962,3.26036) -- (6.33444,3.26036);
\definecolor{c}{rgb}{0,0,0};
\colorlet{c}{kugray};
\draw [c] (6.34926,3.26468) -- (6.34926,3.29684);
\draw [c] (6.34926,3.29684) -- (6.34926,3.32595);
\draw [c] (6.33444,3.29684) -- (6.34926,3.29684);
\draw [c] (6.34926,3.29684) -- (6.36407,3.29684);
\definecolor{c}{rgb}{0,0,0};
\colorlet{c}{kugray};
\draw [c] (6.37889,3.30966) -- (6.37889,3.34032);
\draw [c] (6.37889,3.34032) -- (6.37889,3.36821);
\draw [c] (6.36407,3.34032) -- (6.37889,3.34032);
\draw [c] (6.37889,3.34032) -- (6.39371,3.34032);
\definecolor{c}{rgb}{0,0,0};
\colorlet{c}{kugray};
\draw [c] (6.40853,3.23431) -- (6.40853,3.26778);
\draw [c] (6.40853,3.26778) -- (6.40853,3.29797);
\draw [c] (6.39371,3.26778) -- (6.40853,3.26778);
\draw [c] (6.40853,3.26778) -- (6.42334,3.26778);
\definecolor{c}{rgb}{0,0,0};
\colorlet{c}{kugray};
\draw [c] (6.43816,3.26143) -- (6.43816,3.29357);
\draw [c] (6.43816,3.29357) -- (6.43816,3.32268);
\draw [c] (6.42334,3.29357) -- (6.43816,3.29357);
\draw [c] (6.43816,3.29357) -- (6.45298,3.29357);
\definecolor{c}{rgb}{0,0,0};
\colorlet{c}{kugray};
\draw [c] (6.4678,3.21815) -- (6.4678,3.25281);
\draw [c] (6.4678,3.25281) -- (6.4678,3.28397);
\draw [c] (6.45298,3.25281) -- (6.4678,3.25281);
\draw [c] (6.4678,3.25281) -- (6.48262,3.25281);
\definecolor{c}{rgb}{0,0,0};
\colorlet{c}{kugray};
\draw [c] (6.49743,3.25782) -- (6.49743,3.29048);
\draw [c] (6.49743,3.29048) -- (6.49743,3.32);
\draw [c] (6.48262,3.29048) -- (6.49743,3.29048);
\draw [c] (6.49743,3.29048) -- (6.51225,3.29048);
\definecolor{c}{rgb}{0,0,0};
\colorlet{c}{kugray};
\draw [c] (6.52707,3.29629) -- (6.52707,3.32741);
\draw [c] (6.52707,3.32741) -- (6.52707,3.35568);
\draw [c] (6.51225,3.32741) -- (6.52707,3.32741);
\draw [c] (6.52707,3.32741) -- (6.54189,3.32741);
\definecolor{c}{rgb}{0,0,0};
\colorlet{c}{kugray};
\draw [c] (6.55671,3.28852) -- (6.55671,3.32075);
\draw [c] (6.55671,3.32075) -- (6.55671,3.34993);
\draw [c] (6.54189,3.32075) -- (6.55671,3.32075);
\draw [c] (6.55671,3.32075) -- (6.57152,3.32075);
\definecolor{c}{rgb}{0,0,0};
\colorlet{c}{kugray};
\draw [c] (6.58634,3.11775) -- (6.58634,3.15987);
\draw [c] (6.58634,3.15987) -- (6.58634,3.19692);
\draw [c] (6.57152,3.15987) -- (6.58634,3.15987);
\draw [c] (6.58634,3.15987) -- (6.60116,3.15987);
\definecolor{c}{rgb}{0,0,0};
\colorlet{c}{kugray};
\draw [c] (6.61598,3.23027) -- (6.61598,3.2646);
\draw [c] (6.61598,3.2646) -- (6.61598,3.29549);
\draw [c] (6.60116,3.2646) -- (6.61598,3.2646);
\draw [c] (6.61598,3.2646) -- (6.63079,3.2646);
\definecolor{c}{rgb}{0,0,0};
\colorlet{c}{kugray};
\draw [c] (6.64561,3.24953) -- (6.64561,3.28391);
\draw [c] (6.64561,3.28391) -- (6.64561,3.31483);
\draw [c] (6.63079,3.28391) -- (6.64561,3.28391);
\draw [c] (6.64561,3.28391) -- (6.66043,3.28391);
\definecolor{c}{rgb}{0,0,0};
\colorlet{c}{kugray};
\draw [c] (6.67525,3.22925) -- (6.67525,3.26504);
\draw [c] (6.67525,3.26504) -- (6.67525,3.2971);
\draw [c] (6.66043,3.26504) -- (6.67525,3.26504);
\draw [c] (6.67525,3.26504) -- (6.69007,3.26504);
\definecolor{c}{rgb}{0,0,0};
\colorlet{c}{kugray};
\draw [c] (6.70488,3.27333) -- (6.70488,3.30459);
\draw [c] (6.70488,3.30459) -- (6.70488,3.33297);
\draw [c] (6.69007,3.30459) -- (6.70488,3.30459);
\draw [c] (6.70488,3.30459) -- (6.7197,3.30459);
\definecolor{c}{rgb}{0,0,0};
\colorlet{c}{kugray};
\draw [c] (6.73452,3.2188) -- (6.73452,3.25383);
\draw [c] (6.73452,3.25383) -- (6.73452,3.28527);
\draw [c] (6.7197,3.25383) -- (6.73452,3.25383);
\draw [c] (6.73452,3.25383) -- (6.74934,3.25383);
\definecolor{c}{rgb}{0,0,0};
\colorlet{c}{kugray};
\draw [c] (6.76416,3.19407) -- (6.76416,3.23035);
\draw [c] (6.76416,3.23035) -- (6.76416,3.26281);
\draw [c] (6.74934,3.23035) -- (6.76416,3.23035);
\draw [c] (6.76416,3.23035) -- (6.77897,3.23035);
\definecolor{c}{rgb}{0,0,0};
\colorlet{c}{kugray};
\draw [c] (6.79379,3.2118) -- (6.79379,3.24685);
\draw [c] (6.79379,3.24685) -- (6.79379,3.2783);
\draw [c] (6.77897,3.24685) -- (6.79379,3.24685);
\draw [c] (6.79379,3.24685) -- (6.80861,3.24685);
\definecolor{c}{rgb}{0,0,0};
\colorlet{c}{kugray};
\draw [c] (6.82343,3.20129) -- (6.82343,3.23637);
\draw [c] (6.82343,3.23637) -- (6.82343,3.26785);
\draw [c] (6.80861,3.23637) -- (6.82343,3.23637);
\draw [c] (6.82343,3.23637) -- (6.83824,3.23637);
\definecolor{c}{rgb}{0,0,0};
\colorlet{c}{kugray};
\draw [c] (6.85306,3.27615) -- (6.85306,3.30874);
\draw [c] (6.85306,3.30874) -- (6.85306,3.33821);
\draw [c] (6.83824,3.30874) -- (6.85306,3.30874);
\draw [c] (6.85306,3.30874) -- (6.86788,3.30874);
\definecolor{c}{rgb}{0,0,0};
\colorlet{c}{kugray};
\draw [c] (6.8827,3.28993) -- (6.8827,3.32159);
\draw [c] (6.8827,3.32159) -- (6.8827,3.3503);
\draw [c] (6.86788,3.32159) -- (6.8827,3.32159);
\draw [c] (6.8827,3.32159) -- (6.89752,3.32159);
\definecolor{c}{rgb}{0,0,0};
\colorlet{c}{kugray};
\draw [c] (6.91233,3.22502) -- (6.91233,3.25976);
\draw [c] (6.91233,3.25976) -- (6.91233,3.29097);
\draw [c] (6.89752,3.25976) -- (6.91233,3.25976);
\draw [c] (6.91233,3.25976) -- (6.92715,3.25976);
\definecolor{c}{rgb}{0,0,0};
\colorlet{c}{kugray};
\draw [c] (6.94197,3.27047) -- (6.94197,3.30221);
\draw [c] (6.94197,3.30221) -- (6.94197,3.33098);
\draw [c] (6.92715,3.30221) -- (6.94197,3.30221);
\draw [c] (6.94197,3.30221) -- (6.95679,3.30221);
\definecolor{c}{rgb}{0,0,0};
\colorlet{c}{kugray};
\draw [c] (6.97161,3.23938) -- (6.97161,3.27409);
\draw [c] (6.97161,3.27409) -- (6.97161,3.30529);
\draw [c] (6.95679,3.27409) -- (6.97161,3.27409);
\draw [c] (6.97161,3.27409) -- (6.98642,3.27409);
\definecolor{c}{rgb}{0,0,0};
\colorlet{c}{kugray};
\draw [c] (7.00124,3.2166) -- (7.00124,3.25185);
\draw [c] (7.00124,3.25185) -- (7.00124,3.28348);
\draw [c] (6.98642,3.25185) -- (7.00124,3.25185);
\draw [c] (7.00124,3.25185) -- (7.01606,3.25185);
\definecolor{c}{rgb}{0,0,0};
\colorlet{c}{kugray};
\draw [c] (7.03088,3.27665) -- (7.03088,3.3089);
\draw [c] (7.03088,3.3089) -- (7.03088,3.33809);
\draw [c] (7.01606,3.3089) -- (7.03088,3.3089);
\draw [c] (7.03088,3.3089) -- (7.0457,3.3089);
\definecolor{c}{rgb}{0,0,0};
\colorlet{c}{kugray};
\draw [c] (7.06051,3.14522) -- (7.06051,3.18511);
\draw [c] (7.06051,3.18511) -- (7.06051,3.22041);
\draw [c] (7.0457,3.18511) -- (7.06051,3.18511);
\draw [c] (7.06051,3.18511) -- (7.07533,3.18511);
\definecolor{c}{rgb}{0,0,0};
\colorlet{c}{kugray};
\draw [c] (7.09015,3.25803) -- (7.09015,3.29279);
\draw [c] (7.09015,3.29279) -- (7.09015,3.32402);
\draw [c] (7.07533,3.29279) -- (7.09015,3.29279);
\draw [c] (7.09015,3.29279) -- (7.10497,3.29279);
\definecolor{c}{rgb}{0,0,0};
\colorlet{c}{kugray};
\draw [c] (7.11978,3.1251) -- (7.11978,3.16549);
\draw [c] (7.11978,3.16549) -- (7.11978,3.2012);
\draw [c] (7.10497,3.16549) -- (7.11978,3.16549);
\draw [c] (7.11978,3.16549) -- (7.1346,3.16549);
\definecolor{c}{rgb}{0,0,0};
\colorlet{c}{kugray};
\draw [c] (7.14942,3.20086) -- (7.14942,3.23706);
\draw [c] (7.14942,3.23706) -- (7.14942,3.26946);
\draw [c] (7.1346,3.23706) -- (7.14942,3.23706);
\draw [c] (7.14942,3.23706) -- (7.16424,3.23706);
\definecolor{c}{rgb}{0,0,0};
\colorlet{c}{kugray};
\draw [c] (7.17906,3.23931) -- (7.17906,3.27217);
\draw [c] (7.17906,3.27217) -- (7.17906,3.30187);
\draw [c] (7.16424,3.27217) -- (7.17906,3.27217);
\draw [c] (7.17906,3.27217) -- (7.19387,3.27217);
\definecolor{c}{rgb}{0,0,0};
\colorlet{c}{kugray};
\draw [c] (7.20869,3.22963) -- (7.20869,3.26378);
\draw [c] (7.20869,3.26378) -- (7.20869,3.29452);
\draw [c] (7.19387,3.26378) -- (7.20869,3.26378);
\draw [c] (7.20869,3.26378) -- (7.22351,3.26378);
\definecolor{c}{rgb}{0,0,0};
\colorlet{c}{kugray};
\draw [c] (7.23833,3.17041) -- (7.23833,3.20638);
\draw [c] (7.23833,3.20638) -- (7.23833,3.23859);
\draw [c] (7.22351,3.20638) -- (7.23833,3.20638);
\draw [c] (7.23833,3.20638) -- (7.25315,3.20638);
\definecolor{c}{rgb}{0,0,0};
\colorlet{c}{kugray};
\draw [c] (7.26796,3.20336) -- (7.26796,3.2391);
\draw [c] (7.26796,3.2391) -- (7.26796,3.27111);
\draw [c] (7.25315,3.2391) -- (7.26796,3.2391);
\draw [c] (7.26796,3.2391) -- (7.28278,3.2391);
\definecolor{c}{rgb}{0,0,0};
\colorlet{c}{kugray};
\draw [c] (7.2976,3.13789) -- (7.2976,3.17615);
\draw [c] (7.2976,3.17615) -- (7.2976,3.21018);
\draw [c] (7.28278,3.17615) -- (7.2976,3.17615);
\draw [c] (7.2976,3.17615) -- (7.31242,3.17615);
\definecolor{c}{rgb}{0,0,0};
\colorlet{c}{kugray};
\draw [c] (7.32724,3.18107) -- (7.32724,3.21995);
\draw [c] (7.32724,3.21995) -- (7.32724,3.25447);
\draw [c] (7.31242,3.21995) -- (7.32724,3.21995);
\draw [c] (7.32724,3.21995) -- (7.34205,3.21995);
\definecolor{c}{rgb}{0,0,0};
\colorlet{c}{kugray};
\draw [c] (7.35687,3.17771) -- (7.35687,3.214);
\draw [c] (7.35687,3.214) -- (7.35687,3.24646);
\draw [c] (7.34205,3.214) -- (7.35687,3.214);
\draw [c] (7.35687,3.214) -- (7.37169,3.214);
\definecolor{c}{rgb}{0,0,0};
\colorlet{c}{kugray};
\draw [c] (7.38651,3.20963) -- (7.38651,3.24614);
\draw [c] (7.38651,3.24614) -- (7.38651,3.27878);
\draw [c] (7.37169,3.24614) -- (7.38651,3.24614);
\draw [c] (7.38651,3.24614) -- (7.40132,3.24614);
\definecolor{c}{rgb}{0,0,0};
\colorlet{c}{kugray};
\draw [c] (7.41614,3.23434) -- (7.41614,3.26861);
\draw [c] (7.41614,3.26861) -- (7.41614,3.29944);
\draw [c] (7.40132,3.26861) -- (7.41614,3.26861);
\draw [c] (7.41614,3.26861) -- (7.43096,3.26861);
\definecolor{c}{rgb}{0,0,0};
\colorlet{c}{kugray};
\draw [c] (7.44578,3.26607) -- (7.44578,3.29811);
\draw [c] (7.44578,3.29811) -- (7.44578,3.32713);
\draw [c] (7.43096,3.29811) -- (7.44578,3.29811);
\draw [c] (7.44578,3.29811) -- (7.4606,3.29811);
\definecolor{c}{rgb}{0,0,0};
\colorlet{c}{kugray};
\draw [c] (7.47541,3.23088) -- (7.47541,3.26608);
\draw [c] (7.47541,3.26608) -- (7.47541,3.29768);
\draw [c] (7.4606,3.26608) -- (7.47541,3.26608);
\draw [c] (7.47541,3.26608) -- (7.49023,3.26608);
\definecolor{c}{rgb}{0,0,0};
\colorlet{c}{kugray};
\draw [c] (7.50505,3.19492) -- (7.50505,3.23154);
\draw [c] (7.50505,3.23154) -- (7.50505,3.26427);
\draw [c] (7.49023,3.23154) -- (7.50505,3.23154);
\draw [c] (7.50505,3.23154) -- (7.51987,3.23154);
\definecolor{c}{rgb}{0,0,0};
\colorlet{c}{kugray};
\draw [c] (7.53469,3.16974) -- (7.53469,3.20818);
\draw [c] (7.53469,3.20818) -- (7.53469,3.24234);
\draw [c] (7.51987,3.20818) -- (7.53469,3.20818);
\draw [c] (7.53469,3.20818) -- (7.5495,3.20818);
\definecolor{c}{rgb}{0,0,0};
\colorlet{c}{kugray};
\draw [c] (7.56432,3.23775) -- (7.56432,3.27182);
\draw [c] (7.56432,3.27182) -- (7.56432,3.3025);
\draw [c] (7.5495,3.27182) -- (7.56432,3.27182);
\draw [c] (7.56432,3.27182) -- (7.57914,3.27182);
\definecolor{c}{rgb}{0,0,0};
\colorlet{c}{kugray};
\draw [c] (7.59396,3.27348) -- (7.59396,3.30657);
\draw [c] (7.59396,3.30657) -- (7.59396,3.33644);
\draw [c] (7.57914,3.30657) -- (7.59396,3.30657);
\draw [c] (7.59396,3.30657) -- (7.60877,3.30657);
\definecolor{c}{rgb}{0,0,0};
\colorlet{c}{kugray};
\draw [c] (7.62359,3.20408) -- (7.62359,3.24244);
\draw [c] (7.62359,3.24244) -- (7.62359,3.27655);
\draw [c] (7.60877,3.24244) -- (7.62359,3.24244);
\draw [c] (7.62359,3.24244) -- (7.63841,3.24244);
\definecolor{c}{rgb}{0,0,0};
\colorlet{c}{kugray};
\draw [c] (7.65323,3.20046) -- (7.65323,3.23614);
\draw [c] (7.65323,3.23614) -- (7.65323,3.26812);
\draw [c] (7.63841,3.23614) -- (7.65323,3.23614);
\draw [c] (7.65323,3.23614) -- (7.66805,3.23614);
\definecolor{c}{rgb}{0,0,0};
\colorlet{c}{kugray};
\draw [c] (7.68286,3.11491) -- (7.68286,3.15626);
\draw [c] (7.68286,3.15626) -- (7.68286,3.19271);
\draw [c] (7.66805,3.15626) -- (7.68286,3.15626);
\draw [c] (7.68286,3.15626) -- (7.69768,3.15626);
\definecolor{c}{rgb}{0,0,0};
\colorlet{c}{kugray};
\draw [c] (7.7125,3.20878) -- (7.7125,3.24334);
\draw [c] (7.7125,3.24334) -- (7.7125,3.2744);
\draw [c] (7.69768,3.24334) -- (7.7125,3.24334);
\draw [c] (7.7125,3.24334) -- (7.72732,3.24334);
\definecolor{c}{rgb}{0,0,0};
\colorlet{c}{kugray};
\draw [c] (7.74214,3.18668) -- (7.74214,3.22275);
\draw [c] (7.74214,3.22275) -- (7.74214,3.25502);
\draw [c] (7.72732,3.22275) -- (7.74214,3.22275);
\draw [c] (7.74214,3.22275) -- (7.75695,3.22275);
\definecolor{c}{rgb}{0,0,0};
\colorlet{c}{kugray};
\draw [c] (7.77177,3.17887) -- (7.77177,3.21749);
\draw [c] (7.77177,3.21749) -- (7.77177,3.2518);
\draw [c] (7.75695,3.21749) -- (7.77177,3.21749);
\draw [c] (7.77177,3.21749) -- (7.78659,3.21749);
\definecolor{c}{rgb}{0,0,0};
\colorlet{c}{kugray};
\draw [c] (7.80141,3.1287) -- (7.80141,3.1705);
\draw [c] (7.80141,3.1705) -- (7.80141,3.2073);
\draw [c] (7.78659,3.1705) -- (7.80141,3.1705);
\draw [c] (7.80141,3.1705) -- (7.81623,3.1705);
\definecolor{c}{rgb}{0,0,0};
\colorlet{c}{kugray};
\draw [c] (7.83104,3.19387) -- (7.83104,3.23296);
\draw [c] (7.83104,3.23296) -- (7.83104,3.26765);
\draw [c] (7.81623,3.23296) -- (7.83104,3.23296);
\draw [c] (7.83104,3.23296) -- (7.84586,3.23296);
\definecolor{c}{rgb}{0,0,0};
\colorlet{c}{kugray};
\draw [c] (7.86068,3.19817) -- (7.86068,3.23466);
\draw [c] (7.86068,3.23466) -- (7.86068,3.26728);
\draw [c] (7.84586,3.23466) -- (7.86068,3.23466);
\draw [c] (7.86068,3.23466) -- (7.8755,3.23466);
\definecolor{c}{rgb}{0,0,0};
\colorlet{c}{kugray};
\draw [c] (7.89031,3.19947) -- (7.89031,3.23529);
\draw [c] (7.89031,3.23529) -- (7.89031,3.26737);
\draw [c] (7.8755,3.23529) -- (7.89031,3.23529);
\draw [c] (7.89031,3.23529) -- (7.90513,3.23529);
\definecolor{c}{rgb}{0,0,0};
\colorlet{c}{kugray};
\draw [c] (7.91995,3.17289) -- (7.91995,3.21165);
\draw [c] (7.91995,3.21165) -- (7.91995,3.24608);
\draw [c] (7.90513,3.21165) -- (7.91995,3.21165);
\draw [c] (7.91995,3.21165) -- (7.93477,3.21165);
\definecolor{c}{rgb}{0,0,0};
\colorlet{c}{kugray};
\draw [c] (7.94959,3.16056) -- (7.94959,3.20019);
\draw [c] (7.94959,3.20019) -- (7.94959,3.23529);
\draw [c] (7.93477,3.20019) -- (7.94959,3.20019);
\draw [c] (7.94959,3.20019) -- (7.9644,3.20019);
\definecolor{c}{rgb}{0,0,0};
\colorlet{c}{kugray};
\draw [c] (7.97922,3.17291) -- (7.97922,3.2128);
\draw [c] (7.97922,3.2128) -- (7.97922,3.2481);
\draw [c] (7.9644,3.2128) -- (7.97922,3.2128);
\draw [c] (7.97922,3.2128) -- (7.99404,3.2128);
\definecolor{c}{rgb}{0,0,0};
\colorlet{c}{kugray};
\draw [c] (8.00886,3.21345) -- (8.00886,3.24811);
\draw [c] (8.00886,3.24811) -- (8.00886,3.27926);
\draw [c] (7.99404,3.24811) -- (8.00886,3.24811);
\draw [c] (8.00886,3.24811) -- (8.02368,3.24811);
\definecolor{c}{rgb}{0,0,0};
\colorlet{c}{kugray};
\draw [c] (8.03849,3.16384) -- (8.03849,3.20119);
\draw [c] (8.03849,3.20119) -- (8.03849,3.2345);
\draw [c] (8.02368,3.20119) -- (8.03849,3.20119);
\draw [c] (8.03849,3.20119) -- (8.05331,3.20119);
\definecolor{c}{rgb}{0,0,0};
\colorlet{c}{kugray};
\draw [c] (8.06813,3.2368) -- (8.06813,3.27102);
\draw [c] (8.06813,3.27102) -- (8.06813,3.30182);
\draw [c] (8.05331,3.27102) -- (8.06813,3.27102);
\draw [c] (8.06813,3.27102) -- (8.08295,3.27102);
\definecolor{c}{rgb}{0,0,0};
\colorlet{c}{kugray};
\draw [c] (8.09776,3.20237) -- (8.09776,3.23814);
\draw [c] (8.09776,3.23814) -- (8.09776,3.27017);
\draw [c] (8.08295,3.23814) -- (8.09776,3.23814);
\draw [c] (8.09776,3.23814) -- (8.11258,3.23814);
\definecolor{c}{rgb}{0,0,0};
\colorlet{c}{kugray};
\draw [c] (8.1274,3.16214) -- (8.1274,3.20061);
\draw [c] (8.1274,3.20061) -- (8.1274,3.2348);
\draw [c] (8.11258,3.20061) -- (8.1274,3.20061);
\draw [c] (8.1274,3.20061) -- (8.14222,3.20061);
\definecolor{c}{rgb}{0,0,0};
\colorlet{c}{kugray};
\draw [c] (8.15704,3.17943) -- (8.15704,3.21673);
\draw [c] (8.15704,3.21673) -- (8.15704,3.24999);
\draw [c] (8.14222,3.21673) -- (8.15704,3.21673);
\draw [c] (8.15704,3.21673) -- (8.17185,3.21673);
\definecolor{c}{rgb}{0,0,0};
\colorlet{c}{kugray};
\draw [c] (8.18667,3.2077) -- (8.18667,3.24191);
\draw [c] (8.18667,3.24191) -- (8.18667,3.2727);
\draw [c] (8.17185,3.24191) -- (8.18667,3.24191);
\draw [c] (8.18667,3.24191) -- (8.20149,3.24191);
\definecolor{c}{rgb}{0,0,0};
\colorlet{c}{kugray};
\draw [c] (8.21631,3.13546) -- (8.21631,3.17622);
\draw [c] (8.21631,3.17622) -- (8.21631,3.21221);
\draw [c] (8.20149,3.17622) -- (8.21631,3.17622);
\draw [c] (8.21631,3.17622) -- (8.23113,3.17622);
\definecolor{c}{rgb}{0,0,0};
\colorlet{c}{kugray};
\draw [c] (8.24594,3.14151) -- (8.24594,3.18151);
\draw [c] (8.24594,3.18151) -- (8.24594,3.2169);
\draw [c] (8.23113,3.18151) -- (8.24594,3.18151);
\draw [c] (8.24594,3.18151) -- (8.26076,3.18151);
\definecolor{c}{rgb}{0,0,0};
\colorlet{c}{kugray};
\draw [c] (8.27558,3.17226) -- (8.27558,3.21036);
\draw [c] (8.27558,3.21036) -- (8.27558,3.24426);
\draw [c] (8.26076,3.21036) -- (8.27558,3.21036);
\draw [c] (8.27558,3.21036) -- (8.2904,3.21036);
\definecolor{c}{rgb}{0,0,0};
\colorlet{c}{kugray};
\draw [c] (8.30521,3.21664) -- (8.30521,3.25163);
\draw [c] (8.30521,3.25163) -- (8.30521,3.28304);
\draw [c] (8.2904,3.25163) -- (8.30521,3.25163);
\draw [c] (8.30521,3.25163) -- (8.32003,3.25163);
\definecolor{c}{rgb}{0,0,0};
\colorlet{c}{kugray};
\draw [c] (8.33485,3.18633) -- (8.33485,3.22273);
\draw [c] (8.33485,3.22273) -- (8.33485,3.25528);
\draw [c] (8.32003,3.22273) -- (8.33485,3.22273);
\draw [c] (8.33485,3.22273) -- (8.34967,3.22273);
\definecolor{c}{rgb}{0,0,0};
\colorlet{c}{kugray};
\draw [c] (8.36449,3.10352) -- (8.36449,3.14557);
\draw [c] (8.36449,3.14557) -- (8.36449,3.18257);
\draw [c] (8.34967,3.14557) -- (8.36449,3.14557);
\draw [c] (8.36449,3.14557) -- (8.3793,3.14557);
\definecolor{c}{rgb}{0,0,0};
\colorlet{c}{kugray};
\draw [c] (8.39412,3.14212) -- (8.39412,3.18236);
\draw [c] (8.39412,3.18236) -- (8.39412,3.21794);
\draw [c] (8.3793,3.18236) -- (8.39412,3.18236);
\draw [c] (8.39412,3.18236) -- (8.40894,3.18236);
\definecolor{c}{rgb}{0,0,0};
\colorlet{c}{kugray};
\draw [c] (8.42376,3.14268) -- (8.42376,3.18228);
\draw [c] (8.42376,3.18228) -- (8.42376,3.21737);
\draw [c] (8.40894,3.18228) -- (8.42376,3.18228);
\draw [c] (8.42376,3.18228) -- (8.43858,3.18228);
\definecolor{c}{rgb}{0,0,0};
\colorlet{c}{kugray};
\draw [c] (8.45339,3.17027) -- (8.45339,3.21011);
\draw [c] (8.45339,3.21011) -- (8.45339,3.24537);
\draw [c] (8.43858,3.21011) -- (8.45339,3.21011);
\draw [c] (8.45339,3.21011) -- (8.46821,3.21011);
\definecolor{c}{rgb}{0,0,0};
\colorlet{c}{kugray};
\draw [c] (8.48303,3.19033) -- (8.48303,3.22695);
\draw [c] (8.48303,3.22695) -- (8.48303,3.25968);
\draw [c] (8.46821,3.22695) -- (8.48303,3.22695);
\draw [c] (8.48303,3.22695) -- (8.49785,3.22695);
\definecolor{c}{rgb}{0,0,0};
\colorlet{c}{kugray};
\draw [c] (8.51267,3.1697) -- (8.51267,3.20663);
\draw [c] (8.51267,3.20663) -- (8.51267,3.2396);
\draw [c] (8.49785,3.20663) -- (8.51267,3.20663);
\draw [c] (8.51267,3.20663) -- (8.52748,3.20663);
\definecolor{c}{rgb}{0,0,0};
\colorlet{c}{kugray};
\draw [c] (8.5423,3.16479) -- (8.5423,3.20274);
\draw [c] (8.5423,3.20274) -- (8.5423,3.23651);
\draw [c] (8.52748,3.20274) -- (8.5423,3.20274);
\draw [c] (8.5423,3.20274) -- (8.55712,3.20274);
\definecolor{c}{rgb}{0,0,0};
\colorlet{c}{kugray};
\draw [c] (8.57194,3.12526) -- (8.57194,3.1667);
\draw [c] (8.57194,3.1667) -- (8.57194,3.20322);
\draw [c] (8.55712,3.1667) -- (8.57194,3.1667);
\draw [c] (8.57194,3.1667) -- (8.58675,3.1667);
\definecolor{c}{rgb}{0,0,0};
\colorlet{c}{kugray};
\draw [c] (8.60157,3.15802) -- (8.60157,3.19823);
\draw [c] (8.60157,3.19823) -- (8.60157,3.23378);
\draw [c] (8.58675,3.19823) -- (8.60157,3.19823);
\draw [c] (8.60157,3.19823) -- (8.61639,3.19823);
\definecolor{c}{rgb}{0,0,0};
\colorlet{c}{kugray};
\draw [c] (8.63121,3.10131) -- (8.63121,3.14082);
\draw [c] (8.63121,3.14082) -- (8.63121,3.17583);
\draw [c] (8.61639,3.14082) -- (8.63121,3.14082);
\draw [c] (8.63121,3.14082) -- (8.64603,3.14082);
\definecolor{c}{rgb}{0,0,0};
\colorlet{c}{kugray};
\draw [c] (8.66084,3.10207) -- (8.66084,3.14645);
\draw [c] (8.66084,3.14645) -- (8.66084,3.18523);
\draw [c] (8.64603,3.14645) -- (8.66084,3.14645);
\draw [c] (8.66084,3.14645) -- (8.67566,3.14645);
\definecolor{c}{rgb}{0,0,0};
\colorlet{c}{kugray};
\draw [c] (8.69048,3.15812) -- (8.69048,3.19621);
\draw [c] (8.69048,3.19621) -- (8.69048,3.2301);
\draw [c] (8.67566,3.19621) -- (8.69048,3.19621);
\draw [c] (8.69048,3.19621) -- (8.7053,3.19621);
\definecolor{c}{rgb}{0,0,0};
\colorlet{c}{kugray};
\draw [c] (8.72012,3.0956) -- (8.72012,3.14019);
\draw [c] (8.72012,3.14019) -- (8.72012,3.17912);
\draw [c] (8.7053,3.14019) -- (8.72012,3.14019);
\draw [c] (8.72012,3.14019) -- (8.73493,3.14019);
\definecolor{c}{rgb}{0,0,0};
\colorlet{c}{kugray};
\draw [c] (8.74975,3.08632) -- (8.74975,3.12806);
\draw [c] (8.74975,3.12806) -- (8.74975,3.1648);
\draw [c] (8.73493,3.12806) -- (8.74975,3.12806);
\draw [c] (8.74975,3.12806) -- (8.76457,3.12806);
\definecolor{c}{rgb}{0,0,0};
\colorlet{c}{kugray};
\draw [c] (8.77939,3.12605) -- (8.77939,3.16744);
\draw [c] (8.77939,3.16744) -- (8.77939,3.20392);
\draw [c] (8.76457,3.16744) -- (8.77939,3.16744);
\draw [c] (8.77939,3.16744) -- (8.79421,3.16744);
\definecolor{c}{rgb}{0,0,0};
\colorlet{c}{kugray};
\draw [c] (8.80902,3.19836) -- (8.80902,3.23362);
\draw [c] (8.80902,3.23362) -- (8.80902,3.26525);
\draw [c] (8.79421,3.23362) -- (8.80902,3.23362);
\draw [c] (8.80902,3.23362) -- (8.82384,3.23362);
\definecolor{c}{rgb}{0,0,0};
\colorlet{c}{kugray};
\draw [c] (8.83866,3.1022) -- (8.83866,3.14697);
\draw [c] (8.83866,3.14697) -- (8.83866,3.18604);
\draw [c] (8.82384,3.14697) -- (8.83866,3.14697);
\draw [c] (8.83866,3.14697) -- (8.85348,3.14697);
\definecolor{c}{rgb}{0,0,0};
\colorlet{c}{kugray};
\draw [c] (8.86829,3.20731) -- (8.86829,3.2428);
\draw [c] (8.86829,3.2428) -- (8.86829,3.27462);
\draw [c] (8.85348,3.2428) -- (8.86829,3.2428);
\draw [c] (8.86829,3.2428) -- (8.88311,3.2428);
\definecolor{c}{rgb}{0,0,0};
\colorlet{c}{kugray};
\draw [c] (8.89793,3.19795) -- (8.89793,3.23912);
\draw [c] (8.89793,3.23912) -- (8.89793,3.27543);
\draw [c] (8.88311,3.23912) -- (8.89793,3.23912);
\draw [c] (8.89793,3.23912) -- (8.91275,3.23912);
\definecolor{c}{rgb}{0,0,0};
\colorlet{c}{kugray};
\draw [c] (8.92757,3.07413) -- (8.92757,3.11815);
\draw [c] (8.92757,3.11815) -- (8.92757,3.15665);
\draw [c] (8.91275,3.11815) -- (8.92757,3.11815);
\draw [c] (8.92757,3.11815) -- (8.94238,3.11815);
\definecolor{c}{rgb}{0,0,0};
\colorlet{c}{kugray};
\draw [c] (8.9572,3.05408) -- (8.9572,3.10088);
\draw [c] (8.9572,3.10088) -- (8.9572,3.14149);
\draw [c] (8.94238,3.10088) -- (8.9572,3.10088);
\draw [c] (8.9572,3.10088) -- (8.97202,3.10088);
\definecolor{c}{rgb}{0,0,0};
\colorlet{c}{kugray};
\draw [c] (8.98684,3.16696) -- (8.98684,3.20827);
\draw [c] (8.98684,3.20827) -- (8.98684,3.24469);
\draw [c] (8.97202,3.20827) -- (8.98684,3.20827);
\draw [c] (8.98684,3.20827) -- (9.00166,3.20827);
\definecolor{c}{rgb}{0,0,0};
\colorlet{c}{kugray};
\draw [c] (9.01647,3.06066) -- (9.01647,3.10415);
\draw [c] (9.01647,3.10415) -- (9.01647,3.14225);
\draw [c] (9.00166,3.10415) -- (9.01647,3.10415);
\draw [c] (9.01647,3.10415) -- (9.03129,3.10415);
\definecolor{c}{rgb}{0,0,0};
\colorlet{c}{kugray};
\draw [c] (9.04611,3.03321) -- (9.04611,3.07793);
\draw [c] (9.04611,3.07793) -- (9.04611,3.11697);
\draw [c] (9.03129,3.07793) -- (9.04611,3.07793);
\draw [c] (9.04611,3.07793) -- (9.06093,3.07793);
\definecolor{c}{rgb}{0,0,0};
\colorlet{c}{kugray};
\draw [c] (9.07574,3.01544) -- (9.07574,3.0622);
\draw [c] (9.07574,3.0622) -- (9.07574,3.10279);
\draw [c] (9.06093,3.0622) -- (9.07574,3.0622);
\draw [c] (9.07574,3.0622) -- (9.09056,3.0622);
\definecolor{c}{rgb}{0,0,0};
\colorlet{c}{kugray};
\draw [c] (9.10538,3.11751) -- (9.10538,3.16063);
\draw [c] (9.10538,3.16063) -- (9.10538,3.19845);
\draw [c] (9.09056,3.16063) -- (9.10538,3.16063);
\draw [c] (9.10538,3.16063) -- (9.1202,3.16063);
\definecolor{c}{rgb}{0,0,0};
\colorlet{c}{kugray};
\draw [c] (9.13502,3.05438) -- (9.13502,3.10259);
\draw [c] (9.13502,3.10259) -- (9.13502,3.14427);
\draw [c] (9.1202,3.10259) -- (9.13502,3.10259);
\draw [c] (9.13502,3.10259) -- (9.14983,3.10259);
\definecolor{c}{rgb}{0,0,0};
\colorlet{c}{kugray};
\draw [c] (9.16465,3.01975) -- (9.16465,3.07036);
\draw [c] (9.16465,3.07036) -- (9.16465,3.11381);
\draw [c] (9.14983,3.07036) -- (9.16465,3.07036);
\draw [c] (9.16465,3.07036) -- (9.17947,3.07036);
\definecolor{c}{rgb}{0,0,0};
\colorlet{c}{kugray};
\draw [c] (9.19429,3.09316) -- (9.19429,3.13551);
\draw [c] (9.19429,3.13551) -- (9.19429,3.17274);
\draw [c] (9.17947,3.13551) -- (9.19429,3.13551);
\draw [c] (9.19429,3.13551) -- (9.20911,3.13551);
\definecolor{c}{rgb}{0,0,0};
\colorlet{c}{kugray};
\draw [c] (9.22392,3.06278) -- (9.22392,3.109);
\draw [c] (9.22392,3.109) -- (9.22392,3.14918);
\draw [c] (9.20911,3.109) -- (9.22392,3.109);
\draw [c] (9.22392,3.109) -- (9.23874,3.109);
\definecolor{c}{rgb}{0,0,0};
\colorlet{c}{kugray};
\draw [c] (9.25356,3.08447) -- (9.25356,3.12834);
\draw [c] (9.25356,3.12834) -- (9.25356,3.16674);
\draw [c] (9.23874,3.12834) -- (9.25356,3.12834);
\draw [c] (9.25356,3.12834) -- (9.26838,3.12834);
\definecolor{c}{rgb}{0,0,0};
\colorlet{c}{kugray};
\draw [c] (9.2832,2.99447) -- (9.2832,3.04509);
\draw [c] (9.2832,3.04509) -- (9.2832,3.08855);
\draw [c] (9.26838,3.04509) -- (9.2832,3.04509);
\draw [c] (9.2832,3.04509) -- (9.29801,3.04509);
\definecolor{c}{rgb}{0,0,0};
\colorlet{c}{kugray};
\draw [c] (9.31283,3.07942) -- (9.31283,3.12376);
\draw [c] (9.31283,3.12376) -- (9.31283,3.16252);
\draw [c] (9.29801,3.12376) -- (9.31283,3.12376);
\draw [c] (9.31283,3.12376) -- (9.32765,3.12376);
\definecolor{c}{rgb}{0,0,0};
\colorlet{c}{kugray};
\draw [c] (9.34247,2.98071) -- (9.34247,3.03135);
\draw [c] (9.34247,3.03135) -- (9.34247,3.07482);
\draw [c] (9.32765,3.03135) -- (9.34247,3.03135);
\draw [c] (9.34247,3.03135) -- (9.35728,3.03135);
\definecolor{c}{rgb}{0,0,0};
\colorlet{c}{kugray};
\draw [c] (9.3721,3.09749) -- (9.3721,3.13869);
\draw [c] (9.3721,3.13869) -- (9.3721,3.17501);
\draw [c] (9.35728,3.13869) -- (9.3721,3.13869);
\draw [c] (9.3721,3.13869) -- (9.38692,3.13869);
\definecolor{c}{rgb}{0,0,0};
\colorlet{c}{kugray};
\draw [c] (9.40174,3.0393) -- (9.40174,3.08577);
\draw [c] (9.40174,3.08577) -- (9.40174,3.12613);
\draw [c] (9.38692,3.08577) -- (9.40174,3.08577);
\draw [c] (9.40174,3.08577) -- (9.41656,3.08577);
\definecolor{c}{rgb}{0,0,0};
\colorlet{c}{kugray};
\draw [c] (9.43137,2.98225) -- (9.43137,3.03202);
\draw [c] (9.43137,3.03202) -- (9.43137,3.07485);
\draw [c] (9.41656,3.03202) -- (9.43137,3.03202);
\draw [c] (9.43137,3.03202) -- (9.44619,3.03202);
\definecolor{c}{rgb}{0,0,0};
\colorlet{c}{kugray};
\draw [c] (9.46101,3.05321) -- (9.46101,3.0968);
\draw [c] (9.46101,3.0968) -- (9.46101,3.13497);
\draw [c] (9.44619,3.0968) -- (9.46101,3.0968);
\draw [c] (9.46101,3.0968) -- (9.47583,3.0968);
\definecolor{c}{rgb}{0,0,0};
\colorlet{c}{kugray};
\draw [c] (9.49065,3.04863) -- (9.49065,3.09461);
\draw [c] (9.49065,3.09461) -- (9.49065,3.1346);
\draw [c] (9.47583,3.09461) -- (9.49065,3.09461);
\draw [c] (9.49065,3.09461) -- (9.50546,3.09461);
\definecolor{c}{rgb}{0,0,0};
\colorlet{c}{kugray};
\draw [c] (9.52028,3.06634) -- (9.52028,3.11124);
\draw [c] (9.52028,3.11124) -- (9.52028,3.15042);
\draw [c] (9.50546,3.11124) -- (9.52028,3.11124);
\draw [c] (9.52028,3.11124) -- (9.5351,3.11124);
\definecolor{c}{rgb}{0,0,0};
\colorlet{c}{kugray};
\draw [c] (9.54992,3.06028) -- (9.54992,3.1061);
\draw [c] (9.54992,3.1061) -- (9.54992,3.14597);
\draw [c] (9.5351,3.1061) -- (9.54992,3.1061);
\draw [c] (9.54992,3.1061) -- (9.56474,3.1061);
\definecolor{c}{rgb}{0,0,0};
\colorlet{c}{kugray};
\draw [c] (9.57955,2.92876) -- (9.57955,2.98234);
\draw [c] (9.57955,2.98234) -- (9.57955,3.02797);
\draw [c] (9.56474,2.98234) -- (9.57955,2.98234);
\draw [c] (9.57955,2.98234) -- (9.59437,2.98234);
\definecolor{c}{rgb}{0,0,0};
\colorlet{c}{kugray};
\draw [c] (9.60919,3.05453) -- (9.60919,3.10277);
\draw [c] (9.60919,3.10277) -- (9.60919,3.14446);
\draw [c] (9.59437,3.10277) -- (9.60919,3.10277);
\draw [c] (9.60919,3.10277) -- (9.62401,3.10277);
\definecolor{c}{rgb}{0,0,0};
\colorlet{c}{kugray};
\draw [c] (9.63882,3.06402) -- (9.63882,3.10887);
\draw [c] (9.63882,3.10887) -- (9.63882,3.14802);
\draw [c] (9.62401,3.10887) -- (9.63882,3.10887);
\draw [c] (9.63882,3.10887) -- (9.65364,3.10887);
\definecolor{c}{rgb}{0,0,0};
\colorlet{c}{kugray};
\draw [c] (9.66846,3.03821) -- (9.66846,3.08666);
\draw [c] (9.66846,3.08666) -- (9.66846,3.12851);
\draw [c] (9.65364,3.08666) -- (9.66846,3.08666);
\draw [c] (9.66846,3.08666) -- (9.68328,3.08666);
\definecolor{c}{rgb}{0,0,0};
\colorlet{c}{kugray};
\draw [c] (9.6981,3.05196) -- (9.6981,3.09725);
\draw [c] (9.6981,3.09725) -- (9.6981,3.13673);
\draw [c] (9.68328,3.09725) -- (9.6981,3.09725);
\draw [c] (9.6981,3.09725) -- (9.71291,3.09725);
\definecolor{c}{rgb}{0,0,0};
\colorlet{c}{kugray};
\draw [c] (9.72773,2.99179) -- (9.72773,3.04189);
\draw [c] (9.72773,3.04189) -- (9.72773,3.08497);
\draw [c] (9.71291,3.04189) -- (9.72773,3.04189);
\draw [c] (9.72773,3.04189) -- (9.74255,3.04189);
\definecolor{c}{rgb}{0,0,0};
\colorlet{c}{kugray};
\draw [c] (9.75737,2.97816) -- (9.75737,3.02871);
\draw [c] (9.75737,3.02871) -- (9.75737,3.07212);
\draw [c] (9.74255,3.02871) -- (9.75737,3.02871);
\draw [c] (9.75737,3.02871) -- (9.77219,3.02871);
\definecolor{c}{rgb}{0,0,0};
\colorlet{c}{kugray};
\draw [c] (9.787,3.07791) -- (9.787,3.12253);
\draw [c] (9.787,3.12253) -- (9.787,3.1615);
\draw [c] (9.77219,3.12253) -- (9.787,3.12253);
\draw [c] (9.787,3.12253) -- (9.80182,3.12253);
\definecolor{c}{rgb}{0,0,0};
\colorlet{c}{kugray};
\draw [c] (9.81664,2.90394) -- (9.81664,2.95902);
\draw [c] (9.81664,2.95902) -- (9.81664,3.00573);
\draw [c] (9.80182,2.95902) -- (9.81664,2.95902);
\draw [c] (9.81664,2.95902) -- (9.83146,2.95902);
\definecolor{c}{rgb}{0,0,0};
\colorlet{c}{kugray};
\draw [c] (9.84627,3.10832) -- (9.84627,3.15092);
\draw [c] (9.84627,3.15092) -- (9.84627,3.18834);
\draw [c] (9.83146,3.15092) -- (9.84627,3.15092);
\draw [c] (9.84627,3.15092) -- (9.86109,3.15092);
\definecolor{c}{rgb}{0,0,0};
\colorlet{c}{kugray};
\draw [c] (9.87591,2.95007) -- (9.87591,3.00342);
\draw [c] (9.87591,3.00342) -- (9.87591,3.04887);
\draw [c] (9.86109,3.00342) -- (9.87591,3.00342);
\draw [c] (9.87591,3.00342) -- (9.89073,3.00342);
\definecolor{c}{rgb}{0,0,0};
\colorlet{c}{kugray};
\draw [c] (9.90555,2.93927) -- (9.90555,2.99525);
\draw [c] (9.90555,2.99525) -- (9.90555,3.0426);
\draw [c] (9.89073,2.99525) -- (9.90555,2.99525);
\draw [c] (9.90555,2.99525) -- (9.92036,2.99525);
\definecolor{c}{rgb}{0,0,0};
\colorlet{c}{kugray};
\draw [c] (9.93518,2.97877) -- (9.93518,3.0325);
\draw [c] (9.93518,3.0325) -- (9.93518,3.07823);
\draw [c] (9.92036,3.0325) -- (9.93518,3.0325);
\draw [c] (9.93518,3.0325) -- (9.95,3.0325);
\definecolor{c}{rgb}{0,0,0};
\definecolor{c}{rgb}{1,0.8,0};
\draw [c] (1.51655,5.54612) -- (1.60131,5.40539) -- (1.68607,5.27646) -- (1.77083,5.15727) -- (1.85558,5.04631) -- (1.94034,4.9424) -- (2.0251,4.84466) -- (2.10986,4.75237) -- (2.19462,4.66498) -- (2.27938,4.58203) -- (2.36413,4.50316)
 -- (2.44889,4.42808) -- (2.53365,4.35654) -- (2.61841,4.28835) -- (2.70317,4.22336) -- (2.78793,4.16142) -- (2.87268,4.10244) -- (2.95744,4.04632) -- (3.0422,3.99299) -- (3.12696,3.94238) -- (3.21172,3.89444) -- (3.29648,3.84909) --
 (3.38123,3.80628) -- (3.46599,3.76596) -- (3.55075,3.72806) -- (3.63551,3.69251) -- (3.72027,3.65925) -- (3.80502,3.62819) -- (3.88978,3.59926) -- (3.97454,3.57235) -- (4.0593,3.5474) -- (4.14406,3.52429) -- (4.22882,3.50293) -- (4.31357,3.48322) --
 (4.39833,3.46507) -- (4.48309,3.44838) -- (4.56785,3.43305) -- (4.65261,3.41898) -- (4.73737,3.40609) -- (4.82212,3.39428) -- (4.90688,3.38348) -- (4.99164,3.37359) -- (5.0764,3.36455) -- (5.16116,3.35628) -- (5.24592,3.34871) -- (5.33067,3.34177)
 -- (5.41543,3.3354) -- (5.50019,3.32954) -- (5.58495,3.32415) -- (5.66971,3.31916);
\draw [c] (5.66971,3.31916) -- (5.75447,3.31453) -- (5.83922,3.31022) -- (5.92398,3.30618) -- (6.00874,3.30237) -- (6.0935,3.29876) -- (6.17826,3.29531) -- (6.26301,3.292) -- (6.34777,3.28878) -- (6.43253,3.28563) -- (6.51729,3.28253)
 -- (6.60205,3.27946) -- (6.68681,3.27638) -- (6.77156,3.27328) -- (6.85632,3.27014) -- (6.94108,3.26694) -- (7.02584,3.26367) -- (7.1106,3.2603) -- (7.19536,3.25682) -- (7.28011,3.25323) -- (7.36487,3.2495) -- (7.44963,3.24562) -- (7.53439,3.24159)
 -- (7.61915,3.23739) -- (7.70391,3.23302) -- (7.78866,3.22846) -- (7.87342,3.2237) -- (7.95818,3.21875) -- (8.04294,3.21358) -- (8.1277,3.20821) -- (8.21246,3.2026) -- (8.29721,3.19678) -- (8.38197,3.19071) -- (8.46673,3.18441) -- (8.55149,3.17787)
 -- (8.63625,3.17108) -- (8.72101,3.16405) -- (8.80576,3.15675) -- (8.89052,3.1492) -- (8.97528,3.14139) -- (9.06004,3.13331) -- (9.1448,3.12497) -- (9.22955,3.11636) -- (9.31431,3.10747) -- (9.39907,3.09831) -- (9.48383,3.08888) -- (9.56859,3.07917)
 -- (9.65335,3.06918) -- (9.7381,3.0589) -- (9.82286,3.04835);
\draw [c] (9.82286,3.04835) -- (9.90762,3.03751);
\end{tikzpicture}

\end{infilsf}
\caption{The functions fitted to each of the CalcHEP distributions with different $\Lambda$ values of 0.75 TeV (yellow), 1.00 TeV (red) and $\infty$ (the SM, green). The fit is carried out in the range $150 \le M_{\gamma\gamma} \le 3\,000$ TeV. The fit has $\chi^2$ / Ndf = 938.8 / 814.}\label{simfit}
\end{figure}

The fit estimates the parameters to be, for the SM curve:\allowdisplaybreaks
\begin{align*}
p_1                      & \quad = &  7.81\times10^{-6}   \quad & \pm    0.31\times10^{-6} \\
p_2                      & \quad = &      6.96   \quad & \pm    0.14    \\
p_3                      & \quad = &      6.894   \quad & \pm    0.018    \\
p_4                      & \quad = &    -0.553   \quad & \pm    0.003  \\
\intertext{For the $\Lambda = 1.00$ TeV curve:}
p_5                      & \quad = &     0.014   \quad & \pm    0.002  \\
p_6                      & \quad = &      2540   \quad & \pm    61     \\
p_7                      & \quad = &      476   \quad & \pm    96     \\
p_8                      & \quad = &     -11.6   \quad & \pm    0.6   \\
p_9                      & \quad = &     0.12   \quad & \pm    0.09   \\
\intertext{And for the $\Lambda=0.75$ TeV curve:}
p_{10}                     & \quad = &     0.19   \quad & \pm    0.03   \\
p_{11}                     & \quad = &      2199   \quad & \pm    61     \\
p_{12}                     & \quad = &      740   \quad & \pm    86     \\
p_{13}                     & \quad = &     -7.6   \quad & \pm    0.7    \\
p_{14}                     & \quad = &    -0.39   \quad & \pm    0.08   
\end{align*}

The error bands in fig.~\ref{simfit} represent the 95\% confidence intervals on the functions, as calculated from the covariance matrix produced by the fitting procedure. This confidence interval represents the systematic uncertainty that arises from the uncertainty on the fit parameters.

It is now a simple matter to determine the polynomial coefficients, which are plotted in fig.~\ref{coef}.

\begin{figure}[hbt]
\begin{minipage}[b]{.69\textwidth}
\begin{infilsf}\tiny
\begin{tikzpicture}[x=.16\textwidth,y=.16\textwidth]
\pgfdeclareplotmark{cross} {
\pgfpathmoveto{\pgfpoint{-0.3\pgfplotmarksize}{\pgfplotmarksize}}
\pgfpathlineto{\pgfpoint{+0.3\pgfplotmarksize}{\pgfplotmarksize}}
\pgfpathlineto{\pgfpoint{+0.3\pgfplotmarksize}{0.3\pgfplotmarksize}}
\pgfpathlineto{\pgfpoint{+1\pgfplotmarksize}{0.3\pgfplotmarksize}}
\pgfpathlineto{\pgfpoint{+1\pgfplotmarksize}{-0.3\pgfplotmarksize}}
\pgfpathlineto{\pgfpoint{+0.3\pgfplotmarksize}{-0.3\pgfplotmarksize}}
\pgfpathlineto{\pgfpoint{+0.3\pgfplotmarksize}{-1.\pgfplotmarksize}}
\pgfpathlineto{\pgfpoint{-0.3\pgfplotmarksize}{-1.\pgfplotmarksize}}
\pgfpathlineto{\pgfpoint{-0.3\pgfplotmarksize}{-0.3\pgfplotmarksize}}
\pgfpathlineto{\pgfpoint{-1.\pgfplotmarksize}{-0.3\pgfplotmarksize}}
\pgfpathlineto{\pgfpoint{-1.\pgfplotmarksize}{0.3\pgfplotmarksize}}
\pgfpathlineto{\pgfpoint{-0.3\pgfplotmarksize}{0.3\pgfplotmarksize}}
\pgfpathclose
\pgfusepathqstroke
}
\pgfdeclareplotmark{cross*} {
\pgfpathmoveto{\pgfpoint{-0.3\pgfplotmarksize}{\pgfplotmarksize}}
\pgfpathlineto{\pgfpoint{+0.3\pgfplotmarksize}{\pgfplotmarksize}}
\pgfpathlineto{\pgfpoint{+0.3\pgfplotmarksize}{0.3\pgfplotmarksize}}
\pgfpathlineto{\pgfpoint{+1\pgfplotmarksize}{0.3\pgfplotmarksize}}
\pgfpathlineto{\pgfpoint{+1\pgfplotmarksize}{-0.3\pgfplotmarksize}}
\pgfpathlineto{\pgfpoint{+0.3\pgfplotmarksize}{-0.3\pgfplotmarksize}}
\pgfpathlineto{\pgfpoint{+0.3\pgfplotmarksize}{-1.\pgfplotmarksize}}
\pgfpathlineto{\pgfpoint{-0.3\pgfplotmarksize}{-1.\pgfplotmarksize}}
\pgfpathlineto{\pgfpoint{-0.3\pgfplotmarksize}{-0.3\pgfplotmarksize}}
\pgfpathlineto{\pgfpoint{-1.\pgfplotmarksize}{-0.3\pgfplotmarksize}}
\pgfpathlineto{\pgfpoint{-1.\pgfplotmarksize}{0.3\pgfplotmarksize}}
\pgfpathlineto{\pgfpoint{-0.3\pgfplotmarksize}{0.3\pgfplotmarksize}}
\pgfpathclose
\pgfusepathqfillstroke
}
\pgfdeclareplotmark{newstar} {
\pgfpathmoveto{\pgfqpoint{0pt}{\pgfplotmarksize}}
\pgfpathlineto{\pgfqpointpolar{44}{0.5\pgfplotmarksize}}
\pgfpathlineto{\pgfqpointpolar{18}{\pgfplotmarksize}}
\pgfpathlineto{\pgfqpointpolar{-20}{0.5\pgfplotmarksize}}
\pgfpathlineto{\pgfqpointpolar{-54}{\pgfplotmarksize}}
\pgfpathlineto{\pgfqpointpolar{-90}{0.5\pgfplotmarksize}}
\pgfpathlineto{\pgfqpointpolar{234}{\pgfplotmarksize}}
\pgfpathlineto{\pgfqpointpolar{198}{0.5\pgfplotmarksize}}
\pgfpathlineto{\pgfqpointpolar{162}{\pgfplotmarksize}}
\pgfpathlineto{\pgfqpointpolar{134}{0.5\pgfplotmarksize}}
\pgfpathclose
\pgfusepathqstroke
}
\pgfdeclareplotmark{newstar*} {
\pgfpathmoveto{\pgfqpoint{0pt}{\pgfplotmarksize}}
\pgfpathlineto{\pgfqpointpolar{44}{0.5\pgfplotmarksize}}
\pgfpathlineto{\pgfqpointpolar{18}{\pgfplotmarksize}}
\pgfpathlineto{\pgfqpointpolar{-20}{0.5\pgfplotmarksize}}
\pgfpathlineto{\pgfqpointpolar{-54}{\pgfplotmarksize}}
\pgfpathlineto{\pgfqpointpolar{-90}{0.5\pgfplotmarksize}}
\pgfpathlineto{\pgfqpointpolar{234}{\pgfplotmarksize}}
\pgfpathlineto{\pgfqpointpolar{198}{0.5\pgfplotmarksize}}
\pgfpathlineto{\pgfqpointpolar{162}{\pgfplotmarksize}}
\pgfpathlineto{\pgfqpointpolar{134}{0.5\pgfplotmarksize}}
\pgfpathclose
\pgfusepathqfillstroke
}
\definecolor{c}{rgb}{1,1,1};
\draw [color=c, fill=c] (0,0) rectangle (5.82482,10);
\draw [color=c, fill=c] (0,6.7) rectangle (5.82482,10);
\draw [color=c, fill=c] (0.582482,7.03) rectangle (5.79569,9.967);
\definecolor{c}{rgb}{0,0,0};
\draw [c] (0.582482,7.03) -- (0.582482,9.967) -- (5.79569,9.967) -- (5.79569,7.03) -- (0.582482,7.03);
\definecolor{c}{rgb}{1,1,1};
\draw [color=c, fill=c] (0.582482,7.03) rectangle (5.79569,9.967);
\definecolor{c}{rgb}{0,0,0};
\draw [c] (0.582482,7.03) -- (0.582482,9.967) -- (5.79569,9.967) -- (5.79569,7.03) -- (0.582482,7.03);
\colorlet{c}{natgreen!20};
\draw [c, fill=c] (0.60992,9.86662) -- (0.664796,9.73169) -- (0.719672,9.61154) -- (0.774547,9.50445) -- (0.829423,9.40897) -- (0.884299,9.32374) -- (0.939175,9.24746) -- (0.994051,9.17886) -- (1.04893,9.11677) -- (1.1038,9.06014) --
 (1.15868,9.00805) -- (1.21355,8.95973) -- (1.26843,8.91456) -- (1.32331,8.87203) -- (1.37818,8.83172) -- (1.43306,8.79333) -- (1.48793,8.75657) -- (1.54281,8.72124) -- (1.59769,8.68718) -- (1.65256,8.65423) -- (1.70744,8.62228) -- (1.76231,8.59124)
 -- (1.81719,8.56103) -- (1.87207,8.53157) -- (1.92694,8.5028) -- (1.98182,8.47468) -- (2.03669,8.44717) -- (2.09157,8.42021) -- (2.14645,8.39377) -- (2.20132,8.36704) -- (2.2562,8.34168) -- (2.31107,8.31673) -- (2.36595,8.29219) -- (2.42083,8.26803)
 -- (2.4757,8.24422) -- (2.53058,8.22077) -- (2.58545,8.19764) -- (2.64033,8.17483) -- (2.6952,8.15233) -- (2.75008,8.13011) -- (2.80496,8.10817) -- (2.85983,8.0865) -- (2.91471,8.06509) -- (2.96958,8.04392) -- (3.02446,8.02298) -- (3.07934,8.00228)
 -- (3.13421,7.98179) -- (3.18909,7.96151) -- (3.24396,7.94143) -- (3.29884,7.92155) -- (3.35372,7.90186) -- (3.40859,7.88234) -- (3.46347,7.863) -- (3.51834,7.84383) -- (3.57322,7.82483) -- (3.62809,7.80598) -- (3.68297,7.78728) -- (3.73785,7.76872)
 -- (3.79272,7.75031) -- (3.8476,7.73204) -- (3.90247,7.7139) -- (3.95735,7.69588) -- (4.01223,7.67799) -- (4.0671,7.66022) -- (4.12198,7.64256) -- (4.17685,7.62501) -- (4.23173,7.60757) -- (4.28661,7.59024) -- (4.34148,7.57301) -- (4.39636,7.55587)
 -- (4.45123,7.53883) -- (4.50611,7.52188) -- (4.56099,7.50501) -- (4.61586,7.48823) -- (4.67074,7.47154) -- (4.72561,7.45492) -- (4.78049,7.43838) -- (4.83537,7.42191) -- (4.89024,7.40552) -- (4.94512,7.38919) -- (4.99999,7.37294) --
 (5.05487,7.35674) -- (5.10974,7.34061) -- (5.16462,7.32453) -- (5.2195,7.30852) -- (5.27437,7.29256) -- (5.32925,7.27665) -- (5.38412,7.26079) -- (5.439,7.24498) -- (5.49388,7.22922) -- (5.54875,7.21351) -- (5.60363,7.19784) -- (5.6585,7.18221) --
 (5.71338,7.16661) -- (5.76826,7.15106) -- (5.76826,7.13887) -- (5.71338,7.15465) -- (5.6585,7.17047) -- (5.60363,7.18632) -- (5.54875,7.20222) -- (5.49388,7.21815) -- (5.439,7.23413) -- (5.38412,7.25015) -- (5.32925,7.26623) -- (5.27437,7.28235) --
 (5.2195,7.29852) -- (5.16462,7.31474) -- (5.10974,7.33103) -- (5.05487,7.34736) -- (4.99999,7.36376) -- (4.94512,7.38023) -- (4.89024,7.39675) -- (4.83537,7.41335) -- (4.78049,7.43001) -- (4.72561,7.44675) -- (4.67074,7.46356) -- (4.61586,7.48044)
 -- (4.56099,7.49741) -- (4.50611,7.51447) -- (4.45123,7.5316) -- (4.39636,7.54883) -- (4.34148,7.56615) -- (4.28661,7.58357) -- (4.23173,7.60108) -- (4.17685,7.61869) -- (4.12198,7.63642) -- (4.0671,7.65425) -- (4.01223,7.67219) -- (3.95735,7.69025)
 -- (3.90247,7.70844) -- (3.8476,7.72674) -- (3.79272,7.74518) -- (3.73785,7.76375) -- (3.68297,7.78246) -- (3.62809,7.80132) -- (3.57322,7.82032) -- (3.51834,7.83948) -- (3.46347,7.8588) -- (3.40859,7.87829) -- (3.35372,7.89794) -- (3.29884,7.91778)
 -- (3.24396,7.9378) -- (3.18909,7.95801) -- (3.13421,7.97842) -- (3.07934,7.99903) -- (3.02446,8.01987) -- (2.96958,8.04092) -- (2.91471,8.06221) -- (2.85983,8.08374) -- (2.80496,8.10552) -- (2.75008,8.12756) -- (2.6952,8.14987) -- (2.64033,8.17247)
 -- (2.58545,8.19537) -- (2.53058,8.21857) -- (2.4757,8.24211) -- (2.42083,8.26598) -- (2.36595,8.2902) -- (2.31107,8.3148) -- (2.2562,8.33979) -- (2.20132,8.36519) -- (2.14645,8.39008) -- (2.09157,8.41619) -- (2.03669,8.44275) -- (1.98182,8.46977)
 -- (1.92694,8.49727) -- (1.87207,8.52527) -- (1.81719,8.55377) -- (1.76231,8.58281) -- (1.70744,8.61238) -- (1.65256,8.6425) -- (1.59769,8.67315) -- (1.54281,8.70431) -- (1.48793,8.73594) -- (1.43306,8.76797) -- (1.37818,8.80025) --
 (1.32331,8.83257) -- (1.26843,8.86454) -- (1.21355,8.89554) -- (1.15868,8.92444) -- (1.1038,8.94912) -- (1.04893,8.96506) -- (0.994051,8.96056) -- (0.939175,8.8861) -- (0.884299,7.03) -- (0.829423,7.03) -- (0.774547,7.03) -- (0.719672,7.03) --
 (0.664796,7.03) -- (0.60992,7.03);
\definecolor{c}{rgb}{0,0,0};
\draw [c] (0.582482,7.03) -- (5.79569,7.03);
\draw [c] (1.2066,7.11861) -- (1.2066,7.03);
\draw [c] (1.39016,7.0743) -- (1.39016,7.03);
\draw [c] (1.57373,7.0743) -- (1.57373,7.03);
\draw [c] (1.75729,7.0743) -- (1.75729,7.03);
\draw [c] (1.94085,7.0743) -- (1.94085,7.03);
\draw [c] (2.12442,7.11861) -- (2.12442,7.03);
\draw [c] (2.30798,7.0743) -- (2.30798,7.03);
\draw [c] (2.49155,7.0743) -- (2.49155,7.03);
\draw [c] (2.67511,7.0743) -- (2.67511,7.03);
\draw [c] (2.85867,7.0743) -- (2.85867,7.03);
\draw [c] (3.04224,7.11861) -- (3.04224,7.03);
\draw [c] (3.2258,7.0743) -- (3.2258,7.03);
\draw [c] (3.40936,7.0743) -- (3.40936,7.03);
\draw [c] (3.59293,7.0743) -- (3.59293,7.03);
\draw [c] (3.77649,7.0743) -- (3.77649,7.03);
\draw [c] (3.96006,7.11861) -- (3.96006,7.03);
\draw [c] (4.14362,7.0743) -- (4.14362,7.03);
\draw [c] (4.32718,7.0743) -- (4.32718,7.03);
\draw [c] (4.51075,7.0743) -- (4.51075,7.03);
\draw [c] (4.69431,7.0743) -- (4.69431,7.03);
\draw [c] (4.87787,7.11861) -- (4.87787,7.03);
\draw [c] (5.06144,7.0743) -- (5.06144,7.03);
\draw [c] (5.245,7.0743) -- (5.245,7.03);
\draw [c] (5.42857,7.0743) -- (5.42857,7.03);
\draw [c] (5.61213,7.0743) -- (5.61213,7.03);
\draw [c] (5.79569,7.11861) -- (5.79569,7.03);
\draw [c] (1.2066,7.11861) -- (1.2066,7.03);
\draw [c] (1.02303,7.0743) -- (1.02303,7.03);
\draw [c] (0.839471,7.0743) -- (0.839471,7.03);
\draw [c] (0.655907,7.0743) -- (0.655907,7.03);
\draw [anchor=base] (1.2066,6.8716) node[color=c, rotate=0]{500};
\draw [anchor=base] (2.12442,6.8716) node[color=c, rotate=0]{1000};
\draw [anchor=base] (3.04224,6.8716) node[color=c, rotate=0]{1500};
\draw [anchor=base] (3.96006,6.8716) node[color=c, rotate=0]{2000};
\draw [anchor=base] (4.87787,6.8716) node[color=c, rotate=0]{2500};
\draw [anchor=base] (5.79569,6.8716) node[color=c, rotate=0]{3000};
\draw [c] (0.582482,7.03) -- (0.582482,9.967);
\draw [c] (0.660243,7.03821) -- (0.582482,7.03821);
\draw [c] (0.660243,7.05671) -- (0.582482,7.05671);
\draw [c] (0.738004,7.07325) -- (0.582482,7.07325);
\draw [anchor= east] (0.537048,7.07325) node[color=c, rotate=0]{$10^{-4}$};
\draw [c] (0.660243,7.18212) -- (0.582482,7.18212);
\draw [c] (0.660243,7.24581) -- (0.582482,7.24581);
\draw [c] (0.660243,7.291) -- (0.582482,7.291);
\draw [c] (0.660243,7.32604) -- (0.582482,7.32604);
\draw [c] (0.660243,7.35468) -- (0.582482,7.35468);
\draw [c] (0.660243,7.37889) -- (0.582482,7.37889);
\draw [c] (0.660243,7.39987) -- (0.582482,7.39987);
\draw [c] (0.660243,7.41837) -- (0.582482,7.41837);
\draw [c] (0.738004,7.43491) -- (0.582482,7.43491);
\draw [anchor= east] (0.537048,7.43491) node[color=c, rotate=0]{$10^{-3}$};
\draw [c] (0.660243,7.54379) -- (0.582482,7.54379);
\draw [c] (0.660243,7.60747) -- (0.582482,7.60747);
\draw [c] (0.660243,7.65266) -- (0.582482,7.65266);
\draw [c] (0.660243,7.6877) -- (0.582482,7.6877);
\draw [c] (0.660243,7.71634) -- (0.582482,7.71634);
\draw [c] (0.660243,7.74055) -- (0.582482,7.74055);
\draw [c] (0.660243,7.76153) -- (0.582482,7.76153);
\draw [c] (0.660243,7.78003) -- (0.582482,7.78003);
\draw [c] (0.738004,7.79658) -- (0.582482,7.79658);
\draw [anchor= east] (0.537048,7.79658) node[color=c, rotate=0]{$10^{-2}$};
\draw [c] (0.660243,7.90545) -- (0.582482,7.90545);
\draw [c] (0.660243,7.96913) -- (0.582482,7.96913);
\draw [c] (0.660243,8.01432) -- (0.582482,8.01432);
\draw [c] (0.660243,8.04937) -- (0.582482,8.04937);
\draw [c] (0.660243,8.078) -- (0.582482,8.078);
\draw [c] (0.660243,8.10221) -- (0.582482,8.10221);
\draw [c] (0.660243,8.12319) -- (0.582482,8.12319);
\draw [c] (0.660243,8.14169) -- (0.582482,8.14169);
\draw [c] (0.738004,8.15824) -- (0.582482,8.15824);
\draw [anchor= east] (0.537048,8.15824) node[color=c, rotate=0]{$10^{-1}$};
\draw [c] (0.660243,8.26711) -- (0.582482,8.26711);
\draw [c] (0.660243,8.33079) -- (0.582482,8.33079);
\draw [c] (0.660243,8.37598) -- (0.582482,8.37598);
\draw [c] (0.660243,8.41103) -- (0.582482,8.41103);
\draw [c] (0.660243,8.43966) -- (0.582482,8.43966);
\draw [c] (0.660243,8.46387) -- (0.582482,8.46387);
\draw [c] (0.660243,8.48485) -- (0.582482,8.48485);
\draw [c] (0.660243,8.50335) -- (0.582482,8.50335);
\draw [c] (0.738004,8.5199) -- (0.582482,8.5199);
\draw [anchor= east] (0.537048,8.5199) node[color=c, rotate=0]{1};
\draw [c] (0.660243,8.62877) -- (0.582482,8.62877);
\draw [c] (0.660243,8.69245) -- (0.582482,8.69245);
\draw [c] (0.660243,8.73764) -- (0.582482,8.73764);
\draw [c] (0.660243,8.77269) -- (0.582482,8.77269);
\draw [c] (0.660243,8.80132) -- (0.582482,8.80132);
\draw [c] (0.660243,8.82553) -- (0.582482,8.82553);
\draw [c] (0.660243,8.84651) -- (0.582482,8.84651);
\draw [c] (0.660243,8.86501) -- (0.582482,8.86501);
\draw [c] (0.738004,8.88156) -- (0.582482,8.88156);
\draw [anchor= east] (0.537048,8.88156) node[color=c, rotate=0]{10};
\draw [c] (0.660243,8.99043) -- (0.582482,8.99043);
\draw [c] (0.660243,9.05411) -- (0.582482,9.05411);
\draw [c] (0.660243,9.0993) -- (0.582482,9.0993);
\draw [c] (0.660243,9.13435) -- (0.582482,9.13435);
\draw [c] (0.660243,9.16298) -- (0.582482,9.16298);
\draw [c] (0.660243,9.1872) -- (0.582482,9.1872);
\draw [c] (0.660243,9.20817) -- (0.582482,9.20817);
\draw [c] (0.660243,9.22667) -- (0.582482,9.22667);
\draw [c] (0.738004,9.24322) -- (0.582482,9.24322);
\draw [anchor= east] (0.537048,9.24322) node[color=c, rotate=0]{$10^{2}$};
\draw [c] (0.660243,9.35209) -- (0.582482,9.35209);
\draw [c] (0.660243,9.41577) -- (0.582482,9.41577);
\draw [c] (0.660243,9.46096) -- (0.582482,9.46096);
\draw [c] (0.660243,9.49601) -- (0.582482,9.49601);
\draw [c] (0.660243,9.52464) -- (0.582482,9.52464);
\draw [c] (0.660243,9.54886) -- (0.582482,9.54886);
\draw [c] (0.660243,9.56983) -- (0.582482,9.56983);
\draw [c] (0.660243,9.58833) -- (0.582482,9.58833);
\draw [c] (0.738004,9.60488) -- (0.582482,9.60488);
\draw [anchor= east] (0.537048,9.60488) node[color=c, rotate=0]{$10^{3}$};
\draw [c] (0.660243,9.71375) -- (0.582482,9.71375);
\draw [c] (0.660243,9.77743) -- (0.582482,9.77743);
\draw [c] (0.660243,9.82262) -- (0.582482,9.82262);
\draw [c] (0.660243,9.85767) -- (0.582482,9.85767);
\draw [c] (0.660243,9.8863) -- (0.582482,9.8863);
\draw [c] (0.660243,9.91052) -- (0.582482,9.91052);
\draw [c] (0.660243,9.93149) -- (0.582482,9.93149);
\draw [c] (0.660243,9.94999) -- (0.582482,9.94999);
\draw [c] (0.738004,9.96654) -- (0.582482,9.96654);
\draw [anchor= east] (0.537048,9.96654) node[color=c, rotate=0]{$10^{4}$};
\colorlet{c}{natgreen};
\draw [c] (0.608548,9.51914) -- (0.66068,9.45016) -- (0.712812,9.38649) -- (0.764944,9.32724) -- (0.817076,9.27173) -- (0.869208,9.21944) -- (0.921341,9.16994) -- (0.973473,9.12289) -- (1.0256,9.078) -- (1.07774,9.03505) --
 (1.12987,8.99384) -- (1.182,8.95419) -- (1.23413,8.91597) -- (1.28627,8.87904) -- (1.3384,8.8433) -- (1.39053,8.80866) -- (1.44266,8.77504) -- (1.49479,8.74234) -- (1.54693,8.71052) -- (1.59906,8.67951) -- (1.65119,8.64926) -- (1.70332,8.61971) --
 (1.75545,8.59084) -- (1.80759,8.56258) -- (1.85972,8.53492) -- (1.91185,8.50781) -- (1.96398,8.48123) -- (2.01611,8.45514) -- (2.06825,8.42953) -- (2.12038,8.40436) -- (2.17251,8.37962) -- (2.22464,8.35528) -- (2.27678,8.33133) -- (2.32891,8.30774)
 -- (2.38104,8.28451) -- (2.43317,8.26161) -- (2.4853,8.23903) -- (2.53744,8.21676) -- (2.58957,8.19479) -- (2.6417,8.17309) -- (2.69383,8.15166) -- (2.74596,8.1305) -- (2.7981,8.10958) -- (2.85023,8.08891) -- (2.90236,8.06846) -- (2.95449,8.04824)
 -- (3.00663,8.02823) -- (3.05876,8.00842) -- (3.11089,7.98882) -- (3.16302,7.9694);
\draw [c] (3.16302,7.9694) -- (3.21515,7.95017) -- (3.26729,7.93112) -- (3.31942,7.91224) -- (3.37155,7.89353) -- (3.42368,7.87497) -- (3.47581,7.85657) -- (3.52795,7.83832) -- (3.58008,7.82022) -- (3.63221,7.80225) --
 (3.68434,7.78442) -- (3.73647,7.76672) -- (3.78861,7.74915) -- (3.84074,7.7317) -- (3.89287,7.71437) -- (3.945,7.69715) -- (3.99714,7.68005) -- (4.04927,7.66305) -- (4.1014,7.64616) -- (4.15353,7.62937) -- (4.20566,7.61267) -- (4.2578,7.59607) --
 (4.30993,7.57956) -- (4.36206,7.56314) -- (4.41419,7.54681) -- (4.46632,7.53056) -- (4.51846,7.51439) -- (4.57059,7.4983) -- (4.62272,7.48228) -- (4.67485,7.46634) -- (4.72698,7.45047) -- (4.77912,7.43467) -- (4.83125,7.41893) -- (4.88338,7.40325)
 -- (4.93551,7.38764) -- (4.98765,7.37209) -- (5.03978,7.3566) -- (5.09191,7.34116) -- (5.14404,7.32577) -- (5.19617,7.31044) -- (5.24831,7.29516) -- (5.30044,7.27992) -- (5.35257,7.26473) -- (5.4047,7.24959) -- (5.45683,7.23449) -- (5.50897,7.21943)
 -- (5.5611,7.20441) -- (5.61323,7.18943) -- (5.66536,7.17448) -- (5.7175,7.15957);
\draw [c] (5.7175,7.15957) -- (5.76963,7.14469);
\definecolor{c}{rgb}{1,1,1};
\draw [color=c, fill=c] (0,3.3) rectangle (5.82482,6.7);
\draw [color=c, fill=c] (0.582482,3.64) rectangle (5.79569,6.666);
\definecolor{c}{rgb}{0,0,0};
\draw [c] (0.582482,3.64) -- (0.582482,6.666) -- (5.79569,6.666) -- (5.79569,3.64) -- (0.582482,3.64);
\definecolor{c}{rgb}{1,1,1};
\draw [color=c, fill=c] (0.582482,3.64) rectangle (5.79569,6.666);
\definecolor{c}{rgb}{0,0,0};
\draw [c] (0.582482,3.64) -- (0.582482,6.666) -- (5.79569,6.666) -- (5.79569,3.64) -- (0.582482,3.64);
\draw [c] (0.582482,3.64) -- (5.79569,3.64);
\draw [c] (1.2227,3.73129) -- (1.2227,3.64);
\draw [c] (1.40562,3.68565) -- (1.40562,3.64);
\draw [c] (1.58854,3.68565) -- (1.58854,3.64);
\draw [c] (1.77146,3.68565) -- (1.77146,3.64);
\draw [c] (1.95438,3.68565) -- (1.95438,3.64);
\draw [c] (2.1373,3.73129) -- (2.1373,3.64);
\draw [c] (2.32022,3.68565) -- (2.32022,3.64);
\draw [c] (2.50314,3.68565) -- (2.50314,3.64);
\draw [c] (2.68606,3.68565) -- (2.68606,3.64);
\draw [c] (2.86898,3.68565) -- (2.86898,3.64);
\draw [c] (3.0519,3.73129) -- (3.0519,3.64);
\draw [c] (3.23482,3.68565) -- (3.23482,3.64);
\draw [c] (3.41774,3.68565) -- (3.41774,3.64);
\draw [c] (3.60066,3.68565) -- (3.60066,3.64);
\draw [c] (3.78358,3.68565) -- (3.78358,3.64);
\draw [c] (3.9665,3.73129) -- (3.9665,3.64);
\draw [c] (4.14942,3.68565) -- (4.14942,3.64);
\draw [c] (4.33234,3.68565) -- (4.33234,3.64);
\draw [c] (4.51526,3.68565) -- (4.51526,3.64);
\draw [c] (4.69818,3.68565) -- (4.69818,3.64);
\draw [c] (4.88109,3.73129) -- (4.88109,3.64);
\draw [c] (5.06401,3.68565) -- (5.06401,3.64);
\draw [c] (5.24693,3.68565) -- (5.24693,3.64);
\draw [c] (5.42985,3.68565) -- (5.42985,3.64);
\draw [c] (5.61277,3.68565) -- (5.61277,3.64);
\draw [c] (5.79569,3.73129) -- (5.79569,3.64);
\draw [c] (1.2227,3.73129) -- (1.2227,3.64);
\draw [c] (1.03978,3.68565) -- (1.03978,3.64);
\draw [c] (0.856861,3.68565) -- (0.856861,3.64);
\draw [c] (0.673942,3.68565) -- (0.673942,3.64);
\draw [anchor=base] (1.2227,3.5278) node[color=c, rotate=0]{500};
\draw [anchor=base] (2.1373,3.5278) node[color=c, rotate=0]{1000};
\draw [anchor=base] (3.0519,3.5278) node[color=c, rotate=0]{1500};
\draw [anchor=base] (3.9665,3.5278) node[color=c, rotate=0]{2000};
\draw [anchor=base] (4.88109,3.5278) node[color=c, rotate=0]{2500};
\draw [anchor=base] (5.79569,3.5278) node[color=c, rotate=0]{3000};
\draw [c] (0.582482,3.64) -- (0.582482,6.666);
\draw [c] (0.738004,3.64) -- (0.582482,3.64);
\draw [c] (0.660243,3.70685) -- (0.582482,3.70685);
\draw [c] (0.660243,3.7737) -- (0.582482,3.7737);
\draw [c] (0.660243,3.84056) -- (0.582482,3.84056);
\draw [c] (0.660243,3.90741) -- (0.582482,3.90741);
\draw [c] (0.738004,3.97426) -- (0.582482,3.97426);
\draw [c] (0.660243,4.04111) -- (0.582482,4.04111);
\draw [c] (0.660243,4.10796) -- (0.582482,4.10796);
\draw [c] (0.660243,4.17481) -- (0.582482,4.17481);
\draw [c] (0.660243,4.24167) -- (0.582482,4.24167);
\draw [c] (0.738004,4.30852) -- (0.582482,4.30852);
\draw [c] (0.660243,4.37537) -- (0.582482,4.37537);
\draw [c] (0.660243,4.44222) -- (0.582482,4.44222);
\draw [c] (0.660243,4.50907) -- (0.582482,4.50907);
\draw [c] (0.660243,4.57593) -- (0.582482,4.57593);
\draw [c] (0.738004,4.64278) -- (0.582482,4.64278);
\draw [c] (0.660243,4.70963) -- (0.582482,4.70963);
\draw [c] (0.660243,4.77648) -- (0.582482,4.77648);
\draw [c] (0.660243,4.84333) -- (0.582482,4.84333);
\draw [c] (0.660243,4.91018) -- (0.582482,4.91018);
\draw [c] (0.738004,4.97704) -- (0.582482,4.97704);
\draw [c] (0.660243,5.04389) -- (0.582482,5.04389);
\draw [c] (0.660243,5.11074) -- (0.582482,5.11074);
\draw [c] (0.660243,5.17759) -- (0.582482,5.17759);
\draw [c] (0.660243,5.24444) -- (0.582482,5.24444);
\draw [c] (0.738004,5.3113) -- (0.582482,5.3113);
\draw [c] (0.660243,5.37815) -- (0.582482,5.37815);
\draw [c] (0.660243,5.445) -- (0.582482,5.445);
\draw [c] (0.660243,5.51185) -- (0.582482,5.51185);
\draw [c] (0.660243,5.5787) -- (0.582482,5.5787);
\draw [c] (0.738004,5.64556) -- (0.582482,5.64556);
\draw [c] (0.660243,5.71241) -- (0.582482,5.71241);
\draw [c] (0.660243,5.77926) -- (0.582482,5.77926);
\draw [c] (0.660243,5.84611) -- (0.582482,5.84611);
\draw [c] (0.660243,5.91296) -- (0.582482,5.91296);
\draw [c] (0.738004,5.97981) -- (0.582482,5.97981);
\draw [c] (0.660243,6.04667) -- (0.582482,6.04667);
\draw [c] (0.660243,6.11352) -- (0.582482,6.11352);
\draw [c] (0.660243,6.18037) -- (0.582482,6.18037);
\draw [c] (0.660243,6.24722) -- (0.582482,6.24722);
\draw [c] (0.738004,6.31407) -- (0.582482,6.31407);
\draw [c] (0.660243,6.38093) -- (0.582482,6.38093);
\draw [c] (0.660243,6.44778) -- (0.582482,6.44778);
\draw [c] (0.660243,6.51463) -- (0.582482,6.51463);
\draw [c] (0.660243,6.58148) -- (0.582482,6.58148);
\draw [c] (0.738004,6.64833) -- (0.582482,6.64833);
\draw [c] (0.738004,6.64833) -- (0.582482,6.64833);
\draw [anchor= east] (0.553358,3.64) node[color=c, rotate=0]{0};
\draw [anchor= east] (0.553358,3.97426) node[color=c, rotate=0]{0.05};
\draw [anchor= east] (0.553358,4.30852) node[color=c, rotate=0]{0.1};
\draw [anchor= east] (0.553358,4.64278) node[color=c, rotate=0]{0.15};
\draw [anchor= east] (0.553358,4.97704) node[color=c, rotate=0]{0.2};
\draw [anchor= east] (0.553358,5.3113) node[color=c, rotate=0]{0.25};
\draw [anchor= east] (0.553358,5.64556) node[color=c, rotate=0]{0.3};
\draw [anchor= east] (0.553358,5.97981) node[color=c, rotate=0]{0.35};
\draw [anchor= east] (0.553358,6.31407) node[color=c, rotate=0]{0.4};
\draw [anchor= east] (0.553358,6.64833) node[color=c, rotate=0]{0.45};
\colorlet{c}{natgreen!20};
\draw [c, fill=c] (0.628115,6.666) -- (0.682799,6.666) -- (0.737482,6.666) -- (0.792166,6.666) -- (0.846849,6.666) -- (0.901532,6.666) -- (0.956216,6.666) -- (1.0109,6.666) -- (1.06558,6.666) -- (1.12027,6.666) -- (1.17495,6.666) -- (1.22963,6.666)
 -- (1.28432,6.666) -- (1.339,6.666) -- (1.39368,6.666) -- (1.44837,6.666) -- (1.50305,6.666) -- (1.55773,6.36242) -- (1.61242,5.84398) -- (1.6671,5.48367) -- (1.72178,5.22553) -- (1.77647,5.03457) -- (1.83115,4.88857) -- (1.88583,4.77325) --
 (1.94052,4.67928) -- (1.9952,4.60048) -- (2.04988,4.53272) -- (2.10457,4.4732) -- (2.15925,4.41999) -- (2.21393,4.35542) -- (2.26862,4.31484) -- (2.3233,4.27666) -- (2.37798,4.24061) -- (2.43267,4.20652) -- (2.48735,4.17422) -- (2.54203,4.14357) --
 (2.59672,4.11446) -- (2.6514,4.08678) -- (2.70608,4.06045) -- (2.76077,4.03539) -- (2.81545,4.01152) -- (2.87013,3.98878) -- (2.92482,3.96712) -- (2.9795,3.9465) -- (3.03418,3.92687) -- (3.08887,3.90821) -- (3.14355,3.89049) -- (3.19823,3.87369) --
 (3.25292,3.85778) -- (3.3076,3.84275) -- (3.36228,3.82859) -- (3.41697,3.81527) -- (3.47165,3.80278) -- (3.52633,3.79109) -- (3.58102,3.78018) -- (3.6357,3.77002) -- (3.69038,3.7606) -- (3.74507,3.75186) -- (3.79975,3.7438) -- (3.85443,3.73637) --
 (3.90912,3.72954) -- (3.9638,3.7233) -- (4.01848,3.71759) -- (4.07317,3.71241) -- (4.12785,3.70772) -- (4.18253,3.70349) -- (4.23722,3.69969) -- (4.2919,3.6963) -- (4.34658,3.69329) -- (4.40127,3.69061) -- (4.45595,3.68823) -- (4.51063,3.68611) --
 (4.56532,3.68421) -- (4.62,3.68249) -- (4.67468,3.6809) -- (4.72937,3.67939) -- (4.78405,3.67795) -- (4.83873,3.67652) -- (4.89342,3.67509) -- (4.9481,3.67364) -- (5.00278,3.67216) -- (5.05747,3.67065) -- (5.11215,3.66912) -- (5.16683,3.66758) --
 (5.22152,3.66605) -- (5.2762,3.66456) -- (5.33088,3.66312) -- (5.38557,3.66175) -- (5.44025,3.66044) -- (5.49494,3.65919) -- (5.54962,3.65799) -- (5.6043,3.65681) -- (5.65899,3.65565) -- (5.71367,3.6545) -- (5.76835,3.65336) -- (5.76835,3.64) --
 (5.71367,3.64) -- (5.65899,3.64) -- (5.6043,3.64195) -- (5.54962,3.64408) -- (5.49494,3.64616) -- (5.44025,3.64817) -- (5.38557,3.65006) -- (5.33088,3.65183) -- (5.2762,3.65345) -- (5.22152,3.65492) -- (5.16683,3.65627) -- (5.11215,3.65752) --
 (5.05747,3.6587) -- (5.00278,3.65985) -- (4.9481,3.66098) -- (4.89342,3.66213) -- (4.83873,3.66332) -- (4.78405,3.66457) -- (4.72937,3.66591) -- (4.67468,3.66736) -- (4.62,3.66893) -- (4.56532,3.67065) -- (4.51063,3.67254) -- (4.45595,3.67463) --
 (4.40127,3.67694) -- (4.34658,3.6795) -- (4.2919,3.68236) -- (4.23722,3.68556) -- (4.18253,3.68912) -- (4.12785,3.69311) -- (4.07317,3.69755) -- (4.01848,3.70249) -- (3.9638,3.70797) -- (3.90912,3.71401) -- (3.85443,3.72066) -- (3.79975,3.72793) --
 (3.74507,3.73586) -- (3.69038,3.74444) -- (3.6357,3.75372) -- (3.58102,3.76368) -- (3.52633,3.77436) -- (3.47165,3.78577) -- (3.41697,3.7979) -- (3.36228,3.81079) -- (3.3076,3.82445) -- (3.25292,3.83889) -- (3.19823,3.85414) -- (3.14355,3.87022) --
 (3.08887,3.88714) -- (3.03418,3.90493) -- (2.9795,3.9236) -- (2.92482,3.94318) -- (2.87013,3.96367) -- (2.81545,3.98509) -- (2.76077,4.00746) -- (2.70608,4.03076) -- (2.6514,4.05501) -- (2.59672,4.08021) -- (2.54203,4.10634) -- (2.48735,4.13341) --
 (2.43267,4.16138) -- (2.37798,4.19023) -- (2.3233,4.21991) -- (2.26862,4.25036) -- (2.21393,4.28144) -- (2.15925,4.29169) -- (2.10457,4.31661) -- (2.04988,4.33869) -- (1.9952,4.35615) -- (1.94052,4.36636) -- (1.88583,4.36537) -- (1.83115,4.34719) --
 (1.77647,4.30269) -- (1.72178,4.21779) -- (1.6671,4.0705) -- (1.61242,3.82604) -- (1.55773,3.64) -- (1.50305,3.64) -- (1.44837,3.64) -- (1.39368,3.64) -- (1.339,3.64) -- (1.28432,3.64) -- (1.22963,3.64) -- (1.17495,3.64) -- (1.12027,3.64) --
 (1.06558,3.64) -- (1.0109,3.64) -- (0.956216,3.64) -- (0.901532,3.64) -- (0.846849,3.64) -- (0.792166,3.64) -- (0.737482,3.64) -- (0.682799,3.64) -- (0.628115,3.64);
\colorlet{c}{natgreen};
\draw [c] (0.608548,6.5219) -- (0.66068,6.39581) -- (0.712812,6.27505) -- (0.764944,6.1594) -- (0.817076,6.04862) -- (0.869208,5.94251) -- (0.921341,5.84088) -- (0.973473,5.74352) -- (1.0256,5.65027) -- (1.07774,5.56093) --
 (1.12987,5.47535) -- (1.182,5.39336) -- (1.23413,5.31481) -- (1.28627,5.23955) -- (1.3384,5.16744) -- (1.39053,5.09834) -- (1.44266,5.03213) -- (1.49479,4.96867) -- (1.54693,4.90785) -- (1.59906,4.84955) -- (1.65119,4.79367) -- (1.70332,4.7401) --
 (1.75545,4.68873) -- (1.80759,4.63947) -- (1.85972,4.59224) -- (1.91185,4.54694) -- (1.96398,4.50348) -- (2.01611,4.4618) -- (2.06825,4.42181) -- (2.12038,4.38344) -- (2.17251,4.34662) -- (2.22464,4.31129) -- (2.27678,4.27739) -- (2.32891,4.24485)
 -- (2.38104,4.21363) -- (2.43317,4.18367) -- (2.4853,4.15492) -- (2.53744,4.12733) -- (2.58957,4.10088) -- (2.6417,4.0755) -- (2.69383,4.05118) -- (2.74596,4.02786) -- (2.7981,4.00553) -- (2.85023,3.98415) -- (2.90236,3.96369) -- (2.95449,3.94412)
 -- (3.00663,3.92543) -- (3.05876,3.9076) -- (3.11089,3.89059) -- (3.16302,3.8744);
\draw [c] (3.16302,3.8744) -- (3.21515,3.859) -- (3.26729,3.84438) -- (3.31942,3.83053) -- (3.37155,3.81741) -- (3.42368,3.80503) -- (3.47581,3.79337) -- (3.52795,3.7824) -- (3.58008,3.77211) -- (3.63221,3.76249) -- (3.68434,3.75352)
 -- (3.73647,3.74518) -- (3.78861,3.73744) -- (3.84074,3.7303) -- (3.89287,3.72372) -- (3.945,3.71768) -- (3.99714,3.71216) -- (4.04927,3.70713) -- (4.1014,3.70256) -- (4.15353,3.69843) -- (4.20566,3.6947) -- (4.2578,3.69134) -- (4.30993,3.68833) --
 (4.36206,3.68562) -- (4.41419,3.68319) -- (4.46632,3.68101) -- (4.51846,3.67904) -- (4.57059,3.67726) -- (4.62272,3.67563) -- (4.67485,3.67412) -- (4.72698,3.67272) -- (4.77912,3.67138) -- (4.83125,3.6701) -- (4.88338,3.66885) -- (4.93551,3.66761)
 -- (4.98765,3.66637) -- (5.03978,3.66511) -- (5.09191,3.66383) -- (5.14404,3.66251) -- (5.19617,3.66116) -- (5.24831,3.65976) -- (5.30044,3.65833) -- (5.35257,3.65686) -- (5.4047,3.65535) -- (5.45683,3.65381) -- (5.50897,3.65226) -- (5.5611,3.65068)
 -- (5.61323,3.64911) -- (5.66536,3.64754) -- (5.7175,3.64599);
\draw [c] (5.7175,3.64599) -- (5.76963,3.64446);
\definecolor{c}{rgb}{1,1,1};
\draw [color=c, fill=c] (0,0) rectangle (5.82482,3.3);
\draw [color=c, fill=c] (0.582482,0.33) rectangle (5.79569,3.267);
\definecolor{c}{rgb}{0,0,0};
\draw [c] (0.582482,0.33) -- (0.582482,3.267) -- (5.79569,3.267) -- (5.79569,0.33) -- (0.582482,0.33);
\definecolor{c}{rgb}{1,1,1};
\draw [color=c, fill=c] (0.582482,0.33) rectangle (5.79569,3.267);
\definecolor{c}{rgb}{0,0,0};
\draw [c] (0.582482,0.33) -- (0.582482,3.267) -- (5.79569,3.267) -- (5.79569,0.33) -- (0.582482,0.33);
\draw [c] (0.582482,0.33) -- (5.79569,0.33);
\draw [anchor= east] (5.79569,0.1252) node[color=c,
rotate=0]{$M_{\gamma\gamma}$ [GeV]};
\draw [c] (1.2227,0.418605) -- (1.2227,0.33);
\draw [c] (1.40562,0.374303) -- (1.40562,0.33);
\draw [c] (1.58854,0.374303) -- (1.58854,0.33);
\draw [c] (1.77146,0.374303) -- (1.77146,0.33);
\draw [c] (1.95438,0.374303) -- (1.95438,0.33);
\draw [c] (2.1373,0.418605) -- (2.1373,0.33);
\draw [c] (2.32022,0.374303) -- (2.32022,0.33);
\draw [c] (2.50314,0.374303) -- (2.50314,0.33);
\draw [c] (2.68606,0.374303) -- (2.68606,0.33);
\draw [c] (2.86898,0.374303) -- (2.86898,0.33);
\draw [c] (3.0519,0.418605) -- (3.0519,0.33);
\draw [c] (3.23482,0.374303) -- (3.23482,0.33);
\draw [c] (3.41774,0.374303) -- (3.41774,0.33);
\draw [c] (3.60066,0.374303) -- (3.60066,0.33);
\draw [c] (3.78358,0.374303) -- (3.78358,0.33);
\draw [c] (3.9665,0.418605) -- (3.9665,0.33);
\draw [c] (4.14942,0.374303) -- (4.14942,0.33);
\draw [c] (4.33234,0.374303) -- (4.33234,0.33);
\draw [c] (4.51526,0.374303) -- (4.51526,0.33);
\draw [c] (4.69818,0.374303) -- (4.69818,0.33);
\draw [c] (4.88109,0.418605) -- (4.88109,0.33);
\draw [c] (5.06401,0.374303) -- (5.06401,0.33);
\draw [c] (5.24693,0.374303) -- (5.24693,0.33);
\draw [c] (5.42985,0.374303) -- (5.42985,0.33);
\draw [c] (5.61277,0.374303) -- (5.61277,0.33);
\draw [c] (5.79569,0.418605) -- (5.79569,0.33);
\draw [c] (1.2227,0.418605) -- (1.2227,0.33);
\draw [c] (1.03978,0.374303) -- (1.03978,0.33);
\draw [c] (0.856861,0.374303) -- (0.856861,0.33);
\draw [c] (0.673942,0.374303) -- (0.673942,0.33);
\draw [anchor=base] (1.2227,0.2211) node[color=c, rotate=0]{500};
\draw [anchor=base] (2.1373,0.2211) node[color=c, rotate=0]{1000};
\draw [anchor=base] (3.0519,0.2211) node[color=c, rotate=0]{1500};
\draw [anchor=base] (3.9665,0.2211) node[color=c, rotate=0]{2000};
\draw [anchor=base] (4.88109,0.2211) node[color=c, rotate=0]{2500};
\draw [anchor=base] (5.79569,0.2211) node[color=c, rotate=0]{3000};
\draw [c] (0.582482,0.33) -- (0.582482,3.267);
\draw [c] (0.738004,0.791569) -- (0.582482,0.791569);
\draw [c] (0.660243,0.900864) -- (0.582482,0.900864);
\draw [c] (0.660243,1.01016) -- (0.582482,1.01016);
\draw [c] (0.660243,1.11945) -- (0.582482,1.11945);
\draw [c] (0.660243,1.22875) -- (0.582482,1.22875);
\draw [c] (0.738004,1.33804) -- (0.582482,1.33804);
\draw [c] (0.660243,1.44734) -- (0.582482,1.44734);
\draw [c] (0.660243,1.55663) -- (0.582482,1.55663);
\draw [c] (0.660243,1.66593) -- (0.582482,1.66593);
\draw [c] (0.660243,1.77522) -- (0.582482,1.77522);
\draw [c] (0.738004,1.88452) -- (0.582482,1.88452);
\draw [c] (0.660243,1.99381) -- (0.582482,1.99381);
\draw [c] (0.660243,2.10311) -- (0.582482,2.10311);
\draw [c] (0.660243,2.2124) -- (0.582482,2.2124);
\draw [c] (0.660243,2.3217) -- (0.582482,2.3217);
\draw [c] (0.738004,2.43099) -- (0.582482,2.43099);
\draw [c] (0.660243,2.54029) -- (0.582482,2.54029);
\draw [c] (0.660243,2.64958) -- (0.582482,2.64958);
\draw [c] (0.660243,2.75888) -- (0.582482,2.75888);
\draw [c] (0.660243,2.86818) -- (0.582482,2.86818);
\draw [c] (0.738004,2.97747) -- (0.582482,2.97747);
\draw [c] (0.738004,0.791569) -- (0.582482,0.791569);
\draw [c] (0.660243,0.682273) -- (0.582482,0.682273);
\draw [c] (0.660243,0.572978) -- (0.582482,0.572978);
\draw [c] (0.660243,0.463683) -- (0.582482,0.463683);
\draw [c] (0.660243,0.354388) -- (0.582482,0.354388);
\draw [c] (0.738004,2.97747) -- (0.582482,2.97747);
\draw [c] (0.660243,3.08677) -- (0.582482,3.08677);
\draw [c] (0.660243,3.19606) -- (0.582482,3.19606);
\draw [anchor= east] (0.553358,0.791569) node[color=c, rotate=0]{0.05};
\draw [anchor= east] (0.553358,1.33804) node[color=c, rotate=0]{0.1};
\draw [anchor= east] (0.553358,1.88452) node[color=c, rotate=0]{0.15};
\draw [anchor= east] (0.553358,2.43099) node[color=c, rotate=0]{0.2};
\draw [anchor= east] (0.553358,2.97747) node[color=c, rotate=0]{0.25};
\colorlet{c}{natgreen!20};
\draw [c, fill=c] (0.628115,3.267) -- (0.682799,3.267) -- (0.737482,3.267) -- (0.792166,3.267) -- (0.846849,3.267) -- (0.901532,3.267) -- (0.956216,3.267) -- (1.0109,3.267) -- (1.06558,3.267) -- (1.12027,3.267) -- (1.17495,3.267) -- (1.22963,3.267)
 -- (1.28432,3.267) -- (1.339,3.267) -- (1.39368,3.267) -- (1.44837,3.267) -- (1.50305,3.19954) -- (1.55773,2.69736) -- (1.61242,2.35428) -- (1.6671,2.11229) -- (1.72178,1.93572) -- (1.77647,1.80226) -- (1.83115,1.69779) -- (1.88583,1.61319) --
 (1.94052,1.54254) -- (1.9952,1.48191) -- (2.04988,1.42869) -- (2.10457,1.38109) -- (2.15925,1.33788) -- (2.21393,1.28928) -- (2.26862,1.25455) -- (2.3233,1.22181) -- (2.37798,1.19088) -- (2.43267,1.16165) -- (2.48735,1.13399) -- (2.54203,1.10781) --
 (2.59672,1.08301) -- (2.6514,1.05952) -- (2.70608,1.03726) -- (2.76077,1.01614) -- (2.81545,0.996107) -- (2.87013,0.977086) -- (2.92482,0.959017) -- (2.9795,0.941844) -- (3.03418,0.925515) -- (3.08887,0.909981) -- (3.14355,0.895198) --
 (3.19823,0.881124) -- (3.25292,0.86772) -- (3.3076,0.854948) -- (3.36228,0.842774) -- (3.41697,0.831162) -- (3.47165,0.820079) -- (3.52633,0.809492) -- (3.58102,0.799368) -- (3.6357,0.789673) -- (3.69038,0.780373) -- (3.74507,0.771434) --
 (3.79975,0.762822) -- (3.85443,0.754502) -- (3.90912,0.746437) -- (3.9638,0.738593) -- (4.01848,0.730935) -- (4.07317,0.723429) -- (4.12785,0.716042) -- (4.18253,0.70874) -- (4.23722,0.701494) -- (4.2919,0.694274) -- (4.34658,0.687053) --
 (4.40127,0.679805) -- (4.45595,0.672505) -- (4.51063,0.665132) -- (4.56532,0.657665) -- (4.62,0.650088) -- (4.67468,0.642387) -- (4.72937,0.634551) -- (4.78405,0.626574) -- (4.83873,0.618452) -- (4.89342,0.610188) -- (4.9481,0.601789) --
 (5.00278,0.593265) -- (5.05747,0.584633) -- (5.11215,0.575911) -- (5.16683,0.567121) -- (5.22152,0.558285) -- (5.2762,0.549425) -- (5.33088,0.54056) -- (5.38557,0.531706) -- (5.44025,0.522877) -- (5.49494,0.514083) -- (5.54962,0.505336) --
 (5.6043,0.496645) -- (5.65899,0.488021) -- (5.71367,0.479475) -- (5.76835,0.471017) -- (5.76835,0.455783) -- (5.71367,0.46482) -- (5.65899,0.473972) -- (5.6043,0.483209) -- (5.54962,0.492499) -- (5.49494,0.501809) -- (5.44025,0.511106) --
 (5.38557,0.52036) -- (5.33088,0.529539) -- (5.2762,0.538619) -- (5.22152,0.547579) -- (5.16683,0.556404) -- (5.11215,0.565086) -- (5.05747,0.573621) -- (5.00278,0.582009) -- (4.9481,0.590252) -- (4.89342,0.598353) -- (4.83873,0.606318) --
 (4.78405,0.614153) -- (4.72937,0.621863) -- (4.67468,0.629457) -- (4.62,0.636943) -- (4.56532,0.644332) -- (4.51063,0.651637) -- (4.45595,0.658872) -- (4.40127,0.666053) -- (4.34658,0.673199) -- (4.2919,0.68033) -- (4.23722,0.687469) --
 (4.18253,0.694639) -- (4.12785,0.701866) -- (4.07317,0.709176) -- (4.01848,0.716597) -- (3.9638,0.724157) -- (3.90912,0.731885) -- (3.85443,0.73981) -- (3.79975,0.747962) -- (3.74507,0.756374) -- (3.69038,0.765078) -- (3.6357,0.774107) --
 (3.58102,0.783496) -- (3.52633,0.793281) -- (3.47165,0.803499) -- (3.41697,0.814187) -- (3.36228,0.825383) -- (3.3076,0.837125) -- (3.25292,0.849452) -- (3.19823,0.8624) -- (3.14355,0.876006) -- (3.08887,0.890307) -- (3.03418,0.905337) --
 (2.9795,0.921127) -- (2.92482,0.937706) -- (2.87013,0.955103) -- (2.81545,0.973341) -- (2.76077,0.992439) -- (2.70608,1.01242) -- (2.6514,1.03329) -- (2.59672,1.05506) -- (2.54203,1.07775) -- (2.48735,1.10136) -- (2.43267,1.1259) --
 (2.37798,1.15135) -- (2.3233,1.17772) -- (2.26862,1.20495) -- (2.21393,1.233) -- (2.15925,1.25007) -- (2.10457,1.27555) -- (2.04988,1.29976) -- (1.9952,1.32158) -- (1.94052,1.33936) -- (1.88583,1.35059) -- (1.83115,1.35148) -- (1.77647,1.33624) --
 (1.72178,1.29589) -- (1.6671,1.21644) -- (1.61242,1.07575) -- (1.55773,0.838284) -- (1.50305,0.446198) -- (1.44837,0.33) -- (1.39368,0.33) -- (1.339,0.33) -- (1.28432,0.33) -- (1.22963,0.33) -- (1.17495,0.33) -- (1.12027,0.33) -- (1.06558,0.33) --
 (1.0109,0.33) -- (0.956216,0.33) -- (0.901532,0.33) -- (0.846849,0.33) -- (0.792166,0.33) -- (0.737482,0.33) -- (0.682799,0.33) -- (0.628115,0.33);
\colorlet{c}{natgreen};
\draw [c] (0.608548,3.12714) -- (0.66068,3.02546) -- (0.712812,2.92757) -- (0.764944,2.83333) -- (0.817076,2.74261) -- (0.869208,2.65528) -- (0.921341,2.57121) -- (0.973473,2.49029) -- (1.0256,2.41241) -- (1.07774,2.33744) --
 (1.12987,2.2653) -- (1.182,2.19588) -- (1.23413,2.12909) -- (1.28627,2.06482) -- (1.3384,2.003) -- (1.39053,1.94354) -- (1.44266,1.88636) -- (1.49479,1.83138) -- (1.54693,1.77852) -- (1.59906,1.72771) -- (1.65119,1.67888) -- (1.70332,1.63197) --
 (1.75545,1.5869) -- (1.80759,1.54363) -- (1.85972,1.50207) -- (1.91185,1.46219) -- (1.96398,1.42392) -- (2.01611,1.3872) -- (2.06825,1.35199) -- (2.12038,1.31823) -- (2.17251,1.28588) -- (2.22464,1.25488) -- (2.27678,1.22519) -- (2.32891,1.19676) --
 (2.38104,1.16956) -- (2.43317,1.14353) -- (2.4853,1.11863) -- (2.53744,1.09483) -- (2.58957,1.07208) -- (2.6417,1.05034) -- (2.69383,1.02958) -- (2.74596,1.00975) -- (2.7981,0.99083) -- (2.85023,0.972769) -- (2.90236,0.955538) -- (2.95449,0.939099)
 -- (3.00663,0.923419) -- (3.05876,0.908465) -- (3.11089,0.894201) -- (3.16302,0.880596);
\draw [c] (3.16302,0.880596) -- (3.21515,0.867616) -- (3.26729,0.855229) -- (3.31942,0.843404) -- (3.37155,0.832108) -- (3.42368,0.821311) -- (3.47581,0.810981) -- (3.52795,0.801087) -- (3.58008,0.791599) -- (3.63221,0.782487) --
 (3.68434,0.77372) -- (3.73647,0.765269) -- (3.78861,0.757103) -- (3.84074,0.749194) -- (3.89287,0.741513) -- (3.945,0.73403) -- (3.99714,0.726718) -- (4.04927,0.719549) -- (4.1014,0.712496) -- (4.15353,0.705533) -- (4.20566,0.698635) --
 (4.2578,0.691778) -- (4.30993,0.684937) -- (4.36206,0.678092) -- (4.41419,0.671222) -- (4.46632,0.664308) -- (4.51846,0.657333) -- (4.57059,0.650282) -- (4.62272,0.643141) -- (4.67485,0.635898) -- (4.72698,0.628546) -- (4.77912,0.621076) --
 (4.83125,0.613485) -- (4.88338,0.60577) -- (4.93551,0.597931) -- (4.98765,0.589971) -- (5.03978,0.581893) -- (5.09191,0.573706) -- (5.14404,0.565416) -- (5.19617,0.557035) -- (5.24831,0.548576) -- (5.30044,0.540052) -- (5.35257,0.531478) --
 (5.4047,0.522871) -- (5.45683,0.514248) -- (5.50897,0.505627) -- (5.5611,0.497026) -- (5.61323,0.488464) -- (5.66536,0.47996) -- (5.7175,0.471532);
\draw [c] (5.7175,0.471532) -- (5.76963,0.463197);
\end{tikzpicture}

\end{infilsf}
\end{minipage}\hfill
\begin{minipage}[b]{.3\textwidth}
\caption{From top to bottom, plots of the coefficients to the constant, linear and quadratic terms used to derive distributions of $M_{\gamma\gamma}$ at any $\Lambda$, defined in chapter~\ref{ch.theory}, with uncertainties.}\label{coef}
\end{minipage}
\end{figure}

\section{Maximum likelihood fit}
To determine the most likely value of $\Lambda$, we will use a maximum likelihood fit to attempt to find the value of $\Lambda$ that best fits the data.

A full description of the maximum likelihood method is given in \cite{barlow} or \cite{pdg}. In brief, the probability of observing a number $n_\text{data}$ events, given an expected number $N_\text{exp}$, is, for a single bin, given by Poisson statistics:
\(p(n_\text{data}|N_\text{exp})=\frac{N_\text{exp}^{n_\text{data}}}{n_\text{data}!}e^{-N_\text{exp}}\label{pois}\)
In our case, where the experiment has been carried out, $n_\text{data}$ will be fixed as the value in each bin. Meanwhile, we worked out a method for varying the number of expected events $N_\text{exp}$ as a function of $\Lambda$ above. Thus, the most likely value of $\Lambda$ is the one that maximises equation \eqref{pois}. Taking the logarithm of eq.~\eqref{pois} yields
\(\ln[p(n_\text{data}|N_\text{exp})]=-\ln n_\text{data}!+n_\text{data}\ln N_\text{exp}-N_\text{exp},\)
where obviously the logarithm of $n_\text{data}!$ is a constant for any given bin. Taking the logarithm also allows us to write the likelihood of an ensemble of measured and expected event counts, such as we find in our histograms, as a sum. Finding the value of $\Lambda$ with the highest likelihood associated is now a simple matter of varying our prediction until we discover a maximum.

As a practical matter, since numerical algorithms for extremum finding are actually minimisers, we will write the negative likelihood,
\[-\ln p,\]
and attempt to find a value of $\Lambda$ that minimises this number. 

We can find a confidence interval around the most likely value by determining what the ratio of likelihoods, or difference of log likelihoods, between the most likely value and the value that corresponds to the desired confidence level. The log ratio of the likelihood of a $\chi^2$ distribution at its mean to its likelihood at the 95th percentile is 3.84 \cite{pdg}. We adopt this as the best estimate available for the 95\% confidence level for twice the negative log likelihood ratio for our distributions as well. The factor two arises from taking the logarithm of $\chi^2$.

As an example, figure~\ref{exllr} plots the negative log likelihood ratios associated with searching for the most likely value of $\Lambda$ given our dataset.

\begin{figure}[htp]
\begin{minipage}[b]{.69\textwidth}
\begin{infilsf} \tiny
\begin{tikzpicture}[x=.092\textwidth,y=.092\textwidth]
\pgfdeclareplotmark{cross} {
\pgfpathmoveto{\pgfpoint{-0.3\pgfplotmarksize}{\pgfplotmarksize}}
\pgfpathlineto{\pgfpoint{+0.3\pgfplotmarksize}{\pgfplotmarksize}}
\pgfpathlineto{\pgfpoint{+0.3\pgfplotmarksize}{0.3\pgfplotmarksize}}
\pgfpathlineto{\pgfpoint{+1\pgfplotmarksize}{0.3\pgfplotmarksize}}
\pgfpathlineto{\pgfpoint{+1\pgfplotmarksize}{-0.3\pgfplotmarksize}}
\pgfpathlineto{\pgfpoint{+0.3\pgfplotmarksize}{-0.3\pgfplotmarksize}}
\pgfpathlineto{\pgfpoint{+0.3\pgfplotmarksize}{-1.\pgfplotmarksize}}
\pgfpathlineto{\pgfpoint{-0.3\pgfplotmarksize}{-1.\pgfplotmarksize}}
\pgfpathlineto{\pgfpoint{-0.3\pgfplotmarksize}{-0.3\pgfplotmarksize}}
\pgfpathlineto{\pgfpoint{-1.\pgfplotmarksize}{-0.3\pgfplotmarksize}}
\pgfpathlineto{\pgfpoint{-1.\pgfplotmarksize}{0.3\pgfplotmarksize}}
\pgfpathlineto{\pgfpoint{-0.3\pgfplotmarksize}{0.3\pgfplotmarksize}}
\pgfpathclose
\pgfusepathqstroke
}
\pgfdeclareplotmark{cross*} {
\pgfpathmoveto{\pgfpoint{-0.3\pgfplotmarksize}{\pgfplotmarksize}}
\pgfpathlineto{\pgfpoint{+0.3\pgfplotmarksize}{\pgfplotmarksize}}
\pgfpathlineto{\pgfpoint{+0.3\pgfplotmarksize}{0.3\pgfplotmarksize}}
\pgfpathlineto{\pgfpoint{+1\pgfplotmarksize}{0.3\pgfplotmarksize}}
\pgfpathlineto{\pgfpoint{+1\pgfplotmarksize}{-0.3\pgfplotmarksize}}
\pgfpathlineto{\pgfpoint{+0.3\pgfplotmarksize}{-0.3\pgfplotmarksize}}
\pgfpathlineto{\pgfpoint{+0.3\pgfplotmarksize}{-1.\pgfplotmarksize}}
\pgfpathlineto{\pgfpoint{-0.3\pgfplotmarksize}{-1.\pgfplotmarksize}}
\pgfpathlineto{\pgfpoint{-0.3\pgfplotmarksize}{-0.3\pgfplotmarksize}}
\pgfpathlineto{\pgfpoint{-1.\pgfplotmarksize}{-0.3\pgfplotmarksize}}
\pgfpathlineto{\pgfpoint{-1.\pgfplotmarksize}{0.3\pgfplotmarksize}}
\pgfpathlineto{\pgfpoint{-0.3\pgfplotmarksize}{0.3\pgfplotmarksize}}
\pgfpathclose
\pgfusepathqfillstroke
}
\pgfdeclareplotmark{newstar} {
\pgfpathmoveto{\pgfqpoint{0pt}{\pgfplotmarksize}}
\pgfpathlineto{\pgfqpointpolar{44}{0.5\pgfplotmarksize}}
\pgfpathlineto{\pgfqpointpolar{18}{\pgfplotmarksize}}
\pgfpathlineto{\pgfqpointpolar{-20}{0.5\pgfplotmarksize}}
\pgfpathlineto{\pgfqpointpolar{-54}{\pgfplotmarksize}}
\pgfpathlineto{\pgfqpointpolar{-90}{0.5\pgfplotmarksize}}
\pgfpathlineto{\pgfqpointpolar{234}{\pgfplotmarksize}}
\pgfpathlineto{\pgfqpointpolar{198}{0.5\pgfplotmarksize}}
\pgfpathlineto{\pgfqpointpolar{162}{\pgfplotmarksize}}
\pgfpathlineto{\pgfqpointpolar{134}{0.5\pgfplotmarksize}}
\pgfpathclose
\pgfusepathqstroke
}
\pgfdeclareplotmark{newstar*} {
\pgfpathmoveto{\pgfqpoint{0pt}{\pgfplotmarksize}}
\pgfpathlineto{\pgfqpointpolar{44}{0.5\pgfplotmarksize}}
\pgfpathlineto{\pgfqpointpolar{18}{\pgfplotmarksize}}
\pgfpathlineto{\pgfqpointpolar{-20}{0.5\pgfplotmarksize}}
\pgfpathlineto{\pgfqpointpolar{-54}{\pgfplotmarksize}}
\pgfpathlineto{\pgfqpointpolar{-90}{0.5\pgfplotmarksize}}
\pgfpathlineto{\pgfqpointpolar{234}{\pgfplotmarksize}}
\pgfpathlineto{\pgfqpointpolar{198}{0.5\pgfplotmarksize}}
\pgfpathlineto{\pgfqpointpolar{162}{\pgfplotmarksize}}
\pgfpathlineto{\pgfqpointpolar{134}{0.5\pgfplotmarksize}}
\pgfpathclose
\pgfusepathqfillstroke
}
\definecolor{c}{rgb}{1,1,1};
\draw [color=c, fill=c] (0,0) rectangle (10,6.80516);
\draw [color=c, fill=c] (1,0.680516) rectangle (9.95,6.73711);
\definecolor{c}{rgb}{0,0,0};
\draw [c] (1,0.680516) -- (1,6.73711) -- (9.95,6.73711) -- (9.95,0.680516) -- (1,0.680516);
\definecolor{c}{rgb}{1,1,1};
\draw [color=c, fill=c] (1,0.680516) rectangle (9.95,6.73711);
\definecolor{c}{rgb}{0,0,0};
\draw [c] (1,0.680516) -- (1,6.73711) -- (9.95,6.73711) -- (9.95,0.680516) -- (1,0.680516);
\draw [c] (1,0.680516) -- (9.95,0.680516);
\draw [c] (1.74583,0.863234) -- (1.74583,0.680516);
\draw [c] (1.99444,0.771875) -- (1.99444,0.680516);
\draw [c] (2.24306,0.771875) -- (2.24306,0.680516);
\draw [c] (2.49167,0.771875) -- (2.49167,0.680516);
\draw [c] (2.74028,0.771875) -- (2.74028,0.680516);
\draw [c] (2.98889,0.863234) -- (2.98889,0.680516);
\draw [c] (3.2375,0.771875) -- (3.2375,0.680516);
\draw [c] (3.48611,0.771875) -- (3.48611,0.680516);
\draw [c] (3.73472,0.771875) -- (3.73472,0.680516);
\draw [c] (3.98333,0.771875) -- (3.98333,0.680516);
\draw [c] (4.23194,0.863234) -- (4.23194,0.680516);
\draw [c] (4.48056,0.771875) -- (4.48056,0.680516);
\draw [c] (4.72917,0.771875) -- (4.72917,0.680516);
\draw [c] (4.97778,0.771875) -- (4.97778,0.680516);
\draw [c] (5.22639,0.771875) -- (5.22639,0.680516);
\draw [c] (5.475,0.863234) -- (5.475,0.680516);
\draw [c] (5.72361,0.771875) -- (5.72361,0.680516);
\draw [c] (5.97222,0.771875) -- (5.97222,0.680516);
\draw [c] (6.22083,0.771875) -- (6.22083,0.680516);
\draw [c] (6.46944,0.771875) -- (6.46944,0.680516);
\draw [c] (6.71806,0.863234) -- (6.71806,0.680516);
\draw [c] (6.96667,0.771875) -- (6.96667,0.680516);
\draw [c] (7.21528,0.771875) -- (7.21528,0.680516);
\draw [c] (7.46389,0.771875) -- (7.46389,0.680516);
\draw [c] (7.7125,0.771875) -- (7.7125,0.680516);
\draw [c] (7.96111,0.863234) -- (7.96111,0.680516);
\draw [c] (8.20972,0.771875) -- (8.20972,0.680516);
\draw [c] (8.45833,0.771875) -- (8.45833,0.680516);
\draw [c] (8.70694,0.771875) -- (8.70694,0.680516);
\draw [c] (8.95556,0.771875) -- (8.95556,0.680516);
\draw [c] (9.20417,0.863234) -- (9.20417,0.680516);
\draw [c] (1.74583,0.863234) -- (1.74583,0.680516);
\draw [c] (1.49722,0.771875) -- (1.49722,0.680516);
\draw [c] (1.24861,0.771875) -- (1.24861,0.680516);
\draw [c] (1,0.771875) -- (1,0.680516);
\draw [c] (9.20417,0.863234) -- (9.20417,0.680516);
\draw [c] (9.45278,0.771875) -- (9.45278,0.680516);
\draw [c] (9.70139,0.771875) -- (9.70139,0.680516);
\draw [c] (9.95,0.771875) -- (9.95,0.680516);
\draw [anchor=base] (1.74583,0.455946) node[color=c, rotate=0]{-3};
\draw [anchor=base] (2.98889,0.455946) node[color=c, rotate=0]{-2};
\draw [anchor=base] (4.23194,0.455946) node[color=c, rotate=0]{-1};
\draw [anchor=base] (5.475,0.455946) node[color=c, rotate=0]{0};
\draw [anchor=base] (6.71806,0.455946) node[color=c, rotate=0]{1};
\draw [anchor=base] (7.96111,0.455946) node[color=c, rotate=0]{2};
\draw [anchor=base] (9.20417,0.455946) node[color=c, rotate=0]{3};
\draw [c] (1,0.680516) -- (1,6.73711);
\draw [c] (1.267,1.49583) -- (1,1.49583);
\draw [c] (1.1335,1.73917) -- (1,1.73917);
\draw [c] (1.1335,1.98251) -- (1,1.98251);
\draw [c] (1.1335,2.22585) -- (1,2.22585);
\draw [c] (1.1335,2.46919) -- (1,2.46919);
\draw [c] (1.267,2.71253) -- (1,2.71253);
\draw [c] (1.1335,2.95587) -- (1,2.95587);
\draw [c] (1.1335,3.19921) -- (1,3.19921);
\draw [c] (1.1335,3.44255) -- (1,3.44255);
\draw [c] (1.1335,3.6859) -- (1,3.6859);
\draw [c] (1.267,3.92924) -- (1,3.92924);
\draw [c] (1.1335,4.17258) -- (1,4.17258);
\draw [c] (1.1335,4.41592) -- (1,4.41592);
\draw [c] (1.1335,4.65926) -- (1,4.65926);
\draw [c] (1.1335,4.9026) -- (1,4.9026);
\draw [c] (1.267,5.14594) -- (1,5.14594);
\draw [c] (1.1335,5.38928) -- (1,5.38928);
\draw [c] (1.1335,5.63262) -- (1,5.63262);
\draw [c] (1.1335,5.87596) -- (1,5.87596);
\draw [c] (1.1335,6.11931) -- (1,6.11931);
\draw [c] (1.267,6.36265) -- (1,6.36265);
\draw [c] (1.267,1.49583) -- (1,1.49583);
\draw [c] (1.1335,1.25249) -- (1,1.25249);
\draw [c] (1.1335,1.00914) -- (1,1.00914);
\draw [c] (1.1335,0.765803) -- (1,0.765803);
\draw [c] (1.267,6.36265) -- (1,6.36265);
\draw [c] (1.1335,6.60599) -- (1,6.60599);
\draw [anchor= east] (0.95,1.49583) node[color=c, rotate=0]{0};
\draw [anchor= east] (0.95,2.71253) node[color=c, rotate=0]{5};
\draw [anchor= east] (0.95,3.92924) node[color=c, rotate=0]{10};
\draw [anchor= east] (0.95,5.14594) node[color=c, rotate=0]{15};
\draw [anchor= east] (0.95,6.36265) node[color=c, rotate=0]{20};
\colorlet{c}{natgreen};
\draw [c] (3.39454,6.73711) -- (3.39487,6.73536);
\draw [c] (3.39487,6.73536) -- (3.39562,6.73147) -- (3.39636,6.72758) -- (3.39711,6.7237) -- (3.39785,6.71982) -- (3.3986,6.71594) -- (3.39935,6.71206) -- (3.40009,6.70819) -- (3.40084,6.70432) -- (3.40158,6.70045) -- (3.40233,6.69658)
 -- (3.40307,6.69272) -- (3.40382,6.68886) -- (3.40457,6.685) -- (3.40531,6.68114) -- (3.40606,6.67729) -- (3.4068,6.67344) -- (3.40755,6.66959) -- (3.4083,6.66574) -- (3.40904,6.6619) -- (3.40979,6.65806) -- (3.41053,6.65422) -- (3.41128,6.65038) --
 (3.41202,6.64655) -- (3.41277,6.64272) -- (3.41352,6.63889) -- (3.41426,6.63506) -- (3.41501,6.63124) -- (3.41575,6.62742) -- (3.4165,6.6236) -- (3.41725,6.61978) -- (3.41799,6.61597) -- (3.41874,6.61216) -- (3.41948,6.60835) -- (3.42023,6.60454) --
 (3.42097,6.60074) -- (3.42172,6.59694) -- (3.42247,6.59314) -- (3.42321,6.58934) -- (3.42396,6.58555) -- (3.4247,6.58175) -- (3.42545,6.57797) -- (3.4262,6.57418) -- (3.42694,6.57039) -- (3.42769,6.56661) -- (3.42843,6.56283) -- (3.42918,6.55906) --
 (3.42992,6.55528) -- (3.43067,6.55151) -- (3.43142,6.54774) -- (3.43216,6.54397) -- (3.43291,6.54021) -- (3.43365,6.53645) -- (3.4344,6.53269) -- (3.43515,6.52893) -- (3.43589,6.52518) -- (3.43664,6.52142) -- (3.43738,6.51767) -- (3.43813,6.51393)
 -- (3.43887,6.51018) -- (3.43962,6.50644) -- (3.44037,6.5027) -- (3.44111,6.49896) -- (3.44186,6.49523) -- (3.4426,6.4915) -- (3.44335,6.48777) -- (3.4441,6.48404) -- (3.44484,6.48031) -- (3.44559,6.47659) -- (3.44633,6.47287) -- (3.44708,6.46915)
 -- (3.44782,6.46544) -- (3.44857,6.46172) -- (3.44932,6.45801) -- (3.45006,6.4543) -- (3.45081,6.4506) -- (3.45155,6.4469) -- (3.4523,6.44319) -- (3.45305,6.4395) -- (3.45379,6.4358) -- (3.45454,6.43211) -- (3.45528,6.42842) -- (3.45603,6.42473) --
 (3.45677,6.42104) -- (3.45752,6.41736) -- (3.45827,6.41368) -- (3.45901,6.41) -- (3.45976,6.40632) -- (3.4605,6.40265) -- (3.46125,6.39898) -- (3.462,6.39531) -- (3.46274,6.39164) -- (3.46349,6.38798) -- (3.46423,6.38431) -- (3.46498,6.38065) --
 (3.46572,6.377) -- (3.46647,6.37334) -- (3.46722,6.36969) -- (3.46796,6.36604) -- (3.46871,6.36239) -- (3.46945,6.35875) -- (3.4702,6.35511) -- (3.47095,6.35147) -- (3.47169,6.34783) -- (3.47244,6.34419) -- (3.47318,6.34056) -- (3.47393,6.33693) --
 (3.47467,6.3333) -- (3.47542,6.32968) -- (3.47617,6.32605) -- (3.47691,6.32243) -- (3.47766,6.31881) -- (3.4784,6.3152) -- (3.47915,6.31158) -- (3.4799,6.30797) -- (3.48064,6.30436) -- (3.48139,6.30076) -- (3.48213,6.29715) -- (3.48288,6.29355) --
 (3.48362,6.28995) -- (3.48437,6.28636) -- (3.48512,6.28276) -- (3.48586,6.27917) -- (3.48661,6.27558) -- (3.48735,6.27199) -- (3.4881,6.26841) -- (3.48885,6.26483) -- (3.48959,6.26125) -- (3.49034,6.25767) -- (3.49108,6.25409) -- (3.49183,6.25052)
 -- (3.49257,6.24695) -- (3.49332,6.24338) -- (3.49407,6.23982) -- (3.49481,6.23625) -- (3.49556,6.23269) -- (3.4963,6.22914) -- (3.49705,6.22558) -- (3.4978,6.22203) -- (3.49854,6.21847) -- (3.49929,6.21493) -- (3.50003,6.21138) -- (3.50078,6.20783)
 -- (3.50152,6.20429) -- (3.50227,6.20075) -- (3.50302,6.19722) -- (3.50376,6.19368) -- (3.50451,6.19015) -- (3.50525,6.18662) -- (3.506,6.18309) -- (3.50675,6.17957) -- (3.50749,6.17604) -- (3.50824,6.17252) -- (3.50898,6.16901) -- (3.50973,6.16549)
 -- (3.51047,6.16198) -- (3.51122,6.15847) -- (3.51197,6.15496) -- (3.51271,6.15145) -- (3.51346,6.14795) -- (3.5142,6.14445) -- (3.51495,6.14095) -- (3.5157,6.13745) -- (3.51644,6.13396) -- (3.51719,6.13046) -- (3.51793,6.12697) -- (3.51868,6.12349)
 -- (3.51942,6.12) -- (3.52017,6.11652) -- (3.52092,6.11304) -- (3.52166,6.10956) -- (3.52241,6.10609) -- (3.52315,6.10261) -- (3.5239,6.09914) -- (3.52465,6.09567) -- (3.52539,6.09221) -- (3.52614,6.08874) -- (3.52688,6.08528) -- (3.52763,6.08182)
 -- (3.52837,6.07837) -- (3.52912,6.07491) -- (3.52987,6.07146) -- (3.53061,6.06801) -- (3.53136,6.06456) -- (3.5321,6.06112) -- (3.53285,6.05767) -- (3.5336,6.05423) -- (3.53434,6.0508) -- (3.53509,6.04736) -- (3.53583,6.04393) -- (3.53658,6.0405)
 -- (3.53732,6.03707) -- (3.53807,6.03364) -- (3.53882,6.03022) -- (3.53956,6.0268) -- (3.54031,6.02338) -- (3.54105,6.01996) -- (3.5418,6.01654) -- (3.54255,6.01313) -- (3.54329,6.00972) -- (3.54404,6.00631) -- (3.54478,6.00291) -- (3.54553,5.99951)
 -- (3.54628,5.99611) -- (3.54702,5.99271) -- (3.54777,5.98931) -- (3.54851,5.98592) -- (3.54926,5.98253) -- (3.55,5.97914) -- (3.55075,5.97575) -- (3.5515,5.97237) -- (3.55224,5.96898) -- (3.55299,5.9656) -- (3.55373,5.96223) -- (3.55448,5.95885) --
 (3.55523,5.95548) -- (3.55597,5.95211) -- (3.55672,5.94874) -- (3.55746,5.94537) -- (3.55821,5.94201) -- (3.55895,5.93865) -- (3.5597,5.93529) -- (3.56045,5.93193) -- (3.56119,5.92858) -- (3.56194,5.92522) -- (3.56268,5.92188) -- (3.56343,5.91853)
 -- (3.56418,5.91518) -- (3.56492,5.91184) -- (3.56567,5.9085) -- (3.56641,5.90516) -- (3.56716,5.90182) -- (3.5679,5.89849) -- (3.56865,5.89516) -- (3.5694,5.89183) -- (3.57014,5.8885) -- (3.57089,5.88518) -- (3.57163,5.88186) -- (3.57238,5.87854)
 -- (3.57313,5.87522) -- (3.57387,5.8719) -- (3.57462,5.86859) -- (3.57536,5.86528) -- (3.57611,5.86197) -- (3.57685,5.85866) -- (3.5776,5.85536) -- (3.57835,5.85206) -- (3.57909,5.84876) -- (3.57984,5.84546) -- (3.58058,5.84216) -- (3.58133,5.83887)
 -- (3.58208,5.83558) -- (3.58282,5.83229) -- (3.58357,5.82901) -- (3.58431,5.82572) -- (3.58506,5.82244) -- (3.5858,5.81916) -- (3.58655,5.81589) -- (3.5873,5.81261) -- (3.58804,5.80934) -- (3.58879,5.80607) -- (3.58953,5.8028) -- (3.59028,5.79954)
 -- (3.59103,5.79627) -- (3.59177,5.79301) -- (3.59252,5.78975) -- (3.59326,5.7865) -- (3.59401,5.78324) -- (3.59475,5.77999) -- (3.5955,5.77674) -- (3.59625,5.7735) -- (3.59699,5.77025) -- (3.59774,5.76701) -- (3.59848,5.76377) -- (3.59923,5.76053)
 -- (3.59998,5.75729) -- (3.60072,5.75406) -- (3.60147,5.75083) -- (3.60221,5.7476) -- (3.60296,5.74437) -- (3.6037,5.74115) -- (3.60445,5.73792) -- (3.6052,5.7347) -- (3.60594,5.73148) -- (3.60669,5.72827) -- (3.60743,5.72506) -- (3.60818,5.72184)
 -- (3.60893,5.71863) -- (3.60967,5.71543) -- (3.61042,5.71222) -- (3.61116,5.70902) -- (3.61191,5.70582) -- (3.61265,5.70262) -- (3.6134,5.69943) -- (3.61415,5.69623) -- (3.61489,5.69304) -- (3.61564,5.68985) -- (3.61638,5.68667) --
 (3.61713,5.68348) -- (3.61788,5.6803) -- (3.61862,5.67712) -- (3.61937,5.67394) -- (3.62011,5.67077) -- (3.62086,5.66759) -- (3.6216,5.66442) -- (3.62235,5.66125) -- (3.6231,5.65808) -- (3.62384,5.65492) -- (3.62459,5.65176) -- (3.62533,5.6486) --
 (3.62608,5.64544) -- (3.62683,5.64228) -- (3.62757,5.63913) -- (3.62832,5.63598) -- (3.62906,5.63283) -- (3.62981,5.62968) -- (3.63055,5.62654) -- (3.6313,5.6234) -- (3.63205,5.62026) -- (3.63279,5.61712) -- (3.63354,5.61398) -- (3.63428,5.61085) --
 (3.63503,5.60772) -- (3.63578,5.60459) -- (3.63652,5.60146) -- (3.63727,5.59834) -- (3.63801,5.59521) -- (3.63876,5.59209) -- (3.6395,5.58898) -- (3.64025,5.58586) -- (3.641,5.58275) -- (3.64174,5.57963) -- (3.64249,5.57653) -- (3.64323,5.57342) --
 (3.64398,5.57031) -- (3.64473,5.56721) -- (3.64547,5.56411) -- (3.64622,5.56101) -- (3.64696,5.55791) -- (3.64771,5.55482) -- (3.64845,5.55173) -- (3.6492,5.54864) -- (3.64995,5.54555) -- (3.65069,5.54247) -- (3.65144,5.53938) -- (3.65218,5.5363) --
 (3.65293,5.53322) -- (3.65368,5.53015) -- (3.65442,5.52707) -- (3.65517,5.524) -- (3.65591,5.52093) -- (3.65666,5.51786) -- (3.6574,5.5148) -- (3.65815,5.51173) -- (3.6589,5.50867) -- (3.65964,5.50561) -- (3.66039,5.50256) -- (3.66113,5.4995) --
 (3.66188,5.49645) -- (3.66263,5.4934) -- (3.66337,5.49035) -- (3.66412,5.4873) -- (3.66486,5.48426) -- (3.66561,5.48122) -- (3.66635,5.47818) -- (3.6671,5.47514) -- (3.66785,5.4721) -- (3.66859,5.46907) -- (3.66934,5.46604) -- (3.67008,5.46301) --
 (3.67083,5.45998) -- (3.67158,5.45696) -- (3.67232,5.45394) -- (3.67307,5.45092) -- (3.67381,5.4479) -- (3.67456,5.44488) -- (3.6753,5.44187) -- (3.67605,5.43886) -- (3.6768,5.43585) -- (3.67754,5.43284) -- (3.67829,5.42983) -- (3.67903,5.42683) --
 (3.67978,5.42383) -- (3.68053,5.42083) -- (3.68127,5.41784) -- (3.68202,5.41484) -- (3.68276,5.41185) -- (3.68351,5.40886) -- (3.68425,5.40587) -- (3.685,5.40288) -- (3.68575,5.3999) -- (3.68649,5.39692) -- (3.68724,5.39394) -- (3.68798,5.39096) --
 (3.68873,5.38799) -- (3.68948,5.38501) -- (3.69022,5.38204) -- (3.69097,5.37907) -- (3.69171,5.37611) -- (3.69246,5.37314) -- (3.6932,5.37018) -- (3.69395,5.36722) -- (3.6947,5.36426) -- (3.69544,5.3613) -- (3.69619,5.35835) -- (3.69693,5.3554) --
 (3.69768,5.35245) -- (3.69843,5.3495) -- (3.69917,5.34655) -- (3.69992,5.34361) -- (3.70066,5.34067) -- (3.70141,5.33773) -- (3.70215,5.33479) -- (3.7029,5.33186) -- (3.70365,5.32892) -- (3.70439,5.32599) -- (3.70514,5.32306) -- (3.70588,5.32014) --
 (3.70663,5.31721) -- (3.70738,5.31429) -- (3.70812,5.31137) -- (3.70887,5.30845) -- (3.70961,5.30554) -- (3.71036,5.30262) -- (3.7111,5.29971) -- (3.71185,5.2968) -- (3.7126,5.29389) -- (3.71334,5.29099) -- (3.71409,5.28808) -- (3.71483,5.28518) --
 (3.71558,5.28228) -- (3.71633,5.27939) -- (3.71707,5.27649) -- (3.71782,5.2736) -- (3.71856,5.27071) -- (3.71931,5.26782) -- (3.72005,5.26493) -- (3.7208,5.26205) -- (3.72155,5.25916) -- (3.72229,5.25628) -- (3.72304,5.2534) -- (3.72378,5.25053) --
 (3.72453,5.24765) -- (3.72528,5.24478) -- (3.72602,5.24191) -- (3.72677,5.23904) -- (3.72751,5.23618) -- (3.72826,5.23331) -- (3.729,5.23045) -- (3.72975,5.22759) -- (3.7305,5.22473) -- (3.73124,5.22188) -- (3.73199,5.21902) -- (3.73273,5.21617) --
 (3.73348,5.21332) -- (3.73423,5.21048) -- (3.73497,5.20763) -- (3.73572,5.20479) -- (3.73646,5.20195) -- (3.73721,5.19911) -- (3.73795,5.19627) -- (3.7387,5.19343) -- (3.73945,5.1906) -- (3.74019,5.18777) -- (3.74094,5.18494) -- (3.74168,5.18211) --
 (3.74243,5.17929) -- (3.74318,5.17647) -- (3.74392,5.17365) -- (3.74467,5.17083) -- (3.74541,5.16801) -- (3.74616,5.1652) -- (3.7469,5.16238) -- (3.74765,5.15957) -- (3.7484,5.15676) -- (3.74914,5.15396) -- (3.74989,5.15115) -- (3.75063,5.14835) --
 (3.75138,5.14555) -- (3.75213,5.14275) -- (3.75287,5.13996) -- (3.75362,5.13716) -- (3.75436,5.13437) -- (3.75511,5.13158) -- (3.75585,5.12879) -- (3.7566,5.12601) -- (3.75735,5.12322) -- (3.75809,5.12044) -- (3.75884,5.11766) -- (3.75958,5.11488)
 -- (3.76033,5.11211) -- (3.76108,5.10933) -- (3.76182,5.10656) -- (3.76257,5.10379) -- (3.76331,5.10102) -- (3.76406,5.09826) -- (3.7648,5.0955) -- (3.76555,5.09273) -- (3.7663,5.08997) -- (3.76704,5.08722) -- (3.76779,5.08446) -- (3.76853,5.08171)
 -- (3.76928,5.07896) -- (3.77003,5.07621) -- (3.77077,5.07346) -- (3.77152,5.07071) -- (3.77226,5.06797) -- (3.77301,5.06523) -- (3.77375,5.06249) -- (3.7745,5.05975) -- (3.77525,5.05701) -- (3.77599,5.05428) -- (3.77674,5.05155) --
 (3.77748,5.04882) -- (3.77823,5.04609) -- (3.77898,5.04337) -- (3.77972,5.04064) -- (3.78047,5.03792) -- (3.78121,5.0352) -- (3.78196,5.03248) -- (3.7827,5.02977) -- (3.78345,5.02706) -- (3.7842,5.02434) -- (3.78494,5.02163) -- (3.78569,5.01893) --
 (3.78643,5.01622) -- (3.78718,5.01352) -- (3.78793,5.01082) -- (3.78867,5.00812) -- (3.78942,5.00542) -- (3.79016,5.00272) -- (3.79091,5.00003) -- (3.79165,4.99734) -- (3.7924,4.99465) -- (3.79315,4.99196) -- (3.79389,4.98927) -- (3.79464,4.98659)
 -- (3.79538,4.98391) -- (3.79613,4.98123) -- (3.79688,4.97855) -- (3.79762,4.97587) -- (3.79837,4.9732) -- (3.79911,4.97053) -- (3.79986,4.96786) -- (3.8006,4.96519) -- (3.80135,4.96252) -- (3.8021,4.95986) -- (3.80284,4.9572) -- (3.80359,4.95454)
 -- (3.80433,4.95188) -- (3.80508,4.94922) -- (3.80582,4.94657) -- (3.80657,4.94392) -- (3.80732,4.94127) -- (3.80806,4.93862) -- (3.80881,4.93597) -- (3.80955,4.93333) -- (3.8103,4.93068) -- (3.81105,4.92804) -- (3.81179,4.9254) -- (3.81254,4.92277)
 -- (3.81328,4.92013) -- (3.81403,4.9175) -- (3.81477,4.91487) -- (3.81552,4.91224) -- (3.81627,4.90961) -- (3.81701,4.90699) -- (3.81776,4.90437) -- (3.8185,4.90174) -- (3.81925,4.89912) -- (3.82,4.89651) -- (3.82074,4.89389) -- (3.82149,4.89128) --
 (3.82223,4.88867) -- (3.82298,4.88606) -- (3.82372,4.88345) -- (3.82447,4.88084) -- (3.82522,4.87824) -- (3.82596,4.87564) -- (3.82671,4.87304) -- (3.82745,4.87044) -- (3.8282,4.86784) -- (3.82895,4.86525) -- (3.82969,4.86266) -- (3.83044,4.86007)
 -- (3.83118,4.85748) -- (3.83193,4.85489) -- (3.83267,4.85231) -- (3.83342,4.84972) -- (3.83417,4.84714) -- (3.83491,4.84456) -- (3.83566,4.84199) -- (3.8364,4.83941) -- (3.83715,4.83684) -- (3.8379,4.83427) -- (3.83864,4.8317) -- (3.83939,4.82913)
 -- (3.84013,4.82657) -- (3.84088,4.824) -- (3.84162,4.82144) -- (3.84237,4.81888) -- (3.84312,4.81632) -- (3.84386,4.81377) -- (3.84461,4.81121) -- (3.84535,4.80866) -- (3.8461,4.80611) -- (3.84685,4.80356) -- (3.84759,4.80102) -- (3.84834,4.79847)
 -- (3.84908,4.79593) -- (3.84983,4.79339) -- (3.85057,4.79085) -- (3.85132,4.78831) -- (3.85207,4.78578) -- (3.85281,4.78324) -- (3.85356,4.78071) -- (3.8543,4.77818) -- (3.85505,4.77566) -- (3.8558,4.77313) -- (3.85654,4.77061) -- (3.85729,4.76808)
 -- (3.85803,4.76556) -- (3.85878,4.76305) -- (3.85952,4.76053) -- (3.86027,4.75801) -- (3.86102,4.7555) -- (3.86176,4.75299) -- (3.86251,4.75048) -- (3.86325,4.74798) -- (3.864,4.74547) -- (3.86475,4.74297) -- (3.86549,4.74047) -- (3.86624,4.73797)
 -- (3.86698,4.73547) -- (3.86773,4.73297) -- (3.86847,4.73048) -- (3.86922,4.72799) -- (3.86997,4.7255) -- (3.87071,4.72301) -- (3.87146,4.72052) -- (3.8722,4.71804) -- (3.87295,4.71555) -- (3.8737,4.71307) -- (3.87444,4.71059) -- (3.87519,4.70812)
 -- (3.87593,4.70564) -- (3.87668,4.70317) -- (3.87742,4.7007) -- (3.87817,4.69823) -- (3.87892,4.69576) -- (3.87966,4.69329) -- (3.88041,4.69083) -- (3.88115,4.68836) -- (3.8819,4.6859) -- (3.88265,4.68345) -- (3.88339,4.68099) -- (3.88414,4.67853)
 -- (3.88488,4.67608) -- (3.88563,4.67363) -- (3.88637,4.67118) -- (3.88712,4.66873) -- (3.88787,4.66629) -- (3.88861,4.66384) -- (3.88936,4.6614) -- (3.8901,4.65896) -- (3.89085,4.65652) -- (3.8916,4.65408) -- (3.89234,4.65165) -- (3.89309,4.64921)
 -- (3.89383,4.64678) -- (3.89458,4.64435) -- (3.89532,4.64193) -- (3.89607,4.6395) -- (3.89682,4.63708) -- (3.89756,4.63465) -- (3.89831,4.63223) -- (3.89905,4.62982) -- (3.8998,4.6274) -- (3.90055,4.62498) -- (3.90129,4.62257) -- (3.90204,4.62016)
 -- (3.90278,4.61775) -- (3.90353,4.61534) -- (3.90427,4.61294) -- (3.90502,4.61053) -- (3.90577,4.60813) -- (3.90651,4.60573) -- (3.90726,4.60333) -- (3.908,4.60093) -- (3.90875,4.59854) -- (3.9095,4.59614) -- (3.91024,4.59375) -- (3.91099,4.59136)
 -- (3.91173,4.58897) -- (3.91248,4.58659) -- (3.91322,4.5842) -- (3.91397,4.58182) -- (3.91472,4.57944) -- (3.91546,4.57706) -- (3.91621,4.57468) -- (3.91695,4.57231) -- (3.9177,4.56993) -- (3.91845,4.56756) -- (3.91919,4.56519) -- (3.91994,4.56282)
 -- (3.92068,4.56046) -- (3.92143,4.55809) -- (3.92217,4.55573) -- (3.92292,4.55337) -- (3.92367,4.55101) -- (3.92441,4.54865) -- (3.92516,4.54629) -- (3.9259,4.54394) -- (3.92665,4.54159) -- (3.9274,4.53924) -- (3.92814,4.53689) -- (3.92889,4.53454)
 -- (3.92963,4.5322) -- (3.93038,4.52985) -- (3.93112,4.52751) -- (3.93187,4.52517) -- (3.93262,4.52283) -- (3.93336,4.5205) -- (3.93411,4.51816) -- (3.93485,4.51583) -- (3.9356,4.5135) -- (3.93635,4.51117) -- (3.93709,4.50884) -- (3.93784,4.50651)
 -- (3.93858,4.50419) -- (3.93933,4.50187) -- (3.94007,4.49955) -- (3.94082,4.49723) -- (3.94157,4.49491) -- (3.94231,4.49259) -- (3.94306,4.49028) -- (3.9438,4.48797) -- (3.94455,4.48566) -- (3.9453,4.48335) -- (3.94604,4.48104) -- (3.94679,4.47874)
 -- (3.94753,4.47643) -- (3.94828,4.47413) -- (3.94902,4.47183) -- (3.94977,4.46953) -- (3.95052,4.46724) -- (3.95126,4.46494) -- (3.95201,4.46265) -- (3.95275,4.46036) -- (3.9535,4.45807) -- (3.95425,4.45578) -- (3.95499,4.45349) --
 (3.95574,4.45121) -- (3.95648,4.44893) -- (3.95723,4.44665) -- (3.95797,4.44437) -- (3.95872,4.44209) -- (3.95947,4.43981) -- (3.96021,4.43754) -- (3.96096,4.43527) -- (3.9617,4.433) -- (3.96245,4.43073) -- (3.9632,4.42846) -- (3.96394,4.42619) --
 (3.96469,4.42393) -- (3.96543,4.42167) -- (3.96618,4.41941) -- (3.96692,4.41715) -- (3.96767,4.41489) -- (3.96842,4.41264) -- (3.96916,4.41038) -- (3.96991,4.40813) -- (3.97065,4.40588) -- (3.9714,4.40363) -- (3.97215,4.40138) -- (3.97289,4.39914)
 -- (3.97364,4.39689) -- (3.97438,4.39465) -- (3.97513,4.39241) -- (3.97588,4.39017) -- (3.97662,4.38794) -- (3.97737,4.3857) -- (3.97811,4.38347) -- (3.97886,4.38124) -- (3.9796,4.37901) -- (3.98035,4.37678) -- (3.9811,4.37455) -- (3.98184,4.37233)
 -- (3.98259,4.3701) -- (3.98333,4.36788) -- (3.98408,4.36566) -- (3.98483,4.36344) -- (3.98557,4.36123) -- (3.98632,4.35901) -- (3.98706,4.3568) -- (3.98781,4.35459) -- (3.98855,4.35238) -- (3.9893,4.35017) -- (3.99005,4.34796) -- (3.99079,4.34576)
 -- (3.99154,4.34355) -- (3.99228,4.34135) -- (3.99303,4.33915) -- (3.99378,4.33695) -- (3.99452,4.33476) -- (3.99527,4.33256) -- (3.99601,4.33037) -- (3.99676,4.32818) -- (3.9975,4.32599) -- (3.99825,4.3238) -- (3.999,4.32161) -- (3.99974,4.31943)
 -- (4.00049,4.31724) -- (4.00123,4.31506) -- (4.00198,4.31288) -- (4.00273,4.3107) -- (4.00347,4.30853) -- (4.00422,4.30635) -- (4.00496,4.30418) -- (4.00571,4.302) -- (4.00645,4.29983) -- (4.0072,4.29767) -- (4.00795,4.2955) -- (4.00869,4.29333) --
 (4.00944,4.29117) -- (4.01018,4.28901) -- (4.01093,4.28685) -- (4.01167,4.28469) -- (4.01242,4.28253) -- (4.01317,4.28037) -- (4.01391,4.27822) -- (4.01466,4.27607) -- (4.0154,4.27392) -- (4.01615,4.27177) -- (4.0169,4.26962) -- (4.01764,4.26747) --
 (4.01839,4.26533) -- (4.01913,4.26319) -- (4.01988,4.26104) -- (4.02063,4.25891) -- (4.02137,4.25677) -- (4.02212,4.25463) -- (4.02286,4.2525) -- (4.02361,4.25036) -- (4.02435,4.24823) -- (4.0251,4.2461) -- (4.02585,4.24397) -- (4.02659,4.24185) --
 (4.02734,4.23972) -- (4.02808,4.2376) -- (4.02883,4.23548) -- (4.02957,4.23336) -- (4.03032,4.23124) -- (4.03107,4.22912) -- (4.03181,4.227) -- (4.03256,4.22489) -- (4.0333,4.22278) -- (4.03405,4.22067) -- (4.0348,4.21856) -- (4.03554,4.21645) --
 (4.03629,4.21434) -- (4.03703,4.21224) -- (4.03778,4.21014) -- (4.03853,4.20804) -- (4.03927,4.20594) -- (4.04002,4.20384) -- (4.04076,4.20174) -- (4.04151,4.19965) -- (4.04225,4.19755) -- (4.043,4.19546) -- (4.04375,4.19337) -- (4.04449,4.19128) --
 (4.04524,4.18919) -- (4.04598,4.18711) -- (4.04673,4.18503) -- (4.04747,4.18294) -- (4.04822,4.18086) -- (4.04897,4.17878) -- (4.04971,4.1767) -- (4.05046,4.17463) -- (4.0512,4.17255) -- (4.05195,4.17048) -- (4.0527,4.16841) -- (4.05344,4.16634) --
 (4.05419,4.16427) -- (4.05493,4.1622) -- (4.05568,4.16014) -- (4.05643,4.15807) -- (4.05717,4.15601) -- (4.05792,4.15395) -- (4.05866,4.15189) -- (4.05941,4.14984) -- (4.06015,4.14778) -- (4.0609,4.14572) -- (4.06165,4.14367) -- (4.06239,4.14162) --
 (4.06314,4.13957) -- (4.06388,4.13752) -- (4.06463,4.13548) -- (4.06537,4.13343) -- (4.06612,4.13139) -- (4.06687,4.12934) -- (4.06761,4.1273) -- (4.06836,4.12526) -- (4.0691,4.12323) -- (4.06985,4.12119) -- (4.0706,4.11916) -- (4.07134,4.11712) --
 (4.07209,4.11509) -- (4.07283,4.11306) -- (4.07358,4.11103) -- (4.07433,4.10901) -- (4.07507,4.10698) -- (4.07582,4.10496) -- (4.07656,4.10293) -- (4.07731,4.10091) -- (4.07805,4.09889) -- (4.0788,4.09688) -- (4.07955,4.09486) -- (4.08029,4.09284)
 -- (4.08104,4.09083) -- (4.08178,4.08882) -- (4.08253,4.08681) -- (4.08327,4.0848) -- (4.08402,4.08279) -- (4.08477,4.08079) -- (4.08551,4.07878) -- (4.08626,4.07678) -- (4.087,4.07478) -- (4.08775,4.07278) -- (4.0885,4.07078) -- (4.08924,4.06878)
 -- (4.08999,4.06679) -- (4.09073,4.06479) -- (4.09148,4.0628) -- (4.09223,4.06081) -- (4.09297,4.05882) -- (4.09372,4.05683) -- (4.09446,4.05485) -- (4.09521,4.05286) -- (4.09595,4.05088) -- (4.0967,4.0489) -- (4.09745,4.04692) -- (4.09819,4.04494)
 -- (4.09894,4.04296) -- (4.09968,4.04098) -- (4.10043,4.03901) -- (4.10117,4.03704) -- (4.10192,4.03506) -- (4.10267,4.03309) -- (4.10341,4.03112) -- (4.10416,4.02916) -- (4.1049,4.02719) -- (4.10565,4.02523) -- (4.1064,4.02326) -- (4.10714,4.0213)
 -- (4.10789,4.01934) -- (4.10863,4.01738) -- (4.10938,4.01543) -- (4.11013,4.01347) -- (4.11087,4.01152) -- (4.11162,4.00956) -- (4.11236,4.00761) -- (4.11311,4.00566) -- (4.11385,4.00372) -- (4.1146,4.00177) -- (4.11535,3.99982) --
 (4.11609,3.99788) -- (4.11684,3.99594) -- (4.11758,3.99399) -- (4.11833,3.99206) -- (4.11907,3.99012) -- (4.11982,3.98818) -- (4.12057,3.98624) -- (4.12131,3.98431) -- (4.12206,3.98238) -- (4.1228,3.98045) -- (4.12355,3.97852) -- (4.1243,3.97659) --
 (4.12504,3.97466) -- (4.12579,3.97274) -- (4.12653,3.97081) -- (4.12728,3.96889) -- (4.12803,3.96697) -- (4.12877,3.96505) -- (4.12952,3.96313) -- (4.13026,3.96121) -- (4.13101,3.9593) -- (4.13175,3.95739) -- (4.1325,3.95547) -- (4.13325,3.95356) --
 (4.13399,3.95165) -- (4.13474,3.94974) -- (4.13548,3.94784) -- (4.13623,3.94593) -- (4.13697,3.94403) -- (4.13772,3.94212) -- (4.13847,3.94022) -- (4.13921,3.93832) -- (4.13996,3.93642) -- (4.1407,3.93453) -- (4.14145,3.93263) -- (4.1422,3.93074) --
 (4.14294,3.92884) -- (4.14369,3.92695) -- (4.14443,3.92506) -- (4.14518,3.92317) -- (4.14593,3.92129) -- (4.14667,3.9194) -- (4.14742,3.91752) -- (4.14816,3.91563) -- (4.14891,3.91375) -- (4.14965,3.91187) -- (4.1504,3.90999) -- (4.15115,3.90811) --
 (4.15189,3.90624) -- (4.15264,3.90436) -- (4.15338,3.90249) -- (4.15413,3.90062) -- (4.15487,3.89875) -- (4.15562,3.89688) -- (4.15637,3.89501) -- (4.15711,3.89314) -- (4.15786,3.89128) -- (4.1586,3.88941) -- (4.15935,3.88755) -- (4.1601,3.88569) --
 (4.16084,3.88383) -- (4.16159,3.88197) -- (4.16233,3.88011) -- (4.16308,3.87826) -- (4.16383,3.8764) -- (4.16457,3.87455) -- (4.16532,3.8727) -- (4.16606,3.87085) -- (4.16681,3.869) -- (4.16755,3.86715) -- (4.1683,3.86531) -- (4.16905,3.86346) --
 (4.16979,3.86162) -- (4.17054,3.85978) -- (4.17128,3.85793) -- (4.17203,3.8561) -- (4.17277,3.85426) -- (4.17352,3.85242) -- (4.17427,3.85058) -- (4.17501,3.84875) -- (4.17576,3.84692) -- (4.1765,3.84509) -- (4.17725,3.84326) -- (4.178,3.84143) --
 (4.17874,3.8396) -- (4.17949,3.83777) -- (4.18023,3.83595) -- (4.18098,3.83413) -- (4.18173,3.8323) -- (4.18247,3.83048) -- (4.18322,3.82866) -- (4.18396,3.82685) -- (4.18471,3.82503) -- (4.18545,3.82321) -- (4.1862,3.8214) -- (4.18695,3.81959) --
 (4.18769,3.81777) -- (4.18844,3.81596) -- (4.18918,3.81416) -- (4.18993,3.81235) -- (4.19068,3.81054) -- (4.19142,3.80874) -- (4.19217,3.80693) -- (4.19291,3.80513) -- (4.19366,3.80333) -- (4.1944,3.80153) -- (4.19515,3.79973) -- (4.1959,3.79793) --
 (4.19664,3.79614) -- (4.19739,3.79434) -- (4.19813,3.79255) -- (4.19888,3.79076) -- (4.19963,3.78897) -- (4.20037,3.78718) -- (4.20112,3.78539) -- (4.20186,3.7836) -- (4.20261,3.78182) -- (4.20335,3.78003) -- (4.2041,3.77825) -- (4.20485,3.77647) --
 (4.20559,3.77469) -- (4.20634,3.77291) -- (4.20708,3.77113) -- (4.20783,3.76935) -- (4.20858,3.76758) -- (4.20932,3.76581) -- (4.21007,3.76403) -- (4.21081,3.76226) -- (4.21156,3.76049) -- (4.2123,3.75872) -- (4.21305,3.75695) -- (4.2138,3.75519) --
 (4.21454,3.75342) -- (4.21529,3.75166) -- (4.21603,3.7499) -- (4.21678,3.74814) -- (4.21753,3.74638) -- (4.21827,3.74462) -- (4.21902,3.74286) -- (4.21976,3.7411) -- (4.22051,3.73935) -- (4.22125,3.73759) -- (4.222,3.73584) -- (4.22275,3.73409) --
 (4.22349,3.73234) -- (4.22424,3.73059) -- (4.22498,3.72884) -- (4.22573,3.7271) -- (4.22648,3.72535) -- (4.22722,3.72361) -- (4.22797,3.72187) -- (4.22871,3.72013) -- (4.22946,3.71839) -- (4.2302,3.71665) -- (4.23095,3.71491) -- (4.2317,3.71317) --
 (4.23244,3.71144) -- (4.23319,3.7097) -- (4.23393,3.70797) -- (4.23468,3.70624) -- (4.23542,3.70451) -- (4.23617,3.70278) -- (4.23692,3.70105) -- (4.23766,3.69933) -- (4.23841,3.6976) -- (4.23915,3.69588) -- (4.2399,3.69416) -- (4.24065,3.69243) --
 (4.24139,3.69071) -- (4.24214,3.68899) -- (4.24288,3.68728) -- (4.24363,3.68556) -- (4.24438,3.68384) -- (4.24512,3.68213) -- (4.24587,3.68042) -- (4.24661,3.67871) -- (4.24736,3.67699) -- (4.2481,3.67529) -- (4.24885,3.67358) -- (4.2496,3.67187) --
 (4.25034,3.67016) -- (4.25109,3.66846) -- (4.25183,3.66676) -- (4.25258,3.66505) -- (4.25332,3.66335) -- (4.25407,3.66165) -- (4.25482,3.65995) -- (4.25556,3.65826) -- (4.25631,3.65656) -- (4.25705,3.65487) -- (4.2578,3.65317) -- (4.25855,3.65148)
 -- (4.25929,3.64979) -- (4.26004,3.6481) -- (4.26078,3.64641) -- (4.26153,3.64472) -- (4.26228,3.64303) -- (4.26302,3.64135) -- (4.26377,3.63966) -- (4.26451,3.63798) -- (4.26526,3.6363) -- (4.266,3.63462) -- (4.26675,3.63294) -- (4.2675,3.63126) --
 (4.26824,3.62958) -- (4.26899,3.62791) -- (4.26973,3.62623) -- (4.27048,3.62456) -- (4.27122,3.62288) -- (4.27197,3.62121) -- (4.27272,3.61954) -- (4.27346,3.61787) -- (4.27421,3.61621) -- (4.27495,3.61454) -- (4.2757,3.61287) -- (4.27645,3.61121)
 -- (4.27719,3.60954) -- (4.27794,3.60788) -- (4.27868,3.60622) -- (4.27943,3.60456) -- (4.28018,3.6029) -- (4.28092,3.60124) -- (4.28167,3.59959) -- (4.28241,3.59793) -- (4.28316,3.59628) -- (4.2839,3.59463) -- (4.28465,3.59297) -- (4.2854,3.59132)
 -- (4.28614,3.58967) -- (4.28689,3.58802) -- (4.28763,3.58638) -- (4.28838,3.58473) -- (4.28912,3.58309) -- (4.28987,3.58144) -- (4.29062,3.5798) -- (4.29136,3.57816) -- (4.29211,3.57652) -- (4.29285,3.57488) -- (4.2936,3.57324) -- (4.29435,3.5716)
 -- (4.29509,3.56996) -- (4.29584,3.56833) -- (4.29658,3.5667) -- (4.29733,3.56506) -- (4.29808,3.56343) -- (4.29882,3.5618) -- (4.29957,3.56017) -- (4.30031,3.55854) -- (4.30106,3.55692) -- (4.3018,3.55529) -- (4.30255,3.55366) -- (4.3033,3.55204)
 -- (4.30404,3.55042) -- (4.30479,3.5488) -- (4.30553,3.54718) -- (4.30628,3.54556) -- (4.30702,3.54394) -- (4.30777,3.54232) -- (4.30852,3.5407) -- (4.30926,3.53909) -- (4.31001,3.53748) -- (4.31075,3.53586) -- (4.3115,3.53425) -- (4.31225,3.53264)
 -- (4.31299,3.53103) -- (4.31374,3.52942) -- (4.31448,3.52781) -- (4.31523,3.52621) -- (4.31598,3.5246) -- (4.31672,3.523) -- (4.31747,3.5214) -- (4.31821,3.51979) -- (4.31896,3.51819) -- (4.3197,3.51659) -- (4.32045,3.51499) -- (4.3212,3.5134) --
 (4.32194,3.5118) -- (4.32269,3.5102) -- (4.32343,3.50861) -- (4.32418,3.50702) -- (4.32492,3.50542) -- (4.32567,3.50383) -- (4.32642,3.50224) -- (4.32716,3.50065) -- (4.32791,3.49906) -- (4.32865,3.49748) -- (4.3294,3.49589) -- (4.33015,3.49431) --
 (4.33089,3.49272) -- (4.33164,3.49114) -- (4.33238,3.48956) -- (4.33313,3.48798) -- (4.33388,3.4864) -- (4.33462,3.48482) -- (4.33537,3.48324) -- (4.33611,3.48166) -- (4.33686,3.48009) -- (4.3376,3.47851) -- (4.33835,3.47694) -- (4.3391,3.47537) --
 (4.33984,3.4738) -- (4.34059,3.47223) -- (4.34133,3.47066) -- (4.34208,3.46909) -- (4.34282,3.46752) -- (4.34357,3.46596) -- (4.34432,3.46439) -- (4.34506,3.46283) -- (4.34581,3.46126) -- (4.34655,3.4597) -- (4.3473,3.45814) -- (4.34805,3.45658) --
 (4.34879,3.45502) -- (4.34954,3.45346) -- (4.35028,3.45191) -- (4.35103,3.45035) -- (4.35178,3.4488) -- (4.35252,3.44724) -- (4.35327,3.44569) -- (4.35401,3.44414) -- (4.35476,3.44259) -- (4.3555,3.44104) -- (4.35625,3.43949) -- (4.357,3.43794) --
 (4.35774,3.43639) -- (4.35849,3.43485) -- (4.35923,3.4333) -- (4.35998,3.43176) -- (4.36072,3.43022) -- (4.36147,3.42867) -- (4.36222,3.42713) -- (4.36296,3.42559) -- (4.36371,3.42405) -- (4.36445,3.42252) -- (4.3652,3.42098) -- (4.36595,3.41944) --
 (4.36669,3.41791) -- (4.36744,3.41638) -- (4.36818,3.41484) -- (4.36893,3.41331) -- (4.36968,3.41178) -- (4.37042,3.41025) -- (4.37117,3.40872) -- (4.37191,3.40719) -- (4.37266,3.40567) -- (4.3734,3.40414) -- (4.37415,3.40262) -- (4.3749,3.40109) --
 (4.37564,3.39957) -- (4.37639,3.39805) -- (4.37713,3.39653) -- (4.37788,3.39501) -- (4.37862,3.39349) -- (4.37937,3.39197) -- (4.38012,3.39045) -- (4.38086,3.38893) -- (4.38161,3.38742) -- (4.38235,3.38591) -- (4.3831,3.38439) -- (4.38385,3.38288)
 -- (4.38459,3.38137) -- (4.38534,3.37986) -- (4.38608,3.37835) -- (4.38683,3.37684) -- (4.38758,3.37533) -- (4.38832,3.37383) -- (4.38907,3.37232) -- (4.38981,3.37081) -- (4.39056,3.36931) -- (4.3913,3.36781) -- (4.39205,3.36631) -- (4.3928,3.3648)
 -- (4.39354,3.36331) -- (4.39429,3.36181) -- (4.39503,3.36031) -- (4.39578,3.35881) -- (4.39652,3.35731) -- (4.39727,3.35582) -- (4.39802,3.35432) -- (4.39876,3.35283) -- (4.39951,3.35134) -- (4.40025,3.34985) -- (4.401,3.34836) -- (4.40175,3.34687)
 -- (4.40249,3.34538) -- (4.40324,3.34389) -- (4.40398,3.3424) -- (4.40473,3.34092) -- (4.40548,3.33943) -- (4.40622,3.33795) -- (4.40697,3.33646) -- (4.40771,3.33498) -- (4.40846,3.3335) -- (4.4092,3.33202) -- (4.40995,3.33054) -- (4.4107,3.32906)
 -- (4.41144,3.32758) -- (4.41219,3.32611) -- (4.41293,3.32463) -- (4.41368,3.32316) -- (4.41442,3.32168) -- (4.41517,3.32021) -- (4.41592,3.31874) -- (4.41666,3.31727) -- (4.41741,3.31579) -- (4.41815,3.31433) -- (4.4189,3.31286) --
 (4.41965,3.31139) -- (4.42039,3.30992) -- (4.42114,3.30846) -- (4.42188,3.30699) -- (4.42263,3.30553) -- (4.42338,3.30406) -- (4.42412,3.3026) -- (4.42487,3.30114) -- (4.42561,3.29968) -- (4.42636,3.29822) -- (4.4271,3.29676) -- (4.42785,3.2953) --
 (4.4286,3.29384) -- (4.42934,3.29239) -- (4.43009,3.29093) -- (4.43083,3.28948) -- (4.43158,3.28802) -- (4.43232,3.28657) -- (4.43307,3.28512) -- (4.43382,3.28367) -- (4.43456,3.28222) -- (4.43531,3.28077) -- (4.43605,3.27932) -- (4.4368,3.27787) --
 (4.43755,3.27643) -- (4.43829,3.27498) -- (4.43904,3.27354) -- (4.43978,3.27209) -- (4.44053,3.27065) -- (4.44128,3.26921) -- (4.44202,3.26776) -- (4.44277,3.26632) -- (4.44351,3.26488) -- (4.44426,3.26344) -- (4.445,3.26201) -- (4.44575,3.26057) --
 (4.4465,3.25913) -- (4.44724,3.2577) -- (4.44799,3.25626) -- (4.44873,3.25483) -- (4.44948,3.2534) -- (4.45022,3.25196) -- (4.45097,3.25053) -- (4.45172,3.2491) -- (4.45246,3.24767) -- (4.45321,3.24624) -- (4.45395,3.24482) -- (4.4547,3.24339) --
 (4.45545,3.24196) -- (4.45619,3.24054) -- (4.45694,3.23911) -- (4.45768,3.23769) -- (4.45843,3.23626) -- (4.45918,3.23484) -- (4.45992,3.23342) -- (4.46067,3.232) -- (4.46141,3.23058) -- (4.46216,3.22916) -- (4.4629,3.22774) -- (4.46365,3.22633) --
 (4.4644,3.22491) -- (4.46514,3.2235) -- (4.46589,3.22208) -- (4.46663,3.22067) -- (4.46738,3.21925) -- (4.46812,3.21784) -- (4.46887,3.21643) -- (4.46962,3.21502) -- (4.47036,3.21361) -- (4.47111,3.2122) -- (4.47185,3.21079) -- (4.4726,3.20938) --
 (4.47335,3.20798) -- (4.47409,3.20657) -- (4.47484,3.20517) -- (4.47558,3.20376) -- (4.47633,3.20236) -- (4.47708,3.20096) -- (4.47782,3.19955) -- (4.47857,3.19815) -- (4.47931,3.19675) -- (4.48006,3.19535) -- (4.4808,3.19395) -- (4.48155,3.19256)
 -- (4.4823,3.19116) -- (4.48304,3.18976) -- (4.48379,3.18837) -- (4.48453,3.18697) -- (4.48528,3.18558) -- (4.48602,3.18419) -- (4.48677,3.18279) -- (4.48752,3.1814) -- (4.48826,3.18001) -- (4.48901,3.17862) -- (4.48975,3.17723) -- (4.4905,3.17584)
 -- (4.49125,3.17446) -- (4.49199,3.17307) -- (4.49274,3.17168) -- (4.49348,3.1703) -- (4.49423,3.16891) -- (4.49498,3.16753) -- (4.49572,3.16615) -- (4.49647,3.16476) -- (4.49721,3.16338) -- (4.49796,3.162) -- (4.4987,3.16062) -- (4.49945,3.15924)
 -- (4.5002,3.15786) -- (4.50094,3.15648) -- (4.50169,3.15511) -- (4.50243,3.15373) -- (4.50318,3.15236) -- (4.50392,3.15098) -- (4.50467,3.14961) -- (4.50542,3.14823) -- (4.50616,3.14686) -- (4.50691,3.14549) -- (4.50765,3.14412) -- (4.5084,3.14275)
 -- (4.50915,3.14138) -- (4.50989,3.14001) -- (4.51064,3.13864) -- (4.51138,3.13727) -- (4.51213,3.13591) -- (4.51288,3.13454) -- (4.51362,3.13318) -- (4.51437,3.13181) -- (4.51511,3.13045) -- (4.51586,3.12909) -- (4.5166,3.12772) --
 (4.51735,3.12636) -- (4.5181,3.125) -- (4.51884,3.12364) -- (4.51959,3.12228) -- (4.52033,3.12092) -- (4.52108,3.11957) -- (4.52182,3.11821) -- (4.52257,3.11685) -- (4.52332,3.1155) -- (4.52406,3.11414) -- (4.52481,3.11279) -- (4.52555,3.11143) --
 (4.5263,3.11008) -- (4.52705,3.10873) -- (4.52779,3.10738) -- (4.52854,3.10603) -- (4.52928,3.10468) -- (4.53003,3.10333) -- (4.53078,3.10198) -- (4.53152,3.10063) -- (4.53227,3.09929) -- (4.53301,3.09794) -- (4.53376,3.09659) -- (4.5345,3.09525) --
 (4.53525,3.0939) -- (4.536,3.09256) -- (4.53674,3.09122) -- (4.53749,3.08988) -- (4.53823,3.08853) -- (4.53898,3.08719) -- (4.53972,3.08585) -- (4.54047,3.08451) -- (4.54122,3.08317) -- (4.54196,3.08184) -- (4.54271,3.0805) -- (4.54345,3.07916) --
 (4.5442,3.07783) -- (4.54495,3.07649) -- (4.54569,3.07516) -- (4.54644,3.07382) -- (4.54718,3.07249) -- (4.54793,3.07116) -- (4.54868,3.06983) -- (4.54942,3.06849) -- (4.55017,3.06716) -- (4.55091,3.06583) -- (4.55166,3.0645) -- (4.5524,3.06318) --
 (4.55315,3.06185) -- (4.5539,3.06052) -- (4.55464,3.05919) -- (4.55539,3.05787) -- (4.55613,3.05654) -- (4.55688,3.05522) -- (4.55762,3.0539) -- (4.55837,3.05257) -- (4.55912,3.05125) -- (4.55986,3.04993) -- (4.56061,3.04861) -- (4.56135,3.04729) --
 (4.5621,3.04597) -- (4.56285,3.04465) -- (4.56359,3.04333) -- (4.56434,3.04201) -- (4.56508,3.04069) -- (4.56583,3.03938) -- (4.56658,3.03806) -- (4.56732,3.03675) -- (4.56807,3.03543) -- (4.56881,3.03412) -- (4.56956,3.0328) -- (4.5703,3.03149) --
 (4.57105,3.03018) -- (4.5718,3.02887) -- (4.57254,3.02756) -- (4.57329,3.02625) -- (4.57403,3.02494) -- (4.57478,3.02363) -- (4.57552,3.02232) -- (4.57627,3.02101) -- (4.57702,3.01971) -- (4.57776,3.0184) -- (4.57851,3.0171) -- (4.57925,3.01579) --
 (4.58,3.01449) -- (4.58075,3.01318) -- (4.58149,3.01188) -- (4.58224,3.01058) -- (4.58298,3.00927) -- (4.58373,3.00797) -- (4.58448,3.00667) -- (4.58522,3.00537) -- (4.58597,3.00407) -- (4.58671,3.00278) -- (4.58746,3.00148) -- (4.5882,3.00018) --
 (4.58895,2.99888) -- (4.5897,2.99759) -- (4.59044,2.99629) -- (4.59119,2.995) -- (4.59193,2.9937) -- (4.59268,2.99241) -- (4.59342,2.99111) -- (4.59417,2.98982) -- (4.59492,2.98853) -- (4.59566,2.98724) -- (4.59641,2.98595) -- (4.59715,2.98466) --
 (4.5979,2.98337) -- (4.59865,2.98208) -- (4.59939,2.98079) -- (4.60014,2.9795) -- (4.60088,2.97822) -- (4.60163,2.97693) -- (4.60238,2.97564) -- (4.60312,2.97436) -- (4.60387,2.97307) -- (4.60461,2.97179) -- (4.60536,2.97051) -- (4.6061,2.96922) --
 (4.60685,2.96794) -- (4.6076,2.96666) -- (4.60834,2.96538) -- (4.60909,2.9641) -- (4.60983,2.96282) -- (4.61058,2.96154) -- (4.61132,2.96026) -- (4.61207,2.95898) -- (4.61282,2.9577) -- (4.61356,2.95643) -- (4.61431,2.95515) -- (4.61505,2.95387) --
 (4.6158,2.9526) -- (4.61655,2.95132) -- (4.61729,2.95005) -- (4.61804,2.94878) -- (4.61878,2.9475) -- (4.61953,2.94623) -- (4.62028,2.94496) -- (4.62102,2.94369) -- (4.62177,2.94242) -- (4.62251,2.94115) -- (4.62326,2.93988) -- (4.624,2.93861) --
 (4.62475,2.93734) -- (4.6255,2.93607) -- (4.62624,2.9348) -- (4.62699,2.93354) -- (4.62773,2.93227) -- (4.62848,2.931) -- (4.62923,2.92974) -- (4.62997,2.92848) -- (4.63072,2.92721) -- (4.63146,2.92595) -- (4.63221,2.92469) -- (4.63295,2.92342) --
 (4.6337,2.92216) -- (4.63445,2.9209) -- (4.63519,2.91964) -- (4.63594,2.91838) -- (4.63668,2.91712) -- (4.63743,2.91586) -- (4.63818,2.9146) -- (4.63892,2.91334) -- (4.63967,2.91209) -- (4.64041,2.91083) -- (4.64116,2.90957) -- (4.6419,2.90832) --
 (4.64265,2.90706) -- (4.6434,2.90581) -- (4.64414,2.90455) -- (4.64489,2.9033) -- (4.64563,2.90205) -- (4.64638,2.90079) -- (4.64713,2.89954) -- (4.64787,2.89829) -- (4.64862,2.89704) -- (4.64936,2.89579) -- (4.65011,2.89454) -- (4.65085,2.89329) --
 (4.6516,2.89204) -- (4.65235,2.89079) -- (4.65309,2.88955) -- (4.65384,2.8883) -- (4.65458,2.88705) -- (4.65533,2.88581) -- (4.65608,2.88456) -- (4.65682,2.88331) -- (4.65757,2.88207) -- (4.65831,2.88083) -- (4.65906,2.87958) -- (4.6598,2.87834) --
 (4.66055,2.8771) -- (4.6613,2.87585) -- (4.66204,2.87461) -- (4.66279,2.87337) -- (4.66353,2.87213) -- (4.66428,2.87089) -- (4.66503,2.86965) -- (4.66577,2.86841) -- (4.66652,2.86718) -- (4.66726,2.86594) -- (4.66801,2.8647) -- (4.66875,2.86346) --
 (4.6695,2.86223) -- (4.67025,2.86099) -- (4.67099,2.85976) -- (4.67174,2.85852) -- (4.67248,2.85729) -- (4.67323,2.85605) -- (4.67397,2.85482) -- (4.67472,2.85359) -- (4.67547,2.85235) -- (4.67621,2.85112) -- (4.67696,2.84989) -- (4.6777,2.84866) --
 (4.67845,2.84743) -- (4.6792,2.8462) -- (4.67994,2.84497) -- (4.68069,2.84374) -- (4.68143,2.84251) -- (4.68218,2.84128) -- (4.68293,2.84006) -- (4.68367,2.83883) -- (4.68442,2.8376) -- (4.68516,2.83638) -- (4.68591,2.83515) -- (4.68665,2.83393) --
 (4.6874,2.8327) -- (4.68815,2.83148) -- (4.68889,2.83026) -- (4.68964,2.82903) -- (4.69038,2.82781) -- (4.69113,2.82659) -- (4.69187,2.82537) -- (4.69262,2.82414) -- (4.69337,2.82292) -- (4.69411,2.8217) -- (4.69486,2.82048) -- (4.6956,2.81926) --
 (4.69635,2.81805) -- (4.6971,2.81683) -- (4.69784,2.81561) -- (4.69859,2.81439) -- (4.69933,2.81317) -- (4.70008,2.81196) -- (4.70083,2.81074) -- (4.70157,2.80953) -- (4.70232,2.80831) -- (4.70306,2.8071) -- (4.70381,2.80588) -- (4.70455,2.80467) --
 (4.7053,2.80346) -- (4.70605,2.80224) -- (4.70679,2.80103) -- (4.70754,2.79982) -- (4.70828,2.79861) -- (4.70903,2.7974) -- (4.70977,2.79619) -- (4.71052,2.79498) -- (4.71127,2.79377) -- (4.71201,2.79256) -- (4.71276,2.79135) -- (4.7135,2.79014) --
 (4.71425,2.78893) -- (4.715,2.78773) -- (4.71574,2.78652) -- (4.71649,2.78531) -- (4.71723,2.78411) -- (4.71798,2.7829) -- (4.71873,2.7817) -- (4.71947,2.78049) -- (4.72022,2.77929) -- (4.72096,2.77808) -- (4.72171,2.77688) -- (4.72245,2.77568) --
 (4.7232,2.77448) -- (4.72395,2.77327) -- (4.72469,2.77207) -- (4.72544,2.77087) -- (4.72618,2.76967) -- (4.72693,2.76847) -- (4.72767,2.76727) -- (4.72842,2.76607) -- (4.72917,2.76487) -- (4.72991,2.76367) -- (4.73066,2.76248) -- (4.7314,2.76128) --
 (4.73215,2.76008) -- (4.7329,2.75888) -- (4.73364,2.75769) -- (4.73439,2.75649) -- (4.73513,2.7553) -- (4.73588,2.7541) -- (4.73663,2.75291) -- (4.73737,2.75171) -- (4.73812,2.75052) -- (4.73886,2.74933) -- (4.73961,2.74813) -- (4.74035,2.74694) --
 (4.7411,2.74575) -- (4.74185,2.74456) -- (4.74259,2.74337) -- (4.74334,2.74217) -- (4.74408,2.74098) -- (4.74483,2.73979) -- (4.74557,2.73861) -- (4.74632,2.73742) -- (4.74707,2.73623) -- (4.74781,2.73504) -- (4.74856,2.73385) -- (4.7493,2.73266) --
 (4.75005,2.73148) -- (4.7508,2.73029) -- (4.75154,2.7291) -- (4.75229,2.72792) -- (4.75303,2.72673) -- (4.75378,2.72555) -- (4.75453,2.72436) -- (4.75527,2.72318) -- (4.75602,2.722) -- (4.75676,2.72081) -- (4.75751,2.71963) -- (4.75825,2.71845) --
 (4.759,2.71726) -- (4.75975,2.71608) -- (4.76049,2.7149) -- (4.76124,2.71372) -- (4.76198,2.71254) -- (4.76273,2.71136) -- (4.76347,2.71018) -- (4.76422,2.709) -- (4.76497,2.70782) -- (4.76571,2.70664) -- (4.76646,2.70546) -- (4.7672,2.70429) --
 (4.76795,2.70311) -- (4.7687,2.70193) -- (4.76944,2.70076) -- (4.77019,2.69958) -- (4.77093,2.6984) -- (4.77168,2.69723) -- (4.77243,2.69605) -- (4.77317,2.69488) -- (4.77392,2.69371) -- (4.77466,2.69253) -- (4.77541,2.69136) -- (4.77615,2.69018) --
 (4.7769,2.68901) -- (4.77765,2.68784) -- (4.77839,2.68667) -- (4.77914,2.6855) -- (4.77988,2.68433) -- (4.78063,2.68315) -- (4.78137,2.68198) -- (4.78212,2.68081) -- (4.78287,2.67964) -- (4.78361,2.67848) -- (4.78436,2.67731) -- (4.7851,2.67614) --
 (4.78585,2.67497) -- (4.7866,2.6738) -- (4.78734,2.67264) -- (4.78809,2.67147) -- (4.78883,2.6703) -- (4.78958,2.66914) -- (4.79033,2.66797) -- (4.79107,2.6668) -- (4.79182,2.66564) -- (4.79256,2.66447) -- (4.79331,2.66331) -- (4.79405,2.66215) --
 (4.7948,2.66098) -- (4.79555,2.65982) -- (4.79629,2.65866) -- (4.79704,2.65749) -- (4.79778,2.65633) -- (4.79853,2.65517) -- (4.79927,2.65401) -- (4.80002,2.65285) -- (4.80077,2.65169) -- (4.80151,2.65053) -- (4.80226,2.64937) -- (4.803,2.64821) --
 (4.80375,2.64705) -- (4.8045,2.64589) -- (4.80524,2.64473) -- (4.80599,2.64357) -- (4.80673,2.64242) -- (4.80748,2.64126) -- (4.80823,2.6401) -- (4.80897,2.63894) -- (4.80972,2.63779) -- (4.81046,2.63663) -- (4.81121,2.63548) -- (4.81195,2.63432) --
 (4.8127,2.63317) -- (4.81345,2.63201) -- (4.81419,2.63086) -- (4.81494,2.6297) -- (4.81568,2.62855) -- (4.81643,2.6274) -- (4.81717,2.62625) -- (4.81792,2.62509) -- (4.81867,2.62394) -- (4.81941,2.62279) -- (4.82016,2.62164) -- (4.8209,2.62049) --
 (4.82165,2.61934) -- (4.8224,2.61819) -- (4.82314,2.61704) -- (4.82389,2.61589) -- (4.82463,2.61474) -- (4.82538,2.61359) -- (4.82613,2.61244) -- (4.82687,2.61129) -- (4.82762,2.61014) -- (4.82836,2.609) -- (4.82911,2.60785) -- (4.82985,2.6067) --
 (4.8306,2.60556) -- (4.83135,2.60441) -- (4.83209,2.60326) -- (4.83284,2.60212) -- (4.83358,2.60097) -- (4.83433,2.59983) -- (4.83507,2.59869) -- (4.83582,2.59754) -- (4.83657,2.5964) -- (4.83731,2.59525) -- (4.83806,2.59411) -- (4.8388,2.59297) --
 (4.83955,2.59183) -- (4.8403,2.59068) -- (4.84104,2.58954) -- (4.84179,2.5884) -- (4.84253,2.58726) -- (4.84328,2.58612) -- (4.84403,2.58498) -- (4.84477,2.58384) -- (4.84552,2.5827) -- (4.84626,2.58156) -- (4.84701,2.58042) -- (4.84775,2.57928) --
 (4.8485,2.57814) -- (4.84925,2.57701) -- (4.84999,2.57587) -- (4.85074,2.57473) -- (4.85148,2.57359) -- (4.85223,2.57246) -- (4.85297,2.57132) -- (4.85372,2.57019) -- (4.85447,2.56905) -- (4.85521,2.56791) -- (4.85596,2.56678) -- (4.8567,2.56564) --
 (4.85745,2.56451) -- (4.8582,2.56338) -- (4.85894,2.56224) -- (4.85969,2.56111) -- (4.86043,2.55998) -- (4.86118,2.55884) -- (4.86193,2.55771) -- (4.86267,2.55658) -- (4.86342,2.55545) -- (4.86416,2.55432) -- (4.86491,2.55318) -- (4.86565,2.55205)
 -- (4.8664,2.55092) -- (4.86715,2.54979) -- (4.86789,2.54866) -- (4.86864,2.54753) -- (4.86938,2.5464) -- (4.87013,2.54528) -- (4.87087,2.54415) -- (4.87162,2.54302) -- (4.87237,2.54189) -- (4.87311,2.54076) -- (4.87386,2.53964) -- (4.8746,2.53851)
 -- (4.87535,2.53738) -- (4.8761,2.53626) -- (4.87684,2.53513) -- (4.87759,2.534) -- (4.87833,2.53288) -- (4.87908,2.53175) -- (4.87983,2.53063) -- (4.88057,2.52951) -- (4.88132,2.52838) -- (4.88206,2.52726) -- (4.88281,2.52613) -- (4.88355,2.52501)
 -- (4.8843,2.52389) -- (4.88505,2.52277) -- (4.88579,2.52164) -- (4.88654,2.52052) -- (4.88728,2.5194) -- (4.88803,2.51828) -- (4.88877,2.51716) -- (4.88952,2.51604) -- (4.89027,2.51492) -- (4.89101,2.5138) -- (4.89176,2.51268) -- (4.8925,2.51156)
 -- (4.89325,2.51044) -- (4.894,2.50932) -- (4.89474,2.5082) -- (4.89549,2.50708) -- (4.89623,2.50596) -- (4.89698,2.50485) -- (4.89773,2.50373) -- (4.89847,2.50261) -- (4.89922,2.5015) -- (4.89996,2.50038) -- (4.90071,2.49926) -- (4.90145,2.49815)
 -- (4.9022,2.49703) -- (4.90295,2.49592) -- (4.90369,2.4948) -- (4.90444,2.49369) -- (4.90518,2.49257) -- (4.90593,2.49146) -- (4.90667,2.49035) -- (4.90742,2.48923) -- (4.90817,2.48812) -- (4.90891,2.48701) -- (4.90966,2.4859) -- (4.9104,2.48478)
 -- (4.91115,2.48367) -- (4.9119,2.48256) -- (4.91264,2.48145) -- (4.91339,2.48034) -- (4.91413,2.47923) -- (4.91488,2.47812) -- (4.91563,2.47701) -- (4.91637,2.4759) -- (4.91712,2.47479) -- (4.91786,2.47368) -- (4.91861,2.47257) -- (4.91935,2.47146)
 -- (4.9201,2.47035) -- (4.92085,2.46925) -- (4.92159,2.46814) -- (4.92234,2.46703) -- (4.92308,2.46592) -- (4.92383,2.46482) -- (4.92457,2.46371) -- (4.92532,2.46261) -- (4.92607,2.4615) -- (4.92681,2.46039) -- (4.92756,2.45929) -- (4.9283,2.45818)
 -- (4.92905,2.45708) -- (4.9298,2.45598) -- (4.93054,2.45487) -- (4.93129,2.45377) -- (4.93203,2.45266) -- (4.93278,2.45156) -- (4.93353,2.45046) -- (4.93427,2.44936) -- (4.93502,2.44825) -- (4.93576,2.44715) -- (4.93651,2.44605) --
 (4.93725,2.44495) -- (4.938,2.44385) -- (4.93875,2.44275) -- (4.93949,2.44165) -- (4.94024,2.44055) -- (4.94098,2.43945) -- (4.94173,2.43835) -- (4.94247,2.43725) -- (4.94322,2.43615) -- (4.94397,2.43505) -- (4.94471,2.43395) -- (4.94546,2.43286) --
 (4.9462,2.43176) -- (4.94695,2.43066) -- (4.9477,2.42956) -- (4.94844,2.42847) -- (4.94919,2.42737) -- (4.94993,2.42627) -- (4.95068,2.42518) -- (4.95143,2.42408) -- (4.95217,2.42299) -- (4.95292,2.42189) -- (4.95366,2.4208) -- (4.95441,2.4197) --
 (4.95515,2.41861) -- (4.9559,2.41751) -- (4.95665,2.41642) -- (4.95739,2.41533) -- (4.95814,2.41423) -- (4.95888,2.41314) -- (4.95963,2.41205) -- (4.96037,2.41096) -- (4.96112,2.40986) -- (4.96187,2.40877) -- (4.96261,2.40768) -- (4.96336,2.40659)
 -- (4.9641,2.4055) -- (4.96485,2.40441) -- (4.9656,2.40332) -- (4.96634,2.40223) -- (4.96709,2.40114) -- (4.96783,2.40005) -- (4.96858,2.39896) -- (4.96933,2.39787) -- (4.97007,2.39679) -- (4.97082,2.3957) -- (4.97156,2.39461) -- (4.97231,2.39352)
 -- (4.97305,2.39243) -- (4.9738,2.39135) -- (4.97455,2.39026) -- (4.97529,2.38918) -- (4.97604,2.38809) -- (4.97678,2.387) -- (4.97753,2.38592) -- (4.97827,2.38483) -- (4.97902,2.38375) -- (4.97977,2.38266) -- (4.98051,2.38158) -- (4.98126,2.3805)
 -- (4.982,2.37941) -- (4.98275,2.37833) -- (4.9835,2.37725) -- (4.98424,2.37616) -- (4.98499,2.37508) -- (4.98573,2.374) -- (4.98648,2.37292) -- (4.98723,2.37183) -- (4.98797,2.37075) -- (4.98872,2.36967) -- (4.98946,2.36859) -- (4.99021,2.36751) --
 (4.99095,2.36643) -- (4.9917,2.36535) -- (4.99245,2.36427) -- (4.99319,2.36319) -- (4.99394,2.36211) -- (4.99468,2.36103) -- (4.99543,2.35996) -- (4.99617,2.35888) -- (4.99692,2.3578) -- (4.99767,2.35672) -- (4.99841,2.35565) -- (4.99916,2.35457) --
 (4.9999,2.35349) -- (5.00065,2.35242) -- (5.0014,2.35134) -- (5.00214,2.35026) -- (5.00289,2.34919) -- (5.00363,2.34811) -- (5.00438,2.34704) -- (5.00513,2.34596) -- (5.00587,2.34489) -- (5.00662,2.34382) -- (5.00736,2.34274) -- (5.00811,2.34167) --
 (5.00885,2.3406) -- (5.0096,2.33952) -- (5.01035,2.33845) -- (5.01109,2.33738) -- (5.01184,2.33631) -- (5.01258,2.33523) -- (5.01333,2.33416) -- (5.01407,2.33309) -- (5.01482,2.33202) -- (5.01557,2.33095) -- (5.01631,2.32988) -- (5.01706,2.32881) --
 (5.0178,2.32774) -- (5.01855,2.32667) -- (5.0193,2.3256) -- (5.02004,2.32453) -- (5.02079,2.32347) -- (5.02153,2.3224) -- (5.02228,2.32133) -- (5.02303,2.32026) -- (5.02377,2.3192) -- (5.02452,2.31813) -- (5.02526,2.31706) -- (5.02601,2.316) --
 (5.02675,2.31493) -- (5.0275,2.31386) -- (5.02825,2.3128) -- (5.02899,2.31173) -- (5.02974,2.31067) -- (5.03048,2.3096) -- (5.03123,2.30854) -- (5.03197,2.30748) -- (5.03272,2.30641) -- (5.03347,2.30535) -- (5.03421,2.30429) -- (5.03496,2.30322) --
 (5.0357,2.30216) -- (5.03645,2.3011) -- (5.0372,2.30004) -- (5.03794,2.29898) -- (5.03869,2.29791) -- (5.03943,2.29685) -- (5.04018,2.29579) -- (5.04093,2.29473) -- (5.04167,2.29367) -- (5.04242,2.29261) -- (5.04316,2.29155) -- (5.04391,2.2905) --
 (5.04465,2.28944) -- (5.0454,2.28838) -- (5.04615,2.28732) -- (5.04689,2.28626) -- (5.04764,2.2852) -- (5.04838,2.28415) -- (5.04913,2.28309) -- (5.04987,2.28203) -- (5.05062,2.28098) -- (5.05137,2.27992) -- (5.05211,2.27887) -- (5.05286,2.27781) --
 (5.0536,2.27676) -- (5.05435,2.2757) -- (5.0551,2.27465) -- (5.05584,2.27359) -- (5.05659,2.27254) -- (5.05733,2.27148) -- (5.05808,2.27043) -- (5.05883,2.26938) -- (5.05957,2.26833) -- (5.06032,2.26727) -- (5.06106,2.26622) -- (5.06181,2.26517) --
 (5.06255,2.26412) -- (5.0633,2.26307) -- (5.06405,2.26202) -- (5.06479,2.26097) -- (5.06554,2.25992) -- (5.06628,2.25887) -- (5.06703,2.25782) -- (5.06778,2.25677) -- (5.06852,2.25572) -- (5.06927,2.25467) -- (5.07001,2.25362) -- (5.07076,2.25257)
 -- (5.0715,2.25153) -- (5.07225,2.25048) -- (5.073,2.24943) -- (5.07374,2.24838) -- (5.07449,2.24734) -- (5.07523,2.24629) -- (5.07598,2.24525) -- (5.07673,2.2442) -- (5.07747,2.24316) -- (5.07822,2.24211) -- (5.07896,2.24107) -- (5.07971,2.24002)
 -- (5.08045,2.23898) -- (5.0812,2.23794) -- (5.08195,2.23689) -- (5.08269,2.23585) -- (5.08344,2.23481) -- (5.08418,2.23376) -- (5.08493,2.23272) -- (5.08568,2.23168) -- (5.08642,2.23064) -- (5.08717,2.2296) -- (5.08791,2.22856) -- (5.08866,2.22752)
 -- (5.0894,2.22648) -- (5.09015,2.22544) -- (5.0909,2.2244) -- (5.09164,2.22336) -- (5.09239,2.22232) -- (5.09313,2.22128) -- (5.09388,2.22024) -- (5.09462,2.21921) -- (5.09537,2.21817) -- (5.09612,2.21713) -- (5.09686,2.2161) -- (5.09761,2.21506)
 -- (5.09835,2.21402) -- (5.0991,2.21299) -- (5.09985,2.21195) -- (5.10059,2.21092) -- (5.10134,2.20988) -- (5.10208,2.20885) -- (5.10283,2.20781) -- (5.10358,2.20678) -- (5.10432,2.20575) -- (5.10507,2.20471) -- (5.10581,2.20368) --
 (5.10656,2.20265) -- (5.1073,2.20162) -- (5.10805,2.20058) -- (5.1088,2.19955) -- (5.10954,2.19852) -- (5.11029,2.19749) -- (5.11103,2.19646) -- (5.11178,2.19543) -- (5.11252,2.1944) -- (5.11327,2.19337) -- (5.11402,2.19234) -- (5.11476,2.19131) --
 (5.11551,2.19029) -- (5.11625,2.18926) -- (5.117,2.18823) -- (5.11775,2.1872) -- (5.11849,2.18618) -- (5.11924,2.18515) -- (5.11998,2.18412) -- (5.12073,2.1831) -- (5.12148,2.18207) -- (5.12222,2.18105) -- (5.12297,2.18002) -- (5.12371,2.179) --
 (5.12446,2.17797) -- (5.1252,2.17695) -- (5.12595,2.17592) -- (5.1267,2.1749) -- (5.12744,2.17388) -- (5.12819,2.17285) -- (5.12893,2.17183) -- (5.12968,2.17081) -- (5.13042,2.16979) -- (5.13117,2.16877) -- (5.13192,2.16775) -- (5.13266,2.16673) --
 (5.13341,2.16571) -- (5.13415,2.16469) -- (5.1349,2.16367) -- (5.13565,2.16265) -- (5.13639,2.16163) -- (5.13714,2.16061) -- (5.13788,2.15959) -- (5.13863,2.15858) -- (5.13938,2.15756) -- (5.14012,2.15654) -- (5.14087,2.15553) -- (5.14161,2.15451)
 -- (5.14236,2.15349) -- (5.1431,2.15248) -- (5.14385,2.15146) -- (5.1446,2.15045) -- (5.14534,2.14943) -- (5.14609,2.14842) -- (5.14683,2.14741) -- (5.14758,2.14639) -- (5.14832,2.14538) -- (5.14907,2.14437) -- (5.14982,2.14335) -- (5.15056,2.14234)
 -- (5.15131,2.14133) -- (5.15205,2.14032) -- (5.1528,2.13931) -- (5.15355,2.1383) -- (5.15429,2.13729) -- (5.15504,2.13628) -- (5.15578,2.13527) -- (5.15653,2.13426) -- (5.15728,2.13325) -- (5.15802,2.13224) -- (5.15877,2.13124) -- (5.15951,2.13023)
 -- (5.16026,2.12922) -- (5.161,2.12822) -- (5.16175,2.12721) -- (5.1625,2.1262) -- (5.16324,2.1252) -- (5.16399,2.12419) -- (5.16473,2.12319) -- (5.16548,2.12218) -- (5.16622,2.12118) -- (5.16697,2.12018) -- (5.16772,2.11917) -- (5.16846,2.11817) --
 (5.16921,2.11717) -- (5.16995,2.11616) -- (5.1707,2.11516) -- (5.17145,2.11416) -- (5.17219,2.11316) -- (5.17294,2.11216) -- (5.17368,2.11116) -- (5.17443,2.11016) -- (5.17518,2.10916) -- (5.17592,2.10816) -- (5.17667,2.10716) -- (5.17741,2.10617)
 -- (5.17816,2.10517) -- (5.1789,2.10417) -- (5.17965,2.10317) -- (5.1804,2.10218) -- (5.18114,2.10118) -- (5.18189,2.10018) -- (5.18263,2.09919) -- (5.18338,2.09819) -- (5.18412,2.0972) -- (5.18487,2.0962) -- (5.18562,2.09521) -- (5.18636,2.09422)
 -- (5.18711,2.09322) -- (5.18785,2.09223) -- (5.1886,2.09124) -- (5.18935,2.09025) -- (5.19009,2.08926) -- (5.19084,2.08826) -- (5.19158,2.08727) -- (5.19233,2.08628) -- (5.19308,2.08529) -- (5.19382,2.0843) -- (5.19457,2.08331) -- (5.19531,2.08233)
 -- (5.19606,2.08134) -- (5.1968,2.08035) -- (5.19755,2.07936) -- (5.1983,2.07838) -- (5.19904,2.07739) -- (5.19979,2.0764) -- (5.20053,2.07542) -- (5.20128,2.07443) -- (5.20202,2.07345) -- (5.20277,2.07246) -- (5.20352,2.07148) -- (5.20426,2.07049)
 -- (5.20501,2.06951) -- (5.20575,2.06853) -- (5.2065,2.06754) -- (5.20725,2.06656) -- (5.20799,2.06558) -- (5.20874,2.0646) -- (5.20948,2.06362) -- (5.21023,2.06264) -- (5.21098,2.06166) -- (5.21172,2.06068) -- (5.21247,2.0597) -- (5.21321,2.05872)
 -- (5.21396,2.05774) -- (5.2147,2.05677) -- (5.21545,2.05579) -- (5.2162,2.05481) -- (5.21694,2.05383) -- (5.21769,2.05286) -- (5.21843,2.05188) -- (5.21918,2.05091) -- (5.21992,2.04993) -- (5.22067,2.04896) -- (5.22142,2.04798) -- (5.22216,2.04701)
 -- (5.22291,2.04604) -- (5.22365,2.04507) -- (5.2244,2.04409) -- (5.22515,2.04312) -- (5.22589,2.04215) -- (5.22664,2.04118) -- (5.22738,2.04021) -- (5.22813,2.03924) -- (5.22888,2.03827) -- (5.22962,2.0373) -- (5.23037,2.03633) -- (5.23111,2.03536)
 -- (5.23186,2.0344) -- (5.2326,2.03343) -- (5.23335,2.03246) -- (5.2341,2.0315) -- (5.23484,2.03053) -- (5.23559,2.02956) -- (5.23633,2.0286) -- (5.23708,2.02763) -- (5.23782,2.02667) -- (5.23857,2.02571) -- (5.23932,2.02474) -- (5.24006,2.02378) --
 (5.24081,2.02282) -- (5.24155,2.02186) -- (5.2423,2.0209) -- (5.24305,2.01993) -- (5.24379,2.01897) -- (5.24454,2.01801) -- (5.24528,2.01705) -- (5.24603,2.0161) -- (5.24678,2.01514) -- (5.24752,2.01418) -- (5.24827,2.01322) -- (5.24901,2.01226) --
 (5.24976,2.01131) -- (5.2505,2.01035) -- (5.25125,2.0094) -- (5.252,2.00844) -- (5.25274,2.00749) -- (5.25349,2.00653) -- (5.25423,2.00558) -- (5.25498,2.00462) -- (5.25572,2.00367) -- (5.25647,2.00272) -- (5.25722,2.00177) -- (5.25796,2.00081) --
 (5.25871,1.99986) -- (5.25945,1.99891) -- (5.2602,1.99796) -- (5.26095,1.99701) -- (5.26169,1.99606) -- (5.26244,1.99512) -- (5.26318,1.99417) -- (5.26393,1.99322) -- (5.26468,1.99227) -- (5.26542,1.99133) -- (5.26617,1.99038) -- (5.26691,1.98943)
 -- (5.26766,1.98849) -- (5.2684,1.98755) -- (5.26915,1.9866) -- (5.2699,1.98566) -- (5.27064,1.98471) -- (5.27139,1.98377) -- (5.27213,1.98283) -- (5.27288,1.98189) -- (5.27362,1.98095) -- (5.27437,1.98001) -- (5.27512,1.97907) -- (5.27586,1.97813)
 -- (5.27661,1.97719) -- (5.27735,1.97625) -- (5.2781,1.97531) -- (5.27885,1.97437) -- (5.27959,1.97344) -- (5.28034,1.9725) -- (5.28108,1.97156) -- (5.28183,1.97063) -- (5.28258,1.96969) -- (5.28332,1.96876) -- (5.28407,1.96782) -- (5.28481,1.96689)
 -- (5.28556,1.96596) -- (5.2863,1.96502) -- (5.28705,1.96409) -- (5.2878,1.96316) -- (5.28854,1.96223) -- (5.28929,1.9613) -- (5.29003,1.96037) -- (5.29078,1.95944) -- (5.29152,1.95851) -- (5.29227,1.95758) -- (5.29302,1.95666) -- (5.29376,1.95573)
 -- (5.29451,1.9548) -- (5.29525,1.95388) -- (5.296,1.95295) -- (5.29675,1.95202) -- (5.29749,1.9511) -- (5.29824,1.95018) -- (5.29898,1.94925) -- (5.29973,1.94833) -- (5.30048,1.94741) -- (5.30122,1.94649) -- (5.30197,1.94556) -- (5.30271,1.94464)
 -- (5.30346,1.94372) -- (5.3042,1.9428) -- (5.30495,1.94188) -- (5.3057,1.94097) -- (5.30644,1.94005) -- (5.30719,1.93913) -- (5.30793,1.93821) -- (5.30868,1.9373) -- (5.30942,1.93638) -- (5.31017,1.93546) -- (5.31092,1.93455) -- (5.31166,1.93364)
 -- (5.31241,1.93272) -- (5.31315,1.93181) -- (5.3139,1.9309) -- (5.31465,1.92998) -- (5.31539,1.92907) -- (5.31614,1.92816) -- (5.31688,1.92725) -- (5.31763,1.92634) -- (5.31838,1.92543) -- (5.31912,1.92452) -- (5.31987,1.92362) -- (5.32061,1.92271)
 -- (5.32136,1.9218) -- (5.3221,1.9209) -- (5.32285,1.91999) -- (5.3236,1.91908) -- (5.32434,1.91818) -- (5.32509,1.91728) -- (5.32583,1.91637) -- (5.32658,1.91547) -- (5.32732,1.91457) -- (5.32807,1.91367) -- (5.32882,1.91276) -- (5.32956,1.91186)
 -- (5.33031,1.91096) -- (5.33105,1.91006) -- (5.3318,1.90917) -- (5.33255,1.90827) -- (5.33329,1.90737) -- (5.33404,1.90647) -- (5.33478,1.90558) -- (5.33553,1.90468) -- (5.33628,1.90378) -- (5.33702,1.90289) -- (5.33777,1.902) -- (5.33851,1.9011)
 -- (5.33926,1.90021) -- (5.34,1.89932) -- (5.34075,1.89843) -- (5.3415,1.89753) -- (5.34224,1.89664) -- (5.34299,1.89575) -- (5.34373,1.89486) -- (5.34448,1.89398) -- (5.34522,1.89309) -- (5.34597,1.8922) -- (5.34672,1.89131) -- (5.34746,1.89043) --
 (5.34821,1.88954) -- (5.34895,1.88866) -- (5.3497,1.88777) -- (5.35045,1.88689) -- (5.35119,1.88601) -- (5.35194,1.88512) -- (5.35268,1.88424) -- (5.35343,1.88336) -- (5.35418,1.88248) -- (5.35492,1.8816) -- (5.35567,1.88072) -- (5.35641,1.87984) --
 (5.35716,1.87896) -- (5.3579,1.87808) -- (5.35865,1.87721) -- (5.3594,1.87633) -- (5.36014,1.87546) -- (5.36089,1.87458) -- (5.36163,1.87371) -- (5.36238,1.87283) -- (5.36312,1.87196) -- (5.36387,1.87109) -- (5.36462,1.87021) -- (5.36536,1.86934) --
 (5.36611,1.86847) -- (5.36685,1.8676) -- (5.3676,1.86673) -- (5.36835,1.86586) -- (5.36909,1.865) -- (5.36984,1.86413) -- (5.37058,1.86326) -- (5.37133,1.8624) -- (5.37208,1.86153) -- (5.37282,1.86067) -- (5.37357,1.8598) -- (5.37431,1.85894) --
 (5.37506,1.85808) -- (5.3758,1.85721) -- (5.37655,1.85635) -- (5.3773,1.85549) -- (5.37804,1.85463) -- (5.37879,1.85377) -- (5.37953,1.85291) -- (5.38028,1.85205) -- (5.38102,1.8512) -- (5.38177,1.85034) -- (5.38252,1.84948) -- (5.38326,1.84863) --
 (5.38401,1.84777) -- (5.38475,1.84692) -- (5.3855,1.84606) -- (5.38625,1.84521) -- (5.38699,1.84436) -- (5.38774,1.84351) -- (5.38848,1.84266) -- (5.38923,1.84181) -- (5.38998,1.84096) -- (5.39072,1.84011) -- (5.39147,1.83926) -- (5.39221,1.83841)
 -- (5.39296,1.83757) -- (5.3937,1.83672) -- (5.39445,1.83588) -- (5.3952,1.83503) -- (5.39594,1.83419) -- (5.39669,1.83334) -- (5.39743,1.8325) -- (5.39818,1.83166) -- (5.39892,1.83082) -- (5.39967,1.82998) -- (5.40042,1.82914) -- (5.40116,1.8283)
 -- (5.40191,1.82746) -- (5.40265,1.82662) -- (5.4034,1.82578) -- (5.40415,1.82495) -- (5.40489,1.82411) -- (5.40564,1.82328) -- (5.40638,1.82244) -- (5.40713,1.82161) -- (5.40788,1.82078) -- (5.40862,1.81995) -- (5.40937,1.81911) --
 (5.41011,1.81828) -- (5.41086,1.81745) -- (5.4116,1.81662) -- (5.41235,1.8158) -- (5.4131,1.81497) -- (5.41384,1.81414) -- (5.41459,1.81332) -- (5.41533,1.81249) -- (5.41608,1.81166) -- (5.41682,1.81084) -- (5.41757,1.81002) -- (5.41832,1.80919) --
 (5.41906,1.80837) -- (5.41981,1.80755) -- (5.42055,1.80673) -- (5.4213,1.80591) -- (5.42205,1.80509) -- (5.42279,1.80427) -- (5.42354,1.80346) -- (5.42428,1.80264) -- (5.42503,1.80182) -- (5.42578,1.80101) -- (5.42652,1.80019) -- (5.42727,1.79938)
 -- (5.42801,1.79857) -- (5.42876,1.79775) -- (5.4295,1.79694) -- (5.43025,1.79613) -- (5.431,1.79532) -- (5.43174,1.79451) -- (5.43249,1.7937) -- (5.43323,1.7929) -- (5.43398,1.79209) -- (5.43472,1.79128) -- (5.43547,1.79048) -- (5.43622,1.78967) --
 (5.43696,1.78887) -- (5.43771,1.78807) -- (5.43845,1.78726) -- (5.4392,1.78646) -- (5.43995,1.78566) -- (5.44069,1.78486) -- (5.44144,1.78406) -- (5.44218,1.78326) -- (5.44293,1.78246) -- (5.44368,1.78167) -- (5.44442,1.78087) -- (5.44517,1.78008)
 -- (5.44591,1.77928) -- (5.44666,1.77849) -- (5.4474,1.77769) -- (5.44815,1.7769) -- (5.4489,1.77611) -- (5.44964,1.77532) -- (5.45039,1.77453) -- (5.45113,1.77374) -- (5.45188,1.77295) -- (5.45262,1.77216) -- (5.45337,1.77138) -- (5.45412,1.77059)
 -- (5.45486,1.76981) -- (5.45561,1.76902) -- (5.45635,1.76824) -- (5.4571,1.76745) -- (5.45785,1.76667) -- (5.45859,1.76589) -- (5.45934,1.76511) -- (5.46008,1.76433) -- (5.46083,1.76355) -- (5.46158,1.76277) -- (5.46232,1.762) -- (5.46307,1.76122)
 -- (5.46381,1.76045) -- (5.46456,1.75967) -- (5.4653,1.7589) -- (5.46605,1.75812) -- (5.4668,1.75735) -- (5.46754,1.75658) -- (5.46829,1.75581) -- (5.46903,1.75504) -- (5.46978,1.75427) -- (5.47052,1.7535) -- (5.47127,1.75274) -- (5.47202,1.75197)
 -- (5.47276,1.7512) -- (5.47351,1.75044) -- (5.47425,1.74967) -- (5.475,1.74891) -- (5.47575,1.74815) -- (5.47649,1.74739) -- (5.47724,1.74663) -- (5.47798,1.74587) -- (5.47873,1.74511) -- (5.47948,1.74435) -- (5.48022,1.74359) -- (5.48097,1.74284)
 -- (5.48171,1.74208) -- (5.48246,1.74132) -- (5.4832,1.74057) -- (5.48395,1.73982) -- (5.4847,1.73907) -- (5.48544,1.73831) -- (5.48619,1.73756) -- (5.48693,1.73681) -- (5.48768,1.73607) -- (5.48842,1.73532) -- (5.48917,1.73457) -- (5.48992,1.73382)
 -- (5.49066,1.73308) -- (5.49141,1.73233) -- (5.49215,1.73159) -- (5.4929,1.73085) -- (5.49365,1.73011) -- (5.49439,1.72936) -- (5.49514,1.72862) -- (5.49588,1.72789) -- (5.49663,1.72715) -- (5.49738,1.72641) -- (5.49812,1.72567) --
 (5.49887,1.72494) -- (5.49961,1.7242) -- (5.50036,1.72347) -- (5.5011,1.72274) -- (5.50185,1.722) -- (5.5026,1.72127) -- (5.50334,1.72054) -- (5.50409,1.71981) -- (5.50483,1.71908) -- (5.50558,1.71836) -- (5.50633,1.71763) -- (5.50707,1.7169) --
 (5.50782,1.71618) -- (5.50856,1.71545) -- (5.50931,1.71473) -- (5.51005,1.71401) -- (5.5108,1.71329) -- (5.51155,1.71257) -- (5.51229,1.71185) -- (5.51304,1.71113) -- (5.51378,1.71041) -- (5.51453,1.70969) -- (5.51528,1.70898) -- (5.51602,1.70826)
 -- (5.51677,1.70755) -- (5.51751,1.70683) -- (5.51826,1.70612) -- (5.519,1.70541) -- (5.51975,1.7047) -- (5.5205,1.70399) -- (5.52124,1.70328) -- (5.52199,1.70257) -- (5.52273,1.70187) -- (5.52348,1.70116) -- (5.52423,1.70046) -- (5.52497,1.69975)
 -- (5.52572,1.69905) -- (5.52646,1.69835) -- (5.52721,1.69765) -- (5.52795,1.69695) -- (5.5287,1.69625) -- (5.52945,1.69555) -- (5.53019,1.69485) -- (5.53094,1.69415) -- (5.53168,1.69346) -- (5.53243,1.69276) -- (5.53317,1.69207) --
 (5.53392,1.69138) -- (5.53467,1.69069) -- (5.53541,1.69) -- (5.53616,1.68931) -- (5.5369,1.68862) -- (5.53765,1.68793) -- (5.5384,1.68724) -- (5.53914,1.68656) -- (5.53989,1.68587) -- (5.54063,1.68519) -- (5.54138,1.6845) -- (5.54213,1.68382) --
 (5.54287,1.68314) -- (5.54362,1.68246) -- (5.54436,1.68178) -- (5.54511,1.6811) -- (5.54585,1.68042) -- (5.5466,1.67975) -- (5.54735,1.67907) -- (5.54809,1.6784) -- (5.54884,1.67773) -- (5.54958,1.67705) -- (5.55033,1.67638) -- (5.55107,1.67571) --
 (5.55182,1.67504) -- (5.55257,1.67437) -- (5.55331,1.67371) -- (5.55406,1.67304) -- (5.5548,1.67237) -- (5.55555,1.67171) -- (5.5563,1.67105) -- (5.55704,1.67038) -- (5.55779,1.66972) -- (5.55853,1.66906) -- (5.55928,1.6684) -- (5.56003,1.66774) --
 (5.56077,1.66708) -- (5.56152,1.66643) -- (5.56226,1.66577) -- (5.56301,1.66512) -- (5.56375,1.66446) -- (5.5645,1.66381) -- (5.56525,1.66316) -- (5.56599,1.66251) -- (5.56674,1.66186) -- (5.56748,1.66121) -- (5.56823,1.66057) -- (5.56897,1.65992)
 -- (5.56972,1.65927) -- (5.57047,1.65863) -- (5.57121,1.65799) -- (5.57196,1.65734) -- (5.5727,1.6567) -- (5.57345,1.65606) -- (5.5742,1.65542) -- (5.57494,1.65478) -- (5.57569,1.65415) -- (5.57643,1.65351) -- (5.57718,1.65288) -- (5.57793,1.65224)
 -- (5.57867,1.65161) -- (5.57942,1.65098) -- (5.58016,1.65035) -- (5.58091,1.64972) -- (5.58165,1.64909) -- (5.5824,1.64846) -- (5.58315,1.64783) -- (5.58389,1.64721) -- (5.58464,1.64658) -- (5.58538,1.64596) -- (5.58613,1.64534) --
 (5.58687,1.64471) -- (5.58762,1.64409) -- (5.58837,1.64348) -- (5.58911,1.64286) -- (5.58986,1.64224) -- (5.5906,1.64162) -- (5.59135,1.64101) -- (5.5921,1.64039) -- (5.59284,1.63978) -- (5.59359,1.63917) -- (5.59433,1.63856) -- (5.59508,1.63795) --
 (5.59583,1.63734) -- (5.59657,1.63673) -- (5.59732,1.63613) -- (5.59806,1.63552) -- (5.59881,1.63492) -- (5.59955,1.63431) -- (5.6003,1.63371) -- (5.60105,1.63311) -- (5.60179,1.63251) -- (5.60254,1.63191) -- (5.60328,1.63131) -- (5.60403,1.63072)
 -- (5.60477,1.63012) -- (5.60552,1.62953) -- (5.60627,1.62893) -- (5.60701,1.62834) -- (5.60776,1.62775) -- (5.6085,1.62716) -- (5.60925,1.62657) -- (5.61,1.62598) -- (5.61074,1.62539) -- (5.61149,1.62481) -- (5.61223,1.62422) -- (5.61298,1.62364)
 -- (5.61373,1.62306) -- (5.61447,1.62248) -- (5.61522,1.6219) -- (5.61596,1.62132) -- (5.61671,1.62074) -- (5.61745,1.62016) -- (5.6182,1.61959) -- (5.61895,1.61901) -- (5.61969,1.61844) -- (5.62044,1.61787) -- (5.62118,1.6173) -- (5.62193,1.61673)
 -- (5.62267,1.61616) -- (5.62342,1.61559) -- (5.62417,1.61502) -- (5.62491,1.61446) -- (5.62566,1.61389) -- (5.6264,1.61333) -- (5.62715,1.61277) -- (5.6279,1.61221) -- (5.62864,1.61165) -- (5.62939,1.61109) -- (5.63013,1.61053) -- (5.63088,1.60997)
 -- (5.63163,1.60942) -- (5.63237,1.60886) -- (5.63312,1.60831) -- (5.63386,1.60776) -- (5.63461,1.60721) -- (5.63535,1.60666) -- (5.6361,1.60611) -- (5.63685,1.60556) -- (5.63759,1.60502) -- (5.63834,1.60447) -- (5.63908,1.60393) --
 (5.63983,1.60339) -- (5.64057,1.60284) -- (5.64132,1.6023) -- (5.64207,1.60177) -- (5.64281,1.60123) -- (5.64356,1.60069) -- (5.6443,1.60016) -- (5.64505,1.59962) -- (5.6458,1.59909) -- (5.64654,1.59856) -- (5.64729,1.59803) -- (5.64803,1.5975) --
 (5.64878,1.59697) -- (5.64953,1.59644) -- (5.65027,1.59591) -- (5.65102,1.59539) -- (5.65176,1.59487) -- (5.65251,1.59434) -- (5.65325,1.59382) -- (5.654,1.5933) -- (5.65475,1.59278) -- (5.65549,1.59226) -- (5.65624,1.59175) -- (5.65698,1.59123) --
 (5.65773,1.59072) -- (5.65847,1.59021) -- (5.65922,1.58969) -- (5.65997,1.58918) -- (5.66071,1.58867) -- (5.66146,1.58817) -- (5.6622,1.58766) -- (5.66295,1.58715) -- (5.6637,1.58665) -- (5.66444,1.58615) -- (5.66519,1.58564) -- (5.66593,1.58514) --
 (5.66668,1.58464) -- (5.66743,1.58415) -- (5.66817,1.58365) -- (5.66892,1.58315) -- (5.66966,1.58266) -- (5.67041,1.58216) -- (5.67115,1.58167) -- (5.6719,1.58118) -- (5.67265,1.58069) -- (5.67339,1.5802) -- (5.67414,1.57972) -- (5.67488,1.57923) --
 (5.67563,1.57875) -- (5.67637,1.57826) -- (5.67712,1.57778) -- (5.67787,1.5773) -- (5.67861,1.57682) -- (5.67936,1.57634) -- (5.6801,1.57586) -- (5.68085,1.57539) -- (5.6816,1.57491) -- (5.68234,1.57444) -- (5.68309,1.57397) -- (5.68383,1.57349) --
 (5.68458,1.57303) -- (5.68533,1.57256) -- (5.68607,1.57209) -- (5.68682,1.57162) -- (5.68756,1.57116) -- (5.68831,1.5707) -- (5.68905,1.57023) -- (5.6898,1.56977) -- (5.69055,1.56931) -- (5.69129,1.56885) -- (5.69204,1.5684) -- (5.69278,1.56794) --
 (5.69353,1.56749) -- (5.69427,1.56703) -- (5.69502,1.56658) -- (5.69577,1.56613) -- (5.69651,1.56568) -- (5.69726,1.56523) -- (5.698,1.56479) -- (5.69875,1.56434) -- (5.6995,1.5639) -- (5.70024,1.56345) -- (5.70099,1.56301) -- (5.70173,1.56257) --
 (5.70248,1.56213) -- (5.70323,1.56169) -- (5.70397,1.56126) -- (5.70472,1.56082) -- (5.70546,1.56039) -- (5.70621,1.55996) -- (5.70695,1.55952) -- (5.7077,1.55909) -- (5.70845,1.55867) -- (5.70919,1.55824) -- (5.70994,1.55781) -- (5.71068,1.55739)
 -- (5.71143,1.55696) -- (5.71217,1.55654) -- (5.71292,1.55612) -- (5.71367,1.5557) -- (5.71441,1.55528) -- (5.71516,1.55487) -- (5.7159,1.55445) -- (5.71665,1.55404) -- (5.7174,1.55362) -- (5.71814,1.55321) -- (5.71889,1.5528) -- (5.71963,1.55239)
 -- (5.72038,1.55199) -- (5.72113,1.55158) -- (5.72187,1.55118) -- (5.72262,1.55077) -- (5.72336,1.55037) -- (5.72411,1.54997) -- (5.72485,1.54957) -- (5.7256,1.54917) -- (5.72635,1.54877) -- (5.72709,1.54838) -- (5.72784,1.54799) --
 (5.72858,1.54759) -- (5.72933,1.5472) -- (5.73007,1.54681) -- (5.73082,1.54642) -- (5.73157,1.54603) -- (5.73231,1.54565) -- (5.73306,1.54526) -- (5.7338,1.54488) -- (5.73455,1.5445) -- (5.7353,1.54412) -- (5.73604,1.54374) -- (5.73679,1.54336) --
 (5.73753,1.54299) -- (5.73828,1.54261) -- (5.73903,1.54224) -- (5.73977,1.54186) -- (5.74052,1.54149) -- (5.74126,1.54112) -- (5.74201,1.54076) -- (5.74275,1.54039) -- (5.7435,1.54002) -- (5.74425,1.53966) -- (5.74499,1.5393) -- (5.74574,1.53894) --
 (5.74648,1.53858) -- (5.74723,1.53822) -- (5.74797,1.53786) -- (5.74872,1.5375) -- (5.74947,1.53715) -- (5.75021,1.5368) -- (5.75096,1.53645) -- (5.7517,1.5361) -- (5.75245,1.53575) -- (5.7532,1.5354) -- (5.75394,1.53505) -- (5.75469,1.53471) --
 (5.75543,1.53437) -- (5.75618,1.53402) -- (5.75693,1.53368) -- (5.75767,1.53334) -- (5.75842,1.53301) -- (5.75916,1.53267) -- (5.75991,1.53234) -- (5.76065,1.532) -- (5.7614,1.53167) -- (5.76215,1.53134) -- (5.76289,1.53101) -- (5.76364,1.53068) --
 (5.76438,1.53036) -- (5.76513,1.53003) -- (5.76587,1.52971) -- (5.76662,1.52939) -- (5.76737,1.52907) -- (5.76811,1.52875) -- (5.76886,1.52843) -- (5.7696,1.52812) -- (5.77035,1.5278) -- (5.7711,1.52749) -- (5.77184,1.52718) -- (5.77259,1.52687) --
 (5.77333,1.52656) -- (5.77408,1.52625) -- (5.77483,1.52594) -- (5.77557,1.52564) -- (5.77632,1.52534) -- (5.77706,1.52503) -- (5.77781,1.52473) -- (5.77855,1.52444) -- (5.7793,1.52414) -- (5.78005,1.52384) -- (5.78079,1.52355) -- (5.78154,1.52326)
 -- (5.78228,1.52296) -- (5.78303,1.52267) -- (5.78377,1.52239) -- (5.78452,1.5221) -- (5.78527,1.52181) -- (5.78601,1.52153) -- (5.78676,1.52125) -- (5.7875,1.52096) -- (5.78825,1.52068) -- (5.789,1.52041) -- (5.78974,1.52013) -- (5.79049,1.51985)
 -- (5.79123,1.51958) -- (5.79198,1.51931) -- (5.79273,1.51904) -- (5.79347,1.51877) -- (5.79422,1.5185) -- (5.79496,1.51823) -- (5.79571,1.51797) -- (5.79645,1.51771) -- (5.7972,1.51744) -- (5.79795,1.51718) -- (5.79869,1.51692) -- (5.79944,1.51667)
 -- (5.80018,1.51641) -- (5.80093,1.51616) -- (5.80167,1.5159) -- (5.80242,1.51565) -- (5.80317,1.5154) -- (5.80391,1.51515) -- (5.80466,1.51491) -- (5.8054,1.51466) -- (5.80615,1.51442) -- (5.8069,1.51418) -- (5.80764,1.51394) -- (5.80839,1.5137) --
 (5.80913,1.51346) -- (5.80988,1.51322) -- (5.81063,1.51299) -- (5.81137,1.51275) -- (5.81212,1.51252) -- (5.81286,1.51229) -- (5.81361,1.51206) -- (5.81435,1.51184) -- (5.8151,1.51161) -- (5.81585,1.51139) -- (5.81659,1.51116) -- (5.81734,1.51094)
 -- (5.81808,1.51072) -- (5.81883,1.51051) -- (5.81957,1.51029) -- (5.82032,1.51007) -- (5.82107,1.50986) -- (5.82181,1.50965) -- (5.82256,1.50944) -- (5.8233,1.50923) -- (5.82405,1.50902) -- (5.8248,1.50882) -- (5.82554,1.50861) -- (5.82629,1.50841)
 -- (5.82703,1.50821) -- (5.82778,1.50801) -- (5.82853,1.50781) -- (5.82927,1.50762) -- (5.83002,1.50742) -- (5.83076,1.50723) -- (5.83151,1.50704) -- (5.83225,1.50685) -- (5.833,1.50666) -- (5.83375,1.50647) -- (5.83449,1.50629) -- (5.83524,1.5061)
 -- (5.83598,1.50592) -- (5.83673,1.50574) -- (5.83747,1.50556) -- (5.83822,1.50538) -- (5.83897,1.50521) -- (5.83971,1.50503) -- (5.84046,1.50486) -- (5.8412,1.50469) -- (5.84195,1.50452) -- (5.8427,1.50435) -- (5.84344,1.50419) -- (5.84419,1.50402)
 -- (5.84493,1.50386) -- (5.84568,1.5037) -- (5.84643,1.50354) -- (5.84717,1.50338) -- (5.84792,1.50322) -- (5.84866,1.50307) -- (5.84941,1.50291) -- (5.85015,1.50276) -- (5.8509,1.50261) -- (5.85165,1.50246) -- (5.85239,1.50231) -- (5.85314,1.50217)
 -- (5.85388,1.50203) -- (5.85463,1.50188) -- (5.85537,1.50174) -- (5.85612,1.5016) -- (5.85687,1.50147) -- (5.85761,1.50133) -- (5.85836,1.5012) -- (5.8591,1.50106) -- (5.85985,1.50093) -- (5.8606,1.5008) -- (5.86134,1.50067) -- (5.86209,1.50055) --
 (5.86283,1.50042) -- (5.86358,1.5003) -- (5.86433,1.50018) -- (5.86507,1.50006) -- (5.86582,1.49994) -- (5.86656,1.49983) -- (5.86731,1.49971) -- (5.86805,1.4996) -- (5.8688,1.49949) -- (5.86955,1.49938) -- (5.87029,1.49927) -- (5.87104,1.49916) --
 (5.87178,1.49906) -- (5.87253,1.49895) -- (5.87327,1.49885) -- (5.87402,1.49875) -- (5.87477,1.49865) -- (5.87551,1.49856) -- (5.87626,1.49846) -- (5.877,1.49837) -- (5.87775,1.49828) -- (5.8785,1.49819) -- (5.87924,1.4981) -- (5.87999,1.49801) --
 (5.88073,1.49793) -- (5.88148,1.49784) -- (5.88223,1.49776) -- (5.88297,1.49768) -- (5.88372,1.4976) -- (5.88446,1.49753) -- (5.88521,1.49745) -- (5.88595,1.49738) -- (5.8867,1.49731) -- (5.88745,1.49724) -- (5.88819,1.49717) -- (5.88894,1.4971) --
 (5.88968,1.49704) -- (5.89043,1.49697) -- (5.89117,1.49691) -- (5.89192,1.49685) -- (5.89267,1.49679) -- (5.89341,1.49674) -- (5.89416,1.49668) -- (5.8949,1.49663) -- (5.89565,1.49658) -- (5.8964,1.49653) -- (5.89714,1.49648) -- (5.89789,1.49643) --
 (5.89863,1.49639) -- (5.89938,1.49634) -- (5.90013,1.4963) -- (5.90087,1.49626) -- (5.90162,1.49623) -- (5.90236,1.49619) -- (5.90311,1.49616) -- (5.90385,1.49612) -- (5.9046,1.49609) -- (5.90535,1.49606) -- (5.90609,1.49603) -- (5.90684,1.49601) --
 (5.90758,1.49598) -- (5.90833,1.49596) -- (5.90907,1.49594) -- (5.90982,1.49592) -- (5.91057,1.4959) -- (5.91131,1.49589) -- (5.91206,1.49588) -- (5.9128,1.49586) -- (5.91355,1.49585) -- (5.9143,1.49584) -- (5.91504,1.49584) -- (5.91579,1.49583) --
 (5.91653,1.49583) -- (5.91728,1.49583) -- (5.91803,1.49583) -- (5.91877,1.49583) -- (5.91952,1.49583) -- (5.92026,1.49584) -- (5.92101,1.49584) -- (5.92175,1.49585) -- (5.9225,1.49586) -- (5.92325,1.49587) -- (5.92399,1.49589) -- (5.92474,1.4959) --
 (5.92548,1.49592) -- (5.92623,1.49594) -- (5.92698,1.49596) -- (5.92772,1.49598) -- (5.92847,1.49601) -- (5.92921,1.49604) -- (5.92996,1.49606) -- (5.9307,1.49609) -- (5.93145,1.49612) -- (5.9322,1.49616) -- (5.93294,1.49619) -- (5.93369,1.49623) --
 (5.93443,1.49627) -- (5.93518,1.49631) -- (5.93593,1.49635) -- (5.93667,1.49639) -- (5.93742,1.49644) -- (5.93816,1.49649) -- (5.93891,1.49654) -- (5.93965,1.49659) -- (5.9404,1.49664) -- (5.94115,1.49669) -- (5.94189,1.49675) -- (5.94264,1.49681)
 -- (5.94338,1.49687) -- (5.94413,1.49693) -- (5.94488,1.49699) -- (5.94562,1.49706) -- (5.94637,1.49713) -- (5.94711,1.4972) -- (5.94786,1.49727) -- (5.9486,1.49734) -- (5.94935,1.49741) -- (5.9501,1.49749) -- (5.95084,1.49757) -- (5.95159,1.49765)
 -- (5.95233,1.49773) -- (5.95308,1.49781) -- (5.95382,1.4979) -- (5.95457,1.49798) -- (5.95532,1.49807) -- (5.95606,1.49816) -- (5.95681,1.49825) -- (5.95755,1.49835) -- (5.9583,1.49844) -- (5.95905,1.49854) -- (5.95979,1.49864) -- (5.96054,1.49874)
 -- (5.96128,1.49885) -- (5.96203,1.49895) -- (5.96278,1.49906) -- (5.96352,1.49917) -- (5.96427,1.49928) -- (5.96501,1.49939) -- (5.96576,1.4995) -- (5.9665,1.49962) -- (5.96725,1.49974) -- (5.968,1.49986) -- (5.96874,1.49998) -- (5.96949,1.5001) --
 (5.97023,1.50023) -- (5.97098,1.50035) -- (5.97172,1.50048) -- (5.97247,1.50061) -- (5.97322,1.50075) -- (5.97396,1.50088) -- (5.97471,1.50102) -- (5.97545,1.50115) -- (5.9762,1.50129) -- (5.97695,1.50144) -- (5.97769,1.50158) -- (5.97844,1.50172)
 -- (5.97918,1.50187) -- (5.97993,1.50202) -- (5.98068,1.50217) -- (5.98142,1.50232) -- (5.98217,1.50248) -- (5.98291,1.50264) -- (5.98366,1.50279) -- (5.9844,1.50295) -- (5.98515,1.50312) -- (5.9859,1.50328) -- (5.98664,1.50345) -- (5.98739,1.50361)
 -- (5.98813,1.50378) -- (5.98888,1.50396) -- (5.98962,1.50413) -- (5.99037,1.5043) -- (5.99112,1.50448) -- (5.99186,1.50466) -- (5.99261,1.50484) -- (5.99335,1.50502) -- (5.9941,1.50521) -- (5.99485,1.5054) -- (5.99559,1.50558) -- (5.99634,1.50577)
 -- (5.99708,1.50597) -- (5.99783,1.50616) -- (5.99858,1.50636) -- (5.99932,1.50656) -- (6.00007,1.50676) -- (6.00081,1.50696) -- (6.00156,1.50716) -- (6.0023,1.50737) -- (6.00305,1.50757) -- (6.0038,1.50778) -- (6.00454,1.50799) -- (6.00529,1.50821)
 -- (6.00603,1.50842) -- (6.00678,1.50864) -- (6.00752,1.50886) -- (6.00827,1.50908) -- (6.00902,1.5093) -- (6.00976,1.50953) -- (6.01051,1.50975) -- (6.01125,1.50998) -- (6.012,1.51021) -- (6.01275,1.51044) -- (6.01349,1.51068) -- (6.01424,1.51091)
 -- (6.01498,1.51115) -- (6.01573,1.51139) -- (6.01648,1.51163) -- (6.01722,1.51188) -- (6.01797,1.51212) -- (6.01871,1.51237) -- (6.01946,1.51262) -- (6.0202,1.51287) -- (6.02095,1.51313) -- (6.0217,1.51338) -- (6.02244,1.51364) -- (6.02319,1.5139)
 -- (6.02393,1.51416) -- (6.02468,1.51442) -- (6.02542,1.51469) -- (6.02617,1.51495) -- (6.02692,1.51522) -- (6.02766,1.51549) -- (6.02841,1.51577) -- (6.02915,1.51604) -- (6.0299,1.51632) -- (6.03065,1.5166) -- (6.03139,1.51688) -- (6.03214,1.51716)
 -- (6.03288,1.51744) -- (6.03363,1.51773) -- (6.03438,1.51802) -- (6.03512,1.51831) -- (6.03587,1.5186) -- (6.03661,1.5189) -- (6.03736,1.51919) -- (6.0381,1.51949) -- (6.03885,1.51979) -- (6.0396,1.52009) -- (6.04034,1.5204) -- (6.04109,1.5207) --
 (6.04183,1.52101) -- (6.04258,1.52132) -- (6.04332,1.52163) -- (6.04407,1.52195) -- (6.04482,1.52226) -- (6.04556,1.52258) -- (6.04631,1.5229) -- (6.04705,1.52322) -- (6.0478,1.52355) -- (6.04855,1.52387) -- (6.04929,1.5242) -- (6.05004,1.52453) --
 (6.05078,1.52486) -- (6.05153,1.5252) -- (6.05228,1.52553) -- (6.05302,1.52587) -- (6.05377,1.52621) -- (6.05451,1.52655) -- (6.05526,1.5269) -- (6.056,1.52724) -- (6.05675,1.52759) -- (6.0575,1.52794) -- (6.05824,1.52829) -- (6.05899,1.52865) --
 (6.05973,1.529) -- (6.06048,1.52936) -- (6.06122,1.52972) -- (6.06197,1.53008) -- (6.06272,1.53045) -- (6.06346,1.53081) -- (6.06421,1.53118) -- (6.06495,1.53155) -- (6.0657,1.53192) -- (6.06645,1.5323) -- (6.06719,1.53267) -- (6.06794,1.53305) --
 (6.06868,1.53343) -- (6.06943,1.53381) -- (6.07018,1.5342) -- (6.07092,1.53458) -- (6.07167,1.53497) -- (6.07241,1.53536) -- (6.07316,1.53575) -- (6.0739,1.53615) -- (6.07465,1.53654) -- (6.0754,1.53694) -- (6.07614,1.53734) -- (6.07689,1.53775) --
 (6.07763,1.53815) -- (6.07838,1.53856) -- (6.07912,1.53897) -- (6.07987,1.53938) -- (6.08062,1.53979) -- (6.08136,1.5402) -- (6.08211,1.54062) -- (6.08285,1.54104) -- (6.0836,1.54146) -- (6.08435,1.54188) -- (6.08509,1.54231) -- (6.08584,1.54273) --
 (6.08658,1.54316) -- (6.08733,1.54359) -- (6.08808,1.54403) -- (6.08882,1.54446) -- (6.08957,1.5449) -- (6.09031,1.54534) -- (6.09106,1.54578) -- (6.0918,1.54622) -- (6.09255,1.54667) -- (6.0933,1.54712) -- (6.09404,1.54757) -- (6.09479,1.54802) --
 (6.09553,1.54847) -- (6.09628,1.54893) -- (6.09702,1.54939) -- (6.09777,1.54985) -- (6.09852,1.55031) -- (6.09926,1.55077) -- (6.10001,1.55124) -- (6.10075,1.55171) -- (6.1015,1.55218) -- (6.10225,1.55265) -- (6.10299,1.55312) -- (6.10374,1.5536) --
 (6.10448,1.55408) -- (6.10523,1.55456) -- (6.10598,1.55504) -- (6.10672,1.55553) -- (6.10747,1.55601) -- (6.10821,1.5565) -- (6.10896,1.55699) -- (6.1097,1.55749) -- (6.11045,1.55798) -- (6.1112,1.55848) -- (6.11194,1.55898) -- (6.11269,1.55948) --
 (6.11343,1.55999) -- (6.11418,1.56049) -- (6.11492,1.561) -- (6.11567,1.56151) -- (6.11642,1.56202) -- (6.11716,1.56254) -- (6.11791,1.56305) -- (6.11865,1.56357) -- (6.1194,1.56409) -- (6.12015,1.56461) -- (6.12089,1.56514) -- (6.12164,1.56567) --
 (6.12238,1.5662) -- (6.12313,1.56673) -- (6.12388,1.56726) -- (6.12462,1.5678) -- (6.12537,1.56833) -- (6.12611,1.56887) -- (6.12686,1.56941) -- (6.1276,1.56996) -- (6.12835,1.5705) -- (6.1291,1.57105) -- (6.12984,1.5716) -- (6.13059,1.57216) --
 (6.13133,1.57271) -- (6.13208,1.57327) -- (6.13282,1.57383) -- (6.13357,1.57439) -- (6.13432,1.57495) -- (6.13506,1.57552) -- (6.13581,1.57608) -- (6.13655,1.57665) -- (6.1373,1.57722) -- (6.13805,1.5778) -- (6.13879,1.57837) -- (6.13954,1.57895) --
 (6.14028,1.57953) -- (6.14103,1.58012) -- (6.14178,1.5807) -- (6.14252,1.58129) -- (6.14327,1.58188) -- (6.14401,1.58247) -- (6.14476,1.58306) -- (6.1455,1.58366) -- (6.14625,1.58425) -- (6.147,1.58485) -- (6.14774,1.58545) -- (6.14849,1.58606) --
 (6.14923,1.58666) -- (6.14998,1.58727) -- (6.15072,1.58788) -- (6.15147,1.5885) -- (6.15222,1.58911) -- (6.15296,1.58973) -- (6.15371,1.59035) -- (6.15445,1.59097) -- (6.1552,1.59159) -- (6.15595,1.59222) -- (6.15669,1.59285) -- (6.15744,1.59348) --
 (6.15818,1.59411) -- (6.15893,1.59474) -- (6.15968,1.59538) -- (6.16042,1.59602) -- (6.16117,1.59666) -- (6.16191,1.5973) -- (6.16266,1.59795) -- (6.1634,1.59859) -- (6.16415,1.59924) -- (6.1649,1.5999) -- (6.16564,1.60055) -- (6.16639,1.60121) --
 (6.16713,1.60187) -- (6.16788,1.60253) -- (6.16862,1.60319) -- (6.16937,1.60385) -- (6.17012,1.60452) -- (6.17086,1.60519) -- (6.17161,1.60586) -- (6.17235,1.60654) -- (6.1731,1.60721) -- (6.17385,1.60789) -- (6.17459,1.60857) -- (6.17534,1.60925)
 -- (6.17608,1.60994) -- (6.17683,1.61063) -- (6.17758,1.61131) -- (6.17832,1.61201) -- (6.17907,1.6127) -- (6.17981,1.6134) -- (6.18056,1.61409) -- (6.1813,1.61479) -- (6.18205,1.6155) -- (6.1828,1.6162) -- (6.18354,1.61691) -- (6.18429,1.61762) --
 (6.18503,1.61833) -- (6.18578,1.61904) -- (6.18652,1.61976) -- (6.18727,1.62048) -- (6.18802,1.6212) -- (6.18876,1.62192) -- (6.18951,1.62265) -- (6.19025,1.62337) -- (6.191,1.6241) -- (6.19175,1.62483) -- (6.19249,1.62557) -- (6.19324,1.6263) --
 (6.19398,1.62704) -- (6.19473,1.62778) -- (6.19548,1.62853) -- (6.19622,1.62927) -- (6.19697,1.63002) -- (6.19771,1.63077) -- (6.19846,1.63152) -- (6.1992,1.63227) -- (6.19995,1.63303) -- (6.2007,1.63379) -- (6.20144,1.63455) -- (6.20219,1.63531) --
 (6.20293,1.63608) -- (6.20368,1.63684) -- (6.20442,1.63761) -- (6.20517,1.63839) -- (6.20592,1.63916) -- (6.20666,1.63994) -- (6.20741,1.64072) -- (6.20815,1.6415) -- (6.2089,1.64228) -- (6.20965,1.64307) -- (6.21039,1.64385) -- (6.21114,1.64464) --
 (6.21188,1.64544) -- (6.21263,1.64623) -- (6.21338,1.64703) -- (6.21412,1.64783) -- (6.21487,1.64863) -- (6.21561,1.64943) -- (6.21636,1.65024) -- (6.2171,1.65105) -- (6.21785,1.65186) -- (6.2186,1.65267) -- (6.21934,1.65348) -- (6.22009,1.6543) --
 (6.22083,1.65512) -- (6.22158,1.65594) -- (6.22232,1.65677) -- (6.22307,1.65759) -- (6.22382,1.65842) -- (6.22456,1.65925) -- (6.22531,1.66009) -- (6.22605,1.66092) -- (6.2268,1.66176) -- (6.22755,1.6626) -- (6.22829,1.66344) -- (6.22904,1.66429) --
 (6.22978,1.66514) -- (6.23053,1.66599) -- (6.23128,1.66684) -- (6.23202,1.66769) -- (6.23277,1.66855) -- (6.23351,1.66941) -- (6.23426,1.67027) -- (6.235,1.67113) -- (6.23575,1.672) -- (6.2365,1.67286) -- (6.23724,1.67373) -- (6.23799,1.67461) --
 (6.23873,1.67548) -- (6.23948,1.67636) -- (6.24022,1.67724) -- (6.24097,1.67812) -- (6.24172,1.679) -- (6.24246,1.67989) -- (6.24321,1.68078) -- (6.24395,1.68167) -- (6.2447,1.68256) -- (6.24545,1.68346) -- (6.24619,1.68436) -- (6.24694,1.68526) --
 (6.24768,1.68616) -- (6.24843,1.68707) -- (6.24918,1.68797) -- (6.24992,1.68888) -- (6.25067,1.68979) -- (6.25141,1.69071) -- (6.25216,1.69162) -- (6.2529,1.69254) -- (6.25365,1.69347) -- (6.2544,1.69439) -- (6.25514,1.69532) -- (6.25589,1.69624) --
 (6.25663,1.69717) -- (6.25738,1.69811) -- (6.25812,1.69904) -- (6.25887,1.69998) -- (6.25962,1.70092) -- (6.26036,1.70186) -- (6.26111,1.70281) -- (6.26185,1.70375) -- (6.2626,1.7047) -- (6.26335,1.70566) -- (6.26409,1.70661) -- (6.26484,1.70757) --
 (6.26558,1.70852) -- (6.26633,1.70949) -- (6.26708,1.71045) -- (6.26782,1.71142) -- (6.26857,1.71238) -- (6.26931,1.71335) -- (6.27006,1.71433) -- (6.2708,1.7153) -- (6.27155,1.71628) -- (6.2723,1.71726) -- (6.27304,1.71824) -- (6.27379,1.71923) --
 (6.27453,1.72021) -- (6.27528,1.7212) -- (6.27602,1.7222) -- (6.27677,1.72319) -- (6.27752,1.72419) -- (6.27826,1.72519) -- (6.27901,1.72619) -- (6.27975,1.72719) -- (6.2805,1.7282) -- (6.28125,1.72921) -- (6.28199,1.73022) -- (6.28274,1.73123) --
 (6.28348,1.73225) -- (6.28423,1.73326) -- (6.28498,1.73428) -- (6.28572,1.73531) -- (6.28647,1.73633) -- (6.28721,1.73736) -- (6.28796,1.73839) -- (6.2887,1.73942) -- (6.28945,1.74046) -- (6.2902,1.74149) -- (6.29094,1.74253) -- (6.29169,1.74358) --
 (6.29243,1.74462) -- (6.29318,1.74567) -- (6.29392,1.74672) -- (6.29467,1.74777) -- (6.29542,1.74882) -- (6.29616,1.74988) -- (6.29691,1.75094) -- (6.29765,1.752) -- (6.2984,1.75306) -- (6.29915,1.75413) -- (6.29989,1.7552) -- (6.30064,1.75627) --
 (6.30138,1.75734) -- (6.30213,1.75842) -- (6.30288,1.75949) -- (6.30362,1.76057) -- (6.30437,1.76166) -- (6.30511,1.76274) -- (6.30586,1.76383) -- (6.3066,1.76492) -- (6.30735,1.76601) -- (6.3081,1.76711) -- (6.30884,1.7682) -- (6.30959,1.7693) --
 (6.31033,1.77041) -- (6.31108,1.77151) -- (6.31182,1.77262) -- (6.31257,1.77373) -- (6.31332,1.77484) -- (6.31406,1.77595) -- (6.31481,1.77707) -- (6.31555,1.77819) -- (6.3163,1.77931) -- (6.31705,1.78044) -- (6.31779,1.78156) -- (6.31854,1.78269)
 -- (6.31928,1.78382) -- (6.32003,1.78496) -- (6.32078,1.78609) -- (6.32152,1.78723) -- (6.32227,1.78837) -- (6.32301,1.78952) -- (6.32376,1.79066) -- (6.3245,1.79181) -- (6.32525,1.79296) -- (6.326,1.79411) -- (6.32674,1.79527) -- (6.32749,1.79643)
 -- (6.32823,1.79759) -- (6.32898,1.79875) -- (6.32972,1.79992) -- (6.33047,1.80109) -- (6.33122,1.80226) -- (6.33196,1.80343) -- (6.33271,1.8046) -- (6.33345,1.80578) -- (6.3342,1.80696) -- (6.33495,1.80815) -- (6.33569,1.80933) -- (6.33644,1.81052)
 -- (6.33718,1.81171) -- (6.33793,1.8129) -- (6.33868,1.8141) -- (6.33942,1.81529) -- (6.34017,1.81649) -- (6.34091,1.8177) -- (6.34166,1.8189) -- (6.3424,1.82011) -- (6.34315,1.82132) -- (6.3439,1.82253) -- (6.34464,1.82375) -- (6.34539,1.82496) --
 (6.34613,1.82618) -- (6.34688,1.82741) -- (6.34762,1.82863) -- (6.34837,1.82986) -- (6.34912,1.83109) -- (6.34986,1.83232) -- (6.35061,1.83355) -- (6.35135,1.83479) -- (6.3521,1.83603) -- (6.35285,1.83727) -- (6.35359,1.83852) -- (6.35434,1.83977)
 -- (6.35508,1.84101) -- (6.35583,1.84227) -- (6.35658,1.84352) -- (6.35732,1.84478) -- (6.35807,1.84604) -- (6.35881,1.8473) -- (6.35956,1.84856) -- (6.3603,1.84983) -- (6.36105,1.8511) -- (6.3618,1.85237) -- (6.36254,1.85365) -- (6.36329,1.85493)
 -- (6.36403,1.8562) -- (6.36478,1.85749) -- (6.36553,1.85877) -- (6.36627,1.86006) -- (6.36702,1.86135) -- (6.36776,1.86264) -- (6.36851,1.86393) -- (6.36925,1.86523) -- (6.37,1.86653) -- (6.37075,1.86783) -- (6.37149,1.86914) -- (6.37224,1.87045)
 -- (6.37298,1.87176) -- (6.37373,1.87307) -- (6.37448,1.87438) -- (6.37522,1.8757) -- (6.37597,1.87702) -- (6.37671,1.87834) -- (6.37746,1.87967) -- (6.3782,1.88099) -- (6.37895,1.88232) -- (6.3797,1.88366) -- (6.38044,1.88499) -- (6.38119,1.88633)
 -- (6.38193,1.88767) -- (6.38268,1.88901) -- (6.38343,1.89036) -- (6.38417,1.8917) -- (6.38492,1.89305) -- (6.38566,1.89441) -- (6.38641,1.89576) -- (6.38715,1.89712) -- (6.3879,1.89848) -- (6.38865,1.89984) -- (6.38939,1.90121) -- (6.39014,1.90258)
 -- (6.39088,1.90395) -- (6.39163,1.90532) -- (6.39237,1.90669) -- (6.39312,1.90807) -- (6.39387,1.90945) -- (6.39461,1.91084) -- (6.39536,1.91222) -- (6.3961,1.91361) -- (6.39685,1.915) -- (6.3976,1.91639) -- (6.39834,1.91779) -- (6.39909,1.91919)
 -- (6.39983,1.92059) -- (6.40058,1.92199) -- (6.40133,1.9234) -- (6.40207,1.92481) -- (6.40282,1.92622) -- (6.40356,1.92763) -- (6.40431,1.92905) -- (6.40505,1.93047) -- (6.4058,1.93189) -- (6.40655,1.93331) -- (6.40729,1.93474) -- (6.40804,1.93617)
 -- (6.40878,1.9376) -- (6.40953,1.93904) -- (6.41027,1.94047) -- (6.41102,1.94191) -- (6.41177,1.94335) -- (6.41251,1.9448) -- (6.41326,1.94625) -- (6.414,1.9477) -- (6.41475,1.94915) -- (6.4155,1.9506) -- (6.41624,1.95206) -- (6.41699,1.95352) --
 (6.41773,1.95498) -- (6.41848,1.95645) -- (6.41923,1.95792) -- (6.41997,1.95939) -- (6.42072,1.96086) -- (6.42146,1.96233) -- (6.42221,1.96381) -- (6.42295,1.96529) -- (6.4237,1.96678) -- (6.42445,1.96826) -- (6.42519,1.96975) -- (6.42594,1.97124)
 -- (6.42668,1.97274) -- (6.42743,1.97423) -- (6.42817,1.97573) -- (6.42892,1.97723) -- (6.42967,1.97874) -- (6.43041,1.98024) -- (6.43116,1.98175) -- (6.4319,1.98326) -- (6.43265,1.98478) -- (6.4334,1.9863) -- (6.43414,1.98782) -- (6.43489,1.98934)
 -- (6.43563,1.99086) -- (6.43638,1.99239) -- (6.43713,1.99392) -- (6.43787,1.99545) -- (6.43862,1.99699) -- (6.43936,1.99853) -- (6.44011,2.00007) -- (6.44085,2.00161) -- (6.4416,2.00315) -- (6.44235,2.0047) -- (6.44309,2.00625) -- (6.44384,2.00781)
 -- (6.44458,2.00936) -- (6.44533,2.01092) -- (6.44607,2.01248) -- (6.44682,2.01405) -- (6.44757,2.01561) -- (6.44831,2.01718) -- (6.44906,2.01876) -- (6.4498,2.02033) -- (6.45055,2.02191) -- (6.4513,2.02349) -- (6.45204,2.02507) -- (6.45279,2.02665)
 -- (6.45353,2.02824) -- (6.45428,2.02983) -- (6.45503,2.03143) -- (6.45577,2.03302) -- (6.45652,2.03462) -- (6.45726,2.03622) -- (6.45801,2.03782) -- (6.45875,2.03943) -- (6.4595,2.04104) -- (6.46025,2.04265) -- (6.46099,2.04426) --
 (6.46174,2.04588) -- (6.46248,2.0475) -- (6.46323,2.04912) -- (6.46397,2.05075) -- (6.46472,2.05237) -- (6.46547,2.054) -- (6.46621,2.05564) -- (6.46696,2.05727) -- (6.4677,2.05891) -- (6.46845,2.06055) -- (6.4692,2.06219) -- (6.46994,2.06384) --
 (6.47069,2.06549) -- (6.47143,2.06714) -- (6.47218,2.06879) -- (6.47293,2.07045) -- (6.47367,2.07211) -- (6.47442,2.07377) -- (6.47516,2.07543) -- (6.47591,2.0771) -- (6.47665,2.07877) -- (6.4774,2.08044) -- (6.47815,2.08212) -- (6.47889,2.08379) --
 (6.47964,2.08547) -- (6.48038,2.08716) -- (6.48113,2.08884) -- (6.48187,2.09053) -- (6.48262,2.09222) -- (6.48337,2.09392) -- (6.48411,2.09561) -- (6.48486,2.09731) -- (6.4856,2.09901) -- (6.48635,2.10072) -- (6.4871,2.10242) -- (6.48784,2.10413) --
 (6.48859,2.10585) -- (6.48933,2.10756) -- (6.49008,2.10928) -- (6.49083,2.111) -- (6.49157,2.11272) -- (6.49232,2.11445) -- (6.49306,2.11618) -- (6.49381,2.11791) -- (6.49455,2.11964) -- (6.4953,2.12138) -- (6.49605,2.12312) -- (6.49679,2.12486) --
 (6.49754,2.1266) -- (6.49828,2.12835) -- (6.49903,2.1301) -- (6.49977,2.13185) -- (6.50052,2.13361) -- (6.50127,2.13537) -- (6.50201,2.13713) -- (6.50276,2.13889) -- (6.5035,2.14066) -- (6.50425,2.14243) -- (6.505,2.1442) -- (6.50574,2.14597) --
 (6.50649,2.14775) -- (6.50723,2.14953) -- (6.50798,2.15131) -- (6.50873,2.15309) -- (6.50947,2.15488) -- (6.51022,2.15667) -- (6.51096,2.15847) -- (6.51171,2.16026) -- (6.51245,2.16206) -- (6.5132,2.16386) -- (6.51395,2.16566) -- (6.51469,2.16747)
 -- (6.51544,2.16928) -- (6.51618,2.17109) -- (6.51693,2.17291) -- (6.51767,2.17472) -- (6.51842,2.17655) -- (6.51917,2.17837) -- (6.51991,2.18019) -- (6.52066,2.18202) -- (6.5214,2.18385) -- (6.52215,2.18569) -- (6.5229,2.18752) -- (6.52364,2.18936)
 -- (6.52439,2.1912) -- (6.52513,2.19305) -- (6.52588,2.1949) -- (6.52663,2.19675) -- (6.52737,2.1986) -- (6.52812,2.20045) -- (6.52886,2.20231) -- (6.52961,2.20417) -- (6.53035,2.20604) -- (6.5311,2.2079) -- (6.53185,2.20977) -- (6.53259,2.21165) --
 (6.53334,2.21352) -- (6.53408,2.2154) -- (6.53483,2.21728) -- (6.53557,2.21916) -- (6.53632,2.22105) -- (6.53707,2.22293) -- (6.53781,2.22483) -- (6.53856,2.22672) -- (6.5393,2.22862) -- (6.54005,2.23052) -- (6.5408,2.23242) -- (6.54154,2.23432) --
 (6.54229,2.23623) -- (6.54303,2.23814) -- (6.54378,2.24005) -- (6.54453,2.24197) -- (6.54527,2.24389) -- (6.54602,2.24581) -- (6.54676,2.24773) -- (6.54751,2.24966) -- (6.54825,2.25159) -- (6.549,2.25352) -- (6.54975,2.25546) -- (6.55049,2.25739) --
 (6.55124,2.25933) -- (6.55198,2.26128) -- (6.55273,2.26322) -- (6.55347,2.26517) -- (6.55422,2.26712) -- (6.55497,2.26908) -- (6.55571,2.27104) -- (6.55646,2.273) -- (6.5572,2.27496) -- (6.55795,2.27692) -- (6.5587,2.27889) -- (6.55944,2.28086) --
 (6.56019,2.28284) -- (6.56093,2.28481) -- (6.56168,2.28679) -- (6.56243,2.28877) -- (6.56317,2.29076) -- (6.56392,2.29275) -- (6.56466,2.29474) -- (6.56541,2.29673) -- (6.56615,2.29873) -- (6.5669,2.30072) -- (6.56765,2.30273) -- (6.56839,2.30473)
 -- (6.56914,2.30674) -- (6.56988,2.30875) -- (6.57063,2.31076) -- (6.57137,2.31278) -- (6.57212,2.31479) -- (6.57287,2.31681) -- (6.57361,2.31884) -- (6.57436,2.32086) -- (6.5751,2.32289) -- (6.57585,2.32493) -- (6.5766,2.32696) -- (6.57734,2.329)
 -- (6.57809,2.33104) -- (6.57883,2.33308) -- (6.57958,2.33513) -- (6.58033,2.33718) -- (6.58107,2.33923) -- (6.58182,2.34128) -- (6.58256,2.34334) -- (6.58331,2.3454) -- (6.58405,2.34746) -- (6.5848,2.34953) -- (6.58555,2.3516) -- (6.58629,2.35367)
 -- (6.58704,2.35574) -- (6.58778,2.35782) -- (6.58853,2.3599) -- (6.58927,2.36198) -- (6.59002,2.36407) -- (6.59077,2.36615) -- (6.59151,2.36824) -- (6.59226,2.37034) -- (6.593,2.37243) -- (6.59375,2.37453) -- (6.5945,2.37664) -- (6.59524,2.37874)
 -- (6.59599,2.38085) -- (6.59673,2.38296) -- (6.59748,2.38507) -- (6.59823,2.38719) -- (6.59897,2.38931) -- (6.59972,2.39143) -- (6.60046,2.39355) -- (6.60121,2.39568) -- (6.60195,2.39781) -- (6.6027,2.39994) -- (6.60345,2.40208) --
 (6.60419,2.40422) -- (6.60494,2.40636) -- (6.60568,2.4085) -- (6.60643,2.41065) -- (6.60717,2.4128) -- (6.60792,2.41495) -- (6.60867,2.41711) -- (6.60941,2.41927) -- (6.61016,2.42143) -- (6.6109,2.42359) -- (6.61165,2.42576) -- (6.6124,2.42793) --
 (6.61314,2.4301) -- (6.61389,2.43228) -- (6.61463,2.43446) -- (6.61538,2.43664) -- (6.61613,2.43882) -- (6.61687,2.44101) -- (6.61762,2.4432) -- (6.61836,2.44539) -- (6.61911,2.44758) -- (6.61985,2.44978) -- (6.6206,2.45198) -- (6.62135,2.45419) --
 (6.62209,2.45639) -- (6.62284,2.4586) -- (6.62358,2.46082) -- (6.62433,2.46303) -- (6.62507,2.46525) -- (6.62582,2.46747) -- (6.62657,2.46969) -- (6.62731,2.47192) -- (6.62806,2.47415) -- (6.6288,2.47638) -- (6.62955,2.47862) -- (6.6303,2.48085) --
 (6.63104,2.4831) -- (6.63179,2.48534) -- (6.63253,2.48759) -- (6.63328,2.48984) -- (6.63403,2.49209) -- (6.63477,2.49434) -- (6.63552,2.4966) -- (6.63626,2.49886) -- (6.63701,2.50112) -- (6.63775,2.50339) -- (6.6385,2.50566) -- (6.63925,2.50793) --
 (6.63999,2.51021) -- (6.64074,2.51249) -- (6.64148,2.51477) -- (6.64223,2.51705) -- (6.64297,2.51934) -- (6.64372,2.52163) -- (6.64447,2.52392) -- (6.64521,2.52621) -- (6.64596,2.52851) -- (6.6467,2.53081) -- (6.64745,2.53312) -- (6.6482,2.53542) --
 (6.64894,2.53773) -- (6.64969,2.54005) -- (6.65043,2.54236) -- (6.65118,2.54468) -- (6.65193,2.547) -- (6.65267,2.54932) -- (6.65342,2.55165) -- (6.65416,2.55398) -- (6.65491,2.55631) -- (6.65565,2.55865) -- (6.6564,2.56099) -- (6.65715,2.56333) --
 (6.65789,2.56567) -- (6.65864,2.56802) -- (6.65938,2.57037) -- (6.66013,2.57272) -- (6.66087,2.57508) -- (6.66162,2.57744) -- (6.66237,2.5798) -- (6.66311,2.58216) -- (6.66386,2.58453) -- (6.6646,2.5869) -- (6.66535,2.58927) -- (6.6661,2.59165) --
 (6.66684,2.59403) -- (6.66759,2.59641) -- (6.66833,2.59879) -- (6.66908,2.60118) -- (6.66983,2.60357) -- (6.67057,2.60596) -- (6.67132,2.60836) -- (6.67206,2.61076) -- (6.67281,2.61316) -- (6.67355,2.61557) -- (6.6743,2.61797) -- (6.67505,2.62039)
 -- (6.67579,2.6228) -- (6.67654,2.62522) -- (6.67728,2.62764) -- (6.67803,2.63006) -- (6.67877,2.63248) -- (6.67952,2.63491) -- (6.68027,2.63734) -- (6.68101,2.63978) -- (6.68176,2.64221) -- (6.6825,2.64465) -- (6.68325,2.6471) -- (6.684,2.64954) --
 (6.68474,2.65199) -- (6.68549,2.65444) -- (6.68623,2.6569) -- (6.68698,2.65935) -- (6.68773,2.66181) -- (6.68847,2.66428) -- (6.68922,2.66674) -- (6.68996,2.66921) -- (6.69071,2.67168) -- (6.69145,2.67416) -- (6.6922,2.67664) -- (6.69295,2.67912) --
 (6.69369,2.6816) -- (6.69444,2.68409) -- (6.69518,2.68658) -- (6.69593,2.68907) -- (6.69667,2.69156) -- (6.69742,2.69406) -- (6.69817,2.69656) -- (6.69891,2.69907) -- (6.69966,2.70157) -- (6.7004,2.70408) -- (6.70115,2.7066) -- (6.7019,2.70911) --
 (6.70264,2.71163) -- (6.70339,2.71415) -- (6.70413,2.71668) -- (6.70488,2.71921) -- (6.70563,2.72174) -- (6.70637,2.72427) -- (6.70712,2.72681) -- (6.70786,2.72935) -- (6.70861,2.73189) -- (6.70935,2.73443) -- (6.7101,2.73698) -- (6.71085,2.73953)
 -- (6.71159,2.74209) -- (6.71234,2.74464) -- (6.71308,2.7472) -- (6.71383,2.74977) -- (6.71457,2.75233) -- (6.71532,2.7549) -- (6.71607,2.75747) -- (6.71681,2.76005) -- (6.71756,2.76263) -- (6.7183,2.76521) -- (6.71905,2.76779) -- (6.7198,2.77038)
 -- (6.72054,2.77297) -- (6.72129,2.77556) -- (6.72203,2.77815) -- (6.72278,2.78075) -- (6.72353,2.78335) -- (6.72427,2.78596) -- (6.72502,2.78857) -- (6.72576,2.79118) -- (6.72651,2.79379) -- (6.72725,2.79641) -- (6.728,2.79902) -- (6.72875,2.80165)
 -- (6.72949,2.80427) -- (6.73024,2.8069) -- (6.73098,2.80953) -- (6.73173,2.81216) -- (6.73247,2.8148) -- (6.73322,2.81744) -- (6.73397,2.82008) -- (6.73471,2.82273) -- (6.73546,2.82538) -- (6.7362,2.82803) -- (6.73695,2.83069) -- (6.7377,2.83334)
 -- (6.73844,2.836) -- (6.73919,2.83867) -- (6.73993,2.84133) -- (6.74068,2.844) -- (6.74143,2.84668) -- (6.74217,2.84935) -- (6.74292,2.85203) -- (6.74366,2.85471) -- (6.74441,2.8574) -- (6.74515,2.86008) -- (6.7459,2.86277) -- (6.74665,2.86547) --
 (6.74739,2.86816) -- (6.74814,2.87086) -- (6.74888,2.87357) -- (6.74963,2.87627) -- (6.75037,2.87898) -- (6.75112,2.88169) -- (6.75187,2.88441) -- (6.75261,2.88712) -- (6.75336,2.88984) -- (6.7541,2.89257) -- (6.75485,2.89529) -- (6.7556,2.89802) --
 (6.75634,2.90076) -- (6.75709,2.90349) -- (6.75783,2.90623) -- (6.75858,2.90897) -- (6.75933,2.91171) -- (6.76007,2.91446) -- (6.76082,2.91721) -- (6.76156,2.91997) -- (6.76231,2.92272) -- (6.76305,2.92548) -- (6.7638,2.92824) -- (6.76455,2.93101)
 -- (6.76529,2.93378) -- (6.76604,2.93655) -- (6.76678,2.93932) -- (6.76753,2.9421) -- (6.76827,2.94488) -- (6.76902,2.94766) -- (6.76977,2.95045) -- (6.77051,2.95324) -- (6.77126,2.95603) -- (6.772,2.95883) -- (6.77275,2.96163) -- (6.7735,2.96443)
 -- (6.77424,2.96723) -- (6.77499,2.97004) -- (6.77573,2.97285) -- (6.77648,2.97566) -- (6.77723,2.97848) -- (6.77797,2.9813) -- (6.77872,2.98412) -- (6.77946,2.98695) -- (6.78021,2.98978) -- (6.78095,2.99261) -- (6.7817,2.99544) -- (6.78245,2.99828)
 -- (6.78319,3.00112) -- (6.78394,3.00397) -- (6.78468,3.00681) -- (6.78543,3.00966) -- (6.78617,3.01251) -- (6.78692,3.01537) -- (6.78767,3.01823) -- (6.78841,3.02109) -- (6.78916,3.02396) -- (6.7899,3.02682) -- (6.79065,3.02969) -- (6.7914,3.03257)
 -- (6.79214,3.03545) -- (6.79289,3.03833) -- (6.79363,3.04121) -- (6.79438,3.0441) -- (6.79513,3.04698) -- (6.79587,3.04988) -- (6.79662,3.05277) -- (6.79736,3.05567) -- (6.79811,3.05857) -- (6.79885,3.06148) -- (6.7996,3.06438) -- (6.80035,3.06729)
 -- (6.80109,3.07021) -- (6.80184,3.07312) -- (6.80258,3.07604) -- (6.80333,3.07897) -- (6.80408,3.08189) -- (6.80482,3.08482) -- (6.80557,3.08775) -- (6.80631,3.09069) -- (6.80706,3.09362) -- (6.8078,3.09657) -- (6.80855,3.09951) -- (6.8093,3.10246)
 -- (6.81004,3.10541) -- (6.81079,3.10836) -- (6.81153,3.11132) -- (6.81228,3.11428) -- (6.81302,3.11724) -- (6.81377,3.1202) -- (6.81452,3.12317) -- (6.81526,3.12614) -- (6.81601,3.12912) -- (6.81675,3.13209) -- (6.8175,3.13508) -- (6.81825,3.13806)
 -- (6.81899,3.14105) -- (6.81974,3.14404) -- (6.82048,3.14703) -- (6.82123,3.15002) -- (6.82198,3.15302) -- (6.82272,3.15603) -- (6.82347,3.15903) -- (6.82421,3.16204) -- (6.82496,3.16505) -- (6.8257,3.16806) -- (6.82645,3.17108) -- (6.8272,3.1741)
 -- (6.82794,3.17713) -- (6.82869,3.18015) -- (6.82943,3.18318) -- (6.83018,3.18621) -- (6.83092,3.18925) -- (6.83167,3.19229) -- (6.83242,3.19533) -- (6.83316,3.19838) -- (6.83391,3.20142) -- (6.83465,3.20448) -- (6.8354,3.20753) --
 (6.83615,3.21059) -- (6.83689,3.21365) -- (6.83764,3.21671) -- (6.83838,3.21978) -- (6.83913,3.22285) -- (6.83988,3.22592) -- (6.84062,3.229) -- (6.84137,3.23207) -- (6.84211,3.23516) -- (6.84286,3.23824) -- (6.8436,3.24133) -- (6.84435,3.24442) --
 (6.8451,3.24752) -- (6.84584,3.25061) -- (6.84659,3.25371) -- (6.84733,3.25682) -- (6.84808,3.25992) -- (6.84882,3.26303) -- (6.84957,3.26615) -- (6.85032,3.26926) -- (6.85106,3.27238) -- (6.85181,3.2755) -- (6.85255,3.27863) -- (6.8533,3.28176) --
 (6.85405,3.28489) -- (6.85479,3.28802) -- (6.85554,3.29116) -- (6.85628,3.2943) -- (6.85703,3.29745) -- (6.85778,3.30059) -- (6.85852,3.30374) -- (6.85927,3.3069) -- (6.86001,3.31005) -- (6.86076,3.31321) -- (6.8615,3.31637) -- (6.86225,3.31954) --
 (6.863,3.32271) -- (6.86374,3.32588) -- (6.86449,3.32905) -- (6.86523,3.33223) -- (6.86598,3.33541) -- (6.86672,3.3386) -- (6.86747,3.34179) -- (6.86822,3.34498) -- (6.86896,3.34817) -- (6.86971,3.35137) -- (6.87045,3.35457) -- (6.8712,3.35777) --
 (6.87195,3.36097) -- (6.87269,3.36418) -- (6.87344,3.3674) -- (6.87418,3.37061) -- (6.87493,3.37383) -- (6.87568,3.37705) -- (6.87642,3.38028) -- (6.87717,3.3835) -- (6.87791,3.38673) -- (6.87866,3.38997) -- (6.8794,3.39321) -- (6.88015,3.39645) --
 (6.8809,3.39969) -- (6.88164,3.40294) -- (6.88239,3.40619) -- (6.88313,3.40944) -- (6.88388,3.41269) -- (6.88462,3.41595) -- (6.88537,3.41922) -- (6.88612,3.42248) -- (6.88686,3.42575) -- (6.88761,3.42902) -- (6.88835,3.4323) -- (6.8891,3.43557) --
 (6.88985,3.43885) -- (6.89059,3.44214) -- (6.89134,3.44543) -- (6.89208,3.44872) -- (6.89283,3.45201) -- (6.89358,3.45531) -- (6.89432,3.45861) -- (6.89507,3.46191) -- (6.89581,3.46521) -- (6.89656,3.46852) -- (6.8973,3.47184) -- (6.89805,3.47515)
 -- (6.8988,3.47847) -- (6.89954,3.48179) -- (6.90029,3.48512) -- (6.90103,3.48844) -- (6.90178,3.49178) -- (6.90252,3.49511) -- (6.90327,3.49845) -- (6.90402,3.50179) -- (6.90476,3.50513) -- (6.90551,3.50848) -- (6.90625,3.51183) -- (6.907,3.51518)
 -- (6.90775,3.51854) -- (6.90849,3.5219) -- (6.90924,3.52526) -- (6.90998,3.52863) -- (6.91073,3.53199) -- (6.91148,3.53537) -- (6.91222,3.53874) -- (6.91297,3.54212) -- (6.91371,3.5455) -- (6.91446,3.54889) -- (6.9152,3.55227) -- (6.91595,3.55566)
 -- (6.9167,3.55906) -- (6.91744,3.56246) -- (6.91819,3.56586) -- (6.91893,3.56926) -- (6.91968,3.57267) -- (6.92042,3.57608) -- (6.92117,3.57949) -- (6.92192,3.58291) -- (6.92266,3.58633) -- (6.92341,3.58975) -- (6.92415,3.59317) -- (6.9249,3.5966)
 -- (6.92565,3.60003) -- (6.92639,3.60347) -- (6.92714,3.60691) -- (6.92788,3.61035) -- (6.92863,3.61379) -- (6.92938,3.61724) -- (6.93012,3.62069) -- (6.93087,3.62415) -- (6.93161,3.62761) -- (6.93236,3.63107) -- (6.9331,3.63453) -- (6.93385,3.638)
 -- (6.9346,3.64147) -- (6.93534,3.64494) -- (6.93609,3.64842) -- (6.93683,3.6519) -- (6.93758,3.65538) -- (6.93832,3.65887) -- (6.93907,3.66235) -- (6.93982,3.66585) -- (6.94056,3.66934) -- (6.94131,3.67284) -- (6.94205,3.67634) -- (6.9428,3.67985)
 -- (6.94355,3.68336) -- (6.94429,3.68687) -- (6.94504,3.69038) -- (6.94578,3.6939) -- (6.94653,3.69742) -- (6.94728,3.70095) -- (6.94802,3.70447) -- (6.94877,3.708) -- (6.94951,3.71154) -- (6.95026,3.71507) -- (6.951,3.71861) -- (6.95175,3.72216) --
 (6.9525,3.7257) -- (6.95324,3.72925) -- (6.95399,3.73281) -- (6.95473,3.73636) -- (6.95548,3.73992) -- (6.95622,3.74348) -- (6.95697,3.74705) -- (6.95772,3.75062) -- (6.95846,3.75419) -- (6.95921,3.75777) -- (6.95995,3.76134) -- (6.9607,3.76493) --
 (6.96145,3.76851) -- (6.96219,3.7721) -- (6.96294,3.77569) -- (6.96368,3.77929) -- (6.96443,3.78288) -- (6.96518,3.78648) -- (6.96592,3.79009) -- (6.96667,3.7937) -- (6.96741,3.79731) -- (6.96816,3.80092) -- (6.9689,3.80454) -- (6.96965,3.80816) --
 (6.9704,3.81178) -- (6.97114,3.81541) -- (6.97189,3.81904) -- (6.97263,3.82267) -- (6.97338,3.8263) -- (6.97412,3.82994) -- (6.97487,3.83359) -- (6.97562,3.83723) -- (6.97636,3.84088) -- (6.97711,3.84453) -- (6.97785,3.84819) -- (6.9786,3.85185) --
 (6.97935,3.85551) -- (6.98009,3.85917) -- (6.98084,3.86284) -- (6.98158,3.86651) -- (6.98233,3.87019) -- (6.98308,3.87387) -- (6.98382,3.87755) -- (6.98457,3.88123) -- (6.98531,3.88492) -- (6.98606,3.88861) -- (6.9868,3.89231) -- (6.98755,3.896) --
 (6.9883,3.8997) -- (6.98904,3.90341) -- (6.98979,3.90711) -- (6.99053,3.91082) -- (6.99128,3.91454) -- (6.99202,3.91825) -- (6.99277,3.92197) -- (6.99352,3.9257) -- (6.99426,3.92942) -- (6.99501,3.93315) -- (6.99575,3.93689) -- (6.9965,3.94062) --
 (6.99725,3.94436) -- (6.99799,3.9481) -- (6.99874,3.95185) -- (6.99948,3.9556) -- (7.00023,3.95935) -- (7.00098,3.96311) -- (7.00172,3.96686) -- (7.00247,3.97063) -- (7.00321,3.97439) -- (7.00396,3.97816) -- (7.0047,3.98193) -- (7.00545,3.98571) --
 (7.0062,3.98948) -- (7.00694,3.99327) -- (7.00769,3.99705) -- (7.00843,4.00084) -- (7.00918,4.00463) -- (7.00992,4.00842) -- (7.01067,4.01222) -- (7.01142,4.01602) -- (7.01216,4.01983) -- (7.01291,4.02363) -- (7.01365,4.02744) -- (7.0144,4.03126) --
 (7.01515,4.03507) -- (7.01589,4.03889) -- (7.01664,4.04272) -- (7.01738,4.04654) -- (7.01813,4.05037) -- (7.01888,4.05421) -- (7.01962,4.05804) -- (7.02037,4.06188) -- (7.02111,4.06573) -- (7.02186,4.06957) -- (7.0226,4.07342) -- (7.02335,4.07728)
 -- (7.0241,4.08113) -- (7.02484,4.08499) -- (7.02559,4.08885) -- (7.02633,4.09272) -- (7.02708,4.09659) -- (7.02782,4.10046) -- (7.02857,4.10434) -- (7.02932,4.10822) -- (7.03006,4.1121) -- (7.03081,4.11598) -- (7.03155,4.11987) -- (7.0323,4.12376)
 -- (7.03305,4.12766) -- (7.03379,4.13156) -- (7.03454,4.13546) -- (7.03528,4.13936) -- (7.03603,4.14327) -- (7.03678,4.14718) -- (7.03752,4.1511) -- (7.03827,4.15502) -- (7.03901,4.15894) -- (7.03976,4.16286) -- (7.0405,4.16679) -- (7.04125,4.17072)
 -- (7.042,4.17466) -- (7.04274,4.17859) -- (7.04349,4.18253) -- (7.04423,4.18648) -- (7.04498,4.19043) -- (7.04572,4.19438) -- (7.04647,4.19833) -- (7.04722,4.20229) -- (7.04796,4.20625) -- (7.04871,4.21021) -- (7.04945,4.21418) -- (7.0502,4.21815)
 -- (7.05095,4.22212) -- (7.05169,4.2261) -- (7.05244,4.23008) -- (7.05318,4.23406) -- (7.05393,4.23805) -- (7.05468,4.24204) -- (7.05542,4.24603) -- (7.05617,4.25003) -- (7.05691,4.25403) -- (7.05766,4.25803) -- (7.0584,4.26204) -- (7.05915,4.26605)
 -- (7.0599,4.27006) -- (7.06064,4.27408) -- (7.06139,4.2781) -- (7.06213,4.28212) -- (7.06288,4.28615) -- (7.06362,4.29018) -- (7.06437,4.29421) -- (7.06512,4.29825) -- (7.06586,4.30229) -- (7.06661,4.30633) -- (7.06735,4.31038) -- (7.0681,4.31443)
 -- (7.06885,4.31848) -- (7.06959,4.32253) -- (7.07034,4.32659) -- (7.07108,4.33066) -- (7.07183,4.33472) -- (7.07258,4.33879) -- (7.07332,4.34286) -- (7.07407,4.34694) -- (7.07481,4.35102) -- (7.07556,4.3551) -- (7.0763,4.35919) -- (7.07705,4.36327)
 -- (7.0778,4.36737) -- (7.07854,4.37146) -- (7.07929,4.37556) -- (7.08003,4.37966) -- (7.08078,4.38377) -- (7.08152,4.38788) -- (7.08227,4.39199) -- (7.08302,4.3961) -- (7.08376,4.40022) -- (7.08451,4.40435) -- (7.08525,4.40847) -- (7.086,4.4126) --
 (7.08675,4.41673) -- (7.08749,4.42087) -- (7.08824,4.425) -- (7.08898,4.42915) -- (7.08973,4.43329) -- (7.09048,4.43744) -- (7.09122,4.44159) -- (7.09197,4.44575) -- (7.09271,4.44991) -- (7.09346,4.45407) -- (7.0942,4.45823) -- (7.09495,4.4624) --
 (7.0957,4.46657) -- (7.09644,4.47075) -- (7.09719,4.47493) -- (7.09793,4.47911) -- (7.09868,4.48329) -- (7.09942,4.48748) -- (7.10017,4.49167) -- (7.10092,4.49587) -- (7.10166,4.50006) -- (7.10241,4.50427) -- (7.10315,4.50847) -- (7.1039,4.51268) --
 (7.10465,4.51689) -- (7.10539,4.52111) -- (7.10614,4.52532) -- (7.10688,4.52955) -- (7.10763,4.53377) -- (7.10838,4.538) -- (7.10912,4.54223) -- (7.10987,4.54646) -- (7.11061,4.5507) -- (7.11136,4.55494) -- (7.1121,4.55919) -- (7.11285,4.56344) --
 (7.1136,4.56769) -- (7.11434,4.57194) -- (7.11509,4.5762) -- (7.11583,4.58046) -- (7.11658,4.58473) -- (7.11732,4.589) -- (7.11807,4.59327) -- (7.11882,4.59754) -- (7.11956,4.60182) -- (7.12031,4.6061) -- (7.12105,4.61039) -- (7.1218,4.61468) --
 (7.12255,4.61897) -- (7.12329,4.62326) -- (7.12404,4.62756) -- (7.12478,4.63186) -- (7.12553,4.63617) -- (7.12628,4.64047) -- (7.12702,4.64479) -- (7.12777,4.6491) -- (7.12851,4.65342) -- (7.12926,4.65774) -- (7.13,4.66207) -- (7.13075,4.66639) --
 (7.1315,4.67073) -- (7.13224,4.67506) -- (7.13299,4.6794) -- (7.13373,4.68374) -- (7.13448,4.68809) -- (7.13522,4.69243) -- (7.13597,4.69679) -- (7.13672,4.70114) -- (7.13746,4.7055) -- (7.13821,4.70986) -- (7.13895,4.71423) -- (7.1397,4.7186) --
 (7.14045,4.72297) -- (7.14119,4.72734) -- (7.14194,4.73172) -- (7.14268,4.7361) -- (7.14343,4.74049) -- (7.14418,4.74488) -- (7.14492,4.74927) -- (7.14567,4.75366) -- (7.14641,4.75806) -- (7.14716,4.76247) -- (7.1479,4.76687) -- (7.14865,4.77128) --
 (7.1494,4.77569) -- (7.15014,4.78011) -- (7.15089,4.78453) -- (7.15163,4.78895) -- (7.15238,4.79338) -- (7.15312,4.7978) -- (7.15387,4.80224) -- (7.15462,4.80667) -- (7.15536,4.81111) -- (7.15611,4.81555) -- (7.15685,4.82) -- (7.1576,4.82445) --
 (7.15835,4.8289) -- (7.15909,4.83336) -- (7.15984,4.83782) -- (7.16058,4.84228) -- (7.16133,4.84675) -- (7.16208,4.85121) -- (7.16282,4.85569) -- (7.16357,4.86016) -- (7.16431,4.86464) -- (7.16506,4.86913) -- (7.1658,4.87361) -- (7.16655,4.8781) --
 (7.1673,4.8826) -- (7.16804,4.88709) -- (7.16879,4.89159) -- (7.16953,4.8961) -- (7.17028,4.9006) -- (7.17102,4.90511) -- (7.17177,4.90963) -- (7.17252,4.91414) -- (7.17326,4.91866) -- (7.17401,4.92319) -- (7.17475,4.92771) -- (7.1755,4.93224) --
 (7.17625,4.93678) -- (7.17699,4.94131) -- (7.17774,4.94585) -- (7.17848,4.9504) -- (7.17923,4.95495) -- (7.17998,4.9595) -- (7.18072,4.96405) -- (7.18147,4.96861) -- (7.18221,4.97317) -- (7.18296,4.97773) -- (7.1837,4.9823) -- (7.18445,4.98687) --
 (7.1852,4.99145) -- (7.18594,4.99602) -- (7.18669,5.0006) -- (7.18743,5.00519) -- (7.18818,5.00978) -- (7.18892,5.01437) -- (7.18967,5.01896) -- (7.19042,5.02356) -- (7.19116,5.02816) -- (7.19191,5.03277) -- (7.19265,5.03737) -- (7.1934,5.04199) --
 (7.19415,5.0466) -- (7.19489,5.05122) -- (7.19564,5.05584) -- (7.19638,5.06047) -- (7.19713,5.0651) -- (7.19788,5.06973) -- (7.19862,5.07436) -- (7.19937,5.079) -- (7.20011,5.08364) -- (7.20086,5.08829) -- (7.2016,5.09294) -- (7.20235,5.09759) --
 (7.2031,5.10225) -- (7.20384,5.10691) -- (7.20459,5.11157) -- (7.20533,5.11624) -- (7.20608,5.12091) -- (7.20682,5.12558) -- (7.20757,5.13025) -- (7.20832,5.13493) -- (7.20906,5.13962) -- (7.20981,5.1443) -- (7.21055,5.14899) -- (7.2113,5.15369) --
 (7.21205,5.15838) -- (7.21279,5.16308) -- (7.21354,5.16779) -- (7.21428,5.1725) -- (7.21503,5.17721) -- (7.21578,5.18192) -- (7.21652,5.18664) -- (7.21727,5.19136) -- (7.21801,5.19608) -- (7.21876,5.20081) -- (7.2195,5.20554) -- (7.22025,5.21028) --
 (7.221,5.21501) -- (7.22174,5.21975) -- (7.22249,5.2245) -- (7.22323,5.22925) -- (7.22398,5.234) -- (7.22473,5.23875) -- (7.22547,5.24351) -- (7.22622,5.24827) -- (7.22696,5.25304) -- (7.22771,5.25781) -- (7.22845,5.26258) -- (7.2292,5.26735) --
 (7.22995,5.27213) -- (7.23069,5.27692) -- (7.23144,5.2817) -- (7.23218,5.28649) -- (7.23293,5.29128) -- (7.23368,5.29608) -- (7.23442,5.30088) -- (7.23517,5.30568) -- (7.23591,5.31049) -- (7.23666,5.3153) -- (7.2374,5.32011) -- (7.23815,5.32493) --
 (7.2389,5.32975) -- (7.23964,5.33457) -- (7.24039,5.3394) -- (7.24113,5.34423) -- (7.24188,5.34906) -- (7.24263,5.3539) -- (7.24337,5.35874) -- (7.24412,5.36358) -- (7.24486,5.36843) -- (7.24561,5.37328) -- (7.24635,5.37814) -- (7.2471,5.383) --
 (7.24785,5.38786) -- (7.24859,5.39272) -- (7.24934,5.39759) -- (7.25008,5.40246) -- (7.25083,5.40734) -- (7.25157,5.41222) -- (7.25232,5.4171) -- (7.25307,5.42198) -- (7.25381,5.42687) -- (7.25456,5.43176) -- (7.2553,5.43666) -- (7.25605,5.44156) --
 (7.2568,5.44646) -- (7.25754,5.45137) -- (7.25829,5.45628) -- (7.25903,5.46119) -- (7.25978,5.46611) -- (7.26053,5.47103) -- (7.26127,5.47595) -- (7.26202,5.48088) -- (7.26276,5.48581) -- (7.26351,5.49074) -- (7.26425,5.49568) -- (7.265,5.50062) --
 (7.26575,5.50556) -- (7.26649,5.51051) -- (7.26724,5.51546) -- (7.26798,5.52042) -- (7.26873,5.52537) -- (7.26947,5.53033) -- (7.27022,5.5353) -- (7.27097,5.54027) -- (7.27171,5.54524) -- (7.27246,5.55021) -- (7.2732,5.55519) -- (7.27395,5.56017) --
 (7.2747,5.56516) -- (7.27544,5.57015) -- (7.27619,5.57514) -- (7.27693,5.58014) -- (7.27768,5.58514) -- (7.27843,5.59014) -- (7.27917,5.59515) -- (7.27992,5.60016) -- (7.28066,5.60517) -- (7.28141,5.61019) -- (7.28215,5.61521) -- (7.2829,5.62023) --
 (7.28365,5.62526) -- (7.28439,5.63029) -- (7.28514,5.63532) -- (7.28588,5.64036) -- (7.28663,5.6454) -- (7.28737,5.65045) -- (7.28812,5.65549) -- (7.28887,5.66054) -- (7.28961,5.6656) -- (7.29036,5.67066) -- (7.2911,5.67572) -- (7.29185,5.68079) --
 (7.2926,5.68585) -- (7.29334,5.69093) -- (7.29409,5.696) -- (7.29483,5.70108) -- (7.29558,5.70616) -- (7.29633,5.71125) -- (7.29707,5.71634) -- (7.29782,5.72143) -- (7.29856,5.72653) -- (7.29931,5.73163) -- (7.30005,5.73673) -- (7.3008,5.74184) --
 (7.30155,5.74695) -- (7.30229,5.75206) -- (7.30304,5.75718) -- (7.30378,5.7623) -- (7.30453,5.76743) -- (7.30527,5.77256) -- (7.30602,5.77769) -- (7.30677,5.78282) -- (7.30751,5.78796) -- (7.30826,5.7931) -- (7.309,5.79825) -- (7.30975,5.8034) --
 (7.3105,5.80855) -- (7.31124,5.8137) -- (7.31199,5.81886) -- (7.31273,5.82403) -- (7.31348,5.82919) -- (7.31423,5.83436) -- (7.31497,5.83954) -- (7.31572,5.84471) -- (7.31646,5.84989) -- (7.31721,5.85508) -- (7.31795,5.86026) -- (7.3187,5.86546) --
 (7.31945,5.87065) -- (7.32019,5.87585) -- (7.32094,5.88105) -- (7.32168,5.88625) -- (7.32243,5.89146) -- (7.32317,5.89667) -- (7.32392,5.90189) -- (7.32467,5.90711) -- (7.32541,5.91233) -- (7.32616,5.91755) -- (7.3269,5.92278) -- (7.32765,5.92802)
 -- (7.3284,5.93325) -- (7.32914,5.93849) -- (7.32989,5.94374) -- (7.33063,5.94898) -- (7.33138,5.95423) -- (7.33213,5.95949) -- (7.33287,5.96474) -- (7.33362,5.97) -- (7.33436,5.97527) -- (7.33511,5.98054) -- (7.33585,5.98581) -- (7.3366,5.99108) --
 (7.33735,5.99636) -- (7.33809,6.00164) -- (7.33884,6.00693) -- (7.33958,6.01222) -- (7.34033,6.01751) -- (7.34107,6.02281) -- (7.34182,6.0281) -- (7.34257,6.03341) -- (7.34331,6.03871) -- (7.34406,6.04402) -- (7.3448,6.04934) -- (7.34555,6.05465) --
 (7.3463,6.05997) -- (7.34704,6.0653) -- (7.34779,6.07063) -- (7.34853,6.07596) -- (7.34928,6.08129) -- (7.35003,6.08663) -- (7.35077,6.09197) -- (7.35152,6.09732) -- (7.35226,6.10266) -- (7.35301,6.10802) -- (7.35375,6.11337) -- (7.3545,6.11873) --
 (7.35525,6.12409) -- (7.35599,6.12946) -- (7.35674,6.13483) -- (7.35748,6.1402) -- (7.35823,6.14558) -- (7.35897,6.15096) -- (7.35972,6.15634) -- (7.36047,6.16173) -- (7.36121,6.16712) -- (7.36196,6.17252) -- (7.3627,6.17791) -- (7.36345,6.18331) --
 (7.3642,6.18872) -- (7.36494,6.19413) -- (7.36569,6.19954) -- (7.36643,6.20496) -- (7.36718,6.21037) -- (7.36793,6.2158) -- (7.36867,6.22122) -- (7.36942,6.22665) -- (7.37016,6.23209) -- (7.37091,6.23752) -- (7.37165,6.24296) -- (7.3724,6.24841) --
 (7.37315,6.25385) -- (7.37389,6.2593) -- (7.37464,6.26476) -- (7.37538,6.27022) -- (7.37613,6.27568) -- (7.37687,6.28114) -- (7.37762,6.28661) -- (7.37837,6.29208) -- (7.37911,6.29756) -- (7.37986,6.30304) -- (7.3806,6.30852) -- (7.38135,6.31401) --
 (7.3821,6.3195) -- (7.38284,6.32499) -- (7.38359,6.33049) -- (7.38433,6.33599) -- (7.38508,6.34149) -- (7.38583,6.347) -- (7.38657,6.35251) -- (7.38732,6.35802) -- (7.38806,6.36354) -- (7.38881,6.36906) -- (7.38955,6.37459) -- (7.3903,6.38012) --
 (7.39105,6.38565) -- (7.39179,6.39119) -- (7.39254,6.39672) -- (7.39328,6.40227) -- (7.39403,6.40781) -- (7.39477,6.41336) -- (7.39552,6.41892) -- (7.39627,6.42447) -- (7.39701,6.43004) -- (7.39776,6.4356) -- (7.3985,6.44117) -- (7.39925,6.44674) --
 (7.4,6.45231) -- (7.40074,6.45789) -- (7.40149,6.46347) -- (7.40223,6.46906) -- (7.40298,6.47465) -- (7.40373,6.48024) -- (7.40447,6.48584) -- (7.40522,6.49144) -- (7.40596,6.49704) -- (7.40671,6.50265) -- (7.40745,6.50826) -- (7.4082,6.51387) --
 (7.40895,6.51949) -- (7.40969,6.52511) -- (7.41044,6.53073) -- (7.41118,6.53636) -- (7.41193,6.54199) -- (7.41267,6.54763) -- (7.41342,6.55327) -- (7.41417,6.55891) -- (7.41491,6.56456) -- (7.41566,6.57021) -- (7.4164,6.57586) -- (7.41715,6.58152)
 -- (7.4179,6.58718) -- (7.41864,6.59284) -- (7.41939,6.59851) -- (7.42013,6.60418) -- (7.42088,6.60985) -- (7.42163,6.61553) -- (7.42237,6.62121) -- (7.42312,6.6269) -- (7.42386,6.63259) -- (7.42461,6.63828) -- (7.42535,6.64397) -- (7.4261,6.64967)
 -- (7.42685,6.65538) -- (7.42759,6.66108) -- (7.42834,6.66679) -- (7.42908,6.67251) -- (7.42983,6.67822) -- (7.43057,6.68395) -- (7.43132,6.68967) -- (7.43207,6.6954) -- (7.43281,6.70113) -- (7.43356,6.70686) -- (7.4343,6.7126) -- (7.43505,6.71835)
 -- (7.4358,6.72409) -- (7.43654,6.72984) -- (7.43729,6.73559) -- (7.43748,6.73711);
\colorlet{c}{kugray!30};
\draw [c] (1,1.96304) -- (9.95,1.96304);
\end{tikzpicture}

\end{infilsf}
\end{minipage}
\begin{minipage}[b]{.3\textwidth}
\caption{The negative log likelihood ratio times 2, for the likelihood scan over a range of values for $\Lambda^{-4}$. This likelihood scan does not include any uncertainties. The minimum negative likelihood corresponds to $\Lambda^{-4}=0.356$ TeV$^{-4}$, with a confidence interval of $-0.425 \le \Lambda^{-4} \le 0.916$ TeV$^{-4}$. The horizontal line indicates the log likelihood ratio at the 95\% confidence level.}\label{exllr}
\end{minipage}
\end{figure}

This likelihood fit suggest that the most likely value of $\Lambda$ is 1.29 TeV, with a lower limit on the 95\% confidence interval of 1.02 TeV. That the confidence interval for $\Lambda^{-4}$ spans across zero indicates that the Standard Model case lies within the confidence interval.

\begin{edits}
Additional uncertainties can be incorporated into the determination of the log likelihood ratio by introducing nuisance parameters. Say that, in addition to the likelihood associated with the Poisson distribution of observed events $n_\text{data}$, the likelihood of a given expected event count $N_\text{exp}$ also depends on a second distribution $\hat\mu$. For a single bin, we modify the number of expected events by a number $\mu$ drawn from the $\hat\mu$ distribution, and modify the overall likelihood by the likelihood for drawing that $\mu$:
\(p(n_\text{data}|N_\text{exp},\mu)=p(n_\text{data}|N_\text{exp}+\mu)p(\mu)\)
Then, to find the likelihood at a step in $N_\text{exp}$, we must find the value of $\mu$ that minimises the total negative log likelihood. We assign separate $\hat\mu$ distributions for each bin in a histogram. Thus, for a histogram with bins $i$, the negative log likelihood is given by
\(-LL=\sum_i\ln n_i!-n_i\ln(N_i+\mu_i)+(N_i+\mu_i)-\ln p_i(\mu_i),\)
where, for each bin $i$, $\mu_i$ is selected to minimise this expression. This is known as the profile log likelihood. As is common with error distributions, $\hat\mu$ is usually assumed to be a Gaussian distribution, an assumption that we will also make in the following.
\end{edits}

\subsection{Systematic uncertainties}
In the course of developing the MC sets used, we identified a number of systematic uncertainties that should be included at this point.

We will, in the absence of better information, assume that all systematic uncertainties are uncorrelated, and, unless otherwise noted, that individual uncertainties are completely correlated across all bins. These uncertainties will then be included as nuisance parameters in the log likelihood fit, by constructing a simple Gaussian distribution of appropriate width about $n_\text{data}$ for each systematic in each bin, and then combining them. In practice, this means adding together their standard deviations in quadrature.

To summarise, the systematic uncertainties were:
\begin{itemize}
\item Choice of PDF: Table \ref{mrsttab} summarises the uncertainties determined due to our choice of PDF. Since we have identified separate errors for each $\Lambda$ sample, we will apply this effect to the appropriate $M_{\gamma\gamma}$ function before determining the errors on the coefficient functions. Thus, this effect of this systematic uncertainty is included in the error on the coefficient functions illustrated in fig.~\ref{coef}.
\item Reweighting: By taking the root--square--sum of the error on the weight assigned to each event in a bin in $M_{\gamma\gamma}$, we were able to ascertain the systematic uncertainty due to the reweighting procedures above bin--by--bin for each $\Lambda$ sample.
\item Confidence interval on $M_{\gamma\gamma}$ shape functions: Potentially, these could be evaluated separately for each value of $\Lambda$.% At high $\Lambda^{-1}$, the errors on the linear and quadratic coefficients, which, as visible in fig.~\ref{coef}, remain significant at larger values of $M_{\gamma\gamma}$, contribute more to the uncertainty on the derived distribution.
\item Choice of event generator: We have attempted to minimise the effect of generator choice by actively tuning CalcHEP to produce distributions of events similar to pythia. A systematic error of 9.18\% on the overall event count is included to account for the deviation of the MadGraph sample.
\end{itemize}
In addition, we include systematic uncertainties due to the following detector effects:
\begin{itemize}
\item Photon identification: 1.5\%
\item Uncertainty on luminosity: 2.8\%
\end{itemize}

The systematic uncertainties which were determined for each $\Lambda$ distribution individually, namely the confidence interval on the fitted functions, the PDF uncertainty and the uncertainty due to the reweighting procedure, were applied as systematic errors to the functions fitted to the distributions in figure~\ref{simfit}, and converted to systematic errors on the coefficient functions, plotted in fig.~\ref{coef}, through standard error propagation. Potentially, the resulting errors on the $M_{\gamma\gamma}$ distributions constructed from these coefficient functions could be determined for each $\Lambda$ individually, however, this proves too cumbersome to implement. In stead, we take as a conservative estimate the uncertainty associated with a function constructed for $\Lambda^{-4}=3.0$ TeV$^{-4}$, which is the upper limit on values of $\Lambda^{-4}$ searched by the minimiser. When determining errors associated with a distribution of invariant mass constructed from the parametrisation at a given value of $\Lambda$, the uncertainty associated with the $b$ and $c$ coefficient functions will enter into the uncertainty on the constructed distribution proportionally to $\Lambda^{-4}$ and $(\Lambda^{-4})^2$, respectively. Thus, we expect the distribution constructed for large values of $\Lambda$ to have larger uncertainties associated with them, and choose the errors associated with the highest value of $\Lambda^{-4}$ searched as the conservative estimate for this error. This uncertainty is evaluated for each bin in $M_{\gamma\gamma}$. The uncertainties thus arrived at are combined with the overall uncertainties from the choice of event generator and detector effects, and applied as nuisance parameters in the log likelihood fit. Since these uncertainties include effects that are not assumed to be correlated between bins, the uncertainties on each bin are treated as separate nuisance parameters. The histograms being fitted in the next section contain 90 bins, meaning that the fit takes into account 90 nuisance parameters.

%[Depending on computer time]
For the background sample, by far the dominant source of uncertainty is that arising from limited statistics in the $\gamma$jet sample. As such, we will use it as a first approximation of the total uncertainty.

\section{Setting a limit}
Figure~\ref{resllr} shows the negative profile log likelihood values found scanning over values for $\Lambda^{-4}$ corresponding to $\Lambda=\pm760$ GeV.

\begin{figure}[htp]
\begin{minipage}[b]{.69\textwidth}
\begin{infilsf} \tiny
\begin{tikzpicture}[x=.092\textwidth,y=.092\textwidth]
\pgfdeclareplotmark{cross} {
\pgfpathmoveto{\pgfpoint{-0.3\pgfplotmarksize}{\pgfplotmarksize}}
\pgfpathlineto{\pgfpoint{+0.3\pgfplotmarksize}{\pgfplotmarksize}}
\pgfpathlineto{\pgfpoint{+0.3\pgfplotmarksize}{0.3\pgfplotmarksize}}
\pgfpathlineto{\pgfpoint{+1\pgfplotmarksize}{0.3\pgfplotmarksize}}
\pgfpathlineto{\pgfpoint{+1\pgfplotmarksize}{-0.3\pgfplotmarksize}}
\pgfpathlineto{\pgfpoint{+0.3\pgfplotmarksize}{-0.3\pgfplotmarksize}}
\pgfpathlineto{\pgfpoint{+0.3\pgfplotmarksize}{-1.\pgfplotmarksize}}
\pgfpathlineto{\pgfpoint{-0.3\pgfplotmarksize}{-1.\pgfplotmarksize}}
\pgfpathlineto{\pgfpoint{-0.3\pgfplotmarksize}{-0.3\pgfplotmarksize}}
\pgfpathlineto{\pgfpoint{-1.\pgfplotmarksize}{-0.3\pgfplotmarksize}}
\pgfpathlineto{\pgfpoint{-1.\pgfplotmarksize}{0.3\pgfplotmarksize}}
\pgfpathlineto{\pgfpoint{-0.3\pgfplotmarksize}{0.3\pgfplotmarksize}}
\pgfpathclose
\pgfusepathqstroke
}
\pgfdeclareplotmark{cross*} {
\pgfpathmoveto{\pgfpoint{-0.3\pgfplotmarksize}{\pgfplotmarksize}}
\pgfpathlineto{\pgfpoint{+0.3\pgfplotmarksize}{\pgfplotmarksize}}
\pgfpathlineto{\pgfpoint{+0.3\pgfplotmarksize}{0.3\pgfplotmarksize}}
\pgfpathlineto{\pgfpoint{+1\pgfplotmarksize}{0.3\pgfplotmarksize}}
\pgfpathlineto{\pgfpoint{+1\pgfplotmarksize}{-0.3\pgfplotmarksize}}
\pgfpathlineto{\pgfpoint{+0.3\pgfplotmarksize}{-0.3\pgfplotmarksize}}
\pgfpathlineto{\pgfpoint{+0.3\pgfplotmarksize}{-1.\pgfplotmarksize}}
\pgfpathlineto{\pgfpoint{-0.3\pgfplotmarksize}{-1.\pgfplotmarksize}}
\pgfpathlineto{\pgfpoint{-0.3\pgfplotmarksize}{-0.3\pgfplotmarksize}}
\pgfpathlineto{\pgfpoint{-1.\pgfplotmarksize}{-0.3\pgfplotmarksize}}
\pgfpathlineto{\pgfpoint{-1.\pgfplotmarksize}{0.3\pgfplotmarksize}}
\pgfpathlineto{\pgfpoint{-0.3\pgfplotmarksize}{0.3\pgfplotmarksize}}
\pgfpathclose
\pgfusepathqfillstroke
}
\pgfdeclareplotmark{newstar} {
\pgfpathmoveto{\pgfqpoint{0pt}{\pgfplotmarksize}}
\pgfpathlineto{\pgfqpointpolar{44}{0.5\pgfplotmarksize}}
\pgfpathlineto{\pgfqpointpolar{18}{\pgfplotmarksize}}
\pgfpathlineto{\pgfqpointpolar{-20}{0.5\pgfplotmarksize}}
\pgfpathlineto{\pgfqpointpolar{-54}{\pgfplotmarksize}}
\pgfpathlineto{\pgfqpointpolar{-90}{0.5\pgfplotmarksize}}
\pgfpathlineto{\pgfqpointpolar{234}{\pgfplotmarksize}}
\pgfpathlineto{\pgfqpointpolar{198}{0.5\pgfplotmarksize}}
\pgfpathlineto{\pgfqpointpolar{162}{\pgfplotmarksize}}
\pgfpathlineto{\pgfqpointpolar{134}{0.5\pgfplotmarksize}}
\pgfpathclose
\pgfusepathqstroke
}
\pgfdeclareplotmark{newstar*} {
\pgfpathmoveto{\pgfqpoint{0pt}{\pgfplotmarksize}}
\pgfpathlineto{\pgfqpointpolar{44}{0.5\pgfplotmarksize}}
\pgfpathlineto{\pgfqpointpolar{18}{\pgfplotmarksize}}
\pgfpathlineto{\pgfqpointpolar{-20}{0.5\pgfplotmarksize}}
\pgfpathlineto{\pgfqpointpolar{-54}{\pgfplotmarksize}}
\pgfpathlineto{\pgfqpointpolar{-90}{0.5\pgfplotmarksize}}
\pgfpathlineto{\pgfqpointpolar{234}{\pgfplotmarksize}}
\pgfpathlineto{\pgfqpointpolar{198}{0.5\pgfplotmarksize}}
\pgfpathlineto{\pgfqpointpolar{162}{\pgfplotmarksize}}
\pgfpathlineto{\pgfqpointpolar{134}{0.5\pgfplotmarksize}}
\pgfpathclose
\pgfusepathqfillstroke
}
\definecolor{c}{rgb}{1,1,1};
\draw [color=c, fill=c] (0,0) rectangle (10,6.80516);
\draw [color=c, fill=c] (1,0.680516) rectangle (9.95,6.73711);
\definecolor{c}{rgb}{0,0,0};
\draw [c] (1,0.680516) -- (1,6.73711) -- (9.95,6.73711) -- (9.95,0.680516) -- (1,0.680516);
\definecolor{c}{rgb}{1,1,1};
\draw [color=c, fill=c] (1,0.680516) rectangle (9.95,6.73711);
\definecolor{c}{rgb}{0,0,0};
\draw [c] (1,0.680516) -- (1,6.73711) -- (9.95,6.73711) -- (9.95,0.680516) -- (1,0.680516);
\draw [c] (1,0.680516) -- (9.95,0.680516);
\draw [c] (1.74583,0.863234) -- (1.74583,0.680516);
\draw [c] (1.99444,0.771875) -- (1.99444,0.680516);
\draw [c] (2.24306,0.771875) -- (2.24306,0.680516);
\draw [c] (2.49167,0.771875) -- (2.49167,0.680516);
\draw [c] (2.74028,0.771875) -- (2.74028,0.680516);
\draw [c] (2.98889,0.863234) -- (2.98889,0.680516);
\draw [c] (3.2375,0.771875) -- (3.2375,0.680516);
\draw [c] (3.48611,0.771875) -- (3.48611,0.680516);
\draw [c] (3.73472,0.771875) -- (3.73472,0.680516);
\draw [c] (3.98333,0.771875) -- (3.98333,0.680516);
\draw [c] (4.23194,0.863234) -- (4.23194,0.680516);
\draw [c] (4.48056,0.771875) -- (4.48056,0.680516);
\draw [c] (4.72917,0.771875) -- (4.72917,0.680516);
\draw [c] (4.97778,0.771875) -- (4.97778,0.680516);
\draw [c] (5.22639,0.771875) -- (5.22639,0.680516);
\draw [c] (5.475,0.863234) -- (5.475,0.680516);
\draw [c] (5.72361,0.771875) -- (5.72361,0.680516);
\draw [c] (5.97222,0.771875) -- (5.97222,0.680516);
\draw [c] (6.22083,0.771875) -- (6.22083,0.680516);
\draw [c] (6.46944,0.771875) -- (6.46944,0.680516);
\draw [c] (6.71806,0.863234) -- (6.71806,0.680516);
\draw [c] (6.96667,0.771875) -- (6.96667,0.680516);
\draw [c] (7.21528,0.771875) -- (7.21528,0.680516);
\draw [c] (7.46389,0.771875) -- (7.46389,0.680516);
\draw [c] (7.7125,0.771875) -- (7.7125,0.680516);
\draw [c] (7.96111,0.863234) -- (7.96111,0.680516);
\draw [c] (8.20972,0.771875) -- (8.20972,0.680516);
\draw [c] (8.45833,0.771875) -- (8.45833,0.680516);
\draw [c] (8.70694,0.771875) -- (8.70694,0.680516);
\draw [c] (8.95556,0.771875) -- (8.95556,0.680516);
\draw [c] (9.20417,0.863234) -- (9.20417,0.680516);
\draw [c] (1.74583,0.863234) -- (1.74583,0.680516);
\draw [c] (1.49722,0.771875) -- (1.49722,0.680516);
\draw [c] (1.24861,0.771875) -- (1.24861,0.680516);
\draw [c] (1,0.771875) -- (1,0.680516);
\draw [c] (9.20417,0.863234) -- (9.20417,0.680516);
\draw [c] (9.45278,0.771875) -- (9.45278,0.680516);
\draw [c] (9.70139,0.771875) -- (9.70139,0.680516);
\draw [c] (9.95,0.771875) -- (9.95,0.680516);
\draw [anchor=base] (1.74583,0.455946) node[color=c, rotate=0]{-3};
\draw [anchor=base] (2.98889,0.455946) node[color=c, rotate=0]{-2};
\draw [anchor=base] (4.23194,0.455946) node[color=c, rotate=0]{-1};
\draw [anchor=base] (5.475,0.455946) node[color=c, rotate=0]{0};
\draw [anchor=base] (6.71806,0.455946) node[color=c, rotate=0]{1};
\draw [anchor=base] (7.96111,0.455946) node[color=c, rotate=0]{2};
\draw [anchor=base] (9.20417,0.455946) node[color=c, rotate=0]{3};
\draw [c] (1,0.680516) -- (1,6.73711);
\draw [c] (1.267,1.18721) -- (1,1.18721);
\draw [c] (1.1335,1.38841) -- (1,1.38841);
\draw [c] (1.1335,1.58961) -- (1,1.58961);
\draw [c] (1.1335,1.79081) -- (1,1.79081);
\draw [c] (1.1335,1.99202) -- (1,1.99202);
\draw [c] (1.267,2.19322) -- (1,2.19322);
\draw [c] (1.1335,2.39442) -- (1,2.39442);
\draw [c] (1.1335,2.59562) -- (1,2.59562);
\draw [c] (1.1335,2.79682) -- (1,2.79682);
\draw [c] (1.1335,2.99802) -- (1,2.99802);
\draw [c] (1.267,3.19923) -- (1,3.19923);
\draw [c] (1.1335,3.40043) -- (1,3.40043);
\draw [c] (1.1335,3.60163) -- (1,3.60163);
\draw [c] (1.1335,3.80283) -- (1,3.80283);
\draw [c] (1.1335,4.00403) -- (1,4.00403);
\draw [c] (1.267,4.20524) -- (1,4.20524);
\draw [c] (1.1335,4.40644) -- (1,4.40644);
\draw [c] (1.1335,4.60764) -- (1,4.60764);
\draw [c] (1.1335,4.80884) -- (1,4.80884);
\draw [c] (1.1335,5.01004) -- (1,5.01004);
\draw [c] (1.267,5.21125) -- (1,5.21125);
\draw [c] (1.1335,5.41245) -- (1,5.41245);
\draw [c] (1.1335,5.61365) -- (1,5.61365);
\draw [c] (1.1335,5.81485) -- (1,5.81485);
\draw [c] (1.1335,6.01605) -- (1,6.01605);
\draw [c] (1.267,6.21726) -- (1,6.21726);
\draw [c] (1.267,1.18721) -- (1,1.18721);
\draw [c] (1.1335,0.986006) -- (1,0.986006);
\draw [c] (1.1335,0.784804) -- (1,0.784804);
\draw [c] (1.267,6.21726) -- (1,6.21726);
\draw [c] (1.1335,6.41846) -- (1,6.41846);
\draw [c] (1.1335,6.61966) -- (1,6.61966);
\draw [anchor= east] (0.95,1.18721) node[color=c, rotate=0]{0};
\draw [anchor= east] (0.95,2.19322) node[color=c, rotate=0]{5};
\draw [anchor= east] (0.95,3.19923) node[color=c, rotate=0]{10};
\draw [anchor= east] (0.95,4.20524) node[color=c, rotate=0]{15};
\draw [anchor= east] (0.95,5.21125) node[color=c, rotate=0]{20};
\draw [anchor= east] (0.95,6.21726) node[color=c, rotate=0]{25};
\colorlet{c}{natgreen};
\draw [c] (1.74583,3.82835) -- (1.74658,3.82698) -- (1.74733,3.82557) -- (1.74807,3.82461) -- (1.74882,3.82365) -- (1.74956,3.8227) -- (1.75031,3.82175) -- (1.75105,3.82079) -- (1.7518,3.81984) -- (1.75255,3.81889) -- (1.75329,3.81794)
 -- (1.75404,3.81698) -- (1.75478,3.81603) -- (1.75553,3.81508) -- (1.75628,3.81413) -- (1.75702,3.81319) -- (1.75777,3.81224) -- (1.75851,3.81129) -- (1.75926,3.81034) -- (1.76,3.80939) -- (1.76075,3.80845) -- (1.7615,3.8075) -- (1.76224,3.80656) --
 (1.76299,3.80562) -- (1.76373,3.80467) -- (1.76448,3.80373) -- (1.76523,3.80279) -- (1.76597,3.80185) -- (1.76672,3.80091) -- (1.76746,3.79997) -- (1.76821,3.79903) -- (1.76895,3.79809) -- (1.7697,3.79715) -- (1.77045,3.79621) -- (1.77119,3.79528)
 -- (1.77194,3.79434) -- (1.77268,3.79341) -- (1.77343,3.79247) -- (1.77418,3.79154) -- (1.77492,3.79061) -- (1.77567,3.78967) -- (1.77641,3.78874) -- (1.77716,3.7878) -- (1.7779,3.78687) -- (1.77865,3.78594) -- (1.7794,3.78501) -- (1.78014,3.78408)
 -- (1.78089,3.78316) -- (1.78163,3.78223) -- (1.78238,3.7813) -- (1.78313,3.78037) -- (1.78387,3.77945) -- (1.78462,3.77852) -- (1.78536,3.7776) -- (1.78611,3.77667) -- (1.78685,3.77575) -- (1.7876,3.77482) -- (1.78835,3.7739) -- (1.78909,3.77298)
 -- (1.78984,3.77208) -- (1.79058,3.7712) -- (1.79133,3.77032) -- (1.79208,3.76944) -- (1.79282,3.76856) -- (1.79357,3.76769) -- (1.79431,3.7668) -- (1.79506,3.76593) -- (1.7958,3.76505) -- (1.79655,3.76418) -- (1.7973,3.7633) -- (1.79804,3.76243) --
 (1.79879,3.76155) -- (1.79953,3.76068) -- (1.80028,3.7598) -- (1.80103,3.75893) -- (1.80177,3.75806) -- (1.80252,3.75719) -- (1.80326,3.75631) -- (1.80401,3.75545) -- (1.80475,3.75457) -- (1.8055,3.75371) -- (1.80625,3.75284) -- (1.80699,3.75198) --
 (1.80774,3.7511) -- (1.80848,3.75024) -- (1.80923,3.74937) -- (1.80998,3.74851) -- (1.81072,3.74764) -- (1.81147,3.74678) -- (1.81221,3.74591) -- (1.81296,3.74506) -- (1.8137,3.74419) -- (1.81445,3.74333) -- (1.8152,3.74247) -- (1.81594,3.74161) --
 (1.81669,3.74075) -- (1.81743,3.73989) -- (1.81818,3.73903) -- (1.81893,3.73818) -- (1.81967,3.73731) -- (1.82042,3.73646) -- (1.82116,3.7356) -- (1.82191,3.73475) -- (1.82265,3.73389) -- (1.8234,3.73304) -- (1.82415,3.73218) -- (1.82489,3.73134) --
 (1.82564,3.73048) -- (1.82638,3.72963) -- (1.82713,3.72878) -- (1.82788,3.72793) -- (1.82862,3.72708) -- (1.82937,3.72623) -- (1.83011,3.72538) -- (1.83086,3.72453) -- (1.8316,3.72368) -- (1.83235,3.72283) -- (1.8331,3.72199) -- (1.83384,3.72115) --
 (1.83459,3.7203) -- (1.83533,3.71946) -- (1.83608,3.71861) -- (1.83683,3.71777) -- (1.83757,3.71693) -- (1.83832,3.71608) -- (1.83906,3.71524) -- (1.83981,3.7144) -- (1.84055,3.71356) -- (1.8413,3.71272) -- (1.84205,3.71188) -- (1.84279,3.71105) --
 (1.84354,3.71021) -- (1.84428,3.70938) -- (1.84503,3.70853) -- (1.84578,3.7077) -- (1.84652,3.70686) -- (1.84727,3.70603) -- (1.84801,3.7052) -- (1.84876,3.70436) -- (1.8495,3.70353) -- (1.85025,3.7027) -- (1.851,3.70187) -- (1.85174,3.70103) --
 (1.85249,3.7002) -- (1.85323,3.69937) -- (1.85398,3.69854) -- (1.85473,3.69772) -- (1.85547,3.69689) -- (1.85622,3.69606) -- (1.85696,3.69523) -- (1.85771,3.69441) -- (1.85845,3.69358) -- (1.8592,3.69276) -- (1.85995,3.69193) -- (1.86069,3.69111) --
 (1.86144,3.69029) -- (1.86218,3.68947) -- (1.86293,3.68864) -- (1.86367,3.68782) -- (1.86442,3.687) -- (1.86517,3.68618) -- (1.86591,3.68536) -- (1.86666,3.68455) -- (1.8674,3.68373) -- (1.86815,3.68291) -- (1.8689,3.68209) -- (1.86964,3.68128) --
 (1.87039,3.68046) -- (1.87113,3.67965) -- (1.87188,3.67883) -- (1.87262,3.67802) -- (1.87337,3.6772) -- (1.87412,3.67639) -- (1.87486,3.67558) -- (1.87561,3.67477) -- (1.87635,3.67396) -- (1.8771,3.67315) -- (1.87785,3.67234) -- (1.87859,3.67153) --
 (1.87934,3.67072) -- (1.88008,3.66992) -- (1.88083,3.66911) -- (1.88157,3.6683) -- (1.88232,3.66749) -- (1.88307,3.6667) -- (1.88381,3.66588) -- (1.88456,3.66508) -- (1.8853,3.66428) -- (1.88605,3.66348) -- (1.8868,3.66271) -- (1.88754,3.66195) --
 (1.88829,3.66119) -- (1.88903,3.66043) -- (1.88978,3.65967) -- (1.89052,3.65891) -- (1.89127,3.65815) -- (1.89202,3.65739) -- (1.89276,3.65664) -- (1.89351,3.65588) -- (1.89425,3.65513) -- (1.895,3.65437) -- (1.89575,3.65362) -- (1.89649,3.65286) --
 (1.89724,3.6521) -- (1.89798,3.65135) -- (1.89873,3.6506) -- (1.89947,3.64985) -- (1.90022,3.64909) -- (1.90097,3.64834) -- (1.90171,3.64759) -- (1.90246,3.64685) -- (1.9032,3.64609) -- (1.90395,3.64535) -- (1.9047,3.6446) -- (1.90544,3.64385) --
 (1.90619,3.6431) -- (1.90693,3.64236) -- (1.90768,3.64161) -- (1.90842,3.64087) -- (1.90917,3.64012) -- (1.90992,3.63938) -- (1.91066,3.63863) -- (1.91141,3.63789) -- (1.91215,3.63715) -- (1.9129,3.63641) -- (1.91365,3.63567) -- (1.91439,3.63493) --
 (1.91514,3.63419) -- (1.91588,3.63345) -- (1.91663,3.63271) -- (1.91738,3.63197) -- (1.91812,3.63123) -- (1.91887,3.6305) -- (1.91961,3.62976) -- (1.92036,3.62903) -- (1.9211,3.62829) -- (1.92185,3.62756) -- (1.9226,3.62683) -- (1.92334,3.62609) --
 (1.92409,3.62536) -- (1.92483,3.62463) -- (1.92558,3.6239) -- (1.92633,3.62317) -- (1.92707,3.62244) -- (1.92782,3.62171) -- (1.92856,3.62098) -- (1.92931,3.62025) -- (1.93005,3.61953) -- (1.9308,3.6188) -- (1.93155,3.61807) -- (1.93229,3.61734) --
 (1.93304,3.61662) -- (1.93378,3.61589) -- (1.93453,3.61517) -- (1.93528,3.61445) -- (1.93602,3.61373) -- (1.93677,3.613) -- (1.93751,3.61228) -- (1.93826,3.61156) -- (1.939,3.61084) -- (1.93975,3.61012) -- (1.9405,3.6094) -- (1.94124,3.60869) --
 (1.94199,3.60797) -- (1.94273,3.60725) -- (1.94348,3.60653) -- (1.94423,3.60582) -- (1.94497,3.6051) -- (1.94572,3.60439) -- (1.94646,3.60367) -- (1.94721,3.60296) -- (1.94795,3.60225) -- (1.9487,3.60153) -- (1.94945,3.60082) -- (1.95019,3.60011) --
 (1.95094,3.5994) -- (1.95168,3.59869) -- (1.95243,3.59798) -- (1.95318,3.59727) -- (1.95392,3.59657) -- (1.95467,3.59585) -- (1.95541,3.59515) -- (1.95616,3.59444) -- (1.9569,3.59374) -- (1.95765,3.59303) -- (1.9584,3.59233) -- (1.95914,3.59162) --
 (1.95989,3.59092) -- (1.96063,3.59022) -- (1.96138,3.58953) -- (1.96213,3.58887) -- (1.96287,3.5882) -- (1.96362,3.58755) -- (1.96436,3.58688) -- (1.96511,3.58623) -- (1.96585,3.58556) -- (1.9666,3.58491) -- (1.96735,3.58425) -- (1.96809,3.58359) --
 (1.96884,3.58294) -- (1.96958,3.58229) -- (1.97033,3.58162) -- (1.97108,3.58097) -- (1.97182,3.58032) -- (1.97257,3.57966) -- (1.97331,3.57901) -- (1.97406,3.57836) -- (1.9748,3.57771) -- (1.97555,3.57705) -- (1.9763,3.57641) -- (1.97704,3.57575) --
 (1.97779,3.57511) -- (1.97853,3.57446) -- (1.97928,3.57382) -- (1.98003,3.57316) -- (1.98077,3.57252) -- (1.98152,3.57187) -- (1.98226,3.57122) -- (1.98301,3.57058) -- (1.98375,3.56994) -- (1.9845,3.56929) -- (1.98525,3.56865) -- (1.98599,3.568) --
 (1.98674,3.56736) -- (1.98748,3.56672) -- (1.98823,3.56608) -- (1.98898,3.56544) -- (1.98972,3.5648) -- (1.99047,3.56416) -- (1.99121,3.56352) -- (1.99196,3.56288) -- (1.9927,3.56225) -- (1.99345,3.56161) -- (1.9942,3.56097) -- (1.99494,3.56034) --
 (1.99569,3.5597) -- (1.99643,3.55907) -- (1.99718,3.55843) -- (1.99793,3.5578) -- (1.99867,3.55717) -- (1.99942,3.55654) -- (2.00016,3.5559) -- (2.00091,3.55528) -- (2.00165,3.55464) -- (2.0024,3.55402) -- (2.00315,3.55339) -- (2.00389,3.55276) --
 (2.00464,3.55213) -- (2.00538,3.5515) -- (2.00613,3.55087) -- (2.00688,3.55025) -- (2.00762,3.54962) -- (2.00837,3.549) -- (2.00911,3.54838) -- (2.00986,3.54775) -- (2.0106,3.54713) -- (2.01135,3.5465) -- (2.0121,3.54588) -- (2.01284,3.54526) --
 (2.01359,3.54464) -- (2.01433,3.54402) -- (2.01508,3.5434) -- (2.01583,3.54279) -- (2.01657,3.54216) -- (2.01732,3.54155) -- (2.01806,3.54093) -- (2.01881,3.54032) -- (2.01955,3.5397) -- (2.0203,3.53908) -- (2.02105,3.5385) -- (2.02179,3.53793) --
 (2.02254,3.53736) -- (2.02328,3.53679) -- (2.02403,3.53621) -- (2.02478,3.53564) -- (2.02552,3.53507) -- (2.02627,3.5345) -- (2.02701,3.53393) -- (2.02776,3.53336) -- (2.0285,3.53279) -- (2.02925,3.53222) -- (2.03,3.53165) -- (2.03074,3.53109) --
 (2.03149,3.53052) -- (2.03223,3.52996) -- (2.03298,3.52939) -- (2.03373,3.52883) -- (2.03447,3.52826) -- (2.03522,3.5277) -- (2.03596,3.52714) -- (2.03671,3.52657) -- (2.03745,3.52602) -- (2.0382,3.52545) -- (2.03895,3.52489) -- (2.03969,3.52433) --
 (2.04044,3.52377) -- (2.04118,3.52321) -- (2.04193,3.52265) -- (2.04268,3.5221) -- (2.04342,3.52154) -- (2.04417,3.52098) -- (2.04491,3.52043) -- (2.04566,3.51987) -- (2.0464,3.51932) -- (2.04715,3.51876) -- (2.0479,3.51821) -- (2.04864,3.51766) --
 (2.04939,3.5171) -- (2.05013,3.51655) -- (2.05088,3.516) -- (2.05163,3.51545) -- (2.05237,3.5149) -- (2.05312,3.51435) -- (2.05386,3.5138) -- (2.05461,3.51325) -- (2.05535,3.5127) -- (2.0561,3.51216) -- (2.05685,3.51161) -- (2.05759,3.51107) --
 (2.05834,3.51052) -- (2.05908,3.50997) -- (2.05983,3.50943) -- (2.06058,3.50889) -- (2.06132,3.50835) -- (2.06207,3.5078) -- (2.06281,3.50726) -- (2.06356,3.50672) -- (2.0643,3.50618) -- (2.06505,3.50564) -- (2.0658,3.5051) -- (2.06654,3.50456) --
 (2.06729,3.50402) -- (2.06803,3.50348) -- (2.06878,3.50295) -- (2.06953,3.50243) -- (2.07027,3.50194) -- (2.07102,3.50144) -- (2.07176,3.50095) -- (2.07251,3.50045) -- (2.07325,3.49996) -- (2.074,3.49947) -- (2.07475,3.49897) -- (2.07549,3.49848) --
 (2.07624,3.49799) -- (2.07698,3.4975) -- (2.07773,3.49701) -- (2.07847,3.49652) -- (2.07922,3.49604) -- (2.07997,3.49555) -- (2.08071,3.49506) -- (2.08146,3.49457) -- (2.0822,3.49409) -- (2.08295,3.4936) -- (2.0837,3.49311) -- (2.08444,3.49263) --
 (2.08519,3.49215) -- (2.08593,3.49166) -- (2.08668,3.49118) -- (2.08742,3.4907) -- (2.08817,3.49022) -- (2.08892,3.48973) -- (2.08966,3.48925) -- (2.09041,3.48878) -- (2.09115,3.48829) -- (2.0919,3.48782) -- (2.09265,3.48734) -- (2.09339,3.48686) --
 (2.09414,3.48638) -- (2.09488,3.48591) -- (2.09563,3.48543) -- (2.09637,3.48496) -- (2.09712,3.48448) -- (2.09787,3.48401) -- (2.09861,3.48353) -- (2.09936,3.48306) -- (2.1001,3.48259) -- (2.10085,3.48212) -- (2.1016,3.48165) -- (2.10234,3.48117) --
 (2.10309,3.4807) -- (2.10383,3.48023) -- (2.10458,3.47976) -- (2.10532,3.4793) -- (2.10607,3.47883) -- (2.10682,3.47836) -- (2.10756,3.47789) -- (2.10831,3.47743) -- (2.10905,3.47696) -- (2.1098,3.4765) -- (2.11055,3.47603) -- (2.11129,3.47557) --
 (2.11204,3.47511) -- (2.11278,3.47467) -- (2.11353,3.47425) -- (2.11427,3.47383) -- (2.11502,3.47341) -- (2.11577,3.47299) -- (2.11651,3.47257) -- (2.11726,3.47215) -- (2.118,3.47173) -- (2.11875,3.47132) -- (2.1195,3.4709) -- (2.12024,3.47048) --
 (2.12099,3.47007) -- (2.12173,3.46965) -- (2.12248,3.46924) -- (2.12322,3.46882) -- (2.12397,3.46841) -- (2.12472,3.468) -- (2.12546,3.46759) -- (2.12621,3.46718) -- (2.12695,3.46676) -- (2.1277,3.46635) -- (2.12845,3.46594) -- (2.12919,3.46554) --
 (2.12994,3.46513) -- (2.13068,3.46472) -- (2.13143,3.46431) -- (2.13217,3.4639) -- (2.13292,3.4635) -- (2.13367,3.46309) -- (2.13441,3.46268) -- (2.13516,3.46228) -- (2.1359,3.46188) -- (2.13665,3.46147) -- (2.1374,3.46107) -- (2.13814,3.46067) --
 (2.13889,3.46027) -- (2.13963,3.45986) -- (2.14038,3.45946) -- (2.14112,3.45906) -- (2.14187,3.45866) -- (2.14262,3.45826) -- (2.14336,3.45787) -- (2.14411,3.45747) -- (2.14485,3.45707) -- (2.1456,3.45667) -- (2.14635,3.45628) -- (2.14709,3.45588)
 -- (2.14784,3.45549) -- (2.14858,3.45509) -- (2.14933,3.4547) -- (2.15007,3.4543) -- (2.15082,3.45391) -- (2.15157,3.45352) -- (2.15231,3.45313) -- (2.15306,3.45273) -- (2.1538,3.45236) -- (2.15455,3.45201) -- (2.1553,3.45166) -- (2.15604,3.45131)
 -- (2.15679,3.45097) -- (2.15753,3.45062) -- (2.15828,3.45027) -- (2.15902,3.44993) -- (2.15977,3.44958) -- (2.16052,3.44924) -- (2.16126,3.4489) -- (2.16201,3.44855) -- (2.16275,3.44821) -- (2.1635,3.44787) -- (2.16425,3.44753) -- (2.16499,3.44719)
 -- (2.16574,3.44685) -- (2.16648,3.44651) -- (2.16723,3.44617) -- (2.16797,3.44583) -- (2.16872,3.44549) -- (2.16947,3.44515) -- (2.17021,3.44482) -- (2.17096,3.44448) -- (2.1717,3.44414) -- (2.17245,3.44381) -- (2.1732,3.44347) -- (2.17394,3.44314)
 -- (2.17469,3.4428) -- (2.17543,3.44247) -- (2.17618,3.44214) -- (2.17692,3.44181) -- (2.17767,3.44147) -- (2.17842,3.44114) -- (2.17916,3.44081) -- (2.17991,3.44048) -- (2.18065,3.44015) -- (2.1814,3.43982) -- (2.18215,3.4395) -- (2.18289,3.43917)
 -- (2.18364,3.43884) -- (2.18438,3.43851) -- (2.18513,3.43819) -- (2.18587,3.43786) -- (2.18662,3.43754) -- (2.18737,3.43721) -- (2.18811,3.43689) -- (2.18886,3.43657) -- (2.1896,3.43624) -- (2.19035,3.43592) -- (2.1911,3.4356) -- (2.19184,3.43528)
 -- (2.19259,3.43496) -- (2.19333,3.43464) -- (2.19408,3.43432) -- (2.19482,3.434) -- (2.19557,3.43369) -- (2.19632,3.43341) -- (2.19706,3.43314) -- (2.19781,3.43286) -- (2.19855,3.43259) -- (2.1993,3.43232) -- (2.20005,3.43204) -- (2.20079,3.43177)
 -- (2.20154,3.4315) -- (2.20228,3.43123) -- (2.20303,3.43096) -- (2.20377,3.43068) -- (2.20452,3.43042) -- (2.20527,3.43015) -- (2.20601,3.42988) -- (2.20676,3.42961) -- (2.2075,3.42934) -- (2.20825,3.42907) -- (2.209,3.42881) -- (2.20974,3.42854)
 -- (2.21049,3.42828) -- (2.21123,3.42801) -- (2.21198,3.42775) -- (2.21272,3.42748) -- (2.21347,3.42722) -- (2.21422,3.42696) -- (2.21496,3.42669) -- (2.21571,3.42643) -- (2.21645,3.42617) -- (2.2172,3.42591) -- (2.21795,3.42565) --
 (2.21869,3.42539) -- (2.21944,3.42513) -- (2.22018,3.42487) -- (2.22093,3.42461) -- (2.22167,3.42436) -- (2.22242,3.4241) -- (2.22317,3.42384) -- (2.22391,3.42359) -- (2.22466,3.42333) -- (2.2254,3.42308) -- (2.22615,3.42283) -- (2.2269,3.42257) --
 (2.22764,3.42232) -- (2.22839,3.42206) -- (2.22913,3.42181) -- (2.22988,3.42156) -- (2.23062,3.42131) -- (2.23137,3.42106) -- (2.23212,3.42081) -- (2.23286,3.42056) -- (2.23361,3.42031) -- (2.23435,3.42007) -- (2.2351,3.41982) -- (2.23585,3.41957)
 -- (2.23659,3.41933) -- (2.23734,3.41908) -- (2.23808,3.41883) -- (2.23883,3.41859) -- (2.23957,3.41835) -- (2.24032,3.41815) -- (2.24107,3.41795) -- (2.24181,3.41775) -- (2.24256,3.41755) -- (2.2433,3.41735) -- (2.24405,3.41715) -- (2.2448,3.41695)
 -- (2.24554,3.41675) -- (2.24629,3.41656) -- (2.24703,3.41636) -- (2.24778,3.41616) -- (2.24853,3.41597) -- (2.24927,3.41577) -- (2.25002,3.41558) -- (2.25076,3.41538) -- (2.25151,3.41519) -- (2.25225,3.41499) -- (2.253,3.4148) -- (2.25375,3.41461)
 -- (2.25449,3.41442) -- (2.25524,3.41423) -- (2.25598,3.41403) -- (2.25673,3.41384) -- (2.25748,3.41366) -- (2.25822,3.41347) -- (2.25897,3.41328) -- (2.25971,3.41309) -- (2.26046,3.4129) -- (2.2612,3.41272) -- (2.26195,3.41253) -- (2.2627,3.41234)
 -- (2.26344,3.41216) -- (2.26419,3.41197) -- (2.26493,3.41179) -- (2.26568,3.41161) -- (2.26643,3.41142) -- (2.26717,3.41124) -- (2.26792,3.41106) -- (2.26866,3.41088) -- (2.26941,3.4107) -- (2.27015,3.41051) -- (2.2709,3.41033) -- (2.27165,3.41015)
 -- (2.27239,3.40998) -- (2.27314,3.4098) -- (2.27388,3.40962) -- (2.27463,3.40944) -- (2.27538,3.40927) -- (2.27612,3.40909) -- (2.27687,3.40891) -- (2.27761,3.40874) -- (2.27836,3.40856) -- (2.2791,3.40839) -- (2.27985,3.40822) -- (2.2806,3.40804)
 -- (2.28134,3.40787) -- (2.28209,3.4077) -- (2.28283,3.40753) -- (2.28358,3.40736) -- (2.28433,3.40719) -- (2.28507,3.40702) -- (2.28582,3.40685) -- (2.28656,3.40671) -- (2.28731,3.40658) -- (2.28805,3.40646) -- (2.2888,3.40633) -- (2.28955,3.40621)
 -- (2.29029,3.40609) -- (2.29104,3.40596) -- (2.29178,3.40584) -- (2.29253,3.40572) -- (2.29328,3.4056) -- (2.29402,3.40547) -- (2.29477,3.40535) -- (2.29551,3.40523) -- (2.29626,3.40511) -- (2.297,3.40499) -- (2.29775,3.40488) -- (2.2985,3.40476)
 -- (2.29924,3.40464) -- (2.29999,3.40452) -- (2.30073,3.4044) -- (2.30148,3.40429) -- (2.30223,3.40417) -- (2.30297,3.40406) -- (2.30372,3.40394) -- (2.30446,3.40383) -- (2.30521,3.40372) -- (2.30595,3.4036) -- (2.3067,3.40349) -- (2.30745,3.40338)
 -- (2.30819,3.40327) -- (2.30894,3.40316) -- (2.30968,3.40305) -- (2.31043,3.40294) -- (2.31118,3.40283) -- (2.31192,3.40272) -- (2.31267,3.40261) -- (2.31341,3.4025) -- (2.31416,3.40239) -- (2.3149,3.40229) -- (2.31565,3.40218) -- (2.3164,3.40208)
 -- (2.31714,3.40197) -- (2.31789,3.40186) -- (2.31863,3.40176) -- (2.31938,3.40166) -- (2.32013,3.40155) -- (2.32087,3.40145) -- (2.32162,3.40135) -- (2.32236,3.40125) -- (2.32311,3.40115) -- (2.32385,3.40105) -- (2.3246,3.40094) --
 (2.32535,3.40084) -- (2.32609,3.40075) -- (2.32684,3.40065) -- (2.32758,3.40055) -- (2.32833,3.40045) -- (2.32908,3.40035) -- (2.32982,3.40026) -- (2.33057,3.40016) -- (2.33131,3.40007) -- (2.33206,3.39997) -- (2.3328,3.39988) -- (2.33355,3.39979)
 -- (2.3343,3.39969) -- (2.33504,3.3996) -- (2.33579,3.39953) -- (2.33653,3.39948) -- (2.33728,3.39943) -- (2.33803,3.39939) -- (2.33877,3.39934) -- (2.33952,3.39929) -- (2.34026,3.39924) -- (2.34101,3.3992) -- (2.34175,3.39915) -- (2.3425,3.39911)
 -- (2.34325,3.39906) -- (2.34399,3.39902) -- (2.34474,3.39897) -- (2.34548,3.39893) -- (2.34623,3.39889) -- (2.34698,3.39885) -- (2.34772,3.3988) -- (2.34847,3.39876) -- (2.34921,3.39872) -- (2.34996,3.39868) -- (2.3507,3.39864) -- (2.35145,3.3986)
 -- (2.3522,3.39856) -- (2.35294,3.39853) -- (2.35369,3.39849) -- (2.35443,3.39845) -- (2.35518,3.39842) -- (2.35593,3.39838) -- (2.35667,3.39834) -- (2.35742,3.39831) -- (2.35816,3.39827) -- (2.35891,3.39824) -- (2.35965,3.3982) -- (2.3604,3.39817)
 -- (2.36115,3.39814) -- (2.36189,3.39811) -- (2.36264,3.39808) -- (2.36338,3.39804) -- (2.36413,3.39801) -- (2.36488,3.39798) -- (2.36562,3.39796) -- (2.36637,3.39792) -- (2.36711,3.3979) -- (2.36786,3.39787) -- (2.3686,3.39784) -- (2.36935,3.39781)
 -- (2.3701,3.39779) -- (2.37084,3.39776) -- (2.37159,3.39774) -- (2.37233,3.39771) -- (2.37308,3.39769) -- (2.37383,3.39766) -- (2.37457,3.39764) -- (2.37532,3.39762) -- (2.37606,3.39759) -- (2.37681,3.39757) -- (2.37755,3.39755) -- (2.3783,3.39753)
 -- (2.37905,3.39751) -- (2.37979,3.39749) -- (2.38054,3.39747) -- (2.38128,3.39745) -- (2.38203,3.39743) -- (2.38278,3.39742) -- (2.38352,3.3974) -- (2.38427,3.39738) -- (2.38501,3.39737) -- (2.38576,3.39735) -- (2.3865,3.39733) -- (2.38725,3.39733)
 -- (2.388,3.39736) -- (2.38874,3.39739) -- (2.38949,3.39742) -- (2.39023,3.39745) -- (2.39098,3.39748) -- (2.39173,3.39751) -- (2.39247,3.39754) -- (2.39322,3.39757) -- (2.39396,3.39761) -- (2.39471,3.39764) -- (2.39545,3.39767) -- (2.3962,3.39771)
 -- (2.39695,3.39774) -- (2.39769,3.39777) -- (2.39844,3.39781) -- (2.39918,3.39785) -- (2.39993,3.39788) -- (2.40068,3.39792) -- (2.40142,3.39796) -- (2.40217,3.39799) -- (2.40291,3.39803) -- (2.40366,3.39807) -- (2.4044,3.39811) --
 (2.40515,3.39815) -- (2.4059,3.39819) -- (2.40664,3.39823) -- (2.40739,3.39827) -- (2.40813,3.39831) -- (2.40888,3.39835) -- (2.40963,3.39839) -- (2.41037,3.39844) -- (2.41112,3.39848) -- (2.41186,3.39852) -- (2.41261,3.39857) -- (2.41335,3.39861)
 -- (2.4141,3.39866) -- (2.41485,3.3987) -- (2.41559,3.39875) -- (2.41634,3.3988) -- (2.41708,3.39885) -- (2.41783,3.39889) -- (2.41858,3.39894) -- (2.41932,3.39899) -- (2.42007,3.39904) -- (2.42081,3.39909) -- (2.42156,3.39914) -- (2.4223,3.39919)
 -- (2.42305,3.39924) -- (2.4238,3.3993) -- (2.42454,3.39935) -- (2.42529,3.3994) -- (2.42603,3.39945) -- (2.42678,3.39951) -- (2.42753,3.39956) -- (2.42827,3.39962) -- (2.42902,3.39967) -- (2.42976,3.39973) -- (2.43051,3.39979) -- (2.43125,3.39984)
 -- (2.432,3.3999) -- (2.43275,3.39996) -- (2.43349,3.40001) -- (2.43424,3.40007) -- (2.43498,3.40013) -- (2.43573,3.40019) -- (2.43648,3.40025) -- (2.43722,3.40031) -- (2.43797,3.40037) -- (2.43871,3.40043) -- (2.43946,3.4005) -- (2.4402,3.40056) --
 (2.44095,3.40066) -- (2.4417,3.40077) -- (2.44244,3.40088) -- (2.44319,3.40098) -- (2.44393,3.40109) -- (2.44468,3.4012) -- (2.44543,3.40131) -- (2.44617,3.40142) -- (2.44692,3.40152) -- (2.44766,3.40164) -- (2.44841,3.40175) -- (2.44915,3.40186) --
 (2.4499,3.40197) -- (2.45065,3.40208) -- (2.45139,3.40219) -- (2.45214,3.40231) -- (2.45288,3.40242) -- (2.45363,3.40253) -- (2.45438,3.40265) -- (2.45512,3.40276) -- (2.45587,3.40288) -- (2.45661,3.40299) -- (2.45736,3.40311) -- (2.4581,3.40323) --
 (2.45885,3.40334) -- (2.4596,3.40346) -- (2.46034,3.40358) -- (2.46109,3.4037) -- (2.46183,3.40382) -- (2.46258,3.40394) -- (2.46333,3.40406) -- (2.46407,3.40418) -- (2.46482,3.4043) -- (2.46556,3.40442) -- (2.46631,3.40454) -- (2.46705,3.40467) --
 (2.4678,3.40479) -- (2.46855,3.40491) -- (2.46929,3.40504) -- (2.47004,3.40516) -- (2.47078,3.40529) -- (2.47153,3.40541) -- (2.47228,3.40554) -- (2.47302,3.40566) -- (2.47377,3.40579) -- (2.47451,3.40592) -- (2.47526,3.40604) -- (2.476,3.40617) --
 (2.47675,3.4063) -- (2.4775,3.40643) -- (2.47824,3.40656) -- (2.47899,3.40669) -- (2.47973,3.40682) -- (2.48048,3.40695) -- (2.48123,3.40708) -- (2.48197,3.40722) -- (2.48272,3.40735) -- (2.48346,3.40748) -- (2.48421,3.40762) -- (2.48495,3.40775) --
 (2.4857,3.40788) -- (2.48645,3.40802) -- (2.48719,3.40815) -- (2.48794,3.40829) -- (2.48868,3.40843) -- (2.48943,3.40856) -- (2.49018,3.4087) -- (2.49092,3.40884) -- (2.49167,3.40898) -- (2.49241,3.40912) -- (2.49316,3.40926) -- (2.4939,3.4094) --
 (2.49465,3.40955) -- (2.4954,3.40974) -- (2.49614,3.40992) -- (2.49689,3.4101) -- (2.49763,3.41029) -- (2.49838,3.41047) -- (2.49913,3.41066) -- (2.49987,3.41085) -- (2.50062,3.41103) -- (2.50136,3.41122) -- (2.50211,3.41141) -- (2.50285,3.41159) --
 (2.5036,3.41178) -- (2.50435,3.41197) -- (2.50509,3.41216) -- (2.50584,3.41235) -- (2.50658,3.41254) -- (2.50733,3.41273) -- (2.50807,3.41292) -- (2.50882,3.41311) -- (2.50957,3.41331) -- (2.51031,3.4135) -- (2.51106,3.41369) -- (2.5118,3.41388) --
 (2.51255,3.41408) -- (2.5133,3.41427) -- (2.51404,3.41447) -- (2.51479,3.41466) -- (2.51553,3.41486) -- (2.51628,3.41506) -- (2.51702,3.41525) -- (2.51777,3.41545) -- (2.51852,3.41565) -- (2.51926,3.41585) -- (2.52001,3.41604) -- (2.52075,3.41624)
 -- (2.5215,3.41644) -- (2.52225,3.41664) -- (2.52299,3.41684) -- (2.52374,3.41704) -- (2.52448,3.41724) -- (2.52523,3.41745) -- (2.52597,3.41765) -- (2.52672,3.41785) -- (2.52747,3.41805) -- (2.52821,3.41826) -- (2.52896,3.41846) -- (2.5297,3.41867)
 -- (2.53045,3.41887) -- (2.5312,3.41908) -- (2.53194,3.41928) -- (2.53269,3.41949) -- (2.53343,3.4197) -- (2.53418,3.4199) -- (2.53492,3.42011) -- (2.53567,3.42032) -- (2.53642,3.42053) -- (2.53716,3.42074) -- (2.53791,3.42095) -- (2.53865,3.42116)
 -- (2.5394,3.42137) -- (2.54015,3.42158) -- (2.54089,3.42179) -- (2.54164,3.422) -- (2.54238,3.42222) -- (2.54313,3.42243) -- (2.54387,3.42264) -- (2.54462,3.42286) -- (2.54537,3.42307) -- (2.54611,3.42329) -- (2.54686,3.4235) -- (2.5476,3.42372) --
 (2.54835,3.42393) -- (2.5491,3.42418) -- (2.54984,3.42444) -- (2.55059,3.4247) -- (2.55133,3.42496) -- (2.55208,3.42522) -- (2.55282,3.42549) -- (2.55357,3.42575) -- (2.55432,3.42601) -- (2.55506,3.42627) -- (2.55581,3.42653) -- (2.55655,3.4268) --
 (2.5573,3.42706) -- (2.55805,3.42733) -- (2.55879,3.42759) -- (2.55954,3.42786) -- (2.56028,3.42812) -- (2.56103,3.42839) -- (2.56177,3.42866) -- (2.56252,3.42892) -- (2.56327,3.42919) -- (2.56401,3.42946) -- (2.56476,3.42973) -- (2.5655,3.43) --
 (2.56625,3.43027) -- (2.567,3.43054) -- (2.56774,3.43081) -- (2.56849,3.43108) -- (2.56923,3.43135) -- (2.56998,3.43162) -- (2.57072,3.43189) -- (2.57147,3.43216) -- (2.57222,3.43244) -- (2.57296,3.43271) -- (2.57371,3.43298) -- (2.57445,3.43326) --
 (2.5752,3.43353) -- (2.57595,3.43381) -- (2.57669,3.43408) -- (2.57744,3.43436) -- (2.57818,3.43463) -- (2.57893,3.43491) -- (2.57967,3.43519) -- (2.58042,3.43547) -- (2.58117,3.43574) -- (2.58191,3.43602) -- (2.58266,3.4363) -- (2.5834,3.43658) --
 (2.58415,3.43686) -- (2.5849,3.43714) -- (2.58564,3.43742) -- (2.58639,3.43771) -- (2.58713,3.43799) -- (2.58788,3.43827) -- (2.58862,3.43855) -- (2.58937,3.43884) -- (2.59012,3.43912) -- (2.59086,3.4394) -- (2.59161,3.43969) -- (2.59235,3.43997) --
 (2.5931,3.44026) -- (2.59385,3.44055) -- (2.59459,3.44083) -- (2.59534,3.44112) -- (2.59608,3.44141) -- (2.59683,3.44169) -- (2.59757,3.44198) -- (2.59832,3.44227) -- (2.59907,3.44256) -- (2.59981,3.44285) -- (2.60056,3.44314) -- (2.6013,3.44343) --
 (2.60205,3.44372) -- (2.6028,3.44405) -- (2.60354,3.44439) -- (2.60429,3.44472) -- (2.60503,3.44506) -- (2.60578,3.44539) -- (2.60652,3.44573) -- (2.60727,3.44606) -- (2.60802,3.4464) -- (2.60876,3.44674) -- (2.60951,3.44708) -- (2.61025,3.44741) --
 (2.611,3.44775) -- (2.61175,3.44809) -- (2.61249,3.44843) -- (2.61324,3.44877) -- (2.61398,3.44911) -- (2.61473,3.44945) -- (2.61547,3.44979) -- (2.61622,3.45013) -- (2.61697,3.45047) -- (2.61771,3.45081) -- (2.61846,3.45116) -- (2.6192,3.4515) --
 (2.61995,3.45184) -- (2.6207,3.45218) -- (2.62144,3.45253) -- (2.62219,3.45287) -- (2.62293,3.45322) -- (2.62368,3.45356) -- (2.62442,3.45391) -- (2.62517,3.45426) -- (2.62592,3.4546) -- (2.62666,3.45495) -- (2.62741,3.4553) -- (2.62815,3.45565) --
 (2.6289,3.456) -- (2.62965,3.45635) -- (2.63039,3.4567) -- (2.63114,3.45705) -- (2.63188,3.4574) -- (2.63263,3.45775) -- (2.63337,3.4581) -- (2.63412,3.45845) -- (2.63487,3.4588) -- (2.63561,3.45916) -- (2.63636,3.45951) -- (2.6371,3.45986) --
 (2.63785,3.46022) -- (2.6386,3.46057) -- (2.63934,3.46093) -- (2.64009,3.46128) -- (2.64083,3.46164) -- (2.64158,3.46199) -- (2.64232,3.46235) -- (2.64307,3.46271) -- (2.64382,3.46307) -- (2.64456,3.46342) -- (2.64531,3.46378) -- (2.64605,3.46414)
 -- (2.6468,3.4645) -- (2.64755,3.46486) -- (2.64829,3.46522) -- (2.64904,3.46558) -- (2.64978,3.46594) -- (2.65053,3.4663) -- (2.65127,3.46666) -- (2.65202,3.46703) -- (2.65277,3.46739) -- (2.65351,3.46775) -- (2.65426,3.46812) -- (2.655,3.4685) --
 (2.65575,3.46891) -- (2.6565,3.46931) -- (2.65724,3.46972) -- (2.65799,3.47013) -- (2.65873,3.47054) -- (2.65948,3.47094) -- (2.66022,3.47135) -- (2.66097,3.47176) -- (2.66172,3.47217) -- (2.66246,3.47258) -- (2.66321,3.47299) -- (2.66395,3.4734) --
 (2.6647,3.47382) -- (2.66545,3.47423) -- (2.66619,3.47464) -- (2.66694,3.47505) -- (2.66768,3.47547) -- (2.66843,3.47588) -- (2.66917,3.47629) -- (2.66992,3.47671) -- (2.67067,3.47712) -- (2.67141,3.47754) -- (2.67216,3.47795) -- (2.6729,3.47837) --
 (2.67365,3.47879) -- (2.6744,3.4792) -- (2.67514,3.47962) -- (2.67589,3.48004) -- (2.67663,3.48046) -- (2.67738,3.48087) -- (2.67813,3.48129) -- (2.67887,3.48171) -- (2.67962,3.48213) -- (2.68036,3.48255) -- (2.68111,3.48297) -- (2.68185,3.48339) --
 (2.6826,3.48381) -- (2.68335,3.48423) -- (2.68409,3.48466) -- (2.68484,3.48508) -- (2.68558,3.4855) -- (2.68633,3.48592) -- (2.68708,3.48635) -- (2.68782,3.48677) -- (2.68857,3.4872) -- (2.68931,3.48762) -- (2.69006,3.48805) -- (2.6908,3.48847) --
 (2.69155,3.4889) -- (2.6923,3.48933) -- (2.69304,3.48975) -- (2.69379,3.49018) -- (2.69453,3.49061) -- (2.69528,3.49104) -- (2.69603,3.49147) -- (2.69677,3.4919) -- (2.69752,3.49233) -- (2.69826,3.49276) -- (2.69901,3.49319) -- (2.69975,3.49362) --
 (2.7005,3.49405) -- (2.70125,3.49448) -- (2.70199,3.49491) -- (2.70274,3.49535) -- (2.70348,3.49578) -- (2.70423,3.49621) -- (2.70498,3.49665) -- (2.70572,3.49708) -- (2.70647,3.49754) -- (2.70721,3.49802) -- (2.70796,3.49849) -- (2.7087,3.49897) --
 (2.70945,3.49945) -- (2.7102,3.49993) -- (2.71094,3.50041) -- (2.71169,3.50089) -- (2.71243,3.50137) -- (2.71318,3.50185) -- (2.71393,3.50233) -- (2.71467,3.50281) -- (2.71542,3.50329) -- (2.71616,3.50377) -- (2.71691,3.50425) -- (2.71765,3.50473)
 -- (2.7184,3.50522) -- (2.71915,3.5057) -- (2.71989,3.50619) -- (2.72064,3.50667) -- (2.72138,3.50715) -- (2.72213,3.50764) -- (2.72288,3.50812) -- (2.72362,3.50861) -- (2.72437,3.5091) -- (2.72511,3.50958) -- (2.72586,3.51007) -- (2.7266,3.51056)
 -- (2.72735,3.51104) -- (2.7281,3.51153) -- (2.72884,3.51202) -- (2.72959,3.51251) -- (2.73033,3.513) -- (2.73108,3.51349) -- (2.73183,3.51398) -- (2.73257,3.51447) -- (2.73332,3.51496) -- (2.73406,3.51545) -- (2.73481,3.51594) -- (2.73555,3.51642)
 -- (2.7363,3.51692) -- (2.73705,3.51741) -- (2.73779,3.5179) -- (2.73854,3.5184) -- (2.73928,3.51889) -- (2.74003,3.51939) -- (2.74078,3.51988) -- (2.74152,3.52038) -- (2.74227,3.52087) -- (2.74301,3.52137) -- (2.74376,3.52187) -- (2.7445,3.52236)
 -- (2.74525,3.52286) -- (2.746,3.52336) -- (2.74674,3.52386) -- (2.74749,3.52435) -- (2.74823,3.52485) -- (2.74898,3.52535) -- (2.74973,3.52585) -- (2.75047,3.52635) -- (2.75122,3.52685) -- (2.75196,3.52736) -- (2.75271,3.52786) -- (2.75345,3.52836)
 -- (2.7542,3.52886) -- (2.75495,3.52936) -- (2.75569,3.52987) -- (2.75644,3.53037) -- (2.75718,3.53091) -- (2.75793,3.53146) -- (2.75868,3.53201) -- (2.75942,3.53255) -- (2.76017,3.5331) -- (2.76091,3.53365) -- (2.76166,3.53419) -- (2.7624,3.53474)
 -- (2.76315,3.53529) -- (2.7639,3.53584) -- (2.76464,3.53639) -- (2.76539,3.53693) -- (2.76613,3.53748) -- (2.76688,3.53803) -- (2.76763,3.53859) -- (2.76837,3.53914) -- (2.76912,3.53969) -- (2.76986,3.54024) -- (2.77061,3.54079) --
 (2.77135,3.54134) -- (2.7721,3.5419) -- (2.77285,3.54245) -- (2.77359,3.543) -- (2.77434,3.54356) -- (2.77508,3.54411) -- (2.77583,3.54467) -- (2.77658,3.54522) -- (2.77732,3.54577) -- (2.77807,3.54632) -- (2.77881,3.54688) -- (2.77956,3.54744) --
 (2.7803,3.54799) -- (2.78105,3.54855) -- (2.7818,3.54911) -- (2.78254,3.54967) -- (2.78329,3.55023) -- (2.78403,3.55078) -- (2.78478,3.55134) -- (2.78553,3.5519) -- (2.78627,3.55246) -- (2.78702,3.55302) -- (2.78776,3.55358) -- (2.78851,3.55415) --
 (2.78925,3.55471) -- (2.79,3.55527) -- (2.79075,3.55583) -- (2.79149,3.5564) -- (2.79224,3.55696) -- (2.79298,3.55752) -- (2.79373,3.55809) -- (2.79448,3.55865) -- (2.79522,3.55922) -- (2.79597,3.55978) -- (2.79671,3.56034) -- (2.79746,3.56091) --
 (2.7982,3.56148) -- (2.79895,3.56204) -- (2.7997,3.56261) -- (2.80044,3.56318) -- (2.80119,3.56374) -- (2.80193,3.56431) -- (2.80268,3.56488) -- (2.80343,3.56545) -- (2.80417,3.56602) -- (2.80492,3.56659) -- (2.80566,3.56716) -- (2.80641,3.56775) --
 (2.80715,3.56836) -- (2.8079,3.56897) -- (2.80865,3.56959) -- (2.80939,3.5702) -- (2.81014,3.57081) -- (2.81088,3.57143) -- (2.81163,3.57204) -- (2.81238,3.57265) -- (2.81312,3.57327) -- (2.81387,3.57388) -- (2.81461,3.5745) -- (2.81536,3.57512) --
 (2.8161,3.57573) -- (2.81685,3.57635) -- (2.8176,3.57697) -- (2.81834,3.57758) -- (2.81909,3.5782) -- (2.81983,3.57881) -- (2.82058,3.57943) -- (2.82133,3.58005) -- (2.82207,3.58066) -- (2.82282,3.58127) -- (2.82356,3.58189) -- (2.82431,3.58251) --
 (2.82505,3.58312) -- (2.8258,3.58375) -- (2.82655,3.58437) -- (2.82729,3.58498) -- (2.82804,3.5856) -- (2.82878,3.58623) -- (2.82953,3.58685) -- (2.83028,3.58748) -- (2.83102,3.5881) -- (2.83177,3.58873) -- (2.83251,3.58935) -- (2.83326,3.58998) --
 (2.834,3.5906) -- (2.83475,3.59123) -- (2.8355,3.59185) -- (2.83624,3.59248) -- (2.83699,3.59311) -- (2.83773,3.59373) -- (2.83848,3.59436) -- (2.83923,3.59499) -- (2.83997,3.59562) -- (2.84072,3.59625) -- (2.84146,3.59687) -- (2.84221,3.5975) --
 (2.84295,3.59813) -- (2.8437,3.59876) -- (2.84445,3.59939) -- (2.84519,3.60002) -- (2.84594,3.60066) -- (2.84668,3.60129) -- (2.84743,3.60192) -- (2.84818,3.60255) -- (2.84892,3.60318) -- (2.84967,3.60382) -- (2.85041,3.60445) -- (2.85116,3.60508)
 -- (2.8519,3.60572) -- (2.85265,3.60635) -- (2.8534,3.60699) -- (2.85414,3.60762) -- (2.85489,3.60826) -- (2.85563,3.6089) -- (2.85638,3.60958) -- (2.85713,3.61026) -- (2.85787,3.61094) -- (2.85862,3.61161) -- (2.85936,3.61229) -- (2.86011,3.61297)
 -- (2.86085,3.61365) -- (2.8616,3.61433) -- (2.86235,3.61501) -- (2.86309,3.61569) -- (2.86384,3.61637) -- (2.86458,3.61705) -- (2.86533,3.61773) -- (2.86608,3.61841) -- (2.86682,3.6191) -- (2.86757,3.61978) -- (2.86831,3.62046) -- (2.86906,3.62114)
 -- (2.8698,3.62183) -- (2.87055,3.62251) -- (2.8713,3.62319) -- (2.87204,3.62388) -- (2.87279,3.62456) -- (2.87353,3.62525) -- (2.87428,3.62593) -- (2.87503,3.62662) -- (2.87577,3.6273) -- (2.87652,3.62799) -- (2.87726,3.62868) -- (2.87801,3.62936)
 -- (2.87875,3.63005) -- (2.8795,3.63074) -- (2.88025,3.63143) -- (2.88099,3.63212) -- (2.88174,3.63281) -- (2.88248,3.63349) -- (2.88323,3.63418) -- (2.88398,3.63487) -- (2.88472,3.63556) -- (2.88547,3.63625) -- (2.88621,3.63695) --
 (2.88696,3.63764) -- (2.8877,3.63833) -- (2.88845,3.63902) -- (2.8892,3.63971) -- (2.88994,3.6404) -- (2.89069,3.6411) -- (2.89143,3.64179) -- (2.89218,3.64248) -- (2.89293,3.64318) -- (2.89367,3.64387) -- (2.89442,3.64457) -- (2.89516,3.64526) --
 (2.89591,3.64596) -- (2.89665,3.64665) -- (2.8974,3.64735) -- (2.89815,3.64804) -- (2.89889,3.64874) -- (2.89964,3.64944) -- (2.90038,3.65014) -- (2.90113,3.65083) -- (2.90188,3.65153) -- (2.90262,3.65223) -- (2.90337,3.65293) -- (2.90411,3.65363)
 -- (2.90486,3.65433) -- (2.9056,3.65506) -- (2.90635,3.6558) -- (2.9071,3.65654) -- (2.90784,3.65728) -- (2.90859,3.65803) -- (2.90933,3.65877) -- (2.91008,3.65951) -- (2.91083,3.66025) -- (2.91157,3.661) -- (2.91232,3.66174) -- (2.91306,3.66248) --
 (2.91381,3.66323) -- (2.91455,3.66397) -- (2.9153,3.66472) -- (2.91605,3.66546) -- (2.91679,3.66621) -- (2.91754,3.66693) -- (2.91828,3.66768) -- (2.91903,3.66842) -- (2.91978,3.66917) -- (2.92052,3.6699) -- (2.92127,3.67065) -- (2.92201,3.6714) --
 (2.92276,3.67215) -- (2.9235,3.6729) -- (2.92425,3.67365) -- (2.925,3.67439) -- (2.92574,3.67514) -- (2.92649,3.67589) -- (2.92723,3.67664) -- (2.92798,3.67739) -- (2.92873,3.67814) -- (2.92947,3.67889) -- (2.93022,3.67965) -- (2.93096,3.6804) --
 (2.93171,3.68115) -- (2.93245,3.6819) -- (2.9332,3.68265) -- (2.93395,3.6834) -- (2.93469,3.68416) -- (2.93544,3.68491) -- (2.93618,3.68566) -- (2.93693,3.68641) -- (2.93767,3.68717) -- (2.93842,3.68792) -- (2.93917,3.68868) -- (2.93991,3.68943) --
 (2.94066,3.69019) -- (2.9414,3.69094) -- (2.94215,3.6917) -- (2.9429,3.69246) -- (2.94364,3.6932) -- (2.94439,3.69396) -- (2.94513,3.69472) -- (2.94588,3.69548) -- (2.94662,3.69624) -- (2.94737,3.697) -- (2.94812,3.69776) -- (2.94886,3.69851) --
 (2.94961,3.69927) -- (2.95035,3.70003) -- (2.9511,3.70079) -- (2.95185,3.70155) -- (2.95259,3.70231) -- (2.95334,3.70307) -- (2.95408,3.70384) -- (2.95483,3.7046) -- (2.95557,3.70538) -- (2.95632,3.70618) -- (2.95707,3.70699) -- (2.95781,3.70779) --
 (2.95856,3.70859) -- (2.9593,3.70939) -- (2.96005,3.7102) -- (2.9608,3.711) -- (2.96154,3.7118) -- (2.96229,3.71262) -- (2.96303,3.71342) -- (2.96378,3.71422) -- (2.96452,3.71503) -- (2.96527,3.71584) -- (2.96602,3.71664) -- (2.96676,3.71745) --
 (2.96751,3.71825) -- (2.96825,3.71906) -- (2.969,3.71987) -- (2.96975,3.72068) -- (2.97049,3.72148) -- (2.97124,3.72229) -- (2.97198,3.7231) -- (2.97273,3.72391) -- (2.97347,3.72472) -- (2.97422,3.72554) -- (2.97497,3.72634) -- (2.97571,3.72715) --
 (2.97646,3.72797) -- (2.9772,3.72877) -- (2.97795,3.72958) -- (2.9787,3.73039) -- (2.97944,3.73121) -- (2.98019,3.73202) -- (2.98093,3.73282) -- (2.98168,3.73364) -- (2.98242,3.73446) -- (2.98317,3.73527) -- (2.98392,3.73609) -- (2.98466,3.7369) --
 (2.98541,3.73771) -- (2.98615,3.73853) -- (2.9869,3.73935) -- (2.98765,3.74016) -- (2.98839,3.74098) -- (2.98914,3.74179) -- (2.98988,3.74261) -- (2.99063,3.74343) -- (2.99137,3.74424) -- (2.99212,3.74505) -- (2.99287,3.74587) -- (2.99361,3.74669)
 -- (2.99436,3.74751) -- (2.9951,3.74832) -- (2.99585,3.74914) -- (2.9966,3.74996) -- (2.99734,3.75078) -- (2.99809,3.7516) -- (2.99883,3.75243) -- (2.99958,3.75324) -- (3.00032,3.75406) -- (3.00107,3.75489) -- (3.00182,3.7557) -- (3.00256,3.75652)
 -- (3.00331,3.75735) -- (3.00405,3.75816) -- (3.0048,3.75898) -- (3.00555,3.75981) -- (3.00629,3.76061) -- (3.00704,3.76147) -- (3.00778,3.76233) -- (3.00853,3.7632) -- (3.00927,3.76407) -- (3.01002,3.76494) -- (3.01077,3.7658) -- (3.01151,3.76667)
 -- (3.01226,3.76753) -- (3.013,3.7684) -- (3.01375,3.76926) -- (3.0145,3.77012) -- (3.01524,3.77099) -- (3.01599,3.77185) -- (3.01673,3.77272) -- (3.01748,3.77359) -- (3.01822,3.77446) -- (3.01897,3.77532) -- (3.01972,3.77619) -- (3.02046,3.77706)
 -- (3.02121,3.77792) -- (3.02195,3.77879) -- (3.0227,3.77966) -- (3.02345,3.78053) -- (3.02419,3.7814) -- (3.02494,3.78227) -- (3.02568,3.78314) -- (3.02643,3.78401) -- (3.02717,3.78488) -- (3.02792,3.78575) -- (3.02867,3.78662) -- (3.02941,3.78749)
 -- (3.03016,3.78836) -- (3.0309,3.78923) -- (3.03165,3.79011) -- (3.0324,3.79097) -- (3.03314,3.79185) -- (3.03389,3.79272) -- (3.03463,3.7936) -- (3.03538,3.79447) -- (3.03612,3.79534) -- (3.03687,3.79621) -- (3.03762,3.79709) -- (3.03836,3.79797)
 -- (3.03911,3.79883) -- (3.03985,3.79971) -- (3.0406,3.80059) -- (3.04135,3.80147) -- (3.04209,3.80235) -- (3.04284,3.80322) -- (3.04358,3.80408) -- (3.04433,3.80496) -- (3.04507,3.80584) -- (3.04582,3.80672) -- (3.04657,3.8076) -- (3.04731,3.80848)
 -- (3.04806,3.80936) -- (3.0488,3.81023) -- (3.04955,3.81111) -- (3.0503,3.81199) -- (3.05104,3.81287) -- (3.05179,3.81375) -- (3.05253,3.81464) -- (3.05328,3.81551) -- (3.05402,3.81639) -- (3.05477,3.81727) -- (3.05552,3.81815) -- (3.05626,3.81903)
 -- (3.05701,3.81991) -- (3.05775,3.82079) -- (3.0585,3.82167) -- (3.05925,3.82257) -- (3.05999,3.8235) -- (3.06074,3.82442) -- (3.06148,3.82534) -- (3.06223,3.82626) -- (3.06297,3.82719) -- (3.06372,3.82811) -- (3.06447,3.82903) -- (3.06521,3.82996)
 -- (3.06596,3.83088) -- (3.0667,3.8318) -- (3.06745,3.83273) -- (3.0682,3.83366) -- (3.06894,3.83458) -- (3.06969,3.83549) -- (3.07043,3.83642) -- (3.07118,3.83734) -- (3.07192,3.83827) -- (3.07267,3.8392) -- (3.07342,3.84013) -- (3.07416,3.84106)
 -- (3.07491,3.84198) -- (3.07565,3.8429) -- (3.0764,3.84383) -- (3.07715,3.84476) -- (3.07789,3.84569) -- (3.07864,3.84661) -- (3.07938,3.84754) -- (3.08013,3.84847) -- (3.08087,3.84941) -- (3.08162,3.85034) -- (3.08237,3.85127) -- (3.08311,3.85212)
 -- (3.08386,3.85305) -- (3.0846,3.85398) -- (3.08535,3.8549) -- (3.0861,3.85583) -- (3.08684,3.85676) -- (3.08759,3.85769) -- (3.08833,3.85862) -- (3.08908,3.85955) -- (3.08982,3.86048) -- (3.09057,3.86141) -- (3.09132,3.86234) -- (3.09206,3.86328)
 -- (3.09281,3.86421) -- (3.09355,3.86514) -- (3.0943,3.86608) -- (3.09505,3.86701) -- (3.09579,3.86795) -- (3.09654,3.86889) -- (3.09728,3.86982) -- (3.09803,3.87076) -- (3.09877,3.8717) -- (3.09952,3.87264) -- (3.10027,3.87355) -- (3.10101,3.87449)
 -- (3.10176,3.87543) -- (3.1025,3.87638) -- (3.10325,3.87732) -- (3.104,3.87826) -- (3.10474,3.87921) -- (3.10549,3.88015) -- (3.10623,3.8811) -- (3.10698,3.88204) -- (3.10773,3.88299) -- (3.10847,3.88393) -- (3.10922,3.88488) -- (3.10996,3.88583)
 -- (3.11071,3.88678) -- (3.11145,3.88772) -- (3.1122,3.88864) -- (3.11295,3.88958) -- (3.11369,3.89056) -- (3.11444,3.89154) -- (3.11518,3.89253) -- (3.11593,3.89351) -- (3.11668,3.89449) -- (3.11742,3.89546) -- (3.11817,3.89644) --
 (3.11891,3.89741) -- (3.11966,3.8984) -- (3.1204,3.89938) -- (3.12115,3.90037) -- (3.1219,3.90134) -- (3.12264,3.90233) -- (3.12339,3.9033) -- (3.12413,3.90429) -- (3.12488,3.90528) -- (3.12563,3.90624) -- (3.12637,3.90723) -- (3.12712,3.90822) --
 (3.12786,3.90919) -- (3.12861,3.91017) -- (3.12935,3.91115) -- (3.1301,3.91214) -- (3.13085,3.91311) -- (3.13159,3.9141) -- (3.13234,3.91509) -- (3.13308,3.91608) -- (3.13383,3.91706) -- (3.13458,3.91805) -- (3.13532,3.91904) -- (3.13607,3.92003) --
 (3.13681,3.92102) -- (3.13756,3.92201) -- (3.1383,3.92299) -- (3.13905,3.92398) -- (3.1398,3.92497) -- (3.14054,3.92591) -- (3.14129,3.92687) -- (3.14203,3.92786) -- (3.14278,3.92886) -- (3.14353,3.92985) -- (3.14427,3.93084) -- (3.14502,3.93183) --
 (3.14576,3.93283) -- (3.14651,3.93382) -- (3.14725,3.93482) -- (3.148,3.93582) -- (3.14875,3.93681) -- (3.14949,3.93781) -- (3.15024,3.93881) -- (3.15098,3.93981) -- (3.15173,3.94078) -- (3.15248,3.94177) -- (3.15322,3.94276) -- (3.15397,3.94375) --
 (3.15471,3.94475) -- (3.15546,3.94574) -- (3.1562,3.94673) -- (3.15695,3.94773) -- (3.1577,3.94873) -- (3.15844,3.94961) -- (3.15919,3.9506) -- (3.15993,3.95159) -- (3.16068,3.95258) -- (3.16143,3.95357) -- (3.16217,3.95456) -- (3.16292,3.95555) --
 (3.16366,3.95654) -- (3.16441,3.95753) -- (3.16515,3.95853) -- (3.1659,3.95952) -- (3.16665,3.96052) -- (3.16739,3.96151) -- (3.16814,3.96253) -- (3.16888,3.96357) -- (3.16963,3.9646) -- (3.17038,3.96564) -- (3.17112,3.96668) -- (3.17187,3.96771) --
 (3.17261,3.96875) -- (3.17336,3.96979) -- (3.1741,3.97083) -- (3.17485,3.97187) -- (3.1756,3.97291) -- (3.17634,3.97395) -- (3.17709,3.97499) -- (3.17783,3.97602) -- (3.17858,3.97706) -- (3.17933,3.97811) -- (3.18007,3.97915) -- (3.18082,3.9802) --
 (3.18156,3.98124) -- (3.18231,3.98229) -- (3.18305,3.98334) -- (3.1838,3.98438) -- (3.18455,3.98543) -- (3.18529,3.98648) -- (3.18604,3.98753) -- (3.18678,3.98857) -- (3.18753,3.98954) -- (3.18828,3.99059) -- (3.18902,3.99164) -- (3.18977,3.99269)
 -- (3.19051,3.99373) -- (3.19126,3.99476) -- (3.192,3.99581) -- (3.19275,3.99686) -- (3.1935,3.99791) -- (3.19424,3.99897) -- (3.19499,3.99997) -- (3.19573,4.00102) -- (3.19648,4.00206) -- (3.19723,4.00308) -- (3.19797,4.00412) -- (3.19872,4.00517)
 -- (3.19946,4.00622) -- (3.20021,4.00727) -- (3.20095,4.00832) -- (3.2017,4.00937) -- (3.20245,4.01035) -- (3.20319,4.01139) -- (3.20394,4.01244) -- (3.20468,4.01348) -- (3.20543,4.01452) -- (3.20618,4.01557) -- (3.20692,4.01662) --
 (3.20767,4.01766) -- (3.20841,4.01871) -- (3.20916,4.01976) -- (3.2099,4.02081) -- (3.21065,4.02186) -- (3.2114,4.02291) -- (3.21214,4.02396) -- (3.21289,4.02501) -- (3.21363,4.02606) -- (3.21438,4.02711) -- (3.21513,4.02816) -- (3.21587,4.02922) --
 (3.21662,4.03027) -- (3.21736,4.03129) -- (3.21811,4.0323) -- (3.21885,4.03336) -- (3.2196,4.03441) -- (3.22035,4.03546) -- (3.22109,4.03652) -- (3.22184,4.03757) -- (3.22258,4.03863) -- (3.22333,4.03968) -- (3.22408,4.04077) -- (3.22482,4.04186) --
 (3.22557,4.04294) -- (3.22631,4.044) -- (3.22706,4.04509) -- (3.2278,4.04618) -- (3.22855,4.04727) -- (3.2293,4.04836) -- (3.23004,4.04946) -- (3.23079,4.05055) -- (3.23153,4.05164) -- (3.23228,4.05274) -- (3.23303,4.05384) -- (3.23377,4.05491) --
 (3.23452,4.05594) -- (3.23526,4.05703) -- (3.23601,4.05812) -- (3.23675,4.0592) -- (3.2375,4.06029) -- (3.23825,4.06138) -- (3.23899,4.06248) -- (3.23974,4.06358) -- (3.24048,4.06467) -- (3.24123,4.06577) -- (3.24198,4.06687) -- (3.24272,4.06797) --
 (3.24347,4.06907) -- (3.24421,4.07016) -- (3.24496,4.07121) -- (3.2457,4.07231) -- (3.24645,4.0734) -- (3.2472,4.0745) -- (3.24794,4.07556) -- (3.24869,4.07665) -- (3.24943,4.07775) -- (3.25018,4.07884) -- (3.25093,4.07994) -- (3.25167,4.08104) --
 (3.25242,4.08214) -- (3.25316,4.08309) -- (3.25391,4.08417) -- (3.25465,4.08527) -- (3.2554,4.08638) -- (3.25615,4.08748) -- (3.25689,4.08858) -- (3.25764,4.08968) -- (3.25838,4.09078) -- (3.25913,4.09189) -- (3.25988,4.09299) -- (3.26062,4.0941) --
 (3.26137,4.0952) -- (3.26211,4.09631) -- (3.26286,4.09742) -- (3.2636,4.09852) -- (3.26435,4.09962) -- (3.2651,4.10072) -- (3.26584,4.10183) -- (3.26659,4.10293) -- (3.26733,4.10369) -- (3.26808,4.10479) -- (3.26883,4.1059) -- (3.26957,4.107) --
 (3.27032,4.1081) -- (3.27106,4.1092) -- (3.27181,4.11031) -- (3.27255,4.11141) -- (3.2733,4.11252) -- (3.27405,4.11362) -- (3.27479,4.11473) -- (3.27554,4.11584) -- (3.27628,4.11694) -- (3.27703,4.11805) -- (3.27778,4.11916) -- (3.27852,4.12027) --
 (3.27927,4.12138) -- (3.28001,4.1225) -- (3.28076,4.12365) -- (3.2815,4.1248) -- (3.28225,4.12595) -- (3.283,4.1271) -- (3.28374,4.12825) -- (3.28449,4.1294) -- (3.28523,4.13055) -- (3.28598,4.13171) -- (3.28673,4.13286) -- (3.28747,4.13401) --
 (3.28822,4.13517) -- (3.28896,4.1363) -- (3.28971,4.13737) -- (3.29045,4.13851) -- (3.2912,4.13963) -- (3.29195,4.14077) -- (3.29269,4.14191) -- (3.29344,4.14306) -- (3.29418,4.1442) -- (3.29493,4.14535) -- (3.29568,4.14649) -- (3.29642,4.14764) --
 (3.29717,4.14878) -- (3.29791,4.14993) -- (3.29866,4.15096) -- (3.2994,4.1521) -- (3.30015,4.15324) -- (3.3009,4.15438) -- (3.30164,4.15551) -- (3.30239,4.15666) -- (3.30313,4.1578) -- (3.30388,4.15894) -- (3.30463,4.16008) -- (3.30537,4.16122) --
 (3.30612,4.16237) -- (3.30686,4.16351) -- (3.30761,4.16466) -- (3.30835,4.1658) -- (3.3091,4.16695) -- (3.30985,4.1681) -- (3.31059,4.16924) -- (3.31134,4.17039) -- (3.31208,4.17154) -- (3.31283,4.17269) -- (3.31358,4.17384) -- (3.31432,4.17499) --
 (3.31507,4.17614) -- (3.31581,4.17729) -- (3.31656,4.17844) -- (3.3173,4.1796) -- (3.31805,4.18058) -- (3.3188,4.18172) -- (3.31954,4.18287) -- (3.32029,4.18401) -- (3.32103,4.18501) -- (3.32178,4.18616) -- (3.32253,4.1873) -- (3.32327,4.18844) --
 (3.32402,4.18959) -- (3.32476,4.19073) -- (3.32551,4.19188) -- (3.32625,4.19303) -- (3.327,4.19417) -- (3.32775,4.19532) -- (3.32849,4.19647) -- (3.32924,4.19762) -- (3.32998,4.19877) -- (3.33073,4.19992) -- (3.33148,4.20107) -- (3.33222,4.20222) --
 (3.33297,4.20337) -- (3.33371,4.20452) -- (3.33446,4.20568) -- (3.3352,4.20683) -- (3.33595,4.20799) -- (3.3367,4.20914) -- (3.33744,4.21033) -- (3.33819,4.21152) -- (3.33893,4.21271) -- (3.33968,4.21391) -- (3.34043,4.2151) -- (3.34117,4.2163) --
 (3.34192,4.21749) -- (3.34266,4.21869) -- (3.34341,4.21989) -- (3.34415,4.22108) -- (3.3449,4.22228) -- (3.34565,4.22348) -- (3.34639,4.22468) -- (3.34714,4.22588) -- (3.34788,4.22705) -- (3.34863,4.22825) -- (3.34938,4.22945) -- (3.35012,4.23066)
 -- (3.35087,4.23184) -- (3.35161,4.23261) -- (3.35236,4.23381) -- (3.3531,4.235) -- (3.35385,4.2362) -- (3.3546,4.23735) -- (3.35534,4.23854) -- (3.35609,4.23974) -- (3.35683,4.24094) -- (3.35758,4.24214) -- (3.35833,4.24333) -- (3.35907,4.24453) --
 (3.35982,4.24573) -- (3.36056,4.24693) -- (3.36131,4.24813) -- (3.36205,4.24933) -- (3.3628,4.25054) -- (3.36355,4.25174) -- (3.36429,4.25294) -- (3.36504,4.25414) -- (3.36578,4.25535) -- (3.36653,4.25655) -- (3.36727,4.25776) -- (3.36802,4.25897)
 -- (3.36877,4.26017) -- (3.36951,4.26138) -- (3.37026,4.26258) -- (3.371,4.26378) -- (3.37175,4.26493) -- (3.3725,4.26613) -- (3.37324,4.26732) -- (3.37399,4.26852) -- (3.37473,4.26972) -- (3.37548,4.27091) -- (3.37622,4.27211) -- (3.37697,4.27331)
 -- (3.37772,4.27451) -- (3.37846,4.27547) -- (3.37921,4.27666) -- (3.37995,4.27785) -- (3.3807,4.27903) -- (3.38145,4.28022) -- (3.38219,4.28141) -- (3.38294,4.2826) -- (3.38368,4.28379) -- (3.38443,4.28498) -- (3.38517,4.28617) -- (3.38592,4.28736)
 -- (3.38667,4.28855) -- (3.38741,4.28974) -- (3.38816,4.29093) -- (3.3889,4.29213) -- (3.38965,4.29332) -- (3.3904,4.29452) -- (3.39114,4.29571) -- (3.39189,4.29691) -- (3.39263,4.2981) -- (3.39338,4.2993) -- (3.39412,4.3005) -- (3.39487,4.30174) --
 (3.39562,4.30297) -- (3.39636,4.30421) -- (3.39711,4.30544) -- (3.39785,4.30668) -- (3.3986,4.30791) -- (3.39935,4.30915) -- (3.40009,4.31039) -- (3.40084,4.31163) -- (3.40158,4.31287) -- (3.40233,4.31411) -- (3.40307,4.3152) -- (3.40382,4.31642) --
 (3.40457,4.31767) -- (3.40531,4.31887) -- (3.40606,4.32011) -- (3.4068,4.32135) -- (3.40755,4.32259) -- (3.4083,4.32384) -- (3.40904,4.32508) -- (3.40979,4.32629) -- (3.41053,4.32752) -- (3.41128,4.32876) -- (3.41202,4.32981) -- (3.41277,4.33099) --
 (3.41352,4.33222) -- (3.41426,4.33346) -- (3.41501,4.3347) -- (3.41575,4.33594) -- (3.4165,4.33718) -- (3.41725,4.33842) -- (3.41799,4.33966) -- (3.41874,4.3409) -- (3.41948,4.34214) -- (3.42023,4.34338) -- (3.42097,4.34463) -- (3.42172,4.34587) --
 (3.42247,4.34711) -- (3.42321,4.34836) -- (3.42396,4.3496) -- (3.4247,4.35074) -- (3.42545,4.35197) -- (3.4262,4.35309) -- (3.42694,4.35432) -- (3.42769,4.35555) -- (3.42843,4.35678) -- (3.42918,4.35802) -- (3.42992,4.35925) -- (3.43067,4.36048) --
 (3.43142,4.36172) -- (3.43216,4.36295) -- (3.43291,4.36419) -- (3.43365,4.36543) -- (3.4344,4.36666) -- (3.43515,4.3679) -- (3.43589,4.36914) -- (3.43664,4.37038) -- (3.43738,4.37162) -- (3.43813,4.37286) -- (3.43887,4.3741) -- (3.43962,4.37534) --
 (3.44037,4.37658) -- (3.44111,4.37783) -- (3.44186,4.37907) -- (3.4426,4.38031) -- (3.44335,4.38131) -- (3.4441,4.38255) -- (3.44484,4.3838) -- (3.44559,4.38504) -- (3.44633,4.38629) -- (3.44708,4.38754) -- (3.44782,4.38878) -- (3.44857,4.39002) --
 (3.44932,4.39126) -- (3.45006,4.3925) -- (3.45081,4.39374) -- (3.45155,4.39498) -- (3.4523,4.39624) -- (3.45305,4.39751) -- (3.45379,4.39879) -- (3.45454,4.40007) -- (3.45528,4.40135) -- (3.45603,4.40263) -- (3.45677,4.40388) -- (3.45752,4.40514) --
 (3.45827,4.40642) -- (3.45901,4.4077) -- (3.45976,4.40897) -- (3.4605,4.41025) -- (3.46125,4.41153) -- (3.462,4.41281) -- (3.46274,4.41408) -- (3.46349,4.41536) -- (3.46423,4.41659) -- (3.46498,4.41787) -- (3.46572,4.41915) -- (3.46647,4.42043) --
 (3.46722,4.42154) -- (3.46796,4.4228) -- (3.46871,4.42378) -- (3.46945,4.42505) -- (3.4702,4.42632) -- (3.47095,4.42759) -- (3.47169,4.4288) -- (3.47244,4.43007) -- (3.47318,4.43134) -- (3.47393,4.43262) -- (3.47467,4.43389) -- (3.47542,4.43516) --
 (3.47617,4.43644) -- (3.47691,4.43772) -- (3.47766,4.43899) -- (3.4784,4.44027) -- (3.47915,4.44155) -- (3.4799,4.44282) -- (3.48064,4.4441) -- (3.48139,4.44538) -- (3.48213,4.44666) -- (3.48288,4.44794) -- (3.48362,4.44922) -- (3.48437,4.4505) --
 (3.48512,4.45178) -- (3.48586,4.45307) -- (3.48661,4.45435) -- (3.48735,4.45563) -- (3.4881,4.45692) -- (3.48885,4.4582) -- (3.48959,4.45949) -- (3.49034,4.46078) -- (3.49108,4.46206) -- (3.49183,4.46335) -- (3.49257,4.46445) -- (3.49332,4.46572) --
 (3.49407,4.46685) -- (3.49481,4.46813) -- (3.49556,4.4694) -- (3.4963,4.47068) -- (3.49705,4.47196) -- (3.4978,4.47324) -- (3.49854,4.47452) -- (3.49929,4.4758) -- (3.50003,4.47708) -- (3.50078,4.47836) -- (3.50152,4.47964) -- (3.50227,4.48093) --
 (3.50302,4.48221) -- (3.50376,4.48349) -- (3.50451,4.48478) -- (3.50525,4.48606) -- (3.506,4.48735) -- (3.50675,4.48864) -- (3.50749,4.48992) -- (3.50824,4.49121) -- (3.50898,4.4925) -- (3.50973,4.49372) -- (3.51047,4.49487) -- (3.51122,4.49616) --
 (3.51197,4.49696) -- (3.51271,4.49792) -- (3.51346,4.49924) -- (3.5142,4.50057) -- (3.51495,4.50189) -- (3.5157,4.50321) -- (3.51644,4.50454) -- (3.51719,4.50587) -- (3.51793,4.50719) -- (3.51868,4.50852) -- (3.51942,4.50985) -- (3.52017,4.51118) --
 (3.52092,4.5125) -- (3.52166,4.51383) -- (3.52241,4.51516) -- (3.52315,4.51649) -- (3.5239,4.51775) -- (3.52465,4.51908) -- (3.52539,4.52042) -- (3.52614,4.52175) -- (3.52688,4.52308) -- (3.52763,4.52442) -- (3.52837,4.52575) -- (3.52912,4.52709) --
 (3.52987,4.52842) -- (3.53061,4.52976) -- (3.53136,4.53103) -- (3.5321,4.53184) -- (3.53285,4.53315) -- (3.5336,4.53447) -- (3.53434,4.53578) -- (3.53509,4.53706) -- (3.53583,4.53837) -- (3.53658,4.53969) -- (3.53732,4.54101) -- (3.53807,4.54233) --
 (3.53882,4.54364) -- (3.53956,4.54496) -- (3.54031,4.54628) -- (3.54105,4.5476) -- (3.5418,4.54892) -- (3.54255,4.55012) -- (3.54329,4.55142) -- (3.54404,4.55273) -- (3.54478,4.55404) -- (3.54553,4.55535) -- (3.54628,4.55667) -- (3.54702,4.55798) --
 (3.54777,4.5593) -- (3.54851,4.56061) -- (3.54926,4.56193) -- (3.55,4.56325) -- (3.55075,4.56456) -- (3.5515,4.56588) -- (3.55224,4.5672) -- (3.55299,4.56852) -- (3.55373,4.56933) -- (3.55448,4.57065) -- (3.55523,4.57196) -- (3.55597,4.57328) --
 (3.55672,4.57459) -- (3.55746,4.57591) -- (3.55821,4.57722) -- (3.55895,4.57854) -- (3.5597,4.57986) -- (3.56045,4.58118) -- (3.56119,4.5825) -- (3.56194,4.58382) -- (3.56268,4.58514) -- (3.56343,4.58646) -- (3.56418,4.58778) -- (3.56492,4.58861) --
 (3.56567,4.58992) -- (3.56641,4.59084) -- (3.56716,4.59215) -- (3.5679,4.59346) -- (3.56865,4.59478) -- (3.5694,4.59609) -- (3.57014,4.5974) -- (3.57089,4.59871) -- (3.57163,4.60002) -- (3.57238,4.60134) -- (3.57313,4.60268) -- (3.57387,4.60402) --
 (3.57462,4.60537) -- (3.57536,4.60672) -- (3.57611,4.60807) -- (3.57685,4.60942) -- (3.5776,4.61077) -- (3.57835,4.61212) -- (3.57909,4.61348) -- (3.57984,4.61483) -- (3.58058,4.61618) -- (3.58133,4.61753) -- (3.58208,4.61889) -- (3.58282,4.62024)
 -- (3.58357,4.6216) -- (3.58431,4.62296) -- (3.58506,4.62431) -- (3.5858,4.62559) -- (3.58655,4.62692) -- (3.5873,4.62828) -- (3.58804,4.62964) -- (3.58879,4.631) -- (3.58953,4.63236) -- (3.59028,4.63372) -- (3.59103,4.63508) -- (3.59177,4.63644) --
 (3.59252,4.63781) -- (3.59326,4.63917) -- (3.59401,4.64053) -- (3.59475,4.6419) -- (3.5955,4.64326) -- (3.59625,4.64463) -- (3.59699,4.64599) -- (3.59774,4.64703) -- (3.59848,4.64837) -- (3.59923,4.64974) -- (3.59998,4.6511) -- (3.60072,4.65244) --
 (3.60147,4.65379) -- (3.60221,4.65515) -- (3.60296,4.65647) -- (3.6037,4.65781) -- (3.60445,4.65916) -- (3.6052,4.66051) -- (3.60594,4.66185) -- (3.60669,4.6632) -- (3.60743,4.66455) -- (3.60818,4.6659) -- (3.60893,4.66725) -- (3.60967,4.6686) --
 (3.61042,4.66956) -- (3.61116,4.6709) -- (3.61191,4.67161) -- (3.61265,4.67295) -- (3.6134,4.6743) -- (3.61415,4.67564) -- (3.61489,4.67698) -- (3.61564,4.67833) -- (3.61638,4.67967) -- (3.61713,4.68102) -- (3.61788,4.68236) -- (3.61862,4.68371) --
 (3.61937,4.68506) -- (3.62011,4.6864) -- (3.62086,4.68775) -- (3.6216,4.6891) -- (3.62235,4.69045) -- (3.6231,4.6918) -- (3.62384,4.69315) -- (3.62459,4.6945) -- (3.62533,4.69585) -- (3.62608,4.6972) -- (3.62683,4.69856) -- (3.62757,4.69991) --
 (3.62832,4.70126) -- (3.62906,4.70262) -- (3.62981,4.70397) -- (3.63055,4.70533) -- (3.6313,4.70669) -- (3.63205,4.70804) -- (3.63279,4.70889) -- (3.63354,4.71024) -- (3.63428,4.71158) -- (3.63503,4.71293) -- (3.63578,4.71428) -- (3.63652,4.71562)
 -- (3.63727,4.71697) -- (3.63801,4.71834) -- (3.63876,4.71972) -- (3.6395,4.7211) -- (3.64025,4.72248) -- (3.641,4.72387) -- (3.64174,4.72525) -- (3.64249,4.72663) -- (3.64323,4.72802) -- (3.64398,4.7294) -- (3.64473,4.73079) -- (3.64547,4.73217) --
 (3.64622,4.73295) -- (3.64696,4.73429) -- (3.64771,4.73567) -- (3.64845,4.73704) -- (3.6492,4.73842) -- (3.64995,4.7398) -- (3.65069,4.74117) -- (3.65144,4.74244) -- (3.65218,4.74382) -- (3.65293,4.7452) -- (3.65368,4.74658) -- (3.65442,4.74796) --
 (3.65517,4.74934) -- (3.65591,4.75072) -- (3.65666,4.7521) -- (3.6574,4.75348) -- (3.65815,4.75487) -- (3.6589,4.75542) -- (3.65964,4.75679) -- (3.66039,4.75763) -- (3.66113,4.759) -- (3.66188,4.76038) -- (3.66263,4.76175) -- (3.66337,4.76312) --
 (3.66412,4.7645) -- (3.66486,4.76587) -- (3.66561,4.76725) -- (3.66635,4.76862) -- (3.6671,4.77) -- (3.66785,4.77138) -- (3.66859,4.77275) -- (3.66934,4.77413) -- (3.67008,4.77551) -- (3.67083,4.77689) -- (3.67158,4.77827) -- (3.67232,4.77965) --
 (3.67307,4.78103) -- (3.67381,4.78241) -- (3.67456,4.78289) -- (3.6753,4.78421) -- (3.67605,4.78559) -- (3.6768,4.78697) -- (3.67754,4.78835) -- (3.67829,4.78973) -- (3.67903,4.79111) -- (3.67978,4.7925) -- (3.68053,4.79388) -- (3.68127,4.79527) --
 (3.68202,4.79665) -- (3.68276,4.79804) -- (3.68351,4.79942) -- (3.68425,4.80081) -- (3.685,4.8022) -- (3.68575,4.80358) -- (3.68649,4.80497) -- (3.68724,4.80636) -- (3.68798,4.80775) -- (3.68873,4.80914) -- (3.68948,4.81053) -- (3.69022,4.81143) --
 (3.69097,4.81279) -- (3.69171,4.81416) -- (3.69246,4.81554) -- (3.6932,4.81691) -- (3.69395,4.81828) -- (3.6947,4.81965) -- (3.69544,4.82103) -- (3.69619,4.8224) -- (3.69693,4.82378) -- (3.69768,4.82515) -- (3.69843,4.82653) -- (3.69917,4.8279) --
 (3.69992,4.82928) -- (3.70066,4.83066) -- (3.70141,4.83204) -- (3.70215,4.83342) -- (3.7029,4.8348) -- (3.70365,4.83542) -- (3.70439,4.83677) -- (3.70514,4.83814) -- (3.70588,4.83952) -- (3.70663,4.84089) -- (3.70738,4.84226) -- (3.70812,4.84364) --
 (3.70887,4.84479) -- (3.70961,4.84619) -- (3.71036,4.84759) -- (3.7111,4.84899) -- (3.71185,4.85039) -- (3.7126,4.85179) -- (3.71334,4.85319) -- (3.71409,4.85459) -- (3.71483,4.85599) -- (3.71558,4.8574) -- (3.71633,4.8588) -- (3.71707,4.8602) --
 (3.71782,4.86161) -- (3.71856,4.86301) -- (3.71931,4.86442) -- (3.72005,4.86582) -- (3.7208,4.86723) -- (3.72155,4.86722) -- (3.72229,4.86862) -- (3.72304,4.87001) -- (3.72378,4.87123) -- (3.72453,4.87261) -- (3.72528,4.874) -- (3.72602,4.8754) --
 (3.72677,4.87677) -- (3.72751,4.87817) -- (3.72826,4.87957) -- (3.729,4.88097) -- (3.72975,4.88233) -- (3.7305,4.88373) -- (3.73124,4.88514) -- (3.73199,4.88654) -- (3.73273,4.88794) -- (3.73348,4.88934) -- (3.73423,4.89075) -- (3.73497,4.89215) --
 (3.73572,4.89355) -- (3.73646,4.89496) -- (3.73721,4.89636) -- (3.73795,4.89777) -- (3.7387,4.89825) -- (3.73945,4.89964) -- (3.74019,4.90104) -- (3.74094,4.90243) -- (3.74168,4.90382) -- (3.74243,4.90522) -- (3.74318,4.90661) -- (3.74392,4.90801)
 -- (3.74467,4.9094) -- (3.74541,4.9108) -- (3.74616,4.91219) -- (3.7469,4.91359) -- (3.74765,4.91499) -- (3.7484,4.91639) -- (3.74914,4.91778) -- (3.74989,4.91918) -- (3.75063,4.92058) -- (3.75138,4.92198) -- (3.75213,4.92328) -- (3.75287,4.92454)
 -- (3.75362,4.92593) -- (3.75436,4.92732) -- (3.75511,4.92871) -- (3.75585,4.9301) -- (3.7566,4.93149) -- (3.75735,4.93288) -- (3.75809,4.93427) -- (3.75884,4.93567) -- (3.75958,4.93706) -- (3.76033,4.93845) -- (3.76108,4.93985) -- (3.76182,4.94023)
 -- (3.76257,4.9413) -- (3.76331,4.94269) -- (3.76406,4.94299) -- (3.7648,4.94414) -- (3.76555,4.94541) -- (3.7663,4.94674) -- (3.76704,4.94811) -- (3.76779,4.94949) -- (3.76853,4.95088) -- (3.76928,4.95227) -- (3.77003,4.95366) -- (3.77077,4.95504)
 -- (3.77152,4.95643) -- (3.77226,4.95782) -- (3.77301,4.95922) -- (3.77375,4.96061) -- (3.7745,4.962) -- (3.77525,4.96339) -- (3.77599,4.96478) -- (3.77674,4.96618) -- (3.77748,4.96757) -- (3.77823,4.96896) -- (3.77898,4.97036) -- (3.77972,4.97176)
 -- (3.78047,4.97315) -- (3.78121,4.97453) -- (3.78196,4.97593) -- (3.7827,4.97698) -- (3.78345,4.97837) -- (3.7842,4.97977) -- (3.78494,4.98116) -- (3.78569,4.98255) -- (3.78643,4.98396) -- (3.78718,4.98538) -- (3.78793,4.98681) -- (3.78867,4.98818)
 -- (3.78942,4.98946) -- (3.79016,4.99041) -- (3.79091,4.9917) -- (3.79165,4.99289) -- (3.7924,4.9943) -- (3.79315,4.9957) -- (3.79389,4.99711) -- (3.79464,4.99852) -- (3.79538,4.99992) -- (3.79613,5.00068) -- (3.79688,5.00209) -- (3.79762,5.0035) --
 (3.79837,5.00491) -- (3.79911,5.00632) -- (3.79986,5.00773) -- (3.8006,5.00914) -- (3.80135,5.01055) -- (3.8021,5.01196) -- (3.80284,5.01337) -- (3.80359,5.01479) -- (3.80433,5.01596) -- (3.80508,5.01735) -- (3.80582,5.01873) -- (3.80657,5.02013) --
 (3.80732,5.02138) -- (3.80806,5.02279) -- (3.80881,5.0242) -- (3.80955,5.02561) -- (3.8103,5.02703) -- (3.81105,5.02819) -- (3.81179,5.02944) -- (3.81254,5.03058) -- (3.81328,5.03198) -- (3.81403,5.03338) -- (3.81477,5.0344) -- (3.81552,5.03573) --
 (3.81627,5.03713) -- (3.81701,5.03853) -- (3.81776,5.03994) -- (3.8185,5.04132) -- (3.81925,5.04225) -- (3.82,5.04365) -- (3.82074,5.04505) -- (3.82149,5.04645) -- (3.82223,5.04786) -- (3.82298,5.04926) -- (3.82372,5.05066) -- (3.82447,5.05207) --
 (3.82522,5.05347) -- (3.82596,5.05488) -- (3.82671,5.05628) -- (3.82745,5.05769) -- (3.8282,5.05909) -- (3.82895,5.06024) -- (3.82969,5.06104) -- (3.83044,5.06237) -- (3.83118,5.06377) -- (3.83193,5.06516) -- (3.83267,5.06631) -- (3.83342,5.06769)
 -- (3.83417,5.06908) -- (3.83491,5.07048) -- (3.83566,5.07188) -- (3.8364,5.07328) -- (3.83715,5.07468) -- (3.8379,5.07608) -- (3.83864,5.07748) -- (3.83939,5.07816) -- (3.84013,5.07954) -- (3.84088,5.08094) -- (3.84162,5.08234) -- (3.84237,5.08373)
 -- (3.84312,5.08513) -- (3.84386,5.08614) -- (3.84461,5.0875) -- (3.84535,5.08879) -- (3.8461,5.09016) -- (3.84685,5.09152) -- (3.84759,5.09289) -- (3.84834,5.09426) -- (3.84908,5.09564) -- (3.84983,5.09701) -- (3.85057,5.09838) -- (3.85132,5.09975)
 -- (3.85207,5.10115) -- (3.85281,5.10235) -- (3.85356,5.10375) -- (3.8543,5.10514) -- (3.85505,5.10654) -- (3.8558,5.10779) -- (3.85654,5.10839) -- (3.85729,5.10976) -- (3.85803,5.11112) -- (3.85878,5.11248) -- (3.85952,5.11385) -- (3.86027,5.11521)
 -- (3.86102,5.11659) -- (3.86176,5.11799) -- (3.86251,5.11938) -- (3.86325,5.12077) -- (3.864,5.12217) -- (3.86475,5.12357) -- (3.86549,5.12496) -- (3.86624,5.12636) -- (3.86698,5.12742) -- (3.86773,5.12881) -- (3.86847,5.12974) -- (3.86922,5.13095)
 -- (3.86997,5.13233) -- (3.87071,5.13372) -- (3.87146,5.1351) -- (3.8722,5.13649) -- (3.87295,5.13788) -- (3.8737,5.13926) -- (3.87444,5.14065) -- (3.87519,5.14205) -- (3.87593,5.14306) -- (3.87668,5.14447) -- (3.87742,5.14588) -- (3.87817,5.14686)
 -- (3.87892,5.14825) -- (3.87966,5.14966) -- (3.88041,5.15107) -- (3.88115,5.15248) -- (3.8819,5.15389) -- (3.88265,5.1553) -- (3.88339,5.15672) -- (3.88414,5.15776) -- (3.88488,5.159) -- (3.88563,5.16041) -- (3.88637,5.16182) -- (3.88712,5.1632) --
 (3.88787,5.16425) -- (3.88861,5.16553) -- (3.88936,5.16694) -- (3.8901,5.16834) -- (3.89085,5.16973) -- (3.8916,5.17114) -- (3.89234,5.17229) -- (3.89309,5.17369) -- (3.89383,5.17509) -- (3.89458,5.1765) -- (3.89532,5.17775) -- (3.89607,5.17915) --
 (3.89682,5.18016) -- (3.89756,5.18156) -- (3.89831,5.18259) -- (3.89905,5.1837) -- (3.8998,5.18509) -- (3.90055,5.18649) -- (3.90129,5.18789) -- (3.90204,5.18928) -- (3.90278,5.19048) -- (3.90353,5.19188) -- (3.90427,5.19328) -- (3.90502,5.19464) --
 (3.90577,5.19554) -- (3.90651,5.19684) -- (3.90726,5.19824) -- (3.908,5.19964) -- (3.90875,5.20103) -- (3.9095,5.20243) -- (3.91024,5.20383) -- (3.91099,5.20499) -- (3.91173,5.20627) -- (3.91248,5.20766) -- (3.91322,5.20906) -- (3.91397,5.2103) --
 (3.91472,5.21147) -- (3.91546,5.21286) -- (3.91621,5.21425) -- (3.91695,5.21498) -- (3.9177,5.21629) -- (3.91845,5.21766) -- (3.91919,5.21904) -- (3.91994,5.22043) -- (3.92068,5.22182) -- (3.92143,5.22321) -- (3.92217,5.22456) -- (3.92292,5.22596)
 -- (3.92367,5.22735) -- (3.92441,5.22875) -- (3.92516,5.22952) -- (3.9259,5.23083) -- (3.92665,5.2322) -- (3.9274,5.23359) -- (3.92814,5.23496) -- (3.92889,5.23635) -- (3.92963,5.23774) -- (3.93038,5.23914) -- (3.93112,5.24053) -- (3.93187,5.24158)
 -- (3.93262,5.24259) -- (3.93336,5.24397) -- (3.93411,5.24536) -- (3.93485,5.24675) -- (3.9356,5.24796) -- (3.93635,5.24929) -- (3.93709,5.25037) -- (3.93784,5.25164) -- (3.93858,5.25303) -- (3.93933,5.25441) -- (3.94007,5.25579) --
 (3.94082,5.25705) -- (3.94157,5.25802) -- (3.94231,5.2594) -- (3.94306,5.26072) -- (3.9438,5.2621) -- (3.94455,5.26339) -- (3.9453,5.26424) -- (3.94604,5.26559) -- (3.94679,5.26694) -- (3.94753,5.26831) -- (3.94828,5.26969) -- (3.94902,5.27104) --
 (3.94977,5.27241) -- (3.95052,5.27379) -- (3.95126,5.27508) -- (3.95201,5.27642) -- (3.95275,5.27736) -- (3.9535,5.27853) -- (3.95425,5.27988) -- (3.95499,5.28124) -- (3.95574,5.28258) -- (3.95648,5.28396) -- (3.95723,5.28528) -- (3.95797,5.28663)
 -- (3.95872,5.28782) -- (3.95947,5.28914) -- (3.96021,5.29037) -- (3.96096,5.29136) -- (3.9617,5.2927) -- (3.96245,5.29407) -- (3.9632,5.29544) -- (3.96394,5.29681) -- (3.96469,5.29813) -- (3.96543,5.29909) -- (3.96618,5.30031) -- (3.96692,5.30167)
 -- (3.96767,5.30303) -- (3.96842,5.30437) -- (3.96916,5.30574) -- (3.96991,5.30679) -- (3.97065,5.30811) -- (3.9714,5.30945) -- (3.97215,5.31064) -- (3.97289,5.31198) -- (3.97364,5.31335) -- (3.97438,5.31468) -- (3.97513,5.31605) --
 (3.97588,5.31708) -- (3.97662,5.31817) -- (3.97737,5.3195) -- (3.97811,5.3205) -- (3.97886,5.32161) -- (3.9796,5.32294) -- (3.98035,5.32428) -- (3.9811,5.32559) -- (3.98184,5.32693) -- (3.98259,5.32827) -- (3.98333,5.32961) -- (3.98408,5.33095) --
 (3.98483,5.33234) -- (3.98557,5.33373) -- (3.98632,5.33496) -- (3.98706,5.33635) -- (3.98781,5.33766) -- (3.98855,5.33899) -- (3.9893,5.33954) -- (3.99005,5.34081) -- (3.99079,5.34216) -- (3.99154,5.34352) -- (3.99228,5.34488) -- (3.99303,5.3462) --
 (3.99378,5.34759) -- (3.99452,5.34895) -- (3.99527,5.35034) -- (3.99601,5.35172) -- (3.99676,5.35253) -- (3.9975,5.35371) -- (3.99825,5.35504) -- (3.999,5.35637) -- (3.99974,5.35768) -- (4.00049,5.35906) -- (4.00123,5.36041) -- (4.00198,5.36173) --
 (4.00273,5.36284) -- (4.00347,5.36414) -- (4.00422,5.36547) -- (4.00496,5.36679) -- (4.00571,5.36809) -- (4.00645,5.36903) -- (4.0072,5.37021) -- (4.00795,5.37154) -- (4.00869,5.37288) -- (4.00944,5.37423) -- (4.01018,5.37525) -- (4.01093,5.37643)
 -- (4.01167,5.37775) -- (4.01242,5.37907) -- (4.01317,5.38039) -- (4.01391,5.38166) -- (4.01466,5.38298) -- (4.0154,5.38421) -- (4.01615,5.38525) -- (4.0169,5.38655) -- (4.01764,5.38789) -- (4.01839,5.38923) -- (4.01913,5.39053) -- (4.01988,5.39161)
 -- (4.02063,5.39291) -- (4.02137,5.39424) -- (4.02212,5.39555) -- (4.02286,5.39687) -- (4.02361,5.39821) -- (4.02435,5.39947) -- (4.0251,5.40024) -- (4.02585,5.40153) -- (4.02659,5.40284) -- (4.02734,5.40415) -- (4.02808,5.40542) --
 (4.02883,5.40673) -- (4.02957,5.40804) -- (4.03032,5.40937) -- (4.03107,5.41071) -- (4.03181,5.41203) -- (4.03256,5.41311) -- (4.0333,5.41395) -- (4.03405,5.41524) -- (4.0348,5.41654) -- (4.03554,5.41785) -- (4.03629,5.41913) -- (4.03703,5.42046) --
 (4.03778,5.42174) -- (4.03853,5.42242) -- (4.03927,5.4237) -- (4.04002,5.42499) -- (4.04076,5.42629) -- (4.04151,5.42755) -- (4.04225,5.42882) -- (4.043,5.4301) -- (4.04375,5.43141) -- (4.04449,5.43273) -- (4.04524,5.43406) -- (4.04598,5.43533) --
 (4.04673,5.43662) -- (4.04747,5.43744) -- (4.04822,5.43869) -- (4.04897,5.43999) -- (4.04971,5.4413) -- (4.05046,5.44259) -- (4.0512,5.44388) -- (4.05195,5.4451) -- (4.0527,5.44637) -- (4.05344,5.44754) -- (4.05419,5.44878) -- (4.05493,5.45008) --
 (4.05568,5.45133) -- (4.05643,5.45248) -- (4.05717,5.45369) -- (4.05792,5.4548) -- (4.05866,5.45597) -- (4.05941,5.45725) -- (4.06015,5.45833) -- (4.0609,5.45958) -- (4.06165,5.46086) -- (4.06239,5.46215) -- (4.06314,5.46343) -- (4.06388,5.46455) --
 (4.06463,5.46539) -- (4.06537,5.46665) -- (4.06612,5.46793) -- (4.06687,5.46921) -- (4.06761,5.47047) -- (4.06836,5.47175) -- (4.0691,5.47304) -- (4.06985,5.4743) -- (4.0706,5.4756) -- (4.07134,5.47662) -- (4.07209,5.47788) -- (4.07283,5.47914) --
 (4.07358,5.48041) -- (4.07433,5.48167) -- (4.07507,5.48247) -- (4.07582,5.48373) -- (4.07656,5.485) -- (4.07731,5.48628) -- (4.07805,5.48756) -- (4.0788,5.48885) -- (4.07955,5.49005) -- (4.08029,5.49126) -- (4.08104,5.49242) -- (4.08178,5.49367) --
 (4.08253,5.49492) -- (4.08327,5.49617) -- (4.08402,5.49729) -- (4.08477,5.49853) -- (4.08551,5.49978) -- (4.08626,5.50105) -- (4.087,5.50188) -- (4.08775,5.50308) -- (4.0885,5.50436) -- (4.08924,5.50564) -- (4.08999,5.50684) -- (4.09073,5.50809) --
 (4.09148,5.50935) -- (4.09223,5.51023) -- (4.09297,5.51145) -- (4.09372,5.51269) -- (4.09446,5.51395) -- (4.09521,5.51521) -- (4.09595,5.51648) -- (4.0967,5.51771) -- (4.09745,5.51898) -- (4.09819,5.52024) -- (4.09894,5.52132) -- (4.09968,5.52205)
 -- (4.10043,5.5233) -- (4.10117,5.52455) -- (4.10192,5.52581) -- (4.10267,5.52707) -- (4.10341,5.52834) -- (4.10416,5.52961) -- (4.1049,5.53078) -- (4.10565,5.53204) -- (4.1064,5.53303) -- (4.10714,5.53397) -- (4.10789,5.53521) -- (4.10863,5.53647)
 -- (4.10938,5.53773) -- (4.11013,5.53903) -- (4.11087,5.54028) -- (4.11162,5.54144) -- (4.11236,5.54267) -- (4.11311,5.5439) -- (4.11385,5.54433) -- (4.1146,5.54553) -- (4.11535,5.54674) -- (4.11609,5.54794) -- (4.11684,5.54919) -- (4.11758,5.55044)
 -- (4.11833,5.55164) -- (4.11907,5.55289) -- (4.11982,5.55414) -- (4.12057,5.55539) -- (4.12131,5.55662) -- (4.12206,5.55787) -- (4.1228,5.55913) -- (4.12355,5.56028) -- (4.1243,5.56152) -- (4.12504,5.56278) -- (4.12579,5.56397) -- (4.12653,5.56503)
 -- (4.12728,5.56625) -- (4.12803,5.56747) -- (4.12877,5.5687) -- (4.12952,5.56985) -- (4.13026,5.57074) -- (4.13101,5.57178) -- (4.13175,5.57297) -- (4.1325,5.57418) -- (4.13325,5.57538) -- (4.13399,5.57659) -- (4.13474,5.57782) -- (4.13548,5.57905)
 -- (4.13623,5.58029) -- (4.13697,5.58134) -- (4.13772,5.58217) -- (4.13847,5.58338) -- (4.13921,5.58459) -- (4.13996,5.5858) -- (4.1407,5.58698) -- (4.14145,5.58819) -- (4.1422,5.58941) -- (4.14294,5.59064) -- (4.14369,5.59187) -- (4.14443,5.593) --
 (4.14518,5.59368) -- (4.14593,5.59488) -- (4.14667,5.59608) -- (4.14742,5.59729) -- (4.14816,5.59846) -- (4.14891,5.59967) -- (4.14965,5.60089) -- (4.1504,5.6021) -- (4.15115,5.60332) -- (4.15189,5.60454) -- (4.15264,5.60577) -- (4.15338,5.6064) --
 (4.15413,5.60749) -- (4.15487,5.60869) -- (4.15562,5.60989) -- (4.15637,5.61103) -- (4.15711,5.61219) -- (4.15786,5.61339) -- (4.1586,5.61459) -- (4.15935,5.61578) -- (4.1601,5.617) -- (4.16084,5.61821) -- (4.16159,5.61943) -- (4.16233,5.62064) --
 (4.16308,5.6212) -- (4.16383,5.62237) -- (4.16457,5.62354) -- (4.16532,5.62471) -- (4.16606,5.62588) -- (4.16681,5.62707) -- (4.16755,5.62828) -- (4.1683,5.62948) -- (4.16905,5.6307) -- (4.16979,5.63189) -- (4.17054,5.63305) -- (4.17128,5.63394) --
 (4.17203,5.63478) -- (4.17277,5.63594) -- (4.17352,5.63712) -- (4.17427,5.6383) -- (4.17501,5.63947) -- (4.17576,5.64056) -- (4.1765,5.6416) -- (4.17725,5.64277) -- (4.178,5.64395) -- (4.17874,5.64508) -- (4.17949,5.64626) -- (4.18023,5.64714) --
 (4.18098,5.64828) -- (4.18173,5.64925) -- (4.18247,5.6504) -- (4.18322,5.65156) -- (4.18396,5.65273) -- (4.18471,5.65386) -- (4.18545,5.65503) -- (4.1862,5.65619) -- (4.18695,5.65709) -- (4.18769,5.65815) -- (4.18844,5.65931) -- (4.18918,5.66047) --
 (4.18993,5.66163) -- (4.19068,5.66274) -- (4.19142,5.66369) -- (4.19217,5.66459) -- (4.19291,5.66573) -- (4.19366,5.66687) -- (4.1944,5.66803) -- (4.19515,5.66918) -- (4.1959,5.67034) -- (4.19664,5.67148) -- (4.19739,5.67265) -- (4.19813,5.67383) --
 (4.19888,5.67488) -- (4.19963,5.67574) -- (4.20037,5.67666) -- (4.20112,5.67779) -- (4.20186,5.67893) -- (4.20261,5.68007) -- (4.20335,5.68116) -- (4.2041,5.6823) -- (4.20485,5.68345) -- (4.20559,5.6846) -- (4.20634,5.68569) -- (4.20708,5.68684) --
 (4.20783,5.68798) -- (4.20858,5.68914) -- (4.20932,5.69016) -- (4.21007,5.69084) -- (4.21081,5.6918) -- (4.21156,5.69292) -- (4.2123,5.69404) -- (4.21305,5.69518) -- (4.2138,5.69632) -- (4.21454,5.69746) -- (4.21529,5.69859) -- (4.21603,5.69971) --
 (4.21678,5.70087) -- (4.21753,5.70199) -- (4.21827,5.70313) -- (4.21902,5.70386) -- (4.21976,5.70497) -- (4.22051,5.70611) -- (4.22125,5.70719) -- (4.222,5.7083) -- (4.22275,5.70941) -- (4.22349,5.71055) -- (4.22424,5.71168) -- (4.22498,5.71271) --
 (4.22573,5.71329) -- (4.22648,5.71438) -- (4.22722,5.71549) -- (4.22797,5.71657) -- (4.22871,5.71767) -- (4.22946,5.71879) -- (4.2302,5.71993) -- (4.23095,5.72107) -- (4.2317,5.72221) -- (4.23244,5.72335) -- (4.23319,5.72449) -- (4.23393,5.72556) --
 (4.23468,5.72627) -- (4.23542,5.72708) -- (4.23617,5.72817) -- (4.23692,5.72928) -- (4.23766,5.73039) -- (4.23841,5.7315) -- (4.23915,5.73261) -- (4.2399,5.73374) -- (4.24065,5.73487) -- (4.24139,5.736) -- (4.24214,5.73682) -- (4.24288,5.73791) --
 (4.24363,5.739) -- (4.24438,5.74009) -- (4.24512,5.74116) -- (4.24587,5.74205) -- (4.24661,5.74313) -- (4.24736,5.74422) -- (4.2481,5.74525) -- (4.24885,5.74633) -- (4.2496,5.74741) -- (4.25034,5.74845) -- (4.25109,5.74955) -- (4.25183,5.75065) --
 (4.25258,5.75175) -- (4.25332,5.75256) -- (4.25407,5.75316) -- (4.25482,5.75424) -- (4.25556,5.75534) -- (4.25631,5.75643) -- (4.25705,5.75753) -- (4.2578,5.75861) -- (4.25855,5.75944) -- (4.25929,5.75968) -- (4.26004,5.76074) -- (4.26078,5.76184)
 -- (4.26153,5.76295) -- (4.26228,5.76405) -- (4.26302,5.76516) -- (4.26377,5.76619) -- (4.26451,5.76729) -- (4.26526,5.76837) -- (4.266,5.76948) -- (4.26675,5.77056) -- (4.2675,5.77167) -- (4.26824,5.77278) -- (4.26899,5.77387) -- (4.26973,5.77495)
 -- (4.27048,5.77604) -- (4.27122,5.77667) -- (4.27197,5.77772) -- (4.27272,5.77878) -- (4.27346,5.77984) -- (4.27421,5.78085) -- (4.27495,5.78189) -- (4.2757,5.78293) -- (4.27645,5.78397) -- (4.27719,5.78502) -- (4.27794,5.78609) --
 (4.27868,5.78716) -- (4.27943,5.78777) -- (4.28018,5.78883) -- (4.28092,5.78988) -- (4.28167,5.79093) -- (4.28241,5.79197) -- (4.28316,5.79303) -- (4.2839,5.79387) -- (4.28465,5.79479) -- (4.2854,5.79583) -- (4.28614,5.79688) -- (4.28689,5.79788) --
 (4.28763,5.79892) -- (4.28838,5.79973) -- (4.28912,5.80077) -- (4.28987,5.80181) -- (4.29062,5.80284) -- (4.29136,5.80387) -- (4.29211,5.80491) -- (4.29285,5.80594) -- (4.2936,5.80697) -- (4.29435,5.80798) -- (4.29509,5.80819) -- (4.29584,5.80921)
 -- (4.29658,5.81022) -- (4.29733,5.81123) -- (4.29808,5.81225) -- (4.29882,5.81329) -- (4.29957,5.81429) -- (4.30031,5.81534) -- (4.30106,5.81638) -- (4.3018,5.81743) -- (4.30255,5.81846) -- (4.3033,5.81946) -- (4.30404,5.82044) -- (4.30479,5.82142)
 -- (4.30553,5.82242) -- (4.30628,5.82344) -- (4.30702,5.82446) -- (4.30777,5.82547) -- (4.30852,5.82597) -- (4.30926,5.82668) -- (4.31001,5.82767) -- (4.31075,5.82867) -- (4.3115,5.82969) -- (4.31225,5.8307) -- (4.31299,5.83173) -- (4.31374,5.83276)
 -- (4.31448,5.83377) -- (4.31523,5.83476) -- (4.31598,5.83576) -- (4.31672,5.83677) -- (4.31747,5.83779) -- (4.31821,5.83855) -- (4.31896,5.8392) -- (4.3197,5.84018) -- (4.32045,5.84117) -- (4.3212,5.84215) -- (4.32194,5.84317) -- (4.32269,5.84419)
 -- (4.32343,5.84516) -- (4.32418,5.84613) -- (4.32492,5.84713) -- (4.32567,5.84813) -- (4.32642,5.84914) -- (4.32716,5.84986) -- (4.32791,5.85083) -- (4.32865,5.85181) -- (4.3294,5.85279) -- (4.33015,5.85376) -- (4.33089,5.85475) --
 (4.33164,5.85574) -- (4.33238,5.85673) -- (4.33313,5.85772) -- (4.33388,5.85869) -- (4.33462,5.85896) -- (4.33537,5.85993) -- (4.33611,5.8609) -- (4.33686,5.86188) -- (4.3376,5.86286) -- (4.33835,5.86384) -- (4.3391,5.86483) -- (4.33984,5.86581) --
 (4.34059,5.86674) -- (4.34133,5.86772) -- (4.34208,5.8687) -- (4.34282,5.86962) -- (4.34357,5.87058) -- (4.34432,5.87157) -- (4.34506,5.87255) -- (4.34581,5.87303) -- (4.34655,5.87397) -- (4.3473,5.87494) -- (4.34805,5.8759) -- (4.34879,5.87688) --
 (4.34954,5.87787) -- (4.35028,5.87886) -- (4.35103,5.87952) -- (4.35178,5.88045) -- (4.35252,5.88138) -- (4.35327,5.88233) -- (4.35401,5.88328) -- (4.35476,5.88425) -- (4.3555,5.88522) -- (4.35625,5.88618) -- (4.357,5.88715) -- (4.35774,5.88811) --
 (4.35849,5.88903) -- (4.35923,5.88997) -- (4.35998,5.89041) -- (4.36072,5.89098) -- (4.36147,5.89189) -- (4.36222,5.89282) -- (4.36296,5.89375) -- (4.36371,5.89467) -- (4.36445,5.89561) -- (4.3652,5.89656) -- (4.36595,5.89752) -- (4.36669,5.89808)
 -- (4.36744,5.89901) -- (4.36818,5.89994) -- (4.36893,5.90086) -- (4.36968,5.90178) -- (4.37042,5.90272) -- (4.37117,5.90366) -- (4.37191,5.90455) -- (4.37266,5.90549) -- (4.3734,5.90643) -- (4.37415,5.90737) -- (4.3749,5.90821) -- (4.37564,5.90893)
 -- (4.37639,5.90985) -- (4.37713,5.91078) -- (4.37788,5.91167) -- (4.37862,5.91259) -- (4.37937,5.91352) -- (4.38012,5.91445) -- (4.38086,5.91536) -- (4.38161,5.91596) -- (4.38235,5.91687) -- (4.3831,5.91778) -- (4.38385,5.91869) --
 (4.38459,5.91957) -- (4.38534,5.92023) -- (4.38608,5.92092) -- (4.38683,5.92182) -- (4.38758,5.92272) -- (4.38832,5.9236) -- (4.38907,5.9245) -- (4.38981,5.9254) -- (4.39056,5.92632) -- (4.3913,5.9272) -- (4.39205,5.9281) -- (4.3928,5.929) --
 (4.39354,5.92992) -- (4.39429,5.93045) -- (4.39503,5.93134) -- (4.39578,5.93222) -- (4.39652,5.93278) -- (4.39727,5.9335) -- (4.39802,5.93436) -- (4.39876,5.93524) -- (4.39951,5.93612) -- (4.40025,5.93701) -- (4.401,5.93787) -- (4.40175,5.93876) --
 (4.40249,5.93965) -- (4.40324,5.94055) -- (4.40398,5.94145) -- (4.40473,5.94231) -- (4.40548,5.94321) -- (4.40622,5.94388) -- (4.40697,5.94462) -- (4.40771,5.94522) -- (4.40846,5.94609) -- (4.4092,5.94695) -- (4.40995,5.94782) -- (4.4107,5.94867) --
 (4.41144,5.94954) -- (4.41219,5.9504) -- (4.41293,5.95126) -- (4.41368,5.95212) -- (4.41442,5.95299) -- (4.41517,5.95386) -- (4.41592,5.95471) -- (4.41666,5.95503) -- (4.41741,5.95588) -- (4.41815,5.95674) -- (4.4189,5.95759) -- (4.41965,5.95845) --
 (4.42039,5.95929) -- (4.42114,5.96015) -- (4.42188,5.961) -- (4.42263,5.96186) -- (4.42338,5.96269) -- (4.42412,5.96355) -- (4.42487,5.96441) -- (4.42561,5.96521) -- (4.42636,5.96608) -- (4.4271,5.96695) -- (4.42785,5.96779) -- (4.4286,5.96863) --
 (4.42934,5.96892) -- (4.43009,5.96975) -- (4.43083,5.97059) -- (4.43158,5.97136) -- (4.43232,5.97221) -- (4.43307,5.97305) -- (4.43382,5.9739) -- (4.43456,5.97471) -- (4.43531,5.97556) -- (4.43605,5.9764) -- (4.4368,5.97724) -- (4.43755,5.97809) --
 (4.43829,5.97891) -- (4.43904,5.97975) -- (4.43978,5.98061) -- (4.44053,5.98137) -- (4.44128,5.98188) -- (4.44202,5.98271) -- (4.44277,5.98354) -- (4.44351,5.9843) -- (4.44426,5.98513) -- (4.445,5.98596) -- (4.44575,5.98679) -- (4.4465,5.98758) --
 (4.44724,5.9883) -- (4.44799,5.98865) -- (4.44873,5.98946) -- (4.44948,5.99028) -- (4.45022,5.9911) -- (4.45097,5.99192) -- (4.45172,5.99269) -- (4.45246,5.99353) -- (4.45321,5.99436) -- (4.45395,5.99518) -- (4.4547,5.99563) -- (4.45545,5.99644) --
 (4.45619,5.99724) -- (4.45694,5.99805) -- (4.45768,5.99886) -- (4.45843,5.99967) -- (4.45918,6.00047) -- (4.45992,6.00123) -- (4.46067,6.00204) -- (4.46141,6.00284) -- (4.46216,6.0036) -- (4.4629,6.0044) -- (4.46365,6.00466) -- (4.4644,6.00535) --
 (4.46514,6.00612) -- (4.46589,6.0069) -- (4.46663,6.0077) -- (4.46738,6.00849) -- (4.46812,6.00928) -- (4.46887,6.01002) -- (4.46962,6.01083) -- (4.47036,6.01159) -- (4.47111,6.01199) -- (4.47185,6.01277) -- (4.4726,6.01356) -- (4.47335,6.01434) --
 (4.47409,6.01512) -- (4.47484,6.0159) -- (4.47558,6.01669) -- (4.47633,6.01747) -- (4.47708,6.01825) -- (4.47782,6.01899) -- (4.47857,6.01977) -- (4.47931,6.02055) -- (4.48006,6.02133) -- (4.4808,6.02211) -- (4.48155,6.02248) -- (4.4823,6.02317) --
 (4.48304,6.02392) -- (4.48379,6.02468) -- (4.48453,6.02543) -- (4.48528,6.02619) -- (4.48602,6.02694) -- (4.48677,6.02766) -- (4.48752,6.02843) -- (4.48826,6.02918) -- (4.48901,6.02948) -- (4.48975,6.03022) -- (4.4905,6.03098) -- (4.49125,6.03174)
 -- (4.49199,6.03247) -- (4.49274,6.03321) -- (4.49348,6.03397) -- (4.49423,6.03473) -- (4.49498,6.03549) -- (4.49572,6.03624) -- (4.49647,6.03699) -- (4.49721,6.03714) -- (4.49796,6.03787) -- (4.4987,6.03861) -- (4.49945,6.03936) -- (4.5002,6.0401)
 -- (4.50094,6.04086) -- (4.50169,6.04161) -- (4.50243,6.04237) -- (4.50318,6.04313) -- (4.50392,6.04388) -- (4.50467,6.04433) -- (4.50542,6.04503) -- (4.50616,6.04575) -- (4.50691,6.04649) -- (4.50765,6.04722) -- (4.5084,6.04796) --
 (4.50915,6.04867) -- (4.50989,6.0494) -- (4.51064,6.04985) -- (4.51138,6.05055) -- (4.51213,6.05127) -- (4.51288,6.05198) -- (4.51362,6.05269) -- (4.51437,6.05338) -- (4.51511,6.0541) -- (4.51586,6.05482) -- (4.5166,6.0554) -- (4.51735,6.05548) --
 (4.5181,6.05618) -- (4.51884,6.05688) -- (4.51959,6.05758) -- (4.52033,6.05827) -- (4.52108,6.05897) -- (4.52182,6.05968) -- (4.52257,6.06035) -- (4.52332,6.06107) -- (4.52406,6.06178) -- (4.52481,6.06249) -- (4.52555,6.06314) -- (4.5263,6.06361) --
 (4.52705,6.0643) -- (4.52779,6.06499) -- (4.52854,6.06568) -- (4.52928,6.06635) -- (4.53003,6.06704) -- (4.53078,6.06773) -- (4.53152,6.0684) -- (4.53227,6.06909) -- (4.53301,6.06978) -- (4.53376,6.07047) -- (4.5345,6.07101) -- (4.53525,6.07169) --
 (4.536,6.07237) -- (4.53674,6.07305) -- (4.53749,6.07363) -- (4.53823,6.07409) -- (4.53898,6.07476) -- (4.53972,6.07545) -- (4.54047,6.07608) -- (4.54122,6.07675) -- (4.54196,6.07742) -- (4.54271,6.07806) -- (4.54345,6.07873) -- (4.5442,6.07939) --
 (4.54495,6.08006) -- (4.54569,6.08027) -- (4.54644,6.08091) -- (4.54718,6.08156) -- (4.54793,6.08221) -- (4.54868,6.08285) -- (4.54942,6.0835) -- (4.55017,6.08416) -- (4.55091,6.08482) -- (4.55166,6.08548) -- (4.5524,6.08613) -- (4.55315,6.08679) --
 (4.5539,6.08739) -- (4.55464,6.08805) -- (4.55539,6.08851) -- (4.55613,6.08875) -- (4.55688,6.08938) -- (4.55762,6.09001) -- (4.55837,6.09064) -- (4.55912,6.09127) -- (4.55986,6.09191) -- (4.56061,6.09255) -- (4.56135,6.09319) -- (4.5621,6.09383) --
 (4.56285,6.09448) -- (4.56359,6.09512) -- (4.56434,6.09568) -- (4.56508,6.09631) -- (4.56583,6.09655) -- (4.56658,6.09718) -- (4.56732,6.0978) -- (4.56807,6.09843) -- (4.56881,6.09906) -- (4.56956,6.09968) -- (4.5703,6.10031) -- (4.57105,6.10093) --
 (4.5718,6.10156) -- (4.57254,6.10215) -- (4.57329,6.10278) -- (4.57403,6.1034) -- (4.57478,6.10396) -- (4.57552,6.10458) -- (4.57627,6.1052) -- (4.57702,6.1056) -- (4.57776,6.1062) -- (4.57851,6.10681) -- (4.57925,6.10742) -- (4.58,6.10804) --
 (4.58075,6.10865) -- (4.58149,6.10925) -- (4.58224,6.10969) -- (4.58298,6.10972) -- (4.58373,6.1103) -- (4.58448,6.11089) -- (4.58522,6.11149) -- (4.58597,6.11209) -- (4.58671,6.11269) -- (4.58746,6.11326) -- (4.5882,6.11386) -- (4.58895,6.11446) --
 (4.5897,6.11506) -- (4.59044,6.11566) -- (4.59119,6.11626) -- (4.59193,6.11635) -- (4.59268,6.11692) -- (4.59342,6.1175) -- (4.59417,6.11808) -- (4.59492,6.11863) -- (4.59566,6.11922) -- (4.59641,6.11981) -- (4.59715,6.1204) -- (4.5979,6.12098) --
 (4.59865,6.12153) -- (4.59939,6.12211) -- (4.60014,6.1227) -- (4.60088,6.12328) -- (4.60163,6.12385) -- (4.60238,6.1244) -- (4.60312,6.1249) -- (4.60387,6.1254) -- (4.60461,6.12596) -- (4.60536,6.12652) -- (4.6061,6.12686) -- (4.60685,6.12741) --
 (4.6076,6.12797) -- (4.60834,6.12852) -- (4.60909,6.12908) -- (4.60983,6.12963) -- (4.61058,6.13019) -- (4.61132,6.13071) -- (4.61207,6.13127) -- (4.61282,6.13164) -- (4.61356,6.13161) -- (4.61431,6.13214) -- (4.61505,6.13267) -- (4.6158,6.13321) --
 (4.61655,6.13375) -- (4.61729,6.1343) -- (4.61804,6.13485) -- (4.61878,6.1354) -- (4.61953,6.13594) -- (4.62028,6.13649) -- (4.62102,6.13704) -- (4.62177,6.13759) -- (4.62251,6.13813) -- (4.62326,6.13858) -- (4.624,6.13911) -- (4.62475,6.13965) --
 (4.6255,6.14011) -- (4.62624,6.14066) -- (4.62699,6.14102) -- (4.62773,6.14129) -- (4.62848,6.1418) -- (4.62923,6.14232) -- (4.62997,6.14284) -- (4.63072,6.14334) -- (4.63146,6.14385) -- (4.63221,6.14437) -- (4.63295,6.14488) -- (4.6337,6.1454) --
 (4.63445,6.14591) -- (4.63519,6.14643) -- (4.63594,6.14692) -- (4.63668,6.14694) -- (4.63743,6.14744) -- (4.63818,6.14795) -- (4.63892,6.14845) -- (4.63967,6.14895) -- (4.64041,6.14945) -- (4.64116,6.14995) -- (4.6419,6.15045) -- (4.64265,6.15094)
 -- (4.6434,6.15144) -- (4.64414,6.15193) -- (4.64489,6.15243) -- (4.64563,6.15292) -- (4.64638,6.15342) -- (4.64713,6.15393) -- (4.64787,6.15437) -- (4.64862,6.15488) -- (4.64936,6.15532) -- (4.65011,6.1555) -- (4.65085,6.15598) -- (4.6516,6.15647)
 -- (4.65235,6.15693) -- (4.65309,6.15741) -- (4.65384,6.1579) -- (4.65458,6.15839) -- (4.65533,6.15878) -- (4.65608,6.15921) -- (4.65682,6.15959) -- (4.65757,6.15983) -- (4.65831,6.16029) -- (4.65906,6.16076) -- (4.6598,6.16119) -- (4.66055,6.16167)
 -- (4.6613,6.16215) -- (4.66204,6.16262) -- (4.66279,6.1631) -- (4.66353,6.16352) -- (4.66428,6.16393) -- (4.66503,6.16439) -- (4.66577,6.16483) -- (4.66652,6.16529) -- (4.66726,6.16563) -- (4.66801,6.16605) -- (4.66875,6.1665) -- (4.6695,6.16694)
 -- (4.67025,6.16735) -- (4.67099,6.16779) -- (4.67174,6.16823) -- (4.67248,6.16867) -- (4.67323,6.16912) -- (4.67397,6.16957) -- (4.67472,6.17001) -- (4.67547,6.17011) -- (4.67621,6.17037) -- (4.67696,6.1708) -- (4.6777,6.17123) -- (4.67845,6.17167)
 -- (4.6792,6.17211) -- (4.67994,6.17254) -- (4.68069,6.17298) -- (4.68143,6.17341) -- (4.68218,6.17385) -- (4.68293,6.17428) -- (4.68367,6.17467) -- (4.68442,6.1751) -- (4.68516,6.17488) -- (4.68591,6.17528) -- (4.68665,6.17568) -- (4.6874,6.17609)
 -- (4.68815,6.1765) -- (4.68889,6.17692) -- (4.68964,6.17735) -- (4.69038,6.17778) -- (4.69113,6.17821) -- (4.69187,6.17863) -- (4.69262,6.17905) -- (4.69337,6.17947) -- (4.69411,6.17989) -- (4.69486,6.18027) -- (4.6956,6.18022) -- (4.69635,6.1806)
 -- (4.6971,6.181) -- (4.69784,6.1814) -- (4.69859,6.18175) -- (4.69933,6.18215) -- (4.70008,6.18254) -- (4.70083,6.18294) -- (4.70157,6.18334) -- (4.70232,6.18374) -- (4.70306,6.18414) -- (4.70381,6.18454) -- (4.70455,6.18493) -- (4.7053,6.18532) --
 (4.70605,6.18572) -- (4.70679,6.18612) -- (4.70754,6.18651) -- (4.70828,6.18691) -- (4.70903,6.1873) -- (4.70977,6.18755) -- (4.71052,6.18779) -- (4.71127,6.18815) -- (4.71201,6.18852) -- (4.71276,6.18889) -- (4.7135,6.18927) -- (4.71425,6.18959) --
 (4.715,6.18995) -- (4.71574,6.19032) -- (4.71649,6.19068) -- (4.71723,6.19105) -- (4.71798,6.19143) -- (4.71873,6.1918) -- (4.71947,6.19174) -- (4.72022,6.19204) -- (4.72096,6.19239) -- (4.72171,6.19274) -- (4.72245,6.19303) -- (4.7232,6.19338) --
 (4.72395,6.19372) -- (4.72469,6.19408) -- (4.72544,6.19443) -- (4.72618,6.19477) -- (4.72693,6.19512) -- (4.72767,6.19548) -- (4.72842,6.19583) -- (4.72917,6.19619) -- (4.72991,6.19654) -- (4.73066,6.19689) -- (4.7314,6.19724) -- (4.73215,6.1976) --
 (4.7329,6.19778) -- (4.73364,6.19776) -- (4.73439,6.19809) -- (4.73513,6.19842) -- (4.73588,6.19872) -- (4.73663,6.19905) -- (4.73737,6.19938) -- (4.73812,6.1997) -- (4.73886,6.20003) -- (4.73961,6.20036) -- (4.74035,6.20069) -- (4.7411,6.201) --
 (4.74185,6.20132) -- (4.74259,6.20165) -- (4.74334,6.20197) -- (4.74408,6.20229) -- (4.74483,6.20213) -- (4.74557,6.20245) -- (4.74632,6.20276) -- (4.74707,6.20308) -- (4.74781,6.20339) -- (4.74856,6.2037) -- (4.7493,6.20401) -- (4.75005,6.20432) --
 (4.7508,6.20462) -- (4.75154,6.20493) -- (4.75229,6.20523) -- (4.75303,6.20554) -- (4.75378,6.20585) -- (4.75453,6.20611) -- (4.75527,6.20641) -- (4.75602,6.20671) -- (4.75676,6.2068) -- (4.75751,6.2071) -- (4.75825,6.20739) -- (4.759,6.20764) --
 (4.75975,6.20793) -- (4.76049,6.20823) -- (4.76124,6.20852) -- (4.76198,6.20881) -- (4.76273,6.20911) -- (4.76347,6.2094) -- (4.76422,6.20914) -- (4.76497,6.20942) -- (4.76571,6.20971) -- (4.76646,6.20999) -- (4.7672,6.21026) -- (4.76795,6.21054) --
 (4.7687,6.21082) -- (4.76944,6.21107) -- (4.77019,6.21135) -- (4.77093,6.21163) -- (4.77168,6.21185) -- (4.77243,6.21212) -- (4.77317,6.21239) -- (4.77392,6.21258) -- (4.77466,6.21285) -- (4.77541,6.21311) -- (4.77615,6.21316) -- (4.7769,6.21306) --
 (4.77765,6.21331) -- (4.77839,6.21357) -- (4.77914,6.21383) -- (4.77988,6.21408) -- (4.78063,6.21434) -- (4.78137,6.21459) -- (4.78212,6.21484) -- (4.78287,6.21509) -- (4.78361,6.21534) -- (4.78436,6.21557) -- (4.7851,6.21582) -- (4.78585,6.21606)
 -- (4.7866,6.21631) -- (4.78734,6.21656) -- (4.78809,6.2168) -- (4.78883,6.21688) -- (4.78958,6.21711) -- (4.79033,6.21734) -- (4.79107,6.21757) -- (4.79182,6.21779) -- (4.79256,6.218) -- (4.79331,6.21823) -- (4.79405,6.21846) -- (4.7948,6.21869) --
 (4.79555,6.21892) -- (4.79629,6.21915) -- (4.79704,6.21885) -- (4.79778,6.21898) -- (4.79853,6.21919) -- (4.79927,6.2194) -- (4.80002,6.21962) -- (4.80077,6.21981) -- (4.80151,6.22002) -- (4.80226,6.22022) -- (4.803,6.22043) -- (4.80375,6.22064) --
 (4.8045,6.22084) -- (4.80524,6.22105) -- (4.80599,6.22126) -- (4.80673,6.22146) -- (4.80748,6.22161) -- (4.80823,6.22146) -- (4.80897,6.22165) -- (4.80972,6.22185) -- (4.81046,6.22204) -- (4.81121,6.22222) -- (4.81195,6.22241) -- (4.8127,6.2226) --
 (4.81345,6.22279) -- (4.81419,6.22298) -- (4.81494,6.22317) -- (4.81568,6.22334) -- (4.81643,6.22352) -- (4.81717,6.22371) -- (4.81792,6.22389) -- (4.81867,6.22395) -- (4.81941,6.22412) -- (4.82016,6.2243) -- (4.8209,6.22447) -- (4.82165,6.22465) --
 (4.8224,6.22481) -- (4.82314,6.2249) -- (4.82389,6.22499) -- (4.82463,6.22516) -- (4.82538,6.22533) -- (4.82613,6.22549) -- (4.82687,6.22544) -- (4.82762,6.22554) -- (4.82836,6.22569) -- (4.82911,6.22585) -- (4.82985,6.22601) -- (4.8306,6.22617) --
 (4.83135,6.22632) -- (4.83209,6.22647) -- (4.83284,6.22663) -- (4.83358,6.22676) -- (4.83433,6.22691) -- (4.83507,6.22705) -- (4.83582,6.2272) -- (4.83657,6.22735) -- (4.83731,6.22722) -- (4.83806,6.22736) -- (4.8388,6.22751) -- (4.83955,6.22762) --
 (4.8403,6.22776) -- (4.84104,6.2279) -- (4.84179,6.22804) -- (4.84253,6.22818) -- (4.84328,6.22832) -- (4.84403,6.22845) -- (4.84477,6.22859) -- (4.84552,6.22872) -- (4.84626,6.22885) -- (4.84701,6.22899) -- (4.84775,6.22907) -- (4.8485,6.22886) --
 (4.84925,6.22893) -- (4.84999,6.22905) -- (4.85074,6.22917) -- (4.85148,6.22929) -- (4.85223,6.2294) -- (4.85297,6.22952) -- (4.85372,6.2296) -- (4.85447,6.22972) -- (4.85521,6.22983) -- (4.85596,6.22974) -- (4.8567,6.22982) -- (4.85745,6.22993) --
 (4.8582,6.23003) -- (4.85894,6.23013) -- (4.85969,6.23024) -- (4.86043,6.23034) -- (4.86118,6.23044) -- (4.86193,6.23054) -- (4.86267,6.23064) -- (4.86342,6.23074) -- (4.86416,6.2306) -- (4.86491,6.23069) -- (4.86565,6.23078) -- (4.8664,6.23086) --
 (4.86715,6.23072) -- (4.86789,6.23051) -- (4.86864,6.23058) -- (4.86938,6.23066) -- (4.87013,6.23074) -- (4.87087,6.23082) -- (4.87162,6.23088) -- (4.87237,6.23096) -- (4.87311,6.23104) -- (4.87386,6.23112) -- (4.8746,6.2312) -- (4.87535,6.23128) --
 (4.8761,6.23135) -- (4.87684,6.23143) -- (4.87759,6.2315) -- (4.87833,6.23158) -- (4.87908,6.23163) -- (4.87983,6.2317) -- (4.88057,6.23176) -- (4.88132,6.23151) -- (4.88206,6.23156) -- (4.88281,6.23162) -- (4.88355,6.23167) -- (4.8843,6.23173) --
 (4.88505,6.23179) -- (4.88579,6.23185) -- (4.88654,6.2319) -- (4.88728,6.23196) -- (4.88803,6.23201) -- (4.88877,6.23205) -- (4.88952,6.23211) -- (4.89027,6.23216) -- (4.89101,6.2322) -- (4.89176,6.23221) -- (4.8925,6.23224) -- (4.89325,6.23227) --
 (4.894,6.23231) -- (4.89474,6.23235) -- (4.89549,6.23239) -- (4.89623,6.23196) -- (4.89698,6.23199) -- (4.89773,6.23202) -- (4.89847,6.23205) -- (4.89922,6.23208) -- (4.89996,6.2321) -- (4.90071,6.23212) -- (4.90145,6.23215) -- (4.9022,6.23218) --
 (4.90295,6.2322) -- (4.90369,6.23222) -- (4.90444,6.23225) -- (4.90518,6.23227) -- (4.90593,6.23209) -- (4.90667,6.23203) -- (4.90742,6.23204) -- (4.90817,6.23204) -- (4.90891,6.23204) -- (4.90966,6.23204) -- (4.9104,6.23204) -- (4.91115,6.23205) --
 (4.9119,6.23205) -- (4.91264,6.23204) -- (4.91339,6.23204) -- (4.91413,6.23205) -- (4.91488,6.23205) -- (4.91563,6.23205) -- (4.91637,6.23187) -- (4.91712,6.23168) -- (4.91786,6.23168) -- (4.91861,6.23167) -- (4.91935,6.23166) -- (4.9201,6.23164) --
 (4.92085,6.23163) -- (4.92159,6.23162) -- (4.92234,6.23159) -- (4.92308,6.23158) -- (4.92383,6.23156) -- (4.92457,6.23152) -- (4.92532,6.23149) -- (4.92607,6.23106) -- (4.92681,6.23103) -- (4.92756,6.231) -- (4.9283,6.23097) -- (4.92905,6.23094) --
 (4.9298,6.23091) -- (4.93054,6.23088) -- (4.93129,6.23085) -- (4.93203,6.23082) -- (4.93278,6.23079) -- (4.93353,6.23075) -- (4.93427,6.23072) -- (4.93502,6.23068) -- (4.93576,6.23064) -- (4.93651,6.2306) -- (4.93725,6.23056) -- (4.938,6.23052) --
 (4.93875,6.23048) -- (4.93949,6.23044) -- (4.94024,6.23036) -- (4.94098,6.23032) -- (4.94173,6.23028) -- (4.94247,6.22955) -- (4.94322,6.22949) -- (4.94397,6.22942) -- (4.94471,6.22934) -- (4.94546,6.22928) -- (4.9462,6.22922) -- (4.94695,6.22916)
 -- (4.9477,6.2291) -- (4.94844,6.22904) -- (4.94919,6.22898) -- (4.94993,6.22891) -- (4.95068,6.22884) -- (4.95143,6.22878) -- (4.95217,6.22871) -- (4.95292,6.22864) -- (4.95366,6.22856) -- (4.95441,6.22849) -- (4.95515,6.22842) -- (4.9559,6.22834)
 -- (4.95665,6.22827) -- (4.95739,6.22819) -- (4.95814,6.22811) -- (4.95888,6.22803) -- (4.95963,6.22795) -- (4.96037,6.22739) -- (4.96112,6.22726) -- (4.96187,6.22717) -- (4.96261,6.22707) -- (4.96336,6.22697) -- (4.9641,6.22687) --
 (4.96485,6.22678) -- (4.9656,6.22668) -- (4.96634,6.22658) -- (4.96709,6.22648) -- (4.96783,6.22639) -- (4.96858,6.22629) -- (4.96933,6.22619) -- (4.97007,6.22609) -- (4.97082,6.22598) -- (4.97156,6.22588) -- (4.97231,6.22577) -- (4.97305,6.22567)
 -- (4.9738,6.22556) -- (4.97455,6.22546) -- (4.97529,6.22534) -- (4.97604,6.22482) -- (4.97678,6.2247) -- (4.97753,6.22458) -- (4.97827,6.22446) -- (4.97902,6.22433) -- (4.97977,6.22421) -- (4.98051,6.22409) -- (4.98126,6.22396) -- (4.982,6.22382)
 -- (4.98275,6.2237) -- (4.9835,6.22357) -- (4.98424,6.22345) -- (4.98499,6.22331) -- (4.98573,6.22319) -- (4.98648,6.22306) -- (4.98723,6.22293) -- (4.98797,6.2228) -- (4.98872,6.22264) -- (4.98946,6.22203) -- (4.99021,6.22185) -- (4.99095,6.22171)
 -- (4.9917,6.22156) -- (4.99245,6.22141) -- (4.99319,6.22126) -- (4.99394,6.22112) -- (4.99468,6.22097) -- (4.99543,6.22082) -- (4.99617,6.22067) -- (4.99692,6.22052) -- (4.99767,6.22037) -- (4.99841,6.22022) -- (4.99916,6.22006) -- (4.9999,6.21987)
 -- (5.00065,6.21939) -- (5.0014,6.21923) -- (5.00214,6.21906) -- (5.00289,6.2189) -- (5.00363,6.21872) -- (5.00438,6.21855) -- (5.00513,6.21838) -- (5.00587,6.21821) -- (5.00662,6.21804) -- (5.00736,6.21786) -- (5.00811,6.21769) -- (5.00885,6.21751)
 -- (5.0096,6.21734) -- (5.01035,6.21716) -- (5.01109,6.21698) -- (5.01184,6.2168) -- (5.01258,6.21662) -- (5.01333,6.2164) -- (5.01407,6.21619) -- (5.01482,6.21545) -- (5.01557,6.21526) -- (5.01631,6.21507) -- (5.01706,6.21488) -- (5.0178,6.21469)
 -- (5.01855,6.21449) -- (5.0193,6.2143) -- (5.02004,6.2141) -- (5.02079,6.21391) -- (5.02153,6.21371) -- (5.02228,6.21352) -- (5.02303,6.21329) -- (5.02377,6.2131) -- (5.02452,6.2129) -- (5.02526,6.2127) -- (5.02601,6.21244) -- (5.02675,6.212) --
 (5.0275,6.21178) -- (5.02825,6.21157) -- (5.02899,6.21135) -- (5.02974,6.21114) -- (5.03048,6.21092) -- (5.03123,6.2107) -- (5.03197,6.21049) -- (5.03272,6.21027) -- (5.03347,6.21005) -- (5.03421,6.20982) -- (5.03496,6.2096) -- (5.0357,6.20929) --
 (5.03645,6.20873) -- (5.0372,6.2085) -- (5.03794,6.20826) -- (5.03869,6.20803) -- (5.03943,6.2078) -- (5.04018,6.20756) -- (5.04093,6.20733) -- (5.04167,6.20709) -- (5.04242,6.20686) -- (5.04316,6.20662) -- (5.04391,6.20638) -- (5.04465,6.20614) --
 (5.0454,6.2059) -- (5.04615,6.20565) -- (5.04689,6.20538) -- (5.04764,6.20489) -- (5.04838,6.20463) -- (5.04913,6.20438) -- (5.04987,6.20412) -- (5.05062,6.20387) -- (5.05137,6.20361) -- (5.05211,6.20334) -- (5.05286,6.20308) -- (5.0536,6.20283) --
 (5.05435,6.20257) -- (5.0551,6.20231) -- (5.05584,6.20201) -- (5.05659,6.20169) -- (5.05733,6.20142) -- (5.05808,6.20115) -- (5.05883,6.20088) -- (5.05957,6.2006) -- (5.06032,6.20032) -- (5.06106,6.20005) -- (5.06181,6.19976) -- (5.06255,6.19948) --
 (5.0633,6.1992) -- (5.06405,6.19841) -- (5.06479,6.19812) -- (5.06554,6.19784) -- (5.06628,6.19756) -- (5.06703,6.19727) -- (5.06778,6.19698) -- (5.06852,6.19669) -- (5.06927,6.19641) -- (5.07001,6.19612) -- (5.07076,6.19583) -- (5.0715,6.19554) --
 (5.07225,6.19525) -- (5.073,6.19492) -- (5.07374,6.19432) -- (5.07449,6.194) -- (5.07523,6.1937) -- (5.07598,6.19339) -- (5.07673,6.19306) -- (5.07747,6.19276) -- (5.07822,6.19245) -- (5.07896,6.19214) -- (5.07971,6.19183) -- (5.08045,6.19152) --
 (5.0812,6.19121) -- (5.08195,6.1909) -- (5.08269,6.19059) -- (5.08344,6.19027) -- (5.08418,6.18996) -- (5.08493,6.18964) -- (5.08568,6.18932) -- (5.08642,6.189) -- (5.08717,6.18828) -- (5.08791,6.18783) -- (5.08866,6.18751) -- (5.0894,6.18718) --
 (5.09015,6.18686) -- (5.0909,6.18653) -- (5.09164,6.1862) -- (5.09239,6.18587) -- (5.09313,6.18554) -- (5.09388,6.1852) -- (5.09462,6.18487) -- (5.09537,6.18452) -- (5.09612,6.18414) -- (5.09686,6.1838) -- (5.09761,6.18346) -- (5.09835,6.18312) --
 (5.0991,6.18277) -- (5.09985,6.18223) -- (5.10059,6.18188) -- (5.10134,6.18153) -- (5.10208,6.18118) -- (5.10283,6.18083) -- (5.10358,6.18047) -- (5.10432,6.18012) -- (5.10507,6.17977) -- (5.10581,6.17942) -- (5.10656,6.17901) -- (5.1073,6.17855) --
 (5.10805,6.17804) -- (5.1088,6.17768) -- (5.10954,6.17732) -- (5.11029,6.17691) -- (5.11103,6.17655) -- (5.11178,6.17619) -- (5.11252,6.17582) -- (5.11327,6.17546) -- (5.11402,6.17509) -- (5.11476,6.17469) -- (5.11551,6.17389) -- (5.11625,6.17351)
 -- (5.117,6.17313) -- (5.11775,6.17275) -- (5.11849,6.17237) -- (5.11924,6.17199) -- (5.11998,6.17161) -- (5.12073,6.17123) -- (5.12148,6.17084) -- (5.12222,6.17046) -- (5.12297,6.17007) -- (5.12371,6.16968) -- (5.12446,6.16929) -- (5.1252,6.1689)
 -- (5.12595,6.16851) -- (5.1267,6.16813) -- (5.12744,6.16773) -- (5.12819,6.16706) -- (5.12893,6.16657) -- (5.12968,6.16617) -- (5.13042,6.16577) -- (5.13117,6.16537) -- (5.13192,6.16496) -- (5.13266,6.16456) -- (5.13341,6.16415) --
 (5.13415,6.16374) -- (5.1349,6.16334) -- (5.13565,6.16293) -- (5.13639,6.16252) -- (5.13714,6.16211) -- (5.13788,6.1617) -- (5.13863,6.16129) -- (5.13938,6.16085) -- (5.14012,6.16002) -- (5.14087,6.1596) -- (5.14161,6.15919) -- (5.14236,6.15877) --
 (5.1431,6.15835) -- (5.14385,6.15793) -- (5.1446,6.1575) -- (5.14534,6.15708) -- (5.14609,6.15665) -- (5.14683,6.15622) -- (5.14758,6.15578) -- (5.14832,6.15499) -- (5.14907,6.15456) -- (5.14982,6.15412) -- (5.15056,6.15366) -- (5.15131,6.15322) --
 (5.15205,6.15278) -- (5.1528,6.15234) -- (5.15355,6.15189) -- (5.15429,6.15144) -- (5.15504,6.151) -- (5.15578,6.15055) -- (5.15653,6.15011) -- (5.15728,6.14965) -- (5.15802,6.1492) -- (5.15877,6.14876) -- (5.15951,6.1483) -- (5.16026,6.14784) --
 (5.161,6.14715) -- (5.16175,6.14661) -- (5.1625,6.14615) -- (5.16324,6.14569) -- (5.16399,6.14523) -- (5.16473,6.14477) -- (5.16548,6.14431) -- (5.16622,6.14354) -- (5.16697,6.14298) -- (5.16772,6.14251) -- (5.16846,6.14204) -- (5.16921,6.14157) --
 (5.16995,6.1411) -- (5.1707,6.14062) -- (5.17145,6.14013) -- (5.17219,6.13966) -- (5.17294,6.13918) -- (5.17368,6.1387) -- (5.17443,6.13822) -- (5.17518,6.13774) -- (5.17592,6.13726) -- (5.17667,6.13674) -- (5.17741,6.13595) -- (5.17816,6.1354) --
 (5.1789,6.13491) -- (5.17965,6.13442) -- (5.1804,6.13393) -- (5.18114,6.13344) -- (5.18189,6.13294) -- (5.18263,6.13245) -- (5.18338,6.13192) -- (5.18412,6.13143) -- (5.18487,6.13093) -- (5.18562,6.13043) -- (5.18636,6.1298) -- (5.18711,6.1292) --
 (5.18785,6.1287) -- (5.1886,6.12819) -- (5.18935,6.12768) -- (5.19009,6.12717) -- (5.19084,6.1264) -- (5.19158,6.1258) -- (5.19233,6.12528) -- (5.19308,6.12477) -- (5.19382,6.12426) -- (5.19457,6.12374) -- (5.19531,6.12323) -- (5.19606,6.12271) --
 (5.1968,6.12219) -- (5.19755,6.12167) -- (5.1983,6.12115) -- (5.19904,6.12063) -- (5.19979,6.1201) -- (5.20053,6.11957) -- (5.20128,6.11905) -- (5.20202,6.11851) -- (5.20277,6.11772) -- (5.20352,6.11717) -- (5.20426,6.11663) -- (5.20501,6.1161) --
 (5.20575,6.11556) -- (5.2065,6.11503) -- (5.20725,6.11448) -- (5.20799,6.11394) -- (5.20874,6.1134) -- (5.20948,6.11286) -- (5.21023,6.11207) -- (5.21098,6.11124) -- (5.21172,6.1107) -- (5.21247,6.11015) -- (5.21321,6.1096) -- (5.21396,6.10905) --
 (5.2147,6.10851) -- (5.21545,6.10796) -- (5.2162,6.1074) -- (5.21694,6.10685) -- (5.21769,6.1063) -- (5.21843,6.10571) -- (5.21918,6.10509) -- (5.21992,6.10453) -- (5.22067,6.10397) -- (5.22142,6.10339) -- (5.22216,6.10276) -- (5.22291,6.10203) --
 (5.22365,6.10146) -- (5.2244,6.10089) -- (5.22515,6.10032) -- (5.22589,6.09975) -- (5.22664,6.09914) -- (5.22738,6.09857) -- (5.22813,6.098) -- (5.22888,6.09742) -- (5.22962,6.09685) -- (5.23037,6.09623) -- (5.23111,6.09526) -- (5.23186,6.09468) --
 (5.2326,6.09409) -- (5.23335,6.0935) -- (5.2341,6.09291) -- (5.23484,6.09232) -- (5.23559,6.09173) -- (5.23633,6.09114) -- (5.23708,6.09055) -- (5.23782,6.08996) -- (5.23857,6.08936) -- (5.23932,6.08877) -- (5.24006,6.08818) -- (5.24081,6.08758) --
 (5.24155,6.08684) -- (5.2423,6.08602) -- (5.24305,6.08542) -- (5.24379,6.08482) -- (5.24454,6.08421) -- (5.24528,6.08361) -- (5.24603,6.083) -- (5.24678,6.08235) -- (5.24752,6.08174) -- (5.24827,6.08113) -- (5.24901,6.08052) -- (5.24976,6.0799) --
 (5.2505,6.07929) -- (5.25125,6.07868) -- (5.252,6.07802) -- (5.25274,6.07694) -- (5.25349,6.07632) -- (5.25423,6.0757) -- (5.25498,6.07508) -- (5.25572,6.07445) -- (5.25647,6.07383) -- (5.25722,6.0732) -- (5.25796,6.07257) -- (5.25871,6.07195) --
 (5.25945,6.07132) -- (5.2602,6.07069) -- (5.26095,6.07006) -- (5.26169,6.06929) -- (5.26244,6.06855) -- (5.26318,6.06791) -- (5.26393,6.06727) -- (5.26468,6.06663) -- (5.26542,6.06598) -- (5.26617,6.06533) -- (5.26691,6.06469) -- (5.26766,6.06405)
 -- (5.2684,6.0634) -- (5.26915,6.06275) -- (5.2699,6.06189) -- (5.27064,6.06097) -- (5.27139,6.06032) -- (5.27213,6.05967) -- (5.27288,6.05901) -- (5.27362,6.05833) -- (5.27437,6.05768) -- (5.27512,6.05702) -- (5.27586,6.05636) -- (5.27661,6.0557)
 -- (5.27735,6.05474) -- (5.2781,6.05401) -- (5.27885,6.05335) -- (5.27959,6.05268) -- (5.28034,6.05201) -- (5.28108,6.05134) -- (5.28183,6.05067) -- (5.28258,6.05) -- (5.28332,6.04933) -- (5.28407,6.04866) -- (5.28481,6.04798) -- (5.28556,6.0473) --
 (5.2863,6.04663) -- (5.28705,6.04594) -- (5.2878,6.04526) -- (5.28854,6.04458) -- (5.28929,6.0439) -- (5.29003,6.04269) -- (5.29078,6.04201) -- (5.29152,6.04132) -- (5.29227,6.04063) -- (5.29302,6.03994) -- (5.29376,6.03925) -- (5.29451,6.03856) --
 (5.29525,6.03787) -- (5.296,6.03717) -- (5.29675,6.03648) -- (5.29749,6.03579) -- (5.29824,6.03502) -- (5.29898,6.03409) -- (5.29973,6.03339) -- (5.30048,6.03268) -- (5.30122,6.03196) -- (5.30197,6.03125) -- (5.30271,6.03054) -- (5.30346,6.02978) --
 (5.3042,6.02907) -- (5.30495,6.02835) -- (5.3057,6.02764) -- (5.30644,6.02693) -- (5.30719,6.02622) -- (5.30793,6.0255) -- (5.30868,6.02479) -- (5.30942,6.02407) -- (5.31017,6.02335) -- (5.31092,6.02247) -- (5.31166,6.02137) -- (5.31241,6.02064) --
 (5.31315,6.01992) -- (5.3139,6.01919) -- (5.31465,6.01846) -- (5.31539,6.01773) -- (5.31614,6.017) -- (5.31688,6.01628) -- (5.31763,6.01554) -- (5.31838,6.01481) -- (5.31912,6.01408) -- (5.31987,6.01333) -- (5.32061,6.01259) -- (5.32136,6.01186) --
 (5.3221,6.01085) -- (5.32285,6.0101) -- (5.3236,6.00935) -- (5.32434,6.00856) -- (5.32509,6.00781) -- (5.32583,6.00706) -- (5.32658,6.00629) -- (5.32732,6.00553) -- (5.32807,6.00478) -- (5.32882,6.00403) -- (5.32956,6.00328) -- (5.33031,6.00252) --
 (5.33105,6.00176) -- (5.3318,6.001) -- (5.33255,6.00025) -- (5.33329,5.99948) -- (5.33404,5.99867) -- (5.33478,5.99746) -- (5.33553,5.99669) -- (5.33628,5.99593) -- (5.33702,5.99514) -- (5.33777,5.99437) -- (5.33851,5.99359) -- (5.33926,5.99282) --
 (5.34,5.99204) -- (5.34075,5.99124) -- (5.3415,5.99047) -- (5.34224,5.98969) -- (5.34299,5.98891) -- (5.34373,5.98813) -- (5.34448,5.98735) -- (5.34522,5.9862) -- (5.34597,5.98542) -- (5.34672,5.98463) -- (5.34746,5.98381) -- (5.34821,5.98302) --
 (5.34895,5.98223) -- (5.3497,5.98143) -- (5.35045,5.98064) -- (5.35119,5.97984) -- (5.35194,5.97904) -- (5.35268,5.97825) -- (5.35343,5.97745) -- (5.35418,5.97666) -- (5.35492,5.97586) -- (5.35567,5.97506) -- (5.35641,5.97426) -- (5.35716,5.97346)
 -- (5.3579,5.97266) -- (5.35865,5.97186) -- (5.3594,5.97105) -- (5.36014,5.97023) -- (5.36089,5.9692) -- (5.36163,5.96838) -- (5.36238,5.96756) -- (5.36312,5.96675) -- (5.36387,5.96593) -- (5.36462,5.96509) -- (5.36536,5.96427) -- (5.36611,5.96345)
 -- (5.36685,5.96261) -- (5.3676,5.96178) -- (5.36835,5.96096) -- (5.36909,5.96012) -- (5.36984,5.9593) -- (5.37058,5.95847) -- (5.37133,5.95764) -- (5.37208,5.95681) -- (5.37282,5.95598) -- (5.37357,5.95515) -- (5.37431,5.95432) -- (5.37506,5.95348)
 -- (5.3758,5.95173) -- (5.37655,5.95089) -- (5.3773,5.95005) -- (5.37804,5.94921) -- (5.37879,5.94836) -- (5.37953,5.94752) -- (5.38028,5.94667) -- (5.38102,5.94583) -- (5.38177,5.94498) -- (5.38252,5.94414) -- (5.38326,5.94329) -- (5.38401,5.94244)
 -- (5.38475,5.94159) -- (5.3855,5.94074) -- (5.38625,5.93989) -- (5.38699,5.93903) -- (5.38774,5.93818) -- (5.38848,5.93733) -- (5.38923,5.93647) -- (5.38998,5.93562) -- (5.39072,5.93476) -- (5.39147,5.93385) -- (5.39221,5.93299) --
 (5.39296,5.93213) -- (5.3937,5.93126) -- (5.39445,5.9304) -- (5.3952,5.92953) -- (5.39594,5.92867) -- (5.39669,5.9278) -- (5.39743,5.92693) -- (5.39818,5.92606) -- (5.39892,5.92518) -- (5.39967,5.92431) -- (5.40042,5.92344) -- (5.40116,5.92256) --
 (5.40191,5.92169) -- (5.40265,5.92082) -- (5.4034,5.91994) -- (5.40415,5.91906) -- (5.40489,5.91819) -- (5.40564,5.91719) -- (5.40638,5.91629) -- (5.40713,5.9154) -- (5.40788,5.91451) -- (5.40862,5.91362) -- (5.40937,5.91272) -- (5.41011,5.91183) --
 (5.41086,5.91093) -- (5.4116,5.91004) -- (5.41235,5.909) -- (5.4131,5.90728) -- (5.41384,5.90637) -- (5.41459,5.90546) -- (5.41533,5.90455) -- (5.41608,5.90363) -- (5.41682,5.90272) -- (5.41757,5.90181) -- (5.41832,5.9009) -- (5.41906,5.89999) --
 (5.41981,5.89908) -- (5.42055,5.89817) -- (5.4213,5.89724) -- (5.42205,5.89633) -- (5.42279,5.89541) -- (5.42354,5.89449) -- (5.42428,5.89358) -- (5.42503,5.89266) -- (5.42578,5.89174) -- (5.42652,5.89082) -- (5.42727,5.8899) -- (5.42801,5.88898) --
 (5.42876,5.88806) -- (5.4295,5.88715) -- (5.43025,5.88617) -- (5.431,5.88524) -- (5.43174,5.88431) -- (5.43249,5.88338) -- (5.43323,5.88245) -- (5.43398,5.88151) -- (5.43472,5.88058) -- (5.43547,5.87965) -- (5.43622,5.87871) -- (5.43696,5.87778) --
 (5.43771,5.87684) -- (5.43845,5.8759) -- (5.4392,5.87496) -- (5.43995,5.87403) -- (5.44069,5.87309) -- (5.44144,5.87214) -- (5.44218,5.8712) -- (5.44293,5.87025) -- (5.44368,5.86931) -- (5.44442,5.86837) -- (5.44517,5.86742) -- (5.44591,5.86647) --
 (5.44666,5.8655) -- (5.4474,5.86456) -- (5.44815,5.86361) -- (5.4489,5.86266) -- (5.44964,5.8617) -- (5.45039,5.86075) -- (5.45113,5.85979) -- (5.45188,5.85884) -- (5.45262,5.85788) -- (5.45337,5.85693) -- (5.45412,5.85597) -- (5.45486,5.85494) --
 (5.45561,5.85398) -- (5.45635,5.85301) -- (5.4571,5.85204) -- (5.45785,5.85105) -- (5.45859,5.85008) -- (5.45934,5.84911) -- (5.46008,5.84814) -- (5.46083,5.84717) -- (5.46158,5.8462) -- (5.46232,5.84522) -- (5.46307,5.84425) -- (5.46381,5.84328) --
 (5.46456,5.84231) -- (5.4653,5.84133) -- (5.46605,5.84036) -- (5.4668,5.83938) -- (5.46754,5.8384) -- (5.46829,5.83736) -- (5.46903,5.83638) -- (5.46978,5.83539) -- (5.47052,5.83441) -- (5.47127,5.83342) -- (5.47202,5.83243) -- (5.47276,5.83144) --
 (5.47351,5.83046) -- (5.47425,5.82947) -- (5.475,5.82848) -- (5.47575,5.8275) -- (5.47649,5.82651) -- (5.47724,5.82552) -- (5.47798,5.82453) -- (5.47873,5.82354) -- (5.47948,5.82255) -- (5.48022,5.82135) -- (5.48097,5.82034) -- (5.48171,5.81925) --
 (5.48246,5.81823) -- (5.4832,5.81721) -- (5.48395,5.81618) -- (5.4847,5.81516) -- (5.48544,5.81414) -- (5.48619,5.81312) -- (5.48693,5.8121) -- (5.48768,5.81108) -- (5.48842,5.81005) -- (5.48917,5.80903) -- (5.48992,5.80801) -- (5.49066,5.80697) --
 (5.49141,5.80595) -- (5.49215,5.80492) -- (5.4929,5.8039) -- (5.49365,5.80288) -- (5.49439,5.80186) -- (5.49514,5.80083) -- (5.49588,5.79981) -- (5.49663,5.79876) -- (5.49738,5.79773) -- (5.49812,5.7967) -- (5.49887,5.79567) -- (5.49961,5.79464) --
 (5.50036,5.79362) -- (5.5011,5.79258) -- (5.50185,5.79155) -- (5.5026,5.79052) -- (5.50334,5.78949) -- (5.50409,5.78762) -- (5.50483,5.78636) -- (5.50558,5.78529) -- (5.50633,5.78423) -- (5.50707,5.78316) -- (5.50782,5.7821) -- (5.50856,5.78104) --
 (5.50931,5.77997) -- (5.51005,5.77891) -- (5.5108,5.77785) -- (5.51155,5.77678) -- (5.51229,5.77572) -- (5.51304,5.77465) -- (5.51378,5.77358) -- (5.51453,5.77252) -- (5.51528,5.77146) -- (5.51602,5.77038) -- (5.51677,5.76932) -- (5.51751,5.76825)
 -- (5.51826,5.76719) -- (5.519,5.76612) -- (5.51975,5.76506) -- (5.5205,5.764) -- (5.52124,5.76293) -- (5.52199,5.76186) -- (5.52273,5.7608) -- (5.52348,5.75974) -- (5.52423,5.75864) -- (5.52497,5.75757) -- (5.52572,5.75651) -- (5.52646,5.75544) --
 (5.52721,5.75437) -- (5.52795,5.75331) -- (5.5287,5.75211) -- (5.52945,5.75103) -- (5.53019,5.74995) -- (5.53094,5.74886) -- (5.53168,5.74778) -- (5.53243,5.7467) -- (5.53317,5.7456) -- (5.53392,5.74452) -- (5.53467,5.74343) -- (5.53541,5.74236) --
 (5.53616,5.741) -- (5.5369,5.7399) -- (5.53765,5.73881) -- (5.5384,5.73771) -- (5.53914,5.73662) -- (5.53989,5.73552) -- (5.54063,5.73442) -- (5.54138,5.73333) -- (5.54213,5.73223) -- (5.54287,5.73114) -- (5.54362,5.73005) -- (5.54436,5.72896) --
 (5.54511,5.72787) -- (5.54585,5.72678) -- (5.5466,5.72566) -- (5.54735,5.72457) -- (5.54809,5.72348) -- (5.54884,5.72239) -- (5.54958,5.7213) -- (5.55033,5.72012) -- (5.55107,5.71901) -- (5.55182,5.71791) -- (5.55257,5.7168) -- (5.55331,5.7157) --
 (5.55406,5.71458) -- (5.5548,5.71348) -- (5.55555,5.71238) -- (5.5563,5.71128) -- (5.55704,5.71018) -- (5.55779,5.70869) -- (5.55853,5.70757) -- (5.55928,5.70645) -- (5.56003,5.70533) -- (5.56077,5.70421) -- (5.56152,5.70309) -- (5.56226,5.70197) --
 (5.56301,5.70085) -- (5.56375,5.69973) -- (5.5645,5.6986) -- (5.56525,5.69748) -- (5.56599,5.69636) -- (5.56674,5.69524) -- (5.56748,5.69407) -- (5.56823,5.69294) -- (5.56897,5.69181) -- (5.56972,5.69068) -- (5.57047,5.68955) -- (5.57121,5.68841) --
 (5.57196,5.68727) -- (5.5727,5.68614) -- (5.57345,5.685) -- (5.5742,5.68388) -- (5.57494,5.68233) -- (5.57569,5.68118) -- (5.57643,5.68003) -- (5.57718,5.67888) -- (5.57793,5.67773) -- (5.57867,5.67658) -- (5.57942,5.67543) -- (5.58016,5.67428) --
 (5.58091,5.67313) -- (5.58165,5.67198) -- (5.5824,5.67083) -- (5.58315,5.66968) -- (5.58389,5.66854) -- (5.58464,5.66739) -- (5.58538,5.66624) -- (5.58613,5.66509) -- (5.58687,5.66394) -- (5.58762,5.66277) -- (5.58837,5.66161) -- (5.58911,5.66047)
 -- (5.58986,5.65932) -- (5.5906,5.65818) -- (5.59135,5.65703) -- (5.5921,5.65546) -- (5.59284,5.65429) -- (5.59359,5.65312) -- (5.59433,5.65195) -- (5.59508,5.65077) -- (5.59583,5.6496) -- (5.59657,5.6484) -- (5.59732,5.64723) -- (5.59806,5.64606)
 -- (5.59881,5.6449) -- (5.59955,5.64373) -- (5.6003,5.64257) -- (5.60105,5.6414) -- (5.60179,5.64024) -- (5.60254,5.63907) -- (5.60328,5.6379) -- (5.60403,5.63674) -- (5.60477,5.63557) -- (5.60552,5.63437) -- (5.60627,5.6332) -- (5.60701,5.63203) --
 (5.60776,5.63087) -- (5.6085,5.62968) -- (5.60925,5.62851) -- (5.61,5.62733) -- (5.61074,5.62616) -- (5.61149,5.62499) -- (5.61223,5.62382) -- (5.61298,5.62261) -- (5.61373,5.62143) -- (5.61447,5.62025) -- (5.61522,5.61844) -- (5.61596,5.61723) --
 (5.61671,5.61602) -- (5.61745,5.61476) -- (5.6182,5.61353) -- (5.61895,5.61231) -- (5.61969,5.61109) -- (5.62044,5.60987) -- (5.62118,5.60864) -- (5.62193,5.60742) -- (5.62267,5.6062) -- (5.62342,5.60493) -- (5.62417,5.60369) -- (5.62491,5.60246) --
 (5.62566,5.60123) -- (5.6264,5.60001) -- (5.62715,5.5988) -- (5.6279,5.59758) -- (5.62864,5.59636) -- (5.62939,5.59516) -- (5.63013,5.59395) -- (5.63088,5.59272) -- (5.63163,5.59151) -- (5.63237,5.59029) -- (5.63312,5.58909) -- (5.63386,5.58788) --
 (5.63461,5.58668) -- (5.63535,5.58547) -- (5.6361,5.58427) -- (5.63685,5.58307) -- (5.63759,5.58186) -- (5.63834,5.58066) -- (5.63908,5.57945) -- (5.63983,5.57824) -- (5.64057,5.57703) -- (5.64132,5.57582) -- (5.64207,5.57456) -- (5.64281,5.57335)
 -- (5.64356,5.57214) -- (5.6443,5.57092) -- (5.64505,5.56969) -- (5.6458,5.56848) -- (5.64654,5.56637) -- (5.64729,5.56511) -- (5.64803,5.56387) -- (5.64878,5.56262) -- (5.64953,5.56134) -- (5.65027,5.56009) -- (5.65102,5.55884) -- (5.65176,5.55759)
 -- (5.65251,5.55635) -- (5.65325,5.5551) -- (5.654,5.55385) -- (5.65475,5.55261) -- (5.65549,5.55136) -- (5.65624,5.55011) -- (5.65698,5.54887) -- (5.65773,5.54762) -- (5.65847,5.54634) -- (5.65922,5.54508) -- (5.65997,5.54338) -- (5.66071,5.54211)
 -- (5.66146,5.54083) -- (5.6622,5.53955) -- (5.66295,5.53827) -- (5.6637,5.53698) -- (5.66444,5.53571) -- (5.66519,5.53444) -- (5.66593,5.53317) -- (5.66668,5.53192) -- (5.66743,5.53067) -- (5.66817,5.52941) -- (5.66892,5.52816) -- (5.66966,5.52689)
 -- (5.67041,5.52563) -- (5.67115,5.52436) -- (5.6719,5.52311) -- (5.67265,5.52185) -- (5.67339,5.5206) -- (5.67414,5.51935) -- (5.67488,5.51809) -- (5.67563,5.51684) -- (5.67637,5.51559) -- (5.67712,5.51434) -- (5.67787,5.51308) -- (5.67861,5.51181)
 -- (5.67936,5.51055) -- (5.6801,5.5093) -- (5.68085,5.50736) -- (5.6816,5.50606) -- (5.68234,5.50475) -- (5.68309,5.50291) -- (5.68383,5.50154) -- (5.68458,5.50022) -- (5.68533,5.49891) -- (5.68607,5.49758) -- (5.68682,5.49627) -- (5.68756,5.49496)
 -- (5.68831,5.49363) -- (5.68905,5.49231) -- (5.6898,5.491) -- (5.69055,5.48969) -- (5.69129,5.48838) -- (5.69204,5.48707) -- (5.69278,5.48576) -- (5.69353,5.48442) -- (5.69427,5.48312) -- (5.69502,5.48182) -- (5.69577,5.48052) -- (5.69651,5.47922)
 -- (5.69726,5.47792) -- (5.698,5.47659) -- (5.69875,5.47529) -- (5.6995,5.47399) -- (5.70024,5.47266) -- (5.70099,5.47135) -- (5.70173,5.47005) -- (5.70248,5.46875) -- (5.70323,5.46745) -- (5.70397,5.46613) -- (5.70472,5.46482) -- (5.70546,5.46353)
 -- (5.70621,5.46223) -- (5.70695,5.46094) -- (5.7077,5.45963) -- (5.70845,5.45833) -- (5.70919,5.45704) -- (5.70994,5.45542) -- (5.71068,5.45408) -- (5.71143,5.45274) -- (5.71217,5.45136) -- (5.71292,5.45003) -- (5.71367,5.4487) -- (5.71441,5.44733)
 -- (5.71516,5.446) -- (5.7159,5.44466) -- (5.71665,5.44333) -- (5.7174,5.442) -- (5.71814,5.44067) -- (5.71889,5.43934) -- (5.71963,5.43801) -- (5.72038,5.43669) -- (5.72113,5.43536) -- (5.72187,5.43403) -- (5.72262,5.43271) -- (5.72336,5.43139) --
 (5.72411,5.43007) -- (5.72485,5.42875) -- (5.7256,5.42743) -- (5.72635,5.42611) -- (5.72709,5.4248) -- (5.72784,5.42349) -- (5.72858,5.42183) -- (5.72933,5.42048) -- (5.73007,5.41912) -- (5.73082,5.41777) -- (5.73157,5.41641) -- (5.73231,5.41506) --
 (5.73306,5.4137) -- (5.7338,5.41235) -- (5.73455,5.41101) -- (5.7353,5.40967) -- (5.73604,5.40832) -- (5.73679,5.40698) -- (5.73753,5.40564) -- (5.73828,5.40365) -- (5.73903,5.40188) -- (5.73977,5.4005) -- (5.74052,5.39913) -- (5.74126,5.39772) --
 (5.74201,5.39634) -- (5.74275,5.39497) -- (5.7435,5.39358) -- (5.74425,5.39222) -- (5.74499,5.39086) -- (5.74574,5.3895) -- (5.74648,5.38816) -- (5.74723,5.38681) -- (5.74797,5.38545) -- (5.74872,5.3841) -- (5.74947,5.38274) -- (5.75021,5.38138) --
 (5.75096,5.38003) -- (5.7517,5.37867) -- (5.75245,5.37732) -- (5.7532,5.37596) -- (5.75394,5.37457) -- (5.75469,5.37322) -- (5.75543,5.37181) -- (5.75618,5.37045) -- (5.75693,5.36909) -- (5.75767,5.36771) -- (5.75842,5.36635) -- (5.75916,5.36499) --
 (5.75991,5.36353) -- (5.76065,5.36214) -- (5.7614,5.36075) -- (5.76215,5.35937) -- (5.76289,5.35797) -- (5.76364,5.35659) -- (5.76438,5.35522) -- (5.76513,5.35384) -- (5.76587,5.35247) -- (5.76662,5.35108) -- (5.76737,5.3497) -- (5.76811,5.34833) --
 (5.76886,5.34694) -- (5.7696,5.34557) -- (5.77035,5.34419) -- (5.7711,5.34282) -- (5.77184,5.34145) -- (5.77259,5.34005) -- (5.77333,5.33833) -- (5.77408,5.33571) -- (5.77483,5.33429) -- (5.77557,5.33286) -- (5.77632,5.33145) -- (5.77706,5.33004) --
 (5.77781,5.32864) -- (5.77855,5.32723) -- (5.7793,5.32583) -- (5.78005,5.32443) -- (5.78079,5.32298) -- (5.78154,5.32158) -- (5.78228,5.32018) -- (5.78303,5.31878) -- (5.78377,5.31738) -- (5.78452,5.31598) -- (5.78527,5.31459) -- (5.78601,5.3132) --
 (5.78676,5.31181) -- (5.7875,5.31036) -- (5.78825,5.30896) -- (5.789,5.30756) -- (5.78974,5.30616) -- (5.79049,5.30474) -- (5.79123,5.30333) -- (5.79198,5.30192) -- (5.79273,5.30051) -- (5.79347,5.29911) -- (5.79422,5.29764) -- (5.79496,5.29621) --
 (5.79571,5.29479) -- (5.79645,5.29312) -- (5.7972,5.29168) -- (5.79795,5.29024) -- (5.79869,5.2885) -- (5.79944,5.28689) -- (5.80018,5.28544) -- (5.80093,5.28399) -- (5.80167,5.28255) -- (5.80242,5.28109) -- (5.80317,5.27965) -- (5.80391,5.27816) --
 (5.80466,5.27671) -- (5.8054,5.27526) -- (5.80615,5.27382) -- (5.8069,5.27238) -- (5.80764,5.27094) -- (5.80839,5.26951) -- (5.80913,5.26804) -- (5.80988,5.26661) -- (5.81063,5.26518) -- (5.81137,5.26375) -- (5.81212,5.26233) -- (5.81286,5.26074) --
 (5.81361,5.2593) -- (5.81435,5.25785) -- (5.8151,5.25641) -- (5.81585,5.25496) -- (5.81659,5.25352) -- (5.81734,5.25207) -- (5.81808,5.25063) -- (5.81883,5.24919) -- (5.81957,5.24774) -- (5.82032,5.2463) -- (5.82107,5.24486) -- (5.82181,5.24339) --
 (5.82256,5.24196) -- (5.8233,5.24053) -- (5.82405,5.2391) -- (5.8248,5.23759) -- (5.82554,5.23616) -- (5.82629,5.23472) -- (5.82703,5.23329) -- (5.82778,5.23183) -- (5.82853,5.23041) -- (5.82927,5.22899) -- (5.83002,5.22755) -- (5.83076,5.2261) --
 (5.83151,5.22465) -- (5.83225,5.22317) -- (5.833,5.22172) -- (5.83375,5.22027) -- (5.83449,5.21883) -- (5.83524,5.21736) -- (5.83598,5.2159) -- (5.83673,5.21446) -- (5.83747,5.21301) -- (5.83822,5.21111) -- (5.83897,5.2094) -- (5.83971,5.20791) --
 (5.84046,5.20642) -- (5.8412,5.20494) -- (5.84195,5.20342) -- (5.8427,5.20195) -- (5.84344,5.20046) -- (5.84419,5.19899) -- (5.84493,5.19752) -- (5.84568,5.19604) -- (5.84643,5.19457) -- (5.84717,5.19309) -- (5.84792,5.19162) -- (5.84866,5.19015) --
 (5.84941,5.18868) -- (5.85015,5.18721) -- (5.8509,5.1857) -- (5.85165,5.18423) -- (5.85239,5.18275) -- (5.85314,5.18126) -- (5.85388,5.17979) -- (5.85463,5.17831) -- (5.85537,5.17684) -- (5.85612,5.17538) -- (5.85687,5.17381) -- (5.85761,5.17117) --
 (5.85836,5.16966) -- (5.8591,5.16815) -- (5.85985,5.16664) -- (5.8606,5.16514) -- (5.86134,5.16364) -- (5.86209,5.16216) -- (5.86283,5.16065) -- (5.86358,5.15917) -- (5.86433,5.15769) -- (5.86507,5.15621) -- (5.86582,5.15473) -- (5.86656,5.15326) --
 (5.86731,5.15179) -- (5.86805,5.15033) -- (5.8688,5.14879) -- (5.86955,5.1473) -- (5.87029,5.14582) -- (5.87104,5.14434) -- (5.87178,5.14286) -- (5.87253,5.14126) -- (5.87327,5.13974) -- (5.87402,5.13823) -- (5.87477,5.13672) -- (5.87551,5.13522) --
 (5.87626,5.13372) -- (5.877,5.13222) -- (5.87775,5.13073) -- (5.8785,5.12925) -- (5.87924,5.12772) -- (5.87999,5.12624) -- (5.88073,5.12467) -- (5.88148,5.12314) -- (5.88223,5.12164) -- (5.88297,5.12008) -- (5.88372,5.11855) -- (5.88446,5.11702) --
 (5.88521,5.11551) -- (5.88595,5.11399) -- (5.8867,5.11238) -- (5.88745,5.11083) -- (5.88819,5.10928) -- (5.88894,5.10775) -- (5.88968,5.10623) -- (5.89043,5.10469) -- (5.89117,5.10317) -- (5.89192,5.10164) -- (5.89267,5.10011) -- (5.89341,5.0986) --
 (5.89416,5.09709) -- (5.8949,5.09558) -- (5.89565,5.09406) -- (5.8964,5.09255) -- (5.89714,5.09104) -- (5.89789,5.08953) -- (5.89863,5.08802) -- (5.89938,5.08651) -- (5.90013,5.08498) -- (5.90087,5.0834) -- (5.90162,5.08187) -- (5.90236,5.08035) --
 (5.90311,5.07882) -- (5.90385,5.07726) -- (5.9046,5.07575) -- (5.90535,5.07418) -- (5.90609,5.07264) -- (5.90684,5.07111) -- (5.90758,5.06956) -- (5.90833,5.06803) -- (5.90907,5.0665) -- (5.90982,5.06497) -- (5.91057,5.0634) -- (5.91131,5.06128) --
 (5.91206,5.05937) -- (5.9128,5.05781) -- (5.91355,5.05624) -- (5.9143,5.05466) -- (5.91504,5.05311) -- (5.91579,5.05157) -- (5.91653,5.04997) -- (5.91728,5.04845) -- (5.91803,5.04693) -- (5.91877,5.04541) -- (5.91952,5.04389) -- (5.92026,5.04236) --
 (5.92101,5.04085) -- (5.92175,5.03933) -- (5.9225,5.0378) -- (5.92325,5.03627) -- (5.92399,5.03472) -- (5.92474,5.03309) -- (5.92548,5.03042) -- (5.92623,5.02887) -- (5.92698,5.02733) -- (5.92772,5.02579) -- (5.92847,5.02425) -- (5.92921,5.02271) --
 (5.92996,5.02117) -- (5.9307,5.01964) -- (5.93145,5.01811) -- (5.9322,5.01657) -- (5.93294,5.01504) -- (5.93369,5.01351) -- (5.93443,5.01198) -- (5.93518,5.01034) -- (5.93593,5.00879) -- (5.93667,5.00724) -- (5.93742,5.00569) -- (5.93816,5.0041) --
 (5.93891,5.00253) -- (5.93965,5.00097) -- (5.9404,4.9994) -- (5.94115,4.99785) -- (5.94189,4.99616) -- (5.94264,4.99459) -- (5.94338,4.99302) -- (5.94413,4.99145) -- (5.94488,4.98985) -- (5.94562,4.98827) -- (5.94637,4.98671) -- (5.94711,4.98514) --
 (5.94786,4.98355) -- (5.9486,4.98198) -- (5.94935,4.98042) -- (5.9501,4.97885) -- (5.95084,4.97727) -- (5.95159,4.97571) -- (5.95233,4.97415) -- (5.95308,4.9726) -- (5.95382,4.97105) -- (5.95457,4.9695) -- (5.95532,4.9678) -- (5.95606,4.96623) --
 (5.95681,4.96467) -- (5.95755,4.96308) -- (5.9583,4.96119) -- (5.95905,4.95908) -- (5.95979,4.95749) -- (5.96054,4.9559) -- (5.96128,4.9543) -- (5.96203,4.9527) -- (5.96278,4.95111) -- (5.96352,4.94953) -- (5.96427,4.94793) -- (5.96501,4.94636) --
 (5.96576,4.94479) -- (5.9665,4.94322) -- (5.96725,4.94166) -- (5.968,4.94005) -- (5.96874,4.93848) -- (5.96949,4.9369) -- (5.97023,4.93532) -- (5.97098,4.93374) -- (5.97172,4.93215) -- (5.97247,4.93056) -- (5.97322,4.92897) -- (5.97396,4.92739) --
 (5.97471,4.92581) -- (5.97545,4.92416) -- (5.9762,4.92257) -- (5.97695,4.92094) -- (5.97769,4.91936) -- (5.97844,4.91778) -- (5.97918,4.91621) -- (5.97993,4.91454) -- (5.98068,4.91248) -- (5.98142,4.91079) -- (5.98217,4.90916) -- (5.98291,4.90752)
 -- (5.98366,4.90587) -- (5.9844,4.90425) -- (5.98515,4.90263) -- (5.9859,4.901) -- (5.98664,4.89938) -- (5.98739,4.89776) -- (5.98813,4.89615) -- (5.98888,4.89453) -- (5.98962,4.89292) -- (5.99037,4.8913) -- (5.99112,4.88969) -- (5.99186,4.88809) --
 (5.99261,4.88649) -- (5.99335,4.88485) -- (5.9941,4.88325) -- (5.99485,4.88166) -- (5.99559,4.88008) -- (5.99634,4.87843) -- (5.99708,4.87683) -- (5.99783,4.87524) -- (5.99858,4.87364) -- (5.99932,4.87196) -- (6.00007,4.87035) -- (6.00081,4.86875)
 -- (6.00156,4.86716) -- (6.0023,4.86553) -- (6.00305,4.86392) -- (6.0038,4.8623) -- (6.00454,4.8607) -- (6.00529,4.85909) -- (6.00603,4.85749) -- (6.00678,4.8559) -- (6.00752,4.85432) -- (6.00827,4.85274) -- (6.00902,4.85106) -- (6.00976,4.84854) --
 (6.01051,4.8469) -- (6.01125,4.84528) -- (6.012,4.84367) -- (6.01275,4.84206) -- (6.01349,4.84045) -- (6.01424,4.83883) -- (6.01498,4.8372) -- (6.01573,4.83555) -- (6.01648,4.83395) -- (6.01722,4.83235) -- (6.01797,4.83073) -- (6.01871,4.82912) --
 (6.01946,4.82752) -- (6.0202,4.82593) -- (6.02095,4.82428) -- (6.0217,4.82267) -- (6.02244,4.82105) -- (6.02319,4.8194) -- (6.02393,4.81777) -- (6.02468,4.81612) -- (6.02542,4.81398) -- (6.02617,4.81234) -- (6.02692,4.81071) -- (6.02766,4.80909) --
 (6.02841,4.80744) -- (6.02915,4.80579) -- (6.0299,4.80415) -- (6.03065,4.8025) -- (6.03139,4.80087) -- (6.03214,4.79924) -- (6.03288,4.79761) -- (6.03363,4.79598) -- (6.03438,4.79435) -- (6.03512,4.79272) -- (6.03587,4.79109) -- (6.03661,4.78946) --
 (6.03736,4.78785) -- (6.0381,4.78575) -- (6.03885,4.78409) -- (6.0396,4.78242) -- (6.04034,4.78076) -- (6.04109,4.77907) -- (6.04183,4.77743) -- (6.04258,4.77578) -- (6.04332,4.77414) -- (6.04407,4.77247) -- (6.04482,4.77083) -- (6.04556,4.76919) --
 (6.04631,4.76756) -- (6.04705,4.7659) -- (6.0478,4.76425) -- (6.04855,4.76261) -- (6.04929,4.76095) -- (6.05004,4.75931) -- (6.05078,4.75762) -- (6.05153,4.75597) -- (6.05228,4.75433) -- (6.05302,4.75269) -- (6.05377,4.75106) -- (6.05451,4.74942) --
 (6.05526,4.74778) -- (6.056,4.74614) -- (6.05675,4.74449) -- (6.0575,4.74285) -- (6.05824,4.74118) -- (6.05899,4.73951) -- (6.05973,4.73785) -- (6.06048,4.73606) -- (6.06122,4.73423) -- (6.06197,4.73233) -- (6.06272,4.73063) -- (6.06346,4.72894) --
 (6.06421,4.72719) -- (6.06495,4.72552) -- (6.0657,4.72386) -- (6.06645,4.7222) -- (6.06719,4.72051) -- (6.06794,4.71885) -- (6.06868,4.71719) -- (6.06943,4.71553) -- (6.07018,4.71389) -- (6.07092,4.71221) -- (6.07167,4.71056) -- (6.07241,4.70893) --
 (6.07316,4.70729) -- (6.0739,4.70559) -- (6.07465,4.70392) -- (6.0754,4.70227) -- (6.07614,4.70062) -- (6.07689,4.69899) -- (6.07763,4.69729) -- (6.07838,4.69563) -- (6.07912,4.69398) -- (6.07987,4.69232) -- (6.08062,4.69065) -- (6.08136,4.68899) --
 (6.08211,4.68736) -- (6.08285,4.68566) -- (6.0836,4.68396) -- (6.08435,4.68227) -- (6.08509,4.6806) -- (6.08584,4.67892) -- (6.08658,4.67721) -- (6.08733,4.67552) -- (6.08808,4.67385) -- (6.08882,4.67216) -- (6.08957,4.67048) -- (6.09031,4.66878) --
 (6.09106,4.66712) -- (6.0918,4.66546) -- (6.09255,4.66381) -- (6.0933,4.66215) -- (6.09404,4.6605) -- (6.09479,4.65884) -- (6.09553,4.65719) -- (6.09628,4.65553) -- (6.09702,4.65387) -- (6.09777,4.65222) -- (6.09852,4.65056) -- (6.09926,4.64891) --
 (6.10001,4.64711) -- (6.10075,4.64545) -- (6.1015,4.64379) -- (6.10225,4.64214) -- (6.10299,4.64048) -- (6.10374,4.63883) -- (6.10448,4.63701) -- (6.10523,4.63465) -- (6.10598,4.63298) -- (6.10672,4.63131) -- (6.10747,4.62964) -- (6.10821,4.62795)
 -- (6.10896,4.62628) -- (6.1097,4.62459) -- (6.11045,4.62293) -- (6.1112,4.62125) -- (6.11194,4.61953) -- (6.11269,4.61784) -- (6.11343,4.61616) -- (6.11418,4.61447) -- (6.11492,4.61277) -- (6.11567,4.61109) -- (6.11642,4.6094) -- (6.11716,4.60768)
 -- (6.11791,4.60599) -- (6.11865,4.60431) -- (6.1194,4.60262) -- (6.12015,4.60094) -- (6.12089,4.59927) -- (6.12164,4.59761) -- (6.12238,4.59594) -- (6.12313,4.59428) -- (6.12388,4.59261) -- (6.12462,4.59095) -- (6.12537,4.58922) --
 (6.12611,4.58754) -- (6.12686,4.58586) -- (6.1276,4.5842) -- (6.12835,4.58249) -- (6.1291,4.58078) -- (6.12984,4.57909) -- (6.13059,4.5774) -- (6.13133,4.57568) -- (6.13208,4.57359) -- (6.13282,4.57189) -- (6.13357,4.57018) -- (6.13432,4.5685) --
 (6.13506,4.56682) -- (6.13581,4.56515) -- (6.13655,4.56347) -- (6.1373,4.56177) -- (6.13805,4.56003) -- (6.13879,4.55764) -- (6.13954,4.55591) -- (6.14028,4.55419) -- (6.14103,4.55247) -- (6.14178,4.55076) -- (6.14252,4.54906) -- (6.14327,4.54736)
 -- (6.14401,4.54566) -- (6.14476,4.54396) -- (6.1455,4.54222) -- (6.14625,4.54053) -- (6.147,4.53884) -- (6.14774,4.53715) -- (6.14849,4.53547) -- (6.14923,4.53381) -- (6.14998,4.53215) -- (6.15072,4.53044) -- (6.15147,4.52877) -- (6.15222,4.52711)
 -- (6.15296,4.52542) -- (6.15371,4.52375) -- (6.15445,4.52209) -- (6.1552,4.52043) -- (6.15595,4.51875) -- (6.15669,4.51707) -- (6.15744,4.51542) -- (6.15818,4.51366) -- (6.15893,4.51198) -- (6.15968,4.51029) -- (6.16042,4.50857) --
 (6.16117,4.50687) -- (6.16191,4.5052) -- (6.16266,4.50348) -- (6.1634,4.50176) -- (6.16415,4.50003) -- (6.1649,4.49833) -- (6.16564,4.49661) -- (6.16639,4.4949) -- (6.16713,4.49318) -- (6.16788,4.49149) -- (6.16862,4.48981) -- (6.16937,4.48811) --
 (6.17012,4.48642) -- (6.17086,4.48445) -- (6.17161,4.48268) -- (6.17235,4.48091) -- (6.1731,4.47915) -- (6.17385,4.47735) -- (6.17459,4.47563) -- (6.17534,4.47391) -- (6.17608,4.47216) -- (6.17683,4.47044) -- (6.17758,4.46873) -- (6.17832,4.46702)
 -- (6.17907,4.46531) -- (6.17981,4.46361) -- (6.18056,4.46192) -- (6.1813,4.46024) -- (6.18205,4.45848) -- (6.1828,4.45678) -- (6.18354,4.45509) -- (6.18429,4.45339) -- (6.18503,4.45169) -- (6.18578,4.45) -- (6.18652,4.44826) -- (6.18727,4.44656) --
 (6.18802,4.44488) -- (6.18876,4.44317) -- (6.18951,4.44148) -- (6.19025,4.43978) -- (6.191,4.43804) -- (6.19175,4.43632) -- (6.19249,4.43461) -- (6.19324,4.43291) -- (6.19398,4.43121) -- (6.19473,4.42951) -- (6.19548,4.42781) -- (6.19622,4.42611) --
 (6.19697,4.42441) -- (6.19771,4.42272) -- (6.19846,4.42102) -- (6.1992,4.41932) -- (6.19995,4.41761) -- (6.2007,4.41591) -- (6.20144,4.41418) -- (6.20219,4.41248) -- (6.20293,4.41078) -- (6.20368,4.40861) -- (6.20442,4.40682) -- (6.20517,4.40506) --
 (6.20592,4.40331) -- (6.20666,4.40159) -- (6.20741,4.39988) -- (6.20815,4.39817) -- (6.2089,4.39647) -- (6.20965,4.39477) -- (6.21039,4.39305) -- (6.21114,4.39134) -- (6.21188,4.38964) -- (6.21263,4.38793) -- (6.21338,4.38623) -- (6.21412,4.38453)
 -- (6.21487,4.38282) -- (6.21561,4.3811) -- (6.21636,4.37938) -- (6.2171,4.37766) -- (6.21785,4.37589) -- (6.2186,4.37417) -- (6.21934,4.37245) -- (6.22009,4.37073) -- (6.22083,4.36902) -- (6.22158,4.36731) -- (6.22232,4.3656) -- (6.22307,4.3639) --
 (6.22382,4.36216) -- (6.22456,4.36045) -- (6.22531,4.35875) -- (6.22605,4.35701) -- (6.2268,4.35517) -- (6.22755,4.35293) -- (6.22829,4.3512) -- (6.22904,4.34948) -- (6.22978,4.34775) -- (6.23053,4.34574) -- (6.23128,4.34401) -- (6.23202,4.34229) --
 (6.23277,4.34057) -- (6.23351,4.33885) -- (6.23426,4.33714) -- (6.235,4.33541) -- (6.23575,4.3337) -- (6.2365,4.33198) -- (6.23724,4.33027) -- (6.23799,4.32856) -- (6.23873,4.32682) -- (6.23948,4.32511) -- (6.24022,4.32339) -- (6.24097,4.32168) --
 (6.24172,4.31997) -- (6.24246,4.3182) -- (6.24321,4.3165) -- (6.24395,4.3148) -- (6.2447,4.3131) -- (6.24545,4.31139) -- (6.24619,4.30968) -- (6.24694,4.30797) -- (6.24768,4.30626) -- (6.24843,4.30457) -- (6.24918,4.30288) -- (6.24992,4.3011) --
 (6.25067,4.29927) -- (6.25141,4.2975) -- (6.25216,4.29574) -- (6.2529,4.294) -- (6.25365,4.29226) -- (6.2544,4.29051) -- (6.25514,4.28877) -- (6.25589,4.28702) -- (6.25663,4.28527) -- (6.25738,4.28353) -- (6.25812,4.2818) -- (6.25887,4.28008) --
 (6.25962,4.27837) -- (6.26036,4.27663) -- (6.26111,4.2749) -- (6.26185,4.27317) -- (6.2626,4.27145) -- (6.26335,4.2697) -- (6.26409,4.26798) -- (6.26484,4.26625) -- (6.26558,4.26451) -- (6.26633,4.26278) -- (6.26708,4.26105) -- (6.26782,4.25931) --
 (6.26857,4.2575) -- (6.26931,4.25528) -- (6.27006,4.25351) -- (6.2708,4.25176) -- (6.27155,4.24997) -- (6.2723,4.24824) -- (6.27304,4.24654) -- (6.27379,4.24485) -- (6.27453,4.24316) -- (6.27528,4.2414) -- (6.27602,4.23969) -- (6.27677,4.23798) --
 (6.27752,4.23626) -- (6.27826,4.23455) -- (6.27901,4.23283) -- (6.27975,4.23111) -- (6.2805,4.22942) -- (6.28125,4.2277) -- (6.28199,4.22599) -- (6.28274,4.22427) -- (6.28348,4.22256) -- (6.28423,4.22081) -- (6.28498,4.21904) -- (6.28572,4.21727) --
 (6.28647,4.21551) -- (6.28721,4.21374) -- (6.28796,4.21199) -- (6.2887,4.21026) -- (6.28945,4.20852) -- (6.2902,4.20678) -- (6.29094,4.20502) -- (6.29169,4.20328) -- (6.29243,4.20157) -- (6.29318,4.19986) -- (6.29392,4.19812) -- (6.29467,4.1964) --
 (6.29542,4.19469) -- (6.29616,4.19297) -- (6.29691,4.19126) -- (6.29765,4.18955) -- (6.2984,4.18784) -- (6.29915,4.18608) -- (6.29989,4.18435) -- (6.30064,4.18264) -- (6.30138,4.18085) -- (6.30213,4.17908) -- (6.30288,4.17733) -- (6.30362,4.17559)
 -- (6.30437,4.17388) -- (6.30511,4.17217) -- (6.30586,4.17036) -- (6.3066,4.16862) -- (6.30735,4.16689) -- (6.3081,4.16518) -- (6.30884,4.16347) -- (6.30959,4.16176) -- (6.31033,4.16005) -- (6.31108,4.1583) -- (6.31182,4.15657) -- (6.31257,4.15481)
 -- (6.31332,4.15308) -- (6.31406,4.15135) -- (6.31481,4.14961) -- (6.31555,4.14721) -- (6.3163,4.14542) -- (6.31705,4.14364) -- (6.31779,4.14189) -- (6.31854,4.14015) -- (6.31928,4.13842) -- (6.32003,4.1367) -- (6.32078,4.13498) -- (6.32152,4.13327)
 -- (6.32227,4.13155) -- (6.32301,4.12978) -- (6.32376,4.12806) -- (6.3245,4.12634) -- (6.32525,4.12463) -- (6.326,4.12291) -- (6.32674,4.1212) -- (6.32749,4.11941) -- (6.32823,4.11767) -- (6.32898,4.11593) -- (6.32972,4.11416) -- (6.33047,4.11239)
 -- (6.33122,4.11063) -- (6.33196,4.10887) -- (6.33271,4.10713) -- (6.33345,4.10539) -- (6.3342,4.10366) -- (6.33495,4.10195) -- (6.33569,4.10024) -- (6.33644,4.09851) -- (6.33718,4.09679) -- (6.33793,4.09505) -- (6.33868,4.09331) --
 (6.33942,4.09159) -- (6.34017,4.08985) -- (6.34091,4.08812) -- (6.34166,4.08637) -- (6.3424,4.08459) -- (6.34315,4.08246) -- (6.3439,4.08067) -- (6.34464,4.0789) -- (6.34539,4.07713) -- (6.34613,4.07538) -- (6.34688,4.07363) -- (6.34762,4.07188) --
 (6.34837,4.07013) -- (6.34912,4.06839) -- (6.34986,4.06664) -- (6.35061,4.06486) -- (6.35135,4.06311) -- (6.3521,4.06137) -- (6.35285,4.05965) -- (6.35359,4.05793) -- (6.35434,4.05621) -- (6.35508,4.05447) -- (6.35583,4.05274) -- (6.35658,4.051) --
 (6.35732,4.04928) -- (6.35807,4.04756) -- (6.35881,4.04581) -- (6.35956,4.04408) -- (6.3603,4.04235) -- (6.36105,4.04061) -- (6.3618,4.03888) -- (6.36254,4.03716) -- (6.36329,4.03543) -- (6.36403,4.03369) -- (6.36478,4.03198) -- (6.36553,4.03022) --
 (6.36627,4.02848) -- (6.36702,4.02676) -- (6.36776,4.02505) -- (6.36851,4.0233) -- (6.36925,4.02155) -- (6.37,4.01984) -- (6.37075,4.01811) -- (6.37149,4.01625) -- (6.37224,4.01445) -- (6.37298,4.01266) -- (6.37373,4.01086) -- (6.37448,4.00906) --
 (6.37522,4.00728) -- (6.37597,4.00552) -- (6.37671,4.00373) -- (6.37746,4.00195) -- (6.3782,4.00021) -- (6.37895,3.99847) -- (6.3797,3.99673) -- (6.38044,3.99499) -- (6.38119,3.99325) -- (6.38193,3.99145) -- (6.38268,3.98971) -- (6.38343,3.98799) --
 (6.38417,3.98627) -- (6.38492,3.98456) -- (6.38566,3.98285) -- (6.38641,3.98114) -- (6.38715,3.97942) -- (6.3879,3.97771) -- (6.38865,3.97597) -- (6.38939,3.97424) -- (6.39014,3.97251) -- (6.39088,3.9708) -- (6.39163,3.96905) -- (6.39237,3.96731) --
 (6.39312,3.9656) -- (6.39387,3.96389) -- (6.39461,3.96217) -- (6.39536,3.96044) -- (6.3961,3.95869) -- (6.39685,3.95696) -- (6.3976,3.95524) -- (6.39834,3.95353) -- (6.39909,3.95176) -- (6.39983,3.95003) -- (6.40058,3.94831) -- (6.40133,3.94655) --
 (6.40207,3.94482) -- (6.40282,3.94309) -- (6.40356,3.94135) -- (6.40431,3.93962) -- (6.40505,3.93769) -- (6.4058,3.93591) -- (6.40655,3.93413) -- (6.40729,3.93235) -- (6.40804,3.93057) -- (6.40878,3.9288) -- (6.40953,3.92702) -- (6.41027,3.92526) --
 (6.41102,3.92349) -- (6.41177,3.92172) -- (6.41251,3.91996) -- (6.41326,3.91822) -- (6.414,3.91649) -- (6.41475,3.91472) -- (6.4155,3.91298) -- (6.41624,3.91125) -- (6.41699,3.90951) -- (6.41773,3.90778) -- (6.41848,3.90603) -- (6.41923,3.9043) --
 (6.41997,3.90258) -- (6.42072,3.90085) -- (6.42146,3.89913) -- (6.42221,3.89742) -- (6.42295,3.89569) -- (6.4237,3.89395) -- (6.42445,3.89222) -- (6.42519,3.89047) -- (6.42594,3.88877) -- (6.42668,3.88706) -- (6.42743,3.88534) -- (6.42817,3.88359)
 -- (6.42892,3.88185) -- (6.42967,3.88012) -- (6.43041,3.87836) -- (6.43116,3.87662) -- (6.4319,3.87489) -- (6.43265,3.87314) -- (6.4334,3.87141) -- (6.43414,3.86968) -- (6.43489,3.86797) -- (6.43563,3.86617) -- (6.43638,3.86442) -- (6.43713,3.86268)
 -- (6.43787,3.8604) -- (6.43862,3.85862) -- (6.43936,3.85684) -- (6.44011,3.85507) -- (6.44085,3.8533) -- (6.4416,3.85153) -- (6.44235,3.84976) -- (6.44309,3.84798) -- (6.44384,3.84621) -- (6.44458,3.84444) -- (6.44533,3.84267) -- (6.44607,3.8409)
 -- (6.44682,3.83916) -- (6.44757,3.83743) -- (6.44831,3.83569) -- (6.44906,3.83395) -- (6.4498,3.83222) -- (6.45055,3.83045) -- (6.4513,3.8287) -- (6.45204,3.82696) -- (6.45279,3.82523) -- (6.45353,3.82349) -- (6.45428,3.82176) -- (6.45503,3.82003)
 -- (6.45577,3.8183) -- (6.45652,3.81652) -- (6.45726,3.81477) -- (6.45801,3.81304) -- (6.45875,3.81131) -- (6.4595,3.80958) -- (6.46025,3.80784) -- (6.46099,3.8061) -- (6.46174,3.80436) -- (6.46248,3.80263) -- (6.46323,3.80089) -- (6.46397,3.79915)
 -- (6.46472,3.79742) -- (6.46547,3.79566) -- (6.46621,3.79392) -- (6.46696,3.79219) -- (6.4677,3.79046) -- (6.46845,3.78872) -- (6.4692,3.78702) -- (6.46994,3.78531) -- (6.47069,3.78354) -- (6.47143,3.7818) -- (6.47218,3.78004) -- (6.47293,3.77828)
 -- (6.47367,3.77651) -- (6.47442,3.7747) -- (6.47516,3.77293) -- (6.47591,3.77116) -- (6.47665,3.76939) -- (6.4774,3.76762) -- (6.47815,3.76586) -- (6.47889,3.76409) -- (6.47964,3.76232) -- (6.48038,3.76042) -- (6.48113,3.75864) -- (6.48187,3.75689)
 -- (6.48262,3.75513) -- (6.48337,3.75338) -- (6.48411,3.75163) -- (6.48486,3.74989) -- (6.4856,3.74816) -- (6.48635,3.74643) -- (6.4871,3.74472) -- (6.48784,3.743) -- (6.48859,3.74128) -- (6.48933,3.73956) -- (6.49008,3.73786) -- (6.49083,3.73617)
 -- (6.49157,3.73446) -- (6.49232,3.73275) -- (6.49306,3.73105) -- (6.49381,3.72934) -- (6.49455,3.72764) -- (6.4953,3.72591) -- (6.49605,3.72415) -- (6.49679,3.72246) -- (6.49754,3.72073) -- (6.49828,3.71903) -- (6.49903,3.71732) --
 (6.49977,3.71561) -- (6.50052,3.7139) -- (6.50127,3.71219) -- (6.50201,3.71049) -- (6.50276,3.70878) -- (6.5035,3.70707) -- (6.50425,3.70533) -- (6.505,3.70359) -- (6.50574,3.70183) -- (6.50649,3.70008) -- (6.50723,3.69833) -- (6.50798,3.69659) --
 (6.50873,3.69486) -- (6.50947,3.69311) -- (6.51022,3.69137) -- (6.51096,3.68963) -- (6.51171,3.68791) -- (6.51245,3.6862) -- (6.5132,3.6845) -- (6.51395,3.68279) -- (6.51469,3.68108) -- (6.51544,3.67938) -- (6.51618,3.67767) -- (6.51693,3.67597) --
 (6.51767,3.67426) -- (6.51842,3.67256) -- (6.51917,3.67081) -- (6.51991,3.66909) -- (6.52066,3.66738) -- (6.5214,3.66567) -- (6.52215,3.66396) -- (6.5229,3.66225) -- (6.52364,3.66054) -- (6.52439,3.65882) -- (6.52513,3.6571) -- (6.52588,3.65537) --
 (6.52663,3.65367) -- (6.52737,3.65196) -- (6.52812,3.65023) -- (6.52886,3.64846) -- (6.52961,3.64659) -- (6.53035,3.64483) -- (6.5311,3.64306) -- (6.53185,3.6413) -- (6.53259,3.63953) -- (6.53334,3.63777) -- (6.53408,3.63601) -- (6.53483,3.63403) --
 (6.53557,3.63226) -- (6.53632,3.6305) -- (6.53707,3.62876) -- (6.53781,3.62701) -- (6.53856,3.62527) -- (6.5393,3.62353) -- (6.54005,3.62178) -- (6.5408,3.62005) -- (6.54154,3.61832) -- (6.54229,3.6166) -- (6.54303,3.61489) -- (6.54378,3.61317) --
 (6.54453,3.61147) -- (6.54527,3.60975) -- (6.54602,3.60804) -- (6.54676,3.60633) -- (6.54751,3.60462) -- (6.54825,3.60291) -- (6.549,3.6012) -- (6.54975,3.59949) -- (6.55049,3.59778) -- (6.55124,3.59608) -- (6.55198,3.59438) -- (6.55273,3.59268) --
 (6.55347,3.59097) -- (6.55422,3.58927) -- (6.55497,3.58756) -- (6.55571,3.58585) -- (6.55646,3.58412) -- (6.5572,3.58239) -- (6.55795,3.58053) -- (6.5587,3.57876) -- (6.55944,3.57699) -- (6.56019,3.57522) -- (6.56093,3.57345) -- (6.56168,3.57171) --
 (6.56243,3.56997) -- (6.56317,3.56823) -- (6.56392,3.56649) -- (6.56466,3.56475) -- (6.56541,3.563) -- (6.56615,3.56126) -- (6.5669,3.55952) -- (6.56765,3.55776) -- (6.56839,3.55602) -- (6.56914,3.55428) -- (6.56988,3.55253) -- (6.57063,3.5508) --
 (6.57137,3.54906) -- (6.57212,3.54732) -- (6.57287,3.54561) -- (6.57361,3.54391) -- (6.57436,3.54221) -- (6.5751,3.54053) -- (6.57585,3.53882) -- (6.5766,3.53712) -- (6.57734,3.53542) -- (6.57809,3.53372) -- (6.57883,3.53202) -- (6.57958,3.53034) --
 (6.58033,3.52861) -- (6.58107,3.52691) -- (6.58182,3.52522) -- (6.58256,3.52352) -- (6.58331,3.52183) -- (6.58405,3.52013) -- (6.5848,3.51841) -- (6.58555,3.5167) -- (6.58629,3.51501) -- (6.58704,3.51333) -- (6.58778,3.51163) -- (6.58853,3.50993) --
 (6.58927,3.50823) -- (6.59002,3.50654) -- (6.59077,3.50484) -- (6.59151,3.50315) -- (6.59226,3.50144) -- (6.593,3.49973) -- (6.59375,3.49804) -- (6.5945,3.49635) -- (6.59524,3.49466) -- (6.59599,3.49296) -- (6.59673,3.49127) -- (6.59748,3.48958) --
 (6.59823,3.48789) -- (6.59897,3.4862) -- (6.59972,3.48451) -- (6.60046,3.48281) -- (6.60121,3.48112) -- (6.60195,3.47943) -- (6.6027,3.47774) -- (6.60345,3.47605) -- (6.60419,3.47436) -- (6.60494,3.47266) -- (6.60568,3.47096) -- (6.60643,3.46927) --
 (6.60717,3.46759) -- (6.60792,3.4659) -- (6.60867,3.46416) -- (6.60941,3.46246) -- (6.61016,3.46078) -- (6.6109,3.45909) -- (6.61165,3.45741) -- (6.6124,3.45566) -- (6.61314,3.45397) -- (6.61389,3.45228) -- (6.61463,3.4506) -- (6.61538,3.44892) --
 (6.61613,3.44721) -- (6.61687,3.44553) -- (6.61762,3.44385) -- (6.61836,3.44216) -- (6.61911,3.44048) -- (6.61985,3.43878) -- (6.6206,3.43708) -- (6.62135,3.4354) -- (6.62209,3.43372) -- (6.62284,3.43204) -- (6.62358,3.43034) -- (6.62433,3.42865) --
 (6.62507,3.42697) -- (6.62582,3.42529) -- (6.62657,3.42361) -- (6.62731,3.42191) -- (6.62806,3.4202) -- (6.6288,3.41849) -- (6.62955,3.4168) -- (6.6303,3.41513) -- (6.63104,3.41345) -- (6.63179,3.41176) -- (6.63253,3.41006) -- (6.63328,3.40835) --
 (6.63403,3.40665) -- (6.63477,3.40497) -- (6.63552,3.4033) -- (6.63626,3.40161) -- (6.63701,3.3999) -- (6.63775,3.3982) -- (6.6385,3.39649) -- (6.63925,3.39478) -- (6.63999,3.39308) -- (6.64074,3.39139) -- (6.64148,3.38972) -- (6.64223,3.38804) --
 (6.64297,3.38637) -- (6.64372,3.38465) -- (6.64447,3.38294) -- (6.64521,3.38123) -- (6.64596,3.37952) -- (6.6467,3.37782) -- (6.64745,3.37611) -- (6.6482,3.3744) -- (6.64894,3.3727) -- (6.64969,3.37099) -- (6.65043,3.36929) -- (6.65118,3.36758) --
 (6.65193,3.36588) -- (6.65267,3.36418) -- (6.65342,3.36251) -- (6.65416,3.36084) -- (6.65491,3.35917) -- (6.65565,3.3575) -- (6.6564,3.35568) -- (6.65715,3.35345) -- (6.65789,3.35171) -- (6.65864,3.35002) -- (6.65938,3.34832) -- (6.66013,3.34663) --
 (6.66087,3.34493) -- (6.66162,3.34324) -- (6.66237,3.34155) -- (6.66311,3.33985) -- (6.66386,3.33815) -- (6.6646,3.33648) -- (6.66535,3.33481) -- (6.6661,3.33313) -- (6.66684,3.33145) -- (6.66759,3.32978) -- (6.66833,3.3281) -- (6.66908,3.32643) --
 (6.66983,3.32475) -- (6.67057,3.32308) -- (6.67132,3.3214) -- (6.67206,3.31973) -- (6.67281,3.31805) -- (6.67355,3.31637) -- (6.6743,3.31468) -- (6.67505,3.31292) -- (6.67579,3.3112) -- (6.67654,3.30948) -- (6.67728,3.30776) -- (6.67803,3.30603) --
 (6.67877,3.30431) -- (6.67952,3.30259) -- (6.68027,3.30088) -- (6.68101,3.29916) -- (6.68176,3.29745) -- (6.6825,3.29576) -- (6.68325,3.29408) -- (6.684,3.29239) -- (6.68474,3.29071) -- (6.68549,3.28903) -- (6.68623,3.28735) -- (6.68698,3.28567) --
 (6.68773,3.28399) -- (6.68847,3.28231) -- (6.68922,3.28063) -- (6.68996,3.27895) -- (6.69071,3.27728) -- (6.69145,3.2756) -- (6.6922,3.27392) -- (6.69295,3.27225) -- (6.69369,3.27058) -- (6.69444,3.2689) -- (6.69518,3.26723) -- (6.69593,3.26556) --
 (6.69667,3.26389) -- (6.69742,3.26222) -- (6.69817,3.26055) -- (6.69891,3.25891) -- (6.69966,3.25726) -- (6.7004,3.2556) -- (6.70115,3.25396) -- (6.7019,3.25232) -- (6.70264,3.25069) -- (6.70339,3.24904) -- (6.70413,3.2474) -- (6.70488,3.24576) --
 (6.70563,3.24412) -- (6.70637,3.24247) -- (6.70712,3.24084) -- (6.70786,3.23919) -- (6.70861,3.23755) -- (6.70935,3.23591) -- (6.7101,3.23426) -- (6.71085,3.23262) -- (6.71159,3.23098) -- (6.71234,3.22934) -- (6.71308,3.22769) -- (6.71383,3.22606)
 -- (6.71457,3.22441) -- (6.71532,3.22277) -- (6.71607,3.22113) -- (6.71681,3.21948) -- (6.71756,3.21784) -- (6.7183,3.2162) -- (6.71905,3.21456) -- (6.7198,3.21291) -- (6.72054,3.21128) -- (6.72129,3.20963) -- (6.72203,3.20799) -- (6.72278,3.20635)
 -- (6.72353,3.2047) -- (6.72427,3.20305) -- (6.72502,3.20141) -- (6.72576,3.19978) -- (6.72651,3.19813) -- (6.72725,3.19647) -- (6.728,3.19481) -- (6.72875,3.19315) -- (6.72949,3.19151) -- (6.73024,3.18987) -- (6.73098,3.1882) -- (6.73173,3.18654)
 -- (6.73247,3.18488) -- (6.73322,3.18322) -- (6.73397,3.18157) -- (6.73471,3.17993) -- (6.73546,3.1783) -- (6.7362,3.17663) -- (6.73695,3.17497) -- (6.7377,3.17331) -- (6.73844,3.17165) -- (6.73919,3.16999) -- (6.73993,3.16833) -- (6.74068,3.16667)
 -- (6.74143,3.16501) -- (6.74217,3.16335) -- (6.74292,3.1617) -- (6.74366,3.16004) -- (6.74441,3.15838) -- (6.74515,3.15633) -- (6.7459,3.15432) -- (6.74665,3.15266) -- (6.74739,3.15101) -- (6.74814,3.14935) -- (6.74888,3.14771) -- (6.74963,3.14606)
 -- (6.75037,3.14441) -- (6.75112,3.14276) -- (6.75187,3.1411) -- (6.75261,3.13945) -- (6.75336,3.1378) -- (6.7541,3.13615) -- (6.75485,3.13451) -- (6.7556,3.13285) -- (6.75634,3.13121) -- (6.75709,3.12956) -- (6.75783,3.12792) -- (6.75858,3.12627)
 -- (6.75933,3.12463) -- (6.76007,3.12298) -- (6.76082,3.12134) -- (6.76156,3.1197) -- (6.76231,3.11806) -- (6.76305,3.11642) -- (6.7638,3.11478) -- (6.76455,3.11314) -- (6.76529,3.1115) -- (6.76604,3.10986) -- (6.76678,3.10822) -- (6.76753,3.10659)
 -- (6.76827,3.10495) -- (6.76902,3.10331) -- (6.76977,3.10168) -- (6.77051,3.10005) -- (6.77126,3.09845) -- (6.772,3.09682) -- (6.77275,3.09523) -- (6.7735,3.09361) -- (6.77424,3.09196) -- (6.77499,3.09033) -- (6.77573,3.08871) -- (6.77648,3.08709)
 -- (6.77723,3.08547) -- (6.77797,3.08381) -- (6.77872,3.08219) -- (6.77946,3.08057) -- (6.78021,3.07894) -- (6.78095,3.07732) -- (6.7817,3.0757) -- (6.78245,3.07408) -- (6.78319,3.07246) -- (6.78394,3.07085) -- (6.78468,3.06908) -- (6.78543,3.06742)
 -- (6.78617,3.06576) -- (6.78692,3.0641) -- (6.78767,3.06248) -- (6.78841,3.06085) -- (6.78916,3.05923) -- (6.7899,3.0576) -- (6.79065,3.05598) -- (6.7914,3.05436) -- (6.79214,3.05274) -- (6.79289,3.05065) -- (6.79363,3.04904) -- (6.79438,3.04745)
 -- (6.79513,3.04584) -- (6.79587,3.04424) -- (6.79662,3.04264) -- (6.79736,3.04104) -- (6.79811,3.03944) -- (6.79885,3.03785) -- (6.7996,3.03626) -- (6.80035,3.03466) -- (6.80109,3.03307) -- (6.80184,3.03147) -- (6.80258,3.02988) --
 (6.80333,3.02829) -- (6.80408,3.02669) -- (6.80482,3.02509) -- (6.80557,3.02351) -- (6.80631,3.02191) -- (6.80706,3.02032) -- (6.8078,3.01873) -- (6.80855,3.01713) -- (6.8093,3.01554) -- (6.81004,3.01395) -- (6.81079,3.01235) -- (6.81153,3.01076) --
 (6.81228,3.00918) -- (6.81302,3.00758) -- (6.81377,3.00599) -- (6.81452,3.00441) -- (6.81526,3.00281) -- (6.81601,3.00122) -- (6.81675,2.99964) -- (6.8175,2.99804) -- (6.81825,2.99645) -- (6.81899,2.99487) -- (6.81974,2.99327) -- (6.82048,2.99168)
 -- (6.82123,2.9901) -- (6.82198,2.98852) -- (6.82272,2.98694) -- (6.82347,2.98536) -- (6.82421,2.98378) -- (6.82496,2.9822) -- (6.8257,2.98062) -- (6.82645,2.97904) -- (6.8272,2.97746) -- (6.82794,2.97587) -- (6.82869,2.97426) -- (6.82943,2.97269)
 -- (6.83018,2.97102) -- (6.83092,2.96941) -- (6.83167,2.9678) -- (6.83242,2.96619) -- (6.83316,2.96458) -- (6.83391,2.96297) -- (6.83465,2.96136) -- (6.8354,2.95975) -- (6.83615,2.95814) -- (6.83689,2.95616) -- (6.83764,2.95455) -- (6.83838,2.95293)
 -- (6.83913,2.95132) -- (6.83988,2.94971) -- (6.84062,2.94809) -- (6.84137,2.94648) -- (6.84211,2.94487) -- (6.84286,2.94326) -- (6.8436,2.94165) -- (6.84435,2.94004) -- (6.8451,2.93844) -- (6.84584,2.93683) -- (6.84659,2.93522) -- (6.84733,2.93362)
 -- (6.84808,2.93204) -- (6.84882,2.93047) -- (6.84957,2.9289) -- (6.85032,2.92734) -- (6.85106,2.92576) -- (6.85181,2.92419) -- (6.85255,2.92263) -- (6.8533,2.92106) -- (6.85405,2.91948) -- (6.85479,2.91791) -- (6.85554,2.91634) -- (6.85628,2.91476)
 -- (6.85703,2.91318) -- (6.85778,2.91162) -- (6.85852,2.91005) -- (6.85927,2.90846) -- (6.86001,2.90689) -- (6.86076,2.90532) -- (6.8615,2.90374) -- (6.86225,2.90217) -- (6.863,2.9006) -- (6.86374,2.89904) -- (6.86449,2.89745) -- (6.86523,2.89589)
 -- (6.86598,2.89432) -- (6.86672,2.89276) -- (6.86747,2.8912) -- (6.86822,2.88961) -- (6.86896,2.88801) -- (6.86971,2.88643) -- (6.87045,2.88487) -- (6.8712,2.88329) -- (6.87195,2.88171) -- (6.87269,2.88014) -- (6.87344,2.87854) -- (6.87418,2.87694)
 -- (6.87493,2.87527) -- (6.87568,2.87367) -- (6.87642,2.87207) -- (6.87717,2.87047) -- (6.87791,2.86887) -- (6.87866,2.86727) -- (6.8794,2.86567) -- (6.88015,2.86411) -- (6.8809,2.86254) -- (6.88164,2.86098) -- (6.88239,2.85942) -- (6.88313,2.85786)
 -- (6.88388,2.8563) -- (6.88462,2.85474) -- (6.88537,2.85295) -- (6.88612,2.85139) -- (6.88686,2.84982) -- (6.88761,2.84826) -- (6.88835,2.84669) -- (6.8891,2.84513) -- (6.88985,2.84357) -- (6.89059,2.84201) -- (6.89134,2.84045) -- (6.89208,2.83889)
 -- (6.89283,2.83733) -- (6.89358,2.83577) -- (6.89432,2.83421) -- (6.89507,2.83266) -- (6.89581,2.8311) -- (6.89656,2.82954) -- (6.8973,2.82799) -- (6.89805,2.82644) -- (6.8988,2.82488) -- (6.89954,2.82333) -- (6.90029,2.82177) -- (6.90103,2.82022)
 -- (6.90178,2.81866) -- (6.90252,2.81714) -- (6.90327,2.8156) -- (6.90402,2.81406) -- (6.90476,2.81254) -- (6.90551,2.81102) -- (6.90625,2.8095) -- (6.907,2.80795) -- (6.90775,2.80643) -- (6.90849,2.80488) -- (6.90924,2.80336) -- (6.90998,2.80183)
 -- (6.91073,2.80029) -- (6.91148,2.79876) -- (6.91222,2.79722) -- (6.91297,2.79569) -- (6.91371,2.79415) -- (6.91446,2.79263) -- (6.9152,2.79109) -- (6.91595,2.78957) -- (6.9167,2.78804) -- (6.91744,2.78651) -- (6.91819,2.78497) -- (6.91893,2.78342)
 -- (6.91968,2.78191) -- (6.92042,2.78038) -- (6.92117,2.77884) -- (6.92192,2.77729) -- (6.92266,2.77575) -- (6.92341,2.7742) -- (6.92415,2.77266) -- (6.9249,2.77111) -- (6.92565,2.76957) -- (6.92639,2.76802) -- (6.92714,2.76648) -- (6.92788,2.76484)
 -- (6.92863,2.76329) -- (6.92938,2.76174) -- (6.93012,2.76019) -- (6.93087,2.75864) -- (6.93161,2.7571) -- (6.93236,2.75555) -- (6.9331,2.754) -- (6.93385,2.75246) -- (6.9346,2.75091) -- (6.93534,2.74937) -- (6.93609,2.74746) -- (6.93683,2.74592) --
 (6.93758,2.7444) -- (6.93832,2.74286) -- (6.93907,2.74133) -- (6.93982,2.73982) -- (6.94056,2.73831) -- (6.94131,2.73677) -- (6.94205,2.73525) -- (6.9428,2.73373) -- (6.94355,2.73221) -- (6.94429,2.73069) -- (6.94504,2.72917) -- (6.94578,2.72765) --
 (6.94653,2.72613) -- (6.94728,2.72462) -- (6.94802,2.7231) -- (6.94877,2.72158) -- (6.94951,2.72007) -- (6.95026,2.71856) -- (6.951,2.71704) -- (6.95175,2.71552) -- (6.9525,2.71401) -- (6.95324,2.71249) -- (6.95399,2.71098) -- (6.95473,2.70946) --
 (6.95548,2.70795) -- (6.95622,2.70643) -- (6.95697,2.70492) -- (6.95772,2.7034) -- (6.95846,2.70189) -- (6.95921,2.70037) -- (6.95995,2.69887) -- (6.9607,2.69736) -- (6.96145,2.69584) -- (6.96219,2.69433) -- (6.96294,2.69281) -- (6.96368,2.6913) --
 (6.96443,2.68978) -- (6.96518,2.68827) -- (6.96592,2.68675) -- (6.96667,2.68524) -- (6.96741,2.68372) -- (6.96816,2.68219) -- (6.9689,2.68066) -- (6.96965,2.67914) -- (6.9704,2.67761) -- (6.97114,2.67611) -- (6.97189,2.67462) -- (6.97263,2.67275) --
 (6.97338,2.6712) -- (6.97412,2.66938) -- (6.97487,2.66784) -- (6.97562,2.66629) -- (6.97636,2.66477) -- (6.97711,2.66326) -- (6.97785,2.66177) -- (6.9786,2.6603) -- (6.97935,2.65883) -- (6.98009,2.65735) -- (6.98084,2.65588) -- (6.98158,2.6544) --
 (6.98233,2.65294) -- (6.98308,2.65145) -- (6.98382,2.64997) -- (6.98457,2.64851) -- (6.98531,2.64704) -- (6.98606,2.64557) -- (6.9868,2.6441) -- (6.98755,2.64263) -- (6.9883,2.64116) -- (6.98904,2.6397) -- (6.98979,2.63823) -- (6.99053,2.63676) --
 (6.99128,2.63529) -- (6.99202,2.63382) -- (6.99277,2.63236) -- (6.99352,2.63089) -- (6.99426,2.62942) -- (6.99501,2.62796) -- (6.99575,2.62649) -- (6.9965,2.62502) -- (6.99725,2.62356) -- (6.99799,2.62209) -- (6.99874,2.62063) -- (6.99948,2.61916)
 -- (7.00023,2.6177) -- (7.00098,2.61623) -- (7.00172,2.61476) -- (7.00247,2.6133) -- (7.00321,2.61183) -- (7.00396,2.61037) -- (7.0047,2.6089) -- (7.00545,2.60744) -- (7.0062,2.60597) -- (7.00694,2.60451) -- (7.00769,2.60306) -- (7.00843,2.60159) --
 (7.00918,2.60013) -- (7.00992,2.59867) -- (7.01067,2.5972) -- (7.01142,2.59574) -- (7.01216,2.59427) -- (7.01291,2.5928) -- (7.01365,2.59126) -- (7.0144,2.58977) -- (7.01515,2.58828) -- (7.01589,2.58663) -- (7.01664,2.58516) -- (7.01738,2.5837) --
 (7.01813,2.58224) -- (7.01888,2.58078) -- (7.01962,2.57932) -- (7.02037,2.57785) -- (7.02111,2.57639) -- (7.02186,2.57493) -- (7.0226,2.57347) -- (7.02335,2.57201) -- (7.0241,2.57054) -- (7.02484,2.56908) -- (7.02559,2.56762) -- (7.02633,2.56616) --
 (7.02708,2.5647) -- (7.02782,2.56324) -- (7.02857,2.56178) -- (7.02932,2.56032) -- (7.03006,2.55886) -- (7.03081,2.5574) -- (7.03155,2.55594) -- (7.0323,2.55449) -- (7.03305,2.55303) -- (7.03379,2.55157) -- (7.03454,2.55011) -- (7.03528,2.54865) --
 (7.03603,2.54719) -- (7.03678,2.54569) -- (7.03752,2.54422) -- (7.03827,2.54274) -- (7.03901,2.54126) -- (7.03976,2.53978) -- (7.0405,2.5383) -- (7.04125,2.53682) -- (7.042,2.53535) -- (7.04274,2.53387) -- (7.04349,2.5324) -- (7.04423,2.53092) --
 (7.04498,2.52945) -- (7.04572,2.52798) -- (7.04647,2.52651) -- (7.04722,2.52503) -- (7.04796,2.52356) -- (7.04871,2.52209) -- (7.04945,2.52062) -- (7.0502,2.51916) -- (7.05095,2.51769) -- (7.05169,2.51622) -- (7.05244,2.51475) -- (7.05318,2.51329)
 -- (7.05393,2.51182) -- (7.05468,2.51036) -- (7.05542,2.50889) -- (7.05617,2.50743) -- (7.05691,2.50597) -- (7.05766,2.50453) -- (7.0584,2.50311) -- (7.05915,2.50168) -- (7.0599,2.50026) -- (7.06064,2.49884) -- (7.06139,2.49741) -- (7.06213,2.49599)
 -- (7.06288,2.49457) -- (7.06362,2.49315) -- (7.06437,2.49173) -- (7.06512,2.4903) -- (7.06586,2.48888) -- (7.06661,2.48746) -- (7.06735,2.48604) -- (7.0681,2.48463) -- (7.06885,2.48321) -- (7.06959,2.48179) -- (7.07034,2.48038) -- (7.07108,2.47896)
 -- (7.07183,2.47755) -- (7.07258,2.47613) -- (7.07332,2.47472) -- (7.07407,2.47331) -- (7.07481,2.4719) -- (7.07556,2.47043) -- (7.0763,2.46902) -- (7.07705,2.4676) -- (7.0778,2.46619) -- (7.07854,2.46477) -- (7.07929,2.46336) -- (7.08003,2.46195)
 -- (7.08078,2.46054) -- (7.08152,2.45913) -- (7.08227,2.45772) -- (7.08302,2.45631) -- (7.08376,2.4549) -- (7.08451,2.45349) -- (7.08525,2.45209) -- (7.086,2.45068) -- (7.08675,2.44927) -- (7.08749,2.44787) -- (7.08824,2.44646) -- (7.08898,2.44506)
 -- (7.08973,2.44366) -- (7.09048,2.44226) -- (7.09122,2.44085) -- (7.09197,2.43922) -- (7.09271,2.4378) -- (7.09346,2.43639) -- (7.0942,2.43498) -- (7.09495,2.43357) -- (7.0957,2.43215) -- (7.09644,2.43074) -- (7.09719,2.42881) -- (7.09793,2.42742)
 -- (7.09868,2.42603) -- (7.09942,2.42464) -- (7.10017,2.42325) -- (7.10092,2.42186) -- (7.10166,2.42048) -- (7.10241,2.41908) -- (7.10315,2.41769) -- (7.1039,2.41629) -- (7.10465,2.4149) -- (7.10539,2.41351) -- (7.10614,2.41213) -- (7.10688,2.41073)
 -- (7.10763,2.40935) -- (7.10838,2.40796) -- (7.10912,2.40657) -- (7.10987,2.40519) -- (7.11061,2.40379) -- (7.11136,2.40241) -- (7.1121,2.40101) -- (7.11285,2.39963) -- (7.1136,2.39823) -- (7.11434,2.39684) -- (7.11509,2.39545) -- (7.11583,2.39406)
 -- (7.11658,2.39267) -- (7.11732,2.39129) -- (7.11807,2.38989) -- (7.11882,2.38851) -- (7.11956,2.38712) -- (7.12031,2.38574) -- (7.12105,2.38434) -- (7.1218,2.38296) -- (7.12255,2.38157) -- (7.12329,2.38018) -- (7.12404,2.3788) -- (7.12478,2.37741)
 -- (7.12553,2.37602) -- (7.12628,2.37463) -- (7.12702,2.37325) -- (7.12777,2.37186) -- (7.12851,2.37048) -- (7.12926,2.36909) -- (7.13,2.3677) -- (7.13075,2.36632) -- (7.1315,2.36493) -- (7.13224,2.36351) -- (7.13299,2.36212) -- (7.13373,2.36067) --
 (7.13448,2.3593) -- (7.13522,2.35792) -- (7.13597,2.35654) -- (7.13672,2.35516) -- (7.13746,2.35377) -- (7.13821,2.3524) -- (7.13895,2.35102) -- (7.1397,2.34938) -- (7.14045,2.34797) -- (7.14119,2.34657) -- (7.14194,2.3452) -- (7.14268,2.34383) --
 (7.14343,2.34246) -- (7.14418,2.34109) -- (7.14492,2.33973) -- (7.14567,2.33837) -- (7.14641,2.337) -- (7.14716,2.33564) -- (7.1479,2.33428) -- (7.14865,2.33292) -- (7.1494,2.33156) -- (7.15014,2.3302) -- (7.15089,2.32884) -- (7.15163,2.32748) --
 (7.15238,2.32612) -- (7.15312,2.32477) -- (7.15387,2.32341) -- (7.15462,2.32205) -- (7.15536,2.3207) -- (7.15611,2.31935) -- (7.15685,2.31799) -- (7.1576,2.31664) -- (7.15835,2.31529) -- (7.15909,2.31394) -- (7.15984,2.31259) -- (7.16058,2.31124) --
 (7.16133,2.30989) -- (7.16208,2.30854) -- (7.16282,2.30719) -- (7.16357,2.30585) -- (7.16431,2.3045) -- (7.16506,2.30315) -- (7.1658,2.30181) -- (7.16655,2.30048) -- (7.1673,2.29913) -- (7.16804,2.29779) -- (7.16879,2.29645) -- (7.16953,2.2951) --
 (7.17028,2.29376) -- (7.17102,2.29241) -- (7.17177,2.29107) -- (7.17252,2.28972) -- (7.17326,2.28838) -- (7.17401,2.28704) -- (7.17475,2.28567) -- (7.1755,2.28433) -- (7.17625,2.28298) -- (7.17699,2.28164) -- (7.17774,2.2803) -- (7.17848,2.27896) --
 (7.17923,2.27762) -- (7.17998,2.27628) -- (7.18072,2.27494) -- (7.18147,2.27352) -- (7.18221,2.27219) -- (7.18296,2.27085) -- (7.1837,2.2695) -- (7.18445,2.26816) -- (7.1852,2.26682) -- (7.18594,2.26548) -- (7.18669,2.26414) -- (7.18743,2.2628) --
 (7.18818,2.26147) -- (7.18892,2.26013) -- (7.18967,2.25879) -- (7.19042,2.25746) -- (7.19116,2.25612) -- (7.19191,2.25479) -- (7.19265,2.25346) -- (7.1934,2.25214) -- (7.19415,2.25084) -- (7.19489,2.24952) -- (7.19564,2.24821) -- (7.19638,2.2469) --
 (7.19713,2.24558) -- (7.19788,2.24427) -- (7.19862,2.24295) -- (7.19937,2.24164) -- (7.20011,2.24033) -- (7.20086,2.23901) -- (7.2016,2.2377) -- (7.20235,2.23638) -- (7.2031,2.23507) -- (7.20384,2.23375) -- (7.20459,2.23243) -- (7.20533,2.2311) --
 (7.20608,2.22978) -- (7.20682,2.22845) -- (7.20757,2.22712) -- (7.20832,2.22579) -- (7.20906,2.22447) -- (7.20981,2.22314) -- (7.21055,2.22182) -- (7.2113,2.2204) -- (7.21205,2.21909) -- (7.21279,2.21776) -- (7.21354,2.21643) -- (7.21428,2.2151) --
 (7.21503,2.21379) -- (7.21578,2.21246) -- (7.21652,2.21115) -- (7.21727,2.20983) -- (7.21801,2.20851) -- (7.21876,2.20719) -- (7.2195,2.20587) -- (7.22025,2.20456) -- (7.221,2.20325) -- (7.22174,2.20194) -- (7.22249,2.20064) -- (7.22323,2.19933) --
 (7.22398,2.19802) -- (7.22473,2.1967) -- (7.22547,2.1954) -- (7.22622,2.19408) -- (7.22696,2.19276) -- (7.22771,2.19146) -- (7.22845,2.19014) -- (7.2292,2.18882) -- (7.22995,2.1875) -- (7.23069,2.18619) -- (7.23144,2.18488) -- (7.23218,2.18356) --
 (7.23293,2.18226) -- (7.23368,2.18095) -- (7.23442,2.17967) -- (7.23517,2.17839) -- (7.23591,2.17712) -- (7.23666,2.17584) -- (7.2374,2.17457) -- (7.23815,2.17323) -- (7.2389,2.17195) -- (7.23964,2.17068) -- (7.24039,2.1694) -- (7.24113,2.16813) --
 (7.24188,2.16685) -- (7.24263,2.16558) -- (7.24337,2.1643) -- (7.24412,2.16303) -- (7.24486,2.16176) -- (7.24561,2.16049) -- (7.24635,2.15922) -- (7.2471,2.15795) -- (7.24785,2.15668) -- (7.24859,2.15541) -- (7.24934,2.15414) -- (7.25008,2.15287) --
 (7.25083,2.15161) -- (7.25157,2.15034) -- (7.25232,2.14908) -- (7.25307,2.14781) -- (7.25381,2.14655) -- (7.25456,2.14517) -- (7.2553,2.1439) -- (7.25605,2.14263) -- (7.2568,2.14136) -- (7.25754,2.14009) -- (7.25829,2.13883) -- (7.25903,2.13756) --
 (7.25978,2.13629) -- (7.26053,2.13503) -- (7.26127,2.13376) -- (7.26202,2.13249) -- (7.26276,2.13123) -- (7.26351,2.12997) -- (7.26425,2.1287) -- (7.265,2.12744) -- (7.26575,2.12618) -- (7.26649,2.12492) -- (7.26724,2.12366) -- (7.26798,2.1224) --
 (7.26873,2.12114) -- (7.26947,2.11989) -- (7.27022,2.11863) -- (7.27097,2.11737) -- (7.27171,2.11612) -- (7.27246,2.11486) -- (7.2732,2.11361) -- (7.27395,2.11235) -- (7.2747,2.1111) -- (7.27544,2.10985) -- (7.27619,2.1086) -- (7.27693,2.10736) --
 (7.27768,2.10613) -- (7.27843,2.1049) -- (7.27917,2.10366) -- (7.27992,2.10242) -- (7.28066,2.10118) -- (7.28141,2.09993) -- (7.28215,2.09868) -- (7.2829,2.09745) -- (7.28365,2.0962) -- (7.28439,2.09495) -- (7.28514,2.0937) -- (7.28588,2.09246) --
 (7.28663,2.09122) -- (7.28737,2.08997) -- (7.28812,2.08872) -- (7.28887,2.08747) -- (7.28961,2.08623) -- (7.29036,2.08498) -- (7.2911,2.0837) -- (7.29185,2.08245) -- (7.2926,2.0812) -- (7.29334,2.07995) -- (7.29409,2.0787) -- (7.29483,2.07746) --
 (7.29558,2.07621) -- (7.29633,2.07496) -- (7.29707,2.07372) -- (7.29782,2.07247) -- (7.29856,2.07123) -- (7.29931,2.06998) -- (7.30005,2.06874) -- (7.3008,2.0675) -- (7.30155,2.06626) -- (7.30229,2.06501) -- (7.30304,2.06377) -- (7.30378,2.06253) --
 (7.30453,2.0613) -- (7.30527,2.06006) -- (7.30602,2.05882) -- (7.30677,2.05758) -- (7.30751,2.05635) -- (7.30826,2.05511) -- (7.309,2.05388) -- (7.30975,2.05264) -- (7.3105,2.05128) -- (7.31124,2.05005) -- (7.31199,2.0488) -- (7.31273,2.04756) --
 (7.31348,2.04632) -- (7.31423,2.04508) -- (7.31497,2.04384) -- (7.31572,2.04261) -- (7.31646,2.04137) -- (7.31721,2.04013) -- (7.31795,2.0389) -- (7.3187,2.03769) -- (7.31945,2.0365) -- (7.32019,2.0353) -- (7.32094,2.03411) -- (7.32168,2.03291) --
 (7.32243,2.03171) -- (7.32317,2.03052) -- (7.32392,2.02932) -- (7.32467,2.02813) -- (7.32541,2.02693) -- (7.32616,2.02574) -- (7.3269,2.02455) -- (7.32765,2.02336) -- (7.3284,2.02217) -- (7.32914,2.02098) -- (7.32989,2.01979) -- (7.33063,2.0186) --
 (7.33138,2.01741) -- (7.33213,2.01623) -- (7.33287,2.01504) -- (7.33362,2.01386) -- (7.33436,2.01267) -- (7.33511,2.01149) -- (7.33585,2.0103) -- (7.3366,2.00912) -- (7.33735,2.00794) -- (7.33809,2.00676) -- (7.33884,2.00558) -- (7.33958,2.0044) --
 (7.34033,2.00322) -- (7.34107,2.00204) -- (7.34182,2.00086) -- (7.34257,1.99954) -- (7.34331,1.99834) -- (7.34406,1.99716) -- (7.3448,1.99597) -- (7.34555,1.99479) -- (7.3463,1.99361) -- (7.34704,1.99231) -- (7.34779,1.99116) -- (7.34853,1.98999) --
 (7.34928,1.98882) -- (7.35003,1.98766) -- (7.35077,1.98649) -- (7.35152,1.98531) -- (7.35226,1.98414) -- (7.35301,1.98297) -- (7.35375,1.9818) -- (7.3545,1.98064) -- (7.35525,1.97948) -- (7.35599,1.97831) -- (7.35674,1.97716) -- (7.35748,1.976) --
 (7.35823,1.97484) -- (7.35897,1.97367) -- (7.35972,1.9725) -- (7.36047,1.97133) -- (7.36121,1.97018) -- (7.36196,1.96903) -- (7.3627,1.96787) -- (7.36345,1.9667) -- (7.3642,1.96552) -- (7.36494,1.96435) -- (7.36569,1.96318) -- (7.36643,1.96201) --
 (7.36718,1.96084) -- (7.36793,1.95967) -- (7.36867,1.9585) -- (7.36942,1.95732) -- (7.37016,1.95611) -- (7.37091,1.95493) -- (7.37165,1.95372) -- (7.3724,1.95255) -- (7.37315,1.95138) -- (7.37389,1.95021) -- (7.37464,1.94906) -- (7.37538,1.94788) --
 (7.37613,1.94673) -- (7.37687,1.94556) -- (7.37762,1.94439) -- (7.37837,1.94321) -- (7.37911,1.94205) -- (7.37986,1.94088) -- (7.3806,1.93971) -- (7.38135,1.93854) -- (7.3821,1.93737) -- (7.38284,1.9362) -- (7.38359,1.93504) -- (7.38433,1.93387) --
 (7.38508,1.93271) -- (7.38583,1.93154) -- (7.38657,1.93038) -- (7.38732,1.92923) -- (7.38806,1.92808) -- (7.38881,1.92693) -- (7.38955,1.92578) -- (7.3903,1.92463) -- (7.39105,1.92348) -- (7.39179,1.92232) -- (7.39254,1.92116) -- (7.39328,1.92) --
 (7.39403,1.91884) -- (7.39477,1.91768) -- (7.39552,1.91651) -- (7.39627,1.91535) -- (7.39701,1.91419) -- (7.39776,1.91304) -- (7.3985,1.91188) -- (7.39925,1.91072) -- (7.4,1.90956) -- (7.40074,1.90844) -- (7.40149,1.90732) -- (7.40223,1.9062) --
 (7.40298,1.90508) -- (7.40373,1.90397) -- (7.40447,1.90285) -- (7.40522,1.90174) -- (7.40596,1.90062) -- (7.40671,1.89951) -- (7.40745,1.8984) -- (7.4082,1.89729) -- (7.40895,1.89617) -- (7.40969,1.89506) -- (7.41044,1.89392) -- (7.41118,1.8928) --
 (7.41193,1.89169) -- (7.41267,1.89058) -- (7.41342,1.88947) -- (7.41417,1.88836) -- (7.41491,1.88725) -- (7.41566,1.88614) -- (7.4164,1.88503) -- (7.41715,1.88392) -- (7.4179,1.88281) -- (7.41864,1.88171) -- (7.41939,1.8806) -- (7.42013,1.8795) --
 (7.42088,1.87839) -- (7.42163,1.87729) -- (7.42237,1.87619) -- (7.42312,1.87508) -- (7.42386,1.87398) -- (7.42461,1.87288) -- (7.42535,1.87178) -- (7.4261,1.87068) -- (7.42685,1.86958) -- (7.42759,1.86848) -- (7.42834,1.86739) -- (7.42908,1.86626)
 -- (7.42983,1.86516) -- (7.43057,1.86406) -- (7.43132,1.86296) -- (7.43207,1.86186) -- (7.43281,1.86076) -- (7.43356,1.85967) -- (7.4343,1.85857) -- (7.43505,1.85747) -- (7.4358,1.85638) -- (7.43654,1.85529) -- (7.43729,1.85419) -- (7.43803,1.8531)
 -- (7.43878,1.85201) -- (7.43953,1.85092) -- (7.44027,1.84983) -- (7.44102,1.84874) -- (7.44176,1.84765) -- (7.44251,1.84647) -- (7.44325,1.84539) -- (7.444,1.8443) -- (7.44475,1.8432) -- (7.44549,1.84211) -- (7.44624,1.84102) -- (7.44698,1.83993)
 -- (7.44773,1.83884) -- (7.44847,1.83776) -- (7.44922,1.83667) -- (7.44997,1.83558) -- (7.45071,1.8345) -- (7.45146,1.83341) -- (7.4522,1.83232) -- (7.45295,1.83124) -- (7.4537,1.83016) -- (7.45444,1.82907) -- (7.45519,1.82799) -- (7.45593,1.82691)
 -- (7.45668,1.82583) -- (7.45743,1.82475) -- (7.45817,1.82367) -- (7.45892,1.82246) -- (7.45966,1.82137) -- (7.46041,1.82029) -- (7.46115,1.8192) -- (7.4619,1.81812) -- (7.46265,1.81703) -- (7.46339,1.81595) -- (7.46414,1.81487) -- (7.46488,1.81378)
 -- (7.46563,1.8127) -- (7.46637,1.81162) -- (7.46712,1.81054) -- (7.46787,1.80946) -- (7.46861,1.80839) -- (7.46936,1.80731) -- (7.4701,1.80623) -- (7.47085,1.80516) -- (7.4716,1.80406) -- (7.47234,1.80298) -- (7.47309,1.80191) -- (7.47383,1.80083)
 -- (7.47458,1.79976) -- (7.47533,1.79866) -- (7.47607,1.79759) -- (7.47682,1.79651) -- (7.47756,1.79544) -- (7.47831,1.79437) -- (7.47905,1.79334) -- (7.4798,1.7923) -- (7.48055,1.79127) -- (7.48129,1.79023) -- (7.48204,1.7892) -- (7.48278,1.78817)
 -- (7.48353,1.78714) -- (7.48427,1.7861) -- (7.48502,1.78507) -- (7.48577,1.78405) -- (7.48651,1.78302) -- (7.48726,1.78199) -- (7.488,1.78096) -- (7.48875,1.77993) -- (7.4895,1.77891) -- (7.49024,1.77788) -- (7.49099,1.77686) -- (7.49173,1.77584)
 -- (7.49248,1.77481) -- (7.49323,1.77379) -- (7.49397,1.77277) -- (7.49472,1.77175) -- (7.49546,1.77073) -- (7.49621,1.76957) -- (7.49695,1.76855) -- (7.4977,1.76752) -- (7.49845,1.7665) -- (7.49919,1.76548) -- (7.49994,1.76447) -- (7.50068,1.76344)
 -- (7.50143,1.76244) -- (7.50217,1.76142) -- (7.50292,1.76041) -- (7.50367,1.75939) -- (7.50441,1.75838) -- (7.50516,1.75736) -- (7.5059,1.75636) -- (7.50665,1.75534) -- (7.5074,1.75433) -- (7.50814,1.75333) -- (7.50889,1.75231) -- (7.50963,1.75123)
 -- (7.51038,1.75021) -- (7.51113,1.74919) -- (7.51187,1.74817) -- (7.51262,1.74716) -- (7.51336,1.74614) -- (7.51411,1.74512) -- (7.51485,1.74411) -- (7.5156,1.74309) -- (7.51635,1.74208) -- (7.51709,1.74106) -- (7.51784,1.74005) --
 (7.51858,1.73904) -- (7.51933,1.73802) -- (7.52007,1.73701) -- (7.52082,1.736) -- (7.52157,1.73499) -- (7.52231,1.73398) -- (7.52306,1.73298) -- (7.5238,1.73197) -- (7.52455,1.73096) -- (7.5253,1.72996) -- (7.52604,1.72895) -- (7.52679,1.72795) --
 (7.52753,1.72694) -- (7.52828,1.72594) -- (7.52903,1.72494) -- (7.52977,1.72394) -- (7.53052,1.72293) -- (7.53126,1.72193) -- (7.53201,1.72093) -- (7.53275,1.71993) -- (7.5335,1.71891) -- (7.53425,1.71791) -- (7.53499,1.71691) -- (7.53574,1.71591)
 -- (7.53648,1.71491) -- (7.53723,1.71391) -- (7.53797,1.71291) -- (7.53872,1.71191) -- (7.53947,1.71091) -- (7.54021,1.70992) -- (7.54096,1.70881) -- (7.5417,1.70782) -- (7.54245,1.70682) -- (7.5432,1.70581) -- (7.54394,1.70481) -- (7.54469,1.70381)
 -- (7.54543,1.70281) -- (7.54618,1.70181) -- (7.54693,1.70082) -- (7.54767,1.69982) -- (7.54842,1.69882) -- (7.54916,1.69783) -- (7.54991,1.69683) -- (7.55065,1.69584) -- (7.5514,1.69484) -- (7.55215,1.69385) -- (7.55289,1.69286) --
 (7.55364,1.69186) -- (7.55438,1.69091) -- (7.55513,1.68996) -- (7.55587,1.689) -- (7.55662,1.68805) -- (7.55737,1.6871) -- (7.55811,1.68615) -- (7.55886,1.6852) -- (7.5596,1.68425) -- (7.56035,1.68331) -- (7.5611,1.68236) -- (7.56184,1.68141) --
 (7.56259,1.68047) -- (7.56333,1.67952) -- (7.56408,1.67858) -- (7.56483,1.67764) -- (7.56557,1.67658) -- (7.56632,1.67563) -- (7.56706,1.67469) -- (7.56781,1.67374) -- (7.56855,1.67279) -- (7.5693,1.67185) -- (7.57005,1.6709) -- (7.57079,1.66996) --
 (7.57154,1.66901) -- (7.57228,1.66807) -- (7.57303,1.66713) -- (7.57377,1.66619) -- (7.57452,1.66525) -- (7.57527,1.66431) -- (7.57601,1.66337) -- (7.57676,1.66243) -- (7.5775,1.66149) -- (7.57825,1.66055) -- (7.579,1.65962) -- (7.57974,1.65868) --
 (7.58049,1.65775) -- (7.58123,1.65681) -- (7.58198,1.65588) -- (7.58273,1.65495) -- (7.58347,1.65401) -- (7.58422,1.65308) -- (7.58496,1.65215) -- (7.58571,1.65122) -- (7.58645,1.65029) -- (7.5872,1.64937) -- (7.58795,1.64839) -- (7.58869,1.64747)
 -- (7.58944,1.64654) -- (7.59018,1.6456) -- (7.59093,1.64467) -- (7.59167,1.64374) -- (7.59242,1.64281) -- (7.59317,1.64188) -- (7.59391,1.64095) -- (7.59466,1.64002) -- (7.5954,1.63909) -- (7.59615,1.63817) -- (7.5969,1.63724) -- (7.59764,1.63632)
 -- (7.59839,1.63539) -- (7.59913,1.63447) -- (7.59988,1.63355) -- (7.60063,1.63263) -- (7.60137,1.63171) -- (7.60212,1.63078) -- (7.60286,1.62979) -- (7.60361,1.62886) -- (7.60435,1.62794) -- (7.6051,1.62701) -- (7.60585,1.62609) --
 (7.60659,1.62516) -- (7.60734,1.62424) -- (7.60808,1.62332) -- (7.60883,1.62239) -- (7.60957,1.62147) -- (7.61032,1.62055) -- (7.61107,1.61963) -- (7.61181,1.61871) -- (7.61256,1.6178) -- (7.6133,1.61688) -- (7.61405,1.61596) -- (7.6148,1.61505) --
 (7.61554,1.61413) -- (7.61629,1.61322) -- (7.61703,1.6123) -- (7.61778,1.61139) -- (7.61853,1.61048) -- (7.61927,1.60957) -- (7.62002,1.60868) -- (7.62076,1.60774) -- (7.62151,1.60683) -- (7.62225,1.60591) -- (7.623,1.605) -- (7.62375,1.60408) --
 (7.62449,1.60317) -- (7.62524,1.60226) -- (7.62598,1.60135) -- (7.62673,1.60043) -- (7.62747,1.59952) -- (7.62822,1.59865) -- (7.62897,1.59778) -- (7.62971,1.59691) -- (7.63046,1.59604) -- (7.6312,1.59517) -- (7.63195,1.5943) -- (7.6327,1.59343) --
 (7.63344,1.59257) -- (7.63419,1.5917) -- (7.63493,1.59084) -- (7.63568,1.58997) -- (7.63643,1.58911) -- (7.63717,1.58822) -- (7.63792,1.58736) -- (7.63866,1.58649) -- (7.63941,1.58563) -- (7.64015,1.58476) -- (7.6409,1.5839) -- (7.64165,1.58304) --
 (7.64239,1.58218) -- (7.64314,1.58131) -- (7.64388,1.58045) -- (7.64463,1.57959) -- (7.64537,1.57874) -- (7.64612,1.57788) -- (7.64687,1.57702) -- (7.64761,1.57616) -- (7.64836,1.57531) -- (7.6491,1.57445) -- (7.64985,1.57357) -- (7.6506,1.57271) --
 (7.65134,1.57185) -- (7.65209,1.571) -- (7.65283,1.57014) -- (7.65358,1.56929) -- (7.65433,1.56843) -- (7.65507,1.56758) -- (7.65582,1.56672) -- (7.65656,1.56587) -- (7.65731,1.56502) -- (7.65805,1.56417) -- (7.6588,1.56332) -- (7.65955,1.56247) --
 (7.66029,1.56162) -- (7.66104,1.56077) -- (7.66178,1.55989) -- (7.66253,1.55905) -- (7.66328,1.5582) -- (7.66402,1.55735) -- (7.66477,1.5565) -- (7.66551,1.55565) -- (7.66626,1.55481) -- (7.667,1.55396) -- (7.66775,1.55311) -- (7.6685,1.55227) --
 (7.66924,1.55142) -- (7.66999,1.55058) -- (7.67073,1.54974) -- (7.67148,1.54889) -- (7.67222,1.54805) -- (7.67297,1.54721) -- (7.67372,1.54637) -- (7.67446,1.5455) -- (7.67521,1.54466) -- (7.67595,1.54381) -- (7.6767,1.54297) -- (7.67745,1.54213) --
 (7.67819,1.54129) -- (7.67894,1.54045) -- (7.67968,1.53961) -- (7.68043,1.53877) -- (7.68118,1.53794) -- (7.68192,1.5371) -- (7.68267,1.53626) -- (7.68341,1.53543) -- (7.68416,1.53459) -- (7.6849,1.53376) -- (7.68565,1.53293) -- (7.6864,1.53209) --
 (7.68714,1.53124) -- (7.68789,1.53041) -- (7.68863,1.52957) -- (7.68938,1.52874) -- (7.69012,1.52791) -- (7.69087,1.52708) -- (7.69162,1.52625) -- (7.69236,1.52542) -- (7.69311,1.52459) -- (7.69385,1.52376) -- (7.6946,1.52294) -- (7.69535,1.52211)
 -- (7.69609,1.52127) -- (7.69684,1.52043) -- (7.69758,1.51961) -- (7.69833,1.51882) -- (7.69908,1.51803) -- (7.69982,1.51725) -- (7.70057,1.51646) -- (7.70131,1.51567) -- (7.70206,1.51488) -- (7.7028,1.5141) -- (7.70355,1.51331) -- (7.7043,1.51253)
 -- (7.70504,1.51174) -- (7.70579,1.51096) -- (7.70653,1.51018) -- (7.70728,1.50939) -- (7.70802,1.5086) -- (7.70877,1.50783) -- (7.70952,1.50705) -- (7.71026,1.50626) -- (7.71101,1.50548) -- (7.71175,1.5047) -- (7.7125,1.50392) -- (7.71325,1.50315)
 -- (7.71399,1.50237) -- (7.71474,1.50159) -- (7.71548,1.50079) -- (7.71623,1.50002) -- (7.71698,1.49924) -- (7.71772,1.49846) -- (7.71847,1.49768) -- (7.71921,1.49691) -- (7.71996,1.49613) -- (7.7207,1.49536) -- (7.72145,1.49459) -- (7.7222,1.49381)
 -- (7.72294,1.49304) -- (7.72369,1.49227) -- (7.72443,1.4915) -- (7.72518,1.49073) -- (7.72592,1.48996) -- (7.72667,1.48919) -- (7.72742,1.4884) -- (7.72816,1.48763) -- (7.72891,1.48686) -- (7.72965,1.48609) -- (7.7304,1.48532) -- (7.73115,1.48456)
 -- (7.73189,1.48379) -- (7.73264,1.48302) -- (7.73338,1.48226) -- (7.73413,1.48149) -- (7.73488,1.48073) -- (7.73562,1.47996) -- (7.73637,1.4792) -- (7.73711,1.47844) -- (7.73786,1.47768) -- (7.7386,1.47692) -- (7.73935,1.47613) -- (7.7401,1.47537)
 -- (7.74084,1.47461) -- (7.74159,1.47385) -- (7.74233,1.47309) -- (7.74308,1.47233) -- (7.74382,1.47157) -- (7.74457,1.47081) -- (7.74532,1.47005) -- (7.74606,1.4693) -- (7.74681,1.46854) -- (7.74755,1.46778) -- (7.7483,1.46703) -- (7.74905,1.46628)
 -- (7.74979,1.46549) -- (7.75054,1.46475) -- (7.75128,1.46399) -- (7.75203,1.46323) -- (7.75278,1.46248) -- (7.75352,1.46173) -- (7.75427,1.46097) -- (7.75501,1.46022) -- (7.75576,1.45947) -- (7.7565,1.45872) -- (7.75725,1.45797) -- (7.758,1.45722)
 -- (7.75874,1.45647) -- (7.75949,1.45572) -- (7.76023,1.45496) -- (7.76098,1.45421) -- (7.76172,1.45346) -- (7.76247,1.45271) -- (7.76322,1.45197) -- (7.76396,1.45122) -- (7.76471,1.45048) -- (7.76545,1.44978) -- (7.7662,1.44907) --
 (7.76695,1.44837) -- (7.76769,1.44766) -- (7.76844,1.44696) -- (7.76918,1.44625) -- (7.76993,1.44555) -- (7.77068,1.44484) -- (7.77142,1.44414) -- (7.77217,1.44344) -- (7.77291,1.44273) -- (7.77366,1.44203) -- (7.7744,1.44133) -- (7.77515,1.44063)
 -- (7.7759,1.43993) -- (7.77664,1.43923) -- (7.77739,1.43854) -- (7.77813,1.43784) -- (7.77888,1.43712) -- (7.77962,1.43641) -- (7.78037,1.43571) -- (7.78112,1.43502) -- (7.78186,1.43435) -- (7.78261,1.43366) -- (7.78335,1.43297) -- (7.7841,1.43228)
 -- (7.78485,1.43159) -- (7.78559,1.43089) -- (7.78634,1.4302) -- (7.78708,1.42951) -- (7.78783,1.42881) -- (7.78858,1.42812) -- (7.78932,1.42743) -- (7.79007,1.42674) -- (7.79081,1.42605) -- (7.79156,1.42536) -- (7.7923,1.42467) -- (7.79305,1.42397)
 -- (7.7938,1.42329) -- (7.79454,1.4226) -- (7.79529,1.42191) -- (7.79603,1.42123) -- (7.79678,1.42054) -- (7.79752,1.41986) -- (7.79827,1.41917) -- (7.79902,1.41849) -- (7.79976,1.4178) -- (7.80051,1.41711) -- (7.80125,1.41643) -- (7.802,1.41575) --
 (7.80275,1.41506) -- (7.80349,1.41439) -- (7.80424,1.4137) -- (7.80498,1.41302) -- (7.80573,1.41234) -- (7.80648,1.41166) -- (7.80722,1.41098) -- (7.80797,1.4103) -- (7.80871,1.40962) -- (7.80946,1.40894) -- (7.8102,1.40827) -- (7.81095,1.40759) --
 (7.8117,1.4069) -- (7.81244,1.40623) -- (7.81319,1.40555) -- (7.81393,1.40489) -- (7.81468,1.40421) -- (7.81542,1.40354) -- (7.81617,1.40286) -- (7.81692,1.40219) -- (7.81766,1.40151) -- (7.81841,1.40084) -- (7.81915,1.40016) -- (7.8199,1.39949) --
 (7.82065,1.39883) -- (7.82139,1.39815) -- (7.82214,1.39748) -- (7.82288,1.39681) -- (7.82363,1.39614) -- (7.82438,1.39547) -- (7.82512,1.39481) -- (7.82587,1.39414) -- (7.82661,1.39347) -- (7.82736,1.3928) -- (7.8281,1.39214) -- (7.82885,1.39147) --
 (7.8296,1.3908) -- (7.83034,1.39016) -- (7.83109,1.38954) -- (7.83183,1.38891) -- (7.83258,1.38829) -- (7.83332,1.38767) -- (7.83407,1.38705) -- (7.83482,1.38642) -- (7.83556,1.3858) -- (7.83631,1.38518) -- (7.83705,1.38456) -- (7.8378,1.38394) --
 (7.83855,1.38333) -- (7.83929,1.38271) -- (7.84004,1.38209) -- (7.84078,1.38147) -- (7.84153,1.38086) -- (7.84228,1.38024) -- (7.84302,1.37963) -- (7.84377,1.37901) -- (7.84451,1.3784) -- (7.84526,1.37779) -- (7.846,1.37717) -- (7.84675,1.37656) --
 (7.8475,1.37594) -- (7.84824,1.37533) -- (7.84899,1.37472) -- (7.84973,1.37411) -- (7.85048,1.3735) -- (7.85122,1.37289) -- (7.85197,1.37228) -- (7.85272,1.37167) -- (7.85346,1.37106) -- (7.85421,1.37045) -- (7.85495,1.36984) -- (7.8557,1.36923) --
 (7.85645,1.36863) -- (7.85719,1.36802) -- (7.85794,1.36741) -- (7.85868,1.36681) -- (7.85943,1.3662) -- (7.86018,1.3656) -- (7.86092,1.36499) -- (7.86167,1.36439) -- (7.86241,1.36379) -- (7.86316,1.36319) -- (7.8639,1.36258) -- (7.86465,1.36198) --
 (7.8654,1.36138) -- (7.86614,1.36077) -- (7.86689,1.36017) -- (7.86763,1.35958) -- (7.86838,1.35898) -- (7.86912,1.35838) -- (7.86987,1.35778) -- (7.87062,1.35719) -- (7.87136,1.35659) -- (7.87211,1.35599) -- (7.87285,1.3554) -- (7.8736,1.3548) --
 (7.87435,1.3542) -- (7.87509,1.35361) -- (7.87584,1.35302) -- (7.87658,1.35242) -- (7.87733,1.35183) -- (7.87808,1.35124) -- (7.87882,1.35063) -- (7.87957,1.35004) -- (7.88031,1.34944) -- (7.88106,1.34885) -- (7.8818,1.34827) -- (7.88255,1.34768) --
 (7.8833,1.3471) -- (7.88404,1.34652) -- (7.88479,1.34593) -- (7.88553,1.34534) -- (7.88628,1.34475) -- (7.88702,1.34417) -- (7.88777,1.34358) -- (7.88852,1.34299) -- (7.88926,1.34241) -- (7.89001,1.34182) -- (7.89075,1.34124) -- (7.8915,1.34065) --
 (7.89225,1.34007) -- (7.89299,1.33949) -- (7.89374,1.33893) -- (7.89448,1.33839) -- (7.89523,1.33785) -- (7.89598,1.33731) -- (7.89672,1.33677) -- (7.89747,1.33623) -- (7.89821,1.33569) -- (7.89896,1.33515) -- (7.8997,1.33461) -- (7.90045,1.33408)
 -- (7.9012,1.33354) -- (7.90194,1.333) -- (7.90269,1.33247) -- (7.90343,1.33194) -- (7.90418,1.3314) -- (7.90492,1.33087) -- (7.90567,1.33033) -- (7.90642,1.3298) -- (7.90716,1.32927) -- (7.90791,1.32874) -- (7.90865,1.32821) -- (7.9094,1.32767) --
 (7.91015,1.32715) -- (7.91089,1.32661) -- (7.91164,1.32609) -- (7.91238,1.32556) -- (7.91313,1.32503) -- (7.91388,1.3245) -- (7.91462,1.32397) -- (7.91537,1.32345) -- (7.91611,1.32292) -- (7.91686,1.32239) -- (7.9176,1.32187) -- (7.91835,1.32134) --
 (7.9191,1.32082) -- (7.91984,1.32029) -- (7.92059,1.31977) -- (7.92133,1.31925) -- (7.92208,1.31873) -- (7.92282,1.31821) -- (7.92357,1.31769) -- (7.92432,1.31716) -- (7.92506,1.31665) -- (7.92581,1.31612) -- (7.92655,1.31561) -- (7.9273,1.31508) --
 (7.92805,1.31457) -- (7.92879,1.31405) -- (7.92954,1.31353) -- (7.93028,1.31302) -- (7.93103,1.3125) -- (7.93178,1.31198) -- (7.93252,1.31147) -- (7.93327,1.31095) -- (7.93401,1.31044) -- (7.93476,1.30992) -- (7.9355,1.30941) -- (7.93625,1.30889) --
 (7.937,1.30839) -- (7.93774,1.30788) -- (7.93849,1.30737) -- (7.93923,1.30685) -- (7.93998,1.30634) -- (7.94072,1.30583) -- (7.94147,1.30532) -- (7.94222,1.30481) -- (7.94296,1.30431) -- (7.94371,1.3038) -- (7.94445,1.30329) -- (7.9452,1.30279) --
 (7.94595,1.30227) -- (7.94669,1.30177) -- (7.94744,1.30126) -- (7.94818,1.30076) -- (7.94893,1.30026) -- (7.94968,1.29975) -- (7.95042,1.29925) -- (7.95117,1.29875) -- (7.95191,1.29824) -- (7.95266,1.29774) -- (7.9534,1.29724) -- (7.95415,1.29674)
 -- (7.9549,1.29624) -- (7.95564,1.29575) -- (7.95639,1.29529) -- (7.95713,1.29483) -- (7.95788,1.29437) -- (7.95862,1.29392) -- (7.95937,1.29346) -- (7.96012,1.293) -- (7.96086,1.29255) -- (7.96161,1.29209) -- (7.96235,1.29164) -- (7.9631,1.29119)
 -- (7.96385,1.29074) -- (7.96459,1.29029) -- (7.96534,1.28983) -- (7.96608,1.28938) -- (7.96683,1.28893) -- (7.96758,1.28848) -- (7.96832,1.28803) -- (7.96907,1.28758) -- (7.96981,1.28714) -- (7.97056,1.28669) -- (7.9713,1.28624) --
 (7.97205,1.28579) -- (7.9728,1.28535) -- (7.97354,1.2849) -- (7.97429,1.28446) -- (7.97503,1.28401) -- (7.97578,1.28357) -- (7.97652,1.28312) -- (7.97727,1.28268) -- (7.97802,1.28223) -- (7.97876,1.28179) -- (7.97951,1.28135) -- (7.98025,1.28091) --
 (7.981,1.28047) -- (7.98175,1.28003) -- (7.98249,1.27958) -- (7.98324,1.27915) -- (7.98398,1.27871) -- (7.98473,1.27827) -- (7.98548,1.27783) -- (7.98622,1.27739) -- (7.98697,1.27696) -- (7.98771,1.27652) -- (7.98846,1.27608) -- (7.9892,1.27565) --
 (7.98995,1.27521) -- (7.9907,1.27478) -- (7.99144,1.27435) -- (7.99219,1.27392) -- (7.99293,1.27348) -- (7.99368,1.27305) -- (7.99442,1.27262) -- (7.99517,1.27218) -- (7.99592,1.27176) -- (7.99666,1.27132) -- (7.99741,1.2709) -- (7.99815,1.27046) --
 (7.9989,1.27004) -- (7.99965,1.26961) -- (8.00039,1.26918) -- (8.00114,1.26875) -- (8.00188,1.26833) -- (8.00263,1.2679) -- (8.00338,1.26747) -- (8.00412,1.26705) -- (8.00487,1.26662) -- (8.00561,1.2662) -- (8.00636,1.26578) -- (8.0071,1.26536) --
 (8.00785,1.26493) -- (8.0086,1.26451) -- (8.00934,1.26409) -- (8.01009,1.26367) -- (8.01083,1.26325) -- (8.01158,1.26283) -- (8.01233,1.26241) -- (8.01307,1.26199) -- (8.01382,1.26157) -- (8.01456,1.26115) -- (8.01531,1.26073) -- (8.01605,1.26032)
 -- (8.0168,1.2599) -- (8.01755,1.2595) -- (8.01829,1.25913) -- (8.01904,1.25876) -- (8.01978,1.25838) -- (8.02053,1.25801) -- (8.02127,1.25764) -- (8.02202,1.25727) -- (8.02277,1.2569) -- (8.02351,1.25653) -- (8.02426,1.25616) -- (8.025,1.25579) --
 (8.02575,1.25542) -- (8.0265,1.25506) -- (8.02724,1.25469) -- (8.02799,1.25432) -- (8.02873,1.25395) -- (8.02948,1.25359) -- (8.03022,1.25322) -- (8.03097,1.25286) -- (8.03172,1.25249) -- (8.03246,1.25213) -- (8.03321,1.25177) -- (8.03395,1.2514) --
 (8.0347,1.25104) -- (8.03545,1.25068) -- (8.03619,1.25032) -- (8.03694,1.24996) -- (8.03768,1.2496) -- (8.03843,1.24924) -- (8.03918,1.24888) -- (8.03992,1.24852) -- (8.04067,1.24816) -- (8.04141,1.2478) -- (8.04216,1.24745) -- (8.0429,1.24709) --
 (8.04365,1.24674) -- (8.0444,1.24638) -- (8.04514,1.24602) -- (8.04589,1.24567) -- (8.04663,1.24531) -- (8.04738,1.24496) -- (8.04813,1.24461) -- (8.04887,1.24426) -- (8.04962,1.24391) -- (8.05036,1.24355) -- (8.05111,1.2432) -- (8.05185,1.24286) --
 (8.0526,1.2425) -- (8.05335,1.24215) -- (8.05409,1.24181) -- (8.05484,1.24146) -- (8.05558,1.24111) -- (8.05633,1.24076) -- (8.05707,1.24042) -- (8.05782,1.24007) -- (8.05857,1.23972) -- (8.05931,1.23938) -- (8.06006,1.23904) -- (8.0608,1.23869) --
 (8.06155,1.23835) -- (8.0623,1.23801) -- (8.06304,1.23766) -- (8.06379,1.23732) -- (8.06453,1.23698) -- (8.06528,1.23664) -- (8.06602,1.2363) -- (8.06677,1.23596) -- (8.06752,1.23562) -- (8.06826,1.23528) -- (8.06901,1.23494) -- (8.06975,1.2346) --
 (8.0705,1.23427) -- (8.07125,1.23393) -- (8.07199,1.2336) -- (8.07274,1.23326) -- (8.07348,1.23292) -- (8.07423,1.23259) -- (8.07498,1.23226) -- (8.07572,1.23192) -- (8.07647,1.23159) -- (8.07721,1.23125) -- (8.07796,1.23092) -- (8.0787,1.2306) --
 (8.07945,1.23031) -- (8.0802,1.23002) -- (8.08094,1.22973) -- (8.08169,1.22945) -- (8.08243,1.22915) -- (8.08318,1.22887) -- (8.08393,1.22858) -- (8.08467,1.2283) -- (8.08542,1.22801) -- (8.08616,1.22773) -- (8.08691,1.22745) -- (8.08765,1.22716) --
 (8.0884,1.22688) -- (8.08915,1.2266) -- (8.08989,1.22632) -- (8.09064,1.22604) -- (8.09138,1.22576) -- (8.09213,1.22547) -- (8.09288,1.2252) -- (8.09362,1.22492) -- (8.09437,1.22464) -- (8.09511,1.22436) -- (8.09586,1.22408) -- (8.0966,1.2238) --
 (8.09735,1.22353) -- (8.0981,1.22325) -- (8.09884,1.22298) -- (8.09959,1.22271) -- (8.10033,1.22243) -- (8.10108,1.22216) -- (8.10182,1.22189) -- (8.10257,1.22161) -- (8.10332,1.22134) -- (8.10406,1.22107) -- (8.10481,1.2208) -- (8.10555,1.22053) --
 (8.1063,1.22026) -- (8.10705,1.21999) -- (8.10779,1.21972) -- (8.10854,1.21945) -- (8.10928,1.21918) -- (8.11003,1.21891) -- (8.11077,1.21865) -- (8.11152,1.21838) -- (8.11227,1.21812) -- (8.11301,1.21785) -- (8.11376,1.21759) -- (8.1145,1.21732) --
 (8.11525,1.21706) -- (8.116,1.2168) -- (8.11674,1.21653) -- (8.11749,1.21627) -- (8.11823,1.21601) -- (8.11898,1.21575) -- (8.11973,1.21549) -- (8.12047,1.21523) -- (8.12122,1.21497) -- (8.12196,1.21471) -- (8.12271,1.21445) -- (8.12345,1.21419) --
 (8.1242,1.21394) -- (8.12495,1.21368) -- (8.12569,1.21343) -- (8.12644,1.21317) -- (8.12718,1.21292) -- (8.12793,1.21266) -- (8.12868,1.21241) -- (8.12942,1.21215) -- (8.13017,1.2119) -- (8.13091,1.21165) -- (8.13166,1.2114) -- (8.1324,1.21114) --
 (8.13315,1.21089) -- (8.1339,1.21065) -- (8.13464,1.21039) -- (8.13539,1.21015) -- (8.13613,1.2099) -- (8.13688,1.20965) -- (8.13762,1.2094) -- (8.13837,1.20916) -- (8.13912,1.20891) -- (8.13986,1.20866) -- (8.14061,1.20845) -- (8.14135,1.20825) --
 (8.1421,1.20805) -- (8.14285,1.20784) -- (8.14359,1.20764) -- (8.14434,1.20744) -- (8.14508,1.20724) -- (8.14583,1.20704) -- (8.14657,1.20684) -- (8.14732,1.20664) -- (8.14807,1.20645) -- (8.14881,1.20625) -- (8.14956,1.20605) -- (8.1503,1.20586) --
 (8.15105,1.20566) -- (8.1518,1.20547) -- (8.15254,1.20527) -- (8.15329,1.20508) -- (8.15403,1.20488) -- (8.15478,1.20469) -- (8.15553,1.2045) -- (8.15627,1.2043) -- (8.15702,1.20412) -- (8.15776,1.20392) -- (8.15851,1.20373) -- (8.15925,1.20355) --
 (8.16,1.20335) -- (8.16075,1.20317) -- (8.16149,1.20298) -- (8.16224,1.20279) -- (8.16298,1.2026) -- (8.16373,1.20242) -- (8.16448,1.20223) -- (8.16522,1.20204) -- (8.16597,1.20186) -- (8.16671,1.20168) -- (8.16746,1.20149) -- (8.1682,1.20131) --
 (8.16895,1.20112) -- (8.1697,1.20094) -- (8.17044,1.20076) -- (8.17119,1.20058) -- (8.17193,1.2004) -- (8.17268,1.20022) -- (8.17342,1.20004) -- (8.17417,1.19986) -- (8.17492,1.19968) -- (8.17566,1.1995) -- (8.17641,1.19933) -- (8.17715,1.19915) --
 (8.1779,1.19897) -- (8.17865,1.1988) -- (8.17939,1.19862) -- (8.18014,1.19845) -- (8.18088,1.19827) -- (8.18163,1.1981) -- (8.18237,1.19792) -- (8.18312,1.19776) -- (8.18387,1.19758) -- (8.18461,1.19741) -- (8.18536,1.19724) -- (8.1861,1.19707) --
 (8.18685,1.1969) -- (8.1876,1.19673) -- (8.18834,1.19656) -- (8.18909,1.1964) -- (8.18983,1.19622) -- (8.19058,1.19606) -- (8.19133,1.19589) -- (8.19207,1.19573) -- (8.19282,1.19556) -- (8.19356,1.1954) -- (8.19431,1.19523) -- (8.19505,1.19507) --
 (8.1958,1.19491) -- (8.19655,1.19474) -- (8.19729,1.19458) -- (8.19804,1.19442) -- (8.19878,1.19425) -- (8.19953,1.19409) -- (8.20028,1.19393) -- (8.20102,1.19377) -- (8.20177,1.19364) -- (8.20251,1.19353) -- (8.20326,1.19341) -- (8.204,1.19329) --
 (8.20475,1.19318) -- (8.2055,1.19307) -- (8.20624,1.19296) -- (8.20699,1.19284) -- (8.20773,1.19273) -- (8.20848,1.19262) -- (8.20922,1.19251) -- (8.20997,1.19239) -- (8.21072,1.19229) -- (8.21146,1.19217) -- (8.21221,1.19207) -- (8.21295,1.19196)
 -- (8.2137,1.19185) -- (8.21445,1.19175) -- (8.21519,1.19163) -- (8.21594,1.19153) -- (8.21668,1.19142) -- (8.21743,1.19132) -- (8.21817,1.19121) -- (8.21892,1.19111) -- (8.21967,1.19101) -- (8.22041,1.1909) -- (8.22116,1.1908) -- (8.2219,1.1907) --
 (8.22265,1.1906) -- (8.2234,1.1905) -- (8.22414,1.1904) -- (8.22489,1.1903) -- (8.22563,1.1902) -- (8.22638,1.1901) -- (8.22713,1.19) -- (8.22787,1.1899) -- (8.22862,1.18981) -- (8.22936,1.18971) -- (8.23011,1.18961) -- (8.23085,1.18952) --
 (8.2316,1.18942) -- (8.23235,1.18933) -- (8.23309,1.18924) -- (8.23384,1.18914) -- (8.23458,1.18905) -- (8.23533,1.18896) -- (8.23608,1.18887) -- (8.23682,1.18878) -- (8.23757,1.18869) -- (8.23831,1.1886) -- (8.23906,1.18851) -- (8.2398,1.18842) --
 (8.24055,1.18833) -- (8.2413,1.18824) -- (8.24204,1.18815) -- (8.24279,1.18807) -- (8.24353,1.18798) -- (8.24428,1.1879) -- (8.24502,1.18782) -- (8.24577,1.18773) -- (8.24652,1.18765) -- (8.24726,1.18756) -- (8.24801,1.18748) -- (8.24875,1.1874) --
 (8.2495,1.18732) -- (8.25025,1.18723) -- (8.25099,1.18716) -- (8.25174,1.18707) -- (8.25248,1.187) -- (8.25323,1.18692) -- (8.25397,1.18684) -- (8.25472,1.18676) -- (8.25547,1.18669) -- (8.25621,1.18661) -- (8.25696,1.18653) -- (8.2577,1.18646) --
 (8.25845,1.18638) -- (8.2592,1.18631) -- (8.25994,1.18624) -- (8.26069,1.18616) -- (8.26143,1.18609) -- (8.26218,1.18601) -- (8.26293,1.18598) -- (8.26367,1.18595) -- (8.26442,1.18592) -- (8.26516,1.1859) -- (8.26591,1.18587) -- (8.26665,1.18585) --
 (8.2674,1.18582) -- (8.26815,1.18579) -- (8.26889,1.18577) -- (8.26964,1.18575) -- (8.27038,1.18572) -- (8.27113,1.1857) -- (8.27188,1.18568) -- (8.27262,1.18565) -- (8.27337,1.18563) -- (8.27411,1.18561) -- (8.27486,1.18559) -- (8.2756,1.18557) --
 (8.27635,1.18555) -- (8.2771,1.18553) -- (8.27784,1.18552) -- (8.27859,1.1855) -- (8.27933,1.18548) -- (8.28008,1.18546) -- (8.28082,1.18545) -- (8.28157,1.18543) -- (8.28232,1.18542) -- (8.28306,1.1854) -- (8.28381,1.18539) -- (8.28455,1.18538) --
 (8.2853,1.18537) -- (8.28605,1.18535) -- (8.28679,1.18535) -- (8.28754,1.18533) -- (8.28828,1.18532) -- (8.28903,1.18531) -- (8.28977,1.1853) -- (8.29052,1.1853) -- (8.29127,1.18528) -- (8.29201,1.18528) -- (8.29276,1.18528) -- (8.2935,1.18526) --
 (8.29425,1.18526) -- (8.295,1.18525) -- (8.29574,1.18525) -- (8.29649,1.18525) -- (8.29723,1.18524) -- (8.29798,1.18524) -- (8.29873,1.18524) -- (8.29947,1.18524) -- (8.30022,1.18523) -- (8.30096,1.18523) -- (8.30171,1.18523) -- (8.30245,1.18523) --
 (8.3032,1.18523) -- (8.30395,1.18524) -- (8.30469,1.18524) -- (8.30544,1.18524) -- (8.30618,1.18525) -- (8.30693,1.18525) -- (8.30768,1.18525) -- (8.30842,1.18526) -- (8.30917,1.18527) -- (8.30991,1.18527) -- (8.31066,1.18527) -- (8.3114,1.18528) --
 (8.31215,1.18529) -- (8.3129,1.1853) -- (8.31364,1.18531) -- (8.31439,1.18532) -- (8.31513,1.18533) -- (8.31588,1.18534) -- (8.31662,1.18535) -- (8.31737,1.18536) -- (8.31812,1.18537) -- (8.31886,1.18539) -- (8.31961,1.1854) -- (8.32035,1.18541) --
 (8.3211,1.18543) -- (8.32185,1.18544) -- (8.32259,1.18547) -- (8.32334,1.18553) -- (8.32408,1.1856) -- (8.32483,1.18565) -- (8.32557,1.18571) -- (8.32632,1.18578) -- (8.32707,1.18583) -- (8.32781,1.1859) -- (8.32856,1.18596) -- (8.3293,1.18603) --
 (8.33005,1.18609) -- (8.3308,1.18615) -- (8.33154,1.18622) -- (8.33229,1.18628) -- (8.33303,1.18635) -- (8.33378,1.18642) -- (8.33453,1.18649) -- (8.33527,1.18655) -- (8.33602,1.18662) -- (8.33676,1.18669) -- (8.33751,1.18676) -- (8.33825,1.18683)
 -- (8.339,1.1869) -- (8.33975,1.18698) -- (8.34049,1.18705) -- (8.34124,1.18712) -- (8.34198,1.18719) -- (8.34273,1.18727) -- (8.34348,1.18734) -- (8.34422,1.18742) -- (8.34497,1.1875) -- (8.34571,1.18757) -- (8.34646,1.18765) -- (8.3472,1.18772) --
 (8.34795,1.1878) -- (8.3487,1.18788) -- (8.34944,1.18796) -- (8.35019,1.18804) -- (8.35093,1.18812) -- (8.35168,1.1882) -- (8.35242,1.18828) -- (8.35317,1.18836) -- (8.35392,1.18845) -- (8.35466,1.18853) -- (8.35541,1.18861) -- (8.35615,1.1887) --
 (8.3569,1.18878) -- (8.35765,1.18887) -- (8.35839,1.18896) -- (8.35914,1.18904) -- (8.35988,1.18913) -- (8.36063,1.18922) -- (8.36137,1.18931) -- (8.36212,1.1894) -- (8.36287,1.18948) -- (8.36361,1.18958) -- (8.36436,1.18967) -- (8.3651,1.18976) --
 (8.36585,1.18985) -- (8.3666,1.18994) -- (8.36734,1.19004) -- (8.36809,1.19013) -- (8.36883,1.19023) -- (8.36958,1.19032) -- (8.37033,1.19041) -- (8.37107,1.19051) -- (8.37182,1.19061) -- (8.37256,1.19071) -- (8.37331,1.1908) -- (8.37405,1.1909) --
 (8.3748,1.191) -- (8.37555,1.1911) -- (8.37629,1.1912) -- (8.37704,1.1913) -- (8.37778,1.1914) -- (8.37853,1.1915) -- (8.37928,1.19161) -- (8.38002,1.19171) -- (8.38077,1.19181) -- (8.38151,1.19196) -- (8.38226,1.1921) -- (8.383,1.19225) --
 (8.38375,1.1924) -- (8.3845,1.19255) -- (8.38524,1.1927) -- (8.38599,1.19285) -- (8.38673,1.193) -- (8.38748,1.19315) -- (8.38822,1.1933) -- (8.38897,1.19345) -- (8.38972,1.19361) -- (8.39046,1.19376) -- (8.39121,1.19392) -- (8.39195,1.19407) --
 (8.3927,1.19423) -- (8.39345,1.19438) -- (8.39419,1.19454) -- (8.39494,1.19469) -- (8.39568,1.19485) -- (8.39643,1.19501) -- (8.39717,1.19517) -- (8.39792,1.19533) -- (8.39867,1.19549) -- (8.39941,1.19565) -- (8.40016,1.19582) -- (8.4009,1.19598) --
 (8.40165,1.19614) -- (8.4024,1.1963) -- (8.40314,1.19646) -- (8.40389,1.19663) -- (8.40463,1.1968) -- (8.40538,1.19696) -- (8.40613,1.19713) -- (8.40687,1.19729) -- (8.40762,1.19746) -- (8.40836,1.19763) -- (8.40911,1.1978) -- (8.40985,1.19797) --
 (8.4106,1.19814) -- (8.41135,1.19831) -- (8.41209,1.19848) -- (8.41284,1.19865) -- (8.41358,1.19883) -- (8.41433,1.19899) -- (8.41508,1.19917) -- (8.41582,1.19934) -- (8.41657,1.19952) -- (8.41731,1.19969) -- (8.41806,1.19987) -- (8.4188,1.20004) --
 (8.41955,1.20022) -- (8.4203,1.2004) -- (8.42104,1.20057) -- (8.42179,1.20075) -- (8.42253,1.20093) -- (8.42328,1.20111) -- (8.42402,1.20129) -- (8.42477,1.20148) -- (8.42552,1.20166) -- (8.42626,1.20184) -- (8.42701,1.20202) -- (8.42775,1.2022) --
 (8.4285,1.20239) -- (8.42925,1.20257) -- (8.42999,1.20276) -- (8.43074,1.20295) -- (8.43148,1.20313) -- (8.43223,1.20332) -- (8.43297,1.2035) -- (8.43372,1.20369) -- (8.43447,1.20388) -- (8.43521,1.20407) -- (8.43596,1.20426) -- (8.4367,1.20445) --
 (8.43745,1.20466) -- (8.4382,1.20489) -- (8.43894,1.20512) -- (8.43969,1.20536) -- (8.44043,1.2056) -- (8.44118,1.20583) -- (8.44193,1.20607) -- (8.44267,1.20631) -- (8.44342,1.20654) -- (8.44416,1.20679) -- (8.44491,1.20702) -- (8.44565,1.20727) --
 (8.4464,1.2075) -- (8.44715,1.20775) -- (8.44789,1.20799) -- (8.44864,1.20823) -- (8.44938,1.20847) -- (8.45013,1.20872) -- (8.45088,1.20896) -- (8.45162,1.20921) -- (8.45237,1.20945) -- (8.45311,1.2097) -- (8.45386,1.20994) -- (8.4546,1.21019) --
 (8.45535,1.21044) -- (8.4561,1.21069) -- (8.45684,1.21094) -- (8.45759,1.21119) -- (8.45833,1.21144) -- (8.45908,1.21169) -- (8.45982,1.21194) -- (8.46057,1.21219) -- (8.46132,1.21244) -- (8.46206,1.2127) -- (8.46281,1.21295) -- (8.46355,1.21321) --
 (8.4643,1.21346) -- (8.46505,1.21372) -- (8.46579,1.21397) -- (8.46654,1.21423) -- (8.46728,1.21449) -- (8.46803,1.21475) -- (8.46877,1.215) -- (8.46952,1.21526) -- (8.47027,1.21553) -- (8.47101,1.21578) -- (8.47176,1.21605) -- (8.4725,1.21631) --
 (8.47325,1.21657) -- (8.474,1.21684) -- (8.47474,1.2171) -- (8.47549,1.21736) -- (8.47623,1.21763) -- (8.47698,1.2179) -- (8.47773,1.21816) -- (8.47847,1.21843) -- (8.47922,1.21869) -- (8.47996,1.21897) -- (8.48071,1.21923) -- (8.48145,1.21951) --
 (8.4822,1.21977) -- (8.48295,1.22004) -- (8.48369,1.22031) -- (8.48444,1.22059) -- (8.48518,1.22086) -- (8.48593,1.22113) -- (8.48668,1.22141) -- (8.48742,1.22168) -- (8.48817,1.22196) -- (8.48891,1.22223) -- (8.48966,1.22251) -- (8.4904,1.22279) --
 (8.49115,1.22308) -- (8.4919,1.22341) -- (8.49264,1.22373) -- (8.49339,1.22405) -- (8.49413,1.22437) -- (8.49488,1.22469) -- (8.49562,1.22502) -- (8.49637,1.22534) -- (8.49712,1.22566) -- (8.49786,1.22599) -- (8.49861,1.22632) -- (8.49935,1.22665)
 -- (8.5001,1.22697) -- (8.50085,1.2273) -- (8.50159,1.22763) -- (8.50234,1.22796) -- (8.50308,1.22829) -- (8.50383,1.22862) -- (8.50457,1.22895) -- (8.50532,1.22928) -- (8.50607,1.22961) -- (8.50681,1.22995) -- (8.50756,1.23028) -- (8.5083,1.23062)
 -- (8.50905,1.23095) -- (8.5098,1.23129) -- (8.51054,1.23162) -- (8.51129,1.23196) -- (8.51203,1.2323) -- (8.51278,1.23263) -- (8.51353,1.23297) -- (8.51427,1.23331) -- (8.51502,1.23366) -- (8.51576,1.23399) -- (8.51651,1.23434) -- (8.51725,1.23468)
 -- (8.518,1.23502) -- (8.51875,1.23536) -- (8.51949,1.23571) -- (8.52024,1.23605) -- (8.52098,1.2364) -- (8.52173,1.23674) -- (8.52248,1.23709) -- (8.52322,1.23744) -- (8.52397,1.23778) -- (8.52471,1.23813) -- (8.52546,1.23848) -- (8.5262,1.23883)
 -- (8.52695,1.23918) -- (8.5277,1.23953) -- (8.52844,1.23988) -- (8.52919,1.24024) -- (8.52993,1.24059) -- (8.53068,1.24094) -- (8.53143,1.2413) -- (8.53217,1.24165) -- (8.53292,1.24201) -- (8.53366,1.24236) -- (8.53441,1.24272) -- (8.53515,1.24308)
 -- (8.5359,1.24343) -- (8.53665,1.2438) -- (8.53739,1.24415) -- (8.53814,1.24451) -- (8.53888,1.24487) -- (8.53963,1.24523) -- (8.54037,1.2456) -- (8.54112,1.24596) -- (8.54187,1.24635) -- (8.54261,1.24676) -- (8.54336,1.24717) -- (8.5441,1.24757)
 -- (8.54485,1.24798) -- (8.5456,1.24839) -- (8.54634,1.24879) -- (8.54709,1.2492) -- (8.54783,1.24961) -- (8.54858,1.25003) -- (8.54932,1.25044) -- (8.55007,1.25085) -- (8.55082,1.25127) -- (8.55156,1.25168) -- (8.55231,1.25209) -- (8.55305,1.2525)
 -- (8.5538,1.25292) -- (8.55455,1.25334) -- (8.55529,1.25375) -- (8.55604,1.25417) -- (8.55678,1.25459) -- (8.55753,1.25501) -- (8.55828,1.25543) -- (8.55902,1.25585) -- (8.55977,1.25627) -- (8.56051,1.25669) -- (8.56126,1.25712) -- (8.562,1.25753)
 -- (8.56275,1.25796) -- (8.5635,1.25838) -- (8.56424,1.25881) -- (8.56499,1.25923) -- (8.56573,1.25966) -- (8.56648,1.26009) -- (8.56723,1.26051) -- (8.56797,1.26094) -- (8.56872,1.26137) -- (8.56946,1.2618) -- (8.57021,1.26223) -- (8.57095,1.26266)
 -- (8.5717,1.26309) -- (8.57245,1.26352) -- (8.57319,1.26395) -- (8.57394,1.26439) -- (8.57468,1.26482) -- (8.57543,1.26526) -- (8.57617,1.26569) -- (8.57692,1.26613) -- (8.57767,1.26656) -- (8.57841,1.267) -- (8.57916,1.26744) -- (8.5799,1.26787)
 -- (8.58065,1.26832) -- (8.5814,1.26875) -- (8.58214,1.2692) -- (8.58289,1.26964) -- (8.58363,1.27008) -- (8.58438,1.27052) -- (8.58512,1.27097) -- (8.58587,1.27141) -- (8.58662,1.27185) -- (8.58736,1.2723) -- (8.58811,1.27275) -- (8.58885,1.27319)
 -- (8.5896,1.27367) -- (8.59035,1.27417) -- (8.59109,1.27465) -- (8.59184,1.27515) -- (8.59258,1.27564) -- (8.59333,1.27613) -- (8.59408,1.27663) -- (8.59482,1.27712) -- (8.59557,1.27761) -- (8.59631,1.27811) -- (8.59706,1.27861) -- (8.5978,1.2791)
 -- (8.59855,1.2796) -- (8.5993,1.2801) -- (8.60004,1.2806) -- (8.60079,1.2811) -- (8.60153,1.2816) -- (8.60228,1.2821) -- (8.60303,1.2826) -- (8.60377,1.28311) -- (8.60452,1.28361) -- (8.60526,1.28412) -- (8.60601,1.28463) -- (8.60675,1.28513) --
 (8.6075,1.28563) -- (8.60825,1.28615) -- (8.60899,1.28665) -- (8.60974,1.28716) -- (8.61048,1.28767) -- (8.61123,1.28818) -- (8.61197,1.28869) -- (8.61272,1.2892) -- (8.61347,1.28971) -- (8.61421,1.29023) -- (8.61496,1.29074) -- (8.6157,1.29126) --
 (8.61645,1.29177) -- (8.6172,1.29229) -- (8.61794,1.2928) -- (8.61869,1.29332) -- (8.61943,1.29384) -- (8.62018,1.29436) -- (8.62092,1.29488) -- (8.62167,1.2954) -- (8.62242,1.29592) -- (8.62316,1.29644) -- (8.62391,1.29697) -- (8.62465,1.29749) --
 (8.6254,1.29802) -- (8.62615,1.29854) -- (8.62689,1.29906) -- (8.62764,1.29959) -- (8.62838,1.30012) -- (8.62913,1.30064) -- (8.62988,1.30117) -- (8.63062,1.3017) -- (8.63137,1.30223) -- (8.63211,1.30276) -- (8.63286,1.3033) -- (8.6336,1.30383) --
 (8.63435,1.30436) -- (8.6351,1.30492) -- (8.63584,1.3055) -- (8.63659,1.30608) -- (8.63733,1.30665) -- (8.63808,1.30723) -- (8.63883,1.30781) -- (8.63957,1.30839) -- (8.64032,1.30897) -- (8.64106,1.30955) -- (8.64181,1.31013) -- (8.64255,1.31071) --
 (8.6433,1.3113) -- (8.64405,1.31188) -- (8.64479,1.31247) -- (8.64554,1.31305) -- (8.64628,1.31364) -- (8.64703,1.31422) -- (8.64777,1.31481) -- (8.64852,1.3154) -- (8.64927,1.31599) -- (8.65001,1.31658) -- (8.65076,1.31717) -- (8.6515,1.31776) --
 (8.65225,1.31835) -- (8.653,1.31894) -- (8.65374,1.31954) -- (8.65449,1.32013) -- (8.65523,1.32073) -- (8.65598,1.32132) -- (8.65672,1.32192) -- (8.65747,1.32252) -- (8.65822,1.32311) -- (8.65896,1.32371) -- (8.65971,1.32431) -- (8.66045,1.32491) --
 (8.6612,1.32551) -- (8.66195,1.32611) -- (8.66269,1.32671) -- (8.66344,1.32732) -- (8.66418,1.32792) -- (8.66493,1.32852) -- (8.66568,1.32913) -- (8.66642,1.32974) -- (8.66717,1.33034) -- (8.66791,1.33095) -- (8.66866,1.33156) -- (8.6694,1.33216) --
 (8.67015,1.33278) -- (8.6709,1.33339) -- (8.67164,1.334) -- (8.67239,1.33461) -- (8.67313,1.33522) -- (8.67388,1.33583) -- (8.67463,1.33645) -- (8.67537,1.33706) -- (8.67612,1.33768) -- (8.67686,1.33829) -- (8.67761,1.33891) -- (8.67835,1.33953) --
 (8.6791,1.34015) -- (8.67985,1.34078) -- (8.68059,1.34144) -- (8.68134,1.3421) -- (8.68208,1.34277) -- (8.68283,1.34343) -- (8.68357,1.34409) -- (8.68432,1.34476) -- (8.68507,1.34542) -- (8.68581,1.34609) -- (8.68656,1.34676) -- (8.6873,1.34742) --
 (8.68805,1.34809) -- (8.6888,1.34876) -- (8.68954,1.34943) -- (8.69029,1.3501) -- (8.69103,1.35077) -- (8.69178,1.35145) -- (8.69252,1.35212) -- (8.69327,1.35279) -- (8.69402,1.35346) -- (8.69476,1.35414) -- (8.69551,1.35482) -- (8.69625,1.3555) --
 (8.697,1.35617) -- (8.69775,1.35685) -- (8.69849,1.35753) -- (8.69924,1.35821) -- (8.69998,1.35889) -- (8.70073,1.35957) -- (8.70148,1.36025) -- (8.70222,1.36093) -- (8.70297,1.36162) -- (8.70371,1.3623) -- (8.70446,1.36298) -- (8.7052,1.36367) --
 (8.70595,1.36436) -- (8.7067,1.36504) -- (8.70744,1.36573) -- (8.70819,1.36642) -- (8.70893,1.36711) -- (8.70968,1.3678) -- (8.71043,1.36849) -- (8.71117,1.36918) -- (8.71192,1.36987) -- (8.71266,1.37056) -- (8.71341,1.37126) -- (8.71415,1.37195) --
 (8.7149,1.37265) -- (8.71565,1.37334) -- (8.71639,1.37404) -- (8.71714,1.37474) -- (8.71788,1.37543) -- (8.71863,1.37613) -- (8.71937,1.37683) -- (8.72012,1.37753) -- (8.72087,1.37823) -- (8.72161,1.37893) -- (8.72236,1.37964) -- (8.7231,1.38034) --
 (8.72385,1.38104) -- (8.7246,1.38175) -- (8.72534,1.38245) -- (8.72609,1.38316) -- (8.72683,1.38388) -- (8.72758,1.38463) -- (8.72832,1.38538) -- (8.72907,1.38613) -- (8.72982,1.38688) -- (8.73056,1.38763) -- (8.73131,1.38838) -- (8.73205,1.38913)
 -- (8.7328,1.38989) -- (8.73355,1.39064) -- (8.73429,1.3914) -- (8.73504,1.39215) -- (8.73578,1.39291) -- (8.73653,1.39367) -- (8.73728,1.39443) -- (8.73802,1.39518) -- (8.73877,1.39594) -- (8.73951,1.3967) -- (8.74026,1.39747) -- (8.741,1.39823) --
 (8.74175,1.39899) -- (8.7425,1.39975) -- (8.74324,1.40052) -- (8.74399,1.40128) -- (8.74473,1.40205) -- (8.74548,1.40282) -- (8.74623,1.40358) -- (8.74697,1.40435) -- (8.74772,1.40512) -- (8.74846,1.40589) -- (8.74921,1.40666) -- (8.74995,1.40743)
 -- (8.7507,1.4082) -- (8.75145,1.40897) -- (8.75219,1.40975) -- (8.75294,1.41052) -- (8.75368,1.41129) -- (8.75443,1.41207) -- (8.75517,1.41285) -- (8.75592,1.41362) -- (8.75667,1.4144) -- (8.75741,1.41518) -- (8.75816,1.41596) -- (8.7589,1.41674)
 -- (8.75965,1.41752) -- (8.7604,1.4183) -- (8.76114,1.41908) -- (8.76189,1.41987) -- (8.76263,1.42065) -- (8.76338,1.42143) -- (8.76412,1.42222) -- (8.76487,1.423) -- (8.76562,1.42379) -- (8.76636,1.42458) -- (8.76711,1.42537) -- (8.76785,1.42615)
 -- (8.7686,1.42694) -- (8.76935,1.42774) -- (8.77009,1.42853) -- (8.77084,1.42932) -- (8.77158,1.43011) -- (8.77233,1.4309) -- (8.77308,1.4317) -- (8.77382,1.43249) -- (8.77457,1.43329) -- (8.77531,1.43409) -- (8.77606,1.43488) -- (8.7768,1.43568)
 -- (8.77755,1.43648) -- (8.7783,1.43728) -- (8.77904,1.43808) -- (8.77979,1.4389) -- (8.78053,1.43974) -- (8.78128,1.44058) -- (8.78203,1.44142) -- (8.78277,1.44227) -- (8.78352,1.44311) -- (8.78426,1.44396) -- (8.78501,1.44481) -- (8.78575,1.44565)
 -- (8.7865,1.4465) -- (8.78725,1.44735) -- (8.78799,1.4482) -- (8.78874,1.44905) -- (8.78948,1.4499) -- (8.79023,1.45075) -- (8.79097,1.45161) -- (8.79172,1.45246) -- (8.79247,1.45331) -- (8.79321,1.45417) -- (8.79396,1.45502) -- (8.7947,1.45588) --
 (8.79545,1.45674) -- (8.7962,1.4576) -- (8.79694,1.45845) -- (8.79769,1.45931) -- (8.79843,1.46017) -- (8.79918,1.46104) -- (8.79992,1.4619) -- (8.80067,1.46276) -- (8.80142,1.46362) -- (8.80216,1.46449) -- (8.80291,1.46535) -- (8.80365,1.46622) --
 (8.8044,1.46709) -- (8.80515,1.46795) -- (8.80589,1.46882) -- (8.80664,1.46969) -- (8.80738,1.47056) -- (8.80813,1.47143) -- (8.80888,1.4723) -- (8.80962,1.47317) -- (8.81037,1.47404) -- (8.81111,1.47492) -- (8.81186,1.47579) -- (8.8126,1.47667) --
 (8.81335,1.47754) -- (8.8141,1.47842) -- (8.81484,1.4793) -- (8.81559,1.48018) -- (8.81633,1.48106) -- (8.81708,1.48194) -- (8.81783,1.48282) -- (8.81857,1.4837) -- (8.81932,1.48458) -- (8.82006,1.48546) -- (8.82081,1.48635) -- (8.82155,1.48723) --
 (8.8223,1.48812) -- (8.82305,1.489) -- (8.82379,1.48989) -- (8.82454,1.49078) -- (8.82528,1.49166) -- (8.82603,1.49256) -- (8.82677,1.49344) -- (8.82752,1.49434) -- (8.82827,1.49523) -- (8.82901,1.49612) -- (8.82976,1.49701) -- (8.8305,1.49791) --
 (8.83125,1.4988) -- (8.832,1.4997) -- (8.83274,1.50059) -- (8.83349,1.50149) -- (8.83423,1.50239) -- (8.83498,1.50329) -- (8.83572,1.50419) -- (8.83647,1.50509) -- (8.83722,1.50599) -- (8.83796,1.50689) -- (8.83871,1.50779) -- (8.83945,1.50869) --
 (8.8402,1.5096) -- (8.84095,1.5105) -- (8.84169,1.51141) -- (8.84244,1.51231) -- (8.84318,1.51322) -- (8.84393,1.51417) -- (8.84468,1.51512) -- (8.84542,1.51607) -- (8.84617,1.51702) -- (8.84691,1.51797) -- (8.84766,1.51892) -- (8.8484,1.51987) --
 (8.84915,1.52083) -- (8.8499,1.52178) -- (8.85064,1.52274) -- (8.85139,1.52369) -- (8.85213,1.52465) -- (8.85288,1.52561) -- (8.85363,1.52656) -- (8.85437,1.52752) -- (8.85512,1.52848) -- (8.85586,1.52944) -- (8.85661,1.5304) -- (8.85735,1.53137) --
 (8.8581,1.53233) -- (8.85885,1.53329) -- (8.85959,1.53426) -- (8.86034,1.53522) -- (8.86108,1.53619) -- (8.86183,1.53716) -- (8.86257,1.53812) -- (8.86332,1.53909) -- (8.86407,1.54006) -- (8.86481,1.54103) -- (8.86556,1.542) -- (8.8663,1.54298) --
 (8.86705,1.54395) -- (8.8678,1.54492) -- (8.86854,1.54589) -- (8.86929,1.54687) -- (8.87003,1.54784) -- (8.87078,1.54882) -- (8.87152,1.5498) -- (8.87227,1.55078) -- (8.87302,1.55175) -- (8.87376,1.55273) -- (8.87451,1.55372) -- (8.87525,1.55469) --
 (8.876,1.55568) -- (8.87675,1.55666) -- (8.87749,1.55764) -- (8.87824,1.55863) -- (8.87898,1.55961) -- (8.87973,1.5606) -- (8.88048,1.56158) -- (8.88122,1.56257) -- (8.88197,1.56356) -- (8.88271,1.56455) -- (8.88346,1.56554) -- (8.8842,1.56653) --
 (8.88495,1.56752) -- (8.8857,1.56851) -- (8.88644,1.56951) -- (8.88719,1.5705) -- (8.88793,1.57149) -- (8.88868,1.57249) -- (8.88943,1.57349) -- (8.89017,1.57448) -- (8.89092,1.57548) -- (8.89166,1.57648) -- (8.89241,1.57748) -- (8.89315,1.57848) --
 (8.8939,1.57948) -- (8.89465,1.58048) -- (8.89539,1.58148) -- (8.89614,1.58249) -- (8.89688,1.58349) -- (8.89763,1.5845) -- (8.89837,1.5855) -- (8.89912,1.58651) -- (8.89987,1.58751) -- (8.90061,1.58852) -- (8.90136,1.58953) -- (8.9021,1.59055) --
 (8.90285,1.59155) -- (8.9036,1.59256) -- (8.90434,1.59357) -- (8.90509,1.59459) -- (8.90583,1.5956) -- (8.90658,1.59662) -- (8.90732,1.59763) -- (8.90807,1.59865) -- (8.90882,1.59966) -- (8.90956,1.60068) -- (8.91031,1.6017) -- (8.91105,1.60272) --
 (8.9118,1.60374) -- (8.91255,1.60476) -- (8.91329,1.60578) -- (8.91404,1.6068) -- (8.91478,1.60783) -- (8.91553,1.60885) -- (8.91628,1.60987) -- (8.91702,1.6109) -- (8.91777,1.61193) -- (8.91851,1.61295) -- (8.91926,1.61398) -- (8.92,1.61501) --
 (8.92075,1.61604) -- (8.9215,1.61707) -- (8.92224,1.6181) -- (8.92299,1.61913) -- (8.92373,1.62019) -- (8.92448,1.62126) -- (8.92523,1.62233) -- (8.92597,1.62341) -- (8.92672,1.62449) -- (8.92746,1.62556) -- (8.92821,1.62664) -- (8.92895,1.62772) --
 (8.9297,1.6288) -- (8.93045,1.62987) -- (8.93119,1.63095) -- (8.93194,1.63204) -- (8.93268,1.63312) -- (8.93343,1.6342) -- (8.93417,1.63528) -- (8.93492,1.63637) -- (8.93567,1.63745) -- (8.93641,1.63854) -- (8.93716,1.63963) -- (8.9379,1.64071) --
 (8.93865,1.6418) -- (8.9394,1.64289) -- (8.94014,1.64398) -- (8.94089,1.64507) -- (8.94163,1.64616) -- (8.94238,1.64725) -- (8.94312,1.64835) -- (8.94387,1.64944) -- (8.94462,1.65054) -- (8.94536,1.65163) -- (8.94611,1.65273) -- (8.94685,1.65383) --
 (8.9476,1.65492) -- (8.94835,1.65602) -- (8.94909,1.65712) -- (8.94984,1.65822) -- (8.95058,1.65932) -- (8.95133,1.66042) -- (8.95208,1.66153) -- (8.95282,1.66263) -- (8.95357,1.66374) -- (8.95431,1.66484) -- (8.95506,1.66595) -- (8.9558,1.66705) --
 (8.95655,1.66816) -- (8.9573,1.66927) -- (8.95804,1.67038) -- (8.95879,1.67149) -- (8.95953,1.6726) -- (8.96028,1.67371) -- (8.96103,1.67482) -- (8.96177,1.67594) -- (8.96252,1.67705) -- (8.96326,1.67816) -- (8.96401,1.67928) -- (8.96475,1.68039) --
 (8.9655,1.68151) -- (8.96625,1.68263) -- (8.96699,1.68375) -- (8.96774,1.68487) -- (8.96848,1.68599) -- (8.96923,1.68711) -- (8.96998,1.68823) -- (8.97072,1.68935) -- (8.97147,1.69048) -- (8.97221,1.6916) -- (8.97296,1.69273) -- (8.9737,1.69385) --
 (8.97445,1.69498) -- (8.9752,1.69611) -- (8.97594,1.69724) -- (8.97669,1.69837) -- (8.97743,1.69949) -- (8.97818,1.70063) -- (8.97892,1.70176) -- (8.97967,1.70289) -- (8.98042,1.70403) -- (8.98116,1.70516) -- (8.98191,1.70629) -- (8.98265,1.70743)
 -- (8.9834,1.70857) -- (8.98415,1.7097) -- (8.98489,1.71084) -- (8.98564,1.71198) -- (8.98638,1.71312) -- (8.98713,1.71426) -- (8.98787,1.7154) -- (8.98862,1.71654) -- (8.98937,1.71768) -- (8.99011,1.71883) -- (8.99086,1.71997) -- (8.9916,1.72112)
 -- (8.99235,1.72226) -- (8.9931,1.72341) -- (8.99384,1.72456) -- (8.99459,1.72571) -- (8.99533,1.72686) -- (8.99608,1.728) -- (8.99683,1.72916) -- (8.99757,1.73031) -- (8.99832,1.73146) -- (8.99906,1.73261) -- (8.99981,1.73377) -- (9.00055,1.73492)
 -- (9.0013,1.73608) -- (9.00205,1.73723) -- (9.00279,1.73839) -- (9.00354,1.73955) -- (9.00428,1.74071) -- (9.00503,1.74187) -- (9.00578,1.74303) -- (9.00652,1.74419) -- (9.00727,1.74535) -- (9.00801,1.74651) -- (9.00876,1.74768) -- (9.0095,1.74884)
 -- (9.01025,1.75001) -- (9.011,1.75118) -- (9.01174,1.75234) -- (9.01249,1.75351) -- (9.01323,1.75468) -- (9.01398,1.75585) -- (9.01472,1.75702) -- (9.01547,1.75819) -- (9.01622,1.75936) -- (9.01696,1.76053) -- (9.01771,1.76171) -- (9.01845,1.76288)
 -- (9.0192,1.76405) -- (9.01995,1.76523) -- (9.02069,1.76641) -- (9.02144,1.76758) -- (9.02218,1.76877) -- (9.02293,1.76994) -- (9.02367,1.77112) -- (9.02442,1.7723) -- (9.02517,1.77349) -- (9.02591,1.7747) -- (9.02666,1.77592) -- (9.0274,1.77714)
 -- (9.02815,1.77837) -- (9.0289,1.77959) -- (9.02964,1.78082) -- (9.03039,1.78204) -- (9.03113,1.78327) -- (9.03188,1.7845) -- (9.03263,1.78573) -- (9.03337,1.78696) -- (9.03412,1.78819) -- (9.03486,1.78942) -- (9.03561,1.79065) -- (9.03635,1.79188)
 -- (9.0371,1.79312) -- (9.03785,1.79435) -- (9.03859,1.79559) -- (9.03934,1.79682) -- (9.04008,1.79806) -- (9.04083,1.7993) -- (9.04158,1.80054) -- (9.04232,1.80178) -- (9.04307,1.80302) -- (9.04381,1.80426) -- (9.04456,1.8055) -- (9.0453,1.80674)
 -- (9.04605,1.80799) -- (9.0468,1.80923) -- (9.04754,1.81048) -- (9.04829,1.81172) -- (9.04903,1.81297) -- (9.04978,1.81422) -- (9.05052,1.81547) -- (9.05127,1.81672) -- (9.05202,1.81797) -- (9.05276,1.81922) -- (9.05351,1.82047) --
 (9.05425,1.82172) -- (9.055,1.82297) -- (9.05575,1.82423) -- (9.05649,1.82549) -- (9.05724,1.82674) -- (9.05798,1.828) -- (9.05873,1.82925) -- (9.05947,1.83051) -- (9.06022,1.83177) -- (9.06097,1.83303) -- (9.06171,1.83429) -- (9.06246,1.83555) --
 (9.0632,1.83682) -- (9.06395,1.83808) -- (9.0647,1.83934) -- (9.06544,1.84061) -- (9.06619,1.84187) -- (9.06693,1.84314) -- (9.06768,1.84441) -- (9.06843,1.84568) -- (9.06917,1.84694) -- (9.06992,1.84821) -- (9.07066,1.84948) -- (9.07141,1.85076) --
 (9.07215,1.85203) -- (9.0729,1.8533) -- (9.07365,1.85457) -- (9.07439,1.85585) -- (9.07514,1.85712) -- (9.07588,1.8584) -- (9.07663,1.85967) -- (9.07738,1.86095) -- (9.07812,1.86223) -- (9.07887,1.86351) -- (9.07961,1.86479) -- (9.08036,1.86608) --
 (9.0811,1.86735) -- (9.08185,1.86864) -- (9.0826,1.86992) -- (9.08334,1.87121) -- (9.08409,1.87249) -- (9.08483,1.87378) -- (9.08558,1.87506) -- (9.08632,1.87635) -- (9.08707,1.87764) -- (9.08782,1.87893) -- (9.08856,1.88022) -- (9.08931,1.88151) --
 (9.09005,1.8828) -- (9.0908,1.88409) -- (9.09155,1.88539) -- (9.09229,1.88668) -- (9.09304,1.88797) -- (9.09378,1.88927) -- (9.09453,1.89057) -- (9.09527,1.89186) -- (9.09602,1.89316) -- (9.09677,1.89446) -- (9.09751,1.89576) -- (9.09826,1.89706) --
 (9.099,1.89836) -- (9.09975,1.89966) -- (9.1005,1.90097) -- (9.10124,1.90227) -- (9.10199,1.90357) -- (9.10273,1.90488) -- (9.10348,1.90619) -- (9.10423,1.90749) -- (9.10497,1.9088) -- (9.10572,1.91011) -- (9.10646,1.91142) -- (9.10721,1.91273) --
 (9.10795,1.91404) -- (9.1087,1.91535) -- (9.10945,1.91667) -- (9.11019,1.91798) -- (9.11094,1.91929) -- (9.11168,1.92061) -- (9.11243,1.92192) -- (9.11318,1.92324) -- (9.11392,1.92456) -- (9.11467,1.92588) -- (9.11541,1.92719) -- (9.11616,1.92852)
 -- (9.1169,1.92984) -- (9.11765,1.93116) -- (9.1184,1.93248) -- (9.11914,1.9338) -- (9.11989,1.93513) -- (9.12063,1.93645) -- (9.12138,1.93778) -- (9.12212,1.9391) -- (9.12287,1.94043) -- (9.12362,1.94176) -- (9.12436,1.94309) -- (9.12511,1.94442)
 -- (9.12585,1.94575) -- (9.1266,1.94708) -- (9.12735,1.94841) -- (9.12809,1.94975) -- (9.12884,1.95108) -- (9.12958,1.95241) -- (9.13033,1.95375) -- (9.13107,1.95509) -- (9.13182,1.95642) -- (9.13257,1.95776) -- (9.13331,1.9591) -- (9.13406,1.96044)
 -- (9.1348,1.96178) -- (9.13555,1.96312) -- (9.1363,1.96446) -- (9.13704,1.9658) -- (9.13779,1.96715) -- (9.13853,1.96849) -- (9.13928,1.96984) -- (9.14003,1.97118) -- (9.14077,1.97253) -- (9.14152,1.97388) -- (9.14226,1.97523) -- (9.14301,1.97658)
 -- (9.14375,1.97793) -- (9.1445,1.97928) -- (9.14525,1.98063) -- (9.14599,1.98198) -- (9.14674,1.98333) -- (9.14748,1.98469) -- (9.14823,1.98604) -- (9.14898,1.9874) -- (9.14972,1.98876) -- (9.15047,1.99011) -- (9.15121,1.99147) -- (9.15196,1.99283)
 -- (9.1527,1.99419) -- (9.15345,1.99555) -- (9.1542,1.99694) -- (9.15494,1.99835) -- (9.15569,1.99975) -- (9.15643,2.00115) -- (9.15718,2.00256) -- (9.15792,2.00396) -- (9.15867,2.00537) -- (9.15942,2.00678) -- (9.16016,2.00818) -- (9.16091,2.0096)
 -- (9.16165,2.011) -- (9.1624,2.01241) -- (9.16315,2.01382) -- (9.16389,2.01524) -- (9.16464,2.01665) -- (9.16538,2.01806) -- (9.16613,2.01948) -- (9.16687,2.02089) -- (9.16762,2.02231) -- (9.16837,2.02372) -- (9.16911,2.02514) -- (9.16986,2.02656)
 -- (9.1706,2.02798) -- (9.17135,2.0294) -- (9.1721,2.03082) -- (9.17284,2.03224) -- (9.17359,2.03367) -- (9.17433,2.03509) -- (9.17508,2.03651) -- (9.17583,2.03794) -- (9.17657,2.03937) -- (9.17732,2.04079) -- (9.17806,2.04222) -- (9.17881,2.04365)
 -- (9.17955,2.04508) -- (9.1803,2.04651) -- (9.18105,2.04794) -- (9.18179,2.04937) -- (9.18254,2.0508) -- (9.18328,2.05223) -- (9.18403,2.05367) -- (9.18478,2.0551) -- (9.18552,2.05654) -- (9.18627,2.05797) -- (9.18701,2.05941) -- (9.18776,2.06085)
 -- (9.1885,2.06229) -- (9.18925,2.06373) -- (9.19,2.06517) -- (9.19074,2.06661) -- (9.19149,2.06805) -- (9.19223,2.0695) -- (9.19298,2.07094) -- (9.19372,2.07239) -- (9.19447,2.07383) -- (9.19522,2.07528) -- (9.19596,2.07673) -- (9.19671,2.07817) --
 (9.19745,2.07962) -- (9.1982,2.08107) -- (9.19895,2.08252) -- (9.19969,2.08397) -- (9.20044,2.08543) -- (9.20118,2.08688) -- (9.20193,2.08833) -- (9.20267,2.08979) -- (9.20342,2.09124) -- (9.20417,2.0927);
\colorlet{c}{kugray!30};
\draw [c] (1,1.57352) -- (9.95,1.57352);
\end{tikzpicture}

\end{infilsf}
\end{minipage}
\begin{minipage}[b]{.3\textwidth}
\caption{The negative 2 log likelihood ratio for the profile likelihood scan over a range of values for $\Lambda^{-4}$. This likelihood scan does not include uncertainties on the background. The minimum negative likelihood corresponds to $\Lambda=814$ GeV, with a confidence interval of $763 \le \Lambda \le 900$ GeV. The horizontal line indicates the log likelihood ratio at the 95\% confidence level.}\label{resllr}
\end{minipage}
\end{figure}

As figure~\ref{resllr} states, this result excludes the Standard Model. To achieve this result, we have neglected the considerable statistical uncertainty contributed by the $\gamma$jet sample, as is clear from table~\ref{expected}. This uncertainty can be accounted for by incorporating it as a second set of nuisance parameters associated with the background sample. This will require significantly more CPU time to compute than the result presented above.

Figure~\ref{result} shows the $M_{\gamma\gamma}$ distribution associated with the most likely value of $\Lambda$, along with those associated with the limits on the confidence interval overlaid on the data distribution.

\begin{figure}[htp]
\begin{minipage}[b]{.69\textwidth}
\begin{infilsf} \tiny
\begin{tikzpicture}[x=.092\textwidth,y=.092\textwidth]
\pgfdeclareplotmark{cross} {
\pgfpathmoveto{\pgfpoint{-0.3\pgfplotmarksize}{\pgfplotmarksize}}
\pgfpathlineto{\pgfpoint{+0.3\pgfplotmarksize}{\pgfplotmarksize}}
\pgfpathlineto{\pgfpoint{+0.3\pgfplotmarksize}{0.3\pgfplotmarksize}}
\pgfpathlineto{\pgfpoint{+1\pgfplotmarksize}{0.3\pgfplotmarksize}}
\pgfpathlineto{\pgfpoint{+1\pgfplotmarksize}{-0.3\pgfplotmarksize}}
\pgfpathlineto{\pgfpoint{+0.3\pgfplotmarksize}{-0.3\pgfplotmarksize}}
\pgfpathlineto{\pgfpoint{+0.3\pgfplotmarksize}{-1.\pgfplotmarksize}}
\pgfpathlineto{\pgfpoint{-0.3\pgfplotmarksize}{-1.\pgfplotmarksize}}
\pgfpathlineto{\pgfpoint{-0.3\pgfplotmarksize}{-0.3\pgfplotmarksize}}
\pgfpathlineto{\pgfpoint{-1.\pgfplotmarksize}{-0.3\pgfplotmarksize}}
\pgfpathlineto{\pgfpoint{-1.\pgfplotmarksize}{0.3\pgfplotmarksize}}
\pgfpathlineto{\pgfpoint{-0.3\pgfplotmarksize}{0.3\pgfplotmarksize}}
\pgfpathclose
\pgfusepathqstroke
}
\pgfdeclareplotmark{cross*} {
\pgfpathmoveto{\pgfpoint{-0.3\pgfplotmarksize}{\pgfplotmarksize}}
\pgfpathlineto{\pgfpoint{+0.3\pgfplotmarksize}{\pgfplotmarksize}}
\pgfpathlineto{\pgfpoint{+0.3\pgfplotmarksize}{0.3\pgfplotmarksize}}
\pgfpathlineto{\pgfpoint{+1\pgfplotmarksize}{0.3\pgfplotmarksize}}
\pgfpathlineto{\pgfpoint{+1\pgfplotmarksize}{-0.3\pgfplotmarksize}}
\pgfpathlineto{\pgfpoint{+0.3\pgfplotmarksize}{-0.3\pgfplotmarksize}}
\pgfpathlineto{\pgfpoint{+0.3\pgfplotmarksize}{-1.\pgfplotmarksize}}
\pgfpathlineto{\pgfpoint{-0.3\pgfplotmarksize}{-1.\pgfplotmarksize}}
\pgfpathlineto{\pgfpoint{-0.3\pgfplotmarksize}{-0.3\pgfplotmarksize}}
\pgfpathlineto{\pgfpoint{-1.\pgfplotmarksize}{-0.3\pgfplotmarksize}}
\pgfpathlineto{\pgfpoint{-1.\pgfplotmarksize}{0.3\pgfplotmarksize}}
\pgfpathlineto{\pgfpoint{-0.3\pgfplotmarksize}{0.3\pgfplotmarksize}}
\pgfpathclose
\pgfusepathqfillstroke
}
\pgfdeclareplotmark{newstar} {
\pgfpathmoveto{\pgfqpoint{0pt}{\pgfplotmarksize}}
\pgfpathlineto{\pgfqpointpolar{44}{0.5\pgfplotmarksize}}
\pgfpathlineto{\pgfqpointpolar{18}{\pgfplotmarksize}}
\pgfpathlineto{\pgfqpointpolar{-20}{0.5\pgfplotmarksize}}
\pgfpathlineto{\pgfqpointpolar{-54}{\pgfplotmarksize}}
\pgfpathlineto{\pgfqpointpolar{-90}{0.5\pgfplotmarksize}}
\pgfpathlineto{\pgfqpointpolar{234}{\pgfplotmarksize}}
\pgfpathlineto{\pgfqpointpolar{198}{0.5\pgfplotmarksize}}
\pgfpathlineto{\pgfqpointpolar{162}{\pgfplotmarksize}}
\pgfpathlineto{\pgfqpointpolar{134}{0.5\pgfplotmarksize}}
\pgfpathclose
\pgfusepathqstroke
}
\pgfdeclareplotmark{newstar*} {
\pgfpathmoveto{\pgfqpoint{0pt}{\pgfplotmarksize}}
\pgfpathlineto{\pgfqpointpolar{44}{0.5\pgfplotmarksize}}
\pgfpathlineto{\pgfqpointpolar{18}{\pgfplotmarksize}}
\pgfpathlineto{\pgfqpointpolar{-20}{0.5\pgfplotmarksize}}
\pgfpathlineto{\pgfqpointpolar{-54}{\pgfplotmarksize}}
\pgfpathlineto{\pgfqpointpolar{-90}{0.5\pgfplotmarksize}}
\pgfpathlineto{\pgfqpointpolar{234}{\pgfplotmarksize}}
\pgfpathlineto{\pgfqpointpolar{198}{0.5\pgfplotmarksize}}
\pgfpathlineto{\pgfqpointpolar{162}{\pgfplotmarksize}}
\pgfpathlineto{\pgfqpointpolar{134}{0.5\pgfplotmarksize}}
\pgfpathclose
\pgfusepathqfillstroke
}
\definecolor{c}{rgb}{1,1,1};
\draw [color=c, fill=c] (0,0) rectangle (10,6.80516);
\draw [color=c, fill=c] (1,0.680516) rectangle (9.95,6.73711);
\definecolor{c}{rgb}{0,0,0};
\draw [c] (1,0.680516) -- (1,6.73711) -- (9.95,6.73711) -- (9.95,0.680516) -- (1,0.680516);
\definecolor{c}{rgb}{1,1,1};
\draw [color=c, fill=c] (1,0.680516) rectangle (9.95,6.73711);
\definecolor{c}{rgb}{0,0,0};
\draw [c] (1,0.680516) -- (1,6.73711) -- (9.95,6.73711) -- (9.95,0.680516) -- (1,0.680516);
\colorlet{c}{natgreen};
\draw [c] (1,6.10893) -- (1.09421,6.10893) -- (1.09421,5.90062) -- (1.18842,5.90062) -- (1.18842,5.7055) -- (1.28263,5.7055) -- (1.28263,5.51086) -- (1.37684,5.51086) -- (1.37684,5.33087) -- (1.47105,5.33087) -- (1.47105,5.17331) --
 (1.56526,5.17331) -- (1.56526,5.00725) -- (1.65947,5.00725) -- (1.65947,4.85392) -- (1.75368,4.85392) -- (1.75368,4.73013) -- (1.84789,4.73013) -- (1.84789,4.59002) -- (1.94211,4.59002) -- (1.94211,4.47158) -- (2.03632,4.47158) -- (2.03632,4.35849)
 -- (2.13053,4.35849) -- (2.13053,4.23556) -- (2.22474,4.23556) -- (2.22474,4.16816) -- (2.31895,4.16816) -- (2.31895,4.06824) -- (2.41316,4.06824) -- (2.41316,3.84984) -- (2.50737,3.84984) -- (2.50737,3.77652) -- (2.60158,3.77652) --
 (2.60158,3.75293) -- (2.69579,3.75293) -- (2.69579,3.60051) -- (2.79,3.60051) -- (2.79,3.54415) -- (2.88421,3.54415) -- (2.88421,3.46689) -- (2.97842,3.46689) -- (2.97842,3.28726) -- (3.07263,3.28726) -- (3.07263,3.24792) -- (3.16684,3.24792) --
 (3.16684,3.28726) -- (3.26105,3.28726) -- (3.26105,3.12921) -- (3.35526,3.12921) -- (3.35526,2.75109) -- (3.44947,2.75109) -- (3.44947,2.91841) -- (3.54368,2.91841) -- (3.54368,3.00122) -- (3.63789,3.00122) -- (3.63789,2.46506) -- (3.73211,2.46506)
 -- (3.73211,2.8147) -- (3.82632,2.8147) -- (3.82632,2.29774) -- (3.92053,2.29774) -- (3.92053,2.29774) -- (4.01474,2.29774) -- (4.01474,0.680516) -- (4.10895,0.680516) -- (4.10895,2.29774) -- (4.20316,2.29774) -- (4.20316,2.29774) --
 (4.29737,2.29774) -- (4.29737,2.46506) -- (4.39158,2.46506) -- (4.39158,2.0117) -- (4.48579,2.0117) -- (4.48579,0.680516) -- (4.58,0.680516) -- (4.58,2.0117) -- (4.67421,2.0117) -- (4.67421,2.0117) -- (4.76842,2.0117) -- (4.76842,2.0117) --
 (4.86263,2.0117) -- (4.86263,2.0117) -- (4.95684,2.0117) -- (4.95684,2.0117) -- (5.05105,2.0117) -- (5.05105,2.29774) -- (5.14526,2.29774) -- (5.14526,0.680516) -- (5.23947,0.680516) -- (5.23947,0.680516) -- (5.33368,0.680516) -- (5.33368,0.680516)
 -- (5.42789,0.680516) -- (5.42789,0.680516) -- (5.52211,0.680516) -- (5.52211,0.680516) -- (5.61632,0.680516) -- (5.61632,0.680516) -- (5.71053,0.680516) -- (5.71053,0.680516) -- (5.80474,0.680516) -- (5.80474,0.680516) -- (5.89895,0.680516) --
 (5.89895,0.680516) -- (5.99316,0.680516) -- (5.99316,0.680516) -- (6.08737,0.680516) -- (6.08737,0.680516) -- (6.18158,0.680516) -- (6.18158,0.680516) -- (6.27579,0.680516) -- (6.27579,0.680516) -- (6.37,0.680516) -- (6.37,0.680516) --
 (6.46421,0.680516) -- (6.46421,0.680516) -- (6.55842,0.680516) -- (6.55842,0.680516) -- (6.65263,0.680516) -- (6.65263,0.680516) -- (6.74684,0.680516) -- (6.74684,0.680516) -- (6.84105,0.680516) -- (6.84105,0.680516) -- (6.93526,0.680516) --
 (6.93526,0.680516) -- (7.02947,0.680516) -- (7.02947,0.680516) -- (7.12368,0.680516) -- (7.12368,0.680516) -- (7.21789,0.680516) -- (7.21789,0.680516) -- (7.31211,0.680516) -- (7.31211,0.680516) -- (7.40632,0.680516) -- (7.40632,0.680516) --
 (7.50053,0.680516) -- (7.50053,0.680516) -- (7.59474,0.680516) -- (7.59474,0.680516) -- (7.68895,0.680516) -- (7.68895,0.680516) -- (7.78316,0.680516) -- (7.78316,0.680516) -- (7.87737,0.680516) -- (7.87737,0.680516) -- (7.97158,0.680516) --
 (7.97158,0.680516) -- (8.06579,0.680516) -- (8.06579,0.680516) -- (8.16,0.680516) -- (8.16,0.680516) -- (8.25421,0.680516) -- (8.25421,0.680516) -- (8.34842,0.680516) -- (8.34842,0.680516) -- (8.44263,0.680516) -- (8.44263,0.680516) --
 (8.53684,0.680516) -- (8.53684,0.680516) -- (8.63105,0.680516) -- (8.63105,0.680516) -- (8.72526,0.680516) -- (8.72526,0.680516) -- (8.81947,0.680516) -- (8.81947,0.680516) -- (8.91368,0.680516) -- (8.91368,0.680516) -- (9.00789,0.680516) --
 (9.00789,0.680516) -- (9.10211,0.680516) -- (9.10211,0.680516) -- (9.19632,0.680516) -- (9.19632,0.680516) -- (9.29053,0.680516) -- (9.29053,0.680516) -- (9.38474,0.680516) -- (9.38474,0.680516) -- (9.47895,0.680516) -- (9.47895,0.680516) --
 (9.57316,0.680516) -- (9.57316,0.680516) -- (9.66737,0.680516) -- (9.66737,0.680516) -- (9.76158,0.680516) -- (9.76158,0.680516) -- (9.85579,0.680516) -- (9.85579,0.680516) -- (9.95,0.680516);
\definecolor{c}{rgb}{0,0,0};
\draw [c] (1,0.680516) -- (9.95,0.680516);
\draw [anchor= east] (9.95,0.108883) node[color=c, rotate=0]{$M_{\gamma\gamma}\text{ [GeV]}$};
\draw [c] (2.07148,0.863234) -- (2.07148,0.680516);
\draw [c] (2.38662,0.771875) -- (2.38662,0.680516);
\draw [c] (2.70176,0.771875) -- (2.70176,0.680516);
\draw [c] (3.0169,0.771875) -- (3.0169,0.680516);
\draw [c] (3.33204,0.771875) -- (3.33204,0.680516);
\draw [c] (3.64718,0.863234) -- (3.64718,0.680516);
\draw [c] (3.96232,0.771875) -- (3.96232,0.680516);
\draw [c] (4.27746,0.771875) -- (4.27746,0.680516);
\draw [c] (4.59261,0.771875) -- (4.59261,0.680516);
\draw [c] (4.90775,0.771875) -- (4.90775,0.680516);
\draw [c] (5.22289,0.863234) -- (5.22289,0.680516);
\draw [c] (5.53803,0.771875) -- (5.53803,0.680516);
\draw [c] (5.85317,0.771875) -- (5.85317,0.680516);
\draw [c] (6.16831,0.771875) -- (6.16831,0.680516);
\draw [c] (6.48345,0.771875) -- (6.48345,0.680516);
\draw [c] (6.79859,0.863234) -- (6.79859,0.680516);
\draw [c] (7.11373,0.771875) -- (7.11373,0.680516);
\draw [c] (7.42887,0.771875) -- (7.42887,0.680516);
\draw [c] (7.74401,0.771875) -- (7.74401,0.680516);
\draw [c] (8.05915,0.771875) -- (8.05915,0.680516);
\draw [c] (8.3743,0.863234) -- (8.3743,0.680516);
\draw [c] (8.68944,0.771875) -- (8.68944,0.680516);
\draw [c] (9.00458,0.771875) -- (9.00458,0.680516);
\draw [c] (9.31972,0.771875) -- (9.31972,0.680516);
\draw [c] (9.63486,0.771875) -- (9.63486,0.680516);
\draw [c] (9.95,0.863234) -- (9.95,0.680516);
\draw [c] (2.07148,0.863234) -- (2.07148,0.680516);
\draw [c] (1.75634,0.771875) -- (1.75634,0.680516);
\draw [c] (1.4412,0.771875) -- (1.4412,0.680516);
\draw [c] (1.12606,0.771875) -- (1.12606,0.680516);
\draw [anchor=base] (2.07148,0.353868) node[color=c, rotate=0]{500};
\draw [anchor=base] (3.64718,0.353868) node[color=c, rotate=0]{1000};
\draw [anchor=base] (5.22289,0.353868) node[color=c, rotate=0]{1500};
\draw [anchor=base] (6.79859,0.353868) node[color=c, rotate=0]{2000};
\draw [anchor=base] (8.3743,0.353868) node[color=c, rotate=0]{2500};
\draw [anchor=base] (9.95,0.353868) node[color=c, rotate=0]{3000};
\draw [c] (1,0.680516) -- (1,6.73711);
\draw [anchor= east] (-0.12,6.73711) node[color=c, rotate=90]{Number of events};
\draw [c] (1.1335,0.683402) -- (1,0.683402);
\draw [c] (1.1335,0.775484) -- (1,0.775484);
\draw [c] (1.1335,0.850721) -- (1,0.850721);
\draw [c] (1.1335,0.914333) -- (1,0.914333);
\draw [c] (1.1335,0.969436) -- (1,0.969436);
\draw [c] (1.1335,1.01804) -- (1,1.01804);
\draw [c] (1.267,1.06152) -- (1,1.06152);
\draw [anchor= east] (0.922,1.06152) node[color=c, rotate=0]{$10^{-1}$};
\draw [c] (1.1335,1.34755) -- (1,1.34755);
\draw [c] (1.1335,1.51487) -- (1,1.51487);
\draw [c] (1.1335,1.63359) -- (1,1.63359);
\draw [c] (1.1335,1.72567) -- (1,1.72567);
\draw [c] (1.1335,1.80091) -- (1,1.80091);
\draw [c] (1.1335,1.86452) -- (1,1.86452);
\draw [c] (1.1335,1.91962) -- (1,1.91962);
\draw [c] (1.1335,1.96823) -- (1,1.96823);
\draw [c] (1.267,2.0117) -- (1,2.0117);
\draw [anchor= east] (0.922,2.0117) node[color=c, rotate=0]{1};
\draw [c] (1.1335,2.29774) -- (1,2.29774);
\draw [c] (1.1335,2.46506) -- (1,2.46506);
\draw [c] (1.1335,2.58377) -- (1,2.58377);
\draw [c] (1.1335,2.67586) -- (1,2.67586);
\draw [c] (1.1335,2.75109) -- (1,2.75109);
\draw [c] (1.1335,2.8147) -- (1,2.8147);
\draw [c] (1.1335,2.86981) -- (1,2.86981);
\draw [c] (1.1335,2.91841) -- (1,2.91841);
\draw [c] (1.267,2.96189) -- (1,2.96189);
\draw [anchor= east] (0.922,2.96189) node[color=c, rotate=0]{10};
\draw [c] (1.1335,3.24792) -- (1,3.24792);
\draw [c] (1.1335,3.41524) -- (1,3.41524);
\draw [c] (1.1335,3.53396) -- (1,3.53396);
\draw [c] (1.1335,3.62604) -- (1,3.62604);
\draw [c] (1.1335,3.70128) -- (1,3.70128);
\draw [c] (1.1335,3.76489) -- (1,3.76489);
\draw [c] (1.1335,3.81999) -- (1,3.81999);
\draw [c] (1.1335,3.8686) -- (1,3.8686);
\draw [c] (1.267,3.91208) -- (1,3.91208);
\draw [anchor= east] (0.922,3.91208) node[color=c, rotate=0]{$10^{2}$};
\draw [c] (1.1335,4.19811) -- (1,4.19811);
\draw [c] (1.1335,4.36543) -- (1,4.36543);
\draw [c] (1.1335,4.48415) -- (1,4.48415);
\draw [c] (1.1335,4.57623) -- (1,4.57623);
\draw [c] (1.1335,4.65146) -- (1,4.65146);
\draw [c] (1.1335,4.71508) -- (1,4.71508);
\draw [c] (1.1335,4.77018) -- (1,4.77018);
\draw [c] (1.1335,4.81878) -- (1,4.81878);
\draw [c] (1.267,4.86226) -- (1,4.86226);
\draw [anchor= east] (0.922,4.86226) node[color=c, rotate=0]{$10^{3}$};
\draw [c] (1.1335,5.1483) -- (1,5.1483);
\draw [c] (1.1335,5.31562) -- (1,5.31562);
\draw [c] (1.1335,5.43433) -- (1,5.43433);
\draw [c] (1.1335,5.52641) -- (1,5.52641);
\draw [c] (1.1335,5.60165) -- (1,5.60165);
\draw [c] (1.1335,5.66526) -- (1,5.66526);
\draw [c] (1.1335,5.72037) -- (1,5.72037);
\draw [c] (1.1335,5.76897) -- (1,5.76897);
\draw [c] (1.267,5.81245) -- (1,5.81245);
\draw [anchor= east] (0.922,5.81245) node[color=c, rotate=0]{$10^{4}$};
\draw [c] (1.1335,6.09848) -- (1,6.09848);
\draw [c] (1.1335,6.2658) -- (1,6.2658);
\draw [c] (1.1335,6.38452) -- (1,6.38452);
\draw [c] (1.1335,6.4766) -- (1,6.4766);
\draw [c] (1.1335,6.55184) -- (1,6.55184);
\draw [c] (1.1335,6.61545) -- (1,6.61545);
\draw [c] (1.1335,6.67055) -- (1,6.67055);
\draw [c] (1.1335,6.71916) -- (1,6.71916);
\colorlet{c}{natcomp!60};
\draw [c] (1,6.09392) -- (1.09421,6.09392) -- (1.09421,5.87993) -- (1.18842,5.87993) -- (1.18842,5.6608) -- (1.28263,5.6608) -- (1.28263,5.47695) -- (1.37684,5.47695) -- (1.37684,5.28422) -- (1.47105,5.28422) -- (1.47105,5.1326) --
 (1.56526,5.1326) -- (1.56526,4.99612) -- (1.65947,4.99612) -- (1.65947,4.87302) -- (1.75368,4.87302) -- (1.75368,4.70315) -- (1.84789,4.70315) -- (1.84789,4.62357) -- (1.94211,4.62357) -- (1.94211,4.41074) -- (2.03632,4.41074) -- (2.03632,4.32014)
 -- (2.13053,4.32014) -- (2.13053,4.12959) -- (2.22474,4.12959) -- (2.22474,4.09372) -- (2.31895,4.09372) -- (2.31895,3.88795) -- (2.41316,3.88795) -- (2.41316,3.79669) -- (2.50737,3.79669) -- (2.50737,3.99068) -- (2.60158,3.99068) --
 (2.60158,3.60823) -- (2.69579,3.60823) -- (2.69579,3.52284) -- (2.79,3.52284) -- (2.79,3.55391) -- (2.88421,3.55391) -- (2.88421,3.22581) -- (2.97842,3.22581) -- (2.97842,3.40254) -- (3.07263,3.40254) -- (3.07263,3.15669) -- (3.16684,3.15669) --
 (3.16684,3.05172) -- (3.26105,3.05172) -- (3.26105,2.85857) -- (3.35526,2.85857) -- (3.35526,3.36418) -- (3.44947,3.36418) -- (3.44947,2.81617) -- (3.54368,2.81617) -- (3.54368,2.61781) -- (3.63789,2.61781) -- (3.63789,2.72646) -- (3.73211,2.72646)
 -- (3.73211,2.44853) -- (3.82632,2.44853) -- (3.82632,2.37343) -- (3.92053,2.37343) -- (3.92053,2.35594) -- (4.01474,2.35594) -- (4.01474,2.2579) -- (4.10895,2.2579) -- (4.10895,2.23509) -- (4.20316,2.23509) -- (4.20316,2.12708) -- (4.29737,2.12708)
 -- (4.29737,2.08927) -- (4.39158,2.08927) -- (4.39158,2.03507) -- (4.48579,2.03507) -- (4.48579,2.0022) -- (4.58,2.0022) -- (4.58,1.95797) -- (4.67421,1.95797) -- (4.67421,1.99311) -- (4.76842,1.99311) -- (4.76842,1.87212) -- (4.86263,1.87212) --
 (4.86263,1.83413) -- (4.95684,1.83413) -- (4.95684,1.85384) -- (5.05105,1.85384) -- (5.05105,1.78976) -- (5.14526,1.78976) -- (5.14526,1.74601) -- (5.23947,1.74601) -- (5.23947,1.71828) -- (5.33368,1.71828) -- (5.33368,1.69729) -- (5.42789,1.69729)
 -- (5.42789,1.67875) -- (5.52211,1.67875) -- (5.52211,1.65803) -- (5.61632,1.65803) -- (5.61632,1.63198) -- (5.71053,1.63198) -- (5.71053,1.61947) -- (5.80474,1.61947) -- (5.80474,1.60295) -- (5.89895,1.60295) -- (5.89895,1.57962) --
 (5.99316,1.57962) -- (5.99316,1.61249) -- (6.08737,1.61249) -- (6.08737,1.55368) -- (6.18158,1.55368) -- (6.18158,1.53517) -- (6.27579,1.53517) -- (6.27579,1.5218) -- (6.37,1.5218) -- (6.37,1.51619) -- (6.46421,1.51619) -- (6.46421,1.50604) --
 (6.55842,1.50604) -- (6.55842,1.48533) -- (6.65263,1.48533) -- (6.65263,1.47423) -- (6.74684,1.47423) -- (6.74684,1.46987) -- (6.84105,1.46987) -- (6.84105,1.45335) -- (6.93526,1.45335) -- (6.93526,1.44348) -- (7.02947,1.44348) -- (7.02947,1.43393)
 -- (7.12368,1.43393) -- (7.12368,1.42467) -- (7.21789,1.42467) -- (7.21789,1.41563) -- (7.31211,1.41563) -- (7.31211,1.40679) -- (7.40632,1.40679) -- (7.40632,1.39808) -- (7.50053,1.39808) -- (7.50053,1.38946) -- (7.59474,1.38946) --
 (7.59474,1.38089) -- (7.68895,1.38089) -- (7.68895,1.40674) -- (7.78316,1.40674) -- (7.78316,1.3675) -- (7.87737,1.3675) -- (7.87737,1.35494) -- (7.97158,1.35494) -- (7.97158,1.34616) -- (8.06579,1.34616) -- (8.06579,1.33698) -- (8.16,1.33698) --
 (8.16,1.32766) -- (8.25421,1.32766) -- (8.25421,1.31807) -- (8.34842,1.31807) -- (8.34842,1.30816) -- (8.44263,1.30816) -- (8.44263,1.3088) -- (8.53684,1.3088) -- (8.53684,1.28722) -- (8.63105,1.28722) -- (8.63105,1.27613) -- (8.72526,1.27613) --
 (8.72526,1.26458) -- (8.81947,1.26458) -- (8.81947,1.25256) -- (8.91368,1.25256) -- (8.91368,1.24003) -- (9.00789,1.24003) -- (9.00789,1.22698) -- (9.10211,1.22698) -- (9.10211,1.21338) -- (9.19632,1.21338) -- (9.19632,1.19924) -- (9.29053,1.19924)
 -- (9.29053,1.18453) -- (9.38474,1.18453) -- (9.38474,1.16926) -- (9.47895,1.16926) -- (9.47895,1.15342) -- (9.57316,1.15342) -- (9.57316,1.13701) -- (9.66737,1.13701) -- (9.66737,1.12004) -- (9.76158,1.12004) -- (9.76158,1.10251) --
 (9.85579,1.10251) -- (9.85579,1.08444) -- (9.95,1.08444);
\colorlet{c}{natcomp!20};
\draw [c] (1,6.09395) -- (1.09421,6.09395) -- (1.09421,5.87997) -- (1.18842,5.87997) -- (1.18842,5.66086) -- (1.28263,5.66086) -- (1.28263,5.47704) -- (1.37684,5.47704) -- (1.37684,5.28437) -- (1.47105,5.28437) -- (1.47105,5.13281) --
 (1.56526,5.13281) -- (1.56526,4.9964) -- (1.65947,4.9964) -- (1.65947,4.87338) -- (1.75368,4.87338) -- (1.75368,4.70367) -- (1.84789,4.70367) -- (1.84789,4.62418) -- (1.94211,4.62418) -- (1.94211,4.41171) -- (2.03632,4.41171) -- (2.03632,4.32131) --
 (2.13053,4.32131) -- (2.13053,4.13137) -- (2.22474,4.13137) -- (2.22474,4.09559) -- (2.31895,4.09559) -- (2.31895,3.89091) -- (2.41316,3.89091) -- (2.41316,3.80024) -- (2.50737,3.80024) -- (2.50737,3.99282) -- (2.60158,3.99282) -- (2.60158,3.61342)
 -- (2.69579,3.61342) -- (2.69579,3.52898) -- (2.79,3.52898) -- (2.79,3.55939) -- (2.88421,3.55939) -- (2.88421,3.23741) -- (2.97842,3.23741) -- (2.97842,3.40986) -- (3.07263,3.40986) -- (3.07263,3.1694) -- (3.16684,3.1694) -- (3.16684,3.06746) --
 (3.26105,3.06746) -- (3.26105,2.88254) -- (3.35526,2.88254) -- (3.35526,3.37111) -- (3.44947,3.37111) -- (3.44947,2.84084) -- (3.54368,2.84084) -- (3.54368,2.65566) -- (3.63789,2.65566) -- (3.63789,2.75485) -- (3.73211,2.75485) -- (3.73211,2.50071)
 -- (3.82632,2.50071) -- (3.82632,2.43328) -- (3.92053,2.43328) -- (3.92053,2.4162) -- (4.01474,2.4162) -- (4.01474,2.33059) -- (4.10895,2.33059) -- (4.10895,2.3092) -- (4.20316,2.3092) -- (4.20316,2.2182) -- (4.29737,2.2182) -- (4.29737,2.18527) --
 (4.39158,2.18527) -- (4.39158,2.13983) -- (4.48579,2.13983) -- (4.48579,2.11141) -- (4.58,2.11141) -- (4.58,2.0746) -- (4.67421,2.0746) -- (4.67421,2.0984) -- (4.76842,2.0984) -- (4.76842,2.00437) -- (4.86263,2.00437) -- (4.86263,1.97351) --
 (4.95684,1.97351) -- (4.95684,1.98429) -- (5.05105,1.98429) -- (5.05105,1.93495) -- (5.14526,1.93495) -- (5.14526,1.90093) -- (5.23947,1.90093) -- (5.23947,1.87831) -- (5.33368,1.87831) -- (5.33368,1.86044) -- (5.42789,1.86044) -- (5.42789,1.84435)
 -- (5.52211,1.84435) -- (5.52211,1.82693) -- (5.61632,1.82693) -- (5.61632,1.80614) -- (5.71053,1.80614) -- (5.71053,1.79439) -- (5.80474,1.79439) -- (5.80474,1.78012) -- (5.89895,1.78012) -- (5.89895,1.76158) -- (5.99316,1.76158) --
 (5.99316,1.77982) -- (6.08737,1.77982) -- (6.08737,1.7382) -- (6.18158,1.7382) -- (6.18158,1.72315) -- (6.27579,1.72315) -- (6.27579,1.71147) -- (6.37,1.71147) -- (6.37,1.70478) -- (6.46421,1.70478) -- (6.46421,1.6953) -- (6.55842,1.6953) --
 (6.55842,1.67926) -- (6.65263,1.67926) -- (6.65263,1.66935) -- (6.74684,1.66935) -- (6.74684,1.6637) -- (6.84105,1.6637) -- (6.84105,1.65053) -- (6.93526,1.65053) -- (6.93526,1.64155) -- (7.02947,1.64155) -- (7.02947,1.63281) -- (7.12368,1.63281) --
 (7.12368,1.62427) -- (7.21789,1.62427) -- (7.21789,1.61589) -- (7.31211,1.61589) -- (7.31211,1.60762) -- (7.40632,1.60762) -- (7.40632,1.59943) -- (7.50053,1.59943) -- (7.50053,1.59127) -- (7.59474,1.59127) -- (7.59474,1.58311) -- (7.68895,1.58311)
 -- (7.68895,1.5963) -- (7.78316,1.5963) -- (7.78316,1.56892) -- (7.87737,1.56892) -- (7.87737,1.55812) -- (7.97158,1.55812) -- (7.97158,1.54956) -- (8.06579,1.54956) -- (8.06579,1.54065) -- (8.16,1.54065) -- (8.16,1.53154) -- (8.25421,1.53154) --
 (8.25421,1.52214) -- (8.34842,1.52214) -- (8.34842,1.51241) -- (8.44263,1.51241) -- (8.44263,1.509) -- (8.53684,1.509) -- (8.53684,1.49182) -- (8.63105,1.49182) -- (8.63105,1.48091) -- (8.72526,1.48091) -- (8.72526,1.46955) -- (8.81947,1.46955) --
 (8.81947,1.45771) -- (8.91368,1.45771) -- (8.91368,1.44538) -- (9.00789,1.44538) -- (9.00789,1.43255) -- (9.10211,1.43255) -- (9.10211,1.41919) -- (9.19632,1.41919) -- (9.19632,1.4053) -- (9.29053,1.4053) -- (9.29053,1.39087) -- (9.38474,1.39087) --
 (9.38474,1.37589) -- (9.47895,1.37589) -- (9.47895,1.36037) -- (9.57316,1.36037) -- (9.57316,1.34431) -- (9.66737,1.34431) -- (9.66737,1.3277) -- (9.76158,1.3277) -- (9.76158,1.31057) -- (9.85579,1.31057) -- (9.85579,1.29292) -- (9.95,1.29292);
\colorlet{c}{natcomp!20};
\draw [c] (1,6.0939) -- (1.09421,6.0939) -- (1.09421,5.87989) -- (1.18842,5.87989) -- (1.18842,5.66074) -- (1.28263,5.66074) -- (1.28263,5.47686) -- (1.37684,5.47686) -- (1.37684,5.28408) -- (1.47105,5.28408) -- (1.47105,5.13242) --
 (1.56526,5.13242) -- (1.56526,4.99587) -- (1.65947,4.99587) -- (1.65947,4.8727) -- (1.75368,4.8727) -- (1.75368,4.70268) -- (1.84789,4.70268) -- (1.84789,4.62302) -- (1.94211,4.62302) -- (1.94211,4.40985) -- (2.03632,4.40985) -- (2.03632,4.31908) --
 (2.13053,4.31908) -- (2.13053,4.12797) -- (2.22474,4.12797) -- (2.22474,4.09202) -- (2.31895,4.09202) -- (2.31895,3.88526) -- (2.41316,3.88526) -- (2.41316,3.79346) -- (2.50737,3.79346) -- (2.50737,3.98874) -- (2.60158,3.98874) -- (2.60158,3.60351)
 -- (2.69579,3.60351) -- (2.69579,3.51725) -- (2.79,3.51725) -- (2.79,3.54892) -- (2.88421,3.54892) -- (2.88421,3.21511) -- (2.97842,3.21511) -- (2.97842,3.39587) -- (3.07263,3.39587) -- (3.07263,3.14496) -- (3.16684,3.14496) -- (3.16684,3.03711) --
 (3.26105,3.03711) -- (3.26105,2.83589) -- (3.35526,2.83589) -- (3.35526,3.35789) -- (3.44947,3.35789) -- (3.44947,2.79284) -- (3.54368,2.79284) -- (3.54368,2.58085) -- (3.63789,2.58085) -- (3.63789,2.69942) -- (3.73211,2.69942) -- (3.73211,2.39582)
 -- (3.82632,2.39582) -- (3.82632,2.31181) -- (3.92053,2.31181) -- (3.92053,2.2939) -- (4.01474,2.2939) -- (4.01474,2.18061) -- (4.10895,2.18061) -- (4.10895,2.1561) -- (4.20316,2.1561) -- (4.20316,2.02532) -- (4.29737,2.02532) -- (4.29737,1.98071)
 -- (4.39158,1.98071) -- (4.39158,1.91365) -- (4.48579,1.91365) -- (4.48579,1.87411) -- (4.58,1.87411) -- (4.58,1.81823) -- (4.67421,1.81823) -- (4.67421,1.87149) -- (4.76842,1.87149) -- (4.76842,1.70606) -- (4.86263,1.70606) -- (4.86263,1.65521) --
 (4.95684,1.65521) -- (4.95684,1.69163) -- (5.05105,1.69163) -- (5.05105,1.60035) -- (5.14526,1.60035) -- (5.14526,1.53726) -- (5.23947,1.53726) -- (5.23947,1.49903) -- (5.33368,1.49903) -- (5.33368,1.47162) -- (5.42789,1.47162) -- (5.42789,1.44809)
 -- (5.52211,1.44809) -- (5.52211,1.42033) -- (5.61632,1.42033) -- (5.61632,1.38235) -- (5.71053,1.38235) -- (5.71053,1.36854) -- (5.80474,1.36854) -- (5.80474,1.34708) -- (5.89895,1.34708) -- (5.89895,1.31233) -- (5.99316,1.31233) --
 (5.99316,1.38083) -- (6.08737,1.38083) -- (6.08737,1.28091) -- (6.18158,1.28091) -- (6.18158,1.25387) -- (6.27579,1.25387) -- (6.27579,1.23653) -- (6.37,1.23653) -- (6.37,1.23435) -- (6.46421,1.23435) -- (6.46421,1.22292) -- (6.55842,1.22292) --
 (6.55842,1.18993) -- (6.65263,1.18993) -- (6.65263,1.17599) -- (6.74684,1.17599) -- (6.74684,1.17567) -- (6.84105,1.17567) -- (6.84105,1.15014) -- (6.93526,1.15014) -- (6.93526,1.13811) -- (7.02947,1.13811) -- (7.02947,1.12661) -- (7.12368,1.12661)
 -- (7.12368,1.11558) -- (7.21789,1.11558) -- (7.21789,1.10496) -- (7.31211,1.10496) -- (7.31211,1.09469) -- (7.40632,1.09469) -- (7.40632,1.08471) -- (7.50053,1.08471) -- (7.50053,1.07495) -- (7.59474,1.07495) -- (7.59474,1.06537) --
 (7.68895,1.06537) -- (7.68895,1.12676) -- (7.78316,1.12676) -- (7.78316,1.05467) -- (7.87737,1.05467) -- (7.87737,1.03701) -- (7.97158,1.03701) -- (7.97158,1.02772) -- (8.06579,1.02772) -- (8.06579,1.01784) -- (8.16,1.01784) -- (8.16,1.008) --
 (8.25421,1.008) -- (8.25421,0.997928) -- (8.34842,0.997928) -- (8.34842,0.987561) -- (8.44263,0.987561) -- (8.44263,1.00025) -- (8.53684,1.00025) -- (8.53684,0.965765) -- (8.63105,0.965765) -- (8.63105,0.954251) -- (8.72526,0.954251) --
 (8.72526,0.942275) -- (8.81947,0.942275) -- (8.81947,0.929803) -- (8.91368,0.929803) -- (8.91368,0.916805) -- (9.00789,0.916805) -- (9.00789,0.903255) -- (9.10211,0.903255) -- (9.10211,0.889131) -- (9.19632,0.889131) -- (9.19632,0.874415) --
 (9.29053,0.874415) -- (9.29053,0.859095) -- (9.38474,0.859095) -- (9.38474,0.843161) -- (9.47895,0.843161) -- (9.47895,0.826607) -- (9.57316,0.826607) -- (9.57316,0.809433) -- (9.66737,0.809433) -- (9.66737,0.791641) -- (9.76158,0.791641) --
 (9.76158,0.77324) -- (9.85579,0.77324) -- (9.85579,0.75424) -- (9.95,0.75424);
\definecolor{c}{rgb}{0,0,0};
\draw [anchor= west] (6.86605,6.01242) node[color=c, rotate=0]{Data};
\colorlet{c}{natgreen};
\draw [c] (6.12016,6.01242) -- (6.73442,6.01242);
\definecolor{c}{rgb}{0,0,0};
\draw [anchor= west] (6.86605,5.51576) node[color=c, rotate=0]{Max likelihood};
\colorlet{c}{natcomp!60};
\draw [c] (6.12016,5.51576) -- (6.73442,5.51576);
\definecolor{c}{rgb}{0,0,0};
\draw [anchor= west] (6.86605,5.0191) node[color=c, rotate=0]{Conf. interval limits};
\colorlet{c}{natcomp!20};
\draw [c] (6.12016,5.0191) -- (6.73442,5.0191);
\end{tikzpicture}

\end{infilsf}
\end{minipage}
\begin{minipage}[b]{.3\textwidth}
\caption{The data distribution of $M_{\gamma\gamma}$ overlaid with the constructed distributions at the most likely value of $\Lambda$ and at the limits of the confidence level.}\label{result}
\end{minipage}
\end{figure}