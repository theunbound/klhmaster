\chapter{Analysis}
All of the bits so far are put together, and a limit on the size of
$\Lambda$ is found.

In chapter~\ref{ch.mc}, we produced, among others, a sample of Monte Carlo events that gives the SM prediction for the distribution of events. Since we discovered above, that the \atlas{} $\gamma\gamma$ Monte Carlo sample gives a distribution of events that matches data well, we will compare these two samples to asses how well our SM prediction matches data. In doing so, we encounter a few problems.

First, it appears that the procedure which should correct for pileup in the detector simulation procedure has not functioned as intended. During detector simulation, the events produced for this thesis are assigned only a very limited range of values for the number of interactions per bunch crossing. Pileup reweighting produces a weight for each event so that its distribution in a Monte Carlo dataset mathches the distribution found in data.

Since this appears to be a technical glitch, we will substitute a reweighting in the number of reconstructed primary vertices---vertices considered to originate directly from interactions between protons---in each event. The distribution of numbers of primary vertices in the present MC data set is compared to the one found in the \atlas{} MC set in figure~\ref{pvnnone}.

\begin{figure}[htp]
\begin{minipage}[b]{.69\textwidth}
\begin{infilsf} \tiny
\begin{tikzpicture}[x=.092\textwidth,y=.092\textwidth]
\pgfdeclareplotmark{cross} {
\pgfpathmoveto{\pgfpoint{-0.3\pgfplotmarksize}{\pgfplotmarksize}}
\pgfpathlineto{\pgfpoint{+0.3\pgfplotmarksize}{\pgfplotmarksize}}
\pgfpathlineto{\pgfpoint{+0.3\pgfplotmarksize}{0.3\pgfplotmarksize}}
\pgfpathlineto{\pgfpoint{+1\pgfplotmarksize}{0.3\pgfplotmarksize}}
\pgfpathlineto{\pgfpoint{+1\pgfplotmarksize}{-0.3\pgfplotmarksize}}
\pgfpathlineto{\pgfpoint{+0.3\pgfplotmarksize}{-0.3\pgfplotmarksize}}
\pgfpathlineto{\pgfpoint{+0.3\pgfplotmarksize}{-1.\pgfplotmarksize}}
\pgfpathlineto{\pgfpoint{-0.3\pgfplotmarksize}{-1.\pgfplotmarksize}}
\pgfpathlineto{\pgfpoint{-0.3\pgfplotmarksize}{-0.3\pgfplotmarksize}}
\pgfpathlineto{\pgfpoint{-1.\pgfplotmarksize}{-0.3\pgfplotmarksize}}
\pgfpathlineto{\pgfpoint{-1.\pgfplotmarksize}{0.3\pgfplotmarksize}}
\pgfpathlineto{\pgfpoint{-0.3\pgfplotmarksize}{0.3\pgfplotmarksize}}
\pgfpathclose
\pgfusepathqstroke
}
\pgfdeclareplotmark{cross*} {
\pgfpathmoveto{\pgfpoint{-0.3\pgfplotmarksize}{\pgfplotmarksize}}
\pgfpathlineto{\pgfpoint{+0.3\pgfplotmarksize}{\pgfplotmarksize}}
\pgfpathlineto{\pgfpoint{+0.3\pgfplotmarksize}{0.3\pgfplotmarksize}}
\pgfpathlineto{\pgfpoint{+1\pgfplotmarksize}{0.3\pgfplotmarksize}}
\pgfpathlineto{\pgfpoint{+1\pgfplotmarksize}{-0.3\pgfplotmarksize}}
\pgfpathlineto{\pgfpoint{+0.3\pgfplotmarksize}{-0.3\pgfplotmarksize}}
\pgfpathlineto{\pgfpoint{+0.3\pgfplotmarksize}{-1.\pgfplotmarksize}}
\pgfpathlineto{\pgfpoint{-0.3\pgfplotmarksize}{-1.\pgfplotmarksize}}
\pgfpathlineto{\pgfpoint{-0.3\pgfplotmarksize}{-0.3\pgfplotmarksize}}
\pgfpathlineto{\pgfpoint{-1.\pgfplotmarksize}{-0.3\pgfplotmarksize}}
\pgfpathlineto{\pgfpoint{-1.\pgfplotmarksize}{0.3\pgfplotmarksize}}
\pgfpathlineto{\pgfpoint{-0.3\pgfplotmarksize}{0.3\pgfplotmarksize}}
\pgfpathclose
\pgfusepathqfillstroke
}
\pgfdeclareplotmark{newstar} {
\pgfpathmoveto{\pgfqpoint{0pt}{\pgfplotmarksize}}
\pgfpathlineto{\pgfqpointpolar{44}{0.5\pgfplotmarksize}}
\pgfpathlineto{\pgfqpointpolar{18}{\pgfplotmarksize}}
\pgfpathlineto{\pgfqpointpolar{-20}{0.5\pgfplotmarksize}}
\pgfpathlineto{\pgfqpointpolar{-54}{\pgfplotmarksize}}
\pgfpathlineto{\pgfqpointpolar{-90}{0.5\pgfplotmarksize}}
\pgfpathlineto{\pgfqpointpolar{234}{\pgfplotmarksize}}
\pgfpathlineto{\pgfqpointpolar{198}{0.5\pgfplotmarksize}}
\pgfpathlineto{\pgfqpointpolar{162}{\pgfplotmarksize}}
\pgfpathlineto{\pgfqpointpolar{134}{0.5\pgfplotmarksize}}
\pgfpathclose
\pgfusepathqstroke
}
\pgfdeclareplotmark{newstar*} {
\pgfpathmoveto{\pgfqpoint{0pt}{\pgfplotmarksize}}
\pgfpathlineto{\pgfqpointpolar{44}{0.5\pgfplotmarksize}}
\pgfpathlineto{\pgfqpointpolar{18}{\pgfplotmarksize}}
\pgfpathlineto{\pgfqpointpolar{-20}{0.5\pgfplotmarksize}}
\pgfpathlineto{\pgfqpointpolar{-54}{\pgfplotmarksize}}
\pgfpathlineto{\pgfqpointpolar{-90}{0.5\pgfplotmarksize}}
\pgfpathlineto{\pgfqpointpolar{234}{\pgfplotmarksize}}
\pgfpathlineto{\pgfqpointpolar{198}{0.5\pgfplotmarksize}}
\pgfpathlineto{\pgfqpointpolar{162}{\pgfplotmarksize}}
\pgfpathlineto{\pgfqpointpolar{134}{0.5\pgfplotmarksize}}
\pgfpathclose
\pgfusepathqfillstroke
}
\definecolor{c}{rgb}{1,1,1};
\draw [color=c, fill=c] (0,0) rectangle (10,6.80516);
\draw [color=c, fill=c] (1,0.680516) rectangle (9.95,6.73711);
\definecolor{c}{rgb}{0,0,0};
\draw [c] (1,0.680516) -- (1,6.73711) -- (9.95,6.73711) -- (9.95,0.680516) -- (1,0.680516);
\definecolor{c}{rgb}{1,1,1};
\draw [color=c, fill=c] (1,0.680516) rectangle (9.95,6.73711);
\definecolor{c}{rgb}{0,0,0};
\draw [c] (1,0.680516) -- (1,6.73711) -- (9.95,6.73711) -- (9.95,0.680516) -- (1,0.680516);
\colorlet{c}{natcomp!70};
\draw [c] (2.01048,0.686784) -- (2.01048,0.695341);
\draw [c] (2.01048,0.695341) -- (2.01048,0.703897);
\draw [c] (1.86613,0.695341) -- (2.01048,0.695341);
\draw [c] (2.01048,0.695341) -- (2.15484,0.695341);
\definecolor{c}{rgb}{0,0,0};
\colorlet{c}{natcomp!70};
\draw [c] (2.29919,0.680526) -- (2.29919,0.680532);
\draw [c] (2.29919,0.680532) -- (2.29919,0.680538);
\draw [c] (2.15484,0.680532) -- (2.29919,0.680532);
\draw [c] (2.29919,0.680532) -- (2.44355,0.680532);
\definecolor{c}{rgb}{0,0,0};
\colorlet{c}{natcomp!70};
\draw [c] (2.5879,0.752927) -- (2.5879,0.77446);
\draw [c] (2.5879,0.77446) -- (2.5879,0.795993);
\draw [c] (2.44355,0.77446) -- (2.5879,0.77446);
\draw [c] (2.5879,0.77446) -- (2.73226,0.77446);
\definecolor{c}{rgb}{0,0,0};
\colorlet{c}{natcomp!70};
\draw [c] (2.87661,0.943483) -- (2.87661,0.982067);
\draw [c] (2.87661,0.982067) -- (2.87661,1.02065);
\draw [c] (2.73226,0.982067) -- (2.87661,0.982067);
\draw [c] (2.87661,0.982067) -- (3.02097,0.982067);
\definecolor{c}{rgb}{0,0,0};
\colorlet{c}{natcomp!70};
\draw [c] (3.16532,1.29053) -- (3.16532,1.34793);
\draw [c] (3.16532,1.34793) -- (3.16532,1.40533);
\draw [c] (3.02097,1.34793) -- (3.16532,1.34793);
\draw [c] (3.16532,1.34793) -- (3.30968,1.34793);
\definecolor{c}{rgb}{0,0,0};
\colorlet{c}{natcomp!70};
\draw [c] (3.45403,1.93426) -- (3.45403,2.01543);
\draw [c] (3.45403,2.01543) -- (3.45403,2.09661);
\draw [c] (3.30968,2.01543) -- (3.45403,2.01543);
\draw [c] (3.45403,2.01543) -- (3.59839,2.01543);
\definecolor{c}{rgb}{0,0,0};
\colorlet{c}{natcomp!70};
\draw [c] (3.74274,2.71861) -- (3.74274,2.82141);
\draw [c] (3.74274,2.82141) -- (3.74274,2.9242);
\draw [c] (3.59839,2.82141) -- (3.74274,2.82141);
\draw [c] (3.74274,2.82141) -- (3.8871,2.82141);
\definecolor{c}{rgb}{0,0,0};
\colorlet{c}{natcomp!70};
\draw [c] (4.03145,4.11719) -- (4.03145,4.24993);
\draw [c] (4.03145,4.24993) -- (4.03145,4.38267);
\draw [c] (3.8871,4.24993) -- (4.03145,4.24993);
\draw [c] (4.03145,4.24993) -- (4.17581,4.24993);
\definecolor{c}{rgb}{0,0,0};
\colorlet{c}{natcomp!70};
\draw [c] (4.32016,4.63729) -- (4.32016,4.77952);
\draw [c] (4.32016,4.77952) -- (4.32016,4.92176);
\draw [c] (4.17581,4.77952) -- (4.32016,4.77952);
\draw [c] (4.32016,4.77952) -- (4.46452,4.77952);
\definecolor{c}{rgb}{0,0,0};
\colorlet{c}{natcomp!70};
\draw [c] (4.60887,5.6146) -- (4.60887,5.77315);
\draw [c] (4.60887,5.77315) -- (4.60887,5.93169);
\draw [c] (4.46452,5.77315) -- (4.60887,5.77315);
\draw [c] (4.60887,5.77315) -- (4.75323,5.77315);
\definecolor{c}{rgb}{0,0,0};
\colorlet{c}{natcomp!70};
\draw [c] (4.89758,6.11613) -- (4.89758,6.28241);
\draw [c] (4.89758,6.28241) -- (4.89758,6.4487);
\draw [c] (4.75323,6.28241) -- (4.89758,6.28241);
\draw [c] (4.89758,6.28241) -- (5.04194,6.28241);
\definecolor{c}{rgb}{0,0,0};
\colorlet{c}{natcomp!70};
\draw [c] (5.18629,5.576) -- (5.18629,5.73393);
\draw [c] (5.18629,5.73393) -- (5.18629,5.89186);
\draw [c] (5.04194,5.73393) -- (5.18629,5.73393);
\draw [c] (5.18629,5.73393) -- (5.33065,5.73393);
\definecolor{c}{rgb}{0,0,0};
\colorlet{c}{natcomp!70};
\draw [c] (5.475,5.47333) -- (5.475,5.62963);
\draw [c] (5.475,5.62963) -- (5.475,5.78593);
\draw [c] (5.33065,5.62963) -- (5.475,5.62963);
\draw [c] (5.475,5.62963) -- (5.61935,5.62963);
\definecolor{c}{rgb}{0,0,0};
\colorlet{c}{natcomp!70};
\draw [c] (5.76371,4.43808) -- (5.76371,4.57675);
\draw [c] (5.76371,4.57675) -- (5.76371,4.71543);
\draw [c] (5.61935,4.57675) -- (5.76371,4.57675);
\draw [c] (5.76371,4.57675) -- (5.90806,4.57675);
\definecolor{c}{rgb}{0,0,0};
\colorlet{c}{natcomp!70};
\draw [c] (6.05242,3.40517) -- (6.05242,3.52363);
\draw [c] (6.05242,3.52363) -- (6.05242,3.64209);
\draw [c] (5.90806,3.52363) -- (6.05242,3.52363);
\draw [c] (6.05242,3.52363) -- (6.19677,3.52363);
\definecolor{c}{rgb}{0,0,0};
\colorlet{c}{natcomp!70};
\draw [c] (6.34113,2.64159) -- (6.34113,2.74247);
\draw [c] (6.34113,2.74247) -- (6.34113,2.84335);
\draw [c] (6.19677,2.74247) -- (6.34113,2.74247);
\draw [c] (6.34113,2.74247) -- (6.48548,2.74247);
\definecolor{c}{rgb}{0,0,0};
\colorlet{c}{natcomp!70};
\draw [c] (6.62984,2.05919) -- (6.62984,2.14419);
\draw [c] (6.62984,2.14419) -- (6.62984,2.22918);
\draw [c] (6.48548,2.14419) -- (6.62984,2.14419);
\draw [c] (6.62984,2.14419) -- (6.77419,2.14419);
\definecolor{c}{rgb}{0,0,0};
\colorlet{c}{natcomp!70};
\draw [c] (6.91855,1.74752) -- (6.91855,1.8226);
\draw [c] (6.91855,1.8226) -- (6.91855,1.89768);
\draw [c] (6.77419,1.8226) -- (6.91855,1.8226);
\draw [c] (6.91855,1.8226) -- (7.0629,1.8226);
\definecolor{c}{rgb}{0,0,0};
\colorlet{c}{natcomp!70};
\draw [c] (7.20726,1.16326) -- (7.20726,1.2146);
\draw [c] (7.20726,1.2146) -- (7.20726,1.26594);
\draw [c] (7.0629,1.2146) -- (7.20726,1.2146);
\draw [c] (7.20726,1.2146) -- (7.35161,1.2146);
\definecolor{c}{rgb}{0,0,0};
\colorlet{c}{natcomp!70};
\draw [c] (7.49597,0.925145) -- (7.49597,0.962442);
\draw [c] (7.49597,0.962442) -- (7.49597,0.999739);
\draw [c] (7.35161,0.962442) -- (7.49597,0.962442);
\draw [c] (7.49597,0.962442) -- (7.64032,0.962442);
\definecolor{c}{rgb}{0,0,0};
\colorlet{c}{natcomp!70};
\draw [c] (7.78468,0.847054) -- (7.78468,0.878298);
\draw [c] (7.78468,0.878298) -- (7.78468,0.909542);
\draw [c] (7.64032,0.878298) -- (7.78468,0.878298);
\draw [c] (7.78468,0.878298) -- (7.92903,0.878298);
\definecolor{c}{rgb}{0,0,0};
\colorlet{c}{natcomp!70};
\draw [c] (8.07339,0.7399) -- (8.07339,0.759661);
\draw [c] (8.07339,0.759661) -- (8.07339,0.779421);
\draw [c] (7.92903,0.759661) -- (8.07339,0.759661);
\draw [c] (8.07339,0.759661) -- (8.21774,0.759661);
\definecolor{c}{rgb}{0,0,0};
\colorlet{c}{natcomp!70};
\draw [c] (8.3621,0.710206) -- (8.3621,0.725026);
\draw [c] (8.3621,0.725026) -- (8.3621,0.739846);
\draw [c] (8.21774,0.725026) -- (8.3621,0.725026);
\draw [c] (8.3621,0.725026) -- (8.50645,0.725026);
\definecolor{c}{rgb}{0,0,0};
\colorlet{c}{natcomp!70};
\draw [c] (8.65081,0.686798) -- (8.65081,0.695355);
\draw [c] (8.65081,0.695355) -- (8.65081,0.703911);
\draw [c] (8.50645,0.695355) -- (8.65081,0.695355);
\draw [c] (8.65081,0.695355) -- (8.79516,0.695355);
\definecolor{c}{rgb}{0,0,0};
\colorlet{c}{natcomp!70};
\draw [c] (8.93952,0.683424) -- (8.93952,0.69041);
\draw [c] (8.93952,0.69041) -- (8.93952,0.697396);
\draw [c] (8.79516,0.69041) -- (8.93952,0.69041);
\draw [c] (8.93952,0.69041) -- (9.08387,0.69041);
\definecolor{c}{rgb}{0,0,0};
\colorlet{c}{natcomp!70};
\draw [c] (9.22823,0.690398) -- (9.22823,0.700279);
\draw [c] (9.22823,0.700279) -- (9.22823,0.710159);
\draw [c] (9.08387,0.700279) -- (9.22823,0.700279);
\draw [c] (9.22823,0.700279) -- (9.37258,0.700279);
\definecolor{c}{rgb}{0,0,0};
\colorlet{c}{natcomp!70};
\draw [c] (9.80564,0.680517) -- (9.80564,0.68052);
\draw [c] (9.80564,0.68052) -- (9.80564,0.680524);
\draw [c] (9.66129,0.68052) -- (9.80564,0.68052);
\draw [c] (9.80564,0.68052) -- (9.95,0.68052);
\definecolor{c}{rgb}{0,0,0};
\draw [c] (1,0.680516) -- (9.95,0.680516);
\draw [anchor= east] (9.95,0.108883) node[color=c, rotate=0]{Number of primary vertices};
\draw [c] (1.14435,0.863234) -- (1.14435,0.680516);
\draw [c] (1.43306,0.771875) -- (1.43306,0.680516);
\draw [c] (1.72177,0.771875) -- (1.72177,0.680516);
\draw [c] (2.01048,0.771875) -- (2.01048,0.680516);
\draw [c] (2.29919,0.771875) -- (2.29919,0.680516);
\draw [c] (2.5879,0.863234) -- (2.5879,0.680516);
\draw [c] (2.87661,0.771875) -- (2.87661,0.680516);
\draw [c] (3.16532,0.771875) -- (3.16532,0.680516);
\draw [c] (3.45403,0.771875) -- (3.45403,0.680516);
\draw [c] (3.74274,0.771875) -- (3.74274,0.680516);
\draw [c] (4.03145,0.863234) -- (4.03145,0.680516);
\draw [c] (4.32016,0.771875) -- (4.32016,0.680516);
\draw [c] (4.60887,0.771875) -- (4.60887,0.680516);
\draw [c] (4.89758,0.771875) -- (4.89758,0.680516);
\draw [c] (5.18629,0.771875) -- (5.18629,0.680516);
\draw [c] (5.475,0.863234) -- (5.475,0.680516);
\draw [c] (5.76371,0.771875) -- (5.76371,0.680516);
\draw [c] (6.05242,0.771875) -- (6.05242,0.680516);
\draw [c] (6.34113,0.771875) -- (6.34113,0.680516);
\draw [c] (6.62984,0.771875) -- (6.62984,0.680516);
\draw [c] (6.91855,0.863234) -- (6.91855,0.680516);
\draw [c] (7.20726,0.771875) -- (7.20726,0.680516);
\draw [c] (7.49597,0.771875) -- (7.49597,0.680516);
\draw [c] (7.78468,0.771875) -- (7.78468,0.680516);
\draw [c] (8.07339,0.771875) -- (8.07339,0.680516);
\draw [c] (8.3621,0.863234) -- (8.3621,0.680516);
\draw [c] (8.65081,0.771875) -- (8.65081,0.680516);
\draw [c] (8.93952,0.771875) -- (8.93952,0.680516);
\draw [c] (9.22823,0.771875) -- (9.22823,0.680516);
\draw [c] (9.51694,0.771875) -- (9.51694,0.680516);
\draw [c] (9.80564,0.863234) -- (9.80564,0.680516);
\draw [c] (1.14435,0.863234) -- (1.14435,0.680516);
\draw [c] (9.80564,0.863234) -- (9.80564,0.680516);
\draw [anchor=base] (1.14435,0.353868) node[color=c, rotate=0]{0};
\draw [anchor=base] (2.5879,0.353868) node[color=c, rotate=0]{5};
\draw [anchor=base] (4.03145,0.353868) node[color=c, rotate=0]{10};
\draw [anchor=base] (5.475,0.353868) node[color=c, rotate=0]{15};
\draw [anchor=base] (6.91855,0.353868) node[color=c, rotate=0]{20};
\draw [anchor=base] (8.3621,0.353868) node[color=c, rotate=0]{25};
\draw [anchor=base] (9.80564,0.353868) node[color=c, rotate=0]{30};
\draw [c] (1,0.680516) -- (1,6.73711);
\draw [anchor= east] (-0.12,6.73711) node[color=c, rotate=90]{Normalised number of events};
\draw [c] (1.267,0.680516) -- (1,0.680516);
\draw [c] (1.1335,0.878524) -- (1,0.878524);
\draw [c] (1.1335,1.07653) -- (1,1.07653);
\draw [c] (1.1335,1.27454) -- (1,1.27454);
\draw [c] (1.1335,1.47255) -- (1,1.47255);
\draw [c] (1.267,1.67056) -- (1,1.67056);
\draw [c] (1.1335,1.86857) -- (1,1.86857);
\draw [c] (1.1335,2.06657) -- (1,2.06657);
\draw [c] (1.1335,2.26458) -- (1,2.26458);
\draw [c] (1.1335,2.46259) -- (1,2.46259);
\draw [c] (1.267,2.6606) -- (1,2.6606);
\draw [c] (1.1335,2.85861) -- (1,2.85861);
\draw [c] (1.1335,3.05662) -- (1,3.05662);
\draw [c] (1.1335,3.25462) -- (1,3.25462);
\draw [c] (1.1335,3.45263) -- (1,3.45263);
\draw [c] (1.267,3.65064) -- (1,3.65064);
\draw [c] (1.1335,3.84865) -- (1,3.84865);
\draw [c] (1.1335,4.04666) -- (1,4.04666);
\draw [c] (1.1335,4.24467) -- (1,4.24467);
\draw [c] (1.1335,4.44267) -- (1,4.44267);
\draw [c] (1.267,4.64068) -- (1,4.64068);
\draw [c] (1.1335,4.83869) -- (1,4.83869);
\draw [c] (1.1335,5.0367) -- (1,5.0367);
\draw [c] (1.1335,5.23471) -- (1,5.23471);
\draw [c] (1.1335,5.43272) -- (1,5.43272);
\draw [c] (1.267,5.63072) -- (1,5.63072);
\draw [c] (1.1335,5.82873) -- (1,5.82873);
\draw [c] (1.1335,6.02674) -- (1,6.02674);
\draw [c] (1.1335,6.22475) -- (1,6.22475);
\draw [c] (1.1335,6.42276) -- (1,6.42276);
\draw [c] (1.267,6.62077) -- (1,6.62077);
\draw [c] (1.267,6.62077) -- (1,6.62077);
\draw [anchor= east] (0.95,0.680516) node[color=c, rotate=0]{0};
\draw [anchor= east] (0.95,1.67056) node[color=c, rotate=0]{500};
\draw [anchor= east] (0.95,2.6606) node[color=c, rotate=0]{1000};
\draw [anchor= east] (0.95,3.65064) node[color=c, rotate=0]{1500};
\draw [anchor= east] (0.95,4.64068) node[color=c, rotate=0]{2000};
\draw [anchor= east] (0.95,5.63072) node[color=c, rotate=0]{2500};
\draw [anchor= east] (0.95,6.62077) node[color=c, rotate=0]{3000};
\colorlet{c}{natgreen};
\draw [c] (1.72177,0.680523) -- (1.72177,0.681242);
\draw [c] (1.72177,0.681242) -- (1.72177,0.68196);
\draw [c] (1.57742,0.681242) -- (1.72177,0.681242);
\draw [c] (1.72177,0.681242) -- (1.86613,0.681242);
\definecolor{c}{rgb}{0,0,0};
\colorlet{c}{natgreen};
\draw [c] (2.01048,0.689073) -- (2.01048,0.696007);
\draw [c] (2.01048,0.696007) -- (2.01048,0.702941);
\draw [c] (1.86613,0.696007) -- (2.01048,0.696007);
\draw [c] (2.01048,0.696007) -- (2.15484,0.696007);
\definecolor{c}{rgb}{0,0,0};
\colorlet{c}{natgreen};
\draw [c] (2.29919,0.716299) -- (2.29919,0.731686);
\draw [c] (2.29919,0.731686) -- (2.29919,0.747074);
\draw [c] (2.15484,0.731686) -- (2.29919,0.731686);
\draw [c] (2.29919,0.731686) -- (2.44355,0.731686);
\definecolor{c}{rgb}{0,0,0};
\colorlet{c}{natgreen};
\draw [c] (2.5879,0.871864) -- (2.5879,0.902068);
\draw [c] (2.5879,0.902068) -- (2.5879,0.932273);
\draw [c] (2.44355,0.902068) -- (2.5879,0.902068);
\draw [c] (2.5879,0.902068) -- (2.73226,0.902068);
\definecolor{c}{rgb}{0,0,0};
\colorlet{c}{natgreen};
\draw [c] (2.87661,1.14594) -- (2.87661,1.19239);
\draw [c] (2.87661,1.19239) -- (2.87661,1.23883);
\draw [c] (2.73226,1.19239) -- (2.87661,1.19239);
\draw [c] (2.87661,1.19239) -- (3.02097,1.19239);
\definecolor{c}{rgb}{0,0,0};
\colorlet{c}{natgreen};
\draw [c] (3.16532,1.64654) -- (3.16532,1.715);
\draw [c] (3.16532,1.715) -- (3.16532,1.78347);
\draw [c] (3.02097,1.715) -- (3.16532,1.715);
\draw [c] (3.16532,1.715) -- (3.30968,1.715);
\definecolor{c}{rgb}{0,0,0};
\colorlet{c}{natgreen};
\draw [c] (3.45403,2.17071) -- (3.45403,2.25603);
\draw [c] (3.45403,2.25603) -- (3.45403,2.34135);
\draw [c] (3.30968,2.25603) -- (3.45403,2.25603);
\draw [c] (3.45403,2.25603) -- (3.59839,2.25603);
\definecolor{c}{rgb}{0,0,0};
\colorlet{c}{natgreen};
\draw [c] (3.74274,3.01706) -- (3.74274,3.12356);
\draw [c] (3.74274,3.12356) -- (3.74274,3.23006);
\draw [c] (3.59839,3.12356) -- (3.74274,3.12356);
\draw [c] (3.74274,3.12356) -- (3.8871,3.12356);
\definecolor{c}{rgb}{0,0,0};
\colorlet{c}{natgreen};
\draw [c] (4.03145,3.62689) -- (4.03145,3.74661);
\draw [c] (4.03145,3.74661) -- (4.03145,3.86633);
\draw [c] (3.8871,3.74661) -- (4.03145,3.74661);
\draw [c] (4.03145,3.74661) -- (4.17581,3.74661);
\definecolor{c}{rgb}{0,0,0};
\colorlet{c}{natgreen};
\draw [c] (4.32016,4.27462) -- (4.32016,4.40545);
\draw [c] (4.32016,4.40545) -- (4.32016,4.53628);
\draw [c] (4.17581,4.40545) -- (4.32016,4.40545);
\draw [c] (4.32016,4.40545) -- (4.46452,4.40545);
\definecolor{c}{rgb}{0,0,0};
\colorlet{c}{natgreen};
\draw [c] (4.60887,4.55969) -- (4.60887,4.69362);
\draw [c] (4.60887,4.69362) -- (4.60887,4.82755);
\draw [c] (4.46452,4.69362) -- (4.60887,4.69362);
\draw [c] (4.60887,4.69362) -- (4.75323,4.69362);
\definecolor{c}{rgb}{0,0,0};
\colorlet{c}{natgreen};
\draw [c] (4.89758,4.53815) -- (4.89758,4.66935);
\draw [c] (4.89758,4.66935) -- (4.89758,4.80056);
\draw [c] (4.75323,4.66935) -- (4.89758,4.66935);
\draw [c] (4.89758,4.66935) -- (5.04194,4.66935);
\definecolor{c}{rgb}{0,0,0};
\colorlet{c}{natgreen};
\draw [c] (5.18629,4.99176) -- (5.18629,5.12865);
\draw [c] (5.18629,5.12865) -- (5.18629,5.26553);
\draw [c] (5.04194,5.12865) -- (5.18629,5.12865);
\draw [c] (5.18629,5.12865) -- (5.33065,5.12865);
\definecolor{c}{rgb}{0,0,0};
\colorlet{c}{natgreen};
\draw [c] (5.475,4.29737) -- (5.475,4.41993);
\draw [c] (5.475,4.41993) -- (5.475,4.54248);
\draw [c] (5.33065,4.41993) -- (5.475,4.41993);
\draw [c] (5.475,4.41993) -- (5.61935,4.41993);
\definecolor{c}{rgb}{0,0,0};
\colorlet{c}{natgreen};
\draw [c] (5.76371,4.12649) -- (5.76371,4.24387);
\draw [c] (5.76371,4.24387) -- (5.76371,4.36126);
\draw [c] (5.61935,4.24387) -- (5.76371,4.24387);
\draw [c] (5.76371,4.24387) -- (5.90806,4.24387);
\definecolor{c}{rgb}{0,0,0};
\colorlet{c}{natgreen};
\draw [c] (6.05242,3.79458) -- (6.05242,3.90367);
\draw [c] (6.05242,3.90367) -- (6.05242,4.01276);
\draw [c] (5.90806,3.90367) -- (6.05242,3.90367);
\draw [c] (6.05242,3.90367) -- (6.19677,3.90367);
\definecolor{c}{rgb}{0,0,0};
\colorlet{c}{natgreen};
\draw [c] (6.34113,3.46262) -- (6.34113,3.56305);
\draw [c] (6.34113,3.56305) -- (6.34113,3.66348);
\draw [c] (6.19677,3.56305) -- (6.34113,3.56305);
\draw [c] (6.34113,3.56305) -- (6.48548,3.56305);
\definecolor{c}{rgb}{0,0,0};
\colorlet{c}{natgreen};
\draw [c] (6.62984,2.62966) -- (6.62984,2.71154);
\draw [c] (6.62984,2.71154) -- (6.62984,2.79342);
\draw [c] (6.48548,2.71154) -- (6.62984,2.71154);
\draw [c] (6.62984,2.71154) -- (6.77419,2.71154);
\definecolor{c}{rgb}{0,0,0};
\colorlet{c}{natgreen};
\draw [c] (6.91855,2.28926) -- (6.91855,2.36189);
\draw [c] (6.91855,2.36189) -- (6.91855,2.43452);
\draw [c] (6.77419,2.36189) -- (6.91855,2.36189);
\draw [c] (6.91855,2.36189) -- (7.0629,2.36189);
\definecolor{c}{rgb}{0,0,0};
\colorlet{c}{natgreen};
\draw [c] (7.20726,1.66134) -- (7.20726,1.71656);
\draw [c] (7.20726,1.71656) -- (7.20726,1.77178);
\draw [c] (7.0629,1.71656) -- (7.20726,1.71656);
\draw [c] (7.20726,1.71656) -- (7.35161,1.71656);
\definecolor{c}{rgb}{0,0,0};
\colorlet{c}{natgreen};
\draw [c] (7.49597,1.50108) -- (7.49597,1.55148);
\draw [c] (7.49597,1.55148) -- (7.49597,1.60188);
\draw [c] (7.35161,1.55148) -- (7.49597,1.55148);
\draw [c] (7.49597,1.55148) -- (7.64032,1.55148);
\definecolor{c}{rgb}{0,0,0};
\colorlet{c}{natgreen};
\draw [c] (7.78468,1.21206) -- (7.78468,1.25203);
\draw [c] (7.78468,1.25203) -- (7.78468,1.292);
\draw [c] (7.64032,1.25203) -- (7.78468,1.25203);
\draw [c] (7.78468,1.25203) -- (7.92903,1.25203);
\definecolor{c}{rgb}{0,0,0};
\colorlet{c}{natgreen};
\draw [c] (8.07339,1.01304) -- (8.07339,1.04329);
\draw [c] (8.07339,1.04329) -- (8.07339,1.07354);
\draw [c] (7.92903,1.04329) -- (8.07339,1.04329);
\draw [c] (8.07339,1.04329) -- (8.21774,1.04329);
\definecolor{c}{rgb}{0,0,0};
\colorlet{c}{natgreen};
\draw [c] (8.3621,0.847918) -- (8.3621,0.869254);
\draw [c] (8.3621,0.869254) -- (8.3621,0.890591);
\draw [c] (8.21774,0.869254) -- (8.3621,0.869254);
\draw [c] (8.3621,0.869254) -- (8.50645,0.869254);
\definecolor{c}{rgb}{0,0,0};
\colorlet{c}{natgreen};
\draw [c] (8.65081,0.787574) -- (8.65081,0.80412);
\draw [c] (8.65081,0.80412) -- (8.65081,0.820665);
\draw [c] (8.50645,0.80412) -- (8.65081,0.80412);
\draw [c] (8.65081,0.80412) -- (8.79516,0.80412);
\definecolor{c}{rgb}{0,0,0};
\colorlet{c}{natgreen};
\draw [c] (8.93952,0.739471) -- (8.93952,0.751966);
\draw [c] (8.93952,0.751966) -- (8.93952,0.76446);
\draw [c] (8.79516,0.751966) -- (8.93952,0.751966);
\draw [c] (8.93952,0.751966) -- (9.08387,0.751966);
\definecolor{c}{rgb}{0,0,0};
\colorlet{c}{natgreen};
\draw [c] (9.22823,0.713252) -- (9.22823,0.722645);
\draw [c] (9.22823,0.722645) -- (9.22823,0.732039);
\draw [c] (9.08387,0.722645) -- (9.22823,0.722645);
\draw [c] (9.22823,0.722645) -- (9.37258,0.722645);
\definecolor{c}{rgb}{0,0,0};
\colorlet{c}{natgreen};
\draw [c] (9.51694,0.695598) -- (9.51694,0.701995);
\draw [c] (9.51694,0.701995) -- (9.51694,0.708392);
\draw [c] (9.37258,0.701995) -- (9.51694,0.701995);
\draw [c] (9.51694,0.701995) -- (9.66129,0.701995);
\definecolor{c}{rgb}{0,0,0};
\colorlet{c}{natgreen};
\draw [c] (9.80564,0.682123) -- (9.80564,0.68405);
\draw [c] (9.80564,0.68405) -- (9.80564,0.685978);
\draw [c] (9.66129,0.68405) -- (9.80564,0.68405);
\draw [c] (9.80564,0.68405) -- (9.95,0.68405);
\definecolor{c}{rgb}{0,0,0};
\draw [anchor=base west] (6.96633,6.17962) node[color=c, rotate=0]{ATLAS MC};
\colorlet{c}{natgreen};
\draw [c] (6.13521,6.27149) -- (6.81966,6.27149);
\draw [c] (6.47744,6.149) -- (6.47744,6.39398);
\definecolor{c}{rgb}{0,0,0};
\draw [anchor=base west] (6.96633,5.77131) node[color=c, rotate=0]{Our MC};
\colorlet{c}{natcomp!70};
\draw [c] (6.13521,5.86318) -- (6.81966,5.86318);
\draw [c] (6.47744,5.74069) -- (6.47744,5.98567);
\end{tikzpicture}

\end{infilsf}
\end{minipage}
\begin{minipage}[b]{.3\textwidth}
\caption{The distribution of the number of reconstructed primary vertices in the \atlas{} $\gamma\gamma$ MC set and in the CalcHEP MC set produced for this thesis, normalised to the same number of events.}\label{pvnnone}
\end{minipage}
\end{figure}

[Uncertainty...] [Also, effect on Mgg...?]

The second issue we encounter




. The \atlas{} Monte Carlo set, which we found above reproduces the distribution of events in data well, also includes the process described by the box diagram, shown in fig.~\ref{hiorder}. So to compare the MC set from chapter~\ref{ch.mc} with the \atlas{} MC set, and with data, we must know the contribution from the box diagram. 