\chapter{Analysis}\label{ch.an}

In chapter~\ref{ch.mc}, we produced, among others, a sample of Monte Carlo events that gives the SM prediction for the distribution of events. Since we discovered above, that the \atlas{} $\gamma\gamma$ Monte Carlo sample gives a distribution of events that matches data well, we will compare these two samples to asses how well our SM prediction matches data. In doing so, we encounter a few problems.

First, it appears that the procedure which should correct for pileup in the detector simulation procedure has not functioned as intended. During detector simulation, the events produced for this thesis are assigned only a very limited range of values for the number of interactions per bunch crossing. Pileup reweighting produces a weight for each event so that its distribution in a Monte Carlo dataset mathches the distribution found in data.

Since this appears to be a technical glitch, we will substitute a reweighting in the number of reconstructed primary vertices---vertices considered to originate directly from interactions between protons---in each event. The distribution of numbers of primary vertices in the present MC data set is compared to the one found in the \atlas{} MC set in figure~\ref{pvnnone}.

\begin{figure}[htp]
\begin{minipage}[b]{.69\textwidth}
\begin{infilsf} \tiny
\begin{tikzpicture}[x=.092\textwidth,y=.092\textwidth]
\pgfdeclareplotmark{cross} {
\pgfpathmoveto{\pgfpoint{-0.3\pgfplotmarksize}{\pgfplotmarksize}}
\pgfpathlineto{\pgfpoint{+0.3\pgfplotmarksize}{\pgfplotmarksize}}
\pgfpathlineto{\pgfpoint{+0.3\pgfplotmarksize}{0.3\pgfplotmarksize}}
\pgfpathlineto{\pgfpoint{+1\pgfplotmarksize}{0.3\pgfplotmarksize}}
\pgfpathlineto{\pgfpoint{+1\pgfplotmarksize}{-0.3\pgfplotmarksize}}
\pgfpathlineto{\pgfpoint{+0.3\pgfplotmarksize}{-0.3\pgfplotmarksize}}
\pgfpathlineto{\pgfpoint{+0.3\pgfplotmarksize}{-1.\pgfplotmarksize}}
\pgfpathlineto{\pgfpoint{-0.3\pgfplotmarksize}{-1.\pgfplotmarksize}}
\pgfpathlineto{\pgfpoint{-0.3\pgfplotmarksize}{-0.3\pgfplotmarksize}}
\pgfpathlineto{\pgfpoint{-1.\pgfplotmarksize}{-0.3\pgfplotmarksize}}
\pgfpathlineto{\pgfpoint{-1.\pgfplotmarksize}{0.3\pgfplotmarksize}}
\pgfpathlineto{\pgfpoint{-0.3\pgfplotmarksize}{0.3\pgfplotmarksize}}
\pgfpathclose
\pgfusepathqstroke
}
\pgfdeclareplotmark{cross*} {
\pgfpathmoveto{\pgfpoint{-0.3\pgfplotmarksize}{\pgfplotmarksize}}
\pgfpathlineto{\pgfpoint{+0.3\pgfplotmarksize}{\pgfplotmarksize}}
\pgfpathlineto{\pgfpoint{+0.3\pgfplotmarksize}{0.3\pgfplotmarksize}}
\pgfpathlineto{\pgfpoint{+1\pgfplotmarksize}{0.3\pgfplotmarksize}}
\pgfpathlineto{\pgfpoint{+1\pgfplotmarksize}{-0.3\pgfplotmarksize}}
\pgfpathlineto{\pgfpoint{+0.3\pgfplotmarksize}{-0.3\pgfplotmarksize}}
\pgfpathlineto{\pgfpoint{+0.3\pgfplotmarksize}{-1.\pgfplotmarksize}}
\pgfpathlineto{\pgfpoint{-0.3\pgfplotmarksize}{-1.\pgfplotmarksize}}
\pgfpathlineto{\pgfpoint{-0.3\pgfplotmarksize}{-0.3\pgfplotmarksize}}
\pgfpathlineto{\pgfpoint{-1.\pgfplotmarksize}{-0.3\pgfplotmarksize}}
\pgfpathlineto{\pgfpoint{-1.\pgfplotmarksize}{0.3\pgfplotmarksize}}
\pgfpathlineto{\pgfpoint{-0.3\pgfplotmarksize}{0.3\pgfplotmarksize}}
\pgfpathclose
\pgfusepathqfillstroke
}
\pgfdeclareplotmark{newstar} {
\pgfpathmoveto{\pgfqpoint{0pt}{\pgfplotmarksize}}
\pgfpathlineto{\pgfqpointpolar{44}{0.5\pgfplotmarksize}}
\pgfpathlineto{\pgfqpointpolar{18}{\pgfplotmarksize}}
\pgfpathlineto{\pgfqpointpolar{-20}{0.5\pgfplotmarksize}}
\pgfpathlineto{\pgfqpointpolar{-54}{\pgfplotmarksize}}
\pgfpathlineto{\pgfqpointpolar{-90}{0.5\pgfplotmarksize}}
\pgfpathlineto{\pgfqpointpolar{234}{\pgfplotmarksize}}
\pgfpathlineto{\pgfqpointpolar{198}{0.5\pgfplotmarksize}}
\pgfpathlineto{\pgfqpointpolar{162}{\pgfplotmarksize}}
\pgfpathlineto{\pgfqpointpolar{134}{0.5\pgfplotmarksize}}
\pgfpathclose
\pgfusepathqstroke
}
\pgfdeclareplotmark{newstar*} {
\pgfpathmoveto{\pgfqpoint{0pt}{\pgfplotmarksize}}
\pgfpathlineto{\pgfqpointpolar{44}{0.5\pgfplotmarksize}}
\pgfpathlineto{\pgfqpointpolar{18}{\pgfplotmarksize}}
\pgfpathlineto{\pgfqpointpolar{-20}{0.5\pgfplotmarksize}}
\pgfpathlineto{\pgfqpointpolar{-54}{\pgfplotmarksize}}
\pgfpathlineto{\pgfqpointpolar{-90}{0.5\pgfplotmarksize}}
\pgfpathlineto{\pgfqpointpolar{234}{\pgfplotmarksize}}
\pgfpathlineto{\pgfqpointpolar{198}{0.5\pgfplotmarksize}}
\pgfpathlineto{\pgfqpointpolar{162}{\pgfplotmarksize}}
\pgfpathlineto{\pgfqpointpolar{134}{0.5\pgfplotmarksize}}
\pgfpathclose
\pgfusepathqfillstroke
}
\definecolor{c}{rgb}{1,1,1};
\draw [color=c, fill=c] (0,0) rectangle (10,6.80516);
\draw [color=c, fill=c] (1,0.680516) rectangle (9.95,6.73711);
\definecolor{c}{rgb}{0,0,0};
\draw [c] (1,0.680516) -- (1,6.73711) -- (9.95,6.73711) -- (9.95,0.680516) -- (1,0.680516);
\definecolor{c}{rgb}{1,1,1};
\draw [color=c, fill=c] (1,0.680516) rectangle (9.95,6.73711);
\definecolor{c}{rgb}{0,0,0};
\draw [c] (1,0.680516) -- (1,6.73711) -- (9.95,6.73711) -- (9.95,0.680516) -- (1,0.680516);
\colorlet{c}{natcomp!70};
\draw [c] (2.01048,0.686784) -- (2.01048,0.695341);
\draw [c] (2.01048,0.695341) -- (2.01048,0.703897);
\draw [c] (1.86613,0.695341) -- (2.01048,0.695341);
\draw [c] (2.01048,0.695341) -- (2.15484,0.695341);
\definecolor{c}{rgb}{0,0,0};
\colorlet{c}{natcomp!70};
\draw [c] (2.29919,0.680526) -- (2.29919,0.680532);
\draw [c] (2.29919,0.680532) -- (2.29919,0.680538);
\draw [c] (2.15484,0.680532) -- (2.29919,0.680532);
\draw [c] (2.29919,0.680532) -- (2.44355,0.680532);
\definecolor{c}{rgb}{0,0,0};
\colorlet{c}{natcomp!70};
\draw [c] (2.5879,0.752927) -- (2.5879,0.77446);
\draw [c] (2.5879,0.77446) -- (2.5879,0.795993);
\draw [c] (2.44355,0.77446) -- (2.5879,0.77446);
\draw [c] (2.5879,0.77446) -- (2.73226,0.77446);
\definecolor{c}{rgb}{0,0,0};
\colorlet{c}{natcomp!70};
\draw [c] (2.87661,0.943483) -- (2.87661,0.982067);
\draw [c] (2.87661,0.982067) -- (2.87661,1.02065);
\draw [c] (2.73226,0.982067) -- (2.87661,0.982067);
\draw [c] (2.87661,0.982067) -- (3.02097,0.982067);
\definecolor{c}{rgb}{0,0,0};
\colorlet{c}{natcomp!70};
\draw [c] (3.16532,1.29053) -- (3.16532,1.34793);
\draw [c] (3.16532,1.34793) -- (3.16532,1.40533);
\draw [c] (3.02097,1.34793) -- (3.16532,1.34793);
\draw [c] (3.16532,1.34793) -- (3.30968,1.34793);
\definecolor{c}{rgb}{0,0,0};
\colorlet{c}{natcomp!70};
\draw [c] (3.45403,1.93426) -- (3.45403,2.01543);
\draw [c] (3.45403,2.01543) -- (3.45403,2.09661);
\draw [c] (3.30968,2.01543) -- (3.45403,2.01543);
\draw [c] (3.45403,2.01543) -- (3.59839,2.01543);
\definecolor{c}{rgb}{0,0,0};
\colorlet{c}{natcomp!70};
\draw [c] (3.74274,2.71861) -- (3.74274,2.82141);
\draw [c] (3.74274,2.82141) -- (3.74274,2.9242);
\draw [c] (3.59839,2.82141) -- (3.74274,2.82141);
\draw [c] (3.74274,2.82141) -- (3.8871,2.82141);
\definecolor{c}{rgb}{0,0,0};
\colorlet{c}{natcomp!70};
\draw [c] (4.03145,4.11719) -- (4.03145,4.24993);
\draw [c] (4.03145,4.24993) -- (4.03145,4.38267);
\draw [c] (3.8871,4.24993) -- (4.03145,4.24993);
\draw [c] (4.03145,4.24993) -- (4.17581,4.24993);
\definecolor{c}{rgb}{0,0,0};
\colorlet{c}{natcomp!70};
\draw [c] (4.32016,4.63729) -- (4.32016,4.77952);
\draw [c] (4.32016,4.77952) -- (4.32016,4.92176);
\draw [c] (4.17581,4.77952) -- (4.32016,4.77952);
\draw [c] (4.32016,4.77952) -- (4.46452,4.77952);
\definecolor{c}{rgb}{0,0,0};
\colorlet{c}{natcomp!70};
\draw [c] (4.60887,5.6146) -- (4.60887,5.77315);
\draw [c] (4.60887,5.77315) -- (4.60887,5.93169);
\draw [c] (4.46452,5.77315) -- (4.60887,5.77315);
\draw [c] (4.60887,5.77315) -- (4.75323,5.77315);
\definecolor{c}{rgb}{0,0,0};
\colorlet{c}{natcomp!70};
\draw [c] (4.89758,6.11613) -- (4.89758,6.28241);
\draw [c] (4.89758,6.28241) -- (4.89758,6.4487);
\draw [c] (4.75323,6.28241) -- (4.89758,6.28241);
\draw [c] (4.89758,6.28241) -- (5.04194,6.28241);
\definecolor{c}{rgb}{0,0,0};
\colorlet{c}{natcomp!70};
\draw [c] (5.18629,5.576) -- (5.18629,5.73393);
\draw [c] (5.18629,5.73393) -- (5.18629,5.89186);
\draw [c] (5.04194,5.73393) -- (5.18629,5.73393);
\draw [c] (5.18629,5.73393) -- (5.33065,5.73393);
\definecolor{c}{rgb}{0,0,0};
\colorlet{c}{natcomp!70};
\draw [c] (5.475,5.47333) -- (5.475,5.62963);
\draw [c] (5.475,5.62963) -- (5.475,5.78593);
\draw [c] (5.33065,5.62963) -- (5.475,5.62963);
\draw [c] (5.475,5.62963) -- (5.61935,5.62963);
\definecolor{c}{rgb}{0,0,0};
\colorlet{c}{natcomp!70};
\draw [c] (5.76371,4.43808) -- (5.76371,4.57675);
\draw [c] (5.76371,4.57675) -- (5.76371,4.71543);
\draw [c] (5.61935,4.57675) -- (5.76371,4.57675);
\draw [c] (5.76371,4.57675) -- (5.90806,4.57675);
\definecolor{c}{rgb}{0,0,0};
\colorlet{c}{natcomp!70};
\draw [c] (6.05242,3.40517) -- (6.05242,3.52363);
\draw [c] (6.05242,3.52363) -- (6.05242,3.64209);
\draw [c] (5.90806,3.52363) -- (6.05242,3.52363);
\draw [c] (6.05242,3.52363) -- (6.19677,3.52363);
\definecolor{c}{rgb}{0,0,0};
\colorlet{c}{natcomp!70};
\draw [c] (6.34113,2.64159) -- (6.34113,2.74247);
\draw [c] (6.34113,2.74247) -- (6.34113,2.84335);
\draw [c] (6.19677,2.74247) -- (6.34113,2.74247);
\draw [c] (6.34113,2.74247) -- (6.48548,2.74247);
\definecolor{c}{rgb}{0,0,0};
\colorlet{c}{natcomp!70};
\draw [c] (6.62984,2.05919) -- (6.62984,2.14419);
\draw [c] (6.62984,2.14419) -- (6.62984,2.22918);
\draw [c] (6.48548,2.14419) -- (6.62984,2.14419);
\draw [c] (6.62984,2.14419) -- (6.77419,2.14419);
\definecolor{c}{rgb}{0,0,0};
\colorlet{c}{natcomp!70};
\draw [c] (6.91855,1.74752) -- (6.91855,1.8226);
\draw [c] (6.91855,1.8226) -- (6.91855,1.89768);
\draw [c] (6.77419,1.8226) -- (6.91855,1.8226);
\draw [c] (6.91855,1.8226) -- (7.0629,1.8226);
\definecolor{c}{rgb}{0,0,0};
\colorlet{c}{natcomp!70};
\draw [c] (7.20726,1.16326) -- (7.20726,1.2146);
\draw [c] (7.20726,1.2146) -- (7.20726,1.26594);
\draw [c] (7.0629,1.2146) -- (7.20726,1.2146);
\draw [c] (7.20726,1.2146) -- (7.35161,1.2146);
\definecolor{c}{rgb}{0,0,0};
\colorlet{c}{natcomp!70};
\draw [c] (7.49597,0.925145) -- (7.49597,0.962442);
\draw [c] (7.49597,0.962442) -- (7.49597,0.999739);
\draw [c] (7.35161,0.962442) -- (7.49597,0.962442);
\draw [c] (7.49597,0.962442) -- (7.64032,0.962442);
\definecolor{c}{rgb}{0,0,0};
\colorlet{c}{natcomp!70};
\draw [c] (7.78468,0.847054) -- (7.78468,0.878298);
\draw [c] (7.78468,0.878298) -- (7.78468,0.909542);
\draw [c] (7.64032,0.878298) -- (7.78468,0.878298);
\draw [c] (7.78468,0.878298) -- (7.92903,0.878298);
\definecolor{c}{rgb}{0,0,0};
\colorlet{c}{natcomp!70};
\draw [c] (8.07339,0.7399) -- (8.07339,0.759661);
\draw [c] (8.07339,0.759661) -- (8.07339,0.779421);
\draw [c] (7.92903,0.759661) -- (8.07339,0.759661);
\draw [c] (8.07339,0.759661) -- (8.21774,0.759661);
\definecolor{c}{rgb}{0,0,0};
\colorlet{c}{natcomp!70};
\draw [c] (8.3621,0.710206) -- (8.3621,0.725026);
\draw [c] (8.3621,0.725026) -- (8.3621,0.739846);
\draw [c] (8.21774,0.725026) -- (8.3621,0.725026);
\draw [c] (8.3621,0.725026) -- (8.50645,0.725026);
\definecolor{c}{rgb}{0,0,0};
\colorlet{c}{natcomp!70};
\draw [c] (8.65081,0.686798) -- (8.65081,0.695355);
\draw [c] (8.65081,0.695355) -- (8.65081,0.703911);
\draw [c] (8.50645,0.695355) -- (8.65081,0.695355);
\draw [c] (8.65081,0.695355) -- (8.79516,0.695355);
\definecolor{c}{rgb}{0,0,0};
\colorlet{c}{natcomp!70};
\draw [c] (8.93952,0.683424) -- (8.93952,0.69041);
\draw [c] (8.93952,0.69041) -- (8.93952,0.697396);
\draw [c] (8.79516,0.69041) -- (8.93952,0.69041);
\draw [c] (8.93952,0.69041) -- (9.08387,0.69041);
\definecolor{c}{rgb}{0,0,0};
\colorlet{c}{natcomp!70};
\draw [c] (9.22823,0.690398) -- (9.22823,0.700279);
\draw [c] (9.22823,0.700279) -- (9.22823,0.710159);
\draw [c] (9.08387,0.700279) -- (9.22823,0.700279);
\draw [c] (9.22823,0.700279) -- (9.37258,0.700279);
\definecolor{c}{rgb}{0,0,0};
\colorlet{c}{natcomp!70};
\draw [c] (9.80564,0.680517) -- (9.80564,0.68052);
\draw [c] (9.80564,0.68052) -- (9.80564,0.680524);
\draw [c] (9.66129,0.68052) -- (9.80564,0.68052);
\draw [c] (9.80564,0.68052) -- (9.95,0.68052);
\definecolor{c}{rgb}{0,0,0};
\draw [c] (1,0.680516) -- (9.95,0.680516);
\draw [anchor= east] (9.95,0.108883) node[color=c, rotate=0]{Number of primary vertices};
\draw [c] (1.14435,0.863234) -- (1.14435,0.680516);
\draw [c] (1.43306,0.771875) -- (1.43306,0.680516);
\draw [c] (1.72177,0.771875) -- (1.72177,0.680516);
\draw [c] (2.01048,0.771875) -- (2.01048,0.680516);
\draw [c] (2.29919,0.771875) -- (2.29919,0.680516);
\draw [c] (2.5879,0.863234) -- (2.5879,0.680516);
\draw [c] (2.87661,0.771875) -- (2.87661,0.680516);
\draw [c] (3.16532,0.771875) -- (3.16532,0.680516);
\draw [c] (3.45403,0.771875) -- (3.45403,0.680516);
\draw [c] (3.74274,0.771875) -- (3.74274,0.680516);
\draw [c] (4.03145,0.863234) -- (4.03145,0.680516);
\draw [c] (4.32016,0.771875) -- (4.32016,0.680516);
\draw [c] (4.60887,0.771875) -- (4.60887,0.680516);
\draw [c] (4.89758,0.771875) -- (4.89758,0.680516);
\draw [c] (5.18629,0.771875) -- (5.18629,0.680516);
\draw [c] (5.475,0.863234) -- (5.475,0.680516);
\draw [c] (5.76371,0.771875) -- (5.76371,0.680516);
\draw [c] (6.05242,0.771875) -- (6.05242,0.680516);
\draw [c] (6.34113,0.771875) -- (6.34113,0.680516);
\draw [c] (6.62984,0.771875) -- (6.62984,0.680516);
\draw [c] (6.91855,0.863234) -- (6.91855,0.680516);
\draw [c] (7.20726,0.771875) -- (7.20726,0.680516);
\draw [c] (7.49597,0.771875) -- (7.49597,0.680516);
\draw [c] (7.78468,0.771875) -- (7.78468,0.680516);
\draw [c] (8.07339,0.771875) -- (8.07339,0.680516);
\draw [c] (8.3621,0.863234) -- (8.3621,0.680516);
\draw [c] (8.65081,0.771875) -- (8.65081,0.680516);
\draw [c] (8.93952,0.771875) -- (8.93952,0.680516);
\draw [c] (9.22823,0.771875) -- (9.22823,0.680516);
\draw [c] (9.51694,0.771875) -- (9.51694,0.680516);
\draw [c] (9.80564,0.863234) -- (9.80564,0.680516);
\draw [c] (1.14435,0.863234) -- (1.14435,0.680516);
\draw [c] (9.80564,0.863234) -- (9.80564,0.680516);
\draw [anchor=base] (1.14435,0.353868) node[color=c, rotate=0]{0};
\draw [anchor=base] (2.5879,0.353868) node[color=c, rotate=0]{5};
\draw [anchor=base] (4.03145,0.353868) node[color=c, rotate=0]{10};
\draw [anchor=base] (5.475,0.353868) node[color=c, rotate=0]{15};
\draw [anchor=base] (6.91855,0.353868) node[color=c, rotate=0]{20};
\draw [anchor=base] (8.3621,0.353868) node[color=c, rotate=0]{25};
\draw [anchor=base] (9.80564,0.353868) node[color=c, rotate=0]{30};
\draw [c] (1,0.680516) -- (1,6.73711);
\draw [anchor= east] (-0.12,6.73711) node[color=c, rotate=90]{Normalised number of events};
\draw [c] (1.267,0.680516) -- (1,0.680516);
\draw [c] (1.1335,0.878524) -- (1,0.878524);
\draw [c] (1.1335,1.07653) -- (1,1.07653);
\draw [c] (1.1335,1.27454) -- (1,1.27454);
\draw [c] (1.1335,1.47255) -- (1,1.47255);
\draw [c] (1.267,1.67056) -- (1,1.67056);
\draw [c] (1.1335,1.86857) -- (1,1.86857);
\draw [c] (1.1335,2.06657) -- (1,2.06657);
\draw [c] (1.1335,2.26458) -- (1,2.26458);
\draw [c] (1.1335,2.46259) -- (1,2.46259);
\draw [c] (1.267,2.6606) -- (1,2.6606);
\draw [c] (1.1335,2.85861) -- (1,2.85861);
\draw [c] (1.1335,3.05662) -- (1,3.05662);
\draw [c] (1.1335,3.25462) -- (1,3.25462);
\draw [c] (1.1335,3.45263) -- (1,3.45263);
\draw [c] (1.267,3.65064) -- (1,3.65064);
\draw [c] (1.1335,3.84865) -- (1,3.84865);
\draw [c] (1.1335,4.04666) -- (1,4.04666);
\draw [c] (1.1335,4.24467) -- (1,4.24467);
\draw [c] (1.1335,4.44267) -- (1,4.44267);
\draw [c] (1.267,4.64068) -- (1,4.64068);
\draw [c] (1.1335,4.83869) -- (1,4.83869);
\draw [c] (1.1335,5.0367) -- (1,5.0367);
\draw [c] (1.1335,5.23471) -- (1,5.23471);
\draw [c] (1.1335,5.43272) -- (1,5.43272);
\draw [c] (1.267,5.63072) -- (1,5.63072);
\draw [c] (1.1335,5.82873) -- (1,5.82873);
\draw [c] (1.1335,6.02674) -- (1,6.02674);
\draw [c] (1.1335,6.22475) -- (1,6.22475);
\draw [c] (1.1335,6.42276) -- (1,6.42276);
\draw [c] (1.267,6.62077) -- (1,6.62077);
\draw [c] (1.267,6.62077) -- (1,6.62077);
\draw [anchor= east] (0.95,0.680516) node[color=c, rotate=0]{0};
\draw [anchor= east] (0.95,1.67056) node[color=c, rotate=0]{500};
\draw [anchor= east] (0.95,2.6606) node[color=c, rotate=0]{1000};
\draw [anchor= east] (0.95,3.65064) node[color=c, rotate=0]{1500};
\draw [anchor= east] (0.95,4.64068) node[color=c, rotate=0]{2000};
\draw [anchor= east] (0.95,5.63072) node[color=c, rotate=0]{2500};
\draw [anchor= east] (0.95,6.62077) node[color=c, rotate=0]{3000};
\colorlet{c}{natgreen};
\draw [c] (1.72177,0.680523) -- (1.72177,0.681242);
\draw [c] (1.72177,0.681242) -- (1.72177,0.68196);
\draw [c] (1.57742,0.681242) -- (1.72177,0.681242);
\draw [c] (1.72177,0.681242) -- (1.86613,0.681242);
\definecolor{c}{rgb}{0,0,0};
\colorlet{c}{natgreen};
\draw [c] (2.01048,0.689073) -- (2.01048,0.696007);
\draw [c] (2.01048,0.696007) -- (2.01048,0.702941);
\draw [c] (1.86613,0.696007) -- (2.01048,0.696007);
\draw [c] (2.01048,0.696007) -- (2.15484,0.696007);
\definecolor{c}{rgb}{0,0,0};
\colorlet{c}{natgreen};
\draw [c] (2.29919,0.716299) -- (2.29919,0.731686);
\draw [c] (2.29919,0.731686) -- (2.29919,0.747074);
\draw [c] (2.15484,0.731686) -- (2.29919,0.731686);
\draw [c] (2.29919,0.731686) -- (2.44355,0.731686);
\definecolor{c}{rgb}{0,0,0};
\colorlet{c}{natgreen};
\draw [c] (2.5879,0.871864) -- (2.5879,0.902068);
\draw [c] (2.5879,0.902068) -- (2.5879,0.932273);
\draw [c] (2.44355,0.902068) -- (2.5879,0.902068);
\draw [c] (2.5879,0.902068) -- (2.73226,0.902068);
\definecolor{c}{rgb}{0,0,0};
\colorlet{c}{natgreen};
\draw [c] (2.87661,1.14594) -- (2.87661,1.19239);
\draw [c] (2.87661,1.19239) -- (2.87661,1.23883);
\draw [c] (2.73226,1.19239) -- (2.87661,1.19239);
\draw [c] (2.87661,1.19239) -- (3.02097,1.19239);
\definecolor{c}{rgb}{0,0,0};
\colorlet{c}{natgreen};
\draw [c] (3.16532,1.64654) -- (3.16532,1.715);
\draw [c] (3.16532,1.715) -- (3.16532,1.78347);
\draw [c] (3.02097,1.715) -- (3.16532,1.715);
\draw [c] (3.16532,1.715) -- (3.30968,1.715);
\definecolor{c}{rgb}{0,0,0};
\colorlet{c}{natgreen};
\draw [c] (3.45403,2.17071) -- (3.45403,2.25603);
\draw [c] (3.45403,2.25603) -- (3.45403,2.34135);
\draw [c] (3.30968,2.25603) -- (3.45403,2.25603);
\draw [c] (3.45403,2.25603) -- (3.59839,2.25603);
\definecolor{c}{rgb}{0,0,0};
\colorlet{c}{natgreen};
\draw [c] (3.74274,3.01706) -- (3.74274,3.12356);
\draw [c] (3.74274,3.12356) -- (3.74274,3.23006);
\draw [c] (3.59839,3.12356) -- (3.74274,3.12356);
\draw [c] (3.74274,3.12356) -- (3.8871,3.12356);
\definecolor{c}{rgb}{0,0,0};
\colorlet{c}{natgreen};
\draw [c] (4.03145,3.62689) -- (4.03145,3.74661);
\draw [c] (4.03145,3.74661) -- (4.03145,3.86633);
\draw [c] (3.8871,3.74661) -- (4.03145,3.74661);
\draw [c] (4.03145,3.74661) -- (4.17581,3.74661);
\definecolor{c}{rgb}{0,0,0};
\colorlet{c}{natgreen};
\draw [c] (4.32016,4.27462) -- (4.32016,4.40545);
\draw [c] (4.32016,4.40545) -- (4.32016,4.53628);
\draw [c] (4.17581,4.40545) -- (4.32016,4.40545);
\draw [c] (4.32016,4.40545) -- (4.46452,4.40545);
\definecolor{c}{rgb}{0,0,0};
\colorlet{c}{natgreen};
\draw [c] (4.60887,4.55969) -- (4.60887,4.69362);
\draw [c] (4.60887,4.69362) -- (4.60887,4.82755);
\draw [c] (4.46452,4.69362) -- (4.60887,4.69362);
\draw [c] (4.60887,4.69362) -- (4.75323,4.69362);
\definecolor{c}{rgb}{0,0,0};
\colorlet{c}{natgreen};
\draw [c] (4.89758,4.53815) -- (4.89758,4.66935);
\draw [c] (4.89758,4.66935) -- (4.89758,4.80056);
\draw [c] (4.75323,4.66935) -- (4.89758,4.66935);
\draw [c] (4.89758,4.66935) -- (5.04194,4.66935);
\definecolor{c}{rgb}{0,0,0};
\colorlet{c}{natgreen};
\draw [c] (5.18629,4.99176) -- (5.18629,5.12865);
\draw [c] (5.18629,5.12865) -- (5.18629,5.26553);
\draw [c] (5.04194,5.12865) -- (5.18629,5.12865);
\draw [c] (5.18629,5.12865) -- (5.33065,5.12865);
\definecolor{c}{rgb}{0,0,0};
\colorlet{c}{natgreen};
\draw [c] (5.475,4.29737) -- (5.475,4.41993);
\draw [c] (5.475,4.41993) -- (5.475,4.54248);
\draw [c] (5.33065,4.41993) -- (5.475,4.41993);
\draw [c] (5.475,4.41993) -- (5.61935,4.41993);
\definecolor{c}{rgb}{0,0,0};
\colorlet{c}{natgreen};
\draw [c] (5.76371,4.12649) -- (5.76371,4.24387);
\draw [c] (5.76371,4.24387) -- (5.76371,4.36126);
\draw [c] (5.61935,4.24387) -- (5.76371,4.24387);
\draw [c] (5.76371,4.24387) -- (5.90806,4.24387);
\definecolor{c}{rgb}{0,0,0};
\colorlet{c}{natgreen};
\draw [c] (6.05242,3.79458) -- (6.05242,3.90367);
\draw [c] (6.05242,3.90367) -- (6.05242,4.01276);
\draw [c] (5.90806,3.90367) -- (6.05242,3.90367);
\draw [c] (6.05242,3.90367) -- (6.19677,3.90367);
\definecolor{c}{rgb}{0,0,0};
\colorlet{c}{natgreen};
\draw [c] (6.34113,3.46262) -- (6.34113,3.56305);
\draw [c] (6.34113,3.56305) -- (6.34113,3.66348);
\draw [c] (6.19677,3.56305) -- (6.34113,3.56305);
\draw [c] (6.34113,3.56305) -- (6.48548,3.56305);
\definecolor{c}{rgb}{0,0,0};
\colorlet{c}{natgreen};
\draw [c] (6.62984,2.62966) -- (6.62984,2.71154);
\draw [c] (6.62984,2.71154) -- (6.62984,2.79342);
\draw [c] (6.48548,2.71154) -- (6.62984,2.71154);
\draw [c] (6.62984,2.71154) -- (6.77419,2.71154);
\definecolor{c}{rgb}{0,0,0};
\colorlet{c}{natgreen};
\draw [c] (6.91855,2.28926) -- (6.91855,2.36189);
\draw [c] (6.91855,2.36189) -- (6.91855,2.43452);
\draw [c] (6.77419,2.36189) -- (6.91855,2.36189);
\draw [c] (6.91855,2.36189) -- (7.0629,2.36189);
\definecolor{c}{rgb}{0,0,0};
\colorlet{c}{natgreen};
\draw [c] (7.20726,1.66134) -- (7.20726,1.71656);
\draw [c] (7.20726,1.71656) -- (7.20726,1.77178);
\draw [c] (7.0629,1.71656) -- (7.20726,1.71656);
\draw [c] (7.20726,1.71656) -- (7.35161,1.71656);
\definecolor{c}{rgb}{0,0,0};
\colorlet{c}{natgreen};
\draw [c] (7.49597,1.50108) -- (7.49597,1.55148);
\draw [c] (7.49597,1.55148) -- (7.49597,1.60188);
\draw [c] (7.35161,1.55148) -- (7.49597,1.55148);
\draw [c] (7.49597,1.55148) -- (7.64032,1.55148);
\definecolor{c}{rgb}{0,0,0};
\colorlet{c}{natgreen};
\draw [c] (7.78468,1.21206) -- (7.78468,1.25203);
\draw [c] (7.78468,1.25203) -- (7.78468,1.292);
\draw [c] (7.64032,1.25203) -- (7.78468,1.25203);
\draw [c] (7.78468,1.25203) -- (7.92903,1.25203);
\definecolor{c}{rgb}{0,0,0};
\colorlet{c}{natgreen};
\draw [c] (8.07339,1.01304) -- (8.07339,1.04329);
\draw [c] (8.07339,1.04329) -- (8.07339,1.07354);
\draw [c] (7.92903,1.04329) -- (8.07339,1.04329);
\draw [c] (8.07339,1.04329) -- (8.21774,1.04329);
\definecolor{c}{rgb}{0,0,0};
\colorlet{c}{natgreen};
\draw [c] (8.3621,0.847918) -- (8.3621,0.869254);
\draw [c] (8.3621,0.869254) -- (8.3621,0.890591);
\draw [c] (8.21774,0.869254) -- (8.3621,0.869254);
\draw [c] (8.3621,0.869254) -- (8.50645,0.869254);
\definecolor{c}{rgb}{0,0,0};
\colorlet{c}{natgreen};
\draw [c] (8.65081,0.787574) -- (8.65081,0.80412);
\draw [c] (8.65081,0.80412) -- (8.65081,0.820665);
\draw [c] (8.50645,0.80412) -- (8.65081,0.80412);
\draw [c] (8.65081,0.80412) -- (8.79516,0.80412);
\definecolor{c}{rgb}{0,0,0};
\colorlet{c}{natgreen};
\draw [c] (8.93952,0.739471) -- (8.93952,0.751966);
\draw [c] (8.93952,0.751966) -- (8.93952,0.76446);
\draw [c] (8.79516,0.751966) -- (8.93952,0.751966);
\draw [c] (8.93952,0.751966) -- (9.08387,0.751966);
\definecolor{c}{rgb}{0,0,0};
\colorlet{c}{natgreen};
\draw [c] (9.22823,0.713252) -- (9.22823,0.722645);
\draw [c] (9.22823,0.722645) -- (9.22823,0.732039);
\draw [c] (9.08387,0.722645) -- (9.22823,0.722645);
\draw [c] (9.22823,0.722645) -- (9.37258,0.722645);
\definecolor{c}{rgb}{0,0,0};
\colorlet{c}{natgreen};
\draw [c] (9.51694,0.695598) -- (9.51694,0.701995);
\draw [c] (9.51694,0.701995) -- (9.51694,0.708392);
\draw [c] (9.37258,0.701995) -- (9.51694,0.701995);
\draw [c] (9.51694,0.701995) -- (9.66129,0.701995);
\definecolor{c}{rgb}{0,0,0};
\colorlet{c}{natgreen};
\draw [c] (9.80564,0.682123) -- (9.80564,0.68405);
\draw [c] (9.80564,0.68405) -- (9.80564,0.685978);
\draw [c] (9.66129,0.68405) -- (9.80564,0.68405);
\draw [c] (9.80564,0.68405) -- (9.95,0.68405);
\definecolor{c}{rgb}{0,0,0};
\draw [anchor=base west] (6.96633,6.17962) node[color=c, rotate=0]{ATLAS MC};
\colorlet{c}{natgreen};
\draw [c] (6.13521,6.27149) -- (6.81966,6.27149);
\draw [c] (6.47744,6.149) -- (6.47744,6.39398);
\definecolor{c}{rgb}{0,0,0};
\draw [anchor=base west] (6.96633,5.77131) node[color=c, rotate=0]{Our MC};
\colorlet{c}{natcomp!70};
\draw [c] (6.13521,5.86318) -- (6.81966,5.86318);
\draw [c] (6.47744,5.74069) -- (6.47744,5.98567);
\end{tikzpicture}

\end{infilsf}
\end{minipage}
\begin{minipage}[b]{.3\textwidth}
\caption{The distribution of the number of reconstructed primary vertices in the \atlas{} $\gamma\gamma$ MC set and in the CalcHEP MC set produced for this thesis, normalised to the same number of events.}\label{pvnnone}
\end{minipage}
\end{figure}

[Uncertainty...] [Also, effect on Mgg...?]

The second issue we encounter is in the distribution of $E_T^\text{iso}$, which is much broader in the CalcHEP MC set than in the \atlas{} one, as illustrated in figure~\ref{etpv}. Assuming that the distribution in the CalcHEP sample is just the one in the \atlas{} sample, but broadened and shifted slightly, we develop a mapping function to reverse that effect [\~$E_T^\text{iso}/4+0.2$ GeV]. The result of applying this function is shown in fig.~\ref{etmap}.

\begin{figure}[htp]
\begin{minipage}[b]{.49\textwidth}
\begin{infilsf} \tiny 
\begin{tikzpicture}[x=.092\textwidth,y=.092\textwidth]
\pgfdeclareplotmark{cross} {
\pgfpathmoveto{\pgfpoint{-0.3\pgfplotmarksize}{\pgfplotmarksize}}
\pgfpathlineto{\pgfpoint{+0.3\pgfplotmarksize}{\pgfplotmarksize}}
\pgfpathlineto{\pgfpoint{+0.3\pgfplotmarksize}{0.3\pgfplotmarksize}}
\pgfpathlineto{\pgfpoint{+1\pgfplotmarksize}{0.3\pgfplotmarksize}}
\pgfpathlineto{\pgfpoint{+1\pgfplotmarksize}{-0.3\pgfplotmarksize}}
\pgfpathlineto{\pgfpoint{+0.3\pgfplotmarksize}{-0.3\pgfplotmarksize}}
\pgfpathlineto{\pgfpoint{+0.3\pgfplotmarksize}{-1.\pgfplotmarksize}}
\pgfpathlineto{\pgfpoint{-0.3\pgfplotmarksize}{-1.\pgfplotmarksize}}
\pgfpathlineto{\pgfpoint{-0.3\pgfplotmarksize}{-0.3\pgfplotmarksize}}
\pgfpathlineto{\pgfpoint{-1.\pgfplotmarksize}{-0.3\pgfplotmarksize}}
\pgfpathlineto{\pgfpoint{-1.\pgfplotmarksize}{0.3\pgfplotmarksize}}
\pgfpathlineto{\pgfpoint{-0.3\pgfplotmarksize}{0.3\pgfplotmarksize}}
\pgfpathclose
\pgfusepathqstroke
}
\pgfdeclareplotmark{cross*} {
\pgfpathmoveto{\pgfpoint{-0.3\pgfplotmarksize}{\pgfplotmarksize}}
\pgfpathlineto{\pgfpoint{+0.3\pgfplotmarksize}{\pgfplotmarksize}}
\pgfpathlineto{\pgfpoint{+0.3\pgfplotmarksize}{0.3\pgfplotmarksize}}
\pgfpathlineto{\pgfpoint{+1\pgfplotmarksize}{0.3\pgfplotmarksize}}
\pgfpathlineto{\pgfpoint{+1\pgfplotmarksize}{-0.3\pgfplotmarksize}}
\pgfpathlineto{\pgfpoint{+0.3\pgfplotmarksize}{-0.3\pgfplotmarksize}}
\pgfpathlineto{\pgfpoint{+0.3\pgfplotmarksize}{-1.\pgfplotmarksize}}
\pgfpathlineto{\pgfpoint{-0.3\pgfplotmarksize}{-1.\pgfplotmarksize}}
\pgfpathlineto{\pgfpoint{-0.3\pgfplotmarksize}{-0.3\pgfplotmarksize}}
\pgfpathlineto{\pgfpoint{-1.\pgfplotmarksize}{-0.3\pgfplotmarksize}}
\pgfpathlineto{\pgfpoint{-1.\pgfplotmarksize}{0.3\pgfplotmarksize}}
\pgfpathlineto{\pgfpoint{-0.3\pgfplotmarksize}{0.3\pgfplotmarksize}}
\pgfpathclose
\pgfusepathqfillstroke
}
\pgfdeclareplotmark{newstar} {
\pgfpathmoveto{\pgfqpoint{0pt}{\pgfplotmarksize}}
\pgfpathlineto{\pgfqpointpolar{44}{0.5\pgfplotmarksize}}
\pgfpathlineto{\pgfqpointpolar{18}{\pgfplotmarksize}}
\pgfpathlineto{\pgfqpointpolar{-20}{0.5\pgfplotmarksize}}
\pgfpathlineto{\pgfqpointpolar{-54}{\pgfplotmarksize}}
\pgfpathlineto{\pgfqpointpolar{-90}{0.5\pgfplotmarksize}}
\pgfpathlineto{\pgfqpointpolar{234}{\pgfplotmarksize}}
\pgfpathlineto{\pgfqpointpolar{198}{0.5\pgfplotmarksize}}
\pgfpathlineto{\pgfqpointpolar{162}{\pgfplotmarksize}}
\pgfpathlineto{\pgfqpointpolar{134}{0.5\pgfplotmarksize}}
\pgfpathclose
\pgfusepathqstroke
}
\pgfdeclareplotmark{newstar*} {
\pgfpathmoveto{\pgfqpoint{0pt}{\pgfplotmarksize}}
\pgfpathlineto{\pgfqpointpolar{44}{0.5\pgfplotmarksize}}
\pgfpathlineto{\pgfqpointpolar{18}{\pgfplotmarksize}}
\pgfpathlineto{\pgfqpointpolar{-20}{0.5\pgfplotmarksize}}
\pgfpathlineto{\pgfqpointpolar{-54}{\pgfplotmarksize}}
\pgfpathlineto{\pgfqpointpolar{-90}{0.5\pgfplotmarksize}}
\pgfpathlineto{\pgfqpointpolar{234}{\pgfplotmarksize}}
\pgfpathlineto{\pgfqpointpolar{198}{0.5\pgfplotmarksize}}
\pgfpathlineto{\pgfqpointpolar{162}{\pgfplotmarksize}}
\pgfpathlineto{\pgfqpointpolar{134}{0.5\pgfplotmarksize}}
\pgfpathclose
\pgfusepathqfillstroke
}
\definecolor{c}{rgb}{1,1,1};
\draw [color=c, fill=c] (0,0) rectangle (10,6.80516);
\draw [color=c, fill=c] (1,0.680516) rectangle (9.95,6.73711);
\definecolor{c}{rgb}{0,0,0};
\draw [c] (1,0.680516) -- (1,6.73711) -- (9.95,6.73711) -- (9.95,0.680516) -- (1,0.680516);
\definecolor{c}{rgb}{1,1,1};
\draw [color=c, fill=c] (1,0.680516) rectangle (9.95,6.73711);
\definecolor{c}{rgb}{0,0,0};
\draw [c] (1,0.680516) -- (1,6.73711) -- (9.95,6.73711) -- (9.95,0.680516) -- (1,0.680516);
\colorlet{c}{natgreen};
\draw [c] (1.59395,0.686894) -- (1.59395,0.686894);
\draw [c] (1.59395,0.686894) -- (1.59395,0.686894);
\draw [c] (1.58582,0.686894) -- (1.59395,0.686894);
\draw [c] (1.59395,0.686894) -- (1.60209,0.686894);
\definecolor{c}{rgb}{0,0,0};
\colorlet{c}{natgreen};
\draw [c] (1.96823,0.686894) -- (1.96823,0.686894);
\draw [c] (1.96823,0.686894) -- (1.96823,0.686894);
\draw [c] (1.96009,0.686894) -- (1.96823,0.686894);
\draw [c] (1.96823,0.686894) -- (1.97636,0.686894);
\definecolor{c}{rgb}{0,0,0};
\colorlet{c}{natgreen};
\draw [c] (2.24486,0.686894) -- (2.24486,0.686894);
\draw [c] (2.24486,0.686894) -- (2.24486,0.686894);
\draw [c] (2.23673,0.686894) -- (2.24486,0.686894);
\draw [c] (2.24486,0.686894) -- (2.253,0.686894);
\definecolor{c}{rgb}{0,0,0};
\colorlet{c}{natgreen};
\draw [c] (2.94459,0.692107) -- (2.94459,0.702737);
\draw [c] (2.94459,0.702737) -- (2.94459,0.713368);
\draw [c] (2.93645,0.702737) -- (2.94459,0.702737);
\draw [c] (2.94459,0.702737) -- (2.95273,0.702737);
\definecolor{c}{rgb}{0,0,0};
\colorlet{c}{natgreen};
\draw [c] (2.96086,0.691743) -- (2.96086,0.701841);
\draw [c] (2.96086,0.701841) -- (2.96086,0.711939);
\draw [c] (2.95273,0.701841) -- (2.96086,0.701841);
\draw [c] (2.96086,0.701841) -- (2.969,0.701841);
\definecolor{c}{rgb}{0,0,0};
\colorlet{c}{natgreen};
\draw [c] (3.02595,0.68941) -- (3.02595,0.69722);
\draw [c] (3.02595,0.69722) -- (3.02595,0.70503);
\draw [c] (3.01782,0.69722) -- (3.02595,0.69722);
\draw [c] (3.02595,0.69722) -- (3.03409,0.69722);
\definecolor{c}{rgb}{0,0,0};
\colorlet{c}{natgreen};
\draw [c] (3.0585,0.70998) -- (3.0585,0.729499);
\draw [c] (3.0585,0.729499) -- (3.0585,0.749018);
\draw [c] (3.05036,0.729499) -- (3.0585,0.729499);
\draw [c] (3.0585,0.729499) -- (3.06664,0.729499);
\definecolor{c}{rgb}{0,0,0};
\colorlet{c}{natgreen};
\draw [c] (3.07477,0.73884) -- (3.07477,0.766271);
\draw [c] (3.07477,0.766271) -- (3.07477,0.793702);
\draw [c] (3.06664,0.766271) -- (3.07477,0.766271);
\draw [c] (3.07477,0.766271) -- (3.08291,0.766271);
\definecolor{c}{rgb}{0,0,0};
\colorlet{c}{natgreen};
\draw [c] (3.09105,0.70686) -- (3.09105,0.727728);
\draw [c] (3.09105,0.727728) -- (3.09105,0.748596);
\draw [c] (3.08291,0.727728) -- (3.09105,0.727728);
\draw [c] (3.09105,0.727728) -- (3.09918,0.727728);
\definecolor{c}{rgb}{0,0,0};
\colorlet{c}{natgreen};
\draw [c] (3.10732,0.74735) -- (3.10732,0.773852);
\draw [c] (3.10732,0.773852) -- (3.10732,0.800355);
\draw [c] (3.09918,0.773852) -- (3.10732,0.773852);
\draw [c] (3.10732,0.773852) -- (3.11545,0.773852);
\definecolor{c}{rgb}{0,0,0};
\colorlet{c}{natgreen};
\draw [c] (3.12359,0.721184) -- (3.12359,0.744621);
\draw [c] (3.12359,0.744621) -- (3.12359,0.768058);
\draw [c] (3.11545,0.744621) -- (3.12359,0.744621);
\draw [c] (3.12359,0.744621) -- (3.13173,0.744621);
\definecolor{c}{rgb}{0,0,0};
\colorlet{c}{natgreen};
\draw [c] (3.13986,0.758599) -- (3.13986,0.790979);
\draw [c] (3.13986,0.790979) -- (3.13986,0.82336);
\draw [c] (3.13173,0.790979) -- (3.13986,0.790979);
\draw [c] (3.13986,0.790979) -- (3.148,0.790979);
\definecolor{c}{rgb}{0,0,0};
\colorlet{c}{natgreen};
\draw [c] (3.15614,0.770727) -- (3.15614,0.801482);
\draw [c] (3.15614,0.801482) -- (3.15614,0.832236);
\draw [c] (3.148,0.801482) -- (3.15614,0.801482);
\draw [c] (3.15614,0.801482) -- (3.16427,0.801482);
\definecolor{c}{rgb}{0,0,0};
\colorlet{c}{natgreen};
\draw [c] (3.17241,0.793587) -- (3.17241,0.827995);
\draw [c] (3.17241,0.827995) -- (3.17241,0.862404);
\draw [c] (3.16427,0.827995) -- (3.17241,0.827995);
\draw [c] (3.17241,0.827995) -- (3.18055,0.827995);
\definecolor{c}{rgb}{0,0,0};
\colorlet{c}{natgreen};
\draw [c] (3.18868,0.815002) -- (3.18868,0.854342);
\draw [c] (3.18868,0.854342) -- (3.18868,0.893683);
\draw [c] (3.18055,0.854342) -- (3.18868,0.854342);
\draw [c] (3.18868,0.854342) -- (3.19682,0.854342);
\definecolor{c}{rgb}{0,0,0};
\colorlet{c}{natgreen};
\draw [c] (3.20495,0.920998) -- (3.20495,0.973836);
\draw [c] (3.20495,0.973836) -- (3.20495,1.02667);
\draw [c] (3.19682,0.973836) -- (3.20495,0.973836);
\draw [c] (3.20495,0.973836) -- (3.21309,0.973836);
\definecolor{c}{rgb}{0,0,0};
\colorlet{c}{natgreen};
\draw [c] (3.22123,0.915327) -- (3.22123,0.966615);
\draw [c] (3.22123,0.966615) -- (3.22123,1.0179);
\draw [c] (3.21309,0.966615) -- (3.22123,0.966615);
\draw [c] (3.22123,0.966615) -- (3.22936,0.966615);
\definecolor{c}{rgb}{0,0,0};
\colorlet{c}{natgreen};
\draw [c] (3.2375,1.09644) -- (3.2375,1.16438);
\draw [c] (3.2375,1.16438) -- (3.2375,1.23231);
\draw [c] (3.22936,1.16438) -- (3.2375,1.16438);
\draw [c] (3.2375,1.16438) -- (3.24564,1.16438);
\definecolor{c}{rgb}{0,0,0};
\colorlet{c}{natgreen};
\draw [c] (3.25377,1.19172) -- (3.25377,1.26618);
\draw [c] (3.25377,1.26618) -- (3.25377,1.34065);
\draw [c] (3.24564,1.26618) -- (3.25377,1.26618);
\draw [c] (3.25377,1.26618) -- (3.26191,1.26618);
\definecolor{c}{rgb}{0,0,0};
\colorlet{c}{natgreen};
\draw [c] (3.27005,1.3208) -- (3.27005,1.40739);
\draw [c] (3.27005,1.40739) -- (3.27005,1.49398);
\draw [c] (3.26191,1.40739) -- (3.27005,1.40739);
\draw [c] (3.27005,1.40739) -- (3.27818,1.40739);
\definecolor{c}{rgb}{0,0,0};
\colorlet{c}{natgreen};
\draw [c] (3.28632,1.49265) -- (3.28632,1.58839);
\draw [c] (3.28632,1.58839) -- (3.28632,1.68413);
\draw [c] (3.27818,1.58839) -- (3.28632,1.58839);
\draw [c] (3.28632,1.58839) -- (3.29445,1.58839);
\definecolor{c}{rgb}{0,0,0};
\colorlet{c}{natgreen};
\draw [c] (3.30259,1.75408) -- (3.30259,1.865);
\draw [c] (3.30259,1.865) -- (3.30259,1.97593);
\draw [c] (3.29445,1.865) -- (3.30259,1.865);
\draw [c] (3.30259,1.865) -- (3.31073,1.865);
\definecolor{c}{rgb}{0,0,0};
\colorlet{c}{natgreen};
\draw [c] (3.31886,1.97279) -- (3.31886,2.09474);
\draw [c] (3.31886,2.09474) -- (3.31886,2.21669);
\draw [c] (3.31073,2.09474) -- (3.31886,2.09474);
\draw [c] (3.31886,2.09474) -- (3.327,2.09474);
\definecolor{c}{rgb}{0,0,0};
\colorlet{c}{natgreen};
\draw [c] (3.33514,2.31441) -- (3.33514,2.45017);
\draw [c] (3.33514,2.45017) -- (3.33514,2.58593);
\draw [c] (3.327,2.45017) -- (3.33514,2.45017);
\draw [c] (3.33514,2.45017) -- (3.34327,2.45017);
\definecolor{c}{rgb}{0,0,0};
\colorlet{c}{natgreen};
\draw [c] (3.35141,2.64591) -- (3.35141,2.7972);
\draw [c] (3.35141,2.7972) -- (3.35141,2.9485);
\draw [c] (3.34327,2.7972) -- (3.35141,2.7972);
\draw [c] (3.35141,2.7972) -- (3.35955,2.7972);
\definecolor{c}{rgb}{0,0,0};
\colorlet{c}{natgreen};
\draw [c] (3.36768,3.07353) -- (3.36768,3.24084);
\draw [c] (3.36768,3.24084) -- (3.36768,3.40814);
\draw [c] (3.35955,3.24084) -- (3.36768,3.24084);
\draw [c] (3.36768,3.24084) -- (3.37582,3.24084);
\definecolor{c}{rgb}{0,0,0};
\colorlet{c}{natgreen};
\draw [c] (3.38395,3.50243) -- (3.38395,3.68251);
\draw [c] (3.38395,3.68251) -- (3.38395,3.8626);
\draw [c] (3.37582,3.68251) -- (3.38395,3.68251);
\draw [c] (3.38395,3.68251) -- (3.39209,3.68251);
\definecolor{c}{rgb}{0,0,0};
\colorlet{c}{natgreen};
\draw [c] (3.40023,4.04994) -- (3.40023,4.24798);
\draw [c] (3.40023,4.24798) -- (3.40023,4.44601);
\draw [c] (3.39209,4.24798) -- (3.40023,4.24798);
\draw [c] (3.40023,4.24798) -- (3.40836,4.24798);
\definecolor{c}{rgb}{0,0,0};
\colorlet{c}{natgreen};
\draw [c] (3.4165,4.32972) -- (3.4165,4.53659);
\draw [c] (3.4165,4.53659) -- (3.4165,4.74346);
\draw [c] (3.40836,4.53659) -- (3.4165,4.53659);
\draw [c] (3.4165,4.53659) -- (3.42464,4.53659);
\definecolor{c}{rgb}{0,0,0};
\colorlet{c}{natgreen};
\draw [c] (3.43277,5.58375) -- (3.43277,5.82833);
\draw [c] (3.43277,5.82833) -- (3.43277,6.0729);
\draw [c] (3.42464,5.82833) -- (3.43277,5.82833);
\draw [c] (3.43277,5.82833) -- (3.44091,5.82833);
\definecolor{c}{rgb}{0,0,0};
\colorlet{c}{natgreen};
\draw [c] (3.44905,5.85444) -- (3.44905,6.10271);
\draw [c] (3.44905,6.10271) -- (3.44905,6.35098);
\draw [c] (3.44091,6.10271) -- (3.44905,6.10271);
\draw [c] (3.44905,6.10271) -- (3.45718,6.10271);
\definecolor{c}{rgb}{0,0,0};
\colorlet{c}{natgreen};
\draw [c] (3.46532,5.99755) -- (3.46532,6.2519);
\draw [c] (3.46532,6.2519) -- (3.46532,6.50624);
\draw [c] (3.45718,6.2519) -- (3.46532,6.2519);
\draw [c] (3.46532,6.2519) -- (3.47345,6.2519);
\definecolor{c}{rgb}{0,0,0};
\colorlet{c}{natgreen};
\draw [c] (3.48159,5.75888) -- (3.48159,6.00699);
\draw [c] (3.48159,6.00699) -- (3.48159,6.2551);
\draw [c] (3.47345,6.00699) -- (3.48159,6.00699);
\draw [c] (3.48159,6.00699) -- (3.48973,6.00699);
\definecolor{c}{rgb}{0,0,0};
\colorlet{c}{natgreen};
\draw [c] (3.49786,5.99961) -- (3.49786,6.25327);
\draw [c] (3.49786,6.25327) -- (3.49786,6.50693);
\draw [c] (3.48973,6.25327) -- (3.49786,6.25327);
\draw [c] (3.49786,6.25327) -- (3.506,6.25327);
\definecolor{c}{rgb}{0,0,0};
\colorlet{c}{natgreen};
\draw [c] (3.51414,6.07844) -- (3.51414,6.33275);
\draw [c] (3.51414,6.33275) -- (3.51414,6.58705);
\draw [c] (3.506,6.33275) -- (3.51414,6.33275);
\draw [c] (3.51414,6.33275) -- (3.52227,6.33275);
\definecolor{c}{rgb}{0,0,0};
\colorlet{c}{natgreen};
\draw [c] (3.53041,5.86819) -- (3.53041,6.1164);
\draw [c] (3.53041,6.1164) -- (3.53041,6.36461);
\draw [c] (3.52227,6.1164) -- (3.53041,6.1164);
\draw [c] (3.53041,6.1164) -- (3.53855,6.1164);
\definecolor{c}{rgb}{0,0,0};
\colorlet{c}{natgreen};
\draw [c] (3.54668,5.42558) -- (3.54668,5.66604);
\draw [c] (3.54668,5.66604) -- (3.54668,5.90649);
\draw [c] (3.53855,5.66604) -- (3.54668,5.66604);
\draw [c] (3.54668,5.66604) -- (3.55482,5.66604);
\definecolor{c}{rgb}{0,0,0};
\colorlet{c}{natgreen};
\draw [c] (3.56295,5.15763) -- (3.56295,5.3894);
\draw [c] (3.56295,5.3894) -- (3.56295,5.62118);
\draw [c] (3.55482,5.3894) -- (3.56295,5.3894);
\draw [c] (3.56295,5.3894) -- (3.57109,5.3894);
\definecolor{c}{rgb}{0,0,0};
\colorlet{c}{natgreen};
\draw [c] (3.57923,4.48033) -- (3.57923,4.69298);
\draw [c] (3.57923,4.69298) -- (3.57923,4.90562);
\draw [c] (3.57109,4.69298) -- (3.57923,4.69298);
\draw [c] (3.57923,4.69298) -- (3.58736,4.69298);
\definecolor{c}{rgb}{0,0,0};
\colorlet{c}{natgreen};
\draw [c] (3.5955,4.00128) -- (3.5955,4.20052);
\draw [c] (3.5955,4.20052) -- (3.5955,4.39975);
\draw [c] (3.58736,4.20052) -- (3.5955,4.20052);
\draw [c] (3.5955,4.20052) -- (3.60364,4.20052);
\definecolor{c}{rgb}{0,0,0};
\colorlet{c}{natgreen};
\draw [c] (3.61177,4.5575) -- (3.61177,4.7732);
\draw [c] (3.61177,4.7732) -- (3.61177,4.98891);
\draw [c] (3.60364,4.7732) -- (3.61177,4.7732);
\draw [c] (3.61177,4.7732) -- (3.61991,4.7732);
\definecolor{c}{rgb}{0,0,0};
\colorlet{c}{natgreen};
\draw [c] (3.62805,3.69671) -- (3.62805,3.88706);
\draw [c] (3.62805,3.88706) -- (3.62805,4.07741);
\draw [c] (3.61991,3.88706) -- (3.62805,3.88706);
\draw [c] (3.62805,3.88706) -- (3.63618,3.88706);
\definecolor{c}{rgb}{0,0,0};
\colorlet{c}{natgreen};
\draw [c] (3.64432,3.2541) -- (3.64432,3.43157);
\draw [c] (3.64432,3.43157) -- (3.64432,3.60905);
\draw [c] (3.63618,3.43157) -- (3.64432,3.43157);
\draw [c] (3.64432,3.43157) -- (3.65245,3.43157);
\definecolor{c}{rgb}{0,0,0};
\colorlet{c}{natgreen};
\draw [c] (3.66059,3.29397) -- (3.66059,3.46949);
\draw [c] (3.66059,3.46949) -- (3.66059,3.64502);
\draw [c] (3.65245,3.46949) -- (3.66059,3.46949);
\draw [c] (3.66059,3.46949) -- (3.66873,3.46949);
\definecolor{c}{rgb}{0,0,0};
\colorlet{c}{natgreen};
\draw [c] (3.67686,2.88985) -- (3.67686,3.05307);
\draw [c] (3.67686,3.05307) -- (3.67686,3.21628);
\draw [c] (3.66873,3.05307) -- (3.67686,3.05307);
\draw [c] (3.67686,3.05307) -- (3.685,3.05307);
\definecolor{c}{rgb}{0,0,0};
\colorlet{c}{natgreen};
\draw [c] (3.69314,2.89355) -- (3.69314,3.05681);
\draw [c] (3.69314,3.05681) -- (3.69314,3.22008);
\draw [c] (3.685,3.05681) -- (3.69314,3.05681);
\draw [c] (3.69314,3.05681) -- (3.70127,3.05681);
\definecolor{c}{rgb}{0,0,0};
\colorlet{c}{natgreen};
\draw [c] (3.70941,2.51251) -- (3.70941,2.66097);
\draw [c] (3.70941,2.66097) -- (3.70941,2.80943);
\draw [c] (3.70127,2.66097) -- (3.70941,2.66097);
\draw [c] (3.70941,2.66097) -- (3.71755,2.66097);
\definecolor{c}{rgb}{0,0,0};
\colorlet{c}{natgreen};
\draw [c] (3.72568,2.57195) -- (3.72568,2.72513);
\draw [c] (3.72568,2.72513) -- (3.72568,2.87831);
\draw [c] (3.71755,2.72513) -- (3.72568,2.72513);
\draw [c] (3.72568,2.72513) -- (3.73382,2.72513);
\definecolor{c}{rgb}{0,0,0};
\colorlet{c}{natgreen};
\draw [c] (3.74195,2.17915) -- (3.74195,2.31346);
\draw [c] (3.74195,2.31346) -- (3.74195,2.44777);
\draw [c] (3.73382,2.31346) -- (3.74195,2.31346);
\draw [c] (3.74195,2.31346) -- (3.75009,2.31346);
\definecolor{c}{rgb}{0,0,0};
\colorlet{c}{natgreen};
\draw [c] (3.75823,2.19392) -- (3.75823,2.32903);
\draw [c] (3.75823,2.32903) -- (3.75823,2.46413);
\draw [c] (3.75009,2.32903) -- (3.75823,2.32903);
\draw [c] (3.75823,2.32903) -- (3.76636,2.32903);
\definecolor{c}{rgb}{0,0,0};
\colorlet{c}{natgreen};
\draw [c] (3.7745,1.81437) -- (3.7745,1.92945);
\draw [c] (3.7745,1.92945) -- (3.7745,2.04452);
\draw [c] (3.76636,1.92945) -- (3.7745,1.92945);
\draw [c] (3.7745,1.92945) -- (3.78264,1.92945);
\definecolor{c}{rgb}{0,0,0};
\colorlet{c}{natgreen};
\draw [c] (3.79077,1.94943) -- (3.79077,2.07455);
\draw [c] (3.79077,2.07455) -- (3.79077,2.19966);
\draw [c] (3.78264,2.07455) -- (3.79077,2.07455);
\draw [c] (3.79077,2.07455) -- (3.79891,2.07455);
\definecolor{c}{rgb}{0,0,0};
\colorlet{c}{natgreen};
\draw [c] (3.80705,1.89796) -- (3.80705,2.01752);
\draw [c] (3.80705,2.01752) -- (3.80705,2.13708);
\draw [c] (3.79891,2.01752) -- (3.80705,2.01752);
\draw [c] (3.80705,2.01752) -- (3.81518,2.01752);
\definecolor{c}{rgb}{0,0,0};
\colorlet{c}{natgreen};
\draw [c] (3.82332,1.63297) -- (3.82332,1.7425);
\draw [c] (3.82332,1.7425) -- (3.82332,1.85203);
\draw [c] (3.81518,1.7425) -- (3.82332,1.7425);
\draw [c] (3.82332,1.7425) -- (3.83145,1.7425);
\definecolor{c}{rgb}{0,0,0};
\colorlet{c}{natgreen};
\draw [c] (3.83959,1.88026) -- (3.83959,1.99932);
\draw [c] (3.83959,1.99932) -- (3.83959,2.11838);
\draw [c] (3.83145,1.99932) -- (3.83959,1.99932);
\draw [c] (3.83959,1.99932) -- (3.84773,1.99932);
\definecolor{c}{rgb}{0,0,0};
\colorlet{c}{natgreen};
\draw [c] (3.85586,1.55762) -- (3.85586,1.66135);
\draw [c] (3.85586,1.66135) -- (3.85586,1.76508);
\draw [c] (3.84773,1.66135) -- (3.85586,1.66135);
\draw [c] (3.85586,1.66135) -- (3.864,1.66135);
\definecolor{c}{rgb}{0,0,0};
\colorlet{c}{natgreen};
\draw [c] (3.87214,1.37707) -- (3.87214,1.46506);
\draw [c] (3.87214,1.46506) -- (3.87214,1.55305);
\draw [c] (3.864,1.46506) -- (3.87214,1.46506);
\draw [c] (3.87214,1.46506) -- (3.88027,1.46506);
\definecolor{c}{rgb}{0,0,0};
\colorlet{c}{natgreen};
\draw [c] (3.88841,1.38074) -- (3.88841,1.47519);
\draw [c] (3.88841,1.47519) -- (3.88841,1.56964);
\draw [c] (3.88027,1.47519) -- (3.88841,1.47519);
\draw [c] (3.88841,1.47519) -- (3.89655,1.47519);
\definecolor{c}{rgb}{0,0,0};
\colorlet{c}{natgreen};
\draw [c] (3.90468,1.34633) -- (3.90468,1.43759);
\draw [c] (3.90468,1.43759) -- (3.90468,1.52885);
\draw [c] (3.89655,1.43759) -- (3.90468,1.43759);
\draw [c] (3.90468,1.43759) -- (3.91282,1.43759);
\definecolor{c}{rgb}{0,0,0};
\colorlet{c}{natgreen};
\draw [c] (3.92095,1.20184) -- (3.92095,1.28192);
\draw [c] (3.92095,1.28192) -- (3.92095,1.362);
\draw [c] (3.91282,1.28192) -- (3.92095,1.28192);
\draw [c] (3.92095,1.28192) -- (3.92909,1.28192);
\definecolor{c}{rgb}{0,0,0};
\colorlet{c}{natgreen};
\draw [c] (3.93723,1.14683) -- (3.93723,1.22409);
\draw [c] (3.93723,1.22409) -- (3.93723,1.30136);
\draw [c] (3.92909,1.22409) -- (3.93723,1.22409);
\draw [c] (3.93723,1.22409) -- (3.94536,1.22409);
\definecolor{c}{rgb}{0,0,0};
\colorlet{c}{natgreen};
\draw [c] (3.9535,1.09219) -- (3.9535,1.16486);
\draw [c] (3.9535,1.16486) -- (3.9535,1.23754);
\draw [c] (3.94536,1.16486) -- (3.9535,1.16486);
\draw [c] (3.9535,1.16486) -- (3.96164,1.16486);
\definecolor{c}{rgb}{0,0,0};
\colorlet{c}{natgreen};
\draw [c] (3.96977,1.03834) -- (3.96977,1.10508);
\draw [c] (3.96977,1.10508) -- (3.96977,1.17181);
\draw [c] (3.96164,1.10508) -- (3.96977,1.10508);
\draw [c] (3.96977,1.10508) -- (3.97791,1.10508);
\definecolor{c}{rgb}{0,0,0};
\colorlet{c}{natgreen};
\draw [c] (3.98605,1.09536) -- (3.98605,1.16626);
\draw [c] (3.98605,1.16626) -- (3.98605,1.23716);
\draw [c] (3.97791,1.16626) -- (3.98605,1.16626);
\draw [c] (3.98605,1.16626) -- (3.99418,1.16626);
\definecolor{c}{rgb}{0,0,0};
\colorlet{c}{natgreen};
\draw [c] (4.00232,0.977898) -- (4.00232,1.03829);
\draw [c] (4.00232,1.03829) -- (4.00232,1.09867);
\draw [c] (3.99418,1.03829) -- (4.00232,1.03829);
\draw [c] (4.00232,1.03829) -- (4.01045,1.03829);
\definecolor{c}{rgb}{0,0,0};
\colorlet{c}{natgreen};
\draw [c] (4.01859,1.13448) -- (4.01859,1.21092);
\draw [c] (4.01859,1.21092) -- (4.01859,1.28736);
\draw [c] (4.01045,1.21092) -- (4.01859,1.21092);
\draw [c] (4.01859,1.21092) -- (4.02673,1.21092);
\definecolor{c}{rgb}{0,0,0};
\colorlet{c}{natgreen};
\draw [c] (4.03486,0.932526) -- (4.03486,0.989231);
\draw [c] (4.03486,0.989231) -- (4.03486,1.04594);
\draw [c] (4.02673,0.989231) -- (4.03486,0.989231);
\draw [c] (4.03486,0.989231) -- (4.043,0.989231);
\definecolor{c}{rgb}{0,0,0};
\colorlet{c}{natgreen};
\draw [c] (4.05114,0.909038) -- (4.05114,0.964041);
\draw [c] (4.05114,0.964041) -- (4.05114,1.01904);
\draw [c] (4.043,0.964041) -- (4.05114,0.964041);
\draw [c] (4.05114,0.964041) -- (4.05927,0.964041);
\definecolor{c}{rgb}{0,0,0};
\colorlet{c}{natgreen};
\draw [c] (4.06741,0.905278) -- (4.06741,0.955802);
\draw [c] (4.06741,0.955802) -- (4.06741,1.00633);
\draw [c] (4.05927,0.955802) -- (4.06741,0.955802);
\draw [c] (4.06741,0.955802) -- (4.07555,0.955802);
\definecolor{c}{rgb}{0,0,0};
\colorlet{c}{natgreen};
\draw [c] (4.08368,0.910484) -- (4.08368,0.965967);
\draw [c] (4.08368,0.965967) -- (4.08368,1.02145);
\draw [c] (4.07555,0.965967) -- (4.08368,0.965967);
\draw [c] (4.08368,0.965967) -- (4.09182,0.965967);
\definecolor{c}{rgb}{0,0,0};
\colorlet{c}{natgreen};
\draw [c] (4.09995,0.84669) -- (4.09995,0.893411);
\draw [c] (4.09995,0.893411) -- (4.09995,0.940132);
\draw [c] (4.09182,0.893411) -- (4.09995,0.893411);
\draw [c] (4.09995,0.893411) -- (4.10809,0.893411);
\definecolor{c}{rgb}{0,0,0};
\colorlet{c}{natgreen};
\draw [c] (4.11623,0.816785) -- (4.11623,0.859365);
\draw [c] (4.11623,0.859365) -- (4.11623,0.901946);
\draw [c] (4.10809,0.859365) -- (4.11623,0.859365);
\draw [c] (4.11623,0.859365) -- (4.12436,0.859365);
\definecolor{c}{rgb}{0,0,0};
\colorlet{c}{natgreen};
\draw [c] (4.1325,0.820052) -- (4.1325,0.86177);
\draw [c] (4.1325,0.86177) -- (4.1325,0.903487);
\draw [c] (4.12436,0.86177) -- (4.1325,0.86177);
\draw [c] (4.1325,0.86177) -- (4.14064,0.86177);
\definecolor{c}{rgb}{0,0,0};
\colorlet{c}{natgreen};
\draw [c] (4.14877,0.830401) -- (4.14877,0.871992);
\draw [c] (4.14877,0.871992) -- (4.14877,0.913582);
\draw [c] (4.14064,0.871992) -- (4.14877,0.871992);
\draw [c] (4.14877,0.871992) -- (4.15691,0.871992);
\definecolor{c}{rgb}{0,0,0};
\colorlet{c}{natgreen};
\draw [c] (4.16505,0.79867) -- (4.16505,0.837525);
\draw [c] (4.16505,0.837525) -- (4.16505,0.87638);
\draw [c] (4.15691,0.837525) -- (4.16505,0.837525);
\draw [c] (4.16505,0.837525) -- (4.17318,0.837525);
\definecolor{c}{rgb}{0,0,0};
\colorlet{c}{natgreen};
\draw [c] (4.18132,0.784983) -- (4.18132,0.822264);
\draw [c] (4.18132,0.822264) -- (4.18132,0.859546);
\draw [c] (4.17318,0.822264) -- (4.18132,0.822264);
\draw [c] (4.18132,0.822264) -- (4.18945,0.822264);
\definecolor{c}{rgb}{0,0,0};
\colorlet{c}{natgreen};
\draw [c] (4.19759,0.794485) -- (4.19759,0.831406);
\draw [c] (4.19759,0.831406) -- (4.19759,0.868327);
\draw [c] (4.18945,0.831406) -- (4.19759,0.831406);
\draw [c] (4.19759,0.831406) -- (4.20573,0.831406);
\definecolor{c}{rgb}{0,0,0};
\colorlet{c}{natgreen};
\draw [c] (4.21386,0.787833) -- (4.21386,0.826205);
\draw [c] (4.21386,0.826205) -- (4.21386,0.864577);
\draw [c] (4.20573,0.826205) -- (4.21386,0.826205);
\draw [c] (4.21386,0.826205) -- (4.222,0.826205);
\definecolor{c}{rgb}{0,0,0};
\colorlet{c}{natgreen};
\draw [c] (4.23014,0.779584) -- (4.23014,0.816318);
\draw [c] (4.23014,0.816318) -- (4.23014,0.853052);
\draw [c] (4.222,0.816318) -- (4.23014,0.816318);
\draw [c] (4.23014,0.816318) -- (4.23827,0.816318);
\definecolor{c}{rgb}{0,0,0};
\colorlet{c}{natgreen};
\draw [c] (4.24641,0.773341) -- (4.24641,0.809215);
\draw [c] (4.24641,0.809215) -- (4.24641,0.845089);
\draw [c] (4.23827,0.809215) -- (4.24641,0.809215);
\draw [c] (4.24641,0.809215) -- (4.25455,0.809215);
\definecolor{c}{rgb}{0,0,0};
\colorlet{c}{natgreen};
\draw [c] (4.26268,0.754109) -- (4.26268,0.789001);
\draw [c] (4.26268,0.789001) -- (4.26268,0.823894);
\draw [c] (4.25455,0.789001) -- (4.26268,0.789001);
\draw [c] (4.26268,0.789001) -- (4.27082,0.789001);
\definecolor{c}{rgb}{0,0,0};
\colorlet{c}{natgreen};
\draw [c] (4.27895,0.761187) -- (4.27895,0.795352);
\draw [c] (4.27895,0.795352) -- (4.27895,0.829516);
\draw [c] (4.27082,0.795352) -- (4.27895,0.795352);
\draw [c] (4.27895,0.795352) -- (4.28709,0.795352);
\definecolor{c}{rgb}{0,0,0};
\colorlet{c}{natgreen};
\draw [c] (4.29523,0.703056) -- (4.29523,0.723502);
\draw [c] (4.29523,0.723502) -- (4.29523,0.743949);
\draw [c] (4.28709,0.723502) -- (4.29523,0.723502);
\draw [c] (4.29523,0.723502) -- (4.30336,0.723502);
\definecolor{c}{rgb}{0,0,0};
\colorlet{c}{natgreen};
\draw [c] (4.3115,0.731532) -- (4.3115,0.761049);
\draw [c] (4.3115,0.761049) -- (4.3115,0.790567);
\draw [c] (4.30336,0.761049) -- (4.3115,0.761049);
\draw [c] (4.3115,0.761049) -- (4.31964,0.761049);
\definecolor{c}{rgb}{0,0,0};
\colorlet{c}{natgreen};
\draw [c] (4.32777,0.745916) -- (4.32777,0.773891);
\draw [c] (4.32777,0.773891) -- (4.32777,0.801865);
\draw [c] (4.31964,0.773891) -- (4.32777,0.773891);
\draw [c] (4.32777,0.773891) -- (4.33591,0.773891);
\definecolor{c}{rgb}{0,0,0};
\colorlet{c}{natgreen};
\draw [c] (4.34405,0.743554) -- (4.34405,0.773152);
\draw [c] (4.34405,0.773152) -- (4.34405,0.80275);
\draw [c] (4.33591,0.773152) -- (4.34405,0.773152);
\draw [c] (4.34405,0.773152) -- (4.35218,0.773152);
\definecolor{c}{rgb}{0,0,0};
\colorlet{c}{natgreen};
\draw [c] (4.36032,0.692936) -- (4.36032,0.707758);
\draw [c] (4.36032,0.707758) -- (4.36032,0.72258);
\draw [c] (4.35218,0.707758) -- (4.36032,0.707758);
\draw [c] (4.36032,0.707758) -- (4.36845,0.707758);
\definecolor{c}{rgb}{0,0,0};
\colorlet{c}{natgreen};
\draw [c] (4.37659,0.757202) -- (4.37659,0.792096);
\draw [c] (4.37659,0.792096) -- (4.37659,0.826991);
\draw [c] (4.36845,0.792096) -- (4.37659,0.792096);
\draw [c] (4.37659,0.792096) -- (4.38473,0.792096);
\definecolor{c}{rgb}{0,0,0};
\colorlet{c}{natgreen};
\draw [c] (4.39286,0.702945) -- (4.39286,0.719518);
\draw [c] (4.39286,0.719518) -- (4.39286,0.73609);
\draw [c] (4.38473,0.719518) -- (4.39286,0.719518);
\draw [c] (4.39286,0.719518) -- (4.401,0.719518);
\definecolor{c}{rgb}{0,0,0};
\colorlet{c}{natgreen};
\draw [c] (4.40914,0.723316) -- (4.40914,0.749374);
\draw [c] (4.40914,0.749374) -- (4.40914,0.775431);
\draw [c] (4.401,0.749374) -- (4.40914,0.749374);
\draw [c] (4.40914,0.749374) -- (4.41727,0.749374);
\definecolor{c}{rgb}{0,0,0};
\colorlet{c}{natgreen};
\draw [c] (4.42541,0.70816) -- (4.42541,0.729558);
\draw [c] (4.42541,0.729558) -- (4.42541,0.750956);
\draw [c] (4.41727,0.729558) -- (4.42541,0.729558);
\draw [c] (4.42541,0.729558) -- (4.43355,0.729558);
\definecolor{c}{rgb}{0,0,0};
\colorlet{c}{natgreen};
\draw [c] (4.44168,0.731388) -- (4.44168,0.758337);
\draw [c] (4.44168,0.758337) -- (4.44168,0.785287);
\draw [c] (4.43355,0.758337) -- (4.44168,0.758337);
\draw [c] (4.44168,0.758337) -- (4.44982,0.758337);
\definecolor{c}{rgb}{0,0,0};
\colorlet{c}{natgreen};
\draw [c] (4.45795,0.715949) -- (4.45795,0.737013);
\draw [c] (4.45795,0.737013) -- (4.45795,0.758078);
\draw [c] (4.44982,0.737013) -- (4.45795,0.737013);
\draw [c] (4.45795,0.737013) -- (4.46609,0.737013);
\definecolor{c}{rgb}{0,0,0};
\colorlet{c}{natgreen};
\draw [c] (4.47423,0.708102) -- (4.47423,0.72723);
\draw [c] (4.47423,0.72723) -- (4.47423,0.746359);
\draw [c] (4.46609,0.72723) -- (4.47423,0.72723);
\draw [c] (4.47423,0.72723) -- (4.48236,0.72723);
\definecolor{c}{rgb}{0,0,0};
\colorlet{c}{natgreen};
\draw [c] (4.4905,0.695838) -- (4.4905,0.713323);
\draw [c] (4.4905,0.713323) -- (4.4905,0.730808);
\draw [c] (4.48236,0.713323) -- (4.4905,0.713323);
\draw [c] (4.4905,0.713323) -- (4.49864,0.713323);
\definecolor{c}{rgb}{0,0,0};
\colorlet{c}{natgreen};
\draw [c] (4.50677,0.70065) -- (4.50677,0.71825);
\draw [c] (4.50677,0.71825) -- (4.50677,0.735851);
\draw [c] (4.49864,0.71825) -- (4.50677,0.71825);
\draw [c] (4.50677,0.71825) -- (4.51491,0.71825);
\definecolor{c}{rgb}{0,0,0};
\colorlet{c}{natgreen};
\draw [c] (4.52305,0.692273) -- (4.52305,0.70583);
\draw [c] (4.52305,0.70583) -- (4.52305,0.719387);
\draw [c] (4.51491,0.70583) -- (4.52305,0.70583);
\draw [c] (4.52305,0.70583) -- (4.53118,0.70583);
\definecolor{c}{rgb}{0,0,0};
\colorlet{c}{natgreen};
\draw [c] (4.53932,0.708612) -- (4.53932,0.730843);
\draw [c] (4.53932,0.730843) -- (4.53932,0.753075);
\draw [c] (4.53118,0.730843) -- (4.53932,0.730843);
\draw [c] (4.53932,0.730843) -- (4.54745,0.730843);
\definecolor{c}{rgb}{0,0,0};
\colorlet{c}{natgreen};
\draw [c] (4.55559,0.699596) -- (4.55559,0.720953);
\draw [c] (4.55559,0.720953) -- (4.55559,0.74231);
\draw [c] (4.54745,0.720953) -- (4.55559,0.720953);
\draw [c] (4.55559,0.720953) -- (4.56373,0.720953);
\definecolor{c}{rgb}{0,0,0};
\colorlet{c}{natgreen};
\draw [c] (4.57186,0.709471) -- (4.57186,0.731051);
\draw [c] (4.57186,0.731051) -- (4.57186,0.75263);
\draw [c] (4.56373,0.731051) -- (4.57186,0.731051);
\draw [c] (4.57186,0.731051) -- (4.58,0.731051);
\definecolor{c}{rgb}{0,0,0};
\colorlet{c}{natgreen};
\draw [c] (4.58814,0.69136) -- (4.58814,0.704002);
\draw [c] (4.58814,0.704002) -- (4.58814,0.716643);
\draw [c] (4.58,0.704002) -- (4.58814,0.704002);
\draw [c] (4.58814,0.704002) -- (4.59627,0.704002);
\definecolor{c}{rgb}{0,0,0};
\colorlet{c}{natgreen};
\draw [c] (4.60441,0.688612) -- (4.60441,0.700212);
\draw [c] (4.60441,0.700212) -- (4.60441,0.711812);
\draw [c] (4.59627,0.700212) -- (4.60441,0.700212);
\draw [c] (4.60441,0.700212) -- (4.61255,0.700212);
\definecolor{c}{rgb}{0,0,0};
\colorlet{c}{natgreen};
\draw [c] (4.62068,0.691463) -- (4.62068,0.70493);
\draw [c] (4.62068,0.70493) -- (4.62068,0.718398);
\draw [c] (4.61255,0.70493) -- (4.62068,0.70493);
\draw [c] (4.62068,0.70493) -- (4.62882,0.70493);
\definecolor{c}{rgb}{0,0,0};
\colorlet{c}{natgreen};
\draw [c] (4.65323,0.690572) -- (4.65323,0.702891);
\draw [c] (4.65323,0.702891) -- (4.65323,0.71521);
\draw [c] (4.64509,0.702891) -- (4.65323,0.702891);
\draw [c] (4.65323,0.702891) -- (4.66136,0.702891);
\definecolor{c}{rgb}{0,0,0};
\colorlet{c}{natgreen};
\draw [c] (4.68577,0.686894) -- (4.68577,0.69532);
\draw [c] (4.68577,0.69532) -- (4.68577,0.703746);
\draw [c] (4.67764,0.69532) -- (4.68577,0.69532);
\draw [c] (4.68577,0.69532) -- (4.69391,0.69532);
\definecolor{c}{rgb}{0,0,0};
\colorlet{c}{natgreen};
\draw [c] (4.70205,0.702308) -- (4.70205,0.723366);
\draw [c] (4.70205,0.723366) -- (4.70205,0.744423);
\draw [c] (4.69391,0.723366) -- (4.70205,0.723366);
\draw [c] (4.70205,0.723366) -- (4.71018,0.723366);
\definecolor{c}{rgb}{0,0,0};
\colorlet{c}{natgreen};
\draw [c] (4.73459,0.686894) -- (4.73459,0.687013);
\draw [c] (4.73459,0.687013) -- (4.73459,0.687133);
\draw [c] (4.72645,0.687013) -- (4.73459,0.687013);
\draw [c] (4.73459,0.687013) -- (4.74273,0.687013);
\definecolor{c}{rgb}{0,0,0};
\colorlet{c}{natgreen};
\draw [c] (4.75086,0.686894) -- (4.75086,0.699741);
\draw [c] (4.75086,0.699741) -- (4.75086,0.712588);
\draw [c] (4.74273,0.699741) -- (4.75086,0.699741);
\draw [c] (4.75086,0.699741) -- (4.759,0.699741);
\definecolor{c}{rgb}{0,0,0};
\colorlet{c}{natgreen};
\draw [c] (4.76714,0.689559) -- (4.76714,0.695992);
\draw [c] (4.76714,0.695992) -- (4.76714,0.702426);
\draw [c] (4.759,0.695992) -- (4.76714,0.695992);
\draw [c] (4.76714,0.695992) -- (4.77527,0.695992);
\definecolor{c}{rgb}{0,0,0};
\colorlet{c}{natgreen};
\draw [c] (4.78341,0.688608) -- (4.78341,0.699913);
\draw [c] (4.78341,0.699913) -- (4.78341,0.711218);
\draw [c] (4.77527,0.699913) -- (4.78341,0.699913);
\draw [c] (4.78341,0.699913) -- (4.79155,0.699913);
\definecolor{c}{rgb}{0,0,0};
\colorlet{c}{natgreen};
\draw [c] (4.79968,0.697235) -- (4.79968,0.712428);
\draw [c] (4.79968,0.712428) -- (4.79968,0.72762);
\draw [c] (4.79155,0.712428) -- (4.79968,0.712428);
\draw [c] (4.79968,0.712428) -- (4.80782,0.712428);
\definecolor{c}{rgb}{0,0,0};
\colorlet{c}{natgreen};
\draw [c] (4.83223,0.686894) -- (4.83223,0.691443);
\draw [c] (4.83223,0.691443) -- (4.83223,0.695992);
\draw [c] (4.82409,0.691443) -- (4.83223,0.691443);
\draw [c] (4.83223,0.691443) -- (4.84036,0.691443);
\definecolor{c}{rgb}{0,0,0};
\colorlet{c}{natgreen};
\draw [c] (4.8485,0.686894) -- (4.8485,0.692853);
\draw [c] (4.8485,0.692853) -- (4.8485,0.698811);
\draw [c] (4.84036,0.692853) -- (4.8485,0.692853);
\draw [c] (4.8485,0.692853) -- (4.85664,0.692853);
\definecolor{c}{rgb}{0,0,0};
\colorlet{c}{natgreen};
\draw [c] (4.88105,0.691479) -- (4.88105,0.70509);
\draw [c] (4.88105,0.70509) -- (4.88105,0.718702);
\draw [c] (4.87291,0.70509) -- (4.88105,0.70509);
\draw [c] (4.88105,0.70509) -- (4.88918,0.70509);
\definecolor{c}{rgb}{0,0,0};
\colorlet{c}{natgreen};
\draw [c] (4.89732,0.697187) -- (4.89732,0.714402);
\draw [c] (4.89732,0.714402) -- (4.89732,0.731616);
\draw [c] (4.88918,0.714402) -- (4.89732,0.714402);
\draw [c] (4.89732,0.714402) -- (4.90545,0.714402);
\definecolor{c}{rgb}{0,0,0};
\colorlet{c}{natgreen};
\draw [c] (4.94614,0.701168) -- (4.94614,0.722014);
\draw [c] (4.94614,0.722014) -- (4.94614,0.74286);
\draw [c] (4.938,0.722014) -- (4.94614,0.722014);
\draw [c] (4.94614,0.722014) -- (4.95427,0.722014);
\definecolor{c}{rgb}{0,0,0};
\colorlet{c}{natgreen};
\draw [c] (4.96241,0.686894) -- (4.96241,0.70053);
\draw [c] (4.96241,0.70053) -- (4.96241,0.714166);
\draw [c] (4.95427,0.70053) -- (4.96241,0.70053);
\draw [c] (4.96241,0.70053) -- (4.97055,0.70053);
\definecolor{c}{rgb}{0,0,0};
\colorlet{c}{natgreen};
\draw [c] (5.04377,0.686894) -- (5.04377,0.688764);
\draw [c] (5.04377,0.688764) -- (5.04377,0.690634);
\draw [c] (5.03564,0.688764) -- (5.04377,0.688764);
\draw [c] (5.04377,0.688764) -- (5.05191,0.688764);
\definecolor{c}{rgb}{0,0,0};
\colorlet{c}{natgreen};
\draw [c] (5.06005,0.686894) -- (5.06005,0.698043);
\draw [c] (5.06005,0.698043) -- (5.06005,0.709192);
\draw [c] (5.05191,0.698043) -- (5.06005,0.698043);
\draw [c] (5.06005,0.698043) -- (5.06818,0.698043);
\definecolor{c}{rgb}{0,0,0};
\colorlet{c}{natgreen};
\draw [c] (5.10886,0.686894) -- (5.10886,0.697253);
\draw [c] (5.10886,0.697253) -- (5.10886,0.707613);
\draw [c] (5.10073,0.697253) -- (5.10886,0.697253);
\draw [c] (5.10886,0.697253) -- (5.117,0.697253);
\definecolor{c}{rgb}{0,0,0};
\colorlet{c}{natgreen};
\draw [c] (5.15768,0.686894) -- (5.15768,0.699132);
\draw [c] (5.15768,0.699132) -- (5.15768,0.71137);
\draw [c] (5.14955,0.699132) -- (5.15768,0.699132);
\draw [c] (5.15768,0.699132) -- (5.16582,0.699132);
\definecolor{c}{rgb}{0,0,0};
\colorlet{c}{natgreen};
\draw [c] (5.19023,0.686894) -- (5.19023,0.687595);
\draw [c] (5.19023,0.687595) -- (5.19023,0.688296);
\draw [c] (5.18209,0.687595) -- (5.19023,0.687595);
\draw [c] (5.19023,0.687595) -- (5.19836,0.687595);
\definecolor{c}{rgb}{0,0,0};
\colorlet{c}{natgreen};
\draw [c] (5.30414,0.686894) -- (5.30414,0.692853);
\draw [c] (5.30414,0.692853) -- (5.30414,0.698811);
\draw [c] (5.296,0.692853) -- (5.30414,0.692853);
\draw [c] (5.30414,0.692853) -- (5.31227,0.692853);
\definecolor{c}{rgb}{0,0,0};
\colorlet{c}{natgreen};
\draw [c] (5.33668,0.686894) -- (5.33668,0.694017);
\draw [c] (5.33668,0.694017) -- (5.33668,0.70114);
\draw [c] (5.32855,0.694017) -- (5.33668,0.694017);
\draw [c] (5.33668,0.694017) -- (5.34482,0.694017);
\definecolor{c}{rgb}{0,0,0};
\colorlet{c}{natgreen};
\draw [c] (5.36923,0.686894) -- (5.36923,0.69065);
\draw [c] (5.36923,0.69065) -- (5.36923,0.694407);
\draw [c] (5.36109,0.69065) -- (5.36923,0.69065);
\draw [c] (5.36923,0.69065) -- (5.37736,0.69065);
\definecolor{c}{rgb}{0,0,0};
\colorlet{c}{natgreen};
\draw [c] (5.3855,0.686894) -- (5.3855,0.687595);
\draw [c] (5.3855,0.687595) -- (5.3855,0.688296);
\draw [c] (5.37736,0.687595) -- (5.3855,0.687595);
\draw [c] (5.3855,0.687595) -- (5.39364,0.687595);
\definecolor{c}{rgb}{0,0,0};
\colorlet{c}{natgreen};
\draw [c] (5.40177,0.686894) -- (5.40177,0.694017);
\draw [c] (5.40177,0.694017) -- (5.40177,0.70114);
\draw [c] (5.39364,0.694017) -- (5.40177,0.694017);
\draw [c] (5.40177,0.694017) -- (5.40991,0.694017);
\definecolor{c}{rgb}{0,0,0};
\colorlet{c}{natgreen};
\draw [c] (5.41805,0.686894) -- (5.41805,0.694017);
\draw [c] (5.41805,0.694017) -- (5.41805,0.70114);
\draw [c] (5.40991,0.694017) -- (5.41805,0.694017);
\draw [c] (5.41805,0.694017) -- (5.42618,0.694017);
\definecolor{c}{rgb}{0,0,0};
\colorlet{c}{natgreen};
\draw [c] (5.43432,0.686894) -- (5.43432,0.696318);
\draw [c] (5.43432,0.696318) -- (5.43432,0.705742);
\draw [c] (5.42618,0.696318) -- (5.43432,0.696318);
\draw [c] (5.43432,0.696318) -- (5.44245,0.696318);
\definecolor{c}{rgb}{0,0,0};
\colorlet{c}{natgreen};
\draw [c] (5.45059,0.686894) -- (5.45059,0.691443);
\draw [c] (5.45059,0.691443) -- (5.45059,0.695992);
\draw [c] (5.44245,0.691443) -- (5.45059,0.691443);
\draw [c] (5.45059,0.691443) -- (5.45873,0.691443);
\definecolor{c}{rgb}{0,0,0};
\colorlet{c}{natgreen};
\draw [c] (5.48314,0.686894) -- (5.48314,0.698972);
\draw [c] (5.48314,0.698972) -- (5.48314,0.711049);
\draw [c] (5.475,0.698972) -- (5.48314,0.698972);
\draw [c] (5.48314,0.698972) -- (5.49127,0.698972);
\definecolor{c}{rgb}{0,0,0};
\colorlet{c}{natgreen};
\draw [c] (5.71095,0.686894) -- (5.71095,0.70053);
\draw [c] (5.71095,0.70053) -- (5.71095,0.714166);
\draw [c] (5.70282,0.70053) -- (5.71095,0.70053);
\draw [c] (5.71095,0.70053) -- (5.71909,0.70053);
\definecolor{c}{rgb}{0,0,0};
\colorlet{c}{natgreen};
\draw [c] (5.7435,0.686894) -- (5.7435,0.698334);
\draw [c] (5.7435,0.698334) -- (5.7435,0.709774);
\draw [c] (5.73536,0.698334) -- (5.7435,0.698334);
\draw [c] (5.7435,0.698334) -- (5.75164,0.698334);
\definecolor{c}{rgb}{0,0,0};
\colorlet{c}{natgreen};
\draw [c] (5.79232,0.686894) -- (5.79232,0.69532);
\draw [c] (5.79232,0.69532) -- (5.79232,0.703746);
\draw [c] (5.78418,0.69532) -- (5.79232,0.69532);
\draw [c] (5.79232,0.69532) -- (5.80045,0.69532);
\definecolor{c}{rgb}{0,0,0};
\colorlet{c}{natgreen};
\draw [c] (5.80859,0.686894) -- (5.80859,0.694017);
\draw [c] (5.80859,0.694017) -- (5.80859,0.70114);
\draw [c] (5.80045,0.694017) -- (5.80859,0.694017);
\draw [c] (5.80859,0.694017) -- (5.81673,0.694017);
\definecolor{c}{rgb}{0,0,0};
\colorlet{c}{natgreen};
\draw [c] (5.93877,0.686894) -- (5.93877,0.697253);
\draw [c] (5.93877,0.697253) -- (5.93877,0.707613);
\draw [c] (5.93064,0.697253) -- (5.93877,0.697253);
\draw [c] (5.93877,0.697253) -- (5.94691,0.697253);
\definecolor{c}{rgb}{0,0,0};
\colorlet{c}{natgreen};
\draw [c] (6.39441,0.686894) -- (6.39441,0.698489);
\draw [c] (6.39441,0.698489) -- (6.39441,0.710083);
\draw [c] (6.38627,0.698489) -- (6.39441,0.698489);
\draw [c] (6.39441,0.698489) -- (6.40255,0.698489);
\definecolor{c}{rgb}{0,0,0};
\colorlet{c}{natgreen};
\draw [c] (6.57341,0.686894) -- (6.57341,0.698606);
\draw [c] (6.57341,0.698606) -- (6.57341,0.710318);
\draw [c] (6.56527,0.698606) -- (6.57341,0.698606);
\draw [c] (6.57341,0.698606) -- (6.58155,0.698606);
\definecolor{c}{rgb}{0,0,0};
\colorlet{c}{natgreen};
\draw [c] (7.74505,0.686894) -- (7.74505,0.690097);
\draw [c] (7.74505,0.690097) -- (7.74505,0.6933);
\draw [c] (7.73691,0.690097) -- (7.74505,0.690097);
\draw [c] (7.74505,0.690097) -- (7.75318,0.690097);
\definecolor{c}{rgb}{0,0,0};
\draw [c] (1,0.680516) -- (9.95,0.680516);
\draw [anchor= east] (9.95,0.108883) node[color=c, rotate=0]{$E_{T}^{iso} \text{ [GeV]}$};
\draw [c] (1,0.863234) -- (1,0.680516);
\draw [c] (1.25571,0.771875) -- (1.25571,0.680516);
\draw [c] (1.51143,0.771875) -- (1.51143,0.680516);
\draw [c] (1.76714,0.771875) -- (1.76714,0.680516);
\draw [c] (2.02286,0.771875) -- (2.02286,0.680516);
\draw [c] (2.27857,0.863234) -- (2.27857,0.680516);
\draw [c] (2.53429,0.771875) -- (2.53429,0.680516);
\draw [c] (2.79,0.771875) -- (2.79,0.680516);
\draw [c] (3.04571,0.771875) -- (3.04571,0.680516);
\draw [c] (3.30143,0.771875) -- (3.30143,0.680516);
\draw [c] (3.55714,0.863234) -- (3.55714,0.680516);
\draw [c] (3.81286,0.771875) -- (3.81286,0.680516);
\draw [c] (4.06857,0.771875) -- (4.06857,0.680516);
\draw [c] (4.32429,0.771875) -- (4.32429,0.680516);
\draw [c] (4.58,0.771875) -- (4.58,0.680516);
\draw [c] (4.83571,0.863234) -- (4.83571,0.680516);
\draw [c] (5.09143,0.771875) -- (5.09143,0.680516);
\draw [c] (5.34714,0.771875) -- (5.34714,0.680516);
\draw [c] (5.60286,0.771875) -- (5.60286,0.680516);
\draw [c] (5.85857,0.771875) -- (5.85857,0.680516);
\draw [c] (6.11429,0.863234) -- (6.11429,0.680516);
\draw [c] (6.37,0.771875) -- (6.37,0.680516);
\draw [c] (6.62571,0.771875) -- (6.62571,0.680516);
\draw [c] (6.88143,0.771875) -- (6.88143,0.680516);
\draw [c] (7.13714,0.771875) -- (7.13714,0.680516);
\draw [c] (7.39286,0.863234) -- (7.39286,0.680516);
\draw [c] (7.64857,0.771875) -- (7.64857,0.680516);
\draw [c] (7.90429,0.771875) -- (7.90429,0.680516);
\draw [c] (8.16,0.771875) -- (8.16,0.680516);
\draw [c] (8.41571,0.771875) -- (8.41571,0.680516);
\draw [c] (8.67143,0.863234) -- (8.67143,0.680516);
\draw [c] (8.92714,0.771875) -- (8.92714,0.680516);
\draw [c] (9.18286,0.771875) -- (9.18286,0.680516);
\draw [c] (9.43857,0.771875) -- (9.43857,0.680516);
\draw [c] (9.69429,0.771875) -- (9.69429,0.680516);
\draw [c] (9.95,0.863234) -- (9.95,0.680516);
\draw [c] (9.95,0.863234) -- (9.95,0.680516);
\draw [anchor=base] (1,0.353868) node[color=c, rotate=0]{-20};
\draw [anchor=base] (2.27857,0.353868) node[color=c, rotate=0]{-10};
\draw [anchor=base] (3.55714,0.353868) node[color=c, rotate=0]{0};
\draw [anchor=base] (4.83571,0.353868) node[color=c, rotate=0]{10};
\draw [anchor=base] (6.11429,0.353868) node[color=c, rotate=0]{20};
\draw [anchor=base] (7.39286,0.353868) node[color=c, rotate=0]{30};
\draw [anchor=base] (8.67143,0.353868) node[color=c, rotate=0]{40};
\draw [anchor=base] (9.95,0.353868) node[color=c, rotate=0]{50};
\draw [c] (1,0.680516) -- (1,6.73711);
\draw [anchor= east] (-0.12,6.73711) node[color=c, rotate=90]{Normalised number of events};
\draw [c] (1.267,0.686894) -- (1,0.686894);
\draw [c] (1.1335,0.960568) -- (1,0.960568);
\draw [c] (1.1335,1.23424) -- (1,1.23424);
\draw [c] (1.1335,1.50792) -- (1,1.50792);
\draw [c] (1.267,1.78159) -- (1,1.78159);
\draw [c] (1.1335,2.05527) -- (1,2.05527);
\draw [c] (1.1335,2.32894) -- (1,2.32894);
\draw [c] (1.1335,2.60262) -- (1,2.60262);
\draw [c] (1.267,2.87629) -- (1,2.87629);
\draw [c] (1.1335,3.14996) -- (1,3.14996);
\draw [c] (1.1335,3.42364) -- (1,3.42364);
\draw [c] (1.1335,3.69731) -- (1,3.69731);
\draw [c] (1.267,3.97099) -- (1,3.97099);
\draw [c] (1.1335,4.24466) -- (1,4.24466);
\draw [c] (1.1335,4.51834) -- (1,4.51834);
\draw [c] (1.1335,4.79201) -- (1,4.79201);
\draw [c] (1.267,5.06569) -- (1,5.06569);
\draw [c] (1.1335,5.33936) -- (1,5.33936);
\draw [c] (1.1335,5.61304) -- (1,5.61304);
\draw [c] (1.1335,5.88671) -- (1,5.88671);
\draw [c] (1.267,6.16039) -- (1,6.16039);
\draw [c] (1.267,0.686894) -- (1,0.686894);
\draw [c] (1.267,6.16039) -- (1,6.16039);
\draw [c] (1.1335,6.43406) -- (1,6.43406);
\draw [c] (1.1335,6.70773) -- (1,6.70773);
\draw [anchor= east] (0.95,0.686894) node[color=c, rotate=0]{0};
\draw [anchor= east] (0.95,1.78159) node[color=c, rotate=0]{200};
\draw [anchor= east] (0.95,2.87629) node[color=c, rotate=0]{400};
\draw [anchor= east] (0.95,3.97099) node[color=c, rotate=0]{600};
\draw [anchor= east] (0.95,5.06569) node[color=c, rotate=0]{800};
\draw [anchor= east] (0.95,6.16039) node[color=c, rotate=0]{1000};
\colorlet{c}{natcomp!70};
\draw [c] (1.04068,0.686894) -- (1.04068,0.6869);
\draw [c] (1.04068,0.6869) -- (1.04068,0.686907);
\draw [c] (1.03255,0.6869) -- (1.04068,0.6869);
\draw [c] (1.04068,0.6869) -- (1.04882,0.6869);
\definecolor{c}{rgb}{0,0,0};
\colorlet{c}{natcomp!70};
\draw [c] (1.10577,0.686894) -- (1.10577,0.705403);
\draw [c] (1.10577,0.705403) -- (1.10577,0.723913);
\draw [c] (1.09764,0.705403) -- (1.10577,0.705403);
\draw [c] (1.10577,0.705403) -- (1.11391,0.705403);
\definecolor{c}{rgb}{0,0,0};
\colorlet{c}{natcomp!70};
\draw [c] (1.12205,0.6869) -- (1.12205,0.708668);
\draw [c] (1.12205,0.708668) -- (1.12205,0.730436);
\draw [c] (1.11391,0.708668) -- (1.12205,0.708668);
\draw [c] (1.12205,0.708668) -- (1.13018,0.708668);
\definecolor{c}{rgb}{0,0,0};
\colorlet{c}{natcomp!70};
\draw [c] (1.13832,0.686894) -- (1.13832,0.686899);
\draw [c] (1.13832,0.686899) -- (1.13832,0.686904);
\draw [c] (1.13018,0.686899) -- (1.13832,0.686899);
\draw [c] (1.13832,0.686899) -- (1.14645,0.686899);
\definecolor{c}{rgb}{0,0,0};
\colorlet{c}{natcomp!70};
\draw [c] (1.15459,0.686907) -- (1.15459,0.695046);
\draw [c] (1.15459,0.695046) -- (1.15459,0.703185);
\draw [c] (1.14645,0.695046) -- (1.15459,0.695046);
\draw [c] (1.15459,0.695046) -- (1.16273,0.695046);
\definecolor{c}{rgb}{0,0,0};
\colorlet{c}{natcomp!70};
\draw [c] (1.17086,0.686894) -- (1.17086,0.686904);
\draw [c] (1.17086,0.686904) -- (1.17086,0.686914);
\draw [c] (1.16273,0.686904) -- (1.17086,0.686904);
\draw [c] (1.17086,0.686904) -- (1.179,0.686904);
\definecolor{c}{rgb}{0,0,0};
\colorlet{c}{natcomp!70};
\draw [c] (1.18714,0.686897) -- (1.18714,0.686904);
\draw [c] (1.18714,0.686904) -- (1.18714,0.686911);
\draw [c] (1.179,0.686904) -- (1.18714,0.686904);
\draw [c] (1.18714,0.686904) -- (1.19527,0.686904);
\definecolor{c}{rgb}{0,0,0};
\colorlet{c}{natcomp!70};
\draw [c] (1.20341,0.686906) -- (1.20341,0.705415);
\draw [c] (1.20341,0.705415) -- (1.20341,0.723925);
\draw [c] (1.19527,0.705415) -- (1.20341,0.705415);
\draw [c] (1.20341,0.705415) -- (1.21155,0.705415);
\definecolor{c}{rgb}{0,0,0};
\colorlet{c}{natcomp!70};
\draw [c] (1.21968,0.686898) -- (1.21968,0.68691);
\draw [c] (1.21968,0.68691) -- (1.21968,0.686923);
\draw [c] (1.21155,0.68691) -- (1.21968,0.68691);
\draw [c] (1.21968,0.68691) -- (1.22782,0.68691);
\definecolor{c}{rgb}{0,0,0};
\colorlet{c}{natcomp!70};
\draw [c] (1.25223,0.695733) -- (1.25223,0.723286);
\draw [c] (1.25223,0.723286) -- (1.25223,0.75084);
\draw [c] (1.24409,0.723286) -- (1.25223,0.723286);
\draw [c] (1.25223,0.723286) -- (1.26036,0.723286);
\definecolor{c}{rgb}{0,0,0};
\colorlet{c}{natcomp!70};
\draw [c] (1.2685,0.686901) -- (1.2685,0.686911);
\draw [c] (1.2685,0.686911) -- (1.2685,0.686922);
\draw [c] (1.26036,0.686911) -- (1.2685,0.686911);
\draw [c] (1.2685,0.686911) -- (1.27664,0.686911);
\definecolor{c}{rgb}{0,0,0};
\colorlet{c}{natcomp!70};
\draw [c] (1.28477,0.686898) -- (1.28477,0.686915);
\draw [c] (1.28477,0.686915) -- (1.28477,0.686932);
\draw [c] (1.27664,0.686915) -- (1.28477,0.686915);
\draw [c] (1.28477,0.686915) -- (1.29291,0.686915);
\definecolor{c}{rgb}{0,0,0};
\colorlet{c}{natcomp!70};
\draw [c] (1.31732,0.694427) -- (1.31732,0.721188);
\draw [c] (1.31732,0.721188) -- (1.31732,0.747948);
\draw [c] (1.30918,0.721188) -- (1.31732,0.721188);
\draw [c] (1.31732,0.721188) -- (1.32545,0.721188);
\definecolor{c}{rgb}{0,0,0};
\colorlet{c}{natcomp!70};
\draw [c] (1.34986,0.686899) -- (1.34986,0.69462);
\draw [c] (1.34986,0.69462) -- (1.34986,0.702342);
\draw [c] (1.34173,0.69462) -- (1.34986,0.69462);
\draw [c] (1.34986,0.69462) -- (1.358,0.69462);
\definecolor{c}{rgb}{0,0,0};
\colorlet{c}{natcomp!70};
\draw [c] (1.36614,0.686894) -- (1.36614,0.6869);
\draw [c] (1.36614,0.6869) -- (1.36614,0.686907);
\draw [c] (1.358,0.6869) -- (1.36614,0.6869);
\draw [c] (1.36614,0.6869) -- (1.37427,0.6869);
\definecolor{c}{rgb}{0,0,0};
\colorlet{c}{natcomp!70};
\draw [c] (1.38241,0.68691) -- (1.38241,0.71207);
\draw [c] (1.38241,0.71207) -- (1.38241,0.73723);
\draw [c] (1.37427,0.71207) -- (1.38241,0.71207);
\draw [c] (1.38241,0.71207) -- (1.39055,0.71207);
\definecolor{c}{rgb}{0,0,0};
\colorlet{c}{natcomp!70};
\draw [c] (1.39868,0.696781) -- (1.39868,0.722664);
\draw [c] (1.39868,0.722664) -- (1.39868,0.748546);
\draw [c] (1.39055,0.722664) -- (1.39868,0.722664);
\draw [c] (1.39868,0.722664) -- (1.40682,0.722664);
\definecolor{c}{rgb}{0,0,0};
\colorlet{c}{natcomp!70};
\draw [c] (1.41495,0.686894) -- (1.41495,0.697305);
\draw [c] (1.41495,0.697305) -- (1.41495,0.707716);
\draw [c] (1.40682,0.697305) -- (1.41495,0.697305);
\draw [c] (1.41495,0.697305) -- (1.42309,0.697305);
\definecolor{c}{rgb}{0,0,0};
\colorlet{c}{natcomp!70};
\draw [c] (1.43123,0.6869) -- (1.43123,0.700903);
\draw [c] (1.43123,0.700903) -- (1.43123,0.714905);
\draw [c] (1.42309,0.700903) -- (1.43123,0.700903);
\draw [c] (1.43123,0.700903) -- (1.43936,0.700903);
\definecolor{c}{rgb}{0,0,0};
\colorlet{c}{natcomp!70};
\draw [c] (1.4475,0.686909) -- (1.4475,0.705418);
\draw [c] (1.4475,0.705418) -- (1.4475,0.723928);
\draw [c] (1.43936,0.705418) -- (1.4475,0.705418);
\draw [c] (1.4475,0.705418) -- (1.45564,0.705418);
\definecolor{c}{rgb}{0,0,0};
\colorlet{c}{natcomp!70};
\draw [c] (1.46377,0.686915) -- (1.46377,0.705424);
\draw [c] (1.46377,0.705424) -- (1.46377,0.723933);
\draw [c] (1.45564,0.705424) -- (1.46377,0.705424);
\draw [c] (1.46377,0.705424) -- (1.47191,0.705424);
\definecolor{c}{rgb}{0,0,0};
\colorlet{c}{natcomp!70};
\draw [c] (1.48005,0.6869) -- (1.48005,0.686916);
\draw [c] (1.48005,0.686916) -- (1.48005,0.686932);
\draw [c] (1.47191,0.686916) -- (1.48005,0.686916);
\draw [c] (1.48005,0.686916) -- (1.48818,0.686916);
\definecolor{c}{rgb}{0,0,0};
\colorlet{c}{natcomp!70};
\draw [c] (1.49632,0.686904) -- (1.49632,0.700906);
\draw [c] (1.49632,0.700906) -- (1.49632,0.714908);
\draw [c] (1.48818,0.700906) -- (1.49632,0.700906);
\draw [c] (1.49632,0.700906) -- (1.50445,0.700906);
\definecolor{c}{rgb}{0,0,0};
\colorlet{c}{natcomp!70};
\draw [c] (1.51259,0.686912) -- (1.51259,0.698144);
\draw [c] (1.51259,0.698144) -- (1.51259,0.709377);
\draw [c] (1.50445,0.698144) -- (1.51259,0.698144);
\draw [c] (1.51259,0.698144) -- (1.52073,0.698144);
\definecolor{c}{rgb}{0,0,0};
\colorlet{c}{natcomp!70};
\draw [c] (1.52886,0.686905) -- (1.52886,0.697316);
\draw [c] (1.52886,0.697316) -- (1.52886,0.707727);
\draw [c] (1.52073,0.697316) -- (1.52886,0.697316);
\draw [c] (1.52886,0.697316) -- (1.537,0.697316);
\definecolor{c}{rgb}{0,0,0};
\colorlet{c}{natcomp!70};
\draw [c] (1.54514,0.68691) -- (1.54514,0.686925);
\draw [c] (1.54514,0.686925) -- (1.54514,0.68694);
\draw [c] (1.537,0.686925) -- (1.54514,0.686925);
\draw [c] (1.54514,0.686925) -- (1.55327,0.686925);
\definecolor{c}{rgb}{0,0,0};
\colorlet{c}{natcomp!70};
\draw [c] (1.56141,0.686899) -- (1.56141,0.686912);
\draw [c] (1.56141,0.686912) -- (1.56141,0.686926);
\draw [c] (1.55327,0.686912) -- (1.56141,0.686912);
\draw [c] (1.56141,0.686912) -- (1.56955,0.686912);
\definecolor{c}{rgb}{0,0,0};
\colorlet{c}{natcomp!70};
\draw [c] (1.57768,0.686904) -- (1.57768,0.705413);
\draw [c] (1.57768,0.705413) -- (1.57768,0.723923);
\draw [c] (1.56955,0.705413) -- (1.57768,0.705413);
\draw [c] (1.57768,0.705413) -- (1.58582,0.705413);
\definecolor{c}{rgb}{0,0,0};
\colorlet{c}{natcomp!70};
\draw [c] (1.59395,0.686923) -- (1.59395,0.696817);
\draw [c] (1.59395,0.696817) -- (1.59395,0.706711);
\draw [c] (1.58582,0.696817) -- (1.59395,0.696817);
\draw [c] (1.59395,0.696817) -- (1.60209,0.696817);
\definecolor{c}{rgb}{0,0,0};
\colorlet{c}{natcomp!70};
\draw [c] (1.61023,0.686911) -- (1.61023,0.719865);
\draw [c] (1.61023,0.719865) -- (1.61023,0.752819);
\draw [c] (1.60209,0.719865) -- (1.61023,0.719865);
\draw [c] (1.61023,0.719865) -- (1.61836,0.719865);
\definecolor{c}{rgb}{0,0,0};
\colorlet{c}{natcomp!70};
\draw [c] (1.6265,0.692256) -- (1.6265,0.705147);
\draw [c] (1.6265,0.705147) -- (1.6265,0.718038);
\draw [c] (1.61836,0.705147) -- (1.6265,0.705147);
\draw [c] (1.6265,0.705147) -- (1.63464,0.705147);
\definecolor{c}{rgb}{0,0,0};
\colorlet{c}{natcomp!70};
\draw [c] (1.64277,0.686912) -- (1.64277,0.686932);
\draw [c] (1.64277,0.686932) -- (1.64277,0.686952);
\draw [c] (1.63464,0.686932) -- (1.64277,0.686932);
\draw [c] (1.64277,0.686932) -- (1.65091,0.686932);
\definecolor{c}{rgb}{0,0,0};
\colorlet{c}{natcomp!70};
\draw [c] (1.65905,0.692192) -- (1.65905,0.704906);
\draw [c] (1.65905,0.704906) -- (1.65905,0.717619);
\draw [c] (1.65091,0.704906) -- (1.65905,0.704906);
\draw [c] (1.65905,0.704906) -- (1.66718,0.704906);
\definecolor{c}{rgb}{0,0,0};
\colorlet{c}{natcomp!70};
\draw [c] (1.67532,0.700693) -- (1.67532,0.723056);
\draw [c] (1.67532,0.723056) -- (1.67532,0.74542);
\draw [c] (1.66718,0.723056) -- (1.67532,0.723056);
\draw [c] (1.67532,0.723056) -- (1.68345,0.723056);
\definecolor{c}{rgb}{0,0,0};
\colorlet{c}{natcomp!70};
\draw [c] (1.69159,0.686942) -- (1.69159,0.708221);
\draw [c] (1.69159,0.708221) -- (1.69159,0.729499);
\draw [c] (1.68345,0.708221) -- (1.69159,0.708221);
\draw [c] (1.69159,0.708221) -- (1.69973,0.708221);
\definecolor{c}{rgb}{0,0,0};
\colorlet{c}{natcomp!70};
\draw [c] (1.70786,0.705683) -- (1.70786,0.724842);
\draw [c] (1.70786,0.724842) -- (1.70786,0.744);
\draw [c] (1.69973,0.724842) -- (1.70786,0.724842);
\draw [c] (1.70786,0.724842) -- (1.716,0.724842);
\definecolor{c}{rgb}{0,0,0};
\colorlet{c}{natcomp!70};
\draw [c] (1.72414,0.686939) -- (1.72414,0.700941);
\draw [c] (1.72414,0.700941) -- (1.72414,0.714943);
\draw [c] (1.716,0.700941) -- (1.72414,0.700941);
\draw [c] (1.72414,0.700941) -- (1.73227,0.700941);
\definecolor{c}{rgb}{0,0,0};
\colorlet{c}{natcomp!70};
\draw [c] (1.74041,0.686936) -- (1.74041,0.708215);
\draw [c] (1.74041,0.708215) -- (1.74041,0.729493);
\draw [c] (1.73227,0.708215) -- (1.74041,0.708215);
\draw [c] (1.74041,0.708215) -- (1.74855,0.708215);
\definecolor{c}{rgb}{0,0,0};
\colorlet{c}{natcomp!70};
\draw [c] (1.75668,0.686903) -- (1.75668,0.696018);
\draw [c] (1.75668,0.696018) -- (1.75668,0.705133);
\draw [c] (1.74855,0.696018) -- (1.75668,0.696018);
\draw [c] (1.75668,0.696018) -- (1.76482,0.696018);
\definecolor{c}{rgb}{0,0,0};
\colorlet{c}{natcomp!70};
\draw [c] (1.77295,0.714667) -- (1.77295,0.738024);
\draw [c] (1.77295,0.738024) -- (1.77295,0.761381);
\draw [c] (1.76482,0.738024) -- (1.77295,0.738024);
\draw [c] (1.77295,0.738024) -- (1.78109,0.738024);
\definecolor{c}{rgb}{0,0,0};
\colorlet{c}{natcomp!70};
\draw [c] (1.78923,0.713535) -- (1.78923,0.735176);
\draw [c] (1.78923,0.735176) -- (1.78923,0.756817);
\draw [c] (1.78109,0.735176) -- (1.78923,0.735176);
\draw [c] (1.78923,0.735176) -- (1.79736,0.735176);
\definecolor{c}{rgb}{0,0,0};
\colorlet{c}{natcomp!70};
\draw [c] (1.8055,0.706706) -- (1.8055,0.727668);
\draw [c] (1.8055,0.727668) -- (1.8055,0.748629);
\draw [c] (1.79736,0.727668) -- (1.8055,0.727668);
\draw [c] (1.8055,0.727668) -- (1.81364,0.727668);
\definecolor{c}{rgb}{0,0,0};
\colorlet{c}{natcomp!70};
\draw [c] (1.82177,0.71253) -- (1.82177,0.7417);
\draw [c] (1.82177,0.7417) -- (1.82177,0.77087);
\draw [c] (1.81364,0.7417) -- (1.82177,0.7417);
\draw [c] (1.82177,0.7417) -- (1.82991,0.7417);
\definecolor{c}{rgb}{0,0,0};
\colorlet{c}{natcomp!70};
\draw [c] (1.83805,0.686939) -- (1.83805,0.698172);
\draw [c] (1.83805,0.698172) -- (1.83805,0.709404);
\draw [c] (1.82991,0.698172) -- (1.83805,0.698172);
\draw [c] (1.83805,0.698172) -- (1.84618,0.698172);
\definecolor{c}{rgb}{0,0,0};
\colorlet{c}{natcomp!70};
\draw [c] (1.85432,0.693926) -- (1.85432,0.714558);
\draw [c] (1.85432,0.714558) -- (1.85432,0.73519);
\draw [c] (1.84618,0.714558) -- (1.85432,0.714558);
\draw [c] (1.85432,0.714558) -- (1.86245,0.714558);
\definecolor{c}{rgb}{0,0,0};
\colorlet{c}{natcomp!70};
\draw [c] (1.87059,0.702794) -- (1.87059,0.726598);
\draw [c] (1.87059,0.726598) -- (1.87059,0.750403);
\draw [c] (1.86245,0.726598) -- (1.87059,0.726598);
\draw [c] (1.87059,0.726598) -- (1.87873,0.726598);
\definecolor{c}{rgb}{0,0,0};
\colorlet{c}{natcomp!70};
\draw [c] (1.88686,0.723513) -- (1.88686,0.749566);
\draw [c] (1.88686,0.749566) -- (1.88686,0.775619);
\draw [c] (1.87873,0.749566) -- (1.88686,0.749566);
\draw [c] (1.88686,0.749566) -- (1.895,0.749566);
\definecolor{c}{rgb}{0,0,0};
\colorlet{c}{natcomp!70};
\draw [c] (1.90314,0.703651) -- (1.90314,0.728586);
\draw [c] (1.90314,0.728586) -- (1.90314,0.753521);
\draw [c] (1.895,0.728586) -- (1.90314,0.728586);
\draw [c] (1.90314,0.728586) -- (1.91127,0.728586);
\definecolor{c}{rgb}{0,0,0};
\colorlet{c}{natcomp!70};
\draw [c] (1.91941,0.735552) -- (1.91941,0.762435);
\draw [c] (1.91941,0.762435) -- (1.91941,0.789317);
\draw [c] (1.91127,0.762435) -- (1.91941,0.762435);
\draw [c] (1.91941,0.762435) -- (1.92755,0.762435);
\definecolor{c}{rgb}{0,0,0};
\colorlet{c}{natcomp!70};
\draw [c] (1.93568,0.705159) -- (1.93568,0.723496);
\draw [c] (1.93568,0.723496) -- (1.93568,0.741834);
\draw [c] (1.92755,0.723496) -- (1.93568,0.723496);
\draw [c] (1.93568,0.723496) -- (1.94382,0.723496);
\definecolor{c}{rgb}{0,0,0};
\colorlet{c}{natcomp!70};
\draw [c] (1.95195,0.728073) -- (1.95195,0.759278);
\draw [c] (1.95195,0.759278) -- (1.95195,0.790483);
\draw [c] (1.94382,0.759278) -- (1.95195,0.759278);
\draw [c] (1.95195,0.759278) -- (1.96009,0.759278);
\definecolor{c}{rgb}{0,0,0};
\colorlet{c}{natcomp!70};
\draw [c] (1.96823,0.721308) -- (1.96823,0.745166);
\draw [c] (1.96823,0.745166) -- (1.96823,0.769024);
\draw [c] (1.96009,0.745166) -- (1.96823,0.745166);
\draw [c] (1.96823,0.745166) -- (1.97636,0.745166);
\definecolor{c}{rgb}{0,0,0};
\colorlet{c}{natcomp!70};
\draw [c] (1.9845,0.717672) -- (1.9845,0.74667);
\draw [c] (1.9845,0.74667) -- (1.9845,0.775667);
\draw [c] (1.97636,0.74667) -- (1.9845,0.74667);
\draw [c] (1.9845,0.74667) -- (1.99264,0.74667);
\definecolor{c}{rgb}{0,0,0};
\colorlet{c}{natcomp!70};
\draw [c] (2.00077,0.754545) -- (2.00077,0.791474);
\draw [c] (2.00077,0.791474) -- (2.00077,0.828402);
\draw [c] (1.99264,0.791474) -- (2.00077,0.791474);
\draw [c] (2.00077,0.791474) -- (2.00891,0.791474);
\definecolor{c}{rgb}{0,0,0};
\colorlet{c}{natcomp!70};
\draw [c] (2.01705,0.812339) -- (2.01705,0.861065);
\draw [c] (2.01705,0.861065) -- (2.01705,0.909792);
\draw [c] (2.00891,0.861065) -- (2.01705,0.861065);
\draw [c] (2.01705,0.861065) -- (2.02518,0.861065);
\definecolor{c}{rgb}{0,0,0};
\colorlet{c}{natcomp!70};
\draw [c] (2.03332,0.756859) -- (2.03332,0.788069);
\draw [c] (2.03332,0.788069) -- (2.03332,0.819279);
\draw [c] (2.02518,0.788069) -- (2.03332,0.788069);
\draw [c] (2.03332,0.788069) -- (2.04145,0.788069);
\definecolor{c}{rgb}{0,0,0};
\colorlet{c}{natcomp!70};
\draw [c] (2.04959,0.733143) -- (2.04959,0.772578);
\draw [c] (2.04959,0.772578) -- (2.04959,0.812014);
\draw [c] (2.04145,0.772578) -- (2.04959,0.772578);
\draw [c] (2.04959,0.772578) -- (2.05773,0.772578);
\definecolor{c}{rgb}{0,0,0};
\colorlet{c}{natcomp!70};
\draw [c] (2.06586,0.734846) -- (2.06586,0.767565);
\draw [c] (2.06586,0.767565) -- (2.06586,0.800284);
\draw [c] (2.05773,0.767565) -- (2.06586,0.767565);
\draw [c] (2.06586,0.767565) -- (2.074,0.767565);
\definecolor{c}{rgb}{0,0,0};
\colorlet{c}{natcomp!70};
\draw [c] (2.08214,0.789195) -- (2.08214,0.827149);
\draw [c] (2.08214,0.827149) -- (2.08214,0.865104);
\draw [c] (2.074,0.827149) -- (2.08214,0.827149);
\draw [c] (2.08214,0.827149) -- (2.09027,0.827149);
\definecolor{c}{rgb}{0,0,0};
\colorlet{c}{natcomp!70};
\draw [c] (2.09841,0.788521) -- (2.09841,0.828396);
\draw [c] (2.09841,0.828396) -- (2.09841,0.868271);
\draw [c] (2.09027,0.828396) -- (2.09841,0.828396);
\draw [c] (2.09841,0.828396) -- (2.10655,0.828396);
\definecolor{c}{rgb}{0,0,0};
\colorlet{c}{natcomp!70};
\draw [c] (2.11468,0.759161) -- (2.11468,0.799535);
\draw [c] (2.11468,0.799535) -- (2.11468,0.839908);
\draw [c] (2.10655,0.799535) -- (2.11468,0.799535);
\draw [c] (2.11468,0.799535) -- (2.12282,0.799535);
\definecolor{c}{rgb}{0,0,0};
\colorlet{c}{natcomp!70};
\draw [c] (2.13095,0.7679) -- (2.13095,0.805702);
\draw [c] (2.13095,0.805702) -- (2.13095,0.843503);
\draw [c] (2.12282,0.805702) -- (2.13095,0.805702);
\draw [c] (2.13095,0.805702) -- (2.13909,0.805702);
\definecolor{c}{rgb}{0,0,0};
\colorlet{c}{natcomp!70};
\draw [c] (2.14723,0.746911) -- (2.14723,0.783535);
\draw [c] (2.14723,0.783535) -- (2.14723,0.820159);
\draw [c] (2.13909,0.783535) -- (2.14723,0.783535);
\draw [c] (2.14723,0.783535) -- (2.15536,0.783535);
\definecolor{c}{rgb}{0,0,0};
\colorlet{c}{natcomp!70};
\draw [c] (2.1635,0.772802) -- (2.1635,0.808728);
\draw [c] (2.1635,0.808728) -- (2.1635,0.844654);
\draw [c] (2.15536,0.808728) -- (2.1635,0.808728);
\draw [c] (2.1635,0.808728) -- (2.17164,0.808728);
\definecolor{c}{rgb}{0,0,0};
\colorlet{c}{natcomp!70};
\draw [c] (2.17977,0.810552) -- (2.17977,0.849296);
\draw [c] (2.17977,0.849296) -- (2.17977,0.888039);
\draw [c] (2.17164,0.849296) -- (2.17977,0.849296);
\draw [c] (2.17977,0.849296) -- (2.18791,0.849296);
\definecolor{c}{rgb}{0,0,0};
\colorlet{c}{natcomp!70};
\draw [c] (2.19605,0.778252) -- (2.19605,0.813555);
\draw [c] (2.19605,0.813555) -- (2.19605,0.848858);
\draw [c] (2.18791,0.813555) -- (2.19605,0.813555);
\draw [c] (2.19605,0.813555) -- (2.20418,0.813555);
\definecolor{c}{rgb}{0,0,0};
\colorlet{c}{natcomp!70};
\draw [c] (2.21232,0.789468) -- (2.21232,0.827907);
\draw [c] (2.21232,0.827907) -- (2.21232,0.866346);
\draw [c] (2.20418,0.827907) -- (2.21232,0.827907);
\draw [c] (2.21232,0.827907) -- (2.22045,0.827907);
\definecolor{c}{rgb}{0,0,0};
\colorlet{c}{natcomp!70};
\draw [c] (2.22859,0.815059) -- (2.22859,0.859924);
\draw [c] (2.22859,0.859924) -- (2.22859,0.904788);
\draw [c] (2.22045,0.859924) -- (2.22859,0.859924);
\draw [c] (2.22859,0.859924) -- (2.23673,0.859924);
\definecolor{c}{rgb}{0,0,0};
\colorlet{c}{natcomp!70};
\draw [c] (2.24486,0.802412) -- (2.24486,0.839797);
\draw [c] (2.24486,0.839797) -- (2.24486,0.877183);
\draw [c] (2.23673,0.839797) -- (2.24486,0.839797);
\draw [c] (2.24486,0.839797) -- (2.253,0.839797);
\definecolor{c}{rgb}{0,0,0};
\colorlet{c}{natcomp!70};
\draw [c] (2.26114,0.840272) -- (2.26114,0.885266);
\draw [c] (2.26114,0.885266) -- (2.26114,0.93026);
\draw [c] (2.253,0.885266) -- (2.26114,0.885266);
\draw [c] (2.26114,0.885266) -- (2.26927,0.885266);
\definecolor{c}{rgb}{0,0,0};
\colorlet{c}{natcomp!70};
\draw [c] (2.27741,0.805072) -- (2.27741,0.848826);
\draw [c] (2.27741,0.848826) -- (2.27741,0.892579);
\draw [c] (2.26927,0.848826) -- (2.27741,0.848826);
\draw [c] (2.27741,0.848826) -- (2.28555,0.848826);
\definecolor{c}{rgb}{0,0,0};
\colorlet{c}{natcomp!70};
\draw [c] (2.29368,0.791504) -- (2.29368,0.826715);
\draw [c] (2.29368,0.826715) -- (2.29368,0.861925);
\draw [c] (2.28555,0.826715) -- (2.29368,0.826715);
\draw [c] (2.29368,0.826715) -- (2.30182,0.826715);
\definecolor{c}{rgb}{0,0,0};
\colorlet{c}{natcomp!70};
\draw [c] (2.30995,0.856209) -- (2.30995,0.906613);
\draw [c] (2.30995,0.906613) -- (2.30995,0.957017);
\draw [c] (2.30182,0.906613) -- (2.30995,0.906613);
\draw [c] (2.30995,0.906613) -- (2.31809,0.906613);
\definecolor{c}{rgb}{0,0,0};
\colorlet{c}{natcomp!70};
\draw [c] (2.32623,0.828277) -- (2.32623,0.869435);
\draw [c] (2.32623,0.869435) -- (2.32623,0.910592);
\draw [c] (2.31809,0.869435) -- (2.32623,0.869435);
\draw [c] (2.32623,0.869435) -- (2.33436,0.869435);
\definecolor{c}{rgb}{0,0,0};
\colorlet{c}{natcomp!70};
\draw [c] (2.3425,0.88625) -- (2.3425,0.939801);
\draw [c] (2.3425,0.939801) -- (2.3425,0.993352);
\draw [c] (2.33436,0.939801) -- (2.3425,0.939801);
\draw [c] (2.3425,0.939801) -- (2.35064,0.939801);
\definecolor{c}{rgb}{0,0,0};
\colorlet{c}{natcomp!70};
\draw [c] (2.35877,0.916275) -- (2.35877,0.969867);
\draw [c] (2.35877,0.969867) -- (2.35877,1.02346);
\draw [c] (2.35064,0.969867) -- (2.35877,0.969867);
\draw [c] (2.35877,0.969867) -- (2.36691,0.969867);
\definecolor{c}{rgb}{0,0,0};
\colorlet{c}{natcomp!70};
\draw [c] (2.37505,0.907296) -- (2.37505,0.962405);
\draw [c] (2.37505,0.962405) -- (2.37505,1.01751);
\draw [c] (2.36691,0.962405) -- (2.37505,0.962405);
\draw [c] (2.37505,0.962405) -- (2.38318,0.962405);
\definecolor{c}{rgb}{0,0,0};
\colorlet{c}{natcomp!70};
\draw [c] (2.39132,0.827878) -- (2.39132,0.86913);
\draw [c] (2.39132,0.86913) -- (2.39132,0.910383);
\draw [c] (2.38318,0.86913) -- (2.39132,0.86913);
\draw [c] (2.39132,0.86913) -- (2.39945,0.86913);
\definecolor{c}{rgb}{0,0,0};
\colorlet{c}{natcomp!70};
\draw [c] (2.40759,0.857178) -- (2.40759,0.910918);
\draw [c] (2.40759,0.910918) -- (2.40759,0.964658);
\draw [c] (2.39945,0.910918) -- (2.40759,0.910918);
\draw [c] (2.40759,0.910918) -- (2.41573,0.910918);
\definecolor{c}{rgb}{0,0,0};
\colorlet{c}{natcomp!70};
\draw [c] (2.42386,0.919688) -- (2.42386,0.974446);
\draw [c] (2.42386,0.974446) -- (2.42386,1.0292);
\draw [c] (2.41573,0.974446) -- (2.42386,0.974446);
\draw [c] (2.42386,0.974446) -- (2.432,0.974446);
\definecolor{c}{rgb}{0,0,0};
\colorlet{c}{natcomp!70};
\draw [c] (2.44014,0.898282) -- (2.44014,0.948641);
\draw [c] (2.44014,0.948641) -- (2.44014,0.999);
\draw [c] (2.432,0.948641) -- (2.44014,0.948641);
\draw [c] (2.44014,0.948641) -- (2.44827,0.948641);
\definecolor{c}{rgb}{0,0,0};
\colorlet{c}{natcomp!70};
\draw [c] (2.45641,0.910429) -- (2.45641,0.968628);
\draw [c] (2.45641,0.968628) -- (2.45641,1.02683);
\draw [c] (2.44827,0.968628) -- (2.45641,0.968628);
\draw [c] (2.45641,0.968628) -- (2.46455,0.968628);
\definecolor{c}{rgb}{0,0,0};
\colorlet{c}{natcomp!70};
\draw [c] (2.47268,0.884339) -- (2.47268,0.931852);
\draw [c] (2.47268,0.931852) -- (2.47268,0.979366);
\draw [c] (2.46455,0.931852) -- (2.47268,0.931852);
\draw [c] (2.47268,0.931852) -- (2.48082,0.931852);
\definecolor{c}{rgb}{0,0,0};
\colorlet{c}{natcomp!70};
\draw [c] (2.48895,0.969757) -- (2.48895,1.0318);
\draw [c] (2.48895,1.0318) -- (2.48895,1.09385);
\draw [c] (2.48082,1.0318) -- (2.48895,1.0318);
\draw [c] (2.48895,1.0318) -- (2.49709,1.0318);
\definecolor{c}{rgb}{0,0,0};
\colorlet{c}{natcomp!70};
\draw [c] (2.50523,0.971435) -- (2.50523,1.03092);
\draw [c] (2.50523,1.03092) -- (2.50523,1.0904);
\draw [c] (2.49709,1.03092) -- (2.50523,1.03092);
\draw [c] (2.50523,1.03092) -- (2.51336,1.03092);
\definecolor{c}{rgb}{0,0,0};
\colorlet{c}{natcomp!70};
\draw [c] (2.5215,0.990909) -- (2.5215,1.05577);
\draw [c] (2.5215,1.05577) -- (2.5215,1.12064);
\draw [c] (2.51336,1.05577) -- (2.5215,1.05577);
\draw [c] (2.5215,1.05577) -- (2.52964,1.05577);
\definecolor{c}{rgb}{0,0,0};
\colorlet{c}{natcomp!70};
\draw [c] (2.53777,1.00227) -- (2.53777,1.06689);
\draw [c] (2.53777,1.06689) -- (2.53777,1.13151);
\draw [c] (2.52964,1.06689) -- (2.53777,1.06689);
\draw [c] (2.53777,1.06689) -- (2.54591,1.06689);
\definecolor{c}{rgb}{0,0,0};
\colorlet{c}{natcomp!70};
\draw [c] (2.55405,1.04669) -- (2.55405,1.11011);
\draw [c] (2.55405,1.11011) -- (2.55405,1.17353);
\draw [c] (2.54591,1.11011) -- (2.55405,1.11011);
\draw [c] (2.55405,1.11011) -- (2.56218,1.11011);
\definecolor{c}{rgb}{0,0,0};
\colorlet{c}{natcomp!70};
\draw [c] (2.57032,0.995602) -- (2.57032,1.06059);
\draw [c] (2.57032,1.06059) -- (2.57032,1.12558);
\draw [c] (2.56218,1.06059) -- (2.57032,1.06059);
\draw [c] (2.57032,1.06059) -- (2.57845,1.06059);
\definecolor{c}{rgb}{0,0,0};
\colorlet{c}{natcomp!70};
\draw [c] (2.58659,0.99475) -- (2.58659,1.05464);
\draw [c] (2.58659,1.05464) -- (2.58659,1.11453);
\draw [c] (2.57845,1.05464) -- (2.58659,1.05464);
\draw [c] (2.58659,1.05464) -- (2.59473,1.05464);
\definecolor{c}{rgb}{0,0,0};
\colorlet{c}{natcomp!70};
\draw [c] (2.60286,1.07289) -- (2.60286,1.14199);
\draw [c] (2.60286,1.14199) -- (2.60286,1.21109);
\draw [c] (2.59473,1.14199) -- (2.60286,1.14199);
\draw [c] (2.60286,1.14199) -- (2.611,1.14199);
\definecolor{c}{rgb}{0,0,0};
\colorlet{c}{natcomp!70};
\draw [c] (2.61914,1.13773) -- (2.61914,1.21239);
\draw [c] (2.61914,1.21239) -- (2.61914,1.28705);
\draw [c] (2.611,1.21239) -- (2.61914,1.21239);
\draw [c] (2.61914,1.21239) -- (2.62727,1.21239);
\definecolor{c}{rgb}{0,0,0};
\colorlet{c}{natcomp!70};
\draw [c] (2.63541,0.996344) -- (2.63541,1.05637);
\draw [c] (2.63541,1.05637) -- (2.63541,1.1164);
\draw [c] (2.62727,1.05637) -- (2.63541,1.05637);
\draw [c] (2.63541,1.05637) -- (2.64355,1.05637);
\definecolor{c}{rgb}{0,0,0};
\colorlet{c}{natcomp!70};
\draw [c] (2.65168,0.988957) -- (2.65168,1.0492);
\draw [c] (2.65168,1.0492) -- (2.65168,1.10944);
\draw [c] (2.64355,1.0492) -- (2.65168,1.0492);
\draw [c] (2.65168,1.0492) -- (2.65982,1.0492);
\definecolor{c}{rgb}{0,0,0};
\colorlet{c}{natcomp!70};
\draw [c] (2.66795,1.1185) -- (2.66795,1.18712);
\draw [c] (2.66795,1.18712) -- (2.66795,1.25574);
\draw [c] (2.65982,1.18712) -- (2.66795,1.18712);
\draw [c] (2.66795,1.18712) -- (2.67609,1.18712);
\definecolor{c}{rgb}{0,0,0};
\colorlet{c}{natcomp!70};
\draw [c] (2.68423,1.11432) -- (2.68423,1.19207);
\draw [c] (2.68423,1.19207) -- (2.68423,1.26981);
\draw [c] (2.67609,1.19207) -- (2.68423,1.19207);
\draw [c] (2.68423,1.19207) -- (2.69236,1.19207);
\definecolor{c}{rgb}{0,0,0};
\colorlet{c}{natcomp!70};
\draw [c] (2.7005,1.21466) -- (2.7005,1.29003);
\draw [c] (2.7005,1.29003) -- (2.7005,1.3654);
\draw [c] (2.69236,1.29003) -- (2.7005,1.29003);
\draw [c] (2.7005,1.29003) -- (2.70864,1.29003);
\definecolor{c}{rgb}{0,0,0};
\colorlet{c}{natcomp!70};
\draw [c] (2.71677,1.10429) -- (2.71677,1.17543);
\draw [c] (2.71677,1.17543) -- (2.71677,1.24658);
\draw [c] (2.70864,1.17543) -- (2.71677,1.17543);
\draw [c] (2.71677,1.17543) -- (2.72491,1.17543);
\definecolor{c}{rgb}{0,0,0};
\colorlet{c}{natcomp!70};
\draw [c] (2.73305,1.12661) -- (2.73305,1.2009);
\draw [c] (2.73305,1.2009) -- (2.73305,1.27519);
\draw [c] (2.72491,1.2009) -- (2.73305,1.2009);
\draw [c] (2.73305,1.2009) -- (2.74118,1.2009);
\definecolor{c}{rgb}{0,0,0};
\colorlet{c}{natcomp!70};
\draw [c] (2.74932,1.06158) -- (2.74932,1.12477);
\draw [c] (2.74932,1.12477) -- (2.74932,1.18796);
\draw [c] (2.74118,1.12477) -- (2.74932,1.12477);
\draw [c] (2.74932,1.12477) -- (2.75745,1.12477);
\definecolor{c}{rgb}{0,0,0};
\colorlet{c}{natcomp!70};
\draw [c] (2.76559,1.18504) -- (2.76559,1.26123);
\draw [c] (2.76559,1.26123) -- (2.76559,1.33741);
\draw [c] (2.75745,1.26123) -- (2.76559,1.26123);
\draw [c] (2.76559,1.26123) -- (2.77373,1.26123);
\definecolor{c}{rgb}{0,0,0};
\colorlet{c}{natcomp!70};
\draw [c] (2.78186,1.23072) -- (2.78186,1.31228);
\draw [c] (2.78186,1.31228) -- (2.78186,1.39384);
\draw [c] (2.77373,1.31228) -- (2.78186,1.31228);
\draw [c] (2.78186,1.31228) -- (2.79,1.31228);
\definecolor{c}{rgb}{0,0,0};
\colorlet{c}{natcomp!70};
\draw [c] (2.79814,1.28934) -- (2.79814,1.37187);
\draw [c] (2.79814,1.37187) -- (2.79814,1.45439);
\draw [c] (2.79,1.37187) -- (2.79814,1.37187);
\draw [c] (2.79814,1.37187) -- (2.80627,1.37187);
\definecolor{c}{rgb}{0,0,0};
\colorlet{c}{natcomp!70};
\draw [c] (2.81441,1.32982) -- (2.81441,1.41625);
\draw [c] (2.81441,1.41625) -- (2.81441,1.50267);
\draw [c] (2.80627,1.41625) -- (2.81441,1.41625);
\draw [c] (2.81441,1.41625) -- (2.82255,1.41625);
\definecolor{c}{rgb}{0,0,0};
\colorlet{c}{natcomp!70};
\draw [c] (2.83068,1.30347) -- (2.83068,1.38681);
\draw [c] (2.83068,1.38681) -- (2.83068,1.47015);
\draw [c] (2.82255,1.38681) -- (2.83068,1.38681);
\draw [c] (2.83068,1.38681) -- (2.83882,1.38681);
\definecolor{c}{rgb}{0,0,0};
\colorlet{c}{natcomp!70};
\draw [c] (2.84695,1.23426) -- (2.84695,1.31216);
\draw [c] (2.84695,1.31216) -- (2.84695,1.39006);
\draw [c] (2.83882,1.31216) -- (2.84695,1.31216);
\draw [c] (2.84695,1.31216) -- (2.85509,1.31216);
\definecolor{c}{rgb}{0,0,0};
\colorlet{c}{natcomp!70};
\draw [c] (2.86323,1.36782) -- (2.86323,1.45635);
\draw [c] (2.86323,1.45635) -- (2.86323,1.54489);
\draw [c] (2.85509,1.45635) -- (2.86323,1.45635);
\draw [c] (2.86323,1.45635) -- (2.87136,1.45635);
\definecolor{c}{rgb}{0,0,0};
\colorlet{c}{natcomp!70};
\draw [c] (2.8795,1.31293) -- (2.8795,1.3965);
\draw [c] (2.8795,1.3965) -- (2.8795,1.48008);
\draw [c] (2.87136,1.3965) -- (2.8795,1.3965);
\draw [c] (2.8795,1.3965) -- (2.88764,1.3965);
\definecolor{c}{rgb}{0,0,0};
\colorlet{c}{natcomp!70};
\draw [c] (2.89577,1.42357) -- (2.89577,1.51525);
\draw [c] (2.89577,1.51525) -- (2.89577,1.60692);
\draw [c] (2.88764,1.51525) -- (2.89577,1.51525);
\draw [c] (2.89577,1.51525) -- (2.90391,1.51525);
\definecolor{c}{rgb}{0,0,0};
\colorlet{c}{natcomp!70};
\draw [c] (2.91205,1.43079) -- (2.91205,1.52154);
\draw [c] (2.91205,1.52154) -- (2.91205,1.61228);
\draw [c] (2.90391,1.52154) -- (2.91205,1.52154);
\draw [c] (2.91205,1.52154) -- (2.92018,1.52154);
\definecolor{c}{rgb}{0,0,0};
\colorlet{c}{natcomp!70};
\draw [c] (2.92832,1.34088) -- (2.92832,1.42855);
\draw [c] (2.92832,1.42855) -- (2.92832,1.51622);
\draw [c] (2.92018,1.42855) -- (2.92832,1.42855);
\draw [c] (2.92832,1.42855) -- (2.93645,1.42855);
\definecolor{c}{rgb}{0,0,0};
\colorlet{c}{natcomp!70};
\draw [c] (2.94459,1.41547) -- (2.94459,1.50915);
\draw [c] (2.94459,1.50915) -- (2.94459,1.60282);
\draw [c] (2.93645,1.50915) -- (2.94459,1.50915);
\draw [c] (2.94459,1.50915) -- (2.95273,1.50915);
\definecolor{c}{rgb}{0,0,0};
\colorlet{c}{natcomp!70};
\draw [c] (2.96086,1.58909) -- (2.96086,1.69352);
\draw [c] (2.96086,1.69352) -- (2.96086,1.79794);
\draw [c] (2.95273,1.69352) -- (2.96086,1.69352);
\draw [c] (2.96086,1.69352) -- (2.969,1.69352);
\definecolor{c}{rgb}{0,0,0};
\colorlet{c}{natcomp!70};
\draw [c] (2.97714,1.33395) -- (2.97714,1.41959);
\draw [c] (2.97714,1.41959) -- (2.97714,1.50522);
\draw [c] (2.969,1.41959) -- (2.97714,1.41959);
\draw [c] (2.97714,1.41959) -- (2.98527,1.41959);
\definecolor{c}{rgb}{0,0,0};
\colorlet{c}{natcomp!70};
\draw [c] (2.99341,1.59874) -- (2.99341,1.69978);
\draw [c] (2.99341,1.69978) -- (2.99341,1.80083);
\draw [c] (2.98527,1.69978) -- (2.99341,1.69978);
\draw [c] (2.99341,1.69978) -- (3.00155,1.69978);
\definecolor{c}{rgb}{0,0,0};
\colorlet{c}{natcomp!70};
\draw [c] (3.00968,1.52556) -- (3.00968,1.61936);
\draw [c] (3.00968,1.61936) -- (3.00968,1.71316);
\draw [c] (3.00155,1.61936) -- (3.00968,1.61936);
\draw [c] (3.00968,1.61936) -- (3.01782,1.61936);
\definecolor{c}{rgb}{0,0,0};
\colorlet{c}{natcomp!70};
\draw [c] (3.02595,1.6018) -- (3.02595,1.70136);
\draw [c] (3.02595,1.70136) -- (3.02595,1.80091);
\draw [c] (3.01782,1.70136) -- (3.02595,1.70136);
\draw [c] (3.02595,1.70136) -- (3.03409,1.70136);
\definecolor{c}{rgb}{0,0,0};
\colorlet{c}{natcomp!70};
\draw [c] (3.04223,1.58016) -- (3.04223,1.68072);
\draw [c] (3.04223,1.68072) -- (3.04223,1.78129);
\draw [c] (3.03409,1.68072) -- (3.04223,1.68072);
\draw [c] (3.04223,1.68072) -- (3.05036,1.68072);
\definecolor{c}{rgb}{0,0,0};
\colorlet{c}{natcomp!70};
\draw [c] (3.0585,1.72343) -- (3.0585,1.82892);
\draw [c] (3.0585,1.82892) -- (3.0585,1.93442);
\draw [c] (3.05036,1.82892) -- (3.0585,1.82892);
\draw [c] (3.0585,1.82892) -- (3.06664,1.82892);
\definecolor{c}{rgb}{0,0,0};
\colorlet{c}{natcomp!70};
\draw [c] (3.07477,1.80323) -- (3.07477,1.92123);
\draw [c] (3.07477,1.92123) -- (3.07477,2.03923);
\draw [c] (3.06664,1.92123) -- (3.07477,1.92123);
\draw [c] (3.07477,1.92123) -- (3.08291,1.92123);
\definecolor{c}{rgb}{0,0,0};
\colorlet{c}{natcomp!70};
\draw [c] (3.09105,1.72774) -- (3.09105,1.8397);
\draw [c] (3.09105,1.8397) -- (3.09105,1.95166);
\draw [c] (3.08291,1.8397) -- (3.09105,1.8397);
\draw [c] (3.09105,1.8397) -- (3.09918,1.8397);
\definecolor{c}{rgb}{0,0,0};
\colorlet{c}{natcomp!70};
\draw [c] (3.10732,1.5984) -- (3.10732,1.70303);
\draw [c] (3.10732,1.70303) -- (3.10732,1.80766);
\draw [c] (3.09918,1.70303) -- (3.10732,1.70303);
\draw [c] (3.10732,1.70303) -- (3.11545,1.70303);
\definecolor{c}{rgb}{0,0,0};
\colorlet{c}{natcomp!70};
\draw [c] (3.12359,1.72127) -- (3.12359,1.83007);
\draw [c] (3.12359,1.83007) -- (3.12359,1.93888);
\draw [c] (3.11545,1.83007) -- (3.12359,1.83007);
\draw [c] (3.12359,1.83007) -- (3.13173,1.83007);
\definecolor{c}{rgb}{0,0,0};
\colorlet{c}{natcomp!70};
\draw [c] (3.13986,1.64464) -- (3.13986,1.74758);
\draw [c] (3.13986,1.74758) -- (3.13986,1.85053);
\draw [c] (3.13173,1.74758) -- (3.13986,1.74758);
\draw [c] (3.13986,1.74758) -- (3.148,1.74758);
\definecolor{c}{rgb}{0,0,0};
\colorlet{c}{natcomp!70};
\draw [c] (3.15614,1.69808) -- (3.15614,1.80701);
\draw [c] (3.15614,1.80701) -- (3.15614,1.91594);
\draw [c] (3.148,1.80701) -- (3.15614,1.80701);
\draw [c] (3.15614,1.80701) -- (3.16427,1.80701);
\definecolor{c}{rgb}{0,0,0};
\colorlet{c}{natcomp!70};
\draw [c] (3.17241,1.85831) -- (3.17241,1.98447);
\draw [c] (3.17241,1.98447) -- (3.17241,2.11062);
\draw [c] (3.16427,1.98447) -- (3.17241,1.98447);
\draw [c] (3.17241,1.98447) -- (3.18055,1.98447);
\definecolor{c}{rgb}{0,0,0};
\colorlet{c}{natcomp!70};
\draw [c] (3.18868,1.75745) -- (3.18868,1.87281);
\draw [c] (3.18868,1.87281) -- (3.18868,1.98818);
\draw [c] (3.18055,1.87281) -- (3.18868,1.87281);
\draw [c] (3.18868,1.87281) -- (3.19682,1.87281);
\definecolor{c}{rgb}{0,0,0};
\colorlet{c}{natcomp!70};
\draw [c] (3.20495,1.88058) -- (3.20495,2.00281);
\draw [c] (3.20495,2.00281) -- (3.20495,2.12505);
\draw [c] (3.19682,2.00281) -- (3.20495,2.00281);
\draw [c] (3.20495,2.00281) -- (3.21309,2.00281);
\definecolor{c}{rgb}{0,0,0};
\colorlet{c}{natcomp!70};
\draw [c] (3.22123,1.80611) -- (3.22123,1.92753);
\draw [c] (3.22123,1.92753) -- (3.22123,2.04895);
\draw [c] (3.21309,1.92753) -- (3.22123,1.92753);
\draw [c] (3.22123,1.92753) -- (3.22936,1.92753);
\definecolor{c}{rgb}{0,0,0};
\colorlet{c}{natcomp!70};
\draw [c] (3.2375,1.91345) -- (3.2375,2.03616);
\draw [c] (3.2375,2.03616) -- (3.2375,2.15888);
\draw [c] (3.22936,2.03616) -- (3.2375,2.03616);
\draw [c] (3.2375,2.03616) -- (3.24564,2.03616);
\definecolor{c}{rgb}{0,0,0};
\colorlet{c}{natcomp!70};
\draw [c] (3.25377,1.96375) -- (3.25377,2.09043);
\draw [c] (3.25377,2.09043) -- (3.25377,2.21712);
\draw [c] (3.24564,2.09043) -- (3.25377,2.09043);
\draw [c] (3.25377,2.09043) -- (3.26191,2.09043);
\definecolor{c}{rgb}{0,0,0};
\colorlet{c}{natcomp!70};
\draw [c] (3.27005,1.92288) -- (3.27005,2.04725);
\draw [c] (3.27005,2.04725) -- (3.27005,2.17162);
\draw [c] (3.26191,2.04725) -- (3.27005,2.04725);
\draw [c] (3.27005,2.04725) -- (3.27818,2.04725);
\definecolor{c}{rgb}{0,0,0};
\colorlet{c}{natcomp!70};
\draw [c] (3.28632,1.84932) -- (3.28632,1.97407);
\draw [c] (3.28632,1.97407) -- (3.28632,2.09882);
\draw [c] (3.27818,1.97407) -- (3.28632,1.97407);
\draw [c] (3.28632,1.97407) -- (3.29445,1.97407);
\definecolor{c}{rgb}{0,0,0};
\colorlet{c}{natcomp!70};
\draw [c] (3.30259,1.90077) -- (3.30259,2.02007);
\draw [c] (3.30259,2.02007) -- (3.30259,2.13937);
\draw [c] (3.29445,2.02007) -- (3.30259,2.02007);
\draw [c] (3.30259,2.02007) -- (3.31073,2.02007);
\definecolor{c}{rgb}{0,0,0};
\colorlet{c}{natcomp!70};
\draw [c] (3.31886,1.83797) -- (3.31886,1.96045);
\draw [c] (3.31886,1.96045) -- (3.31886,2.08293);
\draw [c] (3.31073,1.96045) -- (3.31886,1.96045);
\draw [c] (3.31886,1.96045) -- (3.327,1.96045);
\definecolor{c}{rgb}{0,0,0};
\colorlet{c}{natcomp!70};
\draw [c] (3.33514,1.88986) -- (3.33514,2.01298);
\draw [c] (3.33514,2.01298) -- (3.33514,2.1361);
\draw [c] (3.327,2.01298) -- (3.33514,2.01298);
\draw [c] (3.33514,2.01298) -- (3.34327,2.01298);
\definecolor{c}{rgb}{0,0,0};
\colorlet{c}{natcomp!70};
\draw [c] (3.35141,1.97995) -- (3.35141,2.10928);
\draw [c] (3.35141,2.10928) -- (3.35141,2.2386);
\draw [c] (3.34327,2.10928) -- (3.35141,2.10928);
\draw [c] (3.35141,2.10928) -- (3.35955,2.10928);
\definecolor{c}{rgb}{0,0,0};
\colorlet{c}{natcomp!70};
\draw [c] (3.36768,1.69146) -- (3.36768,1.80413);
\draw [c] (3.36768,1.80413) -- (3.36768,1.9168);
\draw [c] (3.35955,1.80413) -- (3.36768,1.80413);
\draw [c] (3.36768,1.80413) -- (3.37582,1.80413);
\definecolor{c}{rgb}{0,0,0};
\colorlet{c}{natcomp!70};
\draw [c] (3.38395,1.76713) -- (3.38395,1.8867);
\draw [c] (3.38395,1.8867) -- (3.38395,2.00627);
\draw [c] (3.37582,1.8867) -- (3.38395,1.8867);
\draw [c] (3.38395,1.8867) -- (3.39209,1.8867);
\definecolor{c}{rgb}{0,0,0};
\colorlet{c}{natcomp!70};
\draw [c] (3.40023,1.73703) -- (3.40023,1.85405);
\draw [c] (3.40023,1.85405) -- (3.40023,1.97106);
\draw [c] (3.39209,1.85405) -- (3.40023,1.85405);
\draw [c] (3.40023,1.85405) -- (3.40836,1.85405);
\definecolor{c}{rgb}{0,0,0};
\colorlet{c}{natcomp!70};
\draw [c] (3.4165,1.75427) -- (3.4165,1.87418);
\draw [c] (3.4165,1.87418) -- (3.4165,1.99409);
\draw [c] (3.40836,1.87418) -- (3.4165,1.87418);
\draw [c] (3.4165,1.87418) -- (3.42464,1.87418);
\definecolor{c}{rgb}{0,0,0};
\colorlet{c}{natcomp!70};
\draw [c] (3.43277,1.59071) -- (3.43277,1.70168);
\draw [c] (3.43277,1.70168) -- (3.43277,1.81264);
\draw [c] (3.42464,1.70168) -- (3.43277,1.70168);
\draw [c] (3.43277,1.70168) -- (3.44091,1.70168);
\definecolor{c}{rgb}{0,0,0};
\colorlet{c}{natcomp!70};
\draw [c] (3.44905,2.0116) -- (3.44905,2.1488);
\draw [c] (3.44905,2.1488) -- (3.44905,2.286);
\draw [c] (3.44091,2.1488) -- (3.44905,2.1488);
\draw [c] (3.44905,2.1488) -- (3.45718,2.1488);
\definecolor{c}{rgb}{0,0,0};
\colorlet{c}{natcomp!70};
\draw [c] (3.46532,1.58379) -- (3.46532,1.69731);
\draw [c] (3.46532,1.69731) -- (3.46532,1.81082);
\draw [c] (3.45718,1.69731) -- (3.46532,1.69731);
\draw [c] (3.46532,1.69731) -- (3.47345,1.69731);
\definecolor{c}{rgb}{0,0,0};
\colorlet{c}{natcomp!70};
\draw [c] (3.48159,1.69179) -- (3.48159,1.8072);
\draw [c] (3.48159,1.8072) -- (3.48159,1.92261);
\draw [c] (3.47345,1.8072) -- (3.48159,1.8072);
\draw [c] (3.48159,1.8072) -- (3.48973,1.8072);
\definecolor{c}{rgb}{0,0,0};
\colorlet{c}{natcomp!70};
\draw [c] (3.49786,1.66795) -- (3.49786,1.7778);
\draw [c] (3.49786,1.7778) -- (3.49786,1.88766);
\draw [c] (3.48973,1.7778) -- (3.49786,1.7778);
\draw [c] (3.49786,1.7778) -- (3.506,1.7778);
\definecolor{c}{rgb}{0,0,0};
\colorlet{c}{natcomp!70};
\draw [c] (3.51414,1.54378) -- (3.51414,1.65414);
\draw [c] (3.51414,1.65414) -- (3.51414,1.76451);
\draw [c] (3.506,1.65414) -- (3.51414,1.65414);
\draw [c] (3.51414,1.65414) -- (3.52227,1.65414);
\definecolor{c}{rgb}{0,0,0};
\colorlet{c}{natcomp!70};
\draw [c] (3.53041,1.63429) -- (3.53041,1.75469);
\draw [c] (3.53041,1.75469) -- (3.53041,1.87509);
\draw [c] (3.52227,1.75469) -- (3.53041,1.75469);
\draw [c] (3.53041,1.75469) -- (3.53855,1.75469);
\definecolor{c}{rgb}{0,0,0};
\colorlet{c}{natcomp!70};
\draw [c] (3.54668,1.49617) -- (3.54668,1.60081);
\draw [c] (3.54668,1.60081) -- (3.54668,1.70545);
\draw [c] (3.53855,1.60081) -- (3.54668,1.60081);
\draw [c] (3.54668,1.60081) -- (3.55482,1.60081);
\definecolor{c}{rgb}{0,0,0};
\colorlet{c}{natcomp!70};
\draw [c] (3.56295,1.58314) -- (3.56295,1.69229);
\draw [c] (3.56295,1.69229) -- (3.56295,1.80144);
\draw [c] (3.55482,1.69229) -- (3.56295,1.69229);
\draw [c] (3.56295,1.69229) -- (3.57109,1.69229);
\definecolor{c}{rgb}{0,0,0};
\colorlet{c}{natcomp!70};
\draw [c] (3.57923,1.46417) -- (3.57923,1.56462);
\draw [c] (3.57923,1.56462) -- (3.57923,1.66507);
\draw [c] (3.57109,1.56462) -- (3.57923,1.56462);
\draw [c] (3.57923,1.56462) -- (3.58736,1.56462);
\definecolor{c}{rgb}{0,0,0};
\colorlet{c}{natcomp!70};
\draw [c] (3.5955,1.7506) -- (3.5955,1.87231);
\draw [c] (3.5955,1.87231) -- (3.5955,1.99402);
\draw [c] (3.58736,1.87231) -- (3.5955,1.87231);
\draw [c] (3.5955,1.87231) -- (3.60364,1.87231);
\definecolor{c}{rgb}{0,0,0};
\colorlet{c}{natcomp!70};
\draw [c] (3.61177,1.53631) -- (3.61177,1.63577);
\draw [c] (3.61177,1.63577) -- (3.61177,1.73524);
\draw [c] (3.60364,1.63577) -- (3.61177,1.63577);
\draw [c] (3.61177,1.63577) -- (3.61991,1.63577);
\definecolor{c}{rgb}{0,0,0};
\colorlet{c}{natcomp!70};
\draw [c] (3.62805,1.60831) -- (3.62805,1.71397);
\draw [c] (3.62805,1.71397) -- (3.62805,1.81962);
\draw [c] (3.61991,1.71397) -- (3.62805,1.71397);
\draw [c] (3.62805,1.71397) -- (3.63618,1.71397);
\definecolor{c}{rgb}{0,0,0};
\colorlet{c}{natcomp!70};
\draw [c] (3.64432,1.41401) -- (3.64432,1.5139);
\draw [c] (3.64432,1.5139) -- (3.64432,1.61378);
\draw [c] (3.63618,1.5139) -- (3.64432,1.5139);
\draw [c] (3.64432,1.5139) -- (3.65245,1.5139);
\definecolor{c}{rgb}{0,0,0};
\colorlet{c}{natcomp!70};
\draw [c] (3.66059,1.47867) -- (3.66059,1.57869);
\draw [c] (3.66059,1.57869) -- (3.66059,1.67871);
\draw [c] (3.65245,1.57869) -- (3.66059,1.57869);
\draw [c] (3.66059,1.57869) -- (3.66873,1.57869);
\definecolor{c}{rgb}{0,0,0};
\colorlet{c}{natcomp!70};
\draw [c] (3.67686,1.43725) -- (3.67686,1.53155);
\draw [c] (3.67686,1.53155) -- (3.67686,1.62584);
\draw [c] (3.66873,1.53155) -- (3.67686,1.53155);
\draw [c] (3.67686,1.53155) -- (3.685,1.53155);
\definecolor{c}{rgb}{0,0,0};
\colorlet{c}{natcomp!70};
\draw [c] (3.69314,1.44595) -- (3.69314,1.54196);
\draw [c] (3.69314,1.54196) -- (3.69314,1.63798);
\draw [c] (3.685,1.54196) -- (3.69314,1.54196);
\draw [c] (3.69314,1.54196) -- (3.70127,1.54196);
\definecolor{c}{rgb}{0,0,0};
\colorlet{c}{natcomp!70};
\draw [c] (3.70941,1.31331) -- (3.70941,1.4008);
\draw [c] (3.70941,1.4008) -- (3.70941,1.48828);
\draw [c] (3.70127,1.4008) -- (3.70941,1.4008);
\draw [c] (3.70941,1.4008) -- (3.71755,1.4008);
\definecolor{c}{rgb}{0,0,0};
\colorlet{c}{natcomp!70};
\draw [c] (3.72568,1.38718) -- (3.72568,1.47882);
\draw [c] (3.72568,1.47882) -- (3.72568,1.57046);
\draw [c] (3.71755,1.47882) -- (3.72568,1.47882);
\draw [c] (3.72568,1.47882) -- (3.73382,1.47882);
\definecolor{c}{rgb}{0,0,0};
\colorlet{c}{natcomp!70};
\draw [c] (3.74195,1.30543) -- (3.74195,1.39264);
\draw [c] (3.74195,1.39264) -- (3.74195,1.47984);
\draw [c] (3.73382,1.39264) -- (3.74195,1.39264);
\draw [c] (3.74195,1.39264) -- (3.75009,1.39264);
\definecolor{c}{rgb}{0,0,0};
\colorlet{c}{natcomp!70};
\draw [c] (3.75823,1.33609) -- (3.75823,1.42593);
\draw [c] (3.75823,1.42593) -- (3.75823,1.51577);
\draw [c] (3.75009,1.42593) -- (3.75823,1.42593);
\draw [c] (3.75823,1.42593) -- (3.76636,1.42593);
\definecolor{c}{rgb}{0,0,0};
\colorlet{c}{natcomp!70};
\draw [c] (3.7745,1.48113) -- (3.7745,1.57802);
\draw [c] (3.7745,1.57802) -- (3.7745,1.67491);
\draw [c] (3.76636,1.57802) -- (3.7745,1.57802);
\draw [c] (3.7745,1.57802) -- (3.78264,1.57802);
\definecolor{c}{rgb}{0,0,0};
\colorlet{c}{natcomp!70};
\draw [c] (3.79077,1.36985) -- (3.79077,1.4665);
\draw [c] (3.79077,1.4665) -- (3.79077,1.56316);
\draw [c] (3.78264,1.4665) -- (3.79077,1.4665);
\draw [c] (3.79077,1.4665) -- (3.79891,1.4665);
\definecolor{c}{rgb}{0,0,0};
\colorlet{c}{natcomp!70};
\draw [c] (3.80705,1.3613) -- (3.80705,1.45155);
\draw [c] (3.80705,1.45155) -- (3.80705,1.5418);
\draw [c] (3.79891,1.45155) -- (3.80705,1.45155);
\draw [c] (3.80705,1.45155) -- (3.81518,1.45155);
\definecolor{c}{rgb}{0,0,0};
\colorlet{c}{natcomp!70};
\draw [c] (3.82332,1.34548) -- (3.82332,1.43349);
\draw [c] (3.82332,1.43349) -- (3.82332,1.5215);
\draw [c] (3.81518,1.43349) -- (3.82332,1.43349);
\draw [c] (3.82332,1.43349) -- (3.83145,1.43349);
\definecolor{c}{rgb}{0,0,0};
\colorlet{c}{natcomp!70};
\draw [c] (3.83959,1.34648) -- (3.83959,1.44227);
\draw [c] (3.83959,1.44227) -- (3.83959,1.53807);
\draw [c] (3.83145,1.44227) -- (3.83959,1.44227);
\draw [c] (3.83959,1.44227) -- (3.84773,1.44227);
\definecolor{c}{rgb}{0,0,0};
\colorlet{c}{natcomp!70};
\draw [c] (3.85586,1.26603) -- (3.85586,1.34858);
\draw [c] (3.85586,1.34858) -- (3.85586,1.43114);
\draw [c] (3.84773,1.34858) -- (3.85586,1.34858);
\draw [c] (3.85586,1.34858) -- (3.864,1.34858);
\definecolor{c}{rgb}{0,0,0};
\colorlet{c}{natcomp!70};
\draw [c] (3.87214,1.19945) -- (3.87214,1.28167);
\draw [c] (3.87214,1.28167) -- (3.87214,1.3639);
\draw [c] (3.864,1.28167) -- (3.87214,1.28167);
\draw [c] (3.87214,1.28167) -- (3.88027,1.28167);
\definecolor{c}{rgb}{0,0,0};
\colorlet{c}{natcomp!70};
\draw [c] (3.88841,1.37139) -- (3.88841,1.46647);
\draw [c] (3.88841,1.46647) -- (3.88841,1.56155);
\draw [c] (3.88027,1.46647) -- (3.88841,1.46647);
\draw [c] (3.88841,1.46647) -- (3.89655,1.46647);
\definecolor{c}{rgb}{0,0,0};
\colorlet{c}{natcomp!70};
\draw [c] (3.90468,1.0781) -- (3.90468,1.14498);
\draw [c] (3.90468,1.14498) -- (3.90468,1.21187);
\draw [c] (3.89655,1.14498) -- (3.90468,1.14498);
\draw [c] (3.90468,1.14498) -- (3.91282,1.14498);
\definecolor{c}{rgb}{0,0,0};
\colorlet{c}{natcomp!70};
\draw [c] (3.92095,1.12409) -- (3.92095,1.19694);
\draw [c] (3.92095,1.19694) -- (3.92095,1.26979);
\draw [c] (3.91282,1.19694) -- (3.92095,1.19694);
\draw [c] (3.92095,1.19694) -- (3.92909,1.19694);
\definecolor{c}{rgb}{0,0,0};
\colorlet{c}{natcomp!70};
\draw [c] (3.93723,1.21279) -- (3.93723,1.29838);
\draw [c] (3.93723,1.29838) -- (3.93723,1.38397);
\draw [c] (3.92909,1.29838) -- (3.93723,1.29838);
\draw [c] (3.93723,1.29838) -- (3.94536,1.29838);
\definecolor{c}{rgb}{0,0,0};
\colorlet{c}{natcomp!70};
\draw [c] (3.9535,1.18838) -- (3.9535,1.26484);
\draw [c] (3.9535,1.26484) -- (3.9535,1.3413);
\draw [c] (3.94536,1.26484) -- (3.9535,1.26484);
\draw [c] (3.9535,1.26484) -- (3.96164,1.26484);
\definecolor{c}{rgb}{0,0,0};
\colorlet{c}{natcomp!70};
\draw [c] (3.96977,1.13244) -- (3.96977,1.20537);
\draw [c] (3.96977,1.20537) -- (3.96977,1.2783);
\draw [c] (3.96164,1.20537) -- (3.96977,1.20537);
\draw [c] (3.96977,1.20537) -- (3.97791,1.20537);
\definecolor{c}{rgb}{0,0,0};
\colorlet{c}{natcomp!70};
\draw [c] (3.98605,1.18644) -- (3.98605,1.26695);
\draw [c] (3.98605,1.26695) -- (3.98605,1.34746);
\draw [c] (3.97791,1.26695) -- (3.98605,1.26695);
\draw [c] (3.98605,1.26695) -- (3.99418,1.26695);
\definecolor{c}{rgb}{0,0,0};
\colorlet{c}{natcomp!70};
\draw [c] (4.00232,1.18905) -- (4.00232,1.27326);
\draw [c] (4.00232,1.27326) -- (4.00232,1.35748);
\draw [c] (3.99418,1.27326) -- (4.00232,1.27326);
\draw [c] (4.00232,1.27326) -- (4.01045,1.27326);
\definecolor{c}{rgb}{0,0,0};
\colorlet{c}{natcomp!70};
\draw [c] (4.01859,1.21515) -- (4.01859,1.29654);
\draw [c] (4.01859,1.29654) -- (4.01859,1.37792);
\draw [c] (4.01045,1.29654) -- (4.01859,1.29654);
\draw [c] (4.01859,1.29654) -- (4.02673,1.29654);
\definecolor{c}{rgb}{0,0,0};
\colorlet{c}{natcomp!70};
\draw [c] (4.03486,1.0567) -- (4.03486,1.12126);
\draw [c] (4.03486,1.12126) -- (4.03486,1.18582);
\draw [c] (4.02673,1.12126) -- (4.03486,1.12126);
\draw [c] (4.03486,1.12126) -- (4.043,1.12126);
\definecolor{c}{rgb}{0,0,0};
\colorlet{c}{natcomp!70};
\draw [c] (4.05114,1.0637) -- (4.05114,1.13565);
\draw [c] (4.05114,1.13565) -- (4.05114,1.2076);
\draw [c] (4.043,1.13565) -- (4.05114,1.13565);
\draw [c] (4.05114,1.13565) -- (4.05927,1.13565);
\definecolor{c}{rgb}{0,0,0};
\colorlet{c}{natcomp!70};
\draw [c] (4.06741,1.12543) -- (4.06741,1.19947);
\draw [c] (4.06741,1.19947) -- (4.06741,1.27351);
\draw [c] (4.05927,1.19947) -- (4.06741,1.19947);
\draw [c] (4.06741,1.19947) -- (4.07555,1.19947);
\definecolor{c}{rgb}{0,0,0};
\colorlet{c}{natcomp!70};
\draw [c] (4.08368,1.11089) -- (4.08368,1.18647);
\draw [c] (4.08368,1.18647) -- (4.08368,1.26206);
\draw [c] (4.07555,1.18647) -- (4.08368,1.18647);
\draw [c] (4.08368,1.18647) -- (4.09182,1.18647);
\definecolor{c}{rgb}{0,0,0};
\colorlet{c}{natcomp!70};
\draw [c] (4.09995,1.09755) -- (4.09995,1.16945);
\draw [c] (4.09995,1.16945) -- (4.09995,1.24135);
\draw [c] (4.09182,1.16945) -- (4.09995,1.16945);
\draw [c] (4.09995,1.16945) -- (4.10809,1.16945);
\definecolor{c}{rgb}{0,0,0};
\colorlet{c}{natcomp!70};
\draw [c] (4.11623,1.07487) -- (4.11623,1.14657);
\draw [c] (4.11623,1.14657) -- (4.11623,1.21826);
\draw [c] (4.10809,1.14657) -- (4.11623,1.14657);
\draw [c] (4.11623,1.14657) -- (4.12436,1.14657);
\definecolor{c}{rgb}{0,0,0};
\colorlet{c}{natcomp!70};
\draw [c] (4.1325,1.03624) -- (4.1325,1.10277);
\draw [c] (4.1325,1.10277) -- (4.1325,1.16931);
\draw [c] (4.12436,1.10277) -- (4.1325,1.10277);
\draw [c] (4.1325,1.10277) -- (4.14064,1.10277);
\definecolor{c}{rgb}{0,0,0};
\colorlet{c}{natcomp!70};
\draw [c] (4.14877,0.952458) -- (4.14877,1.00761);
\draw [c] (4.14877,1.00761) -- (4.14877,1.06276);
\draw [c] (4.14064,1.00761) -- (4.14877,1.00761);
\draw [c] (4.14877,1.00761) -- (4.15691,1.00761);
\definecolor{c}{rgb}{0,0,0};
\colorlet{c}{natcomp!70};
\draw [c] (4.16505,1.04998) -- (4.16505,1.12012);
\draw [c] (4.16505,1.12012) -- (4.16505,1.19027);
\draw [c] (4.15691,1.12012) -- (4.16505,1.12012);
\draw [c] (4.16505,1.12012) -- (4.17318,1.12012);
\definecolor{c}{rgb}{0,0,0};
\colorlet{c}{natcomp!70};
\draw [c] (4.18132,1.08985) -- (4.18132,1.16335);
\draw [c] (4.18132,1.16335) -- (4.18132,1.23686);
\draw [c] (4.17318,1.16335) -- (4.18132,1.16335);
\draw [c] (4.18132,1.16335) -- (4.18945,1.16335);
\definecolor{c}{rgb}{0,0,0};
\colorlet{c}{natcomp!70};
\draw [c] (4.19759,1.06558) -- (4.19759,1.13256);
\draw [c] (4.19759,1.13256) -- (4.19759,1.19953);
\draw [c] (4.18945,1.13256) -- (4.19759,1.13256);
\draw [c] (4.19759,1.13256) -- (4.20573,1.13256);
\definecolor{c}{rgb}{0,0,0};
\colorlet{c}{natcomp!70};
\draw [c] (4.21386,0.99441) -- (4.21386,1.05948);
\draw [c] (4.21386,1.05948) -- (4.21386,1.12455);
\draw [c] (4.20573,1.05948) -- (4.21386,1.05948);
\draw [c] (4.21386,1.05948) -- (4.222,1.05948);
\definecolor{c}{rgb}{0,0,0};
\colorlet{c}{natcomp!70};
\draw [c] (4.23014,1.15782) -- (4.23014,1.23917);
\draw [c] (4.23014,1.23917) -- (4.23014,1.32052);
\draw [c] (4.222,1.23917) -- (4.23014,1.23917);
\draw [c] (4.23014,1.23917) -- (4.23827,1.23917);
\definecolor{c}{rgb}{0,0,0};
\colorlet{c}{natcomp!70};
\draw [c] (4.24641,0.983316) -- (4.24641,1.04763);
\draw [c] (4.24641,1.04763) -- (4.24641,1.11194);
\draw [c] (4.23827,1.04763) -- (4.24641,1.04763);
\draw [c] (4.24641,1.04763) -- (4.25455,1.04763);
\definecolor{c}{rgb}{0,0,0};
\colorlet{c}{natcomp!70};
\draw [c] (4.26268,1.03462) -- (4.26268,1.10754);
\draw [c] (4.26268,1.10754) -- (4.26268,1.18046);
\draw [c] (4.25455,1.10754) -- (4.26268,1.10754);
\draw [c] (4.26268,1.10754) -- (4.27082,1.10754);
\definecolor{c}{rgb}{0,0,0};
\colorlet{c}{natcomp!70};
\draw [c] (4.27895,0.980957) -- (4.27895,1.04168);
\draw [c] (4.27895,1.04168) -- (4.27895,1.1024);
\draw [c] (4.27082,1.04168) -- (4.27895,1.04168);
\draw [c] (4.27895,1.04168) -- (4.28709,1.04168);
\definecolor{c}{rgb}{0,0,0};
\colorlet{c}{natcomp!70};
\draw [c] (4.29523,0.98657) -- (4.29523,1.04335);
\draw [c] (4.29523,1.04335) -- (4.29523,1.10013);
\draw [c] (4.28709,1.04335) -- (4.29523,1.04335);
\draw [c] (4.29523,1.04335) -- (4.30336,1.04335);
\definecolor{c}{rgb}{0,0,0};
\colorlet{c}{natcomp!70};
\draw [c] (4.3115,0.987687) -- (4.3115,1.0472);
\draw [c] (4.3115,1.0472) -- (4.3115,1.10672);
\draw [c] (4.30336,1.0472) -- (4.3115,1.0472);
\draw [c] (4.3115,1.0472) -- (4.31964,1.0472);
\definecolor{c}{rgb}{0,0,0};
\colorlet{c}{natcomp!70};
\draw [c] (4.32777,1.01895) -- (4.32777,1.08324);
\draw [c] (4.32777,1.08324) -- (4.32777,1.14754);
\draw [c] (4.31964,1.08324) -- (4.32777,1.08324);
\draw [c] (4.32777,1.08324) -- (4.33591,1.08324);
\definecolor{c}{rgb}{0,0,0};
\colorlet{c}{natcomp!70};
\draw [c] (4.34405,0.968473) -- (4.34405,1.02884);
\draw [c] (4.34405,1.02884) -- (4.34405,1.08921);
\draw [c] (4.33591,1.02884) -- (4.34405,1.02884);
\draw [c] (4.34405,1.02884) -- (4.35218,1.02884);
\definecolor{c}{rgb}{0,0,0};
\colorlet{c}{natcomp!70};
\draw [c] (4.36032,0.974112) -- (4.36032,1.03613);
\draw [c] (4.36032,1.03613) -- (4.36032,1.09814);
\draw [c] (4.35218,1.03613) -- (4.36032,1.03613);
\draw [c] (4.36032,1.03613) -- (4.36845,1.03613);
\definecolor{c}{rgb}{0,0,0};
\colorlet{c}{natcomp!70};
\draw [c] (4.37659,0.980809) -- (4.37659,1.04153);
\draw [c] (4.37659,1.04153) -- (4.37659,1.10224);
\draw [c] (4.36845,1.04153) -- (4.37659,1.04153);
\draw [c] (4.37659,1.04153) -- (4.38473,1.04153);
\definecolor{c}{rgb}{0,0,0};
\colorlet{c}{natcomp!70};
\draw [c] (4.39286,0.971029) -- (4.39286,1.02981);
\draw [c] (4.39286,1.02981) -- (4.39286,1.08859);
\draw [c] (4.38473,1.02981) -- (4.39286,1.02981);
\draw [c] (4.39286,1.02981) -- (4.401,1.02981);
\definecolor{c}{rgb}{0,0,0};
\colorlet{c}{natcomp!70};
\draw [c] (4.40914,0.95256) -- (4.40914,1.00793);
\draw [c] (4.40914,1.00793) -- (4.40914,1.06331);
\draw [c] (4.401,1.00793) -- (4.40914,1.00793);
\draw [c] (4.40914,1.00793) -- (4.41727,1.00793);
\definecolor{c}{rgb}{0,0,0};
\colorlet{c}{natcomp!70};
\draw [c] (4.42541,1.04433) -- (4.42541,1.1097);
\draw [c] (4.42541,1.1097) -- (4.42541,1.17507);
\draw [c] (4.41727,1.1097) -- (4.42541,1.1097);
\draw [c] (4.42541,1.1097) -- (4.43355,1.1097);
\definecolor{c}{rgb}{0,0,0};
\colorlet{c}{natcomp!70};
\draw [c] (4.44168,0.921556) -- (4.44168,0.976194);
\draw [c] (4.44168,0.976194) -- (4.44168,1.03083);
\draw [c] (4.43355,0.976194) -- (4.44168,0.976194);
\draw [c] (4.44168,0.976194) -- (4.44982,0.976194);
\definecolor{c}{rgb}{0,0,0};
\colorlet{c}{natcomp!70};
\draw [c] (4.45795,0.972593) -- (4.45795,1.04013);
\draw [c] (4.45795,1.04013) -- (4.45795,1.10767);
\draw [c] (4.44982,1.04013) -- (4.45795,1.04013);
\draw [c] (4.45795,1.04013) -- (4.46609,1.04013);
\definecolor{c}{rgb}{0,0,0};
\colorlet{c}{natcomp!70};
\draw [c] (4.47423,0.946413) -- (4.47423,0.999826);
\draw [c] (4.47423,0.999826) -- (4.47423,1.05324);
\draw [c] (4.46609,0.999826) -- (4.47423,0.999826);
\draw [c] (4.47423,0.999826) -- (4.48236,0.999826);
\definecolor{c}{rgb}{0,0,0};
\colorlet{c}{natcomp!70};
\draw [c] (4.4905,0.928871) -- (4.4905,0.983045);
\draw [c] (4.4905,0.983045) -- (4.4905,1.03722);
\draw [c] (4.48236,0.983045) -- (4.4905,0.983045);
\draw [c] (4.4905,0.983045) -- (4.49864,0.983045);
\definecolor{c}{rgb}{0,0,0};
\colorlet{c}{natcomp!70};
\draw [c] (4.50677,0.92616) -- (4.50677,0.979906);
\draw [c] (4.50677,0.979906) -- (4.50677,1.03365);
\draw [c] (4.49864,0.979906) -- (4.50677,0.979906);
\draw [c] (4.50677,0.979906) -- (4.51491,0.979906);
\definecolor{c}{rgb}{0,0,0};
\colorlet{c}{natcomp!70};
\draw [c] (4.52305,0.866944) -- (4.52305,0.914198);
\draw [c] (4.52305,0.914198) -- (4.52305,0.961452);
\draw [c] (4.51491,0.914198) -- (4.52305,0.914198);
\draw [c] (4.52305,0.914198) -- (4.53118,0.914198);
\definecolor{c}{rgb}{0,0,0};
\colorlet{c}{natcomp!70};
\draw [c] (4.53932,0.956535) -- (4.53932,1.01368);
\draw [c] (4.53932,1.01368) -- (4.53932,1.07083);
\draw [c] (4.53118,1.01368) -- (4.53932,1.01368);
\draw [c] (4.53932,1.01368) -- (4.54745,1.01368);
\definecolor{c}{rgb}{0,0,0};
\colorlet{c}{natcomp!70};
\draw [c] (4.55559,1.00276) -- (4.55559,1.06462);
\draw [c] (4.55559,1.06462) -- (4.55559,1.12648);
\draw [c] (4.54745,1.06462) -- (4.55559,1.06462);
\draw [c] (4.55559,1.06462) -- (4.56373,1.06462);
\definecolor{c}{rgb}{0,0,0};
\colorlet{c}{natcomp!70};
\draw [c] (4.57186,0.919644) -- (4.57186,0.985824);
\draw [c] (4.57186,0.985824) -- (4.57186,1.052);
\draw [c] (4.56373,0.985824) -- (4.57186,0.985824);
\draw [c] (4.57186,0.985824) -- (4.58,0.985824);
\definecolor{c}{rgb}{0,0,0};
\colorlet{c}{natcomp!70};
\draw [c] (4.58814,0.882513) -- (4.58814,0.934061);
\draw [c] (4.58814,0.934061) -- (4.58814,0.985609);
\draw [c] (4.58,0.934061) -- (4.58814,0.934061);
\draw [c] (4.58814,0.934061) -- (4.59627,0.934061);
\definecolor{c}{rgb}{0,0,0};
\colorlet{c}{natcomp!70};
\draw [c] (4.60441,0.912457) -- (4.60441,0.967236);
\draw [c] (4.60441,0.967236) -- (4.60441,1.02202);
\draw [c] (4.59627,0.967236) -- (4.60441,0.967236);
\draw [c] (4.60441,0.967236) -- (4.61255,0.967236);
\definecolor{c}{rgb}{0,0,0};
\colorlet{c}{natcomp!70};
\draw [c] (4.62068,0.914155) -- (4.62068,0.969166);
\draw [c] (4.62068,0.969166) -- (4.62068,1.02418);
\draw [c] (4.61255,0.969166) -- (4.62068,0.969166);
\draw [c] (4.62068,0.969166) -- (4.62882,0.969166);
\definecolor{c}{rgb}{0,0,0};
\colorlet{c}{natcomp!70};
\draw [c] (4.63695,0.879878) -- (4.63695,0.938137);
\draw [c] (4.63695,0.938137) -- (4.63695,0.996396);
\draw [c] (4.62882,0.938137) -- (4.63695,0.938137);
\draw [c] (4.63695,0.938137) -- (4.64509,0.938137);
\definecolor{c}{rgb}{0,0,0};
\colorlet{c}{natcomp!70};
\draw [c] (4.65323,0.899783) -- (4.65323,0.956467);
\draw [c] (4.65323,0.956467) -- (4.65323,1.01315);
\draw [c] (4.64509,0.956467) -- (4.65323,0.956467);
\draw [c] (4.65323,0.956467) -- (4.66136,0.956467);
\definecolor{c}{rgb}{0,0,0};
\colorlet{c}{natcomp!70};
\draw [c] (4.6695,0.884772) -- (4.6695,0.936176);
\draw [c] (4.6695,0.936176) -- (4.6695,0.987579);
\draw [c] (4.66136,0.936176) -- (4.6695,0.936176);
\draw [c] (4.6695,0.936176) -- (4.67764,0.936176);
\definecolor{c}{rgb}{0,0,0};
\colorlet{c}{natcomp!70};
\draw [c] (4.68577,0.985752) -- (4.68577,1.05675);
\draw [c] (4.68577,1.05675) -- (4.68577,1.12775);
\draw [c] (4.67764,1.05675) -- (4.68577,1.05675);
\draw [c] (4.68577,1.05675) -- (4.69391,1.05675);
\definecolor{c}{rgb}{0,0,0};
\colorlet{c}{natcomp!70};
\draw [c] (4.70205,0.888065) -- (4.70205,0.939341);
\draw [c] (4.70205,0.939341) -- (4.70205,0.990618);
\draw [c] (4.69391,0.939341) -- (4.70205,0.939341);
\draw [c] (4.70205,0.939341) -- (4.71018,0.939341);
\definecolor{c}{rgb}{0,0,0};
\colorlet{c}{natcomp!70};
\draw [c] (4.71832,0.950968) -- (4.71832,1.01404);
\draw [c] (4.71832,1.01404) -- (4.71832,1.0771);
\draw [c] (4.71018,1.01404) -- (4.71832,1.01404);
\draw [c] (4.71832,1.01404) -- (4.72645,1.01404);
\definecolor{c}{rgb}{0,0,0};
\colorlet{c}{natcomp!70};
\draw [c] (4.73459,0.890061) -- (4.73459,0.949548);
\draw [c] (4.73459,0.949548) -- (4.73459,1.00904);
\draw [c] (4.72645,0.949548) -- (4.73459,0.949548);
\draw [c] (4.73459,0.949548) -- (4.74273,0.949548);
\definecolor{c}{rgb}{0,0,0};
\colorlet{c}{natcomp!70};
\draw [c] (4.75086,0.903141) -- (4.75086,0.959014);
\draw [c] (4.75086,0.959014) -- (4.75086,1.01489);
\draw [c] (4.74273,0.959014) -- (4.75086,0.959014);
\draw [c] (4.75086,0.959014) -- (4.759,0.959014);
\definecolor{c}{rgb}{0,0,0};
\colorlet{c}{natcomp!70};
\draw [c] (4.76714,0.878125) -- (4.76714,0.929474);
\draw [c] (4.76714,0.929474) -- (4.76714,0.980823);
\draw [c] (4.759,0.929474) -- (4.76714,0.929474);
\draw [c] (4.76714,0.929474) -- (4.77527,0.929474);
\definecolor{c}{rgb}{0,0,0};
\colorlet{c}{natcomp!70};
\draw [c] (4.78341,0.831227) -- (4.78341,0.880248);
\draw [c] (4.78341,0.880248) -- (4.78341,0.929269);
\draw [c] (4.77527,0.880248) -- (4.78341,0.880248);
\draw [c] (4.78341,0.880248) -- (4.79155,0.880248);
\definecolor{c}{rgb}{0,0,0};
\colorlet{c}{natcomp!70};
\draw [c] (4.79968,0.864439) -- (4.79968,0.917743);
\draw [c] (4.79968,0.917743) -- (4.79968,0.971047);
\draw [c] (4.79155,0.917743) -- (4.79968,0.917743);
\draw [c] (4.79968,0.917743) -- (4.80782,0.917743);
\definecolor{c}{rgb}{0,0,0};
\colorlet{c}{natcomp!70};
\draw [c] (4.81595,0.852991) -- (4.81595,0.902588);
\draw [c] (4.81595,0.902588) -- (4.81595,0.952185);
\draw [c] (4.80782,0.902588) -- (4.81595,0.902588);
\draw [c] (4.81595,0.902588) -- (4.82409,0.902588);
\definecolor{c}{rgb}{0,0,0};
\colorlet{c}{natcomp!70};
\draw [c] (4.83223,0.838268) -- (4.83223,0.879525);
\draw [c] (4.83223,0.879525) -- (4.83223,0.920782);
\draw [c] (4.82409,0.879525) -- (4.83223,0.879525);
\draw [c] (4.83223,0.879525) -- (4.84036,0.879525);
\definecolor{c}{rgb}{0,0,0};
\colorlet{c}{natcomp!70};
\draw [c] (4.8485,0.849755) -- (4.8485,0.901917);
\draw [c] (4.8485,0.901917) -- (4.8485,0.954078);
\draw [c] (4.84036,0.901917) -- (4.8485,0.901917);
\draw [c] (4.8485,0.901917) -- (4.85664,0.901917);
\definecolor{c}{rgb}{0,0,0};
\colorlet{c}{natcomp!70};
\draw [c] (4.86477,0.88973) -- (4.86477,0.942964);
\draw [c] (4.86477,0.942964) -- (4.86477,0.996198);
\draw [c] (4.85664,0.942964) -- (4.86477,0.942964);
\draw [c] (4.86477,0.942964) -- (4.87291,0.942964);
\definecolor{c}{rgb}{0,0,0};
\colorlet{c}{natcomp!70};
\draw [c] (4.88105,0.852361) -- (4.88105,0.898715);
\draw [c] (4.88105,0.898715) -- (4.88105,0.945068);
\draw [c] (4.87291,0.898715) -- (4.88105,0.898715);
\draw [c] (4.88105,0.898715) -- (4.88918,0.898715);
\definecolor{c}{rgb}{0,0,0};
\colorlet{c}{natcomp!70};
\draw [c] (4.89732,0.889338) -- (4.89732,0.94623);
\draw [c] (4.89732,0.94623) -- (4.89732,1.00312);
\draw [c] (4.88918,0.94623) -- (4.89732,0.94623);
\draw [c] (4.89732,0.94623) -- (4.90545,0.94623);
\definecolor{c}{rgb}{0,0,0};
\colorlet{c}{natcomp!70};
\draw [c] (4.91359,0.818285) -- (4.91359,0.857852);
\draw [c] (4.91359,0.857852) -- (4.91359,0.897419);
\draw [c] (4.90545,0.857852) -- (4.91359,0.857852);
\draw [c] (4.91359,0.857852) -- (4.92173,0.857852);
\definecolor{c}{rgb}{0,0,0};
\colorlet{c}{natcomp!70};
\draw [c] (4.92986,0.839442) -- (4.92986,0.884558);
\draw [c] (4.92986,0.884558) -- (4.92986,0.929674);
\draw [c] (4.92173,0.884558) -- (4.92986,0.884558);
\draw [c] (4.92986,0.884558) -- (4.938,0.884558);
\definecolor{c}{rgb}{0,0,0};
\colorlet{c}{natcomp!70};
\draw [c] (4.94614,0.928982) -- (4.94614,0.983751);
\draw [c] (4.94614,0.983751) -- (4.94614,1.03852);
\draw [c] (4.938,0.983751) -- (4.94614,0.983751);
\draw [c] (4.94614,0.983751) -- (4.95427,0.983751);
\definecolor{c}{rgb}{0,0,0};
\colorlet{c}{natcomp!70};
\draw [c] (4.96241,0.756623) -- (4.96241,0.793018);
\draw [c] (4.96241,0.793018) -- (4.96241,0.829412);
\draw [c] (4.95427,0.793018) -- (4.96241,0.793018);
\draw [c] (4.96241,0.793018) -- (4.97055,0.793018);
\definecolor{c}{rgb}{0,0,0};
\colorlet{c}{natcomp!70};
\draw [c] (4.97868,0.863301) -- (4.97868,0.913408);
\draw [c] (4.97868,0.913408) -- (4.97868,0.963516);
\draw [c] (4.97055,0.913408) -- (4.97868,0.913408);
\draw [c] (4.97868,0.913408) -- (4.98682,0.913408);
\definecolor{c}{rgb}{0,0,0};
\colorlet{c}{natcomp!70};
\draw [c] (4.99495,0.820068) -- (4.99495,0.859989);
\draw [c] (4.99495,0.859989) -- (4.99495,0.89991);
\draw [c] (4.98682,0.859989) -- (4.99495,0.859989);
\draw [c] (4.99495,0.859989) -- (5.00309,0.859989);
\definecolor{c}{rgb}{0,0,0};
\colorlet{c}{natcomp!70};
\draw [c] (5.01123,0.781416) -- (5.01123,0.816202);
\draw [c] (5.01123,0.816202) -- (5.01123,0.850987);
\draw [c] (5.00309,0.816202) -- (5.01123,0.816202);
\draw [c] (5.01123,0.816202) -- (5.01936,0.816202);
\definecolor{c}{rgb}{0,0,0};
\colorlet{c}{natcomp!70};
\draw [c] (5.0275,0.790764) -- (5.0275,0.827697);
\draw [c] (5.0275,0.827697) -- (5.0275,0.86463);
\draw [c] (5.01936,0.827697) -- (5.0275,0.827697);
\draw [c] (5.0275,0.827697) -- (5.03564,0.827697);
\definecolor{c}{rgb}{0,0,0};
\colorlet{c}{natcomp!70};
\draw [c] (5.04377,0.750942) -- (5.04377,0.784172);
\draw [c] (5.04377,0.784172) -- (5.04377,0.817401);
\draw [c] (5.03564,0.784172) -- (5.04377,0.784172);
\draw [c] (5.04377,0.784172) -- (5.05191,0.784172);
\definecolor{c}{rgb}{0,0,0};
\colorlet{c}{natcomp!70};
\draw [c] (5.06005,0.881823) -- (5.06005,0.934976);
\draw [c] (5.06005,0.934976) -- (5.06005,0.988128);
\draw [c] (5.05191,0.934976) -- (5.06005,0.934976);
\draw [c] (5.06005,0.934976) -- (5.06818,0.934976);
\definecolor{c}{rgb}{0,0,0};
\colorlet{c}{natcomp!70};
\draw [c] (5.07632,0.879455) -- (5.07632,0.936439);
\draw [c] (5.07632,0.936439) -- (5.07632,0.993423);
\draw [c] (5.06818,0.936439) -- (5.07632,0.936439);
\draw [c] (5.07632,0.936439) -- (5.08445,0.936439);
\definecolor{c}{rgb}{0,0,0};
\colorlet{c}{natcomp!70};
\draw [c] (5.09259,0.765327) -- (5.09259,0.799544);
\draw [c] (5.09259,0.799544) -- (5.09259,0.833762);
\draw [c] (5.08445,0.799544) -- (5.09259,0.799544);
\draw [c] (5.09259,0.799544) -- (5.10073,0.799544);
\definecolor{c}{rgb}{0,0,0};
\colorlet{c}{natcomp!70};
\draw [c] (5.10886,0.816846) -- (5.10886,0.854773);
\draw [c] (5.10886,0.854773) -- (5.10886,0.8927);
\draw [c] (5.10073,0.854773) -- (5.10886,0.854773);
\draw [c] (5.10886,0.854773) -- (5.117,0.854773);
\definecolor{c}{rgb}{0,0,0};
\colorlet{c}{natcomp!70};
\draw [c] (5.12514,0.878289) -- (5.12514,0.930852);
\draw [c] (5.12514,0.930852) -- (5.12514,0.983415);
\draw [c] (5.117,0.930852) -- (5.12514,0.930852);
\draw [c] (5.12514,0.930852) -- (5.13327,0.930852);
\definecolor{c}{rgb}{0,0,0};
\colorlet{c}{natcomp!70};
\draw [c] (5.14141,0.886117) -- (5.14141,0.93627);
\draw [c] (5.14141,0.93627) -- (5.14141,0.986423);
\draw [c] (5.13327,0.93627) -- (5.14141,0.93627);
\draw [c] (5.14141,0.93627) -- (5.14955,0.93627);
\definecolor{c}{rgb}{0,0,0};
\colorlet{c}{natcomp!70};
\draw [c] (5.15768,0.756901) -- (5.15768,0.7883);
\draw [c] (5.15768,0.7883) -- (5.15768,0.819698);
\draw [c] (5.14955,0.7883) -- (5.15768,0.7883);
\draw [c] (5.15768,0.7883) -- (5.16582,0.7883);
\definecolor{c}{rgb}{0,0,0};
\colorlet{c}{natcomp!70};
\draw [c] (5.17395,0.850388) -- (5.17395,0.903041);
\draw [c] (5.17395,0.903041) -- (5.17395,0.955693);
\draw [c] (5.16582,0.903041) -- (5.17395,0.903041);
\draw [c] (5.17395,0.903041) -- (5.18209,0.903041);
\definecolor{c}{rgb}{0,0,0};
\colorlet{c}{natcomp!70};
\draw [c] (5.19023,0.837397) -- (5.19023,0.880272);
\draw [c] (5.19023,0.880272) -- (5.19023,0.923147);
\draw [c] (5.18209,0.880272) -- (5.19023,0.880272);
\draw [c] (5.19023,0.880272) -- (5.19836,0.880272);
\definecolor{c}{rgb}{0,0,0};
\colorlet{c}{natcomp!70};
\draw [c] (5.2065,0.782891) -- (5.2065,0.819807);
\draw [c] (5.2065,0.819807) -- (5.2065,0.856723);
\draw [c] (5.19836,0.819807) -- (5.2065,0.819807);
\draw [c] (5.2065,0.819807) -- (5.21464,0.819807);
\definecolor{c}{rgb}{0,0,0};
\colorlet{c}{natcomp!70};
\draw [c] (5.22277,0.877003) -- (5.22277,0.925132);
\draw [c] (5.22277,0.925132) -- (5.22277,0.973261);
\draw [c] (5.21464,0.925132) -- (5.22277,0.925132);
\draw [c] (5.22277,0.925132) -- (5.23091,0.925132);
\definecolor{c}{rgb}{0,0,0};
\colorlet{c}{natcomp!70};
\draw [c] (5.23905,0.832507) -- (5.23905,0.877913);
\draw [c] (5.23905,0.877913) -- (5.23905,0.923319);
\draw [c] (5.23091,0.877913) -- (5.23905,0.877913);
\draw [c] (5.23905,0.877913) -- (5.24718,0.877913);
\definecolor{c}{rgb}{0,0,0};
\colorlet{c}{natcomp!70};
\draw [c] (5.25532,0.809327) -- (5.25532,0.853797);
\draw [c] (5.25532,0.853797) -- (5.25532,0.898268);
\draw [c] (5.24718,0.853797) -- (5.25532,0.853797);
\draw [c] (5.25532,0.853797) -- (5.26345,0.853797);
\definecolor{c}{rgb}{0,0,0};
\colorlet{c}{natcomp!70};
\draw [c] (5.27159,0.838939) -- (5.27159,0.884592);
\draw [c] (5.27159,0.884592) -- (5.27159,0.930245);
\draw [c] (5.26345,0.884592) -- (5.27159,0.884592);
\draw [c] (5.27159,0.884592) -- (5.27973,0.884592);
\definecolor{c}{rgb}{0,0,0};
\colorlet{c}{natcomp!70};
\draw [c] (5.28786,0.774997) -- (5.28786,0.815419);
\draw [c] (5.28786,0.815419) -- (5.28786,0.85584);
\draw [c] (5.27973,0.815419) -- (5.28786,0.815419);
\draw [c] (5.28786,0.815419) -- (5.296,0.815419);
\definecolor{c}{rgb}{0,0,0};
\colorlet{c}{natcomp!70};
\draw [c] (5.30414,0.802019) -- (5.30414,0.840859);
\draw [c] (5.30414,0.840859) -- (5.30414,0.879699);
\draw [c] (5.296,0.840859) -- (5.30414,0.840859);
\draw [c] (5.30414,0.840859) -- (5.31227,0.840859);
\definecolor{c}{rgb}{0,0,0};
\colorlet{c}{natcomp!70};
\draw [c] (5.32041,0.734849) -- (5.32041,0.765883);
\draw [c] (5.32041,0.765883) -- (5.32041,0.796917);
\draw [c] (5.31227,0.765883) -- (5.32041,0.765883);
\draw [c] (5.32041,0.765883) -- (5.32855,0.765883);
\definecolor{c}{rgb}{0,0,0};
\colorlet{c}{natcomp!70};
\draw [c] (5.33668,0.76708) -- (5.33668,0.802514);
\draw [c] (5.33668,0.802514) -- (5.33668,0.837949);
\draw [c] (5.32855,0.802514) -- (5.33668,0.802514);
\draw [c] (5.33668,0.802514) -- (5.34482,0.802514);
\definecolor{c}{rgb}{0,0,0};
\colorlet{c}{natcomp!70};
\draw [c] (5.35295,0.820219) -- (5.35295,0.864005);
\draw [c] (5.35295,0.864005) -- (5.35295,0.907791);
\draw [c] (5.34482,0.864005) -- (5.35295,0.864005);
\draw [c] (5.35295,0.864005) -- (5.36109,0.864005);
\definecolor{c}{rgb}{0,0,0};
\colorlet{c}{natcomp!70};
\draw [c] (5.36923,0.792286) -- (5.36923,0.832453);
\draw [c] (5.36923,0.832453) -- (5.36923,0.87262);
\draw [c] (5.36109,0.832453) -- (5.36923,0.832453);
\draw [c] (5.36923,0.832453) -- (5.37736,0.832453);
\definecolor{c}{rgb}{0,0,0};
\colorlet{c}{natcomp!70};
\draw [c] (5.3855,0.779486) -- (5.3855,0.817789);
\draw [c] (5.3855,0.817789) -- (5.3855,0.856092);
\draw [c] (5.37736,0.817789) -- (5.3855,0.817789);
\draw [c] (5.3855,0.817789) -- (5.39364,0.817789);
\definecolor{c}{rgb}{0,0,0};
\colorlet{c}{natcomp!70};
\draw [c] (5.40177,0.758942) -- (5.40177,0.79033);
\draw [c] (5.40177,0.79033) -- (5.40177,0.821719);
\draw [c] (5.39364,0.79033) -- (5.40177,0.79033);
\draw [c] (5.40177,0.79033) -- (5.40991,0.79033);
\definecolor{c}{rgb}{0,0,0};
\colorlet{c}{natcomp!70};
\draw [c] (5.41805,0.828335) -- (5.41805,0.872313);
\draw [c] (5.41805,0.872313) -- (5.41805,0.916292);
\draw [c] (5.40991,0.872313) -- (5.41805,0.872313);
\draw [c] (5.41805,0.872313) -- (5.42618,0.872313);
\definecolor{c}{rgb}{0,0,0};
\colorlet{c}{natcomp!70};
\draw [c] (5.43432,0.796227) -- (5.43432,0.833442);
\draw [c] (5.43432,0.833442) -- (5.43432,0.870657);
\draw [c] (5.42618,0.833442) -- (5.43432,0.833442);
\draw [c] (5.43432,0.833442) -- (5.44245,0.833442);
\definecolor{c}{rgb}{0,0,0};
\colorlet{c}{natcomp!70};
\draw [c] (5.45059,0.824412) -- (5.45059,0.86803);
\draw [c] (5.45059,0.86803) -- (5.45059,0.911649);
\draw [c] (5.44245,0.86803) -- (5.45059,0.86803);
\draw [c] (5.45059,0.86803) -- (5.45873,0.86803);
\definecolor{c}{rgb}{0,0,0};
\colorlet{c}{natcomp!70};
\draw [c] (5.46686,0.749011) -- (5.46686,0.778106);
\draw [c] (5.46686,0.778106) -- (5.46686,0.807201);
\draw [c] (5.45873,0.778106) -- (5.46686,0.778106);
\draw [c] (5.46686,0.778106) -- (5.475,0.778106);
\definecolor{c}{rgb}{0,0,0};
\colorlet{c}{natcomp!70};
\draw [c] (5.48314,0.717158) -- (5.48314,0.738269);
\draw [c] (5.48314,0.738269) -- (5.48314,0.75938);
\draw [c] (5.475,0.738269) -- (5.48314,0.738269);
\draw [c] (5.48314,0.738269) -- (5.49127,0.738269);
\definecolor{c}{rgb}{0,0,0};
\colorlet{c}{natcomp!70};
\draw [c] (5.49941,0.760794) -- (5.49941,0.791025);
\draw [c] (5.49941,0.791025) -- (5.49941,0.821255);
\draw [c] (5.49127,0.791025) -- (5.49941,0.791025);
\draw [c] (5.49941,0.791025) -- (5.50755,0.791025);
\definecolor{c}{rgb}{0,0,0};
\colorlet{c}{natcomp!70};
\draw [c] (5.51568,0.780683) -- (5.51568,0.819473);
\draw [c] (5.51568,0.819473) -- (5.51568,0.858263);
\draw [c] (5.50755,0.819473) -- (5.51568,0.819473);
\draw [c] (5.51568,0.819473) -- (5.52382,0.819473);
\definecolor{c}{rgb}{0,0,0};
\colorlet{c}{natcomp!70};
\draw [c] (5.53195,0.767697) -- (5.53195,0.81179);
\draw [c] (5.53195,0.81179) -- (5.53195,0.855883);
\draw [c] (5.52382,0.81179) -- (5.53195,0.81179);
\draw [c] (5.53195,0.81179) -- (5.54009,0.81179);
\definecolor{c}{rgb}{0,0,0};
\colorlet{c}{natcomp!70};
\draw [c] (5.54823,0.789662) -- (5.54823,0.82657);
\draw [c] (5.54823,0.82657) -- (5.54823,0.863477);
\draw [c] (5.54009,0.82657) -- (5.54823,0.82657);
\draw [c] (5.54823,0.82657) -- (5.55636,0.82657);
\definecolor{c}{rgb}{0,0,0};
\colorlet{c}{natcomp!70};
\draw [c] (5.5645,0.836001) -- (5.5645,0.881788);
\draw [c] (5.5645,0.881788) -- (5.5645,0.927576);
\draw [c] (5.55636,0.881788) -- (5.5645,0.881788);
\draw [c] (5.5645,0.881788) -- (5.57264,0.881788);
\definecolor{c}{rgb}{0,0,0};
\colorlet{c}{natcomp!70};
\draw [c] (5.58077,0.77978) -- (5.58077,0.81503);
\draw [c] (5.58077,0.81503) -- (5.58077,0.85028);
\draw [c] (5.57264,0.81503) -- (5.58077,0.81503);
\draw [c] (5.58077,0.81503) -- (5.58891,0.81503);
\definecolor{c}{rgb}{0,0,0};
\colorlet{c}{natcomp!70};
\draw [c] (5.59705,0.765328) -- (5.59705,0.798276);
\draw [c] (5.59705,0.798276) -- (5.59705,0.831225);
\draw [c] (5.58891,0.798276) -- (5.59705,0.798276);
\draw [c] (5.59705,0.798276) -- (5.60518,0.798276);
\definecolor{c}{rgb}{0,0,0};
\colorlet{c}{natcomp!70};
\draw [c] (5.61332,0.713101) -- (5.61332,0.735885);
\draw [c] (5.61332,0.735885) -- (5.61332,0.758668);
\draw [c] (5.60518,0.735885) -- (5.61332,0.735885);
\draw [c] (5.61332,0.735885) -- (5.62145,0.735885);
\definecolor{c}{rgb}{0,0,0};
\colorlet{c}{natcomp!70};
\draw [c] (5.62959,0.77156) -- (5.62959,0.806931);
\draw [c] (5.62959,0.806931) -- (5.62959,0.842301);
\draw [c] (5.62145,0.806931) -- (5.62959,0.806931);
\draw [c] (5.62959,0.806931) -- (5.63773,0.806931);
\definecolor{c}{rgb}{0,0,0};
\colorlet{c}{natcomp!70};
\draw [c] (5.64586,0.767598) -- (5.64586,0.801089);
\draw [c] (5.64586,0.801089) -- (5.64586,0.83458);
\draw [c] (5.63773,0.801089) -- (5.64586,0.801089);
\draw [c] (5.64586,0.801089) -- (5.654,0.801089);
\definecolor{c}{rgb}{0,0,0};
\colorlet{c}{natcomp!70};
\draw [c] (5.66214,0.812558) -- (5.66214,0.852249);
\draw [c] (5.66214,0.852249) -- (5.66214,0.891941);
\draw [c] (5.654,0.852249) -- (5.66214,0.852249);
\draw [c] (5.66214,0.852249) -- (5.67027,0.852249);
\definecolor{c}{rgb}{0,0,0};
\colorlet{c}{natcomp!70};
\draw [c] (5.67841,0.750819) -- (5.67841,0.790319);
\draw [c] (5.67841,0.790319) -- (5.67841,0.829819);
\draw [c] (5.67027,0.790319) -- (5.67841,0.790319);
\draw [c] (5.67841,0.790319) -- (5.68655,0.790319);
\definecolor{c}{rgb}{0,0,0};
\colorlet{c}{natcomp!70};
\draw [c] (5.69468,0.763809) -- (5.69468,0.800161);
\draw [c] (5.69468,0.800161) -- (5.69468,0.836514);
\draw [c] (5.68655,0.800161) -- (5.69468,0.800161);
\draw [c] (5.69468,0.800161) -- (5.70282,0.800161);
\definecolor{c}{rgb}{0,0,0};
\colorlet{c}{natcomp!70};
\draw [c] (5.71095,0.773509) -- (5.71095,0.810299);
\draw [c] (5.71095,0.810299) -- (5.71095,0.84709);
\draw [c] (5.70282,0.810299) -- (5.71095,0.810299);
\draw [c] (5.71095,0.810299) -- (5.71909,0.810299);
\definecolor{c}{rgb}{0,0,0};
\colorlet{c}{natcomp!70};
\draw [c] (5.72723,0.781311) -- (5.72723,0.815949);
\draw [c] (5.72723,0.815949) -- (5.72723,0.850586);
\draw [c] (5.71909,0.815949) -- (5.72723,0.815949);
\draw [c] (5.72723,0.815949) -- (5.73536,0.815949);
\definecolor{c}{rgb}{0,0,0};
\colorlet{c}{natcomp!70};
\draw [c] (5.7435,0.749041) -- (5.7435,0.783543);
\draw [c] (5.7435,0.783543) -- (5.7435,0.818044);
\draw [c] (5.73536,0.783543) -- (5.7435,0.783543);
\draw [c] (5.7435,0.783543) -- (5.75164,0.783543);
\definecolor{c}{rgb}{0,0,0};
\colorlet{c}{natcomp!70};
\draw [c] (5.75977,0.734525) -- (5.75977,0.760913);
\draw [c] (5.75977,0.760913) -- (5.75977,0.787301);
\draw [c] (5.75164,0.760913) -- (5.75977,0.760913);
\draw [c] (5.75977,0.760913) -- (5.76791,0.760913);
\definecolor{c}{rgb}{0,0,0};
\colorlet{c}{natcomp!70};
\draw [c] (5.77605,0.757329) -- (5.77605,0.795295);
\draw [c] (5.77605,0.795295) -- (5.77605,0.833262);
\draw [c] (5.76791,0.795295) -- (5.77605,0.795295);
\draw [c] (5.77605,0.795295) -- (5.78418,0.795295);
\definecolor{c}{rgb}{0,0,0};
\colorlet{c}{natcomp!70};
\draw [c] (5.79232,0.778456) -- (5.79232,0.824701);
\draw [c] (5.79232,0.824701) -- (5.79232,0.870946);
\draw [c] (5.78418,0.824701) -- (5.79232,0.824701);
\draw [c] (5.79232,0.824701) -- (5.80045,0.824701);
\definecolor{c}{rgb}{0,0,0};
\colorlet{c}{natcomp!70};
\draw [c] (5.80859,0.791499) -- (5.80859,0.828233);
\draw [c] (5.80859,0.828233) -- (5.80859,0.864968);
\draw [c] (5.80045,0.828233) -- (5.80859,0.828233);
\draw [c] (5.80859,0.828233) -- (5.81673,0.828233);
\definecolor{c}{rgb}{0,0,0};
\colorlet{c}{natcomp!70};
\draw [c] (5.82486,0.778777) -- (5.82486,0.81849);
\draw [c] (5.82486,0.81849) -- (5.82486,0.858202);
\draw [c] (5.81673,0.81849) -- (5.82486,0.81849);
\draw [c] (5.82486,0.81849) -- (5.833,0.81849);
\definecolor{c}{rgb}{0,0,0};
\colorlet{c}{natcomp!70};
\draw [c] (5.84114,0.766161) -- (5.84114,0.815256);
\draw [c] (5.84114,0.815256) -- (5.84114,0.864352);
\draw [c] (5.833,0.815256) -- (5.84114,0.815256);
\draw [c] (5.84114,0.815256) -- (5.84927,0.815256);
\definecolor{c}{rgb}{0,0,0};
\colorlet{c}{natcomp!70};
\draw [c] (5.85741,0.739717) -- (5.85741,0.766483);
\draw [c] (5.85741,0.766483) -- (5.85741,0.79325);
\draw [c] (5.84927,0.766483) -- (5.85741,0.766483);
\draw [c] (5.85741,0.766483) -- (5.86555,0.766483);
\definecolor{c}{rgb}{0,0,0};
\colorlet{c}{natcomp!70};
\draw [c] (5.87368,0.767167) -- (5.87368,0.798411);
\draw [c] (5.87368,0.798411) -- (5.87368,0.829655);
\draw [c] (5.86555,0.798411) -- (5.87368,0.798411);
\draw [c] (5.87368,0.798411) -- (5.88182,0.798411);
\definecolor{c}{rgb}{0,0,0};
\colorlet{c}{natcomp!70};
\draw [c] (5.88995,0.7537) -- (5.88995,0.78274);
\draw [c] (5.88995,0.78274) -- (5.88995,0.811781);
\draw [c] (5.88182,0.78274) -- (5.88995,0.78274);
\draw [c] (5.88995,0.78274) -- (5.89809,0.78274);
\definecolor{c}{rgb}{0,0,0};
\colorlet{c}{natcomp!70};
\draw [c] (5.90623,0.787736) -- (5.90623,0.823567);
\draw [c] (5.90623,0.823567) -- (5.90623,0.859398);
\draw [c] (5.89809,0.823567) -- (5.90623,0.823567);
\draw [c] (5.90623,0.823567) -- (5.91436,0.823567);
\definecolor{c}{rgb}{0,0,0};
\colorlet{c}{natcomp!70};
\draw [c] (5.9225,0.7657) -- (5.9225,0.797966);
\draw [c] (5.9225,0.797966) -- (5.9225,0.830233);
\draw [c] (5.91436,0.797966) -- (5.9225,0.797966);
\draw [c] (5.9225,0.797966) -- (5.93064,0.797966);
\definecolor{c}{rgb}{0,0,0};
\colorlet{c}{natcomp!70};
\draw [c] (5.93877,0.776906) -- (5.93877,0.812086);
\draw [c] (5.93877,0.812086) -- (5.93877,0.847266);
\draw [c] (5.93064,0.812086) -- (5.93877,0.812086);
\draw [c] (5.93877,0.812086) -- (5.94691,0.812086);
\definecolor{c}{rgb}{0,0,0};
\colorlet{c}{natcomp!70};
\draw [c] (5.95505,0.727034) -- (5.95505,0.751779);
\draw [c] (5.95505,0.751779) -- (5.95505,0.776524);
\draw [c] (5.94691,0.751779) -- (5.95505,0.751779);
\draw [c] (5.95505,0.751779) -- (5.96318,0.751779);
\definecolor{c}{rgb}{0,0,0};
\colorlet{c}{natcomp!70};
\draw [c] (5.97132,0.718677) -- (5.97132,0.7484);
\draw [c] (5.97132,0.7484) -- (5.97132,0.778123);
\draw [c] (5.96318,0.7484) -- (5.97132,0.7484);
\draw [c] (5.97132,0.7484) -- (5.97945,0.7484);
\definecolor{c}{rgb}{0,0,0};
\colorlet{c}{natcomp!70};
\draw [c] (5.98759,0.723722) -- (5.98759,0.746507);
\draw [c] (5.98759,0.746507) -- (5.98759,0.769292);
\draw [c] (5.97945,0.746507) -- (5.98759,0.746507);
\draw [c] (5.98759,0.746507) -- (5.99573,0.746507);
\definecolor{c}{rgb}{0,0,0};
\colorlet{c}{natcomp!70};
\draw [c] (6.00386,0.756977) -- (6.00386,0.792774);
\draw [c] (6.00386,0.792774) -- (6.00386,0.828571);
\draw [c] (5.99573,0.792774) -- (6.00386,0.792774);
\draw [c] (6.00386,0.792774) -- (6.012,0.792774);
\definecolor{c}{rgb}{0,0,0};
\colorlet{c}{natcomp!70};
\draw [c] (6.02014,0.72654) -- (6.02014,0.750922);
\draw [c] (6.02014,0.750922) -- (6.02014,0.775304);
\draw [c] (6.012,0.750922) -- (6.02014,0.750922);
\draw [c] (6.02014,0.750922) -- (6.02827,0.750922);
\definecolor{c}{rgb}{0,0,0};
\colorlet{c}{natcomp!70};
\draw [c] (6.03641,0.765876) -- (6.03641,0.798169);
\draw [c] (6.03641,0.798169) -- (6.03641,0.830462);
\draw [c] (6.02827,0.798169) -- (6.03641,0.798169);
\draw [c] (6.03641,0.798169) -- (6.04455,0.798169);
\definecolor{c}{rgb}{0,0,0};
\colorlet{c}{natcomp!70};
\draw [c] (6.05268,0.723692) -- (6.05268,0.763257);
\draw [c] (6.05268,0.763257) -- (6.05268,0.802821);
\draw [c] (6.04455,0.763257) -- (6.05268,0.763257);
\draw [c] (6.05268,0.763257) -- (6.06082,0.763257);
\definecolor{c}{rgb}{0,0,0};
\colorlet{c}{natcomp!70};
\draw [c] (6.06895,0.735947) -- (6.06895,0.776286);
\draw [c] (6.06895,0.776286) -- (6.06895,0.816626);
\draw [c] (6.06082,0.776286) -- (6.06895,0.776286);
\draw [c] (6.06895,0.776286) -- (6.07709,0.776286);
\definecolor{c}{rgb}{0,0,0};
\colorlet{c}{natcomp!70};
\draw [c] (6.08523,0.768199) -- (6.08523,0.8023);
\draw [c] (6.08523,0.8023) -- (6.08523,0.836401);
\draw [c] (6.07709,0.8023) -- (6.08523,0.8023);
\draw [c] (6.08523,0.8023) -- (6.09336,0.8023);
\definecolor{c}{rgb}{0,0,0};
\colorlet{c}{natcomp!70};
\draw [c] (6.1015,0.710277) -- (6.1015,0.729228);
\draw [c] (6.1015,0.729228) -- (6.1015,0.748179);
\draw [c] (6.09336,0.729228) -- (6.1015,0.729228);
\draw [c] (6.1015,0.729228) -- (6.10964,0.729228);
\definecolor{c}{rgb}{0,0,0};
\colorlet{c}{natcomp!70};
\draw [c] (6.11777,0.740692) -- (6.11777,0.767798);
\draw [c] (6.11777,0.767798) -- (6.11777,0.794904);
\draw [c] (6.10964,0.767798) -- (6.11777,0.767798);
\draw [c] (6.11777,0.767798) -- (6.12591,0.767798);
\definecolor{c}{rgb}{0,0,0};
\colorlet{c}{natcomp!70};
\draw [c] (6.13405,0.720528) -- (6.13405,0.743887);
\draw [c] (6.13405,0.743887) -- (6.13405,0.767247);
\draw [c] (6.12591,0.743887) -- (6.13405,0.743887);
\draw [c] (6.13405,0.743887) -- (6.14218,0.743887);
\definecolor{c}{rgb}{0,0,0};
\colorlet{c}{natcomp!70};
\draw [c] (6.15032,0.69748) -- (6.15032,0.712307);
\draw [c] (6.15032,0.712307) -- (6.15032,0.727135);
\draw [c] (6.14218,0.712307) -- (6.15032,0.712307);
\draw [c] (6.15032,0.712307) -- (6.15845,0.712307);
\definecolor{c}{rgb}{0,0,0};
\colorlet{c}{natcomp!70};
\draw [c] (6.16659,0.709648) -- (6.16659,0.728192);
\draw [c] (6.16659,0.728192) -- (6.16659,0.746736);
\draw [c] (6.15845,0.728192) -- (6.16659,0.728192);
\draw [c] (6.16659,0.728192) -- (6.17473,0.728192);
\definecolor{c}{rgb}{0,0,0};
\colorlet{c}{natcomp!70};
\draw [c] (6.18286,0.7346) -- (6.18286,0.766423);
\draw [c] (6.18286,0.766423) -- (6.18286,0.798247);
\draw [c] (6.17473,0.766423) -- (6.18286,0.766423);
\draw [c] (6.18286,0.766423) -- (6.191,0.766423);
\definecolor{c}{rgb}{0,0,0};
\colorlet{c}{natcomp!70};
\draw [c] (6.19914,0.725682) -- (6.19914,0.753499);
\draw [c] (6.19914,0.753499) -- (6.19914,0.781317);
\draw [c] (6.191,0.753499) -- (6.19914,0.753499);
\draw [c] (6.19914,0.753499) -- (6.20727,0.753499);
\definecolor{c}{rgb}{0,0,0};
\colorlet{c}{natcomp!70};
\draw [c] (6.21541,0.726686) -- (6.21541,0.759284);
\draw [c] (6.21541,0.759284) -- (6.21541,0.791882);
\draw [c] (6.20727,0.759284) -- (6.21541,0.759284);
\draw [c] (6.21541,0.759284) -- (6.22355,0.759284);
\definecolor{c}{rgb}{0,0,0};
\colorlet{c}{natcomp!70};
\draw [c] (6.23168,0.742176) -- (6.23168,0.777624);
\draw [c] (6.23168,0.777624) -- (6.23168,0.813072);
\draw [c] (6.22355,0.777624) -- (6.23168,0.777624);
\draw [c] (6.23168,0.777624) -- (6.23982,0.777624);
\definecolor{c}{rgb}{0,0,0};
\colorlet{c}{natcomp!70};
\draw [c] (6.24795,0.776333) -- (6.24795,0.816985);
\draw [c] (6.24795,0.816985) -- (6.24795,0.857637);
\draw [c] (6.23982,0.816985) -- (6.24795,0.816985);
\draw [c] (6.24795,0.816985) -- (6.25609,0.816985);
\definecolor{c}{rgb}{0,0,0};
\colorlet{c}{natcomp!70};
\draw [c] (6.26423,0.716593) -- (6.26423,0.737387);
\draw [c] (6.26423,0.737387) -- (6.26423,0.758182);
\draw [c] (6.25609,0.737387) -- (6.26423,0.737387);
\draw [c] (6.26423,0.737387) -- (6.27236,0.737387);
\definecolor{c}{rgb}{0,0,0};
\colorlet{c}{natcomp!70};
\draw [c] (6.2805,0.73157) -- (6.2805,0.759414);
\draw [c] (6.2805,0.759414) -- (6.2805,0.787258);
\draw [c] (6.27236,0.759414) -- (6.2805,0.759414);
\draw [c] (6.2805,0.759414) -- (6.28864,0.759414);
\definecolor{c}{rgb}{0,0,0};
\colorlet{c}{natcomp!70};
\draw [c] (6.29677,0.765117) -- (6.29677,0.800819);
\draw [c] (6.29677,0.800819) -- (6.29677,0.83652);
\draw [c] (6.28864,0.800819) -- (6.29677,0.800819);
\draw [c] (6.29677,0.800819) -- (6.30491,0.800819);
\definecolor{c}{rgb}{0,0,0};
\colorlet{c}{natcomp!70};
\draw [c] (6.31305,0.697057) -- (6.31305,0.711255);
\draw [c] (6.31305,0.711255) -- (6.31305,0.725454);
\draw [c] (6.30491,0.711255) -- (6.31305,0.711255);
\draw [c] (6.31305,0.711255) -- (6.32118,0.711255);
\definecolor{c}{rgb}{0,0,0};
\colorlet{c}{natcomp!70};
\draw [c] (6.32932,0.699766) -- (6.32932,0.721515);
\draw [c] (6.32932,0.721515) -- (6.32932,0.743263);
\draw [c] (6.32118,0.721515) -- (6.32932,0.721515);
\draw [c] (6.32932,0.721515) -- (6.33745,0.721515);
\definecolor{c}{rgb}{0,0,0};
\colorlet{c}{natcomp!70};
\draw [c] (6.34559,0.75193) -- (6.34559,0.784287);
\draw [c] (6.34559,0.784287) -- (6.34559,0.816645);
\draw [c] (6.33745,0.784287) -- (6.34559,0.784287);
\draw [c] (6.34559,0.784287) -- (6.35373,0.784287);
\definecolor{c}{rgb}{0,0,0};
\colorlet{c}{natcomp!70};
\draw [c] (6.36186,0.751842) -- (6.36186,0.788755);
\draw [c] (6.36186,0.788755) -- (6.36186,0.825668);
\draw [c] (6.35373,0.788755) -- (6.36186,0.788755);
\draw [c] (6.36186,0.788755) -- (6.37,0.788755);
\definecolor{c}{rgb}{0,0,0};
\colorlet{c}{natcomp!70};
\draw [c] (6.37814,0.740937) -- (6.37814,0.773723);
\draw [c] (6.37814,0.773723) -- (6.37814,0.806508);
\draw [c] (6.37,0.773723) -- (6.37814,0.773723);
\draw [c] (6.37814,0.773723) -- (6.38627,0.773723);
\definecolor{c}{rgb}{0,0,0};
\colorlet{c}{natcomp!70};
\draw [c] (6.39441,0.698759) -- (6.39441,0.715275);
\draw [c] (6.39441,0.715275) -- (6.39441,0.73179);
\draw [c] (6.38627,0.715275) -- (6.39441,0.715275);
\draw [c] (6.39441,0.715275) -- (6.40255,0.715275);
\definecolor{c}{rgb}{0,0,0};
\colorlet{c}{natcomp!70};
\draw [c] (6.41068,0.710571) -- (6.41068,0.730263);
\draw [c] (6.41068,0.730263) -- (6.41068,0.749955);
\draw [c] (6.40255,0.730263) -- (6.41068,0.730263);
\draw [c] (6.41068,0.730263) -- (6.41882,0.730263);
\definecolor{c}{rgb}{0,0,0};
\colorlet{c}{natcomp!70};
\draw [c] (6.42695,0.733223) -- (6.42695,0.762826);
\draw [c] (6.42695,0.762826) -- (6.42695,0.792429);
\draw [c] (6.41882,0.762826) -- (6.42695,0.762826);
\draw [c] (6.42695,0.762826) -- (6.43509,0.762826);
\definecolor{c}{rgb}{0,0,0};
\colorlet{c}{natcomp!70};
\draw [c] (6.44323,0.722776) -- (6.44323,0.747622);
\draw [c] (6.44323,0.747622) -- (6.44323,0.772467);
\draw [c] (6.43509,0.747622) -- (6.44323,0.747622);
\draw [c] (6.44323,0.747622) -- (6.45136,0.747622);
\definecolor{c}{rgb}{0,0,0};
\colorlet{c}{natcomp!70};
\draw [c] (6.4595,0.739345) -- (6.4595,0.766635);
\draw [c] (6.4595,0.766635) -- (6.4595,0.793925);
\draw [c] (6.45136,0.766635) -- (6.4595,0.766635);
\draw [c] (6.4595,0.766635) -- (6.46764,0.766635);
\definecolor{c}{rgb}{0,0,0};
\colorlet{c}{natcomp!70};
\draw [c] (6.47577,0.710389) -- (6.47577,0.729721);
\draw [c] (6.47577,0.729721) -- (6.47577,0.749052);
\draw [c] (6.46764,0.729721) -- (6.47577,0.729721);
\draw [c] (6.47577,0.729721) -- (6.48391,0.729721);
\definecolor{c}{rgb}{0,0,0};
\colorlet{c}{natcomp!70};
\draw [c] (6.49205,0.741601) -- (6.49205,0.774287);
\draw [c] (6.49205,0.774287) -- (6.49205,0.806974);
\draw [c] (6.48391,0.774287) -- (6.49205,0.774287);
\draw [c] (6.49205,0.774287) -- (6.50018,0.774287);
\definecolor{c}{rgb}{0,0,0};
\colorlet{c}{natcomp!70};
\draw [c] (6.50832,0.704581) -- (6.50832,0.722973);
\draw [c] (6.50832,0.722973) -- (6.50832,0.741364);
\draw [c] (6.50018,0.722973) -- (6.50832,0.722973);
\draw [c] (6.50832,0.722973) -- (6.51645,0.722973);
\definecolor{c}{rgb}{0,0,0};
\colorlet{c}{natcomp!70};
\draw [c] (6.52459,0.698445) -- (6.52459,0.714356);
\draw [c] (6.52459,0.714356) -- (6.52459,0.730267);
\draw [c] (6.51645,0.714356) -- (6.52459,0.714356);
\draw [c] (6.52459,0.714356) -- (6.53273,0.714356);
\definecolor{c}{rgb}{0,0,0};
\colorlet{c}{natcomp!70};
\draw [c] (6.54086,0.713893) -- (6.54086,0.736042);
\draw [c] (6.54086,0.736042) -- (6.54086,0.758192);
\draw [c] (6.53273,0.736042) -- (6.54086,0.736042);
\draw [c] (6.54086,0.736042) -- (6.549,0.736042);
\definecolor{c}{rgb}{0,0,0};
\colorlet{c}{natcomp!70};
\draw [c] (6.55714,0.705456) -- (6.55714,0.724441);
\draw [c] (6.55714,0.724441) -- (6.55714,0.743426);
\draw [c] (6.549,0.724441) -- (6.55714,0.724441);
\draw [c] (6.55714,0.724441) -- (6.56527,0.724441);
\definecolor{c}{rgb}{0,0,0};
\colorlet{c}{natcomp!70};
\draw [c] (6.57341,0.708858) -- (6.57341,0.726682);
\draw [c] (6.57341,0.726682) -- (6.57341,0.744506);
\draw [c] (6.56527,0.726682) -- (6.57341,0.726682);
\draw [c] (6.57341,0.726682) -- (6.58155,0.726682);
\definecolor{c}{rgb}{0,0,0};
\colorlet{c}{natcomp!70};
\draw [c] (6.58968,0.744164) -- (6.58968,0.773304);
\draw [c] (6.58968,0.773304) -- (6.58968,0.802443);
\draw [c] (6.58155,0.773304) -- (6.58968,0.773304);
\draw [c] (6.58968,0.773304) -- (6.59782,0.773304);
\definecolor{c}{rgb}{0,0,0};
\colorlet{c}{natcomp!70};
\draw [c] (6.60595,0.728187) -- (6.60595,0.754684);
\draw [c] (6.60595,0.754684) -- (6.60595,0.781181);
\draw [c] (6.59782,0.754684) -- (6.60595,0.754684);
\draw [c] (6.60595,0.754684) -- (6.61409,0.754684);
\definecolor{c}{rgb}{0,0,0};
\colorlet{c}{natcomp!70};
\draw [c] (6.62223,0.716607) -- (6.62223,0.741593);
\draw [c] (6.62223,0.741593) -- (6.62223,0.766578);
\draw [c] (6.61409,0.741593) -- (6.62223,0.741593);
\draw [c] (6.62223,0.741593) -- (6.63036,0.741593);
\definecolor{c}{rgb}{0,0,0};
\colorlet{c}{natcomp!70};
\draw [c] (6.6385,0.699804) -- (6.6385,0.717549);
\draw [c] (6.6385,0.717549) -- (6.6385,0.735295);
\draw [c] (6.63036,0.717549) -- (6.6385,0.717549);
\draw [c] (6.6385,0.717549) -- (6.64664,0.717549);
\definecolor{c}{rgb}{0,0,0};
\colorlet{c}{natcomp!70};
\draw [c] (6.65477,0.753668) -- (6.65477,0.789033);
\draw [c] (6.65477,0.789033) -- (6.65477,0.824398);
\draw [c] (6.64664,0.789033) -- (6.65477,0.789033);
\draw [c] (6.65477,0.789033) -- (6.66291,0.789033);
\definecolor{c}{rgb}{0,0,0};
\colorlet{c}{natcomp!70};
\draw [c] (6.67105,0.708234) -- (6.67105,0.734765);
\draw [c] (6.67105,0.734765) -- (6.67105,0.761297);
\draw [c] (6.66291,0.734765) -- (6.67105,0.734765);
\draw [c] (6.67105,0.734765) -- (6.67918,0.734765);
\definecolor{c}{rgb}{0,0,0};
\colorlet{c}{natcomp!70};
\draw [c] (6.68732,0.713447) -- (6.68732,0.738205);
\draw [c] (6.68732,0.738205) -- (6.68732,0.762963);
\draw [c] (6.67918,0.738205) -- (6.68732,0.738205);
\draw [c] (6.68732,0.738205) -- (6.69545,0.738205);
\definecolor{c}{rgb}{0,0,0};
\colorlet{c}{natcomp!70};
\draw [c] (6.70359,0.733101) -- (6.70359,0.759424);
\draw [c] (6.70359,0.759424) -- (6.70359,0.785747);
\draw [c] (6.69545,0.759424) -- (6.70359,0.759424);
\draw [c] (6.70359,0.759424) -- (6.71173,0.759424);
\definecolor{c}{rgb}{0,0,0};
\colorlet{c}{natcomp!70};
\draw [c] (6.71986,0.703929) -- (6.71986,0.721255);
\draw [c] (6.71986,0.721255) -- (6.71986,0.738582);
\draw [c] (6.71173,0.721255) -- (6.71986,0.721255);
\draw [c] (6.71986,0.721255) -- (6.728,0.721255);
\definecolor{c}{rgb}{0,0,0};
\colorlet{c}{natcomp!70};
\draw [c] (6.73614,0.704397) -- (6.73614,0.721875);
\draw [c] (6.73614,0.721875) -- (6.73614,0.739353);
\draw [c] (6.728,0.721875) -- (6.73614,0.721875);
\draw [c] (6.73614,0.721875) -- (6.74427,0.721875);
\definecolor{c}{rgb}{0,0,0};
\colorlet{c}{natcomp!70};
\draw [c] (6.75241,0.710501) -- (6.75241,0.729699);
\draw [c] (6.75241,0.729699) -- (6.75241,0.748896);
\draw [c] (6.74427,0.729699) -- (6.75241,0.729699);
\draw [c] (6.75241,0.729699) -- (6.76055,0.729699);
\definecolor{c}{rgb}{0,0,0};
\colorlet{c}{natcomp!70};
\draw [c] (6.76868,0.711589) -- (6.76868,0.731751);
\draw [c] (6.76868,0.731751) -- (6.76868,0.751914);
\draw [c] (6.76055,0.731751) -- (6.76868,0.731751);
\draw [c] (6.76868,0.731751) -- (6.77682,0.731751);
\definecolor{c}{rgb}{0,0,0};
\colorlet{c}{natcomp!70};
\draw [c] (6.78495,0.718017) -- (6.78495,0.745245);
\draw [c] (6.78495,0.745245) -- (6.78495,0.772473);
\draw [c] (6.77682,0.745245) -- (6.78495,0.745245);
\draw [c] (6.78495,0.745245) -- (6.79309,0.745245);
\definecolor{c}{rgb}{0,0,0};
\colorlet{c}{natcomp!70};
\draw [c] (6.80123,0.697887) -- (6.80123,0.713515);
\draw [c] (6.80123,0.713515) -- (6.80123,0.729144);
\draw [c] (6.79309,0.713515) -- (6.80123,0.713515);
\draw [c] (6.80123,0.713515) -- (6.80936,0.713515);
\definecolor{c}{rgb}{0,0,0};
\colorlet{c}{natcomp!70};
\draw [c] (6.8175,0.712023) -- (6.8175,0.732788);
\draw [c] (6.8175,0.732788) -- (6.8175,0.753553);
\draw [c] (6.80936,0.732788) -- (6.8175,0.732788);
\draw [c] (6.8175,0.732788) -- (6.82564,0.732788);
\definecolor{c}{rgb}{0,0,0};
\colorlet{c}{natcomp!70};
\draw [c] (6.83377,0.714232) -- (6.83377,0.736446);
\draw [c] (6.83377,0.736446) -- (6.83377,0.758661);
\draw [c] (6.82564,0.736446) -- (6.83377,0.736446);
\draw [c] (6.83377,0.736446) -- (6.84191,0.736446);
\definecolor{c}{rgb}{0,0,0};
\colorlet{c}{natcomp!70};
\draw [c] (6.85005,0.709462) -- (6.85005,0.739161);
\draw [c] (6.85005,0.739161) -- (6.85005,0.768861);
\draw [c] (6.84191,0.739161) -- (6.85005,0.739161);
\draw [c] (6.85005,0.739161) -- (6.85818,0.739161);
\definecolor{c}{rgb}{0,0,0};
\colorlet{c}{natcomp!70};
\draw [c] (6.86632,0.69326) -- (6.86632,0.708575);
\draw [c] (6.86632,0.708575) -- (6.86632,0.72389);
\draw [c] (6.85818,0.708575) -- (6.86632,0.708575);
\draw [c] (6.86632,0.708575) -- (6.87445,0.708575);
\definecolor{c}{rgb}{0,0,0};
\colorlet{c}{natcomp!70};
\draw [c] (6.88259,0.703644) -- (6.88259,0.720612);
\draw [c] (6.88259,0.720612) -- (6.88259,0.73758);
\draw [c] (6.87445,0.720612) -- (6.88259,0.720612);
\draw [c] (6.88259,0.720612) -- (6.89073,0.720612);
\definecolor{c}{rgb}{0,0,0};
\colorlet{c}{natcomp!70};
\draw [c] (6.89886,0.692271) -- (6.89886,0.705485);
\draw [c] (6.89886,0.705485) -- (6.89886,0.7187);
\draw [c] (6.89073,0.705485) -- (6.89886,0.705485);
\draw [c] (6.89886,0.705485) -- (6.907,0.705485);
\definecolor{c}{rgb}{0,0,0};
\colorlet{c}{natcomp!70};
\draw [c] (6.91514,0.692587) -- (6.91514,0.706325);
\draw [c] (6.91514,0.706325) -- (6.91514,0.720063);
\draw [c] (6.907,0.706325) -- (6.91514,0.706325);
\draw [c] (6.91514,0.706325) -- (6.92327,0.706325);
\definecolor{c}{rgb}{0,0,0};
\colorlet{c}{natcomp!70};
\draw [c] (6.93141,0.686927) -- (6.93141,0.695065);
\draw [c] (6.93141,0.695065) -- (6.93141,0.703204);
\draw [c] (6.92327,0.695065) -- (6.93141,0.695065);
\draw [c] (6.93141,0.695065) -- (6.93955,0.695065);
\definecolor{c}{rgb}{0,0,0};
\colorlet{c}{natcomp!70};
\draw [c] (6.94768,0.704333) -- (6.94768,0.7221);
\draw [c] (6.94768,0.7221) -- (6.94768,0.739867);
\draw [c] (6.93955,0.7221) -- (6.94768,0.7221);
\draw [c] (6.94768,0.7221) -- (6.95582,0.7221);
\definecolor{c}{rgb}{0,0,0};
\colorlet{c}{natcomp!70};
\draw [c] (6.96395,0.686909) -- (6.96395,0.68692);
\draw [c] (6.96395,0.68692) -- (6.96395,0.686931);
\draw [c] (6.95582,0.68692) -- (6.96395,0.68692);
\draw [c] (6.96395,0.68692) -- (6.97209,0.68692);
\definecolor{c}{rgb}{0,0,0};
\colorlet{c}{natcomp!70};
\draw [c] (6.98023,0.718409) -- (6.98023,0.740294);
\draw [c] (6.98023,0.740294) -- (6.98023,0.762179);
\draw [c] (6.97209,0.740294) -- (6.98023,0.740294);
\draw [c] (6.98023,0.740294) -- (6.98836,0.740294);
\definecolor{c}{rgb}{0,0,0};
\colorlet{c}{natcomp!70};
\draw [c] (6.9965,0.686917) -- (6.9965,0.68693);
\draw [c] (6.9965,0.68693) -- (6.9965,0.686943);
\draw [c] (6.98836,0.68693) -- (6.9965,0.68693);
\draw [c] (6.9965,0.68693) -- (7.00464,0.68693);
\definecolor{c}{rgb}{0,0,0};
\colorlet{c}{natcomp!70};
\draw [c] (7.01277,0.713906) -- (7.01277,0.736963);
\draw [c] (7.01277,0.736963) -- (7.01277,0.760019);
\draw [c] (7.00464,0.736963) -- (7.01277,0.736963);
\draw [c] (7.01277,0.736963) -- (7.02091,0.736963);
\definecolor{c}{rgb}{0,0,0};
\colorlet{c}{natcomp!70};
\draw [c] (7.02905,0.703869) -- (7.02905,0.721431);
\draw [c] (7.02905,0.721431) -- (7.02905,0.738992);
\draw [c] (7.02091,0.721431) -- (7.02905,0.721431);
\draw [c] (7.02905,0.721431) -- (7.03718,0.721431);
\definecolor{c}{rgb}{0,0,0};
\colorlet{c}{natcomp!70};
\draw [c] (7.04532,0.705655) -- (7.04532,0.724877);
\draw [c] (7.04532,0.724877) -- (7.04532,0.7441);
\draw [c] (7.03718,0.724877) -- (7.04532,0.724877);
\draw [c] (7.04532,0.724877) -- (7.05345,0.724877);
\definecolor{c}{rgb}{0,0,0};
\colorlet{c}{natcomp!70};
\draw [c] (7.06159,0.703801) -- (7.06159,0.72083);
\draw [c] (7.06159,0.72083) -- (7.06159,0.73786);
\draw [c] (7.05345,0.72083) -- (7.06159,0.72083);
\draw [c] (7.06159,0.72083) -- (7.06973,0.72083);
\definecolor{c}{rgb}{0,0,0};
\colorlet{c}{natcomp!70};
\draw [c] (7.07786,0.719124) -- (7.07786,0.741674);
\draw [c] (7.07786,0.741674) -- (7.07786,0.764224);
\draw [c] (7.06973,0.741674) -- (7.07786,0.741674);
\draw [c] (7.07786,0.741674) -- (7.086,0.741674);
\definecolor{c}{rgb}{0,0,0};
\colorlet{c}{natcomp!70};
\draw [c] (7.09414,0.709945) -- (7.09414,0.738016);
\draw [c] (7.09414,0.738016) -- (7.09414,0.766087);
\draw [c] (7.086,0.738016) -- (7.09414,0.738016);
\draw [c] (7.09414,0.738016) -- (7.10227,0.738016);
\definecolor{c}{rgb}{0,0,0};
\colorlet{c}{natcomp!70};
\draw [c] (7.11041,0.691149) -- (7.11041,0.701332);
\draw [c] (7.11041,0.701332) -- (7.11041,0.711516);
\draw [c] (7.10227,0.701332) -- (7.11041,0.701332);
\draw [c] (7.11041,0.701332) -- (7.11855,0.701332);
\definecolor{c}{rgb}{0,0,0};
\colorlet{c}{natcomp!70};
\draw [c] (7.12668,0.706897) -- (7.12668,0.728052);
\draw [c] (7.12668,0.728052) -- (7.12668,0.749206);
\draw [c] (7.11855,0.728052) -- (7.12668,0.728052);
\draw [c] (7.12668,0.728052) -- (7.13482,0.728052);
\definecolor{c}{rgb}{0,0,0};
\colorlet{c}{natcomp!70};
\draw [c] (7.14295,0.691954) -- (7.14295,0.704175);
\draw [c] (7.14295,0.704175) -- (7.14295,0.716395);
\draw [c] (7.13482,0.704175) -- (7.14295,0.704175);
\draw [c] (7.14295,0.704175) -- (7.15109,0.704175);
\definecolor{c}{rgb}{0,0,0};
\colorlet{c}{natcomp!70};
\draw [c] (7.15923,0.686944) -- (7.15923,0.695807);
\draw [c] (7.15923,0.695807) -- (7.15923,0.704669);
\draw [c] (7.15109,0.695807) -- (7.15923,0.695807);
\draw [c] (7.15923,0.695807) -- (7.16736,0.695807);
\definecolor{c}{rgb}{0,0,0};
\colorlet{c}{natcomp!70};
\draw [c] (7.1755,0.713402) -- (7.1755,0.735241);
\draw [c] (7.1755,0.735241) -- (7.1755,0.75708);
\draw [c] (7.16736,0.735241) -- (7.1755,0.735241);
\draw [c] (7.1755,0.735241) -- (7.18364,0.735241);
\definecolor{c}{rgb}{0,0,0};
\colorlet{c}{natcomp!70};
\draw [c] (7.19177,0.70731) -- (7.19177,0.72784);
\draw [c] (7.19177,0.72784) -- (7.19177,0.74837);
\draw [c] (7.18364,0.72784) -- (7.19177,0.72784);
\draw [c] (7.19177,0.72784) -- (7.19991,0.72784);
\definecolor{c}{rgb}{0,0,0};
\colorlet{c}{natcomp!70};
\draw [c] (7.20805,0.694873) -- (7.20805,0.730967);
\draw [c] (7.20805,0.730967) -- (7.20805,0.767061);
\draw [c] (7.19991,0.730967) -- (7.20805,0.730967);
\draw [c] (7.20805,0.730967) -- (7.21618,0.730967);
\definecolor{c}{rgb}{0,0,0};
\colorlet{c}{natcomp!70};
\draw [c] (7.22432,0.692201) -- (7.22432,0.704986);
\draw [c] (7.22432,0.704986) -- (7.22432,0.71777);
\draw [c] (7.21618,0.704986) -- (7.22432,0.704986);
\draw [c] (7.22432,0.704986) -- (7.23245,0.704986);
\definecolor{c}{rgb}{0,0,0};
\colorlet{c}{natcomp!70};
\draw [c] (7.24059,0.694145) -- (7.24059,0.712713);
\draw [c] (7.24059,0.712713) -- (7.24059,0.731281);
\draw [c] (7.23245,0.712713) -- (7.24059,0.712713);
\draw [c] (7.24059,0.712713) -- (7.24873,0.712713);
\definecolor{c}{rgb}{0,0,0};
\colorlet{c}{natcomp!70};
\draw [c] (7.25686,0.686911) -- (7.25686,0.696805);
\draw [c] (7.25686,0.696805) -- (7.25686,0.706699);
\draw [c] (7.24873,0.696805) -- (7.25686,0.696805);
\draw [c] (7.25686,0.696805) -- (7.265,0.696805);
\definecolor{c}{rgb}{0,0,0};
\colorlet{c}{natcomp!70};
\draw [c] (7.27314,0.686939) -- (7.27314,0.698171);
\draw [c] (7.27314,0.698171) -- (7.27314,0.709404);
\draw [c] (7.265,0.698171) -- (7.27314,0.698171);
\draw [c] (7.27314,0.698171) -- (7.28127,0.698171);
\definecolor{c}{rgb}{0,0,0};
\colorlet{c}{natcomp!70};
\draw [c] (7.28941,0.705817) -- (7.28941,0.724823);
\draw [c] (7.28941,0.724823) -- (7.28941,0.743829);
\draw [c] (7.28127,0.724823) -- (7.28941,0.724823);
\draw [c] (7.28941,0.724823) -- (7.29755,0.724823);
\definecolor{c}{rgb}{0,0,0};
\colorlet{c}{natcomp!70};
\draw [c] (7.30568,0.704012) -- (7.30568,0.721633);
\draw [c] (7.30568,0.721633) -- (7.30568,0.739254);
\draw [c] (7.29755,0.721633) -- (7.30568,0.721633);
\draw [c] (7.30568,0.721633) -- (7.31382,0.721633);
\definecolor{c}{rgb}{0,0,0};
\colorlet{c}{natcomp!70};
\draw [c] (7.32195,0.714809) -- (7.32195,0.753099);
\draw [c] (7.32195,0.753099) -- (7.32195,0.791389);
\draw [c] (7.31382,0.753099) -- (7.32195,0.753099);
\draw [c] (7.32195,0.753099) -- (7.33009,0.753099);
\definecolor{c}{rgb}{0,0,0};
\colorlet{c}{natcomp!70};
\draw [c] (7.33823,0.697923) -- (7.33823,0.713018);
\draw [c] (7.33823,0.713018) -- (7.33823,0.728113);
\draw [c] (7.33009,0.713018) -- (7.33823,0.713018);
\draw [c] (7.33823,0.713018) -- (7.34636,0.713018);
\definecolor{c}{rgb}{0,0,0};
\colorlet{c}{natcomp!70};
\draw [c] (7.3545,0.698691) -- (7.3545,0.715314);
\draw [c] (7.3545,0.715314) -- (7.3545,0.731936);
\draw [c] (7.34636,0.715314) -- (7.3545,0.715314);
\draw [c] (7.3545,0.715314) -- (7.36264,0.715314);
\definecolor{c}{rgb}{0,0,0};
\colorlet{c}{natcomp!70};
\draw [c] (7.37077,0.717852) -- (7.37077,0.745539);
\draw [c] (7.37077,0.745539) -- (7.37077,0.773225);
\draw [c] (7.36264,0.745539) -- (7.37077,0.745539);
\draw [c] (7.37077,0.745539) -- (7.37891,0.745539);
\definecolor{c}{rgb}{0,0,0};
\colorlet{c}{natcomp!70};
\draw [c] (7.38705,0.701315) -- (7.38705,0.726433);
\draw [c] (7.38705,0.726433) -- (7.38705,0.751551);
\draw [c] (7.37891,0.726433) -- (7.38705,0.726433);
\draw [c] (7.38705,0.726433) -- (7.39518,0.726433);
\definecolor{c}{rgb}{0,0,0};
\colorlet{c}{natcomp!70};
\draw [c] (7.40332,0.719987) -- (7.40332,0.743126);
\draw [c] (7.40332,0.743126) -- (7.40332,0.766266);
\draw [c] (7.39518,0.743126) -- (7.40332,0.743126);
\draw [c] (7.40332,0.743126) -- (7.41145,0.743126);
\definecolor{c}{rgb}{0,0,0};
\colorlet{c}{natcomp!70};
\draw [c] (7.41959,0.691933) -- (7.41959,0.704153);
\draw [c] (7.41959,0.704153) -- (7.41959,0.716373);
\draw [c] (7.41145,0.704153) -- (7.41959,0.704153);
\draw [c] (7.41959,0.704153) -- (7.42773,0.704153);
\definecolor{c}{rgb}{0,0,0};
\colorlet{c}{natcomp!70};
\draw [c] (7.45214,0.686902) -- (7.45214,0.695866);
\draw [c] (7.45214,0.695866) -- (7.45214,0.704829);
\draw [c] (7.444,0.695866) -- (7.45214,0.695866);
\draw [c] (7.45214,0.695866) -- (7.46027,0.695866);
\definecolor{c}{rgb}{0,0,0};
\colorlet{c}{natcomp!70};
\draw [c] (7.46841,0.686921) -- (7.46841,0.695885);
\draw [c] (7.46841,0.695885) -- (7.46841,0.704848);
\draw [c] (7.46027,0.695885) -- (7.46841,0.695885);
\draw [c] (7.46841,0.695885) -- (7.47655,0.695885);
\definecolor{c}{rgb}{0,0,0};
\colorlet{c}{natcomp!70};
\draw [c] (7.48468,0.686897) -- (7.48468,0.686905);
\draw [c] (7.48468,0.686905) -- (7.48468,0.686914);
\draw [c] (7.47655,0.686905) -- (7.48468,0.686905);
\draw [c] (7.48468,0.686905) -- (7.49282,0.686905);
\definecolor{c}{rgb}{0,0,0};
\colorlet{c}{natcomp!70};
\draw [c] (7.50095,0.686894) -- (7.50095,0.686899);
\draw [c] (7.50095,0.686899) -- (7.50095,0.686904);
\draw [c] (7.49282,0.686899) -- (7.50095,0.686899);
\draw [c] (7.50095,0.686899) -- (7.50909,0.686899);
\definecolor{c}{rgb}{0,0,0};
\colorlet{c}{natcomp!70};
\draw [c] (7.51723,0.702942) -- (7.51723,0.719147);
\draw [c] (7.51723,0.719147) -- (7.51723,0.735351);
\draw [c] (7.50909,0.719147) -- (7.51723,0.719147);
\draw [c] (7.51723,0.719147) -- (7.52536,0.719147);
\definecolor{c}{rgb}{0,0,0};
\colorlet{c}{natcomp!70};
\draw [c] (7.5335,0.691582) -- (7.5335,0.703001);
\draw [c] (7.5335,0.703001) -- (7.5335,0.71442);
\draw [c] (7.52536,0.703001) -- (7.5335,0.703001);
\draw [c] (7.5335,0.703001) -- (7.54164,0.703001);
\definecolor{c}{rgb}{0,0,0};
\colorlet{c}{natcomp!70};
\draw [c] (7.54977,0.699339) -- (7.54977,0.718609);
\draw [c] (7.54977,0.718609) -- (7.54977,0.737879);
\draw [c] (7.54164,0.718609) -- (7.54977,0.718609);
\draw [c] (7.54977,0.718609) -- (7.55791,0.718609);
\definecolor{c}{rgb}{0,0,0};
\colorlet{c}{natcomp!70};
\draw [c] (7.56605,0.705399) -- (7.56605,0.728064);
\draw [c] (7.56605,0.728064) -- (7.56605,0.750729);
\draw [c] (7.55791,0.728064) -- (7.56605,0.728064);
\draw [c] (7.56605,0.728064) -- (7.57418,0.728064);
\definecolor{c}{rgb}{0,0,0};
\colorlet{c}{natcomp!70};
\draw [c] (7.58232,0.699404) -- (7.58232,0.723597);
\draw [c] (7.58232,0.723597) -- (7.58232,0.74779);
\draw [c] (7.57418,0.723597) -- (7.58232,0.723597);
\draw [c] (7.58232,0.723597) -- (7.59045,0.723597);
\definecolor{c}{rgb}{0,0,0};
\colorlet{c}{natcomp!70};
\draw [c] (7.59859,0.686912) -- (7.59859,0.694113);
\draw [c] (7.59859,0.694113) -- (7.59859,0.701313);
\draw [c] (7.59045,0.694113) -- (7.59859,0.694113);
\draw [c] (7.59859,0.694113) -- (7.60673,0.694113);
\definecolor{c}{rgb}{0,0,0};
\colorlet{c}{natcomp!70};
\draw [c] (7.61486,0.686896) -- (7.61486,0.686902);
\draw [c] (7.61486,0.686902) -- (7.61486,0.686909);
\draw [c] (7.60673,0.686902) -- (7.61486,0.686902);
\draw [c] (7.61486,0.686902) -- (7.623,0.686902);
\definecolor{c}{rgb}{0,0,0};
\colorlet{c}{natcomp!70};
\draw [c] (7.63114,0.721485) -- (7.63114,0.74621);
\draw [c] (7.63114,0.74621) -- (7.63114,0.770934);
\draw [c] (7.623,0.74621) -- (7.63114,0.74621);
\draw [c] (7.63114,0.74621) -- (7.63927,0.74621);
\definecolor{c}{rgb}{0,0,0};
\colorlet{c}{natcomp!70};
\draw [c] (7.64741,0.686934) -- (7.64741,0.695073);
\draw [c] (7.64741,0.695073) -- (7.64741,0.703212);
\draw [c] (7.63927,0.695073) -- (7.64741,0.695073);
\draw [c] (7.64741,0.695073) -- (7.65555,0.695073);
\definecolor{c}{rgb}{0,0,0};
\colorlet{c}{natcomp!70};
\draw [c] (7.66368,0.69193) -- (7.66368,0.704888);
\draw [c] (7.66368,0.704888) -- (7.66368,0.717847);
\draw [c] (7.65555,0.704888) -- (7.66368,0.704888);
\draw [c] (7.66368,0.704888) -- (7.67182,0.704888);
\definecolor{c}{rgb}{0,0,0};
\colorlet{c}{natcomp!70};
\draw [c] (7.67995,0.693227) -- (7.67995,0.708478);
\draw [c] (7.67995,0.708478) -- (7.67995,0.723729);
\draw [c] (7.67182,0.708478) -- (7.67995,0.708478);
\draw [c] (7.67995,0.708478) -- (7.68809,0.708478);
\definecolor{c}{rgb}{0,0,0};
\colorlet{c}{natcomp!70};
\draw [c] (7.69623,0.705019) -- (7.69623,0.723124);
\draw [c] (7.69623,0.723124) -- (7.69623,0.74123);
\draw [c] (7.68809,0.723124) -- (7.69623,0.723124);
\draw [c] (7.69623,0.723124) -- (7.70436,0.723124);
\definecolor{c}{rgb}{0,0,0};
\colorlet{c}{natcomp!70};
\draw [c] (7.7125,0.686901) -- (7.7125,0.686914);
\draw [c] (7.7125,0.686914) -- (7.7125,0.686926);
\draw [c] (7.70436,0.686914) -- (7.7125,0.686914);
\draw [c] (7.7125,0.686914) -- (7.72064,0.686914);
\definecolor{c}{rgb}{0,0,0};
\colorlet{c}{natcomp!70};
\draw [c] (7.72877,0.686911) -- (7.72877,0.697705);
\draw [c] (7.72877,0.697705) -- (7.72877,0.7085);
\draw [c] (7.72064,0.697705) -- (7.72877,0.697705);
\draw [c] (7.72877,0.697705) -- (7.73691,0.697705);
\definecolor{c}{rgb}{0,0,0};
\colorlet{c}{natcomp!70};
\draw [c] (7.74505,0.697486) -- (7.74505,0.712097);
\draw [c] (7.74505,0.712097) -- (7.74505,0.726708);
\draw [c] (7.73691,0.712097) -- (7.74505,0.712097);
\draw [c] (7.74505,0.712097) -- (7.75318,0.712097);
\definecolor{c}{rgb}{0,0,0};
\colorlet{c}{natcomp!70};
\draw [c] (7.76132,0.692362) -- (7.76132,0.708107);
\draw [c] (7.76132,0.708107) -- (7.76132,0.723852);
\draw [c] (7.75318,0.708107) -- (7.76132,0.708107);
\draw [c] (7.76132,0.708107) -- (7.76945,0.708107);
\definecolor{c}{rgb}{0,0,0};
\colorlet{c}{natcomp!70};
\draw [c] (7.77759,0.699709) -- (7.77759,0.717404);
\draw [c] (7.77759,0.717404) -- (7.77759,0.735098);
\draw [c] (7.76945,0.717404) -- (7.77759,0.717404);
\draw [c] (7.77759,0.717404) -- (7.78573,0.717404);
\definecolor{c}{rgb}{0,0,0};
\colorlet{c}{natcomp!70};
\draw [c] (7.79386,0.733221) -- (7.79386,0.759074);
\draw [c] (7.79386,0.759074) -- (7.79386,0.784928);
\draw [c] (7.78573,0.759074) -- (7.79386,0.759074);
\draw [c] (7.79386,0.759074) -- (7.802,0.759074);
\definecolor{c}{rgb}{0,0,0};
\colorlet{c}{natcomp!70};
\draw [c] (7.81014,0.697974) -- (7.81014,0.713528);
\draw [c] (7.81014,0.713528) -- (7.81014,0.729082);
\draw [c] (7.802,0.713528) -- (7.81014,0.713528);
\draw [c] (7.81014,0.713528) -- (7.81827,0.713528);
\definecolor{c}{rgb}{0,0,0};
\colorlet{c}{natcomp!70};
\draw [c] (7.82641,0.692082) -- (7.82641,0.705044);
\draw [c] (7.82641,0.705044) -- (7.82641,0.718006);
\draw [c] (7.81827,0.705044) -- (7.82641,0.705044);
\draw [c] (7.82641,0.705044) -- (7.83455,0.705044);
\definecolor{c}{rgb}{0,0,0};
\colorlet{c}{natcomp!70};
\draw [c] (7.84268,0.69139) -- (7.84268,0.702257);
\draw [c] (7.84268,0.702257) -- (7.84268,0.713124);
\draw [c] (7.83455,0.702257) -- (7.84268,0.702257);
\draw [c] (7.84268,0.702257) -- (7.85082,0.702257);
\definecolor{c}{rgb}{0,0,0};
\colorlet{c}{natcomp!70};
\draw [c] (7.85895,0.68692) -- (7.85895,0.696815);
\draw [c] (7.85895,0.696815) -- (7.85895,0.706709);
\draw [c] (7.85082,0.696815) -- (7.85895,0.696815);
\draw [c] (7.85895,0.696815) -- (7.86709,0.696815);
\definecolor{c}{rgb}{0,0,0};
\colorlet{c}{natcomp!70};
\draw [c] (7.87523,0.699254) -- (7.87523,0.718923);
\draw [c] (7.87523,0.718923) -- (7.87523,0.738592);
\draw [c] (7.86709,0.718923) -- (7.87523,0.718923);
\draw [c] (7.87523,0.718923) -- (7.88336,0.718923);
\definecolor{c}{rgb}{0,0,0};
\colorlet{c}{natcomp!70};
\draw [c] (7.8915,0.711611) -- (7.8915,0.732071);
\draw [c] (7.8915,0.732071) -- (7.8915,0.752532);
\draw [c] (7.88336,0.732071) -- (7.8915,0.732071);
\draw [c] (7.8915,0.732071) -- (7.89964,0.732071);
\definecolor{c}{rgb}{0,0,0};
\colorlet{c}{natcomp!70};
\draw [c] (7.90777,0.715085) -- (7.90777,0.738825);
\draw [c] (7.90777,0.738825) -- (7.90777,0.762565);
\draw [c] (7.89964,0.738825) -- (7.90777,0.738825);
\draw [c] (7.90777,0.738825) -- (7.91591,0.738825);
\definecolor{c}{rgb}{0,0,0};
\colorlet{c}{natcomp!70};
\draw [c] (7.92405,0.686922) -- (7.92405,0.696038);
\draw [c] (7.92405,0.696038) -- (7.92405,0.705153);
\draw [c] (7.91591,0.696038) -- (7.92405,0.696038);
\draw [c] (7.92405,0.696038) -- (7.93218,0.696038);
\definecolor{c}{rgb}{0,0,0};
\colorlet{c}{natcomp!70};
\draw [c] (7.94032,0.686907) -- (7.94032,0.696802);
\draw [c] (7.94032,0.696802) -- (7.94032,0.706696);
\draw [c] (7.93218,0.696802) -- (7.94032,0.696802);
\draw [c] (7.94032,0.696802) -- (7.94845,0.696802);
\definecolor{c}{rgb}{0,0,0};
\colorlet{c}{natcomp!70};
\draw [c] (7.95659,0.691941) -- (7.95659,0.704161);
\draw [c] (7.95659,0.704161) -- (7.95659,0.716381);
\draw [c] (7.94845,0.704161) -- (7.95659,0.704161);
\draw [c] (7.95659,0.704161) -- (7.96473,0.704161);
\definecolor{c}{rgb}{0,0,0};
\colorlet{c}{natcomp!70};
\draw [c] (7.97286,0.686912) -- (7.97286,0.69505);
\draw [c] (7.97286,0.69505) -- (7.97286,0.703189);
\draw [c] (7.96473,0.69505) -- (7.97286,0.69505);
\draw [c] (7.97286,0.69505) -- (7.981,0.69505);
\definecolor{c}{rgb}{0,0,0};
\colorlet{c}{natcomp!70};
\draw [c] (7.98914,0.693662) -- (7.98914,0.710807);
\draw [c] (7.98914,0.710807) -- (7.98914,0.727952);
\draw [c] (7.981,0.710807) -- (7.98914,0.710807);
\draw [c] (7.98914,0.710807) -- (7.99727,0.710807);
\definecolor{c}{rgb}{0,0,0};
\colorlet{c}{natcomp!70};
\draw [c] (8.00541,0.686922) -- (8.00541,0.695785);
\draw [c] (8.00541,0.695785) -- (8.00541,0.704647);
\draw [c] (7.99727,0.695785) -- (8.00541,0.695785);
\draw [c] (8.00541,0.695785) -- (8.01355,0.695785);
\definecolor{c}{rgb}{0,0,0};
\colorlet{c}{natcomp!70};
\draw [c] (8.02168,0.691552) -- (8.02168,0.702771);
\draw [c] (8.02168,0.702771) -- (8.02168,0.71399);
\draw [c] (8.01355,0.702771) -- (8.02168,0.702771);
\draw [c] (8.02168,0.702771) -- (8.02982,0.702771);
\definecolor{c}{rgb}{0,0,0};
\colorlet{c}{natcomp!70};
\draw [c] (8.03795,0.686908) -- (8.03795,0.70091);
\draw [c] (8.03795,0.70091) -- (8.03795,0.714913);
\draw [c] (8.02982,0.70091) -- (8.03795,0.70091);
\draw [c] (8.03795,0.70091) -- (8.04609,0.70091);
\definecolor{c}{rgb}{0,0,0};
\colorlet{c}{natcomp!70};
\draw [c] (8.05423,0.686902) -- (8.05423,0.695765);
\draw [c] (8.05423,0.695765) -- (8.05423,0.704627);
\draw [c] (8.04609,0.695765) -- (8.05423,0.695765);
\draw [c] (8.05423,0.695765) -- (8.06236,0.695765);
\definecolor{c}{rgb}{0,0,0};
\colorlet{c}{natcomp!70};
\draw [c] (8.0705,0.686913) -- (8.0705,0.694635);
\draw [c] (8.0705,0.694635) -- (8.0705,0.702356);
\draw [c] (8.06236,0.694635) -- (8.0705,0.694635);
\draw [c] (8.0705,0.694635) -- (8.07864,0.694635);
\definecolor{c}{rgb}{0,0,0};
\colorlet{c}{natcomp!70};
\draw [c] (8.08677,0.691886) -- (8.08677,0.703918);
\draw [c] (8.08677,0.703918) -- (8.08677,0.715951);
\draw [c] (8.07864,0.703918) -- (8.08677,0.703918);
\draw [c] (8.08677,0.703918) -- (8.09491,0.703918);
\definecolor{c}{rgb}{0,0,0};
\colorlet{c}{natcomp!70};
\draw [c] (8.10305,0.699036) -- (8.10305,0.715859);
\draw [c] (8.10305,0.715859) -- (8.10305,0.732682);
\draw [c] (8.09491,0.715859) -- (8.10305,0.715859);
\draw [c] (8.10305,0.715859) -- (8.11118,0.715859);
\definecolor{c}{rgb}{0,0,0};
\colorlet{c}{natcomp!70};
\draw [c] (8.11932,0.70093) -- (8.11932,0.723385);
\draw [c] (8.11932,0.723385) -- (8.11932,0.74584);
\draw [c] (8.11118,0.723385) -- (8.11932,0.723385);
\draw [c] (8.11932,0.723385) -- (8.12745,0.723385);
\definecolor{c}{rgb}{0,0,0};
\colorlet{c}{natcomp!70};
\draw [c] (8.13559,0.691991) -- (8.13559,0.704542);
\draw [c] (8.13559,0.704542) -- (8.13559,0.717092);
\draw [c] (8.12745,0.704542) -- (8.13559,0.704542);
\draw [c] (8.13559,0.704542) -- (8.14373,0.704542);
\definecolor{c}{rgb}{0,0,0};
\colorlet{c}{natcomp!70};
\draw [c] (8.15186,0.704381) -- (8.15186,0.722348);
\draw [c] (8.15186,0.722348) -- (8.15186,0.740315);
\draw [c] (8.14373,0.722348) -- (8.15186,0.722348);
\draw [c] (8.15186,0.722348) -- (8.16,0.722348);
\definecolor{c}{rgb}{0,0,0};
\colorlet{c}{natcomp!70};
\draw [c] (8.16814,0.692514) -- (8.16814,0.706186);
\draw [c] (8.16814,0.706186) -- (8.16814,0.719858);
\draw [c] (8.16,0.706186) -- (8.16814,0.706186);
\draw [c] (8.16814,0.706186) -- (8.17627,0.706186);
\definecolor{c}{rgb}{0,0,0};
\colorlet{c}{natcomp!70};
\draw [c] (8.18441,0.686908) -- (8.18441,0.697319);
\draw [c] (8.18441,0.697319) -- (8.18441,0.70773);
\draw [c] (8.17627,0.697319) -- (8.18441,0.697319);
\draw [c] (8.18441,0.697319) -- (8.19255,0.697319);
\definecolor{c}{rgb}{0,0,0};
\colorlet{c}{natcomp!70};
\draw [c] (8.20068,0.686908) -- (8.20068,0.708675);
\draw [c] (8.20068,0.708675) -- (8.20068,0.730443);
\draw [c] (8.19255,0.708675) -- (8.20068,0.708675);
\draw [c] (8.20068,0.708675) -- (8.20882,0.708675);
\definecolor{c}{rgb}{0,0,0};
\colorlet{c}{natcomp!70};
\draw [c] (8.21695,0.691366) -- (8.21695,0.702233);
\draw [c] (8.21695,0.702233) -- (8.21695,0.7131);
\draw [c] (8.20882,0.702233) -- (8.21695,0.702233);
\draw [c] (8.21695,0.702233) -- (8.22509,0.702233);
\definecolor{c}{rgb}{0,0,0};
\colorlet{c}{natcomp!70};
\draw [c] (8.23323,0.700598) -- (8.23323,0.719335);
\draw [c] (8.23323,0.719335) -- (8.23323,0.738073);
\draw [c] (8.22509,0.719335) -- (8.23323,0.719335);
\draw [c] (8.23323,0.719335) -- (8.24136,0.719335);
\definecolor{c}{rgb}{0,0,0};
\colorlet{c}{natcomp!70};
\draw [c] (8.2495,0.698239) -- (8.2495,0.713998);
\draw [c] (8.2495,0.713998) -- (8.2495,0.729758);
\draw [c] (8.24136,0.713998) -- (8.2495,0.713998);
\draw [c] (8.2495,0.713998) -- (8.25764,0.713998);
\definecolor{c}{rgb}{0,0,0};
\colorlet{c}{natcomp!70};
\draw [c] (8.26577,0.704207) -- (8.26577,0.72189);
\draw [c] (8.26577,0.72189) -- (8.26577,0.739573);
\draw [c] (8.25764,0.72189) -- (8.26577,0.72189);
\draw [c] (8.26577,0.72189) -- (8.27391,0.72189);
\definecolor{c}{rgb}{0,0,0};
\colorlet{c}{natcomp!70};
\draw [c] (8.28205,0.699406) -- (8.28205,0.719236);
\draw [c] (8.28205,0.719236) -- (8.28205,0.739065);
\draw [c] (8.27391,0.719236) -- (8.28205,0.719236);
\draw [c] (8.28205,0.719236) -- (8.29018,0.719236);
\definecolor{c}{rgb}{0,0,0};
\colorlet{c}{natcomp!70};
\draw [c] (8.29832,0.686897) -- (8.29832,0.686904);
\draw [c] (8.29832,0.686904) -- (8.29832,0.686911);
\draw [c] (8.29018,0.686904) -- (8.29832,0.686904);
\draw [c] (8.29832,0.686904) -- (8.30645,0.686904);
\definecolor{c}{rgb}{0,0,0};
\colorlet{c}{natcomp!70};
\draw [c] (8.31459,0.692156) -- (8.31459,0.704967);
\draw [c] (8.31459,0.704967) -- (8.31459,0.717779);
\draw [c] (8.30645,0.704967) -- (8.31459,0.704967);
\draw [c] (8.31459,0.704967) -- (8.32273,0.704967);
\definecolor{c}{rgb}{0,0,0};
\colorlet{c}{natcomp!70};
\draw [c] (8.33086,0.686894) -- (8.33086,0.686898);
\draw [c] (8.33086,0.686898) -- (8.33086,0.686902);
\draw [c] (8.32273,0.686898) -- (8.33086,0.686898);
\draw [c] (8.33086,0.686898) -- (8.339,0.686898);
\definecolor{c}{rgb}{0,0,0};
\colorlet{c}{natcomp!70};
\draw [c] (8.34714,0.686913) -- (8.34714,0.696808);
\draw [c] (8.34714,0.696808) -- (8.34714,0.706702);
\draw [c] (8.339,0.696808) -- (8.34714,0.696808);
\draw [c] (8.34714,0.696808) -- (8.35527,0.696808);
\definecolor{c}{rgb}{0,0,0};
\colorlet{c}{natcomp!70};
\draw [c] (8.36341,0.6869) -- (8.36341,0.686908);
\draw [c] (8.36341,0.686908) -- (8.36341,0.686916);
\draw [c] (8.35527,0.686908) -- (8.36341,0.686908);
\draw [c] (8.36341,0.686908) -- (8.37155,0.686908);
\definecolor{c}{rgb}{0,0,0};
\colorlet{c}{natcomp!70};
\draw [c] (8.37968,0.709537) -- (8.37968,0.733883);
\draw [c] (8.37968,0.733883) -- (8.37968,0.758228);
\draw [c] (8.37155,0.733883) -- (8.37968,0.733883);
\draw [c] (8.37968,0.733883) -- (8.38782,0.733883);
\definecolor{c}{rgb}{0,0,0};
\colorlet{c}{natcomp!70};
\draw [c] (8.39595,0.686897) -- (8.39595,0.696792);
\draw [c] (8.39595,0.696792) -- (8.39595,0.706686);
\draw [c] (8.38782,0.696792) -- (8.39595,0.696792);
\draw [c] (8.39595,0.696792) -- (8.40409,0.696792);
\definecolor{c}{rgb}{0,0,0};
\colorlet{c}{natcomp!70};
\draw [c] (8.41223,0.686907) -- (8.41223,0.696023);
\draw [c] (8.41223,0.696023) -- (8.41223,0.705138);
\draw [c] (8.40409,0.696023) -- (8.41223,0.696023);
\draw [c] (8.41223,0.696023) -- (8.42036,0.696023);
\definecolor{c}{rgb}{0,0,0};
\colorlet{c}{natcomp!70};
\draw [c] (8.4285,0.691882) -- (8.4285,0.703914);
\draw [c] (8.4285,0.703914) -- (8.4285,0.715947);
\draw [c] (8.42036,0.703914) -- (8.4285,0.703914);
\draw [c] (8.4285,0.703914) -- (8.43664,0.703914);
\definecolor{c}{rgb}{0,0,0};
\colorlet{c}{natcomp!70};
\draw [c] (8.44477,0.686908) -- (8.44477,0.69814);
\draw [c] (8.44477,0.69814) -- (8.44477,0.709372);
\draw [c] (8.43664,0.69814) -- (8.44477,0.69814);
\draw [c] (8.44477,0.69814) -- (8.45291,0.69814);
\definecolor{c}{rgb}{0,0,0};
\colorlet{c}{natcomp!70};
\draw [c] (8.46105,0.686896) -- (8.46105,0.686903);
\draw [c] (8.46105,0.686903) -- (8.46105,0.686909);
\draw [c] (8.45291,0.686903) -- (8.46105,0.686903);
\draw [c] (8.46105,0.686903) -- (8.46918,0.686903);
\definecolor{c}{rgb}{0,0,0};
\colorlet{c}{natcomp!70};
\draw [c] (8.47732,0.686902) -- (8.47732,0.686911);
\draw [c] (8.47732,0.686911) -- (8.47732,0.686919);
\draw [c] (8.46918,0.686911) -- (8.47732,0.686911);
\draw [c] (8.47732,0.686911) -- (8.48545,0.686911);
\definecolor{c}{rgb}{0,0,0};
\colorlet{c}{natcomp!70};
\draw [c] (8.49359,0.686915) -- (8.49359,0.712075);
\draw [c] (8.49359,0.712075) -- (8.49359,0.737235);
\draw [c] (8.48545,0.712075) -- (8.49359,0.712075);
\draw [c] (8.49359,0.712075) -- (8.50173,0.712075);
\definecolor{c}{rgb}{0,0,0};
\colorlet{c}{natcomp!70};
\draw [c] (8.50986,0.686899) -- (8.50986,0.686907);
\draw [c] (8.50986,0.686907) -- (8.50986,0.686915);
\draw [c] (8.50173,0.686907) -- (8.50986,0.686907);
\draw [c] (8.50986,0.686907) -- (8.518,0.686907);
\definecolor{c}{rgb}{0,0,0};
\colorlet{c}{natcomp!70};
\draw [c] (8.52614,0.686912) -- (8.52614,0.694113);
\draw [c] (8.52614,0.694113) -- (8.52614,0.701314);
\draw [c] (8.518,0.694113) -- (8.52614,0.694113);
\draw [c] (8.52614,0.694113) -- (8.53427,0.694113);
\definecolor{c}{rgb}{0,0,0};
\colorlet{c}{natcomp!70};
\draw [c] (8.54241,0.70362) -- (8.54241,0.720773);
\draw [c] (8.54241,0.720773) -- (8.54241,0.737927);
\draw [c] (8.53427,0.720773) -- (8.54241,0.720773);
\draw [c] (8.54241,0.720773) -- (8.55055,0.720773);
\definecolor{c}{rgb}{0,0,0};
\colorlet{c}{natcomp!70};
\draw [c] (8.55868,0.686902) -- (8.55868,0.697697);
\draw [c] (8.55868,0.697697) -- (8.55868,0.708492);
\draw [c] (8.55055,0.697697) -- (8.55868,0.697697);
\draw [c] (8.55868,0.697697) -- (8.56682,0.697697);
\definecolor{c}{rgb}{0,0,0};
\colorlet{c}{natcomp!70};
\draw [c] (8.57495,0.692719) -- (8.57495,0.70709);
\draw [c] (8.57495,0.70709) -- (8.57495,0.721461);
\draw [c] (8.56682,0.70709) -- (8.57495,0.70709);
\draw [c] (8.57495,0.70709) -- (8.58309,0.70709);
\definecolor{c}{rgb}{0,0,0};
\colorlet{c}{natcomp!70};
\draw [c] (8.59123,0.686902) -- (8.59123,0.686909);
\draw [c] (8.59123,0.686909) -- (8.59123,0.686917);
\draw [c] (8.58309,0.686909) -- (8.59123,0.686909);
\draw [c] (8.59123,0.686909) -- (8.59936,0.686909);
\definecolor{c}{rgb}{0,0,0};
\colorlet{c}{natcomp!70};
\draw [c] (8.6075,0.686894) -- (8.6075,0.686897);
\draw [c] (8.6075,0.686897) -- (8.6075,0.686901);
\draw [c] (8.59936,0.686897) -- (8.6075,0.686897);
\draw [c] (8.6075,0.686897) -- (8.61564,0.686897);
\definecolor{c}{rgb}{0,0,0};
\colorlet{c}{natcomp!70};
\draw [c] (8.62377,0.686903) -- (8.62377,0.686912);
\draw [c] (8.62377,0.686912) -- (8.62377,0.686922);
\draw [c] (8.61564,0.686912) -- (8.62377,0.686912);
\draw [c] (8.62377,0.686912) -- (8.63191,0.686912);
\definecolor{c}{rgb}{0,0,0};
\colorlet{c}{natcomp!70};
\draw [c] (8.64005,0.706296) -- (8.64005,0.725795);
\draw [c] (8.64005,0.725795) -- (8.64005,0.745294);
\draw [c] (8.63191,0.725795) -- (8.64005,0.725795);
\draw [c] (8.64005,0.725795) -- (8.64818,0.725795);
\definecolor{c}{rgb}{0,0,0};
\colorlet{c}{natcomp!70};
\draw [c] (8.65632,0.686902) -- (8.65632,0.697676);
\draw [c] (8.65632,0.697676) -- (8.65632,0.708449);
\draw [c] (8.64818,0.697676) -- (8.65632,0.697676);
\draw [c] (8.65632,0.697676) -- (8.66445,0.697676);
\definecolor{c}{rgb}{0,0,0};
\colorlet{c}{natcomp!70};
\draw [c] (8.67259,0.697669) -- (8.67259,0.712756);
\draw [c] (8.67259,0.712756) -- (8.67259,0.727844);
\draw [c] (8.66445,0.712756) -- (8.67259,0.712756);
\draw [c] (8.67259,0.712756) -- (8.68073,0.712756);
\definecolor{c}{rgb}{0,0,0};
\colorlet{c}{natcomp!70};
\draw [c] (8.68886,0.686917) -- (8.68886,0.695056);
\draw [c] (8.68886,0.695056) -- (8.68886,0.703195);
\draw [c] (8.68073,0.695056) -- (8.68886,0.695056);
\draw [c] (8.68886,0.695056) -- (8.697,0.695056);
\definecolor{c}{rgb}{0,0,0};
\colorlet{c}{natcomp!70};
\draw [c] (8.70514,0.686894) -- (8.70514,0.686898);
\draw [c] (8.70514,0.686898) -- (8.70514,0.686902);
\draw [c] (8.697,0.686898) -- (8.70514,0.686898);
\draw [c] (8.70514,0.686898) -- (8.71327,0.686898);
\definecolor{c}{rgb}{0,0,0};
\colorlet{c}{natcomp!70};
\draw [c] (8.72141,0.686896) -- (8.72141,0.686903);
\draw [c] (8.72141,0.686903) -- (8.72141,0.686909);
\draw [c] (8.71327,0.686903) -- (8.72141,0.686903);
\draw [c] (8.72141,0.686903) -- (8.72955,0.686903);
\definecolor{c}{rgb}{0,0,0};
\colorlet{c}{natcomp!70};
\draw [c] (8.73768,0.6869) -- (8.73768,0.686909);
\draw [c] (8.73768,0.686909) -- (8.73768,0.686917);
\draw [c] (8.72955,0.686909) -- (8.73768,0.686909);
\draw [c] (8.73768,0.686909) -- (8.74582,0.686909);
\definecolor{c}{rgb}{0,0,0};
\colorlet{c}{natcomp!70};
\draw [c] (8.75395,0.686898) -- (8.75395,0.697671);
\draw [c] (8.75395,0.697671) -- (8.75395,0.708445);
\draw [c] (8.74582,0.697671) -- (8.75395,0.697671);
\draw [c] (8.75395,0.697671) -- (8.76209,0.697671);
\definecolor{c}{rgb}{0,0,0};
\colorlet{c}{natcomp!70};
\draw [c] (8.77023,0.686896) -- (8.77023,0.686902);
\draw [c] (8.77023,0.686902) -- (8.77023,0.686908);
\draw [c] (8.76209,0.686902) -- (8.77023,0.686902);
\draw [c] (8.77023,0.686902) -- (8.77836,0.686902);
\definecolor{c}{rgb}{0,0,0};
\colorlet{c}{natcomp!70};
\draw [c] (8.80277,0.686894) -- (8.80277,0.697305);
\draw [c] (8.80277,0.697305) -- (8.80277,0.707716);
\draw [c] (8.79464,0.697305) -- (8.80277,0.697305);
\draw [c] (8.80277,0.697305) -- (8.81091,0.697305);
\definecolor{c}{rgb}{0,0,0};
\colorlet{c}{natcomp!70};
\draw [c] (8.81905,0.686894) -- (8.81905,0.686899);
\draw [c] (8.81905,0.686899) -- (8.81905,0.686904);
\draw [c] (8.81091,0.686899) -- (8.81905,0.686899);
\draw [c] (8.81905,0.686899) -- (8.82718,0.686899);
\definecolor{c}{rgb}{0,0,0};
\colorlet{c}{natcomp!70};
\draw [c] (8.83532,0.686894) -- (8.83532,0.686899);
\draw [c] (8.83532,0.686899) -- (8.83532,0.686904);
\draw [c] (8.82718,0.686899) -- (8.83532,0.686899);
\draw [c] (8.83532,0.686899) -- (8.84345,0.686899);
\definecolor{c}{rgb}{0,0,0};
\colorlet{c}{natcomp!70};
\draw [c] (8.85159,0.686907) -- (8.85159,0.696022);
\draw [c] (8.85159,0.696022) -- (8.85159,0.705137);
\draw [c] (8.84345,0.696022) -- (8.85159,0.696022);
\draw [c] (8.85159,0.696022) -- (8.85973,0.696022);
\definecolor{c}{rgb}{0,0,0};
\colorlet{c}{natcomp!70};
\draw [c] (8.86786,0.686894) -- (8.86786,0.686897);
\draw [c] (8.86786,0.686897) -- (8.86786,0.686901);
\draw [c] (8.85973,0.686897) -- (8.86786,0.686897);
\draw [c] (8.86786,0.686897) -- (8.876,0.686897);
\definecolor{c}{rgb}{0,0,0};
\colorlet{c}{natcomp!70};
\draw [c] (8.88414,0.700826) -- (8.88414,0.720637);
\draw [c] (8.88414,0.720637) -- (8.88414,0.740448);
\draw [c] (8.876,0.720637) -- (8.88414,0.720637);
\draw [c] (8.88414,0.720637) -- (8.89227,0.720637);
\definecolor{c}{rgb}{0,0,0};
\colorlet{c}{natcomp!70};
\draw [c] (8.90041,0.686894) -- (8.90041,0.686897);
\draw [c] (8.90041,0.686897) -- (8.90041,0.686901);
\draw [c] (8.89227,0.686897) -- (8.90041,0.686897);
\draw [c] (8.90041,0.686897) -- (8.90855,0.686897);
\definecolor{c}{rgb}{0,0,0};
\colorlet{c}{natcomp!70};
\draw [c] (8.91668,0.722516) -- (8.91668,0.755184);
\draw [c] (8.91668,0.755184) -- (8.91668,0.787852);
\draw [c] (8.90855,0.755184) -- (8.91668,0.755184);
\draw [c] (8.91668,0.755184) -- (8.92482,0.755184);
\definecolor{c}{rgb}{0,0,0};
\colorlet{c}{natcomp!70};
\draw [c] (8.93295,0.6869) -- (8.93295,0.694622);
\draw [c] (8.93295,0.694622) -- (8.93295,0.702344);
\draw [c] (8.92482,0.694622) -- (8.93295,0.694622);
\draw [c] (8.93295,0.694622) -- (8.94109,0.694622);
\definecolor{c}{rgb}{0,0,0};
\colorlet{c}{natcomp!70};
\draw [c] (8.94923,0.686894) -- (8.94923,0.686898);
\draw [c] (8.94923,0.686898) -- (8.94923,0.686902);
\draw [c] (8.94109,0.686898) -- (8.94923,0.686898);
\draw [c] (8.94923,0.686898) -- (8.95736,0.686898);
\definecolor{c}{rgb}{0,0,0};
\colorlet{c}{natcomp!70};
\draw [c] (8.9655,0.686894) -- (8.9655,0.686898);
\draw [c] (8.9655,0.686898) -- (8.9655,0.686902);
\draw [c] (8.95736,0.686898) -- (8.9655,0.686898);
\draw [c] (8.9655,0.686898) -- (8.97364,0.686898);
\definecolor{c}{rgb}{0,0,0};
\colorlet{c}{natcomp!70};
\draw [c] (8.98177,0.686896) -- (8.98177,0.686902);
\draw [c] (8.98177,0.686902) -- (8.98177,0.686907);
\draw [c] (8.97364,0.686902) -- (8.98177,0.686902);
\draw [c] (8.98177,0.686902) -- (8.98991,0.686902);
\definecolor{c}{rgb}{0,0,0};
\colorlet{c}{natcomp!70};
\draw [c] (8.99805,0.686904) -- (8.99805,0.686922);
\draw [c] (8.99805,0.686922) -- (8.99805,0.686939);
\draw [c] (8.98991,0.686922) -- (8.99805,0.686922);
\draw [c] (8.99805,0.686922) -- (9.00618,0.686922);
\definecolor{c}{rgb}{0,0,0};
\colorlet{c}{natcomp!70};
\draw [c] (9.01432,0.686897) -- (9.01432,0.686903);
\draw [c] (9.01432,0.686903) -- (9.01432,0.68691);
\draw [c] (9.00618,0.686903) -- (9.01432,0.686903);
\draw [c] (9.01432,0.686903) -- (9.02245,0.686903);
\definecolor{c}{rgb}{0,0,0};
\colorlet{c}{natcomp!70};
\draw [c] (9.04686,0.708993) -- (9.04686,0.731658);
\draw [c] (9.04686,0.731658) -- (9.04686,0.754323);
\draw [c] (9.03873,0.731658) -- (9.04686,0.731658);
\draw [c] (9.04686,0.731658) -- (9.055,0.731658);
\definecolor{c}{rgb}{0,0,0};
\colorlet{c}{natcomp!70};
\draw [c] (9.07941,0.692085) -- (9.07941,0.704618);
\draw [c] (9.07941,0.704618) -- (9.07941,0.717151);
\draw [c] (9.07127,0.704618) -- (9.07941,0.704618);
\draw [c] (9.07941,0.704618) -- (9.08755,0.704618);
\definecolor{c}{rgb}{0,0,0};
\colorlet{c}{natcomp!70};
\draw [c] (9.09568,0.686897) -- (9.09568,0.695861);
\draw [c] (9.09568,0.695861) -- (9.09568,0.704825);
\draw [c] (9.08755,0.695861) -- (9.09568,0.695861);
\draw [c] (9.09568,0.695861) -- (9.10382,0.695861);
\definecolor{c}{rgb}{0,0,0};
\colorlet{c}{natcomp!70};
\draw [c] (9.12823,0.686894) -- (9.12823,0.686897);
\draw [c] (9.12823,0.686897) -- (9.12823,0.686901);
\draw [c] (9.12009,0.686897) -- (9.12823,0.686897);
\draw [c] (9.12823,0.686897) -- (9.13636,0.686897);
\definecolor{c}{rgb}{0,0,0};
\colorlet{c}{natcomp!70};
\draw [c] (9.1445,0.686899) -- (9.1445,0.696014);
\draw [c] (9.1445,0.696014) -- (9.1445,0.70513);
\draw [c] (9.13636,0.696014) -- (9.1445,0.696014);
\draw [c] (9.1445,0.696014) -- (9.15264,0.696014);
\definecolor{c}{rgb}{0,0,0};
\colorlet{c}{natcomp!70};
\draw [c] (9.16077,0.686897) -- (9.16077,0.686903);
\draw [c] (9.16077,0.686903) -- (9.16077,0.68691);
\draw [c] (9.15264,0.686903) -- (9.16077,0.686903);
\draw [c] (9.16077,0.686903) -- (9.16891,0.686903);
\definecolor{c}{rgb}{0,0,0};
\colorlet{c}{natcomp!70};
\draw [c] (9.17705,0.686894) -- (9.17705,0.705403);
\draw [c] (9.17705,0.705403) -- (9.17705,0.723913);
\draw [c] (9.16891,0.705403) -- (9.17705,0.705403);
\draw [c] (9.17705,0.705403) -- (9.18518,0.705403);
\definecolor{c}{rgb}{0,0,0};
\colorlet{c}{natcomp!70};
\draw [c] (9.19332,0.686897) -- (9.19332,0.686906);
\draw [c] (9.19332,0.686906) -- (9.19332,0.686914);
\draw [c] (9.18518,0.686906) -- (9.19332,0.686906);
\draw [c] (9.19332,0.686906) -- (9.20145,0.686906);
\definecolor{c}{rgb}{0,0,0};
\colorlet{c}{natcomp!70};
\draw [c] (9.20959,0.686906) -- (9.20959,0.694628);
\draw [c] (9.20959,0.694628) -- (9.20959,0.70235);
\draw [c] (9.20145,0.694628) -- (9.20959,0.694628);
\draw [c] (9.20959,0.694628) -- (9.21773,0.694628);
\definecolor{c}{rgb}{0,0,0};
\colorlet{c}{natcomp!70};
\draw [c] (9.22586,0.686898) -- (9.22586,0.697671);
\draw [c] (9.22586,0.697671) -- (9.22586,0.708444);
\draw [c] (9.21773,0.697671) -- (9.22586,0.697671);
\draw [c] (9.22586,0.697671) -- (9.234,0.697671);
\definecolor{c}{rgb}{0,0,0};
\colorlet{c}{natcomp!70};
\draw [c] (9.24214,0.697919) -- (9.24214,0.713014);
\draw [c] (9.24214,0.713014) -- (9.24214,0.72811);
\draw [c] (9.234,0.713014) -- (9.24214,0.713014);
\draw [c] (9.24214,0.713014) -- (9.25027,0.713014);
\definecolor{c}{rgb}{0,0,0};
\colorlet{c}{natcomp!70};
\draw [c] (9.25841,0.686896) -- (9.25841,0.686902);
\draw [c] (9.25841,0.686902) -- (9.25841,0.686908);
\draw [c] (9.25027,0.686902) -- (9.25841,0.686902);
\draw [c] (9.25841,0.686902) -- (9.26655,0.686902);
\definecolor{c}{rgb}{0,0,0};
\colorlet{c}{natcomp!70};
\draw [c] (9.27468,0.686897) -- (9.27468,0.686903);
\draw [c] (9.27468,0.686903) -- (9.27468,0.68691);
\draw [c] (9.26655,0.686903) -- (9.27468,0.686903);
\draw [c] (9.27468,0.686903) -- (9.28282,0.686903);
\definecolor{c}{rgb}{0,0,0};
\colorlet{c}{natcomp!70};
\draw [c] (9.30723,0.691421) -- (9.30723,0.702341);
\draw [c] (9.30723,0.702341) -- (9.30723,0.713261);
\draw [c] (9.29909,0.702341) -- (9.30723,0.702341);
\draw [c] (9.30723,0.702341) -- (9.31536,0.702341);
\definecolor{c}{rgb}{0,0,0};
\colorlet{c}{natcomp!70};
\draw [c] (9.33977,0.686902) -- (9.33977,0.69504);
\draw [c] (9.33977,0.69504) -- (9.33977,0.703179);
\draw [c] (9.33164,0.69504) -- (9.33977,0.69504);
\draw [c] (9.33977,0.69504) -- (9.34791,0.69504);
\definecolor{c}{rgb}{0,0,0};
\colorlet{c}{natcomp!70};
\draw [c] (9.35605,0.693227) -- (9.35605,0.708478);
\draw [c] (9.35605,0.708478) -- (9.35605,0.723729);
\draw [c] (9.34791,0.708478) -- (9.35605,0.708478);
\draw [c] (9.35605,0.708478) -- (9.36418,0.708478);
\definecolor{c}{rgb}{0,0,0};
\colorlet{c}{natcomp!70};
\draw [c] (9.37232,0.686896) -- (9.37232,0.686902);
\draw [c] (9.37232,0.686902) -- (9.37232,0.686908);
\draw [c] (9.36418,0.686902) -- (9.37232,0.686902);
\draw [c] (9.37232,0.686902) -- (9.38046,0.686902);
\definecolor{c}{rgb}{0,0,0};
\colorlet{c}{natcomp!70};
\draw [c] (9.38859,0.686897) -- (9.38859,0.696013);
\draw [c] (9.38859,0.696013) -- (9.38859,0.705128);
\draw [c] (9.38046,0.696013) -- (9.38859,0.696013);
\draw [c] (9.38859,0.696013) -- (9.39673,0.696013);
\definecolor{c}{rgb}{0,0,0};
\colorlet{c}{natcomp!70};
\draw [c] (9.43741,0.686902) -- (9.43741,0.697312);
\draw [c] (9.43741,0.697312) -- (9.43741,0.707723);
\draw [c] (9.42927,0.697312) -- (9.43741,0.697312);
\draw [c] (9.43741,0.697312) -- (9.44555,0.697312);
\definecolor{c}{rgb}{0,0,0};
\colorlet{c}{natcomp!70};
\draw [c] (9.45368,0.686897) -- (9.45368,0.686903);
\draw [c] (9.45368,0.686903) -- (9.45368,0.68691);
\draw [c] (9.44555,0.686903) -- (9.45368,0.686903);
\draw [c] (9.45368,0.686903) -- (9.46182,0.686903);
\definecolor{c}{rgb}{0,0,0};
\colorlet{c}{natcomp!70};
\draw [c] (9.46995,0.692681) -- (9.46995,0.706988);
\draw [c] (9.46995,0.706988) -- (9.46995,0.721296);
\draw [c] (9.46182,0.706988) -- (9.46995,0.706988);
\draw [c] (9.46995,0.706988) -- (9.47809,0.706988);
\definecolor{c}{rgb}{0,0,0};
\colorlet{c}{natcomp!70};
\draw [c] (9.48623,0.686894) -- (9.48623,0.686903);
\draw [c] (9.48623,0.686903) -- (9.48623,0.686911);
\draw [c] (9.47809,0.686903) -- (9.48623,0.686903);
\draw [c] (9.48623,0.686903) -- (9.49436,0.686903);
\definecolor{c}{rgb}{0,0,0};
\colorlet{c}{natcomp!70};
\draw [c] (9.5025,0.686898) -- (9.5025,0.696793);
\draw [c] (9.5025,0.696793) -- (9.5025,0.706687);
\draw [c] (9.49436,0.696793) -- (9.5025,0.696793);
\draw [c] (9.5025,0.696793) -- (9.51064,0.696793);
\definecolor{c}{rgb}{0,0,0};
\colorlet{c}{natcomp!70};
\draw [c] (9.51877,0.686894) -- (9.51877,0.697689);
\draw [c] (9.51877,0.697689) -- (9.51877,0.708483);
\draw [c] (9.51064,0.697689) -- (9.51877,0.697689);
\draw [c] (9.51877,0.697689) -- (9.52691,0.697689);
\definecolor{c}{rgb}{0,0,0};
\colorlet{c}{natcomp!70};
\draw [c] (9.53505,0.686896) -- (9.53505,0.686901);
\draw [c] (9.53505,0.686901) -- (9.53505,0.686906);
\draw [c] (9.52691,0.686901) -- (9.53505,0.686901);
\draw [c] (9.53505,0.686901) -- (9.54318,0.686901);
\definecolor{c}{rgb}{0,0,0};
\colorlet{c}{natcomp!70};
\draw [c] (9.55132,0.686896) -- (9.55132,0.686902);
\draw [c] (9.55132,0.686902) -- (9.55132,0.686908);
\draw [c] (9.54318,0.686902) -- (9.55132,0.686902);
\draw [c] (9.55132,0.686902) -- (9.55945,0.686902);
\definecolor{c}{rgb}{0,0,0};
\colorlet{c}{natcomp!70};
\draw [c] (9.56759,0.686894) -- (9.56759,0.696788);
\draw [c] (9.56759,0.696788) -- (9.56759,0.706683);
\draw [c] (9.55945,0.696788) -- (9.56759,0.696788);
\draw [c] (9.56759,0.696788) -- (9.57573,0.696788);
\definecolor{c}{rgb}{0,0,0};
\colorlet{c}{natcomp!70};
\draw [c] (9.58386,0.686894) -- (9.58386,0.686898);
\draw [c] (9.58386,0.686898) -- (9.58386,0.686902);
\draw [c] (9.57573,0.686898) -- (9.58386,0.686898);
\draw [c] (9.58386,0.686898) -- (9.592,0.686898);
\definecolor{c}{rgb}{0,0,0};
\colorlet{c}{natcomp!70};
\draw [c] (9.60014,0.686899) -- (9.60014,0.695761);
\draw [c] (9.60014,0.695761) -- (9.60014,0.704623);
\draw [c] (9.592,0.695761) -- (9.60014,0.695761);
\draw [c] (9.60014,0.695761) -- (9.60827,0.695761);
\definecolor{c}{rgb}{0,0,0};
\colorlet{c}{natcomp!70};
\draw [c] (9.61641,0.691959) -- (9.61641,0.70451);
\draw [c] (9.61641,0.70451) -- (9.61641,0.717061);
\draw [c] (9.60827,0.70451) -- (9.61641,0.70451);
\draw [c] (9.61641,0.70451) -- (9.62455,0.70451);
\definecolor{c}{rgb}{0,0,0};
\colorlet{c}{natcomp!70};
\draw [c] (9.64895,0.686899) -- (9.64895,0.686907);
\draw [c] (9.64895,0.686907) -- (9.64895,0.686914);
\draw [c] (9.64082,0.686907) -- (9.64895,0.686907);
\draw [c] (9.64895,0.686907) -- (9.65709,0.686907);
\definecolor{c}{rgb}{0,0,0};
\colorlet{c}{natcomp!70};
\draw [c] (9.66523,0.686899) -- (9.66523,0.686907);
\draw [c] (9.66523,0.686907) -- (9.66523,0.686914);
\draw [c] (9.65709,0.686907) -- (9.66523,0.686907);
\draw [c] (9.66523,0.686907) -- (9.67336,0.686907);
\definecolor{c}{rgb}{0,0,0};
\colorlet{c}{natcomp!70};
\draw [c] (9.6815,0.686894) -- (9.6815,0.686897);
\draw [c] (9.6815,0.686897) -- (9.6815,0.686901);
\draw [c] (9.67336,0.686897) -- (9.6815,0.686897);
\draw [c] (9.6815,0.686897) -- (9.68964,0.686897);
\definecolor{c}{rgb}{0,0,0};
\colorlet{c}{natcomp!70};
\draw [c] (9.71405,0.692495) -- (9.71405,0.706167);
\draw [c] (9.71405,0.706167) -- (9.71405,0.719839);
\draw [c] (9.70591,0.706167) -- (9.71405,0.706167);
\draw [c] (9.71405,0.706167) -- (9.72218,0.706167);
\definecolor{c}{rgb}{0,0,0};
\colorlet{c}{natcomp!70};
\draw [c] (9.74659,0.686894) -- (9.74659,0.686897);
\draw [c] (9.74659,0.686897) -- (9.74659,0.686901);
\draw [c] (9.73845,0.686897) -- (9.74659,0.686897);
\draw [c] (9.74659,0.686897) -- (9.75473,0.686897);
\definecolor{c}{rgb}{0,0,0};
\colorlet{c}{natcomp!70};
\draw [c] (9.76286,0.686896) -- (9.76286,0.686901);
\draw [c] (9.76286,0.686901) -- (9.76286,0.686906);
\draw [c] (9.75473,0.686901) -- (9.76286,0.686901);
\draw [c] (9.76286,0.686901) -- (9.771,0.686901);
\definecolor{c}{rgb}{0,0,0};
\colorlet{c}{natcomp!70};
\draw [c] (9.77914,0.686896) -- (9.77914,0.686902);
\draw [c] (9.77914,0.686902) -- (9.77914,0.686908);
\draw [c] (9.771,0.686902) -- (9.77914,0.686902);
\draw [c] (9.77914,0.686902) -- (9.78727,0.686902);
\definecolor{c}{rgb}{0,0,0};
\colorlet{c}{natcomp!70};
\draw [c] (9.79541,0.686898) -- (9.79541,0.696014);
\draw [c] (9.79541,0.696014) -- (9.79541,0.705129);
\draw [c] (9.78727,0.696014) -- (9.79541,0.696014);
\draw [c] (9.79541,0.696014) -- (9.80354,0.696014);
\definecolor{c}{rgb}{0,0,0};
\colorlet{c}{natcomp!70};
\draw [c] (9.81168,0.686894) -- (9.81168,0.686899);
\draw [c] (9.81168,0.686899) -- (9.81168,0.686903);
\draw [c] (9.80354,0.686899) -- (9.81168,0.686899);
\draw [c] (9.81168,0.686899) -- (9.81982,0.686899);
\definecolor{c}{rgb}{0,0,0};
\colorlet{c}{natcomp!70};
\draw [c] (9.82795,0.692367) -- (9.82795,0.70565);
\draw [c] (9.82795,0.70565) -- (9.82795,0.718933);
\draw [c] (9.81982,0.70565) -- (9.82795,0.70565);
\draw [c] (9.82795,0.70565) -- (9.83609,0.70565);
\definecolor{c}{rgb}{0,0,0};
\colorlet{c}{natcomp!70};
\draw [c] (9.84423,0.686894) -- (9.84423,0.686898);
\draw [c] (9.84423,0.686898) -- (9.84423,0.686902);
\draw [c] (9.83609,0.686898) -- (9.84423,0.686898);
\draw [c] (9.84423,0.686898) -- (9.85236,0.686898);
\definecolor{c}{rgb}{0,0,0};
\colorlet{c}{natcomp!70};
\draw [c] (9.8605,0.686894) -- (9.8605,0.686901);
\draw [c] (9.8605,0.686901) -- (9.8605,0.686908);
\draw [c] (9.85236,0.686901) -- (9.8605,0.686901);
\draw [c] (9.8605,0.686901) -- (9.86864,0.686901);
\definecolor{c}{rgb}{0,0,0};
\colorlet{c}{natcomp!70};
\draw [c] (9.87677,0.686894) -- (9.87677,0.686899);
\draw [c] (9.87677,0.686899) -- (9.87677,0.686903);
\draw [c] (9.86864,0.686899) -- (9.87677,0.686899);
\draw [c] (9.87677,0.686899) -- (9.88491,0.686899);
\definecolor{c}{rgb}{0,0,0};
\colorlet{c}{natcomp!70};
\draw [c] (9.89305,0.691538) -- (9.89305,0.702957);
\draw [c] (9.89305,0.702957) -- (9.89305,0.714375);
\draw [c] (9.88491,0.702957) -- (9.89305,0.702957);
\draw [c] (9.89305,0.702957) -- (9.90118,0.702957);
\definecolor{c}{rgb}{0,0,0};
\colorlet{c}{natcomp!70};
\draw [c] (9.90932,0.692367) -- (9.90932,0.70565);
\draw [c] (9.90932,0.70565) -- (9.90932,0.718933);
\draw [c] (9.90118,0.70565) -- (9.90932,0.70565);
\draw [c] (9.90932,0.70565) -- (9.91745,0.70565);
\definecolor{c}{rgb}{0,0,0};
\colorlet{c}{natcomp!70};
\draw [c] (9.94186,0.686899) -- (9.94186,0.686907);
\draw [c] (9.94186,0.686907) -- (9.94186,0.686915);
\draw [c] (9.93373,0.686907) -- (9.94186,0.686907);
\draw [c] (9.94186,0.686907) -- (9.95,0.686907);
\definecolor{c}{rgb}{0,0,0};
\colorlet{c}{natcomp!70};
\draw [c] (1.04068,0.686894) -- (1.04068,0.6869);
\draw [c] (1.04068,0.6869) -- (1.04068,0.686907);
\draw [c] (1.03255,0.6869) -- (1.04068,0.6869);
\draw [c] (1.04068,0.6869) -- (1.04882,0.6869);
\definecolor{c}{rgb}{0,0,0};
\colorlet{c}{natcomp!70};
\draw [c] (1.10577,0.686894) -- (1.10577,0.705403);
\draw [c] (1.10577,0.705403) -- (1.10577,0.723913);
\draw [c] (1.09764,0.705403) -- (1.10577,0.705403);
\draw [c] (1.10577,0.705403) -- (1.11391,0.705403);
\definecolor{c}{rgb}{0,0,0};
\colorlet{c}{natcomp!70};
\draw [c] (1.12205,0.6869) -- (1.12205,0.708668);
\draw [c] (1.12205,0.708668) -- (1.12205,0.730436);
\draw [c] (1.11391,0.708668) -- (1.12205,0.708668);
\draw [c] (1.12205,0.708668) -- (1.13018,0.708668);
\definecolor{c}{rgb}{0,0,0};
\colorlet{c}{natcomp!70};
\draw [c] (1.13832,0.686894) -- (1.13832,0.686899);
\draw [c] (1.13832,0.686899) -- (1.13832,0.686904);
\draw [c] (1.13018,0.686899) -- (1.13832,0.686899);
\draw [c] (1.13832,0.686899) -- (1.14645,0.686899);
\definecolor{c}{rgb}{0,0,0};
\colorlet{c}{natcomp!70};
\draw [c] (1.15459,0.686907) -- (1.15459,0.695046);
\draw [c] (1.15459,0.695046) -- (1.15459,0.703185);
\draw [c] (1.14645,0.695046) -- (1.15459,0.695046);
\draw [c] (1.15459,0.695046) -- (1.16273,0.695046);
\definecolor{c}{rgb}{0,0,0};
\colorlet{c}{natcomp!70};
\draw [c] (1.17086,0.686894) -- (1.17086,0.686904);
\draw [c] (1.17086,0.686904) -- (1.17086,0.686914);
\draw [c] (1.16273,0.686904) -- (1.17086,0.686904);
\draw [c] (1.17086,0.686904) -- (1.179,0.686904);
\definecolor{c}{rgb}{0,0,0};
\colorlet{c}{natcomp!70};
\draw [c] (1.18714,0.686897) -- (1.18714,0.686904);
\draw [c] (1.18714,0.686904) -- (1.18714,0.686911);
\draw [c] (1.179,0.686904) -- (1.18714,0.686904);
\draw [c] (1.18714,0.686904) -- (1.19527,0.686904);
\definecolor{c}{rgb}{0,0,0};
\colorlet{c}{natcomp!70};
\draw [c] (1.20341,0.686906) -- (1.20341,0.705415);
\draw [c] (1.20341,0.705415) -- (1.20341,0.723925);
\draw [c] (1.19527,0.705415) -- (1.20341,0.705415);
\draw [c] (1.20341,0.705415) -- (1.21155,0.705415);
\definecolor{c}{rgb}{0,0,0};
\colorlet{c}{natcomp!70};
\draw [c] (1.21968,0.686898) -- (1.21968,0.68691);
\draw [c] (1.21968,0.68691) -- (1.21968,0.686923);
\draw [c] (1.21155,0.68691) -- (1.21968,0.68691);
\draw [c] (1.21968,0.68691) -- (1.22782,0.68691);
\definecolor{c}{rgb}{0,0,0};
\colorlet{c}{natcomp!70};
\draw [c] (1.25223,0.695733) -- (1.25223,0.723286);
\draw [c] (1.25223,0.723286) -- (1.25223,0.75084);
\draw [c] (1.24409,0.723286) -- (1.25223,0.723286);
\draw [c] (1.25223,0.723286) -- (1.26036,0.723286);
\definecolor{c}{rgb}{0,0,0};
\colorlet{c}{natcomp!70};
\draw [c] (1.2685,0.686901) -- (1.2685,0.686911);
\draw [c] (1.2685,0.686911) -- (1.2685,0.686922);
\draw [c] (1.26036,0.686911) -- (1.2685,0.686911);
\draw [c] (1.2685,0.686911) -- (1.27664,0.686911);
\definecolor{c}{rgb}{0,0,0};
\colorlet{c}{natcomp!70};
\draw [c] (1.28477,0.686898) -- (1.28477,0.686915);
\draw [c] (1.28477,0.686915) -- (1.28477,0.686932);
\draw [c] (1.27664,0.686915) -- (1.28477,0.686915);
\draw [c] (1.28477,0.686915) -- (1.29291,0.686915);
\definecolor{c}{rgb}{0,0,0};
\colorlet{c}{natcomp!70};
\draw [c] (1.31732,0.694427) -- (1.31732,0.721188);
\draw [c] (1.31732,0.721188) -- (1.31732,0.747948);
\draw [c] (1.30918,0.721188) -- (1.31732,0.721188);
\draw [c] (1.31732,0.721188) -- (1.32545,0.721188);
\definecolor{c}{rgb}{0,0,0};
\colorlet{c}{natcomp!70};
\draw [c] (1.34986,0.686899) -- (1.34986,0.69462);
\draw [c] (1.34986,0.69462) -- (1.34986,0.702342);
\draw [c] (1.34173,0.69462) -- (1.34986,0.69462);
\draw [c] (1.34986,0.69462) -- (1.358,0.69462);
\definecolor{c}{rgb}{0,0,0};
\colorlet{c}{natcomp!70};
\draw [c] (1.36614,0.686894) -- (1.36614,0.6869);
\draw [c] (1.36614,0.6869) -- (1.36614,0.686907);
\draw [c] (1.358,0.6869) -- (1.36614,0.6869);
\draw [c] (1.36614,0.6869) -- (1.37427,0.6869);
\definecolor{c}{rgb}{0,0,0};
\colorlet{c}{natcomp!70};
\draw [c] (1.38241,0.68691) -- (1.38241,0.71207);
\draw [c] (1.38241,0.71207) -- (1.38241,0.73723);
\draw [c] (1.37427,0.71207) -- (1.38241,0.71207);
\draw [c] (1.38241,0.71207) -- (1.39055,0.71207);
\definecolor{c}{rgb}{0,0,0};
\colorlet{c}{natcomp!70};
\draw [c] (1.39868,0.696781) -- (1.39868,0.722664);
\draw [c] (1.39868,0.722664) -- (1.39868,0.748546);
\draw [c] (1.39055,0.722664) -- (1.39868,0.722664);
\draw [c] (1.39868,0.722664) -- (1.40682,0.722664);
\definecolor{c}{rgb}{0,0,0};
\colorlet{c}{natcomp!70};
\draw [c] (1.41495,0.686894) -- (1.41495,0.697305);
\draw [c] (1.41495,0.697305) -- (1.41495,0.707716);
\draw [c] (1.40682,0.697305) -- (1.41495,0.697305);
\draw [c] (1.41495,0.697305) -- (1.42309,0.697305);
\definecolor{c}{rgb}{0,0,0};
\colorlet{c}{natcomp!70};
\draw [c] (1.43123,0.6869) -- (1.43123,0.700903);
\draw [c] (1.43123,0.700903) -- (1.43123,0.714905);
\draw [c] (1.42309,0.700903) -- (1.43123,0.700903);
\draw [c] (1.43123,0.700903) -- (1.43936,0.700903);
\definecolor{c}{rgb}{0,0,0};
\colorlet{c}{natcomp!70};
\draw [c] (1.4475,0.686909) -- (1.4475,0.705418);
\draw [c] (1.4475,0.705418) -- (1.4475,0.723928);
\draw [c] (1.43936,0.705418) -- (1.4475,0.705418);
\draw [c] (1.4475,0.705418) -- (1.45564,0.705418);
\definecolor{c}{rgb}{0,0,0};
\colorlet{c}{natcomp!70};
\draw [c] (1.46377,0.686915) -- (1.46377,0.705424);
\draw [c] (1.46377,0.705424) -- (1.46377,0.723933);
\draw [c] (1.45564,0.705424) -- (1.46377,0.705424);
\draw [c] (1.46377,0.705424) -- (1.47191,0.705424);
\definecolor{c}{rgb}{0,0,0};
\colorlet{c}{natcomp!70};
\draw [c] (1.48005,0.6869) -- (1.48005,0.686916);
\draw [c] (1.48005,0.686916) -- (1.48005,0.686932);
\draw [c] (1.47191,0.686916) -- (1.48005,0.686916);
\draw [c] (1.48005,0.686916) -- (1.48818,0.686916);
\definecolor{c}{rgb}{0,0,0};
\colorlet{c}{natcomp!70};
\draw [c] (1.49632,0.686904) -- (1.49632,0.700906);
\draw [c] (1.49632,0.700906) -- (1.49632,0.714908);
\draw [c] (1.48818,0.700906) -- (1.49632,0.700906);
\draw [c] (1.49632,0.700906) -- (1.50445,0.700906);
\definecolor{c}{rgb}{0,0,0};
\colorlet{c}{natcomp!70};
\draw [c] (1.51259,0.686912) -- (1.51259,0.698144);
\draw [c] (1.51259,0.698144) -- (1.51259,0.709377);
\draw [c] (1.50445,0.698144) -- (1.51259,0.698144);
\draw [c] (1.51259,0.698144) -- (1.52073,0.698144);
\definecolor{c}{rgb}{0,0,0};
\colorlet{c}{natcomp!70};
\draw [c] (1.52886,0.686905) -- (1.52886,0.697316);
\draw [c] (1.52886,0.697316) -- (1.52886,0.707727);
\draw [c] (1.52073,0.697316) -- (1.52886,0.697316);
\draw [c] (1.52886,0.697316) -- (1.537,0.697316);
\definecolor{c}{rgb}{0,0,0};
\colorlet{c}{natcomp!70};
\draw [c] (1.54514,0.68691) -- (1.54514,0.686925);
\draw [c] (1.54514,0.686925) -- (1.54514,0.68694);
\draw [c] (1.537,0.686925) -- (1.54514,0.686925);
\draw [c] (1.54514,0.686925) -- (1.55327,0.686925);
\definecolor{c}{rgb}{0,0,0};
\colorlet{c}{natcomp!70};
\draw [c] (1.56141,0.686899) -- (1.56141,0.686912);
\draw [c] (1.56141,0.686912) -- (1.56141,0.686926);
\draw [c] (1.55327,0.686912) -- (1.56141,0.686912);
\draw [c] (1.56141,0.686912) -- (1.56955,0.686912);
\definecolor{c}{rgb}{0,0,0};
\colorlet{c}{natcomp!70};
\draw [c] (1.57768,0.686904) -- (1.57768,0.705413);
\draw [c] (1.57768,0.705413) -- (1.57768,0.723923);
\draw [c] (1.56955,0.705413) -- (1.57768,0.705413);
\draw [c] (1.57768,0.705413) -- (1.58582,0.705413);
\definecolor{c}{rgb}{0,0,0};
\colorlet{c}{natcomp!70};
\draw [c] (1.59395,0.686923) -- (1.59395,0.696817);
\draw [c] (1.59395,0.696817) -- (1.59395,0.706711);
\draw [c] (1.58582,0.696817) -- (1.59395,0.696817);
\draw [c] (1.59395,0.696817) -- (1.60209,0.696817);
\definecolor{c}{rgb}{0,0,0};
\colorlet{c}{natcomp!70};
\draw [c] (1.61023,0.686911) -- (1.61023,0.719865);
\draw [c] (1.61023,0.719865) -- (1.61023,0.752819);
\draw [c] (1.60209,0.719865) -- (1.61023,0.719865);
\draw [c] (1.61023,0.719865) -- (1.61836,0.719865);
\definecolor{c}{rgb}{0,0,0};
\colorlet{c}{natcomp!70};
\draw [c] (1.6265,0.692256) -- (1.6265,0.705147);
\draw [c] (1.6265,0.705147) -- (1.6265,0.718038);
\draw [c] (1.61836,0.705147) -- (1.6265,0.705147);
\draw [c] (1.6265,0.705147) -- (1.63464,0.705147);
\definecolor{c}{rgb}{0,0,0};
\colorlet{c}{natcomp!70};
\draw [c] (1.64277,0.686912) -- (1.64277,0.686932);
\draw [c] (1.64277,0.686932) -- (1.64277,0.686952);
\draw [c] (1.63464,0.686932) -- (1.64277,0.686932);
\draw [c] (1.64277,0.686932) -- (1.65091,0.686932);
\definecolor{c}{rgb}{0,0,0};
\colorlet{c}{natcomp!70};
\draw [c] (1.65905,0.692192) -- (1.65905,0.704906);
\draw [c] (1.65905,0.704906) -- (1.65905,0.717619);
\draw [c] (1.65091,0.704906) -- (1.65905,0.704906);
\draw [c] (1.65905,0.704906) -- (1.66718,0.704906);
\definecolor{c}{rgb}{0,0,0};
\colorlet{c}{natcomp!70};
\draw [c] (1.67532,0.700693) -- (1.67532,0.723056);
\draw [c] (1.67532,0.723056) -- (1.67532,0.74542);
\draw [c] (1.66718,0.723056) -- (1.67532,0.723056);
\draw [c] (1.67532,0.723056) -- (1.68345,0.723056);
\definecolor{c}{rgb}{0,0,0};
\colorlet{c}{natcomp!70};
\draw [c] (1.69159,0.686942) -- (1.69159,0.708221);
\draw [c] (1.69159,0.708221) -- (1.69159,0.729499);
\draw [c] (1.68345,0.708221) -- (1.69159,0.708221);
\draw [c] (1.69159,0.708221) -- (1.69973,0.708221);
\definecolor{c}{rgb}{0,0,0};
\colorlet{c}{natcomp!70};
\draw [c] (1.70786,0.705683) -- (1.70786,0.724842);
\draw [c] (1.70786,0.724842) -- (1.70786,0.744);
\draw [c] (1.69973,0.724842) -- (1.70786,0.724842);
\draw [c] (1.70786,0.724842) -- (1.716,0.724842);
\definecolor{c}{rgb}{0,0,0};
\colorlet{c}{natcomp!70};
\draw [c] (1.72414,0.686939) -- (1.72414,0.700941);
\draw [c] (1.72414,0.700941) -- (1.72414,0.714943);
\draw [c] (1.716,0.700941) -- (1.72414,0.700941);
\draw [c] (1.72414,0.700941) -- (1.73227,0.700941);
\definecolor{c}{rgb}{0,0,0};
\colorlet{c}{natcomp!70};
\draw [c] (1.74041,0.686936) -- (1.74041,0.708215);
\draw [c] (1.74041,0.708215) -- (1.74041,0.729493);
\draw [c] (1.73227,0.708215) -- (1.74041,0.708215);
\draw [c] (1.74041,0.708215) -- (1.74855,0.708215);
\definecolor{c}{rgb}{0,0,0};
\colorlet{c}{natcomp!70};
\draw [c] (1.75668,0.686903) -- (1.75668,0.696018);
\draw [c] (1.75668,0.696018) -- (1.75668,0.705133);
\draw [c] (1.74855,0.696018) -- (1.75668,0.696018);
\draw [c] (1.75668,0.696018) -- (1.76482,0.696018);
\definecolor{c}{rgb}{0,0,0};
\colorlet{c}{natcomp!70};
\draw [c] (1.77295,0.714667) -- (1.77295,0.738024);
\draw [c] (1.77295,0.738024) -- (1.77295,0.761381);
\draw [c] (1.76482,0.738024) -- (1.77295,0.738024);
\draw [c] (1.77295,0.738024) -- (1.78109,0.738024);
\definecolor{c}{rgb}{0,0,0};
\colorlet{c}{natcomp!70};
\draw [c] (1.78923,0.713535) -- (1.78923,0.735176);
\draw [c] (1.78923,0.735176) -- (1.78923,0.756817);
\draw [c] (1.78109,0.735176) -- (1.78923,0.735176);
\draw [c] (1.78923,0.735176) -- (1.79736,0.735176);
\definecolor{c}{rgb}{0,0,0};
\colorlet{c}{natcomp!70};
\draw [c] (1.8055,0.706706) -- (1.8055,0.727668);
\draw [c] (1.8055,0.727668) -- (1.8055,0.748629);
\draw [c] (1.79736,0.727668) -- (1.8055,0.727668);
\draw [c] (1.8055,0.727668) -- (1.81364,0.727668);
\definecolor{c}{rgb}{0,0,0};
\colorlet{c}{natcomp!70};
\draw [c] (1.82177,0.71253) -- (1.82177,0.7417);
\draw [c] (1.82177,0.7417) -- (1.82177,0.77087);
\draw [c] (1.81364,0.7417) -- (1.82177,0.7417);
\draw [c] (1.82177,0.7417) -- (1.82991,0.7417);
\definecolor{c}{rgb}{0,0,0};
\colorlet{c}{natcomp!70};
\draw [c] (1.83805,0.686939) -- (1.83805,0.698172);
\draw [c] (1.83805,0.698172) -- (1.83805,0.709404);
\draw [c] (1.82991,0.698172) -- (1.83805,0.698172);
\draw [c] (1.83805,0.698172) -- (1.84618,0.698172);
\definecolor{c}{rgb}{0,0,0};
\colorlet{c}{natcomp!70};
\draw [c] (1.85432,0.693926) -- (1.85432,0.714558);
\draw [c] (1.85432,0.714558) -- (1.85432,0.73519);
\draw [c] (1.84618,0.714558) -- (1.85432,0.714558);
\draw [c] (1.85432,0.714558) -- (1.86245,0.714558);
\definecolor{c}{rgb}{0,0,0};
\colorlet{c}{natcomp!70};
\draw [c] (1.87059,0.702794) -- (1.87059,0.726598);
\draw [c] (1.87059,0.726598) -- (1.87059,0.750403);
\draw [c] (1.86245,0.726598) -- (1.87059,0.726598);
\draw [c] (1.87059,0.726598) -- (1.87873,0.726598);
\definecolor{c}{rgb}{0,0,0};
\colorlet{c}{natcomp!70};
\draw [c] (1.88686,0.723513) -- (1.88686,0.749566);
\draw [c] (1.88686,0.749566) -- (1.88686,0.775619);
\draw [c] (1.87873,0.749566) -- (1.88686,0.749566);
\draw [c] (1.88686,0.749566) -- (1.895,0.749566);
\definecolor{c}{rgb}{0,0,0};
\colorlet{c}{natcomp!70};
\draw [c] (1.90314,0.703651) -- (1.90314,0.728586);
\draw [c] (1.90314,0.728586) -- (1.90314,0.753521);
\draw [c] (1.895,0.728586) -- (1.90314,0.728586);
\draw [c] (1.90314,0.728586) -- (1.91127,0.728586);
\definecolor{c}{rgb}{0,0,0};
\colorlet{c}{natcomp!70};
\draw [c] (1.91941,0.735552) -- (1.91941,0.762435);
\draw [c] (1.91941,0.762435) -- (1.91941,0.789317);
\draw [c] (1.91127,0.762435) -- (1.91941,0.762435);
\draw [c] (1.91941,0.762435) -- (1.92755,0.762435);
\definecolor{c}{rgb}{0,0,0};
\colorlet{c}{natcomp!70};
\draw [c] (1.93568,0.705159) -- (1.93568,0.723496);
\draw [c] (1.93568,0.723496) -- (1.93568,0.741834);
\draw [c] (1.92755,0.723496) -- (1.93568,0.723496);
\draw [c] (1.93568,0.723496) -- (1.94382,0.723496);
\definecolor{c}{rgb}{0,0,0};
\colorlet{c}{natcomp!70};
\draw [c] (1.95195,0.728073) -- (1.95195,0.759278);
\draw [c] (1.95195,0.759278) -- (1.95195,0.790483);
\draw [c] (1.94382,0.759278) -- (1.95195,0.759278);
\draw [c] (1.95195,0.759278) -- (1.96009,0.759278);
\definecolor{c}{rgb}{0,0,0};
\colorlet{c}{natcomp!70};
\draw [c] (1.96823,0.721308) -- (1.96823,0.745166);
\draw [c] (1.96823,0.745166) -- (1.96823,0.769024);
\draw [c] (1.96009,0.745166) -- (1.96823,0.745166);
\draw [c] (1.96823,0.745166) -- (1.97636,0.745166);
\definecolor{c}{rgb}{0,0,0};
\colorlet{c}{natcomp!70};
\draw [c] (1.9845,0.717672) -- (1.9845,0.74667);
\draw [c] (1.9845,0.74667) -- (1.9845,0.775667);
\draw [c] (1.97636,0.74667) -- (1.9845,0.74667);
\draw [c] (1.9845,0.74667) -- (1.99264,0.74667);
\definecolor{c}{rgb}{0,0,0};
\colorlet{c}{natcomp!70};
\draw [c] (2.00077,0.754545) -- (2.00077,0.791474);
\draw [c] (2.00077,0.791474) -- (2.00077,0.828402);
\draw [c] (1.99264,0.791474) -- (2.00077,0.791474);
\draw [c] (2.00077,0.791474) -- (2.00891,0.791474);
\definecolor{c}{rgb}{0,0,0};
\colorlet{c}{natcomp!70};
\draw [c] (2.01705,0.812339) -- (2.01705,0.861065);
\draw [c] (2.01705,0.861065) -- (2.01705,0.909792);
\draw [c] (2.00891,0.861065) -- (2.01705,0.861065);
\draw [c] (2.01705,0.861065) -- (2.02518,0.861065);
\definecolor{c}{rgb}{0,0,0};
\colorlet{c}{natcomp!70};
\draw [c] (2.03332,0.756859) -- (2.03332,0.788069);
\draw [c] (2.03332,0.788069) -- (2.03332,0.819279);
\draw [c] (2.02518,0.788069) -- (2.03332,0.788069);
\draw [c] (2.03332,0.788069) -- (2.04145,0.788069);
\definecolor{c}{rgb}{0,0,0};
\colorlet{c}{natcomp!70};
\draw [c] (2.04959,0.733143) -- (2.04959,0.772578);
\draw [c] (2.04959,0.772578) -- (2.04959,0.812014);
\draw [c] (2.04145,0.772578) -- (2.04959,0.772578);
\draw [c] (2.04959,0.772578) -- (2.05773,0.772578);
\definecolor{c}{rgb}{0,0,0};
\colorlet{c}{natcomp!70};
\draw [c] (2.06586,0.734846) -- (2.06586,0.767565);
\draw [c] (2.06586,0.767565) -- (2.06586,0.800284);
\draw [c] (2.05773,0.767565) -- (2.06586,0.767565);
\draw [c] (2.06586,0.767565) -- (2.074,0.767565);
\definecolor{c}{rgb}{0,0,0};
\colorlet{c}{natcomp!70};
\draw [c] (2.08214,0.789195) -- (2.08214,0.827149);
\draw [c] (2.08214,0.827149) -- (2.08214,0.865104);
\draw [c] (2.074,0.827149) -- (2.08214,0.827149);
\draw [c] (2.08214,0.827149) -- (2.09027,0.827149);
\definecolor{c}{rgb}{0,0,0};
\colorlet{c}{natcomp!70};
\draw [c] (2.09841,0.788521) -- (2.09841,0.828396);
\draw [c] (2.09841,0.828396) -- (2.09841,0.868271);
\draw [c] (2.09027,0.828396) -- (2.09841,0.828396);
\draw [c] (2.09841,0.828396) -- (2.10655,0.828396);
\definecolor{c}{rgb}{0,0,0};
\colorlet{c}{natcomp!70};
\draw [c] (2.11468,0.759161) -- (2.11468,0.799535);
\draw [c] (2.11468,0.799535) -- (2.11468,0.839908);
\draw [c] (2.10655,0.799535) -- (2.11468,0.799535);
\draw [c] (2.11468,0.799535) -- (2.12282,0.799535);
\definecolor{c}{rgb}{0,0,0};
\colorlet{c}{natcomp!70};
\draw [c] (2.13095,0.7679) -- (2.13095,0.805702);
\draw [c] (2.13095,0.805702) -- (2.13095,0.843503);
\draw [c] (2.12282,0.805702) -- (2.13095,0.805702);
\draw [c] (2.13095,0.805702) -- (2.13909,0.805702);
\definecolor{c}{rgb}{0,0,0};
\colorlet{c}{natcomp!70};
\draw [c] (2.14723,0.746911) -- (2.14723,0.783535);
\draw [c] (2.14723,0.783535) -- (2.14723,0.820159);
\draw [c] (2.13909,0.783535) -- (2.14723,0.783535);
\draw [c] (2.14723,0.783535) -- (2.15536,0.783535);
\definecolor{c}{rgb}{0,0,0};
\colorlet{c}{natcomp!70};
\draw [c] (2.1635,0.772802) -- (2.1635,0.808728);
\draw [c] (2.1635,0.808728) -- (2.1635,0.844654);
\draw [c] (2.15536,0.808728) -- (2.1635,0.808728);
\draw [c] (2.1635,0.808728) -- (2.17164,0.808728);
\definecolor{c}{rgb}{0,0,0};
\colorlet{c}{natcomp!70};
\draw [c] (2.17977,0.810552) -- (2.17977,0.849296);
\draw [c] (2.17977,0.849296) -- (2.17977,0.888039);
\draw [c] (2.17164,0.849296) -- (2.17977,0.849296);
\draw [c] (2.17977,0.849296) -- (2.18791,0.849296);
\definecolor{c}{rgb}{0,0,0};
\colorlet{c}{natcomp!70};
\draw [c] (2.19605,0.778252) -- (2.19605,0.813555);
\draw [c] (2.19605,0.813555) -- (2.19605,0.848858);
\draw [c] (2.18791,0.813555) -- (2.19605,0.813555);
\draw [c] (2.19605,0.813555) -- (2.20418,0.813555);
\definecolor{c}{rgb}{0,0,0};
\colorlet{c}{natcomp!70};
\draw [c] (2.21232,0.789468) -- (2.21232,0.827907);
\draw [c] (2.21232,0.827907) -- (2.21232,0.866346);
\draw [c] (2.20418,0.827907) -- (2.21232,0.827907);
\draw [c] (2.21232,0.827907) -- (2.22045,0.827907);
\definecolor{c}{rgb}{0,0,0};
\colorlet{c}{natcomp!70};
\draw [c] (2.22859,0.815059) -- (2.22859,0.859924);
\draw [c] (2.22859,0.859924) -- (2.22859,0.904788);
\draw [c] (2.22045,0.859924) -- (2.22859,0.859924);
\draw [c] (2.22859,0.859924) -- (2.23673,0.859924);
\definecolor{c}{rgb}{0,0,0};
\colorlet{c}{natcomp!70};
\draw [c] (2.24486,0.802412) -- (2.24486,0.839797);
\draw [c] (2.24486,0.839797) -- (2.24486,0.877183);
\draw [c] (2.23673,0.839797) -- (2.24486,0.839797);
\draw [c] (2.24486,0.839797) -- (2.253,0.839797);
\definecolor{c}{rgb}{0,0,0};
\colorlet{c}{natcomp!70};
\draw [c] (2.26114,0.840272) -- (2.26114,0.885266);
\draw [c] (2.26114,0.885266) -- (2.26114,0.93026);
\draw [c] (2.253,0.885266) -- (2.26114,0.885266);
\draw [c] (2.26114,0.885266) -- (2.26927,0.885266);
\definecolor{c}{rgb}{0,0,0};
\colorlet{c}{natcomp!70};
\draw [c] (2.27741,0.805072) -- (2.27741,0.848826);
\draw [c] (2.27741,0.848826) -- (2.27741,0.892579);
\draw [c] (2.26927,0.848826) -- (2.27741,0.848826);
\draw [c] (2.27741,0.848826) -- (2.28555,0.848826);
\definecolor{c}{rgb}{0,0,0};
\colorlet{c}{natcomp!70};
\draw [c] (2.29368,0.791504) -- (2.29368,0.826715);
\draw [c] (2.29368,0.826715) -- (2.29368,0.861925);
\draw [c] (2.28555,0.826715) -- (2.29368,0.826715);
\draw [c] (2.29368,0.826715) -- (2.30182,0.826715);
\definecolor{c}{rgb}{0,0,0};
\colorlet{c}{natcomp!70};
\draw [c] (2.30995,0.856209) -- (2.30995,0.906613);
\draw [c] (2.30995,0.906613) -- (2.30995,0.957017);
\draw [c] (2.30182,0.906613) -- (2.30995,0.906613);
\draw [c] (2.30995,0.906613) -- (2.31809,0.906613);
\definecolor{c}{rgb}{0,0,0};
\colorlet{c}{natcomp!70};
\draw [c] (2.32623,0.828277) -- (2.32623,0.869435);
\draw [c] (2.32623,0.869435) -- (2.32623,0.910592);
\draw [c] (2.31809,0.869435) -- (2.32623,0.869435);
\draw [c] (2.32623,0.869435) -- (2.33436,0.869435);
\definecolor{c}{rgb}{0,0,0};
\colorlet{c}{natcomp!70};
\draw [c] (2.3425,0.88625) -- (2.3425,0.939801);
\draw [c] (2.3425,0.939801) -- (2.3425,0.993352);
\draw [c] (2.33436,0.939801) -- (2.3425,0.939801);
\draw [c] (2.3425,0.939801) -- (2.35064,0.939801);
\definecolor{c}{rgb}{0,0,0};
\colorlet{c}{natcomp!70};
\draw [c] (2.35877,0.916275) -- (2.35877,0.969867);
\draw [c] (2.35877,0.969867) -- (2.35877,1.02346);
\draw [c] (2.35064,0.969867) -- (2.35877,0.969867);
\draw [c] (2.35877,0.969867) -- (2.36691,0.969867);
\definecolor{c}{rgb}{0,0,0};
\colorlet{c}{natcomp!70};
\draw [c] (2.37505,0.907296) -- (2.37505,0.962405);
\draw [c] (2.37505,0.962405) -- (2.37505,1.01751);
\draw [c] (2.36691,0.962405) -- (2.37505,0.962405);
\draw [c] (2.37505,0.962405) -- (2.38318,0.962405);
\definecolor{c}{rgb}{0,0,0};
\colorlet{c}{natcomp!70};
\draw [c] (2.39132,0.827878) -- (2.39132,0.86913);
\draw [c] (2.39132,0.86913) -- (2.39132,0.910383);
\draw [c] (2.38318,0.86913) -- (2.39132,0.86913);
\draw [c] (2.39132,0.86913) -- (2.39945,0.86913);
\definecolor{c}{rgb}{0,0,0};
\colorlet{c}{natcomp!70};
\draw [c] (2.40759,0.857178) -- (2.40759,0.910918);
\draw [c] (2.40759,0.910918) -- (2.40759,0.964658);
\draw [c] (2.39945,0.910918) -- (2.40759,0.910918);
\draw [c] (2.40759,0.910918) -- (2.41573,0.910918);
\definecolor{c}{rgb}{0,0,0};
\colorlet{c}{natcomp!70};
\draw [c] (2.42386,0.919688) -- (2.42386,0.974446);
\draw [c] (2.42386,0.974446) -- (2.42386,1.0292);
\draw [c] (2.41573,0.974446) -- (2.42386,0.974446);
\draw [c] (2.42386,0.974446) -- (2.432,0.974446);
\definecolor{c}{rgb}{0,0,0};
\colorlet{c}{natcomp!70};
\draw [c] (2.44014,0.898282) -- (2.44014,0.948641);
\draw [c] (2.44014,0.948641) -- (2.44014,0.999);
\draw [c] (2.432,0.948641) -- (2.44014,0.948641);
\draw [c] (2.44014,0.948641) -- (2.44827,0.948641);
\definecolor{c}{rgb}{0,0,0};
\colorlet{c}{natcomp!70};
\draw [c] (2.45641,0.910429) -- (2.45641,0.968628);
\draw [c] (2.45641,0.968628) -- (2.45641,1.02683);
\draw [c] (2.44827,0.968628) -- (2.45641,0.968628);
\draw [c] (2.45641,0.968628) -- (2.46455,0.968628);
\definecolor{c}{rgb}{0,0,0};
\colorlet{c}{natcomp!70};
\draw [c] (2.47268,0.884339) -- (2.47268,0.931852);
\draw [c] (2.47268,0.931852) -- (2.47268,0.979366);
\draw [c] (2.46455,0.931852) -- (2.47268,0.931852);
\draw [c] (2.47268,0.931852) -- (2.48082,0.931852);
\definecolor{c}{rgb}{0,0,0};
\colorlet{c}{natcomp!70};
\draw [c] (2.48895,0.969757) -- (2.48895,1.0318);
\draw [c] (2.48895,1.0318) -- (2.48895,1.09385);
\draw [c] (2.48082,1.0318) -- (2.48895,1.0318);
\draw [c] (2.48895,1.0318) -- (2.49709,1.0318);
\definecolor{c}{rgb}{0,0,0};
\colorlet{c}{natcomp!70};
\draw [c] (2.50523,0.971435) -- (2.50523,1.03092);
\draw [c] (2.50523,1.03092) -- (2.50523,1.0904);
\draw [c] (2.49709,1.03092) -- (2.50523,1.03092);
\draw [c] (2.50523,1.03092) -- (2.51336,1.03092);
\definecolor{c}{rgb}{0,0,0};
\colorlet{c}{natcomp!70};
\draw [c] (2.5215,0.990909) -- (2.5215,1.05577);
\draw [c] (2.5215,1.05577) -- (2.5215,1.12064);
\draw [c] (2.51336,1.05577) -- (2.5215,1.05577);
\draw [c] (2.5215,1.05577) -- (2.52964,1.05577);
\definecolor{c}{rgb}{0,0,0};
\colorlet{c}{natcomp!70};
\draw [c] (2.53777,1.00227) -- (2.53777,1.06689);
\draw [c] (2.53777,1.06689) -- (2.53777,1.13151);
\draw [c] (2.52964,1.06689) -- (2.53777,1.06689);
\draw [c] (2.53777,1.06689) -- (2.54591,1.06689);
\definecolor{c}{rgb}{0,0,0};
\colorlet{c}{natcomp!70};
\draw [c] (2.55405,1.04669) -- (2.55405,1.11011);
\draw [c] (2.55405,1.11011) -- (2.55405,1.17353);
\draw [c] (2.54591,1.11011) -- (2.55405,1.11011);
\draw [c] (2.55405,1.11011) -- (2.56218,1.11011);
\definecolor{c}{rgb}{0,0,0};
\colorlet{c}{natcomp!70};
\draw [c] (2.57032,0.995602) -- (2.57032,1.06059);
\draw [c] (2.57032,1.06059) -- (2.57032,1.12558);
\draw [c] (2.56218,1.06059) -- (2.57032,1.06059);
\draw [c] (2.57032,1.06059) -- (2.57845,1.06059);
\definecolor{c}{rgb}{0,0,0};
\colorlet{c}{natcomp!70};
\draw [c] (2.58659,0.99475) -- (2.58659,1.05464);
\draw [c] (2.58659,1.05464) -- (2.58659,1.11453);
\draw [c] (2.57845,1.05464) -- (2.58659,1.05464);
\draw [c] (2.58659,1.05464) -- (2.59473,1.05464);
\definecolor{c}{rgb}{0,0,0};
\colorlet{c}{natcomp!70};
\draw [c] (2.60286,1.07289) -- (2.60286,1.14199);
\draw [c] (2.60286,1.14199) -- (2.60286,1.21109);
\draw [c] (2.59473,1.14199) -- (2.60286,1.14199);
\draw [c] (2.60286,1.14199) -- (2.611,1.14199);
\definecolor{c}{rgb}{0,0,0};
\colorlet{c}{natcomp!70};
\draw [c] (2.61914,1.13773) -- (2.61914,1.21239);
\draw [c] (2.61914,1.21239) -- (2.61914,1.28705);
\draw [c] (2.611,1.21239) -- (2.61914,1.21239);
\draw [c] (2.61914,1.21239) -- (2.62727,1.21239);
\definecolor{c}{rgb}{0,0,0};
\colorlet{c}{natcomp!70};
\draw [c] (2.63541,0.996344) -- (2.63541,1.05637);
\draw [c] (2.63541,1.05637) -- (2.63541,1.1164);
\draw [c] (2.62727,1.05637) -- (2.63541,1.05637);
\draw [c] (2.63541,1.05637) -- (2.64355,1.05637);
\definecolor{c}{rgb}{0,0,0};
\colorlet{c}{natcomp!70};
\draw [c] (2.65168,0.988957) -- (2.65168,1.0492);
\draw [c] (2.65168,1.0492) -- (2.65168,1.10944);
\draw [c] (2.64355,1.0492) -- (2.65168,1.0492);
\draw [c] (2.65168,1.0492) -- (2.65982,1.0492);
\definecolor{c}{rgb}{0,0,0};
\colorlet{c}{natcomp!70};
\draw [c] (2.66795,1.1185) -- (2.66795,1.18712);
\draw [c] (2.66795,1.18712) -- (2.66795,1.25574);
\draw [c] (2.65982,1.18712) -- (2.66795,1.18712);
\draw [c] (2.66795,1.18712) -- (2.67609,1.18712);
\definecolor{c}{rgb}{0,0,0};
\colorlet{c}{natcomp!70};
\draw [c] (2.68423,1.11432) -- (2.68423,1.19207);
\draw [c] (2.68423,1.19207) -- (2.68423,1.26981);
\draw [c] (2.67609,1.19207) -- (2.68423,1.19207);
\draw [c] (2.68423,1.19207) -- (2.69236,1.19207);
\definecolor{c}{rgb}{0,0,0};
\colorlet{c}{natcomp!70};
\draw [c] (2.7005,1.21466) -- (2.7005,1.29003);
\draw [c] (2.7005,1.29003) -- (2.7005,1.3654);
\draw [c] (2.69236,1.29003) -- (2.7005,1.29003);
\draw [c] (2.7005,1.29003) -- (2.70864,1.29003);
\definecolor{c}{rgb}{0,0,0};
\colorlet{c}{natcomp!70};
\draw [c] (2.71677,1.10429) -- (2.71677,1.17543);
\draw [c] (2.71677,1.17543) -- (2.71677,1.24658);
\draw [c] (2.70864,1.17543) -- (2.71677,1.17543);
\draw [c] (2.71677,1.17543) -- (2.72491,1.17543);
\definecolor{c}{rgb}{0,0,0};
\colorlet{c}{natcomp!70};
\draw [c] (2.73305,1.12661) -- (2.73305,1.2009);
\draw [c] (2.73305,1.2009) -- (2.73305,1.27519);
\draw [c] (2.72491,1.2009) -- (2.73305,1.2009);
\draw [c] (2.73305,1.2009) -- (2.74118,1.2009);
\definecolor{c}{rgb}{0,0,0};
\colorlet{c}{natcomp!70};
\draw [c] (2.74932,1.06158) -- (2.74932,1.12477);
\draw [c] (2.74932,1.12477) -- (2.74932,1.18796);
\draw [c] (2.74118,1.12477) -- (2.74932,1.12477);
\draw [c] (2.74932,1.12477) -- (2.75745,1.12477);
\definecolor{c}{rgb}{0,0,0};
\colorlet{c}{natcomp!70};
\draw [c] (2.76559,1.18504) -- (2.76559,1.26123);
\draw [c] (2.76559,1.26123) -- (2.76559,1.33741);
\draw [c] (2.75745,1.26123) -- (2.76559,1.26123);
\draw [c] (2.76559,1.26123) -- (2.77373,1.26123);
\definecolor{c}{rgb}{0,0,0};
\colorlet{c}{natcomp!70};
\draw [c] (2.78186,1.23072) -- (2.78186,1.31228);
\draw [c] (2.78186,1.31228) -- (2.78186,1.39384);
\draw [c] (2.77373,1.31228) -- (2.78186,1.31228);
\draw [c] (2.78186,1.31228) -- (2.79,1.31228);
\definecolor{c}{rgb}{0,0,0};
\colorlet{c}{natcomp!70};
\draw [c] (2.79814,1.28934) -- (2.79814,1.37187);
\draw [c] (2.79814,1.37187) -- (2.79814,1.45439);
\draw [c] (2.79,1.37187) -- (2.79814,1.37187);
\draw [c] (2.79814,1.37187) -- (2.80627,1.37187);
\definecolor{c}{rgb}{0,0,0};
\colorlet{c}{natcomp!70};
\draw [c] (2.81441,1.32982) -- (2.81441,1.41625);
\draw [c] (2.81441,1.41625) -- (2.81441,1.50267);
\draw [c] (2.80627,1.41625) -- (2.81441,1.41625);
\draw [c] (2.81441,1.41625) -- (2.82255,1.41625);
\definecolor{c}{rgb}{0,0,0};
\colorlet{c}{natcomp!70};
\draw [c] (2.83068,1.30347) -- (2.83068,1.38681);
\draw [c] (2.83068,1.38681) -- (2.83068,1.47015);
\draw [c] (2.82255,1.38681) -- (2.83068,1.38681);
\draw [c] (2.83068,1.38681) -- (2.83882,1.38681);
\definecolor{c}{rgb}{0,0,0};
\colorlet{c}{natcomp!70};
\draw [c] (2.84695,1.23426) -- (2.84695,1.31216);
\draw [c] (2.84695,1.31216) -- (2.84695,1.39006);
\draw [c] (2.83882,1.31216) -- (2.84695,1.31216);
\draw [c] (2.84695,1.31216) -- (2.85509,1.31216);
\definecolor{c}{rgb}{0,0,0};
\colorlet{c}{natcomp!70};
\draw [c] (2.86323,1.36782) -- (2.86323,1.45635);
\draw [c] (2.86323,1.45635) -- (2.86323,1.54489);
\draw [c] (2.85509,1.45635) -- (2.86323,1.45635);
\draw [c] (2.86323,1.45635) -- (2.87136,1.45635);
\definecolor{c}{rgb}{0,0,0};
\colorlet{c}{natcomp!70};
\draw [c] (2.8795,1.31293) -- (2.8795,1.3965);
\draw [c] (2.8795,1.3965) -- (2.8795,1.48008);
\draw [c] (2.87136,1.3965) -- (2.8795,1.3965);
\draw [c] (2.8795,1.3965) -- (2.88764,1.3965);
\definecolor{c}{rgb}{0,0,0};
\colorlet{c}{natcomp!70};
\draw [c] (2.89577,1.42357) -- (2.89577,1.51525);
\draw [c] (2.89577,1.51525) -- (2.89577,1.60692);
\draw [c] (2.88764,1.51525) -- (2.89577,1.51525);
\draw [c] (2.89577,1.51525) -- (2.90391,1.51525);
\definecolor{c}{rgb}{0,0,0};
\colorlet{c}{natcomp!70};
\draw [c] (2.91205,1.43079) -- (2.91205,1.52154);
\draw [c] (2.91205,1.52154) -- (2.91205,1.61228);
\draw [c] (2.90391,1.52154) -- (2.91205,1.52154);
\draw [c] (2.91205,1.52154) -- (2.92018,1.52154);
\definecolor{c}{rgb}{0,0,0};
\colorlet{c}{natcomp!70};
\draw [c] (2.92832,1.34088) -- (2.92832,1.42855);
\draw [c] (2.92832,1.42855) -- (2.92832,1.51622);
\draw [c] (2.92018,1.42855) -- (2.92832,1.42855);
\draw [c] (2.92832,1.42855) -- (2.93645,1.42855);
\definecolor{c}{rgb}{0,0,0};
\colorlet{c}{natcomp!70};
\draw [c] (2.94459,1.41547) -- (2.94459,1.50915);
\draw [c] (2.94459,1.50915) -- (2.94459,1.60282);
\draw [c] (2.93645,1.50915) -- (2.94459,1.50915);
\draw [c] (2.94459,1.50915) -- (2.95273,1.50915);
\definecolor{c}{rgb}{0,0,0};
\colorlet{c}{natcomp!70};
\draw [c] (2.96086,1.58909) -- (2.96086,1.69352);
\draw [c] (2.96086,1.69352) -- (2.96086,1.79794);
\draw [c] (2.95273,1.69352) -- (2.96086,1.69352);
\draw [c] (2.96086,1.69352) -- (2.969,1.69352);
\definecolor{c}{rgb}{0,0,0};
\colorlet{c}{natcomp!70};
\draw [c] (2.97714,1.33395) -- (2.97714,1.41959);
\draw [c] (2.97714,1.41959) -- (2.97714,1.50522);
\draw [c] (2.969,1.41959) -- (2.97714,1.41959);
\draw [c] (2.97714,1.41959) -- (2.98527,1.41959);
\definecolor{c}{rgb}{0,0,0};
\colorlet{c}{natcomp!70};
\draw [c] (2.99341,1.59874) -- (2.99341,1.69978);
\draw [c] (2.99341,1.69978) -- (2.99341,1.80083);
\draw [c] (2.98527,1.69978) -- (2.99341,1.69978);
\draw [c] (2.99341,1.69978) -- (3.00155,1.69978);
\definecolor{c}{rgb}{0,0,0};
\colorlet{c}{natcomp!70};
\draw [c] (3.00968,1.52556) -- (3.00968,1.61936);
\draw [c] (3.00968,1.61936) -- (3.00968,1.71316);
\draw [c] (3.00155,1.61936) -- (3.00968,1.61936);
\draw [c] (3.00968,1.61936) -- (3.01782,1.61936);
\definecolor{c}{rgb}{0,0,0};
\colorlet{c}{natcomp!70};
\draw [c] (3.02595,1.6018) -- (3.02595,1.70136);
\draw [c] (3.02595,1.70136) -- (3.02595,1.80091);
\draw [c] (3.01782,1.70136) -- (3.02595,1.70136);
\draw [c] (3.02595,1.70136) -- (3.03409,1.70136);
\definecolor{c}{rgb}{0,0,0};
\colorlet{c}{natcomp!70};
\draw [c] (3.04223,1.58016) -- (3.04223,1.68072);
\draw [c] (3.04223,1.68072) -- (3.04223,1.78129);
\draw [c] (3.03409,1.68072) -- (3.04223,1.68072);
\draw [c] (3.04223,1.68072) -- (3.05036,1.68072);
\definecolor{c}{rgb}{0,0,0};
\colorlet{c}{natcomp!70};
\draw [c] (3.0585,1.72343) -- (3.0585,1.82892);
\draw [c] (3.0585,1.82892) -- (3.0585,1.93442);
\draw [c] (3.05036,1.82892) -- (3.0585,1.82892);
\draw [c] (3.0585,1.82892) -- (3.06664,1.82892);
\definecolor{c}{rgb}{0,0,0};
\colorlet{c}{natcomp!70};
\draw [c] (3.07477,1.80323) -- (3.07477,1.92123);
\draw [c] (3.07477,1.92123) -- (3.07477,2.03923);
\draw [c] (3.06664,1.92123) -- (3.07477,1.92123);
\draw [c] (3.07477,1.92123) -- (3.08291,1.92123);
\definecolor{c}{rgb}{0,0,0};
\colorlet{c}{natcomp!70};
\draw [c] (3.09105,1.72774) -- (3.09105,1.8397);
\draw [c] (3.09105,1.8397) -- (3.09105,1.95166);
\draw [c] (3.08291,1.8397) -- (3.09105,1.8397);
\draw [c] (3.09105,1.8397) -- (3.09918,1.8397);
\definecolor{c}{rgb}{0,0,0};
\colorlet{c}{natcomp!70};
\draw [c] (3.10732,1.5984) -- (3.10732,1.70303);
\draw [c] (3.10732,1.70303) -- (3.10732,1.80766);
\draw [c] (3.09918,1.70303) -- (3.10732,1.70303);
\draw [c] (3.10732,1.70303) -- (3.11545,1.70303);
\definecolor{c}{rgb}{0,0,0};
\colorlet{c}{natcomp!70};
\draw [c] (3.12359,1.72127) -- (3.12359,1.83007);
\draw [c] (3.12359,1.83007) -- (3.12359,1.93888);
\draw [c] (3.11545,1.83007) -- (3.12359,1.83007);
\draw [c] (3.12359,1.83007) -- (3.13173,1.83007);
\definecolor{c}{rgb}{0,0,0};
\colorlet{c}{natcomp!70};
\draw [c] (3.13986,1.64464) -- (3.13986,1.74758);
\draw [c] (3.13986,1.74758) -- (3.13986,1.85053);
\draw [c] (3.13173,1.74758) -- (3.13986,1.74758);
\draw [c] (3.13986,1.74758) -- (3.148,1.74758);
\definecolor{c}{rgb}{0,0,0};
\colorlet{c}{natcomp!70};
\draw [c] (3.15614,1.69808) -- (3.15614,1.80701);
\draw [c] (3.15614,1.80701) -- (3.15614,1.91594);
\draw [c] (3.148,1.80701) -- (3.15614,1.80701);
\draw [c] (3.15614,1.80701) -- (3.16427,1.80701);
\definecolor{c}{rgb}{0,0,0};
\colorlet{c}{natcomp!70};
\draw [c] (3.17241,1.85831) -- (3.17241,1.98447);
\draw [c] (3.17241,1.98447) -- (3.17241,2.11062);
\draw [c] (3.16427,1.98447) -- (3.17241,1.98447);
\draw [c] (3.17241,1.98447) -- (3.18055,1.98447);
\definecolor{c}{rgb}{0,0,0};
\colorlet{c}{natcomp!70};
\draw [c] (3.18868,1.75745) -- (3.18868,1.87281);
\draw [c] (3.18868,1.87281) -- (3.18868,1.98818);
\draw [c] (3.18055,1.87281) -- (3.18868,1.87281);
\draw [c] (3.18868,1.87281) -- (3.19682,1.87281);
\definecolor{c}{rgb}{0,0,0};
\colorlet{c}{natcomp!70};
\draw [c] (3.20495,1.88058) -- (3.20495,2.00281);
\draw [c] (3.20495,2.00281) -- (3.20495,2.12505);
\draw [c] (3.19682,2.00281) -- (3.20495,2.00281);
\draw [c] (3.20495,2.00281) -- (3.21309,2.00281);
\definecolor{c}{rgb}{0,0,0};
\colorlet{c}{natcomp!70};
\draw [c] (3.22123,1.80611) -- (3.22123,1.92753);
\draw [c] (3.22123,1.92753) -- (3.22123,2.04895);
\draw [c] (3.21309,1.92753) -- (3.22123,1.92753);
\draw [c] (3.22123,1.92753) -- (3.22936,1.92753);
\definecolor{c}{rgb}{0,0,0};
\colorlet{c}{natcomp!70};
\draw [c] (3.2375,1.91345) -- (3.2375,2.03616);
\draw [c] (3.2375,2.03616) -- (3.2375,2.15888);
\draw [c] (3.22936,2.03616) -- (3.2375,2.03616);
\draw [c] (3.2375,2.03616) -- (3.24564,2.03616);
\definecolor{c}{rgb}{0,0,0};
\colorlet{c}{natcomp!70};
\draw [c] (3.25377,1.96375) -- (3.25377,2.09043);
\draw [c] (3.25377,2.09043) -- (3.25377,2.21712);
\draw [c] (3.24564,2.09043) -- (3.25377,2.09043);
\draw [c] (3.25377,2.09043) -- (3.26191,2.09043);
\definecolor{c}{rgb}{0,0,0};
\colorlet{c}{natcomp!70};
\draw [c] (3.27005,1.92288) -- (3.27005,2.04725);
\draw [c] (3.27005,2.04725) -- (3.27005,2.17162);
\draw [c] (3.26191,2.04725) -- (3.27005,2.04725);
\draw [c] (3.27005,2.04725) -- (3.27818,2.04725);
\definecolor{c}{rgb}{0,0,0};
\colorlet{c}{natcomp!70};
\draw [c] (3.28632,1.84932) -- (3.28632,1.97407);
\draw [c] (3.28632,1.97407) -- (3.28632,2.09882);
\draw [c] (3.27818,1.97407) -- (3.28632,1.97407);
\draw [c] (3.28632,1.97407) -- (3.29445,1.97407);
\definecolor{c}{rgb}{0,0,0};
\colorlet{c}{natcomp!70};
\draw [c] (3.30259,1.90077) -- (3.30259,2.02007);
\draw [c] (3.30259,2.02007) -- (3.30259,2.13937);
\draw [c] (3.29445,2.02007) -- (3.30259,2.02007);
\draw [c] (3.30259,2.02007) -- (3.31073,2.02007);
\definecolor{c}{rgb}{0,0,0};
\colorlet{c}{natcomp!70};
\draw [c] (3.31886,1.83797) -- (3.31886,1.96045);
\draw [c] (3.31886,1.96045) -- (3.31886,2.08293);
\draw [c] (3.31073,1.96045) -- (3.31886,1.96045);
\draw [c] (3.31886,1.96045) -- (3.327,1.96045);
\definecolor{c}{rgb}{0,0,0};
\colorlet{c}{natcomp!70};
\draw [c] (3.33514,1.88986) -- (3.33514,2.01298);
\draw [c] (3.33514,2.01298) -- (3.33514,2.1361);
\draw [c] (3.327,2.01298) -- (3.33514,2.01298);
\draw [c] (3.33514,2.01298) -- (3.34327,2.01298);
\definecolor{c}{rgb}{0,0,0};
\colorlet{c}{natcomp!70};
\draw [c] (3.35141,1.97995) -- (3.35141,2.10928);
\draw [c] (3.35141,2.10928) -- (3.35141,2.2386);
\draw [c] (3.34327,2.10928) -- (3.35141,2.10928);
\draw [c] (3.35141,2.10928) -- (3.35955,2.10928);
\definecolor{c}{rgb}{0,0,0};
\colorlet{c}{natcomp!70};
\draw [c] (3.36768,1.69146) -- (3.36768,1.80413);
\draw [c] (3.36768,1.80413) -- (3.36768,1.9168);
\draw [c] (3.35955,1.80413) -- (3.36768,1.80413);
\draw [c] (3.36768,1.80413) -- (3.37582,1.80413);
\definecolor{c}{rgb}{0,0,0};
\colorlet{c}{natcomp!70};
\draw [c] (3.38395,1.76713) -- (3.38395,1.8867);
\draw [c] (3.38395,1.8867) -- (3.38395,2.00627);
\draw [c] (3.37582,1.8867) -- (3.38395,1.8867);
\draw [c] (3.38395,1.8867) -- (3.39209,1.8867);
\definecolor{c}{rgb}{0,0,0};
\colorlet{c}{natcomp!70};
\draw [c] (3.40023,1.73703) -- (3.40023,1.85405);
\draw [c] (3.40023,1.85405) -- (3.40023,1.97106);
\draw [c] (3.39209,1.85405) -- (3.40023,1.85405);
\draw [c] (3.40023,1.85405) -- (3.40836,1.85405);
\definecolor{c}{rgb}{0,0,0};
\colorlet{c}{natcomp!70};
\draw [c] (3.4165,1.75427) -- (3.4165,1.87418);
\draw [c] (3.4165,1.87418) -- (3.4165,1.99409);
\draw [c] (3.40836,1.87418) -- (3.4165,1.87418);
\draw [c] (3.4165,1.87418) -- (3.42464,1.87418);
\definecolor{c}{rgb}{0,0,0};
\colorlet{c}{natcomp!70};
\draw [c] (3.43277,1.59071) -- (3.43277,1.70168);
\draw [c] (3.43277,1.70168) -- (3.43277,1.81264);
\draw [c] (3.42464,1.70168) -- (3.43277,1.70168);
\draw [c] (3.43277,1.70168) -- (3.44091,1.70168);
\definecolor{c}{rgb}{0,0,0};
\colorlet{c}{natcomp!70};
\draw [c] (3.44905,2.0116) -- (3.44905,2.1488);
\draw [c] (3.44905,2.1488) -- (3.44905,2.286);
\draw [c] (3.44091,2.1488) -- (3.44905,2.1488);
\draw [c] (3.44905,2.1488) -- (3.45718,2.1488);
\definecolor{c}{rgb}{0,0,0};
\colorlet{c}{natcomp!70};
\draw [c] (3.46532,1.58379) -- (3.46532,1.69731);
\draw [c] (3.46532,1.69731) -- (3.46532,1.81082);
\draw [c] (3.45718,1.69731) -- (3.46532,1.69731);
\draw [c] (3.46532,1.69731) -- (3.47345,1.69731);
\definecolor{c}{rgb}{0,0,0};
\colorlet{c}{natcomp!70};
\draw [c] (3.48159,1.69179) -- (3.48159,1.8072);
\draw [c] (3.48159,1.8072) -- (3.48159,1.92261);
\draw [c] (3.47345,1.8072) -- (3.48159,1.8072);
\draw [c] (3.48159,1.8072) -- (3.48973,1.8072);
\definecolor{c}{rgb}{0,0,0};
\colorlet{c}{natcomp!70};
\draw [c] (3.49786,1.66795) -- (3.49786,1.7778);
\draw [c] (3.49786,1.7778) -- (3.49786,1.88766);
\draw [c] (3.48973,1.7778) -- (3.49786,1.7778);
\draw [c] (3.49786,1.7778) -- (3.506,1.7778);
\definecolor{c}{rgb}{0,0,0};
\colorlet{c}{natcomp!70};
\draw [c] (3.51414,1.54378) -- (3.51414,1.65414);
\draw [c] (3.51414,1.65414) -- (3.51414,1.76451);
\draw [c] (3.506,1.65414) -- (3.51414,1.65414);
\draw [c] (3.51414,1.65414) -- (3.52227,1.65414);
\definecolor{c}{rgb}{0,0,0};
\colorlet{c}{natcomp!70};
\draw [c] (3.53041,1.63429) -- (3.53041,1.75469);
\draw [c] (3.53041,1.75469) -- (3.53041,1.87509);
\draw [c] (3.52227,1.75469) -- (3.53041,1.75469);
\draw [c] (3.53041,1.75469) -- (3.53855,1.75469);
\definecolor{c}{rgb}{0,0,0};
\colorlet{c}{natcomp!70};
\draw [c] (3.54668,1.49617) -- (3.54668,1.60081);
\draw [c] (3.54668,1.60081) -- (3.54668,1.70545);
\draw [c] (3.53855,1.60081) -- (3.54668,1.60081);
\draw [c] (3.54668,1.60081) -- (3.55482,1.60081);
\definecolor{c}{rgb}{0,0,0};
\colorlet{c}{natcomp!70};
\draw [c] (3.56295,1.58314) -- (3.56295,1.69229);
\draw [c] (3.56295,1.69229) -- (3.56295,1.80144);
\draw [c] (3.55482,1.69229) -- (3.56295,1.69229);
\draw [c] (3.56295,1.69229) -- (3.57109,1.69229);
\definecolor{c}{rgb}{0,0,0};
\colorlet{c}{natcomp!70};
\draw [c] (3.57923,1.46417) -- (3.57923,1.56462);
\draw [c] (3.57923,1.56462) -- (3.57923,1.66507);
\draw [c] (3.57109,1.56462) -- (3.57923,1.56462);
\draw [c] (3.57923,1.56462) -- (3.58736,1.56462);
\definecolor{c}{rgb}{0,0,0};
\colorlet{c}{natcomp!70};
\draw [c] (3.5955,1.7506) -- (3.5955,1.87231);
\draw [c] (3.5955,1.87231) -- (3.5955,1.99402);
\draw [c] (3.58736,1.87231) -- (3.5955,1.87231);
\draw [c] (3.5955,1.87231) -- (3.60364,1.87231);
\definecolor{c}{rgb}{0,0,0};
\colorlet{c}{natcomp!70};
\draw [c] (3.61177,1.53631) -- (3.61177,1.63577);
\draw [c] (3.61177,1.63577) -- (3.61177,1.73524);
\draw [c] (3.60364,1.63577) -- (3.61177,1.63577);
\draw [c] (3.61177,1.63577) -- (3.61991,1.63577);
\definecolor{c}{rgb}{0,0,0};
\colorlet{c}{natcomp!70};
\draw [c] (3.62805,1.60831) -- (3.62805,1.71397);
\draw [c] (3.62805,1.71397) -- (3.62805,1.81962);
\draw [c] (3.61991,1.71397) -- (3.62805,1.71397);
\draw [c] (3.62805,1.71397) -- (3.63618,1.71397);
\definecolor{c}{rgb}{0,0,0};
\colorlet{c}{natcomp!70};
\draw [c] (3.64432,1.41401) -- (3.64432,1.5139);
\draw [c] (3.64432,1.5139) -- (3.64432,1.61378);
\draw [c] (3.63618,1.5139) -- (3.64432,1.5139);
\draw [c] (3.64432,1.5139) -- (3.65245,1.5139);
\definecolor{c}{rgb}{0,0,0};
\colorlet{c}{natcomp!70};
\draw [c] (3.66059,1.47867) -- (3.66059,1.57869);
\draw [c] (3.66059,1.57869) -- (3.66059,1.67871);
\draw [c] (3.65245,1.57869) -- (3.66059,1.57869);
\draw [c] (3.66059,1.57869) -- (3.66873,1.57869);
\definecolor{c}{rgb}{0,0,0};
\colorlet{c}{natcomp!70};
\draw [c] (3.67686,1.43725) -- (3.67686,1.53155);
\draw [c] (3.67686,1.53155) -- (3.67686,1.62584);
\draw [c] (3.66873,1.53155) -- (3.67686,1.53155);
\draw [c] (3.67686,1.53155) -- (3.685,1.53155);
\definecolor{c}{rgb}{0,0,0};
\colorlet{c}{natcomp!70};
\draw [c] (3.69314,1.44595) -- (3.69314,1.54196);
\draw [c] (3.69314,1.54196) -- (3.69314,1.63798);
\draw [c] (3.685,1.54196) -- (3.69314,1.54196);
\draw [c] (3.69314,1.54196) -- (3.70127,1.54196);
\definecolor{c}{rgb}{0,0,0};
\colorlet{c}{natcomp!70};
\draw [c] (3.70941,1.31331) -- (3.70941,1.4008);
\draw [c] (3.70941,1.4008) -- (3.70941,1.48828);
\draw [c] (3.70127,1.4008) -- (3.70941,1.4008);
\draw [c] (3.70941,1.4008) -- (3.71755,1.4008);
\definecolor{c}{rgb}{0,0,0};
\colorlet{c}{natcomp!70};
\draw [c] (3.72568,1.38718) -- (3.72568,1.47882);
\draw [c] (3.72568,1.47882) -- (3.72568,1.57046);
\draw [c] (3.71755,1.47882) -- (3.72568,1.47882);
\draw [c] (3.72568,1.47882) -- (3.73382,1.47882);
\definecolor{c}{rgb}{0,0,0};
\colorlet{c}{natcomp!70};
\draw [c] (3.74195,1.30543) -- (3.74195,1.39264);
\draw [c] (3.74195,1.39264) -- (3.74195,1.47984);
\draw [c] (3.73382,1.39264) -- (3.74195,1.39264);
\draw [c] (3.74195,1.39264) -- (3.75009,1.39264);
\definecolor{c}{rgb}{0,0,0};
\colorlet{c}{natcomp!70};
\draw [c] (3.75823,1.33609) -- (3.75823,1.42593);
\draw [c] (3.75823,1.42593) -- (3.75823,1.51577);
\draw [c] (3.75009,1.42593) -- (3.75823,1.42593);
\draw [c] (3.75823,1.42593) -- (3.76636,1.42593);
\definecolor{c}{rgb}{0,0,0};
\colorlet{c}{natcomp!70};
\draw [c] (3.7745,1.48113) -- (3.7745,1.57802);
\draw [c] (3.7745,1.57802) -- (3.7745,1.67491);
\draw [c] (3.76636,1.57802) -- (3.7745,1.57802);
\draw [c] (3.7745,1.57802) -- (3.78264,1.57802);
\definecolor{c}{rgb}{0,0,0};
\colorlet{c}{natcomp!70};
\draw [c] (3.79077,1.36985) -- (3.79077,1.4665);
\draw [c] (3.79077,1.4665) -- (3.79077,1.56316);
\draw [c] (3.78264,1.4665) -- (3.79077,1.4665);
\draw [c] (3.79077,1.4665) -- (3.79891,1.4665);
\definecolor{c}{rgb}{0,0,0};
\colorlet{c}{natcomp!70};
\draw [c] (3.80705,1.3613) -- (3.80705,1.45155);
\draw [c] (3.80705,1.45155) -- (3.80705,1.5418);
\draw [c] (3.79891,1.45155) -- (3.80705,1.45155);
\draw [c] (3.80705,1.45155) -- (3.81518,1.45155);
\definecolor{c}{rgb}{0,0,0};
\colorlet{c}{natcomp!70};
\draw [c] (3.82332,1.34548) -- (3.82332,1.43349);
\draw [c] (3.82332,1.43349) -- (3.82332,1.5215);
\draw [c] (3.81518,1.43349) -- (3.82332,1.43349);
\draw [c] (3.82332,1.43349) -- (3.83145,1.43349);
\definecolor{c}{rgb}{0,0,0};
\colorlet{c}{natcomp!70};
\draw [c] (3.83959,1.34648) -- (3.83959,1.44227);
\draw [c] (3.83959,1.44227) -- (3.83959,1.53807);
\draw [c] (3.83145,1.44227) -- (3.83959,1.44227);
\draw [c] (3.83959,1.44227) -- (3.84773,1.44227);
\definecolor{c}{rgb}{0,0,0};
\colorlet{c}{natcomp!70};
\draw [c] (3.85586,1.26603) -- (3.85586,1.34858);
\draw [c] (3.85586,1.34858) -- (3.85586,1.43114);
\draw [c] (3.84773,1.34858) -- (3.85586,1.34858);
\draw [c] (3.85586,1.34858) -- (3.864,1.34858);
\definecolor{c}{rgb}{0,0,0};
\colorlet{c}{natcomp!70};
\draw [c] (3.87214,1.19945) -- (3.87214,1.28167);
\draw [c] (3.87214,1.28167) -- (3.87214,1.3639);
\draw [c] (3.864,1.28167) -- (3.87214,1.28167);
\draw [c] (3.87214,1.28167) -- (3.88027,1.28167);
\definecolor{c}{rgb}{0,0,0};
\colorlet{c}{natcomp!70};
\draw [c] (3.88841,1.37139) -- (3.88841,1.46647);
\draw [c] (3.88841,1.46647) -- (3.88841,1.56155);
\draw [c] (3.88027,1.46647) -- (3.88841,1.46647);
\draw [c] (3.88841,1.46647) -- (3.89655,1.46647);
\definecolor{c}{rgb}{0,0,0};
\colorlet{c}{natcomp!70};
\draw [c] (3.90468,1.0781) -- (3.90468,1.14498);
\draw [c] (3.90468,1.14498) -- (3.90468,1.21187);
\draw [c] (3.89655,1.14498) -- (3.90468,1.14498);
\draw [c] (3.90468,1.14498) -- (3.91282,1.14498);
\definecolor{c}{rgb}{0,0,0};
\colorlet{c}{natcomp!70};
\draw [c] (3.92095,1.12409) -- (3.92095,1.19694);
\draw [c] (3.92095,1.19694) -- (3.92095,1.26979);
\draw [c] (3.91282,1.19694) -- (3.92095,1.19694);
\draw [c] (3.92095,1.19694) -- (3.92909,1.19694);
\definecolor{c}{rgb}{0,0,0};
\colorlet{c}{natcomp!70};
\draw [c] (3.93723,1.21279) -- (3.93723,1.29838);
\draw [c] (3.93723,1.29838) -- (3.93723,1.38397);
\draw [c] (3.92909,1.29838) -- (3.93723,1.29838);
\draw [c] (3.93723,1.29838) -- (3.94536,1.29838);
\definecolor{c}{rgb}{0,0,0};
\colorlet{c}{natcomp!70};
\draw [c] (3.9535,1.18838) -- (3.9535,1.26484);
\draw [c] (3.9535,1.26484) -- (3.9535,1.3413);
\draw [c] (3.94536,1.26484) -- (3.9535,1.26484);
\draw [c] (3.9535,1.26484) -- (3.96164,1.26484);
\definecolor{c}{rgb}{0,0,0};
\colorlet{c}{natcomp!70};
\draw [c] (3.96977,1.13244) -- (3.96977,1.20537);
\draw [c] (3.96977,1.20537) -- (3.96977,1.2783);
\draw [c] (3.96164,1.20537) -- (3.96977,1.20537);
\draw [c] (3.96977,1.20537) -- (3.97791,1.20537);
\definecolor{c}{rgb}{0,0,0};
\colorlet{c}{natcomp!70};
\draw [c] (3.98605,1.18644) -- (3.98605,1.26695);
\draw [c] (3.98605,1.26695) -- (3.98605,1.34746);
\draw [c] (3.97791,1.26695) -- (3.98605,1.26695);
\draw [c] (3.98605,1.26695) -- (3.99418,1.26695);
\definecolor{c}{rgb}{0,0,0};
\colorlet{c}{natcomp!70};
\draw [c] (4.00232,1.18905) -- (4.00232,1.27326);
\draw [c] (4.00232,1.27326) -- (4.00232,1.35748);
\draw [c] (3.99418,1.27326) -- (4.00232,1.27326);
\draw [c] (4.00232,1.27326) -- (4.01045,1.27326);
\definecolor{c}{rgb}{0,0,0};
\colorlet{c}{natcomp!70};
\draw [c] (4.01859,1.21515) -- (4.01859,1.29654);
\draw [c] (4.01859,1.29654) -- (4.01859,1.37792);
\draw [c] (4.01045,1.29654) -- (4.01859,1.29654);
\draw [c] (4.01859,1.29654) -- (4.02673,1.29654);
\definecolor{c}{rgb}{0,0,0};
\colorlet{c}{natcomp!70};
\draw [c] (4.03486,1.0567) -- (4.03486,1.12126);
\draw [c] (4.03486,1.12126) -- (4.03486,1.18582);
\draw [c] (4.02673,1.12126) -- (4.03486,1.12126);
\draw [c] (4.03486,1.12126) -- (4.043,1.12126);
\definecolor{c}{rgb}{0,0,0};
\colorlet{c}{natcomp!70};
\draw [c] (4.05114,1.0637) -- (4.05114,1.13565);
\draw [c] (4.05114,1.13565) -- (4.05114,1.2076);
\draw [c] (4.043,1.13565) -- (4.05114,1.13565);
\draw [c] (4.05114,1.13565) -- (4.05927,1.13565);
\definecolor{c}{rgb}{0,0,0};
\colorlet{c}{natcomp!70};
\draw [c] (4.06741,1.12543) -- (4.06741,1.19947);
\draw [c] (4.06741,1.19947) -- (4.06741,1.27351);
\draw [c] (4.05927,1.19947) -- (4.06741,1.19947);
\draw [c] (4.06741,1.19947) -- (4.07555,1.19947);
\definecolor{c}{rgb}{0,0,0};
\colorlet{c}{natcomp!70};
\draw [c] (4.08368,1.11089) -- (4.08368,1.18647);
\draw [c] (4.08368,1.18647) -- (4.08368,1.26206);
\draw [c] (4.07555,1.18647) -- (4.08368,1.18647);
\draw [c] (4.08368,1.18647) -- (4.09182,1.18647);
\definecolor{c}{rgb}{0,0,0};
\colorlet{c}{natcomp!70};
\draw [c] (4.09995,1.09755) -- (4.09995,1.16945);
\draw [c] (4.09995,1.16945) -- (4.09995,1.24135);
\draw [c] (4.09182,1.16945) -- (4.09995,1.16945);
\draw [c] (4.09995,1.16945) -- (4.10809,1.16945);
\definecolor{c}{rgb}{0,0,0};
\colorlet{c}{natcomp!70};
\draw [c] (4.11623,1.07487) -- (4.11623,1.14657);
\draw [c] (4.11623,1.14657) -- (4.11623,1.21826);
\draw [c] (4.10809,1.14657) -- (4.11623,1.14657);
\draw [c] (4.11623,1.14657) -- (4.12436,1.14657);
\definecolor{c}{rgb}{0,0,0};
\colorlet{c}{natcomp!70};
\draw [c] (4.1325,1.03624) -- (4.1325,1.10277);
\draw [c] (4.1325,1.10277) -- (4.1325,1.16931);
\draw [c] (4.12436,1.10277) -- (4.1325,1.10277);
\draw [c] (4.1325,1.10277) -- (4.14064,1.10277);
\definecolor{c}{rgb}{0,0,0};
\colorlet{c}{natcomp!70};
\draw [c] (4.14877,0.952458) -- (4.14877,1.00761);
\draw [c] (4.14877,1.00761) -- (4.14877,1.06276);
\draw [c] (4.14064,1.00761) -- (4.14877,1.00761);
\draw [c] (4.14877,1.00761) -- (4.15691,1.00761);
\definecolor{c}{rgb}{0,0,0};
\colorlet{c}{natcomp!70};
\draw [c] (4.16505,1.04998) -- (4.16505,1.12012);
\draw [c] (4.16505,1.12012) -- (4.16505,1.19027);
\draw [c] (4.15691,1.12012) -- (4.16505,1.12012);
\draw [c] (4.16505,1.12012) -- (4.17318,1.12012);
\definecolor{c}{rgb}{0,0,0};
\colorlet{c}{natcomp!70};
\draw [c] (4.18132,1.08985) -- (4.18132,1.16335);
\draw [c] (4.18132,1.16335) -- (4.18132,1.23686);
\draw [c] (4.17318,1.16335) -- (4.18132,1.16335);
\draw [c] (4.18132,1.16335) -- (4.18945,1.16335);
\definecolor{c}{rgb}{0,0,0};
\colorlet{c}{natcomp!70};
\draw [c] (4.19759,1.06558) -- (4.19759,1.13256);
\draw [c] (4.19759,1.13256) -- (4.19759,1.19953);
\draw [c] (4.18945,1.13256) -- (4.19759,1.13256);
\draw [c] (4.19759,1.13256) -- (4.20573,1.13256);
\definecolor{c}{rgb}{0,0,0};
\colorlet{c}{natcomp!70};
\draw [c] (4.21386,0.99441) -- (4.21386,1.05948);
\draw [c] (4.21386,1.05948) -- (4.21386,1.12455);
\draw [c] (4.20573,1.05948) -- (4.21386,1.05948);
\draw [c] (4.21386,1.05948) -- (4.222,1.05948);
\definecolor{c}{rgb}{0,0,0};
\colorlet{c}{natcomp!70};
\draw [c] (4.23014,1.15782) -- (4.23014,1.23917);
\draw [c] (4.23014,1.23917) -- (4.23014,1.32052);
\draw [c] (4.222,1.23917) -- (4.23014,1.23917);
\draw [c] (4.23014,1.23917) -- (4.23827,1.23917);
\definecolor{c}{rgb}{0,0,0};
\colorlet{c}{natcomp!70};
\draw [c] (4.24641,0.983316) -- (4.24641,1.04763);
\draw [c] (4.24641,1.04763) -- (4.24641,1.11194);
\draw [c] (4.23827,1.04763) -- (4.24641,1.04763);
\draw [c] (4.24641,1.04763) -- (4.25455,1.04763);
\definecolor{c}{rgb}{0,0,0};
\colorlet{c}{natcomp!70};
\draw [c] (4.26268,1.03462) -- (4.26268,1.10754);
\draw [c] (4.26268,1.10754) -- (4.26268,1.18046);
\draw [c] (4.25455,1.10754) -- (4.26268,1.10754);
\draw [c] (4.26268,1.10754) -- (4.27082,1.10754);
\definecolor{c}{rgb}{0,0,0};
\colorlet{c}{natcomp!70};
\draw [c] (4.27895,0.980957) -- (4.27895,1.04168);
\draw [c] (4.27895,1.04168) -- (4.27895,1.1024);
\draw [c] (4.27082,1.04168) -- (4.27895,1.04168);
\draw [c] (4.27895,1.04168) -- (4.28709,1.04168);
\definecolor{c}{rgb}{0,0,0};
\colorlet{c}{natcomp!70};
\draw [c] (4.29523,0.98657) -- (4.29523,1.04335);
\draw [c] (4.29523,1.04335) -- (4.29523,1.10013);
\draw [c] (4.28709,1.04335) -- (4.29523,1.04335);
\draw [c] (4.29523,1.04335) -- (4.30336,1.04335);
\definecolor{c}{rgb}{0,0,0};
\colorlet{c}{natcomp!70};
\draw [c] (4.3115,0.987687) -- (4.3115,1.0472);
\draw [c] (4.3115,1.0472) -- (4.3115,1.10672);
\draw [c] (4.30336,1.0472) -- (4.3115,1.0472);
\draw [c] (4.3115,1.0472) -- (4.31964,1.0472);
\definecolor{c}{rgb}{0,0,0};
\colorlet{c}{natcomp!70};
\draw [c] (4.32777,1.01895) -- (4.32777,1.08324);
\draw [c] (4.32777,1.08324) -- (4.32777,1.14754);
\draw [c] (4.31964,1.08324) -- (4.32777,1.08324);
\draw [c] (4.32777,1.08324) -- (4.33591,1.08324);
\definecolor{c}{rgb}{0,0,0};
\colorlet{c}{natcomp!70};
\draw [c] (4.34405,0.968473) -- (4.34405,1.02884);
\draw [c] (4.34405,1.02884) -- (4.34405,1.08921);
\draw [c] (4.33591,1.02884) -- (4.34405,1.02884);
\draw [c] (4.34405,1.02884) -- (4.35218,1.02884);
\definecolor{c}{rgb}{0,0,0};
\colorlet{c}{natcomp!70};
\draw [c] (4.36032,0.974112) -- (4.36032,1.03613);
\draw [c] (4.36032,1.03613) -- (4.36032,1.09814);
\draw [c] (4.35218,1.03613) -- (4.36032,1.03613);
\draw [c] (4.36032,1.03613) -- (4.36845,1.03613);
\definecolor{c}{rgb}{0,0,0};
\colorlet{c}{natcomp!70};
\draw [c] (4.37659,0.980809) -- (4.37659,1.04153);
\draw [c] (4.37659,1.04153) -- (4.37659,1.10224);
\draw [c] (4.36845,1.04153) -- (4.37659,1.04153);
\draw [c] (4.37659,1.04153) -- (4.38473,1.04153);
\definecolor{c}{rgb}{0,0,0};
\colorlet{c}{natcomp!70};
\draw [c] (4.39286,0.971029) -- (4.39286,1.02981);
\draw [c] (4.39286,1.02981) -- (4.39286,1.08859);
\draw [c] (4.38473,1.02981) -- (4.39286,1.02981);
\draw [c] (4.39286,1.02981) -- (4.401,1.02981);
\definecolor{c}{rgb}{0,0,0};
\colorlet{c}{natcomp!70};
\draw [c] (4.40914,0.95256) -- (4.40914,1.00793);
\draw [c] (4.40914,1.00793) -- (4.40914,1.06331);
\draw [c] (4.401,1.00793) -- (4.40914,1.00793);
\draw [c] (4.40914,1.00793) -- (4.41727,1.00793);
\definecolor{c}{rgb}{0,0,0};
\colorlet{c}{natcomp!70};
\draw [c] (4.42541,1.04433) -- (4.42541,1.1097);
\draw [c] (4.42541,1.1097) -- (4.42541,1.17507);
\draw [c] (4.41727,1.1097) -- (4.42541,1.1097);
\draw [c] (4.42541,1.1097) -- (4.43355,1.1097);
\definecolor{c}{rgb}{0,0,0};
\colorlet{c}{natcomp!70};
\draw [c] (4.44168,0.921556) -- (4.44168,0.976194);
\draw [c] (4.44168,0.976194) -- (4.44168,1.03083);
\draw [c] (4.43355,0.976194) -- (4.44168,0.976194);
\draw [c] (4.44168,0.976194) -- (4.44982,0.976194);
\definecolor{c}{rgb}{0,0,0};
\colorlet{c}{natcomp!70};
\draw [c] (4.45795,0.972593) -- (4.45795,1.04013);
\draw [c] (4.45795,1.04013) -- (4.45795,1.10767);
\draw [c] (4.44982,1.04013) -- (4.45795,1.04013);
\draw [c] (4.45795,1.04013) -- (4.46609,1.04013);
\definecolor{c}{rgb}{0,0,0};
\colorlet{c}{natcomp!70};
\draw [c] (4.47423,0.946413) -- (4.47423,0.999826);
\draw [c] (4.47423,0.999826) -- (4.47423,1.05324);
\draw [c] (4.46609,0.999826) -- (4.47423,0.999826);
\draw [c] (4.47423,0.999826) -- (4.48236,0.999826);
\definecolor{c}{rgb}{0,0,0};
\colorlet{c}{natcomp!70};
\draw [c] (4.4905,0.928871) -- (4.4905,0.983045);
\draw [c] (4.4905,0.983045) -- (4.4905,1.03722);
\draw [c] (4.48236,0.983045) -- (4.4905,0.983045);
\draw [c] (4.4905,0.983045) -- (4.49864,0.983045);
\definecolor{c}{rgb}{0,0,0};
\colorlet{c}{natcomp!70};
\draw [c] (4.50677,0.92616) -- (4.50677,0.979906);
\draw [c] (4.50677,0.979906) -- (4.50677,1.03365);
\draw [c] (4.49864,0.979906) -- (4.50677,0.979906);
\draw [c] (4.50677,0.979906) -- (4.51491,0.979906);
\definecolor{c}{rgb}{0,0,0};
\colorlet{c}{natcomp!70};
\draw [c] (4.52305,0.866944) -- (4.52305,0.914198);
\draw [c] (4.52305,0.914198) -- (4.52305,0.961452);
\draw [c] (4.51491,0.914198) -- (4.52305,0.914198);
\draw [c] (4.52305,0.914198) -- (4.53118,0.914198);
\definecolor{c}{rgb}{0,0,0};
\colorlet{c}{natcomp!70};
\draw [c] (4.53932,0.956535) -- (4.53932,1.01368);
\draw [c] (4.53932,1.01368) -- (4.53932,1.07083);
\draw [c] (4.53118,1.01368) -- (4.53932,1.01368);
\draw [c] (4.53932,1.01368) -- (4.54745,1.01368);
\definecolor{c}{rgb}{0,0,0};
\colorlet{c}{natcomp!70};
\draw [c] (4.55559,1.00276) -- (4.55559,1.06462);
\draw [c] (4.55559,1.06462) -- (4.55559,1.12648);
\draw [c] (4.54745,1.06462) -- (4.55559,1.06462);
\draw [c] (4.55559,1.06462) -- (4.56373,1.06462);
\definecolor{c}{rgb}{0,0,0};
\colorlet{c}{natcomp!70};
\draw [c] (4.57186,0.919644) -- (4.57186,0.985824);
\draw [c] (4.57186,0.985824) -- (4.57186,1.052);
\draw [c] (4.56373,0.985824) -- (4.57186,0.985824);
\draw [c] (4.57186,0.985824) -- (4.58,0.985824);
\definecolor{c}{rgb}{0,0,0};
\colorlet{c}{natcomp!70};
\draw [c] (4.58814,0.882513) -- (4.58814,0.934061);
\draw [c] (4.58814,0.934061) -- (4.58814,0.985609);
\draw [c] (4.58,0.934061) -- (4.58814,0.934061);
\draw [c] (4.58814,0.934061) -- (4.59627,0.934061);
\definecolor{c}{rgb}{0,0,0};
\colorlet{c}{natcomp!70};
\draw [c] (4.60441,0.912457) -- (4.60441,0.967236);
\draw [c] (4.60441,0.967236) -- (4.60441,1.02202);
\draw [c] (4.59627,0.967236) -- (4.60441,0.967236);
\draw [c] (4.60441,0.967236) -- (4.61255,0.967236);
\definecolor{c}{rgb}{0,0,0};
\colorlet{c}{natcomp!70};
\draw [c] (4.62068,0.914155) -- (4.62068,0.969166);
\draw [c] (4.62068,0.969166) -- (4.62068,1.02418);
\draw [c] (4.61255,0.969166) -- (4.62068,0.969166);
\draw [c] (4.62068,0.969166) -- (4.62882,0.969166);
\definecolor{c}{rgb}{0,0,0};
\colorlet{c}{natcomp!70};
\draw [c] (4.63695,0.879878) -- (4.63695,0.938137);
\draw [c] (4.63695,0.938137) -- (4.63695,0.996396);
\draw [c] (4.62882,0.938137) -- (4.63695,0.938137);
\draw [c] (4.63695,0.938137) -- (4.64509,0.938137);
\definecolor{c}{rgb}{0,0,0};
\colorlet{c}{natcomp!70};
\draw [c] (4.65323,0.899783) -- (4.65323,0.956467);
\draw [c] (4.65323,0.956467) -- (4.65323,1.01315);
\draw [c] (4.64509,0.956467) -- (4.65323,0.956467);
\draw [c] (4.65323,0.956467) -- (4.66136,0.956467);
\definecolor{c}{rgb}{0,0,0};
\colorlet{c}{natcomp!70};
\draw [c] (4.6695,0.884772) -- (4.6695,0.936176);
\draw [c] (4.6695,0.936176) -- (4.6695,0.987579);
\draw [c] (4.66136,0.936176) -- (4.6695,0.936176);
\draw [c] (4.6695,0.936176) -- (4.67764,0.936176);
\definecolor{c}{rgb}{0,0,0};
\colorlet{c}{natcomp!70};
\draw [c] (4.68577,0.985752) -- (4.68577,1.05675);
\draw [c] (4.68577,1.05675) -- (4.68577,1.12775);
\draw [c] (4.67764,1.05675) -- (4.68577,1.05675);
\draw [c] (4.68577,1.05675) -- (4.69391,1.05675);
\definecolor{c}{rgb}{0,0,0};
\colorlet{c}{natcomp!70};
\draw [c] (4.70205,0.888065) -- (4.70205,0.939341);
\draw [c] (4.70205,0.939341) -- (4.70205,0.990618);
\draw [c] (4.69391,0.939341) -- (4.70205,0.939341);
\draw [c] (4.70205,0.939341) -- (4.71018,0.939341);
\definecolor{c}{rgb}{0,0,0};
\colorlet{c}{natcomp!70};
\draw [c] (4.71832,0.950968) -- (4.71832,1.01404);
\draw [c] (4.71832,1.01404) -- (4.71832,1.0771);
\draw [c] (4.71018,1.01404) -- (4.71832,1.01404);
\draw [c] (4.71832,1.01404) -- (4.72645,1.01404);
\definecolor{c}{rgb}{0,0,0};
\colorlet{c}{natcomp!70};
\draw [c] (4.73459,0.890061) -- (4.73459,0.949548);
\draw [c] (4.73459,0.949548) -- (4.73459,1.00904);
\draw [c] (4.72645,0.949548) -- (4.73459,0.949548);
\draw [c] (4.73459,0.949548) -- (4.74273,0.949548);
\definecolor{c}{rgb}{0,0,0};
\colorlet{c}{natcomp!70};
\draw [c] (4.75086,0.903141) -- (4.75086,0.959014);
\draw [c] (4.75086,0.959014) -- (4.75086,1.01489);
\draw [c] (4.74273,0.959014) -- (4.75086,0.959014);
\draw [c] (4.75086,0.959014) -- (4.759,0.959014);
\definecolor{c}{rgb}{0,0,0};
\colorlet{c}{natcomp!70};
\draw [c] (4.76714,0.878125) -- (4.76714,0.929474);
\draw [c] (4.76714,0.929474) -- (4.76714,0.980823);
\draw [c] (4.759,0.929474) -- (4.76714,0.929474);
\draw [c] (4.76714,0.929474) -- (4.77527,0.929474);
\definecolor{c}{rgb}{0,0,0};
\colorlet{c}{natcomp!70};
\draw [c] (4.78341,0.831227) -- (4.78341,0.880248);
\draw [c] (4.78341,0.880248) -- (4.78341,0.929269);
\draw [c] (4.77527,0.880248) -- (4.78341,0.880248);
\draw [c] (4.78341,0.880248) -- (4.79155,0.880248);
\definecolor{c}{rgb}{0,0,0};
\colorlet{c}{natcomp!70};
\draw [c] (4.79968,0.864439) -- (4.79968,0.917743);
\draw [c] (4.79968,0.917743) -- (4.79968,0.971047);
\draw [c] (4.79155,0.917743) -- (4.79968,0.917743);
\draw [c] (4.79968,0.917743) -- (4.80782,0.917743);
\definecolor{c}{rgb}{0,0,0};
\colorlet{c}{natcomp!70};
\draw [c] (4.81595,0.852991) -- (4.81595,0.902588);
\draw [c] (4.81595,0.902588) -- (4.81595,0.952185);
\draw [c] (4.80782,0.902588) -- (4.81595,0.902588);
\draw [c] (4.81595,0.902588) -- (4.82409,0.902588);
\definecolor{c}{rgb}{0,0,0};
\colorlet{c}{natcomp!70};
\draw [c] (4.83223,0.838268) -- (4.83223,0.879525);
\draw [c] (4.83223,0.879525) -- (4.83223,0.920782);
\draw [c] (4.82409,0.879525) -- (4.83223,0.879525);
\draw [c] (4.83223,0.879525) -- (4.84036,0.879525);
\definecolor{c}{rgb}{0,0,0};
\colorlet{c}{natcomp!70};
\draw [c] (4.8485,0.849755) -- (4.8485,0.901917);
\draw [c] (4.8485,0.901917) -- (4.8485,0.954078);
\draw [c] (4.84036,0.901917) -- (4.8485,0.901917);
\draw [c] (4.8485,0.901917) -- (4.85664,0.901917);
\definecolor{c}{rgb}{0,0,0};
\colorlet{c}{natcomp!70};
\draw [c] (4.86477,0.88973) -- (4.86477,0.942964);
\draw [c] (4.86477,0.942964) -- (4.86477,0.996198);
\draw [c] (4.85664,0.942964) -- (4.86477,0.942964);
\draw [c] (4.86477,0.942964) -- (4.87291,0.942964);
\definecolor{c}{rgb}{0,0,0};
\colorlet{c}{natcomp!70};
\draw [c] (4.88105,0.852361) -- (4.88105,0.898715);
\draw [c] (4.88105,0.898715) -- (4.88105,0.945068);
\draw [c] (4.87291,0.898715) -- (4.88105,0.898715);
\draw [c] (4.88105,0.898715) -- (4.88918,0.898715);
\definecolor{c}{rgb}{0,0,0};
\colorlet{c}{natcomp!70};
\draw [c] (4.89732,0.889338) -- (4.89732,0.94623);
\draw [c] (4.89732,0.94623) -- (4.89732,1.00312);
\draw [c] (4.88918,0.94623) -- (4.89732,0.94623);
\draw [c] (4.89732,0.94623) -- (4.90545,0.94623);
\definecolor{c}{rgb}{0,0,0};
\colorlet{c}{natcomp!70};
\draw [c] (4.91359,0.818285) -- (4.91359,0.857852);
\draw [c] (4.91359,0.857852) -- (4.91359,0.897419);
\draw [c] (4.90545,0.857852) -- (4.91359,0.857852);
\draw [c] (4.91359,0.857852) -- (4.92173,0.857852);
\definecolor{c}{rgb}{0,0,0};
\colorlet{c}{natcomp!70};
\draw [c] (4.92986,0.839442) -- (4.92986,0.884558);
\draw [c] (4.92986,0.884558) -- (4.92986,0.929674);
\draw [c] (4.92173,0.884558) -- (4.92986,0.884558);
\draw [c] (4.92986,0.884558) -- (4.938,0.884558);
\definecolor{c}{rgb}{0,0,0};
\colorlet{c}{natcomp!70};
\draw [c] (4.94614,0.928982) -- (4.94614,0.983751);
\draw [c] (4.94614,0.983751) -- (4.94614,1.03852);
\draw [c] (4.938,0.983751) -- (4.94614,0.983751);
\draw [c] (4.94614,0.983751) -- (4.95427,0.983751);
\definecolor{c}{rgb}{0,0,0};
\colorlet{c}{natcomp!70};
\draw [c] (4.96241,0.756623) -- (4.96241,0.793018);
\draw [c] (4.96241,0.793018) -- (4.96241,0.829412);
\draw [c] (4.95427,0.793018) -- (4.96241,0.793018);
\draw [c] (4.96241,0.793018) -- (4.97055,0.793018);
\definecolor{c}{rgb}{0,0,0};
\colorlet{c}{natcomp!70};
\draw [c] (4.97868,0.863301) -- (4.97868,0.913408);
\draw [c] (4.97868,0.913408) -- (4.97868,0.963516);
\draw [c] (4.97055,0.913408) -- (4.97868,0.913408);
\draw [c] (4.97868,0.913408) -- (4.98682,0.913408);
\definecolor{c}{rgb}{0,0,0};
\colorlet{c}{natcomp!70};
\draw [c] (4.99495,0.820068) -- (4.99495,0.859989);
\draw [c] (4.99495,0.859989) -- (4.99495,0.89991);
\draw [c] (4.98682,0.859989) -- (4.99495,0.859989);
\draw [c] (4.99495,0.859989) -- (5.00309,0.859989);
\definecolor{c}{rgb}{0,0,0};
\colorlet{c}{natcomp!70};
\draw [c] (5.01123,0.781416) -- (5.01123,0.816202);
\draw [c] (5.01123,0.816202) -- (5.01123,0.850987);
\draw [c] (5.00309,0.816202) -- (5.01123,0.816202);
\draw [c] (5.01123,0.816202) -- (5.01936,0.816202);
\definecolor{c}{rgb}{0,0,0};
\colorlet{c}{natcomp!70};
\draw [c] (5.0275,0.790764) -- (5.0275,0.827697);
\draw [c] (5.0275,0.827697) -- (5.0275,0.86463);
\draw [c] (5.01936,0.827697) -- (5.0275,0.827697);
\draw [c] (5.0275,0.827697) -- (5.03564,0.827697);
\definecolor{c}{rgb}{0,0,0};
\colorlet{c}{natcomp!70};
\draw [c] (5.04377,0.750942) -- (5.04377,0.784172);
\draw [c] (5.04377,0.784172) -- (5.04377,0.817401);
\draw [c] (5.03564,0.784172) -- (5.04377,0.784172);
\draw [c] (5.04377,0.784172) -- (5.05191,0.784172);
\definecolor{c}{rgb}{0,0,0};
\colorlet{c}{natcomp!70};
\draw [c] (5.06005,0.881823) -- (5.06005,0.934976);
\draw [c] (5.06005,0.934976) -- (5.06005,0.988128);
\draw [c] (5.05191,0.934976) -- (5.06005,0.934976);
\draw [c] (5.06005,0.934976) -- (5.06818,0.934976);
\definecolor{c}{rgb}{0,0,0};
\colorlet{c}{natcomp!70};
\draw [c] (5.07632,0.879455) -- (5.07632,0.936439);
\draw [c] (5.07632,0.936439) -- (5.07632,0.993423);
\draw [c] (5.06818,0.936439) -- (5.07632,0.936439);
\draw [c] (5.07632,0.936439) -- (5.08445,0.936439);
\definecolor{c}{rgb}{0,0,0};
\colorlet{c}{natcomp!70};
\draw [c] (5.09259,0.765327) -- (5.09259,0.799544);
\draw [c] (5.09259,0.799544) -- (5.09259,0.833762);
\draw [c] (5.08445,0.799544) -- (5.09259,0.799544);
\draw [c] (5.09259,0.799544) -- (5.10073,0.799544);
\definecolor{c}{rgb}{0,0,0};
\colorlet{c}{natcomp!70};
\draw [c] (5.10886,0.816846) -- (5.10886,0.854773);
\draw [c] (5.10886,0.854773) -- (5.10886,0.8927);
\draw [c] (5.10073,0.854773) -- (5.10886,0.854773);
\draw [c] (5.10886,0.854773) -- (5.117,0.854773);
\definecolor{c}{rgb}{0,0,0};
\colorlet{c}{natcomp!70};
\draw [c] (5.12514,0.878289) -- (5.12514,0.930852);
\draw [c] (5.12514,0.930852) -- (5.12514,0.983415);
\draw [c] (5.117,0.930852) -- (5.12514,0.930852);
\draw [c] (5.12514,0.930852) -- (5.13327,0.930852);
\definecolor{c}{rgb}{0,0,0};
\colorlet{c}{natcomp!70};
\draw [c] (5.14141,0.886117) -- (5.14141,0.93627);
\draw [c] (5.14141,0.93627) -- (5.14141,0.986423);
\draw [c] (5.13327,0.93627) -- (5.14141,0.93627);
\draw [c] (5.14141,0.93627) -- (5.14955,0.93627);
\definecolor{c}{rgb}{0,0,0};
\colorlet{c}{natcomp!70};
\draw [c] (5.15768,0.756901) -- (5.15768,0.7883);
\draw [c] (5.15768,0.7883) -- (5.15768,0.819698);
\draw [c] (5.14955,0.7883) -- (5.15768,0.7883);
\draw [c] (5.15768,0.7883) -- (5.16582,0.7883);
\definecolor{c}{rgb}{0,0,0};
\colorlet{c}{natcomp!70};
\draw [c] (5.17395,0.850388) -- (5.17395,0.903041);
\draw [c] (5.17395,0.903041) -- (5.17395,0.955693);
\draw [c] (5.16582,0.903041) -- (5.17395,0.903041);
\draw [c] (5.17395,0.903041) -- (5.18209,0.903041);
\definecolor{c}{rgb}{0,0,0};
\colorlet{c}{natcomp!70};
\draw [c] (5.19023,0.837397) -- (5.19023,0.880272);
\draw [c] (5.19023,0.880272) -- (5.19023,0.923147);
\draw [c] (5.18209,0.880272) -- (5.19023,0.880272);
\draw [c] (5.19023,0.880272) -- (5.19836,0.880272);
\definecolor{c}{rgb}{0,0,0};
\colorlet{c}{natcomp!70};
\draw [c] (5.2065,0.782891) -- (5.2065,0.819807);
\draw [c] (5.2065,0.819807) -- (5.2065,0.856723);
\draw [c] (5.19836,0.819807) -- (5.2065,0.819807);
\draw [c] (5.2065,0.819807) -- (5.21464,0.819807);
\definecolor{c}{rgb}{0,0,0};
\colorlet{c}{natcomp!70};
\draw [c] (5.22277,0.877003) -- (5.22277,0.925132);
\draw [c] (5.22277,0.925132) -- (5.22277,0.973261);
\draw [c] (5.21464,0.925132) -- (5.22277,0.925132);
\draw [c] (5.22277,0.925132) -- (5.23091,0.925132);
\definecolor{c}{rgb}{0,0,0};
\colorlet{c}{natcomp!70};
\draw [c] (5.23905,0.832507) -- (5.23905,0.877913);
\draw [c] (5.23905,0.877913) -- (5.23905,0.923319);
\draw [c] (5.23091,0.877913) -- (5.23905,0.877913);
\draw [c] (5.23905,0.877913) -- (5.24718,0.877913);
\definecolor{c}{rgb}{0,0,0};
\colorlet{c}{natcomp!70};
\draw [c] (5.25532,0.809327) -- (5.25532,0.853797);
\draw [c] (5.25532,0.853797) -- (5.25532,0.898268);
\draw [c] (5.24718,0.853797) -- (5.25532,0.853797);
\draw [c] (5.25532,0.853797) -- (5.26345,0.853797);
\definecolor{c}{rgb}{0,0,0};
\colorlet{c}{natcomp!70};
\draw [c] (5.27159,0.838939) -- (5.27159,0.884592);
\draw [c] (5.27159,0.884592) -- (5.27159,0.930245);
\draw [c] (5.26345,0.884592) -- (5.27159,0.884592);
\draw [c] (5.27159,0.884592) -- (5.27973,0.884592);
\definecolor{c}{rgb}{0,0,0};
\colorlet{c}{natcomp!70};
\draw [c] (5.28786,0.774997) -- (5.28786,0.815419);
\draw [c] (5.28786,0.815419) -- (5.28786,0.85584);
\draw [c] (5.27973,0.815419) -- (5.28786,0.815419);
\draw [c] (5.28786,0.815419) -- (5.296,0.815419);
\definecolor{c}{rgb}{0,0,0};
\colorlet{c}{natcomp!70};
\draw [c] (5.30414,0.802019) -- (5.30414,0.840859);
\draw [c] (5.30414,0.840859) -- (5.30414,0.879699);
\draw [c] (5.296,0.840859) -- (5.30414,0.840859);
\draw [c] (5.30414,0.840859) -- (5.31227,0.840859);
\definecolor{c}{rgb}{0,0,0};
\colorlet{c}{natcomp!70};
\draw [c] (5.32041,0.734849) -- (5.32041,0.765883);
\draw [c] (5.32041,0.765883) -- (5.32041,0.796917);
\draw [c] (5.31227,0.765883) -- (5.32041,0.765883);
\draw [c] (5.32041,0.765883) -- (5.32855,0.765883);
\definecolor{c}{rgb}{0,0,0};
\colorlet{c}{natcomp!70};
\draw [c] (5.33668,0.76708) -- (5.33668,0.802514);
\draw [c] (5.33668,0.802514) -- (5.33668,0.837949);
\draw [c] (5.32855,0.802514) -- (5.33668,0.802514);
\draw [c] (5.33668,0.802514) -- (5.34482,0.802514);
\definecolor{c}{rgb}{0,0,0};
\colorlet{c}{natcomp!70};
\draw [c] (5.35295,0.820219) -- (5.35295,0.864005);
\draw [c] (5.35295,0.864005) -- (5.35295,0.907791);
\draw [c] (5.34482,0.864005) -- (5.35295,0.864005);
\draw [c] (5.35295,0.864005) -- (5.36109,0.864005);
\definecolor{c}{rgb}{0,0,0};
\colorlet{c}{natcomp!70};
\draw [c] (5.36923,0.792286) -- (5.36923,0.832453);
\draw [c] (5.36923,0.832453) -- (5.36923,0.87262);
\draw [c] (5.36109,0.832453) -- (5.36923,0.832453);
\draw [c] (5.36923,0.832453) -- (5.37736,0.832453);
\definecolor{c}{rgb}{0,0,0};
\colorlet{c}{natcomp!70};
\draw [c] (5.3855,0.779486) -- (5.3855,0.817789);
\draw [c] (5.3855,0.817789) -- (5.3855,0.856092);
\draw [c] (5.37736,0.817789) -- (5.3855,0.817789);
\draw [c] (5.3855,0.817789) -- (5.39364,0.817789);
\definecolor{c}{rgb}{0,0,0};
\colorlet{c}{natcomp!70};
\draw [c] (5.40177,0.758942) -- (5.40177,0.79033);
\draw [c] (5.40177,0.79033) -- (5.40177,0.821719);
\draw [c] (5.39364,0.79033) -- (5.40177,0.79033);
\draw [c] (5.40177,0.79033) -- (5.40991,0.79033);
\definecolor{c}{rgb}{0,0,0};
\colorlet{c}{natcomp!70};
\draw [c] (5.41805,0.828335) -- (5.41805,0.872313);
\draw [c] (5.41805,0.872313) -- (5.41805,0.916292);
\draw [c] (5.40991,0.872313) -- (5.41805,0.872313);
\draw [c] (5.41805,0.872313) -- (5.42618,0.872313);
\definecolor{c}{rgb}{0,0,0};
\colorlet{c}{natcomp!70};
\draw [c] (5.43432,0.796227) -- (5.43432,0.833442);
\draw [c] (5.43432,0.833442) -- (5.43432,0.870657);
\draw [c] (5.42618,0.833442) -- (5.43432,0.833442);
\draw [c] (5.43432,0.833442) -- (5.44245,0.833442);
\definecolor{c}{rgb}{0,0,0};
\colorlet{c}{natcomp!70};
\draw [c] (5.45059,0.824412) -- (5.45059,0.86803);
\draw [c] (5.45059,0.86803) -- (5.45059,0.911649);
\draw [c] (5.44245,0.86803) -- (5.45059,0.86803);
\draw [c] (5.45059,0.86803) -- (5.45873,0.86803);
\definecolor{c}{rgb}{0,0,0};
\colorlet{c}{natcomp!70};
\draw [c] (5.46686,0.749011) -- (5.46686,0.778106);
\draw [c] (5.46686,0.778106) -- (5.46686,0.807201);
\draw [c] (5.45873,0.778106) -- (5.46686,0.778106);
\draw [c] (5.46686,0.778106) -- (5.475,0.778106);
\definecolor{c}{rgb}{0,0,0};
\colorlet{c}{natcomp!70};
\draw [c] (5.48314,0.717158) -- (5.48314,0.738269);
\draw [c] (5.48314,0.738269) -- (5.48314,0.75938);
\draw [c] (5.475,0.738269) -- (5.48314,0.738269);
\draw [c] (5.48314,0.738269) -- (5.49127,0.738269);
\definecolor{c}{rgb}{0,0,0};
\colorlet{c}{natcomp!70};
\draw [c] (5.49941,0.760794) -- (5.49941,0.791025);
\draw [c] (5.49941,0.791025) -- (5.49941,0.821255);
\draw [c] (5.49127,0.791025) -- (5.49941,0.791025);
\draw [c] (5.49941,0.791025) -- (5.50755,0.791025);
\definecolor{c}{rgb}{0,0,0};
\colorlet{c}{natcomp!70};
\draw [c] (5.51568,0.780683) -- (5.51568,0.819473);
\draw [c] (5.51568,0.819473) -- (5.51568,0.858263);
\draw [c] (5.50755,0.819473) -- (5.51568,0.819473);
\draw [c] (5.51568,0.819473) -- (5.52382,0.819473);
\definecolor{c}{rgb}{0,0,0};
\colorlet{c}{natcomp!70};
\draw [c] (5.53195,0.767697) -- (5.53195,0.81179);
\draw [c] (5.53195,0.81179) -- (5.53195,0.855883);
\draw [c] (5.52382,0.81179) -- (5.53195,0.81179);
\draw [c] (5.53195,0.81179) -- (5.54009,0.81179);
\definecolor{c}{rgb}{0,0,0};
\colorlet{c}{natcomp!70};
\draw [c] (5.54823,0.789662) -- (5.54823,0.82657);
\draw [c] (5.54823,0.82657) -- (5.54823,0.863477);
\draw [c] (5.54009,0.82657) -- (5.54823,0.82657);
\draw [c] (5.54823,0.82657) -- (5.55636,0.82657);
\definecolor{c}{rgb}{0,0,0};
\colorlet{c}{natcomp!70};
\draw [c] (5.5645,0.836001) -- (5.5645,0.881788);
\draw [c] (5.5645,0.881788) -- (5.5645,0.927576);
\draw [c] (5.55636,0.881788) -- (5.5645,0.881788);
\draw [c] (5.5645,0.881788) -- (5.57264,0.881788);
\definecolor{c}{rgb}{0,0,0};
\colorlet{c}{natcomp!70};
\draw [c] (5.58077,0.77978) -- (5.58077,0.81503);
\draw [c] (5.58077,0.81503) -- (5.58077,0.85028);
\draw [c] (5.57264,0.81503) -- (5.58077,0.81503);
\draw [c] (5.58077,0.81503) -- (5.58891,0.81503);
\definecolor{c}{rgb}{0,0,0};
\colorlet{c}{natcomp!70};
\draw [c] (5.59705,0.765328) -- (5.59705,0.798276);
\draw [c] (5.59705,0.798276) -- (5.59705,0.831225);
\draw [c] (5.58891,0.798276) -- (5.59705,0.798276);
\draw [c] (5.59705,0.798276) -- (5.60518,0.798276);
\definecolor{c}{rgb}{0,0,0};
\colorlet{c}{natcomp!70};
\draw [c] (5.61332,0.713101) -- (5.61332,0.735885);
\draw [c] (5.61332,0.735885) -- (5.61332,0.758668);
\draw [c] (5.60518,0.735885) -- (5.61332,0.735885);
\draw [c] (5.61332,0.735885) -- (5.62145,0.735885);
\definecolor{c}{rgb}{0,0,0};
\colorlet{c}{natcomp!70};
\draw [c] (5.62959,0.77156) -- (5.62959,0.806931);
\draw [c] (5.62959,0.806931) -- (5.62959,0.842301);
\draw [c] (5.62145,0.806931) -- (5.62959,0.806931);
\draw [c] (5.62959,0.806931) -- (5.63773,0.806931);
\definecolor{c}{rgb}{0,0,0};
\colorlet{c}{natcomp!70};
\draw [c] (5.64586,0.767598) -- (5.64586,0.801089);
\draw [c] (5.64586,0.801089) -- (5.64586,0.83458);
\draw [c] (5.63773,0.801089) -- (5.64586,0.801089);
\draw [c] (5.64586,0.801089) -- (5.654,0.801089);
\definecolor{c}{rgb}{0,0,0};
\colorlet{c}{natcomp!70};
\draw [c] (5.66214,0.812558) -- (5.66214,0.852249);
\draw [c] (5.66214,0.852249) -- (5.66214,0.891941);
\draw [c] (5.654,0.852249) -- (5.66214,0.852249);
\draw [c] (5.66214,0.852249) -- (5.67027,0.852249);
\definecolor{c}{rgb}{0,0,0};
\colorlet{c}{natcomp!70};
\draw [c] (5.67841,0.750819) -- (5.67841,0.790319);
\draw [c] (5.67841,0.790319) -- (5.67841,0.829819);
\draw [c] (5.67027,0.790319) -- (5.67841,0.790319);
\draw [c] (5.67841,0.790319) -- (5.68655,0.790319);
\definecolor{c}{rgb}{0,0,0};
\colorlet{c}{natcomp!70};
\draw [c] (5.69468,0.763809) -- (5.69468,0.800161);
\draw [c] (5.69468,0.800161) -- (5.69468,0.836514);
\draw [c] (5.68655,0.800161) -- (5.69468,0.800161);
\draw [c] (5.69468,0.800161) -- (5.70282,0.800161);
\definecolor{c}{rgb}{0,0,0};
\colorlet{c}{natcomp!70};
\draw [c] (5.71095,0.773509) -- (5.71095,0.810299);
\draw [c] (5.71095,0.810299) -- (5.71095,0.84709);
\draw [c] (5.70282,0.810299) -- (5.71095,0.810299);
\draw [c] (5.71095,0.810299) -- (5.71909,0.810299);
\definecolor{c}{rgb}{0,0,0};
\colorlet{c}{natcomp!70};
\draw [c] (5.72723,0.781311) -- (5.72723,0.815949);
\draw [c] (5.72723,0.815949) -- (5.72723,0.850586);
\draw [c] (5.71909,0.815949) -- (5.72723,0.815949);
\draw [c] (5.72723,0.815949) -- (5.73536,0.815949);
\definecolor{c}{rgb}{0,0,0};
\colorlet{c}{natcomp!70};
\draw [c] (5.7435,0.749041) -- (5.7435,0.783543);
\draw [c] (5.7435,0.783543) -- (5.7435,0.818044);
\draw [c] (5.73536,0.783543) -- (5.7435,0.783543);
\draw [c] (5.7435,0.783543) -- (5.75164,0.783543);
\definecolor{c}{rgb}{0,0,0};
\colorlet{c}{natcomp!70};
\draw [c] (5.75977,0.734525) -- (5.75977,0.760913);
\draw [c] (5.75977,0.760913) -- (5.75977,0.787301);
\draw [c] (5.75164,0.760913) -- (5.75977,0.760913);
\draw [c] (5.75977,0.760913) -- (5.76791,0.760913);
\definecolor{c}{rgb}{0,0,0};
\colorlet{c}{natcomp!70};
\draw [c] (5.77605,0.757329) -- (5.77605,0.795295);
\draw [c] (5.77605,0.795295) -- (5.77605,0.833262);
\draw [c] (5.76791,0.795295) -- (5.77605,0.795295);
\draw [c] (5.77605,0.795295) -- (5.78418,0.795295);
\definecolor{c}{rgb}{0,0,0};
\colorlet{c}{natcomp!70};
\draw [c] (5.79232,0.778456) -- (5.79232,0.824701);
\draw [c] (5.79232,0.824701) -- (5.79232,0.870946);
\draw [c] (5.78418,0.824701) -- (5.79232,0.824701);
\draw [c] (5.79232,0.824701) -- (5.80045,0.824701);
\definecolor{c}{rgb}{0,0,0};
\colorlet{c}{natcomp!70};
\draw [c] (5.80859,0.791499) -- (5.80859,0.828233);
\draw [c] (5.80859,0.828233) -- (5.80859,0.864968);
\draw [c] (5.80045,0.828233) -- (5.80859,0.828233);
\draw [c] (5.80859,0.828233) -- (5.81673,0.828233);
\definecolor{c}{rgb}{0,0,0};
\colorlet{c}{natcomp!70};
\draw [c] (5.82486,0.778777) -- (5.82486,0.81849);
\draw [c] (5.82486,0.81849) -- (5.82486,0.858202);
\draw [c] (5.81673,0.81849) -- (5.82486,0.81849);
\draw [c] (5.82486,0.81849) -- (5.833,0.81849);
\definecolor{c}{rgb}{0,0,0};
\colorlet{c}{natcomp!70};
\draw [c] (5.84114,0.766161) -- (5.84114,0.815256);
\draw [c] (5.84114,0.815256) -- (5.84114,0.864352);
\draw [c] (5.833,0.815256) -- (5.84114,0.815256);
\draw [c] (5.84114,0.815256) -- (5.84927,0.815256);
\definecolor{c}{rgb}{0,0,0};
\colorlet{c}{natcomp!70};
\draw [c] (5.85741,0.739717) -- (5.85741,0.766483);
\draw [c] (5.85741,0.766483) -- (5.85741,0.79325);
\draw [c] (5.84927,0.766483) -- (5.85741,0.766483);
\draw [c] (5.85741,0.766483) -- (5.86555,0.766483);
\definecolor{c}{rgb}{0,0,0};
\colorlet{c}{natcomp!70};
\draw [c] (5.87368,0.767167) -- (5.87368,0.798411);
\draw [c] (5.87368,0.798411) -- (5.87368,0.829655);
\draw [c] (5.86555,0.798411) -- (5.87368,0.798411);
\draw [c] (5.87368,0.798411) -- (5.88182,0.798411);
\definecolor{c}{rgb}{0,0,0};
\colorlet{c}{natcomp!70};
\draw [c] (5.88995,0.7537) -- (5.88995,0.78274);
\draw [c] (5.88995,0.78274) -- (5.88995,0.811781);
\draw [c] (5.88182,0.78274) -- (5.88995,0.78274);
\draw [c] (5.88995,0.78274) -- (5.89809,0.78274);
\definecolor{c}{rgb}{0,0,0};
\colorlet{c}{natcomp!70};
\draw [c] (5.90623,0.787736) -- (5.90623,0.823567);
\draw [c] (5.90623,0.823567) -- (5.90623,0.859398);
\draw [c] (5.89809,0.823567) -- (5.90623,0.823567);
\draw [c] (5.90623,0.823567) -- (5.91436,0.823567);
\definecolor{c}{rgb}{0,0,0};
\colorlet{c}{natcomp!70};
\draw [c] (5.9225,0.7657) -- (5.9225,0.797966);
\draw [c] (5.9225,0.797966) -- (5.9225,0.830233);
\draw [c] (5.91436,0.797966) -- (5.9225,0.797966);
\draw [c] (5.9225,0.797966) -- (5.93064,0.797966);
\definecolor{c}{rgb}{0,0,0};
\colorlet{c}{natcomp!70};
\draw [c] (5.93877,0.776906) -- (5.93877,0.812086);
\draw [c] (5.93877,0.812086) -- (5.93877,0.847266);
\draw [c] (5.93064,0.812086) -- (5.93877,0.812086);
\draw [c] (5.93877,0.812086) -- (5.94691,0.812086);
\definecolor{c}{rgb}{0,0,0};
\colorlet{c}{natcomp!70};
\draw [c] (5.95505,0.727034) -- (5.95505,0.751779);
\draw [c] (5.95505,0.751779) -- (5.95505,0.776524);
\draw [c] (5.94691,0.751779) -- (5.95505,0.751779);
\draw [c] (5.95505,0.751779) -- (5.96318,0.751779);
\definecolor{c}{rgb}{0,0,0};
\colorlet{c}{natcomp!70};
\draw [c] (5.97132,0.718677) -- (5.97132,0.7484);
\draw [c] (5.97132,0.7484) -- (5.97132,0.778123);
\draw [c] (5.96318,0.7484) -- (5.97132,0.7484);
\draw [c] (5.97132,0.7484) -- (5.97945,0.7484);
\definecolor{c}{rgb}{0,0,0};
\colorlet{c}{natcomp!70};
\draw [c] (5.98759,0.723722) -- (5.98759,0.746507);
\draw [c] (5.98759,0.746507) -- (5.98759,0.769292);
\draw [c] (5.97945,0.746507) -- (5.98759,0.746507);
\draw [c] (5.98759,0.746507) -- (5.99573,0.746507);
\definecolor{c}{rgb}{0,0,0};
\colorlet{c}{natcomp!70};
\draw [c] (6.00386,0.756977) -- (6.00386,0.792774);
\draw [c] (6.00386,0.792774) -- (6.00386,0.828571);
\draw [c] (5.99573,0.792774) -- (6.00386,0.792774);
\draw [c] (6.00386,0.792774) -- (6.012,0.792774);
\definecolor{c}{rgb}{0,0,0};
\colorlet{c}{natcomp!70};
\draw [c] (6.02014,0.72654) -- (6.02014,0.750922);
\draw [c] (6.02014,0.750922) -- (6.02014,0.775304);
\draw [c] (6.012,0.750922) -- (6.02014,0.750922);
\draw [c] (6.02014,0.750922) -- (6.02827,0.750922);
\definecolor{c}{rgb}{0,0,0};
\colorlet{c}{natcomp!70};
\draw [c] (6.03641,0.765876) -- (6.03641,0.798169);
\draw [c] (6.03641,0.798169) -- (6.03641,0.830462);
\draw [c] (6.02827,0.798169) -- (6.03641,0.798169);
\draw [c] (6.03641,0.798169) -- (6.04455,0.798169);
\definecolor{c}{rgb}{0,0,0};
\colorlet{c}{natcomp!70};
\draw [c] (6.05268,0.723692) -- (6.05268,0.763257);
\draw [c] (6.05268,0.763257) -- (6.05268,0.802821);
\draw [c] (6.04455,0.763257) -- (6.05268,0.763257);
\draw [c] (6.05268,0.763257) -- (6.06082,0.763257);
\definecolor{c}{rgb}{0,0,0};
\colorlet{c}{natcomp!70};
\draw [c] (6.06895,0.735947) -- (6.06895,0.776286);
\draw [c] (6.06895,0.776286) -- (6.06895,0.816626);
\draw [c] (6.06082,0.776286) -- (6.06895,0.776286);
\draw [c] (6.06895,0.776286) -- (6.07709,0.776286);
\definecolor{c}{rgb}{0,0,0};
\colorlet{c}{natcomp!70};
\draw [c] (6.08523,0.768199) -- (6.08523,0.8023);
\draw [c] (6.08523,0.8023) -- (6.08523,0.836401);
\draw [c] (6.07709,0.8023) -- (6.08523,0.8023);
\draw [c] (6.08523,0.8023) -- (6.09336,0.8023);
\definecolor{c}{rgb}{0,0,0};
\colorlet{c}{natcomp!70};
\draw [c] (6.1015,0.710277) -- (6.1015,0.729228);
\draw [c] (6.1015,0.729228) -- (6.1015,0.748179);
\draw [c] (6.09336,0.729228) -- (6.1015,0.729228);
\draw [c] (6.1015,0.729228) -- (6.10964,0.729228);
\definecolor{c}{rgb}{0,0,0};
\colorlet{c}{natcomp!70};
\draw [c] (6.11777,0.740692) -- (6.11777,0.767798);
\draw [c] (6.11777,0.767798) -- (6.11777,0.794904);
\draw [c] (6.10964,0.767798) -- (6.11777,0.767798);
\draw [c] (6.11777,0.767798) -- (6.12591,0.767798);
\definecolor{c}{rgb}{0,0,0};
\colorlet{c}{natcomp!70};
\draw [c] (6.13405,0.720528) -- (6.13405,0.743887);
\draw [c] (6.13405,0.743887) -- (6.13405,0.767247);
\draw [c] (6.12591,0.743887) -- (6.13405,0.743887);
\draw [c] (6.13405,0.743887) -- (6.14218,0.743887);
\definecolor{c}{rgb}{0,0,0};
\colorlet{c}{natcomp!70};
\draw [c] (6.15032,0.69748) -- (6.15032,0.712307);
\draw [c] (6.15032,0.712307) -- (6.15032,0.727135);
\draw [c] (6.14218,0.712307) -- (6.15032,0.712307);
\draw [c] (6.15032,0.712307) -- (6.15845,0.712307);
\definecolor{c}{rgb}{0,0,0};
\colorlet{c}{natcomp!70};
\draw [c] (6.16659,0.709648) -- (6.16659,0.728192);
\draw [c] (6.16659,0.728192) -- (6.16659,0.746736);
\draw [c] (6.15845,0.728192) -- (6.16659,0.728192);
\draw [c] (6.16659,0.728192) -- (6.17473,0.728192);
\definecolor{c}{rgb}{0,0,0};
\colorlet{c}{natcomp!70};
\draw [c] (6.18286,0.7346) -- (6.18286,0.766423);
\draw [c] (6.18286,0.766423) -- (6.18286,0.798247);
\draw [c] (6.17473,0.766423) -- (6.18286,0.766423);
\draw [c] (6.18286,0.766423) -- (6.191,0.766423);
\definecolor{c}{rgb}{0,0,0};
\colorlet{c}{natcomp!70};
\draw [c] (6.19914,0.725682) -- (6.19914,0.753499);
\draw [c] (6.19914,0.753499) -- (6.19914,0.781317);
\draw [c] (6.191,0.753499) -- (6.19914,0.753499);
\draw [c] (6.19914,0.753499) -- (6.20727,0.753499);
\definecolor{c}{rgb}{0,0,0};
\colorlet{c}{natcomp!70};
\draw [c] (6.21541,0.726686) -- (6.21541,0.759284);
\draw [c] (6.21541,0.759284) -- (6.21541,0.791882);
\draw [c] (6.20727,0.759284) -- (6.21541,0.759284);
\draw [c] (6.21541,0.759284) -- (6.22355,0.759284);
\definecolor{c}{rgb}{0,0,0};
\colorlet{c}{natcomp!70};
\draw [c] (6.23168,0.742176) -- (6.23168,0.777624);
\draw [c] (6.23168,0.777624) -- (6.23168,0.813072);
\draw [c] (6.22355,0.777624) -- (6.23168,0.777624);
\draw [c] (6.23168,0.777624) -- (6.23982,0.777624);
\definecolor{c}{rgb}{0,0,0};
\colorlet{c}{natcomp!70};
\draw [c] (6.24795,0.776333) -- (6.24795,0.816985);
\draw [c] (6.24795,0.816985) -- (6.24795,0.857637);
\draw [c] (6.23982,0.816985) -- (6.24795,0.816985);
\draw [c] (6.24795,0.816985) -- (6.25609,0.816985);
\definecolor{c}{rgb}{0,0,0};
\colorlet{c}{natcomp!70};
\draw [c] (6.26423,0.716593) -- (6.26423,0.737387);
\draw [c] (6.26423,0.737387) -- (6.26423,0.758182);
\draw [c] (6.25609,0.737387) -- (6.26423,0.737387);
\draw [c] (6.26423,0.737387) -- (6.27236,0.737387);
\definecolor{c}{rgb}{0,0,0};
\colorlet{c}{natcomp!70};
\draw [c] (6.2805,0.73157) -- (6.2805,0.759414);
\draw [c] (6.2805,0.759414) -- (6.2805,0.787258);
\draw [c] (6.27236,0.759414) -- (6.2805,0.759414);
\draw [c] (6.2805,0.759414) -- (6.28864,0.759414);
\definecolor{c}{rgb}{0,0,0};
\colorlet{c}{natcomp!70};
\draw [c] (6.29677,0.765117) -- (6.29677,0.800819);
\draw [c] (6.29677,0.800819) -- (6.29677,0.83652);
\draw [c] (6.28864,0.800819) -- (6.29677,0.800819);
\draw [c] (6.29677,0.800819) -- (6.30491,0.800819);
\definecolor{c}{rgb}{0,0,0};
\colorlet{c}{natcomp!70};
\draw [c] (6.31305,0.697057) -- (6.31305,0.711255);
\draw [c] (6.31305,0.711255) -- (6.31305,0.725454);
\draw [c] (6.30491,0.711255) -- (6.31305,0.711255);
\draw [c] (6.31305,0.711255) -- (6.32118,0.711255);
\definecolor{c}{rgb}{0,0,0};
\colorlet{c}{natcomp!70};
\draw [c] (6.32932,0.699766) -- (6.32932,0.721515);
\draw [c] (6.32932,0.721515) -- (6.32932,0.743263);
\draw [c] (6.32118,0.721515) -- (6.32932,0.721515);
\draw [c] (6.32932,0.721515) -- (6.33745,0.721515);
\definecolor{c}{rgb}{0,0,0};
\colorlet{c}{natcomp!70};
\draw [c] (6.34559,0.75193) -- (6.34559,0.784287);
\draw [c] (6.34559,0.784287) -- (6.34559,0.816645);
\draw [c] (6.33745,0.784287) -- (6.34559,0.784287);
\draw [c] (6.34559,0.784287) -- (6.35373,0.784287);
\definecolor{c}{rgb}{0,0,0};
\colorlet{c}{natcomp!70};
\draw [c] (6.36186,0.751842) -- (6.36186,0.788755);
\draw [c] (6.36186,0.788755) -- (6.36186,0.825668);
\draw [c] (6.35373,0.788755) -- (6.36186,0.788755);
\draw [c] (6.36186,0.788755) -- (6.37,0.788755);
\definecolor{c}{rgb}{0,0,0};
\colorlet{c}{natcomp!70};
\draw [c] (6.37814,0.740937) -- (6.37814,0.773723);
\draw [c] (6.37814,0.773723) -- (6.37814,0.806508);
\draw [c] (6.37,0.773723) -- (6.37814,0.773723);
\draw [c] (6.37814,0.773723) -- (6.38627,0.773723);
\definecolor{c}{rgb}{0,0,0};
\colorlet{c}{natcomp!70};
\draw [c] (6.39441,0.698759) -- (6.39441,0.715275);
\draw [c] (6.39441,0.715275) -- (6.39441,0.73179);
\draw [c] (6.38627,0.715275) -- (6.39441,0.715275);
\draw [c] (6.39441,0.715275) -- (6.40255,0.715275);
\definecolor{c}{rgb}{0,0,0};
\colorlet{c}{natcomp!70};
\draw [c] (6.41068,0.710571) -- (6.41068,0.730263);
\draw [c] (6.41068,0.730263) -- (6.41068,0.749955);
\draw [c] (6.40255,0.730263) -- (6.41068,0.730263);
\draw [c] (6.41068,0.730263) -- (6.41882,0.730263);
\definecolor{c}{rgb}{0,0,0};
\colorlet{c}{natcomp!70};
\draw [c] (6.42695,0.733223) -- (6.42695,0.762826);
\draw [c] (6.42695,0.762826) -- (6.42695,0.792429);
\draw [c] (6.41882,0.762826) -- (6.42695,0.762826);
\draw [c] (6.42695,0.762826) -- (6.43509,0.762826);
\definecolor{c}{rgb}{0,0,0};
\colorlet{c}{natcomp!70};
\draw [c] (6.44323,0.722776) -- (6.44323,0.747622);
\draw [c] (6.44323,0.747622) -- (6.44323,0.772467);
\draw [c] (6.43509,0.747622) -- (6.44323,0.747622);
\draw [c] (6.44323,0.747622) -- (6.45136,0.747622);
\definecolor{c}{rgb}{0,0,0};
\colorlet{c}{natcomp!70};
\draw [c] (6.4595,0.739345) -- (6.4595,0.766635);
\draw [c] (6.4595,0.766635) -- (6.4595,0.793925);
\draw [c] (6.45136,0.766635) -- (6.4595,0.766635);
\draw [c] (6.4595,0.766635) -- (6.46764,0.766635);
\definecolor{c}{rgb}{0,0,0};
\colorlet{c}{natcomp!70};
\draw [c] (6.47577,0.710389) -- (6.47577,0.729721);
\draw [c] (6.47577,0.729721) -- (6.47577,0.749052);
\draw [c] (6.46764,0.729721) -- (6.47577,0.729721);
\draw [c] (6.47577,0.729721) -- (6.48391,0.729721);
\definecolor{c}{rgb}{0,0,0};
\colorlet{c}{natcomp!70};
\draw [c] (6.49205,0.741601) -- (6.49205,0.774287);
\draw [c] (6.49205,0.774287) -- (6.49205,0.806974);
\draw [c] (6.48391,0.774287) -- (6.49205,0.774287);
\draw [c] (6.49205,0.774287) -- (6.50018,0.774287);
\definecolor{c}{rgb}{0,0,0};
\colorlet{c}{natcomp!70};
\draw [c] (6.50832,0.704581) -- (6.50832,0.722973);
\draw [c] (6.50832,0.722973) -- (6.50832,0.741364);
\draw [c] (6.50018,0.722973) -- (6.50832,0.722973);
\draw [c] (6.50832,0.722973) -- (6.51645,0.722973);
\definecolor{c}{rgb}{0,0,0};
\colorlet{c}{natcomp!70};
\draw [c] (6.52459,0.698445) -- (6.52459,0.714356);
\draw [c] (6.52459,0.714356) -- (6.52459,0.730267);
\draw [c] (6.51645,0.714356) -- (6.52459,0.714356);
\draw [c] (6.52459,0.714356) -- (6.53273,0.714356);
\definecolor{c}{rgb}{0,0,0};
\colorlet{c}{natcomp!70};
\draw [c] (6.54086,0.713893) -- (6.54086,0.736042);
\draw [c] (6.54086,0.736042) -- (6.54086,0.758192);
\draw [c] (6.53273,0.736042) -- (6.54086,0.736042);
\draw [c] (6.54086,0.736042) -- (6.549,0.736042);
\definecolor{c}{rgb}{0,0,0};
\colorlet{c}{natcomp!70};
\draw [c] (6.55714,0.705456) -- (6.55714,0.724441);
\draw [c] (6.55714,0.724441) -- (6.55714,0.743426);
\draw [c] (6.549,0.724441) -- (6.55714,0.724441);
\draw [c] (6.55714,0.724441) -- (6.56527,0.724441);
\definecolor{c}{rgb}{0,0,0};
\colorlet{c}{natcomp!70};
\draw [c] (6.57341,0.708858) -- (6.57341,0.726682);
\draw [c] (6.57341,0.726682) -- (6.57341,0.744506);
\draw [c] (6.56527,0.726682) -- (6.57341,0.726682);
\draw [c] (6.57341,0.726682) -- (6.58155,0.726682);
\definecolor{c}{rgb}{0,0,0};
\colorlet{c}{natcomp!70};
\draw [c] (6.58968,0.744164) -- (6.58968,0.773304);
\draw [c] (6.58968,0.773304) -- (6.58968,0.802443);
\draw [c] (6.58155,0.773304) -- (6.58968,0.773304);
\draw [c] (6.58968,0.773304) -- (6.59782,0.773304);
\definecolor{c}{rgb}{0,0,0};
\colorlet{c}{natcomp!70};
\draw [c] (6.60595,0.728187) -- (6.60595,0.754684);
\draw [c] (6.60595,0.754684) -- (6.60595,0.781181);
\draw [c] (6.59782,0.754684) -- (6.60595,0.754684);
\draw [c] (6.60595,0.754684) -- (6.61409,0.754684);
\definecolor{c}{rgb}{0,0,0};
\colorlet{c}{natcomp!70};
\draw [c] (6.62223,0.716607) -- (6.62223,0.741593);
\draw [c] (6.62223,0.741593) -- (6.62223,0.766578);
\draw [c] (6.61409,0.741593) -- (6.62223,0.741593);
\draw [c] (6.62223,0.741593) -- (6.63036,0.741593);
\definecolor{c}{rgb}{0,0,0};
\colorlet{c}{natcomp!70};
\draw [c] (6.6385,0.699804) -- (6.6385,0.717549);
\draw [c] (6.6385,0.717549) -- (6.6385,0.735295);
\draw [c] (6.63036,0.717549) -- (6.6385,0.717549);
\draw [c] (6.6385,0.717549) -- (6.64664,0.717549);
\definecolor{c}{rgb}{0,0,0};
\colorlet{c}{natcomp!70};
\draw [c] (6.65477,0.753668) -- (6.65477,0.789033);
\draw [c] (6.65477,0.789033) -- (6.65477,0.824398);
\draw [c] (6.64664,0.789033) -- (6.65477,0.789033);
\draw [c] (6.65477,0.789033) -- (6.66291,0.789033);
\definecolor{c}{rgb}{0,0,0};
\colorlet{c}{natcomp!70};
\draw [c] (6.67105,0.708234) -- (6.67105,0.734765);
\draw [c] (6.67105,0.734765) -- (6.67105,0.761297);
\draw [c] (6.66291,0.734765) -- (6.67105,0.734765);
\draw [c] (6.67105,0.734765) -- (6.67918,0.734765);
\definecolor{c}{rgb}{0,0,0};
\colorlet{c}{natcomp!70};
\draw [c] (6.68732,0.713447) -- (6.68732,0.738205);
\draw [c] (6.68732,0.738205) -- (6.68732,0.762963);
\draw [c] (6.67918,0.738205) -- (6.68732,0.738205);
\draw [c] (6.68732,0.738205) -- (6.69545,0.738205);
\definecolor{c}{rgb}{0,0,0};
\colorlet{c}{natcomp!70};
\draw [c] (6.70359,0.733101) -- (6.70359,0.759424);
\draw [c] (6.70359,0.759424) -- (6.70359,0.785747);
\draw [c] (6.69545,0.759424) -- (6.70359,0.759424);
\draw [c] (6.70359,0.759424) -- (6.71173,0.759424);
\definecolor{c}{rgb}{0,0,0};
\colorlet{c}{natcomp!70};
\draw [c] (6.71986,0.703929) -- (6.71986,0.721255);
\draw [c] (6.71986,0.721255) -- (6.71986,0.738582);
\draw [c] (6.71173,0.721255) -- (6.71986,0.721255);
\draw [c] (6.71986,0.721255) -- (6.728,0.721255);
\definecolor{c}{rgb}{0,0,0};
\colorlet{c}{natcomp!70};
\draw [c] (6.73614,0.704397) -- (6.73614,0.721875);
\draw [c] (6.73614,0.721875) -- (6.73614,0.739353);
\draw [c] (6.728,0.721875) -- (6.73614,0.721875);
\draw [c] (6.73614,0.721875) -- (6.74427,0.721875);
\definecolor{c}{rgb}{0,0,0};
\colorlet{c}{natcomp!70};
\draw [c] (6.75241,0.710501) -- (6.75241,0.729699);
\draw [c] (6.75241,0.729699) -- (6.75241,0.748896);
\draw [c] (6.74427,0.729699) -- (6.75241,0.729699);
\draw [c] (6.75241,0.729699) -- (6.76055,0.729699);
\definecolor{c}{rgb}{0,0,0};
\colorlet{c}{natcomp!70};
\draw [c] (6.76868,0.711589) -- (6.76868,0.731751);
\draw [c] (6.76868,0.731751) -- (6.76868,0.751914);
\draw [c] (6.76055,0.731751) -- (6.76868,0.731751);
\draw [c] (6.76868,0.731751) -- (6.77682,0.731751);
\definecolor{c}{rgb}{0,0,0};
\colorlet{c}{natcomp!70};
\draw [c] (6.78495,0.718017) -- (6.78495,0.745245);
\draw [c] (6.78495,0.745245) -- (6.78495,0.772473);
\draw [c] (6.77682,0.745245) -- (6.78495,0.745245);
\draw [c] (6.78495,0.745245) -- (6.79309,0.745245);
\definecolor{c}{rgb}{0,0,0};
\colorlet{c}{natcomp!70};
\draw [c] (6.80123,0.697887) -- (6.80123,0.713515);
\draw [c] (6.80123,0.713515) -- (6.80123,0.729144);
\draw [c] (6.79309,0.713515) -- (6.80123,0.713515);
\draw [c] (6.80123,0.713515) -- (6.80936,0.713515);
\definecolor{c}{rgb}{0,0,0};
\colorlet{c}{natcomp!70};
\draw [c] (6.8175,0.712023) -- (6.8175,0.732788);
\draw [c] (6.8175,0.732788) -- (6.8175,0.753553);
\draw [c] (6.80936,0.732788) -- (6.8175,0.732788);
\draw [c] (6.8175,0.732788) -- (6.82564,0.732788);
\definecolor{c}{rgb}{0,0,0};
\colorlet{c}{natcomp!70};
\draw [c] (6.83377,0.714232) -- (6.83377,0.736446);
\draw [c] (6.83377,0.736446) -- (6.83377,0.758661);
\draw [c] (6.82564,0.736446) -- (6.83377,0.736446);
\draw [c] (6.83377,0.736446) -- (6.84191,0.736446);
\definecolor{c}{rgb}{0,0,0};
\colorlet{c}{natcomp!70};
\draw [c] (6.85005,0.709462) -- (6.85005,0.739161);
\draw [c] (6.85005,0.739161) -- (6.85005,0.768861);
\draw [c] (6.84191,0.739161) -- (6.85005,0.739161);
\draw [c] (6.85005,0.739161) -- (6.85818,0.739161);
\definecolor{c}{rgb}{0,0,0};
\colorlet{c}{natcomp!70};
\draw [c] (6.86632,0.69326) -- (6.86632,0.708575);
\draw [c] (6.86632,0.708575) -- (6.86632,0.72389);
\draw [c] (6.85818,0.708575) -- (6.86632,0.708575);
\draw [c] (6.86632,0.708575) -- (6.87445,0.708575);
\definecolor{c}{rgb}{0,0,0};
\colorlet{c}{natcomp!70};
\draw [c] (6.88259,0.703644) -- (6.88259,0.720612);
\draw [c] (6.88259,0.720612) -- (6.88259,0.73758);
\draw [c] (6.87445,0.720612) -- (6.88259,0.720612);
\draw [c] (6.88259,0.720612) -- (6.89073,0.720612);
\definecolor{c}{rgb}{0,0,0};
\colorlet{c}{natcomp!70};
\draw [c] (6.89886,0.692271) -- (6.89886,0.705485);
\draw [c] (6.89886,0.705485) -- (6.89886,0.7187);
\draw [c] (6.89073,0.705485) -- (6.89886,0.705485);
\draw [c] (6.89886,0.705485) -- (6.907,0.705485);
\definecolor{c}{rgb}{0,0,0};
\colorlet{c}{natcomp!70};
\draw [c] (6.91514,0.692587) -- (6.91514,0.706325);
\draw [c] (6.91514,0.706325) -- (6.91514,0.720063);
\draw [c] (6.907,0.706325) -- (6.91514,0.706325);
\draw [c] (6.91514,0.706325) -- (6.92327,0.706325);
\definecolor{c}{rgb}{0,0,0};
\colorlet{c}{natcomp!70};
\draw [c] (6.93141,0.686927) -- (6.93141,0.695065);
\draw [c] (6.93141,0.695065) -- (6.93141,0.703204);
\draw [c] (6.92327,0.695065) -- (6.93141,0.695065);
\draw [c] (6.93141,0.695065) -- (6.93955,0.695065);
\definecolor{c}{rgb}{0,0,0};
\colorlet{c}{natcomp!70};
\draw [c] (6.94768,0.704333) -- (6.94768,0.7221);
\draw [c] (6.94768,0.7221) -- (6.94768,0.739867);
\draw [c] (6.93955,0.7221) -- (6.94768,0.7221);
\draw [c] (6.94768,0.7221) -- (6.95582,0.7221);
\definecolor{c}{rgb}{0,0,0};
\colorlet{c}{natcomp!70};
\draw [c] (6.96395,0.686909) -- (6.96395,0.68692);
\draw [c] (6.96395,0.68692) -- (6.96395,0.686931);
\draw [c] (6.95582,0.68692) -- (6.96395,0.68692);
\draw [c] (6.96395,0.68692) -- (6.97209,0.68692);
\definecolor{c}{rgb}{0,0,0};
\colorlet{c}{natcomp!70};
\draw [c] (6.98023,0.718409) -- (6.98023,0.740294);
\draw [c] (6.98023,0.740294) -- (6.98023,0.762179);
\draw [c] (6.97209,0.740294) -- (6.98023,0.740294);
\draw [c] (6.98023,0.740294) -- (6.98836,0.740294);
\definecolor{c}{rgb}{0,0,0};
\colorlet{c}{natcomp!70};
\draw [c] (6.9965,0.686917) -- (6.9965,0.68693);
\draw [c] (6.9965,0.68693) -- (6.9965,0.686943);
\draw [c] (6.98836,0.68693) -- (6.9965,0.68693);
\draw [c] (6.9965,0.68693) -- (7.00464,0.68693);
\definecolor{c}{rgb}{0,0,0};
\colorlet{c}{natcomp!70};
\draw [c] (7.01277,0.713906) -- (7.01277,0.736963);
\draw [c] (7.01277,0.736963) -- (7.01277,0.760019);
\draw [c] (7.00464,0.736963) -- (7.01277,0.736963);
\draw [c] (7.01277,0.736963) -- (7.02091,0.736963);
\definecolor{c}{rgb}{0,0,0};
\colorlet{c}{natcomp!70};
\draw [c] (7.02905,0.703869) -- (7.02905,0.721431);
\draw [c] (7.02905,0.721431) -- (7.02905,0.738992);
\draw [c] (7.02091,0.721431) -- (7.02905,0.721431);
\draw [c] (7.02905,0.721431) -- (7.03718,0.721431);
\definecolor{c}{rgb}{0,0,0};
\colorlet{c}{natcomp!70};
\draw [c] (7.04532,0.705655) -- (7.04532,0.724877);
\draw [c] (7.04532,0.724877) -- (7.04532,0.7441);
\draw [c] (7.03718,0.724877) -- (7.04532,0.724877);
\draw [c] (7.04532,0.724877) -- (7.05345,0.724877);
\definecolor{c}{rgb}{0,0,0};
\colorlet{c}{natcomp!70};
\draw [c] (7.06159,0.703801) -- (7.06159,0.72083);
\draw [c] (7.06159,0.72083) -- (7.06159,0.73786);
\draw [c] (7.05345,0.72083) -- (7.06159,0.72083);
\draw [c] (7.06159,0.72083) -- (7.06973,0.72083);
\definecolor{c}{rgb}{0,0,0};
\colorlet{c}{natcomp!70};
\draw [c] (7.07786,0.719124) -- (7.07786,0.741674);
\draw [c] (7.07786,0.741674) -- (7.07786,0.764224);
\draw [c] (7.06973,0.741674) -- (7.07786,0.741674);
\draw [c] (7.07786,0.741674) -- (7.086,0.741674);
\definecolor{c}{rgb}{0,0,0};
\colorlet{c}{natcomp!70};
\draw [c] (7.09414,0.709945) -- (7.09414,0.738016);
\draw [c] (7.09414,0.738016) -- (7.09414,0.766087);
\draw [c] (7.086,0.738016) -- (7.09414,0.738016);
\draw [c] (7.09414,0.738016) -- (7.10227,0.738016);
\definecolor{c}{rgb}{0,0,0};
\colorlet{c}{natcomp!70};
\draw [c] (7.11041,0.691149) -- (7.11041,0.701332);
\draw [c] (7.11041,0.701332) -- (7.11041,0.711516);
\draw [c] (7.10227,0.701332) -- (7.11041,0.701332);
\draw [c] (7.11041,0.701332) -- (7.11855,0.701332);
\definecolor{c}{rgb}{0,0,0};
\colorlet{c}{natcomp!70};
\draw [c] (7.12668,0.706897) -- (7.12668,0.728052);
\draw [c] (7.12668,0.728052) -- (7.12668,0.749206);
\draw [c] (7.11855,0.728052) -- (7.12668,0.728052);
\draw [c] (7.12668,0.728052) -- (7.13482,0.728052);
\definecolor{c}{rgb}{0,0,0};
\colorlet{c}{natcomp!70};
\draw [c] (7.14295,0.691954) -- (7.14295,0.704175);
\draw [c] (7.14295,0.704175) -- (7.14295,0.716395);
\draw [c] (7.13482,0.704175) -- (7.14295,0.704175);
\draw [c] (7.14295,0.704175) -- (7.15109,0.704175);
\definecolor{c}{rgb}{0,0,0};
\colorlet{c}{natcomp!70};
\draw [c] (7.15923,0.686944) -- (7.15923,0.695807);
\draw [c] (7.15923,0.695807) -- (7.15923,0.704669);
\draw [c] (7.15109,0.695807) -- (7.15923,0.695807);
\draw [c] (7.15923,0.695807) -- (7.16736,0.695807);
\definecolor{c}{rgb}{0,0,0};
\colorlet{c}{natcomp!70};
\draw [c] (7.1755,0.713402) -- (7.1755,0.735241);
\draw [c] (7.1755,0.735241) -- (7.1755,0.75708);
\draw [c] (7.16736,0.735241) -- (7.1755,0.735241);
\draw [c] (7.1755,0.735241) -- (7.18364,0.735241);
\definecolor{c}{rgb}{0,0,0};
\colorlet{c}{natcomp!70};
\draw [c] (7.19177,0.70731) -- (7.19177,0.72784);
\draw [c] (7.19177,0.72784) -- (7.19177,0.74837);
\draw [c] (7.18364,0.72784) -- (7.19177,0.72784);
\draw [c] (7.19177,0.72784) -- (7.19991,0.72784);
\definecolor{c}{rgb}{0,0,0};
\colorlet{c}{natcomp!70};
\draw [c] (7.20805,0.694873) -- (7.20805,0.730967);
\draw [c] (7.20805,0.730967) -- (7.20805,0.767061);
\draw [c] (7.19991,0.730967) -- (7.20805,0.730967);
\draw [c] (7.20805,0.730967) -- (7.21618,0.730967);
\definecolor{c}{rgb}{0,0,0};
\colorlet{c}{natcomp!70};
\draw [c] (7.22432,0.692201) -- (7.22432,0.704986);
\draw [c] (7.22432,0.704986) -- (7.22432,0.71777);
\draw [c] (7.21618,0.704986) -- (7.22432,0.704986);
\draw [c] (7.22432,0.704986) -- (7.23245,0.704986);
\definecolor{c}{rgb}{0,0,0};
\colorlet{c}{natcomp!70};
\draw [c] (7.24059,0.694145) -- (7.24059,0.712713);
\draw [c] (7.24059,0.712713) -- (7.24059,0.731281);
\draw [c] (7.23245,0.712713) -- (7.24059,0.712713);
\draw [c] (7.24059,0.712713) -- (7.24873,0.712713);
\definecolor{c}{rgb}{0,0,0};
\colorlet{c}{natcomp!70};
\draw [c] (7.25686,0.686911) -- (7.25686,0.696805);
\draw [c] (7.25686,0.696805) -- (7.25686,0.706699);
\draw [c] (7.24873,0.696805) -- (7.25686,0.696805);
\draw [c] (7.25686,0.696805) -- (7.265,0.696805);
\definecolor{c}{rgb}{0,0,0};
\colorlet{c}{natcomp!70};
\draw [c] (7.27314,0.686939) -- (7.27314,0.698171);
\draw [c] (7.27314,0.698171) -- (7.27314,0.709404);
\draw [c] (7.265,0.698171) -- (7.27314,0.698171);
\draw [c] (7.27314,0.698171) -- (7.28127,0.698171);
\definecolor{c}{rgb}{0,0,0};
\colorlet{c}{natcomp!70};
\draw [c] (7.28941,0.705817) -- (7.28941,0.724823);
\draw [c] (7.28941,0.724823) -- (7.28941,0.743829);
\draw [c] (7.28127,0.724823) -- (7.28941,0.724823);
\draw [c] (7.28941,0.724823) -- (7.29755,0.724823);
\definecolor{c}{rgb}{0,0,0};
\colorlet{c}{natcomp!70};
\draw [c] (7.30568,0.704012) -- (7.30568,0.721633);
\draw [c] (7.30568,0.721633) -- (7.30568,0.739254);
\draw [c] (7.29755,0.721633) -- (7.30568,0.721633);
\draw [c] (7.30568,0.721633) -- (7.31382,0.721633);
\definecolor{c}{rgb}{0,0,0};
\colorlet{c}{natcomp!70};
\draw [c] (7.32195,0.714809) -- (7.32195,0.753099);
\draw [c] (7.32195,0.753099) -- (7.32195,0.791389);
\draw [c] (7.31382,0.753099) -- (7.32195,0.753099);
\draw [c] (7.32195,0.753099) -- (7.33009,0.753099);
\definecolor{c}{rgb}{0,0,0};
\colorlet{c}{natcomp!70};
\draw [c] (7.33823,0.697923) -- (7.33823,0.713018);
\draw [c] (7.33823,0.713018) -- (7.33823,0.728113);
\draw [c] (7.33009,0.713018) -- (7.33823,0.713018);
\draw [c] (7.33823,0.713018) -- (7.34636,0.713018);
\definecolor{c}{rgb}{0,0,0};
\colorlet{c}{natcomp!70};
\draw [c] (7.3545,0.698691) -- (7.3545,0.715314);
\draw [c] (7.3545,0.715314) -- (7.3545,0.731936);
\draw [c] (7.34636,0.715314) -- (7.3545,0.715314);
\draw [c] (7.3545,0.715314) -- (7.36264,0.715314);
\definecolor{c}{rgb}{0,0,0};
\colorlet{c}{natcomp!70};
\draw [c] (7.37077,0.717852) -- (7.37077,0.745539);
\draw [c] (7.37077,0.745539) -- (7.37077,0.773225);
\draw [c] (7.36264,0.745539) -- (7.37077,0.745539);
\draw [c] (7.37077,0.745539) -- (7.37891,0.745539);
\definecolor{c}{rgb}{0,0,0};
\colorlet{c}{natcomp!70};
\draw [c] (7.38705,0.701315) -- (7.38705,0.726433);
\draw [c] (7.38705,0.726433) -- (7.38705,0.751551);
\draw [c] (7.37891,0.726433) -- (7.38705,0.726433);
\draw [c] (7.38705,0.726433) -- (7.39518,0.726433);
\definecolor{c}{rgb}{0,0,0};
\colorlet{c}{natcomp!70};
\draw [c] (7.40332,0.719987) -- (7.40332,0.743126);
\draw [c] (7.40332,0.743126) -- (7.40332,0.766266);
\draw [c] (7.39518,0.743126) -- (7.40332,0.743126);
\draw [c] (7.40332,0.743126) -- (7.41145,0.743126);
\definecolor{c}{rgb}{0,0,0};
\colorlet{c}{natcomp!70};
\draw [c] (7.41959,0.691933) -- (7.41959,0.704153);
\draw [c] (7.41959,0.704153) -- (7.41959,0.716373);
\draw [c] (7.41145,0.704153) -- (7.41959,0.704153);
\draw [c] (7.41959,0.704153) -- (7.42773,0.704153);
\definecolor{c}{rgb}{0,0,0};
\colorlet{c}{natcomp!70};
\draw [c] (7.45214,0.686902) -- (7.45214,0.695866);
\draw [c] (7.45214,0.695866) -- (7.45214,0.704829);
\draw [c] (7.444,0.695866) -- (7.45214,0.695866);
\draw [c] (7.45214,0.695866) -- (7.46027,0.695866);
\definecolor{c}{rgb}{0,0,0};
\colorlet{c}{natcomp!70};
\draw [c] (7.46841,0.686921) -- (7.46841,0.695885);
\draw [c] (7.46841,0.695885) -- (7.46841,0.704848);
\draw [c] (7.46027,0.695885) -- (7.46841,0.695885);
\draw [c] (7.46841,0.695885) -- (7.47655,0.695885);
\definecolor{c}{rgb}{0,0,0};
\colorlet{c}{natcomp!70};
\draw [c] (7.48468,0.686897) -- (7.48468,0.686905);
\draw [c] (7.48468,0.686905) -- (7.48468,0.686914);
\draw [c] (7.47655,0.686905) -- (7.48468,0.686905);
\draw [c] (7.48468,0.686905) -- (7.49282,0.686905);
\definecolor{c}{rgb}{0,0,0};
\colorlet{c}{natcomp!70};
\draw [c] (7.50095,0.686894) -- (7.50095,0.686899);
\draw [c] (7.50095,0.686899) -- (7.50095,0.686904);
\draw [c] (7.49282,0.686899) -- (7.50095,0.686899);
\draw [c] (7.50095,0.686899) -- (7.50909,0.686899);
\definecolor{c}{rgb}{0,0,0};
\colorlet{c}{natcomp!70};
\draw [c] (7.51723,0.702942) -- (7.51723,0.719147);
\draw [c] (7.51723,0.719147) -- (7.51723,0.735351);
\draw [c] (7.50909,0.719147) -- (7.51723,0.719147);
\draw [c] (7.51723,0.719147) -- (7.52536,0.719147);
\definecolor{c}{rgb}{0,0,0};
\colorlet{c}{natcomp!70};
\draw [c] (7.5335,0.691582) -- (7.5335,0.703001);
\draw [c] (7.5335,0.703001) -- (7.5335,0.71442);
\draw [c] (7.52536,0.703001) -- (7.5335,0.703001);
\draw [c] (7.5335,0.703001) -- (7.54164,0.703001);
\definecolor{c}{rgb}{0,0,0};
\colorlet{c}{natcomp!70};
\draw [c] (7.54977,0.699339) -- (7.54977,0.718609);
\draw [c] (7.54977,0.718609) -- (7.54977,0.737879);
\draw [c] (7.54164,0.718609) -- (7.54977,0.718609);
\draw [c] (7.54977,0.718609) -- (7.55791,0.718609);
\definecolor{c}{rgb}{0,0,0};
\colorlet{c}{natcomp!70};
\draw [c] (7.56605,0.705399) -- (7.56605,0.728064);
\draw [c] (7.56605,0.728064) -- (7.56605,0.750729);
\draw [c] (7.55791,0.728064) -- (7.56605,0.728064);
\draw [c] (7.56605,0.728064) -- (7.57418,0.728064);
\definecolor{c}{rgb}{0,0,0};
\colorlet{c}{natcomp!70};
\draw [c] (7.58232,0.699404) -- (7.58232,0.723597);
\draw [c] (7.58232,0.723597) -- (7.58232,0.74779);
\draw [c] (7.57418,0.723597) -- (7.58232,0.723597);
\draw [c] (7.58232,0.723597) -- (7.59045,0.723597);
\definecolor{c}{rgb}{0,0,0};
\colorlet{c}{natcomp!70};
\draw [c] (7.59859,0.686912) -- (7.59859,0.694113);
\draw [c] (7.59859,0.694113) -- (7.59859,0.701313);
\draw [c] (7.59045,0.694113) -- (7.59859,0.694113);
\draw [c] (7.59859,0.694113) -- (7.60673,0.694113);
\definecolor{c}{rgb}{0,0,0};
\colorlet{c}{natcomp!70};
\draw [c] (7.61486,0.686896) -- (7.61486,0.686902);
\draw [c] (7.61486,0.686902) -- (7.61486,0.686909);
\draw [c] (7.60673,0.686902) -- (7.61486,0.686902);
\draw [c] (7.61486,0.686902) -- (7.623,0.686902);
\definecolor{c}{rgb}{0,0,0};
\colorlet{c}{natcomp!70};
\draw [c] (7.63114,0.721485) -- (7.63114,0.74621);
\draw [c] (7.63114,0.74621) -- (7.63114,0.770934);
\draw [c] (7.623,0.74621) -- (7.63114,0.74621);
\draw [c] (7.63114,0.74621) -- (7.63927,0.74621);
\definecolor{c}{rgb}{0,0,0};
\colorlet{c}{natcomp!70};
\draw [c] (7.64741,0.686934) -- (7.64741,0.695073);
\draw [c] (7.64741,0.695073) -- (7.64741,0.703212);
\draw [c] (7.63927,0.695073) -- (7.64741,0.695073);
\draw [c] (7.64741,0.695073) -- (7.65555,0.695073);
\definecolor{c}{rgb}{0,0,0};
\colorlet{c}{natcomp!70};
\draw [c] (7.66368,0.69193) -- (7.66368,0.704888);
\draw [c] (7.66368,0.704888) -- (7.66368,0.717847);
\draw [c] (7.65555,0.704888) -- (7.66368,0.704888);
\draw [c] (7.66368,0.704888) -- (7.67182,0.704888);
\definecolor{c}{rgb}{0,0,0};
\colorlet{c}{natcomp!70};
\draw [c] (7.67995,0.693227) -- (7.67995,0.708478);
\draw [c] (7.67995,0.708478) -- (7.67995,0.723729);
\draw [c] (7.67182,0.708478) -- (7.67995,0.708478);
\draw [c] (7.67995,0.708478) -- (7.68809,0.708478);
\definecolor{c}{rgb}{0,0,0};
\colorlet{c}{natcomp!70};
\draw [c] (7.69623,0.705019) -- (7.69623,0.723124);
\draw [c] (7.69623,0.723124) -- (7.69623,0.74123);
\draw [c] (7.68809,0.723124) -- (7.69623,0.723124);
\draw [c] (7.69623,0.723124) -- (7.70436,0.723124);
\definecolor{c}{rgb}{0,0,0};
\colorlet{c}{natcomp!70};
\draw [c] (7.7125,0.686901) -- (7.7125,0.686914);
\draw [c] (7.7125,0.686914) -- (7.7125,0.686926);
\draw [c] (7.70436,0.686914) -- (7.7125,0.686914);
\draw [c] (7.7125,0.686914) -- (7.72064,0.686914);
\definecolor{c}{rgb}{0,0,0};
\colorlet{c}{natcomp!70};
\draw [c] (7.72877,0.686911) -- (7.72877,0.697705);
\draw [c] (7.72877,0.697705) -- (7.72877,0.7085);
\draw [c] (7.72064,0.697705) -- (7.72877,0.697705);
\draw [c] (7.72877,0.697705) -- (7.73691,0.697705);
\definecolor{c}{rgb}{0,0,0};
\colorlet{c}{natcomp!70};
\draw [c] (7.74505,0.697486) -- (7.74505,0.712097);
\draw [c] (7.74505,0.712097) -- (7.74505,0.726708);
\draw [c] (7.73691,0.712097) -- (7.74505,0.712097);
\draw [c] (7.74505,0.712097) -- (7.75318,0.712097);
\definecolor{c}{rgb}{0,0,0};
\colorlet{c}{natcomp!70};
\draw [c] (7.76132,0.692362) -- (7.76132,0.708107);
\draw [c] (7.76132,0.708107) -- (7.76132,0.723852);
\draw [c] (7.75318,0.708107) -- (7.76132,0.708107);
\draw [c] (7.76132,0.708107) -- (7.76945,0.708107);
\definecolor{c}{rgb}{0,0,0};
\colorlet{c}{natcomp!70};
\draw [c] (7.77759,0.699709) -- (7.77759,0.717404);
\draw [c] (7.77759,0.717404) -- (7.77759,0.735098);
\draw [c] (7.76945,0.717404) -- (7.77759,0.717404);
\draw [c] (7.77759,0.717404) -- (7.78573,0.717404);
\definecolor{c}{rgb}{0,0,0};
\colorlet{c}{natcomp!70};
\draw [c] (7.79386,0.733221) -- (7.79386,0.759074);
\draw [c] (7.79386,0.759074) -- (7.79386,0.784928);
\draw [c] (7.78573,0.759074) -- (7.79386,0.759074);
\draw [c] (7.79386,0.759074) -- (7.802,0.759074);
\definecolor{c}{rgb}{0,0,0};
\colorlet{c}{natcomp!70};
\draw [c] (7.81014,0.697974) -- (7.81014,0.713528);
\draw [c] (7.81014,0.713528) -- (7.81014,0.729082);
\draw [c] (7.802,0.713528) -- (7.81014,0.713528);
\draw [c] (7.81014,0.713528) -- (7.81827,0.713528);
\definecolor{c}{rgb}{0,0,0};
\colorlet{c}{natcomp!70};
\draw [c] (7.82641,0.692082) -- (7.82641,0.705044);
\draw [c] (7.82641,0.705044) -- (7.82641,0.718006);
\draw [c] (7.81827,0.705044) -- (7.82641,0.705044);
\draw [c] (7.82641,0.705044) -- (7.83455,0.705044);
\definecolor{c}{rgb}{0,0,0};
\colorlet{c}{natcomp!70};
\draw [c] (7.84268,0.69139) -- (7.84268,0.702257);
\draw [c] (7.84268,0.702257) -- (7.84268,0.713124);
\draw [c] (7.83455,0.702257) -- (7.84268,0.702257);
\draw [c] (7.84268,0.702257) -- (7.85082,0.702257);
\definecolor{c}{rgb}{0,0,0};
\colorlet{c}{natcomp!70};
\draw [c] (7.85895,0.68692) -- (7.85895,0.696815);
\draw [c] (7.85895,0.696815) -- (7.85895,0.706709);
\draw [c] (7.85082,0.696815) -- (7.85895,0.696815);
\draw [c] (7.85895,0.696815) -- (7.86709,0.696815);
\definecolor{c}{rgb}{0,0,0};
\colorlet{c}{natcomp!70};
\draw [c] (7.87523,0.699254) -- (7.87523,0.718923);
\draw [c] (7.87523,0.718923) -- (7.87523,0.738592);
\draw [c] (7.86709,0.718923) -- (7.87523,0.718923);
\draw [c] (7.87523,0.718923) -- (7.88336,0.718923);
\definecolor{c}{rgb}{0,0,0};
\colorlet{c}{natcomp!70};
\draw [c] (7.8915,0.711611) -- (7.8915,0.732071);
\draw [c] (7.8915,0.732071) -- (7.8915,0.752532);
\draw [c] (7.88336,0.732071) -- (7.8915,0.732071);
\draw [c] (7.8915,0.732071) -- (7.89964,0.732071);
\definecolor{c}{rgb}{0,0,0};
\colorlet{c}{natcomp!70};
\draw [c] (7.90777,0.715085) -- (7.90777,0.738825);
\draw [c] (7.90777,0.738825) -- (7.90777,0.762565);
\draw [c] (7.89964,0.738825) -- (7.90777,0.738825);
\draw [c] (7.90777,0.738825) -- (7.91591,0.738825);
\definecolor{c}{rgb}{0,0,0};
\colorlet{c}{natcomp!70};
\draw [c] (7.92405,0.686922) -- (7.92405,0.696038);
\draw [c] (7.92405,0.696038) -- (7.92405,0.705153);
\draw [c] (7.91591,0.696038) -- (7.92405,0.696038);
\draw [c] (7.92405,0.696038) -- (7.93218,0.696038);
\definecolor{c}{rgb}{0,0,0};
\colorlet{c}{natcomp!70};
\draw [c] (7.94032,0.686907) -- (7.94032,0.696802);
\draw [c] (7.94032,0.696802) -- (7.94032,0.706696);
\draw [c] (7.93218,0.696802) -- (7.94032,0.696802);
\draw [c] (7.94032,0.696802) -- (7.94845,0.696802);
\definecolor{c}{rgb}{0,0,0};
\colorlet{c}{natcomp!70};
\draw [c] (7.95659,0.691941) -- (7.95659,0.704161);
\draw [c] (7.95659,0.704161) -- (7.95659,0.716381);
\draw [c] (7.94845,0.704161) -- (7.95659,0.704161);
\draw [c] (7.95659,0.704161) -- (7.96473,0.704161);
\definecolor{c}{rgb}{0,0,0};
\colorlet{c}{natcomp!70};
\draw [c] (7.97286,0.686912) -- (7.97286,0.69505);
\draw [c] (7.97286,0.69505) -- (7.97286,0.703189);
\draw [c] (7.96473,0.69505) -- (7.97286,0.69505);
\draw [c] (7.97286,0.69505) -- (7.981,0.69505);
\definecolor{c}{rgb}{0,0,0};
\colorlet{c}{natcomp!70};
\draw [c] (7.98914,0.693662) -- (7.98914,0.710807);
\draw [c] (7.98914,0.710807) -- (7.98914,0.727952);
\draw [c] (7.981,0.710807) -- (7.98914,0.710807);
\draw [c] (7.98914,0.710807) -- (7.99727,0.710807);
\definecolor{c}{rgb}{0,0,0};
\colorlet{c}{natcomp!70};
\draw [c] (8.00541,0.686922) -- (8.00541,0.695785);
\draw [c] (8.00541,0.695785) -- (8.00541,0.704647);
\draw [c] (7.99727,0.695785) -- (8.00541,0.695785);
\draw [c] (8.00541,0.695785) -- (8.01355,0.695785);
\definecolor{c}{rgb}{0,0,0};
\colorlet{c}{natcomp!70};
\draw [c] (8.02168,0.691552) -- (8.02168,0.702771);
\draw [c] (8.02168,0.702771) -- (8.02168,0.71399);
\draw [c] (8.01355,0.702771) -- (8.02168,0.702771);
\draw [c] (8.02168,0.702771) -- (8.02982,0.702771);
\definecolor{c}{rgb}{0,0,0};
\colorlet{c}{natcomp!70};
\draw [c] (8.03795,0.686908) -- (8.03795,0.70091);
\draw [c] (8.03795,0.70091) -- (8.03795,0.714913);
\draw [c] (8.02982,0.70091) -- (8.03795,0.70091);
\draw [c] (8.03795,0.70091) -- (8.04609,0.70091);
\definecolor{c}{rgb}{0,0,0};
\colorlet{c}{natcomp!70};
\draw [c] (8.05423,0.686902) -- (8.05423,0.695765);
\draw [c] (8.05423,0.695765) -- (8.05423,0.704627);
\draw [c] (8.04609,0.695765) -- (8.05423,0.695765);
\draw [c] (8.05423,0.695765) -- (8.06236,0.695765);
\definecolor{c}{rgb}{0,0,0};
\colorlet{c}{natcomp!70};
\draw [c] (8.0705,0.686913) -- (8.0705,0.694635);
\draw [c] (8.0705,0.694635) -- (8.0705,0.702356);
\draw [c] (8.06236,0.694635) -- (8.0705,0.694635);
\draw [c] (8.0705,0.694635) -- (8.07864,0.694635);
\definecolor{c}{rgb}{0,0,0};
\colorlet{c}{natcomp!70};
\draw [c] (8.08677,0.691886) -- (8.08677,0.703918);
\draw [c] (8.08677,0.703918) -- (8.08677,0.715951);
\draw [c] (8.07864,0.703918) -- (8.08677,0.703918);
\draw [c] (8.08677,0.703918) -- (8.09491,0.703918);
\definecolor{c}{rgb}{0,0,0};
\colorlet{c}{natcomp!70};
\draw [c] (8.10305,0.699036) -- (8.10305,0.715859);
\draw [c] (8.10305,0.715859) -- (8.10305,0.732682);
\draw [c] (8.09491,0.715859) -- (8.10305,0.715859);
\draw [c] (8.10305,0.715859) -- (8.11118,0.715859);
\definecolor{c}{rgb}{0,0,0};
\colorlet{c}{natcomp!70};
\draw [c] (8.11932,0.70093) -- (8.11932,0.723385);
\draw [c] (8.11932,0.723385) -- (8.11932,0.74584);
\draw [c] (8.11118,0.723385) -- (8.11932,0.723385);
\draw [c] (8.11932,0.723385) -- (8.12745,0.723385);
\definecolor{c}{rgb}{0,0,0};
\colorlet{c}{natcomp!70};
\draw [c] (8.13559,0.691991) -- (8.13559,0.704542);
\draw [c] (8.13559,0.704542) -- (8.13559,0.717092);
\draw [c] (8.12745,0.704542) -- (8.13559,0.704542);
\draw [c] (8.13559,0.704542) -- (8.14373,0.704542);
\definecolor{c}{rgb}{0,0,0};
\colorlet{c}{natcomp!70};
\draw [c] (8.15186,0.704381) -- (8.15186,0.722348);
\draw [c] (8.15186,0.722348) -- (8.15186,0.740315);
\draw [c] (8.14373,0.722348) -- (8.15186,0.722348);
\draw [c] (8.15186,0.722348) -- (8.16,0.722348);
\definecolor{c}{rgb}{0,0,0};
\colorlet{c}{natcomp!70};
\draw [c] (8.16814,0.692514) -- (8.16814,0.706186);
\draw [c] (8.16814,0.706186) -- (8.16814,0.719858);
\draw [c] (8.16,0.706186) -- (8.16814,0.706186);
\draw [c] (8.16814,0.706186) -- (8.17627,0.706186);
\definecolor{c}{rgb}{0,0,0};
\colorlet{c}{natcomp!70};
\draw [c] (8.18441,0.686908) -- (8.18441,0.697319);
\draw [c] (8.18441,0.697319) -- (8.18441,0.70773);
\draw [c] (8.17627,0.697319) -- (8.18441,0.697319);
\draw [c] (8.18441,0.697319) -- (8.19255,0.697319);
\definecolor{c}{rgb}{0,0,0};
\colorlet{c}{natcomp!70};
\draw [c] (8.20068,0.686908) -- (8.20068,0.708675);
\draw [c] (8.20068,0.708675) -- (8.20068,0.730443);
\draw [c] (8.19255,0.708675) -- (8.20068,0.708675);
\draw [c] (8.20068,0.708675) -- (8.20882,0.708675);
\definecolor{c}{rgb}{0,0,0};
\colorlet{c}{natcomp!70};
\draw [c] (8.21695,0.691366) -- (8.21695,0.702233);
\draw [c] (8.21695,0.702233) -- (8.21695,0.7131);
\draw [c] (8.20882,0.702233) -- (8.21695,0.702233);
\draw [c] (8.21695,0.702233) -- (8.22509,0.702233);
\definecolor{c}{rgb}{0,0,0};
\colorlet{c}{natcomp!70};
\draw [c] (8.23323,0.700598) -- (8.23323,0.719335);
\draw [c] (8.23323,0.719335) -- (8.23323,0.738073);
\draw [c] (8.22509,0.719335) -- (8.23323,0.719335);
\draw [c] (8.23323,0.719335) -- (8.24136,0.719335);
\definecolor{c}{rgb}{0,0,0};
\colorlet{c}{natcomp!70};
\draw [c] (8.2495,0.698239) -- (8.2495,0.713998);
\draw [c] (8.2495,0.713998) -- (8.2495,0.729758);
\draw [c] (8.24136,0.713998) -- (8.2495,0.713998);
\draw [c] (8.2495,0.713998) -- (8.25764,0.713998);
\definecolor{c}{rgb}{0,0,0};
\colorlet{c}{natcomp!70};
\draw [c] (8.26577,0.704207) -- (8.26577,0.72189);
\draw [c] (8.26577,0.72189) -- (8.26577,0.739573);
\draw [c] (8.25764,0.72189) -- (8.26577,0.72189);
\draw [c] (8.26577,0.72189) -- (8.27391,0.72189);
\definecolor{c}{rgb}{0,0,0};
\colorlet{c}{natcomp!70};
\draw [c] (8.28205,0.699406) -- (8.28205,0.719236);
\draw [c] (8.28205,0.719236) -- (8.28205,0.739065);
\draw [c] (8.27391,0.719236) -- (8.28205,0.719236);
\draw [c] (8.28205,0.719236) -- (8.29018,0.719236);
\definecolor{c}{rgb}{0,0,0};
\colorlet{c}{natcomp!70};
\draw [c] (8.29832,0.686897) -- (8.29832,0.686904);
\draw [c] (8.29832,0.686904) -- (8.29832,0.686911);
\draw [c] (8.29018,0.686904) -- (8.29832,0.686904);
\draw [c] (8.29832,0.686904) -- (8.30645,0.686904);
\definecolor{c}{rgb}{0,0,0};
\colorlet{c}{natcomp!70};
\draw [c] (8.31459,0.692156) -- (8.31459,0.704967);
\draw [c] (8.31459,0.704967) -- (8.31459,0.717779);
\draw [c] (8.30645,0.704967) -- (8.31459,0.704967);
\draw [c] (8.31459,0.704967) -- (8.32273,0.704967);
\definecolor{c}{rgb}{0,0,0};
\colorlet{c}{natcomp!70};
\draw [c] (8.33086,0.686894) -- (8.33086,0.686898);
\draw [c] (8.33086,0.686898) -- (8.33086,0.686902);
\draw [c] (8.32273,0.686898) -- (8.33086,0.686898);
\draw [c] (8.33086,0.686898) -- (8.339,0.686898);
\definecolor{c}{rgb}{0,0,0};
\colorlet{c}{natcomp!70};
\draw [c] (8.34714,0.686913) -- (8.34714,0.696808);
\draw [c] (8.34714,0.696808) -- (8.34714,0.706702);
\draw [c] (8.339,0.696808) -- (8.34714,0.696808);
\draw [c] (8.34714,0.696808) -- (8.35527,0.696808);
\definecolor{c}{rgb}{0,0,0};
\colorlet{c}{natcomp!70};
\draw [c] (8.36341,0.6869) -- (8.36341,0.686908);
\draw [c] (8.36341,0.686908) -- (8.36341,0.686916);
\draw [c] (8.35527,0.686908) -- (8.36341,0.686908);
\draw [c] (8.36341,0.686908) -- (8.37155,0.686908);
\definecolor{c}{rgb}{0,0,0};
\colorlet{c}{natcomp!70};
\draw [c] (8.37968,0.709537) -- (8.37968,0.733883);
\draw [c] (8.37968,0.733883) -- (8.37968,0.758228);
\draw [c] (8.37155,0.733883) -- (8.37968,0.733883);
\draw [c] (8.37968,0.733883) -- (8.38782,0.733883);
\definecolor{c}{rgb}{0,0,0};
\colorlet{c}{natcomp!70};
\draw [c] (8.39595,0.686897) -- (8.39595,0.696792);
\draw [c] (8.39595,0.696792) -- (8.39595,0.706686);
\draw [c] (8.38782,0.696792) -- (8.39595,0.696792);
\draw [c] (8.39595,0.696792) -- (8.40409,0.696792);
\definecolor{c}{rgb}{0,0,0};
\colorlet{c}{natcomp!70};
\draw [c] (8.41223,0.686907) -- (8.41223,0.696023);
\draw [c] (8.41223,0.696023) -- (8.41223,0.705138);
\draw [c] (8.40409,0.696023) -- (8.41223,0.696023);
\draw [c] (8.41223,0.696023) -- (8.42036,0.696023);
\definecolor{c}{rgb}{0,0,0};
\colorlet{c}{natcomp!70};
\draw [c] (8.4285,0.691882) -- (8.4285,0.703914);
\draw [c] (8.4285,0.703914) -- (8.4285,0.715947);
\draw [c] (8.42036,0.703914) -- (8.4285,0.703914);
\draw [c] (8.4285,0.703914) -- (8.43664,0.703914);
\definecolor{c}{rgb}{0,0,0};
\colorlet{c}{natcomp!70};
\draw [c] (8.44477,0.686908) -- (8.44477,0.69814);
\draw [c] (8.44477,0.69814) -- (8.44477,0.709372);
\draw [c] (8.43664,0.69814) -- (8.44477,0.69814);
\draw [c] (8.44477,0.69814) -- (8.45291,0.69814);
\definecolor{c}{rgb}{0,0,0};
\colorlet{c}{natcomp!70};
\draw [c] (8.46105,0.686896) -- (8.46105,0.686903);
\draw [c] (8.46105,0.686903) -- (8.46105,0.686909);
\draw [c] (8.45291,0.686903) -- (8.46105,0.686903);
\draw [c] (8.46105,0.686903) -- (8.46918,0.686903);
\definecolor{c}{rgb}{0,0,0};
\colorlet{c}{natcomp!70};
\draw [c] (8.47732,0.686902) -- (8.47732,0.686911);
\draw [c] (8.47732,0.686911) -- (8.47732,0.686919);
\draw [c] (8.46918,0.686911) -- (8.47732,0.686911);
\draw [c] (8.47732,0.686911) -- (8.48545,0.686911);
\definecolor{c}{rgb}{0,0,0};
\colorlet{c}{natcomp!70};
\draw [c] (8.49359,0.686915) -- (8.49359,0.712075);
\draw [c] (8.49359,0.712075) -- (8.49359,0.737235);
\draw [c] (8.48545,0.712075) -- (8.49359,0.712075);
\draw [c] (8.49359,0.712075) -- (8.50173,0.712075);
\definecolor{c}{rgb}{0,0,0};
\colorlet{c}{natcomp!70};
\draw [c] (8.50986,0.686899) -- (8.50986,0.686907);
\draw [c] (8.50986,0.686907) -- (8.50986,0.686915);
\draw [c] (8.50173,0.686907) -- (8.50986,0.686907);
\draw [c] (8.50986,0.686907) -- (8.518,0.686907);
\definecolor{c}{rgb}{0,0,0};
\colorlet{c}{natcomp!70};
\draw [c] (8.52614,0.686912) -- (8.52614,0.694113);
\draw [c] (8.52614,0.694113) -- (8.52614,0.701314);
\draw [c] (8.518,0.694113) -- (8.52614,0.694113);
\draw [c] (8.52614,0.694113) -- (8.53427,0.694113);
\definecolor{c}{rgb}{0,0,0};
\colorlet{c}{natcomp!70};
\draw [c] (8.54241,0.70362) -- (8.54241,0.720773);
\draw [c] (8.54241,0.720773) -- (8.54241,0.737927);
\draw [c] (8.53427,0.720773) -- (8.54241,0.720773);
\draw [c] (8.54241,0.720773) -- (8.55055,0.720773);
\definecolor{c}{rgb}{0,0,0};
\colorlet{c}{natcomp!70};
\draw [c] (8.55868,0.686902) -- (8.55868,0.697697);
\draw [c] (8.55868,0.697697) -- (8.55868,0.708492);
\draw [c] (8.55055,0.697697) -- (8.55868,0.697697);
\draw [c] (8.55868,0.697697) -- (8.56682,0.697697);
\definecolor{c}{rgb}{0,0,0};
\colorlet{c}{natcomp!70};
\draw [c] (8.57495,0.692719) -- (8.57495,0.70709);
\draw [c] (8.57495,0.70709) -- (8.57495,0.721461);
\draw [c] (8.56682,0.70709) -- (8.57495,0.70709);
\draw [c] (8.57495,0.70709) -- (8.58309,0.70709);
\definecolor{c}{rgb}{0,0,0};
\colorlet{c}{natcomp!70};
\draw [c] (8.59123,0.686902) -- (8.59123,0.686909);
\draw [c] (8.59123,0.686909) -- (8.59123,0.686917);
\draw [c] (8.58309,0.686909) -- (8.59123,0.686909);
\draw [c] (8.59123,0.686909) -- (8.59936,0.686909);
\definecolor{c}{rgb}{0,0,0};
\colorlet{c}{natcomp!70};
\draw [c] (8.6075,0.686894) -- (8.6075,0.686897);
\draw [c] (8.6075,0.686897) -- (8.6075,0.686901);
\draw [c] (8.59936,0.686897) -- (8.6075,0.686897);
\draw [c] (8.6075,0.686897) -- (8.61564,0.686897);
\definecolor{c}{rgb}{0,0,0};
\colorlet{c}{natcomp!70};
\draw [c] (8.62377,0.686903) -- (8.62377,0.686912);
\draw [c] (8.62377,0.686912) -- (8.62377,0.686922);
\draw [c] (8.61564,0.686912) -- (8.62377,0.686912);
\draw [c] (8.62377,0.686912) -- (8.63191,0.686912);
\definecolor{c}{rgb}{0,0,0};
\colorlet{c}{natcomp!70};
\draw [c] (8.64005,0.706296) -- (8.64005,0.725795);
\draw [c] (8.64005,0.725795) -- (8.64005,0.745294);
\draw [c] (8.63191,0.725795) -- (8.64005,0.725795);
\draw [c] (8.64005,0.725795) -- (8.64818,0.725795);
\definecolor{c}{rgb}{0,0,0};
\colorlet{c}{natcomp!70};
\draw [c] (8.65632,0.686902) -- (8.65632,0.697676);
\draw [c] (8.65632,0.697676) -- (8.65632,0.708449);
\draw [c] (8.64818,0.697676) -- (8.65632,0.697676);
\draw [c] (8.65632,0.697676) -- (8.66445,0.697676);
\definecolor{c}{rgb}{0,0,0};
\colorlet{c}{natcomp!70};
\draw [c] (8.67259,0.697669) -- (8.67259,0.712756);
\draw [c] (8.67259,0.712756) -- (8.67259,0.727844);
\draw [c] (8.66445,0.712756) -- (8.67259,0.712756);
\draw [c] (8.67259,0.712756) -- (8.68073,0.712756);
\definecolor{c}{rgb}{0,0,0};
\colorlet{c}{natcomp!70};
\draw [c] (8.68886,0.686917) -- (8.68886,0.695056);
\draw [c] (8.68886,0.695056) -- (8.68886,0.703195);
\draw [c] (8.68073,0.695056) -- (8.68886,0.695056);
\draw [c] (8.68886,0.695056) -- (8.697,0.695056);
\definecolor{c}{rgb}{0,0,0};
\colorlet{c}{natcomp!70};
\draw [c] (8.70514,0.686894) -- (8.70514,0.686898);
\draw [c] (8.70514,0.686898) -- (8.70514,0.686902);
\draw [c] (8.697,0.686898) -- (8.70514,0.686898);
\draw [c] (8.70514,0.686898) -- (8.71327,0.686898);
\definecolor{c}{rgb}{0,0,0};
\colorlet{c}{natcomp!70};
\draw [c] (8.72141,0.686896) -- (8.72141,0.686903);
\draw [c] (8.72141,0.686903) -- (8.72141,0.686909);
\draw [c] (8.71327,0.686903) -- (8.72141,0.686903);
\draw [c] (8.72141,0.686903) -- (8.72955,0.686903);
\definecolor{c}{rgb}{0,0,0};
\colorlet{c}{natcomp!70};
\draw [c] (8.73768,0.6869) -- (8.73768,0.686909);
\draw [c] (8.73768,0.686909) -- (8.73768,0.686917);
\draw [c] (8.72955,0.686909) -- (8.73768,0.686909);
\draw [c] (8.73768,0.686909) -- (8.74582,0.686909);
\definecolor{c}{rgb}{0,0,0};
\colorlet{c}{natcomp!70};
\draw [c] (8.75395,0.686898) -- (8.75395,0.697671);
\draw [c] (8.75395,0.697671) -- (8.75395,0.708445);
\draw [c] (8.74582,0.697671) -- (8.75395,0.697671);
\draw [c] (8.75395,0.697671) -- (8.76209,0.697671);
\definecolor{c}{rgb}{0,0,0};
\colorlet{c}{natcomp!70};
\draw [c] (8.77023,0.686896) -- (8.77023,0.686902);
\draw [c] (8.77023,0.686902) -- (8.77023,0.686908);
\draw [c] (8.76209,0.686902) -- (8.77023,0.686902);
\draw [c] (8.77023,0.686902) -- (8.77836,0.686902);
\definecolor{c}{rgb}{0,0,0};
\colorlet{c}{natcomp!70};
\draw [c] (8.80277,0.686894) -- (8.80277,0.697305);
\draw [c] (8.80277,0.697305) -- (8.80277,0.707716);
\draw [c] (8.79464,0.697305) -- (8.80277,0.697305);
\draw [c] (8.80277,0.697305) -- (8.81091,0.697305);
\definecolor{c}{rgb}{0,0,0};
\colorlet{c}{natcomp!70};
\draw [c] (8.81905,0.686894) -- (8.81905,0.686899);
\draw [c] (8.81905,0.686899) -- (8.81905,0.686904);
\draw [c] (8.81091,0.686899) -- (8.81905,0.686899);
\draw [c] (8.81905,0.686899) -- (8.82718,0.686899);
\definecolor{c}{rgb}{0,0,0};
\colorlet{c}{natcomp!70};
\draw [c] (8.83532,0.686894) -- (8.83532,0.686899);
\draw [c] (8.83532,0.686899) -- (8.83532,0.686904);
\draw [c] (8.82718,0.686899) -- (8.83532,0.686899);
\draw [c] (8.83532,0.686899) -- (8.84345,0.686899);
\definecolor{c}{rgb}{0,0,0};
\colorlet{c}{natcomp!70};
\draw [c] (8.85159,0.686907) -- (8.85159,0.696022);
\draw [c] (8.85159,0.696022) -- (8.85159,0.705137);
\draw [c] (8.84345,0.696022) -- (8.85159,0.696022);
\draw [c] (8.85159,0.696022) -- (8.85973,0.696022);
\definecolor{c}{rgb}{0,0,0};
\colorlet{c}{natcomp!70};
\draw [c] (8.86786,0.686894) -- (8.86786,0.686897);
\draw [c] (8.86786,0.686897) -- (8.86786,0.686901);
\draw [c] (8.85973,0.686897) -- (8.86786,0.686897);
\draw [c] (8.86786,0.686897) -- (8.876,0.686897);
\definecolor{c}{rgb}{0,0,0};
\colorlet{c}{natcomp!70};
\draw [c] (8.88414,0.700826) -- (8.88414,0.720637);
\draw [c] (8.88414,0.720637) -- (8.88414,0.740448);
\draw [c] (8.876,0.720637) -- (8.88414,0.720637);
\draw [c] (8.88414,0.720637) -- (8.89227,0.720637);
\definecolor{c}{rgb}{0,0,0};
\colorlet{c}{natcomp!70};
\draw [c] (8.90041,0.686894) -- (8.90041,0.686897);
\draw [c] (8.90041,0.686897) -- (8.90041,0.686901);
\draw [c] (8.89227,0.686897) -- (8.90041,0.686897);
\draw [c] (8.90041,0.686897) -- (8.90855,0.686897);
\definecolor{c}{rgb}{0,0,0};
\colorlet{c}{natcomp!70};
\draw [c] (8.91668,0.722516) -- (8.91668,0.755184);
\draw [c] (8.91668,0.755184) -- (8.91668,0.787852);
\draw [c] (8.90855,0.755184) -- (8.91668,0.755184);
\draw [c] (8.91668,0.755184) -- (8.92482,0.755184);
\definecolor{c}{rgb}{0,0,0};
\colorlet{c}{natcomp!70};
\draw [c] (8.93295,0.6869) -- (8.93295,0.694622);
\draw [c] (8.93295,0.694622) -- (8.93295,0.702344);
\draw [c] (8.92482,0.694622) -- (8.93295,0.694622);
\draw [c] (8.93295,0.694622) -- (8.94109,0.694622);
\definecolor{c}{rgb}{0,0,0};
\colorlet{c}{natcomp!70};
\draw [c] (8.94923,0.686894) -- (8.94923,0.686898);
\draw [c] (8.94923,0.686898) -- (8.94923,0.686902);
\draw [c] (8.94109,0.686898) -- (8.94923,0.686898);
\draw [c] (8.94923,0.686898) -- (8.95736,0.686898);
\definecolor{c}{rgb}{0,0,0};
\colorlet{c}{natcomp!70};
\draw [c] (8.9655,0.686894) -- (8.9655,0.686898);
\draw [c] (8.9655,0.686898) -- (8.9655,0.686902);
\draw [c] (8.95736,0.686898) -- (8.9655,0.686898);
\draw [c] (8.9655,0.686898) -- (8.97364,0.686898);
\definecolor{c}{rgb}{0,0,0};
\colorlet{c}{natcomp!70};
\draw [c] (8.98177,0.686896) -- (8.98177,0.686902);
\draw [c] (8.98177,0.686902) -- (8.98177,0.686907);
\draw [c] (8.97364,0.686902) -- (8.98177,0.686902);
\draw [c] (8.98177,0.686902) -- (8.98991,0.686902);
\definecolor{c}{rgb}{0,0,0};
\colorlet{c}{natcomp!70};
\draw [c] (8.99805,0.686904) -- (8.99805,0.686922);
\draw [c] (8.99805,0.686922) -- (8.99805,0.686939);
\draw [c] (8.98991,0.686922) -- (8.99805,0.686922);
\draw [c] (8.99805,0.686922) -- (9.00618,0.686922);
\definecolor{c}{rgb}{0,0,0};
\colorlet{c}{natcomp!70};
\draw [c] (9.01432,0.686897) -- (9.01432,0.686903);
\draw [c] (9.01432,0.686903) -- (9.01432,0.68691);
\draw [c] (9.00618,0.686903) -- (9.01432,0.686903);
\draw [c] (9.01432,0.686903) -- (9.02245,0.686903);
\definecolor{c}{rgb}{0,0,0};
\colorlet{c}{natcomp!70};
\draw [c] (9.04686,0.708993) -- (9.04686,0.731658);
\draw [c] (9.04686,0.731658) -- (9.04686,0.754323);
\draw [c] (9.03873,0.731658) -- (9.04686,0.731658);
\draw [c] (9.04686,0.731658) -- (9.055,0.731658);
\definecolor{c}{rgb}{0,0,0};
\colorlet{c}{natcomp!70};
\draw [c] (9.07941,0.692085) -- (9.07941,0.704618);
\draw [c] (9.07941,0.704618) -- (9.07941,0.717151);
\draw [c] (9.07127,0.704618) -- (9.07941,0.704618);
\draw [c] (9.07941,0.704618) -- (9.08755,0.704618);
\definecolor{c}{rgb}{0,0,0};
\colorlet{c}{natcomp!70};
\draw [c] (9.09568,0.686897) -- (9.09568,0.695861);
\draw [c] (9.09568,0.695861) -- (9.09568,0.704825);
\draw [c] (9.08755,0.695861) -- (9.09568,0.695861);
\draw [c] (9.09568,0.695861) -- (9.10382,0.695861);
\definecolor{c}{rgb}{0,0,0};
\colorlet{c}{natcomp!70};
\draw [c] (9.12823,0.686894) -- (9.12823,0.686897);
\draw [c] (9.12823,0.686897) -- (9.12823,0.686901);
\draw [c] (9.12009,0.686897) -- (9.12823,0.686897);
\draw [c] (9.12823,0.686897) -- (9.13636,0.686897);
\definecolor{c}{rgb}{0,0,0};
\colorlet{c}{natcomp!70};
\draw [c] (9.1445,0.686899) -- (9.1445,0.696014);
\draw [c] (9.1445,0.696014) -- (9.1445,0.70513);
\draw [c] (9.13636,0.696014) -- (9.1445,0.696014);
\draw [c] (9.1445,0.696014) -- (9.15264,0.696014);
\definecolor{c}{rgb}{0,0,0};
\colorlet{c}{natcomp!70};
\draw [c] (9.16077,0.686897) -- (9.16077,0.686903);
\draw [c] (9.16077,0.686903) -- (9.16077,0.68691);
\draw [c] (9.15264,0.686903) -- (9.16077,0.686903);
\draw [c] (9.16077,0.686903) -- (9.16891,0.686903);
\definecolor{c}{rgb}{0,0,0};
\colorlet{c}{natcomp!70};
\draw [c] (9.17705,0.686894) -- (9.17705,0.705403);
\draw [c] (9.17705,0.705403) -- (9.17705,0.723913);
\draw [c] (9.16891,0.705403) -- (9.17705,0.705403);
\draw [c] (9.17705,0.705403) -- (9.18518,0.705403);
\definecolor{c}{rgb}{0,0,0};
\colorlet{c}{natcomp!70};
\draw [c] (9.19332,0.686897) -- (9.19332,0.686906);
\draw [c] (9.19332,0.686906) -- (9.19332,0.686914);
\draw [c] (9.18518,0.686906) -- (9.19332,0.686906);
\draw [c] (9.19332,0.686906) -- (9.20145,0.686906);
\definecolor{c}{rgb}{0,0,0};
\colorlet{c}{natcomp!70};
\draw [c] (9.20959,0.686906) -- (9.20959,0.694628);
\draw [c] (9.20959,0.694628) -- (9.20959,0.70235);
\draw [c] (9.20145,0.694628) -- (9.20959,0.694628);
\draw [c] (9.20959,0.694628) -- (9.21773,0.694628);
\definecolor{c}{rgb}{0,0,0};
\colorlet{c}{natcomp!70};
\draw [c] (9.22586,0.686898) -- (9.22586,0.697671);
\draw [c] (9.22586,0.697671) -- (9.22586,0.708444);
\draw [c] (9.21773,0.697671) -- (9.22586,0.697671);
\draw [c] (9.22586,0.697671) -- (9.234,0.697671);
\definecolor{c}{rgb}{0,0,0};
\colorlet{c}{natcomp!70};
\draw [c] (9.24214,0.697919) -- (9.24214,0.713014);
\draw [c] (9.24214,0.713014) -- (9.24214,0.72811);
\draw [c] (9.234,0.713014) -- (9.24214,0.713014);
\draw [c] (9.24214,0.713014) -- (9.25027,0.713014);
\definecolor{c}{rgb}{0,0,0};
\colorlet{c}{natcomp!70};
\draw [c] (9.25841,0.686896) -- (9.25841,0.686902);
\draw [c] (9.25841,0.686902) -- (9.25841,0.686908);
\draw [c] (9.25027,0.686902) -- (9.25841,0.686902);
\draw [c] (9.25841,0.686902) -- (9.26655,0.686902);
\definecolor{c}{rgb}{0,0,0};
\colorlet{c}{natcomp!70};
\draw [c] (9.27468,0.686897) -- (9.27468,0.686903);
\draw [c] (9.27468,0.686903) -- (9.27468,0.68691);
\draw [c] (9.26655,0.686903) -- (9.27468,0.686903);
\draw [c] (9.27468,0.686903) -- (9.28282,0.686903);
\definecolor{c}{rgb}{0,0,0};
\colorlet{c}{natcomp!70};
\draw [c] (9.30723,0.691421) -- (9.30723,0.702341);
\draw [c] (9.30723,0.702341) -- (9.30723,0.713261);
\draw [c] (9.29909,0.702341) -- (9.30723,0.702341);
\draw [c] (9.30723,0.702341) -- (9.31536,0.702341);
\definecolor{c}{rgb}{0,0,0};
\colorlet{c}{natcomp!70};
\draw [c] (9.33977,0.686902) -- (9.33977,0.69504);
\draw [c] (9.33977,0.69504) -- (9.33977,0.703179);
\draw [c] (9.33164,0.69504) -- (9.33977,0.69504);
\draw [c] (9.33977,0.69504) -- (9.34791,0.69504);
\definecolor{c}{rgb}{0,0,0};
\colorlet{c}{natcomp!70};
\draw [c] (9.35605,0.693227) -- (9.35605,0.708478);
\draw [c] (9.35605,0.708478) -- (9.35605,0.723729);
\draw [c] (9.34791,0.708478) -- (9.35605,0.708478);
\draw [c] (9.35605,0.708478) -- (9.36418,0.708478);
\definecolor{c}{rgb}{0,0,0};
\colorlet{c}{natcomp!70};
\draw [c] (9.37232,0.686896) -- (9.37232,0.686902);
\draw [c] (9.37232,0.686902) -- (9.37232,0.686908);
\draw [c] (9.36418,0.686902) -- (9.37232,0.686902);
\draw [c] (9.37232,0.686902) -- (9.38046,0.686902);
\definecolor{c}{rgb}{0,0,0};
\colorlet{c}{natcomp!70};
\draw [c] (9.38859,0.686897) -- (9.38859,0.696013);
\draw [c] (9.38859,0.696013) -- (9.38859,0.705128);
\draw [c] (9.38046,0.696013) -- (9.38859,0.696013);
\draw [c] (9.38859,0.696013) -- (9.39673,0.696013);
\definecolor{c}{rgb}{0,0,0};
\colorlet{c}{natcomp!70};
\draw [c] (9.43741,0.686902) -- (9.43741,0.697312);
\draw [c] (9.43741,0.697312) -- (9.43741,0.707723);
\draw [c] (9.42927,0.697312) -- (9.43741,0.697312);
\draw [c] (9.43741,0.697312) -- (9.44555,0.697312);
\definecolor{c}{rgb}{0,0,0};
\colorlet{c}{natcomp!70};
\draw [c] (9.45368,0.686897) -- (9.45368,0.686903);
\draw [c] (9.45368,0.686903) -- (9.45368,0.68691);
\draw [c] (9.44555,0.686903) -- (9.45368,0.686903);
\draw [c] (9.45368,0.686903) -- (9.46182,0.686903);
\definecolor{c}{rgb}{0,0,0};
\colorlet{c}{natcomp!70};
\draw [c] (9.46995,0.692681) -- (9.46995,0.706988);
\draw [c] (9.46995,0.706988) -- (9.46995,0.721296);
\draw [c] (9.46182,0.706988) -- (9.46995,0.706988);
\draw [c] (9.46995,0.706988) -- (9.47809,0.706988);
\definecolor{c}{rgb}{0,0,0};
\colorlet{c}{natcomp!70};
\draw [c] (9.48623,0.686894) -- (9.48623,0.686903);
\draw [c] (9.48623,0.686903) -- (9.48623,0.686911);
\draw [c] (9.47809,0.686903) -- (9.48623,0.686903);
\draw [c] (9.48623,0.686903) -- (9.49436,0.686903);
\definecolor{c}{rgb}{0,0,0};
\colorlet{c}{natcomp!70};
\draw [c] (9.5025,0.686898) -- (9.5025,0.696793);
\draw [c] (9.5025,0.696793) -- (9.5025,0.706687);
\draw [c] (9.49436,0.696793) -- (9.5025,0.696793);
\draw [c] (9.5025,0.696793) -- (9.51064,0.696793);
\definecolor{c}{rgb}{0,0,0};
\colorlet{c}{natcomp!70};
\draw [c] (9.51877,0.686894) -- (9.51877,0.697689);
\draw [c] (9.51877,0.697689) -- (9.51877,0.708483);
\draw [c] (9.51064,0.697689) -- (9.51877,0.697689);
\draw [c] (9.51877,0.697689) -- (9.52691,0.697689);
\definecolor{c}{rgb}{0,0,0};
\colorlet{c}{natcomp!70};
\draw [c] (9.53505,0.686896) -- (9.53505,0.686901);
\draw [c] (9.53505,0.686901) -- (9.53505,0.686906);
\draw [c] (9.52691,0.686901) -- (9.53505,0.686901);
\draw [c] (9.53505,0.686901) -- (9.54318,0.686901);
\definecolor{c}{rgb}{0,0,0};
\colorlet{c}{natcomp!70};
\draw [c] (9.55132,0.686896) -- (9.55132,0.686902);
\draw [c] (9.55132,0.686902) -- (9.55132,0.686908);
\draw [c] (9.54318,0.686902) -- (9.55132,0.686902);
\draw [c] (9.55132,0.686902) -- (9.55945,0.686902);
\definecolor{c}{rgb}{0,0,0};
\colorlet{c}{natcomp!70};
\draw [c] (9.56759,0.686894) -- (9.56759,0.696788);
\draw [c] (9.56759,0.696788) -- (9.56759,0.706683);
\draw [c] (9.55945,0.696788) -- (9.56759,0.696788);
\draw [c] (9.56759,0.696788) -- (9.57573,0.696788);
\definecolor{c}{rgb}{0,0,0};
\colorlet{c}{natcomp!70};
\draw [c] (9.58386,0.686894) -- (9.58386,0.686898);
\draw [c] (9.58386,0.686898) -- (9.58386,0.686902);
\draw [c] (9.57573,0.686898) -- (9.58386,0.686898);
\draw [c] (9.58386,0.686898) -- (9.592,0.686898);
\definecolor{c}{rgb}{0,0,0};
\colorlet{c}{natcomp!70};
\draw [c] (9.60014,0.686899) -- (9.60014,0.695761);
\draw [c] (9.60014,0.695761) -- (9.60014,0.704623);
\draw [c] (9.592,0.695761) -- (9.60014,0.695761);
\draw [c] (9.60014,0.695761) -- (9.60827,0.695761);
\definecolor{c}{rgb}{0,0,0};
\colorlet{c}{natcomp!70};
\draw [c] (9.61641,0.691959) -- (9.61641,0.70451);
\draw [c] (9.61641,0.70451) -- (9.61641,0.717061);
\draw [c] (9.60827,0.70451) -- (9.61641,0.70451);
\draw [c] (9.61641,0.70451) -- (9.62455,0.70451);
\definecolor{c}{rgb}{0,0,0};
\colorlet{c}{natcomp!70};
\draw [c] (9.64895,0.686899) -- (9.64895,0.686907);
\draw [c] (9.64895,0.686907) -- (9.64895,0.686914);
\draw [c] (9.64082,0.686907) -- (9.64895,0.686907);
\draw [c] (9.64895,0.686907) -- (9.65709,0.686907);
\definecolor{c}{rgb}{0,0,0};
\colorlet{c}{natcomp!70};
\draw [c] (9.66523,0.686899) -- (9.66523,0.686907);
\draw [c] (9.66523,0.686907) -- (9.66523,0.686914);
\draw [c] (9.65709,0.686907) -- (9.66523,0.686907);
\draw [c] (9.66523,0.686907) -- (9.67336,0.686907);
\definecolor{c}{rgb}{0,0,0};
\colorlet{c}{natcomp!70};
\draw [c] (9.6815,0.686894) -- (9.6815,0.686897);
\draw [c] (9.6815,0.686897) -- (9.6815,0.686901);
\draw [c] (9.67336,0.686897) -- (9.6815,0.686897);
\draw [c] (9.6815,0.686897) -- (9.68964,0.686897);
\definecolor{c}{rgb}{0,0,0};
\colorlet{c}{natcomp!70};
\draw [c] (9.71405,0.692495) -- (9.71405,0.706167);
\draw [c] (9.71405,0.706167) -- (9.71405,0.719839);
\draw [c] (9.70591,0.706167) -- (9.71405,0.706167);
\draw [c] (9.71405,0.706167) -- (9.72218,0.706167);
\definecolor{c}{rgb}{0,0,0};
\colorlet{c}{natcomp!70};
\draw [c] (9.74659,0.686894) -- (9.74659,0.686897);
\draw [c] (9.74659,0.686897) -- (9.74659,0.686901);
\draw [c] (9.73845,0.686897) -- (9.74659,0.686897);
\draw [c] (9.74659,0.686897) -- (9.75473,0.686897);
\definecolor{c}{rgb}{0,0,0};
\colorlet{c}{natcomp!70};
\draw [c] (9.76286,0.686896) -- (9.76286,0.686901);
\draw [c] (9.76286,0.686901) -- (9.76286,0.686906);
\draw [c] (9.75473,0.686901) -- (9.76286,0.686901);
\draw [c] (9.76286,0.686901) -- (9.771,0.686901);
\definecolor{c}{rgb}{0,0,0};
\colorlet{c}{natcomp!70};
\draw [c] (9.77914,0.686896) -- (9.77914,0.686902);
\draw [c] (9.77914,0.686902) -- (9.77914,0.686908);
\draw [c] (9.771,0.686902) -- (9.77914,0.686902);
\draw [c] (9.77914,0.686902) -- (9.78727,0.686902);
\definecolor{c}{rgb}{0,0,0};
\colorlet{c}{natcomp!70};
\draw [c] (9.79541,0.686898) -- (9.79541,0.696014);
\draw [c] (9.79541,0.696014) -- (9.79541,0.705129);
\draw [c] (9.78727,0.696014) -- (9.79541,0.696014);
\draw [c] (9.79541,0.696014) -- (9.80354,0.696014);
\definecolor{c}{rgb}{0,0,0};
\colorlet{c}{natcomp!70};
\draw [c] (9.81168,0.686894) -- (9.81168,0.686899);
\draw [c] (9.81168,0.686899) -- (9.81168,0.686903);
\draw [c] (9.80354,0.686899) -- (9.81168,0.686899);
\draw [c] (9.81168,0.686899) -- (9.81982,0.686899);
\definecolor{c}{rgb}{0,0,0};
\colorlet{c}{natcomp!70};
\draw [c] (9.82795,0.692367) -- (9.82795,0.70565);
\draw [c] (9.82795,0.70565) -- (9.82795,0.718933);
\draw [c] (9.81982,0.70565) -- (9.82795,0.70565);
\draw [c] (9.82795,0.70565) -- (9.83609,0.70565);
\definecolor{c}{rgb}{0,0,0};
\colorlet{c}{natcomp!70};
\draw [c] (9.84423,0.686894) -- (9.84423,0.686898);
\draw [c] (9.84423,0.686898) -- (9.84423,0.686902);
\draw [c] (9.83609,0.686898) -- (9.84423,0.686898);
\draw [c] (9.84423,0.686898) -- (9.85236,0.686898);
\definecolor{c}{rgb}{0,0,0};
\colorlet{c}{natcomp!70};
\draw [c] (9.8605,0.686894) -- (9.8605,0.686901);
\draw [c] (9.8605,0.686901) -- (9.8605,0.686908);
\draw [c] (9.85236,0.686901) -- (9.8605,0.686901);
\draw [c] (9.8605,0.686901) -- (9.86864,0.686901);
\definecolor{c}{rgb}{0,0,0};
\colorlet{c}{natcomp!70};
\draw [c] (9.87677,0.686894) -- (9.87677,0.686899);
\draw [c] (9.87677,0.686899) -- (9.87677,0.686903);
\draw [c] (9.86864,0.686899) -- (9.87677,0.686899);
\draw [c] (9.87677,0.686899) -- (9.88491,0.686899);
\definecolor{c}{rgb}{0,0,0};
\colorlet{c}{natcomp!70};
\draw [c] (9.89305,0.691538) -- (9.89305,0.702957);
\draw [c] (9.89305,0.702957) -- (9.89305,0.714375);
\draw [c] (9.88491,0.702957) -- (9.89305,0.702957);
\draw [c] (9.89305,0.702957) -- (9.90118,0.702957);
\definecolor{c}{rgb}{0,0,0};
\colorlet{c}{natcomp!70};
\draw [c] (9.90932,0.692367) -- (9.90932,0.70565);
\draw [c] (9.90932,0.70565) -- (9.90932,0.718933);
\draw [c] (9.90118,0.70565) -- (9.90932,0.70565);
\draw [c] (9.90932,0.70565) -- (9.91745,0.70565);
\definecolor{c}{rgb}{0,0,0};
\colorlet{c}{natcomp!70};
\draw [c] (9.94186,0.686899) -- (9.94186,0.686907);
\draw [c] (9.94186,0.686907) -- (9.94186,0.686915);
\draw [c] (9.93373,0.686907) -- (9.94186,0.686907);
\draw [c] (9.94186,0.686907) -- (9.95,0.686907);
\definecolor{c}{rgb}{0,0,0};
\draw [anchor=base west] (6.62249,6.01827) node[color=c, rotate=0]{ALTAS MC};
\colorlet{c}{natgreen};
\draw [c] (5.87661,6.12464) -- (6.49087,6.12464);
\draw [c] (6.18374,5.98281) -- (6.18374,6.26648);
\definecolor{c}{rgb}{0,0,0};
\draw [anchor=base west] (6.62249,5.54549) node[color=c, rotate=0]{CalcHEP MC};
\colorlet{c}{natcomp!70};
\draw [c] (5.87661,5.65186) -- (6.49087,5.65186);
\draw [c] (6.18374,5.51003) -- (6.18374,5.7937);
\end{tikzpicture}

\end{infilsf}
\end{minipage}
\hfill
\begin{minipage}[b]{.49\textwidth}
\begin{infilsf} \tiny 
\begin{tikzpicture}[x=.092\textwidth,y=.092\textwidth]
\pgfdeclareplotmark{cross} {
\pgfpathmoveto{\pgfpoint{-0.3\pgfplotmarksize}{\pgfplotmarksize}}
\pgfpathlineto{\pgfpoint{+0.3\pgfplotmarksize}{\pgfplotmarksize}}
\pgfpathlineto{\pgfpoint{+0.3\pgfplotmarksize}{0.3\pgfplotmarksize}}
\pgfpathlineto{\pgfpoint{+1\pgfplotmarksize}{0.3\pgfplotmarksize}}
\pgfpathlineto{\pgfpoint{+1\pgfplotmarksize}{-0.3\pgfplotmarksize}}
\pgfpathlineto{\pgfpoint{+0.3\pgfplotmarksize}{-0.3\pgfplotmarksize}}
\pgfpathlineto{\pgfpoint{+0.3\pgfplotmarksize}{-1.\pgfplotmarksize}}
\pgfpathlineto{\pgfpoint{-0.3\pgfplotmarksize}{-1.\pgfplotmarksize}}
\pgfpathlineto{\pgfpoint{-0.3\pgfplotmarksize}{-0.3\pgfplotmarksize}}
\pgfpathlineto{\pgfpoint{-1.\pgfplotmarksize}{-0.3\pgfplotmarksize}}
\pgfpathlineto{\pgfpoint{-1.\pgfplotmarksize}{0.3\pgfplotmarksize}}
\pgfpathlineto{\pgfpoint{-0.3\pgfplotmarksize}{0.3\pgfplotmarksize}}
\pgfpathclose
\pgfusepathqstroke
}
\pgfdeclareplotmark{cross*} {
\pgfpathmoveto{\pgfpoint{-0.3\pgfplotmarksize}{\pgfplotmarksize}}
\pgfpathlineto{\pgfpoint{+0.3\pgfplotmarksize}{\pgfplotmarksize}}
\pgfpathlineto{\pgfpoint{+0.3\pgfplotmarksize}{0.3\pgfplotmarksize}}
\pgfpathlineto{\pgfpoint{+1\pgfplotmarksize}{0.3\pgfplotmarksize}}
\pgfpathlineto{\pgfpoint{+1\pgfplotmarksize}{-0.3\pgfplotmarksize}}
\pgfpathlineto{\pgfpoint{+0.3\pgfplotmarksize}{-0.3\pgfplotmarksize}}
\pgfpathlineto{\pgfpoint{+0.3\pgfplotmarksize}{-1.\pgfplotmarksize}}
\pgfpathlineto{\pgfpoint{-0.3\pgfplotmarksize}{-1.\pgfplotmarksize}}
\pgfpathlineto{\pgfpoint{-0.3\pgfplotmarksize}{-0.3\pgfplotmarksize}}
\pgfpathlineto{\pgfpoint{-1.\pgfplotmarksize}{-0.3\pgfplotmarksize}}
\pgfpathlineto{\pgfpoint{-1.\pgfplotmarksize}{0.3\pgfplotmarksize}}
\pgfpathlineto{\pgfpoint{-0.3\pgfplotmarksize}{0.3\pgfplotmarksize}}
\pgfpathclose
\pgfusepathqfillstroke
}
\pgfdeclareplotmark{newstar} {
\pgfpathmoveto{\pgfqpoint{0pt}{\pgfplotmarksize}}
\pgfpathlineto{\pgfqpointpolar{44}{0.5\pgfplotmarksize}}
\pgfpathlineto{\pgfqpointpolar{18}{\pgfplotmarksize}}
\pgfpathlineto{\pgfqpointpolar{-20}{0.5\pgfplotmarksize}}
\pgfpathlineto{\pgfqpointpolar{-54}{\pgfplotmarksize}}
\pgfpathlineto{\pgfqpointpolar{-90}{0.5\pgfplotmarksize}}
\pgfpathlineto{\pgfqpointpolar{234}{\pgfplotmarksize}}
\pgfpathlineto{\pgfqpointpolar{198}{0.5\pgfplotmarksize}}
\pgfpathlineto{\pgfqpointpolar{162}{\pgfplotmarksize}}
\pgfpathlineto{\pgfqpointpolar{134}{0.5\pgfplotmarksize}}
\pgfpathclose
\pgfusepathqstroke
}
\pgfdeclareplotmark{newstar*} {
\pgfpathmoveto{\pgfqpoint{0pt}{\pgfplotmarksize}}
\pgfpathlineto{\pgfqpointpolar{44}{0.5\pgfplotmarksize}}
\pgfpathlineto{\pgfqpointpolar{18}{\pgfplotmarksize}}
\pgfpathlineto{\pgfqpointpolar{-20}{0.5\pgfplotmarksize}}
\pgfpathlineto{\pgfqpointpolar{-54}{\pgfplotmarksize}}
\pgfpathlineto{\pgfqpointpolar{-90}{0.5\pgfplotmarksize}}
\pgfpathlineto{\pgfqpointpolar{234}{\pgfplotmarksize}}
\pgfpathlineto{\pgfqpointpolar{198}{0.5\pgfplotmarksize}}
\pgfpathlineto{\pgfqpointpolar{162}{\pgfplotmarksize}}
\pgfpathlineto{\pgfqpointpolar{134}{0.5\pgfplotmarksize}}
\pgfpathclose
\pgfusepathqfillstroke
}
\definecolor{c}{rgb}{1,1,1};
\draw [color=c, fill=c] (0,0) rectangle (10,6.80516);
\draw [color=c, fill=c] (1,0.680516) rectangle (9.95,6.73711);
\definecolor{c}{rgb}{0,0,0};
\draw [c] (1,0.680516) -- (1,6.73711) -- (9.95,6.73711) -- (9.95,0.680516) -- (1,0.680516);
\definecolor{c}{rgb}{1,1,1};
\draw [color=c, fill=c] (1,0.680516) rectangle (9.95,6.73711);
\definecolor{c}{rgb}{0,0,0};
\draw [c] (1,0.680516) -- (1,6.73711) -- (9.95,6.73711) -- (9.95,0.680516) -- (1,0.680516);
\colorlet{c}{natgreen};
\draw [c] (1.09022,0.740799) -- (1.09022,0.753375);
\draw [c] (1.09022,0.753375) -- (1.09022,0.76595);
\draw [c] (1.07218,0.753375) -- (1.09022,0.753375);
\draw [c] (1.09022,0.753375) -- (1.10827,0.753375);
\definecolor{c}{rgb}{0,0,0};
\colorlet{c}{natgreen};
\draw [c] (1.12631,0.740369) -- (1.12631,0.752314);
\draw [c] (1.12631,0.752314) -- (1.12631,0.76426);
\draw [c] (1.10827,0.752314) -- (1.12631,0.752314);
\draw [c] (1.12631,0.752314) -- (1.14435,0.752314);
\definecolor{c}{rgb}{0,0,0};
\colorlet{c}{natgreen};
\draw [c] (1.30675,0.737608) -- (1.30675,0.746848);
\draw [c] (1.30675,0.746848) -- (1.30675,0.756087);
\draw [c] (1.28871,0.746848) -- (1.30675,0.746848);
\draw [c] (1.30675,0.746848) -- (1.3248,0.746848);
\definecolor{c}{rgb}{0,0,0};
\colorlet{c}{natgreen};
\draw [c] (1.37893,0.755182) -- (1.37893,0.777154);
\draw [c] (1.37893,0.777154) -- (1.37893,0.799127);
\draw [c] (1.36089,0.777154) -- (1.37893,0.777154);
\draw [c] (1.37893,0.777154) -- (1.39698,0.777154);
\definecolor{c}{rgb}{0,0,0};
\colorlet{c}{natgreen};
\draw [c] (1.41502,0.755947) -- (1.41502,0.779138);
\draw [c] (1.41502,0.779138) -- (1.41502,0.802329);
\draw [c] (1.39698,0.779138) -- (1.41502,0.779138);
\draw [c] (1.41502,0.779138) -- (1.43306,0.779138);
\definecolor{c}{rgb}{0,0,0};
\colorlet{c}{natgreen};
\draw [c] (1.45111,0.768124) -- (1.45111,0.791905);
\draw [c] (1.45111,0.791905) -- (1.45111,0.815686);
\draw [c] (1.43306,0.791905) -- (1.45111,0.791905);
\draw [c] (1.45111,0.791905) -- (1.46915,0.791905);
\definecolor{c}{rgb}{0,0,0};
\colorlet{c}{natgreen};
\draw [c] (1.4872,0.758252) -- (1.4872,0.782938);
\draw [c] (1.4872,0.782938) -- (1.4872,0.807624);
\draw [c] (1.46915,0.782938) -- (1.4872,0.782938);
\draw [c] (1.4872,0.782938) -- (1.50524,0.782938);
\definecolor{c}{rgb}{0,0,0};
\colorlet{c}{natgreen};
\draw [c] (1.52329,0.78146) -- (1.52329,0.808304);
\draw [c] (1.52329,0.808304) -- (1.52329,0.835148);
\draw [c] (1.50524,0.808304) -- (1.52329,0.808304);
\draw [c] (1.52329,0.808304) -- (1.54133,0.808304);
\definecolor{c}{rgb}{0,0,0};
\colorlet{c}{natgreen};
\draw [c] (1.55938,0.77573) -- (1.55938,0.802259);
\draw [c] (1.55938,0.802259) -- (1.55938,0.828788);
\draw [c] (1.54133,0.802259) -- (1.55938,0.802259);
\draw [c] (1.55938,0.802259) -- (1.57742,0.802259);
\definecolor{c}{rgb}{0,0,0};
\colorlet{c}{natgreen};
\draw [c] (1.59546,0.761209) -- (1.59546,0.784068);
\draw [c] (1.59546,0.784068) -- (1.59546,0.806926);
\draw [c] (1.57742,0.784068) -- (1.59546,0.784068);
\draw [c] (1.59546,0.784068) -- (1.61351,0.784068);
\definecolor{c}{rgb}{0,0,0};
\colorlet{c}{natgreen};
\draw [c] (1.63155,0.802522) -- (1.63155,0.838187);
\draw [c] (1.63155,0.838187) -- (1.63155,0.873852);
\draw [c] (1.61351,0.838187) -- (1.63155,0.838187);
\draw [c] (1.63155,0.838187) -- (1.6496,0.838187);
\definecolor{c}{rgb}{0,0,0};
\colorlet{c}{natgreen};
\draw [c] (1.66764,0.829492) -- (1.66764,0.865601);
\draw [c] (1.66764,0.865601) -- (1.66764,0.901709);
\draw [c] (1.6496,0.865601) -- (1.66764,0.865601);
\draw [c] (1.66764,0.865601) -- (1.68569,0.865601);
\definecolor{c}{rgb}{0,0,0};
\colorlet{c}{natgreen};
\draw [c] (1.70373,0.80146) -- (1.70373,0.833741);
\draw [c] (1.70373,0.833741) -- (1.70373,0.866023);
\draw [c] (1.68569,0.833741) -- (1.70373,0.833741);
\draw [c] (1.70373,0.833741) -- (1.72177,0.833741);
\definecolor{c}{rgb}{0,0,0};
\colorlet{c}{natgreen};
\draw [c] (1.73982,0.803179) -- (1.73982,0.835265);
\draw [c] (1.73982,0.835265) -- (1.73982,0.867351);
\draw [c] (1.72177,0.835265) -- (1.73982,0.835265);
\draw [c] (1.73982,0.835265) -- (1.75786,0.835265);
\definecolor{c}{rgb}{0,0,0};
\colorlet{c}{natgreen};
\draw [c] (1.77591,0.898175) -- (1.77591,0.946579);
\draw [c] (1.77591,0.946579) -- (1.77591,0.994983);
\draw [c] (1.75786,0.946579) -- (1.77591,0.946579);
\draw [c] (1.77591,0.946579) -- (1.79395,0.946579);
\definecolor{c}{rgb}{0,0,0};
\colorlet{c}{natgreen};
\draw [c] (1.812,0.953061) -- (1.812,1.00914);
\draw [c] (1.812,1.00914) -- (1.812,1.06522);
\draw [c] (1.79395,1.00914) -- (1.812,1.00914);
\draw [c] (1.812,1.00914) -- (1.83004,1.00914);
\definecolor{c}{rgb}{0,0,0};
\colorlet{c}{natgreen};
\draw [c] (1.84808,0.939044) -- (1.84808,0.99126);
\draw [c] (1.84808,0.99126) -- (1.84808,1.04348);
\draw [c] (1.83004,0.99126) -- (1.84808,0.99126);
\draw [c] (1.84808,0.99126) -- (1.86613,0.99126);
\definecolor{c}{rgb}{0,0,0};
\colorlet{c}{natgreen};
\draw [c] (1.88417,1.08459) -- (1.88417,1.15614);
\draw [c] (1.88417,1.15614) -- (1.88417,1.22769);
\draw [c] (1.86613,1.15614) -- (1.88417,1.15614);
\draw [c] (1.88417,1.15614) -- (1.90222,1.15614);
\definecolor{c}{rgb}{0,0,0};
\colorlet{c}{natgreen};
\draw [c] (1.92026,1.07599) -- (1.92026,1.14389);
\draw [c] (1.92026,1.14389) -- (1.92026,1.21179);
\draw [c] (1.90222,1.14389) -- (1.92026,1.14389);
\draw [c] (1.92026,1.14389) -- (1.93831,1.14389);
\definecolor{c}{rgb}{0,0,0};
\colorlet{c}{natgreen};
\draw [c] (1.95635,1.20977) -- (1.95635,1.28799);
\draw [c] (1.95635,1.28799) -- (1.95635,1.36621);
\draw [c] (1.93831,1.28799) -- (1.95635,1.28799);
\draw [c] (1.95635,1.28799) -- (1.9744,1.28799);
\definecolor{c}{rgb}{0,0,0};
\colorlet{c}{natgreen};
\draw [c] (1.99244,1.34676) -- (1.99244,1.43945);
\draw [c] (1.99244,1.43945) -- (1.99244,1.53215);
\draw [c] (1.9744,1.43945) -- (1.99244,1.43945);
\draw [c] (1.99244,1.43945) -- (2.01048,1.43945);
\definecolor{c}{rgb}{0,0,0};
\colorlet{c}{natgreen};
\draw [c] (2.02853,1.38683) -- (2.02853,1.48215);
\draw [c] (2.02853,1.48215) -- (2.02853,1.57747);
\draw [c] (2.01048,1.48215) -- (2.02853,1.48215);
\draw [c] (2.02853,1.48215) -- (2.04657,1.48215);
\definecolor{c}{rgb}{0,0,0};
\colorlet{c}{natgreen};
\draw [c] (2.06462,1.59757) -- (2.06462,1.70779);
\draw [c] (2.06462,1.70779) -- (2.06462,1.81801);
\draw [c] (2.04657,1.70779) -- (2.06462,1.70779);
\draw [c] (2.06462,1.70779) -- (2.08266,1.70779);
\definecolor{c}{rgb}{0,0,0};
\colorlet{c}{natgreen};
\draw [c] (2.10071,1.84589) -- (2.10071,1.96852);
\draw [c] (2.10071,1.96852) -- (2.10071,2.09115);
\draw [c] (2.08266,1.96852) -- (2.10071,1.96852);
\draw [c] (2.10071,1.96852) -- (2.11875,1.96852);
\definecolor{c}{rgb}{0,0,0};
\colorlet{c}{natgreen};
\draw [c] (2.13679,1.88974) -- (2.13679,2.01623);
\draw [c] (2.13679,2.01623) -- (2.13679,2.14272);
\draw [c] (2.11875,2.01623) -- (2.13679,2.01623);
\draw [c] (2.13679,2.01623) -- (2.15484,2.01623);
\definecolor{c}{rgb}{0,0,0};
\colorlet{c}{natgreen};
\draw [c] (2.17288,2.27127) -- (2.17288,2.41567);
\draw [c] (2.17288,2.41567) -- (2.17288,2.56008);
\draw [c] (2.15484,2.41567) -- (2.17288,2.41567);
\draw [c] (2.17288,2.41567) -- (2.19093,2.41567);
\definecolor{c}{rgb}{0,0,0};
\colorlet{c}{natgreen};
\draw [c] (2.20897,2.40762) -- (2.20897,2.55977);
\draw [c] (2.20897,2.55977) -- (2.20897,2.71191);
\draw [c] (2.19093,2.55977) -- (2.20897,2.55977);
\draw [c] (2.20897,2.55977) -- (2.22702,2.55977);
\definecolor{c}{rgb}{0,0,0};
\colorlet{c}{natgreen};
\draw [c] (2.24506,2.6453) -- (2.24506,2.80823);
\draw [c] (2.24506,2.80823) -- (2.24506,2.97115);
\draw [c] (2.22702,2.80823) -- (2.24506,2.80823);
\draw [c] (2.24506,2.80823) -- (2.2631,2.80823);
\definecolor{c}{rgb}{0,0,0};
\colorlet{c}{natgreen};
\draw [c] (2.28115,3.10578) -- (2.28115,3.28792);
\draw [c] (2.28115,3.28792) -- (2.28115,3.47006);
\draw [c] (2.2631,3.28792) -- (2.28115,3.28792);
\draw [c] (2.28115,3.28792) -- (2.29919,3.28792);
\definecolor{c}{rgb}{0,0,0};
\colorlet{c}{natgreen};
\draw [c] (2.31724,3.25624) -- (2.31724,3.44303);
\draw [c] (2.31724,3.44303) -- (2.31724,3.62982);
\draw [c] (2.29919,3.44303) -- (2.31724,3.44303);
\draw [c] (2.31724,3.44303) -- (2.33528,3.44303);
\definecolor{c}{rgb}{0,0,0};
\colorlet{c}{natgreen};
\draw [c] (2.35333,3.66962) -- (2.35333,3.87126);
\draw [c] (2.35333,3.87126) -- (2.35333,4.07289);
\draw [c] (2.33528,3.87126) -- (2.35333,3.87126);
\draw [c] (2.35333,3.87126) -- (2.37137,3.87126);
\definecolor{c}{rgb}{0,0,0};
\colorlet{c}{natgreen};
\draw [c] (2.38942,4.0976) -- (2.38942,4.3147);
\draw [c] (2.38942,4.3147) -- (2.38942,4.5318);
\draw [c] (2.37137,4.3147) -- (2.38942,4.3147);
\draw [c] (2.38942,4.3147) -- (2.40746,4.3147);
\definecolor{c}{rgb}{0,0,0};
\colorlet{c}{natgreen};
\draw [c] (2.4255,4.48404) -- (2.4255,4.71294);
\draw [c] (2.4255,4.71294) -- (2.4255,4.94183);
\draw [c] (2.40746,4.71294) -- (2.4255,4.71294);
\draw [c] (2.4255,4.71294) -- (2.44355,4.71294);
\definecolor{c}{rgb}{0,0,0};
\colorlet{c}{natgreen};
\draw [c] (2.46159,5.32319) -- (2.46159,5.58255);
\draw [c] (2.46159,5.58255) -- (2.46159,5.84191);
\draw [c] (2.44355,5.58255) -- (2.46159,5.58255);
\draw [c] (2.46159,5.58255) -- (2.47964,5.58255);
\definecolor{c}{rgb}{0,0,0};
\colorlet{c}{natgreen};
\draw [c] (2.49768,5.48167) -- (2.49768,5.74099);
\draw [c] (2.49768,5.74099) -- (2.49768,6.00031);
\draw [c] (2.47964,5.74099) -- (2.49768,5.74099);
\draw [c] (2.49768,5.74099) -- (2.51573,5.74099);
\definecolor{c}{rgb}{0,0,0};
\colorlet{c}{natgreen};
\draw [c] (2.53377,5.79817) -- (2.53377,6.06855);
\draw [c] (2.53377,6.06855) -- (2.53377,6.33893);
\draw [c] (2.51573,6.06855) -- (2.53377,6.06855);
\draw [c] (2.53377,6.06855) -- (2.55181,6.06855);
\definecolor{c}{rgb}{0,0,0};
\colorlet{c}{natgreen};
\draw [c] (2.56986,5.60846) -- (2.56986,5.87462);
\draw [c] (2.56986,5.87462) -- (2.56986,6.14078);
\draw [c] (2.55181,5.87462) -- (2.56986,5.87462);
\draw [c] (2.56986,5.87462) -- (2.5879,5.87462);
\definecolor{c}{rgb}{0,0,0};
\colorlet{c}{natgreen};
\draw [c] (2.60595,5.15422) -- (2.60595,5.4071);
\draw [c] (2.60595,5.4071) -- (2.60595,5.65997);
\draw [c] (2.5879,5.4071) -- (2.60595,5.4071);
\draw [c] (2.60595,5.4071) -- (2.62399,5.4071);
\definecolor{c}{rgb}{0,0,0};
\colorlet{c}{natgreen};
\draw [c] (2.64204,5.79256) -- (2.64204,6.06206);
\draw [c] (2.64204,6.06206) -- (2.64204,6.33155);
\draw [c] (2.62399,6.06206) -- (2.64204,6.06206);
\draw [c] (2.64204,6.06206) -- (2.66008,6.06206);
\definecolor{c}{rgb}{0,0,0};
\colorlet{c}{natgreen};
\draw [c] (2.67813,5.8431) -- (2.67813,6.11415);
\draw [c] (2.67813,6.11415) -- (2.67813,6.38519);
\draw [c] (2.66008,6.11415) -- (2.67813,6.11415);
\draw [c] (2.67813,6.11415) -- (2.69617,6.11415);
\definecolor{c}{rgb}{0,0,0};
\colorlet{c}{natgreen};
\draw [c] (2.71421,5.63737) -- (2.71421,5.90035);
\draw [c] (2.71421,5.90035) -- (2.71421,6.16332);
\draw [c] (2.69617,5.90035) -- (2.71421,5.90035);
\draw [c] (2.71421,5.90035) -- (2.73226,5.90035);
\definecolor{c}{rgb}{0,0,0};
\colorlet{c}{natgreen};
\draw [c] (2.7503,5.25438) -- (2.7503,5.50765);
\draw [c] (2.7503,5.50765) -- (2.7503,5.76092);
\draw [c] (2.73226,5.50765) -- (2.7503,5.50765);
\draw [c] (2.7503,5.50765) -- (2.76835,5.50765);
\definecolor{c}{rgb}{0,0,0};
\colorlet{c}{natgreen};
\draw [c] (2.78639,5.03639) -- (2.78639,5.2877);
\draw [c] (2.78639,5.2877) -- (2.78639,5.53901);
\draw [c] (2.76835,5.2877) -- (2.78639,5.2877);
\draw [c] (2.78639,5.2877) -- (2.80444,5.2877);
\definecolor{c}{rgb}{0,0,0};
\colorlet{c}{natgreen};
\draw [c] (2.82248,4.89291) -- (2.82248,5.13628);
\draw [c] (2.82248,5.13628) -- (2.82248,5.37966);
\draw [c] (2.80444,5.13628) -- (2.82248,5.13628);
\draw [c] (2.82248,5.13628) -- (2.84052,5.13628);
\definecolor{c}{rgb}{0,0,0};
\colorlet{c}{natgreen};
\draw [c] (2.85857,4.3619) -- (2.85857,4.58945);
\draw [c] (2.85857,4.58945) -- (2.85857,4.81699);
\draw [c] (2.84052,4.58945) -- (2.85857,4.58945);
\draw [c] (2.85857,4.58945) -- (2.87661,4.58945);
\definecolor{c}{rgb}{0,0,0};
\colorlet{c}{natgreen};
\draw [c] (2.89466,3.86365) -- (2.89466,4.07372);
\draw [c] (2.89466,4.07372) -- (2.89466,4.28379);
\draw [c] (2.87661,4.07372) -- (2.89466,4.07372);
\draw [c] (2.89466,4.07372) -- (2.9127,4.07372);
\definecolor{c}{rgb}{0,0,0};
\colorlet{c}{natgreen};
\draw [c] (2.93075,3.79834) -- (2.93075,4.0075);
\draw [c] (2.93075,4.0075) -- (2.93075,4.21665);
\draw [c] (2.9127,4.0075) -- (2.93075,4.0075);
\draw [c] (2.93075,4.0075) -- (2.94879,4.0075);
\definecolor{c}{rgb}{0,0,0};
\colorlet{c}{natgreen};
\draw [c] (2.96683,4.24582) -- (2.96683,4.47087);
\draw [c] (2.96683,4.47087) -- (2.96683,4.69591);
\draw [c] (2.94879,4.47087) -- (2.96683,4.47087);
\draw [c] (2.96683,4.47087) -- (2.98488,4.47087);
\definecolor{c}{rgb}{0,0,0};
\colorlet{c}{natgreen};
\draw [c] (3.00292,3.50606) -- (3.00292,3.70501);
\draw [c] (3.00292,3.70501) -- (3.00292,3.90397);
\draw [c] (2.98488,3.70501) -- (3.00292,3.70501);
\draw [c] (3.00292,3.70501) -- (3.02097,3.70501);
\definecolor{c}{rgb}{0,0,0};
\colorlet{c}{natgreen};
\draw [c] (3.03901,3.34697) -- (3.03901,3.54204);
\draw [c] (3.03901,3.54204) -- (3.03901,3.73711);
\draw [c] (3.02097,3.54204) -- (3.03901,3.54204);
\draw [c] (3.03901,3.54204) -- (3.05706,3.54204);
\definecolor{c}{rgb}{0,0,0};
\colorlet{c}{natgreen};
\draw [c] (3.0751,3.09791) -- (3.0751,3.28161);
\draw [c] (3.0751,3.28161) -- (3.0751,3.46531);
\draw [c] (3.05706,3.28161) -- (3.0751,3.28161);
\draw [c] (3.0751,3.28161) -- (3.09315,3.28161);
\definecolor{c}{rgb}{0,0,0};
\colorlet{c}{natgreen};
\draw [c] (3.11119,2.9437) -- (3.11119,3.12193);
\draw [c] (3.11119,3.12193) -- (3.11119,3.30016);
\draw [c] (3.09315,3.12193) -- (3.11119,3.12193);
\draw [c] (3.11119,3.12193) -- (3.12923,3.12193);
\definecolor{c}{rgb}{0,0,0};
\colorlet{c}{natgreen};
\draw [c] (3.14728,2.78324) -- (3.14728,2.95447);
\draw [c] (3.14728,2.95447) -- (3.14728,3.1257);
\draw [c] (3.12923,2.95447) -- (3.14728,2.95447);
\draw [c] (3.14728,2.95447) -- (3.16532,2.95447);
\definecolor{c}{rgb}{0,0,0};
\colorlet{c}{natgreen};
\draw [c] (3.18337,2.77691) -- (3.18337,2.94856);
\draw [c] (3.18337,2.94856) -- (3.18337,3.1202);
\draw [c] (3.16532,2.94856) -- (3.18337,2.94856);
\draw [c] (3.18337,2.94856) -- (3.20141,2.94856);
\definecolor{c}{rgb}{0,0,0};
\colorlet{c}{natgreen};
\draw [c] (3.21946,2.48398) -- (3.21946,2.64232);
\draw [c] (3.21946,2.64232) -- (3.21946,2.80067);
\draw [c] (3.20141,2.64232) -- (3.21946,2.64232);
\draw [c] (3.21946,2.64232) -- (3.2375,2.64232);
\definecolor{c}{rgb}{0,0,0};
\colorlet{c}{natgreen};
\draw [c] (3.25554,2.42662) -- (3.25554,2.58401);
\draw [c] (3.25554,2.58401) -- (3.25554,2.7414);
\draw [c] (3.2375,2.58401) -- (3.25554,2.58401);
\draw [c] (3.25554,2.58401) -- (3.27359,2.58401);
\definecolor{c}{rgb}{0,0,0};
\colorlet{c}{natgreen};
\draw [c] (3.29163,2.45462) -- (3.29163,2.61483);
\draw [c] (3.29163,2.61483) -- (3.29163,2.77503);
\draw [c] (3.27359,2.61483) -- (3.29163,2.61483);
\draw [c] (3.29163,2.61483) -- (3.30968,2.61483);
\definecolor{c}{rgb}{0,0,0};
\colorlet{c}{natgreen};
\draw [c] (3.32772,2.09209) -- (3.32772,2.23211);
\draw [c] (3.32772,2.23211) -- (3.32772,2.37213);
\draw [c] (3.30968,2.23211) -- (3.32772,2.23211);
\draw [c] (3.32772,2.23211) -- (3.34577,2.23211);
\definecolor{c}{rgb}{0,0,0};
\colorlet{c}{natgreen};
\draw [c] (3.36381,2.05646) -- (3.36381,2.19661);
\draw [c] (3.36381,2.19661) -- (3.36381,2.33676);
\draw [c] (3.34577,2.19661) -- (3.36381,2.19661);
\draw [c] (3.36381,2.19661) -- (3.38185,2.19661);
\definecolor{c}{rgb}{0,0,0};
\colorlet{c}{natgreen};
\draw [c] (3.3999,1.90968) -- (3.3999,2.0376);
\draw [c] (3.3999,2.0376) -- (3.3999,2.16552);
\draw [c] (3.38185,2.0376) -- (3.3999,2.0376);
\draw [c] (3.3999,2.0376) -- (3.41794,2.0376);
\definecolor{c}{rgb}{0,0,0};
\colorlet{c}{natgreen};
\draw [c] (3.43599,1.8398) -- (3.43599,1.96403);
\draw [c] (3.43599,1.96403) -- (3.43599,2.08826);
\draw [c] (3.41794,1.96403) -- (3.43599,1.96403);
\draw [c] (3.43599,1.96403) -- (3.45403,1.96403);
\definecolor{c}{rgb}{0,0,0};
\colorlet{c}{natgreen};
\draw [c] (3.47208,1.89732) -- (3.47208,2.02992);
\draw [c] (3.47208,2.02992) -- (3.47208,2.16253);
\draw [c] (3.45403,2.02992) -- (3.47208,2.02992);
\draw [c] (3.47208,2.02992) -- (3.49012,2.02992);
\definecolor{c}{rgb}{0,0,0};
\colorlet{c}{natgreen};
\draw [c] (3.50817,1.76818) -- (3.50817,1.88907);
\draw [c] (3.50817,1.88907) -- (3.50817,2.00995);
\draw [c] (3.49012,1.88907) -- (3.50817,1.88907);
\draw [c] (3.50817,1.88907) -- (3.52621,1.88907);
\definecolor{c}{rgb}{0,0,0};
\colorlet{c}{natgreen};
\draw [c] (3.54425,1.65461) -- (3.54425,1.77173);
\draw [c] (3.54425,1.77173) -- (3.54425,1.88885);
\draw [c] (3.52621,1.77173) -- (3.54425,1.77173);
\draw [c] (3.54425,1.77173) -- (3.5623,1.77173);
\definecolor{c}{rgb}{0,0,0};
\colorlet{c}{natgreen};
\draw [c] (3.58034,1.83018) -- (3.58034,1.9561);
\draw [c] (3.58034,1.9561) -- (3.58034,2.08202);
\draw [c] (3.5623,1.9561) -- (3.58034,1.9561);
\draw [c] (3.58034,1.9561) -- (3.59839,1.9561);
\definecolor{c}{rgb}{0,0,0};
\colorlet{c}{natgreen};
\draw [c] (3.61643,1.73218) -- (3.61643,1.85168);
\draw [c] (3.61643,1.85168) -- (3.61643,1.97118);
\draw [c] (3.59839,1.85168) -- (3.61643,1.85168);
\draw [c] (3.61643,1.85168) -- (3.63448,1.85168);
\definecolor{c}{rgb}{0,0,0};
\colorlet{c}{natgreen};
\draw [c] (3.65252,1.5139) -- (3.65252,1.62264);
\draw [c] (3.65252,1.62264) -- (3.65252,1.73139);
\draw [c] (3.63448,1.62264) -- (3.65252,1.62264);
\draw [c] (3.65252,1.62264) -- (3.67056,1.62264);
\definecolor{c}{rgb}{0,0,0};
\colorlet{c}{natgreen};
\draw [c] (3.68861,1.3318) -- (3.68861,1.42091);
\draw [c] (3.68861,1.42091) -- (3.68861,1.51002);
\draw [c] (3.67056,1.42091) -- (3.68861,1.42091);
\draw [c] (3.68861,1.42091) -- (3.70665,1.42091);
\definecolor{c}{rgb}{0,0,0};
\colorlet{c}{natgreen};
\draw [c] (3.7247,1.36919) -- (3.7247,1.46681);
\draw [c] (3.7247,1.46681) -- (3.7247,1.56444);
\draw [c] (3.70665,1.46681) -- (3.7247,1.46681);
\draw [c] (3.7247,1.46681) -- (3.74274,1.46681);
\definecolor{c}{rgb}{0,0,0};
\colorlet{c}{natgreen};
\draw [c] (3.76079,1.42997) -- (3.76079,1.53404);
\draw [c] (3.76079,1.53404) -- (3.76079,1.63811);
\draw [c] (3.74274,1.53404) -- (3.76079,1.53404);
\draw [c] (3.76079,1.53404) -- (3.77883,1.53404);
\definecolor{c}{rgb}{0,0,0};
\colorlet{c}{natgreen};
\draw [c] (3.79688,1.27256) -- (3.79688,1.36142);
\draw [c] (3.79688,1.36142) -- (3.79688,1.45027);
\draw [c] (3.77883,1.36142) -- (3.79688,1.36142);
\draw [c] (3.79688,1.36142) -- (3.81492,1.36142);
\definecolor{c}{rgb}{0,0,0};
\colorlet{c}{natgreen};
\draw [c] (3.83296,1.21638) -- (3.83296,1.30191);
\draw [c] (3.83296,1.30191) -- (3.83296,1.38745);
\draw [c] (3.81492,1.30191) -- (3.83296,1.30191);
\draw [c] (3.83296,1.30191) -- (3.85101,1.30191);
\definecolor{c}{rgb}{0,0,0};
\colorlet{c}{natgreen};
\draw [c] (3.86905,1.12283) -- (3.86905,1.20232);
\draw [c] (3.86905,1.20232) -- (3.86905,1.28182);
\draw [c] (3.85101,1.20232) -- (3.86905,1.20232);
\draw [c] (3.86905,1.20232) -- (3.8871,1.20232);
\definecolor{c}{rgb}{0,0,0};
\colorlet{c}{natgreen};
\draw [c] (3.90514,1.13958) -- (3.90514,1.21936);
\draw [c] (3.90514,1.21936) -- (3.90514,1.29914);
\draw [c] (3.8871,1.21936) -- (3.90514,1.21936);
\draw [c] (3.90514,1.21936) -- (3.92319,1.21936);
\definecolor{c}{rgb}{0,0,0};
\colorlet{c}{natgreen};
\draw [c] (3.94123,1.05205) -- (3.94123,1.1216);
\draw [c] (3.94123,1.1216) -- (3.94123,1.19116);
\draw [c] (3.92319,1.1216) -- (3.94123,1.1216);
\draw [c] (3.94123,1.1216) -- (3.95927,1.1216);
\definecolor{c}{rgb}{0,0,0};
\colorlet{c}{natgreen};
\draw [c] (3.97732,1.07434) -- (3.97732,1.14602);
\draw [c] (3.97732,1.14602) -- (3.97732,1.2177);
\draw [c] (3.95927,1.14602) -- (3.97732,1.14602);
\draw [c] (3.97732,1.14602) -- (3.99536,1.14602);
\definecolor{c}{rgb}{0,0,0};
\colorlet{c}{natgreen};
\draw [c] (4.01341,1.1326) -- (4.01341,1.20974);
\draw [c] (4.01341,1.20974) -- (4.01341,1.28687);
\draw [c] (3.99536,1.20974) -- (4.01341,1.20974);
\draw [c] (4.01341,1.20974) -- (4.03145,1.20974);
\definecolor{c}{rgb}{0,0,0};
\colorlet{c}{natgreen};
\draw [c] (4.0495,0.939214) -- (4.0495,0.995054);
\draw [c] (4.0495,0.995054) -- (4.0495,1.05089);
\draw [c] (4.03145,0.995054) -- (4.0495,0.995054);
\draw [c] (4.0495,0.995054) -- (4.06754,0.995054);
\definecolor{c}{rgb}{0,0,0};
\colorlet{c}{natgreen};
\draw [c] (4.08558,1.20029) -- (4.08558,1.28407);
\draw [c] (4.08558,1.28407) -- (4.08558,1.36784);
\draw [c] (4.06754,1.28407) -- (4.08558,1.28407);
\draw [c] (4.08558,1.28407) -- (4.10363,1.28407);
\definecolor{c}{rgb}{0,0,0};
\colorlet{c}{natgreen};
\draw [c] (4.12167,1.0113) -- (4.12167,1.07931);
\draw [c] (4.12167,1.07931) -- (4.12167,1.14733);
\draw [c] (4.10363,1.07931) -- (4.12167,1.07931);
\draw [c] (4.12167,1.07931) -- (4.13972,1.07931);
\definecolor{c}{rgb}{0,0,0};
\colorlet{c}{natgreen};
\draw [c] (4.15776,0.981008) -- (4.15776,1.04309);
\draw [c] (4.15776,1.04309) -- (4.15776,1.10517);
\draw [c] (4.13972,1.04309) -- (4.15776,1.04309);
\draw [c] (4.15776,1.04309) -- (4.17581,1.04309);
\definecolor{c}{rgb}{0,0,0};
\colorlet{c}{natgreen};
\draw [c] (4.19385,0.886966) -- (4.19385,0.937799);
\draw [c] (4.19385,0.937799) -- (4.19385,0.988632);
\draw [c] (4.17581,0.937799) -- (4.19385,0.937799);
\draw [c] (4.19385,0.937799) -- (4.21189,0.937799);
\definecolor{c}{rgb}{0,0,0};
\colorlet{c}{natgreen};
\draw [c] (4.22994,0.996004) -- (4.22994,1.0585);
\draw [c] (4.22994,1.0585) -- (4.22994,1.121);
\draw [c] (4.21189,1.0585) -- (4.22994,1.0585);
\draw [c] (4.22994,1.0585) -- (4.24798,1.0585);
\definecolor{c}{rgb}{0,0,0};
\colorlet{c}{natgreen};
\draw [c] (4.26603,0.916623) -- (4.26603,0.96946);
\draw [c] (4.26603,0.96946) -- (4.26603,1.0223);
\draw [c] (4.24798,0.96946) -- (4.26603,0.96946);
\draw [c] (4.26603,0.96946) -- (4.28407,0.96946);
\definecolor{c}{rgb}{0,0,0};
\colorlet{c}{natgreen};
\draw [c] (4.30212,0.950391) -- (4.30212,1.01253);
\draw [c] (4.30212,1.01253) -- (4.30212,1.07467);
\draw [c] (4.28407,1.01253) -- (4.30212,1.01253);
\draw [c] (4.30212,1.01253) -- (4.32016,1.01253);
\definecolor{c}{rgb}{0,0,0};
\colorlet{c}{natgreen};
\draw [c] (4.33821,0.870271) -- (4.33821,0.915271);
\draw [c] (4.33821,0.915271) -- (4.33821,0.960271);
\draw [c] (4.32016,0.915271) -- (4.33821,0.915271);
\draw [c] (4.33821,0.915271) -- (4.35625,0.915271);
\definecolor{c}{rgb}{0,0,0};
\colorlet{c}{natgreen};
\draw [c] (4.37429,0.832558) -- (4.37429,0.87354);
\draw [c] (4.37429,0.87354) -- (4.37429,0.914522);
\draw [c] (4.35625,0.87354) -- (4.37429,0.87354);
\draw [c] (4.37429,0.87354) -- (4.39234,0.87354);
\definecolor{c}{rgb}{0,0,0};
\colorlet{c}{natgreen};
\draw [c] (4.41038,0.865532) -- (4.41038,0.913695);
\draw [c] (4.41038,0.913695) -- (4.41038,0.961857);
\draw [c] (4.39234,0.913695) -- (4.41038,0.913695);
\draw [c] (4.41038,0.913695) -- (4.42843,0.913695);
\definecolor{c}{rgb}{0,0,0};
\colorlet{c}{natgreen};
\draw [c] (4.44647,0.891549) -- (4.44647,0.93864);
\draw [c] (4.44647,0.93864) -- (4.44647,0.985731);
\draw [c] (4.42843,0.93864) -- (4.44647,0.93864);
\draw [c] (4.44647,0.93864) -- (4.46452,0.93864);
\definecolor{c}{rgb}{0,0,0};
\colorlet{c}{natgreen};
\draw [c] (4.48256,0.808217) -- (4.48256,0.84252);
\draw [c] (4.48256,0.84252) -- (4.48256,0.876822);
\draw [c] (4.46452,0.84252) -- (4.48256,0.84252);
\draw [c] (4.48256,0.84252) -- (4.5006,0.84252);
\definecolor{c}{rgb}{0,0,0};
\colorlet{c}{natgreen};
\draw [c] (4.51865,0.844653) -- (4.51865,0.886753);
\draw [c] (4.51865,0.886753) -- (4.51865,0.928852);
\draw [c] (4.5006,0.886753) -- (4.51865,0.886753);
\draw [c] (4.51865,0.886753) -- (4.53669,0.886753);
\definecolor{c}{rgb}{0,0,0};
\colorlet{c}{natgreen};
\draw [c] (4.55474,0.817059) -- (4.55474,0.854363);
\draw [c] (4.55474,0.854363) -- (4.55474,0.891668);
\draw [c] (4.53669,0.854363) -- (4.55474,0.854363);
\draw [c] (4.55474,0.854363) -- (4.57278,0.854363);
\definecolor{c}{rgb}{0,0,0};
\colorlet{c}{natgreen};
\draw [c] (4.59083,0.798305) -- (4.59083,0.832897);
\draw [c] (4.59083,0.832897) -- (4.59083,0.86749);
\draw [c] (4.57278,0.832897) -- (4.59083,0.832897);
\draw [c] (4.59083,0.832897) -- (4.60887,0.832897);
\definecolor{c}{rgb}{0,0,0};
\colorlet{c}{natgreen};
\draw [c] (4.62692,0.850203) -- (4.62692,0.892336);
\draw [c] (4.62692,0.892336) -- (4.62692,0.934469);
\draw [c] (4.60887,0.892336) -- (4.62692,0.892336);
\draw [c] (4.62692,0.892336) -- (4.64496,0.892336);
\definecolor{c}{rgb}{0,0,0};
\colorlet{c}{natgreen};
\draw [c] (4.663,0.837422) -- (4.663,0.880898);
\draw [c] (4.663,0.880898) -- (4.663,0.924374);
\draw [c] (4.64496,0.880898) -- (4.663,0.880898);
\draw [c] (4.663,0.880898) -- (4.68105,0.880898);
\definecolor{c}{rgb}{0,0,0};
\colorlet{c}{natgreen};
\draw [c] (4.69909,0.79697) -- (4.69909,0.828716);
\draw [c] (4.69909,0.828716) -- (4.69909,0.860462);
\draw [c] (4.68105,0.828716) -- (4.69909,0.828716);
\draw [c] (4.69909,0.828716) -- (4.71714,0.828716);
\definecolor{c}{rgb}{0,0,0};
\colorlet{c}{natgreen};
\draw [c] (4.73518,0.840524) -- (4.73518,0.884851);
\draw [c] (4.73518,0.884851) -- (4.73518,0.929178);
\draw [c] (4.71714,0.884851) -- (4.73518,0.884851);
\draw [c] (4.73518,0.884851) -- (4.75323,0.884851);
\definecolor{c}{rgb}{0,0,0};
\colorlet{c}{natgreen};
\draw [c] (4.77127,0.780186) -- (4.77127,0.813231);
\draw [c] (4.77127,0.813231) -- (4.77127,0.846275);
\draw [c] (4.75323,0.813231) -- (4.77127,0.813231);
\draw [c] (4.77127,0.813231) -- (4.78931,0.813231);
\definecolor{c}{rgb}{0,0,0};
\colorlet{c}{natgreen};
\draw [c] (4.80736,0.813031) -- (4.80736,0.853508);
\draw [c] (4.80736,0.853508) -- (4.80736,0.893985);
\draw [c] (4.78931,0.853508) -- (4.80736,0.853508);
\draw [c] (4.80736,0.853508) -- (4.8254,0.853508);
\definecolor{c}{rgb}{0,0,0};
\colorlet{c}{natgreen};
\draw [c] (4.84345,0.793976) -- (4.84345,0.828606);
\draw [c] (4.84345,0.828606) -- (4.84345,0.863236);
\draw [c] (4.8254,0.828606) -- (4.84345,0.828606);
\draw [c] (4.84345,0.828606) -- (4.86149,0.828606);
\definecolor{c}{rgb}{0,0,0};
\colorlet{c}{natgreen};
\draw [c] (4.87954,0.758384) -- (4.87954,0.784644);
\draw [c] (4.87954,0.784644) -- (4.87954,0.810903);
\draw [c] (4.86149,0.784644) -- (4.87954,0.784644);
\draw [c] (4.87954,0.784644) -- (4.89758,0.784644);
\definecolor{c}{rgb}{0,0,0};
\colorlet{c}{natgreen};
\draw [c] (4.91563,0.758837) -- (4.91563,0.787174);
\draw [c] (4.91563,0.787174) -- (4.91563,0.815512);
\draw [c] (4.89758,0.787174) -- (4.91563,0.787174);
\draw [c] (4.91563,0.787174) -- (4.93367,0.787174);
\definecolor{c}{rgb}{0,0,0};
\colorlet{c}{natgreen};
\draw [c] (4.95171,0.761736) -- (4.95171,0.785406);
\draw [c] (4.95171,0.785406) -- (4.95171,0.809077);
\draw [c] (4.93367,0.785406) -- (4.95171,0.785406);
\draw [c] (4.95171,0.785406) -- (4.96976,0.785406);
\definecolor{c}{rgb}{0,0,0};
\colorlet{c}{natgreen};
\draw [c] (4.9878,0.800978) -- (4.9878,0.832609);
\draw [c] (4.9878,0.832609) -- (4.9878,0.86424);
\draw [c] (4.96976,0.832609) -- (4.9878,0.832609);
\draw [c] (4.9878,0.832609) -- (5.00585,0.832609);
\definecolor{c}{rgb}{0,0,0};
\colorlet{c}{natgreen};
\draw [c] (5.02389,0.775133) -- (5.02389,0.80439);
\draw [c] (5.02389,0.80439) -- (5.02389,0.833647);
\draw [c] (5.00585,0.80439) -- (5.02389,0.80439);
\draw [c] (5.02389,0.80439) -- (5.04194,0.80439);
\definecolor{c}{rgb}{0,0,0};
\colorlet{c}{natgreen};
\draw [c] (5.05998,0.742255) -- (5.05998,0.760704);
\draw [c] (5.05998,0.760704) -- (5.05998,0.779153);
\draw [c] (5.04194,0.760704) -- (5.05998,0.760704);
\draw [c] (5.05998,0.760704) -- (5.07802,0.760704);
\definecolor{c}{rgb}{0,0,0};
\colorlet{c}{natgreen};
\draw [c] (5.09607,0.751018) -- (5.09607,0.773695);
\draw [c] (5.09607,0.773695) -- (5.09607,0.796372);
\draw [c] (5.07802,0.773695) -- (5.09607,0.773695);
\draw [c] (5.09607,0.773695) -- (5.11411,0.773695);
\definecolor{c}{rgb}{0,0,0};
\colorlet{c}{natgreen};
\draw [c] (5.13216,0.812429) -- (5.13216,0.851758);
\draw [c] (5.13216,0.851758) -- (5.13216,0.891088);
\draw [c] (5.11411,0.851758) -- (5.13216,0.851758);
\draw [c] (5.13216,0.851758) -- (5.1502,0.851758);
\definecolor{c}{rgb}{0,0,0};
\colorlet{c}{natgreen};
\draw [c] (5.16825,0.747876) -- (5.16825,0.76617);
\draw [c] (5.16825,0.76617) -- (5.16825,0.784464);
\draw [c] (5.1502,0.76617) -- (5.16825,0.76617);
\draw [c] (5.16825,0.76617) -- (5.18629,0.76617);
\definecolor{c}{rgb}{0,0,0};
\colorlet{c}{natgreen};
\draw [c] (5.20433,0.757096) -- (5.20433,0.780973);
\draw [c] (5.20433,0.780973) -- (5.20433,0.804849);
\draw [c] (5.18629,0.780973) -- (5.20433,0.780973);
\draw [c] (5.20433,0.780973) -- (5.22238,0.780973);
\definecolor{c}{rgb}{0,0,0};
\colorlet{c}{natgreen};
\draw [c] (5.24042,0.780723) -- (5.24042,0.812674);
\draw [c] (5.24042,0.812674) -- (5.24042,0.844625);
\draw [c] (5.22238,0.812674) -- (5.24042,0.812674);
\draw [c] (5.24042,0.812674) -- (5.25847,0.812674);
\definecolor{c}{rgb}{0,0,0};
\colorlet{c}{natgreen};
\draw [c] (5.27651,0.746262) -- (5.27651,0.76547);
\draw [c] (5.27651,0.76547) -- (5.27651,0.784677);
\draw [c] (5.25847,0.76547) -- (5.27651,0.76547);
\draw [c] (5.27651,0.76547) -- (5.29456,0.76547);
\definecolor{c}{rgb}{0,0,0};
\colorlet{c}{natgreen};
\draw [c] (5.3126,0.77607) -- (5.3126,0.803785);
\draw [c] (5.3126,0.803785) -- (5.3126,0.8315);
\draw [c] (5.29456,0.803785) -- (5.3126,0.803785);
\draw [c] (5.3126,0.803785) -- (5.33065,0.803785);
\definecolor{c}{rgb}{0,0,0};
\colorlet{c}{natgreen};
\draw [c] (5.34869,0.75608) -- (5.34869,0.778446);
\draw [c] (5.34869,0.778446) -- (5.34869,0.800813);
\draw [c] (5.33065,0.778446) -- (5.34869,0.778446);
\draw [c] (5.34869,0.778446) -- (5.36673,0.778446);
\definecolor{c}{rgb}{0,0,0};
\colorlet{c}{natgreen};
\draw [c] (5.38478,0.752922) -- (5.38478,0.773923);
\draw [c] (5.38478,0.773923) -- (5.38478,0.794924);
\draw [c] (5.36673,0.773923) -- (5.38478,0.773923);
\draw [c] (5.38478,0.773923) -- (5.40282,0.773923);
\definecolor{c}{rgb}{0,0,0};
\colorlet{c}{natgreen};
\draw [c] (5.42087,0.742951) -- (5.42087,0.760468);
\draw [c] (5.42087,0.760468) -- (5.42087,0.777986);
\draw [c] (5.40282,0.760468) -- (5.42087,0.760468);
\draw [c] (5.42087,0.760468) -- (5.43891,0.760468);
\definecolor{c}{rgb}{0,0,0};
\colorlet{c}{natgreen};
\draw [c] (5.45696,0.742899) -- (5.45696,0.762868);
\draw [c] (5.45696,0.762868) -- (5.45696,0.782838);
\draw [c] (5.43891,0.762868) -- (5.45696,0.762868);
\draw [c] (5.45696,0.762868) -- (5.475,0.762868);
\definecolor{c}{rgb}{0,0,0};
\colorlet{c}{natgreen};
\draw [c] (5.49304,0.750905) -- (5.49304,0.771726);
\draw [c] (5.49304,0.771726) -- (5.49304,0.792547);
\draw [c] (5.475,0.771726) -- (5.49304,0.771726);
\draw [c] (5.49304,0.771726) -- (5.51109,0.771726);
\definecolor{c}{rgb}{0,0,0};
\colorlet{c}{natgreen};
\draw [c] (5.52913,0.735553) -- (5.52913,0.742652);
\draw [c] (5.52913,0.742652) -- (5.52913,0.749751);
\draw [c] (5.51109,0.742652) -- (5.52913,0.742652);
\draw [c] (5.52913,0.742652) -- (5.54718,0.742652);
\definecolor{c}{rgb}{0,0,0};
\colorlet{c}{natgreen};
\draw [c] (5.56522,0.752318) -- (5.56522,0.776656);
\draw [c] (5.56522,0.776656) -- (5.56522,0.800993);
\draw [c] (5.54718,0.776656) -- (5.56522,0.776656);
\draw [c] (5.56522,0.776656) -- (5.58327,0.776656);
\definecolor{c}{rgb}{0,0,0};
\colorlet{c}{natgreen};
\draw [c] (5.60131,0.752888) -- (5.60131,0.779509);
\draw [c] (5.60131,0.779509) -- (5.60131,0.80613);
\draw [c] (5.58327,0.779509) -- (5.60131,0.779509);
\draw [c] (5.60131,0.779509) -- (5.61935,0.779509);
\definecolor{c}{rgb}{0,0,0};
\colorlet{c}{natgreen};
\draw [c] (5.6374,0.754683) -- (5.6374,0.778493);
\draw [c] (5.6374,0.778493) -- (5.6374,0.802304);
\draw [c] (5.61935,0.778493) -- (5.6374,0.778493);
\draw [c] (5.6374,0.778493) -- (5.65544,0.778493);
\definecolor{c}{rgb}{0,0,0};
\colorlet{c}{natgreen};
\draw [c] (5.67349,0.737725) -- (5.67349,0.748389);
\draw [c] (5.67349,0.748389) -- (5.67349,0.759052);
\draw [c] (5.65544,0.748389) -- (5.67349,0.748389);
\draw [c] (5.67349,0.748389) -- (5.69153,0.748389);
\definecolor{c}{rgb}{0,0,0};
\colorlet{c}{natgreen};
\draw [c] (5.70958,0.734632) -- (5.70958,0.741681);
\draw [c] (5.70958,0.741681) -- (5.70958,0.74873);
\draw [c] (5.69153,0.741681) -- (5.70958,0.741681);
\draw [c] (5.70958,0.741681) -- (5.72762,0.741681);
\definecolor{c}{rgb}{0,0,0};
\colorlet{c}{natgreen};
\draw [c] (5.74567,0.744543) -- (5.74567,0.763576);
\draw [c] (5.74567,0.763576) -- (5.74567,0.782609);
\draw [c] (5.72762,0.763576) -- (5.74567,0.763576);
\draw [c] (5.74567,0.763576) -- (5.76371,0.763576);
\definecolor{c}{rgb}{0,0,0};
\colorlet{c}{natgreen};
\draw [c] (5.81784,0.740037) -- (5.81784,0.755969);
\draw [c] (5.81784,0.755969) -- (5.81784,0.771901);
\draw [c] (5.7998,0.755969) -- (5.81784,0.755969);
\draw [c] (5.81784,0.755969) -- (5.83589,0.755969);
\definecolor{c}{rgb}{0,0,0};
\colorlet{c}{natgreen};
\draw [c] (5.89002,0.738984) -- (5.89002,0.753556);
\draw [c] (5.89002,0.753556) -- (5.89002,0.768129);
\draw [c] (5.87198,0.753556) -- (5.89002,0.753556);
\draw [c] (5.89002,0.753556) -- (5.90806,0.753556);
\definecolor{c}{rgb}{0,0,0};
\colorlet{c}{natgreen};
\draw [c] (5.99829,0.734632) -- (5.99829,0.7446);
\draw [c] (5.99829,0.7446) -- (5.99829,0.754568);
\draw [c] (5.98024,0.7446) -- (5.99829,0.7446);
\draw [c] (5.99829,0.7446) -- (6.01633,0.7446);
\definecolor{c}{rgb}{0,0,0};
\colorlet{c}{natgreen};
\draw [c] (6.03438,0.752867) -- (6.03438,0.777778);
\draw [c] (6.03438,0.777778) -- (6.03438,0.802688);
\draw [c] (6.01633,0.777778) -- (6.03438,0.777778);
\draw [c] (6.03438,0.777778) -- (6.05242,0.777778);
\definecolor{c}{rgb}{0,0,0};
\colorlet{c}{natgreen};
\draw [c] (6.14264,0.734633) -- (6.14264,0.734774);
\draw [c] (6.14264,0.734774) -- (6.14264,0.734916);
\draw [c] (6.1246,0.734774) -- (6.14264,0.734774);
\draw [c] (6.14264,0.734774) -- (6.16069,0.734774);
\definecolor{c}{rgb}{0,0,0};
\colorlet{c}{natgreen};
\draw [c] (6.17873,0.734632) -- (6.17873,0.74983);
\draw [c] (6.17873,0.74983) -- (6.17873,0.765027);
\draw [c] (6.16069,0.74983) -- (6.17873,0.74983);
\draw [c] (6.17873,0.74983) -- (6.19677,0.74983);
\definecolor{c}{rgb}{0,0,0};
\colorlet{c}{natgreen};
\draw [c] (6.21482,0.737785) -- (6.21482,0.745395);
\draw [c] (6.21482,0.745395) -- (6.21482,0.753006);
\draw [c] (6.19677,0.745395) -- (6.21482,0.745395);
\draw [c] (6.21482,0.745395) -- (6.23286,0.745395);
\definecolor{c}{rgb}{0,0,0};
\colorlet{c}{natgreen};
\draw [c] (6.25091,0.73666) -- (6.25091,0.750033);
\draw [c] (6.25091,0.750033) -- (6.25091,0.763407);
\draw [c] (6.23286,0.750033) -- (6.25091,0.750033);
\draw [c] (6.25091,0.750033) -- (6.26895,0.750033);
\definecolor{c}{rgb}{0,0,0};
\colorlet{c}{natgreen};
\draw [c] (6.287,0.734632) -- (6.287,0.747821);
\draw [c] (6.287,0.747821) -- (6.287,0.76101);
\draw [c] (6.26895,0.747821) -- (6.287,0.747821);
\draw [c] (6.287,0.747821) -- (6.30504,0.747821);
\definecolor{c}{rgb}{0,0,0};
\colorlet{c}{natgreen};
\draw [c] (6.32308,0.739441) -- (6.32308,0.751649);
\draw [c] (6.32308,0.751649) -- (6.32308,0.763857);
\draw [c] (6.30504,0.751649) -- (6.32308,0.751649);
\draw [c] (6.32308,0.751649) -- (6.34113,0.751649);
\definecolor{c}{rgb}{0,0,0};
\colorlet{c}{natgreen};
\draw [c] (6.39526,0.734632) -- (6.39526,0.740014);
\draw [c] (6.39526,0.740014) -- (6.39526,0.745395);
\draw [c] (6.37722,0.740014) -- (6.39526,0.740014);
\draw [c] (6.39526,0.740014) -- (6.41331,0.740014);
\definecolor{c}{rgb}{0,0,0};
\colorlet{c}{natgreen};
\draw [c] (6.43135,0.734632) -- (6.43135,0.741681);
\draw [c] (6.43135,0.741681) -- (6.43135,0.74873);
\draw [c] (6.41331,0.741681) -- (6.43135,0.741681);
\draw [c] (6.43135,0.741681) -- (6.4494,0.741681);
\definecolor{c}{rgb}{0,0,0};
\colorlet{c}{natgreen};
\draw [c] (6.53962,0.740056) -- (6.53962,0.756158);
\draw [c] (6.53962,0.756158) -- (6.53962,0.77226);
\draw [c] (6.52157,0.756158) -- (6.53962,0.756158);
\draw [c] (6.53962,0.756158) -- (6.55766,0.756158);
\definecolor{c}{rgb}{0,0,0};
\colorlet{c}{natgreen};
\draw [c] (6.57571,0.746809) -- (6.57571,0.767173);
\draw [c] (6.57571,0.767173) -- (6.57571,0.787537);
\draw [c] (6.55766,0.767173) -- (6.57571,0.767173);
\draw [c] (6.57571,0.767173) -- (6.59375,0.767173);
\definecolor{c}{rgb}{0,0,0};
\colorlet{c}{natgreen};
\draw [c] (6.72006,0.74203) -- (6.72006,0.762645);
\draw [c] (6.72006,0.762645) -- (6.72006,0.78326);
\draw [c] (6.70202,0.762645) -- (6.72006,0.762645);
\draw [c] (6.72006,0.762645) -- (6.7381,0.762645);
\definecolor{c}{rgb}{0,0,0};
\colorlet{c}{natgreen};
\draw [c] (6.75615,0.74324) -- (6.75615,0.764297);
\draw [c] (6.75615,0.764297) -- (6.75615,0.785353);
\draw [c] (6.7381,0.764297) -- (6.75615,0.764297);
\draw [c] (6.75615,0.764297) -- (6.77419,0.764297);
\definecolor{c}{rgb}{0,0,0};
\colorlet{c}{natgreen};
\draw [c] (7.00877,0.734632) -- (7.00877,0.736845);
\draw [c] (7.00877,0.736845) -- (7.00877,0.739057);
\draw [c] (6.99073,0.736845) -- (7.00877,0.736845);
\draw [c] (7.00877,0.736845) -- (7.02681,0.736845);
\definecolor{c}{rgb}{0,0,0};
\colorlet{c}{natgreen};
\draw [c] (7.08095,0.734632) -- (7.08095,0.747821);
\draw [c] (7.08095,0.747821) -- (7.08095,0.76101);
\draw [c] (7.0629,0.747821) -- (7.08095,0.747821);
\draw [c] (7.08095,0.747821) -- (7.09899,0.747821);
\definecolor{c}{rgb}{0,0,0};
\colorlet{c}{natgreen};
\draw [c] (7.18921,0.734632) -- (7.18921,0.746887);
\draw [c] (7.18921,0.746887) -- (7.18921,0.759142);
\draw [c] (7.17117,0.746887) -- (7.18921,0.746887);
\draw [c] (7.18921,0.746887) -- (7.20726,0.746887);
\definecolor{c}{rgb}{0,0,0};
\colorlet{c}{natgreen};
\draw [c] (7.33357,0.734632) -- (7.33357,0.749109);
\draw [c] (7.33357,0.749109) -- (7.33357,0.763586);
\draw [c] (7.31552,0.749109) -- (7.33357,0.749109);
\draw [c] (7.33357,0.749109) -- (7.35161,0.749109);
\definecolor{c}{rgb}{0,0,0};
\colorlet{c}{natgreen};
\draw [c] (7.44183,0.734632) -- (7.44183,0.735461);
\draw [c] (7.44183,0.735461) -- (7.44183,0.736291);
\draw [c] (7.42379,0.735461) -- (7.44183,0.735461);
\draw [c] (7.44183,0.735461) -- (7.45988,0.735461);
\definecolor{c}{rgb}{0,0,0};
\colorlet{c}{natgreen};
\draw [c] (7.73054,0.734632) -- (7.73054,0.741681);
\draw [c] (7.73054,0.741681) -- (7.73054,0.74873);
\draw [c] (7.7125,0.741681) -- (7.73054,0.741681);
\draw [c] (7.73054,0.741681) -- (7.74859,0.741681);
\definecolor{c}{rgb}{0,0,0};
\colorlet{c}{natgreen};
\draw [c] (7.83881,0.734632) -- (7.83881,0.743059);
\draw [c] (7.83881,0.743059) -- (7.83881,0.751485);
\draw [c] (7.82077,0.743059) -- (7.83881,0.743059);
\draw [c] (7.83881,0.743059) -- (7.85685,0.743059);
\definecolor{c}{rgb}{0,0,0};
\colorlet{c}{natgreen};
\draw [c] (7.91099,0.734632) -- (7.91099,0.739076);
\draw [c] (7.91099,0.739076) -- (7.91099,0.74352);
\draw [c] (7.89294,0.739076) -- (7.91099,0.739076);
\draw [c] (7.91099,0.739076) -- (7.92903,0.739076);
\definecolor{c}{rgb}{0,0,0};
\colorlet{c}{natgreen};
\draw [c] (7.94708,0.734632) -- (7.94708,0.735461);
\draw [c] (7.94708,0.735461) -- (7.94708,0.736291);
\draw [c] (7.92903,0.735461) -- (7.94708,0.735461);
\draw [c] (7.94708,0.735461) -- (7.96512,0.735461);
\definecolor{c}{rgb}{0,0,0};
\colorlet{c}{natgreen};
\draw [c] (8.01925,0.734632) -- (8.01925,0.743059);
\draw [c] (8.01925,0.743059) -- (8.01925,0.751485);
\draw [c] (8.00121,0.743059) -- (8.01925,0.743059);
\draw [c] (8.01925,0.743059) -- (8.0373,0.743059);
\definecolor{c}{rgb}{0,0,0};
\colorlet{c}{natgreen};
\draw [c] (8.05534,0.734632) -- (8.05534,0.743059);
\draw [c] (8.05534,0.743059) -- (8.05534,0.751485);
\draw [c] (8.0373,0.743059) -- (8.05534,0.743059);
\draw [c] (8.05534,0.743059) -- (8.07339,0.743059);
\definecolor{c}{rgb}{0,0,0};
\colorlet{c}{natgreen};
\draw [c] (8.09143,0.734632) -- (8.09143,0.745781);
\draw [c] (8.09143,0.745781) -- (8.09143,0.756929);
\draw [c] (8.07339,0.745781) -- (8.09143,0.745781);
\draw [c] (8.09143,0.745781) -- (8.10948,0.745781);
\definecolor{c}{rgb}{0,0,0};
\colorlet{c}{natgreen};
\draw [c] (8.12752,0.734632) -- (8.12752,0.740014);
\draw [c] (8.12752,0.740014) -- (8.12752,0.745395);
\draw [c] (8.10948,0.740014) -- (8.12752,0.740014);
\draw [c] (8.12752,0.740014) -- (8.14556,0.740014);
\definecolor{c}{rgb}{0,0,0};
\colorlet{c}{natgreen};
\draw [c] (8.23579,0.734632) -- (8.23579,0.74892);
\draw [c] (8.23579,0.74892) -- (8.23579,0.763207);
\draw [c] (8.21774,0.74892) -- (8.23579,0.74892);
\draw [c] (8.23579,0.74892) -- (8.25383,0.74892);
\definecolor{c}{rgb}{0,0,0};
\colorlet{c}{natgreen};
\draw [c] (8.88538,0.734632) -- (8.88538,0.750764);
\draw [c] (8.88538,0.750764) -- (8.88538,0.766895);
\draw [c] (8.86734,0.750764) -- (8.88538,0.750764);
\draw [c] (8.88538,0.750764) -- (8.90343,0.750764);
\definecolor{c}{rgb}{0,0,0};
\colorlet{c}{natgreen};
\draw [c] (8.95756,0.734632) -- (8.95756,0.748166);
\draw [c] (8.95756,0.748166) -- (8.95756,0.761699);
\draw [c] (8.93952,0.748166) -- (8.95756,0.748166);
\draw [c] (8.95756,0.748166) -- (8.97561,0.748166);
\definecolor{c}{rgb}{0,0,0};
\colorlet{c}{natgreen};
\draw [c] (9.10192,0.734632) -- (9.10192,0.7446);
\draw [c] (9.10192,0.7446) -- (9.10192,0.754568);
\draw [c] (9.08387,0.7446) -- (9.10192,0.7446);
\draw [c] (9.10192,0.7446) -- (9.11996,0.7446);
\definecolor{c}{rgb}{0,0,0};
\colorlet{c}{natgreen};
\draw [c] (9.17409,0.734632) -- (9.17409,0.743059);
\draw [c] (9.17409,0.743059) -- (9.17409,0.751485);
\draw [c] (9.15605,0.743059) -- (9.17409,0.743059);
\draw [c] (9.17409,0.743059) -- (9.19214,0.743059);
\definecolor{c}{rgb}{0,0,0};
\colorlet{c}{natgreen};
\draw [c] (9.53498,0.734632) -- (9.53498,0.746887);
\draw [c] (9.53498,0.746887) -- (9.53498,0.759142);
\draw [c] (9.51694,0.746887) -- (9.53498,0.746887);
\draw [c] (9.53498,0.746887) -- (9.55302,0.746887);
\definecolor{c}{rgb}{0,0,0};
\draw [c] (1,0.680516) -- (9.95,0.680516);
\draw [anchor= east] (9.95,-0.0816619) node[color=c, rotate=0]{$E_{T}^{\text{iso}}\text{ [GeV]}$};
\draw [c] (1,0.863234) -- (1,0.680516);
\draw [c] (1.36089,0.771875) -- (1.36089,0.680516);
\draw [c] (1.72177,0.771875) -- (1.72177,0.680516);
\draw [c] (2.08266,0.771875) -- (2.08266,0.680516);
\draw [c] (2.44355,0.771875) -- (2.44355,0.680516);
\draw [c] (2.80444,0.863234) -- (2.80444,0.680516);
\draw [c] (3.16532,0.771875) -- (3.16532,0.680516);
\draw [c] (3.52621,0.771875) -- (3.52621,0.680516);
\draw [c] (3.8871,0.771875) -- (3.8871,0.680516);
\draw [c] (4.24798,0.771875) -- (4.24798,0.680516);
\draw [c] (4.60887,0.863234) -- (4.60887,0.680516);
\draw [c] (4.96976,0.771875) -- (4.96976,0.680516);
\draw [c] (5.33065,0.771875) -- (5.33065,0.680516);
\draw [c] (5.69153,0.771875) -- (5.69153,0.680516);
\draw [c] (6.05242,0.771875) -- (6.05242,0.680516);
\draw [c] (6.41331,0.863234) -- (6.41331,0.680516);
\draw [c] (6.77419,0.771875) -- (6.77419,0.680516);
\draw [c] (7.13508,0.771875) -- (7.13508,0.680516);
\draw [c] (7.49597,0.771875) -- (7.49597,0.680516);
\draw [c] (7.85685,0.771875) -- (7.85685,0.680516);
\draw [c] (8.21774,0.863234) -- (8.21774,0.680516);
\draw [c] (8.21774,0.863234) -- (8.21774,0.680516);
\draw [c] (8.57863,0.771875) -- (8.57863,0.680516);
\draw [c] (8.93952,0.771875) -- (8.93952,0.680516);
\draw [c] (9.3004,0.771875) -- (9.3004,0.680516);
\draw [c] (9.66129,0.771875) -- (9.66129,0.680516);
\draw [anchor=base] (1,0.285817) node[color=c, rotate=0]{-5};
\draw [anchor=base] (2.80444,0.285817) node[color=c, rotate=0]{0};
\draw [anchor=base] (4.60887,0.285817) node[color=c, rotate=0]{5};
\draw [anchor=base] (6.41331,0.285817) node[color=c, rotate=0]{10};
\draw [anchor=base] (8.21774,0.285817) node[color=c, rotate=0]{15};
\draw [c] (1,0.680516) -- (1,6.73711);
\draw [anchor= east] (-0.4,6.73711) node[color=c, rotate=90]{Normalised number of events};
\draw [c] (1.267,0.734632) -- (1,0.734632);
\draw [c] (1.1335,0.864132) -- (1,0.864132);
\draw [c] (1.1335,0.993631) -- (1,0.993631);
\draw [c] (1.1335,1.12313) -- (1,1.12313);
\draw [c] (1.1335,1.25263) -- (1,1.25263);
\draw [c] (1.267,1.38213) -- (1,1.38213);
\draw [c] (1.1335,1.51163) -- (1,1.51163);
\draw [c] (1.1335,1.64113) -- (1,1.64113);
\draw [c] (1.1335,1.77063) -- (1,1.77063);
\draw [c] (1.1335,1.90013) -- (1,1.90013);
\draw [c] (1.267,2.02963) -- (1,2.02963);
\draw [c] (1.1335,2.15913) -- (1,2.15913);
\draw [c] (1.1335,2.28863) -- (1,2.28863);
\draw [c] (1.1335,2.41812) -- (1,2.41812);
\draw [c] (1.1335,2.54762) -- (1,2.54762);
\draw [c] (1.267,2.67712) -- (1,2.67712);
\draw [c] (1.1335,2.80662) -- (1,2.80662);
\draw [c] (1.1335,2.93612) -- (1,2.93612);
\draw [c] (1.1335,3.06562) -- (1,3.06562);
\draw [c] (1.1335,3.19512) -- (1,3.19512);
\draw [c] (1.267,3.32462) -- (1,3.32462);
\draw [c] (1.1335,3.45412) -- (1,3.45412);
\draw [c] (1.1335,3.58362) -- (1,3.58362);
\draw [c] (1.1335,3.71312) -- (1,3.71312);
\draw [c] (1.1335,3.84262) -- (1,3.84262);
\draw [c] (1.267,3.97212) -- (1,3.97212);
\draw [c] (1.1335,4.10162) -- (1,4.10162);
\draw [c] (1.1335,4.23112) -- (1,4.23112);
\draw [c] (1.1335,4.36062) -- (1,4.36062);
\draw [c] (1.1335,4.49012) -- (1,4.49012);
\draw [c] (1.267,4.61962) -- (1,4.61962);
\draw [c] (1.1335,4.74911) -- (1,4.74911);
\draw [c] (1.1335,4.87861) -- (1,4.87861);
\draw [c] (1.1335,5.00811) -- (1,5.00811);
\draw [c] (1.1335,5.13761) -- (1,5.13761);
\draw [c] (1.267,5.26711) -- (1,5.26711);
\draw [c] (1.1335,5.39661) -- (1,5.39661);
\draw [c] (1.1335,5.52611) -- (1,5.52611);
\draw [c] (1.1335,5.65561) -- (1,5.65561);
\draw [c] (1.1335,5.78511) -- (1,5.78511);
\draw [c] (1.267,5.91461) -- (1,5.91461);
\draw [c] (1.1335,6.04411) -- (1,6.04411);
\draw [c] (1.1335,6.17361) -- (1,6.17361);
\draw [c] (1.1335,6.30311) -- (1,6.30311);
\draw [c] (1.1335,6.43261) -- (1,6.43261);
\draw [c] (1.267,6.56211) -- (1,6.56211);
\draw [c] (1.267,0.734632) -- (1,0.734632);
\draw [c] (1.267,6.56211) -- (1,6.56211);
\draw [c] (1.1335,6.69161) -- (1,6.69161);
\draw [anchor= east] (0.95,0.734632) node[color=c, rotate=0]{0};
\draw [anchor= east] (0.95,1.38213) node[color=c, rotate=0]{100};
\draw [anchor= east] (0.95,2.02963) node[color=c, rotate=0]{200};
\draw [anchor= east] (0.95,2.67712) node[color=c, rotate=0]{300};
\draw [anchor= east] (0.95,3.32462) node[color=c, rotate=0]{400};
\draw [anchor= east] (0.95,3.97212) node[color=c, rotate=0]{500};
\draw [anchor= east] (0.95,4.61962) node[color=c, rotate=0]{600};
\draw [anchor= east] (0.95,5.26711) node[color=c, rotate=0]{700};
\draw [anchor= east] (0.95,5.91461) node[color=c, rotate=0]{800};
\draw [anchor= east] (0.95,6.56211) node[color=c, rotate=0]{900};
\colorlet{c}{natcomp!70};
\draw [c] (1.01804,0.734643) -- (1.01804,0.756364);
\draw [c] (1.01804,0.756364) -- (1.01804,0.778085);
\draw [c] (1,0.756364) -- (1.01804,0.756364);
\draw [c] (1.01804,0.756364) -- (1.03609,0.756364);
\definecolor{c}{rgb}{0,0,0};
\colorlet{c}{natcomp!70};
\draw [c] (1.05413,0.734632) -- (1.05413,0.734644);
\draw [c] (1.05413,0.734644) -- (1.05413,0.734656);
\draw [c] (1.03609,0.734644) -- (1.05413,0.734644);
\draw [c] (1.05413,0.734644) -- (1.07218,0.734644);
\definecolor{c}{rgb}{0,0,0};
\colorlet{c}{natcomp!70};
\draw [c] (1.19849,0.734643) -- (1.19849,0.764169);
\draw [c] (1.19849,0.764169) -- (1.19849,0.793695);
\draw [c] (1.18044,0.764169) -- (1.19849,0.764169);
\draw [c] (1.19849,0.764169) -- (1.21653,0.764169);
\definecolor{c}{rgb}{0,0,0};
\colorlet{c}{natcomp!70};
\draw [c] (1.23458,0.734636) -- (1.23458,0.734645);
\draw [c] (1.23458,0.734645) -- (1.23458,0.734654);
\draw [c] (1.21653,0.734645) -- (1.23458,0.734645);
\draw [c] (1.23458,0.734645) -- (1.25262,0.734645);
\definecolor{c}{rgb}{0,0,0};
\colorlet{c}{natcomp!70};
\draw [c] (1.27067,0.73464) -- (1.27067,0.756361);
\draw [c] (1.27067,0.756361) -- (1.27067,0.778083);
\draw [c] (1.25262,0.756361) -- (1.27067,0.756361);
\draw [c] (1.27067,0.756361) -- (1.28871,0.756361);
\definecolor{c}{rgb}{0,0,0};
\colorlet{c}{natcomp!70};
\draw [c] (1.30675,0.74249) -- (1.30675,0.769762);
\draw [c] (1.30675,0.769762) -- (1.30675,0.797034);
\draw [c] (1.28871,0.769762) -- (1.30675,0.769762);
\draw [c] (1.30675,0.769762) -- (1.3248,0.769762);
\definecolor{c}{rgb}{0,0,0};
\colorlet{c}{natcomp!70};
\draw [c] (1.34284,0.734678) -- (1.34284,0.756399);
\draw [c] (1.34284,0.756399) -- (1.34284,0.77812);
\draw [c] (1.3248,0.756399) -- (1.34284,0.756399);
\draw [c] (1.34284,0.756399) -- (1.36089,0.756399);
\definecolor{c}{rgb}{0,0,0};
\colorlet{c}{natcomp!70};
\draw [c] (1.37893,0.74505) -- (1.37893,0.777385);
\draw [c] (1.37893,0.777385) -- (1.37893,0.80972);
\draw [c] (1.36089,0.777385) -- (1.37893,0.777385);
\draw [c] (1.37893,0.777385) -- (1.39698,0.777385);
\definecolor{c}{rgb}{0,0,0};
\colorlet{c}{natcomp!70};
\draw [c] (1.41502,0.751247) -- (1.41502,0.783932);
\draw [c] (1.41502,0.783932) -- (1.41502,0.816618);
\draw [c] (1.39698,0.783932) -- (1.41502,0.783932);
\draw [c] (1.41502,0.783932) -- (1.43306,0.783932);
\definecolor{c}{rgb}{0,0,0};
\colorlet{c}{natcomp!70};
\draw [c] (1.45111,0.774293) -- (1.45111,0.81838);
\draw [c] (1.45111,0.81838) -- (1.45111,0.862466);
\draw [c] (1.43306,0.81838) -- (1.45111,0.81838);
\draw [c] (1.45111,0.81838) -- (1.46915,0.81838);
\definecolor{c}{rgb}{0,0,0};
\colorlet{c}{natcomp!70};
\draw [c] (1.4872,0.759731) -- (1.4872,0.794569);
\draw [c] (1.4872,0.794569) -- (1.4872,0.829406);
\draw [c] (1.46915,0.794569) -- (1.4872,0.794569);
\draw [c] (1.4872,0.794569) -- (1.50524,0.794569);
\definecolor{c}{rgb}{0,0,0};
\colorlet{c}{natcomp!70};
\draw [c] (1.52329,0.752172) -- (1.52329,0.776524);
\draw [c] (1.52329,0.776524) -- (1.52329,0.800876);
\draw [c] (1.50524,0.776524) -- (1.52329,0.776524);
\draw [c] (1.52329,0.776524) -- (1.54133,0.776524);
\definecolor{c}{rgb}{0,0,0};
\colorlet{c}{natcomp!70};
\draw [c] (1.55938,0.734713) -- (1.55938,0.756435);
\draw [c] (1.55938,0.756435) -- (1.55938,0.778156);
\draw [c] (1.54133,0.756435) -- (1.55938,0.756435);
\draw [c] (1.55938,0.756435) -- (1.57742,0.756435);
\definecolor{c}{rgb}{0,0,0};
\colorlet{c}{natcomp!70};
\draw [c] (1.59546,0.763306) -- (1.59546,0.806425);
\draw [c] (1.59546,0.806425) -- (1.59546,0.849545);
\draw [c] (1.57742,0.806425) -- (1.59546,0.806425);
\draw [c] (1.59546,0.806425) -- (1.61351,0.806425);
\definecolor{c}{rgb}{0,0,0};
\colorlet{c}{natcomp!70};
\draw [c] (1.63155,0.784078) -- (1.63155,0.823256);
\draw [c] (1.63155,0.823256) -- (1.63155,0.862434);
\draw [c] (1.61351,0.823256) -- (1.63155,0.823256);
\draw [c] (1.63155,0.823256) -- (1.6496,0.823256);
\definecolor{c}{rgb}{0,0,0};
\colorlet{c}{natcomp!70};
\draw [c] (1.66764,0.800855) -- (1.66764,0.840914);
\draw [c] (1.66764,0.840914) -- (1.66764,0.880972);
\draw [c] (1.6496,0.840914) -- (1.66764,0.840914);
\draw [c] (1.66764,0.840914) -- (1.68569,0.840914);
\definecolor{c}{rgb}{0,0,0};
\colorlet{c}{natcomp!70};
\draw [c] (1.70373,0.898383) -- (1.70373,0.953898);
\draw [c] (1.70373,0.953898) -- (1.70373,1.00941);
\draw [c] (1.68569,0.953898) -- (1.70373,0.953898);
\draw [c] (1.70373,0.953898) -- (1.72177,0.953898);
\definecolor{c}{rgb}{0,0,0};
\colorlet{c}{natcomp!70};
\draw [c] (1.73982,0.798717) -- (1.73982,0.84012);
\draw [c] (1.73982,0.84012) -- (1.73982,0.881522);
\draw [c] (1.72177,0.84012) -- (1.73982,0.84012);
\draw [c] (1.73982,0.84012) -- (1.75786,0.84012);
\definecolor{c}{rgb}{0,0,0};
\colorlet{c}{natcomp!70};
\draw [c] (1.77591,0.920053) -- (1.77591,0.975511);
\draw [c] (1.77591,0.975511) -- (1.77591,1.03097);
\draw [c] (1.75786,0.975511) -- (1.77591,0.975511);
\draw [c] (1.77591,0.975511) -- (1.79395,0.975511);
\definecolor{c}{rgb}{0,0,0};
\colorlet{c}{natcomp!70};
\draw [c] (1.812,0.936516) -- (1.812,0.998493);
\draw [c] (1.812,0.998493) -- (1.812,1.06047);
\draw [c] (1.79395,0.998493) -- (1.812,0.998493);
\draw [c] (1.812,0.998493) -- (1.83004,0.998493);
\definecolor{c}{rgb}{0,0,0};
\colorlet{c}{natcomp!70};
\draw [c] (1.84808,1.18919) -- (1.84808,1.28414);
\draw [c] (1.84808,1.28414) -- (1.84808,1.3791);
\draw [c] (1.83004,1.28414) -- (1.84808,1.28414);
\draw [c] (1.84808,1.28414) -- (1.86613,1.28414);
\definecolor{c}{rgb}{0,0,0};
\colorlet{c}{natcomp!70};
\draw [c] (1.88417,1.12118) -- (1.88417,1.20085);
\draw [c] (1.88417,1.20085) -- (1.88417,1.28052);
\draw [c] (1.86613,1.20085) -- (1.88417,1.20085);
\draw [c] (1.88417,1.20085) -- (1.90222,1.20085);
\definecolor{c}{rgb}{0,0,0};
\colorlet{c}{natcomp!70};
\draw [c] (1.92026,1.14624) -- (1.92026,1.22811);
\draw [c] (1.92026,1.22811) -- (1.92026,1.30998);
\draw [c] (1.90222,1.22811) -- (1.92026,1.22811);
\draw [c] (1.92026,1.22811) -- (1.93831,1.22811);
\definecolor{c}{rgb}{0,0,0};
\colorlet{c}{natcomp!70};
\draw [c] (1.95635,1.30269) -- (1.95635,1.39258);
\draw [c] (1.95635,1.39258) -- (1.95635,1.48248);
\draw [c] (1.93831,1.39258) -- (1.95635,1.39258);
\draw [c] (1.95635,1.39258) -- (1.9744,1.39258);
\definecolor{c}{rgb}{0,0,0};
\colorlet{c}{natcomp!70};
\draw [c] (1.99244,1.33385) -- (1.99244,1.42513);
\draw [c] (1.99244,1.42513) -- (1.99244,1.51641);
\draw [c] (1.9744,1.42513) -- (1.99244,1.42513);
\draw [c] (1.99244,1.42513) -- (2.01048,1.42513);
\definecolor{c}{rgb}{0,0,0};
\colorlet{c}{natcomp!70};
\draw [c] (2.02853,1.50746) -- (2.02853,1.6117);
\draw [c] (2.02853,1.6117) -- (2.02853,1.71594);
\draw [c] (2.01048,1.6117) -- (2.02853,1.6117);
\draw [c] (2.02853,1.6117) -- (2.04657,1.6117);
\definecolor{c}{rgb}{0,0,0};
\colorlet{c}{natcomp!70};
\draw [c] (2.06462,1.66914) -- (2.06462,1.7831);
\draw [c] (2.06462,1.7831) -- (2.06462,1.89706);
\draw [c] (2.04657,1.7831) -- (2.06462,1.7831);
\draw [c] (2.06462,1.7831) -- (2.08266,1.7831);
\definecolor{c}{rgb}{0,0,0};
\colorlet{c}{natcomp!70};
\draw [c] (2.10071,1.71336) -- (2.10071,1.83223);
\draw [c] (2.10071,1.83223) -- (2.10071,1.9511);
\draw [c] (2.08266,1.83223) -- (2.10071,1.83223);
\draw [c] (2.10071,1.83223) -- (2.11875,1.83223);
\definecolor{c}{rgb}{0,0,0};
\colorlet{c}{natcomp!70};
\draw [c] (2.13679,2.01604) -- (2.13679,2.14978);
\draw [c] (2.13679,2.14978) -- (2.13679,2.28352);
\draw [c] (2.11875,2.14978) -- (2.13679,2.14978);
\draw [c] (2.13679,2.14978) -- (2.15484,2.14978);
\definecolor{c}{rgb}{0,0,0};
\colorlet{c}{natcomp!70};
\draw [c] (2.17288,2.19175) -- (2.17288,2.33083);
\draw [c] (2.17288,2.33083) -- (2.17288,2.46991);
\draw [c] (2.15484,2.33083) -- (2.17288,2.33083);
\draw [c] (2.17288,2.33083) -- (2.19093,2.33083);
\definecolor{c}{rgb}{0,0,0};
\colorlet{c}{natcomp!70};
\draw [c] (2.20897,2.44823) -- (2.20897,2.5985);
\draw [c] (2.20897,2.5985) -- (2.20897,2.74878);
\draw [c] (2.19093,2.5985) -- (2.20897,2.5985);
\draw [c] (2.20897,2.5985) -- (2.22702,2.5985);
\definecolor{c}{rgb}{0,0,0};
\colorlet{c}{natcomp!70};
\draw [c] (2.24506,2.63778) -- (2.24506,2.79551);
\draw [c] (2.24506,2.79551) -- (2.24506,2.95324);
\draw [c] (2.22702,2.79551) -- (2.24506,2.79551);
\draw [c] (2.24506,2.79551) -- (2.2631,2.79551);
\definecolor{c}{rgb}{0,0,0};
\colorlet{c}{natcomp!70};
\draw [c] (2.28115,2.66197) -- (2.28115,2.81917);
\draw [c] (2.28115,2.81917) -- (2.28115,2.97637);
\draw [c] (2.2631,2.81917) -- (2.28115,2.81917);
\draw [c] (2.28115,2.81917) -- (2.29919,2.81917);
\definecolor{c}{rgb}{0,0,0};
\colorlet{c}{natcomp!70};
\draw [c] (2.31724,3.3542) -- (2.31724,3.53769);
\draw [c] (2.31724,3.53769) -- (2.31724,3.72117);
\draw [c] (2.29919,3.53769) -- (2.31724,3.53769);
\draw [c] (2.31724,3.53769) -- (2.33528,3.53769);
\definecolor{c}{rgb}{0,0,0};
\colorlet{c}{natcomp!70};
\draw [c] (2.35333,3.50262) -- (2.35333,3.68753);
\draw [c] (2.35333,3.68753) -- (2.35333,3.87244);
\draw [c] (2.33528,3.68753) -- (2.35333,3.68753);
\draw [c] (2.35333,3.68753) -- (2.37137,3.68753);
\definecolor{c}{rgb}{0,0,0};
\colorlet{c}{natcomp!70};
\draw [c] (2.38942,3.94866) -- (2.38942,4.15279);
\draw [c] (2.38942,4.15279) -- (2.38942,4.35691);
\draw [c] (2.37137,4.15279) -- (2.38942,4.15279);
\draw [c] (2.38942,4.15279) -- (2.40746,4.15279);
\definecolor{c}{rgb}{0,0,0};
\colorlet{c}{natcomp!70};
\draw [c] (2.4255,4.34127) -- (2.4255,4.55624);
\draw [c] (2.4255,4.55624) -- (2.4255,4.77121);
\draw [c] (2.40746,4.55624) -- (2.4255,4.55624);
\draw [c] (2.4255,4.55624) -- (2.44355,4.55624);
\definecolor{c}{rgb}{0,0,0};
\colorlet{c}{natcomp!70};
\draw [c] (2.46159,4.60059) -- (2.46159,4.81922);
\draw [c] (2.46159,4.81922) -- (2.46159,5.03785);
\draw [c] (2.44355,4.81922) -- (2.46159,4.81922);
\draw [c] (2.46159,4.81922) -- (2.47964,4.81922);
\definecolor{c}{rgb}{0,0,0};
\colorlet{c}{natcomp!70};
\draw [c] (2.49768,5.39548) -- (2.49768,5.64371);
\draw [c] (2.49768,5.64371) -- (2.49768,5.89194);
\draw [c] (2.47964,5.64371) -- (2.49768,5.64371);
\draw [c] (2.49768,5.64371) -- (2.51573,5.64371);
\definecolor{c}{rgb}{0,0,0};
\colorlet{c}{natcomp!70};
\draw [c] (2.53377,5.06867) -- (2.53377,5.3066);
\draw [c] (2.53377,5.3066) -- (2.53377,5.54453);
\draw [c] (2.51573,5.3066) -- (2.53377,5.3066);
\draw [c] (2.53377,5.3066) -- (2.55181,5.3066);
\definecolor{c}{rgb}{0,0,0};
\colorlet{c}{natcomp!70};
\draw [c] (2.56986,5.88276) -- (2.56986,6.15509);
\draw [c] (2.56986,6.15509) -- (2.56986,6.42741);
\draw [c] (2.55181,6.15509) -- (2.56986,6.15509);
\draw [c] (2.56986,6.15509) -- (2.5879,6.15509);
\definecolor{c}{rgb}{0,0,0};
\colorlet{c}{natcomp!70};
\draw [c] (2.60595,6.14439) -- (2.60595,6.41923);
\draw [c] (2.60595,6.41923) -- (2.60595,6.69406);
\draw [c] (2.5879,6.41923) -- (2.60595,6.41923);
\draw [c] (2.60595,6.41923) -- (2.62399,6.41923);
\definecolor{c}{rgb}{0,0,0};
\colorlet{c}{natcomp!70};
\draw [c] (2.64204,6.13672) -- (2.64204,6.41349);
\draw [c] (2.64204,6.41349) -- (2.64204,6.69026);
\draw [c] (2.62399,6.41349) -- (2.64204,6.41349);
\draw [c] (2.64204,6.41349) -- (2.66008,6.41349);
\definecolor{c}{rgb}{0,0,0};
\colorlet{c}{natcomp!70};
\draw [c] (2.67813,5.66473) -- (2.67813,5.93277);
\draw [c] (2.67813,5.93277) -- (2.67813,6.20081);
\draw [c] (2.66008,5.93277) -- (2.67813,5.93277);
\draw [c] (2.67813,5.93277) -- (2.69617,5.93277);
\definecolor{c}{rgb}{0,0,0};
\colorlet{c}{natcomp!70};
\draw [c] (2.71421,5.38953) -- (2.71421,5.65592);
\draw [c] (2.71421,5.65592) -- (2.71421,5.92231);
\draw [c] (2.69617,5.65592) -- (2.71421,5.65592);
\draw [c] (2.71421,5.65592) -- (2.73226,5.65592);
\definecolor{c}{rgb}{0,0,0};
\colorlet{c}{natcomp!70};
\draw [c] (2.7503,4.96565) -- (2.7503,5.21733);
\draw [c] (2.7503,5.21733) -- (2.7503,5.46902);
\draw [c] (2.73226,5.21733) -- (2.7503,5.21733);
\draw [c] (2.7503,5.21733) -- (2.76835,5.21733);
\definecolor{c}{rgb}{0,0,0};
\colorlet{c}{natcomp!70};
\draw [c] (2.78639,4.72809) -- (2.78639,4.97622);
\draw [c] (2.78639,4.97622) -- (2.78639,5.22435);
\draw [c] (2.76835,4.97622) -- (2.78639,4.97622);
\draw [c] (2.78639,4.97622) -- (2.80444,4.97622);
\definecolor{c}{rgb}{0,0,0};
\colorlet{c}{natcomp!70};
\draw [c] (2.82248,4.72571) -- (2.82248,4.96387);
\draw [c] (2.82248,4.96387) -- (2.82248,5.20204);
\draw [c] (2.80444,4.96387) -- (2.82248,4.96387);
\draw [c] (2.82248,4.96387) -- (2.84052,4.96387);
\definecolor{c}{rgb}{0,0,0};
\colorlet{c}{natcomp!70};
\draw [c] (2.85857,4.18523) -- (2.85857,4.403);
\draw [c] (2.85857,4.403) -- (2.85857,4.62076);
\draw [c] (2.84052,4.403) -- (2.85857,4.403);
\draw [c] (2.85857,4.403) -- (2.87661,4.403);
\definecolor{c}{rgb}{0,0,0};
\colorlet{c}{natcomp!70};
\draw [c] (2.89466,3.65899) -- (2.89466,3.85657);
\draw [c] (2.89466,3.85657) -- (2.89466,4.05415);
\draw [c] (2.87661,3.85657) -- (2.89466,3.85657);
\draw [c] (2.89466,3.85657) -- (2.9127,3.85657);
\definecolor{c}{rgb}{0,0,0};
\colorlet{c}{natcomp!70};
\draw [c] (2.93075,3.93162) -- (2.93075,4.14189);
\draw [c] (2.93075,4.14189) -- (2.93075,4.35217);
\draw [c] (2.9127,4.14189) -- (2.93075,4.14189);
\draw [c] (2.93075,4.14189) -- (2.94879,4.14189);
\definecolor{c}{rgb}{0,0,0};
\colorlet{c}{natcomp!70};
\draw [c] (2.96683,3.4435) -- (2.96683,3.63714);
\draw [c] (2.96683,3.63714) -- (2.96683,3.83078);
\draw [c] (2.94879,3.63714) -- (2.96683,3.63714);
\draw [c] (2.96683,3.63714) -- (2.98488,3.63714);
\definecolor{c}{rgb}{0,0,0};
\colorlet{c}{natcomp!70};
\draw [c] (3.00292,3.01529) -- (3.00292,3.1925);
\draw [c] (3.00292,3.1925) -- (3.00292,3.36972);
\draw [c] (2.98488,3.1925) -- (3.00292,3.1925);
\draw [c] (3.00292,3.1925) -- (3.02097,3.1925);
\definecolor{c}{rgb}{0,0,0};
\colorlet{c}{natcomp!70};
\draw [c] (3.03901,3.01986) -- (3.03901,3.19798);
\draw [c] (3.03901,3.19798) -- (3.03901,3.3761);
\draw [c] (3.02097,3.19798) -- (3.03901,3.19798);
\draw [c] (3.03901,3.19798) -- (3.05706,3.19798);
\definecolor{c}{rgb}{0,0,0};
\colorlet{c}{natcomp!70};
\draw [c] (3.0751,2.88122) -- (3.0751,3.05473);
\draw [c] (3.0751,3.05473) -- (3.0751,3.22824);
\draw [c] (3.05706,3.05473) -- (3.0751,3.05473);
\draw [c] (3.0751,3.05473) -- (3.09315,3.05473);
\definecolor{c}{rgb}{0,0,0};
\colorlet{c}{natcomp!70};
\draw [c] (3.11119,2.58374) -- (3.11119,2.74401);
\draw [c] (3.11119,2.74401) -- (3.11119,2.90429);
\draw [c] (3.09315,2.74401) -- (3.11119,2.74401);
\draw [c] (3.11119,2.74401) -- (3.12923,2.74401);
\definecolor{c}{rgb}{0,0,0};
\colorlet{c}{natcomp!70};
\draw [c] (3.14728,2.30093) -- (3.14728,2.44861);
\draw [c] (3.14728,2.44861) -- (3.14728,2.59629);
\draw [c] (3.12923,2.44861) -- (3.14728,2.44861);
\draw [c] (3.14728,2.44861) -- (3.16532,2.44861);
\definecolor{c}{rgb}{0,0,0};
\colorlet{c}{natcomp!70};
\draw [c] (3.18337,2.3983) -- (3.18337,2.55001);
\draw [c] (3.18337,2.55001) -- (3.18337,2.70172);
\draw [c] (3.16532,2.55001) -- (3.18337,2.55001);
\draw [c] (3.18337,2.55001) -- (3.20141,2.55001);
\definecolor{c}{rgb}{0,0,0};
\colorlet{c}{natcomp!70};
\draw [c] (3.21946,2.43359) -- (3.21946,2.5937);
\draw [c] (3.21946,2.5937) -- (3.21946,2.75381);
\draw [c] (3.20141,2.5937) -- (3.21946,2.5937);
\draw [c] (3.21946,2.5937) -- (3.2375,2.5937);
\definecolor{c}{rgb}{0,0,0};
\colorlet{c}{natcomp!70};
\draw [c] (3.25554,2.18533) -- (3.25554,2.32113);
\draw [c] (3.25554,2.32113) -- (3.25554,2.45694);
\draw [c] (3.2375,2.32113) -- (3.25554,2.32113);
\draw [c] (3.25554,2.32113) -- (3.27359,2.32113);
\definecolor{c}{rgb}{0,0,0};
\colorlet{c}{natcomp!70};
\draw [c] (3.29163,2.0191) -- (3.29163,2.15263);
\draw [c] (3.29163,2.15263) -- (3.29163,2.28615);
\draw [c] (3.27359,2.15263) -- (3.29163,2.15263);
\draw [c] (3.29163,2.15263) -- (3.30968,2.15263);
\definecolor{c}{rgb}{0,0,0};
\colorlet{c}{natcomp!70};
\draw [c] (3.32772,2.11944) -- (3.32772,2.256);
\draw [c] (3.32772,2.256) -- (3.32772,2.39256);
\draw [c] (3.30968,2.256) -- (3.32772,2.256);
\draw [c] (3.32772,2.256) -- (3.34577,2.256);
\definecolor{c}{rgb}{0,0,0};
\colorlet{c}{natcomp!70};
\draw [c] (3.36381,1.88428) -- (3.36381,2.00683);
\draw [c] (3.36381,2.00683) -- (3.36381,2.12938);
\draw [c] (3.34577,2.00683) -- (3.36381,2.00683);
\draw [c] (3.36381,2.00683) -- (3.38185,2.00683);
\definecolor{c}{rgb}{0,0,0};
\colorlet{c}{natcomp!70};
\draw [c] (3.3999,1.92813) -- (3.3999,2.05525);
\draw [c] (3.3999,2.05525) -- (3.3999,2.18238);
\draw [c] (3.38185,2.05525) -- (3.3999,2.05525);
\draw [c] (3.3999,2.05525) -- (3.41794,2.05525);
\definecolor{c}{rgb}{0,0,0};
\colorlet{c}{natcomp!70};
\draw [c] (3.43599,1.74122) -- (3.43599,1.86638);
\draw [c] (3.43599,1.86638) -- (3.43599,1.99155);
\draw [c] (3.41794,1.86638) -- (3.43599,1.86638);
\draw [c] (3.43599,1.86638) -- (3.45403,1.86638);
\definecolor{c}{rgb}{0,0,0};
\colorlet{c}{natcomp!70};
\draw [c] (3.47208,1.81489) -- (3.47208,1.94716);
\draw [c] (3.47208,1.94716) -- (3.47208,2.07944);
\draw [c] (3.45403,1.94716) -- (3.47208,1.94716);
\draw [c] (3.47208,1.94716) -- (3.49012,1.94716);
\definecolor{c}{rgb}{0,0,0};
\colorlet{c}{natcomp!70};
\draw [c] (3.50817,1.85856) -- (3.50817,1.99328);
\draw [c] (3.50817,1.99328) -- (3.50817,2.128);
\draw [c] (3.49012,1.99328) -- (3.50817,1.99328);
\draw [c] (3.50817,1.99328) -- (3.52621,1.99328);
\definecolor{c}{rgb}{0,0,0};
\colorlet{c}{natcomp!70};
\draw [c] (3.54425,1.51474) -- (3.54425,1.62467);
\draw [c] (3.54425,1.62467) -- (3.54425,1.7346);
\draw [c] (3.52621,1.62467) -- (3.54425,1.62467);
\draw [c] (3.54425,1.62467) -- (3.5623,1.62467);
\definecolor{c}{rgb}{0,0,0};
\colorlet{c}{natcomp!70};
\draw [c] (3.58034,1.58337) -- (3.58034,1.69582);
\draw [c] (3.58034,1.69582) -- (3.58034,1.80827);
\draw [c] (3.5623,1.69582) -- (3.58034,1.69582);
\draw [c] (3.58034,1.69582) -- (3.59839,1.69582);
\definecolor{c}{rgb}{0,0,0};
\colorlet{c}{natcomp!70};
\draw [c] (3.61643,1.54794) -- (3.61643,1.65528);
\draw [c] (3.61643,1.65528) -- (3.61643,1.76262);
\draw [c] (3.59839,1.65528) -- (3.61643,1.65528);
\draw [c] (3.61643,1.65528) -- (3.63448,1.65528);
\definecolor{c}{rgb}{0,0,0};
\colorlet{c}{natcomp!70};
\draw [c] (3.65252,1.42522) -- (3.65252,1.52375);
\draw [c] (3.65252,1.52375) -- (3.65252,1.62228);
\draw [c] (3.63448,1.52375) -- (3.65252,1.52375);
\draw [c] (3.65252,1.52375) -- (3.67056,1.52375);
\definecolor{c}{rgb}{0,0,0};
\colorlet{c}{natcomp!70};
\draw [c] (3.68861,1.31417) -- (3.68861,1.40383);
\draw [c] (3.68861,1.40383) -- (3.68861,1.4935);
\draw [c] (3.67056,1.40383) -- (3.68861,1.40383);
\draw [c] (3.68861,1.40383) -- (3.70665,1.40383);
\definecolor{c}{rgb}{0,0,0};
\colorlet{c}{natcomp!70};
\draw [c] (3.7247,1.39879) -- (3.7247,1.50121);
\draw [c] (3.7247,1.50121) -- (3.7247,1.60364);
\draw [c] (3.70665,1.50121) -- (3.7247,1.50121);
\draw [c] (3.7247,1.50121) -- (3.74274,1.50121);
\definecolor{c}{rgb}{0,0,0};
\colorlet{c}{natcomp!70};
\draw [c] (3.76079,1.47938) -- (3.76079,1.57989);
\draw [c] (3.76079,1.57989) -- (3.76079,1.68041);
\draw [c] (3.74274,1.57989) -- (3.76079,1.57989);
\draw [c] (3.76079,1.57989) -- (3.77883,1.57989);
\definecolor{c}{rgb}{0,0,0};
\colorlet{c}{natcomp!70};
\draw [c] (3.79688,1.37064) -- (3.79688,1.46625);
\draw [c] (3.79688,1.46625) -- (3.79688,1.56186);
\draw [c] (3.77883,1.46625) -- (3.79688,1.46625);
\draw [c] (3.79688,1.46625) -- (3.81492,1.46625);
\definecolor{c}{rgb}{0,0,0};
\colorlet{c}{natcomp!70};
\draw [c] (3.83296,1.48496) -- (3.83296,1.58921);
\draw [c] (3.83296,1.58921) -- (3.83296,1.69345);
\draw [c] (3.81492,1.58921) -- (3.83296,1.58921);
\draw [c] (3.83296,1.58921) -- (3.85101,1.58921);
\definecolor{c}{rgb}{0,0,0};
\colorlet{c}{natcomp!70};
\draw [c] (3.86905,1.16297) -- (3.86905,1.24409);
\draw [c] (3.86905,1.24409) -- (3.86905,1.3252);
\draw [c] (3.85101,1.24409) -- (3.86905,1.24409);
\draw [c] (3.86905,1.24409) -- (3.8871,1.24409);
\definecolor{c}{rgb}{0,0,0};
\colorlet{c}{natcomp!70};
\draw [c] (3.90514,1.26804) -- (3.90514,1.35812);
\draw [c] (3.90514,1.35812) -- (3.90514,1.4482);
\draw [c] (3.8871,1.35812) -- (3.90514,1.35812);
\draw [c] (3.90514,1.35812) -- (3.92319,1.35812);
\definecolor{c}{rgb}{0,0,0};
\colorlet{c}{natcomp!70};
\draw [c] (3.94123,1.22039) -- (3.94123,1.30135);
\draw [c] (3.94123,1.30135) -- (3.94123,1.38232);
\draw [c] (3.92319,1.30135) -- (3.94123,1.30135);
\draw [c] (3.94123,1.30135) -- (3.95927,1.30135);
\definecolor{c}{rgb}{0,0,0};
\colorlet{c}{natcomp!70};
\draw [c] (3.97732,1.12845) -- (3.97732,1.20076);
\draw [c] (3.97732,1.20076) -- (3.97732,1.27306);
\draw [c] (3.95927,1.20076) -- (3.97732,1.20076);
\draw [c] (3.97732,1.20076) -- (3.99536,1.20076);
\definecolor{c}{rgb}{0,0,0};
\colorlet{c}{natcomp!70};
\draw [c] (4.01341,1.29099) -- (4.01341,1.38407);
\draw [c] (4.01341,1.38407) -- (4.01341,1.47715);
\draw [c] (3.99536,1.38407) -- (4.01341,1.38407);
\draw [c] (4.01341,1.38407) -- (4.03145,1.38407);
\definecolor{c}{rgb}{0,0,0};
\colorlet{c}{natcomp!70};
\draw [c] (4.0495,1.08142) -- (4.0495,1.15343);
\draw [c] (4.0495,1.15343) -- (4.0495,1.22545);
\draw [c] (4.03145,1.15343) -- (4.0495,1.15343);
\draw [c] (4.0495,1.15343) -- (4.06754,1.15343);
\definecolor{c}{rgb}{0,0,0};
\colorlet{c}{natcomp!70};
\draw [c] (4.08558,1.18677) -- (4.08558,1.27028);
\draw [c] (4.08558,1.27028) -- (4.08558,1.35379);
\draw [c] (4.06754,1.27028) -- (4.08558,1.27028);
\draw [c] (4.08558,1.27028) -- (4.10363,1.27028);
\definecolor{c}{rgb}{0,0,0};
\colorlet{c}{natcomp!70};
\draw [c] (4.12167,1.14251) -- (4.12167,1.21997);
\draw [c] (4.12167,1.21997) -- (4.12167,1.29744);
\draw [c] (4.10363,1.21997) -- (4.12167,1.21997);
\draw [c] (4.12167,1.21997) -- (4.13972,1.21997);
\definecolor{c}{rgb}{0,0,0};
\colorlet{c}{natcomp!70};
\draw [c] (4.15776,1.11761) -- (4.15776,1.20294);
\draw [c] (4.15776,1.20294) -- (4.15776,1.28827);
\draw [c] (4.13972,1.20294) -- (4.15776,1.20294);
\draw [c] (4.15776,1.20294) -- (4.17581,1.20294);
\definecolor{c}{rgb}{0,0,0};
\colorlet{c}{natcomp!70};
\draw [c] (4.19385,1.17105) -- (4.19385,1.25955);
\draw [c] (4.19385,1.25955) -- (4.19385,1.34804);
\draw [c] (4.17581,1.25955) -- (4.19385,1.25955);
\draw [c] (4.19385,1.25955) -- (4.21189,1.25955);
\definecolor{c}{rgb}{0,0,0};
\colorlet{c}{natcomp!70};
\draw [c] (4.22994,1.10263) -- (4.22994,1.17065);
\draw [c] (4.22994,1.17065) -- (4.22994,1.23867);
\draw [c] (4.21189,1.17065) -- (4.22994,1.17065);
\draw [c] (4.22994,1.17065) -- (4.24798,1.17065);
\definecolor{c}{rgb}{0,0,0};
\colorlet{c}{natcomp!70};
\draw [c] (4.26603,1.06985) -- (4.26603,1.1406);
\draw [c] (4.26603,1.1406) -- (4.26603,1.21135);
\draw [c] (4.24798,1.1406) -- (4.26603,1.1406);
\draw [c] (4.26603,1.1406) -- (4.28407,1.1406);
\definecolor{c}{rgb}{0,0,0};
\colorlet{c}{natcomp!70};
\draw [c] (4.30212,1.04355) -- (4.30212,1.1102);
\draw [c] (4.30212,1.1102) -- (4.30212,1.17684);
\draw [c] (4.28407,1.1102) -- (4.30212,1.1102);
\draw [c] (4.30212,1.1102) -- (4.32016,1.1102);
\definecolor{c}{rgb}{0,0,0};
\colorlet{c}{natcomp!70};
\draw [c] (4.33821,1.01865) -- (4.33821,1.09854);
\draw [c] (4.33821,1.09854) -- (4.33821,1.17843);
\draw [c] (4.32016,1.09854) -- (4.33821,1.09854);
\draw [c] (4.33821,1.09854) -- (4.35625,1.09854);
\definecolor{c}{rgb}{0,0,0};
\colorlet{c}{natcomp!70};
\draw [c] (4.37429,0.906879) -- (4.37429,0.955253);
\draw [c] (4.37429,0.955253) -- (4.37429,1.00363);
\draw [c] (4.35625,0.955253) -- (4.37429,0.955253);
\draw [c] (4.37429,0.955253) -- (4.39234,0.955253);
\definecolor{c}{rgb}{0,0,0};
\colorlet{c}{natcomp!70};
\draw [c] (4.41038,0.926954) -- (4.41038,0.983752);
\draw [c] (4.41038,0.983752) -- (4.41038,1.04055);
\draw [c] (4.39234,0.983752) -- (4.41038,0.983752);
\draw [c] (4.41038,0.983752) -- (4.42843,0.983752);
\definecolor{c}{rgb}{0,0,0};
\colorlet{c}{natcomp!70};
\draw [c] (4.44647,1.03078) -- (4.44647,1.10645);
\draw [c] (4.44647,1.10645) -- (4.44647,1.18211);
\draw [c] (4.42843,1.10645) -- (4.44647,1.10645);
\draw [c] (4.44647,1.10645) -- (4.46452,1.10645);
\definecolor{c}{rgb}{0,0,0};
\colorlet{c}{natcomp!70};
\draw [c] (4.48256,0.96438) -- (4.48256,1.02395);
\draw [c] (4.48256,1.02395) -- (4.48256,1.08353);
\draw [c] (4.46452,1.02395) -- (4.48256,1.02395);
\draw [c] (4.48256,1.02395) -- (4.5006,1.02395);
\definecolor{c}{rgb}{0,0,0};
\colorlet{c}{natcomp!70};
\draw [c] (4.51865,1.01953) -- (4.51865,1.09213);
\draw [c] (4.51865,1.09213) -- (4.51865,1.16473);
\draw [c] (4.5006,1.09213) -- (4.51865,1.09213);
\draw [c] (4.51865,1.09213) -- (4.53669,1.09213);
\definecolor{c}{rgb}{0,0,0};
\colorlet{c}{natcomp!70};
\draw [c] (4.55474,0.89496) -- (4.55474,0.945912);
\draw [c] (4.55474,0.945912) -- (4.55474,0.996863);
\draw [c] (4.53669,0.945912) -- (4.55474,0.945912);
\draw [c] (4.55474,0.945912) -- (4.57278,0.945912);
\definecolor{c}{rgb}{0,0,0};
\colorlet{c}{natcomp!70};
\draw [c] (4.59083,0.986726) -- (4.59083,1.04818);
\draw [c] (4.59083,1.04818) -- (4.59083,1.10964);
\draw [c] (4.57278,1.04818) -- (4.59083,1.04818);
\draw [c] (4.59083,1.04818) -- (4.60887,1.04818);
\definecolor{c}{rgb}{0,0,0};
\colorlet{c}{natcomp!70};
\draw [c] (4.62692,0.828778) -- (4.62692,0.867519);
\draw [c] (4.62692,0.867519) -- (4.62692,0.90626);
\draw [c] (4.60887,0.867519) -- (4.62692,0.867519);
\draw [c] (4.62692,0.867519) -- (4.64496,0.867519);
\definecolor{c}{rgb}{0,0,0};
\colorlet{c}{natcomp!70};
\draw [c] (4.663,0.929865) -- (4.663,0.982585);
\draw [c] (4.663,0.982585) -- (4.663,1.03531);
\draw [c] (4.64496,0.982585) -- (4.663,0.982585);
\draw [c] (4.663,0.982585) -- (4.68105,0.982585);
\definecolor{c}{rgb}{0,0,0};
\colorlet{c}{natcomp!70};
\draw [c] (4.69909,0.937887) -- (4.69909,1.0001);
\draw [c] (4.69909,1.0001) -- (4.69909,1.06232);
\draw [c] (4.68105,1.0001) -- (4.69909,1.0001);
\draw [c] (4.69909,1.0001) -- (4.71714,1.0001);
\definecolor{c}{rgb}{0,0,0};
\colorlet{c}{natcomp!70};
\draw [c] (4.73518,0.910001) -- (4.73518,0.961309);
\draw [c] (4.73518,0.961309) -- (4.73518,1.01262);
\draw [c] (4.71714,0.961309) -- (4.73518,0.961309);
\draw [c] (4.73518,0.961309) -- (4.75323,0.961309);
\definecolor{c}{rgb}{0,0,0};
\colorlet{c}{natcomp!70};
\draw [c] (4.77127,0.868019) -- (4.77127,0.914493);
\draw [c] (4.77127,0.914493) -- (4.77127,0.960967);
\draw [c] (4.75323,0.914493) -- (4.77127,0.914493);
\draw [c] (4.77127,0.914493) -- (4.78931,0.914493);
\definecolor{c}{rgb}{0,0,0};
\colorlet{c}{natcomp!70};
\draw [c] (4.80736,0.878272) -- (4.80736,0.930742);
\draw [c] (4.80736,0.930742) -- (4.80736,0.983212);
\draw [c] (4.78931,0.930742) -- (4.80736,0.930742);
\draw [c] (4.80736,0.930742) -- (4.8254,0.930742);
\definecolor{c}{rgb}{0,0,0};
\colorlet{c}{natcomp!70};
\draw [c] (4.84345,0.809312) -- (4.84345,0.84424);
\draw [c] (4.84345,0.84424) -- (4.84345,0.879168);
\draw [c] (4.8254,0.84424) -- (4.84345,0.84424);
\draw [c] (4.84345,0.84424) -- (4.86149,0.84424);
\definecolor{c}{rgb}{0,0,0};
\colorlet{c}{natcomp!70};
\draw [c] (4.87954,0.762677) -- (4.87954,0.78561);
\draw [c] (4.87954,0.78561) -- (4.87954,0.808544);
\draw [c] (4.86149,0.78561) -- (4.87954,0.78561);
\draw [c] (4.87954,0.78561) -- (4.89758,0.78561);
\definecolor{c}{rgb}{0,0,0};
\colorlet{c}{natcomp!70};
\draw [c] (4.91563,0.835344) -- (4.91563,0.874984);
\draw [c] (4.91563,0.874984) -- (4.91563,0.914624);
\draw [c] (4.89758,0.874984) -- (4.91563,0.874984);
\draw [c] (4.91563,0.874984) -- (4.93367,0.874984);
\definecolor{c}{rgb}{0,0,0};
\colorlet{c}{natcomp!70};
\draw [c] (4.95171,0.87042) -- (4.95171,0.914521);
\draw [c] (4.95171,0.914521) -- (4.95171,0.958622);
\draw [c] (4.93367,0.914521) -- (4.95171,0.914521);
\draw [c] (4.95171,0.914521) -- (4.96976,0.914521);
\definecolor{c}{rgb}{0,0,0};
\colorlet{c}{natcomp!70};
\draw [c] (4.9878,0.826334) -- (4.9878,0.870592);
\draw [c] (4.9878,0.870592) -- (4.9878,0.91485);
\draw [c] (4.96976,0.870592) -- (4.9878,0.870592);
\draw [c] (4.9878,0.870592) -- (5.00585,0.870592);
\definecolor{c}{rgb}{0,0,0};
\colorlet{c}{natcomp!70};
\draw [c] (5.02389,0.822381) -- (5.02389,0.86059);
\draw [c] (5.02389,0.86059) -- (5.02389,0.898798);
\draw [c] (5.00585,0.86059) -- (5.02389,0.86059);
\draw [c] (5.02389,0.86059) -- (5.04194,0.86059);
\definecolor{c}{rgb}{0,0,0};
\colorlet{c}{natcomp!70};
\draw [c] (5.05998,0.788662) -- (5.05998,0.838806);
\draw [c] (5.05998,0.838806) -- (5.05998,0.88895);
\draw [c] (5.04194,0.838806) -- (5.05998,0.838806);
\draw [c] (5.05998,0.838806) -- (5.07802,0.838806);
\definecolor{c}{rgb}{0,0,0};
\colorlet{c}{natcomp!70};
\draw [c] (5.09607,0.81529) -- (5.09607,0.850679);
\draw [c] (5.09607,0.850679) -- (5.09607,0.886068);
\draw [c] (5.07802,0.850679) -- (5.09607,0.850679);
\draw [c] (5.09607,0.850679) -- (5.11411,0.850679);
\definecolor{c}{rgb}{0,0,0};
\colorlet{c}{natcomp!70};
\draw [c] (5.13216,0.877018) -- (5.13216,0.93804);
\draw [c] (5.13216,0.93804) -- (5.13216,0.999062);
\draw [c] (5.11411,0.93804) -- (5.13216,0.93804);
\draw [c] (5.13216,0.93804) -- (5.1502,0.93804);
\definecolor{c}{rgb}{0,0,0};
\colorlet{c}{natcomp!70};
\draw [c] (5.16825,0.815589) -- (5.16825,0.856866);
\draw [c] (5.16825,0.856866) -- (5.16825,0.898142);
\draw [c] (5.1502,0.856866) -- (5.16825,0.856866);
\draw [c] (5.16825,0.856866) -- (5.18629,0.856866);
\definecolor{c}{rgb}{0,0,0};
\colorlet{c}{natcomp!70};
\draw [c] (5.20433,0.740854) -- (5.20433,0.755731);
\draw [c] (5.20433,0.755731) -- (5.20433,0.770607);
\draw [c] (5.18629,0.755731) -- (5.20433,0.755731);
\draw [c] (5.20433,0.755731) -- (5.22238,0.755731);
\definecolor{c}{rgb}{0,0,0};
\colorlet{c}{natcomp!70};
\draw [c] (5.24042,0.812639) -- (5.24042,0.85358);
\draw [c] (5.24042,0.85358) -- (5.24042,0.89452);
\draw [c] (5.22238,0.85358) -- (5.24042,0.85358);
\draw [c] (5.24042,0.85358) -- (5.25847,0.85358);
\definecolor{c}{rgb}{0,0,0};
\colorlet{c}{natcomp!70};
\draw [c] (5.27651,0.803111) -- (5.27651,0.844052);
\draw [c] (5.27651,0.844052) -- (5.27651,0.884993);
\draw [c] (5.25847,0.844052) -- (5.27651,0.844052);
\draw [c] (5.27651,0.844052) -- (5.29456,0.844052);
\definecolor{c}{rgb}{0,0,0};
\colorlet{c}{natcomp!70};
\draw [c] (5.3126,0.800844) -- (5.3126,0.834967);
\draw [c] (5.3126,0.834967) -- (5.3126,0.869089);
\draw [c] (5.29456,0.834967) -- (5.3126,0.834967);
\draw [c] (5.3126,0.834967) -- (5.33065,0.834967);
\definecolor{c}{rgb}{0,0,0};
\colorlet{c}{natcomp!70};
\draw [c] (5.34869,0.784644) -- (5.34869,0.815177);
\draw [c] (5.34869,0.815177) -- (5.34869,0.845709);
\draw [c] (5.33065,0.815177) -- (5.34869,0.815177);
\draw [c] (5.34869,0.815177) -- (5.36673,0.815177);
\definecolor{c}{rgb}{0,0,0};
\colorlet{c}{natcomp!70};
\draw [c] (5.38478,0.802543) -- (5.38478,0.837564);
\draw [c] (5.38478,0.837564) -- (5.38478,0.872585);
\draw [c] (5.36673,0.837564) -- (5.38478,0.837564);
\draw [c] (5.38478,0.837564) -- (5.40282,0.837564);
\definecolor{c}{rgb}{0,0,0};
\colorlet{c}{natcomp!70};
\draw [c] (5.42087,0.838708) -- (5.42087,0.877281);
\draw [c] (5.42087,0.877281) -- (5.42087,0.915853);
\draw [c] (5.40282,0.877281) -- (5.42087,0.877281);
\draw [c] (5.42087,0.877281) -- (5.43891,0.877281);
\definecolor{c}{rgb}{0,0,0};
\colorlet{c}{natcomp!70};
\draw [c] (5.45696,0.828845) -- (5.45696,0.868725);
\draw [c] (5.45696,0.868725) -- (5.45696,0.908606);
\draw [c] (5.43891,0.868725) -- (5.45696,0.868725);
\draw [c] (5.45696,0.868725) -- (5.475,0.868725);
\definecolor{c}{rgb}{0,0,0};
\colorlet{c}{natcomp!70};
\draw [c] (5.49304,0.776612) -- (5.49304,0.806357);
\draw [c] (5.49304,0.806357) -- (5.49304,0.836102);
\draw [c] (5.475,0.806357) -- (5.49304,0.806357);
\draw [c] (5.49304,0.806357) -- (5.51109,0.806357);
\definecolor{c}{rgb}{0,0,0};
\colorlet{c}{natcomp!70};
\draw [c] (5.52913,0.765895) -- (5.52913,0.792266);
\draw [c] (5.52913,0.792266) -- (5.52913,0.818637);
\draw [c] (5.51109,0.792266) -- (5.52913,0.792266);
\draw [c] (5.52913,0.792266) -- (5.54718,0.792266);
\definecolor{c}{rgb}{0,0,0};
\colorlet{c}{natcomp!70};
\draw [c] (5.56522,0.772359) -- (5.56522,0.799594);
\draw [c] (5.56522,0.799594) -- (5.56522,0.82683);
\draw [c] (5.54718,0.799594) -- (5.56522,0.799594);
\draw [c] (5.56522,0.799594) -- (5.58327,0.799594);
\definecolor{c}{rgb}{0,0,0};
\colorlet{c}{natcomp!70};
\draw [c] (5.60131,0.796115) -- (5.60131,0.832186);
\draw [c] (5.60131,0.832186) -- (5.60131,0.868257);
\draw [c] (5.58327,0.832186) -- (5.60131,0.832186);
\draw [c] (5.60131,0.832186) -- (5.61935,0.832186);
\definecolor{c}{rgb}{0,0,0};
\colorlet{c}{natcomp!70};
\draw [c] (5.6374,0.781953) -- (5.6374,0.811129);
\draw [c] (5.6374,0.811129) -- (5.6374,0.840305);
\draw [c] (5.61935,0.811129) -- (5.6374,0.811129);
\draw [c] (5.6374,0.811129) -- (5.65544,0.811129);
\definecolor{c}{rgb}{0,0,0};
\colorlet{c}{natcomp!70};
\draw [c] (5.67349,0.81513) -- (5.67349,0.856508);
\draw [c] (5.67349,0.856508) -- (5.67349,0.897886);
\draw [c] (5.65544,0.856508) -- (5.67349,0.856508);
\draw [c] (5.67349,0.856508) -- (5.69153,0.856508);
\definecolor{c}{rgb}{0,0,0};
\colorlet{c}{natcomp!70};
\draw [c] (5.70958,0.792858) -- (5.70958,0.826425);
\draw [c] (5.70958,0.826425) -- (5.70958,0.859993);
\draw [c] (5.69153,0.826425) -- (5.70958,0.826425);
\draw [c] (5.70958,0.826425) -- (5.72762,0.826425);
\definecolor{c}{rgb}{0,0,0};
\colorlet{c}{natcomp!70};
\draw [c] (5.74567,0.749354) -- (5.74567,0.771177);
\draw [c] (5.74567,0.771177) -- (5.74567,0.793);
\draw [c] (5.72762,0.771177) -- (5.74567,0.771177);
\draw [c] (5.74567,0.771177) -- (5.76371,0.771177);
\definecolor{c}{rgb}{0,0,0};
\colorlet{c}{natcomp!70};
\draw [c] (5.78175,0.776801) -- (5.78175,0.80719);
\draw [c] (5.78175,0.80719) -- (5.78175,0.837579);
\draw [c] (5.76371,0.80719) -- (5.78175,0.80719);
\draw [c] (5.78175,0.80719) -- (5.7998,0.80719);
\definecolor{c}{rgb}{0,0,0};
\colorlet{c}{natcomp!70};
\draw [c] (5.81784,0.745067) -- (5.81784,0.777402);
\draw [c] (5.81784,0.777402) -- (5.81784,0.809737);
\draw [c] (5.7998,0.777402) -- (5.81784,0.777402);
\draw [c] (5.81784,0.777402) -- (5.83589,0.777402);
\definecolor{c}{rgb}{0,0,0};
\colorlet{c}{natcomp!70};
\draw [c] (5.85393,0.761061) -- (5.85393,0.782893);
\draw [c] (5.85393,0.782893) -- (5.85393,0.804725);
\draw [c] (5.83589,0.782893) -- (5.85393,0.782893);
\draw [c] (5.85393,0.782893) -- (5.87198,0.782893);
\definecolor{c}{rgb}{0,0,0};
\colorlet{c}{natcomp!70};
\draw [c] (5.89002,0.749941) -- (5.89002,0.771033);
\draw [c] (5.89002,0.771033) -- (5.89002,0.792126);
\draw [c] (5.87198,0.771033) -- (5.89002,0.771033);
\draw [c] (5.89002,0.771033) -- (5.90806,0.771033);
\definecolor{c}{rgb}{0,0,0};
\colorlet{c}{natcomp!70};
\draw [c] (5.92611,0.77436) -- (5.92611,0.80203);
\draw [c] (5.92611,0.80203) -- (5.92611,0.829699);
\draw [c] (5.90806,0.80203) -- (5.92611,0.80203);
\draw [c] (5.92611,0.80203) -- (5.94415,0.80203);
\definecolor{c}{rgb}{0,0,0};
\colorlet{c}{natcomp!70};
\draw [c] (5.9622,0.74754) -- (5.9622,0.765501);
\draw [c] (5.9622,0.765501) -- (5.9622,0.783462);
\draw [c] (5.94415,0.765501) -- (5.9622,0.765501);
\draw [c] (5.9622,0.765501) -- (5.98024,0.765501);
\definecolor{c}{rgb}{0,0,0};
\colorlet{c}{natcomp!70};
\draw [c] (5.99829,0.734669) -- (5.99829,0.747312);
\draw [c] (5.99829,0.747312) -- (5.99829,0.759955);
\draw [c] (5.98024,0.747312) -- (5.99829,0.747312);
\draw [c] (5.99829,0.747312) -- (6.01633,0.747312);
\definecolor{c}{rgb}{0,0,0};
\colorlet{c}{natcomp!70};
\draw [c] (6.03438,0.734644) -- (6.03438,0.746862);
\draw [c] (6.03438,0.746862) -- (6.03438,0.759079);
\draw [c] (6.01633,0.746862) -- (6.03438,0.746862);
\draw [c] (6.03438,0.746862) -- (6.05242,0.746862);
\definecolor{c}{rgb}{0,0,0};
\colorlet{c}{natcomp!70};
\draw [c] (6.07046,0.75936) -- (6.07046,0.784951);
\draw [c] (6.07046,0.784951) -- (6.07046,0.810543);
\draw [c] (6.05242,0.784951) -- (6.07046,0.784951);
\draw [c] (6.07046,0.784951) -- (6.08851,0.784951);
\definecolor{c}{rgb}{0,0,0};
\colorlet{c}{natcomp!70};
\draw [c] (6.10655,0.784459) -- (6.10655,0.823852);
\draw [c] (6.10655,0.823852) -- (6.10655,0.863245);
\draw [c] (6.08851,0.823852) -- (6.10655,0.823852);
\draw [c] (6.10655,0.823852) -- (6.1246,0.823852);
\definecolor{c}{rgb}{0,0,0};
\colorlet{c}{natcomp!70};
\draw [c] (6.14264,0.734662) -- (6.14264,0.734685);
\draw [c] (6.14264,0.734685) -- (6.14264,0.734709);
\draw [c] (6.1246,0.734685) -- (6.14264,0.734685);
\draw [c] (6.14264,0.734685) -- (6.16069,0.734685);
\definecolor{c}{rgb}{0,0,0};
\colorlet{c}{natcomp!70};
\draw [c] (6.17873,0.769006) -- (6.17873,0.797565);
\draw [c] (6.17873,0.797565) -- (6.17873,0.826124);
\draw [c] (6.16069,0.797565) -- (6.17873,0.797565);
\draw [c] (6.17873,0.797565) -- (6.19677,0.797565);
\definecolor{c}{rgb}{0,0,0};
\colorlet{c}{natcomp!70};
\draw [c] (6.21482,0.740767) -- (6.21482,0.75556);
\draw [c] (6.21482,0.75556) -- (6.21482,0.770352);
\draw [c] (6.19677,0.75556) -- (6.21482,0.75556);
\draw [c] (6.21482,0.75556) -- (6.23286,0.75556);
\definecolor{c}{rgb}{0,0,0};
\colorlet{c}{natcomp!70};
\draw [c] (6.25091,0.742869) -- (6.25091,0.767082);
\draw [c] (6.25091,0.767082) -- (6.25091,0.791294);
\draw [c] (6.23286,0.767082) -- (6.25091,0.767082);
\draw [c] (6.25091,0.767082) -- (6.26895,0.767082);
\definecolor{c}{rgb}{0,0,0};
\colorlet{c}{natcomp!70};
\draw [c] (6.287,0.763439) -- (6.287,0.787014);
\draw [c] (6.287,0.787014) -- (6.287,0.810589);
\draw [c] (6.26895,0.787014) -- (6.287,0.787014);
\draw [c] (6.287,0.787014) -- (6.30504,0.787014);
\definecolor{c}{rgb}{0,0,0};
\colorlet{c}{natcomp!70};
\draw [c] (6.32308,0.739956) -- (6.32308,0.752771);
\draw [c] (6.32308,0.752771) -- (6.32308,0.765586);
\draw [c] (6.30504,0.752771) -- (6.32308,0.752771);
\draw [c] (6.32308,0.752771) -- (6.34113,0.752771);
\definecolor{c}{rgb}{0,0,0};
\colorlet{c}{natcomp!70};
\draw [c] (6.35917,0.749245) -- (6.35917,0.769531);
\draw [c] (6.35917,0.769531) -- (6.35917,0.789818);
\draw [c] (6.34113,0.769531) -- (6.35917,0.769531);
\draw [c] (6.35917,0.769531) -- (6.37722,0.769531);
\definecolor{c}{rgb}{0,0,0};
\colorlet{c}{natcomp!70};
\draw [c] (6.39526,0.734642) -- (6.39526,0.745339);
\draw [c] (6.39526,0.745339) -- (6.39526,0.756036);
\draw [c] (6.37722,0.745339) -- (6.39526,0.745339);
\draw [c] (6.39526,0.745339) -- (6.41331,0.745339);
\definecolor{c}{rgb}{0,0,0};
\colorlet{c}{natcomp!70};
\draw [c] (6.43135,0.749697) -- (6.43135,0.770462);
\draw [c] (6.43135,0.770462) -- (6.43135,0.791227);
\draw [c] (6.41331,0.770462) -- (6.43135,0.770462);
\draw [c] (6.43135,0.770462) -- (6.4494,0.770462);
\definecolor{c}{rgb}{0,0,0};
\colorlet{c}{natcomp!70};
\draw [c] (6.46744,0.741741) -- (6.46744,0.758925);
\draw [c] (6.46744,0.758925) -- (6.46744,0.77611);
\draw [c] (6.4494,0.758925) -- (6.46744,0.758925);
\draw [c] (6.46744,0.758925) -- (6.48548,0.758925);
\definecolor{c}{rgb}{0,0,0};
\colorlet{c}{natcomp!70};
\draw [c] (6.50353,0.741076) -- (6.50353,0.756664);
\draw [c] (6.50353,0.756664) -- (6.50353,0.772252);
\draw [c] (6.48548,0.756664) -- (6.50353,0.756664);
\draw [c] (6.50353,0.756664) -- (6.52157,0.756664);
\definecolor{c}{rgb}{0,0,0};
\colorlet{c}{natcomp!70};
\draw [c] (6.53962,0.740597) -- (6.53962,0.755326);
\draw [c] (6.53962,0.755326) -- (6.53962,0.770055);
\draw [c] (6.52157,0.755326) -- (6.53962,0.755326);
\draw [c] (6.53962,0.755326) -- (6.55766,0.755326);
\definecolor{c}{rgb}{0,0,0};
\colorlet{c}{natcomp!70};
\draw [c] (6.57571,0.741224) -- (6.57571,0.757269);
\draw [c] (6.57571,0.757269) -- (6.57571,0.773313);
\draw [c] (6.55766,0.757269) -- (6.57571,0.757269);
\draw [c] (6.57571,0.757269) -- (6.59375,0.757269);
\definecolor{c}{rgb}{0,0,0};
\colorlet{c}{natcomp!70};
\draw [c] (6.61179,0.734641) -- (6.61179,0.73465);
\draw [c] (6.61179,0.73465) -- (6.61179,0.734658);
\draw [c] (6.59375,0.73465) -- (6.61179,0.73465);
\draw [c] (6.61179,0.73465) -- (6.62984,0.73465);
\definecolor{c}{rgb}{0,0,0};
\colorlet{c}{natcomp!70};
\draw [c] (6.64788,0.748451) -- (6.64788,0.767356);
\draw [c] (6.64788,0.767356) -- (6.64788,0.786261);
\draw [c] (6.62984,0.767356) -- (6.64788,0.767356);
\draw [c] (6.64788,0.767356) -- (6.66593,0.767356);
\definecolor{c}{rgb}{0,0,0};
\colorlet{c}{natcomp!70};
\draw [c] (6.68397,0.734651) -- (6.68397,0.745051);
\draw [c] (6.68397,0.745051) -- (6.68397,0.755451);
\draw [c] (6.66593,0.745051) -- (6.68397,0.745051);
\draw [c] (6.68397,0.745051) -- (6.70202,0.745051);
\definecolor{c}{rgb}{0,0,0};
\colorlet{c}{natcomp!70};
\draw [c] (6.72006,0.747383) -- (6.72006,0.765114);
\draw [c] (6.72006,0.765114) -- (6.72006,0.782845);
\draw [c] (6.70202,0.765114) -- (6.72006,0.765114);
\draw [c] (6.72006,0.765114) -- (6.7381,0.765114);
\definecolor{c}{rgb}{0,0,0};
\colorlet{c}{natcomp!70};
\draw [c] (6.75615,0.734642) -- (6.75615,0.744194);
\draw [c] (6.75615,0.744194) -- (6.75615,0.753745);
\draw [c] (6.7381,0.744194) -- (6.75615,0.744194);
\draw [c] (6.75615,0.744194) -- (6.77419,0.744194);
\definecolor{c}{rgb}{0,0,0};
\colorlet{c}{natcomp!70};
\draw [c] (6.79224,0.745364) -- (6.79224,0.760001);
\draw [c] (6.79224,0.760001) -- (6.79224,0.774637);
\draw [c] (6.77419,0.760001) -- (6.79224,0.760001);
\draw [c] (6.79224,0.760001) -- (6.81028,0.760001);
\definecolor{c}{rgb}{0,0,0};
\colorlet{c}{natcomp!70};
\draw [c] (6.82833,0.748794) -- (6.82833,0.768589);
\draw [c] (6.82833,0.768589) -- (6.82833,0.788384);
\draw [c] (6.81028,0.768589) -- (6.82833,0.768589);
\draw [c] (6.82833,0.768589) -- (6.84637,0.768589);
\definecolor{c}{rgb}{0,0,0};
\colorlet{c}{natcomp!70};
\draw [c] (6.86442,0.734632) -- (6.86442,0.744183);
\draw [c] (6.86442,0.744183) -- (6.86442,0.753735);
\draw [c] (6.84637,0.744183) -- (6.86442,0.744183);
\draw [c] (6.86442,0.744183) -- (6.88246,0.744183);
\definecolor{c}{rgb}{0,0,0};
\colorlet{c}{natcomp!70};
\draw [c] (6.9005,0.74204) -- (6.9005,0.761487);
\draw [c] (6.9005,0.761487) -- (6.9005,0.780933);
\draw [c] (6.88246,0.761487) -- (6.9005,0.761487);
\draw [c] (6.9005,0.761487) -- (6.91855,0.761487);
\definecolor{c}{rgb}{0,0,0};
\colorlet{c}{natcomp!70};
\draw [c] (6.93659,0.734641) -- (6.93659,0.745338);
\draw [c] (6.93659,0.745338) -- (6.93659,0.756035);
\draw [c] (6.91855,0.745338) -- (6.93659,0.745338);
\draw [c] (6.93659,0.745338) -- (6.95464,0.745338);
\definecolor{c}{rgb}{0,0,0};
\colorlet{c}{natcomp!70};
\draw [c] (6.97268,0.734636) -- (6.97268,0.734649);
\draw [c] (6.97268,0.734649) -- (6.97268,0.734662);
\draw [c] (6.95464,0.734649) -- (6.97268,0.734649);
\draw [c] (6.97268,0.734649) -- (6.99073,0.734649);
\definecolor{c}{rgb}{0,0,0};
\colorlet{c}{natcomp!70};
\draw [c] (7.00877,0.743265) -- (7.00877,0.76582);
\draw [c] (7.00877,0.76582) -- (7.00877,0.788375);
\draw [c] (6.99073,0.76582) -- (7.00877,0.76582);
\draw [c] (7.00877,0.76582) -- (7.02681,0.76582);
\definecolor{c}{rgb}{0,0,0};
\colorlet{c}{natcomp!70};
\draw [c] (7.04486,0.740152) -- (7.04486,0.753784);
\draw [c] (7.04486,0.753784) -- (7.04486,0.767416);
\draw [c] (7.02681,0.753784) -- (7.04486,0.753784);
\draw [c] (7.04486,0.753784) -- (7.0629,0.753784);
\definecolor{c}{rgb}{0,0,0};
\colorlet{c}{natcomp!70};
\draw [c] (7.08095,0.73465) -- (7.08095,0.746868);
\draw [c] (7.08095,0.746868) -- (7.08095,0.759085);
\draw [c] (7.0629,0.746868) -- (7.08095,0.746868);
\draw [c] (7.08095,0.746868) -- (7.09899,0.746868);
\definecolor{c}{rgb}{0,0,0};
\colorlet{c}{natcomp!70};
\draw [c] (7.11704,0.734641) -- (7.11704,0.744192);
\draw [c] (7.11704,0.744192) -- (7.11704,0.753744);
\draw [c] (7.09899,0.744192) -- (7.11704,0.744192);
\draw [c] (7.11704,0.744192) -- (7.13508,0.744192);
\definecolor{c}{rgb}{0,0,0};
\colorlet{c}{natcomp!70};
\draw [c] (7.15312,0.734653) -- (7.15312,0.746265);
\draw [c] (7.15312,0.746265) -- (7.15312,0.757876);
\draw [c] (7.13508,0.746265) -- (7.15312,0.746265);
\draw [c] (7.15312,0.746265) -- (7.17117,0.746265);
\definecolor{c}{rgb}{0,0,0};
\colorlet{c}{natcomp!70};
\draw [c] (7.18921,0.734635) -- (7.18921,0.734641);
\draw [c] (7.18921,0.734641) -- (7.18921,0.734648);
\draw [c] (7.17117,0.734641) -- (7.18921,0.734641);
\draw [c] (7.18921,0.734641) -- (7.20726,0.734641);
\definecolor{c}{rgb}{0,0,0};
\colorlet{c}{natcomp!70};
\draw [c] (7.2253,0.734641) -- (7.2253,0.760185);
\draw [c] (7.2253,0.760185) -- (7.2253,0.78573);
\draw [c] (7.20726,0.760185) -- (7.2253,0.760185);
\draw [c] (7.2253,0.760185) -- (7.24335,0.760185);
\definecolor{c}{rgb}{0,0,0};
\colorlet{c}{natcomp!70};
\draw [c] (7.26139,0.734632) -- (7.26139,0.734637);
\draw [c] (7.26139,0.734637) -- (7.26139,0.734642);
\draw [c] (7.24335,0.734637) -- (7.26139,0.734637);
\draw [c] (7.26139,0.734637) -- (7.27944,0.734637);
\definecolor{c}{rgb}{0,0,0};
\colorlet{c}{natcomp!70};
\draw [c] (7.29748,0.734642) -- (7.29748,0.745339);
\draw [c] (7.29748,0.745339) -- (7.29748,0.756036);
\draw [c] (7.27944,0.745339) -- (7.29748,0.745339);
\draw [c] (7.29748,0.745339) -- (7.31552,0.745339);
\definecolor{c}{rgb}{0,0,0};
\colorlet{c}{natcomp!70};
\draw [c] (7.33357,0.734637) -- (7.33357,0.745037);
\draw [c] (7.33357,0.745037) -- (7.33357,0.755437);
\draw [c] (7.31552,0.745037) -- (7.33357,0.745037);
\draw [c] (7.33357,0.745037) -- (7.35161,0.745037);
\definecolor{c}{rgb}{0,0,0};
\colorlet{c}{natcomp!70};
\draw [c] (7.36966,0.734641) -- (7.36966,0.746252);
\draw [c] (7.36966,0.746252) -- (7.36966,0.757864);
\draw [c] (7.35161,0.746252) -- (7.36966,0.746252);
\draw [c] (7.36966,0.746252) -- (7.3877,0.746252);
\definecolor{c}{rgb}{0,0,0};
\colorlet{c}{natcomp!70};
\draw [c] (7.40575,0.734656) -- (7.40575,0.743718);
\draw [c] (7.40575,0.743718) -- (7.40575,0.752779);
\draw [c] (7.3877,0.743718) -- (7.40575,0.743718);
\draw [c] (7.40575,0.743718) -- (7.42379,0.743718);
\definecolor{c}{rgb}{0,0,0};
\colorlet{c}{natcomp!70};
\draw [c] (7.44183,0.734632) -- (7.44183,0.734637);
\draw [c] (7.44183,0.734637) -- (7.44183,0.734641);
\draw [c] (7.42379,0.734637) -- (7.44183,0.734637);
\draw [c] (7.44183,0.734637) -- (7.45988,0.734637);
\definecolor{c}{rgb}{0,0,0};
\colorlet{c}{natcomp!70};
\draw [c] (7.58619,0.734637) -- (7.58619,0.73465);
\draw [c] (7.58619,0.73465) -- (7.58619,0.734663);
\draw [c] (7.56815,0.73465) -- (7.58619,0.73465);
\draw [c] (7.58619,0.73465) -- (7.60423,0.73465);
\definecolor{c}{rgb}{0,0,0};
\colorlet{c}{natcomp!70};
\draw [c] (7.62228,0.740087) -- (7.62228,0.753488);
\draw [c] (7.62228,0.753488) -- (7.62228,0.766888);
\draw [c] (7.60423,0.753488) -- (7.62228,0.753488);
\draw [c] (7.62228,0.753488) -- (7.64032,0.753488);
\definecolor{c}{rgb}{0,0,0};
\colorlet{c}{natcomp!70};
\draw [c] (7.73054,0.740987) -- (7.73054,0.756833);
\draw [c] (7.73054,0.756833) -- (7.73054,0.772678);
\draw [c] (7.7125,0.756833) -- (7.73054,0.756833);
\draw [c] (7.73054,0.756833) -- (7.74859,0.756833);
\definecolor{c}{rgb}{0,0,0};
\colorlet{c}{natcomp!70};
\draw [c] (7.76663,0.74081) -- (7.76663,0.755729);
\draw [c] (7.76663,0.755729) -- (7.76663,0.770649);
\draw [c] (7.74859,0.755729) -- (7.76663,0.755729);
\draw [c] (7.76663,0.755729) -- (7.78468,0.755729);
\definecolor{c}{rgb}{0,0,0};
\colorlet{c}{natcomp!70};
\draw [c] (7.80272,0.734636) -- (7.80272,0.734645);
\draw [c] (7.80272,0.734645) -- (7.80272,0.734655);
\draw [c] (7.78468,0.734645) -- (7.80272,0.734645);
\draw [c] (7.80272,0.734645) -- (7.82077,0.734645);
\definecolor{c}{rgb}{0,0,0};
\colorlet{c}{natcomp!70};
\draw [c] (7.83881,0.734638) -- (7.83881,0.734652);
\draw [c] (7.83881,0.734652) -- (7.83881,0.734666);
\draw [c] (7.82077,0.734652) -- (7.83881,0.734652);
\draw [c] (7.83881,0.734652) -- (7.85685,0.734652);
\definecolor{c}{rgb}{0,0,0};
\colorlet{c}{natcomp!70};
\draw [c] (7.8749,0.734638) -- (7.8749,0.744189);
\draw [c] (7.8749,0.744189) -- (7.8749,0.75374);
\draw [c] (7.85685,0.744189) -- (7.8749,0.744189);
\draw [c] (7.8749,0.744189) -- (7.89294,0.744189);
\definecolor{c}{rgb}{0,0,0};
\colorlet{c}{natcomp!70};
\draw [c] (7.91099,0.734635) -- (7.91099,0.734642);
\draw [c] (7.91099,0.734642) -- (7.91099,0.734649);
\draw [c] (7.89294,0.734642) -- (7.91099,0.734642);
\draw [c] (7.91099,0.734642) -- (7.92903,0.734642);
\definecolor{c}{rgb}{0,0,0};
\colorlet{c}{natcomp!70};
\draw [c] (7.94708,0.734632) -- (7.94708,0.746244);
\draw [c] (7.94708,0.746244) -- (7.94708,0.757855);
\draw [c] (7.92903,0.746244) -- (7.94708,0.746244);
\draw [c] (7.94708,0.746244) -- (7.96512,0.746244);
\definecolor{c}{rgb}{0,0,0};
\colorlet{c}{natcomp!70};
\draw [c] (7.98317,0.734657) -- (7.98317,0.745354);
\draw [c] (7.98317,0.745354) -- (7.98317,0.756051);
\draw [c] (7.96512,0.745354) -- (7.98317,0.745354);
\draw [c] (7.98317,0.745354) -- (8.00121,0.745354);
\definecolor{c}{rgb}{0,0,0};
\colorlet{c}{natcomp!70};
\draw [c] (8.01925,0.734632) -- (8.01925,0.734638);
\draw [c] (8.01925,0.734638) -- (8.01925,0.734644);
\draw [c] (8.00121,0.734638) -- (8.01925,0.734638);
\draw [c] (8.01925,0.734638) -- (8.0373,0.734638);
\definecolor{c}{rgb}{0,0,0};
\colorlet{c}{natcomp!70};
\draw [c] (8.05534,0.734632) -- (8.05534,0.734637);
\draw [c] (8.05534,0.734637) -- (8.05534,0.734642);
\draw [c] (8.0373,0.734637) -- (8.05534,0.734637);
\draw [c] (8.05534,0.734637) -- (8.07339,0.734637);
\definecolor{c}{rgb}{0,0,0};
\colorlet{c}{natcomp!70};
\draw [c] (8.09143,0.734632) -- (8.09143,0.7473);
\draw [c] (8.09143,0.7473) -- (8.09143,0.759968);
\draw [c] (8.07339,0.7473) -- (8.09143,0.7473);
\draw [c] (8.09143,0.7473) -- (8.10948,0.7473);
\definecolor{c}{rgb}{0,0,0};
\colorlet{c}{natcomp!70};
\draw [c] (8.16361,0.734632) -- (8.16361,0.743694);
\draw [c] (8.16361,0.743694) -- (8.16361,0.752755);
\draw [c] (8.14556,0.743694) -- (8.16361,0.743694);
\draw [c] (8.16361,0.743694) -- (8.18165,0.743694);
\definecolor{c}{rgb}{0,0,0};
\colorlet{c}{natcomp!70};
\draw [c] (8.23579,0.734632) -- (8.23579,0.734637);
\draw [c] (8.23579,0.734637) -- (8.23579,0.734642);
\draw [c] (8.21774,0.734637) -- (8.23579,0.734637);
\draw [c] (8.23579,0.734637) -- (8.25383,0.734637);
\definecolor{c}{rgb}{0,0,0};
\colorlet{c}{natcomp!70};
\draw [c] (8.27188,0.734637) -- (8.27188,0.745156);
\draw [c] (8.27188,0.745156) -- (8.27188,0.755676);
\draw [c] (8.25383,0.745156) -- (8.27188,0.745156);
\draw [c] (8.27188,0.745156) -- (8.28992,0.745156);
\definecolor{c}{rgb}{0,0,0};
\colorlet{c}{natcomp!70};
\draw [c] (8.34405,0.734651) -- (8.34405,0.747294);
\draw [c] (8.34405,0.747294) -- (8.34405,0.759937);
\draw [c] (8.32601,0.747294) -- (8.34405,0.747294);
\draw [c] (8.34405,0.747294) -- (8.3621,0.747294);
\definecolor{c}{rgb}{0,0,0};
\colorlet{c}{natcomp!70};
\draw [c] (8.45232,0.734632) -- (8.45232,0.734637);
\draw [c] (8.45232,0.734637) -- (8.45232,0.734642);
\draw [c] (8.43427,0.734637) -- (8.45232,0.734637);
\draw [c] (8.45232,0.734637) -- (8.47036,0.734637);
\definecolor{c}{rgb}{0,0,0};
\colorlet{c}{natcomp!70};
\draw [c] (8.48841,0.734637) -- (8.48841,0.744188);
\draw [c] (8.48841,0.744188) -- (8.48841,0.75374);
\draw [c] (8.47036,0.744188) -- (8.48841,0.744188);
\draw [c] (8.48841,0.744188) -- (8.50645,0.744188);
\definecolor{c}{rgb}{0,0,0};
\colorlet{c}{natcomp!70};
\draw [c] (8.74103,0.734632) -- (8.74103,0.734637);
\draw [c] (8.74103,0.734637) -- (8.74103,0.734641);
\draw [c] (8.72298,0.734637) -- (8.74103,0.734637);
\draw [c] (8.74103,0.734637) -- (8.75907,0.734637);
\definecolor{c}{rgb}{0,0,0};
\colorlet{c}{natcomp!70};
\draw [c] (8.77712,0.734632) -- (8.77712,0.734636);
\draw [c] (8.77712,0.734636) -- (8.77712,0.73464);
\draw [c] (8.75907,0.734636) -- (8.77712,0.734636);
\draw [c] (8.77712,0.734636) -- (8.79516,0.734636);
\definecolor{c}{rgb}{0,0,0};
\colorlet{c}{natcomp!70};
\draw [c] (8.81321,0.734632) -- (8.81321,0.734637);
\draw [c] (8.81321,0.734637) -- (8.81321,0.734642);
\draw [c] (8.79516,0.734637) -- (8.81321,0.734637);
\draw [c] (8.81321,0.734637) -- (8.83125,0.734637);
\definecolor{c}{rgb}{0,0,0};
\colorlet{c}{natcomp!70};
\draw [c] (8.88538,0.734632) -- (8.88538,0.743082);
\draw [c] (8.88538,0.743082) -- (8.88538,0.751533);
\draw [c] (8.86734,0.743082) -- (8.88538,0.743082);
\draw [c] (8.88538,0.743082) -- (8.90343,0.743082);
\definecolor{c}{rgb}{0,0,0};
\colorlet{c}{natcomp!70};
\draw [c] (8.95756,0.734632) -- (8.95756,0.734638);
\draw [c] (8.95756,0.734638) -- (8.95756,0.734644);
\draw [c] (8.93952,0.734638) -- (8.95756,0.734638);
\draw [c] (8.95756,0.734638) -- (8.97561,0.734638);
\definecolor{c}{rgb}{0,0,0};
\colorlet{c}{natcomp!70};
\draw [c] (9.10192,0.734632) -- (9.10192,0.734637);
\draw [c] (9.10192,0.734637) -- (9.10192,0.734641);
\draw [c] (9.08387,0.734637) -- (9.10192,0.734637);
\draw [c] (9.10192,0.734637) -- (9.11996,0.734637);
\definecolor{c}{rgb}{0,0,0};
\colorlet{c}{natcomp!70};
\draw [c] (9.21018,0.734632) -- (9.21018,0.753592);
\draw [c] (9.21018,0.753592) -- (9.21018,0.772551);
\draw [c] (9.19214,0.753592) -- (9.21018,0.753592);
\draw [c] (9.21018,0.753592) -- (9.22823,0.753592);
\definecolor{c}{rgb}{0,0,0};
\colorlet{c}{natcomp!70};
\draw [c] (9.28236,0.734632) -- (9.28236,0.745329);
\draw [c] (9.28236,0.745329) -- (9.28236,0.756027);
\draw [c] (9.26431,0.745329) -- (9.28236,0.745329);
\draw [c] (9.28236,0.745329) -- (9.3004,0.745329);
\definecolor{c}{rgb}{0,0,0};
\colorlet{c}{natcomp!70};
\draw [c] (9.31845,0.734632) -- (9.31845,0.73464);
\draw [c] (9.31845,0.73464) -- (9.31845,0.734648);
\draw [c] (9.3004,0.73464) -- (9.31845,0.73464);
\draw [c] (9.31845,0.73464) -- (9.33649,0.73464);
\definecolor{c}{rgb}{0,0,0};
\colorlet{c}{natcomp!70};
\draw [c] (9.4628,0.734632) -- (9.4628,0.746244);
\draw [c] (9.4628,0.746244) -- (9.4628,0.757855);
\draw [c] (9.44476,0.746244) -- (9.4628,0.746244);
\draw [c] (9.4628,0.746244) -- (9.48085,0.746244);
\definecolor{c}{rgb}{0,0,0};
\colorlet{c}{natcomp!70};
\draw [c] (9.49889,0.734632) -- (9.49889,0.744183);
\draw [c] (9.49889,0.744183) -- (9.49889,0.753735);
\draw [c] (9.48085,0.744183) -- (9.49889,0.744183);
\draw [c] (9.49889,0.744183) -- (9.51694,0.744183);
\definecolor{c}{rgb}{0,0,0};
\colorlet{c}{natcomp!70};
\draw [c] (9.57107,0.734632) -- (9.57107,0.745032);
\draw [c] (9.57107,0.745032) -- (9.57107,0.755432);
\draw [c] (9.55302,0.745032) -- (9.57107,0.745032);
\draw [c] (9.57107,0.745032) -- (9.58911,0.745032);
\definecolor{c}{rgb}{0,0,0};
\colorlet{c}{natcomp!70};
\draw [c] (9.67933,0.734632) -- (9.67933,0.734636);
\draw [c] (9.67933,0.734636) -- (9.67933,0.73464);
\draw [c] (9.66129,0.734636) -- (9.67933,0.734636);
\draw [c] (9.67933,0.734636) -- (9.69738,0.734636);
\definecolor{c}{rgb}{0,0,0};
\colorlet{c}{natcomp!70};
\draw [c] (9.7876,0.734632) -- (9.7876,0.734636);
\draw [c] (9.7876,0.734636) -- (9.7876,0.73464);
\draw [c] (9.76956,0.734636) -- (9.7876,0.734636);
\draw [c] (9.7876,0.734636) -- (9.80564,0.734636);
\definecolor{c}{rgb}{0,0,0};
\colorlet{c}{natcomp!70};
\draw [c] (9.82369,0.734632) -- (9.82369,0.734636);
\draw [c] (9.82369,0.734636) -- (9.82369,0.73464);
\draw [c] (9.80564,0.734636) -- (9.82369,0.734636);
\draw [c] (9.82369,0.734636) -- (9.84173,0.734636);
\definecolor{c}{rgb}{0,0,0};
\colorlet{c}{natcomp!70};
\draw [c] (9.93196,0.734632) -- (9.93196,0.746244);
\draw [c] (9.93196,0.746244) -- (9.93196,0.757855);
\draw [c] (9.91391,0.746244) -- (9.93196,0.746244);
\draw [c] (9.93196,0.746244) -- (9.95,0.746244);
\definecolor{c}{rgb}{0,0,0};
\draw [anchor=base west] (6.09599,5.83614) node[color=c, rotate=0]{ATLAS MC};
\colorlet{c}{natgreen};
\draw [c] (5.14004,5.96347) -- (5.92729,5.96347);
\draw [c] (5.53367,5.7937) -- (5.53367,6.13324);
\definecolor{c}{rgb}{0,0,0};
\draw [anchor=base west] (6.09599,5.27024) node[color=c, rotate=0]{CalcHEP MC};
\colorlet{c}{natcomp!70};
\draw [c] (5.14004,5.39756) -- (5.92729,5.39756);
\draw [c] (5.53367,5.22779) -- (5.53367,5.56734);
\end{tikzpicture}

\end{infilsf}
\end{minipage}
\begin{minipage}[b]{.49\textwidth}
\subcaption{Before application of mapping function.\label{etpv}}
\end{minipage}\hfill
\begin{minipage}[b]{.49\textwidth}
\subcaption{After application of mapping function.\label{etmap}}
\end{minipage}
\caption{The distribution of $E_T^\text{iso}$ in the \atlas{} MC set compared with the distribution in the CalcHEP MC set. In \subcaptionref{etmap}, a mapping function which applies a scale and offset to the values for the CalcHEP MC has been applied.}
\end{figure}

These distributions are now very close to being identical. We correct the remaining discrepancy by reweighting the CalcHEP sample [uncertainty, effect on Mgg...?]

Finally, the CalcHEP sample only included events produced by the tree level process, whereas the \atlas{} sample also includes the contribution from the box diagram shown in fig.~\ref{hiorder}. So, to meaningfully compare the two, we must know the contribution from the box diagram. This, we glean from another \atlas{} MC sample\footnote{that would be that appendix again}, which provides a $M_{\gamma\gamma}$ distribution illustrated in fig.~\ref{boxmgg}. As with the estimated background, this distribution has insufficient statistics to accurately represent the shape of the distribution in the interesting region above 1\,000 GeV, forcing us to extrapolate the shape of this distribution as well.

\begin{figure}[htp]
\begin{minipage}[b]{.69\textwidth}
\begin{infilsf} \tiny
\begin{tikzpicture}[x=.092\textwidth,y=.092\textwidth]
\pgfdeclareplotmark{cross} {
\pgfpathmoveto{\pgfpoint{-0.3\pgfplotmarksize}{\pgfplotmarksize}}
\pgfpathlineto{\pgfpoint{+0.3\pgfplotmarksize}{\pgfplotmarksize}}
\pgfpathlineto{\pgfpoint{+0.3\pgfplotmarksize}{0.3\pgfplotmarksize}}
\pgfpathlineto{\pgfpoint{+1\pgfplotmarksize}{0.3\pgfplotmarksize}}
\pgfpathlineto{\pgfpoint{+1\pgfplotmarksize}{-0.3\pgfplotmarksize}}
\pgfpathlineto{\pgfpoint{+0.3\pgfplotmarksize}{-0.3\pgfplotmarksize}}
\pgfpathlineto{\pgfpoint{+0.3\pgfplotmarksize}{-1.\pgfplotmarksize}}
\pgfpathlineto{\pgfpoint{-0.3\pgfplotmarksize}{-1.\pgfplotmarksize}}
\pgfpathlineto{\pgfpoint{-0.3\pgfplotmarksize}{-0.3\pgfplotmarksize}}
\pgfpathlineto{\pgfpoint{-1.\pgfplotmarksize}{-0.3\pgfplotmarksize}}
\pgfpathlineto{\pgfpoint{-1.\pgfplotmarksize}{0.3\pgfplotmarksize}}
\pgfpathlineto{\pgfpoint{-0.3\pgfplotmarksize}{0.3\pgfplotmarksize}}
\pgfpathclose
\pgfusepathqstroke
}
\pgfdeclareplotmark{cross*} {
\pgfpathmoveto{\pgfpoint{-0.3\pgfplotmarksize}{\pgfplotmarksize}}
\pgfpathlineto{\pgfpoint{+0.3\pgfplotmarksize}{\pgfplotmarksize}}
\pgfpathlineto{\pgfpoint{+0.3\pgfplotmarksize}{0.3\pgfplotmarksize}}
\pgfpathlineto{\pgfpoint{+1\pgfplotmarksize}{0.3\pgfplotmarksize}}
\pgfpathlineto{\pgfpoint{+1\pgfplotmarksize}{-0.3\pgfplotmarksize}}
\pgfpathlineto{\pgfpoint{+0.3\pgfplotmarksize}{-0.3\pgfplotmarksize}}
\pgfpathlineto{\pgfpoint{+0.3\pgfplotmarksize}{-1.\pgfplotmarksize}}
\pgfpathlineto{\pgfpoint{-0.3\pgfplotmarksize}{-1.\pgfplotmarksize}}
\pgfpathlineto{\pgfpoint{-0.3\pgfplotmarksize}{-0.3\pgfplotmarksize}}
\pgfpathlineto{\pgfpoint{-1.\pgfplotmarksize}{-0.3\pgfplotmarksize}}
\pgfpathlineto{\pgfpoint{-1.\pgfplotmarksize}{0.3\pgfplotmarksize}}
\pgfpathlineto{\pgfpoint{-0.3\pgfplotmarksize}{0.3\pgfplotmarksize}}
\pgfpathclose
\pgfusepathqfillstroke
}
\pgfdeclareplotmark{newstar} {
\pgfpathmoveto{\pgfqpoint{0pt}{\pgfplotmarksize}}
\pgfpathlineto{\pgfqpointpolar{44}{0.5\pgfplotmarksize}}
\pgfpathlineto{\pgfqpointpolar{18}{\pgfplotmarksize}}
\pgfpathlineto{\pgfqpointpolar{-20}{0.5\pgfplotmarksize}}
\pgfpathlineto{\pgfqpointpolar{-54}{\pgfplotmarksize}}
\pgfpathlineto{\pgfqpointpolar{-90}{0.5\pgfplotmarksize}}
\pgfpathlineto{\pgfqpointpolar{234}{\pgfplotmarksize}}
\pgfpathlineto{\pgfqpointpolar{198}{0.5\pgfplotmarksize}}
\pgfpathlineto{\pgfqpointpolar{162}{\pgfplotmarksize}}
\pgfpathlineto{\pgfqpointpolar{134}{0.5\pgfplotmarksize}}
\pgfpathclose
\pgfusepathqstroke
}
\pgfdeclareplotmark{newstar*} {
\pgfpathmoveto{\pgfqpoint{0pt}{\pgfplotmarksize}}
\pgfpathlineto{\pgfqpointpolar{44}{0.5\pgfplotmarksize}}
\pgfpathlineto{\pgfqpointpolar{18}{\pgfplotmarksize}}
\pgfpathlineto{\pgfqpointpolar{-20}{0.5\pgfplotmarksize}}
\pgfpathlineto{\pgfqpointpolar{-54}{\pgfplotmarksize}}
\pgfpathlineto{\pgfqpointpolar{-90}{0.5\pgfplotmarksize}}
\pgfpathlineto{\pgfqpointpolar{234}{\pgfplotmarksize}}
\pgfpathlineto{\pgfqpointpolar{198}{0.5\pgfplotmarksize}}
\pgfpathlineto{\pgfqpointpolar{162}{\pgfplotmarksize}}
\pgfpathlineto{\pgfqpointpolar{134}{0.5\pgfplotmarksize}}
\pgfpathclose
\pgfusepathqfillstroke
}
\definecolor{c}{rgb}{1,1,1};
\draw [color=c, fill=c] (0,0) rectangle (10,6.80516);
\draw [color=c, fill=c] (1,0.680516) rectangle (9.95,6.73711);
\definecolor{c}{rgb}{0,0,0};
\draw [c] (1,0.680516) -- (1,6.73711) -- (9.95,6.73711) -- (9.95,0.680516) -- (1,0.680516);
\definecolor{c}{rgb}{1,1,1};
\draw [color=c, fill=c] (1,0.680516) rectangle (9.95,6.73711);
\definecolor{c}{rgb}{0,0,0};
\draw [c] (1,0.680516) -- (1,6.73711) -- (9.95,6.73711) -- (9.95,0.680516) -- (1,0.680516);
\colorlet{c}{natgreen};
\draw [c] (1.6646,0.680516) -- (1.6646,2.3049);
\draw [c] (1.6646,2.3049) -- (1.6646,2.69635);
\draw [c] (1.6203,2.3049) -- (1.6646,2.3049);
\draw [c] (1.6646,2.3049) -- (1.70891,2.3049);
\definecolor{c}{rgb}{0,0,0};
\colorlet{c}{natgreen};
\draw [c] (1.75322,2.38132) -- (1.75322,2.95036);
\draw [c] (1.75322,2.95036) -- (1.75322,3.22799);
\draw [c] (1.70891,2.95036) -- (1.75322,2.95036);
\draw [c] (1.75322,2.95036) -- (1.79752,2.95036);
\definecolor{c}{rgb}{0,0,0};
\colorlet{c}{natgreen};
\draw [c] (1.84183,4.31431) -- (1.84183,4.43557);
\draw [c] (1.84183,4.43557) -- (1.84183,4.53534);
\draw [c] (1.79752,4.43557) -- (1.84183,4.43557);
\draw [c] (1.84183,4.43557) -- (1.88614,4.43557);
\definecolor{c}{rgb}{0,0,0};
\colorlet{c}{natgreen};
\draw [c] (1.93045,6.01844) -- (1.93045,6.04598);
\draw [c] (1.93045,6.04598) -- (1.93045,6.07223);
\draw [c] (1.88614,6.04598) -- (1.93045,6.04598);
\draw [c] (1.93045,6.04598) -- (1.97475,6.04598);
\definecolor{c}{rgb}{0,0,0};
\colorlet{c}{natgreen};
\draw [c] (2.01906,6.4381) -- (2.01906,6.45705);
\draw [c] (2.01906,6.45705) -- (2.01906,6.47539);
\draw [c] (1.97475,6.45705) -- (2.01906,6.45705);
\draw [c] (2.01906,6.45705) -- (2.06337,6.45705);
\definecolor{c}{rgb}{0,0,0};
\colorlet{c}{natgreen};
\draw [c] (2.10767,6.48095) -- (2.10767,6.49923);
\draw [c] (2.10767,6.49923) -- (2.10767,6.51694);
\draw [c] (2.06337,6.49923) -- (2.10767,6.49923);
\draw [c] (2.10767,6.49923) -- (2.15198,6.49923);
\definecolor{c}{rgb}{0,0,0};
\colorlet{c}{natgreen};
\draw [c] (2.19629,6.42817) -- (2.19629,6.44727);
\draw [c] (2.19629,6.44727) -- (2.19629,6.46573);
\draw [c] (2.15198,6.44727) -- (2.19629,6.44727);
\draw [c] (2.19629,6.44727) -- (2.24059,6.44727);
\definecolor{c}{rgb}{0,0,0};
\colorlet{c}{natgreen};
\draw [c] (2.2849,6.34203) -- (2.2849,6.36277);
\draw [c] (2.2849,6.36277) -- (2.2849,6.38277);
\draw [c] (2.24059,6.36277) -- (2.2849,6.36277);
\draw [c] (2.2849,6.36277) -- (2.32921,6.36277);
\definecolor{c}{rgb}{0,0,0};
\colorlet{c}{natgreen};
\draw [c] (2.37351,6.22705) -- (2.37351,6.24993);
\draw [c] (2.37351,6.24993) -- (2.37351,6.27192);
\draw [c] (2.32921,6.24993) -- (2.37351,6.24993);
\draw [c] (2.37351,6.24993) -- (2.41782,6.24993);
\definecolor{c}{rgb}{0,0,0};
\colorlet{c}{natgreen};
\draw [c] (2.46213,6.10396) -- (2.46213,6.12959);
\draw [c] (2.46213,6.12959) -- (2.46213,6.15411);
\draw [c] (2.41782,6.12959) -- (2.46213,6.12959);
\draw [c] (2.46213,6.12959) -- (2.50644,6.12959);
\definecolor{c}{rgb}{0,0,0};
\colorlet{c}{natgreen};
\draw [c] (2.55074,5.923) -- (2.55074,5.95292);
\draw [c] (2.55074,5.95292) -- (2.55074,5.98134);
\draw [c] (2.50644,5.95292) -- (2.55074,5.95292);
\draw [c] (2.55074,5.95292) -- (2.59505,5.95292);
\definecolor{c}{rgb}{0,0,0};
\colorlet{c}{natgreen};
\draw [c] (2.63936,5.81438) -- (2.63936,5.84748);
\draw [c] (2.63936,5.84748) -- (2.63936,5.87875);
\draw [c] (2.59505,5.84748) -- (2.63936,5.84748);
\draw [c] (2.63936,5.84748) -- (2.68366,5.84748);
\definecolor{c}{rgb}{0,0,0};
\colorlet{c}{natgreen};
\draw [c] (2.72797,5.6363) -- (2.72797,5.67497);
\draw [c] (2.72797,5.67497) -- (2.72797,5.71115);
\draw [c] (2.68366,5.67497) -- (2.72797,5.67497);
\draw [c] (2.72797,5.67497) -- (2.77228,5.67497);
\definecolor{c}{rgb}{0,0,0};
\colorlet{c}{natgreen};
\draw [c] (2.81658,5.53226) -- (2.81658,5.57507);
\draw [c] (2.81658,5.57507) -- (2.81658,5.61487);
\draw [c] (2.77228,5.57507) -- (2.81658,5.57507);
\draw [c] (2.81658,5.57507) -- (2.86089,5.57507);
\definecolor{c}{rgb}{0,0,0};
\colorlet{c}{natgreen};
\draw [c] (2.9052,5.35892) -- (2.9052,5.4081);
\draw [c] (2.9052,5.4081) -- (2.9052,5.45334);
\draw [c] (2.86089,5.4081) -- (2.9052,5.4081);
\draw [c] (2.9052,5.4081) -- (2.94951,5.4081);
\definecolor{c}{rgb}{0,0,0};
\colorlet{c}{natgreen};
\draw [c] (2.99381,5.24072) -- (2.99381,5.29613);
\draw [c] (2.99381,5.29613) -- (2.99381,5.34658);
\draw [c] (2.94951,5.29613) -- (2.99381,5.29613);
\draw [c] (2.99381,5.29613) -- (3.03812,5.29613);
\definecolor{c}{rgb}{0,0,0};
\colorlet{c}{natgreen};
\draw [c] (3.08243,5.21899) -- (3.08243,5.27515);
\draw [c] (3.08243,5.27515) -- (3.08243,5.32623);
\draw [c] (3.03812,5.27515) -- (3.08243,5.27515);
\draw [c] (3.08243,5.27515) -- (3.12673,5.27515);
\definecolor{c}{rgb}{0,0,0};
\colorlet{c}{natgreen};
\draw [c] (3.17104,5.09339) -- (3.17104,5.1552);
\draw [c] (3.17104,5.1552) -- (3.17104,5.21091);
\draw [c] (3.12673,5.1552) -- (3.17104,5.1552);
\draw [c] (3.17104,5.1552) -- (3.21535,5.1552);
\definecolor{c}{rgb}{0,0,0};
\colorlet{c}{natgreen};
\draw [c] (3.25965,4.96656) -- (3.25965,5.03664);
\draw [c] (3.25965,5.03664) -- (3.25965,5.09898);
\draw [c] (3.21535,5.03664) -- (3.25965,5.03664);
\draw [c] (3.25965,5.03664) -- (3.30396,5.03664);
\definecolor{c}{rgb}{0,0,0};
\colorlet{c}{natgreen};
\draw [c] (3.34827,4.84513) -- (3.34827,4.92243);
\draw [c] (3.34827,4.92243) -- (3.34827,4.99041);
\draw [c] (3.30396,4.92243) -- (3.34827,4.92243);
\draw [c] (3.34827,4.92243) -- (3.39257,4.92243);
\definecolor{c}{rgb}{0,0,0};
\colorlet{c}{natgreen};
\draw [c] (3.43688,4.68676) -- (3.43688,4.7774);
\draw [c] (3.43688,4.7774) -- (3.43688,4.85548);
\draw [c] (3.39257,4.7774) -- (3.43688,4.7774);
\draw [c] (3.43688,4.7774) -- (3.48119,4.7774);
\definecolor{c}{rgb}{0,0,0};
\colorlet{c}{natgreen};
\draw [c] (3.5255,4.69232) -- (3.5255,4.78254);
\draw [c] (3.5255,4.78254) -- (3.5255,4.86032);
\draw [c] (3.48119,4.78254) -- (3.5255,4.78254);
\draw [c] (3.5255,4.78254) -- (3.5698,4.78254);
\definecolor{c}{rgb}{0,0,0};
\colorlet{c}{natgreen};
\draw [c] (3.61411,4.44789) -- (3.61411,4.55535);
\draw [c] (3.61411,4.55535) -- (3.61411,4.64559);
\draw [c] (3.5698,4.55535) -- (3.61411,4.55535);
\draw [c] (3.61411,4.55535) -- (3.65842,4.55535);
\definecolor{c}{rgb}{0,0,0};
\colorlet{c}{natgreen};
\draw [c] (3.70272,4.38517) -- (3.70272,4.50101);
\draw [c] (3.70272,4.50101) -- (3.70272,4.59709);
\draw [c] (3.65842,4.50101) -- (3.70272,4.50101);
\draw [c] (3.70272,4.50101) -- (3.74703,4.50101);
\definecolor{c}{rgb}{0,0,0};
\colorlet{c}{natgreen};
\draw [c] (3.79134,4.43828) -- (3.79134,4.54987);
\draw [c] (3.79134,4.54987) -- (3.79134,4.64301);
\draw [c] (3.74703,4.54987) -- (3.79134,4.54987);
\draw [c] (3.79134,4.54987) -- (3.83564,4.54987);
\definecolor{c}{rgb}{0,0,0};
\colorlet{c}{natgreen};
\draw [c] (3.87995,4.17957) -- (3.87995,4.32023);
\draw [c] (3.87995,4.32023) -- (3.87995,4.43275);
\draw [c] (3.83564,4.32023) -- (3.87995,4.32023);
\draw [c] (3.87995,4.32023) -- (3.92426,4.32023);
\definecolor{c}{rgb}{0,0,0};
\colorlet{c}{natgreen};
\draw [c] (3.96856,3.71331) -- (3.96856,3.92083);
\draw [c] (3.96856,3.92083) -- (3.96856,4.07224);
\draw [c] (3.92426,3.92083) -- (3.96856,3.92083);
\draw [c] (3.96856,3.92083) -- (4.01287,3.92083);
\definecolor{c}{rgb}{0,0,0};
\colorlet{c}{natgreen};
\draw [c] (4.05718,3.73488) -- (4.05718,3.93623);
\draw [c] (4.05718,3.93623) -- (4.05718,4.08436);
\draw [c] (4.01287,3.93623) -- (4.05718,3.93623);
\draw [c] (4.05718,3.93623) -- (4.10149,3.93623);
\definecolor{c}{rgb}{0,0,0};
\colorlet{c}{natgreen};
\draw [c] (4.14579,3.98827) -- (4.14579,4.14568);
\draw [c] (4.14579,4.14568) -- (4.14579,4.26864);
\draw [c] (4.10149,4.14568) -- (4.14579,4.14568);
\draw [c] (4.14579,4.14568) -- (4.1901,4.14568);
\definecolor{c}{rgb}{0,0,0};
\colorlet{c}{natgreen};
\draw [c] (4.23441,3.70996) -- (4.23441,3.92222);
\draw [c] (4.23441,3.92222) -- (4.23441,4.07613);
\draw [c] (4.1901,3.92222) -- (4.23441,3.92222);
\draw [c] (4.23441,3.92222) -- (4.27871,3.92222);
\definecolor{c}{rgb}{0,0,0};
\colorlet{c}{natgreen};
\draw [c] (4.32302,3.33856) -- (4.32302,3.6243);
\draw [c] (4.32302,3.6243) -- (4.32302,3.81313);
\draw [c] (4.27871,3.6243) -- (4.32302,3.6243);
\draw [c] (4.32302,3.6243) -- (4.36733,3.6243);
\definecolor{c}{rgb}{0,0,0};
\colorlet{c}{natgreen};
\draw [c] (4.41163,3.32705) -- (4.41163,3.6277);
\draw [c] (4.41163,3.6277) -- (4.41163,3.82286);
\draw [c] (4.36733,3.6277) -- (4.41163,3.6277);
\draw [c] (4.41163,3.6277) -- (4.45594,3.6277);
\definecolor{c}{rgb}{0,0,0};
\colorlet{c}{natgreen};
\draw [c] (4.50025,3.83238) -- (4.50025,4.02204);
\draw [c] (4.50025,4.02204) -- (4.50025,4.16377);
\draw [c] (4.45594,4.02204) -- (4.50025,4.02204);
\draw [c] (4.50025,4.02204) -- (4.54455,4.02204);
\definecolor{c}{rgb}{0,0,0};
\colorlet{c}{natgreen};
\draw [c] (4.58886,3.27564) -- (4.58886,3.5691);
\draw [c] (4.58886,3.5691) -- (4.58886,3.76124);
\draw [c] (4.54455,3.5691) -- (4.58886,3.5691);
\draw [c] (4.58886,3.5691) -- (4.63317,3.5691);
\definecolor{c}{rgb}{0,0,0};
\colorlet{c}{natgreen};
\draw [c] (4.67748,2.74147) -- (4.67748,3.24921);
\draw [c] (4.67748,3.24921) -- (4.67748,3.51219);
\draw [c] (4.63317,3.24921) -- (4.67748,3.24921);
\draw [c] (4.67748,3.24921) -- (4.72178,3.24921);
\definecolor{c}{rgb}{0,0,0};
\colorlet{c}{natgreen};
\draw [c] (4.76609,3.60758) -- (4.76609,3.85031);
\draw [c] (4.76609,3.85031) -- (4.76609,4.01953);
\draw [c] (4.72178,3.85031) -- (4.76609,3.85031);
\draw [c] (4.76609,3.85031) -- (4.8104,3.85031);
\definecolor{c}{rgb}{0,0,0};
\colorlet{c}{natgreen};
\draw [c] (4.94332,2.30513) -- (4.94332,3.00293);
\draw [c] (4.94332,3.00293) -- (4.94332,3.30569);
\draw [c] (4.89901,3.00293) -- (4.94332,3.00293);
\draw [c] (4.94332,3.00293) -- (4.98762,3.00293);
\definecolor{c}{rgb}{0,0,0};
\colorlet{c}{natgreen};
\draw [c] (5.12054,3.19183) -- (5.12054,3.50284);
\draw [c] (5.12054,3.50284) -- (5.12054,3.70225);
\draw [c] (5.07624,3.50284) -- (5.12054,3.50284);
\draw [c] (5.12054,3.50284) -- (5.16485,3.50284);
\definecolor{c}{rgb}{0,0,0};
\colorlet{c}{natgreen};
\draw [c] (5.20916,2.95994) -- (5.20916,3.3768);
\draw [c] (5.20916,3.3768) -- (5.20916,3.61401);
\draw [c] (5.16485,3.3768) -- (5.20916,3.3768);
\draw [c] (5.20916,3.3768) -- (5.25347,3.3768);
\definecolor{c}{rgb}{0,0,0};
\colorlet{c}{natgreen};
\draw [c] (5.29777,2.67687) -- (5.29777,3.18022);
\draw [c] (5.29777,3.18022) -- (5.29777,3.44206);
\draw [c] (5.25347,3.18022) -- (5.29777,3.18022);
\draw [c] (5.29777,3.18022) -- (5.34208,3.18022);
\definecolor{c}{rgb}{0,0,0};
\colorlet{c}{natgreen};
\draw [c] (5.38639,0.680516) -- (5.38639,2.77962);
\draw [c] (5.38639,2.77962) -- (5.38639,3.17107);
\draw [c] (5.34208,2.77962) -- (5.38639,2.77962);
\draw [c] (5.38639,2.77962) -- (5.43069,2.77962);
\definecolor{c}{rgb}{0,0,0};
\colorlet{c}{natgreen};
\draw [c] (5.475,2.32505) -- (5.475,3.01853);
\draw [c] (5.475,3.01853) -- (5.475,3.32055);
\draw [c] (5.43069,3.01853) -- (5.475,3.01853);
\draw [c] (5.475,3.01853) -- (5.51931,3.01853);
\definecolor{c}{rgb}{0,0,0};
\colorlet{c}{natgreen};
\draw [c] (5.56361,0.680516) -- (5.56361,2.25224);
\draw [c] (5.56361,2.25224) -- (5.56361,2.64369);
\draw [c] (5.51931,2.25224) -- (5.56361,2.25224);
\draw [c] (5.56361,2.25224) -- (5.60792,2.25224);
\definecolor{c}{rgb}{0,0,0};
\colorlet{c}{natgreen};
\draw [c] (5.65223,2.72587) -- (5.65223,3.2156);
\draw [c] (5.65223,3.2156) -- (5.65223,3.47388);
\draw [c] (5.60792,3.2156) -- (5.65223,3.2156);
\draw [c] (5.65223,3.2156) -- (5.69653,3.2156);
\definecolor{c}{rgb}{0,0,0};
\colorlet{c}{natgreen};
\draw [c] (5.74084,2.27391) -- (5.74084,2.96766);
\draw [c] (5.74084,2.96766) -- (5.74084,3.26973);
\draw [c] (5.69653,2.96766) -- (5.74084,2.96766);
\draw [c] (5.74084,2.96766) -- (5.78515,2.96766);
\definecolor{c}{rgb}{0,0,0};
\colorlet{c}{natgreen};
\draw [c] (5.82946,0.680516) -- (5.82946,2.51889);
\draw [c] (5.82946,2.51889) -- (5.82946,2.91034);
\draw [c] (5.78515,2.51889) -- (5.82946,2.51889);
\draw [c] (5.82946,2.51889) -- (5.87376,2.51889);
\definecolor{c}{rgb}{0,0,0};
\colorlet{c}{natgreen};
\draw [c] (6.00668,0.680516) -- (6.00668,2.62777);
\draw [c] (6.00668,2.62777) -- (6.00668,3.01922);
\draw [c] (5.96238,2.62777) -- (6.00668,2.62777);
\draw [c] (6.00668,2.62777) -- (6.05099,2.62777);
\definecolor{c}{rgb}{0,0,0};
\colorlet{c}{natgreen};
\draw [c] (6.0953,0.680516) -- (6.0953,2.62558);
\draw [c] (6.0953,2.62558) -- (6.0953,3.01404);
\draw [c] (6.05099,2.62558) -- (6.0953,2.62558);
\draw [c] (6.0953,2.62558) -- (6.1396,2.62558);
\definecolor{c}{rgb}{0,0,0};
\colorlet{c}{natgreen};
\draw [c] (6.36114,0.680516) -- (6.36114,1.60554);
\draw [c] (6.36114,1.60554) -- (6.36114,1.99699);
\draw [c] (6.31683,1.60554) -- (6.36114,1.60554);
\draw [c] (6.36114,1.60554) -- (6.40545,1.60554);
\definecolor{c}{rgb}{0,0,0};
\colorlet{c}{natgreen};
\draw [c] (6.62698,0.680516) -- (6.62698,2.65744);
\draw [c] (6.62698,2.65744) -- (6.62698,3.04889);
\draw [c] (6.58267,2.65744) -- (6.62698,2.65744);
\draw [c] (6.62698,2.65744) -- (6.67129,2.65744);
\definecolor{c}{rgb}{0,0,0};
\colorlet{c}{natgreen};
\draw [c] (6.71559,2.78195) -- (6.71559,3.27658);
\draw [c] (6.71559,3.27658) -- (6.71559,3.53615);
\draw [c] (6.67129,3.27658) -- (6.71559,3.27658);
\draw [c] (6.71559,3.27658) -- (6.7599,3.27658);
\definecolor{c}{rgb}{0,0,0};
\colorlet{c}{natgreen};
\draw [c] (6.80421,0.680516) -- (6.80421,2.68178);
\draw [c] (6.80421,2.68178) -- (6.80421,3.07323);
\draw [c] (6.7599,2.68178) -- (6.80421,2.68178);
\draw [c] (6.80421,2.68178) -- (6.84852,2.68178);
\definecolor{c}{rgb}{0,0,0};
\colorlet{c}{natgreen};
\draw [c] (7.07005,0.680516) -- (7.07005,2.5647);
\draw [c] (7.07005,2.5647) -- (7.07005,2.95615);
\draw [c] (7.02574,2.5647) -- (7.07005,2.5647);
\draw [c] (7.07005,2.5647) -- (7.11436,2.5647);
\definecolor{c}{rgb}{0,0,0};
\colorlet{c}{natgreen};
\draw [c] (7.4245,0.680516) -- (7.4245,2.60897);
\draw [c] (7.4245,2.60897) -- (7.4245,3.00042);
\draw [c] (7.3802,2.60897) -- (7.4245,2.60897);
\draw [c] (7.4245,2.60897) -- (7.46881,2.60897);
\definecolor{c}{rgb}{0,0,0};
\colorlet{c}{natgreen};
\draw [c] (8.39926,0.680516) -- (8.39926,1.05001);
\draw [c] (8.39926,1.05001) -- (8.39926,1.44146);
\draw [c] (8.35495,1.05001) -- (8.39926,1.05001);
\draw [c] (8.39926,1.05001) -- (8.44356,1.05001);
\definecolor{c}{rgb}{0,0,0};
\draw [c] (1,0.680516) -- (9.95,0.680516);
\draw [anchor= east] (9.95,0.108883) node[color=c, rotate=0]{$M_{\gamma\gamma}\text{ [GeV]}$};
\draw [c] (1,0.863234) -- (1,0.680516);
\draw [c] (1.44307,0.771875) -- (1.44307,0.680516);
\draw [c] (1.88614,0.771875) -- (1.88614,0.680516);
\draw [c] (2.32921,0.771875) -- (2.32921,0.680516);
\draw [c] (2.77228,0.863234) -- (2.77228,0.680516);
\draw [c] (3.21535,0.771875) -- (3.21535,0.680516);
\draw [c] (3.65842,0.771875) -- (3.65842,0.680516);
\draw [c] (4.10149,0.771875) -- (4.10149,0.680516);
\draw [c] (4.54455,0.863234) -- (4.54455,0.680516);
\draw [c] (4.98762,0.771875) -- (4.98762,0.680516);
\draw [c] (5.43069,0.771875) -- (5.43069,0.680516);
\draw [c] (5.87376,0.771875) -- (5.87376,0.680516);
\draw [c] (6.31683,0.863234) -- (6.31683,0.680516);
\draw [c] (6.7599,0.771875) -- (6.7599,0.680516);
\draw [c] (7.20297,0.771875) -- (7.20297,0.680516);
\draw [c] (7.64604,0.771875) -- (7.64604,0.680516);
\draw [c] (8.08911,0.863234) -- (8.08911,0.680516);
\draw [c] (8.53218,0.771875) -- (8.53218,0.680516);
\draw [c] (8.97525,0.771875) -- (8.97525,0.680516);
\draw [c] (9.41832,0.771875) -- (9.41832,0.680516);
\draw [c] (9.86139,0.863234) -- (9.86139,0.680516);
\draw [c] (9.86139,0.863234) -- (9.86139,0.680516);
\draw [anchor=base] (1,0.353868) node[color=c, rotate=0]{0};
\draw [anchor=base] (2.77228,0.353868) node[color=c, rotate=0]{200};
\draw [anchor=base] (4.54455,0.353868) node[color=c, rotate=0]{400};
\draw [anchor=base] (6.31683,0.353868) node[color=c, rotate=0]{600};
\draw [anchor=base] (8.08911,0.353868) node[color=c, rotate=0]{800};
\draw [anchor=base] (9.86139,0.353868) node[color=c, rotate=0]{1000};
\draw [c] (1,0.680516) -- (1,6.73711);
\draw [anchor= east] (-0.12,6.73711) node[color=c, rotate=90]{Number of events};
\draw [c] (1.1335,0.7327) -- (1,0.7327);
\draw [c] (1.1335,0.895167) -- (1,0.895167);
\draw [c] (1.1335,1.02119) -- (1,1.02119);
\draw [c] (1.1335,1.12415) -- (1,1.12415);
\draw [c] (1.1335,1.21121) -- (1,1.21121);
\draw [c] (1.1335,1.28662) -- (1,1.28662);
\draw [c] (1.1335,1.35314) -- (1,1.35314);
\draw [c] (1.267,1.41264) -- (1,1.41264);
\draw [anchor= east] (0.922,1.41264) node[color=c, rotate=0]{$10^{-1}$};
\draw [c] (1.1335,1.80409) -- (1,1.80409);
\draw [c] (1.1335,2.03307) -- (1,2.03307);
\draw [c] (1.1335,2.19554) -- (1,2.19554);
\draw [c] (1.1335,2.32156) -- (1,2.32156);
\draw [c] (1.1335,2.42453) -- (1,2.42453);
\draw [c] (1.1335,2.51158) -- (1,2.51158);
\draw [c] (1.1335,2.58699) -- (1,2.58699);
\draw [c] (1.1335,2.65351) -- (1,2.65351);
\draw [c] (1.267,2.71301) -- (1,2.71301);
\draw [anchor= east] (0.922,2.71301) node[color=c, rotate=0]{1};
\draw [c] (1.1335,3.10446) -- (1,3.10446);
\draw [c] (1.1335,3.33345) -- (1,3.33345);
\draw [c] (1.1335,3.49592) -- (1,3.49592);
\draw [c] (1.1335,3.62193) -- (1,3.62193);
\draw [c] (1.1335,3.7249) -- (1,3.7249);
\draw [c] (1.1335,3.81196) -- (1,3.81196);
\draw [c] (1.1335,3.88737) -- (1,3.88737);
\draw [c] (1.1335,3.95388) -- (1,3.95388);
\draw [c] (1.267,4.01339) -- (1,4.01339);
\draw [anchor= east] (0.922,4.01339) node[color=c, rotate=0]{10};
\draw [c] (1.1335,4.40484) -- (1,4.40484);
\draw [c] (1.1335,4.63382) -- (1,4.63382);
\draw [c] (1.1335,4.79629) -- (1,4.79629);
\draw [c] (1.1335,4.92231) -- (1,4.92231);
\draw [c] (1.1335,5.02527) -- (1,5.02527);
\draw [c] (1.1335,5.11233) -- (1,5.11233);
\draw [c] (1.1335,5.18774) -- (1,5.18774);
\draw [c] (1.1335,5.25426) -- (1,5.25426);
\draw [c] (1.267,5.31376) -- (1,5.31376);
\draw [anchor= east] (0.922,5.31376) node[color=c, rotate=0]{$10^{2}$};
\draw [c] (1.1335,5.70521) -- (1,5.70521);
\draw [c] (1.1335,5.9342) -- (1,5.9342);
\draw [c] (1.1335,6.09666) -- (1,6.09666);
\draw [c] (1.1335,6.22268) -- (1,6.22268);
\draw [c] (1.1335,6.32565) -- (1,6.32565);
\draw [c] (1.1335,6.4127) -- (1,6.4127);
\draw [c] (1.1335,6.48812) -- (1,6.48812);
\draw [c] (1.1335,6.55463) -- (1,6.55463);
\draw [c] (1.267,6.61414) -- (1,6.61414);
\draw [anchor= east] (0.922,6.61414) node[color=c, rotate=0]{$10^{3}$};
\end{tikzpicture}

\end{infilsf}
\end{minipage}
\begin{minipage}[b]{.3\textwidth}
\caption{The distribution of invariant masses in the \atlas{} box diagram data set.}\label{boxmgg}
\end{minipage}
\end{figure}

Adding this to the CalcHEP sample, we get the distribution in fig.~\ref{ggcomp}. As should be evident, there is still a deficit in the CalcHEP sample. This might be because the box diagram contribution added to the CalcHEP sample was produced by a different generator than was used to generate the \atlas{} $\gamma\gamma$ sample, or it might be due to interference between the diagrams which simply adding together samples does not capture. In any case, the difference will be obscure by the uncertainty on the $\gamma$jet sample, which must be added to both.

\begin{figure}[htp]
\begin{minipage}[b]{.69\textwidth}
\begin{infilsf} \tiny
\begin{tikzpicture}[x=.092\textwidth,y=.092\textwidth]
\pgfdeclareplotmark{cross} {
\pgfpathmoveto{\pgfpoint{-0.3\pgfplotmarksize}{\pgfplotmarksize}}
\pgfpathlineto{\pgfpoint{+0.3\pgfplotmarksize}{\pgfplotmarksize}}
\pgfpathlineto{\pgfpoint{+0.3\pgfplotmarksize}{0.3\pgfplotmarksize}}
\pgfpathlineto{\pgfpoint{+1\pgfplotmarksize}{0.3\pgfplotmarksize}}
\pgfpathlineto{\pgfpoint{+1\pgfplotmarksize}{-0.3\pgfplotmarksize}}
\pgfpathlineto{\pgfpoint{+0.3\pgfplotmarksize}{-0.3\pgfplotmarksize}}
\pgfpathlineto{\pgfpoint{+0.3\pgfplotmarksize}{-1.\pgfplotmarksize}}
\pgfpathlineto{\pgfpoint{-0.3\pgfplotmarksize}{-1.\pgfplotmarksize}}
\pgfpathlineto{\pgfpoint{-0.3\pgfplotmarksize}{-0.3\pgfplotmarksize}}
\pgfpathlineto{\pgfpoint{-1.\pgfplotmarksize}{-0.3\pgfplotmarksize}}
\pgfpathlineto{\pgfpoint{-1.\pgfplotmarksize}{0.3\pgfplotmarksize}}
\pgfpathlineto{\pgfpoint{-0.3\pgfplotmarksize}{0.3\pgfplotmarksize}}
\pgfpathclose
\pgfusepathqstroke
}
\pgfdeclareplotmark{cross*} {
\pgfpathmoveto{\pgfpoint{-0.3\pgfplotmarksize}{\pgfplotmarksize}}
\pgfpathlineto{\pgfpoint{+0.3\pgfplotmarksize}{\pgfplotmarksize}}
\pgfpathlineto{\pgfpoint{+0.3\pgfplotmarksize}{0.3\pgfplotmarksize}}
\pgfpathlineto{\pgfpoint{+1\pgfplotmarksize}{0.3\pgfplotmarksize}}
\pgfpathlineto{\pgfpoint{+1\pgfplotmarksize}{-0.3\pgfplotmarksize}}
\pgfpathlineto{\pgfpoint{+0.3\pgfplotmarksize}{-0.3\pgfplotmarksize}}
\pgfpathlineto{\pgfpoint{+0.3\pgfplotmarksize}{-1.\pgfplotmarksize}}
\pgfpathlineto{\pgfpoint{-0.3\pgfplotmarksize}{-1.\pgfplotmarksize}}
\pgfpathlineto{\pgfpoint{-0.3\pgfplotmarksize}{-0.3\pgfplotmarksize}}
\pgfpathlineto{\pgfpoint{-1.\pgfplotmarksize}{-0.3\pgfplotmarksize}}
\pgfpathlineto{\pgfpoint{-1.\pgfplotmarksize}{0.3\pgfplotmarksize}}
\pgfpathlineto{\pgfpoint{-0.3\pgfplotmarksize}{0.3\pgfplotmarksize}}
\pgfpathclose
\pgfusepathqfillstroke
}
\pgfdeclareplotmark{newstar} {
\pgfpathmoveto{\pgfqpoint{0pt}{\pgfplotmarksize}}
\pgfpathlineto{\pgfqpointpolar{44}{0.5\pgfplotmarksize}}
\pgfpathlineto{\pgfqpointpolar{18}{\pgfplotmarksize}}
\pgfpathlineto{\pgfqpointpolar{-20}{0.5\pgfplotmarksize}}
\pgfpathlineto{\pgfqpointpolar{-54}{\pgfplotmarksize}}
\pgfpathlineto{\pgfqpointpolar{-90}{0.5\pgfplotmarksize}}
\pgfpathlineto{\pgfqpointpolar{234}{\pgfplotmarksize}}
\pgfpathlineto{\pgfqpointpolar{198}{0.5\pgfplotmarksize}}
\pgfpathlineto{\pgfqpointpolar{162}{\pgfplotmarksize}}
\pgfpathlineto{\pgfqpointpolar{134}{0.5\pgfplotmarksize}}
\pgfpathclose
\pgfusepathqstroke
}
\pgfdeclareplotmark{newstar*} {
\pgfpathmoveto{\pgfqpoint{0pt}{\pgfplotmarksize}}
\pgfpathlineto{\pgfqpointpolar{44}{0.5\pgfplotmarksize}}
\pgfpathlineto{\pgfqpointpolar{18}{\pgfplotmarksize}}
\pgfpathlineto{\pgfqpointpolar{-20}{0.5\pgfplotmarksize}}
\pgfpathlineto{\pgfqpointpolar{-54}{\pgfplotmarksize}}
\pgfpathlineto{\pgfqpointpolar{-90}{0.5\pgfplotmarksize}}
\pgfpathlineto{\pgfqpointpolar{234}{\pgfplotmarksize}}
\pgfpathlineto{\pgfqpointpolar{198}{0.5\pgfplotmarksize}}
\pgfpathlineto{\pgfqpointpolar{162}{\pgfplotmarksize}}
\pgfpathlineto{\pgfqpointpolar{134}{0.5\pgfplotmarksize}}
\pgfpathclose
\pgfusepathqfillstroke
}
\definecolor{c}{rgb}{1,1,1};
\draw [color=c, fill=c] (0,0) rectangle (10,6.80516);
\draw [color=c, fill=c] (1,0.680516) rectangle (9.95,6.73711);
\definecolor{c}{rgb}{0,0,0};
\draw [c] (1,0.680516) -- (1,6.73711) -- (9.95,6.73711) -- (9.95,0.680516) -- (1,0.680516);
\definecolor{c}{rgb}{1,1,1};
\draw [color=c, fill=c] (1,0.680516) rectangle (9.95,6.73711);
\definecolor{c}{rgb}{0,0,0};
\draw [c] (1,0.680516) -- (1,6.73711) -- (9.95,6.73711) -- (9.95,0.680516) -- (1,0.680516);
\colorlet{c}{natcomp!70};
\draw [c] (1.34131,1.78352) -- (1.34131,2.51541);
\draw [c] (1.34131,2.51541) -- (1.34131,2.82687);
\draw [c] (1.30339,2.51541) -- (1.34131,2.51541);
\draw [c] (1.34131,2.51541) -- (1.37924,2.51541);
\definecolor{c}{rgb}{0,0,0};
\colorlet{c}{natcomp!70};
\draw [c] (1.41716,1.62511) -- (1.41716,2.33052);
\draw [c] (1.41716,2.33052) -- (1.41716,2.63755);
\draw [c] (1.37924,2.33052) -- (1.41716,2.33052);
\draw [c] (1.41716,2.33052) -- (1.45508,2.33052);
\definecolor{c}{rgb}{0,0,0};
\colorlet{c}{natcomp!70};
\draw [c] (1.56886,1.90599) -- (1.56886,2.61743);
\draw [c] (1.56886,2.61743) -- (1.56886,2.92549);
\draw [c] (1.53093,2.61743) -- (1.56886,2.61743);
\draw [c] (1.56886,2.61743) -- (1.60678,2.61743);
\definecolor{c}{rgb}{0,0,0};
\colorlet{c}{natcomp!70};
\draw [c] (1.6447,2.52941) -- (1.6447,2.8612);
\draw [c] (1.6447,2.8612) -- (1.6447,3.07008);
\draw [c] (1.60678,2.8612) -- (1.6447,2.8612);
\draw [c] (1.6447,2.8612) -- (1.68263,2.8612);
\definecolor{c}{rgb}{0,0,0};
\colorlet{c}{natcomp!70};
\draw [c] (1.72055,3.76948) -- (1.72055,3.89028);
\draw [c] (1.72055,3.89028) -- (1.72055,3.99002);
\draw [c] (1.68263,3.89028) -- (1.72055,3.89028);
\draw [c] (1.72055,3.89028) -- (1.75847,3.89028);
\definecolor{c}{rgb}{0,0,0};
\colorlet{c}{natcomp!70};
\draw [c] (1.7964,5.80305) -- (1.7964,5.82466);
\draw [c] (1.7964,5.82466) -- (1.7964,5.84548);
\draw [c] (1.75847,5.82466) -- (1.7964,5.82466);
\draw [c] (1.7964,5.82466) -- (1.83432,5.82466);
\definecolor{c}{rgb}{0,0,0};
\colorlet{c}{natcomp!70};
\draw [c] (1.87225,6.24998) -- (1.87225,6.26438);
\draw [c] (1.87225,6.26438) -- (1.87225,6.27841);
\draw [c] (1.83432,6.26438) -- (1.87225,6.26438);
\draw [c] (1.87225,6.26438) -- (1.91017,6.26438);
\definecolor{c}{rgb}{0,0,0};
\colorlet{c}{natcomp!70};
\draw [c] (1.94809,6.34388) -- (1.94809,6.35726);
\draw [c] (1.94809,6.35726) -- (1.94809,6.37035);
\draw [c] (1.91017,6.35726) -- (1.94809,6.35726);
\draw [c] (1.94809,6.35726) -- (1.98602,6.35726);
\definecolor{c}{rgb}{0,0,0};
\colorlet{c}{natcomp!70};
\draw [c] (2.02394,6.30879) -- (2.02394,6.32269);
\draw [c] (2.02394,6.32269) -- (2.02394,6.33626);
\draw [c] (1.98602,6.32269) -- (2.02394,6.32269);
\draw [c] (2.02394,6.32269) -- (2.06186,6.32269);
\definecolor{c}{rgb}{0,0,0};
\colorlet{c}{natcomp!70};
\draw [c] (2.09979,6.25019) -- (2.09979,6.26512);
\draw [c] (2.09979,6.26512) -- (2.09979,6.27967);
\draw [c] (2.06186,6.26512) -- (2.09979,6.26512);
\draw [c] (2.09979,6.26512) -- (2.13771,6.26512);
\definecolor{c}{rgb}{0,0,0};
\colorlet{c}{natcomp!70};
\draw [c] (2.17564,6.16207) -- (2.17564,6.17792);
\draw [c] (2.17564,6.17792) -- (2.17564,6.19334);
\draw [c] (2.13771,6.17792) -- (2.17564,6.17792);
\draw [c] (2.17564,6.17792) -- (2.21356,6.17792);
\definecolor{c}{rgb}{0,0,0};
\colorlet{c}{natcomp!70};
\draw [c] (2.25148,6.07193) -- (2.25148,6.08917);
\draw [c] (2.25148,6.08917) -- (2.25148,6.10592);
\draw [c] (2.21356,6.08917) -- (2.25148,6.08917);
\draw [c] (2.25148,6.08917) -- (2.28941,6.08917);
\definecolor{c}{rgb}{0,0,0};
\colorlet{c}{natcomp!70};
\draw [c] (2.32733,5.94806) -- (2.32733,5.96764);
\draw [c] (2.32733,5.96764) -- (2.32733,5.98657);
\draw [c] (2.28941,5.96764) -- (2.32733,5.96764);
\draw [c] (2.32733,5.96764) -- (2.36525,5.96764);
\definecolor{c}{rgb}{0,0,0};
\colorlet{c}{natcomp!70};
\draw [c] (2.40318,5.82449) -- (2.40318,5.846);
\draw [c] (2.40318,5.846) -- (2.40318,5.86674);
\draw [c] (2.36525,5.846) -- (2.40318,5.846);
\draw [c] (2.40318,5.846) -- (2.4411,5.846);
\definecolor{c}{rgb}{0,0,0};
\colorlet{c}{natcomp!70};
\draw [c] (2.47903,5.71169) -- (2.47903,5.73544);
\draw [c] (2.47903,5.73544) -- (2.47903,5.75825);
\draw [c] (2.4411,5.73544) -- (2.47903,5.73544);
\draw [c] (2.47903,5.73544) -- (2.51695,5.73544);
\definecolor{c}{rgb}{0,0,0};
\colorlet{c}{natcomp!70};
\draw [c] (2.55487,5.64532) -- (2.55487,5.67137);
\draw [c] (2.55487,5.67137) -- (2.55487,5.69629);
\draw [c] (2.51695,5.67137) -- (2.55487,5.67137);
\draw [c] (2.55487,5.67137) -- (2.5928,5.67137);
\definecolor{c}{rgb}{0,0,0};
\colorlet{c}{natcomp!70};
\draw [c] (2.63072,5.52352) -- (2.63072,5.55176);
\draw [c] (2.63072,5.55176) -- (2.63072,5.57869);
\draw [c] (2.5928,5.55176) -- (2.63072,5.55176);
\draw [c] (2.63072,5.55176) -- (2.66864,5.55176);
\definecolor{c}{rgb}{0,0,0};
\colorlet{c}{natcomp!70};
\draw [c] (2.70657,5.50217) -- (2.70657,5.53122);
\draw [c] (2.70657,5.53122) -- (2.70657,5.55886);
\draw [c] (2.66864,5.53122) -- (2.70657,5.53122);
\draw [c] (2.70657,5.53122) -- (2.74449,5.53122);
\definecolor{c}{rgb}{0,0,0};
\colorlet{c}{natcomp!70};
\draw [c] (2.78242,5.40445) -- (2.78242,5.43645);
\draw [c] (2.78242,5.43645) -- (2.78242,5.46676);
\draw [c] (2.74449,5.43645) -- (2.78242,5.43645);
\draw [c] (2.78242,5.43645) -- (2.82034,5.43645);
\definecolor{c}{rgb}{0,0,0};
\colorlet{c}{natcomp!70};
\draw [c] (2.85826,5.28652) -- (2.85826,5.32084);
\draw [c] (2.85826,5.32084) -- (2.85826,5.35322);
\draw [c] (2.82034,5.32084) -- (2.85826,5.32084);
\draw [c] (2.85826,5.32084) -- (2.89619,5.32084);
\definecolor{c}{rgb}{0,0,0};
\colorlet{c}{natcomp!70};
\draw [c] (2.93411,5.09026) -- (2.93411,5.13066);
\draw [c] (2.93411,5.13066) -- (2.93411,5.16841);
\draw [c] (2.89619,5.13066) -- (2.93411,5.13066);
\draw [c] (2.93411,5.13066) -- (2.97203,5.13066);
\definecolor{c}{rgb}{0,0,0};
\colorlet{c}{natcomp!70};
\draw [c] (3.00996,5.07889) -- (3.00996,5.12198);
\draw [c] (3.00996,5.12198) -- (3.00996,5.16205);
\draw [c] (2.97203,5.12198) -- (3.00996,5.12198);
\draw [c] (3.00996,5.12198) -- (3.04788,5.12198);
\definecolor{c}{rgb}{0,0,0};
\colorlet{c}{natcomp!70};
\draw [c] (3.08581,4.99109) -- (3.08581,5.03784);
\draw [c] (3.08581,5.03784) -- (3.08581,5.08107);
\draw [c] (3.04788,5.03784) -- (3.08581,5.03784);
\draw [c] (3.08581,5.03784) -- (3.12373,5.03784);
\definecolor{c}{rgb}{0,0,0};
\colorlet{c}{natcomp!70};
\draw [c] (3.16165,4.99427) -- (3.16165,5.04101);
\draw [c] (3.16165,5.04101) -- (3.16165,5.08423);
\draw [c] (3.12373,5.04101) -- (3.16165,5.04101);
\draw [c] (3.16165,5.04101) -- (3.19958,5.04101);
\definecolor{c}{rgb}{0,0,0};
\colorlet{c}{natcomp!70};
\draw [c] (3.2375,4.84683) -- (3.2375,4.89782);
\draw [c] (3.2375,4.89782) -- (3.2375,4.94464);
\draw [c] (3.19958,4.89782) -- (3.2375,4.89782);
\draw [c] (3.2375,4.89782) -- (3.27542,4.89782);
\definecolor{c}{rgb}{0,0,0};
\colorlet{c}{natcomp!70};
\draw [c] (3.31335,4.77703) -- (3.31335,4.8307);
\draw [c] (3.31335,4.8307) -- (3.31335,4.87977);
\draw [c] (3.27542,4.8307) -- (3.31335,4.8307);
\draw [c] (3.31335,4.8307) -- (3.35127,4.8307);
\definecolor{c}{rgb}{0,0,0};
\colorlet{c}{natcomp!70};
\draw [c] (3.38919,4.63709) -- (3.38919,4.69689);
\draw [c] (3.38919,4.69689) -- (3.38919,4.75105);
\draw [c] (3.35127,4.69689) -- (3.38919,4.69689);
\draw [c] (3.38919,4.69689) -- (3.42712,4.69689);
\definecolor{c}{rgb}{0,0,0};
\colorlet{c}{natcomp!70};
\draw [c] (3.46504,4.62899) -- (3.46504,4.69223);
\draw [c] (3.46504,4.69223) -- (3.46504,4.74918);
\draw [c] (3.42712,4.69223) -- (3.46504,4.69223);
\draw [c] (3.46504,4.69223) -- (3.50297,4.69223);
\definecolor{c}{rgb}{0,0,0};
\colorlet{c}{natcomp!70};
\draw [c] (3.54089,4.5342) -- (3.54089,4.60227);
\draw [c] (3.54089,4.60227) -- (3.54089,4.66313);
\draw [c] (3.50297,4.60227) -- (3.54089,4.60227);
\draw [c] (3.54089,4.60227) -- (3.57881,4.60227);
\definecolor{c}{rgb}{0,0,0};
\colorlet{c}{natcomp!70};
\draw [c] (3.61674,4.40663) -- (3.61674,4.47821);
\draw [c] (3.61674,4.47821) -- (3.61674,4.54185);
\draw [c] (3.57881,4.47821) -- (3.61674,4.47821);
\draw [c] (3.61674,4.47821) -- (3.65466,4.47821);
\definecolor{c}{rgb}{0,0,0};
\colorlet{c}{natcomp!70};
\draw [c] (3.69258,4.3837) -- (3.69258,4.46006);
\draw [c] (3.69258,4.46006) -- (3.69258,4.52744);
\draw [c] (3.65466,4.46006) -- (3.69258,4.46006);
\draw [c] (3.69258,4.46006) -- (3.73051,4.46006);
\definecolor{c}{rgb}{0,0,0};
\colorlet{c}{natcomp!70};
\draw [c] (3.76843,4.26121) -- (3.76843,4.35345);
\draw [c] (3.76843,4.35345) -- (3.76843,4.43289);
\draw [c] (3.73051,4.35345) -- (3.76843,4.35345);
\draw [c] (3.76843,4.35345) -- (3.80636,4.35345);
\definecolor{c}{rgb}{0,0,0};
\colorlet{c}{natcomp!70};
\draw [c] (3.84428,4.14632) -- (3.84428,4.24826);
\draw [c] (3.84428,4.24826) -- (3.84428,4.3348);
\draw [c] (3.80636,4.24826) -- (3.84428,4.24826);
\draw [c] (3.84428,4.24826) -- (3.8822,4.24826);
\definecolor{c}{rgb}{0,0,0};
\colorlet{c}{natcomp!70};
\draw [c] (3.92013,4.18657) -- (3.92013,4.28491);
\draw [c] (3.92013,4.28491) -- (3.92013,4.36884);
\draw [c] (3.8822,4.28491) -- (3.92013,4.28491);
\draw [c] (3.92013,4.28491) -- (3.95805,4.28491);
\definecolor{c}{rgb}{0,0,0};
\colorlet{c}{natcomp!70};
\draw [c] (3.99597,4.31132) -- (3.99597,4.39416);
\draw [c] (3.99597,4.39416) -- (3.99597,4.46653);
\draw [c] (3.95805,4.39416) -- (3.99597,4.39416);
\draw [c] (3.99597,4.39416) -- (4.0339,4.39416);
\definecolor{c}{rgb}{0,0,0};
\colorlet{c}{natcomp!70};
\draw [c] (4.07182,3.79936) -- (4.07182,3.92106);
\draw [c] (4.07182,3.92106) -- (4.07182,4.02142);
\draw [c] (4.0339,3.92106) -- (4.07182,3.92106);
\draw [c] (4.07182,3.92106) -- (4.10975,3.92106);
\definecolor{c}{rgb}{0,0,0};
\colorlet{c}{natcomp!70};
\draw [c] (4.14767,4.04423) -- (4.14767,4.15122);
\draw [c] (4.14767,4.15122) -- (4.14767,4.24136);
\draw [c] (4.10975,4.15122) -- (4.14767,4.15122);
\draw [c] (4.14767,4.15122) -- (4.18559,4.15122);
\definecolor{c}{rgb}{0,0,0};
\colorlet{c}{natcomp!70};
\draw [c] (4.22352,4.17213) -- (4.22352,4.27444);
\draw [c] (4.22352,4.27444) -- (4.22352,4.36124);
\draw [c] (4.18559,4.27444) -- (4.22352,4.27444);
\draw [c] (4.22352,4.27444) -- (4.26144,4.27444);
\definecolor{c}{rgb}{0,0,0};
\colorlet{c}{natcomp!70};
\draw [c] (4.29936,3.93105) -- (4.29936,3.95136);
\draw [c] (4.29936,3.95136) -- (4.29936,3.97098);
\draw [c] (4.26144,3.95136) -- (4.29936,3.95136);
\draw [c] (4.29936,3.95136) -- (4.33729,3.95136);
\definecolor{c}{rgb}{0,0,0};
\colorlet{c}{natcomp!70};
\draw [c] (4.37521,3.86494) -- (4.37521,3.89839);
\draw [c] (4.37521,3.89839) -- (4.37521,3.92999);
\draw [c] (4.33729,3.89839) -- (4.37521,3.89839);
\draw [c] (4.37521,3.89839) -- (4.41314,3.89839);
\definecolor{c}{rgb}{0,0,0};
\colorlet{c}{natcomp!70};
\draw [c] (4.45106,3.84202) -- (4.45106,3.86382);
\draw [c] (4.45106,3.86382) -- (4.45106,3.88482);
\draw [c] (4.41314,3.86382) -- (4.45106,3.86382);
\draw [c] (4.45106,3.86382) -- (4.48898,3.86382);
\definecolor{c}{rgb}{0,0,0};
\colorlet{c}{natcomp!70};
\draw [c] (4.52691,3.84293) -- (4.52691,3.8867);
\draw [c] (4.52691,3.8867) -- (4.52691,3.92736);
\draw [c] (4.48898,3.8867) -- (4.52691,3.8867);
\draw [c] (4.52691,3.8867) -- (4.56483,3.8867);
\definecolor{c}{rgb}{0,0,0};
\colorlet{c}{natcomp!70};
\draw [c] (4.60275,3.75875) -- (4.60275,3.80805);
\draw [c] (4.60275,3.80805) -- (4.60275,3.85345);
\draw [c] (4.56483,3.80805) -- (4.60275,3.80805);
\draw [c] (4.60275,3.80805) -- (4.64068,3.80805);
\definecolor{c}{rgb}{0,0,0};
\colorlet{c}{natcomp!70};
\draw [c] (4.6786,3.66225) -- (4.6786,3.71021);
\draw [c] (4.6786,3.71021) -- (4.6786,3.75446);
\draw [c] (4.64068,3.71021) -- (4.6786,3.71021);
\draw [c] (4.6786,3.71021) -- (4.71653,3.71021);
\definecolor{c}{rgb}{0,0,0};
\colorlet{c}{natcomp!70};
\draw [c] (4.75445,3.57524) -- (4.75445,3.62382);
\draw [c] (4.75445,3.62382) -- (4.75445,3.6686);
\draw [c] (4.71653,3.62382) -- (4.75445,3.62382);
\draw [c] (4.75445,3.62382) -- (4.79237,3.62382);
\definecolor{c}{rgb}{0,0,0};
\colorlet{c}{natcomp!70};
\draw [c] (4.8303,3.55668) -- (4.8303,3.60867);
\draw [c] (4.8303,3.60867) -- (4.8303,3.65634);
\draw [c] (4.79237,3.60867) -- (4.8303,3.60867);
\draw [c] (4.8303,3.60867) -- (4.86822,3.60867);
\definecolor{c}{rgb}{0,0,0};
\colorlet{c}{natcomp!70};
\draw [c] (4.90614,3.51853) -- (4.90614,3.55247);
\draw [c] (4.90614,3.55247) -- (4.90614,3.58452);
\draw [c] (4.86822,3.55247) -- (4.90614,3.55247);
\draw [c] (4.90614,3.55247) -- (4.94407,3.55247);
\definecolor{c}{rgb}{0,0,0};
\colorlet{c}{natcomp!70};
\draw [c] (4.98199,3.52238) -- (4.98199,3.58359);
\draw [c] (4.98199,3.58359) -- (4.98199,3.6389);
\draw [c] (4.94407,3.58359) -- (4.98199,3.58359);
\draw [c] (4.98199,3.58359) -- (5.01992,3.58359);
\definecolor{c}{rgb}{0,0,0};
\colorlet{c}{natcomp!70};
\draw [c] (5.05784,3.43016) -- (5.05784,3.48934);
\draw [c] (5.05784,3.48934) -- (5.05784,3.54298);
\draw [c] (5.01992,3.48934) -- (5.05784,3.48934);
\draw [c] (5.05784,3.48934) -- (5.09576,3.48934);
\definecolor{c}{rgb}{0,0,0};
\colorlet{c}{natcomp!70};
\draw [c] (5.13369,3.32771) -- (5.13369,3.37927);
\draw [c] (5.13369,3.37927) -- (5.13369,3.42658);
\draw [c] (5.09576,3.37927) -- (5.13369,3.37927);
\draw [c] (5.13369,3.37927) -- (5.17161,3.37927);
\definecolor{c}{rgb}{0,0,0};
\colorlet{c}{natcomp!70};
\draw [c] (5.20953,3.23557) -- (5.20953,3.27213);
\draw [c] (5.20953,3.27213) -- (5.20953,3.3065);
\draw [c] (5.17161,3.27213) -- (5.20953,3.27213);
\draw [c] (5.20953,3.27213) -- (5.24746,3.27213);
\definecolor{c}{rgb}{0,0,0};
\colorlet{c}{natcomp!70};
\draw [c] (5.28538,3.29755) -- (5.28538,3.35759);
\draw [c] (5.28538,3.35759) -- (5.28538,3.41194);
\draw [c] (5.24746,3.35759) -- (5.28538,3.35759);
\draw [c] (5.28538,3.35759) -- (5.32331,3.35759);
\definecolor{c}{rgb}{0,0,0};
\colorlet{c}{natcomp!70};
\draw [c] (5.36123,3.1904) -- (5.36123,3.25923);
\draw [c] (5.36123,3.25923) -- (5.36123,3.32068);
\draw [c] (5.32331,3.25923) -- (5.36123,3.25923);
\draw [c] (5.36123,3.25923) -- (5.39915,3.25923);
\definecolor{c}{rgb}{0,0,0};
\colorlet{c}{natcomp!70};
\draw [c] (5.43708,3.14764) -- (5.43708,3.18783);
\draw [c] (5.43708,3.18783) -- (5.43708,3.22538);
\draw [c] (5.39915,3.18783) -- (5.43708,3.18783);
\draw [c] (5.43708,3.18783) -- (5.475,3.18783);
\definecolor{c}{rgb}{0,0,0};
\colorlet{c}{natcomp!70};
\draw [c] (5.51292,3.10655) -- (5.51292,3.14909);
\draw [c] (5.51292,3.14909) -- (5.51292,3.18868);
\draw [c] (5.475,3.14909) -- (5.51292,3.14909);
\draw [c] (5.51292,3.14909) -- (5.55085,3.14909);
\definecolor{c}{rgb}{0,0,0};
\colorlet{c}{natcomp!70};
\draw [c] (5.58877,3.12957) -- (5.58877,3.17285);
\draw [c] (5.58877,3.17285) -- (5.58877,3.21309);
\draw [c] (5.55085,3.17285) -- (5.58877,3.17285);
\draw [c] (5.58877,3.17285) -- (5.6267,3.17285);
\definecolor{c}{rgb}{0,0,0};
\colorlet{c}{natcomp!70};
\draw [c] (5.66462,2.98182) -- (5.66462,3.0269);
\draw [c] (5.66462,3.0269) -- (5.66462,3.0687);
\draw [c] (5.6267,3.0269) -- (5.66462,3.0269);
\draw [c] (5.66462,3.0269) -- (5.70254,3.0269);
\definecolor{c}{rgb}{0,0,0};
\colorlet{c}{natcomp!70};
\draw [c] (5.74047,2.95982) -- (5.74047,3.00795);
\draw [c] (5.74047,3.00795) -- (5.74047,3.05235);
\draw [c] (5.70254,3.00795) -- (5.74047,3.00795);
\draw [c] (5.74047,3.00795) -- (5.77839,3.00795);
\definecolor{c}{rgb}{0,0,0};
\colorlet{c}{natcomp!70};
\draw [c] (5.81631,2.9584) -- (5.81631,3.05897);
\draw [c] (5.81631,3.05897) -- (5.81631,3.14452);
\draw [c] (5.77839,3.05897) -- (5.81631,3.05897);
\draw [c] (5.81631,3.05897) -- (5.85424,3.05897);
\definecolor{c}{rgb}{0,0,0};
\colorlet{c}{natcomp!70};
\draw [c] (5.89216,3.10587) -- (5.89216,3.23102);
\draw [c] (5.89216,3.23102) -- (5.89216,3.3337);
\draw [c] (5.85424,3.23102) -- (5.89216,3.23102);
\draw [c] (5.89216,3.23102) -- (5.93008,3.23102);
\definecolor{c}{rgb}{0,0,0};
\colorlet{c}{natcomp!70};
\draw [c] (5.96801,2.82054) -- (5.96801,2.94787);
\draw [c] (5.96801,2.94787) -- (5.96801,3.05201);
\draw [c] (5.93008,2.94787) -- (5.96801,2.94787);
\draw [c] (5.96801,2.94787) -- (6.00593,2.94787);
\definecolor{c}{rgb}{0,0,0};
\colorlet{c}{natcomp!70};
\draw [c] (6.04386,2.72523) -- (6.04386,2.78281);
\draw [c] (6.04386,2.78281) -- (6.04386,2.83514);
\draw [c] (6.00593,2.78281) -- (6.04386,2.78281);
\draw [c] (6.04386,2.78281) -- (6.08178,2.78281);
\definecolor{c}{rgb}{0,0,0};
\colorlet{c}{natcomp!70};
\draw [c] (6.1197,2.78902) -- (6.1197,2.8457);
\draw [c] (6.1197,2.8457) -- (6.1197,2.89729);
\draw [c] (6.08178,2.8457) -- (6.1197,2.8457);
\draw [c] (6.1197,2.8457) -- (6.15763,2.8457);
\definecolor{c}{rgb}{0,0,0};
\colorlet{c}{natcomp!70};
\draw [c] (6.19555,2.76443) -- (6.19555,2.88354);
\draw [c] (6.19555,2.88354) -- (6.19555,2.98212);
\draw [c] (6.15763,2.88354) -- (6.19555,2.88354);
\draw [c] (6.19555,2.88354) -- (6.23347,2.88354);
\definecolor{c}{rgb}{0,0,0};
\colorlet{c}{natcomp!70};
\draw [c] (6.2714,2.71588) -- (6.2714,2.77508);
\draw [c] (6.2714,2.77508) -- (6.2714,2.82873);
\draw [c] (6.23347,2.77508) -- (6.2714,2.77508);
\draw [c] (6.2714,2.77508) -- (6.30932,2.77508);
\definecolor{c}{rgb}{0,0,0};
\colorlet{c}{natcomp!70};
\draw [c] (6.34725,2.55336) -- (6.34725,2.61892);
\draw [c] (6.34725,2.61892) -- (6.34725,2.67775);
\draw [c] (6.30932,2.61892) -- (6.34725,2.61892);
\draw [c] (6.34725,2.61892) -- (6.38517,2.61892);
\definecolor{c}{rgb}{0,0,0};
\colorlet{c}{natcomp!70};
\draw [c] (6.42309,2.54103) -- (6.42309,2.60764);
\draw [c] (6.42309,2.60764) -- (6.42309,2.66732);
\draw [c] (6.38517,2.60764) -- (6.42309,2.60764);
\draw [c] (6.42309,2.60764) -- (6.46102,2.60764);
\definecolor{c}{rgb}{0,0,0};
\colorlet{c}{natcomp!70};
\draw [c] (6.49894,2.57973) -- (6.49894,2.7434);
\draw [c] (6.49894,2.7434) -- (6.49894,2.87061);
\draw [c] (6.46102,2.7434) -- (6.49894,2.7434);
\draw [c] (6.49894,2.7434) -- (6.53686,2.7434);
\definecolor{c}{rgb}{0,0,0};
\colorlet{c}{natcomp!70};
\draw [c] (6.57479,2.28663) -- (6.57479,2.36458);
\draw [c] (6.57479,2.36458) -- (6.57479,2.4332);
\draw [c] (6.53686,2.36458) -- (6.57479,2.36458);
\draw [c] (6.57479,2.36458) -- (6.61271,2.36458);
\definecolor{c}{rgb}{0,0,0};
\colorlet{c}{natcomp!70};
\draw [c] (6.65064,2.40822) -- (6.65064,2.48613);
\draw [c] (6.65064,2.48613) -- (6.65064,2.55472);
\draw [c] (6.61271,2.48613) -- (6.65064,2.48613);
\draw [c] (6.65064,2.48613) -- (6.68856,2.48613);
\definecolor{c}{rgb}{0,0,0};
\colorlet{c}{natcomp!70};
\draw [c] (6.72648,2.47109) -- (6.72648,2.54861);
\draw [c] (6.72648,2.54861) -- (6.72648,2.6169);
\draw [c] (6.68856,2.54861) -- (6.72648,2.54861);
\draw [c] (6.72648,2.54861) -- (6.76441,2.54861);
\definecolor{c}{rgb}{0,0,0};
\colorlet{c}{natcomp!70};
\draw [c] (6.80233,2.34361) -- (6.80233,2.42923);
\draw [c] (6.80233,2.42923) -- (6.80233,2.50371);
\draw [c] (6.76441,2.42923) -- (6.80233,2.42923);
\draw [c] (6.80233,2.42923) -- (6.84025,2.42923);
\definecolor{c}{rgb}{0,0,0};
\colorlet{c}{natcomp!70};
\draw [c] (6.87818,2.39771) -- (6.87818,2.47467);
\draw [c] (6.87818,2.47467) -- (6.87818,2.54252);
\draw [c] (6.84025,2.47467) -- (6.87818,2.47467);
\draw [c] (6.87818,2.47467) -- (6.9161,2.47467);
\definecolor{c}{rgb}{0,0,0};
\colorlet{c}{natcomp!70};
\draw [c] (6.95403,2.13205) -- (6.95403,2.22216);
\draw [c] (6.95403,2.22216) -- (6.95403,2.30002);
\draw [c] (6.9161,2.22216) -- (6.95403,2.22216);
\draw [c] (6.95403,2.22216) -- (6.99195,2.22216);
\definecolor{c}{rgb}{0,0,0};
\colorlet{c}{natcomp!70};
\draw [c] (7.02987,2.25139) -- (7.02987,2.33724);
\draw [c] (7.02987,2.33724) -- (7.02987,2.41191);
\draw [c] (6.99195,2.33724) -- (7.02987,2.33724);
\draw [c] (7.02987,2.33724) -- (7.0678,2.33724);
\definecolor{c}{rgb}{0,0,0};
\colorlet{c}{natcomp!70};
\draw [c] (7.10572,2.32922) -- (7.10572,2.41443);
\draw [c] (7.10572,2.41443) -- (7.10572,2.48861);
\draw [c] (7.0678,2.41443) -- (7.10572,2.41443);
\draw [c] (7.10572,2.41443) -- (7.14364,2.41443);
\definecolor{c}{rgb}{0,0,0};
\colorlet{c}{natcomp!70};
\draw [c] (7.18157,2.12343) -- (7.18157,2.21978);
\draw [c] (7.18157,2.21978) -- (7.18157,2.30226);
\draw [c] (7.14364,2.21978) -- (7.18157,2.21978);
\draw [c] (7.18157,2.21978) -- (7.21949,2.21978);
\definecolor{c}{rgb}{0,0,0};
\colorlet{c}{natcomp!70};
\draw [c] (7.25742,2.03344) -- (7.25742,2.13853);
\draw [c] (7.25742,2.13853) -- (7.25742,2.22732);
\draw [c] (7.21949,2.13853) -- (7.25742,2.13853);
\draw [c] (7.25742,2.13853) -- (7.29534,2.13853);
\definecolor{c}{rgb}{0,0,0};
\colorlet{c}{natcomp!70};
\draw [c] (7.33326,2.02634) -- (7.33326,2.12717);
\draw [c] (7.33326,2.12717) -- (7.33326,2.2129);
\draw [c] (7.29534,2.12717) -- (7.33326,2.12717);
\draw [c] (7.33326,2.12717) -- (7.37119,2.12717);
\definecolor{c}{rgb}{0,0,0};
\colorlet{c}{natcomp!70};
\draw [c] (7.40911,1.83146) -- (7.40911,1.95577);
\draw [c] (7.40911,1.95577) -- (7.40911,2.05788);
\draw [c] (7.37119,1.95577) -- (7.40911,1.95577);
\draw [c] (7.40911,1.95577) -- (7.44703,1.95577);
\definecolor{c}{rgb}{0,0,0};
\colorlet{c}{natcomp!70};
\draw [c] (7.48496,2.08552) -- (7.48496,2.18317);
\draw [c] (7.48496,2.18317) -- (7.48496,2.26659);
\draw [c] (7.44703,2.18317) -- (7.48496,2.18317);
\draw [c] (7.48496,2.18317) -- (7.52288,2.18317);
\definecolor{c}{rgb}{0,0,0};
\colorlet{c}{natcomp!70};
\draw [c] (7.56081,2.1242) -- (7.56081,2.21961);
\draw [c] (7.56081,2.21961) -- (7.56081,2.3014);
\draw [c] (7.52288,2.21961) -- (7.56081,2.21961);
\draw [c] (7.56081,2.21961) -- (7.59873,2.21961);
\definecolor{c}{rgb}{0,0,0};
\colorlet{c}{natcomp!70};
\draw [c] (7.63665,1.9047) -- (7.63665,2.02183);
\draw [c] (7.63665,2.02183) -- (7.63665,2.11905);
\draw [c] (7.59873,2.02183) -- (7.63665,2.02183);
\draw [c] (7.63665,2.02183) -- (7.67458,2.02183);
\definecolor{c}{rgb}{0,0,0};
\colorlet{c}{natcomp!70};
\draw [c] (7.7125,1.94031) -- (7.7125,2.06341);
\draw [c] (7.7125,2.06341) -- (7.7125,2.16472);
\draw [c] (7.67458,2.06341) -- (7.7125,2.06341);
\draw [c] (7.7125,2.06341) -- (7.75042,2.06341);
\definecolor{c}{rgb}{0,0,0};
\colorlet{c}{natcomp!70};
\draw [c] (7.78835,1.72392) -- (7.78835,1.86154);
\draw [c] (7.78835,1.86154) -- (7.78835,1.97246);
\draw [c] (7.75042,1.86154) -- (7.78835,1.86154);
\draw [c] (7.78835,1.86154) -- (7.82627,1.86154);
\definecolor{c}{rgb}{0,0,0};
\colorlet{c}{natcomp!70};
\draw [c] (7.86419,1.73133) -- (7.86419,1.86386);
\draw [c] (7.86419,1.86386) -- (7.86419,1.97145);
\draw [c] (7.82627,1.86386) -- (7.86419,1.86386);
\draw [c] (7.86419,1.86386) -- (7.90212,1.86386);
\definecolor{c}{rgb}{0,0,0};
\colorlet{c}{natcomp!70};
\draw [c] (7.94004,1.51339) -- (7.94004,1.66499);
\draw [c] (7.94004,1.66499) -- (7.94004,1.78479);
\draw [c] (7.90212,1.66499) -- (7.94004,1.66499);
\draw [c] (7.94004,1.66499) -- (7.97797,1.66499);
\definecolor{c}{rgb}{0,0,0};
\colorlet{c}{natcomp!70};
\draw [c] (8.01589,1.80733) -- (8.01589,1.9271);
\draw [c] (8.01589,1.9271) -- (8.01589,2.02614);
\draw [c] (7.97797,1.9271) -- (8.01589,1.9271);
\draw [c] (8.01589,1.9271) -- (8.05381,1.9271);
\definecolor{c}{rgb}{0,0,0};
\colorlet{c}{natcomp!70};
\draw [c] (8.09174,1.74933) -- (8.09174,1.87737);
\draw [c] (8.09174,1.87737) -- (8.09174,1.982);
\draw [c] (8.05381,1.87737) -- (8.09174,1.87737);
\draw [c] (8.09174,1.87737) -- (8.12966,1.87737);
\definecolor{c}{rgb}{0,0,0};
\colorlet{c}{natcomp!70};
\draw [c] (8.16758,1.58389) -- (8.16758,1.7371);
\draw [c] (8.16758,1.7371) -- (8.16758,1.85791);
\draw [c] (8.12966,1.7371) -- (8.16758,1.7371);
\draw [c] (8.16758,1.7371) -- (8.20551,1.7371);
\definecolor{c}{rgb}{0,0,0};
\colorlet{c}{natcomp!70};
\draw [c] (8.24343,1.69622) -- (8.24343,1.83434);
\draw [c] (8.24343,1.83434) -- (8.24343,1.94559);
\draw [c] (8.20551,1.83434) -- (8.24343,1.83434);
\draw [c] (8.24343,1.83434) -- (8.28136,1.83434);
\definecolor{c}{rgb}{0,0,0};
\colorlet{c}{natcomp!70};
\draw [c] (8.31928,1.69369) -- (8.31928,1.82769);
\draw [c] (8.31928,1.82769) -- (8.31928,1.93625);
\draw [c] (8.28136,1.82769) -- (8.31928,1.82769);
\draw [c] (8.31928,1.82769) -- (8.3572,1.82769);
\definecolor{c}{rgb}{0,0,0};
\colorlet{c}{natcomp!70};
\draw [c] (8.39513,1.35148) -- (8.39513,1.52488);
\draw [c] (8.39513,1.52488) -- (8.39513,1.65787);
\draw [c] (8.3572,1.52488) -- (8.39513,1.52488);
\draw [c] (8.39513,1.52488) -- (8.43305,1.52488);
\definecolor{c}{rgb}{0,0,0};
\colorlet{c}{natcomp!70};
\draw [c] (8.47097,1.69673) -- (8.47097,1.81679);
\draw [c] (8.47097,1.81679) -- (8.47097,1.91602);
\draw [c] (8.43305,1.81679) -- (8.47097,1.81679);
\draw [c] (8.47097,1.81679) -- (8.5089,1.81679);
\definecolor{c}{rgb}{0,0,0};
\colorlet{c}{natcomp!70};
\draw [c] (8.54682,1.53076) -- (8.54682,1.65084);
\draw [c] (8.54682,1.65084) -- (8.54682,1.75008);
\draw [c] (8.5089,1.65084) -- (8.54682,1.65084);
\draw [c] (8.54682,1.65084) -- (8.58475,1.65084);
\definecolor{c}{rgb}{0,0,0};
\colorlet{c}{natcomp!70};
\draw [c] (8.62267,1.44173) -- (8.62267,1.5394);
\draw [c] (8.62267,1.5394) -- (8.62267,1.62283);
\draw [c] (8.58475,1.5394) -- (8.62267,1.5394);
\draw [c] (8.62267,1.5394) -- (8.66059,1.5394);
\definecolor{c}{rgb}{0,0,0};
\colorlet{c}{natcomp!70};
\draw [c] (8.69852,1.44392) -- (8.69852,1.46659);
\draw [c] (8.69852,1.46659) -- (8.69852,1.4884);
\draw [c] (8.66059,1.46659) -- (8.69852,1.46659);
\draw [c] (8.69852,1.46659) -- (8.73644,1.46659);
\definecolor{c}{rgb}{0,0,0};
\colorlet{c}{natcomp!70};
\draw [c] (8.77436,1.45513) -- (8.77436,1.47712);
\draw [c] (8.77436,1.47712) -- (8.77436,1.4983);
\draw [c] (8.73644,1.47712) -- (8.77436,1.47712);
\draw [c] (8.77436,1.47712) -- (8.81229,1.47712);
\definecolor{c}{rgb}{0,0,0};
\colorlet{c}{natcomp!70};
\draw [c] (8.85021,1.45819) -- (8.85021,1.48066);
\draw [c] (8.85021,1.48066) -- (8.85021,1.50228);
\draw [c] (8.81229,1.48066) -- (8.85021,1.48066);
\draw [c] (8.85021,1.48066) -- (8.88814,1.48066);
\definecolor{c}{rgb}{0,0,0};
\colorlet{c}{natcomp!70};
\draw [c] (8.92606,1.43318) -- (8.92606,1.45542);
\draw [c] (8.92606,1.45542) -- (8.92606,1.47683);
\draw [c] (8.88814,1.45542) -- (8.92606,1.45542);
\draw [c] (8.92606,1.45542) -- (8.96398,1.45542);
\definecolor{c}{rgb}{0,0,0};
\colorlet{c}{natcomp!70};
\draw [c] (9.00191,1.40407) -- (9.00191,1.42727);
\draw [c] (9.00191,1.42727) -- (9.00191,1.44958);
\draw [c] (8.96398,1.42727) -- (9.00191,1.42727);
\draw [c] (9.00191,1.42727) -- (9.03983,1.42727);
\definecolor{c}{rgb}{0,0,0};
\colorlet{c}{natcomp!70};
\draw [c] (9.07775,1.34916) -- (9.07775,1.37304);
\draw [c] (9.07775,1.37304) -- (9.07775,1.39597);
\draw [c] (9.03983,1.37304) -- (9.07775,1.37304);
\draw [c] (9.07775,1.37304) -- (9.11568,1.37304);
\definecolor{c}{rgb}{0,0,0};
\colorlet{c}{natcomp!70};
\draw [c] (9.1536,1.349) -- (9.1536,1.37351);
\draw [c] (9.1536,1.37351) -- (9.1536,1.39702);
\draw [c] (9.11568,1.37351) -- (9.1536,1.37351);
\draw [c] (9.1536,1.37351) -- (9.19153,1.37351);
\definecolor{c}{rgb}{0,0,0};
\colorlet{c}{natcomp!70};
\draw [c] (9.22945,1.28331) -- (9.22945,1.30947);
\draw [c] (9.22945,1.30947) -- (9.22945,1.3345);
\draw [c] (9.19153,1.30947) -- (9.22945,1.30947);
\draw [c] (9.22945,1.30947) -- (9.26737,1.30947);
\definecolor{c}{rgb}{0,0,0};
\colorlet{c}{natcomp!70};
\draw [c] (9.3053,1.32295) -- (9.3053,1.34851);
\draw [c] (9.3053,1.34851) -- (9.3053,1.37299);
\draw [c] (9.26737,1.34851) -- (9.3053,1.34851);
\draw [c] (9.3053,1.34851) -- (9.34322,1.34851);
\definecolor{c}{rgb}{0,0,0};
\colorlet{c}{natcomp!70};
\draw [c] (9.38114,1.21349) -- (9.38114,1.24116);
\draw [c] (9.38114,1.24116) -- (9.38114,1.26756);
\draw [c] (9.34322,1.24116) -- (9.38114,1.24116);
\draw [c] (9.38114,1.24116) -- (9.41907,1.24116);
\definecolor{c}{rgb}{0,0,0};
\colorlet{c}{natcomp!70};
\draw [c] (9.45699,1.22504) -- (9.45699,1.25181);
\draw [c] (9.45699,1.25181) -- (9.45699,1.27738);
\draw [c] (9.41907,1.25181) -- (9.45699,1.25181);
\draw [c] (9.45699,1.25181) -- (9.49492,1.25181);
\definecolor{c}{rgb}{0,0,0};
\colorlet{c}{natcomp!70};
\draw [c] (9.53284,1.2621) -- (9.53284,1.28829);
\draw [c] (9.53284,1.28829) -- (9.53284,1.31333);
\draw [c] (9.49492,1.28829) -- (9.53284,1.28829);
\draw [c] (9.53284,1.28829) -- (9.57076,1.28829);
\definecolor{c}{rgb}{0,0,0};
\colorlet{c}{natcomp!70};
\draw [c] (9.60869,1.1794) -- (9.60869,1.20761);
\draw [c] (9.60869,1.20761) -- (9.60869,1.23449);
\draw [c] (9.57076,1.20761) -- (9.60869,1.20761);
\draw [c] (9.60869,1.20761) -- (9.64661,1.20761);
\definecolor{c}{rgb}{0,0,0};
\colorlet{c}{natcomp!70};
\draw [c] (9.68453,1.17427) -- (9.68453,1.20235);
\draw [c] (9.68453,1.20235) -- (9.68453,1.22911);
\draw [c] (9.64661,1.20235) -- (9.68453,1.20235);
\draw [c] (9.68453,1.20235) -- (9.72246,1.20235);
\definecolor{c}{rgb}{0,0,0};
\colorlet{c}{natcomp!70};
\draw [c] (9.76038,1.13334) -- (9.76038,1.1631);
\draw [c] (9.76038,1.1631) -- (9.76038,1.19139);
\draw [c] (9.72246,1.1631) -- (9.76038,1.1631);
\draw [c] (9.76038,1.1631) -- (9.79831,1.1631);
\definecolor{c}{rgb}{0,0,0};
\colorlet{c}{natcomp!70};
\draw [c] (9.83623,1.0783) -- (9.83623,1.10911);
\draw [c] (9.83623,1.10911) -- (9.83623,1.13834);
\draw [c] (9.79831,1.10911) -- (9.83623,1.10911);
\draw [c] (9.83623,1.10911) -- (9.87415,1.10911);
\definecolor{c}{rgb}{0,0,0};
\colorlet{c}{natcomp!70};
\draw [c] (9.91208,1.11033) -- (9.91208,1.14063);
\draw [c] (9.91208,1.14063) -- (9.91208,1.16941);
\draw [c] (9.87415,1.14063) -- (9.91208,1.14063);
\draw [c] (9.91208,1.14063) -- (9.95,1.14063);
\definecolor{c}{rgb}{0,0,0};
\draw [c] (1,0.680516) -- (9.95,0.680516);
\draw [anchor= east] (9.95,0.108883) node[color=c, rotate=0]{$M_{\gamma\gamma}\text{ [GeV]}$};
\draw [c] (1,0.863234) -- (1,0.680516);
\draw [c] (1.37924,0.771875) -- (1.37924,0.680516);
\draw [c] (1.75847,0.771875) -- (1.75847,0.680516);
\draw [c] (2.13771,0.771875) -- (2.13771,0.680516);
\draw [c] (2.51695,0.863234) -- (2.51695,0.680516);
\draw [c] (2.89619,0.771875) -- (2.89619,0.680516);
\draw [c] (3.27542,0.771875) -- (3.27542,0.680516);
\draw [c] (3.65466,0.771875) -- (3.65466,0.680516);
\draw [c] (4.0339,0.863234) -- (4.0339,0.680516);
\draw [c] (4.41314,0.771875) -- (4.41314,0.680516);
\draw [c] (4.79237,0.771875) -- (4.79237,0.680516);
\draw [c] (5.17161,0.771875) -- (5.17161,0.680516);
\draw [c] (5.55085,0.863234) -- (5.55085,0.680516);
\draw [c] (5.93008,0.771875) -- (5.93008,0.680516);
\draw [c] (6.30932,0.771875) -- (6.30932,0.680516);
\draw [c] (6.68856,0.771875) -- (6.68856,0.680516);
\draw [c] (7.0678,0.863234) -- (7.0678,0.680516);
\draw [c] (7.44703,0.771875) -- (7.44703,0.680516);
\draw [c] (7.82627,0.771875) -- (7.82627,0.680516);
\draw [c] (8.20551,0.771875) -- (8.20551,0.680516);
\draw [c] (8.58475,0.863234) -- (8.58475,0.680516);
\draw [c] (8.58475,0.863234) -- (8.58475,0.680516);
\draw [c] (8.96398,0.771875) -- (8.96398,0.680516);
\draw [c] (9.34322,0.771875) -- (9.34322,0.680516);
\draw [c] (9.72246,0.771875) -- (9.72246,0.680516);
\draw [anchor=base] (1,0.353868) node[color=c, rotate=0]{0};
\draw [anchor=base] (2.51695,0.353868) node[color=c, rotate=0]{200};
\draw [anchor=base] (4.0339,0.353868) node[color=c, rotate=0]{400};
\draw [anchor=base] (5.55085,0.353868) node[color=c, rotate=0]{600};
\draw [anchor=base] (7.0678,0.353868) node[color=c, rotate=0]{800};
\draw [anchor=base] (8.58475,0.353868) node[color=c, rotate=0]{1000};
\draw [c] (1,0.680516) -- (1,6.73711);
\draw [anchor= east] (-0.12,6.73711) node[color=c, rotate=90]{Number of events};
\draw [c] (1.267,0.688075) -- (1,0.688075);
\draw [anchor= east] (0.922,0.688075) node[color=c, rotate=0]{$10^{-1}$};
\draw [c] (1.1335,1.08586) -- (1,1.08586);
\draw [c] (1.1335,1.31855) -- (1,1.31855);
\draw [c] (1.1335,1.48365) -- (1,1.48365);
\draw [c] (1.1335,1.61171) -- (1,1.61171);
\draw [c] (1.1335,1.71634) -- (1,1.71634);
\draw [c] (1.1335,1.80481) -- (1,1.80481);
\draw [c] (1.1335,1.88144) -- (1,1.88144);
\draw [c] (1.1335,1.94903) -- (1,1.94903);
\draw [c] (1.267,2.0095) -- (1,2.0095);
\draw [anchor= east] (0.922,2.0095) node[color=c, rotate=0]{1};
\draw [c] (1.1335,2.40729) -- (1,2.40729);
\draw [c] (1.1335,2.63998) -- (1,2.63998);
\draw [c] (1.1335,2.80507) -- (1,2.80507);
\draw [c] (1.1335,2.93313) -- (1,2.93313);
\draw [c] (1.1335,3.03776) -- (1,3.03776);
\draw [c] (1.1335,3.12623) -- (1,3.12623);
\draw [c] (1.1335,3.20286) -- (1,3.20286);
\draw [c] (1.1335,3.27046) -- (1,3.27046);
\draw [c] (1.267,3.33092) -- (1,3.33092);
\draw [anchor= east] (0.922,3.33092) node[color=c, rotate=0]{10};
\draw [c] (1.1335,3.72871) -- (1,3.72871);
\draw [c] (1.1335,3.9614) -- (1,3.9614);
\draw [c] (1.1335,4.1265) -- (1,4.1265);
\draw [c] (1.1335,4.25456) -- (1,4.25456);
\draw [c] (1.1335,4.35919) -- (1,4.35919);
\draw [c] (1.1335,4.44765) -- (1,4.44765);
\draw [c] (1.1335,4.52428) -- (1,4.52428);
\draw [c] (1.1335,4.59188) -- (1,4.59188);
\draw [c] (1.267,4.65234) -- (1,4.65234);
\draw [anchor= east] (0.922,4.65234) node[color=c, rotate=0]{$10^{2}$};
\draw [c] (1.1335,5.05013) -- (1,5.05013);
\draw [c] (1.1335,5.28282) -- (1,5.28282);
\draw [c] (1.1335,5.44792) -- (1,5.44792);
\draw [c] (1.1335,5.57598) -- (1,5.57598);
\draw [c] (1.1335,5.68061) -- (1,5.68061);
\draw [c] (1.1335,5.76908) -- (1,5.76908);
\draw [c] (1.1335,5.84571) -- (1,5.84571);
\draw [c] (1.1335,5.9133) -- (1,5.9133);
\draw [c] (1.267,5.97377) -- (1,5.97377);
\draw [anchor= east] (0.922,5.97377) node[color=c, rotate=0]{$10^{3}$};
\draw [c] (1.1335,6.37155) -- (1,6.37155);
\draw [c] (1.1335,6.60425) -- (1,6.60425);
\colorlet{c}{natgreen};
\draw [c] (1.56886,0.680516) -- (1.56886,2.32133);
\draw [c] (1.56886,2.32133) -- (1.56886,2.71912);
\draw [c] (1.53093,2.32133) -- (1.56886,2.32133);
\draw [c] (1.56886,2.32133) -- (1.60678,2.32133);
\definecolor{c}{rgb}{0,0,0};
\colorlet{c}{natgreen};
\draw [c] (1.6447,2.15154) -- (1.6447,2.77477);
\draw [c] (1.6447,2.77477) -- (1.6447,3.06646);
\draw [c] (1.60678,2.77477) -- (1.6447,2.77477);
\draw [c] (1.6447,2.77477) -- (1.68263,2.77477);
\definecolor{c}{rgb}{0,0,0};
\colorlet{c}{natgreen};
\draw [c] (1.72055,4.09914) -- (1.72055,4.23107);
\draw [c] (1.72055,4.23107) -- (1.72055,4.33827);
\draw [c] (1.68263,4.23107) -- (1.72055,4.23107);
\draw [c] (1.72055,4.23107) -- (1.75847,4.23107);
\definecolor{c}{rgb}{0,0,0};
\colorlet{c}{natgreen};
\draw [c] (1.7964,5.88673) -- (1.7964,5.91485);
\draw [c] (1.7964,5.91485) -- (1.7964,5.94166);
\draw [c] (1.75847,5.91485) -- (1.7964,5.91485);
\draw [c] (1.7964,5.91485) -- (1.83432,5.91485);
\definecolor{c}{rgb}{0,0,0};
\colorlet{c}{natgreen};
\draw [c] (1.87225,6.36331) -- (1.87225,6.38193);
\draw [c] (1.87225,6.38193) -- (1.87225,6.39997);
\draw [c] (1.83432,6.38193) -- (1.87225,6.38193);
\draw [c] (1.87225,6.38193) -- (1.91017,6.38193);
\definecolor{c}{rgb}{0,0,0};
\colorlet{c}{natgreen};
\draw [c] (1.94809,6.44614) -- (1.94809,6.46344);
\draw [c] (1.94809,6.46344) -- (1.94809,6.48024);
\draw [c] (1.91017,6.46344) -- (1.94809,6.46344);
\draw [c] (1.94809,6.46344) -- (1.98602,6.46344);
\definecolor{c}{rgb}{0,0,0};
\colorlet{c}{natgreen};
\draw [c] (2.02394,6.37812) -- (2.02394,6.39641);
\draw [c] (2.02394,6.39641) -- (2.02394,6.41414);
\draw [c] (1.98602,6.39641) -- (2.02394,6.39641);
\draw [c] (2.02394,6.39641) -- (2.06186,6.39641);
\definecolor{c}{rgb}{0,0,0};
\colorlet{c}{natgreen};
\draw [c] (2.09979,6.3115) -- (2.09979,6.3308);
\draw [c] (2.09979,6.3308) -- (2.09979,6.34947);
\draw [c] (2.06186,6.3308) -- (2.09979,6.3308);
\draw [c] (2.09979,6.3308) -- (2.13771,6.3308);
\definecolor{c}{rgb}{0,0,0};
\colorlet{c}{natgreen};
\draw [c] (2.17564,6.22934) -- (2.17564,6.25001);
\draw [c] (2.17564,6.25001) -- (2.17564,6.26996);
\draw [c] (2.13771,6.25001) -- (2.17564,6.25001);
\draw [c] (2.17564,6.25001) -- (2.21356,6.25001);
\definecolor{c}{rgb}{0,0,0};
\colorlet{c}{natgreen};
\draw [c] (2.25148,6.12923) -- (2.25148,6.15189);
\draw [c] (2.25148,6.15189) -- (2.25148,6.17369);
\draw [c] (2.21356,6.15189) -- (2.25148,6.15189);
\draw [c] (2.25148,6.15189) -- (2.28941,6.15189);
\definecolor{c}{rgb}{0,0,0};
\colorlet{c}{natgreen};
\draw [c] (2.32733,5.99974) -- (2.32733,6.02512);
\draw [c] (2.32733,6.02512) -- (2.32733,6.04941);
\draw [c] (2.28941,6.02512) -- (2.32733,6.02512);
\draw [c] (2.32733,6.02512) -- (2.36525,6.02512);
\definecolor{c}{rgb}{0,0,0};
\colorlet{c}{natgreen};
\draw [c] (2.40318,5.93984) -- (2.40318,5.96669);
\draw [c] (2.40318,5.96669) -- (2.40318,5.99234);
\draw [c] (2.36525,5.96669) -- (2.40318,5.96669);
\draw [c] (2.40318,5.96669) -- (2.4411,5.96669);
\definecolor{c}{rgb}{0,0,0};
\colorlet{c}{natgreen};
\draw [c] (2.47903,5.86198) -- (2.47903,5.89044);
\draw [c] (2.47903,5.89044) -- (2.47903,5.91755);
\draw [c] (2.4411,5.89044) -- (2.47903,5.89044);
\draw [c] (2.47903,5.89044) -- (2.51695,5.89044);
\definecolor{c}{rgb}{0,0,0};
\colorlet{c}{natgreen};
\draw [c] (2.55487,5.728) -- (2.55487,5.76007);
\draw [c] (2.55487,5.76007) -- (2.55487,5.79044);
\draw [c] (2.51695,5.76007) -- (2.55487,5.76007);
\draw [c] (2.55487,5.76007) -- (2.5928,5.76007);
\definecolor{c}{rgb}{0,0,0};
\colorlet{c}{natgreen};
\draw [c] (2.63072,5.662) -- (2.63072,5.69603);
\draw [c] (2.63072,5.69603) -- (2.63072,5.72815);
\draw [c] (2.5928,5.69603) -- (2.63072,5.69603);
\draw [c] (2.63072,5.69603) -- (2.66864,5.69603);
\definecolor{c}{rgb}{0,0,0};
\colorlet{c}{natgreen};
\draw [c] (2.70657,5.52065) -- (2.70657,5.55932);
\draw [c] (2.70657,5.55932) -- (2.70657,5.59556);
\draw [c] (2.66864,5.55932) -- (2.70657,5.55932);
\draw [c] (2.70657,5.55932) -- (2.74449,5.55932);
\definecolor{c}{rgb}{0,0,0};
\colorlet{c}{natgreen};
\draw [c] (2.78242,5.39535) -- (2.78242,5.43811);
\draw [c] (2.78242,5.43811) -- (2.78242,5.4779);
\draw [c] (2.74449,5.43811) -- (2.78242,5.43811);
\draw [c] (2.78242,5.43811) -- (2.82034,5.43811);
\definecolor{c}{rgb}{0,0,0};
\colorlet{c}{natgreen};
\draw [c] (2.85826,5.25104) -- (2.85826,5.2996);
\draw [c] (2.85826,5.2996) -- (2.85826,5.34438);
\draw [c] (2.82034,5.2996) -- (2.85826,5.2996);
\draw [c] (2.85826,5.2996) -- (2.89619,5.2996);
\definecolor{c}{rgb}{0,0,0};
\colorlet{c}{natgreen};
\draw [c] (2.93411,5.28231) -- (2.93411,5.32982);
\draw [c] (2.93411,5.32982) -- (2.93411,5.37369);
\draw [c] (2.89619,5.32982) -- (2.93411,5.32982);
\draw [c] (2.93411,5.32982) -- (2.97203,5.32982);
\definecolor{c}{rgb}{0,0,0};
\colorlet{c}{natgreen};
\draw [c] (3.00996,5.21897) -- (3.00996,5.2689);
\draw [c] (3.00996,5.2689) -- (3.00996,5.31483);
\draw [c] (2.97203,5.2689) -- (3.00996,5.2689);
\draw [c] (3.00996,5.2689) -- (3.04788,5.2689);
\definecolor{c}{rgb}{0,0,0};
\colorlet{c}{natgreen};
\draw [c] (3.08581,5.21023) -- (3.08581,5.2608);
\draw [c] (3.08581,5.2608) -- (3.08581,5.30728);
\draw [c] (3.04788,5.2608) -- (3.08581,5.2608);
\draw [c] (3.08581,5.2608) -- (3.12373,5.2608);
\definecolor{c}{rgb}{0,0,0};
\colorlet{c}{natgreen};
\draw [c] (3.16165,4.96886) -- (3.16165,5.03107);
\draw [c] (3.16165,5.03107) -- (3.16165,5.08718);
\draw [c] (3.12373,5.03107) -- (3.16165,5.03107);
\draw [c] (3.16165,5.03107) -- (3.19958,5.03107);
\definecolor{c}{rgb}{0,0,0};
\colorlet{c}{natgreen};
\draw [c] (3.2375,4.94124) -- (3.2375,5.00492);
\draw [c] (3.2375,5.00492) -- (3.2375,5.06224);
\draw [c] (3.19958,5.00492) -- (3.2375,5.00492);
\draw [c] (3.2375,5.00492) -- (3.27542,5.00492);
\definecolor{c}{rgb}{0,0,0};
\colorlet{c}{natgreen};
\draw [c] (3.31335,4.69119) -- (3.31335,4.76971);
\draw [c] (3.31335,4.76971) -- (3.31335,4.83877);
\draw [c] (3.27542,4.76971) -- (3.31335,4.76971);
\draw [c] (3.31335,4.76971) -- (3.35127,4.76971);
\definecolor{c}{rgb}{0,0,0};
\colorlet{c}{natgreen};
\draw [c] (3.38919,4.67612) -- (3.38919,4.75548);
\draw [c] (3.38919,4.75548) -- (3.38919,4.82519);
\draw [c] (3.35127,4.75548) -- (3.38919,4.75548);
\draw [c] (3.38919,4.75548) -- (3.42712,4.75548);
\definecolor{c}{rgb}{0,0,0};
\colorlet{c}{natgreen};
\draw [c] (3.46504,4.74061) -- (3.46504,4.81591);
\draw [c] (3.46504,4.81591) -- (3.46504,4.88246);
\draw [c] (3.42712,4.81591) -- (3.46504,4.81591);
\draw [c] (3.46504,4.81591) -- (3.50297,4.81591);
\definecolor{c}{rgb}{0,0,0};
\colorlet{c}{natgreen};
\draw [c] (3.54089,4.54627) -- (3.54089,4.63667);
\draw [c] (3.54089,4.63667) -- (3.54089,4.71475);
\draw [c] (3.50297,4.63667) -- (3.54089,4.63667);
\draw [c] (3.54089,4.63667) -- (3.57881,4.63667);
\definecolor{c}{rgb}{0,0,0};
\colorlet{c}{natgreen};
\draw [c] (3.61674,4.44837) -- (3.61674,4.54601);
\draw [c] (3.61674,4.54601) -- (3.61674,4.62944);
\draw [c] (3.57881,4.54601) -- (3.61674,4.54601);
\draw [c] (3.61674,4.54601) -- (3.65466,4.54601);
\definecolor{c}{rgb}{0,0,0};
\colorlet{c}{natgreen};
\draw [c] (3.69258,4.55513) -- (3.69258,4.64683);
\draw [c] (3.69258,4.64683) -- (3.69258,4.72588);
\draw [c] (3.65466,4.64683) -- (3.69258,4.64683);
\draw [c] (3.69258,4.64683) -- (3.73051,4.64683);
\definecolor{c}{rgb}{0,0,0};
\colorlet{c}{natgreen};
\draw [c] (3.76843,4.48937) -- (3.76843,4.58443);
\draw [c] (3.76843,4.58443) -- (3.76843,4.66596);
\draw [c] (3.73051,4.58443) -- (3.76843,4.58443);
\draw [c] (3.76843,4.58443) -- (3.80636,4.58443);
\definecolor{c}{rgb}{0,0,0};
\colorlet{c}{natgreen};
\draw [c] (3.84428,4.1449) -- (3.84428,4.26971);
\draw [c] (3.84428,4.26971) -- (3.84428,4.37217);
\draw [c] (3.80636,4.26971) -- (3.84428,4.26971);
\draw [c] (3.84428,4.26971) -- (3.8822,4.26971);
\definecolor{c}{rgb}{0,0,0};
\colorlet{c}{natgreen};
\draw [c] (3.92013,4.28662) -- (3.92013,4.40131);
\draw [c] (3.92013,4.40131) -- (3.92013,4.49685);
\draw [c] (3.8822,4.40131) -- (3.92013,4.40131);
\draw [c] (3.92013,4.40131) -- (3.95805,4.40131);
\definecolor{c}{rgb}{0,0,0};
\colorlet{c}{natgreen};
\draw [c] (3.99597,4.2373) -- (3.99597,4.35773);
\draw [c] (3.99597,4.35773) -- (3.99597,4.45722);
\draw [c] (3.95805,4.35773) -- (3.99597,4.35773);
\draw [c] (3.99597,4.35773) -- (4.0339,4.35773);
\definecolor{c}{rgb}{0,0,0};
\colorlet{c}{natgreen};
\draw [c] (4.07182,4.17735) -- (4.07182,4.30431);
\draw [c] (4.07182,4.30431) -- (4.07182,4.40821);
\draw [c] (4.0339,4.30431) -- (4.07182,4.30431);
\draw [c] (4.07182,4.30431) -- (4.10975,4.30431);
\definecolor{c}{rgb}{0,0,0};
\colorlet{c}{natgreen};
\draw [c] (4.14767,3.58812) -- (4.14767,3.79113);
\draw [c] (4.14767,3.79113) -- (4.14767,3.94079);
\draw [c] (4.10975,3.79113) -- (4.14767,3.79113);
\draw [c] (4.14767,3.79113) -- (4.18559,3.79113);
\definecolor{c}{rgb}{0,0,0};
\colorlet{c}{natgreen};
\draw [c] (4.22352,3.94542) -- (4.22352,4.0949);
\draw [c] (4.22352,4.0949) -- (4.22352,4.21338);
\draw [c] (4.18559,4.0949) -- (4.22352,4.0949);
\draw [c] (4.22352,4.0949) -- (4.26144,4.0949);
\definecolor{c}{rgb}{0,0,0};
\colorlet{c}{natgreen};
\draw [c] (4.29936,4.19993) -- (4.29936,4.3255);
\draw [c] (4.29936,4.3255) -- (4.29936,4.42847);
\draw [c] (4.26144,4.3255) -- (4.29936,4.3255);
\draw [c] (4.29936,4.3255) -- (4.33729,4.3255);
\definecolor{c}{rgb}{0,0,0};
\colorlet{c}{natgreen};
\draw [c] (4.37521,3.55265) -- (4.37521,3.76314);
\draw [c] (4.37521,3.76314) -- (4.37521,3.91681);
\draw [c] (4.33729,3.76314) -- (4.37521,3.76314);
\draw [c] (4.37521,3.76314) -- (4.41314,3.76314);
\definecolor{c}{rgb}{0,0,0};
\colorlet{c}{natgreen};
\draw [c] (4.45106,3.92155) -- (4.45106,4.07489);
\draw [c] (4.45106,4.07489) -- (4.45106,4.19578);
\draw [c] (4.41314,4.07489) -- (4.45106,4.07489);
\draw [c] (4.45106,4.07489) -- (4.48898,4.07489);
\definecolor{c}{rgb}{0,0,0};
\colorlet{c}{natgreen};
\draw [c] (4.52691,3.54663) -- (4.52691,3.75691);
\draw [c] (4.52691,3.75691) -- (4.52691,3.91046);
\draw [c] (4.48898,3.75691) -- (4.52691,3.75691);
\draw [c] (4.52691,3.75691) -- (4.56483,3.75691);
\definecolor{c}{rgb}{0,0,0};
\colorlet{c}{natgreen};
\draw [c] (4.60275,3.66877) -- (4.60275,3.8548);
\draw [c] (4.60275,3.8548) -- (4.60275,3.99507);
\draw [c] (4.56483,3.8548) -- (4.60275,3.8548);
\draw [c] (4.60275,3.8548) -- (4.64068,3.8548);
\definecolor{c}{rgb}{0,0,0};
\colorlet{c}{natgreen};
\draw [c] (4.6786,3.38407) -- (4.6786,3.62745);
\draw [c] (4.6786,3.62745) -- (4.6786,3.79781);
\draw [c] (4.64068,3.62745) -- (4.6786,3.62745);
\draw [c] (4.6786,3.62745) -- (4.71653,3.62745);
\definecolor{c}{rgb}{0,0,0};
\colorlet{c}{natgreen};
\draw [c] (4.75445,3.33733) -- (4.75445,3.56304);
\draw [c] (4.75445,3.56304) -- (4.75445,3.72461);
\draw [c] (4.71653,3.56304) -- (4.75445,3.56304);
\draw [c] (4.75445,3.56304) -- (4.79237,3.56304);
\definecolor{c}{rgb}{0,0,0};
\colorlet{c}{natgreen};
\draw [c] (4.8303,3.82832) -- (4.8303,3.99795);
\draw [c] (4.8303,3.99795) -- (4.8303,4.12871);
\draw [c] (4.79237,3.99795) -- (4.8303,3.99795);
\draw [c] (4.8303,3.99795) -- (4.86822,3.99795);
\definecolor{c}{rgb}{0,0,0};
\colorlet{c}{natgreen};
\draw [c] (4.90614,3.33704) -- (4.90614,3.58231);
\draw [c] (4.90614,3.58231) -- (4.90614,3.7536);
\draw [c] (4.86822,3.58231) -- (4.90614,3.58231);
\draw [c] (4.90614,3.58231) -- (4.94407,3.58231);
\definecolor{c}{rgb}{0,0,0};
\colorlet{c}{natgreen};
\draw [c] (4.98199,3.5684) -- (4.98199,3.77461);
\draw [c] (4.98199,3.77461) -- (4.98199,3.92599);
\draw [c] (4.94407,3.77461) -- (4.98199,3.77461);
\draw [c] (4.98199,3.77461) -- (5.01992,3.77461);
\definecolor{c}{rgb}{0,0,0};
\colorlet{c}{natgreen};
\draw [c] (5.05784,3.37991) -- (5.05784,3.63276);
\draw [c] (5.05784,3.63276) -- (5.05784,3.80768);
\draw [c] (5.01992,3.63276) -- (5.05784,3.63276);
\draw [c] (5.05784,3.63276) -- (5.09576,3.63276);
\definecolor{c}{rgb}{0,0,0};
\colorlet{c}{natgreen};
\draw [c] (5.13369,3.09137) -- (5.13369,3.38505);
\draw [c] (5.13369,3.38505) -- (5.13369,3.57837);
\draw [c] (5.09576,3.38505) -- (5.13369,3.38505);
\draw [c] (5.13369,3.38505) -- (5.17161,3.38505);
\definecolor{c}{rgb}{0,0,0};
\colorlet{c}{natgreen};
\draw [c] (5.20953,2.62672) -- (5.20953,3.12268);
\draw [c] (5.20953,3.12268) -- (5.20953,3.38468);
\draw [c] (5.17161,3.12268) -- (5.20953,3.12268);
\draw [c] (5.20953,3.12268) -- (5.24746,3.12268);
\definecolor{c}{rgb}{0,0,0};
\colorlet{c}{natgreen};
\draw [c] (5.28538,3.30387) -- (5.28538,3.56756);
\draw [c] (5.28538,3.56756) -- (5.28538,3.74755);
\draw [c] (5.24746,3.56756) -- (5.28538,3.56756);
\draw [c] (5.28538,3.56756) -- (5.32331,3.56756);
\definecolor{c}{rgb}{0,0,0};
\colorlet{c}{natgreen};
\draw [c] (5.36123,3.06923) -- (5.36123,3.39873);
\draw [c] (5.36123,3.39873) -- (5.36123,3.60673);
\draw [c] (5.32331,3.39873) -- (5.36123,3.39873);
\draw [c] (5.36123,3.39873) -- (5.39915,3.39873);
\definecolor{c}{rgb}{0,0,0};
\colorlet{c}{natgreen};
\draw [c] (5.43708,2.75701) -- (5.43708,3.13388);
\draw [c] (5.43708,3.13388) -- (5.43708,3.35943);
\draw [c] (5.39915,3.13388) -- (5.43708,3.13388);
\draw [c] (5.43708,3.13388) -- (5.475,3.13388);
\definecolor{c}{rgb}{0,0,0};
\colorlet{c}{natgreen};
\draw [c] (5.58877,2.38329) -- (5.58877,2.91028);
\draw [c] (5.58877,2.91028) -- (5.58877,3.18029);
\draw [c] (5.55085,2.91028) -- (5.58877,2.91028);
\draw [c] (5.58877,2.91028) -- (5.6267,2.91028);
\definecolor{c}{rgb}{0,0,0};
\colorlet{c}{natgreen};
\draw [c] (5.66462,2.31794) -- (5.66462,2.87603);
\draw [c] (5.66462,2.87603) -- (5.66462,3.15354);
\draw [c] (5.6267,2.87603) -- (5.66462,2.87603);
\draw [c] (5.66462,2.87603) -- (5.70254,2.87603);
\definecolor{c}{rgb}{0,0,0};
\colorlet{c}{natgreen};
\draw [c] (5.74047,2.64865) -- (5.74047,3.14387);
\draw [c] (5.74047,3.14387) -- (5.74047,3.40568);
\draw [c] (5.70254,3.14387) -- (5.74047,3.14387);
\draw [c] (5.74047,3.14387) -- (5.77839,3.14387);
\definecolor{c}{rgb}{0,0,0};
\colorlet{c}{natgreen};
\draw [c] (5.81631,2.63248) -- (5.81631,3.01946);
\draw [c] (5.81631,3.01946) -- (5.81631,3.2485);
\draw [c] (5.77839,3.01946) -- (5.81631,3.01946);
\draw [c] (5.81631,3.01946) -- (5.85424,3.01946);
\definecolor{c}{rgb}{0,0,0};
\colorlet{c}{natgreen};
\draw [c] (5.89216,3.02506) -- (5.89216,3.35228);
\draw [c] (5.89216,3.35228) -- (5.89216,3.55938);
\draw [c] (5.85424,3.35228) -- (5.89216,3.35228);
\draw [c] (5.89216,3.35228) -- (5.93008,3.35228);
\definecolor{c}{rgb}{0,0,0};
\colorlet{c}{natgreen};
\draw [c] (5.96801,2.96904) -- (5.96801,3.3376);
\draw [c] (5.96801,3.3376) -- (5.96801,3.5602);
\draw [c] (5.93008,3.3376) -- (5.96801,3.3376);
\draw [c] (5.96801,3.3376) -- (6.00593,3.3376);
\definecolor{c}{rgb}{0,0,0};
\colorlet{c}{natgreen};
\draw [c] (6.04386,2.13625) -- (6.04386,2.84097);
\draw [c] (6.04386,2.84097) -- (6.04386,3.14788);
\draw [c] (6.00593,2.84097) -- (6.04386,2.84097);
\draw [c] (6.04386,2.84097) -- (6.08178,2.84097);
\definecolor{c}{rgb}{0,0,0};
\colorlet{c}{natgreen};
\draw [c] (6.1197,0.680516) -- (6.1197,2.38875);
\draw [c] (6.1197,2.38875) -- (6.1197,2.78654);
\draw [c] (6.08178,2.38875) -- (6.1197,2.38875);
\draw [c] (6.1197,2.38875) -- (6.15763,2.38875);
\definecolor{c}{rgb}{0,0,0};
\colorlet{c}{natgreen};
\draw [c] (6.19555,2.54233) -- (6.19555,3.0409);
\draw [c] (6.19555,3.0409) -- (6.19555,3.3036);
\draw [c] (6.15763,3.0409) -- (6.19555,3.0409);
\draw [c] (6.19555,3.0409) -- (6.23347,3.0409);
\definecolor{c}{rgb}{0,0,0};
\colorlet{c}{natgreen};
\draw [c] (6.2714,2.75147) -- (6.2714,3.15831);
\draw [c] (6.2714,3.15831) -- (6.2714,3.39399);
\draw [c] (6.23347,3.15831) -- (6.2714,3.15831);
\draw [c] (6.2714,3.15831) -- (6.30932,3.15831);
\definecolor{c}{rgb}{0,0,0};
\colorlet{c}{natgreen};
\draw [c] (6.34725,1.87883) -- (6.34725,2.64933);
\draw [c] (6.34725,2.64933) -- (6.34725,2.96681);
\draw [c] (6.30932,2.64933) -- (6.34725,2.64933);
\draw [c] (6.34725,2.64933) -- (6.38517,2.64933);
\definecolor{c}{rgb}{0,0,0};
\colorlet{c}{natgreen};
\draw [c] (6.42309,2.61419) -- (6.42309,3.10871);
\draw [c] (6.42309,3.10871) -- (6.42309,3.37033);
\draw [c] (6.38517,3.10871) -- (6.42309,3.10871);
\draw [c] (6.42309,3.10871) -- (6.46102,3.10871);
\definecolor{c}{rgb}{0,0,0};
\colorlet{c}{natgreen};
\draw [c] (6.57479,1.35007) -- (6.57479,2.57712);
\draw [c] (6.57479,2.57712) -- (6.57479,2.94005);
\draw [c] (6.53686,2.57712) -- (6.57479,2.57712);
\draw [c] (6.57479,2.57712) -- (6.61271,2.57712);
\definecolor{c}{rgb}{0,0,0};
\colorlet{c}{natgreen};
\draw [c] (6.65064,2.86211) -- (6.65064,3.23882);
\draw [c] (6.65064,3.23882) -- (6.65064,3.4643);
\draw [c] (6.61271,3.23882) -- (6.65064,3.23882);
\draw [c] (6.65064,3.23882) -- (6.68856,3.23882);
\definecolor{c}{rgb}{0,0,0};
\colorlet{c}{natgreen};
\draw [c] (6.87818,0.680516) -- (6.87818,1.90334);
\draw [c] (6.87818,1.90334) -- (6.87818,2.30113);
\draw [c] (6.84025,1.90334) -- (6.87818,1.90334);
\draw [c] (6.87818,1.90334) -- (6.9161,1.90334);
\definecolor{c}{rgb}{0,0,0};
\colorlet{c}{natgreen};
\draw [c] (7.02987,1.97897) -- (7.02987,2.7321);
\draw [c] (7.02987,2.7321) -- (7.02987,3.04692);
\draw [c] (6.99195,2.7321) -- (7.02987,2.7321);
\draw [c] (7.02987,2.7321) -- (7.0678,2.7321);
\definecolor{c}{rgb}{0,0,0};
\colorlet{c}{natgreen};
\draw [c] (7.18157,0.680516) -- (7.18157,2.58979);
\draw [c] (7.18157,2.58979) -- (7.18157,2.98757);
\draw [c] (7.14364,2.58979) -- (7.18157,2.58979);
\draw [c] (7.18157,2.58979) -- (7.21949,2.58979);
\definecolor{c}{rgb}{0,0,0};
\colorlet{c}{natgreen};
\draw [c] (7.33326,0.680516) -- (7.33326,2.24583);
\draw [c] (7.33326,2.24583) -- (7.33326,2.64362);
\draw [c] (7.29534,2.24583) -- (7.33326,2.24583);
\draw [c] (7.33326,2.24583) -- (7.37119,2.24583);
\definecolor{c}{rgb}{0,0,0};
\colorlet{c}{natgreen};
\draw [c] (7.63665,0.680516) -- (7.63665,2.43257);
\draw [c] (7.63665,2.43257) -- (7.63665,2.83036);
\draw [c] (7.59873,2.43257) -- (7.63665,2.43257);
\draw [c] (7.63665,2.43257) -- (7.67458,2.43257);
\definecolor{c}{rgb}{0,0,0};
\colorlet{c}{natgreen};
\draw [c] (7.78835,0.680516) -- (7.78835,2.25707);
\draw [c] (7.78835,2.25707) -- (7.78835,2.65486);
\draw [c] (7.75042,2.25707) -- (7.78835,2.25707);
\draw [c] (7.78835,2.25707) -- (7.82627,2.25707);
\definecolor{c}{rgb}{0,0,0};
\colorlet{c}{natgreen};
\draw [c] (7.86419,0.680516) -- (7.86419,2.4637);
\draw [c] (7.86419,2.4637) -- (7.86419,2.86148);
\draw [c] (7.82627,2.4637) -- (7.86419,2.4637);
\draw [c] (7.86419,2.4637) -- (7.90212,2.4637);
\definecolor{c}{rgb}{0,0,0};
\colorlet{c}{natgreen};
\draw [c] (8.01589,2.10948) -- (8.01589,2.81591);
\draw [c] (8.01589,2.81591) -- (8.01589,3.12312);
\draw [c] (7.97797,2.81591) -- (8.01589,2.81591);
\draw [c] (8.01589,2.81591) -- (8.05381,2.81591);
\definecolor{c}{rgb}{0,0,0};
\colorlet{c}{natgreen};
\draw [c] (8.09174,2.16055) -- (8.09174,2.86528);
\draw [c] (8.09174,2.86528) -- (8.09174,3.1722);
\draw [c] (8.05381,2.86528) -- (8.09174,2.86528);
\draw [c] (8.09174,2.86528) -- (8.12966,2.86528);
\definecolor{c}{rgb}{0,0,0};
\colorlet{c}{natgreen};
\draw [c] (8.24343,1.97776) -- (8.24343,2.72975);
\draw [c] (8.24343,2.72975) -- (8.24343,3.04439);
\draw [c] (8.20551,2.72975) -- (8.24343,2.72975);
\draw [c] (8.24343,2.72975) -- (8.28136,2.72975);
\definecolor{c}{rgb}{0,0,0};
\colorlet{c}{natgreen};
\draw [c] (8.39513,0.680516) -- (8.39513,2.49913);
\draw [c] (8.39513,2.49913) -- (8.39513,2.89692);
\draw [c] (8.3572,2.49913) -- (8.39513,2.49913);
\draw [c] (8.39513,2.49913) -- (8.43305,2.49913);
\definecolor{c}{rgb}{0,0,0};
\colorlet{c}{natgreen};
\draw [c] (9.60869,0.680516) -- (9.60869,2.58979);
\draw [c] (9.60869,2.58979) -- (9.60869,2.98757);
\draw [c] (9.57076,2.58979) -- (9.60869,2.58979);
\draw [c] (9.60869,2.58979) -- (9.64661,2.58979);
\definecolor{c}{rgb}{0,0,0};
\draw [anchor= west] (6.52221,6.06375) node[color=c, rotate=0]{ATLAS MC};
\colorlet{c}{natgreen};
\draw [c] (5.77633,6.06375) -- (6.39058,6.06375);
\draw [c] (6.08345,5.87679) -- (6.08345,6.25072);
\definecolor{c}{rgb}{0,0,0};
\draw [anchor= west] (6.52221,5.44054) node[color=c, rotate=0]{CalcHEP + box};
\colorlet{c}{natcomp!70};
\draw [c] (5.77633,5.44054) -- (6.39058,5.44054);
\draw [c] (6.08345,5.25358) -- (6.08345,5.62751);
\end{tikzpicture}

\end{infilsf}
\end{minipage}
\begin{minipage}[b]{.3\textwidth}
\caption{Comparing the \atlas{} distribution with the one produced by CalcHEP, combined with a distribution for the box diagram contribution.}\label{ggcomp}
\end{minipage}
\end{figure}

As the data background and box diagram distributions have the same sort of shape, it is natural to combine them before attempting to extrapolate a shape. The function we fit is [some power function... $\chi^2$ fit with probability \~$10^{-5}$, I think...] result in fig.~\ref{bckfit}.

\begin{figure}[htp]
\begin{minipage}[b]{.69\textwidth}
\begin{infilsf} \tiny
\begin{tikzpicture}[x=.092\textwidth,y=.092\textwidth]
\pgfdeclareplotmark{cross} {
\pgfpathmoveto{\pgfpoint{-0.3\pgfplotmarksize}{\pgfplotmarksize}}
\pgfpathlineto{\pgfpoint{+0.3\pgfplotmarksize}{\pgfplotmarksize}}
\pgfpathlineto{\pgfpoint{+0.3\pgfplotmarksize}{0.3\pgfplotmarksize}}
\pgfpathlineto{\pgfpoint{+1\pgfplotmarksize}{0.3\pgfplotmarksize}}
\pgfpathlineto{\pgfpoint{+1\pgfplotmarksize}{-0.3\pgfplotmarksize}}
\pgfpathlineto{\pgfpoint{+0.3\pgfplotmarksize}{-0.3\pgfplotmarksize}}
\pgfpathlineto{\pgfpoint{+0.3\pgfplotmarksize}{-1.\pgfplotmarksize}}
\pgfpathlineto{\pgfpoint{-0.3\pgfplotmarksize}{-1.\pgfplotmarksize}}
\pgfpathlineto{\pgfpoint{-0.3\pgfplotmarksize}{-0.3\pgfplotmarksize}}
\pgfpathlineto{\pgfpoint{-1.\pgfplotmarksize}{-0.3\pgfplotmarksize}}
\pgfpathlineto{\pgfpoint{-1.\pgfplotmarksize}{0.3\pgfplotmarksize}}
\pgfpathlineto{\pgfpoint{-0.3\pgfplotmarksize}{0.3\pgfplotmarksize}}
\pgfpathclose
\pgfusepathqstroke
}
\pgfdeclareplotmark{cross*} {
\pgfpathmoveto{\pgfpoint{-0.3\pgfplotmarksize}{\pgfplotmarksize}}
\pgfpathlineto{\pgfpoint{+0.3\pgfplotmarksize}{\pgfplotmarksize}}
\pgfpathlineto{\pgfpoint{+0.3\pgfplotmarksize}{0.3\pgfplotmarksize}}
\pgfpathlineto{\pgfpoint{+1\pgfplotmarksize}{0.3\pgfplotmarksize}}
\pgfpathlineto{\pgfpoint{+1\pgfplotmarksize}{-0.3\pgfplotmarksize}}
\pgfpathlineto{\pgfpoint{+0.3\pgfplotmarksize}{-0.3\pgfplotmarksize}}
\pgfpathlineto{\pgfpoint{+0.3\pgfplotmarksize}{-1.\pgfplotmarksize}}
\pgfpathlineto{\pgfpoint{-0.3\pgfplotmarksize}{-1.\pgfplotmarksize}}
\pgfpathlineto{\pgfpoint{-0.3\pgfplotmarksize}{-0.3\pgfplotmarksize}}
\pgfpathlineto{\pgfpoint{-1.\pgfplotmarksize}{-0.3\pgfplotmarksize}}
\pgfpathlineto{\pgfpoint{-1.\pgfplotmarksize}{0.3\pgfplotmarksize}}
\pgfpathlineto{\pgfpoint{-0.3\pgfplotmarksize}{0.3\pgfplotmarksize}}
\pgfpathclose
\pgfusepathqfillstroke
}
\pgfdeclareplotmark{newstar} {
\pgfpathmoveto{\pgfqpoint{0pt}{\pgfplotmarksize}}
\pgfpathlineto{\pgfqpointpolar{44}{0.5\pgfplotmarksize}}
\pgfpathlineto{\pgfqpointpolar{18}{\pgfplotmarksize}}
\pgfpathlineto{\pgfqpointpolar{-20}{0.5\pgfplotmarksize}}
\pgfpathlineto{\pgfqpointpolar{-54}{\pgfplotmarksize}}
\pgfpathlineto{\pgfqpointpolar{-90}{0.5\pgfplotmarksize}}
\pgfpathlineto{\pgfqpointpolar{234}{\pgfplotmarksize}}
\pgfpathlineto{\pgfqpointpolar{198}{0.5\pgfplotmarksize}}
\pgfpathlineto{\pgfqpointpolar{162}{\pgfplotmarksize}}
\pgfpathlineto{\pgfqpointpolar{134}{0.5\pgfplotmarksize}}
\pgfpathclose
\pgfusepathqstroke
}
\pgfdeclareplotmark{newstar*} {
\pgfpathmoveto{\pgfqpoint{0pt}{\pgfplotmarksize}}
\pgfpathlineto{\pgfqpointpolar{44}{0.5\pgfplotmarksize}}
\pgfpathlineto{\pgfqpointpolar{18}{\pgfplotmarksize}}
\pgfpathlineto{\pgfqpointpolar{-20}{0.5\pgfplotmarksize}}
\pgfpathlineto{\pgfqpointpolar{-54}{\pgfplotmarksize}}
\pgfpathlineto{\pgfqpointpolar{-90}{0.5\pgfplotmarksize}}
\pgfpathlineto{\pgfqpointpolar{234}{\pgfplotmarksize}}
\pgfpathlineto{\pgfqpointpolar{198}{0.5\pgfplotmarksize}}
\pgfpathlineto{\pgfqpointpolar{162}{\pgfplotmarksize}}
\pgfpathlineto{\pgfqpointpolar{134}{0.5\pgfplotmarksize}}
\pgfpathclose
\pgfusepathqfillstroke
}
\definecolor{c}{rgb}{1,1,1};
\draw [color=c, fill=c] (0,0) rectangle (10,6.80516);
\draw [color=c, fill=c] (1,0.680516) rectangle (9.95,6.73711);
\definecolor{c}{rgb}{0,0,0};
\draw [c] (1,0.680516) -- (1,6.73711) -- (9.95,6.73711) -- (9.95,0.680516) -- (1,0.680516);
\definecolor{c}{rgb}{1,1,1};
\draw [color=c, fill=c] (1,0.680516) rectangle (9.95,6.73711);
\definecolor{c}{rgb}{0,0,0};
\draw [c] (1,0.680516) -- (1,6.73711) -- (9.95,6.73711) -- (9.95,0.680516) -- (1,0.680516);
\colorlet{c}{natgreen};
\draw [c] (1.04431,3.58796) -- (1.04431,3.68717);
\draw [c] (1.04431,3.68717) -- (1.04431,3.77009);
\draw [c] (1,3.68717) -- (1.04431,3.68717);
\draw [c] (1.04431,3.68717) -- (1.08861,3.68717);
\definecolor{c}{rgb}{0,0,0};
\colorlet{c}{natgreen};
\draw [c] (1.13292,4.52557) -- (1.13292,4.56494);
\draw [c] (1.13292,4.56494) -- (1.13292,4.60147);
\draw [c] (1.08861,4.56494) -- (1.13292,4.56494);
\draw [c] (1.13292,4.56494) -- (1.17723,4.56494);
\definecolor{c}{rgb}{0,0,0};
\colorlet{c}{natgreen};
\draw [c] (1.22153,4.55305) -- (1.22153,4.59136);
\draw [c] (1.22153,4.59136) -- (1.22153,4.62698);
\draw [c] (1.17723,4.59136) -- (1.22153,4.59136);
\draw [c] (1.22153,4.59136) -- (1.26584,4.59136);
\definecolor{c}{rgb}{0,0,0};
\colorlet{c}{natgreen};
\draw [c] (1.31015,4.84177) -- (1.31015,4.87058);
\draw [c] (1.31015,4.87058) -- (1.31015,4.89784);
\draw [c] (1.26584,4.87058) -- (1.31015,4.87058);
\draw [c] (1.31015,4.87058) -- (1.35446,4.87058);
\definecolor{c}{rgb}{0,0,0};
\colorlet{c}{natgreen};
\draw [c] (1.39876,4.85866) -- (1.39876,4.887);
\draw [c] (1.39876,4.887) -- (1.39876,4.91384);
\draw [c] (1.35446,4.887) -- (1.39876,4.887);
\draw [c] (1.39876,4.887) -- (1.44307,4.887);
\definecolor{c}{rgb}{0,0,0};
\colorlet{c}{natgreen};
\draw [c] (1.48738,4.90421) -- (1.48738,4.9313);
\draw [c] (1.48738,4.9313) -- (1.48738,4.95702);
\draw [c] (1.44307,4.9313) -- (1.48738,4.9313);
\draw [c] (1.48738,4.9313) -- (1.53168,4.9313);
\definecolor{c}{rgb}{0,0,0};
\colorlet{c}{natgreen};
\draw [c] (1.57599,4.86959) -- (1.57599,4.89763);
\draw [c] (1.57599,4.89763) -- (1.57599,4.92419);
\draw [c] (1.53168,4.89763) -- (1.57599,4.89763);
\draw [c] (1.57599,4.89763) -- (1.6203,4.89763);
\definecolor{c}{rgb}{0,0,0};
\colorlet{c}{natgreen};
\draw [c] (1.6646,4.99569) -- (1.6646,5.02044);
\draw [c] (1.6646,5.02044) -- (1.6646,5.04403);
\draw [c] (1.6203,5.02044) -- (1.6646,5.02044);
\draw [c] (1.6646,5.02044) -- (1.70891,5.02044);
\definecolor{c}{rgb}{0,0,0};
\colorlet{c}{natgreen};
\draw [c] (1.75322,5.16291) -- (1.75322,5.18389);
\draw [c] (1.75322,5.18389) -- (1.75322,5.20403);
\draw [c] (1.70891,5.18389) -- (1.75322,5.18389);
\draw [c] (1.75322,5.18389) -- (1.79752,5.18389);
\definecolor{c}{rgb}{0,0,0};
\colorlet{c}{natgreen};
\draw [c] (1.84183,5.12662) -- (1.84183,5.14828);
\draw [c] (1.84183,5.14828) -- (1.84183,5.16906);
\draw [c] (1.79752,5.14828) -- (1.84183,5.14828);
\draw [c] (1.84183,5.14828) -- (1.88614,5.14828);
\definecolor{c}{rgb}{0,0,0};
\colorlet{c}{natgreen};
\draw [c] (1.93045,5.86287) -- (1.93045,5.87324);
\draw [c] (1.93045,5.87324) -- (1.93045,5.88341);
\draw [c] (1.88614,5.87324) -- (1.93045,5.87324);
\draw [c] (1.93045,5.87324) -- (1.97475,5.87324);
\definecolor{c}{rgb}{0,0,0};
\colorlet{c}{natgreen};
\draw [c] (2.01906,6.27256) -- (2.01906,6.27949);
\draw [c] (2.01906,6.27949) -- (2.01906,6.28633);
\draw [c] (1.97475,6.27949) -- (2.01906,6.27949);
\draw [c] (2.01906,6.27949) -- (2.06337,6.27949);
\definecolor{c}{rgb}{0,0,0};
\colorlet{c}{natgreen};
\draw [c] (2.10767,6.37754) -- (2.10767,6.3838);
\draw [c] (2.10767,6.3838) -- (2.10767,6.38998);
\draw [c] (2.06337,6.3838) -- (2.10767,6.3838);
\draw [c] (2.10767,6.3838) -- (2.15198,6.3838);
\definecolor{c}{rgb}{0,0,0};
\colorlet{c}{natgreen};
\draw [c] (2.19629,6.40128) -- (2.19629,6.4074);
\draw [c] (2.19629,6.4074) -- (2.19629,6.41345);
\draw [c] (2.15198,6.4074) -- (2.19629,6.4074);
\draw [c] (2.19629,6.4074) -- (2.24059,6.4074);
\definecolor{c}{rgb}{0,0,0};
\colorlet{c}{natgreen};
\draw [c] (2.2849,6.36434) -- (2.2849,6.3707);
\draw [c] (2.2849,6.3707) -- (2.2849,6.37697);
\draw [c] (2.24059,6.3707) -- (2.2849,6.3707);
\draw [c] (2.2849,6.3707) -- (2.32921,6.3707);
\definecolor{c}{rgb}{0,0,0};
\colorlet{c}{natgreen};
\draw [c] (2.37351,6.33688) -- (2.37351,6.34342);
\draw [c] (2.37351,6.34342) -- (2.37351,6.34987);
\draw [c] (2.32921,6.34342) -- (2.37351,6.34342);
\draw [c] (2.37351,6.34342) -- (2.41782,6.34342);
\definecolor{c}{rgb}{0,0,0};
\colorlet{c}{natgreen};
\draw [c] (2.46213,6.23385) -- (2.46213,6.24109);
\draw [c] (2.46213,6.24109) -- (2.46213,6.24822);
\draw [c] (2.41782,6.24109) -- (2.46213,6.24109);
\draw [c] (2.46213,6.24109) -- (2.50644,6.24109);
\definecolor{c}{rgb}{0,0,0};
\colorlet{c}{natgreen};
\draw [c] (2.55074,6.16569) -- (2.55074,6.17343);
\draw [c] (2.55074,6.17343) -- (2.55074,6.18107);
\draw [c] (2.50644,6.17343) -- (2.55074,6.17343);
\draw [c] (2.55074,6.17343) -- (2.59505,6.17343);
\definecolor{c}{rgb}{0,0,0};
\colorlet{c}{natgreen};
\draw [c] (2.63936,6.09344) -- (2.63936,6.10177);
\draw [c] (2.63936,6.10177) -- (2.63936,6.10996);
\draw [c] (2.59505,6.10177) -- (2.63936,6.10177);
\draw [c] (2.63936,6.10177) -- (2.68366,6.10177);
\definecolor{c}{rgb}{0,0,0};
\colorlet{c}{natgreen};
\draw [c] (2.72797,6.00711) -- (2.72797,6.01618);
\draw [c] (2.72797,6.01618) -- (2.72797,6.02509);
\draw [c] (2.68366,6.01618) -- (2.72797,6.01618);
\draw [c] (2.72797,6.01618) -- (2.77228,6.01618);
\definecolor{c}{rgb}{0,0,0};
\colorlet{c}{natgreen};
\draw [c] (2.81658,5.93403) -- (2.81658,5.94378);
\draw [c] (2.81658,5.94378) -- (2.81658,5.95336);
\draw [c] (2.77228,5.94378) -- (2.81658,5.94378);
\draw [c] (2.81658,5.94378) -- (2.86089,5.94378);
\definecolor{c}{rgb}{0,0,0};
\colorlet{c}{natgreen};
\draw [c] (2.9052,5.84107) -- (2.9052,5.85176);
\draw [c] (2.9052,5.85176) -- (2.9052,5.86224);
\draw [c] (2.86089,5.85176) -- (2.9052,5.85176);
\draw [c] (2.9052,5.85176) -- (2.94951,5.85176);
\definecolor{c}{rgb}{0,0,0};
\colorlet{c}{natgreen};
\draw [c] (2.99381,5.75037) -- (2.99381,5.76207);
\draw [c] (2.99381,5.76207) -- (2.99381,5.77351);
\draw [c] (2.94951,5.76207) -- (2.99381,5.76207);
\draw [c] (2.99381,5.76207) -- (3.03812,5.76207);
\definecolor{c}{rgb}{0,0,0};
\colorlet{c}{natgreen};
\draw [c] (3.08243,5.71391) -- (3.08243,5.72604);
\draw [c] (3.08243,5.72604) -- (3.08243,5.73789);
\draw [c] (3.03812,5.72604) -- (3.08243,5.72604);
\draw [c] (3.08243,5.72604) -- (3.12673,5.72604);
\definecolor{c}{rgb}{0,0,0};
\colorlet{c}{natgreen};
\draw [c] (3.17104,5.61884) -- (3.17104,5.63216);
\draw [c] (3.17104,5.63216) -- (3.17104,5.64513);
\draw [c] (3.12673,5.63216) -- (3.17104,5.63216);
\draw [c] (3.17104,5.63216) -- (3.21535,5.63216);
\definecolor{c}{rgb}{0,0,0};
\colorlet{c}{natgreen};
\draw [c] (3.25965,5.53981) -- (3.25965,5.55422);
\draw [c] (3.25965,5.55422) -- (3.25965,5.56823);
\draw [c] (3.21535,5.55422) -- (3.25965,5.55422);
\draw [c] (3.25965,5.55422) -- (3.30396,5.55422);
\definecolor{c}{rgb}{0,0,0};
\colorlet{c}{natgreen};
\draw [c] (3.34827,5.4505) -- (3.34827,5.46623);
\draw [c] (3.34827,5.46623) -- (3.34827,5.48148);
\draw [c] (3.30396,5.46623) -- (3.34827,5.46623);
\draw [c] (3.34827,5.46623) -- (3.39257,5.46623);
\definecolor{c}{rgb}{0,0,0};
\colorlet{c}{natgreen};
\draw [c] (3.43688,5.3618) -- (3.43688,5.37899);
\draw [c] (3.43688,5.37899) -- (3.43688,5.39562);
\draw [c] (3.39257,5.37899) -- (3.43688,5.37899);
\draw [c] (3.43688,5.37899) -- (3.48119,5.37899);
\definecolor{c}{rgb}{0,0,0};
\colorlet{c}{natgreen};
\draw [c] (3.5255,5.3192) -- (3.5255,5.33712);
\draw [c] (3.5255,5.33712) -- (3.5255,5.35443);
\draw [c] (3.48119,5.33712) -- (3.5255,5.33712);
\draw [c] (3.5255,5.33712) -- (3.5698,5.33712);
\definecolor{c}{rgb}{0,0,0};
\colorlet{c}{natgreen};
\draw [c] (3.61411,5.29761) -- (3.61411,5.31592);
\draw [c] (3.61411,5.31592) -- (3.61411,5.33358);
\draw [c] (3.5698,5.31592) -- (3.61411,5.31592);
\draw [c] (3.61411,5.31592) -- (3.65842,5.31592);
\definecolor{c}{rgb}{0,0,0};
\colorlet{c}{natgreen};
\draw [c] (3.70272,5.20597) -- (3.70272,5.22601);
\draw [c] (3.70272,5.22601) -- (3.70272,5.2453);
\draw [c] (3.65842,5.22601) -- (3.70272,5.22601);
\draw [c] (3.70272,5.22601) -- (3.74703,5.22601);
\definecolor{c}{rgb}{0,0,0};
\colorlet{c}{natgreen};
\draw [c] (3.79134,5.13236) -- (3.79134,5.15391);
\draw [c] (3.79134,5.15391) -- (3.79134,5.17457);
\draw [c] (3.74703,5.15391) -- (3.79134,5.15391);
\draw [c] (3.79134,5.15391) -- (3.83564,5.15391);
\definecolor{c}{rgb}{0,0,0};
\colorlet{c}{natgreen};
\draw [c] (3.87995,5.04025) -- (3.87995,5.06387);
\draw [c] (3.87995,5.06387) -- (3.87995,5.08644);
\draw [c] (3.83564,5.06387) -- (3.87995,5.06387);
\draw [c] (3.87995,5.06387) -- (3.92426,5.06387);
\definecolor{c}{rgb}{0,0,0};
\colorlet{c}{natgreen};
\draw [c] (3.96856,5.03979) -- (3.96856,5.06345);
\draw [c] (3.96856,5.06345) -- (3.96856,5.08605);
\draw [c] (3.92426,5.06345) -- (3.96856,5.06345);
\draw [c] (3.96856,5.06345) -- (4.01287,5.06345);
\definecolor{c}{rgb}{0,0,0};
\colorlet{c}{natgreen};
\draw [c] (4.05718,4.9159) -- (4.05718,4.94262);
\draw [c] (4.05718,4.94262) -- (4.05718,4.96799);
\draw [c] (4.01287,4.94262) -- (4.05718,4.94262);
\draw [c] (4.05718,4.94262) -- (4.10149,4.94262);
\definecolor{c}{rgb}{0,0,0};
\colorlet{c}{natgreen};
\draw [c] (4.14579,4.77187) -- (4.14579,4.80257);
\draw [c] (4.14579,4.80257) -- (4.14579,4.83152);
\draw [c] (4.10149,4.80257) -- (4.14579,4.80257);
\draw [c] (4.14579,4.80257) -- (4.1901,4.80257);
\definecolor{c}{rgb}{0,0,0};
\colorlet{c}{natgreen};
\draw [c] (4.23441,4.76339) -- (4.23441,4.79444);
\draw [c] (4.23441,4.79444) -- (4.23441,4.8237);
\draw [c] (4.1901,4.79444) -- (4.23441,4.79444);
\draw [c] (4.23441,4.79444) -- (4.27871,4.79444);
\definecolor{c}{rgb}{0,0,0};
\colorlet{c}{natgreen};
\draw [c] (4.32302,4.75203) -- (4.32302,4.78345);
\draw [c] (4.32302,4.78345) -- (4.32302,4.81304);
\draw [c] (4.27871,4.78345) -- (4.32302,4.78345);
\draw [c] (4.32302,4.78345) -- (4.36733,4.78345);
\definecolor{c}{rgb}{0,0,0};
\colorlet{c}{natgreen};
\draw [c] (4.41163,4.63179) -- (4.41163,4.66718);
\draw [c] (4.41163,4.66718) -- (4.41163,4.70026);
\draw [c] (4.36733,4.66718) -- (4.41163,4.66718);
\draw [c] (4.41163,4.66718) -- (4.45594,4.66718);
\definecolor{c}{rgb}{0,0,0};
\colorlet{c}{natgreen};
\draw [c] (4.50025,4.62157) -- (4.50025,4.65723);
\draw [c] (4.50025,4.65723) -- (4.50025,4.69055);
\draw [c] (4.45594,4.65723) -- (4.50025,4.65723);
\draw [c] (4.50025,4.65723) -- (4.54455,4.65723);
\definecolor{c}{rgb}{0,0,0};
\colorlet{c}{natgreen};
\draw [c] (4.58886,4.64707) -- (4.58886,4.6819);
\draw [c] (4.58886,4.6819) -- (4.58886,4.71448);
\draw [c] (4.54455,4.6819) -- (4.58886,4.6819);
\draw [c] (4.58886,4.6819) -- (4.63317,4.6819);
\definecolor{c}{rgb}{0,0,0};
\colorlet{c}{natgreen};
\draw [c] (4.67748,4.39624) -- (4.67748,4.44093);
\draw [c] (4.67748,4.44093) -- (4.67748,4.48199);
\draw [c] (4.63317,4.44093) -- (4.67748,4.44093);
\draw [c] (4.67748,4.44093) -- (4.72178,4.44093);
\definecolor{c}{rgb}{0,0,0};
\colorlet{c}{natgreen};
\draw [c] (4.76609,4.37532) -- (4.76609,4.42087);
\draw [c] (4.76609,4.42087) -- (4.76609,4.46266);
\draw [c] (4.72178,4.42087) -- (4.76609,4.42087);
\draw [c] (4.76609,4.42087) -- (4.8104,4.42087);
\definecolor{c}{rgb}{0,0,0};
\colorlet{c}{natgreen};
\draw [c] (4.8547,4.42175) -- (4.8547,4.46537);
\draw [c] (4.8547,4.46537) -- (4.8547,4.50552);
\draw [c] (4.8104,4.46537) -- (4.8547,4.46537);
\draw [c] (4.8547,4.46537) -- (4.89901,4.46537);
\definecolor{c}{rgb}{0,0,0};
\colorlet{c}{natgreen};
\draw [c] (4.94332,4.31787) -- (4.94332,4.36614);
\draw [c] (4.94332,4.36614) -- (4.94332,4.4102);
\draw [c] (4.89901,4.36614) -- (4.94332,4.36614);
\draw [c] (4.94332,4.36614) -- (4.98762,4.36614);
\definecolor{c}{rgb}{0,0,0};
\colorlet{c}{natgreen};
\draw [c] (5.03193,4.39308) -- (5.03193,4.43794);
\draw [c] (5.03193,4.43794) -- (5.03193,4.47915);
\draw [c] (4.98762,4.43794) -- (5.03193,4.43794);
\draw [c] (5.03193,4.43794) -- (5.07624,4.43794);
\definecolor{c}{rgb}{0,0,0};
\colorlet{c}{natgreen};
\draw [c] (5.12054,4.17587) -- (5.12054,4.23112);
\draw [c] (5.12054,4.23112) -- (5.12054,4.28093);
\draw [c] (5.07624,4.23112) -- (5.12054,4.23112);
\draw [c] (5.12054,4.23112) -- (5.16485,4.23112);
\definecolor{c}{rgb}{0,0,0};
\colorlet{c}{natgreen};
\draw [c] (5.20916,4.1169) -- (5.20916,4.17568);
\draw [c] (5.20916,4.17568) -- (5.20916,4.22834);
\draw [c] (5.16485,4.17568) -- (5.20916,4.17568);
\draw [c] (5.20916,4.17568) -- (5.25347,4.17568);
\definecolor{c}{rgb}{0,0,0};
\colorlet{c}{natgreen};
\draw [c] (5.29777,4.12189) -- (5.29777,4.18035);
\draw [c] (5.29777,4.18035) -- (5.29777,4.23276);
\draw [c] (5.25347,4.18035) -- (5.29777,4.18035);
\draw [c] (5.29777,4.18035) -- (5.34208,4.18035);
\definecolor{c}{rgb}{0,0,0};
\colorlet{c}{natgreen};
\draw [c] (5.38639,4.21247) -- (5.38639,4.26613);
\draw [c] (5.38639,4.26613) -- (5.38639,4.31463);
\draw [c] (5.34208,4.26613) -- (5.38639,4.26613);
\draw [c] (5.38639,4.26613) -- (5.43069,4.26613);
\definecolor{c}{rgb}{0,0,0};
\colorlet{c}{natgreen};
\draw [c] (5.475,4.14696) -- (5.475,4.20408);
\draw [c] (5.475,4.20408) -- (5.475,4.2554);
\draw [c] (5.43069,4.20408) -- (5.475,4.20408);
\draw [c] (5.475,4.20408) -- (5.51931,4.20408);
\definecolor{c}{rgb}{0,0,0};
\colorlet{c}{natgreen};
\draw [c] (5.56361,3.9376) -- (5.56361,4.00776);
\draw [c] (5.56361,4.00776) -- (5.56361,4.06937);
\draw [c] (5.51931,4.00776) -- (5.56361,4.00776);
\draw [c] (5.56361,4.00776) -- (5.60792,4.00776);
\definecolor{c}{rgb}{0,0,0};
\colorlet{c}{natgreen};
\draw [c] (5.65223,3.87671) -- (5.65223,3.95106);
\draw [c] (5.65223,3.95106) -- (5.65223,4.01587);
\draw [c] (5.60792,3.95106) -- (5.65223,3.95106);
\draw [c] (5.65223,3.95106) -- (5.69653,3.95106);
\definecolor{c}{rgb}{0,0,0};
\colorlet{c}{natgreen};
\draw [c] (5.74084,4.07352) -- (5.74084,4.13487);
\draw [c] (5.74084,4.13487) -- (5.74084,4.18959);
\draw [c] (5.69653,4.13487) -- (5.74084,4.13487);
\draw [c] (5.74084,4.13487) -- (5.78515,4.13487);
\definecolor{c}{rgb}{0,0,0};
\colorlet{c}{natgreen};
\draw [c] (5.82946,3.67762) -- (5.82946,3.7682);
\draw [c] (5.82946,3.7682) -- (5.82946,3.84501);
\draw [c] (5.78515,3.7682) -- (5.82946,3.7682);
\draw [c] (5.82946,3.7682) -- (5.87376,3.7682);
\definecolor{c}{rgb}{0,0,0};
\colorlet{c}{natgreen};
\draw [c] (5.91807,3.94968) -- (5.91807,4.01916);
\draw [c] (5.91807,4.01916) -- (5.91807,4.08024);
\draw [c] (5.87376,4.01916) -- (5.91807,4.01916);
\draw [c] (5.91807,4.01916) -- (5.96238,4.01916);
\definecolor{c}{rgb}{0,0,0};
\colorlet{c}{natgreen};
\draw [c] (6.00668,3.75403) -- (6.00668,3.83816);
\draw [c] (6.00668,3.83816) -- (6.00668,3.91028);
\draw [c] (5.96238,3.83816) -- (6.00668,3.83816);
\draw [c] (6.00668,3.83816) -- (6.05099,3.83816);
\definecolor{c}{rgb}{0,0,0};
\colorlet{c}{natgreen};
\draw [c] (6.0953,3.74843) -- (6.0953,3.83301);
\draw [c] (6.0953,3.83301) -- (6.0953,3.90546);
\draw [c] (6.05099,3.83301) -- (6.0953,3.83301);
\draw [c] (6.0953,3.83301) -- (6.1396,3.83301);
\definecolor{c}{rgb}{0,0,0};
\colorlet{c}{natgreen};
\draw [c] (6.18391,3.4963) -- (6.18391,3.60488);
\draw [c] (6.18391,3.60488) -- (6.18391,3.69423);
\draw [c] (6.1396,3.60488) -- (6.18391,3.60488);
\draw [c] (6.18391,3.60488) -- (6.22822,3.60488);
\definecolor{c}{rgb}{0,0,0};
\colorlet{c}{natgreen};
\draw [c] (6.27252,3.50919) -- (6.27252,3.6164);
\draw [c] (6.27252,3.6164) -- (6.27252,3.70483);
\draw [c] (6.22822,3.6164) -- (6.27252,3.6164);
\draw [c] (6.27252,3.6164) -- (6.31683,3.6164);
\definecolor{c}{rgb}{0,0,0};
\colorlet{c}{natgreen};
\draw [c] (6.36114,3.2093) -- (6.36114,3.3527);
\draw [c] (6.36114,3.3527) -- (6.36114,3.46433);
\draw [c] (6.31683,3.3527) -- (6.36114,3.3527);
\draw [c] (6.36114,3.3527) -- (6.40545,3.3527);
\definecolor{c}{rgb}{0,0,0};
\colorlet{c}{natgreen};
\draw [c] (6.44975,3.97014) -- (6.44975,4.03823);
\draw [c] (6.44975,4.03823) -- (6.44975,4.09824);
\draw [c] (6.40545,4.03823) -- (6.44975,4.03823);
\draw [c] (6.44975,4.03823) -- (6.49406,4.03823);
\definecolor{c}{rgb}{0,0,0};
\colorlet{c}{natgreen};
\draw [c] (6.53837,2.88887) -- (6.53837,3.08578);
\draw [c] (6.53837,3.08578) -- (6.53837,3.2272);
\draw [c] (6.49406,3.08578) -- (6.53837,3.08578);
\draw [c] (6.53837,3.08578) -- (6.58267,3.08578);
\definecolor{c}{rgb}{0,0,0};
\colorlet{c}{natgreen};
\draw [c] (6.62698,3.59236) -- (6.62698,3.69101);
\draw [c] (6.62698,3.69101) -- (6.62698,3.77354);
\draw [c] (6.58267,3.69101) -- (6.62698,3.69101);
\draw [c] (6.62698,3.69101) -- (6.67129,3.69101);
\definecolor{c}{rgb}{0,0,0};
\colorlet{c}{natgreen};
\draw [c] (6.71559,3.44265) -- (6.71559,3.55662);
\draw [c] (6.71559,3.55662) -- (6.71559,3.6496);
\draw [c] (6.67129,3.55662) -- (6.71559,3.55662);
\draw [c] (6.71559,3.55662) -- (6.7599,3.55662);
\definecolor{c}{rgb}{0,0,0};
\colorlet{c}{natgreen};
\draw [c] (6.80421,3.38595) -- (6.80421,3.50683);
\draw [c] (6.80421,3.50683) -- (6.80421,3.60434);
\draw [c] (6.7599,3.50683) -- (6.80421,3.50683);
\draw [c] (6.80421,3.50683) -- (6.84852,3.50683);
\definecolor{c}{rgb}{0,0,0};
\colorlet{c}{natgreen};
\draw [c] (6.89282,3.38837) -- (6.89282,3.5091);
\draw [c] (6.89282,3.5091) -- (6.89282,3.60651);
\draw [c] (6.84852,3.5091) -- (6.89282,3.5091);
\draw [c] (6.89282,3.5091) -- (6.93713,3.5091);
\definecolor{c}{rgb}{0,0,0};
\colorlet{c}{natgreen};
\draw [c] (6.98144,3.3475) -- (6.98144,3.47318);
\draw [c] (6.98144,3.47318) -- (6.98144,3.57378);
\draw [c] (6.93713,3.47318) -- (6.98144,3.47318);
\draw [c] (6.98144,3.47318) -- (7.02574,3.47318);
\definecolor{c}{rgb}{0,0,0};
\colorlet{c}{natgreen};
\draw [c] (7.07005,2.39988) -- (7.07005,2.70986);
\draw [c] (7.07005,2.70986) -- (7.07005,2.90074);
\draw [c] (7.02574,2.70986) -- (7.07005,2.70986);
\draw [c] (7.07005,2.70986) -- (7.11436,2.70986);
\definecolor{c}{rgb}{0,0,0};
\colorlet{c}{natgreen};
\draw [c] (7.15866,3.27796) -- (7.15866,3.41252);
\draw [c] (7.15866,3.41252) -- (7.15866,3.51872);
\draw [c] (7.11436,3.41252) -- (7.15866,3.41252);
\draw [c] (7.15866,3.41252) -- (7.20297,3.41252);
\definecolor{c}{rgb}{0,0,0};
\colorlet{c}{natgreen};
\draw [c] (7.24728,3.27734) -- (7.24728,3.41197);
\draw [c] (7.24728,3.41197) -- (7.24728,3.51823);
\draw [c] (7.20297,3.41197) -- (7.24728,3.41197);
\draw [c] (7.24728,3.41197) -- (7.29158,3.41197);
\definecolor{c}{rgb}{0,0,0};
\colorlet{c}{natgreen};
\draw [c] (7.33589,3.29526) -- (7.33589,3.42755);
\draw [c] (7.33589,3.42755) -- (7.33589,3.53234);
\draw [c] (7.29158,3.42755) -- (7.33589,3.42755);
\draw [c] (7.33589,3.42755) -- (7.3802,3.42755);
\definecolor{c}{rgb}{0,0,0};
\colorlet{c}{natgreen};
\draw [c] (7.4245,2.95689) -- (7.4245,3.13997);
\draw [c] (7.4245,3.13997) -- (7.4245,3.27415);
\draw [c] (7.3802,3.13997) -- (7.4245,3.13997);
\draw [c] (7.4245,3.13997) -- (7.46881,3.13997);
\definecolor{c}{rgb}{0,0,0};
\colorlet{c}{natgreen};
\draw [c] (7.51312,2.63192) -- (7.51312,2.88466);
\draw [c] (7.51312,2.88466) -- (7.51312,3.05249);
\draw [c] (7.46881,2.88466) -- (7.51312,2.88466);
\draw [c] (7.51312,2.88466) -- (7.55743,2.88466);
\definecolor{c}{rgb}{0,0,0};
\colorlet{c}{natgreen};
\draw [c] (7.60173,3.10883) -- (7.60173,3.26766);
\draw [c] (7.60173,3.26766) -- (7.60173,3.38838);
\draw [c] (7.55743,3.26766) -- (7.60173,3.26766);
\draw [c] (7.60173,3.26766) -- (7.64604,3.26766);
\definecolor{c}{rgb}{0,0,0};
\colorlet{c}{natgreen};
\draw [c] (7.69035,1.68208) -- (7.69035,2.29607);
\draw [c] (7.69035,2.29607) -- (7.69035,2.56555);
\draw [c] (7.64604,2.29607) -- (7.69035,2.29607);
\draw [c] (7.69035,2.29607) -- (7.73465,2.29607);
\definecolor{c}{rgb}{0,0,0};
\colorlet{c}{natgreen};
\draw [c] (7.77896,3.37021) -- (7.77896,3.49311);
\draw [c] (7.77896,3.49311) -- (7.77896,3.59193);
\draw [c] (7.73465,3.49311) -- (7.77896,3.49311);
\draw [c] (7.77896,3.49311) -- (7.82327,3.49311);
\definecolor{c}{rgb}{0,0,0};
\colorlet{c}{natgreen};
\draw [c] (7.86757,2.70656) -- (7.86757,2.94167);
\draw [c] (7.86757,2.94167) -- (7.86757,3.10162);
\draw [c] (7.82327,2.94167) -- (7.86757,2.94167);
\draw [c] (7.86757,2.94167) -- (7.91188,2.94167);
\definecolor{c}{rgb}{0,0,0};
\colorlet{c}{natgreen};
\draw [c] (7.95619,1.24756) -- (7.95619,2.13237);
\draw [c] (7.95619,2.13237) -- (7.95619,2.43724);
\draw [c] (7.91188,2.13237) -- (7.95619,2.13237);
\draw [c] (7.95619,2.13237) -- (8.00049,2.13237);
\definecolor{c}{rgb}{0,0,0};
\colorlet{c}{natgreen};
\draw [c] (8.0448,2.33128) -- (8.0448,2.66863);
\draw [c] (8.0448,2.66863) -- (8.0448,2.86933);
\draw [c] (8.00049,2.66863) -- (8.0448,2.66863);
\draw [c] (8.0448,2.66863) -- (8.08911,2.66863);
\definecolor{c}{rgb}{0,0,0};
\colorlet{c}{natgreen};
\draw [c] (8.13342,1.23328) -- (8.13342,2.12817);
\draw [c] (8.13342,2.12817) -- (8.13342,2.43399);
\draw [c] (8.08911,2.12817) -- (8.13342,2.12817);
\draw [c] (8.13342,2.12817) -- (8.17772,2.12817);
\definecolor{c}{rgb}{0,0,0};
\colorlet{c}{natgreen};
\draw [c] (8.22203,2.37726) -- (8.22203,2.70014);
\draw [c] (8.22203,2.70014) -- (8.22203,2.89574);
\draw [c] (8.17772,2.70014) -- (8.22203,2.70014);
\draw [c] (8.22203,2.70014) -- (8.26634,2.70014);
\definecolor{c}{rgb}{0,0,0};
\colorlet{c}{natgreen};
\draw [c] (8.31064,2.28945) -- (8.31064,2.64049);
\draw [c] (8.31064,2.64049) -- (8.31064,2.84583);
\draw [c] (8.26634,2.64049) -- (8.31064,2.64049);
\draw [c] (8.31064,2.64049) -- (8.35495,2.64049);
\definecolor{c}{rgb}{0,0,0};
\colorlet{c}{natgreen};
\draw [c] (8.39926,1.52931) -- (8.39926,2.2217);
\draw [c] (8.39926,2.2217) -- (8.39926,2.50372);
\draw [c] (8.35495,2.2217) -- (8.39926,2.2217);
\draw [c] (8.39926,2.2217) -- (8.44356,2.2217);
\definecolor{c}{rgb}{0,0,0};
\colorlet{c}{natgreen};
\draw [c] (8.48787,2.69897) -- (8.48787,2.93582);
\draw [c] (8.48787,2.93582) -- (8.48787,3.09656);
\draw [c] (8.44356,2.93582) -- (8.48787,2.93582);
\draw [c] (8.48787,2.93582) -- (8.53218,2.93582);
\definecolor{c}{rgb}{0,0,0};
\colorlet{c}{natgreen};
\draw [c] (8.6651,1.23778) -- (8.6651,2.12949);
\draw [c] (8.6651,2.12949) -- (8.6651,2.43501);
\draw [c] (8.62079,2.12949) -- (8.6651,2.12949);
\draw [c] (8.6651,2.12949) -- (8.70941,2.12949);
\definecolor{c}{rgb}{0,0,0};
\colorlet{c}{natgreen};
\draw [c] (8.75371,0.680516) -- (8.75371,1.90572);
\draw [c] (8.75371,1.90572) -- (8.75371,2.265);
\draw [c] (8.70941,1.90572) -- (8.75371,1.90572);
\draw [c] (8.75371,1.90572) -- (8.79802,1.90572);
\definecolor{c}{rgb}{0,0,0};
\colorlet{c}{natgreen};
\draw [c] (8.84233,2.55842) -- (8.84233,2.82975);
\draw [c] (8.84233,2.82975) -- (8.84233,3.00548);
\draw [c] (8.79802,2.82975) -- (8.84233,2.82975);
\draw [c] (8.84233,2.82975) -- (8.88663,2.82975);
\definecolor{c}{rgb}{0,0,0};
\colorlet{c}{natgreen};
\draw [c] (8.93094,1.03156) -- (8.93094,2.07613);
\draw [c] (8.93094,2.07613) -- (8.93094,2.39391);
\draw [c] (8.88663,2.07613) -- (8.93094,2.07613);
\draw [c] (8.93094,2.07613) -- (8.97525,2.07613);
\definecolor{c}{rgb}{0,0,0};
\colorlet{c}{natgreen};
\draw [c] (9.01955,2.12561) -- (9.01955,2.53533);
\draw [c] (9.01955,2.53533) -- (9.01955,2.75882);
\draw [c] (8.97525,2.53533) -- (9.01955,2.53533);
\draw [c] (9.01955,2.53533) -- (9.06386,2.53533);
\definecolor{c}{rgb}{0,0,0};
\colorlet{c}{natgreen};
\draw [c] (9.10817,0.680516) -- (9.10817,1.8254);
\draw [c] (9.10817,1.8254) -- (9.10817,2.20548);
\draw [c] (9.06386,1.8254) -- (9.10817,1.8254);
\draw [c] (9.10817,1.8254) -- (9.15248,1.8254);
\definecolor{c}{rgb}{0,0,0};
\colorlet{c}{natgreen};
\draw [c] (9.19678,1.66756) -- (9.19678,2.28945);
\draw [c] (9.19678,2.28945) -- (9.19678,2.56029);
\draw [c] (9.15248,2.28945) -- (9.19678,2.28945);
\draw [c] (9.19678,2.28945) -- (9.24109,2.28945);
\definecolor{c}{rgb}{0,0,0};
\colorlet{c}{natgreen};
\draw [c] (9.46262,0.680516) -- (9.46262,1.93841);
\draw [c] (9.46262,1.93841) -- (9.46262,2.28945);
\draw [c] (9.41832,1.93841) -- (9.46262,1.93841);
\draw [c] (9.46262,1.93841) -- (9.50693,1.93841);
\definecolor{c}{rgb}{0,0,0};
\colorlet{c}{natgreen};
\draw [c] (9.90569,0.680516) -- (9.90569,1.38202);
\draw [c] (9.90569,1.38202) -- (9.90569,1.89102);
\draw [c] (9.86139,1.38202) -- (9.90569,1.38202);
\draw [c] (9.90569,1.38202) -- (9.95,1.38202);
\definecolor{c}{rgb}{0,0,0};
\draw [c] (1,0.680516) -- (9.95,0.680516);
\draw [anchor= east] (9.95,0.108883) node[color=c, rotate=0]{$M_{\gamma\gamma}\text{ [GeV]}$};
\draw [c] (1,0.863234) -- (1,0.680516);
\draw [c] (1.44307,0.771875) -- (1.44307,0.680516);
\draw [c] (1.88614,0.771875) -- (1.88614,0.680516);
\draw [c] (2.32921,0.771875) -- (2.32921,0.680516);
\draw [c] (2.77228,0.863234) -- (2.77228,0.680516);
\draw [c] (3.21535,0.771875) -- (3.21535,0.680516);
\draw [c] (3.65842,0.771875) -- (3.65842,0.680516);
\draw [c] (4.10149,0.771875) -- (4.10149,0.680516);
\draw [c] (4.54455,0.863234) -- (4.54455,0.680516);
\draw [c] (4.98762,0.771875) -- (4.98762,0.680516);
\draw [c] (5.43069,0.771875) -- (5.43069,0.680516);
\draw [c] (5.87376,0.771875) -- (5.87376,0.680516);
\draw [c] (6.31683,0.863234) -- (6.31683,0.680516);
\draw [c] (6.7599,0.771875) -- (6.7599,0.680516);
\draw [c] (7.20297,0.771875) -- (7.20297,0.680516);
\draw [c] (7.64604,0.771875) -- (7.64604,0.680516);
\draw [c] (8.08911,0.863234) -- (8.08911,0.680516);
\draw [c] (8.53218,0.771875) -- (8.53218,0.680516);
\draw [c] (8.97525,0.771875) -- (8.97525,0.680516);
\draw [c] (9.41832,0.771875) -- (9.41832,0.680516);
\draw [c] (9.86139,0.863234) -- (9.86139,0.680516);
\draw [c] (9.86139,0.863234) -- (9.86139,0.680516);
\draw [anchor=base] (1,0.353868) node[color=c, rotate=0]{0};
\draw [anchor=base] (2.77228,0.353868) node[color=c, rotate=0]{200};
\draw [anchor=base] (4.54455,0.353868) node[color=c, rotate=0]{400};
\draw [anchor=base] (6.31683,0.353868) node[color=c, rotate=0]{600};
\draw [anchor=base] (8.08911,0.353868) node[color=c, rotate=0]{800};
\draw [anchor=base] (9.86139,0.353868) node[color=c, rotate=0]{1000};
\draw [c] (1,0.680516) -- (1,6.73711);
\draw [anchor= east] (-0.12,6.73711) node[color=c, rotate=90]{Number of events};
\draw [c] (1.1335,0.718918) -- (1,0.718918);
\draw [c] (1.267,0.772277) -- (1,0.772277);
\draw [anchor= east] (0.922,0.772277) node[color=c, rotate=0]{$10^{-1}$};
\draw [c] (1.1335,1.12332) -- (1,1.12332);
\draw [c] (1.1335,1.32866) -- (1,1.32866);
\draw [c] (1.1335,1.47436) -- (1,1.47436);
\draw [c] (1.1335,1.58737) -- (1,1.58737);
\draw [c] (1.1335,1.6797) -- (1,1.6797);
\draw [c] (1.1335,1.75777) -- (1,1.75777);
\draw [c] (1.1335,1.8254) -- (1,1.8254);
\draw [c] (1.1335,1.88505) -- (1,1.88505);
\draw [c] (1.267,1.93841) -- (1,1.93841);
\draw [anchor= east] (0.922,1.93841) node[color=c, rotate=0]{1};
\draw [c] (1.1335,2.28945) -- (1,2.28945);
\draw [c] (1.1335,2.49479) -- (1,2.49479);
\draw [c] (1.1335,2.64049) -- (1,2.64049);
\draw [c] (1.1335,2.7535) -- (1,2.7535);
\draw [c] (1.1335,2.84583) -- (1,2.84583);
\draw [c] (1.1335,2.9239) -- (1,2.9239);
\draw [c] (1.1335,2.99153) -- (1,2.99153);
\draw [c] (1.1335,3.05118) -- (1,3.05118);
\draw [c] (1.267,3.10454) -- (1,3.10454);
\draw [anchor= east] (0.922,3.10454) node[color=c, rotate=0]{10};
\draw [c] (1.1335,3.45558) -- (1,3.45558);
\draw [c] (1.1335,3.66092) -- (1,3.66092);
\draw [c] (1.1335,3.80662) -- (1,3.80662);
\draw [c] (1.1335,3.91963) -- (1,3.91963);
\draw [c] (1.1335,4.01196) -- (1,4.01196);
\draw [c] (1.1335,4.09003) -- (1,4.09003);
\draw [c] (1.1335,4.15766) -- (1,4.15766);
\draw [c] (1.1335,4.21731) -- (1,4.21731);
\draw [c] (1.267,4.27067) -- (1,4.27067);
\draw [anchor= east] (0.922,4.27067) node[color=c, rotate=0]{$10^{2}$};
\draw [c] (1.1335,4.62171) -- (1,4.62171);
\draw [c] (1.1335,4.82705) -- (1,4.82705);
\draw [c] (1.1335,4.97275) -- (1,4.97275);
\draw [c] (1.1335,5.08576) -- (1,5.08576);
\draw [c] (1.1335,5.1781) -- (1,5.1781);
\draw [c] (1.1335,5.25616) -- (1,5.25616);
\draw [c] (1.1335,5.32379) -- (1,5.32379);
\draw [c] (1.1335,5.38344) -- (1,5.38344);
\draw [c] (1.267,5.4368) -- (1,5.4368);
\draw [anchor= east] (0.922,5.4368) node[color=c, rotate=0]{$10^{3}$};
\draw [c] (1.1335,5.78784) -- (1,5.78784);
\draw [c] (1.1335,5.99319) -- (1,5.99319);
\draw [c] (1.1335,6.13888) -- (1,6.13888);
\draw [c] (1.1335,6.25189) -- (1,6.25189);
\draw [c] (1.1335,6.34423) -- (1,6.34423);
\draw [c] (1.1335,6.42229) -- (1,6.42229);
\draw [c] (1.1335,6.48992) -- (1,6.48992);
\draw [c] (1.1335,6.54957) -- (1,6.54957);
\draw [c] (1.267,6.60293) -- (1,6.60293);
\draw [anchor= east] (0.922,6.60293) node[color=c, rotate=0]{$10^{4}$};
\colorlet{c}{natcomp!70};
\draw [c] (2.36687,6.34851) -- (2.44219,6.27507) -- (2.51751,6.20302) -- (2.59283,6.1323) -- (2.66816,6.06287) -- (2.74348,5.99466) -- (2.8188,5.92763) -- (2.89412,5.86173) -- (2.96944,5.7969) -- (3.04476,5.73311) -- (3.12009,5.6703)
 -- (3.19541,5.60845) -- (3.27073,5.5475) -- (3.34605,5.48743) -- (3.42137,5.42819) -- (3.4967,5.36977) -- (3.57202,5.31211) -- (3.64734,5.25521) -- (3.72266,5.19902) -- (3.79798,5.14352) -- (3.8733,5.08869) -- (3.94863,5.03451) -- (4.02395,4.98094)
 -- (4.09927,4.92798) -- (4.17459,4.87559) -- (4.24991,4.82377) -- (4.32524,4.77249) -- (4.40056,4.72173) -- (4.47588,4.67148) -- (4.5512,4.62172) -- (4.62652,4.57244) -- (4.70184,4.52363) -- (4.77717,4.47526) -- (4.85249,4.42733) --
 (4.92781,4.37982) -- (5.00313,4.33272) -- (5.07845,4.28602) -- (5.15377,4.23971) -- (5.2291,4.19378) -- (5.30442,4.14821) -- (5.37974,4.103) -- (5.45506,4.05814) -- (5.53038,4.01362) -- (5.60571,3.96943) -- (5.68103,3.92556) -- (5.75635,3.88201) --
 (5.83167,3.83876) -- (5.90699,3.79581) -- (5.98231,3.75315) -- (6.05764,3.71077);
\draw [c] (6.05764,3.71077) -- (6.13296,3.66868) -- (6.20828,3.62685) -- (6.2836,3.58529) -- (6.35892,3.54399) -- (6.43425,3.50294) -- (6.50957,3.46213) -- (6.58489,3.42157) -- (6.66021,3.38125) -- (6.73553,3.34115) --
 (6.81085,3.30129) -- (6.88618,3.26164) -- (6.9615,3.22221) -- (7.03682,3.18299) -- (7.11214,3.14398) -- (7.18746,3.10517) -- (7.26278,3.06656) -- (7.33811,3.02815) -- (7.41343,2.98993) -- (7.48875,2.95189) -- (7.56407,2.91404) -- (7.63939,2.87637)
 -- (7.71472,2.83887) -- (7.79004,2.80155) -- (7.86536,2.7644) -- (7.94068,2.72741) -- (8.016,2.69059) -- (8.09132,2.65392) -- (8.16665,2.61742) -- (8.24197,2.58107) -- (8.31729,2.54487) -- (8.39261,2.50882) -- (8.46793,2.47291) -- (8.54325,2.43715)
 -- (8.61858,2.40153) -- (8.6939,2.36605) -- (8.76922,2.3307) -- (8.84454,2.29549) -- (8.91986,2.26041) -- (8.99519,2.22545) -- (9.07051,2.19063) -- (9.14583,2.15593) -- (9.22115,2.12135) -- (9.29647,2.08689) -- (9.37179,2.05255) -- (9.44712,2.01832)
 -- (9.52244,1.98421) -- (9.59776,1.95021) -- (9.67308,1.91633) -- (9.7484,1.88255);
\draw [c] (9.7484,1.88255) -- (9.82373,1.84888);
\draw [c] (2.36687,6.34851) -- (2.44219,6.27507) -- (2.51751,6.20302) -- (2.59283,6.1323) -- (2.66816,6.06287) -- (2.74348,5.99466) -- (2.8188,5.92763) -- (2.89412,5.86173) -- (2.96944,5.7969) -- (3.04476,5.73311) -- (3.12009,5.6703)
 -- (3.19541,5.60845) -- (3.27073,5.5475) -- (3.34605,5.48743) -- (3.42137,5.42819) -- (3.4967,5.36977) -- (3.57202,5.31211) -- (3.64734,5.25521) -- (3.72266,5.19902) -- (3.79798,5.14352) -- (3.8733,5.08869) -- (3.94863,5.03451) -- (4.02395,4.98094)
 -- (4.09927,4.92798) -- (4.17459,4.87559) -- (4.24991,4.82377) -- (4.32524,4.77249) -- (4.40056,4.72173) -- (4.47588,4.67148) -- (4.5512,4.62172) -- (4.62652,4.57244) -- (4.70184,4.52363) -- (4.77717,4.47526) -- (4.85249,4.42733) --
 (4.92781,4.37982) -- (5.00313,4.33272) -- (5.07845,4.28602) -- (5.15377,4.23971) -- (5.2291,4.19378) -- (5.30442,4.14821) -- (5.37974,4.103) -- (5.45506,4.05814) -- (5.53038,4.01362) -- (5.60571,3.96943) -- (5.68103,3.92556) -- (5.75635,3.88201) --
 (5.83167,3.83876) -- (5.90699,3.79581) -- (5.98231,3.75315) -- (6.05764,3.71077);
\draw [c] (6.05764,3.71077) -- (6.13296,3.66868) -- (6.20828,3.62685) -- (6.2836,3.58529) -- (6.35892,3.54399) -- (6.43425,3.50294) -- (6.50957,3.46213) -- (6.58489,3.42157) -- (6.66021,3.38125) -- (6.73553,3.34115) --
 (6.81085,3.30129) -- (6.88618,3.26164) -- (6.9615,3.22221) -- (7.03682,3.18299) -- (7.11214,3.14398) -- (7.18746,3.10517) -- (7.26278,3.06656) -- (7.33811,3.02815) -- (7.41343,2.98993) -- (7.48875,2.95189) -- (7.56407,2.91404) -- (7.63939,2.87637)
 -- (7.71472,2.83887) -- (7.79004,2.80155) -- (7.86536,2.7644) -- (7.94068,2.72741) -- (8.016,2.69059) -- (8.09132,2.65392) -- (8.16665,2.61742) -- (8.24197,2.58107) -- (8.31729,2.54487) -- (8.39261,2.50882) -- (8.46793,2.47291) -- (8.54325,2.43715)
 -- (8.61858,2.40153) -- (8.6939,2.36605) -- (8.76922,2.3307) -- (8.84454,2.29549) -- (8.91986,2.26041) -- (8.99519,2.22545) -- (9.07051,2.19063) -- (9.14583,2.15593) -- (9.22115,2.12135) -- (9.29647,2.08689) -- (9.37179,2.05255) -- (9.44712,2.01832)
 -- (9.52244,1.98421) -- (9.59776,1.95021) -- (9.67308,1.91633) -- (9.7484,1.88255);
\draw [c] (9.7484,1.88255) -- (9.82373,1.84888);
\end{tikzpicture}

\end{infilsf}
\end{minipage}
\begin{minipage}[b]{.3\textwidth}
\caption{Extrapolating the truncated backgrounds by fitting a function.}\label{bckfit}
\end{minipage}
\end{figure}

\section{ Fit}
Recalling eq./eqref{rizzo}, we note that the contribution from the new term is proportional to a constant ($\Lambda^{-2}$) squared. Thus, the content of any bin must be expressible as a second order polynomial [...this should be in the theory chapter, or the sim chapter...]

[Something about the least likelihood method]

Because the fit method is very sensitive to the statistical fluctuations from bin to bin in our predicted invariant mass distributions, we opt to remove those fluctuations by fitting the distributions with a function. Since we take it as a given that the shape of the distribution consists of a base shape plus an excess from the contact interaction, we will fit them with a common power law, plus a gaussian plus an exponential for each of the non--SM distributions [I have a formula around here somewhere...] the result of the fit is illustrated in fig.~\ref{simfit}

\begin{figure}[hbt]
\begin{infilsf}\tiny
\begin{tikzpicture}[x=.092\textwidth,y=.092\textwidth]
\pgfdeclareplotmark{cross} {
\pgfpathmoveto{\pgfpoint{-0.3\pgfplotmarksize}{\pgfplotmarksize}}
\pgfpathlineto{\pgfpoint{+0.3\pgfplotmarksize}{\pgfplotmarksize}}
\pgfpathlineto{\pgfpoint{+0.3\pgfplotmarksize}{0.3\pgfplotmarksize}}
\pgfpathlineto{\pgfpoint{+1\pgfplotmarksize}{0.3\pgfplotmarksize}}
\pgfpathlineto{\pgfpoint{+1\pgfplotmarksize}{-0.3\pgfplotmarksize}}
\pgfpathlineto{\pgfpoint{+0.3\pgfplotmarksize}{-0.3\pgfplotmarksize}}
\pgfpathlineto{\pgfpoint{+0.3\pgfplotmarksize}{-1.\pgfplotmarksize}}
\pgfpathlineto{\pgfpoint{-0.3\pgfplotmarksize}{-1.\pgfplotmarksize}}
\pgfpathlineto{\pgfpoint{-0.3\pgfplotmarksize}{-0.3\pgfplotmarksize}}
\pgfpathlineto{\pgfpoint{-1.\pgfplotmarksize}{-0.3\pgfplotmarksize}}
\pgfpathlineto{\pgfpoint{-1.\pgfplotmarksize}{0.3\pgfplotmarksize}}
\pgfpathlineto{\pgfpoint{-0.3\pgfplotmarksize}{0.3\pgfplotmarksize}}
\pgfpathclose
\pgfusepathqstroke
}
\pgfdeclareplotmark{cross*} {
\pgfpathmoveto{\pgfpoint{-0.3\pgfplotmarksize}{\pgfplotmarksize}}
\pgfpathlineto{\pgfpoint{+0.3\pgfplotmarksize}{\pgfplotmarksize}}
\pgfpathlineto{\pgfpoint{+0.3\pgfplotmarksize}{0.3\pgfplotmarksize}}
\pgfpathlineto{\pgfpoint{+1\pgfplotmarksize}{0.3\pgfplotmarksize}}
\pgfpathlineto{\pgfpoint{+1\pgfplotmarksize}{-0.3\pgfplotmarksize}}
\pgfpathlineto{\pgfpoint{+0.3\pgfplotmarksize}{-0.3\pgfplotmarksize}}
\pgfpathlineto{\pgfpoint{+0.3\pgfplotmarksize}{-1.\pgfplotmarksize}}
\pgfpathlineto{\pgfpoint{-0.3\pgfplotmarksize}{-1.\pgfplotmarksize}}
\pgfpathlineto{\pgfpoint{-0.3\pgfplotmarksize}{-0.3\pgfplotmarksize}}
\pgfpathlineto{\pgfpoint{-1.\pgfplotmarksize}{-0.3\pgfplotmarksize}}
\pgfpathlineto{\pgfpoint{-1.\pgfplotmarksize}{0.3\pgfplotmarksize}}
\pgfpathlineto{\pgfpoint{-0.3\pgfplotmarksize}{0.3\pgfplotmarksize}}
\pgfpathclose
\pgfusepathqfillstroke
}
\pgfdeclareplotmark{newstar} {
\pgfpathmoveto{\pgfqpoint{0pt}{\pgfplotmarksize}}
\pgfpathlineto{\pgfqpointpolar{44}{0.5\pgfplotmarksize}}
\pgfpathlineto{\pgfqpointpolar{18}{\pgfplotmarksize}}
\pgfpathlineto{\pgfqpointpolar{-20}{0.5\pgfplotmarksize}}
\pgfpathlineto{\pgfqpointpolar{-54}{\pgfplotmarksize}}
\pgfpathlineto{\pgfqpointpolar{-90}{0.5\pgfplotmarksize}}
\pgfpathlineto{\pgfqpointpolar{234}{\pgfplotmarksize}}
\pgfpathlineto{\pgfqpointpolar{198}{0.5\pgfplotmarksize}}
\pgfpathlineto{\pgfqpointpolar{162}{\pgfplotmarksize}}
\pgfpathlineto{\pgfqpointpolar{134}{0.5\pgfplotmarksize}}
\pgfpathclose
\pgfusepathqstroke
}
\pgfdeclareplotmark{newstar*} {
\pgfpathmoveto{\pgfqpoint{0pt}{\pgfplotmarksize}}
\pgfpathlineto{\pgfqpointpolar{44}{0.5\pgfplotmarksize}}
\pgfpathlineto{\pgfqpointpolar{18}{\pgfplotmarksize}}
\pgfpathlineto{\pgfqpointpolar{-20}{0.5\pgfplotmarksize}}
\pgfpathlineto{\pgfqpointpolar{-54}{\pgfplotmarksize}}
\pgfpathlineto{\pgfqpointpolar{-90}{0.5\pgfplotmarksize}}
\pgfpathlineto{\pgfqpointpolar{234}{\pgfplotmarksize}}
\pgfpathlineto{\pgfqpointpolar{198}{0.5\pgfplotmarksize}}
\pgfpathlineto{\pgfqpointpolar{162}{\pgfplotmarksize}}
\pgfpathlineto{\pgfqpointpolar{134}{0.5\pgfplotmarksize}}
\pgfpathclose
\pgfusepathqfillstroke
}
\definecolor{c}{rgb}{1,1,1};
\draw [color=c, fill=c] (0,0) rectangle (10,5.96817);
\draw [color=c, fill=c] (1,0.596817) rectangle (9.95,5.90849);
\definecolor{c}{rgb}{0,0,0};
\draw [c] (1,0.596817) -- (1,5.90849) -- (9.95,5.90849) -- (9.95,0.596817) -- (1,0.596817);
\definecolor{c}{rgb}{1,1,1};
\draw [color=c, fill=c] (1,0.596817) rectangle (9.95,5.90849);
\definecolor{c}{rgb}{0,0,0};
\draw [c] (1,0.596817) -- (1,5.90849) -- (9.95,5.90849) -- (9.95,0.596817) -- (1,0.596817);
\colorlet{c}{kugray};
\draw [c] (1.13336,3.41516) -- (1.13336,3.80777);
\draw [c] (1.13336,3.80777) -- (1.13336,3.97485);
\draw [c] (1.11854,3.80777) -- (1.13336,3.80777);
\draw [c] (1.13336,3.80777) -- (1.14818,3.80777);
\definecolor{c}{rgb}{0,0,0};
\colorlet{c}{kugray};
\draw [c] (1.163,3.33018) -- (1.163,3.70859);
\draw [c] (1.163,3.70859) -- (1.163,3.87329);
\draw [c] (1.14818,3.70859) -- (1.163,3.70859);
\draw [c] (1.163,3.70859) -- (1.17781,3.70859);
\definecolor{c}{rgb}{0,0,0};
\colorlet{c}{kugray};
\draw [c] (1.22227,3.26026) -- (1.22227,3.80576);
\draw [c] (1.22227,3.80576) -- (1.22227,3.99181);
\draw [c] (1.20745,3.80576) -- (1.22227,3.80576);
\draw [c] (1.22227,3.80576) -- (1.23709,3.80576);
\definecolor{c}{rgb}{0,0,0};
\colorlet{c}{kugray};
\draw [c] (1.2519,3.59515) -- (1.2519,3.86294);
\draw [c] (1.2519,3.86294) -- (1.2519,4.00395);
\draw [c] (1.23709,3.86294) -- (1.2519,3.86294);
\draw [c] (1.2519,3.86294) -- (1.26672,3.86294);
\definecolor{c}{rgb}{0,0,0};
\colorlet{c}{kugray};
\draw [c] (1.28154,3.81131) -- (1.28154,4.05468);
\draw [c] (1.28154,4.05468) -- (1.28154,4.18889);
\draw [c] (1.26672,4.05468) -- (1.28154,4.05468);
\draw [c] (1.28154,4.05468) -- (1.29636,4.05468);
\definecolor{c}{rgb}{0,0,0};
\colorlet{c}{kugray};
\draw [c] (1.31118,5.36749) -- (1.31118,5.38512);
\draw [c] (1.31118,5.38512) -- (1.31118,5.40179);
\draw [c] (1.29636,5.38512) -- (1.31118,5.38512);
\draw [c] (1.31118,5.38512) -- (1.32599,5.38512);
\definecolor{c}{rgb}{0,0,0};
\colorlet{c}{kugray};
\draw [c] (1.34081,5.61993) -- (1.34081,5.63126);
\draw [c] (1.34081,5.63126) -- (1.34081,5.6422);
\draw [c] (1.32599,5.63126) -- (1.34081,5.63126);
\draw [c] (1.34081,5.63126) -- (1.35563,5.63126);
\definecolor{c}{rgb}{0,0,0};
\colorlet{c}{kugray};
\draw [c] (1.37045,5.69182) -- (1.37045,5.70194);
\draw [c] (1.37045,5.70194) -- (1.37045,5.71174);
\draw [c] (1.35563,5.70194) -- (1.37045,5.70194);
\draw [c] (1.37045,5.70194) -- (1.38526,5.70194);
\definecolor{c}{rgb}{0,0,0};
\colorlet{c}{kugray};
\draw [c] (1.40008,5.67982) -- (1.40008,5.69024);
\draw [c] (1.40008,5.69024) -- (1.40008,5.70031);
\draw [c] (1.38526,5.69024) -- (1.40008,5.69024);
\draw [c] (1.40008,5.69024) -- (1.4149,5.69024);
\definecolor{c}{rgb}{0,0,0};
\colorlet{c}{kugray};
\draw [c] (1.42972,5.65827) -- (1.42972,5.66928);
\draw [c] (1.42972,5.66928) -- (1.42972,5.67991);
\draw [c] (1.4149,5.66928) -- (1.42972,5.66928);
\draw [c] (1.42972,5.66928) -- (1.44454,5.66928);
\definecolor{c}{rgb}{0,0,0};
\colorlet{c}{kugray};
\draw [c] (1.45935,5.62) -- (1.45935,5.63141);
\draw [c] (1.45935,5.63141) -- (1.45935,5.6424);
\draw [c] (1.44454,5.63141) -- (1.45935,5.63141);
\draw [c] (1.45935,5.63141) -- (1.47417,5.63141);
\definecolor{c}{rgb}{0,0,0};
\colorlet{c}{kugray};
\draw [c] (1.48899,5.58164) -- (1.48899,5.5938);
\draw [c] (1.48899,5.5938) -- (1.48899,5.60549);
\draw [c] (1.47417,5.5938) -- (1.48899,5.5938);
\draw [c] (1.48899,5.5938) -- (1.50381,5.5938);
\definecolor{c}{rgb}{0,0,0};
\colorlet{c}{kugray};
\draw [c] (1.51863,5.53046) -- (1.51863,5.54392);
\draw [c] (1.51863,5.54392) -- (1.51863,5.55682);
\draw [c] (1.50381,5.54392) -- (1.51863,5.54392);
\draw [c] (1.51863,5.54392) -- (1.53344,5.54392);
\definecolor{c}{rgb}{0,0,0};
\colorlet{c}{kugray};
\draw [c] (1.54826,5.46022) -- (1.54826,5.47505);
\draw [c] (1.54826,5.47505) -- (1.54826,5.48921);
\draw [c] (1.53344,5.47505) -- (1.54826,5.47505);
\draw [c] (1.54826,5.47505) -- (1.56308,5.47505);
\definecolor{c}{rgb}{0,0,0};
\colorlet{c}{kugray};
\draw [c] (1.5779,5.41506) -- (1.5779,5.43093);
\draw [c] (1.5779,5.43093) -- (1.5779,5.44602);
\draw [c] (1.56308,5.43093) -- (1.5779,5.43093);
\draw [c] (1.5779,5.43093) -- (1.59272,5.43093);
\definecolor{c}{rgb}{0,0,0};
\colorlet{c}{kugray};
\draw [c] (1.60753,5.38705) -- (1.60753,5.40429);
\draw [c] (1.60753,5.40429) -- (1.60753,5.42061);
\draw [c] (1.59272,5.40429) -- (1.60753,5.40429);
\draw [c] (1.60753,5.40429) -- (1.62235,5.40429);
\definecolor{c}{rgb}{0,0,0};
\colorlet{c}{kugray};
\draw [c] (1.63717,5.33118) -- (1.63717,5.34945);
\draw [c] (1.63717,5.34945) -- (1.63717,5.36669);
\draw [c] (1.62235,5.34945) -- (1.63717,5.34945);
\draw [c] (1.63717,5.34945) -- (1.65199,5.34945);
\definecolor{c}{rgb}{0,0,0};
\colorlet{c}{kugray};
\draw [c] (1.6668,5.33501) -- (1.6668,5.35318);
\draw [c] (1.6668,5.35318) -- (1.6668,5.37033);
\draw [c] (1.65199,5.35318) -- (1.6668,5.35318);
\draw [c] (1.6668,5.35318) -- (1.68162,5.35318);
\definecolor{c}{rgb}{0,0,0};
\colorlet{c}{kugray};
\draw [c] (1.69644,5.27038) -- (1.69644,5.29099);
\draw [c] (1.69644,5.29099) -- (1.69644,5.3103);
\draw [c] (1.68162,5.29099) -- (1.69644,5.29099);
\draw [c] (1.69644,5.29099) -- (1.71126,5.29099);
\definecolor{c}{rgb}{0,0,0};
\colorlet{c}{kugray};
\draw [c] (1.72608,5.20804) -- (1.72608,5.23001);
\draw [c] (1.72608,5.23001) -- (1.72608,5.25052);
\draw [c] (1.71126,5.23001) -- (1.72608,5.23001);
\draw [c] (1.72608,5.23001) -- (1.74089,5.23001);
\definecolor{c}{rgb}{0,0,0};
\colorlet{c}{kugray};
\draw [c] (1.75571,5.08901) -- (1.75571,5.11549);
\draw [c] (1.75571,5.11549) -- (1.75571,5.13987);
\draw [c] (1.74089,5.11549) -- (1.75571,5.11549);
\draw [c] (1.75571,5.11549) -- (1.77053,5.11549);
\definecolor{c}{rgb}{0,0,0};
\colorlet{c}{kugray};
\draw [c] (1.78535,5.10181) -- (1.78535,5.12931);
\draw [c] (1.78535,5.12931) -- (1.78535,5.15455);
\draw [c] (1.77053,5.12931) -- (1.78535,5.12931);
\draw [c] (1.78535,5.12931) -- (1.80017,5.12931);
\definecolor{c}{rgb}{0,0,0};
\colorlet{c}{kugray};
\draw [c] (1.81498,5.06396) -- (1.81498,5.09313);
\draw [c] (1.81498,5.09313) -- (1.81498,5.11978);
\draw [c] (1.80017,5.09313) -- (1.81498,5.09313);
\draw [c] (1.81498,5.09313) -- (1.8298,5.09313);
\definecolor{c}{rgb}{0,0,0};
\colorlet{c}{kugray};
\draw [c] (1.84462,5.06537) -- (1.84462,5.09456);
\draw [c] (1.84462,5.09456) -- (1.84462,5.12123);
\draw [c] (1.8298,5.09456) -- (1.84462,5.09456);
\draw [c] (1.84462,5.09456) -- (1.85944,5.09456);
\definecolor{c}{rgb}{0,0,0};
\colorlet{c}{kugray};
\draw [c] (1.87425,4.99725) -- (1.87425,5.02835);
\draw [c] (1.87425,5.02835) -- (1.87425,5.05659);
\draw [c] (1.85944,5.02835) -- (1.87425,5.02835);
\draw [c] (1.87425,5.02835) -- (1.88907,5.02835);
\definecolor{c}{rgb}{0,0,0};
\colorlet{c}{kugray};
\draw [c] (1.90389,4.95832) -- (1.90389,4.99102);
\draw [c] (1.90389,4.99102) -- (1.90389,5.02058);
\draw [c] (1.88907,4.99102) -- (1.90389,4.99102);
\draw [c] (1.90389,4.99102) -- (1.91871,4.99102);
\definecolor{c}{rgb}{0,0,0};
\colorlet{c}{kugray};
\draw [c] (1.93353,4.85544) -- (1.93353,4.89383);
\draw [c] (1.93353,4.89383) -- (1.93353,4.92796);
\draw [c] (1.91871,4.89383) -- (1.93353,4.89383);
\draw [c] (1.93353,4.89383) -- (1.94834,4.89383);
\definecolor{c}{rgb}{0,0,0};
\colorlet{c}{kugray};
\draw [c] (1.96316,4.88333) -- (1.96316,4.92161);
\draw [c] (1.96316,4.92161) -- (1.96316,4.95565);
\draw [c] (1.94834,4.92161) -- (1.96316,4.92161);
\draw [c] (1.96316,4.92161) -- (1.97798,4.92161);
\definecolor{c}{rgb}{0,0,0};
\colorlet{c}{kugray};
\draw [c] (1.9928,4.85825) -- (1.9928,4.89733);
\draw [c] (1.9928,4.89733) -- (1.9928,4.93201);
\draw [c] (1.97798,4.89733) -- (1.9928,4.89733);
\draw [c] (1.9928,4.89733) -- (2.00762,4.89733);
\definecolor{c}{rgb}{0,0,0};
\colorlet{c}{kugray};
\draw [c] (2.02243,4.78026) -- (2.02243,4.82199);
\draw [c] (2.02243,4.82199) -- (2.02243,4.85874);
\draw [c] (2.00762,4.82199) -- (2.02243,4.82199);
\draw [c] (2.02243,4.82199) -- (2.03725,4.82199);
\definecolor{c}{rgb}{0,0,0};
\colorlet{c}{kugray};
\draw [c] (2.05207,4.74394) -- (2.05207,4.7911);
\draw [c] (2.05207,4.7911) -- (2.05207,4.83199);
\draw [c] (2.03725,4.7911) -- (2.05207,4.7911);
\draw [c] (2.05207,4.7911) -- (2.06689,4.7911);
\definecolor{c}{rgb}{0,0,0};
\colorlet{c}{kugray};
\draw [c] (2.08171,4.69059) -- (2.08171,4.74618);
\draw [c] (2.08171,4.74618) -- (2.08171,4.79324);
\draw [c] (2.06689,4.74618) -- (2.08171,4.74618);
\draw [c] (2.08171,4.74618) -- (2.09652,4.74618);
\definecolor{c}{rgb}{0,0,0};
\colorlet{c}{kugray};
\draw [c] (2.11134,4.64498) -- (2.11134,4.70439);
\draw [c] (2.11134,4.70439) -- (2.11134,4.75417);
\draw [c] (2.09652,4.70439) -- (2.11134,4.70439);
\draw [c] (2.11134,4.70439) -- (2.12616,4.70439);
\definecolor{c}{rgb}{0,0,0};
\colorlet{c}{kugray};
\draw [c] (2.14098,4.66916) -- (2.14098,4.726);
\draw [c] (2.14098,4.726) -- (2.14098,4.77396);
\draw [c] (2.12616,4.726) -- (2.14098,4.726);
\draw [c] (2.14098,4.726) -- (2.15579,4.726);
\definecolor{c}{rgb}{0,0,0};
\colorlet{c}{kugray};
\draw [c] (2.17061,4.71202) -- (2.17061,4.76221);
\draw [c] (2.17061,4.76221) -- (2.17061,4.80536);
\draw [c] (2.15579,4.76221) -- (2.17061,4.76221);
\draw [c] (2.17061,4.76221) -- (2.18543,4.76221);
\definecolor{c}{rgb}{0,0,0};
\colorlet{c}{kugray};
\draw [c] (2.20025,4.43289) -- (2.20025,4.50711);
\draw [c] (2.20025,4.50711) -- (2.20025,4.56686);
\draw [c] (2.18543,4.50711) -- (2.20025,4.50711);
\draw [c] (2.20025,4.50711) -- (2.21507,4.50711);
\definecolor{c}{rgb}{0,0,0};
\colorlet{c}{kugray};
\draw [c] (2.22988,4.60564) -- (2.22988,4.66562);
\draw [c] (2.22988,4.66562) -- (2.22988,4.7158);
\draw [c] (2.21507,4.66562) -- (2.22988,4.66562);
\draw [c] (2.22988,4.66562) -- (2.2447,4.66562);
\definecolor{c}{rgb}{0,0,0};
\colorlet{c}{kugray};
\draw [c] (2.25952,4.64159) -- (2.25952,4.70326);
\draw [c] (2.25952,4.70326) -- (2.25952,4.75461);
\draw [c] (2.2447,4.70326) -- (2.25952,4.70326);
\draw [c] (2.25952,4.70326) -- (2.27434,4.70326);
\definecolor{c}{rgb}{0,0,0};
\colorlet{c}{kugray};
\draw [c] (2.28916,4.56718) -- (2.28916,4.57808);
\draw [c] (2.28916,4.57808) -- (2.28916,4.5886);
\draw [c] (2.27434,4.57808) -- (2.28916,4.57808);
\draw [c] (2.28916,4.57808) -- (2.30397,4.57808);
\definecolor{c}{rgb}{0,0,0};
\colorlet{c}{kugray};
\draw [c] (2.31879,4.51802) -- (2.31879,4.5299);
\draw [c] (2.31879,4.5299) -- (2.31879,4.54134);
\draw [c] (2.30397,4.5299) -- (2.31879,4.5299);
\draw [c] (2.31879,4.5299) -- (2.33361,4.5299);
\definecolor{c}{rgb}{0,0,0};
\colorlet{c}{kugray};
\draw [c] (2.34843,4.51942) -- (2.34843,4.53112);
\draw [c] (2.34843,4.53112) -- (2.34843,4.54238);
\draw [c] (2.33361,4.53112) -- (2.34843,4.53112);
\draw [c] (2.34843,4.53112) -- (2.36325,4.53112);
\definecolor{c}{rgb}{0,0,0};
\colorlet{c}{kugray};
\draw [c] (2.37806,4.47939) -- (2.37806,4.492);
\draw [c] (2.37806,4.492) -- (2.37806,4.50411);
\draw [c] (2.36325,4.492) -- (2.37806,4.492);
\draw [c] (2.37806,4.492) -- (2.39288,4.492);
\definecolor{c}{rgb}{0,0,0};
\colorlet{c}{kugray};
\draw [c] (2.4077,4.44091) -- (2.4077,4.45439);
\draw [c] (2.4077,4.45439) -- (2.4077,4.46731);
\draw [c] (2.39288,4.45439) -- (2.4077,4.45439);
\draw [c] (2.4077,4.45439) -- (2.42252,4.45439);
\definecolor{c}{rgb}{0,0,0};
\colorlet{c}{kugray};
\draw [c] (2.43733,4.39585) -- (2.43733,4.41002);
\draw [c] (2.43733,4.41002) -- (2.43733,4.42357);
\draw [c] (2.42252,4.41002) -- (2.43733,4.41002);
\draw [c] (2.43733,4.41002) -- (2.45215,4.41002);
\definecolor{c}{rgb}{0,0,0};
\colorlet{c}{kugray};
\draw [c] (2.46697,4.36562) -- (2.46697,4.38084);
\draw [c] (2.46697,4.38084) -- (2.46697,4.39535);
\draw [c] (2.45215,4.38084) -- (2.46697,4.38084);
\draw [c] (2.46697,4.38084) -- (2.48179,4.38084);
\definecolor{c}{rgb}{0,0,0};
\colorlet{c}{kugray};
\draw [c] (2.49661,4.34442) -- (2.49661,4.3598);
\draw [c] (2.49661,4.3598) -- (2.49661,4.37444);
\draw [c] (2.48179,4.3598) -- (2.49661,4.3598);
\draw [c] (2.49661,4.3598) -- (2.51142,4.3598);
\definecolor{c}{rgb}{0,0,0};
\colorlet{c}{kugray};
\draw [c] (2.52624,4.33876) -- (2.52624,4.3547);
\draw [c] (2.52624,4.3547) -- (2.52624,4.36985);
\draw [c] (2.51142,4.3547) -- (2.52624,4.3547);
\draw [c] (2.52624,4.3547) -- (2.54106,4.3547);
\definecolor{c}{rgb}{0,0,0};
\colorlet{c}{kugray};
\draw [c] (2.55588,4.31164) -- (2.55588,4.32829);
\draw [c] (2.55588,4.32829) -- (2.55588,4.34409);
\draw [c] (2.54106,4.32829) -- (2.55588,4.32829);
\draw [c] (2.55588,4.32829) -- (2.5707,4.32829);
\definecolor{c}{rgb}{0,0,0};
\colorlet{c}{kugray};
\draw [c] (2.58551,4.27319) -- (2.58551,4.29119);
\draw [c] (2.58551,4.29119) -- (2.58551,4.30819);
\draw [c] (2.5707,4.29119) -- (2.58551,4.29119);
\draw [c] (2.58551,4.29119) -- (2.60033,4.29119);
\definecolor{c}{rgb}{0,0,0};
\colorlet{c}{kugray};
\draw [c] (2.61515,4.23142) -- (2.61515,4.25043);
\draw [c] (2.61515,4.25043) -- (2.61515,4.26834);
\draw [c] (2.60033,4.25043) -- (2.61515,4.25043);
\draw [c] (2.61515,4.25043) -- (2.62997,4.25043);
\definecolor{c}{rgb}{0,0,0};
\colorlet{c}{kugray};
\draw [c] (2.64478,4.1941) -- (2.64478,4.21371);
\draw [c] (2.64478,4.21371) -- (2.64478,4.23215);
\draw [c] (2.62997,4.21371) -- (2.64478,4.21371);
\draw [c] (2.64478,4.21371) -- (2.6596,4.21371);
\definecolor{c}{rgb}{0,0,0};
\colorlet{c}{kugray};
\draw [c] (2.67442,4.21383) -- (2.67442,4.23319);
\draw [c] (2.67442,4.23319) -- (2.67442,4.2514);
\draw [c] (2.6596,4.23319) -- (2.67442,4.23319);
\draw [c] (2.67442,4.23319) -- (2.68924,4.23319);
\definecolor{c}{rgb}{0,0,0};
\colorlet{c}{kugray};
\draw [c] (2.70406,4.15434) -- (2.70406,4.17536);
\draw [c] (2.70406,4.17536) -- (2.70406,4.19503);
\draw [c] (2.68924,4.17536) -- (2.70406,4.17536);
\draw [c] (2.70406,4.17536) -- (2.71887,4.17536);
\definecolor{c}{rgb}{0,0,0};
\colorlet{c}{kugray};
\draw [c] (2.73369,4.14693) -- (2.73369,4.16849);
\draw [c] (2.73369,4.16849) -- (2.73369,4.18863);
\draw [c] (2.71887,4.16849) -- (2.73369,4.16849);
\draw [c] (2.73369,4.16849) -- (2.74851,4.16849);
\definecolor{c}{rgb}{0,0,0};
\colorlet{c}{kugray};
\draw [c] (2.76333,4.12489) -- (2.76333,4.1477);
\draw [c] (2.76333,4.1477) -- (2.76333,4.16894);
\draw [c] (2.74851,4.1477) -- (2.76333,4.1477);
\draw [c] (2.76333,4.1477) -- (2.77815,4.1477);
\definecolor{c}{rgb}{0,0,0};
\colorlet{c}{kugray};
\draw [c] (2.79296,4.13183) -- (2.79296,4.15469);
\draw [c] (2.79296,4.15469) -- (2.79296,4.17597);
\draw [c] (2.77815,4.15469) -- (2.79296,4.15469);
\draw [c] (2.79296,4.15469) -- (2.80778,4.15469);
\definecolor{c}{rgb}{0,0,0};
\colorlet{c}{kugray};
\draw [c] (2.8226,4.05797) -- (2.8226,4.08216);
\draw [c] (2.8226,4.08216) -- (2.8226,4.10458);
\draw [c] (2.80778,4.08216) -- (2.8226,4.08216);
\draw [c] (2.8226,4.08216) -- (2.83742,4.08216);
\definecolor{c}{rgb}{0,0,0};
\colorlet{c}{kugray};
\draw [c] (2.85224,4.04617) -- (2.85224,4.07199);
\draw [c] (2.85224,4.07199) -- (2.85224,4.09581);
\draw [c] (2.83742,4.07199) -- (2.85224,4.07199);
\draw [c] (2.85224,4.07199) -- (2.86705,4.07199);
\definecolor{c}{rgb}{0,0,0};
\colorlet{c}{kugray};
\draw [c] (2.88187,4.02533) -- (2.88187,4.05093);
\draw [c] (2.88187,4.05093) -- (2.88187,4.07457);
\draw [c] (2.86705,4.05093) -- (2.88187,4.05093);
\draw [c] (2.88187,4.05093) -- (2.89669,4.05093);
\definecolor{c}{rgb}{0,0,0};
\colorlet{c}{kugray};
\draw [c] (2.91151,4.0461) -- (2.91151,4.07164);
\draw [c] (2.91151,4.07164) -- (2.91151,4.09522);
\draw [c] (2.89669,4.07164) -- (2.91151,4.07164);
\draw [c] (2.91151,4.07164) -- (2.92632,4.07164);
\definecolor{c}{rgb}{0,0,0};
\colorlet{c}{kugray};
\draw [c] (2.94114,3.94743) -- (2.94114,3.97699);
\draw [c] (2.94114,3.97699) -- (2.94114,4.00396);
\draw [c] (2.92632,3.97699) -- (2.94114,3.97699);
\draw [c] (2.94114,3.97699) -- (2.95596,3.97699);
\definecolor{c}{rgb}{0,0,0};
\colorlet{c}{kugray};
\draw [c] (2.97078,3.92033) -- (2.97078,3.95122);
\draw [c] (2.97078,3.95122) -- (2.97078,3.97929);
\draw [c] (2.95596,3.95122) -- (2.97078,3.95122);
\draw [c] (2.97078,3.95122) -- (2.9856,3.95122);
\definecolor{c}{rgb}{0,0,0};
\colorlet{c}{kugray};
\draw [c] (3.00041,3.95455) -- (3.00041,3.98496);
\draw [c] (3.00041,3.98496) -- (3.00041,4.01263);
\draw [c] (2.9856,3.98496) -- (3.00041,3.98496);
\draw [c] (3.00041,3.98496) -- (3.01523,3.98496);
\definecolor{c}{rgb}{0,0,0};
\colorlet{c}{kugray};
\draw [c] (3.03005,3.91612) -- (3.03005,3.94874);
\draw [c] (3.03005,3.94874) -- (3.03005,3.97824);
\draw [c] (3.01523,3.94874) -- (3.03005,3.94874);
\draw [c] (3.03005,3.94874) -- (3.04487,3.94874);
\definecolor{c}{rgb}{0,0,0};
\colorlet{c}{kugray};
\draw [c] (3.05969,3.91532) -- (3.05969,3.94707);
\draw [c] (3.05969,3.94707) -- (3.05969,3.97585);
\draw [c] (3.04487,3.94707) -- (3.05969,3.94707);
\draw [c] (3.05969,3.94707) -- (3.0745,3.94707);
\definecolor{c}{rgb}{0,0,0};
\colorlet{c}{kugray};
\draw [c] (3.08932,3.82813) -- (3.08932,3.8633);
\draw [c] (3.08932,3.8633) -- (3.08932,3.89486);
\draw [c] (3.0745,3.8633) -- (3.08932,3.8633);
\draw [c] (3.08932,3.8633) -- (3.10414,3.8633);
\definecolor{c}{rgb}{0,0,0};
\colorlet{c}{kugray};
\draw [c] (3.11896,3.82152) -- (3.11896,3.85725);
\draw [c] (3.11896,3.85725) -- (3.11896,3.88926);
\draw [c] (3.10414,3.85725) -- (3.11896,3.85725);
\draw [c] (3.11896,3.85725) -- (3.13377,3.85725);
\definecolor{c}{rgb}{0,0,0};
\colorlet{c}{kugray};
\draw [c] (3.14859,3.81099) -- (3.14859,3.849);
\draw [c] (3.14859,3.849) -- (3.14859,3.88283);
\draw [c] (3.13377,3.849) -- (3.14859,3.849);
\draw [c] (3.14859,3.849) -- (3.16341,3.849);
\definecolor{c}{rgb}{0,0,0};
\colorlet{c}{kugray};
\draw [c] (3.17823,3.68505) -- (3.17823,3.72686);
\draw [c] (3.17823,3.72686) -- (3.17823,3.76367);
\draw [c] (3.16341,3.72686) -- (3.17823,3.72686);
\draw [c] (3.17823,3.72686) -- (3.19305,3.72686);
\definecolor{c}{rgb}{0,0,0};
\colorlet{c}{kugray};
\draw [c] (3.20786,3.75027) -- (3.20786,3.79207);
\draw [c] (3.20786,3.79207) -- (3.20786,3.82886);
\draw [c] (3.19305,3.79207) -- (3.20786,3.79207);
\draw [c] (3.20786,3.79207) -- (3.22268,3.79207);
\definecolor{c}{rgb}{0,0,0};
\colorlet{c}{kugray};
\draw [c] (3.2375,3.78399) -- (3.2375,3.82558);
\draw [c] (3.2375,3.82558) -- (3.2375,3.86222);
\draw [c] (3.22268,3.82558) -- (3.2375,3.82558);
\draw [c] (3.2375,3.82558) -- (3.25232,3.82558);
\definecolor{c}{rgb}{0,0,0};
\colorlet{c}{kugray};
\draw [c] (3.26714,3.71561) -- (3.26714,3.76154);
\draw [c] (3.26714,3.76154) -- (3.26714,3.80149);
\draw [c] (3.25232,3.76154) -- (3.26714,3.76154);
\draw [c] (3.26714,3.76154) -- (3.28195,3.76154);
\definecolor{c}{rgb}{0,0,0};
\colorlet{c}{kugray};
\draw [c] (3.29677,3.74464) -- (3.29677,3.78592);
\draw [c] (3.29677,3.78592) -- (3.29677,3.82231);
\draw [c] (3.28195,3.78592) -- (3.29677,3.78592);
\draw [c] (3.29677,3.78592) -- (3.31159,3.78592);
\definecolor{c}{rgb}{0,0,0};
\colorlet{c}{kugray};
\draw [c] (3.32641,3.60212) -- (3.32641,3.65046);
\draw [c] (3.32641,3.65046) -- (3.32641,3.69223);
\draw [c] (3.31159,3.65046) -- (3.32641,3.65046);
\draw [c] (3.32641,3.65046) -- (3.34123,3.65046);
\definecolor{c}{rgb}{0,0,0};
\colorlet{c}{kugray};
\draw [c] (3.35604,3.66614) -- (3.35604,3.71219);
\draw [c] (3.35604,3.71219) -- (3.35604,3.75225);
\draw [c] (3.34123,3.71219) -- (3.35604,3.71219);
\draw [c] (3.35604,3.71219) -- (3.37086,3.71219);
\definecolor{c}{rgb}{0,0,0};
\colorlet{c}{kugray};
\draw [c] (3.38568,3.70789) -- (3.38568,3.7536);
\draw [c] (3.38568,3.7536) -- (3.38568,3.7934);
\draw [c] (3.37086,3.7536) -- (3.38568,3.7536);
\draw [c] (3.38568,3.7536) -- (3.4005,3.7536);
\definecolor{c}{rgb}{0,0,0};
\colorlet{c}{kugray};
\draw [c] (3.41531,3.59749) -- (3.41531,3.64919);
\draw [c] (3.41531,3.64919) -- (3.41531,3.69343);
\draw [c] (3.4005,3.64919) -- (3.41531,3.64919);
\draw [c] (3.41531,3.64919) -- (3.43013,3.64919);
\definecolor{c}{rgb}{0,0,0};
\colorlet{c}{kugray};
\draw [c] (3.44495,3.54922) -- (3.44495,3.6056);
\draw [c] (3.44495,3.6056) -- (3.44495,3.65323);
\draw [c] (3.43013,3.6056) -- (3.44495,3.6056);
\draw [c] (3.44495,3.6056) -- (3.45977,3.6056);
\definecolor{c}{rgb}{0,0,0};
\colorlet{c}{kugray};
\draw [c] (3.47459,3.5315) -- (3.47459,3.58602);
\draw [c] (3.47459,3.58602) -- (3.47459,3.63231);
\draw [c] (3.45977,3.58602) -- (3.47459,3.58602);
\draw [c] (3.47459,3.58602) -- (3.4894,3.58602);
\definecolor{c}{rgb}{0,0,0};
\colorlet{c}{kugray};
\draw [c] (3.50422,3.44087) -- (3.50422,3.50756);
\draw [c] (3.50422,3.50756) -- (3.50422,3.56234);
\draw [c] (3.4894,3.50756) -- (3.50422,3.50756);
\draw [c] (3.50422,3.50756) -- (3.51904,3.50756);
\definecolor{c}{rgb}{0,0,0};
\colorlet{c}{kugray};
\draw [c] (3.53386,3.57716) -- (3.53386,3.62954);
\draw [c] (3.53386,3.62954) -- (3.53386,3.67429);
\draw [c] (3.51904,3.62954) -- (3.53386,3.62954);
\draw [c] (3.53386,3.62954) -- (3.54868,3.62954);
\definecolor{c}{rgb}{0,0,0};
\colorlet{c}{kugray};
\draw [c] (3.56349,3.59791) -- (3.56349,3.64909);
\draw [c] (3.56349,3.64909) -- (3.56349,3.69297);
\draw [c] (3.54868,3.64909) -- (3.56349,3.64909);
\draw [c] (3.56349,3.64909) -- (3.57831,3.64909);
\definecolor{c}{rgb}{0,0,0};
\colorlet{c}{kugray};
\draw [c] (3.59313,3.48016) -- (3.59313,3.54299);
\draw [c] (3.59313,3.54299) -- (3.59313,3.59515);
\draw [c] (3.57831,3.54299) -- (3.59313,3.54299);
\draw [c] (3.59313,3.54299) -- (3.60795,3.54299);
\definecolor{c}{rgb}{0,0,0};
\colorlet{c}{kugray};
\draw [c] (3.62276,3.49926) -- (3.62276,3.5653);
\draw [c] (3.62276,3.5653) -- (3.62276,3.61965);
\draw [c] (3.60795,3.5653) -- (3.62276,3.5653);
\draw [c] (3.62276,3.5653) -- (3.63758,3.5653);
\definecolor{c}{rgb}{0,0,0};
\colorlet{c}{kugray};
\draw [c] (3.6524,3.38318) -- (3.6524,3.45701);
\draw [c] (3.6524,3.45701) -- (3.6524,3.51651);
\draw [c] (3.63758,3.45701) -- (3.6524,3.45701);
\draw [c] (3.6524,3.45701) -- (3.66722,3.45701);
\definecolor{c}{rgb}{0,0,0};
\colorlet{c}{kugray};
\draw [c] (3.68204,3.38716) -- (3.68204,3.45825);
\draw [c] (3.68204,3.45825) -- (3.68204,3.51597);
\draw [c] (3.66722,3.45825) -- (3.68204,3.45825);
\draw [c] (3.68204,3.45825) -- (3.69685,3.45825);
\definecolor{c}{rgb}{0,0,0};
\colorlet{c}{kugray};
\draw [c] (3.71167,3.27025) -- (3.71167,3.35157);
\draw [c] (3.71167,3.35157) -- (3.71167,3.41584);
\draw [c] (3.69685,3.35157) -- (3.71167,3.35157);
\draw [c] (3.71167,3.35157) -- (3.72649,3.35157);
\definecolor{c}{rgb}{0,0,0};
\colorlet{c}{kugray};
\draw [c] (3.74131,3.42793) -- (3.74131,3.49218);
\draw [c] (3.74131,3.49218) -- (3.74131,3.54531);
\draw [c] (3.72649,3.49218) -- (3.74131,3.49218);
\draw [c] (3.74131,3.49218) -- (3.75613,3.49218);
\definecolor{c}{rgb}{0,0,0};
\colorlet{c}{kugray};
\draw [c] (3.77094,3.39681) -- (3.77094,3.4655);
\draw [c] (3.77094,3.4655) -- (3.77094,3.52163);
\draw [c] (3.75613,3.4655) -- (3.77094,3.4655);
\draw [c] (3.77094,3.4655) -- (3.78576,3.4655);
\definecolor{c}{rgb}{0,0,0};
\colorlet{c}{kugray};
\draw [c] (3.80058,3.30807) -- (3.80058,3.39026);
\draw [c] (3.80058,3.39026) -- (3.80058,3.45506);
\draw [c] (3.78576,3.39026) -- (3.80058,3.39026);
\draw [c] (3.80058,3.39026) -- (3.8154,3.39026);
\definecolor{c}{rgb}{0,0,0};
\colorlet{c}{kugray};
\draw [c] (3.83022,3.36833) -- (3.83022,3.44242);
\draw [c] (3.83022,3.44242) -- (3.83022,3.50209);
\draw [c] (3.8154,3.44242) -- (3.83022,3.44242);
\draw [c] (3.83022,3.44242) -- (3.84503,3.44242);
\definecolor{c}{rgb}{0,0,0};
\colorlet{c}{kugray};
\draw [c] (3.85985,3.36697) -- (3.85985,3.43885);
\draw [c] (3.85985,3.43885) -- (3.85985,3.49708);
\draw [c] (3.84503,3.43885) -- (3.85985,3.43885);
\draw [c] (3.85985,3.43885) -- (3.87467,3.43885);
\definecolor{c}{rgb}{0,0,0};
\colorlet{c}{kugray};
\draw [c] (3.88949,3.18339) -- (3.88949,3.27641);
\draw [c] (3.88949,3.27641) -- (3.88949,3.34775);
\draw [c] (3.87467,3.27641) -- (3.88949,3.27641);
\draw [c] (3.88949,3.27641) -- (3.9043,3.27641);
\definecolor{c}{rgb}{0,0,0};
\colorlet{c}{kugray};
\draw [c] (3.91912,3.36859) -- (3.91912,3.433);
\draw [c] (3.91912,3.433) -- (3.91912,3.48624);
\draw [c] (3.9043,3.433) -- (3.91912,3.433);
\draw [c] (3.91912,3.433) -- (3.93394,3.433);
\definecolor{c}{rgb}{0,0,0};
\colorlet{c}{kugray};
\draw [c] (3.94876,3.27957) -- (3.94876,3.34398);
\draw [c] (3.94876,3.34398) -- (3.94876,3.39722);
\draw [c] (3.93394,3.34398) -- (3.94876,3.34398);
\draw [c] (3.94876,3.34398) -- (3.96358,3.34398);
\definecolor{c}{rgb}{0,0,0};
\colorlet{c}{kugray};
\draw [c] (3.97839,3.23181) -- (3.97839,3.2842);
\draw [c] (3.97839,3.2842) -- (3.97839,3.32896);
\draw [c] (3.96358,3.2842) -- (3.97839,3.2842);
\draw [c] (3.97839,3.2842) -- (3.99321,3.2842);
\definecolor{c}{rgb}{0,0,0};
\colorlet{c}{kugray};
\draw [c] (4.00803,3.23298) -- (4.00803,3.24514);
\draw [c] (4.00803,3.24514) -- (4.00803,3.25684);
\draw [c] (3.99321,3.24514) -- (4.00803,3.24514);
\draw [c] (4.00803,3.24514) -- (4.02285,3.24514);
\definecolor{c}{rgb}{0,0,0};
\colorlet{c}{kugray};
\draw [c] (4.03767,3.23899) -- (4.03767,3.25079);
\draw [c] (4.03767,3.25079) -- (4.03767,3.26215);
\draw [c] (4.02285,3.25079) -- (4.03767,3.25079);
\draw [c] (4.03767,3.25079) -- (4.05248,3.25079);
\definecolor{c}{rgb}{0,0,0};
\colorlet{c}{kugray};
\draw [c] (4.0673,3.24063) -- (4.0673,3.25269);
\draw [c] (4.0673,3.25269) -- (4.0673,3.26429);
\draw [c] (4.05248,3.25269) -- (4.0673,3.25269);
\draw [c] (4.0673,3.25269) -- (4.08212,3.25269);
\definecolor{c}{rgb}{0,0,0};
\colorlet{c}{kugray};
\draw [c] (4.09694,3.22722) -- (4.09694,3.23915);
\draw [c] (4.09694,3.23915) -- (4.09694,3.25063);
\draw [c] (4.08212,3.23915) -- (4.09694,3.23915);
\draw [c] (4.09694,3.23915) -- (4.11175,3.23915);
\definecolor{c}{rgb}{0,0,0};
\colorlet{c}{kugray};
\draw [c] (4.12657,3.2116) -- (4.12657,3.22405);
\draw [c] (4.12657,3.22405) -- (4.12657,3.23601);
\draw [c] (4.11175,3.22405) -- (4.12657,3.22405);
\draw [c] (4.12657,3.22405) -- (4.14139,3.22405);
\definecolor{c}{rgb}{0,0,0};
\colorlet{c}{kugray};
\draw [c] (4.15621,3.18215) -- (4.15621,3.19496);
\draw [c] (4.15621,3.19496) -- (4.15621,3.20726);
\draw [c] (4.14139,3.19496) -- (4.15621,3.19496);
\draw [c] (4.15621,3.19496) -- (4.17103,3.19496);
\definecolor{c}{rgb}{0,0,0};
\colorlet{c}{kugray};
\draw [c] (4.18584,3.18206) -- (4.18584,3.19521);
\draw [c] (4.18584,3.19521) -- (4.18584,3.20782);
\draw [c] (4.17103,3.19521) -- (4.18584,3.19521);
\draw [c] (4.18584,3.19521) -- (4.20066,3.19521);
\definecolor{c}{rgb}{0,0,0};
\colorlet{c}{kugray};
\draw [c] (4.21548,3.14682) -- (4.21548,3.16086);
\draw [c] (4.21548,3.16086) -- (4.21548,3.17428);
\draw [c] (4.20066,3.16086) -- (4.21548,3.16086);
\draw [c] (4.21548,3.16086) -- (4.2303,3.16086);
\definecolor{c}{rgb}{0,0,0};
\colorlet{c}{kugray};
\draw [c] (4.24512,3.16809) -- (4.24512,3.1818);
\draw [c] (4.24512,3.1818) -- (4.24512,3.19493);
\draw [c] (4.2303,3.1818) -- (4.24512,3.1818);
\draw [c] (4.24512,3.1818) -- (4.25993,3.1818);
\definecolor{c}{rgb}{0,0,0};
\colorlet{c}{kugray};
\draw [c] (4.27475,3.10937) -- (4.27475,3.12421);
\draw [c] (4.27475,3.12421) -- (4.27475,3.13837);
\draw [c] (4.25993,3.12421) -- (4.27475,3.12421);
\draw [c] (4.27475,3.12421) -- (4.28957,3.12421);
\definecolor{c}{rgb}{0,0,0};
\colorlet{c}{kugray};
\draw [c] (4.30439,3.11556) -- (4.30439,3.12992);
\draw [c] (4.30439,3.12992) -- (4.30439,3.14364);
\draw [c] (4.28957,3.12992) -- (4.30439,3.12992);
\draw [c] (4.30439,3.12992) -- (4.31921,3.12992);
\definecolor{c}{rgb}{0,0,0};
\colorlet{c}{kugray};
\draw [c] (4.33402,3.13544) -- (4.33402,3.14949);
\draw [c] (4.33402,3.14949) -- (4.33402,3.16293);
\draw [c] (4.31921,3.14949) -- (4.33402,3.14949);
\draw [c] (4.33402,3.14949) -- (4.34884,3.14949);
\definecolor{c}{rgb}{0,0,0};
\colorlet{c}{kugray};
\draw [c] (4.36366,3.09108) -- (4.36366,3.10621);
\draw [c] (4.36366,3.10621) -- (4.36366,3.12063);
\draw [c] (4.34884,3.10621) -- (4.36366,3.10621);
\draw [c] (4.36366,3.10621) -- (4.37848,3.10621);
\definecolor{c}{rgb}{0,0,0};
\colorlet{c}{kugray};
\draw [c] (4.39329,3.08833) -- (4.39329,3.10339);
\draw [c] (4.39329,3.10339) -- (4.39329,3.11775);
\draw [c] (4.37848,3.10339) -- (4.39329,3.10339);
\draw [c] (4.39329,3.10339) -- (4.40811,3.10339);
\definecolor{c}{rgb}{0,0,0};
\colorlet{c}{kugray};
\draw [c] (4.42293,3.06637) -- (4.42293,3.08234);
\draw [c] (4.42293,3.08234) -- (4.42293,3.09751);
\draw [c] (4.40811,3.08234) -- (4.42293,3.08234);
\draw [c] (4.42293,3.08234) -- (4.43775,3.08234);
\definecolor{c}{rgb}{0,0,0};
\colorlet{c}{kugray};
\draw [c] (4.45257,3.03685) -- (4.45257,3.05337);
\draw [c] (4.45257,3.05337) -- (4.45257,3.06905);
\draw [c] (4.43775,3.05337) -- (4.45257,3.05337);
\draw [c] (4.45257,3.05337) -- (4.46738,3.05337);
\definecolor{c}{rgb}{0,0,0};
\colorlet{c}{kugray};
\draw [c] (4.4822,3.05403) -- (4.4822,3.07028);
\draw [c] (4.4822,3.07028) -- (4.4822,3.08572);
\draw [c] (4.46738,3.07028) -- (4.4822,3.07028);
\draw [c] (4.4822,3.07028) -- (4.49702,3.07028);
\definecolor{c}{rgb}{0,0,0};
\colorlet{c}{kugray};
\draw [c] (4.51184,3.02475) -- (4.51184,3.04166);
\draw [c] (4.51184,3.04166) -- (4.51184,3.05768);
\draw [c] (4.49702,3.04166) -- (4.51184,3.04166);
\draw [c] (4.51184,3.04166) -- (4.52666,3.04166);
\definecolor{c}{rgb}{0,0,0};
\colorlet{c}{kugray};
\draw [c] (4.54147,2.97429) -- (4.54147,2.99215);
\draw [c] (4.54147,2.99215) -- (4.54147,3.00903);
\draw [c] (4.52666,2.99215) -- (4.54147,2.99215);
\draw [c] (4.54147,2.99215) -- (4.55629,2.99215);
\definecolor{c}{rgb}{0,0,0};
\colorlet{c}{kugray};
\draw [c] (4.57111,2.97145) -- (4.57111,2.9901);
\draw [c] (4.57111,2.9901) -- (4.57111,3.0077);
\draw [c] (4.55629,2.9901) -- (4.57111,2.9901);
\draw [c] (4.57111,2.9901) -- (4.58593,2.9901);
\definecolor{c}{rgb}{0,0,0};
\colorlet{c}{kugray};
\draw [c] (4.60075,2.99637) -- (4.60075,3.01434);
\draw [c] (4.60075,3.01434) -- (4.60075,3.03131);
\draw [c] (4.58593,3.01434) -- (4.60075,3.01434);
\draw [c] (4.60075,3.01434) -- (4.61556,3.01434);
\definecolor{c}{rgb}{0,0,0};
\colorlet{c}{kugray};
\draw [c] (4.63038,2.94705) -- (4.63038,2.96575);
\draw [c] (4.63038,2.96575) -- (4.63038,2.98338);
\draw [c] (4.61556,2.96575) -- (4.63038,2.96575);
\draw [c] (4.63038,2.96575) -- (4.6452,2.96575);
\definecolor{c}{rgb}{0,0,0};
\colorlet{c}{kugray};
\draw [c] (4.66002,2.94213) -- (4.66002,2.96119);
\draw [c] (4.66002,2.96119) -- (4.66002,2.97915);
\draw [c] (4.6452,2.96119) -- (4.66002,2.96119);
\draw [c] (4.66002,2.96119) -- (4.67483,2.96119);
\definecolor{c}{rgb}{0,0,0};
\colorlet{c}{kugray};
\draw [c] (4.68965,2.92699) -- (4.68965,2.94611);
\draw [c] (4.68965,2.94611) -- (4.68965,2.96412);
\draw [c] (4.67483,2.94611) -- (4.68965,2.94611);
\draw [c] (4.68965,2.94611) -- (4.70447,2.94611);
\definecolor{c}{rgb}{0,0,0};
\colorlet{c}{kugray};
\draw [c] (4.71929,2.90533) -- (4.71929,2.92537);
\draw [c] (4.71929,2.92537) -- (4.71929,2.94418);
\draw [c] (4.70447,2.92537) -- (4.71929,2.92537);
\draw [c] (4.71929,2.92537) -- (4.73411,2.92537);
\definecolor{c}{rgb}{0,0,0};
\colorlet{c}{kugray};
\draw [c] (4.74892,2.89803) -- (4.74892,2.9188);
\draw [c] (4.74892,2.9188) -- (4.74892,2.93826);
\draw [c] (4.73411,2.9188) -- (4.74892,2.9188);
\draw [c] (4.74892,2.9188) -- (4.76374,2.9188);
\definecolor{c}{rgb}{0,0,0};
\colorlet{c}{kugray};
\draw [c] (4.77856,2.88491) -- (4.77856,2.90571);
\draw [c] (4.77856,2.90571) -- (4.77856,2.92519);
\draw [c] (4.76374,2.90571) -- (4.77856,2.90571);
\draw [c] (4.77856,2.90571) -- (4.79338,2.90571);
\definecolor{c}{rgb}{0,0,0};
\colorlet{c}{kugray};
\draw [c] (4.8082,2.83295) -- (4.8082,2.85717);
\draw [c] (4.8082,2.85717) -- (4.8082,2.87962);
\draw [c] (4.79338,2.85717) -- (4.8082,2.85717);
\draw [c] (4.8082,2.85717) -- (4.82301,2.85717);
\definecolor{c}{rgb}{0,0,0};
\colorlet{c}{kugray};
\draw [c] (4.83783,2.84099) -- (4.83783,2.86366);
\draw [c] (4.83783,2.86366) -- (4.83783,2.88478);
\draw [c] (4.82301,2.86366) -- (4.83783,2.86366);
\draw [c] (4.83783,2.86366) -- (4.85265,2.86366);
\definecolor{c}{rgb}{0,0,0};
\colorlet{c}{kugray};
\draw [c] (4.86747,2.84516) -- (4.86747,2.86807);
\draw [c] (4.86747,2.86807) -- (4.86747,2.88939);
\draw [c] (4.85265,2.86807) -- (4.86747,2.86807);
\draw [c] (4.86747,2.86807) -- (4.88228,2.86807);
\definecolor{c}{rgb}{0,0,0};
\colorlet{c}{kugray};
\draw [c] (4.8971,2.82427) -- (4.8971,2.84804);
\draw [c] (4.8971,2.84804) -- (4.8971,2.87011);
\draw [c] (4.88228,2.84804) -- (4.8971,2.84804);
\draw [c] (4.8971,2.84804) -- (4.91192,2.84804);
\definecolor{c}{rgb}{0,0,0};
\colorlet{c}{kugray};
\draw [c] (4.92674,2.7958) -- (4.92674,2.82013);
\draw [c] (4.92674,2.82013) -- (4.92674,2.84268);
\draw [c] (4.91192,2.82013) -- (4.92674,2.82013);
\draw [c] (4.92674,2.82013) -- (4.94156,2.82013);
\definecolor{c}{rgb}{0,0,0};
\colorlet{c}{kugray};
\draw [c] (4.95637,2.72162) -- (4.95637,2.74908);
\draw [c] (4.95637,2.74908) -- (4.95637,2.7743);
\draw [c] (4.94156,2.74908) -- (4.95637,2.74908);
\draw [c] (4.95637,2.74908) -- (4.97119,2.74908);
\definecolor{c}{rgb}{0,0,0};
\colorlet{c}{kugray};
\draw [c] (4.98601,2.74467) -- (4.98601,2.77317);
\draw [c] (4.98601,2.77317) -- (4.98601,2.79924);
\draw [c] (4.97119,2.77317) -- (4.98601,2.77317);
\draw [c] (4.98601,2.77317) -- (5.00083,2.77317);
\definecolor{c}{rgb}{0,0,0};
\colorlet{c}{kugray};
\draw [c] (5.01565,2.72133) -- (5.01565,2.74919);
\draw [c] (5.01565,2.74919) -- (5.01565,2.77473);
\draw [c] (5.00083,2.74919) -- (5.01565,2.74919);
\draw [c] (5.01565,2.74919) -- (5.03046,2.74919);
\definecolor{c}{rgb}{0,0,0};
\colorlet{c}{kugray};
\draw [c] (5.04528,2.68622) -- (5.04528,2.71565);
\draw [c] (5.04528,2.71565) -- (5.04528,2.74251);
\draw [c] (5.03046,2.71565) -- (5.04528,2.71565);
\draw [c] (5.04528,2.71565) -- (5.0601,2.71565);
\definecolor{c}{rgb}{0,0,0};
\colorlet{c}{kugray};
\draw [c] (5.07492,2.69413) -- (5.07492,2.72282);
\draw [c] (5.07492,2.72282) -- (5.07492,2.74907);
\draw [c] (5.0601,2.72282) -- (5.07492,2.72282);
\draw [c] (5.07492,2.72282) -- (5.08974,2.72282);
\definecolor{c}{rgb}{0,0,0};
\colorlet{c}{kugray};
\draw [c] (5.10455,2.68206) -- (5.10455,2.71119);
\draw [c] (5.10455,2.71119) -- (5.10455,2.73781);
\draw [c] (5.08974,2.71119) -- (5.10455,2.71119);
\draw [c] (5.10455,2.71119) -- (5.11937,2.71119);
\definecolor{c}{rgb}{0,0,0};
\colorlet{c}{kugray};
\draw [c] (5.13419,2.68307) -- (5.13419,2.7116);
\draw [c] (5.13419,2.7116) -- (5.13419,2.7377);
\draw [c] (5.11937,2.7116) -- (5.13419,2.7116);
\draw [c] (5.13419,2.7116) -- (5.14901,2.7116);
\definecolor{c}{rgb}{0,0,0};
\colorlet{c}{kugray};
\draw [c] (5.16382,2.72714) -- (5.16382,2.75436);
\draw [c] (5.16382,2.75436) -- (5.16382,2.77936);
\draw [c] (5.14901,2.75436) -- (5.16382,2.75436);
\draw [c] (5.16382,2.75436) -- (5.17864,2.75436);
\definecolor{c}{rgb}{0,0,0};
\colorlet{c}{kugray};
\draw [c] (5.19346,2.64861) -- (5.19346,2.67901);
\draw [c] (5.19346,2.67901) -- (5.19346,2.70667);
\draw [c] (5.17864,2.67901) -- (5.19346,2.67901);
\draw [c] (5.19346,2.67901) -- (5.20828,2.67901);
\definecolor{c}{rgb}{0,0,0};
\colorlet{c}{kugray};
\draw [c] (5.2231,2.688) -- (5.2231,2.71811);
\draw [c] (5.2231,2.71811) -- (5.2231,2.74554);
\draw [c] (5.20828,2.71811) -- (5.2231,2.71811);
\draw [c] (5.2231,2.71811) -- (5.23791,2.71811);
\definecolor{c}{rgb}{0,0,0};
\colorlet{c}{kugray};
\draw [c] (5.25273,2.56862) -- (5.25273,2.60483);
\draw [c] (5.25273,2.60483) -- (5.25273,2.63722);
\draw [c] (5.23791,2.60483) -- (5.25273,2.60483);
\draw [c] (5.25273,2.60483) -- (5.26755,2.60483);
\definecolor{c}{rgb}{0,0,0};
\colorlet{c}{kugray};
\draw [c] (5.28237,2.61362) -- (5.28237,2.64718);
\draw [c] (5.28237,2.64718) -- (5.28237,2.67745);
\draw [c] (5.26755,2.64718) -- (5.28237,2.64718);
\draw [c] (5.28237,2.64718) -- (5.29719,2.64718);
\definecolor{c}{rgb}{0,0,0};
\colorlet{c}{kugray};
\draw [c] (5.312,2.58142) -- (5.312,2.61649);
\draw [c] (5.312,2.61649) -- (5.312,2.64798);
\draw [c] (5.29719,2.61649) -- (5.312,2.61649);
\draw [c] (5.312,2.61649) -- (5.32682,2.61649);
\definecolor{c}{rgb}{0,0,0};
\colorlet{c}{kugray};
\draw [c] (5.34164,2.59936) -- (5.34164,2.63325);
\draw [c] (5.34164,2.63325) -- (5.34164,2.66377);
\draw [c] (5.32682,2.63325) -- (5.34164,2.63325);
\draw [c] (5.34164,2.63325) -- (5.35646,2.63325);
\definecolor{c}{rgb}{0,0,0};
\colorlet{c}{kugray};
\draw [c] (5.37127,2.59301) -- (5.37127,2.62735);
\draw [c] (5.37127,2.62735) -- (5.37127,2.65824);
\draw [c] (5.35646,2.62735) -- (5.37127,2.62735);
\draw [c] (5.37127,2.62735) -- (5.38609,2.62735);
\definecolor{c}{rgb}{0,0,0};
\colorlet{c}{kugray};
\draw [c] (5.40091,2.57979) -- (5.40091,2.61666);
\draw [c] (5.40091,2.61666) -- (5.40091,2.64958);
\draw [c] (5.38609,2.61666) -- (5.40091,2.61666);
\draw [c] (5.40091,2.61666) -- (5.41573,2.61666);
\definecolor{c}{rgb}{0,0,0};
\colorlet{c}{kugray};
\draw [c] (5.43055,2.5888) -- (5.43055,2.62522);
\draw [c] (5.43055,2.62522) -- (5.43055,2.65779);
\draw [c] (5.41573,2.62522) -- (5.43055,2.62522);
\draw [c] (5.43055,2.62522) -- (5.44536,2.62522);
\definecolor{c}{rgb}{0,0,0};
\colorlet{c}{kugray};
\draw [c] (5.46018,2.49892) -- (5.46018,2.53839);
\draw [c] (5.46018,2.53839) -- (5.46018,2.57337);
\draw [c] (5.44536,2.53839) -- (5.46018,2.53839);
\draw [c] (5.46018,2.53839) -- (5.475,2.53839);
\definecolor{c}{rgb}{0,0,0};
\colorlet{c}{kugray};
\draw [c] (5.48982,2.51137) -- (5.48982,2.5487);
\draw [c] (5.48982,2.5487) -- (5.48982,2.58199);
\draw [c] (5.475,2.5487) -- (5.48982,2.5487);
\draw [c] (5.48982,2.5487) -- (5.50464,2.5487);
\definecolor{c}{rgb}{0,0,0};
\colorlet{c}{kugray};
\draw [c] (5.51945,2.5345) -- (5.51945,2.57323);
\draw [c] (5.51945,2.57323) -- (5.51945,2.60762);
\draw [c] (5.50464,2.57323) -- (5.51945,2.57323);
\draw [c] (5.51945,2.57323) -- (5.53427,2.57323);
\definecolor{c}{rgb}{0,0,0};
\colorlet{c}{kugray};
\draw [c] (5.54909,2.49349) -- (5.54909,2.533);
\draw [c] (5.54909,2.533) -- (5.54909,2.56801);
\draw [c] (5.53427,2.533) -- (5.54909,2.533);
\draw [c] (5.54909,2.533) -- (5.56391,2.533);
\definecolor{c}{rgb}{0,0,0};
\colorlet{c}{kugray};
\draw [c] (5.57873,2.46256) -- (5.57873,2.50333);
\draw [c] (5.57873,2.50333) -- (5.57873,2.53932);
\draw [c] (5.56391,2.50333) -- (5.57873,2.50333);
\draw [c] (5.57873,2.50333) -- (5.59354,2.50333);
\definecolor{c}{rgb}{0,0,0};
\colorlet{c}{kugray};
\draw [c] (5.60836,2.50823) -- (5.60836,2.54684);
\draw [c] (5.60836,2.54684) -- (5.60836,2.58115);
\draw [c] (5.59354,2.54684) -- (5.60836,2.54684);
\draw [c] (5.60836,2.54684) -- (5.62318,2.54684);
\definecolor{c}{rgb}{0,0,0};
\colorlet{c}{kugray};
\draw [c] (5.638,2.50569) -- (5.638,2.54708);
\draw [c] (5.638,2.54708) -- (5.638,2.58356);
\draw [c] (5.62318,2.54708) -- (5.638,2.54708);
\draw [c] (5.638,2.54708) -- (5.65281,2.54708);
\definecolor{c}{rgb}{0,0,0};
\colorlet{c}{kugray};
\draw [c] (5.66763,2.48101) -- (5.66763,2.5229);
\draw [c] (5.66763,2.5229) -- (5.66763,2.55976);
\draw [c] (5.65281,2.5229) -- (5.66763,2.5229);
\draw [c] (5.66763,2.5229) -- (5.68245,2.5229);
\definecolor{c}{rgb}{0,0,0};
\colorlet{c}{kugray};
\draw [c] (5.69727,2.3796) -- (5.69727,2.42684);
\draw [c] (5.69727,2.42684) -- (5.69727,2.46778);
\draw [c] (5.68245,2.42684) -- (5.69727,2.42684);
\draw [c] (5.69727,2.42684) -- (5.71209,2.42684);
\definecolor{c}{rgb}{0,0,0};
\colorlet{c}{kugray};
\draw [c] (5.7269,2.30877) -- (5.7269,2.35869);
\draw [c] (5.7269,2.35869) -- (5.7269,2.40163);
\draw [c] (5.71209,2.35869) -- (5.7269,2.35869);
\draw [c] (5.7269,2.35869) -- (5.74172,2.35869);
\definecolor{c}{rgb}{0,0,0};
\colorlet{c}{kugray};
\draw [c] (5.75654,2.43196) -- (5.75654,2.47986);
\draw [c] (5.75654,2.47986) -- (5.75654,2.5213);
\draw [c] (5.74172,2.47986) -- (5.75654,2.47986);
\draw [c] (5.75654,2.47986) -- (5.77136,2.47986);
\definecolor{c}{rgb}{0,0,0};
\colorlet{c}{kugray};
\draw [c] (5.78618,2.43169) -- (5.78618,2.47526);
\draw [c] (5.78618,2.47526) -- (5.78618,2.51342);
\draw [c] (5.77136,2.47526) -- (5.78618,2.47526);
\draw [c] (5.78618,2.47526) -- (5.80099,2.47526);
\definecolor{c}{rgb}{0,0,0};
\colorlet{c}{kugray};
\draw [c] (5.81581,2.39907) -- (5.81581,2.44526);
\draw [c] (5.81581,2.44526) -- (5.81581,2.48541);
\draw [c] (5.80099,2.44526) -- (5.81581,2.44526);
\draw [c] (5.81581,2.44526) -- (5.83063,2.44526);
\definecolor{c}{rgb}{0,0,0};
\colorlet{c}{kugray};
\draw [c] (5.84545,2.43121) -- (5.84545,2.47744);
\draw [c] (5.84545,2.47744) -- (5.84545,2.51762);
\draw [c] (5.83063,2.47744) -- (5.84545,2.47744);
\draw [c] (5.84545,2.47744) -- (5.86026,2.47744);
\definecolor{c}{rgb}{0,0,0};
\colorlet{c}{kugray};
\draw [c] (5.87508,2.29827) -- (5.87508,2.35564);
\draw [c] (5.87508,2.35564) -- (5.87508,2.40397);
\draw [c] (5.86026,2.35564) -- (5.87508,2.35564);
\draw [c] (5.87508,2.35564) -- (5.8899,2.35564);
\definecolor{c}{rgb}{0,0,0};
\colorlet{c}{kugray};
\draw [c] (5.90472,2.35345) -- (5.90472,2.40527);
\draw [c] (5.90472,2.40527) -- (5.90472,2.44961);
\draw [c] (5.8899,2.40527) -- (5.90472,2.40527);
\draw [c] (5.90472,2.40527) -- (5.91954,2.40527);
\definecolor{c}{rgb}{0,0,0};
\colorlet{c}{kugray};
\draw [c] (5.93435,2.28972) -- (5.93435,2.34179);
\draw [c] (5.93435,2.34179) -- (5.93435,2.38632);
\draw [c] (5.91954,2.34179) -- (5.93435,2.34179);
\draw [c] (5.93435,2.34179) -- (5.94917,2.34179);
\definecolor{c}{rgb}{0,0,0};
\colorlet{c}{kugray};
\draw [c] (5.96399,2.28856) -- (5.96399,2.34875);
\draw [c] (5.96399,2.34875) -- (5.96399,2.39907);
\draw [c] (5.94917,2.34875) -- (5.96399,2.34875);
\draw [c] (5.96399,2.34875) -- (5.97881,2.34875);
\definecolor{c}{rgb}{0,0,0};
\colorlet{c}{kugray};
\draw [c] (5.99363,2.34369) -- (5.99363,2.39484);
\draw [c] (5.99363,2.39484) -- (5.99363,2.43869);
\draw [c] (5.97881,2.39484) -- (5.99363,2.39484);
\draw [c] (5.99363,2.39484) -- (6.00844,2.39484);
\definecolor{c}{rgb}{0,0,0};
\colorlet{c}{kugray};
\draw [c] (6.02326,2.22004) -- (6.02326,2.27835);
\draw [c] (6.02326,2.27835) -- (6.02326,2.32735);
\draw [c] (6.00844,2.27835) -- (6.02326,2.27835);
\draw [c] (6.02326,2.27835) -- (6.03808,2.27835);
\definecolor{c}{rgb}{0,0,0};
\colorlet{c}{kugray};
\draw [c] (6.0529,2.24812) -- (6.0529,2.30663);
\draw [c] (6.0529,2.30663) -- (6.0529,2.35577);
\draw [c] (6.03808,2.30663) -- (6.0529,2.30663);
\draw [c] (6.0529,2.30663) -- (6.06772,2.30663);
\definecolor{c}{rgb}{0,0,0};
\colorlet{c}{kugray};
\draw [c] (6.08253,2.21392) -- (6.08253,2.27979);
\draw [c] (6.08253,2.27979) -- (6.08253,2.33402);
\draw [c] (6.06772,2.27979) -- (6.08253,2.27979);
\draw [c] (6.08253,2.27979) -- (6.09735,2.27979);
\definecolor{c}{rgb}{0,0,0};
\colorlet{c}{kugray};
\draw [c] (6.11217,2.28422) -- (6.11217,2.34425);
\draw [c] (6.11217,2.34425) -- (6.11217,2.39446);
\draw [c] (6.09735,2.34425) -- (6.11217,2.34425);
\draw [c] (6.11217,2.34425) -- (6.12699,2.34425);
\definecolor{c}{rgb}{0,0,0};
\colorlet{c}{kugray};
\draw [c] (6.1418,2.13526) -- (6.1418,2.20119);
\draw [c] (6.1418,2.20119) -- (6.1418,2.25547);
\draw [c] (6.12699,2.20119) -- (6.1418,2.20119);
\draw [c] (6.1418,2.20119) -- (6.15662,2.20119);
\definecolor{c}{rgb}{0,0,0};
\colorlet{c}{kugray};
\draw [c] (6.17144,2.21632) -- (6.17144,2.28052);
\draw [c] (6.17144,2.28052) -- (6.17144,2.33362);
\draw [c] (6.15662,2.28052) -- (6.17144,2.28052);
\draw [c] (6.17144,2.28052) -- (6.18626,2.28052);
\definecolor{c}{rgb}{0,0,0};
\colorlet{c}{kugray};
\draw [c] (6.20108,2.25826) -- (6.20108,2.32268);
\draw [c] (6.20108,2.32268) -- (6.20108,2.37592);
\draw [c] (6.18626,2.32268) -- (6.20108,2.32268);
\draw [c] (6.20108,2.32268) -- (6.21589,2.32268);
\definecolor{c}{rgb}{0,0,0};
\colorlet{c}{kugray};
\draw [c] (6.23071,2.21696) -- (6.23071,2.28153);
\draw [c] (6.23071,2.28153) -- (6.23071,2.33487);
\draw [c] (6.21589,2.28153) -- (6.23071,2.28153);
\draw [c] (6.23071,2.28153) -- (6.24553,2.28153);
\definecolor{c}{rgb}{0,0,0};
\colorlet{c}{kugray};
\draw [c] (6.26035,2.11717) -- (6.26035,2.18912);
\draw [c] (6.26035,2.18912) -- (6.26035,2.2474);
\draw [c] (6.24553,2.18912) -- (6.26035,2.18912);
\draw [c] (6.26035,2.18912) -- (6.27517,2.18912);
\definecolor{c}{rgb}{0,0,0};
\colorlet{c}{kugray};
\draw [c] (6.28998,2.21915) -- (6.28998,2.27945);
\draw [c] (6.28998,2.27945) -- (6.28998,2.32985);
\draw [c] (6.27517,2.27945) -- (6.28998,2.27945);
\draw [c] (6.28998,2.27945) -- (6.3048,2.27945);
\definecolor{c}{rgb}{0,0,0};
\colorlet{c}{kugray};
\draw [c] (6.31962,2.15067) -- (6.31962,2.22057);
\draw [c] (6.31962,2.22057) -- (6.31962,2.2775);
\draw [c] (6.3048,2.22057) -- (6.31962,2.22057);
\draw [c] (6.31962,2.22057) -- (6.33444,2.22057);
\definecolor{c}{rgb}{0,0,0};
\colorlet{c}{kugray};
\draw [c] (6.34926,2.08129) -- (6.34926,2.15905);
\draw [c] (6.34926,2.15905) -- (6.34926,2.22107);
\draw [c] (6.33444,2.15905) -- (6.34926,2.15905);
\draw [c] (6.34926,2.15905) -- (6.36407,2.15905);
\definecolor{c}{rgb}{0,0,0};
\colorlet{c}{kugray};
\draw [c] (6.37889,2.26481) -- (6.37889,2.32632);
\draw [c] (6.37889,2.32632) -- (6.37889,2.37756);
\draw [c] (6.36407,2.32632) -- (6.37889,2.32632);
\draw [c] (6.37889,2.32632) -- (6.39371,2.32632);
\definecolor{c}{rgb}{0,0,0};
\colorlet{c}{kugray};
\draw [c] (6.40853,2.10103) -- (6.40853,2.17287);
\draw [c] (6.40853,2.17287) -- (6.40853,2.23108);
\draw [c] (6.39371,2.17287) -- (6.40853,2.17287);
\draw [c] (6.40853,2.17287) -- (6.42334,2.17287);
\definecolor{c}{rgb}{0,0,0};
\colorlet{c}{kugray};
\draw [c] (6.43816,2.01805) -- (6.43816,2.10352);
\draw [c] (6.43816,2.10352) -- (6.43816,2.17034);
\draw [c] (6.42334,2.10352) -- (6.43816,2.10352);
\draw [c] (6.43816,2.10352) -- (6.45298,2.10352);
\definecolor{c}{rgb}{0,0,0};
\colorlet{c}{kugray};
\draw [c] (6.4678,2.13798) -- (6.4678,2.21314);
\draw [c] (6.4678,2.21314) -- (6.4678,2.27351);
\draw [c] (6.45298,2.21314) -- (6.4678,2.21314);
\draw [c] (6.4678,2.21314) -- (6.48262,2.21314);
\definecolor{c}{rgb}{0,0,0};
\colorlet{c}{kugray};
\draw [c] (6.49743,2.0628) -- (6.49743,2.13789);
\draw [c] (6.49743,2.13789) -- (6.49743,2.19821);
\draw [c] (6.48262,2.13789) -- (6.49743,2.13789);
\draw [c] (6.49743,2.13789) -- (6.51225,2.13789);
\definecolor{c}{rgb}{0,0,0};
\colorlet{c}{kugray};
\draw [c] (6.52707,2.03036) -- (6.52707,2.11014);
\draw [c] (6.52707,2.11014) -- (6.52707,2.17345);
\draw [c] (6.51225,2.11014) -- (6.52707,2.11014);
\draw [c] (6.52707,2.11014) -- (6.54189,2.11014);
\definecolor{c}{rgb}{0,0,0};
\colorlet{c}{kugray};
\draw [c] (6.55671,1.95941) -- (6.55671,2.05375);
\draw [c] (6.55671,2.05375) -- (6.55671,2.12586);
\draw [c] (6.54189,2.05375) -- (6.55671,2.05375);
\draw [c] (6.55671,2.05375) -- (6.57152,2.05375);
\definecolor{c}{rgb}{0,0,0};
\colorlet{c}{kugray};
\draw [c] (6.58634,2.0092) -- (6.58634,2.09944);
\draw [c] (6.58634,2.09944) -- (6.58634,2.16914);
\draw [c] (6.57152,2.09944) -- (6.58634,2.09944);
\draw [c] (6.58634,2.09944) -- (6.60116,2.09944);
\definecolor{c}{rgb}{0,0,0};
\colorlet{c}{kugray};
\draw [c] (6.61598,2.06823) -- (6.61598,2.14649);
\draw [c] (6.61598,2.14649) -- (6.61598,2.20883);
\draw [c] (6.60116,2.14649) -- (6.61598,2.14649);
\draw [c] (6.61598,2.14649) -- (6.63079,2.14649);
\definecolor{c}{rgb}{0,0,0};
\colorlet{c}{kugray};
\draw [c] (6.64561,1.91105) -- (6.64561,2.01133);
\draw [c] (6.64561,2.01133) -- (6.64561,2.08685);
\draw [c] (6.63079,2.01133) -- (6.64561,2.01133);
\draw [c] (6.64561,2.01133) -- (6.66043,2.01133);
\definecolor{c}{rgb}{0,0,0};
\colorlet{c}{kugray};
\draw [c] (6.67525,1.90644) -- (6.67525,2.01096);
\draw [c] (6.67525,2.01096) -- (6.67525,2.08884);
\draw [c] (6.66043,2.01096) -- (6.67525,2.01096);
\draw [c] (6.67525,2.01096) -- (6.69007,2.01096);
\definecolor{c}{rgb}{0,0,0};
\colorlet{c}{kugray};
\draw [c] (6.70488,1.98499) -- (6.70488,2.07197);
\draw [c] (6.70488,2.07197) -- (6.70488,2.13971);
\draw [c] (6.69007,2.07197) -- (6.70488,2.07197);
\draw [c] (6.70488,2.07197) -- (6.7197,2.07197);
\definecolor{c}{rgb}{0,0,0};
\colorlet{c}{kugray};
\draw [c] (6.73452,1.98323) -- (6.73452,2.07762);
\draw [c] (6.73452,2.07762) -- (6.73452,2.14976);
\draw [c] (6.7197,2.07762) -- (6.73452,2.07762);
\draw [c] (6.73452,2.07762) -- (6.74934,2.07762);
\definecolor{c}{rgb}{0,0,0};
\colorlet{c}{kugray};
\draw [c] (6.76416,1.8404) -- (6.76416,1.94614);
\draw [c] (6.76416,1.94614) -- (6.76416,2.0247);
\draw [c] (6.74934,1.94614) -- (6.76416,1.94614);
\draw [c] (6.76416,1.94614) -- (6.77897,1.94614);
\definecolor{c}{rgb}{0,0,0};
\colorlet{c}{kugray};
\draw [c] (6.79379,1.95326) -- (6.79379,2.04637);
\draw [c] (6.79379,2.04637) -- (6.79379,2.11775);
\draw [c] (6.77897,2.04637) -- (6.79379,2.04637);
\draw [c] (6.79379,2.04637) -- (6.80861,2.04637);
\definecolor{c}{rgb}{0,0,0};
\colorlet{c}{kugray};
\draw [c] (6.82343,1.84998) -- (6.82343,1.96142);
\draw [c] (6.82343,1.96142) -- (6.82343,2.04307);
\draw [c] (6.80861,1.96142) -- (6.82343,1.96142);
\draw [c] (6.82343,1.96142) -- (6.83824,1.96142);
\definecolor{c}{rgb}{0,0,0};
\colorlet{c}{kugray};
\draw [c] (6.85306,1.97353) -- (6.85306,2.06824);
\draw [c] (6.85306,2.06824) -- (6.85306,2.14058);
\draw [c] (6.83824,2.06824) -- (6.85306,2.06824);
\draw [c] (6.85306,2.06824) -- (6.86788,2.06824);
\definecolor{c}{rgb}{0,0,0};
\colorlet{c}{kugray};
\draw [c] (6.8827,1.94959) -- (6.8827,2.05345);
\draw [c] (6.8827,2.05345) -- (6.8827,2.13097);
\draw [c] (6.86788,2.05345) -- (6.8827,2.05345);
\draw [c] (6.8827,2.05345) -- (6.89752,2.05345);
\definecolor{c}{rgb}{0,0,0};
\colorlet{c}{kugray};
\draw [c] (6.91233,1.64738) -- (6.91233,1.796);
\draw [c] (6.91233,1.796) -- (6.91233,1.8958);
\draw [c] (6.89752,1.796) -- (6.91233,1.796);
\draw [c] (6.91233,1.796) -- (6.92715,1.796);
\definecolor{c}{rgb}{0,0,0};
\colorlet{c}{kugray};
\draw [c] (6.94197,1.81861) -- (6.94197,1.94633);
\draw [c] (6.94197,1.94633) -- (6.94197,2.03632);
\draw [c] (6.92715,1.94633) -- (6.94197,1.94633);
\draw [c] (6.94197,1.94633) -- (6.95679,1.94633);
\definecolor{c}{rgb}{0,0,0};
\colorlet{c}{kugray};
\draw [c] (6.97161,1.45467) -- (6.97161,1.67503);
\draw [c] (6.97161,1.67503) -- (6.97161,1.80215);
\draw [c] (6.95679,1.67503) -- (6.97161,1.67503);
\draw [c] (6.97161,1.67503) -- (6.98642,1.67503);
\definecolor{c}{rgb}{0,0,0};
\colorlet{c}{kugray};
\draw [c] (7.00124,1.92607) -- (7.00124,2.03252);
\draw [c] (7.00124,2.03252) -- (7.00124,2.11147);
\draw [c] (6.98642,2.03252) -- (7.00124,2.03252);
\draw [c] (7.00124,2.03252) -- (7.01606,2.03252);
\definecolor{c}{rgb}{0,0,0};
\colorlet{c}{kugray};
\draw [c] (7.03088,1.70142) -- (7.03088,1.83023);
\draw [c] (7.03088,1.83023) -- (7.03088,1.92076);
\draw [c] (7.01606,1.83023) -- (7.03088,1.83023);
\draw [c] (7.03088,1.83023) -- (7.0457,1.83023);
\definecolor{c}{rgb}{0,0,0};
\colorlet{c}{kugray};
\draw [c] (7.06051,1.69915) -- (7.06051,1.84157);
\draw [c] (7.06051,1.84157) -- (7.06051,1.93856);
\draw [c] (7.0457,1.84157) -- (7.06051,1.84157);
\draw [c] (7.06051,1.84157) -- (7.07533,1.84157);
\definecolor{c}{rgb}{0,0,0};
\colorlet{c}{kugray};
\draw [c] (7.09015,1.30062) -- (7.09015,1.58241);
\draw [c] (7.09015,1.58241) -- (7.09015,1.72703);
\draw [c] (7.07533,1.58241) -- (7.09015,1.58241);
\draw [c] (7.09015,1.58241) -- (7.10497,1.58241);
\definecolor{c}{rgb}{0,0,0};
\colorlet{c}{kugray};
\draw [c] (7.11978,1.6185) -- (7.11978,1.77873);
\draw [c] (7.11978,1.77873) -- (7.11978,1.88358);
\draw [c] (7.10497,1.77873) -- (7.11978,1.77873);
\draw [c] (7.11978,1.77873) -- (7.1346,1.77873);
\definecolor{c}{rgb}{0,0,0};
\colorlet{c}{kugray};
\draw [c] (7.14942,1.83806) -- (7.14942,1.95138);
\draw [c] (7.14942,1.95138) -- (7.14942,2.03403);
\draw [c] (7.1346,1.95138) -- (7.14942,1.95138);
\draw [c] (7.14942,1.95138) -- (7.16424,1.95138);
\definecolor{c}{rgb}{0,0,0};
\colorlet{c}{kugray};
\draw [c] (7.17906,1.68124) -- (7.17906,1.87494);
\draw [c] (7.17906,1.87494) -- (7.17906,1.99292);
\draw [c] (7.16424,1.87494) -- (7.17906,1.87494);
\draw [c] (7.17906,1.87494) -- (7.19387,1.87494);
\definecolor{c}{rgb}{0,0,0};
\colorlet{c}{kugray};
\draw [c] (7.20869,1.63815) -- (7.20869,1.80529);
\draw [c] (7.20869,1.80529) -- (7.20869,1.91302);
\draw [c] (7.19387,1.80529) -- (7.20869,1.80529);
\draw [c] (7.20869,1.80529) -- (7.22351,1.80529);
\definecolor{c}{rgb}{0,0,0};
\colorlet{c}{kugray};
\draw [c] (7.23833,1.534) -- (7.23833,1.72717);
\draw [c] (7.23833,1.72717) -- (7.23833,1.84495);
\draw [c] (7.22351,1.72717) -- (7.23833,1.72717);
\draw [c] (7.23833,1.72717) -- (7.25315,1.72717);
\definecolor{c}{rgb}{0,0,0};
\colorlet{c}{kugray};
\draw [c] (7.26796,1.44669) -- (7.26796,1.66186);
\draw [c] (7.26796,1.66186) -- (7.26796,1.78728);
\draw [c] (7.25315,1.66186) -- (7.26796,1.66186);
\draw [c] (7.26796,1.66186) -- (7.28278,1.66186);
\definecolor{c}{rgb}{0,0,0};
\colorlet{c}{kugray};
\draw [c] (7.2976,1.67008) -- (7.2976,1.80777);
\draw [c] (7.2976,1.80777) -- (7.2976,1.90257);
\draw [c] (7.28278,1.80777) -- (7.2976,1.80777);
\draw [c] (7.2976,1.80777) -- (7.31242,1.80777);
\definecolor{c}{rgb}{0,0,0};
\colorlet{c}{kugray};
\draw [c] (7.32724,1.55019) -- (7.32724,1.78423);
\draw [c] (7.32724,1.78423) -- (7.32724,1.91564);
\draw [c] (7.31242,1.78423) -- (7.32724,1.78423);
\draw [c] (7.32724,1.78423) -- (7.34205,1.78423);
\definecolor{c}{rgb}{0,0,0};
\colorlet{c}{kugray};
\draw [c] (7.35687,1.70986) -- (7.35687,1.88005);
\draw [c] (7.35687,1.88005) -- (7.35687,1.98902);
\draw [c] (7.34205,1.88005) -- (7.35687,1.88005);
\draw [c] (7.35687,1.88005) -- (7.37169,1.88005);
\definecolor{c}{rgb}{0,0,0};
\colorlet{c}{kugray};
\draw [c] (7.38651,1.39569) -- (7.38651,1.61683);
\draw [c] (7.38651,1.61683) -- (7.38651,1.7442);
\draw [c] (7.37169,1.61683) -- (7.38651,1.61683);
\draw [c] (7.38651,1.61683) -- (7.40132,1.61683);
\definecolor{c}{rgb}{0,0,0};
\colorlet{c}{kugray};
\draw [c] (7.41614,1.65215) -- (7.41614,1.82337);
\draw [c] (7.41614,1.82337) -- (7.41614,1.93274);
\draw [c] (7.40132,1.82337) -- (7.41614,1.82337);
\draw [c] (7.41614,1.82337) -- (7.43096,1.82337);
\definecolor{c}{rgb}{0,0,0};
\colorlet{c}{kugray};
\draw [c] (7.44578,1.61682) -- (7.44578,1.76694);
\draw [c] (7.44578,1.76694) -- (7.44578,1.86741);
\draw [c] (7.43096,1.76694) -- (7.44578,1.76694);
\draw [c] (7.44578,1.76694) -- (7.4606,1.76694);
\definecolor{c}{rgb}{0,0,0};
\colorlet{c}{kugray};
\draw [c] (7.47541,0.903192) -- (7.47541,1.30472);
\draw [c] (7.47541,1.30472) -- (7.47541,1.47322);
\draw [c] (7.4606,1.30472) -- (7.47541,1.30472);
\draw [c] (7.47541,1.30472) -- (7.49023,1.30472);
\definecolor{c}{rgb}{0,0,0};
\colorlet{c}{kugray};
\draw [c] (7.50505,1.17257) -- (7.50505,1.55106);
\draw [c] (7.50505,1.55106) -- (7.50505,1.71578);
\draw [c] (7.49023,1.55106) -- (7.50505,1.55106);
\draw [c] (7.50505,1.55106) -- (7.51987,1.55106);
\definecolor{c}{rgb}{0,0,0};
\colorlet{c}{kugray};
\draw [c] (7.53469,1.26039) -- (7.53469,1.5276);
\draw [c] (7.53469,1.5276) -- (7.53469,1.66846);
\draw [c] (7.51987,1.5276) -- (7.53469,1.5276);
\draw [c] (7.53469,1.5276) -- (7.5495,1.5276);
\definecolor{c}{rgb}{0,0,0};
\colorlet{c}{kugray};
\draw [c] (7.56432,1.35089) -- (7.56432,1.65389);
\draw [c] (7.56432,1.65389) -- (7.56432,1.80359);
\draw [c] (7.5495,1.65389) -- (7.56432,1.65389);
\draw [c] (7.56432,1.65389) -- (7.57914,1.65389);
\definecolor{c}{rgb}{0,0,0};
\colorlet{c}{kugray};
\draw [c] (7.59396,1.63685) -- (7.59396,1.81433);
\draw [c] (7.59396,1.81433) -- (7.59396,1.92619);
\draw [c] (7.57914,1.81433) -- (7.59396,1.81433);
\draw [c] (7.59396,1.81433) -- (7.60877,1.81433);
\definecolor{c}{rgb}{0,0,0};
\colorlet{c}{kugray};
\draw [c] (7.62359,1.2207) -- (7.62359,1.48705);
\draw [c] (7.62359,1.48705) -- (7.62359,1.62768);
\draw [c] (7.60877,1.48705) -- (7.62359,1.48705);
\draw [c] (7.62359,1.48705) -- (7.63841,1.48705);
\definecolor{c}{rgb}{0,0,0};
\colorlet{c}{kugray};
\draw [c] (7.65323,1.15339) -- (7.65323,1.44836);
\draw [c] (7.65323,1.44836) -- (7.65323,1.5962);
\draw [c] (7.63841,1.44836) -- (7.65323,1.44836);
\draw [c] (7.65323,1.44836) -- (7.66805,1.44836);
\definecolor{c}{rgb}{0,0,0};
\colorlet{c}{kugray};
\draw [c] (7.68286,0.824113) -- (7.68286,1.32118);
\draw [c] (7.68286,1.32118) -- (7.68286,1.50231);
\draw [c] (7.66805,1.32118) -- (7.68286,1.32118);
\draw [c] (7.68286,1.32118) -- (7.69768,1.32118);
\definecolor{c}{rgb}{0,0,0};
\colorlet{c}{kugray};
\draw [c] (7.7125,1.26126) -- (7.7125,1.5339);
\draw [c] (7.7125,1.5339) -- (7.7125,1.67619);
\draw [c] (7.69768,1.5339) -- (7.7125,1.5339);
\draw [c] (7.7125,1.5339) -- (7.72732,1.5339);
\definecolor{c}{rgb}{0,0,0};
\colorlet{c}{kugray};
\draw [c] (7.74214,1.48045) -- (7.74214,1.72479);
\draw [c] (7.74214,1.72479) -- (7.74214,1.85928);
\draw [c] (7.72732,1.72479) -- (7.74214,1.72479);
\draw [c] (7.74214,1.72479) -- (7.75695,1.72479);
\definecolor{c}{rgb}{0,0,0};
\colorlet{c}{kugray};
\draw [c] (7.77177,1.11205) -- (7.77177,1.40937);
\draw [c] (7.77177,1.40937) -- (7.77177,1.55776);
\draw [c] (7.75695,1.40937) -- (7.77177,1.40937);
\draw [c] (7.77177,1.40937) -- (7.78659,1.40937);
\definecolor{c}{rgb}{0,0,0};
\colorlet{c}{kugray};
\draw [c] (7.80141,1.34931) -- (7.80141,1.63981);
\draw [c] (7.80141,1.63981) -- (7.80141,1.78657);
\draw [c] (7.78659,1.63981) -- (7.80141,1.63981);
\draw [c] (7.80141,1.63981) -- (7.81623,1.63981);
\definecolor{c}{rgb}{0,0,0};
\colorlet{c}{kugray};
\draw [c] (7.83104,1.0292) -- (7.83104,1.41463);
\draw [c] (7.83104,1.41463) -- (7.83104,1.58052);
\draw [c] (7.81623,1.41463) -- (7.83104,1.41463);
\draw [c] (7.83104,1.41463) -- (7.84586,1.41463);
\definecolor{c}{rgb}{0,0,0};
\colorlet{c}{kugray};
\draw [c] (7.86068,0.596817) -- (7.86068,1.16974);
\draw [c] (7.86068,1.16974) -- (7.86068,1.38313);
\draw [c] (7.84586,1.16974) -- (7.86068,1.16974);
\draw [c] (7.86068,1.16974) -- (7.8755,1.16974);
\definecolor{c}{rgb}{0,0,0};
\colorlet{c}{kugray};
\draw [c] (7.89031,1.22506) -- (7.89031,1.49977);
\draw [c] (7.89031,1.49977) -- (7.89031,1.64259);
\draw [c] (7.8755,1.49977) -- (7.89031,1.49977);
\draw [c] (7.89031,1.49977) -- (7.90513,1.49977);
\definecolor{c}{rgb}{0,0,0};
\colorlet{c}{kugray};
\draw [c] (7.91995,1.37554) -- (7.91995,1.58913);
\draw [c] (7.91995,1.58913) -- (7.91995,1.71402);
\draw [c] (7.90513,1.58913) -- (7.91995,1.58913);
\draw [c] (7.91995,1.58913) -- (7.93477,1.58913);
\definecolor{c}{rgb}{0,0,0};
\colorlet{c}{kugray};
\draw [c] (7.94959,0.970556) -- (7.94959,1.35289);
\draw [c] (7.94959,1.35289) -- (7.94959,1.51826);
\draw [c] (7.93477,1.35289) -- (7.94959,1.35289);
\draw [c] (7.94959,1.35289) -- (7.9644,1.35289);
\definecolor{c}{rgb}{0,0,0};
\colorlet{c}{kugray};
\draw [c] (7.97922,1.42663) -- (7.97922,1.62932);
\draw [c] (7.97922,1.62932) -- (7.97922,1.75049);
\draw [c] (7.9644,1.62932) -- (7.97922,1.62932);
\draw [c] (7.97922,1.62932) -- (7.99404,1.62932);
\definecolor{c}{rgb}{0,0,0};
\colorlet{c}{kugray};
\draw [c] (8.00886,1.24387) -- (8.00886,1.51218);
\draw [c] (8.00886,1.51218) -- (8.00886,1.65334);
\draw [c] (7.99404,1.51218) -- (8.00886,1.51218);
\draw [c] (8.00886,1.51218) -- (8.02368,1.51218);
\definecolor{c}{rgb}{0,0,0};
\colorlet{c}{kugray};
\draw [c] (8.03849,1.45206) -- (8.03849,1.64214);
\draw [c] (8.03849,1.64214) -- (8.03849,1.75879);
\draw [c] (8.02368,1.64214) -- (8.03849,1.64214);
\draw [c] (8.03849,1.64214) -- (8.05331,1.64214);
\definecolor{c}{rgb}{0,0,0};
\colorlet{c}{kugray};
\draw [c] (8.09776,1.35157) -- (8.09776,1.5665);
\draw [c] (8.09776,1.5665) -- (8.09776,1.69184);
\draw [c] (8.08295,1.5665) -- (8.09776,1.5665);
\draw [c] (8.09776,1.5665) -- (8.11258,1.5665);
\definecolor{c}{rgb}{0,0,0};
\colorlet{c}{kugray};
\draw [c] (8.1274,1.02943) -- (8.1274,1.41864);
\draw [c] (8.1274,1.41864) -- (8.1274,1.58516);
\draw [c] (8.11258,1.41864) -- (8.1274,1.41864);
\draw [c] (8.1274,1.41864) -- (8.14222,1.41864);
\definecolor{c}{rgb}{0,0,0};
\colorlet{c}{kugray};
\draw [c] (8.18667,1.3217) -- (8.18667,1.60896);
\draw [c] (8.18667,1.60896) -- (8.18667,1.75493);
\draw [c] (8.17185,1.60896) -- (8.18667,1.60896);
\draw [c] (8.18667,1.60896) -- (8.20149,1.60896);
\definecolor{c}{rgb}{0,0,0};
\colorlet{c}{kugray};
\draw [c] (8.21631,1.44991) -- (8.21631,1.72488);
\draw [c] (8.21631,1.72488) -- (8.21631,1.86777);
\draw [c] (8.20149,1.72488) -- (8.21631,1.72488);
\draw [c] (8.21631,1.72488) -- (8.23113,1.72488);
\definecolor{c}{rgb}{0,0,0};
\colorlet{c}{kugray};
\draw [c] (8.24594,0.596817) -- (8.24594,1.18479);
\draw [c] (8.24594,1.18479) -- (8.24594,1.39818);
\draw [c] (8.23113,1.18479) -- (8.24594,1.18479);
\draw [c] (8.24594,1.18479) -- (8.26076,1.18479);
\definecolor{c}{rgb}{0,0,0};
\colorlet{c}{kugray};
\draw [c] (8.27558,1.33422) -- (8.27558,1.70646);
\draw [c] (8.27558,1.70646) -- (8.27558,1.8701);
\draw [c] (8.26076,1.70646) -- (8.27558,1.70646);
\draw [c] (8.27558,1.70646) -- (8.2904,1.70646);
\definecolor{c}{rgb}{0,0,0};
\colorlet{c}{kugray};
\draw [c] (8.30521,1.01154) -- (8.30521,1.39412);
\draw [c] (8.30521,1.39412) -- (8.30521,1.55954);
\draw [c] (8.2904,1.39412) -- (8.30521,1.39412);
\draw [c] (8.30521,1.39412) -- (8.32003,1.39412);
\definecolor{c}{rgb}{0,0,0};
\colorlet{c}{kugray};
\draw [c] (8.33485,1.16538) -- (8.33485,1.451);
\draw [c] (8.33485,1.451) -- (8.33485,1.59657);
\draw [c] (8.32003,1.451) -- (8.33485,1.451);
\draw [c] (8.33485,1.451) -- (8.34967,1.451);
\definecolor{c}{rgb}{0,0,0};
\colorlet{c}{kugray};
\draw [c] (8.36449,0.596817) -- (8.36449,1.27192);
\draw [c] (8.36449,1.27192) -- (8.36449,1.48532);
\draw [c] (8.34967,1.27192) -- (8.36449,1.27192);
\draw [c] (8.36449,1.27192) -- (8.3793,1.27192);
\definecolor{c}{rgb}{0,0,0};
\colorlet{c}{kugray};
\draw [c] (8.39412,0.596817) -- (8.39412,1.1669);
\draw [c] (8.39412,1.1669) -- (8.39412,1.38029);
\draw [c] (8.3793,1.1669) -- (8.39412,1.1669);
\draw [c] (8.39412,1.1669) -- (8.40894,1.1669);
\definecolor{c}{rgb}{0,0,0};
\colorlet{c}{kugray};
\draw [c] (8.45339,1.13633) -- (8.45339,1.56977);
\draw [c] (8.45339,1.56977) -- (8.45339,1.74299);
\draw [c] (8.43858,1.56977) -- (8.45339,1.56977);
\draw [c] (8.45339,1.56977) -- (8.46821,1.56977);
\definecolor{c}{rgb}{0,0,0};
\colorlet{c}{kugray};
\draw [c] (8.48303,0.596817) -- (8.48303,1.22594);
\draw [c] (8.48303,1.22594) -- (8.48303,1.43933);
\draw [c] (8.46821,1.22594) -- (8.48303,1.22594);
\draw [c] (8.48303,1.22594) -- (8.49785,1.22594);
\definecolor{c}{rgb}{0,0,0};
\colorlet{c}{kugray};
\draw [c] (8.5423,0.596817) -- (8.5423,1.08304);
\draw [c] (8.5423,1.08304) -- (8.5423,1.29643);
\draw [c] (8.52748,1.08304) -- (8.5423,1.08304);
\draw [c] (8.5423,1.08304) -- (8.55712,1.08304);
\definecolor{c}{rgb}{0,0,0};
\colorlet{c}{kugray};
\draw [c] (8.57194,0.596817) -- (8.57194,1.19595);
\draw [c] (8.57194,1.19595) -- (8.57194,1.40934);
\draw [c] (8.55712,1.19595) -- (8.57194,1.19595);
\draw [c] (8.57194,1.19595) -- (8.58675,1.19595);
\definecolor{c}{rgb}{0,0,0};
\colorlet{c}{kugray};
\draw [c] (8.66084,0.884632) -- (8.66084,1.27717);
\draw [c] (8.66084,1.27717) -- (8.66084,1.44423);
\draw [c] (8.64603,1.27717) -- (8.66084,1.27717);
\draw [c] (8.66084,1.27717) -- (8.67566,1.27717);
\definecolor{c}{rgb}{0,0,0};
\colorlet{c}{kugray};
\draw [c] (8.69048,0.596817) -- (8.69048,1.18479);
\draw [c] (8.69048,1.18479) -- (8.69048,1.39818);
\draw [c] (8.67566,1.18479) -- (8.69048,1.18479);
\draw [c] (8.69048,1.18479) -- (8.7053,1.18479);
\definecolor{c}{rgb}{0,0,0};
\colorlet{c}{kugray};
\draw [c] (8.72012,0.596817) -- (8.72012,1.21188);
\draw [c] (8.72012,1.21188) -- (8.72012,1.42527);
\draw [c] (8.7053,1.21188) -- (8.72012,1.21188);
\draw [c] (8.72012,1.21188) -- (8.73493,1.21188);
\definecolor{c}{rgb}{0,0,0};
\colorlet{c}{kugray};
\draw [c] (8.74975,0.596817) -- (8.74975,1.16598);
\draw [c] (8.74975,1.16598) -- (8.74975,1.37937);
\draw [c] (8.73493,1.16598) -- (8.74975,1.16598);
\draw [c] (8.74975,1.16598) -- (8.76457,1.16598);
\definecolor{c}{rgb}{0,0,0};
\colorlet{c}{kugray};
\draw [c] (8.77939,0.895648) -- (8.77939,1.31039);
\draw [c] (8.77939,1.31039) -- (8.77939,1.48091);
\draw [c] (8.76457,1.31039) -- (8.77939,1.31039);
\draw [c] (8.77939,1.31039) -- (8.79421,1.31039);
\definecolor{c}{rgb}{0,0,0};
\colorlet{c}{kugray};
\draw [c] (8.80902,1.29252) -- (8.80902,1.5583);
\draw [c] (8.80902,1.5583) -- (8.80902,1.69878);
\draw [c] (8.79421,1.5583) -- (8.80902,1.5583);
\draw [c] (8.80902,1.5583) -- (8.82384,1.5583);
\definecolor{c}{rgb}{0,0,0};
\colorlet{c}{kugray};
\draw [c] (8.83866,0.596817) -- (8.83866,1.24072);
\draw [c] (8.83866,1.24072) -- (8.83866,1.45411);
\draw [c] (8.82384,1.24072) -- (8.83866,1.24072);
\draw [c] (8.83866,1.24072) -- (8.85348,1.24072);
\definecolor{c}{rgb}{0,0,0};
\colorlet{c}{kugray};
\draw [c] (8.86829,0.596817) -- (8.86829,0.976443);
\draw [c] (8.86829,0.976443) -- (8.86829,1.18983);
\draw [c] (8.85348,0.976443) -- (8.86829,0.976443);
\draw [c] (8.86829,0.976443) -- (8.88311,0.976443);
\definecolor{c}{rgb}{0,0,0};
\colorlet{c}{kugray};
\draw [c] (8.92757,0.596817) -- (8.92757,1.19654);
\draw [c] (8.92757,1.19654) -- (8.92757,1.40993);
\draw [c] (8.91275,1.19654) -- (8.92757,1.19654);
\draw [c] (8.92757,1.19654) -- (8.94238,1.19654);
\definecolor{c}{rgb}{0,0,0};
\colorlet{c}{kugray};
\draw [c] (9.01647,0.596817) -- (9.01647,1.23526);
\draw [c] (9.01647,1.23526) -- (9.01647,1.44865);
\draw [c] (9.00166,1.23526) -- (9.01647,1.23526);
\draw [c] (9.01647,1.23526) -- (9.03129,1.23526);
\definecolor{c}{rgb}{0,0,0};
\colorlet{c}{kugray};
\draw [c] (9.07574,0.596817) -- (9.07574,1.10612);
\draw [c] (9.07574,1.10612) -- (9.07574,1.31951);
\draw [c] (9.06093,1.10612) -- (9.07574,1.10612);
\draw [c] (9.07574,1.10612) -- (9.09056,1.10612);
\definecolor{c}{rgb}{0,0,0};
\colorlet{c}{kugray};
\draw [c] (9.10538,0.596817) -- (9.10538,1.18963);
\draw [c] (9.10538,1.18963) -- (9.10538,1.40302);
\draw [c] (9.09056,1.18963) -- (9.10538,1.18963);
\draw [c] (9.10538,1.18963) -- (9.1202,1.18963);
\definecolor{c}{rgb}{0,0,0};
\colorlet{c}{kugray};
\draw [c] (9.13502,0.596817) -- (9.13502,1.14382);
\draw [c] (9.13502,1.14382) -- (9.13502,1.35721);
\draw [c] (9.1202,1.14382) -- (9.13502,1.14382);
\draw [c] (9.13502,1.14382) -- (9.14983,1.14382);
\definecolor{c}{rgb}{0,0,0};
\colorlet{c}{kugray};
\draw [c] (9.22392,1.0928) -- (9.22392,1.47755);
\draw [c] (9.22392,1.47755) -- (9.22392,1.64333);
\draw [c] (9.20911,1.47755) -- (9.22392,1.47755);
\draw [c] (9.22392,1.47755) -- (9.23874,1.47755);
\definecolor{c}{rgb}{0,0,0};
\colorlet{c}{kugray};
\draw [c] (9.2832,0.596817) -- (9.2832,1.19105);
\draw [c] (9.2832,1.19105) -- (9.2832,1.40444);
\draw [c] (9.26838,1.19105) -- (9.2832,1.19105);
\draw [c] (9.2832,1.19105) -- (9.29801,1.19105);
\definecolor{c}{rgb}{0,0,0};
\colorlet{c}{kugray};
\draw [c] (9.34247,0.596817) -- (9.34247,1.0722);
\draw [c] (9.34247,1.0722) -- (9.34247,1.28559);
\draw [c] (9.32765,1.0722) -- (9.34247,1.0722);
\draw [c] (9.34247,1.0722) -- (9.35728,1.0722);
\definecolor{c}{rgb}{0,0,0};
\colorlet{c}{kugray};
\draw [c] (9.54992,0.596817) -- (9.54992,1.17593);
\draw [c] (9.54992,1.17593) -- (9.54992,1.38932);
\draw [c] (9.5351,1.17593) -- (9.54992,1.17593);
\draw [c] (9.54992,1.17593) -- (9.56474,1.17593);
\definecolor{c}{rgb}{0,0,0};
\colorlet{c}{kugray};
\draw [c] (9.60919,0.596817) -- (9.60919,1.45456);
\draw [c] (9.60919,1.45456) -- (9.60919,1.66795);
\draw [c] (9.59437,1.45456) -- (9.60919,1.45456);
\draw [c] (9.60919,1.45456) -- (9.62401,1.45456);
\definecolor{c}{rgb}{0,0,0};
\colorlet{c}{kugray};
\draw [c] (9.63882,0.596817) -- (9.63882,1.11893);
\draw [c] (9.63882,1.11893) -- (9.63882,1.33232);
\draw [c] (9.62401,1.11893) -- (9.63882,1.11893);
\draw [c] (9.63882,1.11893) -- (9.65364,1.11893);
\definecolor{c}{rgb}{0,0,0};
\colorlet{c}{kugray};
\draw [c] (9.72773,0.596817) -- (9.72773,1.59979);
\draw [c] (9.72773,1.59979) -- (9.72773,1.81318);
\draw [c] (9.71291,1.59979) -- (9.72773,1.59979);
\draw [c] (9.72773,1.59979) -- (9.74255,1.59979);
\definecolor{c}{rgb}{0,0,0};
\colorlet{c}{kugray};
\draw [c] (9.787,0.596817) -- (9.787,1.11893);
\draw [c] (9.787,1.11893) -- (9.787,1.33232);
\draw [c] (9.77219,1.11893) -- (9.787,1.11893);
\draw [c] (9.787,1.11893) -- (9.80182,1.11893);
\definecolor{c}{rgb}{0,0,0};
\colorlet{c}{kugray};
\draw [c] (9.90555,0.596817) -- (9.90555,1.17321);
\draw [c] (9.90555,1.17321) -- (9.90555,1.3866);
\draw [c] (9.89073,1.17321) -- (9.90555,1.17321);
\draw [c] (9.90555,1.17321) -- (9.92036,1.17321);
\definecolor{c}{rgb}{0,0,0};
\colorlet{c}{kugray};
\draw [c] (9.93518,0.596817) -- (9.93518,1.17077);
\draw [c] (9.93518,1.17077) -- (9.93518,1.38417);
\draw [c] (9.92036,1.17077) -- (9.93518,1.17077);
\draw [c] (9.93518,1.17077) -- (9.95,1.17077);
\definecolor{c}{rgb}{0,0,0};
\colorlet{c}{natgreen};
\draw [c] (1.51655,5.54533) -- (1.60131,5.40419) -- (1.68607,5.2747) -- (1.77083,5.15478) -- (1.85558,5.04288) -- (1.94034,4.93778) -- (2.0251,4.83855) -- (2.10986,4.74443) -- (2.19462,4.65481) -- (2.27938,4.56919) -- (2.36413,4.48712)
 -- (2.44889,4.40827) -- (2.53365,4.33233) -- (2.61841,4.25902) -- (2.70317,4.18813) -- (2.78793,4.11945) -- (2.87268,4.05281) -- (2.95744,3.98806) -- (3.0422,3.92505) -- (3.12696,3.86367) -- (3.21172,3.80381) -- (3.29648,3.74536) --
 (3.38123,3.68824) -- (3.46599,3.63236) -- (3.55075,3.57765) -- (3.63551,3.52404) -- (3.72027,3.47148) -- (3.80502,3.4199) -- (3.88978,3.36925) -- (3.97454,3.31949) -- (4.0593,3.27056) -- (4.14406,3.22243) -- (4.22882,3.17506) -- (4.31357,3.12841) --
 (4.39833,3.08245) -- (4.48309,3.03714) -- (4.56785,2.99247) -- (4.65261,2.94839) -- (4.73737,2.90489) -- (4.82212,2.86194) -- (4.90688,2.81952) -- (4.99164,2.7776) -- (5.0764,2.73617) -- (5.16116,2.6952) -- (5.24592,2.65468) -- (5.33067,2.6146) --
 (5.41543,2.57492) -- (5.50019,2.53565) -- (5.58495,2.49676) -- (5.66971,2.45824);
\draw [c] (5.66971,2.45824) -- (5.75447,2.42007) -- (5.83922,2.38225) -- (5.92398,2.34476) -- (6.00874,2.30759) -- (6.0935,2.27073) -- (6.17826,2.23416) -- (6.26301,2.19789) -- (6.34777,2.16188) -- (6.43253,2.12615) --
 (6.51729,2.09068) -- (6.60205,2.05545) -- (6.68681,2.02047) -- (6.77156,1.98572) -- (6.85632,1.9512) -- (6.94108,1.91689) -- (7.02584,1.88279) -- (7.1106,1.8489) -- (7.19536,1.81521) -- (7.28011,1.78171) -- (7.36487,1.74839) -- (7.44963,1.71525) --
 (7.53439,1.68228) -- (7.61915,1.64948) -- (7.70391,1.61684) -- (7.78866,1.58435) -- (7.87342,1.55202) -- (7.95818,1.51983) -- (8.04294,1.48778) -- (8.1277,1.45586) -- (8.21246,1.42408) -- (8.29721,1.39242) -- (8.38197,1.36088) -- (8.46673,1.32946)
 -- (8.55149,1.29816) -- (8.63625,1.26696) -- (8.72101,1.23587) -- (8.80576,1.20487) -- (8.89052,1.17398) -- (8.97528,1.14317) -- (9.06004,1.11246) -- (9.1448,1.08183) -- (9.22955,1.05129) -- (9.31431,1.02082) -- (9.39907,0.990431) --
 (9.48383,0.960111) -- (9.56859,0.929859) -- (9.65335,0.899673) -- (9.7381,0.869549) -- (9.82286,0.839483);
\draw [c] (9.82286,0.839483) -- (9.90762,0.809473);
\definecolor{c}{rgb}{0,0,0};
\draw [c] (1,0.596817) -- (9.95,0.596817);
\draw [anchor= east] (9.95,0.0954907) node[color=c, rotate=0]{$M_{\gamma\gamma}\text{ [GeV]}$};
\draw [c] (1.02964,0.757062) -- (1.02964,0.596817);
\draw [c] (1.32599,0.67694) -- (1.32599,0.596817);
\draw [c] (1.62235,0.67694) -- (1.62235,0.596817);
\draw [c] (1.91871,0.67694) -- (1.91871,0.596817);
\draw [c] (2.21507,0.67694) -- (2.21507,0.596817);
\draw [c] (2.51142,0.757062) -- (2.51142,0.596817);
\draw [c] (2.80778,0.67694) -- (2.80778,0.596817);
\draw [c] (3.10414,0.67694) -- (3.10414,0.596817);
\draw [c] (3.4005,0.67694) -- (3.4005,0.596817);
\draw [c] (3.69685,0.67694) -- (3.69685,0.596817);
\draw [c] (3.99321,0.757062) -- (3.99321,0.596817);
\draw [c] (4.28957,0.67694) -- (4.28957,0.596817);
\draw [c] (4.58593,0.67694) -- (4.58593,0.596817);
\draw [c] (4.88228,0.67694) -- (4.88228,0.596817);
\draw [c] (5.17864,0.67694) -- (5.17864,0.596817);
\draw [c] (5.475,0.757062) -- (5.475,0.596817);
\draw [c] (5.77136,0.67694) -- (5.77136,0.596817);
\draw [c] (6.06772,0.67694) -- (6.06772,0.596817);
\draw [c] (6.36407,0.67694) -- (6.36407,0.596817);
\draw [c] (6.66043,0.67694) -- (6.66043,0.596817);
\draw [c] (6.95679,0.757062) -- (6.95679,0.596817);
\draw [c] (7.25315,0.67694) -- (7.25315,0.596817);
\draw [c] (7.5495,0.67694) -- (7.5495,0.596817);
\draw [c] (7.84586,0.67694) -- (7.84586,0.596817);
\draw [c] (8.14222,0.67694) -- (8.14222,0.596817);
\draw [c] (8.43858,0.757062) -- (8.43858,0.596817);
\draw [c] (8.73493,0.67694) -- (8.73493,0.596817);
\draw [c] (9.03129,0.67694) -- (9.03129,0.596817);
\draw [c] (9.32765,0.67694) -- (9.32765,0.596817);
\draw [c] (9.62401,0.67694) -- (9.62401,0.596817);
\draw [c] (9.92036,0.757062) -- (9.92036,0.596817);
\draw [c] (1.02964,0.757062) -- (1.02964,0.596817);
\draw [c] (9.92036,0.757062) -- (9.92036,0.596817);
\draw [anchor=base] (1.02964,0.310345) node[color=c, rotate=0]{0};
\draw [anchor=base] (2.51142,0.310345) node[color=c, rotate=0]{500};
\draw [anchor=base] (3.99321,0.310345) node[color=c, rotate=0]{1000};
\draw [anchor=base] (5.475,0.310345) node[color=c, rotate=0]{1500};
\draw [anchor=base] (6.95679,0.310345) node[color=c, rotate=0]{2000};
\draw [anchor=base] (8.43858,0.310345) node[color=c, rotate=0]{2500};
\draw [anchor=base] (9.92036,0.310345) node[color=c, rotate=0]{3000};
\draw [c] (1,0.596817) -- (1,5.90849);
\draw [anchor= east] (-0.12,5.90849) node[color=c, rotate=90]{Number of events};
\draw [c] (1.1335,0.632213) -- (1,0.632213);
\draw [c] (1.1335,0.668474) -- (1,0.668474);
\draw [c] (1.267,0.70091) -- (1,0.70091);
\draw [anchor= east] (0.922,0.70091) node[color=c, rotate=0]{$10^{-4}$};
\draw [c] (1.1335,0.9143) -- (1,0.9143);
\draw [c] (1.1335,1.03913) -- (1,1.03913);
\draw [c] (1.1335,1.12769) -- (1,1.12769);
\draw [c] (1.1335,1.19639) -- (1,1.19639);
\draw [c] (1.1335,1.25252) -- (1,1.25252);
\draw [c] (1.1335,1.29997) -- (1,1.29997);
\draw [c] (1.1335,1.34108) -- (1,1.34108);
\draw [c] (1.1335,1.37734) -- (1,1.37734);
\draw [c] (1.267,1.40978) -- (1,1.40978);
\draw [anchor= east] (0.922,1.40978) node[color=c, rotate=0]{$10^{-3}$};
\draw [c] (1.1335,1.62317) -- (1,1.62317);
\draw [c] (1.1335,1.74799) -- (1,1.74799);
\draw [c] (1.1335,1.83656) -- (1,1.83656);
\draw [c] (1.1335,1.90525) -- (1,1.90525);
\draw [c] (1.1335,1.96138) -- (1,1.96138);
\draw [c] (1.1335,2.00884) -- (1,2.00884);
\draw [c] (1.1335,2.04995) -- (1,2.04995);
\draw [c] (1.1335,2.08621) -- (1,2.08621);
\draw [c] (1.267,2.11864) -- (1,2.11864);
\draw [anchor= east] (0.922,2.11864) node[color=c, rotate=0]{$10^{-2}$};
\draw [c] (1.1335,2.33203) -- (1,2.33203);
\draw [c] (1.1335,2.45686) -- (1,2.45686);
\draw [c] (1.1335,2.54542) -- (1,2.54542);
\draw [c] (1.1335,2.61412) -- (1,2.61412);
\draw [c] (1.1335,2.67025) -- (1,2.67025);
\draw [c] (1.1335,2.71771) -- (1,2.71771);
\draw [c] (1.1335,2.75882) -- (1,2.75882);
\draw [c] (1.1335,2.79508) -- (1,2.79508);
\draw [c] (1.267,2.82751) -- (1,2.82751);
\draw [anchor= east] (0.922,2.82751) node[color=c, rotate=0]{$10^{-1}$};
\draw [c] (1.1335,3.0409) -- (1,3.0409);
\draw [c] (1.1335,3.16573) -- (1,3.16573);
\draw [c] (1.1335,3.25429) -- (1,3.25429);
\draw [c] (1.1335,3.32299) -- (1,3.32299);
\draw [c] (1.1335,3.37912) -- (1,3.37912);
\draw [c] (1.1335,3.42657) -- (1,3.42657);
\draw [c] (1.1335,3.46768) -- (1,3.46768);
\draw [c] (1.1335,3.50394) -- (1,3.50394);
\draw [c] (1.267,3.53638) -- (1,3.53638);
\draw [anchor= east] (0.922,3.53638) node[color=c, rotate=0]{1};
\draw [c] (1.1335,3.74977) -- (1,3.74977);
\draw [c] (1.1335,3.87459) -- (1,3.87459);
\draw [c] (1.1335,3.96316) -- (1,3.96316);
\draw [c] (1.1335,4.03186) -- (1,4.03186);
\draw [c] (1.1335,4.08799) -- (1,4.08799);
\draw [c] (1.1335,4.13544) -- (1,4.13544);
\draw [c] (1.1335,4.17655) -- (1,4.17655);
\draw [c] (1.1335,4.21281) -- (1,4.21281);
\draw [c] (1.267,4.24525) -- (1,4.24525);
\draw [anchor= east] (0.922,4.24525) node[color=c, rotate=0]{10};
\draw [c] (1.1335,4.45864) -- (1,4.45864);
\draw [c] (1.1335,4.58346) -- (1,4.58346);
\draw [c] (1.1335,4.67203) -- (1,4.67203);
\draw [c] (1.1335,4.74072) -- (1,4.74072);
\draw [c] (1.1335,4.79685) -- (1,4.79685);
\draw [c] (1.1335,4.84431) -- (1,4.84431);
\draw [c] (1.1335,4.88542) -- (1,4.88542);
\draw [c] (1.1335,4.92168) -- (1,4.92168);
\draw [c] (1.267,4.95411) -- (1,4.95411);
\draw [anchor= east] (0.922,4.95411) node[color=c, rotate=0]{$10^{2}$};
\draw [c] (1.1335,5.1675) -- (1,5.1675);
\draw [c] (1.1335,5.29233) -- (1,5.29233);
\draw [c] (1.1335,5.38089) -- (1,5.38089);
\draw [c] (1.1335,5.44959) -- (1,5.44959);
\draw [c] (1.1335,5.50572) -- (1,5.50572);
\draw [c] (1.1335,5.55318) -- (1,5.55318);
\draw [c] (1.1335,5.59428) -- (1,5.59428);
\draw [c] (1.1335,5.63055) -- (1,5.63055);
\draw [c] (1.267,5.66298) -- (1,5.66298);
\draw [anchor= east] (0.922,5.66298) node[color=c, rotate=0]{$10^{3}$};
\draw [c] (1.1335,5.87637) -- (1,5.87637);
\colorlet{c}{natgreen!40};
\draw [c, fill=c] (1.45951,0.596817) -- (1.48904,4.38696) -- (1.51857,5.54721) -- (1.54811,5.49618) -- (1.57764,5.44687) -- (1.60717,5.39917) -- (1.6367,5.35293) -- (1.66623,5.30805) -- (1.69576,5.26443) -- (1.7253,5.22198) -- (1.75483,5.18061) --
 (1.78436,5.14026) -- (1.81389,5.10086) -- (1.84342,5.06235) -- (1.87296,5.02469) -- (1.90249,4.98782) -- (1.93202,4.9517) -- (1.96155,4.91631) -- (1.99108,4.88159) -- (2.02061,4.84752) -- (2.05015,4.81408) -- (2.07968,4.78122) -- (2.10921,4.74894)
 -- (2.13874,4.71719) -- (2.16827,4.68597) -- (2.1978,4.65524) -- (2.22734,4.625) -- (2.25687,4.59522) -- (2.2864,4.56589) -- (2.31593,4.53698) -- (2.34546,4.50849) -- (2.375,4.48039) -- (2.40453,4.45268) -- (2.43406,4.42535) -- (2.46359,4.39838) --
 (2.49312,4.37175) -- (2.52265,4.34546) -- (2.55219,4.3195) -- (2.58172,4.29385) -- (2.61125,4.26852) -- (2.64078,4.24348) -- (2.67031,4.21873) -- (2.69984,4.19426) -- (2.72938,4.17006) -- (2.75891,4.14613) -- (2.78844,4.12245) -- (2.81797,4.09903)
 -- (2.8475,4.07584) -- (2.87704,4.0529) -- (2.90657,4.03018) -- (2.9361,4.00769) -- (2.96563,3.98541) -- (2.99516,3.96335) -- (3.02469,3.94149) -- (3.05423,3.91984) -- (3.08376,3.89838) -- (3.11329,3.87712) -- (3.14282,3.85604) -- (3.17235,3.83514)
 -- (3.20189,3.81442) -- (3.23142,3.79387) -- (3.26095,3.7735) -- (3.29048,3.75328) -- (3.32001,3.73323) -- (3.34954,3.71334) -- (3.37908,3.6936) -- (3.40861,3.67402) -- (3.43814,3.65458) -- (3.46767,3.63528) -- (3.4972,3.61613) -- (3.52673,3.59712)
 -- (3.55627,3.57824) -- (3.5858,3.55949) -- (3.61533,3.54087) -- (3.64486,3.52238) -- (3.67439,3.50402) -- (3.70393,3.48578) -- (3.73346,3.46765) -- (3.76299,3.44965) -- (3.79252,3.43176) -- (3.82205,3.41399) -- (3.85158,3.39632) --
 (3.88112,3.37876) -- (3.91065,3.36132) -- (3.94018,3.34397) -- (3.96971,3.32673) -- (3.99924,3.30959) -- (4.02877,3.29255) -- (4.05831,3.27561) -- (4.08784,3.25877) -- (4.11737,3.24202) -- (4.1469,3.22536) -- (4.17643,3.2088) -- (4.20597,3.19232) --
 (4.2355,3.17594) -- (4.26503,3.15964) -- (4.29456,3.14343) -- (4.32409,3.1273) -- (4.35362,3.11126) -- (4.38316,3.09529) -- (4.41269,3.07941) -- (4.44222,3.06361) -- (4.47175,3.04789) -- (4.50128,3.03225) -- (4.53082,3.01668) -- (4.56035,3.00119) --
 (4.58988,2.98577) -- (4.61941,2.97043) -- (4.64894,2.95515) -- (4.67847,2.93995) -- (4.70801,2.92482) -- (4.73754,2.90977) -- (4.76707,2.89477) -- (4.7966,2.87985) -- (4.82613,2.865) -- (4.85566,2.85021) -- (4.8852,2.83548) -- (4.91473,2.82083) --
 (4.94426,2.80623) -- (4.97379,2.7917) -- (5.00332,2.77723) -- (5.03286,2.76282) -- (5.06239,2.74848) -- (5.09192,2.73419) -- (5.12145,2.71996) -- (5.15098,2.70579) -- (5.18051,2.69168) -- (5.21005,2.67763) -- (5.23958,2.66364) -- (5.26911,2.6497) --
 (5.29864,2.63581) -- (5.32817,2.62198) -- (5.3577,2.60821) -- (5.38724,2.59448) -- (5.41677,2.58081) -- (5.4463,2.5672) -- (5.47583,2.55363) -- (5.50536,2.54011) -- (5.5349,2.52665) -- (5.56443,2.51323) -- (5.59396,2.49987) -- (5.62349,2.48655) --
 (5.65302,2.47328) -- (5.68255,2.46005) -- (5.71209,2.44688) -- (5.74162,2.43375) -- (5.77115,2.42066) -- (5.80068,2.40762) -- (5.83021,2.39462) -- (5.85975,2.38167) -- (5.88928,2.36876) -- (5.91881,2.3559) -- (5.94834,2.34307) -- (5.97787,2.33029)
 -- (6.0074,2.31755) -- (6.03694,2.30485) -- (6.06647,2.29219) -- (6.096,2.27957) -- (6.12553,2.26699) -- (6.15506,2.25444) -- (6.18459,2.24194) -- (6.21413,2.22947) -- (6.24366,2.21704) -- (6.27319,2.20465) -- (6.30272,2.1923) -- (6.33225,2.17998)
 -- (6.36179,2.16769) -- (6.39132,2.15544) -- (6.42085,2.14323) -- (6.45038,2.13105) -- (6.47991,2.1189) -- (6.50944,2.10679) -- (6.53898,2.09471) -- (6.56851,2.08267) -- (6.59804,2.07065) -- (6.62757,2.05867) -- (6.6571,2.04672) -- (6.68663,2.03481)
 -- (6.71617,2.02292) -- (6.7457,2.01106) -- (6.77523,1.99923) -- (6.80476,1.98744) -- (6.83429,1.97567) -- (6.86383,1.96393) -- (6.89336,1.95222) -- (6.92289,1.94054) -- (6.95242,1.92889) -- (6.98195,1.91727) -- (7.01148,1.90567) -- (7.04102,1.8941)
 -- (7.07055,1.88256) -- (7.10008,1.87104) -- (7.12961,1.85955) -- (7.15914,1.84809) -- (7.18867,1.83665) -- (7.21821,1.82524) -- (7.24774,1.81385) -- (7.27727,1.80249) -- (7.3068,1.79115) -- (7.33633,1.77984) -- (7.36587,1.76855) -- (7.3954,1.75728)
 -- (7.42493,1.74604) -- (7.45446,1.73482) -- (7.48399,1.72363) -- (7.51352,1.71245) -- (7.54306,1.7013) -- (7.57259,1.69018) -- (7.60212,1.67907) -- (7.63165,1.66798) -- (7.66118,1.65692) -- (7.69072,1.64588) -- (7.72025,1.63486) --
 (7.74978,1.62386) -- (7.77931,1.61288) -- (7.80884,1.60192) -- (7.83837,1.59098) -- (7.86791,1.58006) -- (7.89744,1.56916) -- (7.92697,1.55828) -- (7.9565,1.54742) -- (7.98603,1.53658) -- (8.01556,1.52576) -- (8.0451,1.51495) -- (8.07463,1.50417) --
 (8.10416,1.4934) -- (8.13369,1.48265) -- (8.16322,1.47191) -- (8.19276,1.4612) -- (8.22229,1.4505) -- (8.25182,1.43982) -- (8.28135,1.42915) -- (8.31088,1.41851) -- (8.34041,1.40788) -- (8.36995,1.39726) -- (8.39948,1.38666) -- (8.42901,1.37608) --
 (8.45854,1.36551) -- (8.48807,1.35496) -- (8.5176,1.34443) -- (8.54714,1.3339) -- (8.57667,1.3234) -- (8.6062,1.31291) -- (8.63573,1.30243) -- (8.66526,1.29197) -- (8.6948,1.28152) -- (8.72433,1.27109) -- (8.75386,1.26067) -- (8.78339,1.25026) --
 (8.81292,1.23987) -- (8.84245,1.22949) -- (8.87199,1.21912) -- (8.90152,1.20877) -- (8.93105,1.19843) -- (8.96058,1.1881) -- (8.99011,1.17779) -- (9.01964,1.16748) -- (9.04918,1.15719) -- (9.07871,1.14692) -- (9.10824,1.13665) -- (9.13777,1.12639)
 -- (9.1673,1.11615) -- (9.19684,1.10592) -- (9.22637,1.0957) -- (9.2559,1.08549) -- (9.28543,1.07529) -- (9.31496,1.06511) -- (9.34449,1.05493) -- (9.37403,1.04476) -- (9.40356,1.03461) -- (9.43309,1.02446) -- (9.46262,1.01433) -- (9.49215,1.0042)
 -- (9.52169,0.994085) -- (9.55122,0.983979) -- (9.58075,0.973882) -- (9.61028,0.963795) -- (9.63981,0.953717) -- (9.66934,0.943648) -- (9.69888,0.933588) -- (9.72841,0.923537) -- (9.75794,0.913495) -- (9.78747,0.90346) -- (9.817,0.893435) --
 (9.84653,0.883417) -- (9.87607,0.873408) -- (9.9056,0.810189) -- (9.93513,0.596817) -- (9.93513,0.596817) -- (9.9056,0.810189) -- (9.87607,0.756916) -- (9.84653,0.768025) -- (9.817,0.779137) -- (9.78747,0.790251) -- (9.75794,0.801367) --
 (9.72841,0.812487) -- (9.69888,0.823609) -- (9.66934,0.834734) -- (9.63981,0.845862) -- (9.61028,0.856994) -- (9.58075,0.868129) -- (9.55122,0.879268) -- (9.52169,0.89041) -- (9.49215,0.901557) -- (9.46262,0.912708) -- (9.43309,0.923863) --
 (9.40356,0.935022) -- (9.37403,0.946186) -- (9.34449,0.957355) -- (9.31496,0.968529) -- (9.28543,0.979708) -- (9.2559,0.990893) -- (9.22637,1.00208) -- (9.19684,1.01328) -- (9.1673,1.02448) -- (9.13777,1.03569) -- (9.10824,1.0469) --
 (9.07871,1.05812) -- (9.04918,1.06935) -- (9.01964,1.08058) -- (8.99011,1.09182) -- (8.96058,1.10306) -- (8.93105,1.11432) -- (8.90152,1.12558) -- (8.87199,1.13685) -- (8.84245,1.14813) -- (8.81292,1.15941) -- (8.78339,1.1707) -- (8.75386,1.182) --
 (8.72433,1.19331) -- (8.6948,1.20463) -- (8.66526,1.21596) -- (8.63573,1.22729) -- (8.6062,1.23863) -- (8.57667,1.24999) -- (8.54714,1.26135) -- (8.5176,1.27272) -- (8.48807,1.28411) -- (8.45854,1.2955) -- (8.42901,1.3069) -- (8.39948,1.31832) --
 (8.36995,1.32974) -- (8.34041,1.34118) -- (8.31088,1.35262) -- (8.28135,1.36408) -- (8.25182,1.37555) -- (8.22229,1.38703) -- (8.19276,1.39852) -- (8.16322,1.41003) -- (8.13369,1.42154) -- (8.10416,1.43307) -- (8.07463,1.44461) -- (8.0451,1.45617)
 -- (8.01556,1.46774) -- (7.98603,1.47932) -- (7.9565,1.49091) -- (7.92697,1.50252) -- (7.89744,1.51414) -- (7.86791,1.52578) -- (7.83837,1.53743) -- (7.80884,1.5491) -- (7.77931,1.56078) -- (7.74978,1.57247) -- (7.72025,1.58418) -- (7.69072,1.59591)
 -- (7.66118,1.60765) -- (7.63165,1.61941) -- (7.60212,1.63118) -- (7.57259,1.64297) -- (7.54306,1.65478) -- (7.51352,1.6666) -- (7.48399,1.67844) -- (7.45446,1.6903) -- (7.42493,1.70217) -- (7.3954,1.71407) -- (7.36587,1.72598) -- (7.33633,1.73791)
 -- (7.3068,1.74986) -- (7.27727,1.76183) -- (7.24774,1.77381) -- (7.21821,1.78582) -- (7.18867,1.79785) -- (7.15914,1.80989) -- (7.12961,1.82196) -- (7.10008,1.83405) -- (7.07055,1.84615) -- (7.04102,1.85828) -- (7.01148,1.87043) --
 (6.98195,1.88261) -- (6.95242,1.8948) -- (6.92289,1.90701) -- (6.89336,1.91925) -- (6.86383,1.93151) -- (6.83429,1.9438) -- (6.80476,1.95611) -- (6.77523,1.96844) -- (6.7457,1.9808) -- (6.71617,1.99318) -- (6.68663,2.00558) -- (6.6571,2.01801) --
 (6.62757,2.03047) -- (6.59804,2.04295) -- (6.56851,2.05545) -- (6.53898,2.06799) -- (6.50944,2.08055) -- (6.47991,2.09313) -- (6.45038,2.10575) -- (6.42085,2.11839) -- (6.39132,2.13106) -- (6.36179,2.14376) -- (6.33225,2.15649) -- (6.30272,2.16925)
 -- (6.27319,2.18203) -- (6.24366,2.19485) -- (6.21413,2.2077) -- (6.18459,2.22057) -- (6.15506,2.23348) -- (6.12553,2.24642) -- (6.096,2.2594) -- (6.06647,2.2724) -- (6.03694,2.28544) -- (6.0074,2.29851) -- (5.97787,2.31161) -- (5.94834,2.32475) --
 (5.91881,2.33793) -- (5.88928,2.35113) -- (5.85975,2.36438) -- (5.83021,2.37766) -- (5.80068,2.39098) -- (5.77115,2.40433) -- (5.74162,2.41772) -- (5.71209,2.43115) -- (5.68255,2.44462) -- (5.65302,2.45812) -- (5.62349,2.47167) -- (5.59396,2.48526)
 -- (5.56443,2.49888) -- (5.5349,2.51255) -- (5.50536,2.52626) -- (5.47583,2.54002) -- (5.4463,2.55381) -- (5.41677,2.56765) -- (5.38724,2.58154) -- (5.3577,2.59546) -- (5.32817,2.60944) -- (5.29864,2.62346) -- (5.26911,2.63753) -- (5.23958,2.65165)
 -- (5.21005,2.66581) -- (5.18051,2.68003) -- (5.15098,2.69429) -- (5.12145,2.70861) -- (5.09192,2.72297) -- (5.06239,2.73739) -- (5.03286,2.75187) -- (5.00332,2.7664) -- (4.97379,2.78098) -- (4.94426,2.79562) -- (4.91473,2.81032) -- (4.8852,2.82507)
 -- (4.85566,2.83989) -- (4.82613,2.85476) -- (4.7966,2.8697) -- (4.76707,2.8847) -- (4.73754,2.89976) -- (4.70801,2.91488) -- (4.67847,2.93008) -- (4.64894,2.94534) -- (4.61941,2.96066) -- (4.58988,2.97606) -- (4.56035,2.99153) -- (4.53082,3.00707)
 -- (4.50128,3.02268) -- (4.47175,3.03837) -- (4.44222,3.05413) -- (4.41269,3.06997) -- (4.38316,3.08589) -- (4.35362,3.10188) -- (4.32409,3.11796) -- (4.29456,3.13412) -- (4.26503,3.15037) -- (4.2355,3.1667) -- (4.20597,3.18312) -- (4.17643,3.19962)
 -- (4.1469,3.21622) -- (4.11737,3.23291) -- (4.08784,3.2497) -- (4.05831,3.26657) -- (4.02877,3.28355) -- (3.99924,3.30062) -- (3.96971,3.3178) -- (3.94018,3.33508) -- (3.91065,3.35246) -- (3.88112,3.36995) -- (3.85158,3.38755) -- (3.82205,3.40525)
 -- (3.79252,3.42308) -- (3.76299,3.44101) -- (3.73346,3.45906) -- (3.70393,3.47723) -- (3.67439,3.49553) -- (3.64486,3.51394) -- (3.61533,3.53249) -- (3.5858,3.55116) -- (3.55627,3.56996) -- (3.52673,3.5889) -- (3.4972,3.60797) -- (3.46767,3.62719)
 -- (3.43814,3.64654) -- (3.40861,3.66604) -- (3.37908,3.68569) -- (3.34954,3.70549) -- (3.32001,3.72545) -- (3.29048,3.74556) -- (3.26095,3.76584) -- (3.23142,3.78628) -- (3.20189,3.80689) -- (3.17235,3.82767) -- (3.14282,3.84863) --
 (3.11329,3.86977) -- (3.08376,3.8911) -- (3.05423,3.91261) -- (3.02469,3.93432) -- (2.99516,3.95623) -- (2.96563,3.97834) -- (2.9361,4.00066) -- (2.90657,4.0232) -- (2.87704,4.04595) -- (2.8475,4.06893) -- (2.81797,4.09214) -- (2.78844,4.11559) --
 (2.75891,4.13928) -- (2.72938,4.16323) -- (2.69984,4.18743) -- (2.67031,4.2119) -- (2.64078,4.23665) -- (2.61125,4.26168) -- (2.58172,4.287) -- (2.55219,4.31262) -- (2.52265,4.33855) -- (2.49312,4.3648) -- (2.46359,4.39139) -- (2.43406,4.41832) --
 (2.40453,4.44561) -- (2.375,4.47326) -- (2.34546,4.5013) -- (2.31593,4.52973) -- (2.2864,4.55858) -- (2.25687,4.58785) -- (2.22734,4.61757) -- (2.1978,4.64776) -- (2.16827,4.67842) -- (2.13874,4.70959) -- (2.10921,4.74129) -- (2.07968,4.77353) --
 (2.05015,4.80635) -- (2.02061,4.83976) -- (1.99108,4.87381) -- (1.96155,4.90851) -- (1.93202,4.9439) -- (1.90249,4.98002) -- (1.87296,5.0169) -- (1.84342,5.05458) -- (1.81389,5.0931) -- (1.78436,5.13251) -- (1.75483,5.17286) -- (1.7253,5.2142) --
 (1.69576,5.25657) -- (1.66623,5.30005) -- (1.6367,5.34468) -- (1.60717,5.39055) -- (1.57764,5.43771) -- (1.54811,5.48626) -- (1.51857,5.5363) -- (1.48904,0.596817) -- (1.45951,0.596817);
\definecolor{c}{rgb}{0,0,0};
\colorlet{c}{natcomp!40};
\draw [c, fill=c] (1.45951,0.596817) -- (1.48904,4.38693) -- (1.51857,5.54742) -- (1.54811,5.49642) -- (1.57764,5.44715) -- (1.60717,5.39949) -- (1.6367,5.3533) -- (1.66623,5.30848) -- (1.69576,5.26492) -- (1.7253,5.22253) -- (1.75483,5.18123) --
 (1.78436,5.14096) -- (1.81389,5.10164) -- (1.84342,5.06323) -- (1.87296,5.02567) -- (1.90249,4.98891) -- (1.93202,4.95292) -- (1.96155,4.91766) -- (1.99108,4.88309) -- (2.02061,4.84917) -- (2.05015,4.8159) -- (2.07968,4.78322) -- (2.10921,4.75113)
 -- (2.13874,4.7196) -- (2.16827,4.6886) -- (2.1978,4.65812) -- (2.22734,4.62814) -- (2.25687,4.59863) -- (2.2864,4.56959) -- (2.31593,4.541) -- (2.34546,4.51285) -- (2.375,4.48512) -- (2.40453,4.45779) -- (2.43406,4.43087) -- (2.46359,4.40432) --
 (2.49312,4.37816) -- (2.52265,4.35236) -- (2.55219,4.32691) -- (2.58172,4.30181) -- (2.61125,4.27705) -- (2.64078,4.25262) -- (2.67031,4.22852) -- (2.69984,4.20473) -- (2.72938,4.18125) -- (2.75891,4.15807) -- (2.78844,4.13519) -- (2.81797,4.1126)
 -- (2.8475,4.0903) -- (2.87704,4.06828) -- (2.90657,4.04653) -- (2.9361,4.02506) -- (2.96563,4.00385) -- (2.99516,3.98291) -- (3.02469,3.96222) -- (3.05423,3.94179) -- (3.08376,3.92162) -- (3.11329,3.90169) -- (3.14282,3.88201) -- (3.17235,3.86257)
 -- (3.20189,3.84337) -- (3.23142,3.82441) -- (3.26095,3.80568) -- (3.29048,3.78719) -- (3.32001,3.76893) -- (3.34954,3.7509) -- (3.37908,3.7331) -- (3.40861,3.71552) -- (3.43814,3.69817) -- (3.46767,3.68103) -- (3.4972,3.66412) -- (3.52673,3.64743)
 -- (3.55627,3.63095) -- (3.5858,3.6147) -- (3.61533,3.59865) -- (3.64486,3.58282) -- (3.67439,3.5672) -- (3.70393,3.5518) -- (3.73346,3.5366) -- (3.76299,3.52161) -- (3.79252,3.50683) -- (3.82205,3.49226) -- (3.85158,3.47789) -- (3.88112,3.46372) --
 (3.91065,3.44975) -- (3.94018,3.43599) -- (3.96971,3.42242) -- (3.99924,3.40905) -- (4.02877,3.39587) -- (4.05831,3.3829) -- (4.08784,3.37011) -- (4.11737,3.35751) -- (4.1469,3.34511) -- (4.17643,3.33289) -- (4.20597,3.32085) -- (4.2355,3.309) --
 (4.26503,3.29733) -- (4.29456,3.28584) -- (4.32409,3.27453) -- (4.35362,3.26339) -- (4.38316,3.25243) -- (4.41269,3.24163) -- (4.44222,3.23101) -- (4.47175,3.22055) -- (4.50128,3.21026) -- (4.53082,3.20013) -- (4.56035,3.19015) -- (4.58988,3.18034)
 -- (4.61941,3.17067) -- (4.64894,3.16116) -- (4.67847,3.1518) -- (4.70801,3.14259) -- (4.73754,3.13352) -- (4.76707,3.12459) -- (4.7966,3.1158) -- (4.82613,3.10714) -- (4.85566,3.09862) -- (4.8852,3.09023) -- (4.91473,3.08197) -- (4.94426,3.07384)
 -- (4.97379,3.06583) -- (5.00332,3.05794) -- (5.03286,3.05016) -- (5.06239,3.04251) -- (5.09192,3.03496) -- (5.12145,3.02753) -- (5.15098,3.02021) -- (5.18051,3.01299) -- (5.21005,3.00587) -- (5.23958,2.99886) -- (5.26911,2.99194) --
 (5.29864,2.98512) -- (5.32817,2.97839) -- (5.3577,2.97176) -- (5.38724,2.96521) -- (5.41677,2.95876) -- (5.4463,2.95238) -- (5.47583,2.94609) -- (5.50536,2.93988) -- (5.5349,2.93375) -- (5.56443,2.9277) -- (5.59396,2.92172) -- (5.62349,2.91582) --
 (5.65302,2.90998) -- (5.68255,2.90422) -- (5.71209,2.89852) -- (5.74162,2.89289) -- (5.77115,2.88732) -- (5.80068,2.88181) -- (5.83021,2.87636) -- (5.85975,2.87098) -- (5.88928,2.86565) -- (5.91881,2.86037) -- (5.94834,2.85515) -- (5.97787,2.84999)
 -- (6.0074,2.84487) -- (6.03694,2.83981) -- (6.06647,2.83479) -- (6.096,2.82982) -- (6.12553,2.8249) -- (6.15506,2.82002) -- (6.18459,2.81519) -- (6.21413,2.8104) -- (6.24366,2.80565) -- (6.27319,2.80094) -- (6.30272,2.79627) -- (6.33225,2.79163) --
 (6.36179,2.78704) -- (6.39132,2.78248) -- (6.42085,2.77796) -- (6.45038,2.77347) -- (6.47991,2.76901) -- (6.50944,2.76459) -- (6.53898,2.7602) -- (6.56851,2.75583) -- (6.59804,2.7515) -- (6.62757,2.7472) -- (6.6571,2.74292) -- (6.68663,2.73868) --
 (6.71617,2.73446) -- (6.7457,2.73026) -- (6.77523,2.72609) -- (6.80476,2.72195) -- (6.83429,2.71783) -- (6.86383,2.71373) -- (6.89336,2.70966) -- (6.92289,2.7056) -- (6.95242,2.70157) -- (6.98195,2.69756) -- (7.01148,2.69357) -- (7.04102,2.6896) --
 (7.07055,2.68565) -- (7.10008,2.68172) -- (7.12961,2.6778) -- (7.15914,2.67391) -- (7.18867,2.67003) -- (7.21821,2.66616) -- (7.24774,2.66231) -- (7.27727,2.65848) -- (7.3068,2.65467) -- (7.33633,2.65086) -- (7.36587,2.64708) -- (7.3954,2.6433) --
 (7.42493,2.63954) -- (7.45446,2.6358) -- (7.48399,2.63206) -- (7.51352,2.62834) -- (7.54306,2.62463) -- (7.57259,2.62093) -- (7.60212,2.61724) -- (7.63165,2.61357) -- (7.66118,2.6099) -- (7.69072,2.60625) -- (7.72025,2.6026) -- (7.74978,2.59897) --
 (7.77931,2.59534) -- (7.80884,2.59173) -- (7.83837,2.58812) -- (7.86791,2.58452) -- (7.89744,2.58093) -- (7.92697,2.57735) -- (7.9565,2.57377) -- (7.98603,2.5702) -- (8.01556,2.56664) -- (8.0451,2.56309) -- (8.07463,2.55954) -- (8.10416,2.556) --
 (8.13369,2.55247) -- (8.16322,2.54894) -- (8.19276,2.54541) -- (8.22229,2.5419) -- (8.25182,2.53838) -- (8.28135,2.53488) -- (8.31088,2.53138) -- (8.34041,2.52788) -- (8.36995,2.52438) -- (8.39948,2.52089) -- (8.42901,2.51741) -- (8.45854,2.51392)
 -- (8.48807,2.51045) -- (8.5176,2.50697) -- (8.54714,2.50349) -- (8.57667,2.50002) -- (8.6062,2.49655) -- (8.63573,2.49309) -- (8.66526,2.48962) -- (8.6948,2.48616) -- (8.72433,2.48269) -- (8.75386,2.47923) -- (8.78339,2.47577) -- (8.81292,2.47231)
 -- (8.84245,2.46884) -- (8.87199,2.46538) -- (8.90152,2.46192) -- (8.93105,2.45845) -- (8.96058,2.45499) -- (8.99011,2.45152) -- (9.01964,2.44805) -- (9.04918,2.44457) -- (9.07871,2.4411) -- (9.10824,2.43762) -- (9.13777,2.43413) -- (9.1673,2.43064)
 -- (9.19684,2.42715) -- (9.22637,2.42365) -- (9.2559,2.42015) -- (9.28543,2.41664) -- (9.31496,2.41313) -- (9.34449,2.4096) -- (9.37403,2.40607) -- (9.40356,2.40253) -- (9.43309,2.39899) -- (9.46262,2.39543) -- (9.49215,2.39187) -- (9.52169,2.3883)
 -- (9.55122,2.38471) -- (9.58075,2.38112) -- (9.61028,2.37752) -- (9.63981,2.3739) -- (9.66934,2.37027) -- (9.69888,2.36663) -- (9.72841,2.36298) -- (9.75794,2.35931) -- (9.78747,2.35564) -- (9.817,2.35194) -- (9.84653,2.34824) -- (9.87607,2.34452)
 -- (9.9056,2.32667) -- (9.93513,0.596817) -- (9.93513,0.596817) -- (9.9056,2.32667) -- (9.87607,2.31564) -- (9.84653,2.3194) -- (9.817,2.32313) -- (9.78747,2.32685) -- (9.75794,2.33056) -- (9.72841,2.33425) -- (9.69888,2.33793) -- (9.66934,2.3416)
 -- (9.63981,2.34526) -- (9.61028,2.34891) -- (9.58075,2.35256) -- (9.55122,2.35619) -- (9.52169,2.35983) -- (9.49215,2.36346) -- (9.46262,2.36708) -- (9.43309,2.3707) -- (9.40356,2.37432) -- (9.37403,2.37794) -- (9.34449,2.38155) --
 (9.31496,2.38516) -- (9.28543,2.38878) -- (9.2559,2.39239) -- (9.22637,2.396) -- (9.19684,2.39962) -- (9.1673,2.40323) -- (9.13777,2.40684) -- (9.10824,2.41046) -- (9.07871,2.41408) -- (9.04918,2.4177) -- (9.01964,2.42132) -- (8.99011,2.42494) --
 (8.96058,2.42857) -- (8.93105,2.4322) -- (8.90152,2.43583) -- (8.87199,2.43946) -- (8.84245,2.4431) -- (8.81292,2.44673) -- (8.78339,2.45038) -- (8.75386,2.45402) -- (8.72433,2.45767) -- (8.6948,2.46132) -- (8.66526,2.46498) -- (8.63573,2.46863) --
 (8.6062,2.4723) -- (8.57667,2.47596) -- (8.54714,2.47963) -- (8.5176,2.4833) -- (8.48807,2.48698) -- (8.45854,2.49066) -- (8.42901,2.49435) -- (8.39948,2.49804) -- (8.36995,2.50173) -- (8.34041,2.50543) -- (8.31088,2.50913) -- (8.28135,2.51284) --
 (8.25182,2.51655) -- (8.22229,2.52027) -- (8.19276,2.52399) -- (8.16322,2.52771) -- (8.13369,2.53145) -- (8.10416,2.53518) -- (8.07463,2.53892) -- (8.0451,2.54267) -- (8.01556,2.54643) -- (7.98603,2.55019) -- (7.9565,2.55395) -- (7.92697,2.55772) --
 (7.89744,2.5615) -- (7.86791,2.56528) -- (7.83837,2.56907) -- (7.80884,2.57287) -- (7.77931,2.57668) -- (7.74978,2.58049) -- (7.72025,2.58431) -- (7.69072,2.58814) -- (7.66118,2.59197) -- (7.63165,2.59581) -- (7.60212,2.59967) -- (7.57259,2.60353)
 -- (7.54306,2.60739) -- (7.51352,2.61127) -- (7.48399,2.61516) -- (7.45446,2.61906) -- (7.42493,2.62296) -- (7.3954,2.62688) -- (7.36587,2.63081) -- (7.33633,2.63475) -- (7.3068,2.6387) -- (7.27727,2.64266) -- (7.24774,2.64664) -- (7.21821,2.65063)
 -- (7.18867,2.65463) -- (7.15914,2.65864) -- (7.12961,2.66267) -- (7.10008,2.66671) -- (7.07055,2.67076) -- (7.04102,2.67483) -- (7.01148,2.67892) -- (6.98195,2.68302) -- (6.95242,2.68714) -- (6.92289,2.69128) -- (6.89336,2.69543) --
 (6.86383,2.6996) -- (6.83429,2.7038) -- (6.80476,2.70801) -- (6.77523,2.71224) -- (6.7457,2.71649) -- (6.71617,2.72077) -- (6.68663,2.72506) -- (6.6571,2.72938) -- (6.62757,2.73373) -- (6.59804,2.7381) -- (6.56851,2.74249) -- (6.53898,2.74691) --
 (6.50944,2.75136) -- (6.47991,2.75584) -- (6.45038,2.76034) -- (6.42085,2.76488) -- (6.39132,2.76945) -- (6.36179,2.77405) -- (6.33225,2.77868) -- (6.30272,2.78335) -- (6.27319,2.78805) -- (6.24366,2.79279) -- (6.21413,2.79757) -- (6.18459,2.80239)
 -- (6.15506,2.80725) -- (6.12553,2.81215) -- (6.096,2.81709) -- (6.06647,2.82208) -- (6.03694,2.82711) -- (6.0074,2.8322) -- (5.97787,2.83733) -- (5.94834,2.84251) -- (5.91881,2.84774) -- (5.88928,2.85303) -- (5.85975,2.85838) -- (5.83021,2.86378)
 -- (5.80068,2.86924) -- (5.77115,2.87476) -- (5.74162,2.88034) -- (5.71209,2.88599) -- (5.68255,2.89171) -- (5.65302,2.89749) -- (5.62349,2.90334) -- (5.59396,2.90927) -- (5.56443,2.91527) -- (5.5349,2.92134) -- (5.50536,2.9275) -- (5.47583,2.93373)
 -- (5.4463,2.94005) -- (5.41677,2.94645) -- (5.38724,2.95294) -- (5.3577,2.95952) -- (5.32817,2.96619) -- (5.29864,2.97296) -- (5.26911,2.97982) -- (5.23958,2.98678) -- (5.21005,2.99385) -- (5.18051,3.00101) -- (5.15098,3.00828) -- (5.12145,3.01567)
 -- (5.09192,3.02316) -- (5.06239,3.03076) -- (5.03286,3.03849) -- (5.00332,3.04633) -- (4.97379,3.05429) -- (4.94426,3.06238) -- (4.91473,3.07059) -- (4.8852,3.07893) -- (4.85566,3.0874) -- (4.82613,3.09601) -- (4.7966,3.10475) -- (4.76707,3.11363)
 -- (4.73754,3.12265) -- (4.70801,3.13182) -- (4.67847,3.14113) -- (4.64894,3.15059) -- (4.61941,3.1602) -- (4.58988,3.16997) -- (4.56035,3.17988) -- (4.53082,3.18996) -- (4.50128,3.2002) -- (4.47175,3.2106) -- (4.44222,3.22116) -- (4.41269,3.23189)
 -- (4.38316,3.24279) -- (4.35362,3.25386) -- (4.32409,3.26511) -- (4.29456,3.27652) -- (4.26503,3.28812) -- (4.2355,3.29989) -- (4.20597,3.31184) -- (4.17643,3.32397) -- (4.1469,3.33629) -- (4.11737,3.34879) -- (4.08784,3.36148) -- (4.05831,3.37436)
 -- (4.02877,3.38743) -- (3.99924,3.40069) -- (3.96971,3.41414) -- (3.94018,3.42779) -- (3.91065,3.44164) -- (3.88112,3.45568) -- (3.85158,3.46992) -- (3.82205,3.48436) -- (3.79252,3.49901) -- (3.76299,3.51386) -- (3.73346,3.52891) --
 (3.70393,3.54417) -- (3.67439,3.55963) -- (3.64486,3.57531) -- (3.61533,3.5912) -- (3.5858,3.60729) -- (3.55627,3.6236) -- (3.52673,3.64013) -- (3.4972,3.65687) -- (3.46767,3.67383) -- (3.43814,3.69101) -- (3.40861,3.70841) -- (3.37908,3.72603) --
 (3.34954,3.74388) -- (3.32001,3.76195) -- (3.29048,3.78025) -- (3.26095,3.79878) -- (3.23142,3.81755) -- (3.20189,3.83654) -- (3.17235,3.85578) -- (3.14282,3.87525) -- (3.11329,3.89496) -- (3.08376,3.91492) -- (3.05423,3.93513) -- (3.02469,3.95559)
 -- (2.99516,3.97629) -- (2.96563,3.99726) -- (2.9361,4.01849) -- (2.90657,4.03998) -- (2.87704,4.06173) -- (2.8475,4.08376) -- (2.81797,4.10607) -- (2.78844,4.12866) -- (2.75891,4.15154) -- (2.72938,4.1747) -- (2.69984,4.19817) -- (2.67031,4.22194)
 -- (2.64078,4.24602) -- (2.61125,4.27042) -- (2.58172,4.29515) -- (2.55219,4.32021) -- (2.52265,4.34561) -- (2.49312,4.37136) -- (2.46359,4.39747) -- (2.43406,4.42396) -- (2.40453,4.45083) -- (2.375,4.47809) -- (2.34546,4.50575) -- (2.31593,4.53384)
 -- (2.2864,4.56236) -- (2.25687,4.59133) -- (2.22734,4.62077) -- (2.1978,4.65069) -- (2.16827,4.68111) -- (2.13874,4.71204) -- (2.10921,4.74352) -- (2.07968,4.77557) -- (2.05015,4.8082) -- (2.02061,4.84144) -- (1.99108,4.87533) -- (1.96155,4.90988)
 -- (1.93202,4.94514) -- (1.90249,4.98113) -- (1.87296,5.0179) -- (1.84342,5.05547) -- (1.81389,5.0939) -- (1.78436,5.13322) -- (1.75483,5.17349) -- (1.7253,5.21475) -- (1.69576,5.25706) -- (1.66623,5.30048) -- (1.6367,5.34506) -- (1.60717,5.39088)
 -- (1.57764,5.438) -- (1.54811,5.48651) -- (1.51857,5.53651) -- (1.48904,0.596817) -- (1.45951,0.596817);
\definecolor{c}{rgb}{0,0,0};
\colorlet{c}{natblue!40};
\draw [c, fill=c] (1.45951,0.596817) -- (1.48904,4.38722) -- (1.51857,5.54801) -- (1.54811,5.4971) -- (1.57764,5.44794) -- (1.60717,5.40039) -- (1.6367,5.35433) -- (1.66623,5.30964) -- (1.69576,5.26623) -- (1.7253,5.22401) -- (1.75483,5.1829) --
 (1.78436,5.14283) -- (1.81389,5.10373) -- (1.84342,5.06555) -- (1.87296,5.02825) -- (1.90249,4.99178) -- (1.93202,4.95609) -- (1.96155,4.92115) -- (1.99108,4.88693) -- (2.02061,4.85341) -- (2.05015,4.82054) -- (2.07968,4.78831) -- (2.10921,4.75669)
 -- (2.13874,4.72567) -- (2.16827,4.69522) -- (2.1978,4.66532) -- (2.22734,4.63596) -- (2.25687,4.60712) -- (2.2864,4.5788) -- (2.31593,4.55096) -- (2.34546,4.52361) -- (2.375,4.49674) -- (2.40453,4.47032) -- (2.43406,4.44436) -- (2.46359,4.41883) --
 (2.49312,4.39374) -- (2.52265,4.36908) -- (2.55219,4.34483) -- (2.58172,4.321) -- (2.61125,4.29757) -- (2.64078,4.27454) -- (2.67031,4.25189) -- (2.69984,4.22964) -- (2.72938,4.20776) -- (2.75891,4.18626) -- (2.78844,4.16512) -- (2.81797,4.14436) --
 (2.8475,4.12395) -- (2.87704,4.10389) -- (2.90657,4.08419) -- (2.9361,4.06484) -- (2.96563,4.04583) -- (2.99516,4.02716) -- (3.02469,4.00883) -- (3.05423,3.99083) -- (3.08376,3.97316) -- (3.11329,3.95582) -- (3.14282,3.93881) -- (3.17235,3.92211) --
 (3.20189,3.90573) -- (3.23142,3.88967) -- (3.26095,3.87392) -- (3.29048,3.85848) -- (3.32001,3.84335) -- (3.34954,3.82852) -- (3.37908,3.81399) -- (3.40861,3.79976) -- (3.43814,3.78582) -- (3.46767,3.77218) -- (3.4972,3.75882) -- (3.52673,3.74575)
 -- (3.55627,3.73297) -- (3.5858,3.72046) -- (3.61533,3.70823) -- (3.64486,3.69627) -- (3.67439,3.68459) -- (3.70393,3.67317) -- (3.73346,3.66201) -- (3.76299,3.65112) -- (3.79252,3.64047) -- (3.82205,3.63009) -- (3.85158,3.61995) --
 (3.88112,3.61005) -- (3.91065,3.6004) -- (3.94018,3.59099) -- (3.96971,3.58181) -- (3.99924,3.57285) -- (4.02877,3.56413) -- (4.05831,3.55563) -- (4.08784,3.54735) -- (4.11737,3.53928) -- (4.1469,3.53142) -- (4.17643,3.52378) -- (4.20597,3.51633) --
 (4.2355,3.50908) -- (4.26503,3.50203) -- (4.29456,3.49518) -- (4.32409,3.48851) -- (4.35362,3.48202) -- (4.38316,3.47572) -- (4.41269,3.46959) -- (4.44222,3.46364) -- (4.47175,3.45785) -- (4.50128,3.45223) -- (4.53082,3.44678) -- (4.56035,3.44148)
 -- (4.58988,3.43634) -- (4.61941,3.43135) -- (4.64894,3.42651) -- (4.67847,3.42181) -- (4.70801,3.41726) -- (4.73754,3.41284) -- (4.76707,3.40856) -- (4.7966,3.40441) -- (4.82613,3.40039) -- (4.85566,3.39649) -- (4.8852,3.39271) -- (4.91473,3.38906)
 -- (4.94426,3.38552) -- (4.97379,3.38209) -- (5.00332,3.37877) -- (5.03286,3.37555) -- (5.06239,3.37244) -- (5.09192,3.36942) -- (5.12145,3.3665) -- (5.15098,3.36368) -- (5.18051,3.36094) -- (5.21005,3.35829) -- (5.23958,3.35573) --
 (5.26911,3.35324) -- (5.29864,3.35084) -- (5.32817,3.34851) -- (5.3577,3.34625) -- (5.38724,3.34406) -- (5.41677,3.34193) -- (5.4463,3.33987) -- (5.47583,3.33787) -- (5.50536,3.33593) -- (5.5349,3.33404) -- (5.56443,3.33221) -- (5.59396,3.33043) --
 (5.62349,3.3287) -- (5.65302,3.32701) -- (5.68255,3.32537) -- (5.71209,3.32377) -- (5.74162,3.32221) -- (5.77115,3.32069) -- (5.80068,3.3192) -- (5.83021,3.31775) -- (5.85975,3.31633) -- (5.88928,3.31494) -- (5.91881,3.31358) -- (5.94834,3.31224) --
 (5.97787,3.31094) -- (6.0074,3.30965) -- (6.03694,3.30838) -- (6.06647,3.30714) -- (6.096,3.30591) -- (6.12553,3.30471) -- (6.15506,3.30351) -- (6.18459,3.30234) -- (6.21413,3.30117) -- (6.24366,3.30002) -- (6.27319,3.29888) -- (6.30272,3.29775) --
 (6.33225,3.29663) -- (6.36179,3.29551) -- (6.39132,3.2944) -- (6.42085,3.2933) -- (6.45038,3.2922) -- (6.47991,3.2911) -- (6.50944,3.29001) -- (6.53898,3.28892) -- (6.56851,3.28783) -- (6.59804,3.28673) -- (6.62757,3.28564) -- (6.6571,3.28455) --
 (6.68663,3.28345) -- (6.71617,3.28235) -- (6.7457,3.28124) -- (6.77523,3.28013) -- (6.80476,3.27902) -- (6.83429,3.2779) -- (6.86383,3.27677) -- (6.89336,3.27563) -- (6.92289,3.27449) -- (6.95242,3.27334) -- (6.98195,3.27217) -- (7.01148,3.271) --
 (7.04102,3.26982) -- (7.07055,3.26863) -- (7.10008,3.26742) -- (7.12961,3.2662) -- (7.15914,3.26497) -- (7.18867,3.26373) -- (7.21821,3.26248) -- (7.24774,3.26121) -- (7.27727,3.25992) -- (7.3068,3.25862) -- (7.33633,3.25731) -- (7.36587,3.25598) --
 (7.3954,3.25464) -- (7.42493,3.25327) -- (7.45446,3.25189) -- (7.48399,3.2505) -- (7.51352,3.24908) -- (7.54306,3.24765) -- (7.57259,3.2462) -- (7.60212,3.24473) -- (7.63165,3.24324) -- (7.66118,3.24173) -- (7.69072,3.2402) -- (7.72025,3.23865) --
 (7.74978,3.23708) -- (7.77931,3.23549) -- (7.80884,3.23388) -- (7.83837,3.23224) -- (7.86791,3.23059) -- (7.89744,3.22891) -- (7.92697,3.2272) -- (7.9565,3.22548) -- (7.98603,3.22373) -- (8.01556,3.22195) -- (8.0451,3.22015) -- (8.07463,3.21833) --
 (8.10416,3.21648) -- (8.13369,3.2146) -- (8.16322,3.2127) -- (8.19276,3.21078) -- (8.22229,3.20882) -- (8.25182,3.20684) -- (8.28135,3.20484) -- (8.31088,3.2028) -- (8.34041,3.20074) -- (8.36995,3.19865) -- (8.39948,3.19653) -- (8.42901,3.19439) --
 (8.45854,3.19221) -- (8.48807,3.19) -- (8.5176,3.18777) -- (8.54714,3.18551) -- (8.57667,3.18321) -- (8.6062,3.18089) -- (8.63573,3.17854) -- (8.66526,3.17615) -- (8.6948,3.17374) -- (8.72433,3.17129) -- (8.75386,3.16881) -- (8.78339,3.1663) --
 (8.81292,3.16376) -- (8.84245,3.16119) -- (8.87199,3.15859) -- (8.90152,3.15595) -- (8.93105,3.15328) -- (8.96058,3.15058) -- (8.99011,3.14785) -- (9.01964,3.14508) -- (9.04918,3.14228) -- (9.07871,3.13945) -- (9.10824,3.13658) -- (9.13777,3.13368)
 -- (9.1673,3.13075) -- (9.19684,3.12779) -- (9.22637,3.12479) -- (9.2559,3.12176) -- (9.28543,3.11869) -- (9.31496,3.11559) -- (9.34449,3.11245) -- (9.37403,3.10928) -- (9.40356,3.10608) -- (9.43309,3.10285) -- (9.46262,3.09957) -- (9.49215,3.09627)
 -- (9.52169,3.09293) -- (9.55122,3.08955) -- (9.58075,3.08615) -- (9.61028,3.0827) -- (9.63981,3.07922) -- (9.66934,3.07571) -- (9.69888,3.07217) -- (9.72841,3.06858) -- (9.75794,3.06497) -- (9.78747,3.06132) -- (9.817,3.05763) -- (9.84653,3.05391)
 -- (9.87607,3.05016) -- (9.9056,3.03778) -- (9.93513,0.596817) -- (9.93513,0.596817) -- (9.9056,3.03778) -- (9.87607,3.03276) -- (9.84653,3.03655) -- (9.817,3.0403) -- (9.78747,3.04402) -- (9.75794,3.04771) -- (9.72841,3.05136) -- (9.69888,3.05498)
 -- (9.66934,3.05857) -- (9.63981,3.06212) -- (9.61028,3.06564) -- (9.58075,3.06912) -- (9.55122,3.07257) -- (9.52169,3.07599) -- (9.49215,3.07937) -- (9.46262,3.08273) -- (9.43309,3.08605) -- (9.40356,3.08933) -- (9.37403,3.09258) --
 (9.34449,3.0958) -- (9.31496,3.09899) -- (9.28543,3.10215) -- (9.2559,3.10527) -- (9.22637,3.10836) -- (9.19684,3.11142) -- (9.1673,3.11445) -- (9.13777,3.11744) -- (9.10824,3.1204) -- (9.07871,3.12333) -- (9.04918,3.12623) -- (9.01964,3.1291) --
 (8.99011,3.13194) -- (8.96058,3.13474) -- (8.93105,3.13751) -- (8.90152,3.14025) -- (8.87199,3.14296) -- (8.84245,3.14564) -- (8.81292,3.14829) -- (8.78339,3.15091) -- (8.75386,3.1535) -- (8.72433,3.15605) -- (8.6948,3.15858) -- (8.66526,3.16107) --
 (8.63573,3.16354) -- (8.6062,3.16597) -- (8.57667,3.16837) -- (8.54714,3.17075) -- (8.5176,3.17309) -- (8.48807,3.1754) -- (8.45854,3.17769) -- (8.42901,3.17994) -- (8.39948,3.18216) -- (8.36995,3.18436) -- (8.34041,3.18653) -- (8.31088,3.18866) --
 (8.28135,3.19077) -- (8.25182,3.19285) -- (8.22229,3.1949) -- (8.19276,3.19692) -- (8.16322,3.19891) -- (8.13369,3.20088) -- (8.10416,3.20281) -- (8.07463,3.20472) -- (8.0451,3.2066) -- (8.01556,3.20845) -- (7.98603,3.21028) -- (7.9565,3.21208) --
 (7.92697,3.21385) -- (7.89744,3.21559) -- (7.86791,3.21731) -- (7.83837,3.219) -- (7.80884,3.22067) -- (7.77931,3.22231) -- (7.74978,3.22392) -- (7.72025,3.22552) -- (7.69072,3.22708) -- (7.66118,3.22862) -- (7.63165,3.23014) -- (7.60212,3.23163) --
 (7.57259,3.2331) -- (7.54306,3.23455) -- (7.51352,3.23598) -- (7.48399,3.23738) -- (7.45446,3.23876) -- (7.42493,3.24012) -- (7.3954,3.24146) -- (7.36587,3.24279) -- (7.33633,3.24409) -- (7.3068,3.24537) -- (7.27727,3.24663) -- (7.24774,3.24788) --
 (7.21821,3.24911) -- (7.18867,3.25032) -- (7.15914,3.25152) -- (7.12961,3.25271) -- (7.10008,3.25387) -- (7.07055,3.25503) -- (7.04102,3.25617) -- (7.01148,3.2573) -- (6.98195,3.25842) -- (6.95242,3.25953) -- (6.92289,3.26062) -- (6.89336,3.26171)
 -- (6.86383,3.26279) -- (6.83429,3.26387) -- (6.80476,3.26493) -- (6.77523,3.266) -- (6.7457,3.26705) -- (6.71617,3.26811) -- (6.68663,3.26916) -- (6.6571,3.2702) -- (6.62757,3.27125) -- (6.59804,3.2723) -- (6.56851,3.27335) -- (6.53898,3.2744) --
 (6.50944,3.27546) -- (6.47991,3.27652) -- (6.45038,3.27758) -- (6.42085,3.27865) -- (6.39132,3.27973) -- (6.36179,3.28082) -- (6.33225,3.28192) -- (6.30272,3.28303) -- (6.27319,3.28415) -- (6.24366,3.28529) -- (6.21413,3.28644) -- (6.18459,3.28761)
 -- (6.15506,3.2888) -- (6.12553,3.29) -- (6.096,3.29123) -- (6.06647,3.29247) -- (6.03694,3.29374) -- (6.0074,3.29504) -- (5.97787,3.29636) -- (5.94834,3.29771) -- (5.91881,3.29909) -- (5.88928,3.3005) -- (5.85975,3.30194) -- (5.83021,3.30341) --
 (5.80068,3.30492) -- (5.77115,3.30647) -- (5.74162,3.30805) -- (5.71209,3.30968) -- (5.68255,3.31134) -- (5.65302,3.31305) -- (5.62349,3.31481) -- (5.59396,3.31661) -- (5.56443,3.31847) -- (5.5349,3.32037) -- (5.50536,3.32232) -- (5.47583,3.32433)
 -- (5.4463,3.3264) -- (5.41677,3.32853) -- (5.38724,3.33071) -- (5.3577,3.33296) -- (5.32817,3.33528) -- (5.29864,3.33766) -- (5.26911,3.34011) -- (5.23958,3.34263) -- (5.21005,3.34523) -- (5.18051,3.3479) -- (5.15098,3.35065) -- (5.12145,3.35349)
 -- (5.09192,3.35641) -- (5.06239,3.35941) -- (5.03286,3.36251) -- (5.00332,3.3657) -- (4.97379,3.36898) -- (4.94426,3.37237) -- (4.91473,3.37585) -- (4.8852,3.37945) -- (4.85566,3.38315) -- (4.82613,3.38697) -- (4.7966,3.3909) -- (4.76707,3.39496)
 -- (4.73754,3.39913) -- (4.70801,3.40344) -- (4.67847,3.40788) -- (4.64894,3.41246) -- (4.61941,3.41718) -- (4.58988,3.42204) -- (4.56035,3.42705) -- (4.53082,3.43222) -- (4.50128,3.43754) -- (4.47175,3.44303) -- (4.44222,3.44869) --
 (4.41269,3.45451) -- (4.38316,3.46052) -- (4.35362,3.4667) -- (4.32409,3.47307) -- (4.29456,3.47963) -- (4.26503,3.48639) -- (4.2355,3.49334) -- (4.20597,3.5005) -- (4.17643,3.50787) -- (4.1469,3.51545) -- (4.11737,3.52325) -- (4.08784,3.53127) --
 (4.05831,3.53951) -- (4.02877,3.54799) -- (3.99924,3.5567) -- (3.96971,3.56565) -- (3.94018,3.57484) -- (3.91065,3.58429) -- (3.88112,3.59398) -- (3.85158,3.60392) -- (3.82205,3.61413) -- (3.79252,3.6246) -- (3.76299,3.63533) -- (3.73346,3.64634) --
 (3.70393,3.65762) -- (3.67439,3.66917) -- (3.64486,3.681) -- (3.61533,3.69312) -- (3.5858,3.70552) -- (3.55627,3.71821) -- (3.52673,3.73119) -- (3.4972,3.74446) -- (3.46767,3.75803) -- (3.43814,3.7719) -- (3.40861,3.78607) -- (3.37908,3.80055) --
 (3.34954,3.81533) -- (3.32001,3.83041) -- (3.29048,3.84581) -- (3.26095,3.86152) -- (3.23142,3.87754) -- (3.20189,3.89388) -- (3.17235,3.91054) -- (3.14282,3.92751) -- (3.11329,3.94481) -- (3.08376,3.96243) -- (3.05423,3.98038) -- (3.02469,3.99865)
 -- (2.99516,4.01726) -- (2.96563,4.0362) -- (2.9361,4.05547) -- (2.90657,4.07508) -- (2.87704,4.09502) -- (2.8475,4.11531) -- (2.81797,4.13595) -- (2.78844,4.15693) -- (2.75891,4.17827) -- (2.72938,4.19996) -- (2.69984,4.22201) -- (2.67031,4.24443)
 -- (2.64078,4.26721) -- (2.61125,4.29037) -- (2.58172,4.3139) -- (2.55219,4.33783) -- (2.52265,4.36214) -- (2.49312,4.38686) -- (2.46359,4.41198) -- (2.43406,4.43752) -- (2.40453,4.46349) -- (2.375,4.4899) -- (2.34546,4.51675) -- (2.31593,4.54407)
 -- (2.2864,4.57186) -- (2.25687,4.60013) -- (2.22734,4.62891) -- (2.1978,4.65821) -- (2.16827,4.68804) -- (2.13874,4.71843) -- (2.10921,4.74939) -- (2.07968,4.78094) -- (2.05015,4.81312) -- (2.02061,4.84593) -- (1.99108,4.87942) -- (1.96155,4.9136)
 -- (1.93202,4.94851) -- (1.90249,4.98418) -- (1.87296,5.02064) -- (1.84342,5.05794) -- (1.81389,5.09611) -- (1.78436,5.1352) -- (1.75483,5.17525) -- (1.7253,5.21632) -- (1.69576,5.25845) -- (1.66623,5.3017) -- (1.6367,5.34613) -- (1.60717,5.39181)
 -- (1.57764,5.43881) -- (1.54811,5.48721) -- (1.51857,5.53712) -- (1.48904,0.596817) -- (1.45951,0.596817);
\colorlet{c}{kugray};
\draw [c] (1.13336,3.41516) -- (1.13336,3.80777);
\draw [c] (1.13336,3.80777) -- (1.13336,3.97485);
\draw [c] (1.11854,3.80777) -- (1.13336,3.80777);
\draw [c] (1.13336,3.80777) -- (1.14818,3.80777);
\definecolor{c}{rgb}{0,0,0};
\colorlet{c}{kugray};
\draw [c] (1.163,3.33018) -- (1.163,3.70859);
\draw [c] (1.163,3.70859) -- (1.163,3.87329);
\draw [c] (1.14818,3.70859) -- (1.163,3.70859);
\draw [c] (1.163,3.70859) -- (1.17781,3.70859);
\definecolor{c}{rgb}{0,0,0};
\colorlet{c}{kugray};
\draw [c] (1.22227,3.26026) -- (1.22227,3.80576);
\draw [c] (1.22227,3.80576) -- (1.22227,3.99181);
\draw [c] (1.20745,3.80576) -- (1.22227,3.80576);
\draw [c] (1.22227,3.80576) -- (1.23709,3.80576);
\definecolor{c}{rgb}{0,0,0};
\colorlet{c}{kugray};
\draw [c] (1.2519,3.59515) -- (1.2519,3.86294);
\draw [c] (1.2519,3.86294) -- (1.2519,4.00395);
\draw [c] (1.23709,3.86294) -- (1.2519,3.86294);
\draw [c] (1.2519,3.86294) -- (1.26672,3.86294);
\definecolor{c}{rgb}{0,0,0};
\colorlet{c}{kugray};
\draw [c] (1.28154,3.81131) -- (1.28154,4.05468);
\draw [c] (1.28154,4.05468) -- (1.28154,4.18889);
\draw [c] (1.26672,4.05468) -- (1.28154,4.05468);
\draw [c] (1.28154,4.05468) -- (1.29636,4.05468);
\definecolor{c}{rgb}{0,0,0};
\colorlet{c}{kugray};
\draw [c] (1.31118,5.36749) -- (1.31118,5.38512);
\draw [c] (1.31118,5.38512) -- (1.31118,5.40179);
\draw [c] (1.29636,5.38512) -- (1.31118,5.38512);
\draw [c] (1.31118,5.38512) -- (1.32599,5.38512);
\definecolor{c}{rgb}{0,0,0};
\colorlet{c}{kugray};
\draw [c] (1.34081,5.61993) -- (1.34081,5.63126);
\draw [c] (1.34081,5.63126) -- (1.34081,5.6422);
\draw [c] (1.32599,5.63126) -- (1.34081,5.63126);
\draw [c] (1.34081,5.63126) -- (1.35563,5.63126);
\definecolor{c}{rgb}{0,0,0};
\colorlet{c}{kugray};
\draw [c] (1.37045,5.69182) -- (1.37045,5.70194);
\draw [c] (1.37045,5.70194) -- (1.37045,5.71174);
\draw [c] (1.35563,5.70194) -- (1.37045,5.70194);
\draw [c] (1.37045,5.70194) -- (1.38526,5.70194);
\definecolor{c}{rgb}{0,0,0};
\colorlet{c}{kugray};
\draw [c] (1.40008,5.67982) -- (1.40008,5.69024);
\draw [c] (1.40008,5.69024) -- (1.40008,5.70031);
\draw [c] (1.38526,5.69024) -- (1.40008,5.69024);
\draw [c] (1.40008,5.69024) -- (1.4149,5.69024);
\definecolor{c}{rgb}{0,0,0};
\colorlet{c}{kugray};
\draw [c] (1.42972,5.65827) -- (1.42972,5.66928);
\draw [c] (1.42972,5.66928) -- (1.42972,5.67991);
\draw [c] (1.4149,5.66928) -- (1.42972,5.66928);
\draw [c] (1.42972,5.66928) -- (1.44454,5.66928);
\definecolor{c}{rgb}{0,0,0};
\colorlet{c}{kugray};
\draw [c] (1.45935,5.62) -- (1.45935,5.63141);
\draw [c] (1.45935,5.63141) -- (1.45935,5.6424);
\draw [c] (1.44454,5.63141) -- (1.45935,5.63141);
\draw [c] (1.45935,5.63141) -- (1.47417,5.63141);
\definecolor{c}{rgb}{0,0,0};
\colorlet{c}{kugray};
\draw [c] (1.48899,5.58164) -- (1.48899,5.5938);
\draw [c] (1.48899,5.5938) -- (1.48899,5.60549);
\draw [c] (1.47417,5.5938) -- (1.48899,5.5938);
\draw [c] (1.48899,5.5938) -- (1.50381,5.5938);
\definecolor{c}{rgb}{0,0,0};
\colorlet{c}{kugray};
\draw [c] (1.51863,5.53046) -- (1.51863,5.54392);
\draw [c] (1.51863,5.54392) -- (1.51863,5.55682);
\draw [c] (1.50381,5.54392) -- (1.51863,5.54392);
\draw [c] (1.51863,5.54392) -- (1.53344,5.54392);
\definecolor{c}{rgb}{0,0,0};
\colorlet{c}{kugray};
\draw [c] (1.54826,5.46022) -- (1.54826,5.47505);
\draw [c] (1.54826,5.47505) -- (1.54826,5.48921);
\draw [c] (1.53344,5.47505) -- (1.54826,5.47505);
\draw [c] (1.54826,5.47505) -- (1.56308,5.47505);
\definecolor{c}{rgb}{0,0,0};
\colorlet{c}{kugray};
\draw [c] (1.5779,5.41506) -- (1.5779,5.43093);
\draw [c] (1.5779,5.43093) -- (1.5779,5.44602);
\draw [c] (1.56308,5.43093) -- (1.5779,5.43093);
\draw [c] (1.5779,5.43093) -- (1.59272,5.43093);
\definecolor{c}{rgb}{0,0,0};
\colorlet{c}{kugray};
\draw [c] (1.60753,5.38705) -- (1.60753,5.40429);
\draw [c] (1.60753,5.40429) -- (1.60753,5.42061);
\draw [c] (1.59272,5.40429) -- (1.60753,5.40429);
\draw [c] (1.60753,5.40429) -- (1.62235,5.40429);
\definecolor{c}{rgb}{0,0,0};
\colorlet{c}{kugray};
\draw [c] (1.63717,5.33118) -- (1.63717,5.34945);
\draw [c] (1.63717,5.34945) -- (1.63717,5.36669);
\draw [c] (1.62235,5.34945) -- (1.63717,5.34945);
\draw [c] (1.63717,5.34945) -- (1.65199,5.34945);
\definecolor{c}{rgb}{0,0,0};
\colorlet{c}{kugray};
\draw [c] (1.6668,5.33501) -- (1.6668,5.35318);
\draw [c] (1.6668,5.35318) -- (1.6668,5.37033);
\draw [c] (1.65199,5.35318) -- (1.6668,5.35318);
\draw [c] (1.6668,5.35318) -- (1.68162,5.35318);
\definecolor{c}{rgb}{0,0,0};
\colorlet{c}{kugray};
\draw [c] (1.69644,5.27038) -- (1.69644,5.29099);
\draw [c] (1.69644,5.29099) -- (1.69644,5.3103);
\draw [c] (1.68162,5.29099) -- (1.69644,5.29099);
\draw [c] (1.69644,5.29099) -- (1.71126,5.29099);
\definecolor{c}{rgb}{0,0,0};
\colorlet{c}{kugray};
\draw [c] (1.72608,5.20804) -- (1.72608,5.23001);
\draw [c] (1.72608,5.23001) -- (1.72608,5.25052);
\draw [c] (1.71126,5.23001) -- (1.72608,5.23001);
\draw [c] (1.72608,5.23001) -- (1.74089,5.23001);
\definecolor{c}{rgb}{0,0,0};
\colorlet{c}{kugray};
\draw [c] (1.75571,5.08901) -- (1.75571,5.11549);
\draw [c] (1.75571,5.11549) -- (1.75571,5.13987);
\draw [c] (1.74089,5.11549) -- (1.75571,5.11549);
\draw [c] (1.75571,5.11549) -- (1.77053,5.11549);
\definecolor{c}{rgb}{0,0,0};
\colorlet{c}{kugray};
\draw [c] (1.78535,5.10181) -- (1.78535,5.12931);
\draw [c] (1.78535,5.12931) -- (1.78535,5.15455);
\draw [c] (1.77053,5.12931) -- (1.78535,5.12931);
\draw [c] (1.78535,5.12931) -- (1.80017,5.12931);
\definecolor{c}{rgb}{0,0,0};
\colorlet{c}{kugray};
\draw [c] (1.81498,5.06396) -- (1.81498,5.09313);
\draw [c] (1.81498,5.09313) -- (1.81498,5.11978);
\draw [c] (1.80017,5.09313) -- (1.81498,5.09313);
\draw [c] (1.81498,5.09313) -- (1.8298,5.09313);
\definecolor{c}{rgb}{0,0,0};
\colorlet{c}{kugray};
\draw [c] (1.84462,5.06537) -- (1.84462,5.09456);
\draw [c] (1.84462,5.09456) -- (1.84462,5.12123);
\draw [c] (1.8298,5.09456) -- (1.84462,5.09456);
\draw [c] (1.84462,5.09456) -- (1.85944,5.09456);
\definecolor{c}{rgb}{0,0,0};
\colorlet{c}{kugray};
\draw [c] (1.87425,4.99725) -- (1.87425,5.02835);
\draw [c] (1.87425,5.02835) -- (1.87425,5.05659);
\draw [c] (1.85944,5.02835) -- (1.87425,5.02835);
\draw [c] (1.87425,5.02835) -- (1.88907,5.02835);
\definecolor{c}{rgb}{0,0,0};
\colorlet{c}{kugray};
\draw [c] (1.90389,4.95832) -- (1.90389,4.99102);
\draw [c] (1.90389,4.99102) -- (1.90389,5.02058);
\draw [c] (1.88907,4.99102) -- (1.90389,4.99102);
\draw [c] (1.90389,4.99102) -- (1.91871,4.99102);
\definecolor{c}{rgb}{0,0,0};
\colorlet{c}{kugray};
\draw [c] (1.93353,4.85544) -- (1.93353,4.89383);
\draw [c] (1.93353,4.89383) -- (1.93353,4.92796);
\draw [c] (1.91871,4.89383) -- (1.93353,4.89383);
\draw [c] (1.93353,4.89383) -- (1.94834,4.89383);
\definecolor{c}{rgb}{0,0,0};
\colorlet{c}{kugray};
\draw [c] (1.96316,4.88333) -- (1.96316,4.92161);
\draw [c] (1.96316,4.92161) -- (1.96316,4.95565);
\draw [c] (1.94834,4.92161) -- (1.96316,4.92161);
\draw [c] (1.96316,4.92161) -- (1.97798,4.92161);
\definecolor{c}{rgb}{0,0,0};
\colorlet{c}{kugray};
\draw [c] (1.9928,4.85825) -- (1.9928,4.89733);
\draw [c] (1.9928,4.89733) -- (1.9928,4.93201);
\draw [c] (1.97798,4.89733) -- (1.9928,4.89733);
\draw [c] (1.9928,4.89733) -- (2.00762,4.89733);
\definecolor{c}{rgb}{0,0,0};
\colorlet{c}{kugray};
\draw [c] (2.02243,4.78026) -- (2.02243,4.82199);
\draw [c] (2.02243,4.82199) -- (2.02243,4.85874);
\draw [c] (2.00762,4.82199) -- (2.02243,4.82199);
\draw [c] (2.02243,4.82199) -- (2.03725,4.82199);
\definecolor{c}{rgb}{0,0,0};
\colorlet{c}{kugray};
\draw [c] (2.05207,4.74394) -- (2.05207,4.7911);
\draw [c] (2.05207,4.7911) -- (2.05207,4.83199);
\draw [c] (2.03725,4.7911) -- (2.05207,4.7911);
\draw [c] (2.05207,4.7911) -- (2.06689,4.7911);
\definecolor{c}{rgb}{0,0,0};
\colorlet{c}{kugray};
\draw [c] (2.08171,4.69059) -- (2.08171,4.74618);
\draw [c] (2.08171,4.74618) -- (2.08171,4.79324);
\draw [c] (2.06689,4.74618) -- (2.08171,4.74618);
\draw [c] (2.08171,4.74618) -- (2.09652,4.74618);
\definecolor{c}{rgb}{0,0,0};
\colorlet{c}{kugray};
\draw [c] (2.11134,4.64498) -- (2.11134,4.70439);
\draw [c] (2.11134,4.70439) -- (2.11134,4.75417);
\draw [c] (2.09652,4.70439) -- (2.11134,4.70439);
\draw [c] (2.11134,4.70439) -- (2.12616,4.70439);
\definecolor{c}{rgb}{0,0,0};
\colorlet{c}{kugray};
\draw [c] (2.14098,4.66916) -- (2.14098,4.726);
\draw [c] (2.14098,4.726) -- (2.14098,4.77396);
\draw [c] (2.12616,4.726) -- (2.14098,4.726);
\draw [c] (2.14098,4.726) -- (2.15579,4.726);
\definecolor{c}{rgb}{0,0,0};
\colorlet{c}{kugray};
\draw [c] (2.17061,4.71202) -- (2.17061,4.76221);
\draw [c] (2.17061,4.76221) -- (2.17061,4.80536);
\draw [c] (2.15579,4.76221) -- (2.17061,4.76221);
\draw [c] (2.17061,4.76221) -- (2.18543,4.76221);
\definecolor{c}{rgb}{0,0,0};
\colorlet{c}{kugray};
\draw [c] (2.20025,4.43289) -- (2.20025,4.50711);
\draw [c] (2.20025,4.50711) -- (2.20025,4.56686);
\draw [c] (2.18543,4.50711) -- (2.20025,4.50711);
\draw [c] (2.20025,4.50711) -- (2.21507,4.50711);
\definecolor{c}{rgb}{0,0,0};
\colorlet{c}{kugray};
\draw [c] (2.22988,4.60564) -- (2.22988,4.66562);
\draw [c] (2.22988,4.66562) -- (2.22988,4.7158);
\draw [c] (2.21507,4.66562) -- (2.22988,4.66562);
\draw [c] (2.22988,4.66562) -- (2.2447,4.66562);
\definecolor{c}{rgb}{0,0,0};
\colorlet{c}{kugray};
\draw [c] (2.25952,4.64159) -- (2.25952,4.70326);
\draw [c] (2.25952,4.70326) -- (2.25952,4.75461);
\draw [c] (2.2447,4.70326) -- (2.25952,4.70326);
\draw [c] (2.25952,4.70326) -- (2.27434,4.70326);
\definecolor{c}{rgb}{0,0,0};
\colorlet{c}{kugray};
\draw [c] (2.28916,4.56718) -- (2.28916,4.57808);
\draw [c] (2.28916,4.57808) -- (2.28916,4.5886);
\draw [c] (2.27434,4.57808) -- (2.28916,4.57808);
\draw [c] (2.28916,4.57808) -- (2.30397,4.57808);
\definecolor{c}{rgb}{0,0,0};
\colorlet{c}{kugray};
\draw [c] (2.31879,4.51802) -- (2.31879,4.5299);
\draw [c] (2.31879,4.5299) -- (2.31879,4.54134);
\draw [c] (2.30397,4.5299) -- (2.31879,4.5299);
\draw [c] (2.31879,4.5299) -- (2.33361,4.5299);
\definecolor{c}{rgb}{0,0,0};
\colorlet{c}{kugray};
\draw [c] (2.34843,4.51942) -- (2.34843,4.53112);
\draw [c] (2.34843,4.53112) -- (2.34843,4.54238);
\draw [c] (2.33361,4.53112) -- (2.34843,4.53112);
\draw [c] (2.34843,4.53112) -- (2.36325,4.53112);
\definecolor{c}{rgb}{0,0,0};
\colorlet{c}{kugray};
\draw [c] (2.37806,4.47939) -- (2.37806,4.492);
\draw [c] (2.37806,4.492) -- (2.37806,4.50411);
\draw [c] (2.36325,4.492) -- (2.37806,4.492);
\draw [c] (2.37806,4.492) -- (2.39288,4.492);
\definecolor{c}{rgb}{0,0,0};
\colorlet{c}{kugray};
\draw [c] (2.4077,4.44091) -- (2.4077,4.45439);
\draw [c] (2.4077,4.45439) -- (2.4077,4.46731);
\draw [c] (2.39288,4.45439) -- (2.4077,4.45439);
\draw [c] (2.4077,4.45439) -- (2.42252,4.45439);
\definecolor{c}{rgb}{0,0,0};
\colorlet{c}{kugray};
\draw [c] (2.43733,4.39585) -- (2.43733,4.41002);
\draw [c] (2.43733,4.41002) -- (2.43733,4.42357);
\draw [c] (2.42252,4.41002) -- (2.43733,4.41002);
\draw [c] (2.43733,4.41002) -- (2.45215,4.41002);
\definecolor{c}{rgb}{0,0,0};
\colorlet{c}{kugray};
\draw [c] (2.46697,4.36562) -- (2.46697,4.38084);
\draw [c] (2.46697,4.38084) -- (2.46697,4.39535);
\draw [c] (2.45215,4.38084) -- (2.46697,4.38084);
\draw [c] (2.46697,4.38084) -- (2.48179,4.38084);
\definecolor{c}{rgb}{0,0,0};
\colorlet{c}{kugray};
\draw [c] (2.49661,4.34442) -- (2.49661,4.3598);
\draw [c] (2.49661,4.3598) -- (2.49661,4.37444);
\draw [c] (2.48179,4.3598) -- (2.49661,4.3598);
\draw [c] (2.49661,4.3598) -- (2.51142,4.3598);
\definecolor{c}{rgb}{0,0,0};
\colorlet{c}{kugray};
\draw [c] (2.52624,4.33876) -- (2.52624,4.3547);
\draw [c] (2.52624,4.3547) -- (2.52624,4.36985);
\draw [c] (2.51142,4.3547) -- (2.52624,4.3547);
\draw [c] (2.52624,4.3547) -- (2.54106,4.3547);
\definecolor{c}{rgb}{0,0,0};
\colorlet{c}{kugray};
\draw [c] (2.55588,4.31164) -- (2.55588,4.32829);
\draw [c] (2.55588,4.32829) -- (2.55588,4.34409);
\draw [c] (2.54106,4.32829) -- (2.55588,4.32829);
\draw [c] (2.55588,4.32829) -- (2.5707,4.32829);
\definecolor{c}{rgb}{0,0,0};
\colorlet{c}{kugray};
\draw [c] (2.58551,4.27319) -- (2.58551,4.29119);
\draw [c] (2.58551,4.29119) -- (2.58551,4.30819);
\draw [c] (2.5707,4.29119) -- (2.58551,4.29119);
\draw [c] (2.58551,4.29119) -- (2.60033,4.29119);
\definecolor{c}{rgb}{0,0,0};
\colorlet{c}{kugray};
\draw [c] (2.61515,4.23142) -- (2.61515,4.25043);
\draw [c] (2.61515,4.25043) -- (2.61515,4.26834);
\draw [c] (2.60033,4.25043) -- (2.61515,4.25043);
\draw [c] (2.61515,4.25043) -- (2.62997,4.25043);
\definecolor{c}{rgb}{0,0,0};
\colorlet{c}{kugray};
\draw [c] (2.64478,4.1941) -- (2.64478,4.21371);
\draw [c] (2.64478,4.21371) -- (2.64478,4.23215);
\draw [c] (2.62997,4.21371) -- (2.64478,4.21371);
\draw [c] (2.64478,4.21371) -- (2.6596,4.21371);
\definecolor{c}{rgb}{0,0,0};
\colorlet{c}{kugray};
\draw [c] (2.67442,4.21383) -- (2.67442,4.23319);
\draw [c] (2.67442,4.23319) -- (2.67442,4.2514);
\draw [c] (2.6596,4.23319) -- (2.67442,4.23319);
\draw [c] (2.67442,4.23319) -- (2.68924,4.23319);
\definecolor{c}{rgb}{0,0,0};
\colorlet{c}{kugray};
\draw [c] (2.70406,4.15434) -- (2.70406,4.17536);
\draw [c] (2.70406,4.17536) -- (2.70406,4.19503);
\draw [c] (2.68924,4.17536) -- (2.70406,4.17536);
\draw [c] (2.70406,4.17536) -- (2.71887,4.17536);
\definecolor{c}{rgb}{0,0,0};
\colorlet{c}{kugray};
\draw [c] (2.73369,4.14693) -- (2.73369,4.16849);
\draw [c] (2.73369,4.16849) -- (2.73369,4.18863);
\draw [c] (2.71887,4.16849) -- (2.73369,4.16849);
\draw [c] (2.73369,4.16849) -- (2.74851,4.16849);
\definecolor{c}{rgb}{0,0,0};
\colorlet{c}{kugray};
\draw [c] (2.76333,4.12489) -- (2.76333,4.1477);
\draw [c] (2.76333,4.1477) -- (2.76333,4.16894);
\draw [c] (2.74851,4.1477) -- (2.76333,4.1477);
\draw [c] (2.76333,4.1477) -- (2.77815,4.1477);
\definecolor{c}{rgb}{0,0,0};
\colorlet{c}{kugray};
\draw [c] (2.79296,4.13183) -- (2.79296,4.15469);
\draw [c] (2.79296,4.15469) -- (2.79296,4.17597);
\draw [c] (2.77815,4.15469) -- (2.79296,4.15469);
\draw [c] (2.79296,4.15469) -- (2.80778,4.15469);
\definecolor{c}{rgb}{0,0,0};
\colorlet{c}{kugray};
\draw [c] (2.8226,4.05797) -- (2.8226,4.08216);
\draw [c] (2.8226,4.08216) -- (2.8226,4.10458);
\draw [c] (2.80778,4.08216) -- (2.8226,4.08216);
\draw [c] (2.8226,4.08216) -- (2.83742,4.08216);
\definecolor{c}{rgb}{0,0,0};
\colorlet{c}{kugray};
\draw [c] (2.85224,4.04617) -- (2.85224,4.07199);
\draw [c] (2.85224,4.07199) -- (2.85224,4.09581);
\draw [c] (2.83742,4.07199) -- (2.85224,4.07199);
\draw [c] (2.85224,4.07199) -- (2.86705,4.07199);
\definecolor{c}{rgb}{0,0,0};
\colorlet{c}{kugray};
\draw [c] (2.88187,4.02533) -- (2.88187,4.05093);
\draw [c] (2.88187,4.05093) -- (2.88187,4.07457);
\draw [c] (2.86705,4.05093) -- (2.88187,4.05093);
\draw [c] (2.88187,4.05093) -- (2.89669,4.05093);
\definecolor{c}{rgb}{0,0,0};
\colorlet{c}{kugray};
\draw [c] (2.91151,4.0461) -- (2.91151,4.07164);
\draw [c] (2.91151,4.07164) -- (2.91151,4.09522);
\draw [c] (2.89669,4.07164) -- (2.91151,4.07164);
\draw [c] (2.91151,4.07164) -- (2.92632,4.07164);
\definecolor{c}{rgb}{0,0,0};
\colorlet{c}{kugray};
\draw [c] (2.94114,3.94743) -- (2.94114,3.97699);
\draw [c] (2.94114,3.97699) -- (2.94114,4.00396);
\draw [c] (2.92632,3.97699) -- (2.94114,3.97699);
\draw [c] (2.94114,3.97699) -- (2.95596,3.97699);
\definecolor{c}{rgb}{0,0,0};
\colorlet{c}{kugray};
\draw [c] (2.97078,3.92033) -- (2.97078,3.95122);
\draw [c] (2.97078,3.95122) -- (2.97078,3.97929);
\draw [c] (2.95596,3.95122) -- (2.97078,3.95122);
\draw [c] (2.97078,3.95122) -- (2.9856,3.95122);
\definecolor{c}{rgb}{0,0,0};
\colorlet{c}{kugray};
\draw [c] (3.00041,3.95455) -- (3.00041,3.98496);
\draw [c] (3.00041,3.98496) -- (3.00041,4.01263);
\draw [c] (2.9856,3.98496) -- (3.00041,3.98496);
\draw [c] (3.00041,3.98496) -- (3.01523,3.98496);
\definecolor{c}{rgb}{0,0,0};
\colorlet{c}{kugray};
\draw [c] (3.03005,3.91612) -- (3.03005,3.94874);
\draw [c] (3.03005,3.94874) -- (3.03005,3.97824);
\draw [c] (3.01523,3.94874) -- (3.03005,3.94874);
\draw [c] (3.03005,3.94874) -- (3.04487,3.94874);
\definecolor{c}{rgb}{0,0,0};
\colorlet{c}{kugray};
\draw [c] (3.05969,3.91532) -- (3.05969,3.94707);
\draw [c] (3.05969,3.94707) -- (3.05969,3.97585);
\draw [c] (3.04487,3.94707) -- (3.05969,3.94707);
\draw [c] (3.05969,3.94707) -- (3.0745,3.94707);
\definecolor{c}{rgb}{0,0,0};
\colorlet{c}{kugray};
\draw [c] (3.08932,3.82813) -- (3.08932,3.8633);
\draw [c] (3.08932,3.8633) -- (3.08932,3.89486);
\draw [c] (3.0745,3.8633) -- (3.08932,3.8633);
\draw [c] (3.08932,3.8633) -- (3.10414,3.8633);
\definecolor{c}{rgb}{0,0,0};
\colorlet{c}{kugray};
\draw [c] (3.11896,3.82152) -- (3.11896,3.85725);
\draw [c] (3.11896,3.85725) -- (3.11896,3.88926);
\draw [c] (3.10414,3.85725) -- (3.11896,3.85725);
\draw [c] (3.11896,3.85725) -- (3.13377,3.85725);
\definecolor{c}{rgb}{0,0,0};
\colorlet{c}{kugray};
\draw [c] (3.14859,3.81099) -- (3.14859,3.849);
\draw [c] (3.14859,3.849) -- (3.14859,3.88283);
\draw [c] (3.13377,3.849) -- (3.14859,3.849);
\draw [c] (3.14859,3.849) -- (3.16341,3.849);
\definecolor{c}{rgb}{0,0,0};
\colorlet{c}{kugray};
\draw [c] (3.17823,3.68505) -- (3.17823,3.72686);
\draw [c] (3.17823,3.72686) -- (3.17823,3.76367);
\draw [c] (3.16341,3.72686) -- (3.17823,3.72686);
\draw [c] (3.17823,3.72686) -- (3.19305,3.72686);
\definecolor{c}{rgb}{0,0,0};
\colorlet{c}{kugray};
\draw [c] (3.20786,3.75027) -- (3.20786,3.79207);
\draw [c] (3.20786,3.79207) -- (3.20786,3.82886);
\draw [c] (3.19305,3.79207) -- (3.20786,3.79207);
\draw [c] (3.20786,3.79207) -- (3.22268,3.79207);
\definecolor{c}{rgb}{0,0,0};
\colorlet{c}{kugray};
\draw [c] (3.2375,3.78399) -- (3.2375,3.82558);
\draw [c] (3.2375,3.82558) -- (3.2375,3.86222);
\draw [c] (3.22268,3.82558) -- (3.2375,3.82558);
\draw [c] (3.2375,3.82558) -- (3.25232,3.82558);
\definecolor{c}{rgb}{0,0,0};
\colorlet{c}{kugray};
\draw [c] (3.26714,3.71561) -- (3.26714,3.76154);
\draw [c] (3.26714,3.76154) -- (3.26714,3.80149);
\draw [c] (3.25232,3.76154) -- (3.26714,3.76154);
\draw [c] (3.26714,3.76154) -- (3.28195,3.76154);
\definecolor{c}{rgb}{0,0,0};
\colorlet{c}{kugray};
\draw [c] (3.29677,3.74464) -- (3.29677,3.78592);
\draw [c] (3.29677,3.78592) -- (3.29677,3.82231);
\draw [c] (3.28195,3.78592) -- (3.29677,3.78592);
\draw [c] (3.29677,3.78592) -- (3.31159,3.78592);
\definecolor{c}{rgb}{0,0,0};
\colorlet{c}{kugray};
\draw [c] (3.32641,3.60212) -- (3.32641,3.65046);
\draw [c] (3.32641,3.65046) -- (3.32641,3.69223);
\draw [c] (3.31159,3.65046) -- (3.32641,3.65046);
\draw [c] (3.32641,3.65046) -- (3.34123,3.65046);
\definecolor{c}{rgb}{0,0,0};
\colorlet{c}{kugray};
\draw [c] (3.35604,3.66614) -- (3.35604,3.71219);
\draw [c] (3.35604,3.71219) -- (3.35604,3.75225);
\draw [c] (3.34123,3.71219) -- (3.35604,3.71219);
\draw [c] (3.35604,3.71219) -- (3.37086,3.71219);
\definecolor{c}{rgb}{0,0,0};
\colorlet{c}{kugray};
\draw [c] (3.38568,3.70789) -- (3.38568,3.7536);
\draw [c] (3.38568,3.7536) -- (3.38568,3.7934);
\draw [c] (3.37086,3.7536) -- (3.38568,3.7536);
\draw [c] (3.38568,3.7536) -- (3.4005,3.7536);
\definecolor{c}{rgb}{0,0,0};
\colorlet{c}{kugray};
\draw [c] (3.41531,3.59749) -- (3.41531,3.64919);
\draw [c] (3.41531,3.64919) -- (3.41531,3.69343);
\draw [c] (3.4005,3.64919) -- (3.41531,3.64919);
\draw [c] (3.41531,3.64919) -- (3.43013,3.64919);
\definecolor{c}{rgb}{0,0,0};
\colorlet{c}{kugray};
\draw [c] (3.44495,3.54922) -- (3.44495,3.6056);
\draw [c] (3.44495,3.6056) -- (3.44495,3.65323);
\draw [c] (3.43013,3.6056) -- (3.44495,3.6056);
\draw [c] (3.44495,3.6056) -- (3.45977,3.6056);
\definecolor{c}{rgb}{0,0,0};
\colorlet{c}{kugray};
\draw [c] (3.47459,3.5315) -- (3.47459,3.58602);
\draw [c] (3.47459,3.58602) -- (3.47459,3.63231);
\draw [c] (3.45977,3.58602) -- (3.47459,3.58602);
\draw [c] (3.47459,3.58602) -- (3.4894,3.58602);
\definecolor{c}{rgb}{0,0,0};
\colorlet{c}{kugray};
\draw [c] (3.50422,3.44087) -- (3.50422,3.50756);
\draw [c] (3.50422,3.50756) -- (3.50422,3.56234);
\draw [c] (3.4894,3.50756) -- (3.50422,3.50756);
\draw [c] (3.50422,3.50756) -- (3.51904,3.50756);
\definecolor{c}{rgb}{0,0,0};
\colorlet{c}{kugray};
\draw [c] (3.53386,3.57716) -- (3.53386,3.62954);
\draw [c] (3.53386,3.62954) -- (3.53386,3.67429);
\draw [c] (3.51904,3.62954) -- (3.53386,3.62954);
\draw [c] (3.53386,3.62954) -- (3.54868,3.62954);
\definecolor{c}{rgb}{0,0,0};
\colorlet{c}{kugray};
\draw [c] (3.56349,3.59791) -- (3.56349,3.64909);
\draw [c] (3.56349,3.64909) -- (3.56349,3.69297);
\draw [c] (3.54868,3.64909) -- (3.56349,3.64909);
\draw [c] (3.56349,3.64909) -- (3.57831,3.64909);
\definecolor{c}{rgb}{0,0,0};
\colorlet{c}{kugray};
\draw [c] (3.59313,3.48016) -- (3.59313,3.54299);
\draw [c] (3.59313,3.54299) -- (3.59313,3.59515);
\draw [c] (3.57831,3.54299) -- (3.59313,3.54299);
\draw [c] (3.59313,3.54299) -- (3.60795,3.54299);
\definecolor{c}{rgb}{0,0,0};
\colorlet{c}{kugray};
\draw [c] (3.62276,3.49926) -- (3.62276,3.5653);
\draw [c] (3.62276,3.5653) -- (3.62276,3.61965);
\draw [c] (3.60795,3.5653) -- (3.62276,3.5653);
\draw [c] (3.62276,3.5653) -- (3.63758,3.5653);
\definecolor{c}{rgb}{0,0,0};
\colorlet{c}{kugray};
\draw [c] (3.6524,3.38318) -- (3.6524,3.45701);
\draw [c] (3.6524,3.45701) -- (3.6524,3.51651);
\draw [c] (3.63758,3.45701) -- (3.6524,3.45701);
\draw [c] (3.6524,3.45701) -- (3.66722,3.45701);
\definecolor{c}{rgb}{0,0,0};
\colorlet{c}{kugray};
\draw [c] (3.68204,3.38716) -- (3.68204,3.45825);
\draw [c] (3.68204,3.45825) -- (3.68204,3.51597);
\draw [c] (3.66722,3.45825) -- (3.68204,3.45825);
\draw [c] (3.68204,3.45825) -- (3.69685,3.45825);
\definecolor{c}{rgb}{0,0,0};
\colorlet{c}{kugray};
\draw [c] (3.71167,3.27025) -- (3.71167,3.35157);
\draw [c] (3.71167,3.35157) -- (3.71167,3.41584);
\draw [c] (3.69685,3.35157) -- (3.71167,3.35157);
\draw [c] (3.71167,3.35157) -- (3.72649,3.35157);
\definecolor{c}{rgb}{0,0,0};
\colorlet{c}{kugray};
\draw [c] (3.74131,3.42793) -- (3.74131,3.49218);
\draw [c] (3.74131,3.49218) -- (3.74131,3.54531);
\draw [c] (3.72649,3.49218) -- (3.74131,3.49218);
\draw [c] (3.74131,3.49218) -- (3.75613,3.49218);
\definecolor{c}{rgb}{0,0,0};
\colorlet{c}{kugray};
\draw [c] (3.77094,3.39681) -- (3.77094,3.4655);
\draw [c] (3.77094,3.4655) -- (3.77094,3.52163);
\draw [c] (3.75613,3.4655) -- (3.77094,3.4655);
\draw [c] (3.77094,3.4655) -- (3.78576,3.4655);
\definecolor{c}{rgb}{0,0,0};
\colorlet{c}{kugray};
\draw [c] (3.80058,3.30807) -- (3.80058,3.39026);
\draw [c] (3.80058,3.39026) -- (3.80058,3.45506);
\draw [c] (3.78576,3.39026) -- (3.80058,3.39026);
\draw [c] (3.80058,3.39026) -- (3.8154,3.39026);
\definecolor{c}{rgb}{0,0,0};
\colorlet{c}{kugray};
\draw [c] (3.83022,3.36833) -- (3.83022,3.44242);
\draw [c] (3.83022,3.44242) -- (3.83022,3.50209);
\draw [c] (3.8154,3.44242) -- (3.83022,3.44242);
\draw [c] (3.83022,3.44242) -- (3.84503,3.44242);
\definecolor{c}{rgb}{0,0,0};
\colorlet{c}{kugray};
\draw [c] (3.85985,3.36697) -- (3.85985,3.43885);
\draw [c] (3.85985,3.43885) -- (3.85985,3.49708);
\draw [c] (3.84503,3.43885) -- (3.85985,3.43885);
\draw [c] (3.85985,3.43885) -- (3.87467,3.43885);
\definecolor{c}{rgb}{0,0,0};
\colorlet{c}{kugray};
\draw [c] (3.88949,3.18339) -- (3.88949,3.27641);
\draw [c] (3.88949,3.27641) -- (3.88949,3.34775);
\draw [c] (3.87467,3.27641) -- (3.88949,3.27641);
\draw [c] (3.88949,3.27641) -- (3.9043,3.27641);
\definecolor{c}{rgb}{0,0,0};
\colorlet{c}{kugray};
\draw [c] (3.91912,3.36859) -- (3.91912,3.433);
\draw [c] (3.91912,3.433) -- (3.91912,3.48624);
\draw [c] (3.9043,3.433) -- (3.91912,3.433);
\draw [c] (3.91912,3.433) -- (3.93394,3.433);
\definecolor{c}{rgb}{0,0,0};
\colorlet{c}{kugray};
\draw [c] (3.94876,3.27957) -- (3.94876,3.34398);
\draw [c] (3.94876,3.34398) -- (3.94876,3.39722);
\draw [c] (3.93394,3.34398) -- (3.94876,3.34398);
\draw [c] (3.94876,3.34398) -- (3.96358,3.34398);
\definecolor{c}{rgb}{0,0,0};
\colorlet{c}{kugray};
\draw [c] (3.97839,3.23181) -- (3.97839,3.2842);
\draw [c] (3.97839,3.2842) -- (3.97839,3.32896);
\draw [c] (3.96358,3.2842) -- (3.97839,3.2842);
\draw [c] (3.97839,3.2842) -- (3.99321,3.2842);
\definecolor{c}{rgb}{0,0,0};
\colorlet{c}{kugray};
\draw [c] (4.00803,3.23298) -- (4.00803,3.24514);
\draw [c] (4.00803,3.24514) -- (4.00803,3.25684);
\draw [c] (3.99321,3.24514) -- (4.00803,3.24514);
\draw [c] (4.00803,3.24514) -- (4.02285,3.24514);
\definecolor{c}{rgb}{0,0,0};
\colorlet{c}{kugray};
\draw [c] (4.03767,3.23899) -- (4.03767,3.25079);
\draw [c] (4.03767,3.25079) -- (4.03767,3.26215);
\draw [c] (4.02285,3.25079) -- (4.03767,3.25079);
\draw [c] (4.03767,3.25079) -- (4.05248,3.25079);
\definecolor{c}{rgb}{0,0,0};
\colorlet{c}{kugray};
\draw [c] (4.0673,3.24063) -- (4.0673,3.25269);
\draw [c] (4.0673,3.25269) -- (4.0673,3.26429);
\draw [c] (4.05248,3.25269) -- (4.0673,3.25269);
\draw [c] (4.0673,3.25269) -- (4.08212,3.25269);
\definecolor{c}{rgb}{0,0,0};
\colorlet{c}{kugray};
\draw [c] (4.09694,3.22722) -- (4.09694,3.23915);
\draw [c] (4.09694,3.23915) -- (4.09694,3.25063);
\draw [c] (4.08212,3.23915) -- (4.09694,3.23915);
\draw [c] (4.09694,3.23915) -- (4.11175,3.23915);
\definecolor{c}{rgb}{0,0,0};
\colorlet{c}{kugray};
\draw [c] (4.12657,3.2116) -- (4.12657,3.22405);
\draw [c] (4.12657,3.22405) -- (4.12657,3.23601);
\draw [c] (4.11175,3.22405) -- (4.12657,3.22405);
\draw [c] (4.12657,3.22405) -- (4.14139,3.22405);
\definecolor{c}{rgb}{0,0,0};
\colorlet{c}{kugray};
\draw [c] (4.15621,3.18215) -- (4.15621,3.19496);
\draw [c] (4.15621,3.19496) -- (4.15621,3.20726);
\draw [c] (4.14139,3.19496) -- (4.15621,3.19496);
\draw [c] (4.15621,3.19496) -- (4.17103,3.19496);
\definecolor{c}{rgb}{0,0,0};
\colorlet{c}{kugray};
\draw [c] (4.18584,3.18206) -- (4.18584,3.19521);
\draw [c] (4.18584,3.19521) -- (4.18584,3.20782);
\draw [c] (4.17103,3.19521) -- (4.18584,3.19521);
\draw [c] (4.18584,3.19521) -- (4.20066,3.19521);
\definecolor{c}{rgb}{0,0,0};
\colorlet{c}{kugray};
\draw [c] (4.21548,3.14682) -- (4.21548,3.16086);
\draw [c] (4.21548,3.16086) -- (4.21548,3.17428);
\draw [c] (4.20066,3.16086) -- (4.21548,3.16086);
\draw [c] (4.21548,3.16086) -- (4.2303,3.16086);
\definecolor{c}{rgb}{0,0,0};
\colorlet{c}{kugray};
\draw [c] (4.24512,3.16809) -- (4.24512,3.1818);
\draw [c] (4.24512,3.1818) -- (4.24512,3.19493);
\draw [c] (4.2303,3.1818) -- (4.24512,3.1818);
\draw [c] (4.24512,3.1818) -- (4.25993,3.1818);
\definecolor{c}{rgb}{0,0,0};
\colorlet{c}{kugray};
\draw [c] (4.27475,3.10937) -- (4.27475,3.12421);
\draw [c] (4.27475,3.12421) -- (4.27475,3.13837);
\draw [c] (4.25993,3.12421) -- (4.27475,3.12421);
\draw [c] (4.27475,3.12421) -- (4.28957,3.12421);
\definecolor{c}{rgb}{0,0,0};
\colorlet{c}{kugray};
\draw [c] (4.30439,3.11556) -- (4.30439,3.12992);
\draw [c] (4.30439,3.12992) -- (4.30439,3.14364);
\draw [c] (4.28957,3.12992) -- (4.30439,3.12992);
\draw [c] (4.30439,3.12992) -- (4.31921,3.12992);
\definecolor{c}{rgb}{0,0,0};
\colorlet{c}{kugray};
\draw [c] (4.33402,3.13544) -- (4.33402,3.14949);
\draw [c] (4.33402,3.14949) -- (4.33402,3.16293);
\draw [c] (4.31921,3.14949) -- (4.33402,3.14949);
\draw [c] (4.33402,3.14949) -- (4.34884,3.14949);
\definecolor{c}{rgb}{0,0,0};
\colorlet{c}{kugray};
\draw [c] (4.36366,3.09108) -- (4.36366,3.10621);
\draw [c] (4.36366,3.10621) -- (4.36366,3.12063);
\draw [c] (4.34884,3.10621) -- (4.36366,3.10621);
\draw [c] (4.36366,3.10621) -- (4.37848,3.10621);
\definecolor{c}{rgb}{0,0,0};
\colorlet{c}{kugray};
\draw [c] (4.39329,3.08833) -- (4.39329,3.10339);
\draw [c] (4.39329,3.10339) -- (4.39329,3.11775);
\draw [c] (4.37848,3.10339) -- (4.39329,3.10339);
\draw [c] (4.39329,3.10339) -- (4.40811,3.10339);
\definecolor{c}{rgb}{0,0,0};
\colorlet{c}{kugray};
\draw [c] (4.42293,3.06637) -- (4.42293,3.08234);
\draw [c] (4.42293,3.08234) -- (4.42293,3.09751);
\draw [c] (4.40811,3.08234) -- (4.42293,3.08234);
\draw [c] (4.42293,3.08234) -- (4.43775,3.08234);
\definecolor{c}{rgb}{0,0,0};
\colorlet{c}{kugray};
\draw [c] (4.45257,3.03685) -- (4.45257,3.05337);
\draw [c] (4.45257,3.05337) -- (4.45257,3.06905);
\draw [c] (4.43775,3.05337) -- (4.45257,3.05337);
\draw [c] (4.45257,3.05337) -- (4.46738,3.05337);
\definecolor{c}{rgb}{0,0,0};
\colorlet{c}{kugray};
\draw [c] (4.4822,3.05403) -- (4.4822,3.07028);
\draw [c] (4.4822,3.07028) -- (4.4822,3.08572);
\draw [c] (4.46738,3.07028) -- (4.4822,3.07028);
\draw [c] (4.4822,3.07028) -- (4.49702,3.07028);
\definecolor{c}{rgb}{0,0,0};
\colorlet{c}{kugray};
\draw [c] (4.51184,3.02475) -- (4.51184,3.04166);
\draw [c] (4.51184,3.04166) -- (4.51184,3.05768);
\draw [c] (4.49702,3.04166) -- (4.51184,3.04166);
\draw [c] (4.51184,3.04166) -- (4.52666,3.04166);
\definecolor{c}{rgb}{0,0,0};
\colorlet{c}{kugray};
\draw [c] (4.54147,2.97429) -- (4.54147,2.99215);
\draw [c] (4.54147,2.99215) -- (4.54147,3.00903);
\draw [c] (4.52666,2.99215) -- (4.54147,2.99215);
\draw [c] (4.54147,2.99215) -- (4.55629,2.99215);
\definecolor{c}{rgb}{0,0,0};
\colorlet{c}{kugray};
\draw [c] (4.57111,2.97145) -- (4.57111,2.9901);
\draw [c] (4.57111,2.9901) -- (4.57111,3.0077);
\draw [c] (4.55629,2.9901) -- (4.57111,2.9901);
\draw [c] (4.57111,2.9901) -- (4.58593,2.9901);
\definecolor{c}{rgb}{0,0,0};
\colorlet{c}{kugray};
\draw [c] (4.60075,2.99637) -- (4.60075,3.01434);
\draw [c] (4.60075,3.01434) -- (4.60075,3.03131);
\draw [c] (4.58593,3.01434) -- (4.60075,3.01434);
\draw [c] (4.60075,3.01434) -- (4.61556,3.01434);
\definecolor{c}{rgb}{0,0,0};
\colorlet{c}{kugray};
\draw [c] (4.63038,2.94705) -- (4.63038,2.96575);
\draw [c] (4.63038,2.96575) -- (4.63038,2.98338);
\draw [c] (4.61556,2.96575) -- (4.63038,2.96575);
\draw [c] (4.63038,2.96575) -- (4.6452,2.96575);
\definecolor{c}{rgb}{0,0,0};
\colorlet{c}{kugray};
\draw [c] (4.66002,2.94213) -- (4.66002,2.96119);
\draw [c] (4.66002,2.96119) -- (4.66002,2.97915);
\draw [c] (4.6452,2.96119) -- (4.66002,2.96119);
\draw [c] (4.66002,2.96119) -- (4.67483,2.96119);
\definecolor{c}{rgb}{0,0,0};
\colorlet{c}{kugray};
\draw [c] (4.68965,2.92699) -- (4.68965,2.94611);
\draw [c] (4.68965,2.94611) -- (4.68965,2.96412);
\draw [c] (4.67483,2.94611) -- (4.68965,2.94611);
\draw [c] (4.68965,2.94611) -- (4.70447,2.94611);
\definecolor{c}{rgb}{0,0,0};
\colorlet{c}{kugray};
\draw [c] (4.71929,2.90533) -- (4.71929,2.92537);
\draw [c] (4.71929,2.92537) -- (4.71929,2.94418);
\draw [c] (4.70447,2.92537) -- (4.71929,2.92537);
\draw [c] (4.71929,2.92537) -- (4.73411,2.92537);
\definecolor{c}{rgb}{0,0,0};
\colorlet{c}{kugray};
\draw [c] (4.74892,2.89803) -- (4.74892,2.9188);
\draw [c] (4.74892,2.9188) -- (4.74892,2.93826);
\draw [c] (4.73411,2.9188) -- (4.74892,2.9188);
\draw [c] (4.74892,2.9188) -- (4.76374,2.9188);
\definecolor{c}{rgb}{0,0,0};
\colorlet{c}{kugray};
\draw [c] (4.77856,2.88491) -- (4.77856,2.90571);
\draw [c] (4.77856,2.90571) -- (4.77856,2.92519);
\draw [c] (4.76374,2.90571) -- (4.77856,2.90571);
\draw [c] (4.77856,2.90571) -- (4.79338,2.90571);
\definecolor{c}{rgb}{0,0,0};
\colorlet{c}{kugray};
\draw [c] (4.8082,2.83295) -- (4.8082,2.85717);
\draw [c] (4.8082,2.85717) -- (4.8082,2.87962);
\draw [c] (4.79338,2.85717) -- (4.8082,2.85717);
\draw [c] (4.8082,2.85717) -- (4.82301,2.85717);
\definecolor{c}{rgb}{0,0,0};
\colorlet{c}{kugray};
\draw [c] (4.83783,2.84099) -- (4.83783,2.86366);
\draw [c] (4.83783,2.86366) -- (4.83783,2.88478);
\draw [c] (4.82301,2.86366) -- (4.83783,2.86366);
\draw [c] (4.83783,2.86366) -- (4.85265,2.86366);
\definecolor{c}{rgb}{0,0,0};
\colorlet{c}{kugray};
\draw [c] (4.86747,2.84516) -- (4.86747,2.86807);
\draw [c] (4.86747,2.86807) -- (4.86747,2.88939);
\draw [c] (4.85265,2.86807) -- (4.86747,2.86807);
\draw [c] (4.86747,2.86807) -- (4.88228,2.86807);
\definecolor{c}{rgb}{0,0,0};
\colorlet{c}{kugray};
\draw [c] (4.8971,2.82427) -- (4.8971,2.84804);
\draw [c] (4.8971,2.84804) -- (4.8971,2.87011);
\draw [c] (4.88228,2.84804) -- (4.8971,2.84804);
\draw [c] (4.8971,2.84804) -- (4.91192,2.84804);
\definecolor{c}{rgb}{0,0,0};
\colorlet{c}{kugray};
\draw [c] (4.92674,2.7958) -- (4.92674,2.82013);
\draw [c] (4.92674,2.82013) -- (4.92674,2.84268);
\draw [c] (4.91192,2.82013) -- (4.92674,2.82013);
\draw [c] (4.92674,2.82013) -- (4.94156,2.82013);
\definecolor{c}{rgb}{0,0,0};
\colorlet{c}{kugray};
\draw [c] (4.95637,2.72162) -- (4.95637,2.74908);
\draw [c] (4.95637,2.74908) -- (4.95637,2.7743);
\draw [c] (4.94156,2.74908) -- (4.95637,2.74908);
\draw [c] (4.95637,2.74908) -- (4.97119,2.74908);
\definecolor{c}{rgb}{0,0,0};
\colorlet{c}{kugray};
\draw [c] (4.98601,2.74467) -- (4.98601,2.77317);
\draw [c] (4.98601,2.77317) -- (4.98601,2.79924);
\draw [c] (4.97119,2.77317) -- (4.98601,2.77317);
\draw [c] (4.98601,2.77317) -- (5.00083,2.77317);
\definecolor{c}{rgb}{0,0,0};
\colorlet{c}{kugray};
\draw [c] (5.01565,2.72133) -- (5.01565,2.74919);
\draw [c] (5.01565,2.74919) -- (5.01565,2.77473);
\draw [c] (5.00083,2.74919) -- (5.01565,2.74919);
\draw [c] (5.01565,2.74919) -- (5.03046,2.74919);
\definecolor{c}{rgb}{0,0,0};
\colorlet{c}{kugray};
\draw [c] (5.04528,2.68622) -- (5.04528,2.71565);
\draw [c] (5.04528,2.71565) -- (5.04528,2.74251);
\draw [c] (5.03046,2.71565) -- (5.04528,2.71565);
\draw [c] (5.04528,2.71565) -- (5.0601,2.71565);
\definecolor{c}{rgb}{0,0,0};
\colorlet{c}{kugray};
\draw [c] (5.07492,2.69413) -- (5.07492,2.72282);
\draw [c] (5.07492,2.72282) -- (5.07492,2.74907);
\draw [c] (5.0601,2.72282) -- (5.07492,2.72282);
\draw [c] (5.07492,2.72282) -- (5.08974,2.72282);
\definecolor{c}{rgb}{0,0,0};
\colorlet{c}{kugray};
\draw [c] (5.10455,2.68206) -- (5.10455,2.71119);
\draw [c] (5.10455,2.71119) -- (5.10455,2.73781);
\draw [c] (5.08974,2.71119) -- (5.10455,2.71119);
\draw [c] (5.10455,2.71119) -- (5.11937,2.71119);
\definecolor{c}{rgb}{0,0,0};
\colorlet{c}{kugray};
\draw [c] (5.13419,2.68307) -- (5.13419,2.7116);
\draw [c] (5.13419,2.7116) -- (5.13419,2.7377);
\draw [c] (5.11937,2.7116) -- (5.13419,2.7116);
\draw [c] (5.13419,2.7116) -- (5.14901,2.7116);
\definecolor{c}{rgb}{0,0,0};
\colorlet{c}{kugray};
\draw [c] (5.16382,2.72714) -- (5.16382,2.75436);
\draw [c] (5.16382,2.75436) -- (5.16382,2.77936);
\draw [c] (5.14901,2.75436) -- (5.16382,2.75436);
\draw [c] (5.16382,2.75436) -- (5.17864,2.75436);
\definecolor{c}{rgb}{0,0,0};
\colorlet{c}{kugray};
\draw [c] (5.19346,2.64861) -- (5.19346,2.67901);
\draw [c] (5.19346,2.67901) -- (5.19346,2.70667);
\draw [c] (5.17864,2.67901) -- (5.19346,2.67901);
\draw [c] (5.19346,2.67901) -- (5.20828,2.67901);
\definecolor{c}{rgb}{0,0,0};
\colorlet{c}{kugray};
\draw [c] (5.2231,2.688) -- (5.2231,2.71811);
\draw [c] (5.2231,2.71811) -- (5.2231,2.74554);
\draw [c] (5.20828,2.71811) -- (5.2231,2.71811);
\draw [c] (5.2231,2.71811) -- (5.23791,2.71811);
\definecolor{c}{rgb}{0,0,0};
\colorlet{c}{kugray};
\draw [c] (5.25273,2.56862) -- (5.25273,2.60483);
\draw [c] (5.25273,2.60483) -- (5.25273,2.63722);
\draw [c] (5.23791,2.60483) -- (5.25273,2.60483);
\draw [c] (5.25273,2.60483) -- (5.26755,2.60483);
\definecolor{c}{rgb}{0,0,0};
\colorlet{c}{kugray};
\draw [c] (5.28237,2.61362) -- (5.28237,2.64718);
\draw [c] (5.28237,2.64718) -- (5.28237,2.67745);
\draw [c] (5.26755,2.64718) -- (5.28237,2.64718);
\draw [c] (5.28237,2.64718) -- (5.29719,2.64718);
\definecolor{c}{rgb}{0,0,0};
\colorlet{c}{kugray};
\draw [c] (5.312,2.58142) -- (5.312,2.61649);
\draw [c] (5.312,2.61649) -- (5.312,2.64798);
\draw [c] (5.29719,2.61649) -- (5.312,2.61649);
\draw [c] (5.312,2.61649) -- (5.32682,2.61649);
\definecolor{c}{rgb}{0,0,0};
\colorlet{c}{kugray};
\draw [c] (5.34164,2.59936) -- (5.34164,2.63325);
\draw [c] (5.34164,2.63325) -- (5.34164,2.66377);
\draw [c] (5.32682,2.63325) -- (5.34164,2.63325);
\draw [c] (5.34164,2.63325) -- (5.35646,2.63325);
\definecolor{c}{rgb}{0,0,0};
\colorlet{c}{kugray};
\draw [c] (5.37127,2.59301) -- (5.37127,2.62735);
\draw [c] (5.37127,2.62735) -- (5.37127,2.65824);
\draw [c] (5.35646,2.62735) -- (5.37127,2.62735);
\draw [c] (5.37127,2.62735) -- (5.38609,2.62735);
\definecolor{c}{rgb}{0,0,0};
\colorlet{c}{kugray};
\draw [c] (5.40091,2.57979) -- (5.40091,2.61666);
\draw [c] (5.40091,2.61666) -- (5.40091,2.64958);
\draw [c] (5.38609,2.61666) -- (5.40091,2.61666);
\draw [c] (5.40091,2.61666) -- (5.41573,2.61666);
\definecolor{c}{rgb}{0,0,0};
\colorlet{c}{kugray};
\draw [c] (5.43055,2.5888) -- (5.43055,2.62522);
\draw [c] (5.43055,2.62522) -- (5.43055,2.65779);
\draw [c] (5.41573,2.62522) -- (5.43055,2.62522);
\draw [c] (5.43055,2.62522) -- (5.44536,2.62522);
\definecolor{c}{rgb}{0,0,0};
\colorlet{c}{kugray};
\draw [c] (5.46018,2.49892) -- (5.46018,2.53839);
\draw [c] (5.46018,2.53839) -- (5.46018,2.57337);
\draw [c] (5.44536,2.53839) -- (5.46018,2.53839);
\draw [c] (5.46018,2.53839) -- (5.475,2.53839);
\definecolor{c}{rgb}{0,0,0};
\colorlet{c}{kugray};
\draw [c] (5.48982,2.51137) -- (5.48982,2.5487);
\draw [c] (5.48982,2.5487) -- (5.48982,2.58199);
\draw [c] (5.475,2.5487) -- (5.48982,2.5487);
\draw [c] (5.48982,2.5487) -- (5.50464,2.5487);
\definecolor{c}{rgb}{0,0,0};
\colorlet{c}{kugray};
\draw [c] (5.51945,2.5345) -- (5.51945,2.57323);
\draw [c] (5.51945,2.57323) -- (5.51945,2.60762);
\draw [c] (5.50464,2.57323) -- (5.51945,2.57323);
\draw [c] (5.51945,2.57323) -- (5.53427,2.57323);
\definecolor{c}{rgb}{0,0,0};
\colorlet{c}{kugray};
\draw [c] (5.54909,2.49349) -- (5.54909,2.533);
\draw [c] (5.54909,2.533) -- (5.54909,2.56801);
\draw [c] (5.53427,2.533) -- (5.54909,2.533);
\draw [c] (5.54909,2.533) -- (5.56391,2.533);
\definecolor{c}{rgb}{0,0,0};
\colorlet{c}{kugray};
\draw [c] (5.57873,2.46256) -- (5.57873,2.50333);
\draw [c] (5.57873,2.50333) -- (5.57873,2.53932);
\draw [c] (5.56391,2.50333) -- (5.57873,2.50333);
\draw [c] (5.57873,2.50333) -- (5.59354,2.50333);
\definecolor{c}{rgb}{0,0,0};
\colorlet{c}{kugray};
\draw [c] (5.60836,2.50823) -- (5.60836,2.54684);
\draw [c] (5.60836,2.54684) -- (5.60836,2.58115);
\draw [c] (5.59354,2.54684) -- (5.60836,2.54684);
\draw [c] (5.60836,2.54684) -- (5.62318,2.54684);
\definecolor{c}{rgb}{0,0,0};
\colorlet{c}{kugray};
\draw [c] (5.638,2.50569) -- (5.638,2.54708);
\draw [c] (5.638,2.54708) -- (5.638,2.58356);
\draw [c] (5.62318,2.54708) -- (5.638,2.54708);
\draw [c] (5.638,2.54708) -- (5.65281,2.54708);
\definecolor{c}{rgb}{0,0,0};
\colorlet{c}{kugray};
\draw [c] (5.66763,2.48101) -- (5.66763,2.5229);
\draw [c] (5.66763,2.5229) -- (5.66763,2.55976);
\draw [c] (5.65281,2.5229) -- (5.66763,2.5229);
\draw [c] (5.66763,2.5229) -- (5.68245,2.5229);
\definecolor{c}{rgb}{0,0,0};
\colorlet{c}{kugray};
\draw [c] (5.69727,2.3796) -- (5.69727,2.42684);
\draw [c] (5.69727,2.42684) -- (5.69727,2.46778);
\draw [c] (5.68245,2.42684) -- (5.69727,2.42684);
\draw [c] (5.69727,2.42684) -- (5.71209,2.42684);
\definecolor{c}{rgb}{0,0,0};
\colorlet{c}{kugray};
\draw [c] (5.7269,2.30877) -- (5.7269,2.35869);
\draw [c] (5.7269,2.35869) -- (5.7269,2.40163);
\draw [c] (5.71209,2.35869) -- (5.7269,2.35869);
\draw [c] (5.7269,2.35869) -- (5.74172,2.35869);
\definecolor{c}{rgb}{0,0,0};
\colorlet{c}{kugray};
\draw [c] (5.75654,2.43196) -- (5.75654,2.47986);
\draw [c] (5.75654,2.47986) -- (5.75654,2.5213);
\draw [c] (5.74172,2.47986) -- (5.75654,2.47986);
\draw [c] (5.75654,2.47986) -- (5.77136,2.47986);
\definecolor{c}{rgb}{0,0,0};
\colorlet{c}{kugray};
\draw [c] (5.78618,2.43169) -- (5.78618,2.47526);
\draw [c] (5.78618,2.47526) -- (5.78618,2.51342);
\draw [c] (5.77136,2.47526) -- (5.78618,2.47526);
\draw [c] (5.78618,2.47526) -- (5.80099,2.47526);
\definecolor{c}{rgb}{0,0,0};
\colorlet{c}{kugray};
\draw [c] (5.81581,2.39907) -- (5.81581,2.44526);
\draw [c] (5.81581,2.44526) -- (5.81581,2.48541);
\draw [c] (5.80099,2.44526) -- (5.81581,2.44526);
\draw [c] (5.81581,2.44526) -- (5.83063,2.44526);
\definecolor{c}{rgb}{0,0,0};
\colorlet{c}{kugray};
\draw [c] (5.84545,2.43121) -- (5.84545,2.47744);
\draw [c] (5.84545,2.47744) -- (5.84545,2.51762);
\draw [c] (5.83063,2.47744) -- (5.84545,2.47744);
\draw [c] (5.84545,2.47744) -- (5.86026,2.47744);
\definecolor{c}{rgb}{0,0,0};
\colorlet{c}{kugray};
\draw [c] (5.87508,2.29827) -- (5.87508,2.35564);
\draw [c] (5.87508,2.35564) -- (5.87508,2.40397);
\draw [c] (5.86026,2.35564) -- (5.87508,2.35564);
\draw [c] (5.87508,2.35564) -- (5.8899,2.35564);
\definecolor{c}{rgb}{0,0,0};
\colorlet{c}{kugray};
\draw [c] (5.90472,2.35345) -- (5.90472,2.40527);
\draw [c] (5.90472,2.40527) -- (5.90472,2.44961);
\draw [c] (5.8899,2.40527) -- (5.90472,2.40527);
\draw [c] (5.90472,2.40527) -- (5.91954,2.40527);
\definecolor{c}{rgb}{0,0,0};
\colorlet{c}{kugray};
\draw [c] (5.93435,2.28972) -- (5.93435,2.34179);
\draw [c] (5.93435,2.34179) -- (5.93435,2.38632);
\draw [c] (5.91954,2.34179) -- (5.93435,2.34179);
\draw [c] (5.93435,2.34179) -- (5.94917,2.34179);
\definecolor{c}{rgb}{0,0,0};
\colorlet{c}{kugray};
\draw [c] (5.96399,2.28856) -- (5.96399,2.34875);
\draw [c] (5.96399,2.34875) -- (5.96399,2.39907);
\draw [c] (5.94917,2.34875) -- (5.96399,2.34875);
\draw [c] (5.96399,2.34875) -- (5.97881,2.34875);
\definecolor{c}{rgb}{0,0,0};
\colorlet{c}{kugray};
\draw [c] (5.99363,2.34369) -- (5.99363,2.39484);
\draw [c] (5.99363,2.39484) -- (5.99363,2.43869);
\draw [c] (5.97881,2.39484) -- (5.99363,2.39484);
\draw [c] (5.99363,2.39484) -- (6.00844,2.39484);
\definecolor{c}{rgb}{0,0,0};
\colorlet{c}{kugray};
\draw [c] (6.02326,2.22004) -- (6.02326,2.27835);
\draw [c] (6.02326,2.27835) -- (6.02326,2.32735);
\draw [c] (6.00844,2.27835) -- (6.02326,2.27835);
\draw [c] (6.02326,2.27835) -- (6.03808,2.27835);
\definecolor{c}{rgb}{0,0,0};
\colorlet{c}{kugray};
\draw [c] (6.0529,2.24812) -- (6.0529,2.30663);
\draw [c] (6.0529,2.30663) -- (6.0529,2.35577);
\draw [c] (6.03808,2.30663) -- (6.0529,2.30663);
\draw [c] (6.0529,2.30663) -- (6.06772,2.30663);
\definecolor{c}{rgb}{0,0,0};
\colorlet{c}{kugray};
\draw [c] (6.08253,2.21392) -- (6.08253,2.27979);
\draw [c] (6.08253,2.27979) -- (6.08253,2.33402);
\draw [c] (6.06772,2.27979) -- (6.08253,2.27979);
\draw [c] (6.08253,2.27979) -- (6.09735,2.27979);
\definecolor{c}{rgb}{0,0,0};
\colorlet{c}{kugray};
\draw [c] (6.11217,2.28422) -- (6.11217,2.34425);
\draw [c] (6.11217,2.34425) -- (6.11217,2.39446);
\draw [c] (6.09735,2.34425) -- (6.11217,2.34425);
\draw [c] (6.11217,2.34425) -- (6.12699,2.34425);
\definecolor{c}{rgb}{0,0,0};
\colorlet{c}{kugray};
\draw [c] (6.1418,2.13526) -- (6.1418,2.20119);
\draw [c] (6.1418,2.20119) -- (6.1418,2.25547);
\draw [c] (6.12699,2.20119) -- (6.1418,2.20119);
\draw [c] (6.1418,2.20119) -- (6.15662,2.20119);
\definecolor{c}{rgb}{0,0,0};
\colorlet{c}{kugray};
\draw [c] (6.17144,2.21632) -- (6.17144,2.28052);
\draw [c] (6.17144,2.28052) -- (6.17144,2.33362);
\draw [c] (6.15662,2.28052) -- (6.17144,2.28052);
\draw [c] (6.17144,2.28052) -- (6.18626,2.28052);
\definecolor{c}{rgb}{0,0,0};
\colorlet{c}{kugray};
\draw [c] (6.20108,2.25826) -- (6.20108,2.32268);
\draw [c] (6.20108,2.32268) -- (6.20108,2.37592);
\draw [c] (6.18626,2.32268) -- (6.20108,2.32268);
\draw [c] (6.20108,2.32268) -- (6.21589,2.32268);
\definecolor{c}{rgb}{0,0,0};
\colorlet{c}{kugray};
\draw [c] (6.23071,2.21696) -- (6.23071,2.28153);
\draw [c] (6.23071,2.28153) -- (6.23071,2.33487);
\draw [c] (6.21589,2.28153) -- (6.23071,2.28153);
\draw [c] (6.23071,2.28153) -- (6.24553,2.28153);
\definecolor{c}{rgb}{0,0,0};
\colorlet{c}{kugray};
\draw [c] (6.26035,2.11717) -- (6.26035,2.18912);
\draw [c] (6.26035,2.18912) -- (6.26035,2.2474);
\draw [c] (6.24553,2.18912) -- (6.26035,2.18912);
\draw [c] (6.26035,2.18912) -- (6.27517,2.18912);
\definecolor{c}{rgb}{0,0,0};
\colorlet{c}{kugray};
\draw [c] (6.28998,2.21915) -- (6.28998,2.27945);
\draw [c] (6.28998,2.27945) -- (6.28998,2.32985);
\draw [c] (6.27517,2.27945) -- (6.28998,2.27945);
\draw [c] (6.28998,2.27945) -- (6.3048,2.27945);
\definecolor{c}{rgb}{0,0,0};
\colorlet{c}{kugray};
\draw [c] (6.31962,2.15067) -- (6.31962,2.22057);
\draw [c] (6.31962,2.22057) -- (6.31962,2.2775);
\draw [c] (6.3048,2.22057) -- (6.31962,2.22057);
\draw [c] (6.31962,2.22057) -- (6.33444,2.22057);
\definecolor{c}{rgb}{0,0,0};
\colorlet{c}{kugray};
\draw [c] (6.34926,2.08129) -- (6.34926,2.15905);
\draw [c] (6.34926,2.15905) -- (6.34926,2.22107);
\draw [c] (6.33444,2.15905) -- (6.34926,2.15905);
\draw [c] (6.34926,2.15905) -- (6.36407,2.15905);
\definecolor{c}{rgb}{0,0,0};
\colorlet{c}{kugray};
\draw [c] (6.37889,2.26481) -- (6.37889,2.32632);
\draw [c] (6.37889,2.32632) -- (6.37889,2.37756);
\draw [c] (6.36407,2.32632) -- (6.37889,2.32632);
\draw [c] (6.37889,2.32632) -- (6.39371,2.32632);
\definecolor{c}{rgb}{0,0,0};
\colorlet{c}{kugray};
\draw [c] (6.40853,2.10103) -- (6.40853,2.17287);
\draw [c] (6.40853,2.17287) -- (6.40853,2.23108);
\draw [c] (6.39371,2.17287) -- (6.40853,2.17287);
\draw [c] (6.40853,2.17287) -- (6.42334,2.17287);
\definecolor{c}{rgb}{0,0,0};
\colorlet{c}{kugray};
\draw [c] (6.43816,2.01805) -- (6.43816,2.10352);
\draw [c] (6.43816,2.10352) -- (6.43816,2.17034);
\draw [c] (6.42334,2.10352) -- (6.43816,2.10352);
\draw [c] (6.43816,2.10352) -- (6.45298,2.10352);
\definecolor{c}{rgb}{0,0,0};
\colorlet{c}{kugray};
\draw [c] (6.4678,2.13798) -- (6.4678,2.21314);
\draw [c] (6.4678,2.21314) -- (6.4678,2.27351);
\draw [c] (6.45298,2.21314) -- (6.4678,2.21314);
\draw [c] (6.4678,2.21314) -- (6.48262,2.21314);
\definecolor{c}{rgb}{0,0,0};
\colorlet{c}{kugray};
\draw [c] (6.49743,2.0628) -- (6.49743,2.13789);
\draw [c] (6.49743,2.13789) -- (6.49743,2.19821);
\draw [c] (6.48262,2.13789) -- (6.49743,2.13789);
\draw [c] (6.49743,2.13789) -- (6.51225,2.13789);
\definecolor{c}{rgb}{0,0,0};
\colorlet{c}{kugray};
\draw [c] (6.52707,2.03036) -- (6.52707,2.11014);
\draw [c] (6.52707,2.11014) -- (6.52707,2.17345);
\draw [c] (6.51225,2.11014) -- (6.52707,2.11014);
\draw [c] (6.52707,2.11014) -- (6.54189,2.11014);
\definecolor{c}{rgb}{0,0,0};
\colorlet{c}{kugray};
\draw [c] (6.55671,1.95941) -- (6.55671,2.05375);
\draw [c] (6.55671,2.05375) -- (6.55671,2.12586);
\draw [c] (6.54189,2.05375) -- (6.55671,2.05375);
\draw [c] (6.55671,2.05375) -- (6.57152,2.05375);
\definecolor{c}{rgb}{0,0,0};
\colorlet{c}{kugray};
\draw [c] (6.58634,2.0092) -- (6.58634,2.09944);
\draw [c] (6.58634,2.09944) -- (6.58634,2.16914);
\draw [c] (6.57152,2.09944) -- (6.58634,2.09944);
\draw [c] (6.58634,2.09944) -- (6.60116,2.09944);
\definecolor{c}{rgb}{0,0,0};
\colorlet{c}{kugray};
\draw [c] (6.61598,2.06823) -- (6.61598,2.14649);
\draw [c] (6.61598,2.14649) -- (6.61598,2.20883);
\draw [c] (6.60116,2.14649) -- (6.61598,2.14649);
\draw [c] (6.61598,2.14649) -- (6.63079,2.14649);
\definecolor{c}{rgb}{0,0,0};
\colorlet{c}{kugray};
\draw [c] (6.64561,1.91105) -- (6.64561,2.01133);
\draw [c] (6.64561,2.01133) -- (6.64561,2.08685);
\draw [c] (6.63079,2.01133) -- (6.64561,2.01133);
\draw [c] (6.64561,2.01133) -- (6.66043,2.01133);
\definecolor{c}{rgb}{0,0,0};
\colorlet{c}{kugray};
\draw [c] (6.67525,1.90644) -- (6.67525,2.01096);
\draw [c] (6.67525,2.01096) -- (6.67525,2.08884);
\draw [c] (6.66043,2.01096) -- (6.67525,2.01096);
\draw [c] (6.67525,2.01096) -- (6.69007,2.01096);
\definecolor{c}{rgb}{0,0,0};
\colorlet{c}{kugray};
\draw [c] (6.70488,1.98499) -- (6.70488,2.07197);
\draw [c] (6.70488,2.07197) -- (6.70488,2.13971);
\draw [c] (6.69007,2.07197) -- (6.70488,2.07197);
\draw [c] (6.70488,2.07197) -- (6.7197,2.07197);
\definecolor{c}{rgb}{0,0,0};
\colorlet{c}{kugray};
\draw [c] (6.73452,1.98323) -- (6.73452,2.07762);
\draw [c] (6.73452,2.07762) -- (6.73452,2.14976);
\draw [c] (6.7197,2.07762) -- (6.73452,2.07762);
\draw [c] (6.73452,2.07762) -- (6.74934,2.07762);
\definecolor{c}{rgb}{0,0,0};
\colorlet{c}{kugray};
\draw [c] (6.76416,1.8404) -- (6.76416,1.94614);
\draw [c] (6.76416,1.94614) -- (6.76416,2.0247);
\draw [c] (6.74934,1.94614) -- (6.76416,1.94614);
\draw [c] (6.76416,1.94614) -- (6.77897,1.94614);
\definecolor{c}{rgb}{0,0,0};
\colorlet{c}{kugray};
\draw [c] (6.79379,1.95326) -- (6.79379,2.04637);
\draw [c] (6.79379,2.04637) -- (6.79379,2.11775);
\draw [c] (6.77897,2.04637) -- (6.79379,2.04637);
\draw [c] (6.79379,2.04637) -- (6.80861,2.04637);
\definecolor{c}{rgb}{0,0,0};
\colorlet{c}{kugray};
\draw [c] (6.82343,1.84998) -- (6.82343,1.96142);
\draw [c] (6.82343,1.96142) -- (6.82343,2.04307);
\draw [c] (6.80861,1.96142) -- (6.82343,1.96142);
\draw [c] (6.82343,1.96142) -- (6.83824,1.96142);
\definecolor{c}{rgb}{0,0,0};
\colorlet{c}{kugray};
\draw [c] (6.85306,1.97353) -- (6.85306,2.06824);
\draw [c] (6.85306,2.06824) -- (6.85306,2.14058);
\draw [c] (6.83824,2.06824) -- (6.85306,2.06824);
\draw [c] (6.85306,2.06824) -- (6.86788,2.06824);
\definecolor{c}{rgb}{0,0,0};
\colorlet{c}{kugray};
\draw [c] (6.8827,1.94959) -- (6.8827,2.05345);
\draw [c] (6.8827,2.05345) -- (6.8827,2.13097);
\draw [c] (6.86788,2.05345) -- (6.8827,2.05345);
\draw [c] (6.8827,2.05345) -- (6.89752,2.05345);
\definecolor{c}{rgb}{0,0,0};
\colorlet{c}{kugray};
\draw [c] (6.91233,1.64738) -- (6.91233,1.796);
\draw [c] (6.91233,1.796) -- (6.91233,1.8958);
\draw [c] (6.89752,1.796) -- (6.91233,1.796);
\draw [c] (6.91233,1.796) -- (6.92715,1.796);
\definecolor{c}{rgb}{0,0,0};
\colorlet{c}{kugray};
\draw [c] (6.94197,1.81861) -- (6.94197,1.94633);
\draw [c] (6.94197,1.94633) -- (6.94197,2.03632);
\draw [c] (6.92715,1.94633) -- (6.94197,1.94633);
\draw [c] (6.94197,1.94633) -- (6.95679,1.94633);
\definecolor{c}{rgb}{0,0,0};
\colorlet{c}{kugray};
\draw [c] (6.97161,1.45467) -- (6.97161,1.67503);
\draw [c] (6.97161,1.67503) -- (6.97161,1.80215);
\draw [c] (6.95679,1.67503) -- (6.97161,1.67503);
\draw [c] (6.97161,1.67503) -- (6.98642,1.67503);
\definecolor{c}{rgb}{0,0,0};
\colorlet{c}{kugray};
\draw [c] (7.00124,1.92607) -- (7.00124,2.03252);
\draw [c] (7.00124,2.03252) -- (7.00124,2.11147);
\draw [c] (6.98642,2.03252) -- (7.00124,2.03252);
\draw [c] (7.00124,2.03252) -- (7.01606,2.03252);
\definecolor{c}{rgb}{0,0,0};
\colorlet{c}{kugray};
\draw [c] (7.03088,1.70142) -- (7.03088,1.83023);
\draw [c] (7.03088,1.83023) -- (7.03088,1.92076);
\draw [c] (7.01606,1.83023) -- (7.03088,1.83023);
\draw [c] (7.03088,1.83023) -- (7.0457,1.83023);
\definecolor{c}{rgb}{0,0,0};
\colorlet{c}{kugray};
\draw [c] (7.06051,1.69915) -- (7.06051,1.84157);
\draw [c] (7.06051,1.84157) -- (7.06051,1.93856);
\draw [c] (7.0457,1.84157) -- (7.06051,1.84157);
\draw [c] (7.06051,1.84157) -- (7.07533,1.84157);
\definecolor{c}{rgb}{0,0,0};
\colorlet{c}{kugray};
\draw [c] (7.09015,1.30062) -- (7.09015,1.58241);
\draw [c] (7.09015,1.58241) -- (7.09015,1.72703);
\draw [c] (7.07533,1.58241) -- (7.09015,1.58241);
\draw [c] (7.09015,1.58241) -- (7.10497,1.58241);
\definecolor{c}{rgb}{0,0,0};
\colorlet{c}{kugray};
\draw [c] (7.11978,1.6185) -- (7.11978,1.77873);
\draw [c] (7.11978,1.77873) -- (7.11978,1.88358);
\draw [c] (7.10497,1.77873) -- (7.11978,1.77873);
\draw [c] (7.11978,1.77873) -- (7.1346,1.77873);
\definecolor{c}{rgb}{0,0,0};
\colorlet{c}{kugray};
\draw [c] (7.14942,1.83806) -- (7.14942,1.95138);
\draw [c] (7.14942,1.95138) -- (7.14942,2.03403);
\draw [c] (7.1346,1.95138) -- (7.14942,1.95138);
\draw [c] (7.14942,1.95138) -- (7.16424,1.95138);
\definecolor{c}{rgb}{0,0,0};
\colorlet{c}{kugray};
\draw [c] (7.17906,1.68124) -- (7.17906,1.87494);
\draw [c] (7.17906,1.87494) -- (7.17906,1.99292);
\draw [c] (7.16424,1.87494) -- (7.17906,1.87494);
\draw [c] (7.17906,1.87494) -- (7.19387,1.87494);
\definecolor{c}{rgb}{0,0,0};
\colorlet{c}{kugray};
\draw [c] (7.20869,1.63815) -- (7.20869,1.80529);
\draw [c] (7.20869,1.80529) -- (7.20869,1.91302);
\draw [c] (7.19387,1.80529) -- (7.20869,1.80529);
\draw [c] (7.20869,1.80529) -- (7.22351,1.80529);
\definecolor{c}{rgb}{0,0,0};
\colorlet{c}{kugray};
\draw [c] (7.23833,1.534) -- (7.23833,1.72717);
\draw [c] (7.23833,1.72717) -- (7.23833,1.84495);
\draw [c] (7.22351,1.72717) -- (7.23833,1.72717);
\draw [c] (7.23833,1.72717) -- (7.25315,1.72717);
\definecolor{c}{rgb}{0,0,0};
\colorlet{c}{kugray};
\draw [c] (7.26796,1.44669) -- (7.26796,1.66186);
\draw [c] (7.26796,1.66186) -- (7.26796,1.78728);
\draw [c] (7.25315,1.66186) -- (7.26796,1.66186);
\draw [c] (7.26796,1.66186) -- (7.28278,1.66186);
\definecolor{c}{rgb}{0,0,0};
\colorlet{c}{kugray};
\draw [c] (7.2976,1.67008) -- (7.2976,1.80777);
\draw [c] (7.2976,1.80777) -- (7.2976,1.90257);
\draw [c] (7.28278,1.80777) -- (7.2976,1.80777);
\draw [c] (7.2976,1.80777) -- (7.31242,1.80777);
\definecolor{c}{rgb}{0,0,0};
\colorlet{c}{kugray};
\draw [c] (7.32724,1.55019) -- (7.32724,1.78423);
\draw [c] (7.32724,1.78423) -- (7.32724,1.91564);
\draw [c] (7.31242,1.78423) -- (7.32724,1.78423);
\draw [c] (7.32724,1.78423) -- (7.34205,1.78423);
\definecolor{c}{rgb}{0,0,0};
\colorlet{c}{kugray};
\draw [c] (7.35687,1.70986) -- (7.35687,1.88005);
\draw [c] (7.35687,1.88005) -- (7.35687,1.98902);
\draw [c] (7.34205,1.88005) -- (7.35687,1.88005);
\draw [c] (7.35687,1.88005) -- (7.37169,1.88005);
\definecolor{c}{rgb}{0,0,0};
\colorlet{c}{kugray};
\draw [c] (7.38651,1.39569) -- (7.38651,1.61683);
\draw [c] (7.38651,1.61683) -- (7.38651,1.7442);
\draw [c] (7.37169,1.61683) -- (7.38651,1.61683);
\draw [c] (7.38651,1.61683) -- (7.40132,1.61683);
\definecolor{c}{rgb}{0,0,0};
\colorlet{c}{kugray};
\draw [c] (7.41614,1.65215) -- (7.41614,1.82337);
\draw [c] (7.41614,1.82337) -- (7.41614,1.93274);
\draw [c] (7.40132,1.82337) -- (7.41614,1.82337);
\draw [c] (7.41614,1.82337) -- (7.43096,1.82337);
\definecolor{c}{rgb}{0,0,0};
\colorlet{c}{kugray};
\draw [c] (7.44578,1.61682) -- (7.44578,1.76694);
\draw [c] (7.44578,1.76694) -- (7.44578,1.86741);
\draw [c] (7.43096,1.76694) -- (7.44578,1.76694);
\draw [c] (7.44578,1.76694) -- (7.4606,1.76694);
\definecolor{c}{rgb}{0,0,0};
\colorlet{c}{kugray};
\draw [c] (7.47541,0.903192) -- (7.47541,1.30472);
\draw [c] (7.47541,1.30472) -- (7.47541,1.47322);
\draw [c] (7.4606,1.30472) -- (7.47541,1.30472);
\draw [c] (7.47541,1.30472) -- (7.49023,1.30472);
\definecolor{c}{rgb}{0,0,0};
\colorlet{c}{kugray};
\draw [c] (7.50505,1.17257) -- (7.50505,1.55106);
\draw [c] (7.50505,1.55106) -- (7.50505,1.71578);
\draw [c] (7.49023,1.55106) -- (7.50505,1.55106);
\draw [c] (7.50505,1.55106) -- (7.51987,1.55106);
\definecolor{c}{rgb}{0,0,0};
\colorlet{c}{kugray};
\draw [c] (7.53469,1.26039) -- (7.53469,1.5276);
\draw [c] (7.53469,1.5276) -- (7.53469,1.66846);
\draw [c] (7.51987,1.5276) -- (7.53469,1.5276);
\draw [c] (7.53469,1.5276) -- (7.5495,1.5276);
\definecolor{c}{rgb}{0,0,0};
\colorlet{c}{kugray};
\draw [c] (7.56432,1.35089) -- (7.56432,1.65389);
\draw [c] (7.56432,1.65389) -- (7.56432,1.80359);
\draw [c] (7.5495,1.65389) -- (7.56432,1.65389);
\draw [c] (7.56432,1.65389) -- (7.57914,1.65389);
\definecolor{c}{rgb}{0,0,0};
\colorlet{c}{kugray};
\draw [c] (7.59396,1.63685) -- (7.59396,1.81433);
\draw [c] (7.59396,1.81433) -- (7.59396,1.92619);
\draw [c] (7.57914,1.81433) -- (7.59396,1.81433);
\draw [c] (7.59396,1.81433) -- (7.60877,1.81433);
\definecolor{c}{rgb}{0,0,0};
\colorlet{c}{kugray};
\draw [c] (7.62359,1.2207) -- (7.62359,1.48705);
\draw [c] (7.62359,1.48705) -- (7.62359,1.62768);
\draw [c] (7.60877,1.48705) -- (7.62359,1.48705);
\draw [c] (7.62359,1.48705) -- (7.63841,1.48705);
\definecolor{c}{rgb}{0,0,0};
\colorlet{c}{kugray};
\draw [c] (7.65323,1.15339) -- (7.65323,1.44836);
\draw [c] (7.65323,1.44836) -- (7.65323,1.5962);
\draw [c] (7.63841,1.44836) -- (7.65323,1.44836);
\draw [c] (7.65323,1.44836) -- (7.66805,1.44836);
\definecolor{c}{rgb}{0,0,0};
\colorlet{c}{kugray};
\draw [c] (7.68286,0.824113) -- (7.68286,1.32118);
\draw [c] (7.68286,1.32118) -- (7.68286,1.50231);
\draw [c] (7.66805,1.32118) -- (7.68286,1.32118);
\draw [c] (7.68286,1.32118) -- (7.69768,1.32118);
\definecolor{c}{rgb}{0,0,0};
\colorlet{c}{kugray};
\draw [c] (7.7125,1.26126) -- (7.7125,1.5339);
\draw [c] (7.7125,1.5339) -- (7.7125,1.67619);
\draw [c] (7.69768,1.5339) -- (7.7125,1.5339);
\draw [c] (7.7125,1.5339) -- (7.72732,1.5339);
\definecolor{c}{rgb}{0,0,0};
\colorlet{c}{kugray};
\draw [c] (7.74214,1.48045) -- (7.74214,1.72479);
\draw [c] (7.74214,1.72479) -- (7.74214,1.85928);
\draw [c] (7.72732,1.72479) -- (7.74214,1.72479);
\draw [c] (7.74214,1.72479) -- (7.75695,1.72479);
\definecolor{c}{rgb}{0,0,0};
\colorlet{c}{kugray};
\draw [c] (7.77177,1.11205) -- (7.77177,1.40937);
\draw [c] (7.77177,1.40937) -- (7.77177,1.55776);
\draw [c] (7.75695,1.40937) -- (7.77177,1.40937);
\draw [c] (7.77177,1.40937) -- (7.78659,1.40937);
\definecolor{c}{rgb}{0,0,0};
\colorlet{c}{kugray};
\draw [c] (7.80141,1.34931) -- (7.80141,1.63981);
\draw [c] (7.80141,1.63981) -- (7.80141,1.78657);
\draw [c] (7.78659,1.63981) -- (7.80141,1.63981);
\draw [c] (7.80141,1.63981) -- (7.81623,1.63981);
\definecolor{c}{rgb}{0,0,0};
\colorlet{c}{kugray};
\draw [c] (7.83104,1.0292) -- (7.83104,1.41463);
\draw [c] (7.83104,1.41463) -- (7.83104,1.58052);
\draw [c] (7.81623,1.41463) -- (7.83104,1.41463);
\draw [c] (7.83104,1.41463) -- (7.84586,1.41463);
\definecolor{c}{rgb}{0,0,0};
\colorlet{c}{kugray};
\draw [c] (7.86068,0.596817) -- (7.86068,1.16974);
\draw [c] (7.86068,1.16974) -- (7.86068,1.38313);
\draw [c] (7.84586,1.16974) -- (7.86068,1.16974);
\draw [c] (7.86068,1.16974) -- (7.8755,1.16974);
\definecolor{c}{rgb}{0,0,0};
\colorlet{c}{kugray};
\draw [c] (7.89031,1.22506) -- (7.89031,1.49977);
\draw [c] (7.89031,1.49977) -- (7.89031,1.64259);
\draw [c] (7.8755,1.49977) -- (7.89031,1.49977);
\draw [c] (7.89031,1.49977) -- (7.90513,1.49977);
\definecolor{c}{rgb}{0,0,0};
\colorlet{c}{kugray};
\draw [c] (7.91995,1.37554) -- (7.91995,1.58913);
\draw [c] (7.91995,1.58913) -- (7.91995,1.71402);
\draw [c] (7.90513,1.58913) -- (7.91995,1.58913);
\draw [c] (7.91995,1.58913) -- (7.93477,1.58913);
\definecolor{c}{rgb}{0,0,0};
\colorlet{c}{kugray};
\draw [c] (7.94959,0.970556) -- (7.94959,1.35289);
\draw [c] (7.94959,1.35289) -- (7.94959,1.51826);
\draw [c] (7.93477,1.35289) -- (7.94959,1.35289);
\draw [c] (7.94959,1.35289) -- (7.9644,1.35289);
\definecolor{c}{rgb}{0,0,0};
\colorlet{c}{kugray};
\draw [c] (7.97922,1.42663) -- (7.97922,1.62932);
\draw [c] (7.97922,1.62932) -- (7.97922,1.75049);
\draw [c] (7.9644,1.62932) -- (7.97922,1.62932);
\draw [c] (7.97922,1.62932) -- (7.99404,1.62932);
\definecolor{c}{rgb}{0,0,0};
\colorlet{c}{kugray};
\draw [c] (8.00886,1.24387) -- (8.00886,1.51218);
\draw [c] (8.00886,1.51218) -- (8.00886,1.65334);
\draw [c] (7.99404,1.51218) -- (8.00886,1.51218);
\draw [c] (8.00886,1.51218) -- (8.02368,1.51218);
\definecolor{c}{rgb}{0,0,0};
\colorlet{c}{kugray};
\draw [c] (8.03849,1.45206) -- (8.03849,1.64214);
\draw [c] (8.03849,1.64214) -- (8.03849,1.75879);
\draw [c] (8.02368,1.64214) -- (8.03849,1.64214);
\draw [c] (8.03849,1.64214) -- (8.05331,1.64214);
\definecolor{c}{rgb}{0,0,0};
\colorlet{c}{kugray};
\draw [c] (8.09776,1.35157) -- (8.09776,1.5665);
\draw [c] (8.09776,1.5665) -- (8.09776,1.69184);
\draw [c] (8.08295,1.5665) -- (8.09776,1.5665);
\draw [c] (8.09776,1.5665) -- (8.11258,1.5665);
\definecolor{c}{rgb}{0,0,0};
\colorlet{c}{kugray};
\draw [c] (8.1274,1.02943) -- (8.1274,1.41864);
\draw [c] (8.1274,1.41864) -- (8.1274,1.58516);
\draw [c] (8.11258,1.41864) -- (8.1274,1.41864);
\draw [c] (8.1274,1.41864) -- (8.14222,1.41864);
\definecolor{c}{rgb}{0,0,0};
\colorlet{c}{kugray};
\draw [c] (8.18667,1.3217) -- (8.18667,1.60896);
\draw [c] (8.18667,1.60896) -- (8.18667,1.75493);
\draw [c] (8.17185,1.60896) -- (8.18667,1.60896);
\draw [c] (8.18667,1.60896) -- (8.20149,1.60896);
\definecolor{c}{rgb}{0,0,0};
\colorlet{c}{kugray};
\draw [c] (8.21631,1.44991) -- (8.21631,1.72488);
\draw [c] (8.21631,1.72488) -- (8.21631,1.86777);
\draw [c] (8.20149,1.72488) -- (8.21631,1.72488);
\draw [c] (8.21631,1.72488) -- (8.23113,1.72488);
\definecolor{c}{rgb}{0,0,0};
\colorlet{c}{kugray};
\draw [c] (8.24594,0.596817) -- (8.24594,1.18479);
\draw [c] (8.24594,1.18479) -- (8.24594,1.39818);
\draw [c] (8.23113,1.18479) -- (8.24594,1.18479);
\draw [c] (8.24594,1.18479) -- (8.26076,1.18479);
\definecolor{c}{rgb}{0,0,0};
\colorlet{c}{kugray};
\draw [c] (8.27558,1.33422) -- (8.27558,1.70646);
\draw [c] (8.27558,1.70646) -- (8.27558,1.8701);
\draw [c] (8.26076,1.70646) -- (8.27558,1.70646);
\draw [c] (8.27558,1.70646) -- (8.2904,1.70646);
\definecolor{c}{rgb}{0,0,0};
\colorlet{c}{kugray};
\draw [c] (8.30521,1.01154) -- (8.30521,1.39412);
\draw [c] (8.30521,1.39412) -- (8.30521,1.55954);
\draw [c] (8.2904,1.39412) -- (8.30521,1.39412);
\draw [c] (8.30521,1.39412) -- (8.32003,1.39412);
\definecolor{c}{rgb}{0,0,0};
\colorlet{c}{kugray};
\draw [c] (8.33485,1.16538) -- (8.33485,1.451);
\draw [c] (8.33485,1.451) -- (8.33485,1.59657);
\draw [c] (8.32003,1.451) -- (8.33485,1.451);
\draw [c] (8.33485,1.451) -- (8.34967,1.451);
\definecolor{c}{rgb}{0,0,0};
\colorlet{c}{kugray};
\draw [c] (8.36449,0.596817) -- (8.36449,1.27192);
\draw [c] (8.36449,1.27192) -- (8.36449,1.48532);
\draw [c] (8.34967,1.27192) -- (8.36449,1.27192);
\draw [c] (8.36449,1.27192) -- (8.3793,1.27192);
\definecolor{c}{rgb}{0,0,0};
\colorlet{c}{kugray};
\draw [c] (8.39412,0.596817) -- (8.39412,1.1669);
\draw [c] (8.39412,1.1669) -- (8.39412,1.38029);
\draw [c] (8.3793,1.1669) -- (8.39412,1.1669);
\draw [c] (8.39412,1.1669) -- (8.40894,1.1669);
\definecolor{c}{rgb}{0,0,0};
\colorlet{c}{kugray};
\draw [c] (8.45339,1.13633) -- (8.45339,1.56977);
\draw [c] (8.45339,1.56977) -- (8.45339,1.74299);
\draw [c] (8.43858,1.56977) -- (8.45339,1.56977);
\draw [c] (8.45339,1.56977) -- (8.46821,1.56977);
\definecolor{c}{rgb}{0,0,0};
\colorlet{c}{kugray};
\draw [c] (8.48303,0.596817) -- (8.48303,1.22594);
\draw [c] (8.48303,1.22594) -- (8.48303,1.43933);
\draw [c] (8.46821,1.22594) -- (8.48303,1.22594);
\draw [c] (8.48303,1.22594) -- (8.49785,1.22594);
\definecolor{c}{rgb}{0,0,0};
\colorlet{c}{kugray};
\draw [c] (8.5423,0.596817) -- (8.5423,1.08304);
\draw [c] (8.5423,1.08304) -- (8.5423,1.29643);
\draw [c] (8.52748,1.08304) -- (8.5423,1.08304);
\draw [c] (8.5423,1.08304) -- (8.55712,1.08304);
\definecolor{c}{rgb}{0,0,0};
\colorlet{c}{kugray};
\draw [c] (8.57194,0.596817) -- (8.57194,1.19595);
\draw [c] (8.57194,1.19595) -- (8.57194,1.40934);
\draw [c] (8.55712,1.19595) -- (8.57194,1.19595);
\draw [c] (8.57194,1.19595) -- (8.58675,1.19595);
\definecolor{c}{rgb}{0,0,0};
\colorlet{c}{kugray};
\draw [c] (8.66084,0.884632) -- (8.66084,1.27717);
\draw [c] (8.66084,1.27717) -- (8.66084,1.44423);
\draw [c] (8.64603,1.27717) -- (8.66084,1.27717);
\draw [c] (8.66084,1.27717) -- (8.67566,1.27717);
\definecolor{c}{rgb}{0,0,0};
\colorlet{c}{kugray};
\draw [c] (8.69048,0.596817) -- (8.69048,1.18479);
\draw [c] (8.69048,1.18479) -- (8.69048,1.39818);
\draw [c] (8.67566,1.18479) -- (8.69048,1.18479);
\draw [c] (8.69048,1.18479) -- (8.7053,1.18479);
\definecolor{c}{rgb}{0,0,0};
\colorlet{c}{kugray};
\draw [c] (8.72012,0.596817) -- (8.72012,1.21188);
\draw [c] (8.72012,1.21188) -- (8.72012,1.42527);
\draw [c] (8.7053,1.21188) -- (8.72012,1.21188);
\draw [c] (8.72012,1.21188) -- (8.73493,1.21188);
\definecolor{c}{rgb}{0,0,0};
\colorlet{c}{kugray};
\draw [c] (8.74975,0.596817) -- (8.74975,1.16598);
\draw [c] (8.74975,1.16598) -- (8.74975,1.37937);
\draw [c] (8.73493,1.16598) -- (8.74975,1.16598);
\draw [c] (8.74975,1.16598) -- (8.76457,1.16598);
\definecolor{c}{rgb}{0,0,0};
\colorlet{c}{kugray};
\draw [c] (8.77939,0.895648) -- (8.77939,1.31039);
\draw [c] (8.77939,1.31039) -- (8.77939,1.48091);
\draw [c] (8.76457,1.31039) -- (8.77939,1.31039);
\draw [c] (8.77939,1.31039) -- (8.79421,1.31039);
\definecolor{c}{rgb}{0,0,0};
\colorlet{c}{kugray};
\draw [c] (8.80902,1.29252) -- (8.80902,1.5583);
\draw [c] (8.80902,1.5583) -- (8.80902,1.69878);
\draw [c] (8.79421,1.5583) -- (8.80902,1.5583);
\draw [c] (8.80902,1.5583) -- (8.82384,1.5583);
\definecolor{c}{rgb}{0,0,0};
\colorlet{c}{kugray};
\draw [c] (8.83866,0.596817) -- (8.83866,1.24072);
\draw [c] (8.83866,1.24072) -- (8.83866,1.45411);
\draw [c] (8.82384,1.24072) -- (8.83866,1.24072);
\draw [c] (8.83866,1.24072) -- (8.85348,1.24072);
\definecolor{c}{rgb}{0,0,0};
\colorlet{c}{kugray};
\draw [c] (8.86829,0.596817) -- (8.86829,0.976443);
\draw [c] (8.86829,0.976443) -- (8.86829,1.18983);
\draw [c] (8.85348,0.976443) -- (8.86829,0.976443);
\draw [c] (8.86829,0.976443) -- (8.88311,0.976443);
\definecolor{c}{rgb}{0,0,0};
\colorlet{c}{kugray};
\draw [c] (8.92757,0.596817) -- (8.92757,1.19654);
\draw [c] (8.92757,1.19654) -- (8.92757,1.40993);
\draw [c] (8.91275,1.19654) -- (8.92757,1.19654);
\draw [c] (8.92757,1.19654) -- (8.94238,1.19654);
\definecolor{c}{rgb}{0,0,0};
\colorlet{c}{kugray};
\draw [c] (9.01647,0.596817) -- (9.01647,1.23526);
\draw [c] (9.01647,1.23526) -- (9.01647,1.44865);
\draw [c] (9.00166,1.23526) -- (9.01647,1.23526);
\draw [c] (9.01647,1.23526) -- (9.03129,1.23526);
\definecolor{c}{rgb}{0,0,0};
\colorlet{c}{kugray};
\draw [c] (9.07574,0.596817) -- (9.07574,1.10612);
\draw [c] (9.07574,1.10612) -- (9.07574,1.31951);
\draw [c] (9.06093,1.10612) -- (9.07574,1.10612);
\draw [c] (9.07574,1.10612) -- (9.09056,1.10612);
\definecolor{c}{rgb}{0,0,0};
\colorlet{c}{kugray};
\draw [c] (9.10538,0.596817) -- (9.10538,1.18963);
\draw [c] (9.10538,1.18963) -- (9.10538,1.40302);
\draw [c] (9.09056,1.18963) -- (9.10538,1.18963);
\draw [c] (9.10538,1.18963) -- (9.1202,1.18963);
\definecolor{c}{rgb}{0,0,0};
\colorlet{c}{kugray};
\draw [c] (9.13502,0.596817) -- (9.13502,1.14382);
\draw [c] (9.13502,1.14382) -- (9.13502,1.35721);
\draw [c] (9.1202,1.14382) -- (9.13502,1.14382);
\draw [c] (9.13502,1.14382) -- (9.14983,1.14382);
\definecolor{c}{rgb}{0,0,0};
\colorlet{c}{kugray};
\draw [c] (9.22392,1.0928) -- (9.22392,1.47755);
\draw [c] (9.22392,1.47755) -- (9.22392,1.64333);
\draw [c] (9.20911,1.47755) -- (9.22392,1.47755);
\draw [c] (9.22392,1.47755) -- (9.23874,1.47755);
\definecolor{c}{rgb}{0,0,0};
\colorlet{c}{kugray};
\draw [c] (9.2832,0.596817) -- (9.2832,1.19105);
\draw [c] (9.2832,1.19105) -- (9.2832,1.40444);
\draw [c] (9.26838,1.19105) -- (9.2832,1.19105);
\draw [c] (9.2832,1.19105) -- (9.29801,1.19105);
\definecolor{c}{rgb}{0,0,0};
\colorlet{c}{kugray};
\draw [c] (9.34247,0.596817) -- (9.34247,1.0722);
\draw [c] (9.34247,1.0722) -- (9.34247,1.28559);
\draw [c] (9.32765,1.0722) -- (9.34247,1.0722);
\draw [c] (9.34247,1.0722) -- (9.35728,1.0722);
\definecolor{c}{rgb}{0,0,0};
\colorlet{c}{kugray};
\draw [c] (9.54992,0.596817) -- (9.54992,1.17593);
\draw [c] (9.54992,1.17593) -- (9.54992,1.38932);
\draw [c] (9.5351,1.17593) -- (9.54992,1.17593);
\draw [c] (9.54992,1.17593) -- (9.56474,1.17593);
\definecolor{c}{rgb}{0,0,0};
\colorlet{c}{kugray};
\draw [c] (9.60919,0.596817) -- (9.60919,1.45456);
\draw [c] (9.60919,1.45456) -- (9.60919,1.66795);
\draw [c] (9.59437,1.45456) -- (9.60919,1.45456);
\draw [c] (9.60919,1.45456) -- (9.62401,1.45456);
\definecolor{c}{rgb}{0,0,0};
\colorlet{c}{kugray};
\draw [c] (9.63882,0.596817) -- (9.63882,1.11893);
\draw [c] (9.63882,1.11893) -- (9.63882,1.33232);
\draw [c] (9.62401,1.11893) -- (9.63882,1.11893);
\draw [c] (9.63882,1.11893) -- (9.65364,1.11893);
\definecolor{c}{rgb}{0,0,0};
\colorlet{c}{kugray};
\draw [c] (9.72773,0.596817) -- (9.72773,1.59979);
\draw [c] (9.72773,1.59979) -- (9.72773,1.81318);
\draw [c] (9.71291,1.59979) -- (9.72773,1.59979);
\draw [c] (9.72773,1.59979) -- (9.74255,1.59979);
\definecolor{c}{rgb}{0,0,0};
\colorlet{c}{kugray};
\draw [c] (9.787,0.596817) -- (9.787,1.11893);
\draw [c] (9.787,1.11893) -- (9.787,1.33232);
\draw [c] (9.77219,1.11893) -- (9.787,1.11893);
\draw [c] (9.787,1.11893) -- (9.80182,1.11893);
\definecolor{c}{rgb}{0,0,0};
\colorlet{c}{kugray};
\draw [c] (9.90555,0.596817) -- (9.90555,1.17321);
\draw [c] (9.90555,1.17321) -- (9.90555,1.3866);
\draw [c] (9.89073,1.17321) -- (9.90555,1.17321);
\draw [c] (9.90555,1.17321) -- (9.92036,1.17321);
\definecolor{c}{rgb}{0,0,0};
\colorlet{c}{kugray};
\draw [c] (9.93518,0.596817) -- (9.93518,1.17077);
\draw [c] (9.93518,1.17077) -- (9.93518,1.38417);
\draw [c] (9.92036,1.17077) -- (9.93518,1.17077);
\draw [c] (9.93518,1.17077) -- (9.95,1.17077);
\definecolor{c}{rgb}{0,0,0};
\colorlet{c}{natgreen};
\draw [c] (1.51655,5.54533) -- (1.60131,5.40419) -- (1.68607,5.2747) -- (1.77083,5.15478) -- (1.85558,5.04288) -- (1.94034,4.93778) -- (2.0251,4.83855) -- (2.10986,4.74443) -- (2.19462,4.65481) -- (2.27938,4.56919) -- (2.36413,4.48712)
 -- (2.44889,4.40827) -- (2.53365,4.33233) -- (2.61841,4.25902) -- (2.70317,4.18813) -- (2.78793,4.11945) -- (2.87268,4.05281) -- (2.95744,3.98806) -- (3.0422,3.92505) -- (3.12696,3.86367) -- (3.21172,3.80381) -- (3.29648,3.74536) --
 (3.38123,3.68824) -- (3.46599,3.63236) -- (3.55075,3.57765) -- (3.63551,3.52404) -- (3.72027,3.47148) -- (3.80502,3.4199) -- (3.88978,3.36925) -- (3.97454,3.31949) -- (4.0593,3.27056) -- (4.14406,3.22243) -- (4.22882,3.17506) -- (4.31357,3.12841) --
 (4.39833,3.08245) -- (4.48309,3.03714) -- (4.56785,2.99247) -- (4.65261,2.94839) -- (4.73737,2.90489) -- (4.82212,2.86194) -- (4.90688,2.81952) -- (4.99164,2.7776) -- (5.0764,2.73617) -- (5.16116,2.6952) -- (5.24592,2.65468) -- (5.33067,2.6146) --
 (5.41543,2.57492) -- (5.50019,2.53565) -- (5.58495,2.49676) -- (5.66971,2.45824);
\draw [c] (5.66971,2.45824) -- (5.75447,2.42007) -- (5.83922,2.38225) -- (5.92398,2.34476) -- (6.00874,2.30759) -- (6.0935,2.27073) -- (6.17826,2.23416) -- (6.26301,2.19789) -- (6.34777,2.16188) -- (6.43253,2.12615) --
 (6.51729,2.09068) -- (6.60205,2.05545) -- (6.68681,2.02047) -- (6.77156,1.98572) -- (6.85632,1.9512) -- (6.94108,1.91689) -- (7.02584,1.88279) -- (7.1106,1.8489) -- (7.19536,1.81521) -- (7.28011,1.78171) -- (7.36487,1.74839) -- (7.44963,1.71525) --
 (7.53439,1.68228) -- (7.61915,1.64948) -- (7.70391,1.61684) -- (7.78866,1.58435) -- (7.87342,1.55202) -- (7.95818,1.51983) -- (8.04294,1.48778) -- (8.1277,1.45586) -- (8.21246,1.42408) -- (8.29721,1.39242) -- (8.38197,1.36088) -- (8.46673,1.32946)
 -- (8.55149,1.29816) -- (8.63625,1.26696) -- (8.72101,1.23587) -- (8.80576,1.20487) -- (8.89052,1.17398) -- (8.97528,1.14317) -- (9.06004,1.11246) -- (9.1448,1.08183) -- (9.22955,1.05129) -- (9.31431,1.02082) -- (9.39907,0.990431) --
 (9.48383,0.960111) -- (9.56859,0.929859) -- (9.65335,0.899673) -- (9.7381,0.869549) -- (9.82286,0.839483);
\draw [c] (9.82286,0.839483) -- (9.90762,0.809473);
\colorlet{c}{kugray};
\draw [c] (1.07409,0.596817) -- (1.07409,1.40897);
\draw [c] (1.07409,1.40897) -- (1.07409,1.62236);
\draw [c] (1.05927,1.40897) -- (1.07409,1.40897);
\draw [c] (1.07409,1.40897) -- (1.08891,1.40897);
\definecolor{c}{rgb}{0,0,0};
\colorlet{c}{kugray};
\draw [c] (1.10373,0.596817) -- (1.10373,3.52791);
\draw [c] (1.10373,3.52791) -- (1.10373,3.7413);
\draw [c] (1.08891,3.52791) -- (1.10373,3.52791);
\draw [c] (1.10373,3.52791) -- (1.11854,3.52791);
\definecolor{c}{rgb}{0,0,0};
\colorlet{c}{kugray};
\draw [c] (1.13336,3.30584) -- (1.13336,3.68425);
\draw [c] (1.13336,3.68425) -- (1.13336,3.84896);
\draw [c] (1.11854,3.68425) -- (1.13336,3.68425);
\draw [c] (1.13336,3.68425) -- (1.14818,3.68425);
\definecolor{c}{rgb}{0,0,0};
\colorlet{c}{kugray};
\draw [c] (1.163,0.596817) -- (1.163,3.53823);
\draw [c] (1.163,3.53823) -- (1.163,3.75162);
\draw [c] (1.14818,3.53823) -- (1.163,3.53823);
\draw [c] (1.163,3.53823) -- (1.17781,3.53823);
\definecolor{c}{rgb}{0,0,0};
\colorlet{c}{kugray};
\draw [c] (1.19263,3.61695) -- (1.19263,3.89008);
\draw [c] (1.19263,3.89008) -- (1.19263,4.03249);
\draw [c] (1.17781,3.89008) -- (1.19263,3.89008);
\draw [c] (1.19263,3.89008) -- (1.20745,3.89008);
\definecolor{c}{rgb}{0,0,0};
\colorlet{c}{kugray};
\draw [c] (1.22227,3.83075) -- (1.22227,4.01677);
\draw [c] (1.22227,4.01677) -- (1.22227,4.13191);
\draw [c] (1.20745,4.01677) -- (1.22227,4.01677);
\draw [c] (1.22227,4.01677) -- (1.23709,4.01677);
\definecolor{c}{rgb}{0,0,0};
\colorlet{c}{kugray};
\draw [c] (1.2519,3.71523) -- (1.2519,3.93196);
\draw [c] (1.2519,3.93196) -- (1.2519,4.05789);
\draw [c] (1.23709,3.93196) -- (1.2519,3.93196);
\draw [c] (1.2519,3.93196) -- (1.26672,3.93196);
\definecolor{c}{rgb}{0,0,0};
\colorlet{c}{kugray};
\draw [c] (1.28154,3.75372) -- (1.28154,3.97484);
\draw [c] (1.28154,3.97484) -- (1.28154,4.1022);
\draw [c] (1.26672,3.97484) -- (1.28154,3.97484);
\draw [c] (1.28154,3.97484) -- (1.29636,3.97484);
\definecolor{c}{rgb}{0,0,0};
\colorlet{c}{kugray};
\draw [c] (1.31118,5.36683) -- (1.31118,5.38443);
\draw [c] (1.31118,5.38443) -- (1.31118,5.40108);
\draw [c] (1.29636,5.38443) -- (1.31118,5.38443);
\draw [c] (1.31118,5.38443) -- (1.32599,5.38443);
\definecolor{c}{rgb}{0,0,0};
\colorlet{c}{kugray};
\draw [c] (1.34081,5.64934) -- (1.34081,5.66037);
\draw [c] (1.34081,5.66037) -- (1.34081,5.67102);
\draw [c] (1.32599,5.66037) -- (1.34081,5.66037);
\draw [c] (1.34081,5.66037) -- (1.35563,5.66037);
\definecolor{c}{rgb}{0,0,0};
\colorlet{c}{kugray};
\draw [c] (1.37045,5.69165) -- (1.37045,5.70189);
\draw [c] (1.37045,5.70189) -- (1.37045,5.7118);
\draw [c] (1.35563,5.70189) -- (1.37045,5.70189);
\draw [c] (1.37045,5.70189) -- (1.38526,5.70189);
\definecolor{c}{rgb}{0,0,0};
\colorlet{c}{kugray};
\draw [c] (1.40008,5.69477) -- (1.40008,5.70516);
\draw [c] (1.40008,5.70516) -- (1.40008,5.71522);
\draw [c] (1.38526,5.70516) -- (1.40008,5.70516);
\draw [c] (1.40008,5.70516) -- (1.4149,5.70516);
\definecolor{c}{rgb}{0,0,0};
\colorlet{c}{kugray};
\draw [c] (1.42972,5.6476) -- (1.42972,5.65871);
\draw [c] (1.42972,5.65871) -- (1.42972,5.66943);
\draw [c] (1.4149,5.65871) -- (1.42972,5.65871);
\draw [c] (1.42972,5.65871) -- (1.44454,5.65871);
\definecolor{c}{rgb}{0,0,0};
\colorlet{c}{kugray};
\draw [c] (1.45935,5.62032) -- (1.45935,5.63176);
\draw [c] (1.45935,5.63176) -- (1.45935,5.6428);
\draw [c] (1.44454,5.63176) -- (1.45935,5.63176);
\draw [c] (1.45935,5.63176) -- (1.47417,5.63176);
\definecolor{c}{rgb}{0,0,0};
\colorlet{c}{kugray};
\draw [c] (1.48899,5.56116) -- (1.48899,5.57382);
\draw [c] (1.48899,5.57382) -- (1.48899,5.58597);
\draw [c] (1.47417,5.57382) -- (1.48899,5.57382);
\draw [c] (1.48899,5.57382) -- (1.50381,5.57382);
\definecolor{c}{rgb}{0,0,0};
\colorlet{c}{kugray};
\draw [c] (1.51863,5.51885) -- (1.51863,5.53274);
\draw [c] (1.51863,5.53274) -- (1.51863,5.54604);
\draw [c] (1.50381,5.53274) -- (1.51863,5.53274);
\draw [c] (1.51863,5.53274) -- (1.53344,5.53274);
\definecolor{c}{rgb}{0,0,0};
\colorlet{c}{kugray};
\draw [c] (1.54826,5.46762) -- (1.54826,5.48249);
\draw [c] (1.54826,5.48249) -- (1.54826,5.49667);
\draw [c] (1.53344,5.48249) -- (1.54826,5.48249);
\draw [c] (1.54826,5.48249) -- (1.56308,5.48249);
\definecolor{c}{rgb}{0,0,0};
\colorlet{c}{kugray};
\draw [c] (1.5779,5.40546) -- (1.5779,5.42176);
\draw [c] (1.5779,5.42176) -- (1.5779,5.43724);
\draw [c] (1.56308,5.42176) -- (1.5779,5.42176);
\draw [c] (1.5779,5.42176) -- (1.59272,5.42176);
\definecolor{c}{rgb}{0,0,0};
\colorlet{c}{kugray};
\draw [c] (1.60753,5.41122) -- (1.60753,5.42791);
\draw [c] (1.60753,5.42791) -- (1.60753,5.44375);
\draw [c] (1.59272,5.42791) -- (1.60753,5.42791);
\draw [c] (1.60753,5.42791) -- (1.62235,5.42791);
\definecolor{c}{rgb}{0,0,0};
\colorlet{c}{kugray};
\draw [c] (1.63717,5.33643) -- (1.63717,5.35499);
\draw [c] (1.63717,5.35499) -- (1.63717,5.37248);
\draw [c] (1.62235,5.35499) -- (1.63717,5.35499);
\draw [c] (1.63717,5.35499) -- (1.65199,5.35499);
\definecolor{c}{rgb}{0,0,0};
\colorlet{c}{kugray};
\draw [c] (1.6668,5.32522) -- (1.6668,5.34431);
\draw [c] (1.6668,5.34431) -- (1.6668,5.36228);
\draw [c] (1.65199,5.34431) -- (1.6668,5.34431);
\draw [c] (1.6668,5.34431) -- (1.68162,5.34431);
\definecolor{c}{rgb}{0,0,0};
\colorlet{c}{kugray};
\draw [c] (1.69644,5.24802) -- (1.69644,5.26862);
\draw [c] (1.69644,5.26862) -- (1.69644,5.28793);
\draw [c] (1.68162,5.26862) -- (1.69644,5.26862);
\draw [c] (1.69644,5.26862) -- (1.71126,5.26862);
\definecolor{c}{rgb}{0,0,0};
\colorlet{c}{kugray};
\draw [c] (1.72608,5.18795) -- (1.72608,5.21178);
\draw [c] (1.72608,5.21178) -- (1.72608,5.23389);
\draw [c] (1.71126,5.21178) -- (1.72608,5.21178);
\draw [c] (1.72608,5.21178) -- (1.74089,5.21178);
\definecolor{c}{rgb}{0,0,0};
\colorlet{c}{kugray};
\draw [c] (1.75571,5.164) -- (1.75571,5.18723);
\draw [c] (1.75571,5.18723) -- (1.75571,5.20882);
\draw [c] (1.74089,5.18723) -- (1.75571,5.18723);
\draw [c] (1.75571,5.18723) -- (1.77053,5.18723);
\definecolor{c}{rgb}{0,0,0};
\colorlet{c}{kugray};
\draw [c] (1.78535,5.12784) -- (1.78535,5.15344);
\draw [c] (1.78535,5.15344) -- (1.78535,5.17707);
\draw [c] (1.77053,5.15344) -- (1.78535,5.15344);
\draw [c] (1.78535,5.15344) -- (1.80017,5.15344);
\definecolor{c}{rgb}{0,0,0};
\colorlet{c}{kugray};
\draw [c] (1.81498,5.07595) -- (1.81498,5.10297);
\draw [c] (1.81498,5.10297) -- (1.81498,5.1278);
\draw [c] (1.80017,5.10297) -- (1.81498,5.10297);
\draw [c] (1.81498,5.10297) -- (1.8298,5.10297);
\definecolor{c}{rgb}{0,0,0};
\colorlet{c}{kugray};
\draw [c] (1.84462,5.05921) -- (1.84462,5.0876);
\draw [c] (1.84462,5.0876) -- (1.84462,5.11359);
\draw [c] (1.8298,5.0876) -- (1.84462,5.0876);
\draw [c] (1.84462,5.0876) -- (1.85944,5.0876);
\definecolor{c}{rgb}{0,0,0};
\colorlet{c}{kugray};
\draw [c] (1.87425,4.98897) -- (1.87425,5.02142);
\draw [c] (1.87425,5.02142) -- (1.87425,5.05078);
\draw [c] (1.85944,5.02142) -- (1.87425,5.02142);
\draw [c] (1.87425,5.02142) -- (1.88907,5.02142);
\definecolor{c}{rgb}{0,0,0};
\colorlet{c}{kugray};
\draw [c] (1.90389,4.87515) -- (1.90389,4.914);
\draw [c] (1.90389,4.914) -- (1.90389,4.94849);
\draw [c] (1.88907,4.914) -- (1.90389,4.914);
\draw [c] (1.90389,4.914) -- (1.91871,4.914);
\definecolor{c}{rgb}{0,0,0};
\colorlet{c}{kugray};
\draw [c] (1.93353,4.89763) -- (1.93353,4.9339);
\draw [c] (1.93353,4.9339) -- (1.93353,4.96634);
\draw [c] (1.91871,4.9339) -- (1.93353,4.9339);
\draw [c] (1.93353,4.9339) -- (1.94834,4.9339);
\definecolor{c}{rgb}{0,0,0};
\colorlet{c}{kugray};
\draw [c] (1.96316,4.88868) -- (1.96316,4.92763);
\draw [c] (1.96316,4.92763) -- (1.96316,4.96221);
\draw [c] (1.94834,4.92763) -- (1.96316,4.92763);
\draw [c] (1.96316,4.92763) -- (1.97798,4.92763);
\definecolor{c}{rgb}{0,0,0};
\colorlet{c}{kugray};
\draw [c] (1.9928,4.78948) -- (1.9928,4.83208);
\draw [c] (1.9928,4.83208) -- (1.9928,4.8695);
\draw [c] (1.97798,4.83208) -- (1.9928,4.83208);
\draw [c] (1.9928,4.83208) -- (2.00762,4.83208);
\definecolor{c}{rgb}{0,0,0};
\colorlet{c}{kugray};
\draw [c] (2.02243,4.71758) -- (2.02243,4.76449);
\draw [c] (2.02243,4.76449) -- (2.02243,4.80519);
\draw [c] (2.00762,4.76449) -- (2.02243,4.76449);
\draw [c] (2.02243,4.76449) -- (2.03725,4.76449);
\definecolor{c}{rgb}{0,0,0};
\colorlet{c}{kugray};
\draw [c] (2.05207,4.70827) -- (2.05207,4.75577);
\draw [c] (2.05207,4.75577) -- (2.05207,4.79692);
\draw [c] (2.03725,4.75577) -- (2.05207,4.75577);
\draw [c] (2.05207,4.75577) -- (2.06689,4.75577);
\definecolor{c}{rgb}{0,0,0};
\colorlet{c}{kugray};
\draw [c] (2.08171,4.68783) -- (2.08171,4.73949);
\draw [c] (2.08171,4.73949) -- (2.08171,4.78371);
\draw [c] (2.06689,4.73949) -- (2.08171,4.73949);
\draw [c] (2.08171,4.73949) -- (2.09652,4.73949);
\definecolor{c}{rgb}{0,0,0};
\colorlet{c}{kugray};
\draw [c] (2.11134,4.66301) -- (2.11134,4.71819);
\draw [c] (2.11134,4.71819) -- (2.11134,4.76496);
\draw [c] (2.09652,4.71819) -- (2.11134,4.71819);
\draw [c] (2.11134,4.71819) -- (2.12616,4.71819);
\definecolor{c}{rgb}{0,0,0};
\colorlet{c}{kugray};
\draw [c] (2.14098,4.66355) -- (2.14098,4.71683);
\draw [c] (2.14098,4.71683) -- (2.14098,4.76224);
\draw [c] (2.12616,4.71683) -- (2.14098,4.71683);
\draw [c] (2.14098,4.71683) -- (2.15579,4.71683);
\definecolor{c}{rgb}{0,0,0};
\colorlet{c}{kugray};
\draw [c] (2.17061,4.50979) -- (2.17061,4.57305);
\draw [c] (2.17061,4.57305) -- (2.17061,4.6255);
\draw [c] (2.15579,4.57305) -- (2.17061,4.57305);
\draw [c] (2.17061,4.57305) -- (2.18543,4.57305);
\definecolor{c}{rgb}{0,0,0};
\colorlet{c}{kugray};
\draw [c] (2.20025,4.5238) -- (2.20025,4.5936);
\draw [c] (2.20025,4.5936) -- (2.20025,4.65046);
\draw [c] (2.18543,4.5936) -- (2.20025,4.5936);
\draw [c] (2.20025,4.5936) -- (2.21507,4.5936);
\definecolor{c}{rgb}{0,0,0};
\colorlet{c}{kugray};
\draw [c] (2.22988,4.49841) -- (2.22988,4.56586);
\draw [c] (2.22988,4.56586) -- (2.22988,4.62114);
\draw [c] (2.21507,4.56586) -- (2.22988,4.56586);
\draw [c] (2.22988,4.56586) -- (2.2447,4.56586);
\definecolor{c}{rgb}{0,0,0};
\colorlet{c}{kugray};
\draw [c] (2.25952,4.51369) -- (2.25952,4.58591);
\draw [c] (2.25952,4.58591) -- (2.25952,4.64436);
\draw [c] (2.2447,4.58591) -- (2.25952,4.58591);
\draw [c] (2.25952,4.58591) -- (2.27434,4.58591);
\definecolor{c}{rgb}{0,0,0};
\colorlet{c}{kugray};
\draw [c] (2.28916,4.55805) -- (2.28916,4.56892);
\draw [c] (2.28916,4.56892) -- (2.28916,4.57942);
\draw [c] (2.27434,4.56892) -- (2.28916,4.56892);
\draw [c] (2.28916,4.56892) -- (2.30397,4.56892);
\definecolor{c}{rgb}{0,0,0};
\colorlet{c}{kugray};
\draw [c] (2.31879,4.50775) -- (2.31879,4.51917);
\draw [c] (2.31879,4.51917) -- (2.31879,4.53019);
\draw [c] (2.30397,4.51917) -- (2.31879,4.51917);
\draw [c] (2.31879,4.51917) -- (2.33361,4.51917);
\definecolor{c}{rgb}{0,0,0};
\colorlet{c}{kugray};
\draw [c] (2.34843,4.49289) -- (2.34843,4.50481);
\draw [c] (2.34843,4.50481) -- (2.34843,4.5163);
\draw [c] (2.33361,4.50481) -- (2.34843,4.50481);
\draw [c] (2.34843,4.50481) -- (2.36325,4.50481);
\definecolor{c}{rgb}{0,0,0};
\colorlet{c}{kugray};
\draw [c] (2.37806,4.46995) -- (2.37806,4.48228);
\draw [c] (2.37806,4.48228) -- (2.37806,4.49413);
\draw [c] (2.36325,4.48228) -- (2.37806,4.48228);
\draw [c] (2.37806,4.48228) -- (2.39288,4.48228);
\definecolor{c}{rgb}{0,0,0};
\colorlet{c}{kugray};
\draw [c] (2.4077,4.4553) -- (2.4077,4.46803);
\draw [c] (2.4077,4.46803) -- (2.4077,4.48025);
\draw [c] (2.39288,4.46803) -- (2.4077,4.46803);
\draw [c] (2.4077,4.46803) -- (2.42252,4.46803);
\definecolor{c}{rgb}{0,0,0};
\colorlet{c}{kugray};
\draw [c] (2.43733,4.4243) -- (2.43733,4.43793);
\draw [c] (2.43733,4.43793) -- (2.43733,4.45097);
\draw [c] (2.42252,4.43793) -- (2.43733,4.43793);
\draw [c] (2.43733,4.43793) -- (2.45215,4.43793);
\definecolor{c}{rgb}{0,0,0};
\colorlet{c}{kugray};
\draw [c] (2.46697,4.40165) -- (2.46697,4.41557);
\draw [c] (2.46697,4.41557) -- (2.46697,4.42889);
\draw [c] (2.45215,4.41557) -- (2.46697,4.41557);
\draw [c] (2.46697,4.41557) -- (2.48179,4.41557);
\definecolor{c}{rgb}{0,0,0};
\colorlet{c}{kugray};
\draw [c] (2.49661,4.3514) -- (2.49661,4.36606);
\draw [c] (2.49661,4.36606) -- (2.49661,4.38005);
\draw [c] (2.48179,4.36606) -- (2.49661,4.36606);
\draw [c] (2.49661,4.36606) -- (2.51142,4.36606);
\definecolor{c}{rgb}{0,0,0};
\colorlet{c}{kugray};
\draw [c] (2.52624,4.33895) -- (2.52624,4.3543);
\draw [c] (2.52624,4.3543) -- (2.52624,4.36892);
\draw [c] (2.51142,4.3543) -- (2.52624,4.3543);
\draw [c] (2.52624,4.3543) -- (2.54106,4.3543);
\definecolor{c}{rgb}{0,0,0};
\colorlet{c}{kugray};
\draw [c] (2.55588,4.30184) -- (2.55588,4.31852);
\draw [c] (2.55588,4.31852) -- (2.55588,4.33435);
\draw [c] (2.54106,4.31852) -- (2.55588,4.31852);
\draw [c] (2.55588,4.31852) -- (2.5707,4.31852);
\definecolor{c}{rgb}{0,0,0};
\colorlet{c}{kugray};
\draw [c] (2.58551,4.29419) -- (2.58551,4.31073);
\draw [c] (2.58551,4.31073) -- (2.58551,4.32643);
\draw [c] (2.5707,4.31073) -- (2.58551,4.31073);
\draw [c] (2.58551,4.31073) -- (2.60033,4.31073);
\definecolor{c}{rgb}{0,0,0};
\colorlet{c}{kugray};
\draw [c] (2.61515,4.23639) -- (2.61515,4.25491);
\draw [c] (2.61515,4.25491) -- (2.61515,4.27238);
\draw [c] (2.60033,4.25491) -- (2.61515,4.25491);
\draw [c] (2.61515,4.25491) -- (2.62997,4.25491);
\definecolor{c}{rgb}{0,0,0};
\colorlet{c}{kugray};
\draw [c] (2.64478,4.2342) -- (2.64478,4.2521);
\draw [c] (2.64478,4.2521) -- (2.64478,4.26903);
\draw [c] (2.62997,4.2521) -- (2.64478,4.2521);
\draw [c] (2.64478,4.2521) -- (2.6596,4.2521);
\definecolor{c}{rgb}{0,0,0};
\colorlet{c}{kugray};
\draw [c] (2.67442,4.19709) -- (2.67442,4.21658);
\draw [c] (2.67442,4.21658) -- (2.67442,4.23491);
\draw [c] (2.6596,4.21658) -- (2.67442,4.21658);
\draw [c] (2.67442,4.21658) -- (2.68924,4.21658);
\definecolor{c}{rgb}{0,0,0};
\colorlet{c}{kugray};
\draw [c] (2.70406,4.15645) -- (2.70406,4.17742);
\draw [c] (2.70406,4.17742) -- (2.70406,4.19706);
\draw [c] (2.68924,4.17742) -- (2.70406,4.17742);
\draw [c] (2.70406,4.17742) -- (2.71887,4.17742);
\definecolor{c}{rgb}{0,0,0};
\colorlet{c}{kugray};
\draw [c] (2.73369,4.10839) -- (2.73369,4.13042);
\draw [c] (2.73369,4.13042) -- (2.73369,4.15097);
\draw [c] (2.71887,4.13042) -- (2.73369,4.13042);
\draw [c] (2.73369,4.13042) -- (2.74851,4.13042);
\definecolor{c}{rgb}{0,0,0};
\colorlet{c}{kugray};
\draw [c] (2.76333,4.13783) -- (2.76333,4.15915);
\draw [c] (2.76333,4.15915) -- (2.76333,4.1791);
\draw [c] (2.74851,4.15915) -- (2.76333,4.15915);
\draw [c] (2.76333,4.15915) -- (2.77815,4.15915);
\definecolor{c}{rgb}{0,0,0};
\colorlet{c}{kugray};
\draw [c] (2.79296,4.09048) -- (2.79296,4.11309);
\draw [c] (2.79296,4.11309) -- (2.79296,4.13416);
\draw [c] (2.77815,4.11309) -- (2.79296,4.11309);
\draw [c] (2.79296,4.11309) -- (2.80778,4.11309);
\definecolor{c}{rgb}{0,0,0};
\colorlet{c}{kugray};
\draw [c] (2.8226,4.09688) -- (2.8226,4.11923);
\draw [c] (2.8226,4.11923) -- (2.8226,4.14007);
\draw [c] (2.80778,4.11923) -- (2.8226,4.11923);
\draw [c] (2.8226,4.11923) -- (2.83742,4.11923);
\definecolor{c}{rgb}{0,0,0};
\colorlet{c}{kugray};
\draw [c] (2.85224,4.00498) -- (2.85224,4.03051);
\draw [c] (2.85224,4.03051) -- (2.85224,4.05409);
\draw [c] (2.83742,4.03051) -- (2.85224,4.03051);
\draw [c] (2.85224,4.03051) -- (2.86705,4.03051);
\definecolor{c}{rgb}{0,0,0};
\colorlet{c}{kugray};
\draw [c] (2.88187,4.03101) -- (2.88187,4.05663);
\draw [c] (2.88187,4.05663) -- (2.88187,4.08028);
\draw [c] (2.86705,4.05663) -- (2.88187,4.05663);
\draw [c] (2.88187,4.05663) -- (2.89669,4.05663);
\definecolor{c}{rgb}{0,0,0};
\colorlet{c}{kugray};
\draw [c] (2.91151,4.01624) -- (2.91151,4.04198);
\draw [c] (2.91151,4.04198) -- (2.91151,4.06572);
\draw [c] (2.89669,4.04198) -- (2.91151,4.04198);
\draw [c] (2.91151,4.04198) -- (2.92632,4.04198);
\definecolor{c}{rgb}{0,0,0};
\colorlet{c}{kugray};
\draw [c] (2.94114,4.00464) -- (2.94114,4.03106);
\draw [c] (2.94114,4.03106) -- (2.94114,4.0554);
\draw [c] (2.92632,4.03106) -- (2.94114,4.03106);
\draw [c] (2.94114,4.03106) -- (2.95596,4.03106);
\definecolor{c}{rgb}{0,0,0};
\colorlet{c}{kugray};
\draw [c] (2.97078,3.992) -- (2.97078,4.01847);
\draw [c] (2.97078,4.01847) -- (2.97078,4.04285);
\draw [c] (2.95596,4.01847) -- (2.97078,4.01847);
\draw [c] (2.97078,4.01847) -- (2.9856,4.01847);
\definecolor{c}{rgb}{0,0,0};
\colorlet{c}{kugray};
\draw [c] (3.00041,3.95995) -- (3.00041,3.9878);
\draw [c] (3.00041,3.9878) -- (3.00041,4.01334);
\draw [c] (2.9856,3.9878) -- (3.00041,3.9878);
\draw [c] (3.00041,3.9878) -- (3.01523,3.9878);
\definecolor{c}{rgb}{0,0,0};
\colorlet{c}{kugray};
\draw [c] (3.03005,3.92381) -- (3.03005,3.95391);
\draw [c] (3.03005,3.95391) -- (3.03005,3.98133);
\draw [c] (3.01523,3.95391) -- (3.03005,3.95391);
\draw [c] (3.03005,3.95391) -- (3.04487,3.95391);
\definecolor{c}{rgb}{0,0,0};
\colorlet{c}{kugray};
\draw [c] (3.05969,3.92723) -- (3.05969,3.9572);
\draw [c] (3.05969,3.9572) -- (3.05969,3.9845);
\draw [c] (3.04487,3.9572) -- (3.05969,3.9572);
\draw [c] (3.05969,3.9572) -- (3.0745,3.9572);
\definecolor{c}{rgb}{0,0,0};
\colorlet{c}{kugray};
\draw [c] (3.08932,3.89686) -- (3.08932,3.92737);
\draw [c] (3.08932,3.92737) -- (3.08932,3.95512);
\draw [c] (3.0745,3.92737) -- (3.08932,3.92737);
\draw [c] (3.08932,3.92737) -- (3.10414,3.92737);
\definecolor{c}{rgb}{0,0,0};
\colorlet{c}{kugray};
\draw [c] (3.11896,3.87075) -- (3.11896,3.9025);
\draw [c] (3.11896,3.9025) -- (3.11896,3.93128);
\draw [c] (3.10414,3.9025) -- (3.11896,3.9025);
\draw [c] (3.11896,3.9025) -- (3.13377,3.9025);
\definecolor{c}{rgb}{0,0,0};
\colorlet{c}{kugray};
\draw [c] (3.14859,3.84537) -- (3.14859,3.88076);
\draw [c] (3.14859,3.88076) -- (3.14859,3.9125);
\draw [c] (3.13377,3.88076) -- (3.14859,3.88076);
\draw [c] (3.14859,3.88076) -- (3.16341,3.88076);
\definecolor{c}{rgb}{0,0,0};
\colorlet{c}{kugray};
\draw [c] (3.17823,3.8829) -- (3.17823,3.91571);
\draw [c] (3.17823,3.91571) -- (3.17823,3.94536);
\draw [c] (3.16341,3.91571) -- (3.17823,3.91571);
\draw [c] (3.17823,3.91571) -- (3.19305,3.91571);
\definecolor{c}{rgb}{0,0,0};
\colorlet{c}{kugray};
\draw [c] (3.20786,3.84639) -- (3.20786,3.8804);
\draw [c] (3.20786,3.8804) -- (3.20786,3.91102);
\draw [c] (3.19305,3.8804) -- (3.20786,3.8804);
\draw [c] (3.20786,3.8804) -- (3.22268,3.8804);
\definecolor{c}{rgb}{0,0,0};
\colorlet{c}{kugray};
\draw [c] (3.2375,3.74035) -- (3.2375,3.78287);
\draw [c] (3.2375,3.78287) -- (3.2375,3.82022);
\draw [c] (3.22268,3.78287) -- (3.2375,3.78287);
\draw [c] (3.2375,3.78287) -- (3.25232,3.78287);
\definecolor{c}{rgb}{0,0,0};
\colorlet{c}{kugray};
\draw [c] (3.26714,3.68918) -- (3.26714,3.73432);
\draw [c] (3.26714,3.73432) -- (3.26714,3.77368);
\draw [c] (3.25232,3.73432) -- (3.26714,3.73432);
\draw [c] (3.26714,3.73432) -- (3.28195,3.73432);
\definecolor{c}{rgb}{0,0,0};
\colorlet{c}{kugray};
\draw [c] (3.29677,3.79443) -- (3.29677,3.83114);
\draw [c] (3.29677,3.83114) -- (3.29677,3.86393);
\draw [c] (3.28195,3.83114) -- (3.29677,3.83114);
\draw [c] (3.29677,3.83114) -- (3.31159,3.83114);
\definecolor{c}{rgb}{0,0,0};
\colorlet{c}{kugray};
\draw [c] (3.32641,3.68348) -- (3.32641,3.72745);
\draw [c] (3.32641,3.72745) -- (3.32641,3.76591);
\draw [c] (3.31159,3.72745) -- (3.32641,3.72745);
\draw [c] (3.32641,3.72745) -- (3.34123,3.72745);
\definecolor{c}{rgb}{0,0,0};
\colorlet{c}{kugray};
\draw [c] (3.35604,3.6893) -- (3.35604,3.73394);
\draw [c] (3.35604,3.73394) -- (3.35604,3.77293);
\draw [c] (3.34123,3.73394) -- (3.35604,3.73394);
\draw [c] (3.35604,3.73394) -- (3.37086,3.73394);
\definecolor{c}{rgb}{0,0,0};
\colorlet{c}{kugray};
\draw [c] (3.38568,3.69206) -- (3.38568,3.73661);
\draw [c] (3.38568,3.73661) -- (3.38568,3.77553);
\draw [c] (3.37086,3.73661) -- (3.38568,3.73661);
\draw [c] (3.38568,3.73661) -- (3.4005,3.73661);
\definecolor{c}{rgb}{0,0,0};
\colorlet{c}{kugray};
\draw [c] (3.41531,3.63598) -- (3.41531,3.68186);
\draw [c] (3.41531,3.68186) -- (3.41531,3.72179);
\draw [c] (3.4005,3.68186) -- (3.41531,3.68186);
\draw [c] (3.41531,3.68186) -- (3.43013,3.68186);
\definecolor{c}{rgb}{0,0,0};
\colorlet{c}{kugray};
\draw [c] (3.44495,3.6432) -- (3.44495,3.68882);
\draw [c] (3.44495,3.68882) -- (3.44495,3.72854);
\draw [c] (3.43013,3.68882) -- (3.44495,3.68882);
\draw [c] (3.44495,3.68882) -- (3.45977,3.68882);
\definecolor{c}{rgb}{0,0,0};
\colorlet{c}{kugray};
\draw [c] (3.47459,3.70732) -- (3.47459,3.7514);
\draw [c] (3.47459,3.7514) -- (3.47459,3.78995);
\draw [c] (3.45977,3.7514) -- (3.47459,3.7514);
\draw [c] (3.47459,3.7514) -- (3.4894,3.7514);
\definecolor{c}{rgb}{0,0,0};
\colorlet{c}{kugray};
\draw [c] (3.50422,3.66603) -- (3.50422,3.71303);
\draw [c] (3.50422,3.71303) -- (3.50422,3.7538);
\draw [c] (3.4894,3.71303) -- (3.50422,3.71303);
\draw [c] (3.50422,3.71303) -- (3.51904,3.71303);
\definecolor{c}{rgb}{0,0,0};
\colorlet{c}{kugray};
\draw [c] (3.53386,3.56775) -- (3.53386,3.61999);
\draw [c] (3.53386,3.61999) -- (3.53386,3.66463);
\draw [c] (3.51904,3.61999) -- (3.53386,3.61999);
\draw [c] (3.53386,3.61999) -- (3.54868,3.61999);
\definecolor{c}{rgb}{0,0,0};
\colorlet{c}{kugray};
\draw [c] (3.56349,3.5261) -- (3.56349,3.58634);
\draw [c] (3.56349,3.58634) -- (3.56349,3.6367);
\draw [c] (3.54868,3.58634) -- (3.56349,3.58634);
\draw [c] (3.56349,3.58634) -- (3.57831,3.58634);
\definecolor{c}{rgb}{0,0,0};
\colorlet{c}{kugray};
\draw [c] (3.59313,3.53612) -- (3.59313,3.59351);
\draw [c] (3.59313,3.59351) -- (3.59313,3.64187);
\draw [c] (3.57831,3.59351) -- (3.59313,3.59351);
\draw [c] (3.59313,3.59351) -- (3.60795,3.59351);
\definecolor{c}{rgb}{0,0,0};
\colorlet{c}{kugray};
\draw [c] (3.62276,3.60477) -- (3.62276,3.65738);
\draw [c] (3.62276,3.65738) -- (3.62276,3.70229);
\draw [c] (3.60795,3.65738) -- (3.62276,3.65738);
\draw [c] (3.62276,3.65738) -- (3.63758,3.65738);
\definecolor{c}{rgb}{0,0,0};
\colorlet{c}{kugray};
\draw [c] (3.6524,3.56534) -- (3.6524,3.6218);
\draw [c] (3.6524,3.6218) -- (3.6524,3.66949);
\draw [c] (3.63758,3.6218) -- (3.6524,3.6218);
\draw [c] (3.6524,3.6218) -- (3.66722,3.6218);
\definecolor{c}{rgb}{0,0,0};
\colorlet{c}{kugray};
\draw [c] (3.68204,3.48401) -- (3.68204,3.54474);
\draw [c] (3.68204,3.54474) -- (3.68204,3.59544);
\draw [c] (3.66722,3.54474) -- (3.68204,3.54474);
\draw [c] (3.68204,3.54474) -- (3.69685,3.54474);
\definecolor{c}{rgb}{0,0,0};
\colorlet{c}{kugray};
\draw [c] (3.71167,3.47019) -- (3.71167,3.52933);
\draw [c] (3.71167,3.52933) -- (3.71167,3.57892);
\draw [c] (3.69685,3.52933) -- (3.71167,3.52933);
\draw [c] (3.71167,3.52933) -- (3.72649,3.52933);
\definecolor{c}{rgb}{0,0,0};
\colorlet{c}{kugray};
\draw [c] (3.74131,3.47175) -- (3.74131,3.53495);
\draw [c] (3.74131,3.53495) -- (3.74131,3.58736);
\draw [c] (3.72649,3.53495) -- (3.74131,3.53495);
\draw [c] (3.74131,3.53495) -- (3.75613,3.53495);
\definecolor{c}{rgb}{0,0,0};
\colorlet{c}{kugray};
\draw [c] (3.77094,3.33813) -- (3.77094,3.41569);
\draw [c] (3.77094,3.41569) -- (3.77094,3.47758);
\draw [c] (3.75613,3.41569) -- (3.77094,3.41569);
\draw [c] (3.77094,3.41569) -- (3.78576,3.41569);
\definecolor{c}{rgb}{0,0,0};
\colorlet{c}{kugray};
\draw [c] (3.80058,3.43206) -- (3.80058,3.5008);
\draw [c] (3.80058,3.5008) -- (3.80058,3.55696);
\draw [c] (3.78576,3.5008) -- (3.80058,3.5008);
\draw [c] (3.80058,3.5008) -- (3.8154,3.5008);
\definecolor{c}{rgb}{0,0,0};
\colorlet{c}{kugray};
\draw [c] (3.83022,3.33768) -- (3.83022,3.41833);
\draw [c] (3.83022,3.41833) -- (3.83022,3.48217);
\draw [c] (3.8154,3.41833) -- (3.83022,3.41833);
\draw [c] (3.83022,3.41833) -- (3.84503,3.41833);
\definecolor{c}{rgb}{0,0,0};
\colorlet{c}{kugray};
\draw [c] (3.85985,3.50616) -- (3.85985,3.56185);
\draw [c] (3.85985,3.56185) -- (3.85985,3.60899);
\draw [c] (3.84503,3.56185) -- (3.85985,3.56185);
\draw [c] (3.85985,3.56185) -- (3.87467,3.56185);
\definecolor{c}{rgb}{0,0,0};
\colorlet{c}{kugray};
\draw [c] (3.88949,3.33314) -- (3.88949,3.40343);
\draw [c] (3.88949,3.40343) -- (3.88949,3.46061);
\draw [c] (3.87467,3.40343) -- (3.88949,3.40343);
\draw [c] (3.88949,3.40343) -- (3.9043,3.40343);
\definecolor{c}{rgb}{0,0,0};
\colorlet{c}{kugray};
\draw [c] (3.91912,3.32482) -- (3.91912,3.38647);
\draw [c] (3.91912,3.38647) -- (3.91912,3.43781);
\draw [c] (3.9043,3.38647) -- (3.91912,3.38647);
\draw [c] (3.91912,3.38647) -- (3.93394,3.38647);
\definecolor{c}{rgb}{0,0,0};
\colorlet{c}{kugray};
\draw [c] (3.94876,3.35526) -- (3.94876,3.41219);
\draw [c] (3.94876,3.41219) -- (3.94876,3.46022);
\draw [c] (3.93394,3.41219) -- (3.94876,3.41219);
\draw [c] (3.94876,3.41219) -- (3.96358,3.41219);
\definecolor{c}{rgb}{0,0,0};
\colorlet{c}{kugray};
\draw [c] (3.97839,3.3937) -- (3.97839,3.43987);
\draw [c] (3.97839,3.43987) -- (3.97839,3.48);
\draw [c] (3.96358,3.43987) -- (3.97839,3.43987);
\draw [c] (3.97839,3.43987) -- (3.99321,3.43987);
\definecolor{c}{rgb}{0,0,0};
\colorlet{c}{kugray};
\draw [c] (4.00803,3.39966) -- (4.00803,3.4339);
\draw [c] (4.00803,3.4339) -- (4.00803,3.4647);
\draw [c] (3.99321,3.4339) -- (4.00803,3.4339);
\draw [c] (4.00803,3.4339) -- (4.02285,3.4339);
\definecolor{c}{rgb}{0,0,0};
\colorlet{c}{kugray};
\draw [c] (4.03767,3.3675) -- (4.03767,3.38466);
\draw [c] (4.03767,3.38466) -- (4.03767,3.4009);
\draw [c] (4.02285,3.38466) -- (4.03767,3.38466);
\draw [c] (4.03767,3.38466) -- (4.05248,3.38466);
\definecolor{c}{rgb}{0,0,0};
\colorlet{c}{kugray};
\draw [c] (4.0673,3.36697) -- (4.0673,3.3818);
\draw [c] (4.0673,3.3818) -- (4.0673,3.39594);
\draw [c] (4.05248,3.3818) -- (4.0673,3.3818);
\draw [c] (4.0673,3.3818) -- (4.08212,3.3818);
\definecolor{c}{rgb}{0,0,0};
\colorlet{c}{kugray};
\draw [c] (4.09694,3.34139) -- (4.09694,3.35685);
\draw [c] (4.09694,3.35685) -- (4.09694,3.37158);
\draw [c] (4.08212,3.35685) -- (4.09694,3.35685);
\draw [c] (4.09694,3.35685) -- (4.11175,3.35685);
\definecolor{c}{rgb}{0,0,0};
\colorlet{c}{kugray};
\draw [c] (4.12657,3.34259) -- (4.12657,3.35775);
\draw [c] (4.12657,3.35775) -- (4.12657,3.37219);
\draw [c] (4.11175,3.35775) -- (4.12657,3.35775);
\draw [c] (4.12657,3.35775) -- (4.14139,3.35775);
\definecolor{c}{rgb}{0,0,0};
\colorlet{c}{kugray};
\draw [c] (4.15621,3.33872) -- (4.15621,3.35421);
\draw [c] (4.15621,3.35421) -- (4.15621,3.36896);
\draw [c] (4.14139,3.35421) -- (4.15621,3.35421);
\draw [c] (4.15621,3.35421) -- (4.17103,3.35421);
\definecolor{c}{rgb}{0,0,0};
\colorlet{c}{kugray};
\draw [c] (4.18584,3.34948) -- (4.18584,3.36456);
\draw [c] (4.18584,3.36456) -- (4.18584,3.37895);
\draw [c] (4.17103,3.36456) -- (4.18584,3.36456);
\draw [c] (4.18584,3.36456) -- (4.20066,3.36456);
\definecolor{c}{rgb}{0,0,0};
\colorlet{c}{kugray};
\draw [c] (4.21548,3.30398) -- (4.21548,3.32047);
\draw [c] (4.21548,3.32047) -- (4.21548,3.33611);
\draw [c] (4.20066,3.32047) -- (4.21548,3.32047);
\draw [c] (4.21548,3.32047) -- (4.2303,3.32047);
\definecolor{c}{rgb}{0,0,0};
\colorlet{c}{kugray};
\draw [c] (4.24512,3.29408) -- (4.24512,3.31043);
\draw [c] (4.24512,3.31043) -- (4.24512,3.32595);
\draw [c] (4.2303,3.31043) -- (4.24512,3.31043);
\draw [c] (4.24512,3.31043) -- (4.25993,3.31043);
\definecolor{c}{rgb}{0,0,0};
\colorlet{c}{kugray};
\draw [c] (4.27475,3.3132) -- (4.27475,3.32917);
\draw [c] (4.27475,3.32917) -- (4.27475,3.34436);
\draw [c] (4.25993,3.32917) -- (4.27475,3.32917);
\draw [c] (4.27475,3.32917) -- (4.28957,3.32917);
\definecolor{c}{rgb}{0,0,0};
\colorlet{c}{kugray};
\draw [c] (4.30439,3.28435) -- (4.30439,3.30161);
\draw [c] (4.30439,3.30161) -- (4.30439,3.31795);
\draw [c] (4.28957,3.30161) -- (4.30439,3.30161);
\draw [c] (4.30439,3.30161) -- (4.31921,3.30161);
\definecolor{c}{rgb}{0,0,0};
\colorlet{c}{kugray};
\draw [c] (4.33402,3.2635) -- (4.33402,3.28087);
\draw [c] (4.33402,3.28087) -- (4.33402,3.29731);
\draw [c] (4.31921,3.28087) -- (4.33402,3.28087);
\draw [c] (4.33402,3.28087) -- (4.34884,3.28087);
\definecolor{c}{rgb}{0,0,0};
\colorlet{c}{kugray};
\draw [c] (4.36366,3.26382) -- (4.36366,3.28126);
\draw [c] (4.36366,3.28126) -- (4.36366,3.29776);
\draw [c] (4.34884,3.28126) -- (4.36366,3.28126);
\draw [c] (4.36366,3.28126) -- (4.37848,3.28126);
\definecolor{c}{rgb}{0,0,0};
\colorlet{c}{kugray};
\draw [c] (4.39329,3.26487) -- (4.39329,3.28278);
\draw [c] (4.39329,3.28278) -- (4.39329,3.29971);
\draw [c] (4.37848,3.28278) -- (4.39329,3.28278);
\draw [c] (4.39329,3.28278) -- (4.40811,3.28278);
\definecolor{c}{rgb}{0,0,0};
\colorlet{c}{kugray};
\draw [c] (4.42293,3.21601) -- (4.42293,3.23471);
\draw [c] (4.42293,3.23471) -- (4.42293,3.25234);
\draw [c] (4.40811,3.23471) -- (4.42293,3.23471);
\draw [c] (4.42293,3.23471) -- (4.43775,3.23471);
\definecolor{c}{rgb}{0,0,0};
\colorlet{c}{kugray};
\draw [c] (4.45257,3.20125) -- (4.45257,3.2207);
\draw [c] (4.45257,3.2207) -- (4.45257,3.239);
\draw [c] (4.43775,3.2207) -- (4.45257,3.2207);
\draw [c] (4.45257,3.2207) -- (4.46738,3.2207);
\definecolor{c}{rgb}{0,0,0};
\colorlet{c}{kugray};
\draw [c] (4.4822,3.21568) -- (4.4822,3.23454);
\draw [c] (4.4822,3.23454) -- (4.4822,3.25231);
\draw [c] (4.46738,3.23454) -- (4.4822,3.23454);
\draw [c] (4.4822,3.23454) -- (4.49702,3.23454);
\definecolor{c}{rgb}{0,0,0};
\colorlet{c}{kugray};
\draw [c] (4.51184,3.21203) -- (4.51184,3.23093);
\draw [c] (4.51184,3.23093) -- (4.51184,3.24874);
\draw [c] (4.49702,3.23093) -- (4.51184,3.23093);
\draw [c] (4.51184,3.23093) -- (4.52666,3.23093);
\definecolor{c}{rgb}{0,0,0};
\colorlet{c}{kugray};
\draw [c] (4.54147,3.1933) -- (4.54147,3.21207);
\draw [c] (4.54147,3.21207) -- (4.54147,3.22977);
\draw [c] (4.52666,3.21207) -- (4.54147,3.21207);
\draw [c] (4.54147,3.21207) -- (4.55629,3.21207);
\definecolor{c}{rgb}{0,0,0};
\colorlet{c}{kugray};
\draw [c] (4.57111,3.16612) -- (4.57111,3.18637);
\draw [c] (4.57111,3.18637) -- (4.57111,3.20537);
\draw [c] (4.55629,3.18637) -- (4.57111,3.18637);
\draw [c] (4.57111,3.18637) -- (4.58593,3.18637);
\definecolor{c}{rgb}{0,0,0};
\colorlet{c}{kugray};
\draw [c] (4.60075,3.144) -- (4.60075,3.1653);
\draw [c] (4.60075,3.1653) -- (4.60075,3.18522);
\draw [c] (4.58593,3.1653) -- (4.60075,3.1653);
\draw [c] (4.60075,3.1653) -- (4.61556,3.1653);
\definecolor{c}{rgb}{0,0,0};
\colorlet{c}{kugray};
\draw [c] (4.63038,3.13842) -- (4.63038,3.15957);
\draw [c] (4.63038,3.15957) -- (4.63038,3.17935);
\draw [c] (4.61556,3.15957) -- (4.63038,3.15957);
\draw [c] (4.63038,3.15957) -- (4.6452,3.15957);
\definecolor{c}{rgb}{0,0,0};
\colorlet{c}{kugray};
\draw [c] (4.66002,3.10825) -- (4.66002,3.12968);
\draw [c] (4.66002,3.12968) -- (4.66002,3.14972);
\draw [c] (4.6452,3.12968) -- (4.66002,3.12968);
\draw [c] (4.66002,3.12968) -- (4.67483,3.12968);
\definecolor{c}{rgb}{0,0,0};
\colorlet{c}{kugray};
\draw [c] (4.68965,3.12507) -- (4.68965,3.14699);
\draw [c] (4.68965,3.14699) -- (4.68965,3.16746);
\draw [c] (4.67483,3.14699) -- (4.68965,3.14699);
\draw [c] (4.68965,3.14699) -- (4.70447,3.14699);
\definecolor{c}{rgb}{0,0,0};
\colorlet{c}{kugray};
\draw [c] (4.71929,3.07502) -- (4.71929,3.09766);
\draw [c] (4.71929,3.09766) -- (4.71929,3.11875);
\draw [c] (4.70447,3.09766) -- (4.71929,3.09766);
\draw [c] (4.71929,3.09766) -- (4.73411,3.09766);
\definecolor{c}{rgb}{0,0,0};
\colorlet{c}{kugray};
\draw [c] (4.74892,3.12089) -- (4.74892,3.1424);
\draw [c] (4.74892,3.1424) -- (4.74892,3.1625);
\draw [c] (4.73411,3.1424) -- (4.74892,3.1424);
\draw [c] (4.74892,3.1424) -- (4.76374,3.1424);
\definecolor{c}{rgb}{0,0,0};
\colorlet{c}{kugray};
\draw [c] (4.77856,3.07673) -- (4.77856,3.09957);
\draw [c] (4.77856,3.09957) -- (4.77856,3.12084);
\draw [c] (4.76374,3.09957) -- (4.77856,3.09957);
\draw [c] (4.77856,3.09957) -- (4.79338,3.09957);
\definecolor{c}{rgb}{0,0,0};
\colorlet{c}{kugray};
\draw [c] (4.8082,3.11216) -- (4.8082,3.13437);
\draw [c] (4.8082,3.13437) -- (4.8082,3.15507);
\draw [c] (4.79338,3.13437) -- (4.8082,3.13437);
\draw [c] (4.8082,3.13437) -- (4.82301,3.13437);
\definecolor{c}{rgb}{0,0,0};
\colorlet{c}{kugray};
\draw [c] (4.83783,3.12021) -- (4.83783,3.14174);
\draw [c] (4.83783,3.14174) -- (4.83783,3.16186);
\draw [c] (4.82301,3.14174) -- (4.83783,3.14174);
\draw [c] (4.83783,3.14174) -- (4.85265,3.14174);
\definecolor{c}{rgb}{0,0,0};
\colorlet{c}{kugray};
\draw [c] (4.86747,3.04958) -- (4.86747,3.07475);
\draw [c] (4.86747,3.07475) -- (4.86747,3.09801);
\draw [c] (4.85265,3.07475) -- (4.86747,3.07475);
\draw [c] (4.86747,3.07475) -- (4.88228,3.07475);
\definecolor{c}{rgb}{0,0,0};
\colorlet{c}{kugray};
\draw [c] (4.8971,3.02902) -- (4.8971,3.05448);
\draw [c] (4.8971,3.05448) -- (4.8971,3.078);
\draw [c] (4.88228,3.05448) -- (4.8971,3.05448);
\draw [c] (4.8971,3.05448) -- (4.91192,3.05448);
\definecolor{c}{rgb}{0,0,0};
\colorlet{c}{kugray};
\draw [c] (4.92674,3.05931) -- (4.92674,3.08346);
\draw [c] (4.92674,3.08346) -- (4.92674,3.10584);
\draw [c] (4.91192,3.08346) -- (4.92674,3.08346);
\draw [c] (4.92674,3.08346) -- (4.94156,3.08346);
\definecolor{c}{rgb}{0,0,0};
\colorlet{c}{kugray};
\draw [c] (4.95637,3.04459) -- (4.95637,3.06898);
\draw [c] (4.95637,3.06898) -- (4.95637,3.09157);
\draw [c] (4.94156,3.06898) -- (4.95637,3.06898);
\draw [c] (4.95637,3.06898) -- (4.97119,3.06898);
\definecolor{c}{rgb}{0,0,0};
\colorlet{c}{kugray};
\draw [c] (4.98601,3.02119) -- (4.98601,3.04833);
\draw [c] (4.98601,3.04833) -- (4.98601,3.07328);
\draw [c] (4.97119,3.04833) -- (4.98601,3.04833);
\draw [c] (4.98601,3.04833) -- (5.00083,3.04833);
\definecolor{c}{rgb}{0,0,0};
\colorlet{c}{kugray};
\draw [c] (5.01565,3.03957) -- (5.01565,3.06478);
\draw [c] (5.01565,3.06478) -- (5.01565,3.08808);
\draw [c] (5.00083,3.06478) -- (5.01565,3.06478);
\draw [c] (5.01565,3.06478) -- (5.03046,3.06478);
\definecolor{c}{rgb}{0,0,0};
\colorlet{c}{kugray};
\draw [c] (5.04528,2.9954) -- (5.04528,3.02095);
\draw [c] (5.04528,3.02095) -- (5.04528,3.04453);
\draw [c] (5.03046,3.02095) -- (5.04528,3.02095);
\draw [c] (5.04528,3.02095) -- (5.0601,3.02095);
\definecolor{c}{rgb}{0,0,0};
\colorlet{c}{kugray};
\draw [c] (5.07492,3.00475) -- (5.07492,3.03087);
\draw [c] (5.07492,3.03087) -- (5.07492,3.05495);
\draw [c] (5.0601,3.03087) -- (5.07492,3.03087);
\draw [c] (5.07492,3.03087) -- (5.08974,3.03087);
\definecolor{c}{rgb}{0,0,0};
\colorlet{c}{kugray};
\draw [c] (5.10455,3.04579) -- (5.10455,3.07035);
\draw [c] (5.10455,3.07035) -- (5.10455,3.0931);
\draw [c] (5.08974,3.07035) -- (5.10455,3.07035);
\draw [c] (5.10455,3.07035) -- (5.11937,3.07035);
\definecolor{c}{rgb}{0,0,0};
\colorlet{c}{kugray};
\draw [c] (5.13419,2.99892) -- (5.13419,3.02565);
\draw [c] (5.13419,3.02565) -- (5.13419,3.05025);
\draw [c] (5.11937,3.02565) -- (5.13419,3.02565);
\draw [c] (5.13419,3.02565) -- (5.14901,3.02565);
\definecolor{c}{rgb}{0,0,0};
\colorlet{c}{kugray};
\draw [c] (5.16382,2.99685) -- (5.16382,3.02306);
\draw [c] (5.16382,3.02306) -- (5.16382,3.0472);
\draw [c] (5.14901,3.02306) -- (5.16382,3.02306);
\draw [c] (5.16382,3.02306) -- (5.17864,3.02306);
\definecolor{c}{rgb}{0,0,0};
\colorlet{c}{kugray};
\draw [c] (5.19346,2.98544) -- (5.19346,3.0123);
\draw [c] (5.19346,3.0123) -- (5.19346,3.03701);
\draw [c] (5.17864,3.0123) -- (5.19346,3.0123);
\draw [c] (5.19346,3.0123) -- (5.20828,3.0123);
\definecolor{c}{rgb}{0,0,0};
\colorlet{c}{kugray};
\draw [c] (5.2231,3.00482) -- (5.2231,3.0317);
\draw [c] (5.2231,3.0317) -- (5.2231,3.05642);
\draw [c] (5.20828,3.0317) -- (5.2231,3.0317);
\draw [c] (5.2231,3.0317) -- (5.23791,3.0317);
\definecolor{c}{rgb}{0,0,0};
\colorlet{c}{kugray};
\draw [c] (5.25273,2.9865) -- (5.25273,3.01355);
\draw [c] (5.25273,3.01355) -- (5.25273,3.03841);
\draw [c] (5.23791,3.01355) -- (5.25273,3.01355);
\draw [c] (5.25273,3.01355) -- (5.26755,3.01355);
\definecolor{c}{rgb}{0,0,0};
\colorlet{c}{kugray};
\draw [c] (5.28237,2.99411) -- (5.28237,3.02124);
\draw [c] (5.28237,3.02124) -- (5.28237,3.04618);
\draw [c] (5.26755,3.02124) -- (5.28237,3.02124);
\draw [c] (5.28237,3.02124) -- (5.29719,3.02124);
\definecolor{c}{rgb}{0,0,0};
\colorlet{c}{kugray};
\draw [c] (5.312,2.94471) -- (5.312,2.97506);
\draw [c] (5.312,2.97506) -- (5.312,3.00268);
\draw [c] (5.29719,2.97506) -- (5.312,2.97506);
\draw [c] (5.312,2.97506) -- (5.32682,2.97506);
\definecolor{c}{rgb}{0,0,0};
\colorlet{c}{kugray};
\draw [c] (5.34164,2.92698) -- (5.34164,2.95685);
\draw [c] (5.34164,2.95685) -- (5.34164,2.98407);
\draw [c] (5.32682,2.95685) -- (5.34164,2.95685);
\draw [c] (5.34164,2.95685) -- (5.35646,2.95685);
\definecolor{c}{rgb}{0,0,0};
\colorlet{c}{kugray};
\draw [c] (5.37127,2.88114) -- (5.37127,2.91303);
\draw [c] (5.37127,2.91303) -- (5.37127,2.94192);
\draw [c] (5.35646,2.91303) -- (5.37127,2.91303);
\draw [c] (5.37127,2.91303) -- (5.38609,2.91303);
\definecolor{c}{rgb}{0,0,0};
\colorlet{c}{kugray};
\draw [c] (5.40091,2.91147) -- (5.40091,2.942);
\draw [c] (5.40091,2.942) -- (5.40091,2.96977);
\draw [c] (5.38609,2.942) -- (5.40091,2.942);
\draw [c] (5.40091,2.942) -- (5.41573,2.942);
\definecolor{c}{rgb}{0,0,0};
\colorlet{c}{kugray};
\draw [c] (5.43055,2.89934) -- (5.43055,2.93053);
\draw [c] (5.43055,2.93053) -- (5.43055,2.95885);
\draw [c] (5.41573,2.93053) -- (5.43055,2.93053);
\draw [c] (5.43055,2.93053) -- (5.44536,2.93053);
\definecolor{c}{rgb}{0,0,0};
\colorlet{c}{kugray};
\draw [c] (5.46018,2.83663) -- (5.46018,2.86962);
\draw [c] (5.46018,2.86962) -- (5.46018,2.89942);
\draw [c] (5.44536,2.86962) -- (5.46018,2.86962);
\draw [c] (5.46018,2.86962) -- (5.475,2.86962);
\definecolor{c}{rgb}{0,0,0};
\colorlet{c}{kugray};
\draw [c] (5.48982,2.89582) -- (5.48982,2.92714);
\draw [c] (5.48982,2.92714) -- (5.48982,2.95557);
\draw [c] (5.475,2.92714) -- (5.48982,2.92714);
\draw [c] (5.48982,2.92714) -- (5.50464,2.92714);
\definecolor{c}{rgb}{0,0,0};
\colorlet{c}{kugray};
\draw [c] (5.51945,2.86857) -- (5.51945,2.90037);
\draw [c] (5.51945,2.90037) -- (5.51945,2.9292);
\draw [c] (5.50464,2.90037) -- (5.51945,2.90037);
\draw [c] (5.51945,2.90037) -- (5.53427,2.90037);
\definecolor{c}{rgb}{0,0,0};
\colorlet{c}{kugray};
\draw [c] (5.54909,2.91849) -- (5.54909,2.94886);
\draw [c] (5.54909,2.94886) -- (5.54909,2.97651);
\draw [c] (5.53427,2.94886) -- (5.54909,2.94886);
\draw [c] (5.54909,2.94886) -- (5.56391,2.94886);
\definecolor{c}{rgb}{0,0,0};
\colorlet{c}{kugray};
\draw [c] (5.57873,2.84325) -- (5.57873,2.87551);
\draw [c] (5.57873,2.87551) -- (5.57873,2.90471);
\draw [c] (5.56391,2.87551) -- (5.57873,2.87551);
\draw [c] (5.57873,2.87551) -- (5.59354,2.87551);
\definecolor{c}{rgb}{0,0,0};
\colorlet{c}{kugray};
\draw [c] (5.60836,2.86623) -- (5.60836,2.89815);
\draw [c] (5.60836,2.89815) -- (5.60836,2.92707);
\draw [c] (5.59354,2.89815) -- (5.60836,2.89815);
\draw [c] (5.60836,2.89815) -- (5.62318,2.89815);
\definecolor{c}{rgb}{0,0,0};
\colorlet{c}{kugray};
\draw [c] (5.638,2.91304) -- (5.638,2.94486);
\draw [c] (5.638,2.94486) -- (5.638,2.97369);
\draw [c] (5.62318,2.94486) -- (5.638,2.94486);
\draw [c] (5.638,2.94486) -- (5.65281,2.94486);
\definecolor{c}{rgb}{0,0,0};
\colorlet{c}{kugray};
\draw [c] (5.66763,2.90882) -- (5.66763,2.93949);
\draw [c] (5.66763,2.93949) -- (5.66763,2.96738);
\draw [c] (5.65281,2.93949) -- (5.66763,2.93949);
\draw [c] (5.66763,2.93949) -- (5.68245,2.93949);
\definecolor{c}{rgb}{0,0,0};
\colorlet{c}{kugray};
\draw [c] (5.69727,2.89105) -- (5.69727,2.92252);
\draw [c] (5.69727,2.92252) -- (5.69727,2.95107);
\draw [c] (5.68245,2.92252) -- (5.69727,2.92252);
\draw [c] (5.69727,2.92252) -- (5.71209,2.92252);
\definecolor{c}{rgb}{0,0,0};
\colorlet{c}{kugray};
\draw [c] (5.7269,2.77815) -- (5.7269,2.81636);
\draw [c] (5.7269,2.81636) -- (5.7269,2.85035);
\draw [c] (5.71209,2.81636) -- (5.7269,2.81636);
\draw [c] (5.7269,2.81636) -- (5.74172,2.81636);
\definecolor{c}{rgb}{0,0,0};
\colorlet{c}{kugray};
\draw [c] (5.75654,2.94321) -- (5.75654,2.97327);
\draw [c] (5.75654,2.97327) -- (5.75654,3.00065);
\draw [c] (5.74172,2.97327) -- (5.75654,2.97327);
\draw [c] (5.75654,2.97327) -- (5.77136,2.97327);
\definecolor{c}{rgb}{0,0,0};
\colorlet{c}{kugray};
\draw [c] (5.78618,2.8141) -- (5.78618,2.84956);
\draw [c] (5.78618,2.84956) -- (5.78618,2.88135);
\draw [c] (5.77136,2.84956) -- (5.78618,2.84956);
\draw [c] (5.78618,2.84956) -- (5.80099,2.84956);
\definecolor{c}{rgb}{0,0,0};
\colorlet{c}{kugray};
\draw [c] (5.81581,2.84857) -- (5.81581,2.88346);
\draw [c] (5.81581,2.88346) -- (5.81581,2.91479);
\draw [c] (5.80099,2.88346) -- (5.81581,2.88346);
\draw [c] (5.81581,2.88346) -- (5.83063,2.88346);
\definecolor{c}{rgb}{0,0,0};
\colorlet{c}{kugray};
\draw [c] (5.84545,2.83643) -- (5.84545,2.87155);
\draw [c] (5.84545,2.87155) -- (5.84545,2.90308);
\draw [c] (5.83063,2.87155) -- (5.84545,2.87155);
\draw [c] (5.84545,2.87155) -- (5.86026,2.87155);
\definecolor{c}{rgb}{0,0,0};
\colorlet{c}{kugray};
\draw [c] (5.87508,2.75529) -- (5.87508,2.79332);
\draw [c] (5.87508,2.79332) -- (5.87508,2.82715);
\draw [c] (5.86026,2.79332) -- (5.87508,2.79332);
\draw [c] (5.87508,2.79332) -- (5.8899,2.79332);
\definecolor{c}{rgb}{0,0,0};
\colorlet{c}{kugray};
\draw [c] (5.90472,2.77226) -- (5.90472,2.81175);
\draw [c] (5.90472,2.81175) -- (5.90472,2.84675);
\draw [c] (5.8899,2.81175) -- (5.90472,2.81175);
\draw [c] (5.90472,2.81175) -- (5.91954,2.81175);
\definecolor{c}{rgb}{0,0,0};
\colorlet{c}{kugray};
\draw [c] (5.93435,2.82058) -- (5.93435,2.85588);
\draw [c] (5.93435,2.85588) -- (5.93435,2.88754);
\draw [c] (5.91954,2.85588) -- (5.93435,2.85588);
\draw [c] (5.93435,2.85588) -- (5.94917,2.85588);
\definecolor{c}{rgb}{0,0,0};
\colorlet{c}{kugray};
\draw [c] (5.96399,2.83477) -- (5.96399,2.87069);
\draw [c] (5.96399,2.87069) -- (5.96399,2.90285);
\draw [c] (5.94917,2.87069) -- (5.96399,2.87069);
\draw [c] (5.96399,2.87069) -- (5.97881,2.87069);
\definecolor{c}{rgb}{0,0,0};
\colorlet{c}{kugray};
\draw [c] (5.99363,2.78914) -- (5.99363,2.8266);
\draw [c] (5.99363,2.8266) -- (5.99363,2.85999);
\draw [c] (5.97881,2.8266) -- (5.99363,2.8266);
\draw [c] (5.99363,2.8266) -- (6.00844,2.8266);
\definecolor{c}{rgb}{0,0,0};
\colorlet{c}{kugray};
\draw [c] (6.02326,2.82633) -- (6.02326,2.86318);
\draw [c] (6.02326,2.86318) -- (6.02326,2.89609);
\draw [c] (6.00844,2.86318) -- (6.02326,2.86318);
\draw [c] (6.02326,2.86318) -- (6.03808,2.86318);
\definecolor{c}{rgb}{0,0,0};
\colorlet{c}{kugray};
\draw [c] (6.0529,2.72961) -- (6.0529,2.7694);
\draw [c] (6.0529,2.7694) -- (6.0529,2.80462);
\draw [c] (6.03808,2.7694) -- (6.0529,2.7694);
\draw [c] (6.0529,2.7694) -- (6.06772,2.7694);
\definecolor{c}{rgb}{0,0,0};
\colorlet{c}{kugray};
\draw [c] (6.08253,2.84391) -- (6.08253,2.87833);
\draw [c] (6.08253,2.87833) -- (6.08253,2.90928);
\draw [c] (6.06772,2.87833) -- (6.08253,2.87833);
\draw [c] (6.08253,2.87833) -- (6.09735,2.87833);
\definecolor{c}{rgb}{0,0,0};
\colorlet{c}{kugray};
\draw [c] (6.11217,2.75513) -- (6.11217,2.79539);
\draw [c] (6.11217,2.79539) -- (6.11217,2.83099);
\draw [c] (6.09735,2.79539) -- (6.11217,2.79539);
\draw [c] (6.11217,2.79539) -- (6.12699,2.79539);
\definecolor{c}{rgb}{0,0,0};
\colorlet{c}{kugray};
\draw [c] (6.1418,2.67424) -- (6.1418,2.71934);
\draw [c] (6.1418,2.71934) -- (6.1418,2.75867);
\draw [c] (6.12699,2.71934) -- (6.1418,2.71934);
\draw [c] (6.1418,2.71934) -- (6.15662,2.71934);
\definecolor{c}{rgb}{0,0,0};
\colorlet{c}{kugray};
\draw [c] (6.17144,2.76048) -- (6.17144,2.79878);
\draw [c] (6.17144,2.79878) -- (6.17144,2.83284);
\draw [c] (6.15662,2.79878) -- (6.17144,2.79878);
\draw [c] (6.17144,2.79878) -- (6.18626,2.79878);
\definecolor{c}{rgb}{0,0,0};
\colorlet{c}{kugray};
\draw [c] (6.20108,2.73144) -- (6.20108,2.77033);
\draw [c] (6.20108,2.77033) -- (6.20108,2.80485);
\draw [c] (6.18626,2.77033) -- (6.20108,2.77033);
\draw [c] (6.20108,2.77033) -- (6.21589,2.77033);
\definecolor{c}{rgb}{0,0,0};
\colorlet{c}{kugray};
\draw [c] (6.23071,2.71313) -- (6.23071,2.75551);
\draw [c] (6.23071,2.75551) -- (6.23071,2.79276);
\draw [c] (6.21589,2.75551) -- (6.23071,2.75551);
\draw [c] (6.23071,2.75551) -- (6.24553,2.75551);
\definecolor{c}{rgb}{0,0,0};
\colorlet{c}{kugray};
\draw [c] (6.26035,2.73793) -- (6.26035,2.77947);
\draw [c] (6.26035,2.77947) -- (6.26035,2.81607);
\draw [c] (6.24553,2.77947) -- (6.26035,2.77947);
\draw [c] (6.26035,2.77947) -- (6.27517,2.77947);
\definecolor{c}{rgb}{0,0,0};
\colorlet{c}{kugray};
\draw [c] (6.28998,2.74513) -- (6.28998,2.7844);
\draw [c] (6.28998,2.7844) -- (6.28998,2.81923);
\draw [c] (6.27517,2.7844) -- (6.28998,2.7844);
\draw [c] (6.28998,2.7844) -- (6.3048,2.7844);
\definecolor{c}{rgb}{0,0,0};
\colorlet{c}{kugray};
\draw [c] (6.31962,2.7653) -- (6.31962,2.80355);
\draw [c] (6.31962,2.80355) -- (6.31962,2.83757);
\draw [c] (6.3048,2.80355) -- (6.31962,2.80355);
\draw [c] (6.31962,2.80355) -- (6.33444,2.80355);
\definecolor{c}{rgb}{0,0,0};
\colorlet{c}{kugray};
\draw [c] (6.34926,2.73889) -- (6.34926,2.78116);
\draw [c] (6.34926,2.78116) -- (6.34926,2.81833);
\draw [c] (6.33444,2.78116) -- (6.34926,2.78116);
\draw [c] (6.34926,2.78116) -- (6.36407,2.78116);
\definecolor{c}{rgb}{0,0,0};
\colorlet{c}{kugray};
\draw [c] (6.37889,2.70662) -- (6.37889,2.74897);
\draw [c] (6.37889,2.74897) -- (6.37889,2.7862);
\draw [c] (6.36407,2.74897) -- (6.37889,2.74897);
\draw [c] (6.37889,2.74897) -- (6.39371,2.74897);
\definecolor{c}{rgb}{0,0,0};
\colorlet{c}{kugray};
\draw [c] (6.40853,2.70883) -- (6.40853,2.74986);
\draw [c] (6.40853,2.74986) -- (6.40853,2.78606);
\draw [c] (6.39371,2.74986) -- (6.40853,2.74986);
\draw [c] (6.40853,2.74986) -- (6.42334,2.74986);
\definecolor{c}{rgb}{0,0,0};
\colorlet{c}{kugray};
\draw [c] (6.43816,2.74777) -- (6.43816,2.79061);
\draw [c] (6.43816,2.79061) -- (6.43816,2.82822);
\draw [c] (6.42334,2.79061) -- (6.43816,2.79061);
\draw [c] (6.43816,2.79061) -- (6.45298,2.79061);
\definecolor{c}{rgb}{0,0,0};
\colorlet{c}{kugray};
\draw [c] (6.4678,2.73848) -- (6.4678,2.78112);
\draw [c] (6.4678,2.78112) -- (6.4678,2.81857);
\draw [c] (6.45298,2.78112) -- (6.4678,2.78112);
\draw [c] (6.4678,2.78112) -- (6.48262,2.78112);
\definecolor{c}{rgb}{0,0,0};
\colorlet{c}{kugray};
\draw [c] (6.49743,2.68882) -- (6.49743,2.73113);
\draw [c] (6.49743,2.73113) -- (6.49743,2.76832);
\draw [c] (6.48262,2.73113) -- (6.49743,2.73113);
\draw [c] (6.49743,2.73113) -- (6.51225,2.73113);
\definecolor{c}{rgb}{0,0,0};
\colorlet{c}{kugray};
\draw [c] (6.52707,2.69846) -- (6.52707,2.74161);
\draw [c] (6.52707,2.74161) -- (6.52707,2.77946);
\draw [c] (6.51225,2.74161) -- (6.52707,2.74161);
\draw [c] (6.52707,2.74161) -- (6.54189,2.74161);
\definecolor{c}{rgb}{0,0,0};
\colorlet{c}{kugray};
\draw [c] (6.55671,2.70662) -- (6.55671,2.74955);
\draw [c] (6.55671,2.74955) -- (6.55671,2.78723);
\draw [c] (6.54189,2.74955) -- (6.55671,2.74955);
\draw [c] (6.55671,2.74955) -- (6.57152,2.74955);
\definecolor{c}{rgb}{0,0,0};
\colorlet{c}{kugray};
\draw [c] (6.58634,2.70574) -- (6.58634,2.75106);
\draw [c] (6.58634,2.75106) -- (6.58634,2.79057);
\draw [c] (6.57152,2.75106) -- (6.58634,2.75106);
\draw [c] (6.58634,2.75106) -- (6.60116,2.75106);
\definecolor{c}{rgb}{0,0,0};
\colorlet{c}{kugray};
\draw [c] (6.61598,2.68906) -- (6.61598,2.73402);
\draw [c] (6.61598,2.73402) -- (6.61598,2.77325);
\draw [c] (6.60116,2.73402) -- (6.61598,2.73402);
\draw [c] (6.61598,2.73402) -- (6.63079,2.73402);
\definecolor{c}{rgb}{0,0,0};
\colorlet{c}{kugray};
\draw [c] (6.64561,2.74536) -- (6.64561,2.78499);
\draw [c] (6.64561,2.78499) -- (6.64561,2.82009);
\draw [c] (6.63079,2.78499) -- (6.64561,2.78499);
\draw [c] (6.64561,2.78499) -- (6.66043,2.78499);
\definecolor{c}{rgb}{0,0,0};
\colorlet{c}{kugray};
\draw [c] (6.67525,2.72189) -- (6.67525,2.76667);
\draw [c] (6.67525,2.76667) -- (6.67525,2.80576);
\draw [c] (6.66043,2.76667) -- (6.67525,2.76667);
\draw [c] (6.67525,2.76667) -- (6.69007,2.76667);
\definecolor{c}{rgb}{0,0,0};
\colorlet{c}{kugray};
\draw [c] (6.70488,2.68289) -- (6.70488,2.72648);
\draw [c] (6.70488,2.72648) -- (6.70488,2.76466);
\draw [c] (6.69007,2.72648) -- (6.70488,2.72648);
\draw [c] (6.70488,2.72648) -- (6.7197,2.72648);
\definecolor{c}{rgb}{0,0,0};
\colorlet{c}{kugray};
\draw [c] (6.73452,2.59638) -- (6.73452,2.64724);
\draw [c] (6.73452,2.64724) -- (6.73452,2.69088);
\draw [c] (6.7197,2.64724) -- (6.73452,2.64724);
\draw [c] (6.73452,2.64724) -- (6.74934,2.64724);
\definecolor{c}{rgb}{0,0,0};
\colorlet{c}{kugray};
\draw [c] (6.76416,2.67667) -- (6.76416,2.72138);
\draw [c] (6.76416,2.72138) -- (6.76416,2.76041);
\draw [c] (6.74934,2.72138) -- (6.76416,2.72138);
\draw [c] (6.76416,2.72138) -- (6.77897,2.72138);
\definecolor{c}{rgb}{0,0,0};
\colorlet{c}{kugray};
\draw [c] (6.79379,2.66082) -- (6.79379,2.70714);
\draw [c] (6.79379,2.70714) -- (6.79379,2.74739);
\draw [c] (6.77897,2.70714) -- (6.79379,2.70714);
\draw [c] (6.79379,2.70714) -- (6.80861,2.70714);
\definecolor{c}{rgb}{0,0,0};
\colorlet{c}{kugray};
\draw [c] (6.82343,2.63019) -- (6.82343,2.67693);
\draw [c] (6.82343,2.67693) -- (6.82343,2.7175);
\draw [c] (6.80861,2.67693) -- (6.82343,2.67693);
\draw [c] (6.82343,2.67693) -- (6.83824,2.67693);
\definecolor{c}{rgb}{0,0,0};
\colorlet{c}{kugray};
\draw [c] (6.85306,2.55239) -- (6.85306,2.60372);
\draw [c] (6.85306,2.60372) -- (6.85306,2.64771);
\draw [c] (6.83824,2.60372) -- (6.85306,2.60372);
\draw [c] (6.85306,2.60372) -- (6.86788,2.60372);
\definecolor{c}{rgb}{0,0,0};
\colorlet{c}{kugray};
\draw [c] (6.8827,2.62294) -- (6.8827,2.67113);
\draw [c] (6.8827,2.67113) -- (6.8827,2.71279);
\draw [c] (6.86788,2.67113) -- (6.8827,2.67113);
\draw [c] (6.8827,2.67113) -- (6.89752,2.67113);
\definecolor{c}{rgb}{0,0,0};
\colorlet{c}{kugray};
\draw [c] (6.91233,2.6117) -- (6.91233,2.65948);
\draw [c] (6.91233,2.65948) -- (6.91233,2.70082);
\draw [c] (6.89752,2.65948) -- (6.91233,2.65948);
\draw [c] (6.91233,2.65948) -- (6.92715,2.65948);
\definecolor{c}{rgb}{0,0,0};
\colorlet{c}{kugray};
\draw [c] (6.94197,2.73243) -- (6.94197,2.77935);
\draw [c] (6.94197,2.77935) -- (6.94197,2.82005);
\draw [c] (6.92715,2.77935) -- (6.94197,2.77935);
\draw [c] (6.94197,2.77935) -- (6.95679,2.77935);
\definecolor{c}{rgb}{0,0,0};
\colorlet{c}{kugray};
\draw [c] (6.97161,2.48019) -- (6.97161,2.53947);
\draw [c] (6.97161,2.53947) -- (6.97161,2.58916);
\draw [c] (6.95679,2.53947) -- (6.97161,2.53947);
\draw [c] (6.97161,2.53947) -- (6.98642,2.53947);
\definecolor{c}{rgb}{0,0,0};
\colorlet{c}{kugray};
\draw [c] (7.00124,2.61681) -- (7.00124,2.66764);
\draw [c] (7.00124,2.66764) -- (7.00124,2.71126);
\draw [c] (6.98642,2.66764) -- (7.00124,2.66764);
\draw [c] (7.00124,2.66764) -- (7.01606,2.66764);
\definecolor{c}{rgb}{0,0,0};
\colorlet{c}{kugray};
\draw [c] (7.03088,2.47861) -- (7.03088,2.53761);
\draw [c] (7.03088,2.53761) -- (7.03088,2.5871);
\draw [c] (7.01606,2.53761) -- (7.03088,2.53761);
\draw [c] (7.03088,2.53761) -- (7.0457,2.53761);
\definecolor{c}{rgb}{0,0,0};
\colorlet{c}{kugray};
\draw [c] (7.06051,2.59337) -- (7.06051,2.64276);
\draw [c] (7.06051,2.64276) -- (7.06051,2.68531);
\draw [c] (7.0457,2.64276) -- (7.06051,2.64276);
\draw [c] (7.06051,2.64276) -- (7.07533,2.64276);
\definecolor{c}{rgb}{0,0,0};
\colorlet{c}{kugray};
\draw [c] (7.09015,2.57461) -- (7.09015,2.62915);
\draw [c] (7.09015,2.62915) -- (7.09015,2.67546);
\draw [c] (7.07533,2.62915) -- (7.09015,2.62915);
\draw [c] (7.09015,2.62915) -- (7.10497,2.62915);
\definecolor{c}{rgb}{0,0,0};
\colorlet{c}{kugray};
\draw [c] (7.11978,2.6974) -- (7.11978,2.74332);
\draw [c] (7.11978,2.74332) -- (7.11978,2.78328);
\draw [c] (7.10497,2.74332) -- (7.11978,2.74332);
\draw [c] (7.11978,2.74332) -- (7.1346,2.74332);
\definecolor{c}{rgb}{0,0,0};
\colorlet{c}{kugray};
\draw [c] (7.14942,2.66302) -- (7.14942,2.70937);
\draw [c] (7.14942,2.70937) -- (7.14942,2.74965);
\draw [c] (7.1346,2.70937) -- (7.14942,2.70937);
\draw [c] (7.14942,2.70937) -- (7.16424,2.70937);
\definecolor{c}{rgb}{0,0,0};
\colorlet{c}{kugray};
\draw [c] (7.17906,2.58815) -- (7.17906,2.64597);
\draw [c] (7.17906,2.64597) -- (7.17906,2.69464);
\draw [c] (7.16424,2.64597) -- (7.17906,2.64597);
\draw [c] (7.17906,2.64597) -- (7.19387,2.64597);
\definecolor{c}{rgb}{0,0,0};
\colorlet{c}{kugray};
\draw [c] (7.20869,2.50752) -- (7.20869,2.5657);
\draw [c] (7.20869,2.5657) -- (7.20869,2.61462);
\draw [c] (7.19387,2.5657) -- (7.20869,2.5657);
\draw [c] (7.20869,2.5657) -- (7.22351,2.5657);
\definecolor{c}{rgb}{0,0,0};
\colorlet{c}{kugray};
\draw [c] (7.23833,2.52123) -- (7.23833,2.57436);
\draw [c] (7.23833,2.57436) -- (7.23833,2.61966);
\draw [c] (7.22351,2.57436) -- (7.23833,2.57436);
\draw [c] (7.23833,2.57436) -- (7.25315,2.57436);
\definecolor{c}{rgb}{0,0,0};
\colorlet{c}{kugray};
\draw [c] (7.26796,2.55873) -- (7.26796,2.6113);
\draw [c] (7.26796,2.6113) -- (7.26796,2.65619);
\draw [c] (7.25315,2.6113) -- (7.26796,2.6113);
\draw [c] (7.26796,2.6113) -- (7.28278,2.6113);
\definecolor{c}{rgb}{0,0,0};
\colorlet{c}{kugray};
\draw [c] (7.2976,2.59094) -- (7.2976,2.64085);
\draw [c] (7.2976,2.64085) -- (7.2976,2.68379);
\draw [c] (7.28278,2.64085) -- (7.2976,2.64085);
\draw [c] (7.2976,2.64085) -- (7.31242,2.64085);
\definecolor{c}{rgb}{0,0,0};
\colorlet{c}{kugray};
\draw [c] (7.32724,2.66777) -- (7.32724,2.71706);
\draw [c] (7.32724,2.71706) -- (7.32724,2.75953);
\draw [c] (7.31242,2.71706) -- (7.32724,2.71706);
\draw [c] (7.32724,2.71706) -- (7.34205,2.71706);
\definecolor{c}{rgb}{0,0,0};
\colorlet{c}{kugray};
\draw [c] (7.35687,2.59037) -- (7.35687,2.6421);
\draw [c] (7.35687,2.6421) -- (7.35687,2.68637);
\draw [c] (7.34205,2.6421) -- (7.35687,2.6421);
\draw [c] (7.35687,2.6421) -- (7.37169,2.6421);
\definecolor{c}{rgb}{0,0,0};
\colorlet{c}{kugray};
\draw [c] (7.38651,2.5739) -- (7.38651,2.62653);
\draw [c] (7.38651,2.62653) -- (7.38651,2.67146);
\draw [c] (7.37169,2.62653) -- (7.38651,2.62653);
\draw [c] (7.38651,2.62653) -- (7.40132,2.62653);
\definecolor{c}{rgb}{0,0,0};
\colorlet{c}{kugray};
\draw [c] (7.41614,2.56099) -- (7.41614,2.61279);
\draw [c] (7.41614,2.61279) -- (7.41614,2.65711);
\draw [c] (7.40132,2.61279) -- (7.41614,2.61279);
\draw [c] (7.41614,2.61279) -- (7.43096,2.61279);
\definecolor{c}{rgb}{0,0,0};
\colorlet{c}{kugray};
\draw [c] (7.44578,2.61977) -- (7.44578,2.67016);
\draw [c] (7.44578,2.67016) -- (7.44578,2.71346);
\draw [c] (7.43096,2.67016) -- (7.44578,2.67016);
\draw [c] (7.44578,2.67016) -- (7.4606,2.67016);
\definecolor{c}{rgb}{0,0,0};
\colorlet{c}{kugray};
\draw [c] (7.47541,2.56963) -- (7.47541,2.62141);
\draw [c] (7.47541,2.62141) -- (7.47541,2.66572);
\draw [c] (7.4606,2.62141) -- (7.47541,2.62141);
\draw [c] (7.47541,2.62141) -- (7.49023,2.62141);
\definecolor{c}{rgb}{0,0,0};
\colorlet{c}{kugray};
\draw [c] (7.50505,2.56848) -- (7.50505,2.62408);
\draw [c] (7.50505,2.62408) -- (7.50505,2.67115);
\draw [c] (7.49023,2.62408) -- (7.50505,2.62408);
\draw [c] (7.50505,2.62408) -- (7.51987,2.62408);
\definecolor{c}{rgb}{0,0,0};
\colorlet{c}{kugray};
\draw [c] (7.53469,2.58512) -- (7.53469,2.63799);
\draw [c] (7.53469,2.63799) -- (7.53469,2.6831);
\draw [c] (7.51987,2.63799) -- (7.53469,2.63799);
\draw [c] (7.53469,2.63799) -- (7.5495,2.63799);
\definecolor{c}{rgb}{0,0,0};
\colorlet{c}{kugray};
\draw [c] (7.56432,2.57531) -- (7.56432,2.63381);
\draw [c] (7.56432,2.63381) -- (7.56432,2.68295);
\draw [c] (7.5495,2.63381) -- (7.56432,2.63381);
\draw [c] (7.56432,2.63381) -- (7.57914,2.63381);
\definecolor{c}{rgb}{0,0,0};
\colorlet{c}{kugray};
\draw [c] (7.59396,2.59086) -- (7.59396,2.64416);
\draw [c] (7.59396,2.64416) -- (7.59396,2.68958);
\draw [c] (7.57914,2.64416) -- (7.59396,2.64416);
\draw [c] (7.59396,2.64416) -- (7.60877,2.64416);
\definecolor{c}{rgb}{0,0,0};
\colorlet{c}{kugray};
\draw [c] (7.62359,2.69067) -- (7.62359,2.73229);
\draw [c] (7.62359,2.73229) -- (7.62359,2.76895);
\draw [c] (7.60877,2.73229) -- (7.62359,2.73229);
\draw [c] (7.62359,2.73229) -- (7.63841,2.73229);
\definecolor{c}{rgb}{0,0,0};
\colorlet{c}{kugray};
\draw [c] (7.65323,2.54373) -- (7.65323,2.60191);
\draw [c] (7.65323,2.60191) -- (7.65323,2.65082);
\draw [c] (7.63841,2.60191) -- (7.65323,2.60191);
\draw [c] (7.65323,2.60191) -- (7.66805,2.60191);
\definecolor{c}{rgb}{0,0,0};
\colorlet{c}{kugray};
\draw [c] (7.68286,2.62652) -- (7.68286,2.67485);
\draw [c] (7.68286,2.67485) -- (7.68286,2.71662);
\draw [c] (7.66805,2.67485) -- (7.68286,2.67485);
\draw [c] (7.68286,2.67485) -- (7.69768,2.67485);
\definecolor{c}{rgb}{0,0,0};
\colorlet{c}{kugray};
\draw [c] (7.7125,2.5223) -- (7.7125,2.58056);
\draw [c] (7.7125,2.58056) -- (7.7125,2.62954);
\draw [c] (7.69768,2.58056) -- (7.7125,2.58056);
\draw [c] (7.7125,2.58056) -- (7.72732,2.58056);
\definecolor{c}{rgb}{0,0,0};
\colorlet{c}{kugray};
\draw [c] (7.74214,2.52932) -- (7.74214,2.58746);
\draw [c] (7.74214,2.58746) -- (7.74214,2.63635);
\draw [c] (7.72732,2.58746) -- (7.74214,2.58746);
\draw [c] (7.74214,2.58746) -- (7.75695,2.58746);
\definecolor{c}{rgb}{0,0,0};
\colorlet{c}{kugray};
\draw [c] (7.77177,2.60875) -- (7.77177,2.65775);
\draw [c] (7.77177,2.65775) -- (7.77177,2.70001);
\draw [c] (7.75695,2.65775) -- (7.77177,2.65775);
\draw [c] (7.77177,2.65775) -- (7.78659,2.65775);
\definecolor{c}{rgb}{0,0,0};
\colorlet{c}{kugray};
\draw [c] (7.80141,2.53381) -- (7.80141,2.58535);
\draw [c] (7.80141,2.58535) -- (7.80141,2.62948);
\draw [c] (7.78659,2.58535) -- (7.80141,2.58535);
\draw [c] (7.80141,2.58535) -- (7.81623,2.58535);
\definecolor{c}{rgb}{0,0,0};
\colorlet{c}{kugray};
\draw [c] (7.83104,2.56688) -- (7.83104,2.62207);
\draw [c] (7.83104,2.62207) -- (7.83104,2.66885);
\draw [c] (7.81623,2.62207) -- (7.83104,2.62207);
\draw [c] (7.83104,2.62207) -- (7.84586,2.62207);
\definecolor{c}{rgb}{0,0,0};
\colorlet{c}{kugray};
\draw [c] (7.86068,2.53374) -- (7.86068,2.58902);
\draw [c] (7.86068,2.58902) -- (7.86068,2.63587);
\draw [c] (7.84586,2.58902) -- (7.86068,2.58902);
\draw [c] (7.86068,2.58902) -- (7.8755,2.58902);
\definecolor{c}{rgb}{0,0,0};
\colorlet{c}{kugray};
\draw [c] (7.89031,2.54177) -- (7.89031,2.59754);
\draw [c] (7.89031,2.59754) -- (7.89031,2.64474);
\draw [c] (7.8755,2.59754) -- (7.89031,2.59754);
\draw [c] (7.89031,2.59754) -- (7.90513,2.59754);
\definecolor{c}{rgb}{0,0,0};
\colorlet{c}{kugray};
\draw [c] (7.91995,2.66944) -- (7.91995,2.71223);
\draw [c] (7.91995,2.71223) -- (7.91995,2.74979);
\draw [c] (7.90513,2.71223) -- (7.91995,2.71223);
\draw [c] (7.91995,2.71223) -- (7.93477,2.71223);
\definecolor{c}{rgb}{0,0,0};
\colorlet{c}{kugray};
\draw [c] (7.94959,2.58273) -- (7.94959,2.63198);
\draw [c] (7.94959,2.63198) -- (7.94959,2.67442);
\draw [c] (7.93477,2.63198) -- (7.94959,2.63198);
\draw [c] (7.94959,2.63198) -- (7.9644,2.63198);
\definecolor{c}{rgb}{0,0,0};
\colorlet{c}{kugray};
\draw [c] (7.97922,2.51141) -- (7.97922,2.56884);
\draw [c] (7.97922,2.56884) -- (7.97922,2.61723);
\draw [c] (7.9644,2.56884) -- (7.97922,2.56884);
\draw [c] (7.97922,2.56884) -- (7.99404,2.56884);
\definecolor{c}{rgb}{0,0,0};
\colorlet{c}{kugray};
\draw [c] (8.00886,2.4778) -- (8.00886,2.5402);
\draw [c] (8.00886,2.5402) -- (8.00886,2.59206);
\draw [c] (7.99404,2.5402) -- (8.00886,2.5402);
\draw [c] (8.00886,2.5402) -- (8.02368,2.5402);
\definecolor{c}{rgb}{0,0,0};
\colorlet{c}{kugray};
\draw [c] (8.03849,2.52697) -- (8.03849,2.58159);
\draw [c] (8.03849,2.58159) -- (8.03849,2.62797);
\draw [c] (8.02368,2.58159) -- (8.03849,2.58159);
\draw [c] (8.03849,2.58159) -- (8.05331,2.58159);
\definecolor{c}{rgb}{0,0,0};
\colorlet{c}{kugray};
\draw [c] (8.06813,2.51298) -- (8.06813,2.57152);
\draw [c] (8.06813,2.57152) -- (8.06813,2.6207);
\draw [c] (8.05331,2.57152) -- (8.06813,2.57152);
\draw [c] (8.06813,2.57152) -- (8.08295,2.57152);
\definecolor{c}{rgb}{0,0,0};
\colorlet{c}{kugray};
\draw [c] (8.09776,2.42785) -- (8.09776,2.49484);
\draw [c] (8.09776,2.49484) -- (8.09776,2.54982);
\draw [c] (8.08295,2.49484) -- (8.09776,2.49484);
\draw [c] (8.09776,2.49484) -- (8.11258,2.49484);
\definecolor{c}{rgb}{0,0,0};
\colorlet{c}{kugray};
\draw [c] (8.1274,2.54013) -- (8.1274,2.59386);
\draw [c] (8.1274,2.59386) -- (8.1274,2.6396);
\draw [c] (8.11258,2.59386) -- (8.1274,2.59386);
\draw [c] (8.1274,2.59386) -- (8.14222,2.59386);
\definecolor{c}{rgb}{0,0,0};
\colorlet{c}{kugray};
\draw [c] (8.15704,2.5211) -- (8.15704,2.5755);
\draw [c] (8.15704,2.5755) -- (8.15704,2.62172);
\draw [c] (8.14222,2.5755) -- (8.15704,2.5755);
\draw [c] (8.15704,2.5755) -- (8.17185,2.5755);
\definecolor{c}{rgb}{0,0,0};
\colorlet{c}{kugray};
\draw [c] (8.18667,2.48828) -- (8.18667,2.54723);
\draw [c] (8.18667,2.54723) -- (8.18667,2.59668);
\draw [c] (8.17185,2.54723) -- (8.18667,2.54723);
\draw [c] (8.18667,2.54723) -- (8.20149,2.54723);
\definecolor{c}{rgb}{0,0,0};
\colorlet{c}{kugray};
\draw [c] (8.21631,2.51647) -- (8.21631,2.57511);
\draw [c] (8.21631,2.57511) -- (8.21631,2.62434);
\draw [c] (8.20149,2.57511) -- (8.21631,2.57511);
\draw [c] (8.21631,2.57511) -- (8.23113,2.57511);
\definecolor{c}{rgb}{0,0,0};
\colorlet{c}{kugray};
\draw [c] (8.24594,2.4989) -- (8.24594,2.55998);
\draw [c] (8.24594,2.55998) -- (8.24594,2.61092);
\draw [c] (8.23113,2.55998) -- (8.24594,2.55998);
\draw [c] (8.24594,2.55998) -- (8.26076,2.55998);
\definecolor{c}{rgb}{0,0,0};
\colorlet{c}{kugray};
\draw [c] (8.27558,2.43272) -- (8.27558,2.4971);
\draw [c] (8.27558,2.4971) -- (8.27558,2.55033);
\draw [c] (8.26076,2.4971) -- (8.27558,2.4971);
\draw [c] (8.27558,2.4971) -- (8.2904,2.4971);
\definecolor{c}{rgb}{0,0,0};
\colorlet{c}{kugray};
\draw [c] (8.30521,2.47691) -- (8.30521,2.53483);
\draw [c] (8.30521,2.53483) -- (8.30521,2.58355);
\draw [c] (8.2904,2.53483) -- (8.30521,2.53483);
\draw [c] (8.30521,2.53483) -- (8.32003,2.53483);
\definecolor{c}{rgb}{0,0,0};
\colorlet{c}{kugray};
\draw [c] (8.33485,2.49351) -- (8.33485,2.55549);
\draw [c] (8.33485,2.55549) -- (8.33485,2.60705);
\draw [c] (8.32003,2.55549) -- (8.33485,2.55549);
\draw [c] (8.33485,2.55549) -- (8.34967,2.55549);
\definecolor{c}{rgb}{0,0,0};
\colorlet{c}{kugray};
\draw [c] (8.36449,2.4196) -- (8.36449,2.48391);
\draw [c] (8.36449,2.48391) -- (8.36449,2.53708);
\draw [c] (8.34967,2.48391) -- (8.36449,2.48391);
\draw [c] (8.36449,2.48391) -- (8.3793,2.48391);
\definecolor{c}{rgb}{0,0,0};
\colorlet{c}{kugray};
\draw [c] (8.39412,2.39842) -- (8.39412,2.46762);
\draw [c] (8.39412,2.46762) -- (8.39412,2.52409);
\draw [c] (8.3793,2.46762) -- (8.39412,2.46762);
\draw [c] (8.39412,2.46762) -- (8.40894,2.46762);
\definecolor{c}{rgb}{0,0,0};
\colorlet{c}{kugray};
\draw [c] (8.42376,2.50236) -- (8.42376,2.56041);
\draw [c] (8.42376,2.56041) -- (8.42376,2.60923);
\draw [c] (8.40894,2.56041) -- (8.42376,2.56041);
\draw [c] (8.42376,2.56041) -- (8.43858,2.56041);
\definecolor{c}{rgb}{0,0,0};
\colorlet{c}{kugray};
\draw [c] (8.45339,2.4599) -- (8.45339,2.5192);
\draw [c] (8.45339,2.5192) -- (8.45339,2.56891);
\draw [c] (8.43858,2.5192) -- (8.45339,2.5192);
\draw [c] (8.45339,2.5192) -- (8.46821,2.5192);
\definecolor{c}{rgb}{0,0,0};
\colorlet{c}{kugray};
\draw [c] (8.48303,2.55597) -- (8.48303,2.61113);
\draw [c] (8.48303,2.61113) -- (8.48303,2.65789);
\draw [c] (8.46821,2.61113) -- (8.48303,2.61113);
\draw [c] (8.48303,2.61113) -- (8.49785,2.61113);
\definecolor{c}{rgb}{0,0,0};
\colorlet{c}{kugray};
\draw [c] (8.51267,2.3936) -- (8.51267,2.46748);
\draw [c] (8.51267,2.46748) -- (8.51267,2.52702);
\draw [c] (8.49785,2.46748) -- (8.51267,2.46748);
\draw [c] (8.51267,2.46748) -- (8.52748,2.46748);
\definecolor{c}{rgb}{0,0,0};
\colorlet{c}{kugray};
\draw [c] (8.5423,2.42909) -- (8.5423,2.50139);
\draw [c] (8.5423,2.50139) -- (8.5423,2.5599);
\draw [c] (8.52748,2.50139) -- (8.5423,2.50139);
\draw [c] (8.5423,2.50139) -- (8.55712,2.50139);
\definecolor{c}{rgb}{0,0,0};
\colorlet{c}{kugray};
\draw [c] (8.57194,2.35401) -- (8.57194,2.42763);
\draw [c] (8.57194,2.42763) -- (8.57194,2.487);
\draw [c] (8.55712,2.42763) -- (8.57194,2.42763);
\draw [c] (8.57194,2.42763) -- (8.58675,2.42763);
\definecolor{c}{rgb}{0,0,0};
\colorlet{c}{kugray};
\draw [c] (8.60157,2.37123) -- (8.60157,2.4445);
\draw [c] (8.60157,2.4445) -- (8.60157,2.50364);
\draw [c] (8.58675,2.4445) -- (8.60157,2.4445);
\draw [c] (8.60157,2.4445) -- (8.61639,2.4445);
\definecolor{c}{rgb}{0,0,0};
\colorlet{c}{kugray};
\draw [c] (8.63121,2.43393) -- (8.63121,2.50355);
\draw [c] (8.63121,2.50355) -- (8.63121,2.5603);
\draw [c] (8.61639,2.50355) -- (8.63121,2.50355);
\draw [c] (8.63121,2.50355) -- (8.64603,2.50355);
\definecolor{c}{rgb}{0,0,0};
\colorlet{c}{kugray};
\draw [c] (8.66084,2.47327) -- (8.66084,2.53303);
\draw [c] (8.66084,2.53303) -- (8.66084,2.58305);
\draw [c] (8.64603,2.53303) -- (8.66084,2.53303);
\draw [c] (8.66084,2.53303) -- (8.67566,2.53303);
\definecolor{c}{rgb}{0,0,0};
\colorlet{c}{kugray};
\draw [c] (8.69048,2.46275) -- (8.69048,2.52243);
\draw [c] (8.69048,2.52243) -- (8.69048,2.5724);
\draw [c] (8.67566,2.52243) -- (8.69048,2.52243);
\draw [c] (8.69048,2.52243) -- (8.7053,2.52243);
\definecolor{c}{rgb}{0,0,0};
\colorlet{c}{kugray};
\draw [c] (8.72012,2.33119) -- (8.72012,2.41226);
\draw [c] (8.72012,2.41226) -- (8.72012,2.47637);
\draw [c] (8.7053,2.41226) -- (8.72012,2.41226);
\draw [c] (8.72012,2.41226) -- (8.73493,2.41226);
\definecolor{c}{rgb}{0,0,0};
\colorlet{c}{kugray};
\draw [c] (8.74975,2.3572) -- (8.74975,2.4275);
\draw [c] (8.74975,2.4275) -- (8.74975,2.48469);
\draw [c] (8.73493,2.4275) -- (8.74975,2.4275);
\draw [c] (8.74975,2.4275) -- (8.76457,2.4275);
\definecolor{c}{rgb}{0,0,0};
\colorlet{c}{kugray};
\draw [c] (8.77939,2.41881) -- (8.77939,2.49045);
\draw [c] (8.77939,2.49045) -- (8.77939,2.54853);
\draw [c] (8.76457,2.49045) -- (8.77939,2.49045);
\draw [c] (8.77939,2.49045) -- (8.79421,2.49045);
\definecolor{c}{rgb}{0,0,0};
\colorlet{c}{kugray};
\draw [c] (8.80902,2.52182) -- (8.80902,2.57585);
\draw [c] (8.80902,2.57585) -- (8.80902,2.62181);
\draw [c] (8.79421,2.57585) -- (8.80902,2.57585);
\draw [c] (8.80902,2.57585) -- (8.82384,2.57585);
\definecolor{c}{rgb}{0,0,0};
\colorlet{c}{kugray};
\draw [c] (8.83866,2.47132) -- (8.83866,2.53141);
\draw [c] (8.83866,2.53141) -- (8.83866,2.58166);
\draw [c] (8.82384,2.53141) -- (8.83866,2.53141);
\draw [c] (8.83866,2.53141) -- (8.85348,2.53141);
\definecolor{c}{rgb}{0,0,0};
\colorlet{c}{kugray};
\draw [c] (8.86829,2.49235) -- (8.86829,2.5502);
\draw [c] (8.86829,2.5502) -- (8.86829,2.59887);
\draw [c] (8.85348,2.5502) -- (8.86829,2.5502);
\draw [c] (8.86829,2.5502) -- (8.88311,2.5502);
\definecolor{c}{rgb}{0,0,0};
\colorlet{c}{kugray};
\draw [c] (8.89793,2.47756) -- (8.89793,2.54006);
\draw [c] (8.89793,2.54006) -- (8.89793,2.59198);
\draw [c] (8.88311,2.54006) -- (8.89793,2.54006);
\draw [c] (8.89793,2.54006) -- (8.91275,2.54006);
\definecolor{c}{rgb}{0,0,0};
\colorlet{c}{kugray};
\draw [c] (8.92757,2.39734) -- (8.92757,2.46961);
\draw [c] (8.92757,2.46961) -- (8.92757,2.52811);
\draw [c] (8.91275,2.46961) -- (8.92757,2.46961);
\draw [c] (8.92757,2.46961) -- (8.94238,2.46961);
\definecolor{c}{rgb}{0,0,0};
\colorlet{c}{kugray};
\draw [c] (8.9572,2.42281) -- (8.9572,2.48693);
\draw [c] (8.9572,2.48693) -- (8.9572,2.53997);
\draw [c] (8.94238,2.48693) -- (8.9572,2.48693);
\draw [c] (8.9572,2.48693) -- (8.97202,2.48693);
\definecolor{c}{rgb}{0,0,0};
\colorlet{c}{kugray};
\draw [c] (8.98684,2.45799) -- (8.98684,2.51852);
\draw [c] (8.98684,2.51852) -- (8.98684,2.56909);
\draw [c] (8.97202,2.51852) -- (8.98684,2.51852);
\draw [c] (8.98684,2.51852) -- (9.00166,2.51852);
\definecolor{c}{rgb}{0,0,0};
\colorlet{c}{kugray};
\draw [c] (9.01647,2.30805) -- (9.01647,2.39137);
\draw [c] (9.01647,2.39137) -- (9.01647,2.45687);
\draw [c] (9.00166,2.39137) -- (9.01647,2.39137);
\draw [c] (9.01647,2.39137) -- (9.03129,2.39137);
\definecolor{c}{rgb}{0,0,0};
\colorlet{c}{kugray};
\draw [c] (9.04611,2.42762) -- (9.04611,2.49931);
\draw [c] (9.04611,2.49931) -- (9.04611,2.55742);
\draw [c] (9.03129,2.49931) -- (9.04611,2.49931);
\draw [c] (9.04611,2.49931) -- (9.06093,2.49931);
\definecolor{c}{rgb}{0,0,0};
\colorlet{c}{kugray};
\draw [c] (9.07574,2.43243) -- (9.07574,2.5203);
\draw [c] (9.07574,2.5203) -- (9.07574,2.58858);
\draw [c] (9.06093,2.5203) -- (9.07574,2.5203);
\draw [c] (9.07574,2.5203) -- (9.09056,2.5203);
\definecolor{c}{rgb}{0,0,0};
\colorlet{c}{kugray};
\draw [c] (9.10538,2.33902) -- (9.10538,2.41416);
\draw [c] (9.10538,2.41416) -- (9.10538,2.47451);
\draw [c] (9.09056,2.41416) -- (9.10538,2.41416);
\draw [c] (9.10538,2.41416) -- (9.1202,2.41416);
\definecolor{c}{rgb}{0,0,0};
\colorlet{c}{kugray};
\draw [c] (9.13502,2.4421) -- (9.13502,2.50747);
\draw [c] (9.13502,2.50747) -- (9.13502,2.56135);
\draw [c] (9.1202,2.50747) -- (9.13502,2.50747);
\draw [c] (9.13502,2.50747) -- (9.14983,2.50747);
\definecolor{c}{rgb}{0,0,0};
\colorlet{c}{kugray};
\draw [c] (9.16465,2.44597) -- (9.16465,2.51695);
\draw [c] (9.16465,2.51695) -- (9.16465,2.5746);
\draw [c] (9.14983,2.51695) -- (9.16465,2.51695);
\draw [c] (9.16465,2.51695) -- (9.17947,2.51695);
\definecolor{c}{rgb}{0,0,0};
\colorlet{c}{kugray};
\draw [c] (9.19429,2.3813) -- (9.19429,2.45249);
\draw [c] (9.19429,2.45249) -- (9.19429,2.51027);
\draw [c] (9.17947,2.45249) -- (9.19429,2.45249);
\draw [c] (9.19429,2.45249) -- (9.20911,2.45249);
\definecolor{c}{rgb}{0,0,0};
\colorlet{c}{kugray};
\draw [c] (9.22392,2.26959) -- (9.22392,2.35173);
\draw [c] (9.22392,2.35173) -- (9.22392,2.4165);
\draw [c] (9.20911,2.35173) -- (9.22392,2.35173);
\draw [c] (9.22392,2.35173) -- (9.23874,2.35173);
\definecolor{c}{rgb}{0,0,0};
\colorlet{c}{kugray};
\draw [c] (9.25356,2.47023) -- (9.25356,2.53233);
\draw [c] (9.25356,2.53233) -- (9.25356,2.58399);
\draw [c] (9.23874,2.53233) -- (9.25356,2.53233);
\draw [c] (9.25356,2.53233) -- (9.26838,2.53233);
\definecolor{c}{rgb}{0,0,0};
\colorlet{c}{kugray};
\draw [c] (9.2832,2.25761) -- (9.2832,2.34259);
\draw [c] (9.2832,2.34259) -- (9.2832,2.40911);
\draw [c] (9.26838,2.34259) -- (9.2832,2.34259);
\draw [c] (9.2832,2.34259) -- (9.29801,2.34259);
\definecolor{c}{rgb}{0,0,0};
\colorlet{c}{kugray};
\draw [c] (9.31283,2.38074) -- (9.31283,2.45146);
\draw [c] (9.31283,2.45146) -- (9.31283,2.50893);
\draw [c] (9.29801,2.45146) -- (9.31283,2.45146);
\draw [c] (9.31283,2.45146) -- (9.32765,2.45146);
\definecolor{c}{rgb}{0,0,0};
\colorlet{c}{kugray};
\draw [c] (9.34247,2.30157) -- (9.34247,2.38622);
\draw [c] (9.34247,2.38622) -- (9.34247,2.45254);
\draw [c] (9.32765,2.38622) -- (9.34247,2.38622);
\draw [c] (9.34247,2.38622) -- (9.35728,2.38622);
\definecolor{c}{rgb}{0,0,0};
\colorlet{c}{kugray};
\draw [c] (9.3721,2.44079) -- (9.3721,2.50347);
\draw [c] (9.3721,2.50347) -- (9.3721,2.55552);
\draw [c] (9.35728,2.50347) -- (9.3721,2.50347);
\draw [c] (9.3721,2.50347) -- (9.38692,2.50347);
\definecolor{c}{rgb}{0,0,0};
\colorlet{c}{kugray};
\draw [c] (9.40174,2.35916) -- (9.40174,2.42967);
\draw [c] (9.40174,2.42967) -- (9.40174,2.487);
\draw [c] (9.38692,2.42967) -- (9.40174,2.42967);
\draw [c] (9.40174,2.42967) -- (9.41656,2.42967);
\definecolor{c}{rgb}{0,0,0};
\colorlet{c}{kugray};
\draw [c] (9.43137,2.34873) -- (9.43137,2.418);
\draw [c] (9.43137,2.418) -- (9.43137,2.4745);
\draw [c] (9.41656,2.418) -- (9.43137,2.418);
\draw [c] (9.43137,2.418) -- (9.44619,2.418);
\definecolor{c}{rgb}{0,0,0};
\colorlet{c}{kugray};
\draw [c] (9.46101,2.48472) -- (9.46101,2.54922);
\draw [c] (9.46101,2.54922) -- (9.46101,2.60252);
\draw [c] (9.44619,2.54922) -- (9.46101,2.54922);
\draw [c] (9.46101,2.54922) -- (9.47583,2.54922);
\definecolor{c}{rgb}{0,0,0};
\colorlet{c}{kugray};
\draw [c] (9.49065,2.28137) -- (9.49065,2.3622);
\draw [c] (9.49065,2.3622) -- (9.49065,2.42615);
\draw [c] (9.47583,2.3622) -- (9.49065,2.3622);
\draw [c] (9.49065,2.3622) -- (9.50546,2.3622);
\definecolor{c}{rgb}{0,0,0};
\colorlet{c}{kugray};
\draw [c] (9.52028,2.44204) -- (9.52028,2.50736);
\draw [c] (9.52028,2.50736) -- (9.52028,2.56121);
\draw [c] (9.50546,2.50736) -- (9.52028,2.50736);
\draw [c] (9.52028,2.50736) -- (9.5351,2.50736);
\definecolor{c}{rgb}{0,0,0};
\colorlet{c}{kugray};
\draw [c] (9.54992,2.16356) -- (9.54992,2.26322);
\draw [c] (9.54992,2.26322) -- (9.54992,2.33839);
\draw [c] (9.5351,2.26322) -- (9.54992,2.26322);
\draw [c] (9.54992,2.26322) -- (9.56474,2.26322);
\definecolor{c}{rgb}{0,0,0};
\colorlet{c}{kugray};
\draw [c] (9.57955,2.38059) -- (9.57955,2.45112);
\draw [c] (9.57955,2.45112) -- (9.57955,2.50847);
\draw [c] (9.56474,2.45112) -- (9.57955,2.45112);
\draw [c] (9.57955,2.45112) -- (9.59437,2.45112);
\definecolor{c}{rgb}{0,0,0};
\colorlet{c}{kugray};
\draw [c] (9.60919,2.39358) -- (9.60919,2.4702);
\draw [c] (9.60919,2.4702) -- (9.60919,2.5315);
\draw [c] (9.59437,2.4702) -- (9.60919,2.4702);
\draw [c] (9.60919,2.4702) -- (9.62401,2.4702);
\definecolor{c}{rgb}{0,0,0};
\colorlet{c}{kugray};
\draw [c] (9.63882,2.215) -- (9.63882,2.30298);
\draw [c] (9.63882,2.30298) -- (9.63882,2.37132);
\draw [c] (9.62401,2.30298) -- (9.63882,2.30298);
\draw [c] (9.63882,2.30298) -- (9.65364,2.30298);
\definecolor{c}{rgb}{0,0,0};
\colorlet{c}{kugray};
\draw [c] (9.66846,2.31943) -- (9.66846,2.39972);
\draw [c] (9.66846,2.39972) -- (9.66846,2.46334);
\draw [c] (9.65364,2.39972) -- (9.66846,2.39972);
\draw [c] (9.66846,2.39972) -- (9.68328,2.39972);
\definecolor{c}{rgb}{0,0,0};
\colorlet{c}{kugray};
\draw [c] (9.6981,2.36691) -- (9.6981,2.44427);
\draw [c] (9.6981,2.44427) -- (9.6981,2.50605);
\draw [c] (9.68328,2.44427) -- (9.6981,2.44427);
\draw [c] (9.6981,2.44427) -- (9.71291,2.44427);
\definecolor{c}{rgb}{0,0,0};
\colorlet{c}{kugray};
\draw [c] (9.72773,2.36518) -- (9.72773,2.44376);
\draw [c] (9.72773,2.44376) -- (9.72773,2.50632);
\draw [c] (9.71291,2.44376) -- (9.72773,2.44376);
\draw [c] (9.72773,2.44376) -- (9.74255,2.44376);
\definecolor{c}{rgb}{0,0,0};
\colorlet{c}{kugray};
\draw [c] (9.75737,2.17203) -- (9.75737,2.267);
\draw [c] (9.75737,2.267) -- (9.75737,2.33947);
\draw [c] (9.74255,2.267) -- (9.75737,2.267);
\draw [c] (9.75737,2.267) -- (9.77219,2.267);
\definecolor{c}{rgb}{0,0,0};
\colorlet{c}{kugray};
\draw [c] (9.787,2.27887) -- (9.787,2.35679);
\draw [c] (9.787,2.35679) -- (9.787,2.41892);
\draw [c] (9.77219,2.35679) -- (9.787,2.35679);
\draw [c] (9.787,2.35679) -- (9.80182,2.35679);
\definecolor{c}{rgb}{0,0,0};
\colorlet{c}{kugray};
\draw [c] (9.81664,2.11839) -- (9.81664,2.23312);
\draw [c] (9.81664,2.23312) -- (9.81664,2.31651);
\draw [c] (9.80182,2.23312) -- (9.81664,2.23312);
\draw [c] (9.81664,2.23312) -- (9.83146,2.23312);
\definecolor{c}{rgb}{0,0,0};
\colorlet{c}{kugray};
\draw [c] (9.84627,2.36042) -- (9.84627,2.42873);
\draw [c] (9.84627,2.42873) -- (9.84627,2.4846);
\draw [c] (9.83146,2.42873) -- (9.84627,2.42873);
\draw [c] (9.84627,2.42873) -- (9.86109,2.42873);
\definecolor{c}{rgb}{0,0,0};
\colorlet{c}{kugray};
\draw [c] (9.87591,2.33174) -- (9.87591,2.40637);
\draw [c] (9.87591,2.40637) -- (9.87591,2.46639);
\draw [c] (9.86109,2.40637) -- (9.87591,2.40637);
\draw [c] (9.87591,2.40637) -- (9.89073,2.40637);
\definecolor{c}{rgb}{0,0,0};
\colorlet{c}{kugray};
\draw [c] (9.90555,2.36337) -- (9.90555,2.43811);
\draw [c] (9.90555,2.43811) -- (9.90555,2.4982);
\draw [c] (9.89073,2.43811) -- (9.90555,2.43811);
\draw [c] (9.90555,2.43811) -- (9.92036,2.43811);
\definecolor{c}{rgb}{0,0,0};
\colorlet{c}{kugray};
\draw [c] (9.93518,2.22165) -- (9.93518,2.328);
\draw [c] (9.93518,2.328) -- (9.93518,2.4069);
\draw [c] (9.92036,2.328) -- (9.93518,2.328);
\draw [c] (9.93518,2.328) -- (9.95,2.328);
\definecolor{c}{rgb}{0,0,0};
\colorlet{c}{natcomp};
\draw [c] (1.51655,5.54553) -- (1.60131,5.4045) -- (1.68607,5.27517) -- (1.77083,5.15545) -- (1.85558,5.0438) -- (1.94034,4.93904) -- (2.0251,4.84024) -- (2.10986,4.74665) -- (2.19462,4.65769) -- (2.27938,4.57286) -- (2.36413,4.49176)
 -- (2.44889,4.41407) -- (2.53365,4.33949) -- (2.61841,4.26781) -- (2.70317,4.19881) -- (2.78793,4.13234) -- (2.87268,4.06825) -- (2.95744,4.00643) -- (3.0422,3.94677) -- (3.12696,3.8892) -- (3.21172,3.83363) -- (3.29648,3.78001) -- (3.38123,3.72829)
 -- (3.46599,3.67842) -- (3.55075,3.63037) -- (3.63551,3.58409) -- (3.72027,3.53955) -- (3.80502,3.49673) -- (3.88978,3.45559) -- (3.97454,3.4161) -- (4.0593,3.37822) -- (4.14406,3.34192) -- (4.22882,3.30715) -- (4.31357,3.27388) -- (4.39833,3.24205)
 -- (4.48309,3.21162) -- (4.56785,3.18254) -- (4.65261,3.15475) -- (4.73737,3.12818) -- (4.82212,3.1028) -- (4.90688,3.07853) -- (4.99164,3.05531) -- (5.0764,3.03308) -- (5.16116,3.01179) -- (5.24592,2.99138) -- (5.33067,2.97179) -- (5.41543,2.95296)
 -- (5.50019,2.93484) -- (5.58495,2.91738) -- (5.66971,2.90053);
\draw [c] (5.66971,2.90053) -- (5.75447,2.88424) -- (5.83922,2.86848) -- (5.92398,2.8532) -- (6.00874,2.83837) -- (6.0935,2.82394) -- (6.17826,2.80989) -- (6.26301,2.79618) -- (6.34777,2.7828) -- (6.43253,2.7697) -- (6.51729,2.75687)
 -- (6.60205,2.74428) -- (6.68681,2.73192) -- (6.77156,2.71977) -- (6.85632,2.7078) -- (6.94108,2.69601) -- (7.02584,2.68437) -- (7.1106,2.67289) -- (7.19536,2.66153) -- (7.28011,2.6503) -- (7.36487,2.63918) -- (7.44963,2.62817) -- (7.53439,2.61724)
 -- (7.61915,2.60641) -- (7.70391,2.59566) -- (7.78866,2.58498) -- (7.87342,2.57436) -- (7.95818,2.56381) -- (8.04294,2.55332) -- (8.1277,2.54287) -- (8.21246,2.53247) -- (8.29721,2.52212) -- (8.38197,2.5118) -- (8.46673,2.50152) -- (8.55149,2.49127)
 -- (8.63625,2.48104) -- (8.72101,2.47084) -- (8.80576,2.46065) -- (8.89052,2.45047) -- (8.97528,2.4403) -- (9.06004,2.43013) -- (9.1448,2.41995) -- (9.22955,2.40976) -- (9.31431,2.39954) -- (9.39907,2.38929) -- (9.48383,2.379) -- (9.56859,2.36866)
 -- (9.65335,2.35824) -- (9.7381,2.34774) -- (9.82286,2.33714);
\draw [c] (9.82286,2.33714) -- (9.90762,2.32641);
\colorlet{c}{kugray};
\draw [c] (1.01482,0.596817) -- (1.01482,3.61122);
\draw [c] (1.01482,3.61122) -- (1.01482,3.82461);
\draw [c] (1,3.61122) -- (1.01482,3.61122);
\draw [c] (1.01482,3.61122) -- (1.02964,3.61122);
\definecolor{c}{rgb}{0,0,0};
\colorlet{c}{kugray};
\draw [c] (1.04445,0.596817) -- (1.04445,1.81958);
\draw [c] (1.04445,1.81958) -- (1.04445,2.03297);
\draw [c] (1.02964,1.81958) -- (1.04445,1.81958);
\draw [c] (1.04445,1.81958) -- (1.05927,1.81958);
\definecolor{c}{rgb}{0,0,0};
\colorlet{c}{kugray};
\draw [c] (1.07409,0.596817) -- (1.07409,3.48891);
\draw [c] (1.07409,3.48891) -- (1.07409,3.7023);
\draw [c] (1.05927,3.48891) -- (1.07409,3.48891);
\draw [c] (1.07409,3.48891) -- (1.08891,3.48891);
\definecolor{c}{rgb}{0,0,0};
\colorlet{c}{kugray};
\draw [c] (1.10373,3.60302) -- (1.10373,3.87778);
\draw [c] (1.10373,3.87778) -- (1.10373,4.02061);
\draw [c] (1.08891,3.87778) -- (1.10373,3.87778);
\draw [c] (1.10373,3.87778) -- (1.11854,3.87778);
\definecolor{c}{rgb}{0,0,0};
\colorlet{c}{kugray};
\draw [c] (1.13336,3.42057) -- (1.13336,3.80196);
\draw [c] (1.13336,3.80196) -- (1.13336,3.96717);
\draw [c] (1.11854,3.80196) -- (1.13336,3.80196);
\draw [c] (1.13336,3.80196) -- (1.14818,3.80196);
\definecolor{c}{rgb}{0,0,0};
\colorlet{c}{kugray};
\draw [c] (1.163,0.596817) -- (1.163,2.01527);
\draw [c] (1.163,2.01527) -- (1.163,2.22866);
\draw [c] (1.14818,2.01527) -- (1.163,2.01527);
\draw [c] (1.163,2.01527) -- (1.17781,2.01527);
\definecolor{c}{rgb}{0,0,0};
\colorlet{c}{kugray};
\draw [c] (1.19263,0.596817) -- (1.19263,3.51401);
\draw [c] (1.19263,3.51401) -- (1.19263,3.7274);
\draw [c] (1.17781,3.51401) -- (1.19263,3.51401);
\draw [c] (1.19263,3.51401) -- (1.20745,3.51401);
\definecolor{c}{rgb}{0,0,0};
\colorlet{c}{kugray};
\draw [c] (1.22227,3.43062) -- (1.22227,3.83166);
\draw [c] (1.22227,3.83166) -- (1.22227,4.00008);
\draw [c] (1.20745,3.83166) -- (1.22227,3.83166);
\draw [c] (1.22227,3.83166) -- (1.23709,3.83166);
\definecolor{c}{rgb}{0,0,0};
\colorlet{c}{kugray};
\draw [c] (1.2519,3.8397) -- (1.2519,4.0877);
\draw [c] (1.2519,4.0877) -- (1.2519,4.22325);
\draw [c] (1.23709,4.0877) -- (1.2519,4.0877);
\draw [c] (1.2519,4.0877) -- (1.26672,4.0877);
\definecolor{c}{rgb}{0,0,0};
\colorlet{c}{kugray};
\draw [c] (1.28154,3.98744) -- (1.28154,4.15251);
\draw [c] (1.28154,4.15251) -- (1.28154,4.25938);
\draw [c] (1.26672,4.15251) -- (1.28154,4.15251);
\draw [c] (1.28154,4.15251) -- (1.29636,4.15251);
\definecolor{c}{rgb}{0,0,0};
\colorlet{c}{kugray};
\draw [c] (1.31118,5.3327) -- (1.31118,5.34989);
\draw [c] (1.31118,5.34989) -- (1.31118,5.36617);
\draw [c] (1.29636,5.34989) -- (1.31118,5.34989);
\draw [c] (1.31118,5.34989) -- (1.32599,5.34989);
\definecolor{c}{rgb}{0,0,0};
\colorlet{c}{kugray};
\draw [c] (1.34081,5.6418) -- (1.34081,5.6522);
\draw [c] (1.34081,5.6522) -- (1.34081,5.66227);
\draw [c] (1.32599,5.6522) -- (1.34081,5.6522);
\draw [c] (1.34081,5.6522) -- (1.35563,5.6522);
\definecolor{c}{rgb}{0,0,0};
\colorlet{c}{kugray};
\draw [c] (1.37045,5.68448) -- (1.37045,5.69437);
\draw [c] (1.37045,5.69437) -- (1.37045,5.70394);
\draw [c] (1.35563,5.69437) -- (1.37045,5.69437);
\draw [c] (1.37045,5.69437) -- (1.38526,5.69437);
\definecolor{c}{rgb}{0,0,0};
\colorlet{c}{kugray};
\draw [c] (1.40008,5.69232) -- (1.40008,5.70203);
\draw [c] (1.40008,5.70203) -- (1.40008,5.71145);
\draw [c] (1.38526,5.70203) -- (1.40008,5.70203);
\draw [c] (1.40008,5.70203) -- (1.4149,5.70203);
\definecolor{c}{rgb}{0,0,0};
\colorlet{c}{kugray};
\draw [c] (1.42972,5.65625) -- (1.42972,5.66643);
\draw [c] (1.42972,5.66643) -- (1.42972,5.67629);
\draw [c] (1.4149,5.66643) -- (1.42972,5.66643);
\draw [c] (1.42972,5.66643) -- (1.44454,5.66643);
\definecolor{c}{rgb}{0,0,0};
\colorlet{c}{kugray};
\draw [c] (1.45935,5.61758) -- (1.45935,5.62829);
\draw [c] (1.45935,5.62829) -- (1.45935,5.63863);
\draw [c] (1.44454,5.62829) -- (1.45935,5.62829);
\draw [c] (1.45935,5.62829) -- (1.47417,5.62829);
\definecolor{c}{rgb}{0,0,0};
\colorlet{c}{kugray};
\draw [c] (1.48899,5.57445) -- (1.48899,5.58624);
\draw [c] (1.48899,5.58624) -- (1.48899,5.5976);
\draw [c] (1.47417,5.58624) -- (1.48899,5.58624);
\draw [c] (1.48899,5.58624) -- (1.50381,5.58624);
\definecolor{c}{rgb}{0,0,0};
\colorlet{c}{kugray};
\draw [c] (1.51863,5.53272) -- (1.51863,5.54514);
\draw [c] (1.51863,5.54514) -- (1.51863,5.55708);
\draw [c] (1.50381,5.54514) -- (1.51863,5.54514);
\draw [c] (1.51863,5.54514) -- (1.53344,5.54514);
\definecolor{c}{rgb}{0,0,0};
\colorlet{c}{kugray};
\draw [c] (1.54826,5.48596) -- (1.54826,5.49969);
\draw [c] (1.54826,5.49969) -- (1.54826,5.51284);
\draw [c] (1.53344,5.49969) -- (1.54826,5.49969);
\draw [c] (1.54826,5.49969) -- (1.56308,5.49969);
\definecolor{c}{rgb}{0,0,0};
\colorlet{c}{kugray};
\draw [c] (1.5779,5.4138) -- (1.5779,5.42885);
\draw [c] (1.5779,5.42885) -- (1.5779,5.44319);
\draw [c] (1.56308,5.42885) -- (1.5779,5.42885);
\draw [c] (1.5779,5.42885) -- (1.59272,5.42885);
\definecolor{c}{rgb}{0,0,0};
\colorlet{c}{kugray};
\draw [c] (1.60753,5.41345) -- (1.60753,5.42901);
\draw [c] (1.60753,5.42901) -- (1.60753,5.44382);
\draw [c] (1.59272,5.42901) -- (1.60753,5.42901);
\draw [c] (1.60753,5.42901) -- (1.62235,5.42901);
\definecolor{c}{rgb}{0,0,0};
\colorlet{c}{kugray};
\draw [c] (1.63717,5.34137) -- (1.63717,5.35821);
\draw [c] (1.63717,5.35821) -- (1.63717,5.37418);
\draw [c] (1.62235,5.35821) -- (1.63717,5.35821);
\draw [c] (1.63717,5.35821) -- (1.65199,5.35821);
\definecolor{c}{rgb}{0,0,0};
\colorlet{c}{kugray};
\draw [c] (1.6668,5.31592) -- (1.6668,5.33358);
\draw [c] (1.6668,5.33358) -- (1.6668,5.35028);
\draw [c] (1.65199,5.33358) -- (1.6668,5.33358);
\draw [c] (1.6668,5.33358) -- (1.68162,5.33358);
\definecolor{c}{rgb}{0,0,0};
\colorlet{c}{kugray};
\draw [c] (1.69644,5.25517) -- (1.69644,5.2753);
\draw [c] (1.69644,5.2753) -- (1.69644,5.2942);
\draw [c] (1.68162,5.2753) -- (1.69644,5.2753);
\draw [c] (1.69644,5.2753) -- (1.71126,5.2753);
\definecolor{c}{rgb}{0,0,0};
\colorlet{c}{kugray};
\draw [c] (1.72608,5.20002) -- (1.72608,5.2216);
\draw [c] (1.72608,5.2216) -- (1.72608,5.24177);
\draw [c] (1.71126,5.2216) -- (1.72608,5.2216);
\draw [c] (1.72608,5.2216) -- (1.74089,5.2216);
\definecolor{c}{rgb}{0,0,0};
\colorlet{c}{kugray};
\draw [c] (1.75571,5.14158) -- (1.75571,5.16518);
\draw [c] (1.75571,5.16518) -- (1.75571,5.18709);
\draw [c] (1.74089,5.16518) -- (1.75571,5.16518);
\draw [c] (1.75571,5.16518) -- (1.77053,5.16518);
\definecolor{c}{rgb}{0,0,0};
\colorlet{c}{kugray};
\draw [c] (1.78535,5.11657) -- (1.78535,5.14067);
\draw [c] (1.78535,5.14067) -- (1.78535,5.16302);
\draw [c] (1.77053,5.14067) -- (1.78535,5.14067);
\draw [c] (1.78535,5.14067) -- (1.80017,5.14067);
\definecolor{c}{rgb}{0,0,0};
\colorlet{c}{kugray};
\draw [c] (1.81498,5.10912) -- (1.81498,5.13446);
\draw [c] (1.81498,5.13446) -- (1.81498,5.15788);
\draw [c] (1.80017,5.13446) -- (1.81498,5.13446);
\draw [c] (1.81498,5.13446) -- (1.8298,5.13446);
\definecolor{c}{rgb}{0,0,0};
\colorlet{c}{kugray};
\draw [c] (1.84462,5.02683) -- (1.84462,5.05542);
\draw [c] (1.84462,5.05542) -- (1.84462,5.08158);
\draw [c] (1.8298,5.05542) -- (1.84462,5.05542);
\draw [c] (1.84462,5.05542) -- (1.85944,5.05542);
\definecolor{c}{rgb}{0,0,0};
\colorlet{c}{kugray};
\draw [c] (1.87425,4.99299) -- (1.87425,5.02669);
\draw [c] (1.87425,5.02669) -- (1.87425,5.05705);
\draw [c] (1.85944,5.02669) -- (1.87425,5.02669);
\draw [c] (1.87425,5.02669) -- (1.88907,5.02669);
\definecolor{c}{rgb}{0,0,0};
\colorlet{c}{kugray};
\draw [c] (1.90389,4.94322) -- (1.90389,4.97626);
\draw [c] (1.90389,4.97626) -- (1.90389,5.00609);
\draw [c] (1.88907,4.97626) -- (1.90389,4.97626);
\draw [c] (1.90389,4.97626) -- (1.91871,4.97626);
\definecolor{c}{rgb}{0,0,0};
\colorlet{c}{kugray};
\draw [c] (1.93353,4.96266) -- (1.93353,4.9937);
\draw [c] (1.93353,4.9937) -- (1.93353,5.02189);
\draw [c] (1.91871,4.9937) -- (1.93353,4.9937);
\draw [c] (1.93353,4.9937) -- (1.94834,4.9937);
\definecolor{c}{rgb}{0,0,0};
\colorlet{c}{kugray};
\draw [c] (1.96316,4.86451) -- (1.96316,4.90192);
\draw [c] (1.96316,4.90192) -- (1.96316,4.93528);
\draw [c] (1.94834,4.90192) -- (1.96316,4.90192);
\draw [c] (1.96316,4.90192) -- (1.97798,4.90192);
\definecolor{c}{rgb}{0,0,0};
\colorlet{c}{kugray};
\draw [c] (1.9928,4.89806) -- (1.9928,4.93394);
\draw [c] (1.9928,4.93394) -- (1.9928,4.96608);
\draw [c] (1.97798,4.93394) -- (1.9928,4.93394);
\draw [c] (1.9928,4.93394) -- (2.00762,4.93394);
\definecolor{c}{rgb}{0,0,0};
\colorlet{c}{kugray};
\draw [c] (2.02243,4.68898) -- (2.02243,4.73645);
\draw [c] (2.02243,4.73645) -- (2.02243,4.77757);
\draw [c] (2.00762,4.73645) -- (2.02243,4.73645);
\draw [c] (2.02243,4.73645) -- (2.03725,4.73645);
\definecolor{c}{rgb}{0,0,0};
\colorlet{c}{kugray};
\draw [c] (2.05207,4.74386) -- (2.05207,4.78796);
\draw [c] (2.05207,4.78796) -- (2.05207,4.82652);
\draw [c] (2.03725,4.78796) -- (2.05207,4.78796);
\draw [c] (2.05207,4.78796) -- (2.06689,4.78796);
\definecolor{c}{rgb}{0,0,0};
\colorlet{c}{kugray};
\draw [c] (2.08171,4.73578) -- (2.08171,4.78026);
\draw [c] (2.08171,4.78026) -- (2.08171,4.81912);
\draw [c] (2.06689,4.78026) -- (2.08171,4.78026);
\draw [c] (2.08171,4.78026) -- (2.09652,4.78026);
\definecolor{c}{rgb}{0,0,0};
\colorlet{c}{kugray};
\draw [c] (2.11134,4.6862) -- (2.11134,4.73798);
\draw [c] (2.11134,4.73798) -- (2.11134,4.7823);
\draw [c] (2.09652,4.73798) -- (2.11134,4.73798);
\draw [c] (2.11134,4.73798) -- (2.12616,4.73798);
\definecolor{c}{rgb}{0,0,0};
\colorlet{c}{kugray};
\draw [c] (2.14098,4.54138) -- (2.14098,4.6098);
\draw [c] (2.14098,4.6098) -- (2.14098,4.66574);
\draw [c] (2.12616,4.6098) -- (2.14098,4.6098);
\draw [c] (2.14098,4.6098) -- (2.15579,4.6098);
\definecolor{c}{rgb}{0,0,0};
\colorlet{c}{kugray};
\draw [c] (2.17061,4.64226) -- (2.17061,4.69994);
\draw [c] (2.17061,4.69994) -- (2.17061,4.74849);
\draw [c] (2.15579,4.69994) -- (2.17061,4.69994);
\draw [c] (2.17061,4.69994) -- (2.18543,4.69994);
\definecolor{c}{rgb}{0,0,0};
\colorlet{c}{kugray};
\draw [c] (2.20025,4.53677) -- (2.20025,4.60067);
\draw [c] (2.20025,4.60067) -- (2.20025,4.65356);
\draw [c] (2.18543,4.60067) -- (2.20025,4.60067);
\draw [c] (2.20025,4.60067) -- (2.21507,4.60067);
\definecolor{c}{rgb}{0,0,0};
\colorlet{c}{kugray};
\draw [c] (2.22988,4.61221) -- (2.22988,4.66583);
\draw [c] (2.22988,4.66583) -- (2.22988,4.71148);
\draw [c] (2.21507,4.66583) -- (2.22988,4.66583);
\draw [c] (2.22988,4.66583) -- (2.2447,4.66583);
\definecolor{c}{rgb}{0,0,0};
\colorlet{c}{kugray};
\draw [c] (2.25952,4.53829) -- (2.25952,4.60341);
\draw [c] (2.25952,4.60341) -- (2.25952,4.65714);
\draw [c] (2.2447,4.60341) -- (2.25952,4.60341);
\draw [c] (2.25952,4.60341) -- (2.27434,4.60341);
\definecolor{c}{rgb}{0,0,0};
\colorlet{c}{kugray};
\draw [c] (2.28916,4.56974) -- (2.28916,4.58026);
\draw [c] (2.28916,4.58026) -- (2.28916,4.59044);
\draw [c] (2.27434,4.58026) -- (2.28916,4.58026);
\draw [c] (2.28916,4.58026) -- (2.30397,4.58026);
\definecolor{c}{rgb}{0,0,0};
\colorlet{c}{kugray};
\draw [c] (2.31879,4.53602) -- (2.31879,4.54723);
\draw [c] (2.31879,4.54723) -- (2.31879,4.55804);
\draw [c] (2.30397,4.54723) -- (2.31879,4.54723);
\draw [c] (2.31879,4.54723) -- (2.33361,4.54723);
\definecolor{c}{rgb}{0,0,0};
\colorlet{c}{kugray};
\draw [c] (2.34843,4.52541) -- (2.34843,4.53687);
\draw [c] (2.34843,4.53687) -- (2.34843,4.54792);
\draw [c] (2.33361,4.53687) -- (2.34843,4.53687);
\draw [c] (2.34843,4.53687) -- (2.36325,4.53687);
\definecolor{c}{rgb}{0,0,0};
\colorlet{c}{kugray};
\draw [c] (2.37806,4.47644) -- (2.37806,4.48842);
\draw [c] (2.37806,4.48842) -- (2.37806,4.49994);
\draw [c] (2.36325,4.48842) -- (2.37806,4.48842);
\draw [c] (2.37806,4.48842) -- (2.39288,4.48842);
\definecolor{c}{rgb}{0,0,0};
\colorlet{c}{kugray};
\draw [c] (2.4077,4.45724) -- (2.4077,4.46981);
\draw [c] (2.4077,4.46981) -- (2.4077,4.48188);
\draw [c] (2.39288,4.46981) -- (2.4077,4.46981);
\draw [c] (2.4077,4.46981) -- (2.42252,4.46981);
\definecolor{c}{rgb}{0,0,0};
\colorlet{c}{kugray};
\draw [c] (2.43733,4.43036) -- (2.43733,4.44342);
\draw [c] (2.43733,4.44342) -- (2.43733,4.45594);
\draw [c] (2.42252,4.44342) -- (2.43733,4.44342);
\draw [c] (2.43733,4.44342) -- (2.45215,4.44342);
\definecolor{c}{rgb}{0,0,0};
\colorlet{c}{kugray};
\draw [c] (2.46697,4.40509) -- (2.46697,4.41861);
\draw [c] (2.46697,4.41861) -- (2.46697,4.43156);
\draw [c] (2.45215,4.41861) -- (2.46697,4.41861);
\draw [c] (2.46697,4.41861) -- (2.48179,4.41861);
\definecolor{c}{rgb}{0,0,0};
\colorlet{c}{kugray};
\draw [c] (2.49661,4.37004) -- (2.49661,4.38467);
\draw [c] (2.49661,4.38467) -- (2.49661,4.39863);
\draw [c] (2.48179,4.38467) -- (2.49661,4.38467);
\draw [c] (2.49661,4.38467) -- (2.51142,4.38467);
\definecolor{c}{rgb}{0,0,0};
\colorlet{c}{kugray};
\draw [c] (2.52624,4.32084) -- (2.52624,4.33659);
\draw [c] (2.52624,4.33659) -- (2.52624,4.35157);
\draw [c] (2.51142,4.33659) -- (2.52624,4.33659);
\draw [c] (2.52624,4.33659) -- (2.54106,4.33659);
\definecolor{c}{rgb}{0,0,0};
\colorlet{c}{kugray};
\draw [c] (2.55588,4.30399) -- (2.55588,4.32008);
\draw [c] (2.55588,4.32008) -- (2.55588,4.33538);
\draw [c] (2.54106,4.32008) -- (2.55588,4.32008);
\draw [c] (2.55588,4.32008) -- (2.5707,4.32008);
\definecolor{c}{rgb}{0,0,0};
\colorlet{c}{kugray};
\draw [c] (2.58551,4.30302) -- (2.58551,4.31916);
\draw [c] (2.58551,4.31916) -- (2.58551,4.3345);
\draw [c] (2.5707,4.31916) -- (2.58551,4.31916);
\draw [c] (2.58551,4.31916) -- (2.60033,4.31916);
\definecolor{c}{rgb}{0,0,0};
\colorlet{c}{kugray};
\draw [c] (2.61515,4.25843) -- (2.61515,4.27581);
\draw [c] (2.61515,4.27581) -- (2.61515,4.29226);
\draw [c] (2.60033,4.27581) -- (2.61515,4.27581);
\draw [c] (2.61515,4.27581) -- (2.62997,4.27581);
\definecolor{c}{rgb}{0,0,0};
\colorlet{c}{kugray};
\draw [c] (2.64478,4.24047) -- (2.64478,4.25814);
\draw [c] (2.64478,4.25814) -- (2.64478,4.27485);
\draw [c] (2.62997,4.25814) -- (2.64478,4.25814);
\draw [c] (2.64478,4.25814) -- (2.6596,4.25814);
\definecolor{c}{rgb}{0,0,0};
\colorlet{c}{kugray};
\draw [c] (2.67442,4.23829) -- (2.67442,4.2564);
\draw [c] (2.67442,4.2564) -- (2.67442,4.27351);
\draw [c] (2.6596,4.2564) -- (2.67442,4.2564);
\draw [c] (2.67442,4.2564) -- (2.68924,4.2564);
\definecolor{c}{rgb}{0,0,0};
\colorlet{c}{kugray};
\draw [c] (2.70406,4.23694) -- (2.70406,4.25493);
\draw [c] (2.70406,4.25493) -- (2.70406,4.27193);
\draw [c] (2.68924,4.25493) -- (2.70406,4.25493);
\draw [c] (2.70406,4.25493) -- (2.71887,4.25493);
\definecolor{c}{rgb}{0,0,0};
\colorlet{c}{kugray};
\draw [c] (2.73369,4.17951) -- (2.73369,4.19937);
\draw [c] (2.73369,4.19937) -- (2.73369,4.21803);
\draw [c] (2.71887,4.19937) -- (2.73369,4.19937);
\draw [c] (2.73369,4.19937) -- (2.74851,4.19937);
\definecolor{c}{rgb}{0,0,0};
\colorlet{c}{kugray};
\draw [c] (2.76333,4.15517) -- (2.76333,4.17566);
\draw [c] (2.76333,4.17566) -- (2.76333,4.19487);
\draw [c] (2.74851,4.17566) -- (2.76333,4.17566);
\draw [c] (2.76333,4.17566) -- (2.77815,4.17566);
\definecolor{c}{rgb}{0,0,0};
\colorlet{c}{kugray};
\draw [c] (2.79296,4.13183) -- (2.79296,4.15328);
\draw [c] (2.79296,4.15328) -- (2.79296,4.17334);
\draw [c] (2.77815,4.15328) -- (2.79296,4.15328);
\draw [c] (2.79296,4.15328) -- (2.80778,4.15328);
\definecolor{c}{rgb}{0,0,0};
\colorlet{c}{kugray};
\draw [c] (2.8226,4.09575) -- (2.8226,4.11831);
\draw [c] (2.8226,4.11831) -- (2.8226,4.13932);
\draw [c] (2.80778,4.11831) -- (2.8226,4.11831);
\draw [c] (2.8226,4.11831) -- (2.83742,4.11831);
\definecolor{c}{rgb}{0,0,0};
\colorlet{c}{kugray};
\draw [c] (2.85224,4.0827) -- (2.85224,4.10635);
\draw [c] (2.85224,4.10635) -- (2.85224,4.12832);
\draw [c] (2.83742,4.10635) -- (2.85224,4.10635);
\draw [c] (2.85224,4.10635) -- (2.86705,4.10635);
\definecolor{c}{rgb}{0,0,0};
\colorlet{c}{kugray};
\draw [c] (2.88187,4.07789) -- (2.88187,4.1009);
\draw [c] (2.88187,4.1009) -- (2.88187,4.12231);
\draw [c] (2.86705,4.1009) -- (2.88187,4.1009);
\draw [c] (2.88187,4.1009) -- (2.89669,4.1009);
\definecolor{c}{rgb}{0,0,0};
\colorlet{c}{kugray};
\draw [c] (2.91151,4.04064) -- (2.91151,4.06493);
\draw [c] (2.91151,4.06493) -- (2.91151,4.08745);
\draw [c] (2.89669,4.06493) -- (2.91151,4.06493);
\draw [c] (2.91151,4.06493) -- (2.92632,4.06493);
\definecolor{c}{rgb}{0,0,0};
\colorlet{c}{kugray};
\draw [c] (2.94114,3.99628) -- (2.94114,4.02251);
\draw [c] (2.94114,4.02251) -- (2.94114,4.04669);
\draw [c] (2.92632,4.02251) -- (2.94114,4.02251);
\draw [c] (2.94114,4.02251) -- (2.95596,4.02251);
\definecolor{c}{rgb}{0,0,0};
\colorlet{c}{kugray};
\draw [c] (2.97078,4.01184) -- (2.97078,4.0375);
\draw [c] (2.97078,4.0375) -- (2.97078,4.06119);
\draw [c] (2.95596,4.0375) -- (2.97078,4.0375);
\draw [c] (2.97078,4.0375) -- (2.9856,4.0375);
\definecolor{c}{rgb}{0,0,0};
\colorlet{c}{kugray};
\draw [c] (3.00041,3.96161) -- (3.00041,3.99102);
\draw [c] (3.00041,3.99102) -- (3.00041,4.01787);
\draw [c] (2.9856,3.99102) -- (3.00041,3.99102);
\draw [c] (3.00041,3.99102) -- (3.01523,3.99102);
\definecolor{c}{rgb}{0,0,0};
\colorlet{c}{kugray};
\draw [c] (3.03005,3.96889) -- (3.03005,3.99707);
\draw [c] (3.03005,3.99707) -- (3.03005,4.02289);
\draw [c] (3.01523,3.99707) -- (3.03005,3.99707);
\draw [c] (3.03005,3.99707) -- (3.04487,3.99707);
\definecolor{c}{rgb}{0,0,0};
\colorlet{c}{kugray};
\draw [c] (3.05969,3.98359) -- (3.05969,4.01138);
\draw [c] (3.05969,4.01138) -- (3.05969,4.03686);
\draw [c] (3.04487,4.01138) -- (3.05969,4.01138);
\draw [c] (3.05969,4.01138) -- (3.0745,4.01138);
\definecolor{c}{rgb}{0,0,0};
\colorlet{c}{kugray};
\draw [c] (3.08932,3.94149) -- (3.08932,3.97026);
\draw [c] (3.08932,3.97026) -- (3.08932,3.99657);
\draw [c] (3.0745,3.97026) -- (3.08932,3.97026);
\draw [c] (3.08932,3.97026) -- (3.10414,3.97026);
\definecolor{c}{rgb}{0,0,0};
\colorlet{c}{kugray};
\draw [c] (3.11896,3.91737) -- (3.11896,3.94858);
\draw [c] (3.11896,3.94858) -- (3.11896,3.97692);
\draw [c] (3.10414,3.94858) -- (3.11896,3.94858);
\draw [c] (3.11896,3.94858) -- (3.13377,3.94858);
\definecolor{c}{rgb}{0,0,0};
\colorlet{c}{kugray};
\draw [c] (3.14859,3.9151) -- (3.14859,3.945);
\draw [c] (3.14859,3.945) -- (3.14859,3.97226);
\draw [c] (3.13377,3.945) -- (3.14859,3.945);
\draw [c] (3.14859,3.945) -- (3.16341,3.945);
\definecolor{c}{rgb}{0,0,0};
\colorlet{c}{kugray};
\draw [c] (3.17823,3.85367) -- (3.17823,3.88787);
\draw [c] (3.17823,3.88787) -- (3.17823,3.91865);
\draw [c] (3.16341,3.88787) -- (3.17823,3.88787);
\draw [c] (3.17823,3.88787) -- (3.19305,3.88787);
\definecolor{c}{rgb}{0,0,0};
\colorlet{c}{kugray};
\draw [c] (3.20786,3.83015) -- (3.20786,3.86574);
\draw [c] (3.20786,3.86574) -- (3.20786,3.89765);
\draw [c] (3.19305,3.86574) -- (3.20786,3.86574);
\draw [c] (3.20786,3.86574) -- (3.22268,3.86574);
\definecolor{c}{rgb}{0,0,0};
\colorlet{c}{kugray};
\draw [c] (3.2375,3.83715) -- (3.2375,3.87247);
\draw [c] (3.2375,3.87247) -- (3.2375,3.90415);
\draw [c] (3.22268,3.87247) -- (3.2375,3.87247);
\draw [c] (3.2375,3.87247) -- (3.25232,3.87247);
\definecolor{c}{rgb}{0,0,0};
\colorlet{c}{kugray};
\draw [c] (3.26714,3.73623) -- (3.26714,3.77648);
\draw [c] (3.26714,3.77648) -- (3.26714,3.81206);
\draw [c] (3.25232,3.77648) -- (3.26714,3.77648);
\draw [c] (3.26714,3.77648) -- (3.28195,3.77648);
\definecolor{c}{rgb}{0,0,0};
\colorlet{c}{kugray};
\draw [c] (3.29677,3.80384) -- (3.29677,3.84266);
\draw [c] (3.29677,3.84266) -- (3.29677,3.87713);
\draw [c] (3.28195,3.84266) -- (3.29677,3.84266);
\draw [c] (3.29677,3.84266) -- (3.31159,3.84266);
\definecolor{c}{rgb}{0,0,0};
\colorlet{c}{kugray};
\draw [c] (3.32641,3.76887) -- (3.32641,3.80715);
\draw [c] (3.32641,3.80715) -- (3.32641,3.84119);
\draw [c] (3.31159,3.80715) -- (3.32641,3.80715);
\draw [c] (3.32641,3.80715) -- (3.34123,3.80715);
\definecolor{c}{rgb}{0,0,0};
\colorlet{c}{kugray};
\draw [c] (3.35604,3.75484) -- (3.35604,3.79665);
\draw [c] (3.35604,3.79665) -- (3.35604,3.83345);
\draw [c] (3.34123,3.79665) -- (3.35604,3.79665);
\draw [c] (3.35604,3.79665) -- (3.37086,3.79665);
\definecolor{c}{rgb}{0,0,0};
\colorlet{c}{kugray};
\draw [c] (3.38568,3.74591) -- (3.38568,3.7847);
\draw [c] (3.38568,3.7847) -- (3.38568,3.81915);
\draw [c] (3.37086,3.7847) -- (3.38568,3.7847);
\draw [c] (3.38568,3.7847) -- (3.4005,3.7847);
\definecolor{c}{rgb}{0,0,0};
\colorlet{c}{kugray};
\draw [c] (3.41531,3.79899) -- (3.41531,3.83877);
\draw [c] (3.41531,3.83877) -- (3.41531,3.874);
\draw [c] (3.4005,3.83877) -- (3.41531,3.83877);
\draw [c] (3.41531,3.83877) -- (3.43013,3.83877);
\definecolor{c}{rgb}{0,0,0};
\colorlet{c}{kugray};
\draw [c] (3.44495,3.68988) -- (3.44495,3.73418);
\draw [c] (3.44495,3.73418) -- (3.44495,3.7729);
\draw [c] (3.43013,3.73418) -- (3.44495,3.73418);
\draw [c] (3.44495,3.73418) -- (3.45977,3.73418);
\definecolor{c}{rgb}{0,0,0};
\colorlet{c}{kugray};
\draw [c] (3.47459,3.63044) -- (3.47459,3.67993);
\draw [c] (3.47459,3.67993) -- (3.47459,3.72255);
\draw [c] (3.45977,3.67993) -- (3.47459,3.67993);
\draw [c] (3.47459,3.67993) -- (3.4894,3.67993);
\definecolor{c}{rgb}{0,0,0};
\colorlet{c}{kugray};
\draw [c] (3.50422,3.78508) -- (3.50422,3.82542);
\draw [c] (3.50422,3.82542) -- (3.50422,3.86107);
\draw [c] (3.4894,3.82542) -- (3.50422,3.82542);
\draw [c] (3.50422,3.82542) -- (3.51904,3.82542);
\definecolor{c}{rgb}{0,0,0};
\colorlet{c}{kugray};
\draw [c] (3.53386,3.72102) -- (3.53386,3.76148);
\draw [c] (3.53386,3.76148) -- (3.53386,3.79724);
\draw [c] (3.51904,3.76148) -- (3.53386,3.76148);
\draw [c] (3.53386,3.76148) -- (3.54868,3.76148);
\definecolor{c}{rgb}{0,0,0};
\colorlet{c}{kugray};
\draw [c] (3.56349,3.66874) -- (3.56349,3.71592);
\draw [c] (3.56349,3.71592) -- (3.56349,3.75683);
\draw [c] (3.54868,3.71592) -- (3.56349,3.71592);
\draw [c] (3.56349,3.71592) -- (3.57831,3.71592);
\definecolor{c}{rgb}{0,0,0};
\colorlet{c}{kugray};
\draw [c] (3.59313,3.62349) -- (3.59313,3.66941);
\draw [c] (3.59313,3.66941) -- (3.59313,3.70935);
\draw [c] (3.57831,3.66941) -- (3.59313,3.66941);
\draw [c] (3.59313,3.66941) -- (3.60795,3.66941);
\definecolor{c}{rgb}{0,0,0};
\colorlet{c}{kugray};
\draw [c] (3.62276,3.72066) -- (3.62276,3.7665);
\draw [c] (3.62276,3.7665) -- (3.62276,3.80639);
\draw [c] (3.60795,3.7665) -- (3.62276,3.7665);
\draw [c] (3.62276,3.7665) -- (3.63758,3.7665);
\definecolor{c}{rgb}{0,0,0};
\colorlet{c}{kugray};
\draw [c] (3.6524,3.60108) -- (3.6524,3.65117);
\draw [c] (3.6524,3.65117) -- (3.6524,3.69424);
\draw [c] (3.63758,3.65117) -- (3.6524,3.65117);
\draw [c] (3.6524,3.65117) -- (3.66722,3.65117);
\definecolor{c}{rgb}{0,0,0};
\colorlet{c}{kugray};
\draw [c] (3.68204,3.59622) -- (3.68204,3.64503);
\draw [c] (3.68204,3.64503) -- (3.68204,3.68715);
\draw [c] (3.66722,3.64503) -- (3.68204,3.64503);
\draw [c] (3.68204,3.64503) -- (3.69685,3.64503);
\definecolor{c}{rgb}{0,0,0};
\colorlet{c}{kugray};
\draw [c] (3.71167,3.61286) -- (3.71167,3.66979);
\draw [c] (3.71167,3.66979) -- (3.71167,3.71782);
\draw [c] (3.69685,3.66979) -- (3.71167,3.66979);
\draw [c] (3.71167,3.66979) -- (3.72649,3.66979);
\definecolor{c}{rgb}{0,0,0};
\colorlet{c}{kugray};
\draw [c] (3.74131,3.57945) -- (3.74131,3.62832);
\draw [c] (3.74131,3.62832) -- (3.74131,3.67048);
\draw [c] (3.72649,3.62832) -- (3.74131,3.62832);
\draw [c] (3.74131,3.62832) -- (3.75613,3.62832);
\definecolor{c}{rgb}{0,0,0};
\colorlet{c}{kugray};
\draw [c] (3.77094,3.5864) -- (3.77094,3.6364);
\draw [c] (3.77094,3.6364) -- (3.77094,3.67939);
\draw [c] (3.75613,3.6364) -- (3.77094,3.6364);
\draw [c] (3.77094,3.6364) -- (3.78576,3.6364);
\definecolor{c}{rgb}{0,0,0};
\colorlet{c}{kugray};
\draw [c] (3.80058,3.58352) -- (3.80058,3.63583);
\draw [c] (3.80058,3.63583) -- (3.80058,3.68052);
\draw [c] (3.78576,3.63583) -- (3.80058,3.63583);
\draw [c] (3.80058,3.63583) -- (3.8154,3.63583);
\definecolor{c}{rgb}{0,0,0};
\colorlet{c}{kugray};
\draw [c] (3.83022,3.54338) -- (3.83022,3.60015);
\draw [c] (3.83022,3.60015) -- (3.83022,3.64806);
\draw [c] (3.8154,3.60015) -- (3.83022,3.60015);
\draw [c] (3.83022,3.60015) -- (3.84503,3.60015);
\definecolor{c}{rgb}{0,0,0};
\colorlet{c}{kugray};
\draw [c] (3.85985,3.54491) -- (3.85985,3.59885);
\draw [c] (3.85985,3.59885) -- (3.85985,3.64473);
\draw [c] (3.84503,3.59885) -- (3.85985,3.59885);
\draw [c] (3.85985,3.59885) -- (3.87467,3.59885);
\definecolor{c}{rgb}{0,0,0};
\colorlet{c}{kugray};
\draw [c] (3.88949,3.51526) -- (3.88949,3.57482);
\draw [c] (3.88949,3.57482) -- (3.88949,3.62471);
\draw [c] (3.87467,3.57482) -- (3.88949,3.57482);
\draw [c] (3.88949,3.57482) -- (3.9043,3.57482);
\definecolor{c}{rgb}{0,0,0};
\colorlet{c}{kugray};
\draw [c] (3.91912,3.60926) -- (3.91912,3.65245);
\draw [c] (3.91912,3.65245) -- (3.91912,3.69032);
\draw [c] (3.9043,3.65245) -- (3.91912,3.65245);
\draw [c] (3.91912,3.65245) -- (3.93394,3.65245);
\definecolor{c}{rgb}{0,0,0};
\colorlet{c}{kugray};
\draw [c] (3.94876,3.62613) -- (3.94876,3.66623);
\draw [c] (3.94876,3.66623) -- (3.94876,3.7017);
\draw [c] (3.93394,3.66623) -- (3.94876,3.66623);
\draw [c] (3.94876,3.66623) -- (3.96358,3.66623);
\definecolor{c}{rgb}{0,0,0};
\colorlet{c}{kugray};
\draw [c] (3.97839,3.57716) -- (3.97839,3.60885);
\draw [c] (3.97839,3.60885) -- (3.97839,3.63758);
\draw [c] (3.96358,3.60885) -- (3.97839,3.60885);
\draw [c] (3.97839,3.60885) -- (3.99321,3.60885);
\definecolor{c}{rgb}{0,0,0};
\colorlet{c}{kugray};
\draw [c] (4.00803,3.51864) -- (4.00803,3.54777);
\draw [c] (4.00803,3.54777) -- (4.00803,3.57438);
\draw [c] (3.99321,3.54777) -- (4.00803,3.54777);
\draw [c] (4.00803,3.54777) -- (4.02285,3.54777);
\definecolor{c}{rgb}{0,0,0};
\colorlet{c}{kugray};
\draw [c] (4.03767,3.54945) -- (4.03767,3.5713);
\draw [c] (4.03767,3.5713) -- (4.03767,3.59171);
\draw [c] (4.02285,3.5713) -- (4.03767,3.5713);
\draw [c] (4.03767,3.5713) -- (4.05248,3.5713);
\definecolor{c}{rgb}{0,0,0};
\colorlet{c}{kugray};
\draw [c] (4.0673,3.52301) -- (4.0673,3.54604);
\draw [c] (4.0673,3.54604) -- (4.0673,3.56746);
\draw [c] (4.05248,3.54604) -- (4.0673,3.54604);
\draw [c] (4.0673,3.54604) -- (4.08212,3.54604);
\definecolor{c}{rgb}{0,0,0};
\colorlet{c}{kugray};
\draw [c] (4.09694,3.48934) -- (4.09694,3.51203);
\draw [c] (4.09694,3.51203) -- (4.09694,3.53317);
\draw [c] (4.08212,3.51203) -- (4.09694,3.51203);
\draw [c] (4.09694,3.51203) -- (4.11175,3.51203);
\definecolor{c}{rgb}{0,0,0};
\colorlet{c}{kugray};
\draw [c] (4.12657,3.50377) -- (4.12657,3.52636);
\draw [c] (4.12657,3.52636) -- (4.12657,3.5474);
\draw [c] (4.11175,3.52636) -- (4.12657,3.52636);
\draw [c] (4.12657,3.52636) -- (4.14139,3.52636);
\definecolor{c}{rgb}{0,0,0};
\colorlet{c}{kugray};
\draw [c] (4.15621,3.51321) -- (4.15621,3.53483);
\draw [c] (4.15621,3.53483) -- (4.15621,3.55503);
\draw [c] (4.14139,3.53483) -- (4.15621,3.53483);
\draw [c] (4.15621,3.53483) -- (4.17103,3.53483);
\definecolor{c}{rgb}{0,0,0};
\colorlet{c}{kugray};
\draw [c] (4.18584,3.5001) -- (4.18584,3.52281);
\draw [c] (4.18584,3.52281) -- (4.18584,3.54395);
\draw [c] (4.17103,3.52281) -- (4.18584,3.52281);
\draw [c] (4.18584,3.52281) -- (4.20066,3.52281);
\definecolor{c}{rgb}{0,0,0};
\colorlet{c}{kugray};
\draw [c] (4.21548,3.46277) -- (4.21548,3.48687);
\draw [c] (4.21548,3.48687) -- (4.21548,3.50921);
\draw [c] (4.20066,3.48687) -- (4.21548,3.48687);
\draw [c] (4.21548,3.48687) -- (4.2303,3.48687);
\definecolor{c}{rgb}{0,0,0};
\colorlet{c}{kugray};
\draw [c] (4.24512,3.50438) -- (4.24512,3.52639);
\draw [c] (4.24512,3.52639) -- (4.24512,3.54694);
\draw [c] (4.2303,3.52639) -- (4.24512,3.52639);
\draw [c] (4.24512,3.52639) -- (4.25993,3.52639);
\definecolor{c}{rgb}{0,0,0};
\colorlet{c}{kugray};
\draw [c] (4.27475,3.47759) -- (4.27475,3.50131);
\draw [c] (4.27475,3.50131) -- (4.27475,3.52334);
\draw [c] (4.25993,3.50131) -- (4.27475,3.50131);
\draw [c] (4.27475,3.50131) -- (4.28957,3.50131);
\definecolor{c}{rgb}{0,0,0};
\colorlet{c}{kugray};
\draw [c] (4.30439,3.4529) -- (4.30439,3.47803);
\draw [c] (4.30439,3.47803) -- (4.30439,3.50126);
\draw [c] (4.28957,3.47803) -- (4.30439,3.47803);
\draw [c] (4.30439,3.47803) -- (4.31921,3.47803);
\definecolor{c}{rgb}{0,0,0};
\colorlet{c}{kugray};
\draw [c] (4.33402,3.41852) -- (4.33402,3.44434);
\draw [c] (4.33402,3.44434) -- (4.33402,3.46816);
\draw [c] (4.31921,3.44434) -- (4.33402,3.44434);
\draw [c] (4.33402,3.44434) -- (4.34884,3.44434);
\definecolor{c}{rgb}{0,0,0};
\colorlet{c}{kugray};
\draw [c] (4.36366,3.41395) -- (4.36366,3.4399);
\draw [c] (4.36366,3.4399) -- (4.36366,3.46384);
\draw [c] (4.34884,3.4399) -- (4.36366,3.4399);
\draw [c] (4.36366,3.4399) -- (4.37848,3.4399);
\definecolor{c}{rgb}{0,0,0};
\colorlet{c}{kugray};
\draw [c] (4.39329,3.46193) -- (4.39329,3.48588);
\draw [c] (4.39329,3.48588) -- (4.39329,3.5081);
\draw [c] (4.37848,3.48588) -- (4.39329,3.48588);
\draw [c] (4.39329,3.48588) -- (4.40811,3.48588);
\definecolor{c}{rgb}{0,0,0};
\colorlet{c}{kugray};
\draw [c] (4.42293,3.43862) -- (4.42293,3.46556);
\draw [c] (4.42293,3.46556) -- (4.42293,3.49033);
\draw [c] (4.40811,3.46556) -- (4.42293,3.46556);
\draw [c] (4.42293,3.46556) -- (4.43775,3.46556);
\definecolor{c}{rgb}{0,0,0};
\colorlet{c}{kugray};
\draw [c] (4.45257,3.40413) -- (4.45257,3.43062);
\draw [c] (4.45257,3.43062) -- (4.45257,3.45502);
\draw [c] (4.43775,3.43062) -- (4.45257,3.43062);
\draw [c] (4.45257,3.43062) -- (4.46738,3.43062);
\definecolor{c}{rgb}{0,0,0};
\colorlet{c}{kugray};
\draw [c] (4.4822,3.45965) -- (4.4822,3.48371);
\draw [c] (4.4822,3.48371) -- (4.4822,3.50602);
\draw [c] (4.46738,3.48371) -- (4.4822,3.48371);
\draw [c] (4.4822,3.48371) -- (4.49702,3.48371);
\definecolor{c}{rgb}{0,0,0};
\colorlet{c}{kugray};
\draw [c] (4.51184,3.43161) -- (4.51184,3.45645);
\draw [c] (4.51184,3.45645) -- (4.51184,3.47944);
\draw [c] (4.49702,3.45645) -- (4.51184,3.45645);
\draw [c] (4.51184,3.45645) -- (4.52666,3.45645);
\definecolor{c}{rgb}{0,0,0};
\colorlet{c}{kugray};
\draw [c] (4.54147,3.41238) -- (4.54147,3.43844);
\draw [c] (4.54147,3.43844) -- (4.54147,3.46247);
\draw [c] (4.52666,3.43844) -- (4.54147,3.43844);
\draw [c] (4.54147,3.43844) -- (4.55629,3.43844);
\definecolor{c}{rgb}{0,0,0};
\colorlet{c}{kugray};
\draw [c] (4.57111,3.40389) -- (4.57111,3.43019);
\draw [c] (4.57111,3.43019) -- (4.57111,3.45442);
\draw [c] (4.55629,3.43019) -- (4.57111,3.43019);
\draw [c] (4.57111,3.43019) -- (4.58593,3.43019);
\definecolor{c}{rgb}{0,0,0};
\colorlet{c}{kugray};
\draw [c] (4.60075,3.40524) -- (4.60075,3.43191);
\draw [c] (4.60075,3.43191) -- (4.60075,3.45646);
\draw [c] (4.58593,3.43191) -- (4.60075,3.43191);
\draw [c] (4.60075,3.43191) -- (4.61556,3.43191);
\definecolor{c}{rgb}{0,0,0};
\colorlet{c}{kugray};
\draw [c] (4.63038,3.40284) -- (4.63038,3.42898);
\draw [c] (4.63038,3.42898) -- (4.63038,3.45308);
\draw [c] (4.61556,3.42898) -- (4.63038,3.42898);
\draw [c] (4.63038,3.42898) -- (4.6452,3.42898);
\definecolor{c}{rgb}{0,0,0};
\colorlet{c}{kugray};
\draw [c] (4.66002,3.40215) -- (4.66002,3.42856);
\draw [c] (4.66002,3.42856) -- (4.66002,3.45289);
\draw [c] (4.6452,3.42856) -- (4.66002,3.42856);
\draw [c] (4.66002,3.42856) -- (4.67483,3.42856);
\definecolor{c}{rgb}{0,0,0};
\colorlet{c}{kugray};
\draw [c] (4.68965,3.33528) -- (4.68965,3.36407);
\draw [c] (4.68965,3.36407) -- (4.68965,3.3904);
\draw [c] (4.67483,3.36407) -- (4.68965,3.36407);
\draw [c] (4.68965,3.36407) -- (4.70447,3.36407);
\definecolor{c}{rgb}{0,0,0};
\colorlet{c}{kugray};
\draw [c] (4.71929,3.39699) -- (4.71929,3.42459);
\draw [c] (4.71929,3.42459) -- (4.71929,3.44992);
\draw [c] (4.70447,3.42459) -- (4.71929,3.42459);
\draw [c] (4.71929,3.42459) -- (4.73411,3.42459);
\definecolor{c}{rgb}{0,0,0};
\colorlet{c}{kugray};
\draw [c] (4.74892,3.37826) -- (4.74892,3.40524);
\draw [c] (4.74892,3.40524) -- (4.74892,3.43005);
\draw [c] (4.73411,3.40524) -- (4.74892,3.40524);
\draw [c] (4.74892,3.40524) -- (4.76374,3.40524);
\definecolor{c}{rgb}{0,0,0};
\colorlet{c}{kugray};
\draw [c] (4.77856,3.42016) -- (4.77856,3.44594);
\draw [c] (4.77856,3.44594) -- (4.77856,3.46973);
\draw [c] (4.76374,3.44594) -- (4.77856,3.44594);
\draw [c] (4.77856,3.44594) -- (4.79338,3.44594);
\definecolor{c}{rgb}{0,0,0};
\colorlet{c}{kugray};
\draw [c] (4.8082,3.38648) -- (4.8082,3.41303);
\draw [c] (4.8082,3.41303) -- (4.8082,3.43746);
\draw [c] (4.79338,3.41303) -- (4.8082,3.41303);
\draw [c] (4.8082,3.41303) -- (4.82301,3.41303);
\definecolor{c}{rgb}{0,0,0};
\colorlet{c}{kugray};
\draw [c] (4.83783,3.39861) -- (4.83783,3.42469);
\draw [c] (4.83783,3.42469) -- (4.83783,3.44873);
\draw [c] (4.82301,3.42469) -- (4.83783,3.42469);
\draw [c] (4.83783,3.42469) -- (4.85265,3.42469);
\definecolor{c}{rgb}{0,0,0};
\colorlet{c}{kugray};
\draw [c] (4.86747,3.30875) -- (4.86747,3.33911);
\draw [c] (4.86747,3.33911) -- (4.86747,3.36675);
\draw [c] (4.85265,3.33911) -- (4.86747,3.33911);
\draw [c] (4.86747,3.33911) -- (4.88228,3.33911);
\definecolor{c}{rgb}{0,0,0};
\colorlet{c}{kugray};
\draw [c] (4.8971,3.35508) -- (4.8971,3.38333);
\draw [c] (4.8971,3.38333) -- (4.8971,3.4092);
\draw [c] (4.88228,3.38333) -- (4.8971,3.38333);
\draw [c] (4.8971,3.38333) -- (4.91192,3.38333);
\definecolor{c}{rgb}{0,0,0};
\colorlet{c}{kugray};
\draw [c] (4.92674,3.30415) -- (4.92674,3.33311);
\draw [c] (4.92674,3.33311) -- (4.92674,3.35958);
\draw [c] (4.91192,3.33311) -- (4.92674,3.33311);
\draw [c] (4.92674,3.33311) -- (4.94156,3.33311);
\definecolor{c}{rgb}{0,0,0};
\colorlet{c}{kugray};
\draw [c] (4.95637,3.33255) -- (4.95637,3.36143);
\draw [c] (4.95637,3.36143) -- (4.95637,3.38782);
\draw [c] (4.94156,3.36143) -- (4.95637,3.36143);
\draw [c] (4.95637,3.36143) -- (4.97119,3.36143);
\definecolor{c}{rgb}{0,0,0};
\colorlet{c}{kugray};
\draw [c] (4.98601,3.37358) -- (4.98601,3.40123);
\draw [c] (4.98601,3.40123) -- (4.98601,3.42661);
\draw [c] (4.97119,3.40123) -- (4.98601,3.40123);
\draw [c] (4.98601,3.40123) -- (5.00083,3.40123);
\definecolor{c}{rgb}{0,0,0};
\colorlet{c}{kugray};
\draw [c] (5.01565,3.35642) -- (5.01565,3.38431);
\draw [c] (5.01565,3.38431) -- (5.01565,3.40989);
\draw [c] (5.00083,3.38431) -- (5.01565,3.38431);
\draw [c] (5.01565,3.38431) -- (5.03046,3.38431);
\definecolor{c}{rgb}{0,0,0};
\colorlet{c}{kugray};
\draw [c] (5.04528,3.31958) -- (5.04528,3.34899);
\draw [c] (5.04528,3.34899) -- (5.04528,3.37583);
\draw [c] (5.03046,3.34899) -- (5.04528,3.34899);
\draw [c] (5.04528,3.34899) -- (5.0601,3.34899);
\definecolor{c}{rgb}{0,0,0};
\colorlet{c}{kugray};
\draw [c] (5.07492,3.3449) -- (5.07492,3.37413);
\draw [c] (5.07492,3.37413) -- (5.07492,3.40082);
\draw [c] (5.0601,3.37413) -- (5.07492,3.37413);
\draw [c] (5.07492,3.37413) -- (5.08974,3.37413);
\definecolor{c}{rgb}{0,0,0};
\colorlet{c}{kugray};
\draw [c] (5.10455,3.35313) -- (5.10455,3.38196);
\draw [c] (5.10455,3.38196) -- (5.10455,3.40832);
\draw [c] (5.08974,3.38196) -- (5.10455,3.38196);
\draw [c] (5.10455,3.38196) -- (5.11937,3.38196);
\definecolor{c}{rgb}{0,0,0};
\colorlet{c}{kugray};
\draw [c] (5.13419,3.35694) -- (5.13419,3.38661);
\draw [c] (5.13419,3.38661) -- (5.13419,3.41367);
\draw [c] (5.11937,3.38661) -- (5.13419,3.38661);
\draw [c] (5.13419,3.38661) -- (5.14901,3.38661);
\definecolor{c}{rgb}{0,0,0};
\colorlet{c}{kugray};
\draw [c] (5.16382,3.34144) -- (5.16382,3.36991);
\draw [c] (5.16382,3.36991) -- (5.16382,3.39596);
\draw [c] (5.14901,3.36991) -- (5.16382,3.36991);
\draw [c] (5.16382,3.36991) -- (5.17864,3.36991);
\definecolor{c}{rgb}{0,0,0};
\colorlet{c}{kugray};
\draw [c] (5.19346,3.37738) -- (5.19346,3.40395);
\draw [c] (5.19346,3.40395) -- (5.19346,3.42841);
\draw [c] (5.17864,3.40395) -- (5.19346,3.40395);
\draw [c] (5.19346,3.40395) -- (5.20828,3.40395);
\definecolor{c}{rgb}{0,0,0};
\colorlet{c}{kugray};
\draw [c] (5.2231,3.33251) -- (5.2231,3.36208);
\draw [c] (5.2231,3.36208) -- (5.2231,3.38905);
\draw [c] (5.20828,3.36208) -- (5.2231,3.36208);
\draw [c] (5.2231,3.36208) -- (5.23791,3.36208);
\definecolor{c}{rgb}{0,0,0};
\colorlet{c}{kugray};
\draw [c] (5.25273,3.32212) -- (5.25273,3.3512);
\draw [c] (5.25273,3.3512) -- (5.25273,3.37777);
\draw [c] (5.23791,3.3512) -- (5.25273,3.3512);
\draw [c] (5.25273,3.3512) -- (5.26755,3.3512);
\definecolor{c}{rgb}{0,0,0};
\colorlet{c}{kugray};
\draw [c] (5.28237,3.37237) -- (5.28237,3.40019);
\draw [c] (5.28237,3.40019) -- (5.28237,3.42569);
\draw [c] (5.26755,3.40019) -- (5.28237,3.40019);
\draw [c] (5.28237,3.40019) -- (5.29719,3.40019);
\definecolor{c}{rgb}{0,0,0};
\colorlet{c}{kugray};
\draw [c] (5.312,3.32807) -- (5.312,3.3584);
\draw [c] (5.312,3.3584) -- (5.312,3.38601);
\draw [c] (5.29719,3.3584) -- (5.312,3.3584);
\draw [c] (5.312,3.3584) -- (5.32682,3.3584);
\definecolor{c}{rgb}{0,0,0};
\colorlet{c}{kugray};
\draw [c] (5.34164,3.30394) -- (5.34164,3.33512);
\draw [c] (5.34164,3.33512) -- (5.34164,3.36344);
\draw [c] (5.32682,3.33512) -- (5.34164,3.33512);
\draw [c] (5.34164,3.33512) -- (5.35646,3.33512);
\definecolor{c}{rgb}{0,0,0};
\colorlet{c}{kugray};
\draw [c] (5.37127,3.27081) -- (5.37127,3.30238);
\draw [c] (5.37127,3.30238) -- (5.37127,3.33101);
\draw [c] (5.35646,3.30238) -- (5.37127,3.30238);
\draw [c] (5.37127,3.30238) -- (5.38609,3.30238);
\definecolor{c}{rgb}{0,0,0};
\colorlet{c}{kugray};
\draw [c] (5.40091,3.34778) -- (5.40091,3.37729);
\draw [c] (5.40091,3.37729) -- (5.40091,3.40421);
\draw [c] (5.38609,3.37729) -- (5.40091,3.37729);
\draw [c] (5.40091,3.37729) -- (5.41573,3.37729);
\definecolor{c}{rgb}{0,0,0};
\colorlet{c}{kugray};
\draw [c] (5.43055,3.31745) -- (5.43055,3.34765);
\draw [c] (5.43055,3.34765) -- (5.43055,3.37514);
\draw [c] (5.41573,3.34765) -- (5.43055,3.34765);
\draw [c] (5.43055,3.34765) -- (5.44536,3.34765);
\definecolor{c}{rgb}{0,0,0};
\colorlet{c}{kugray};
\draw [c] (5.46018,3.27009) -- (5.46018,3.30118);
\draw [c] (5.46018,3.30118) -- (5.46018,3.32942);
\draw [c] (5.44536,3.30118) -- (5.46018,3.30118);
\draw [c] (5.46018,3.30118) -- (5.475,3.30118);
\definecolor{c}{rgb}{0,0,0};
\colorlet{c}{kugray};
\draw [c] (5.48982,3.30023) -- (5.48982,3.32999);
\draw [c] (5.48982,3.32999) -- (5.48982,3.35712);
\draw [c] (5.475,3.32999) -- (5.48982,3.32999);
\draw [c] (5.48982,3.32999) -- (5.50464,3.32999);
\definecolor{c}{rgb}{0,0,0};
\colorlet{c}{kugray};
\draw [c] (5.51945,3.27827) -- (5.51945,3.30971);
\draw [c] (5.51945,3.30971) -- (5.51945,3.33824);
\draw [c] (5.50464,3.30971) -- (5.51945,3.30971);
\draw [c] (5.51945,3.30971) -- (5.53427,3.30971);
\definecolor{c}{rgb}{0,0,0};
\colorlet{c}{kugray};
\draw [c] (5.54909,3.27339) -- (5.54909,3.30456);
\draw [c] (5.54909,3.30456) -- (5.54909,3.33287);
\draw [c] (5.53427,3.30456) -- (5.54909,3.30456);
\draw [c] (5.54909,3.30456) -- (5.56391,3.30456);
\definecolor{c}{rgb}{0,0,0};
\colorlet{c}{kugray};
\draw [c] (5.57873,3.2928) -- (5.57873,3.32215);
\draw [c] (5.57873,3.32215) -- (5.57873,3.34893);
\draw [c] (5.56391,3.32215) -- (5.57873,3.32215);
\draw [c] (5.57873,3.32215) -- (5.59354,3.32215);
\definecolor{c}{rgb}{0,0,0};
\colorlet{c}{kugray};
\draw [c] (5.60836,3.32728) -- (5.60836,3.35666);
\draw [c] (5.60836,3.35666) -- (5.60836,3.38348);
\draw [c] (5.59354,3.35666) -- (5.60836,3.35666);
\draw [c] (5.60836,3.35666) -- (5.62318,3.35666);
\definecolor{c}{rgb}{0,0,0};
\colorlet{c}{kugray};
\draw [c] (5.638,3.35064) -- (5.638,3.3789);
\draw [c] (5.638,3.3789) -- (5.638,3.40479);
\draw [c] (5.62318,3.3789) -- (5.638,3.3789);
\draw [c] (5.638,3.3789) -- (5.65281,3.3789);
\definecolor{c}{rgb}{0,0,0};
\colorlet{c}{kugray};
\draw [c] (5.66763,3.27119) -- (5.66763,3.30362);
\draw [c] (5.66763,3.30362) -- (5.66763,3.33296);
\draw [c] (5.65281,3.30362) -- (5.66763,3.30362);
\draw [c] (5.66763,3.30362) -- (5.68245,3.30362);
\definecolor{c}{rgb}{0,0,0};
\colorlet{c}{kugray};
\draw [c] (5.69727,3.30087) -- (5.69727,3.33324);
\draw [c] (5.69727,3.33324) -- (5.69727,3.36252);
\draw [c] (5.68245,3.33324) -- (5.69727,3.33324);
\draw [c] (5.69727,3.33324) -- (5.71209,3.33324);
\definecolor{c}{rgb}{0,0,0};
\colorlet{c}{kugray};
\draw [c] (5.7269,3.35191) -- (5.7269,3.37987);
\draw [c] (5.7269,3.37987) -- (5.7269,3.4055);
\draw [c] (5.71209,3.37987) -- (5.7269,3.37987);
\draw [c] (5.7269,3.37987) -- (5.74172,3.37987);
\definecolor{c}{rgb}{0,0,0};
\colorlet{c}{kugray};
\draw [c] (5.75654,3.32022) -- (5.75654,3.34998);
\draw [c] (5.75654,3.34998) -- (5.75654,3.37711);
\draw [c] (5.74172,3.34998) -- (5.75654,3.34998);
\draw [c] (5.75654,3.34998) -- (5.77136,3.34998);
\definecolor{c}{rgb}{0,0,0};
\colorlet{c}{kugray};
\draw [c] (5.78618,3.29616) -- (5.78618,3.32751);
\draw [c] (5.78618,3.32751) -- (5.78618,3.35595);
\draw [c] (5.77136,3.32751) -- (5.78618,3.32751);
\draw [c] (5.78618,3.32751) -- (5.80099,3.32751);
\definecolor{c}{rgb}{0,0,0};
\colorlet{c}{kugray};
\draw [c] (5.81581,3.25342) -- (5.81581,3.28535);
\draw [c] (5.81581,3.28535) -- (5.81581,3.31427);
\draw [c] (5.80099,3.28535) -- (5.81581,3.28535);
\draw [c] (5.81581,3.28535) -- (5.83063,3.28535);
\definecolor{c}{rgb}{0,0,0};
\colorlet{c}{kugray};
\draw [c] (5.84545,3.26747) -- (5.84545,3.30131);
\draw [c] (5.84545,3.30131) -- (5.84545,3.3318);
\draw [c] (5.83063,3.30131) -- (5.84545,3.30131);
\draw [c] (5.84545,3.30131) -- (5.86026,3.30131);
\definecolor{c}{rgb}{0,0,0};
\colorlet{c}{kugray};
\draw [c] (5.87508,3.27773) -- (5.87508,3.31106);
\draw [c] (5.87508,3.31106) -- (5.87508,3.34112);
\draw [c] (5.86026,3.31106) -- (5.87508,3.31106);
\draw [c] (5.87508,3.31106) -- (5.8899,3.31106);
\definecolor{c}{rgb}{0,0,0};
\colorlet{c}{kugray};
\draw [c] (5.90472,3.33645) -- (5.90472,3.36537);
\draw [c] (5.90472,3.36537) -- (5.90472,3.3918);
\draw [c] (5.8899,3.36537) -- (5.90472,3.36537);
\draw [c] (5.90472,3.36537) -- (5.91954,3.36537);
\definecolor{c}{rgb}{0,0,0};
\colorlet{c}{kugray};
\draw [c] (5.93435,3.324) -- (5.93435,3.35624);
\draw [c] (5.93435,3.35624) -- (5.93435,3.38541);
\draw [c] (5.91954,3.35624) -- (5.93435,3.35624);
\draw [c] (5.93435,3.35624) -- (5.94917,3.35624);
\definecolor{c}{rgb}{0,0,0};
\colorlet{c}{kugray};
\draw [c] (5.96399,3.26085) -- (5.96399,3.29364);
\draw [c] (5.96399,3.29364) -- (5.96399,3.32327);
\draw [c] (5.94917,3.29364) -- (5.96399,3.29364);
\draw [c] (5.96399,3.29364) -- (5.97881,3.29364);
\definecolor{c}{rgb}{0,0,0};
\colorlet{c}{kugray};
\draw [c] (5.99363,3.26893) -- (5.99363,3.30106);
\draw [c] (5.99363,3.30106) -- (5.99363,3.33016);
\draw [c] (5.97881,3.30106) -- (5.99363,3.30106);
\draw [c] (5.99363,3.30106) -- (6.00844,3.30106);
\definecolor{c}{rgb}{0,0,0};
\colorlet{c}{kugray};
\draw [c] (6.02326,3.32056) -- (6.02326,3.35257);
\draw [c] (6.02326,3.35257) -- (6.02326,3.38156);
\draw [c] (6.00844,3.35257) -- (6.02326,3.35257);
\draw [c] (6.02326,3.35257) -- (6.03808,3.35257);
\definecolor{c}{rgb}{0,0,0};
\colorlet{c}{kugray};
\draw [c] (6.0529,3.27284) -- (6.0529,3.30377);
\draw [c] (6.0529,3.30377) -- (6.0529,3.33188);
\draw [c] (6.03808,3.30377) -- (6.0529,3.30377);
\draw [c] (6.0529,3.30377) -- (6.06772,3.30377);
\definecolor{c}{rgb}{0,0,0};
\colorlet{c}{kugray};
\draw [c] (6.08253,3.28251) -- (6.08253,3.31642);
\draw [c] (6.08253,3.31642) -- (6.08253,3.34697);
\draw [c] (6.06772,3.31642) -- (6.08253,3.31642);
\draw [c] (6.08253,3.31642) -- (6.09735,3.31642);
\definecolor{c}{rgb}{0,0,0};
\colorlet{c}{kugray};
\draw [c] (6.11217,3.31512) -- (6.11217,3.34443);
\draw [c] (6.11217,3.34443) -- (6.11217,3.3712);
\draw [c] (6.09735,3.34443) -- (6.11217,3.34443);
\draw [c] (6.11217,3.34443) -- (6.12699,3.34443);
\definecolor{c}{rgb}{0,0,0};
\colorlet{c}{kugray};
\draw [c] (6.1418,3.24222) -- (6.1418,3.27469);
\draw [c] (6.1418,3.27469) -- (6.1418,3.30406);
\draw [c] (6.12699,3.27469) -- (6.1418,3.27469);
\draw [c] (6.1418,3.27469) -- (6.15662,3.27469);
\definecolor{c}{rgb}{0,0,0};
\colorlet{c}{kugray};
\draw [c] (6.17144,3.19853) -- (6.17144,3.23317);
\draw [c] (6.17144,3.23317) -- (6.17144,3.26431);
\draw [c] (6.15662,3.23317) -- (6.17144,3.23317);
\draw [c] (6.17144,3.23317) -- (6.18626,3.23317);
\definecolor{c}{rgb}{0,0,0};
\colorlet{c}{kugray};
\draw [c] (6.20108,3.26243) -- (6.20108,3.29519);
\draw [c] (6.20108,3.29519) -- (6.20108,3.32479);
\draw [c] (6.18626,3.29519) -- (6.20108,3.29519);
\draw [c] (6.20108,3.29519) -- (6.21589,3.29519);
\definecolor{c}{rgb}{0,0,0};
\colorlet{c}{kugray};
\draw [c] (6.23071,3.26744) -- (6.23071,3.30118);
\draw [c] (6.23071,3.30118) -- (6.23071,3.33159);
\draw [c] (6.21589,3.30118) -- (6.23071,3.30118);
\draw [c] (6.23071,3.30118) -- (6.24553,3.30118);
\definecolor{c}{rgb}{0,0,0};
\colorlet{c}{kugray};
\draw [c] (6.26035,3.29406) -- (6.26035,3.32675);
\draw [c] (6.26035,3.32675) -- (6.26035,3.35629);
\draw [c] (6.24553,3.32675) -- (6.26035,3.32675);
\draw [c] (6.26035,3.32675) -- (6.27517,3.32675);
\definecolor{c}{rgb}{0,0,0};
\colorlet{c}{kugray};
\draw [c] (6.28998,3.23898) -- (6.28998,3.27334);
\draw [c] (6.28998,3.27334) -- (6.28998,3.30426);
\draw [c] (6.27517,3.27334) -- (6.28998,3.27334);
\draw [c] (6.28998,3.27334) -- (6.3048,3.27334);
\definecolor{c}{rgb}{0,0,0};
\colorlet{c}{kugray};
\draw [c] (6.31962,3.22624) -- (6.31962,3.26036);
\draw [c] (6.31962,3.26036) -- (6.31962,3.29107);
\draw [c] (6.3048,3.26036) -- (6.31962,3.26036);
\draw [c] (6.31962,3.26036) -- (6.33444,3.26036);
\definecolor{c}{rgb}{0,0,0};
\colorlet{c}{kugray};
\draw [c] (6.34926,3.26468) -- (6.34926,3.29684);
\draw [c] (6.34926,3.29684) -- (6.34926,3.32595);
\draw [c] (6.33444,3.29684) -- (6.34926,3.29684);
\draw [c] (6.34926,3.29684) -- (6.36407,3.29684);
\definecolor{c}{rgb}{0,0,0};
\colorlet{c}{kugray};
\draw [c] (6.37889,3.30966) -- (6.37889,3.34032);
\draw [c] (6.37889,3.34032) -- (6.37889,3.36821);
\draw [c] (6.36407,3.34032) -- (6.37889,3.34032);
\draw [c] (6.37889,3.34032) -- (6.39371,3.34032);
\definecolor{c}{rgb}{0,0,0};
\colorlet{c}{kugray};
\draw [c] (6.40853,3.23431) -- (6.40853,3.26778);
\draw [c] (6.40853,3.26778) -- (6.40853,3.29797);
\draw [c] (6.39371,3.26778) -- (6.40853,3.26778);
\draw [c] (6.40853,3.26778) -- (6.42334,3.26778);
\definecolor{c}{rgb}{0,0,0};
\colorlet{c}{kugray};
\draw [c] (6.43816,3.26143) -- (6.43816,3.29357);
\draw [c] (6.43816,3.29357) -- (6.43816,3.32268);
\draw [c] (6.42334,3.29357) -- (6.43816,3.29357);
\draw [c] (6.43816,3.29357) -- (6.45298,3.29357);
\definecolor{c}{rgb}{0,0,0};
\colorlet{c}{kugray};
\draw [c] (6.4678,3.21815) -- (6.4678,3.25281);
\draw [c] (6.4678,3.25281) -- (6.4678,3.28397);
\draw [c] (6.45298,3.25281) -- (6.4678,3.25281);
\draw [c] (6.4678,3.25281) -- (6.48262,3.25281);
\definecolor{c}{rgb}{0,0,0};
\colorlet{c}{kugray};
\draw [c] (6.49743,3.25782) -- (6.49743,3.29048);
\draw [c] (6.49743,3.29048) -- (6.49743,3.32);
\draw [c] (6.48262,3.29048) -- (6.49743,3.29048);
\draw [c] (6.49743,3.29048) -- (6.51225,3.29048);
\definecolor{c}{rgb}{0,0,0};
\colorlet{c}{kugray};
\draw [c] (6.52707,3.29629) -- (6.52707,3.32741);
\draw [c] (6.52707,3.32741) -- (6.52707,3.35568);
\draw [c] (6.51225,3.32741) -- (6.52707,3.32741);
\draw [c] (6.52707,3.32741) -- (6.54189,3.32741);
\definecolor{c}{rgb}{0,0,0};
\colorlet{c}{kugray};
\draw [c] (6.55671,3.28852) -- (6.55671,3.32075);
\draw [c] (6.55671,3.32075) -- (6.55671,3.34993);
\draw [c] (6.54189,3.32075) -- (6.55671,3.32075);
\draw [c] (6.55671,3.32075) -- (6.57152,3.32075);
\definecolor{c}{rgb}{0,0,0};
\colorlet{c}{kugray};
\draw [c] (6.58634,3.11775) -- (6.58634,3.15987);
\draw [c] (6.58634,3.15987) -- (6.58634,3.19692);
\draw [c] (6.57152,3.15987) -- (6.58634,3.15987);
\draw [c] (6.58634,3.15987) -- (6.60116,3.15987);
\definecolor{c}{rgb}{0,0,0};
\colorlet{c}{kugray};
\draw [c] (6.61598,3.23027) -- (6.61598,3.2646);
\draw [c] (6.61598,3.2646) -- (6.61598,3.29549);
\draw [c] (6.60116,3.2646) -- (6.61598,3.2646);
\draw [c] (6.61598,3.2646) -- (6.63079,3.2646);
\definecolor{c}{rgb}{0,0,0};
\colorlet{c}{kugray};
\draw [c] (6.64561,3.24953) -- (6.64561,3.28391);
\draw [c] (6.64561,3.28391) -- (6.64561,3.31483);
\draw [c] (6.63079,3.28391) -- (6.64561,3.28391);
\draw [c] (6.64561,3.28391) -- (6.66043,3.28391);
\definecolor{c}{rgb}{0,0,0};
\colorlet{c}{kugray};
\draw [c] (6.67525,3.22925) -- (6.67525,3.26504);
\draw [c] (6.67525,3.26504) -- (6.67525,3.2971);
\draw [c] (6.66043,3.26504) -- (6.67525,3.26504);
\draw [c] (6.67525,3.26504) -- (6.69007,3.26504);
\definecolor{c}{rgb}{0,0,0};
\colorlet{c}{kugray};
\draw [c] (6.70488,3.27333) -- (6.70488,3.30459);
\draw [c] (6.70488,3.30459) -- (6.70488,3.33297);
\draw [c] (6.69007,3.30459) -- (6.70488,3.30459);
\draw [c] (6.70488,3.30459) -- (6.7197,3.30459);
\definecolor{c}{rgb}{0,0,0};
\colorlet{c}{kugray};
\draw [c] (6.73452,3.2188) -- (6.73452,3.25383);
\draw [c] (6.73452,3.25383) -- (6.73452,3.28527);
\draw [c] (6.7197,3.25383) -- (6.73452,3.25383);
\draw [c] (6.73452,3.25383) -- (6.74934,3.25383);
\definecolor{c}{rgb}{0,0,0};
\colorlet{c}{kugray};
\draw [c] (6.76416,3.19407) -- (6.76416,3.23035);
\draw [c] (6.76416,3.23035) -- (6.76416,3.26281);
\draw [c] (6.74934,3.23035) -- (6.76416,3.23035);
\draw [c] (6.76416,3.23035) -- (6.77897,3.23035);
\definecolor{c}{rgb}{0,0,0};
\colorlet{c}{kugray};
\draw [c] (6.79379,3.2118) -- (6.79379,3.24685);
\draw [c] (6.79379,3.24685) -- (6.79379,3.2783);
\draw [c] (6.77897,3.24685) -- (6.79379,3.24685);
\draw [c] (6.79379,3.24685) -- (6.80861,3.24685);
\definecolor{c}{rgb}{0,0,0};
\colorlet{c}{kugray};
\draw [c] (6.82343,3.20129) -- (6.82343,3.23637);
\draw [c] (6.82343,3.23637) -- (6.82343,3.26785);
\draw [c] (6.80861,3.23637) -- (6.82343,3.23637);
\draw [c] (6.82343,3.23637) -- (6.83824,3.23637);
\definecolor{c}{rgb}{0,0,0};
\colorlet{c}{kugray};
\draw [c] (6.85306,3.27615) -- (6.85306,3.30874);
\draw [c] (6.85306,3.30874) -- (6.85306,3.33821);
\draw [c] (6.83824,3.30874) -- (6.85306,3.30874);
\draw [c] (6.85306,3.30874) -- (6.86788,3.30874);
\definecolor{c}{rgb}{0,0,0};
\colorlet{c}{kugray};
\draw [c] (6.8827,3.28993) -- (6.8827,3.32159);
\draw [c] (6.8827,3.32159) -- (6.8827,3.3503);
\draw [c] (6.86788,3.32159) -- (6.8827,3.32159);
\draw [c] (6.8827,3.32159) -- (6.89752,3.32159);
\definecolor{c}{rgb}{0,0,0};
\colorlet{c}{kugray};
\draw [c] (6.91233,3.22502) -- (6.91233,3.25976);
\draw [c] (6.91233,3.25976) -- (6.91233,3.29097);
\draw [c] (6.89752,3.25976) -- (6.91233,3.25976);
\draw [c] (6.91233,3.25976) -- (6.92715,3.25976);
\definecolor{c}{rgb}{0,0,0};
\colorlet{c}{kugray};
\draw [c] (6.94197,3.27047) -- (6.94197,3.30221);
\draw [c] (6.94197,3.30221) -- (6.94197,3.33098);
\draw [c] (6.92715,3.30221) -- (6.94197,3.30221);
\draw [c] (6.94197,3.30221) -- (6.95679,3.30221);
\definecolor{c}{rgb}{0,0,0};
\colorlet{c}{kugray};
\draw [c] (6.97161,3.23938) -- (6.97161,3.27409);
\draw [c] (6.97161,3.27409) -- (6.97161,3.30529);
\draw [c] (6.95679,3.27409) -- (6.97161,3.27409);
\draw [c] (6.97161,3.27409) -- (6.98642,3.27409);
\definecolor{c}{rgb}{0,0,0};
\colorlet{c}{kugray};
\draw [c] (7.00124,3.2166) -- (7.00124,3.25185);
\draw [c] (7.00124,3.25185) -- (7.00124,3.28348);
\draw [c] (6.98642,3.25185) -- (7.00124,3.25185);
\draw [c] (7.00124,3.25185) -- (7.01606,3.25185);
\definecolor{c}{rgb}{0,0,0};
\colorlet{c}{kugray};
\draw [c] (7.03088,3.27665) -- (7.03088,3.3089);
\draw [c] (7.03088,3.3089) -- (7.03088,3.33809);
\draw [c] (7.01606,3.3089) -- (7.03088,3.3089);
\draw [c] (7.03088,3.3089) -- (7.0457,3.3089);
\definecolor{c}{rgb}{0,0,0};
\colorlet{c}{kugray};
\draw [c] (7.06051,3.14522) -- (7.06051,3.18511);
\draw [c] (7.06051,3.18511) -- (7.06051,3.22041);
\draw [c] (7.0457,3.18511) -- (7.06051,3.18511);
\draw [c] (7.06051,3.18511) -- (7.07533,3.18511);
\definecolor{c}{rgb}{0,0,0};
\colorlet{c}{kugray};
\draw [c] (7.09015,3.25803) -- (7.09015,3.29279);
\draw [c] (7.09015,3.29279) -- (7.09015,3.32402);
\draw [c] (7.07533,3.29279) -- (7.09015,3.29279);
\draw [c] (7.09015,3.29279) -- (7.10497,3.29279);
\definecolor{c}{rgb}{0,0,0};
\colorlet{c}{kugray};
\draw [c] (7.11978,3.1251) -- (7.11978,3.16549);
\draw [c] (7.11978,3.16549) -- (7.11978,3.2012);
\draw [c] (7.10497,3.16549) -- (7.11978,3.16549);
\draw [c] (7.11978,3.16549) -- (7.1346,3.16549);
\definecolor{c}{rgb}{0,0,0};
\colorlet{c}{kugray};
\draw [c] (7.14942,3.20086) -- (7.14942,3.23706);
\draw [c] (7.14942,3.23706) -- (7.14942,3.26946);
\draw [c] (7.1346,3.23706) -- (7.14942,3.23706);
\draw [c] (7.14942,3.23706) -- (7.16424,3.23706);
\definecolor{c}{rgb}{0,0,0};
\colorlet{c}{kugray};
\draw [c] (7.17906,3.23931) -- (7.17906,3.27217);
\draw [c] (7.17906,3.27217) -- (7.17906,3.30187);
\draw [c] (7.16424,3.27217) -- (7.17906,3.27217);
\draw [c] (7.17906,3.27217) -- (7.19387,3.27217);
\definecolor{c}{rgb}{0,0,0};
\colorlet{c}{kugray};
\draw [c] (7.20869,3.22963) -- (7.20869,3.26378);
\draw [c] (7.20869,3.26378) -- (7.20869,3.29452);
\draw [c] (7.19387,3.26378) -- (7.20869,3.26378);
\draw [c] (7.20869,3.26378) -- (7.22351,3.26378);
\definecolor{c}{rgb}{0,0,0};
\colorlet{c}{kugray};
\draw [c] (7.23833,3.17041) -- (7.23833,3.20638);
\draw [c] (7.23833,3.20638) -- (7.23833,3.23859);
\draw [c] (7.22351,3.20638) -- (7.23833,3.20638);
\draw [c] (7.23833,3.20638) -- (7.25315,3.20638);
\definecolor{c}{rgb}{0,0,0};
\colorlet{c}{kugray};
\draw [c] (7.26796,3.20336) -- (7.26796,3.2391);
\draw [c] (7.26796,3.2391) -- (7.26796,3.27111);
\draw [c] (7.25315,3.2391) -- (7.26796,3.2391);
\draw [c] (7.26796,3.2391) -- (7.28278,3.2391);
\definecolor{c}{rgb}{0,0,0};
\colorlet{c}{kugray};
\draw [c] (7.2976,3.13789) -- (7.2976,3.17615);
\draw [c] (7.2976,3.17615) -- (7.2976,3.21018);
\draw [c] (7.28278,3.17615) -- (7.2976,3.17615);
\draw [c] (7.2976,3.17615) -- (7.31242,3.17615);
\definecolor{c}{rgb}{0,0,0};
\colorlet{c}{kugray};
\draw [c] (7.32724,3.18107) -- (7.32724,3.21995);
\draw [c] (7.32724,3.21995) -- (7.32724,3.25447);
\draw [c] (7.31242,3.21995) -- (7.32724,3.21995);
\draw [c] (7.32724,3.21995) -- (7.34205,3.21995);
\definecolor{c}{rgb}{0,0,0};
\colorlet{c}{kugray};
\draw [c] (7.35687,3.17771) -- (7.35687,3.214);
\draw [c] (7.35687,3.214) -- (7.35687,3.24646);
\draw [c] (7.34205,3.214) -- (7.35687,3.214);
\draw [c] (7.35687,3.214) -- (7.37169,3.214);
\definecolor{c}{rgb}{0,0,0};
\colorlet{c}{kugray};
\draw [c] (7.38651,3.20963) -- (7.38651,3.24614);
\draw [c] (7.38651,3.24614) -- (7.38651,3.27878);
\draw [c] (7.37169,3.24614) -- (7.38651,3.24614);
\draw [c] (7.38651,3.24614) -- (7.40132,3.24614);
\definecolor{c}{rgb}{0,0,0};
\colorlet{c}{kugray};
\draw [c] (7.41614,3.23434) -- (7.41614,3.26861);
\draw [c] (7.41614,3.26861) -- (7.41614,3.29944);
\draw [c] (7.40132,3.26861) -- (7.41614,3.26861);
\draw [c] (7.41614,3.26861) -- (7.43096,3.26861);
\definecolor{c}{rgb}{0,0,0};
\colorlet{c}{kugray};
\draw [c] (7.44578,3.26607) -- (7.44578,3.29811);
\draw [c] (7.44578,3.29811) -- (7.44578,3.32713);
\draw [c] (7.43096,3.29811) -- (7.44578,3.29811);
\draw [c] (7.44578,3.29811) -- (7.4606,3.29811);
\definecolor{c}{rgb}{0,0,0};
\colorlet{c}{kugray};
\draw [c] (7.47541,3.23088) -- (7.47541,3.26608);
\draw [c] (7.47541,3.26608) -- (7.47541,3.29768);
\draw [c] (7.4606,3.26608) -- (7.47541,3.26608);
\draw [c] (7.47541,3.26608) -- (7.49023,3.26608);
\definecolor{c}{rgb}{0,0,0};
\colorlet{c}{kugray};
\draw [c] (7.50505,3.19492) -- (7.50505,3.23154);
\draw [c] (7.50505,3.23154) -- (7.50505,3.26427);
\draw [c] (7.49023,3.23154) -- (7.50505,3.23154);
\draw [c] (7.50505,3.23154) -- (7.51987,3.23154);
\definecolor{c}{rgb}{0,0,0};
\colorlet{c}{kugray};
\draw [c] (7.53469,3.16974) -- (7.53469,3.20818);
\draw [c] (7.53469,3.20818) -- (7.53469,3.24234);
\draw [c] (7.51987,3.20818) -- (7.53469,3.20818);
\draw [c] (7.53469,3.20818) -- (7.5495,3.20818);
\definecolor{c}{rgb}{0,0,0};
\colorlet{c}{kugray};
\draw [c] (7.56432,3.23775) -- (7.56432,3.27182);
\draw [c] (7.56432,3.27182) -- (7.56432,3.3025);
\draw [c] (7.5495,3.27182) -- (7.56432,3.27182);
\draw [c] (7.56432,3.27182) -- (7.57914,3.27182);
\definecolor{c}{rgb}{0,0,0};
\colorlet{c}{kugray};
\draw [c] (7.59396,3.27348) -- (7.59396,3.30657);
\draw [c] (7.59396,3.30657) -- (7.59396,3.33644);
\draw [c] (7.57914,3.30657) -- (7.59396,3.30657);
\draw [c] (7.59396,3.30657) -- (7.60877,3.30657);
\definecolor{c}{rgb}{0,0,0};
\colorlet{c}{kugray};
\draw [c] (7.62359,3.20408) -- (7.62359,3.24244);
\draw [c] (7.62359,3.24244) -- (7.62359,3.27655);
\draw [c] (7.60877,3.24244) -- (7.62359,3.24244);
\draw [c] (7.62359,3.24244) -- (7.63841,3.24244);
\definecolor{c}{rgb}{0,0,0};
\colorlet{c}{kugray};
\draw [c] (7.65323,3.20046) -- (7.65323,3.23614);
\draw [c] (7.65323,3.23614) -- (7.65323,3.26812);
\draw [c] (7.63841,3.23614) -- (7.65323,3.23614);
\draw [c] (7.65323,3.23614) -- (7.66805,3.23614);
\definecolor{c}{rgb}{0,0,0};
\colorlet{c}{kugray};
\draw [c] (7.68286,3.11491) -- (7.68286,3.15626);
\draw [c] (7.68286,3.15626) -- (7.68286,3.19271);
\draw [c] (7.66805,3.15626) -- (7.68286,3.15626);
\draw [c] (7.68286,3.15626) -- (7.69768,3.15626);
\definecolor{c}{rgb}{0,0,0};
\colorlet{c}{kugray};
\draw [c] (7.7125,3.20878) -- (7.7125,3.24334);
\draw [c] (7.7125,3.24334) -- (7.7125,3.2744);
\draw [c] (7.69768,3.24334) -- (7.7125,3.24334);
\draw [c] (7.7125,3.24334) -- (7.72732,3.24334);
\definecolor{c}{rgb}{0,0,0};
\colorlet{c}{kugray};
\draw [c] (7.74214,3.18668) -- (7.74214,3.22275);
\draw [c] (7.74214,3.22275) -- (7.74214,3.25502);
\draw [c] (7.72732,3.22275) -- (7.74214,3.22275);
\draw [c] (7.74214,3.22275) -- (7.75695,3.22275);
\definecolor{c}{rgb}{0,0,0};
\colorlet{c}{kugray};
\draw [c] (7.77177,3.17887) -- (7.77177,3.21749);
\draw [c] (7.77177,3.21749) -- (7.77177,3.2518);
\draw [c] (7.75695,3.21749) -- (7.77177,3.21749);
\draw [c] (7.77177,3.21749) -- (7.78659,3.21749);
\definecolor{c}{rgb}{0,0,0};
\colorlet{c}{kugray};
\draw [c] (7.80141,3.1287) -- (7.80141,3.1705);
\draw [c] (7.80141,3.1705) -- (7.80141,3.2073);
\draw [c] (7.78659,3.1705) -- (7.80141,3.1705);
\draw [c] (7.80141,3.1705) -- (7.81623,3.1705);
\definecolor{c}{rgb}{0,0,0};
\colorlet{c}{kugray};
\draw [c] (7.83104,3.19387) -- (7.83104,3.23296);
\draw [c] (7.83104,3.23296) -- (7.83104,3.26765);
\draw [c] (7.81623,3.23296) -- (7.83104,3.23296);
\draw [c] (7.83104,3.23296) -- (7.84586,3.23296);
\definecolor{c}{rgb}{0,0,0};
\colorlet{c}{kugray};
\draw [c] (7.86068,3.19817) -- (7.86068,3.23466);
\draw [c] (7.86068,3.23466) -- (7.86068,3.26728);
\draw [c] (7.84586,3.23466) -- (7.86068,3.23466);
\draw [c] (7.86068,3.23466) -- (7.8755,3.23466);
\definecolor{c}{rgb}{0,0,0};
\colorlet{c}{kugray};
\draw [c] (7.89031,3.19947) -- (7.89031,3.23529);
\draw [c] (7.89031,3.23529) -- (7.89031,3.26737);
\draw [c] (7.8755,3.23529) -- (7.89031,3.23529);
\draw [c] (7.89031,3.23529) -- (7.90513,3.23529);
\definecolor{c}{rgb}{0,0,0};
\colorlet{c}{kugray};
\draw [c] (7.91995,3.17289) -- (7.91995,3.21165);
\draw [c] (7.91995,3.21165) -- (7.91995,3.24608);
\draw [c] (7.90513,3.21165) -- (7.91995,3.21165);
\draw [c] (7.91995,3.21165) -- (7.93477,3.21165);
\definecolor{c}{rgb}{0,0,0};
\colorlet{c}{kugray};
\draw [c] (7.94959,3.16056) -- (7.94959,3.20019);
\draw [c] (7.94959,3.20019) -- (7.94959,3.23529);
\draw [c] (7.93477,3.20019) -- (7.94959,3.20019);
\draw [c] (7.94959,3.20019) -- (7.9644,3.20019);
\definecolor{c}{rgb}{0,0,0};
\colorlet{c}{kugray};
\draw [c] (7.97922,3.17291) -- (7.97922,3.2128);
\draw [c] (7.97922,3.2128) -- (7.97922,3.2481);
\draw [c] (7.9644,3.2128) -- (7.97922,3.2128);
\draw [c] (7.97922,3.2128) -- (7.99404,3.2128);
\definecolor{c}{rgb}{0,0,0};
\colorlet{c}{kugray};
\draw [c] (8.00886,3.21345) -- (8.00886,3.24811);
\draw [c] (8.00886,3.24811) -- (8.00886,3.27926);
\draw [c] (7.99404,3.24811) -- (8.00886,3.24811);
\draw [c] (8.00886,3.24811) -- (8.02368,3.24811);
\definecolor{c}{rgb}{0,0,0};
\colorlet{c}{kugray};
\draw [c] (8.03849,3.16384) -- (8.03849,3.20119);
\draw [c] (8.03849,3.20119) -- (8.03849,3.2345);
\draw [c] (8.02368,3.20119) -- (8.03849,3.20119);
\draw [c] (8.03849,3.20119) -- (8.05331,3.20119);
\definecolor{c}{rgb}{0,0,0};
\colorlet{c}{kugray};
\draw [c] (8.06813,3.2368) -- (8.06813,3.27102);
\draw [c] (8.06813,3.27102) -- (8.06813,3.30182);
\draw [c] (8.05331,3.27102) -- (8.06813,3.27102);
\draw [c] (8.06813,3.27102) -- (8.08295,3.27102);
\definecolor{c}{rgb}{0,0,0};
\colorlet{c}{kugray};
\draw [c] (8.09776,3.20237) -- (8.09776,3.23814);
\draw [c] (8.09776,3.23814) -- (8.09776,3.27017);
\draw [c] (8.08295,3.23814) -- (8.09776,3.23814);
\draw [c] (8.09776,3.23814) -- (8.11258,3.23814);
\definecolor{c}{rgb}{0,0,0};
\colorlet{c}{kugray};
\draw [c] (8.1274,3.16214) -- (8.1274,3.20061);
\draw [c] (8.1274,3.20061) -- (8.1274,3.2348);
\draw [c] (8.11258,3.20061) -- (8.1274,3.20061);
\draw [c] (8.1274,3.20061) -- (8.14222,3.20061);
\definecolor{c}{rgb}{0,0,0};
\colorlet{c}{kugray};
\draw [c] (8.15704,3.17943) -- (8.15704,3.21673);
\draw [c] (8.15704,3.21673) -- (8.15704,3.24999);
\draw [c] (8.14222,3.21673) -- (8.15704,3.21673);
\draw [c] (8.15704,3.21673) -- (8.17185,3.21673);
\definecolor{c}{rgb}{0,0,0};
\colorlet{c}{kugray};
\draw [c] (8.18667,3.2077) -- (8.18667,3.24191);
\draw [c] (8.18667,3.24191) -- (8.18667,3.2727);
\draw [c] (8.17185,3.24191) -- (8.18667,3.24191);
\draw [c] (8.18667,3.24191) -- (8.20149,3.24191);
\definecolor{c}{rgb}{0,0,0};
\colorlet{c}{kugray};
\draw [c] (8.21631,3.13546) -- (8.21631,3.17622);
\draw [c] (8.21631,3.17622) -- (8.21631,3.21221);
\draw [c] (8.20149,3.17622) -- (8.21631,3.17622);
\draw [c] (8.21631,3.17622) -- (8.23113,3.17622);
\definecolor{c}{rgb}{0,0,0};
\colorlet{c}{kugray};
\draw [c] (8.24594,3.14151) -- (8.24594,3.18151);
\draw [c] (8.24594,3.18151) -- (8.24594,3.2169);
\draw [c] (8.23113,3.18151) -- (8.24594,3.18151);
\draw [c] (8.24594,3.18151) -- (8.26076,3.18151);
\definecolor{c}{rgb}{0,0,0};
\colorlet{c}{kugray};
\draw [c] (8.27558,3.17226) -- (8.27558,3.21036);
\draw [c] (8.27558,3.21036) -- (8.27558,3.24426);
\draw [c] (8.26076,3.21036) -- (8.27558,3.21036);
\draw [c] (8.27558,3.21036) -- (8.2904,3.21036);
\definecolor{c}{rgb}{0,0,0};
\colorlet{c}{kugray};
\draw [c] (8.30521,3.21664) -- (8.30521,3.25163);
\draw [c] (8.30521,3.25163) -- (8.30521,3.28304);
\draw [c] (8.2904,3.25163) -- (8.30521,3.25163);
\draw [c] (8.30521,3.25163) -- (8.32003,3.25163);
\definecolor{c}{rgb}{0,0,0};
\colorlet{c}{kugray};
\draw [c] (8.33485,3.18633) -- (8.33485,3.22273);
\draw [c] (8.33485,3.22273) -- (8.33485,3.25528);
\draw [c] (8.32003,3.22273) -- (8.33485,3.22273);
\draw [c] (8.33485,3.22273) -- (8.34967,3.22273);
\definecolor{c}{rgb}{0,0,0};
\colorlet{c}{kugray};
\draw [c] (8.36449,3.10352) -- (8.36449,3.14557);
\draw [c] (8.36449,3.14557) -- (8.36449,3.18257);
\draw [c] (8.34967,3.14557) -- (8.36449,3.14557);
\draw [c] (8.36449,3.14557) -- (8.3793,3.14557);
\definecolor{c}{rgb}{0,0,0};
\colorlet{c}{kugray};
\draw [c] (8.39412,3.14212) -- (8.39412,3.18236);
\draw [c] (8.39412,3.18236) -- (8.39412,3.21794);
\draw [c] (8.3793,3.18236) -- (8.39412,3.18236);
\draw [c] (8.39412,3.18236) -- (8.40894,3.18236);
\definecolor{c}{rgb}{0,0,0};
\colorlet{c}{kugray};
\draw [c] (8.42376,3.14268) -- (8.42376,3.18228);
\draw [c] (8.42376,3.18228) -- (8.42376,3.21737);
\draw [c] (8.40894,3.18228) -- (8.42376,3.18228);
\draw [c] (8.42376,3.18228) -- (8.43858,3.18228);
\definecolor{c}{rgb}{0,0,0};
\colorlet{c}{kugray};
\draw [c] (8.45339,3.17027) -- (8.45339,3.21011);
\draw [c] (8.45339,3.21011) -- (8.45339,3.24537);
\draw [c] (8.43858,3.21011) -- (8.45339,3.21011);
\draw [c] (8.45339,3.21011) -- (8.46821,3.21011);
\definecolor{c}{rgb}{0,0,0};
\colorlet{c}{kugray};
\draw [c] (8.48303,3.19033) -- (8.48303,3.22695);
\draw [c] (8.48303,3.22695) -- (8.48303,3.25968);
\draw [c] (8.46821,3.22695) -- (8.48303,3.22695);
\draw [c] (8.48303,3.22695) -- (8.49785,3.22695);
\definecolor{c}{rgb}{0,0,0};
\colorlet{c}{kugray};
\draw [c] (8.51267,3.1697) -- (8.51267,3.20663);
\draw [c] (8.51267,3.20663) -- (8.51267,3.2396);
\draw [c] (8.49785,3.20663) -- (8.51267,3.20663);
\draw [c] (8.51267,3.20663) -- (8.52748,3.20663);
\definecolor{c}{rgb}{0,0,0};
\colorlet{c}{kugray};
\draw [c] (8.5423,3.16479) -- (8.5423,3.20274);
\draw [c] (8.5423,3.20274) -- (8.5423,3.23651);
\draw [c] (8.52748,3.20274) -- (8.5423,3.20274);
\draw [c] (8.5423,3.20274) -- (8.55712,3.20274);
\definecolor{c}{rgb}{0,0,0};
\colorlet{c}{kugray};
\draw [c] (8.57194,3.12526) -- (8.57194,3.1667);
\draw [c] (8.57194,3.1667) -- (8.57194,3.20322);
\draw [c] (8.55712,3.1667) -- (8.57194,3.1667);
\draw [c] (8.57194,3.1667) -- (8.58675,3.1667);
\definecolor{c}{rgb}{0,0,0};
\colorlet{c}{kugray};
\draw [c] (8.60157,3.15802) -- (8.60157,3.19823);
\draw [c] (8.60157,3.19823) -- (8.60157,3.23378);
\draw [c] (8.58675,3.19823) -- (8.60157,3.19823);
\draw [c] (8.60157,3.19823) -- (8.61639,3.19823);
\definecolor{c}{rgb}{0,0,0};
\colorlet{c}{kugray};
\draw [c] (8.63121,3.10131) -- (8.63121,3.14082);
\draw [c] (8.63121,3.14082) -- (8.63121,3.17583);
\draw [c] (8.61639,3.14082) -- (8.63121,3.14082);
\draw [c] (8.63121,3.14082) -- (8.64603,3.14082);
\definecolor{c}{rgb}{0,0,0};
\colorlet{c}{kugray};
\draw [c] (8.66084,3.10207) -- (8.66084,3.14645);
\draw [c] (8.66084,3.14645) -- (8.66084,3.18523);
\draw [c] (8.64603,3.14645) -- (8.66084,3.14645);
\draw [c] (8.66084,3.14645) -- (8.67566,3.14645);
\definecolor{c}{rgb}{0,0,0};
\colorlet{c}{kugray};
\draw [c] (8.69048,3.15812) -- (8.69048,3.19621);
\draw [c] (8.69048,3.19621) -- (8.69048,3.2301);
\draw [c] (8.67566,3.19621) -- (8.69048,3.19621);
\draw [c] (8.69048,3.19621) -- (8.7053,3.19621);
\definecolor{c}{rgb}{0,0,0};
\colorlet{c}{kugray};
\draw [c] (8.72012,3.0956) -- (8.72012,3.14019);
\draw [c] (8.72012,3.14019) -- (8.72012,3.17912);
\draw [c] (8.7053,3.14019) -- (8.72012,3.14019);
\draw [c] (8.72012,3.14019) -- (8.73493,3.14019);
\definecolor{c}{rgb}{0,0,0};
\colorlet{c}{kugray};
\draw [c] (8.74975,3.08632) -- (8.74975,3.12806);
\draw [c] (8.74975,3.12806) -- (8.74975,3.1648);
\draw [c] (8.73493,3.12806) -- (8.74975,3.12806);
\draw [c] (8.74975,3.12806) -- (8.76457,3.12806);
\definecolor{c}{rgb}{0,0,0};
\colorlet{c}{kugray};
\draw [c] (8.77939,3.12605) -- (8.77939,3.16744);
\draw [c] (8.77939,3.16744) -- (8.77939,3.20392);
\draw [c] (8.76457,3.16744) -- (8.77939,3.16744);
\draw [c] (8.77939,3.16744) -- (8.79421,3.16744);
\definecolor{c}{rgb}{0,0,0};
\colorlet{c}{kugray};
\draw [c] (8.80902,3.19836) -- (8.80902,3.23362);
\draw [c] (8.80902,3.23362) -- (8.80902,3.26525);
\draw [c] (8.79421,3.23362) -- (8.80902,3.23362);
\draw [c] (8.80902,3.23362) -- (8.82384,3.23362);
\definecolor{c}{rgb}{0,0,0};
\colorlet{c}{kugray};
\draw [c] (8.83866,3.1022) -- (8.83866,3.14697);
\draw [c] (8.83866,3.14697) -- (8.83866,3.18604);
\draw [c] (8.82384,3.14697) -- (8.83866,3.14697);
\draw [c] (8.83866,3.14697) -- (8.85348,3.14697);
\definecolor{c}{rgb}{0,0,0};
\colorlet{c}{kugray};
\draw [c] (8.86829,3.20731) -- (8.86829,3.2428);
\draw [c] (8.86829,3.2428) -- (8.86829,3.27462);
\draw [c] (8.85348,3.2428) -- (8.86829,3.2428);
\draw [c] (8.86829,3.2428) -- (8.88311,3.2428);
\definecolor{c}{rgb}{0,0,0};
\colorlet{c}{kugray};
\draw [c] (8.89793,3.19795) -- (8.89793,3.23912);
\draw [c] (8.89793,3.23912) -- (8.89793,3.27543);
\draw [c] (8.88311,3.23912) -- (8.89793,3.23912);
\draw [c] (8.89793,3.23912) -- (8.91275,3.23912);
\definecolor{c}{rgb}{0,0,0};
\colorlet{c}{kugray};
\draw [c] (8.92757,3.07413) -- (8.92757,3.11815);
\draw [c] (8.92757,3.11815) -- (8.92757,3.15665);
\draw [c] (8.91275,3.11815) -- (8.92757,3.11815);
\draw [c] (8.92757,3.11815) -- (8.94238,3.11815);
\definecolor{c}{rgb}{0,0,0};
\colorlet{c}{kugray};
\draw [c] (8.9572,3.05408) -- (8.9572,3.10088);
\draw [c] (8.9572,3.10088) -- (8.9572,3.14149);
\draw [c] (8.94238,3.10088) -- (8.9572,3.10088);
\draw [c] (8.9572,3.10088) -- (8.97202,3.10088);
\definecolor{c}{rgb}{0,0,0};
\colorlet{c}{kugray};
\draw [c] (8.98684,3.16696) -- (8.98684,3.20827);
\draw [c] (8.98684,3.20827) -- (8.98684,3.24469);
\draw [c] (8.97202,3.20827) -- (8.98684,3.20827);
\draw [c] (8.98684,3.20827) -- (9.00166,3.20827);
\definecolor{c}{rgb}{0,0,0};
\colorlet{c}{kugray};
\draw [c] (9.01647,3.06066) -- (9.01647,3.10415);
\draw [c] (9.01647,3.10415) -- (9.01647,3.14225);
\draw [c] (9.00166,3.10415) -- (9.01647,3.10415);
\draw [c] (9.01647,3.10415) -- (9.03129,3.10415);
\definecolor{c}{rgb}{0,0,0};
\colorlet{c}{kugray};
\draw [c] (9.04611,3.03321) -- (9.04611,3.07793);
\draw [c] (9.04611,3.07793) -- (9.04611,3.11697);
\draw [c] (9.03129,3.07793) -- (9.04611,3.07793);
\draw [c] (9.04611,3.07793) -- (9.06093,3.07793);
\definecolor{c}{rgb}{0,0,0};
\colorlet{c}{kugray};
\draw [c] (9.07574,3.01544) -- (9.07574,3.0622);
\draw [c] (9.07574,3.0622) -- (9.07574,3.10279);
\draw [c] (9.06093,3.0622) -- (9.07574,3.0622);
\draw [c] (9.07574,3.0622) -- (9.09056,3.0622);
\definecolor{c}{rgb}{0,0,0};
\colorlet{c}{kugray};
\draw [c] (9.10538,3.11751) -- (9.10538,3.16063);
\draw [c] (9.10538,3.16063) -- (9.10538,3.19845);
\draw [c] (9.09056,3.16063) -- (9.10538,3.16063);
\draw [c] (9.10538,3.16063) -- (9.1202,3.16063);
\definecolor{c}{rgb}{0,0,0};
\colorlet{c}{kugray};
\draw [c] (9.13502,3.05438) -- (9.13502,3.10259);
\draw [c] (9.13502,3.10259) -- (9.13502,3.14427);
\draw [c] (9.1202,3.10259) -- (9.13502,3.10259);
\draw [c] (9.13502,3.10259) -- (9.14983,3.10259);
\definecolor{c}{rgb}{0,0,0};
\colorlet{c}{kugray};
\draw [c] (9.16465,3.01975) -- (9.16465,3.07036);
\draw [c] (9.16465,3.07036) -- (9.16465,3.11381);
\draw [c] (9.14983,3.07036) -- (9.16465,3.07036);
\draw [c] (9.16465,3.07036) -- (9.17947,3.07036);
\definecolor{c}{rgb}{0,0,0};
\colorlet{c}{kugray};
\draw [c] (9.19429,3.09316) -- (9.19429,3.13551);
\draw [c] (9.19429,3.13551) -- (9.19429,3.17274);
\draw [c] (9.17947,3.13551) -- (9.19429,3.13551);
\draw [c] (9.19429,3.13551) -- (9.20911,3.13551);
\definecolor{c}{rgb}{0,0,0};
\colorlet{c}{kugray};
\draw [c] (9.22392,3.06278) -- (9.22392,3.109);
\draw [c] (9.22392,3.109) -- (9.22392,3.14918);
\draw [c] (9.20911,3.109) -- (9.22392,3.109);
\draw [c] (9.22392,3.109) -- (9.23874,3.109);
\definecolor{c}{rgb}{0,0,0};
\colorlet{c}{kugray};
\draw [c] (9.25356,3.08447) -- (9.25356,3.12834);
\draw [c] (9.25356,3.12834) -- (9.25356,3.16674);
\draw [c] (9.23874,3.12834) -- (9.25356,3.12834);
\draw [c] (9.25356,3.12834) -- (9.26838,3.12834);
\definecolor{c}{rgb}{0,0,0};
\colorlet{c}{kugray};
\draw [c] (9.2832,2.99447) -- (9.2832,3.04509);
\draw [c] (9.2832,3.04509) -- (9.2832,3.08855);
\draw [c] (9.26838,3.04509) -- (9.2832,3.04509);
\draw [c] (9.2832,3.04509) -- (9.29801,3.04509);
\definecolor{c}{rgb}{0,0,0};
\colorlet{c}{kugray};
\draw [c] (9.31283,3.07942) -- (9.31283,3.12376);
\draw [c] (9.31283,3.12376) -- (9.31283,3.16252);
\draw [c] (9.29801,3.12376) -- (9.31283,3.12376);
\draw [c] (9.31283,3.12376) -- (9.32765,3.12376);
\definecolor{c}{rgb}{0,0,0};
\colorlet{c}{kugray};
\draw [c] (9.34247,2.98071) -- (9.34247,3.03135);
\draw [c] (9.34247,3.03135) -- (9.34247,3.07482);
\draw [c] (9.32765,3.03135) -- (9.34247,3.03135);
\draw [c] (9.34247,3.03135) -- (9.35728,3.03135);
\definecolor{c}{rgb}{0,0,0};
\colorlet{c}{kugray};
\draw [c] (9.3721,3.09749) -- (9.3721,3.13869);
\draw [c] (9.3721,3.13869) -- (9.3721,3.17501);
\draw [c] (9.35728,3.13869) -- (9.3721,3.13869);
\draw [c] (9.3721,3.13869) -- (9.38692,3.13869);
\definecolor{c}{rgb}{0,0,0};
\colorlet{c}{kugray};
\draw [c] (9.40174,3.0393) -- (9.40174,3.08577);
\draw [c] (9.40174,3.08577) -- (9.40174,3.12613);
\draw [c] (9.38692,3.08577) -- (9.40174,3.08577);
\draw [c] (9.40174,3.08577) -- (9.41656,3.08577);
\definecolor{c}{rgb}{0,0,0};
\colorlet{c}{kugray};
\draw [c] (9.43137,2.98225) -- (9.43137,3.03202);
\draw [c] (9.43137,3.03202) -- (9.43137,3.07485);
\draw [c] (9.41656,3.03202) -- (9.43137,3.03202);
\draw [c] (9.43137,3.03202) -- (9.44619,3.03202);
\definecolor{c}{rgb}{0,0,0};
\colorlet{c}{kugray};
\draw [c] (9.46101,3.05321) -- (9.46101,3.0968);
\draw [c] (9.46101,3.0968) -- (9.46101,3.13497);
\draw [c] (9.44619,3.0968) -- (9.46101,3.0968);
\draw [c] (9.46101,3.0968) -- (9.47583,3.0968);
\definecolor{c}{rgb}{0,0,0};
\colorlet{c}{kugray};
\draw [c] (9.49065,3.04863) -- (9.49065,3.09461);
\draw [c] (9.49065,3.09461) -- (9.49065,3.1346);
\draw [c] (9.47583,3.09461) -- (9.49065,3.09461);
\draw [c] (9.49065,3.09461) -- (9.50546,3.09461);
\definecolor{c}{rgb}{0,0,0};
\colorlet{c}{kugray};
\draw [c] (9.52028,3.06634) -- (9.52028,3.11124);
\draw [c] (9.52028,3.11124) -- (9.52028,3.15042);
\draw [c] (9.50546,3.11124) -- (9.52028,3.11124);
\draw [c] (9.52028,3.11124) -- (9.5351,3.11124);
\definecolor{c}{rgb}{0,0,0};
\colorlet{c}{kugray};
\draw [c] (9.54992,3.06028) -- (9.54992,3.1061);
\draw [c] (9.54992,3.1061) -- (9.54992,3.14597);
\draw [c] (9.5351,3.1061) -- (9.54992,3.1061);
\draw [c] (9.54992,3.1061) -- (9.56474,3.1061);
\definecolor{c}{rgb}{0,0,0};
\colorlet{c}{kugray};
\draw [c] (9.57955,2.92876) -- (9.57955,2.98234);
\draw [c] (9.57955,2.98234) -- (9.57955,3.02797);
\draw [c] (9.56474,2.98234) -- (9.57955,2.98234);
\draw [c] (9.57955,2.98234) -- (9.59437,2.98234);
\definecolor{c}{rgb}{0,0,0};
\colorlet{c}{kugray};
\draw [c] (9.60919,3.05453) -- (9.60919,3.10277);
\draw [c] (9.60919,3.10277) -- (9.60919,3.14446);
\draw [c] (9.59437,3.10277) -- (9.60919,3.10277);
\draw [c] (9.60919,3.10277) -- (9.62401,3.10277);
\definecolor{c}{rgb}{0,0,0};
\colorlet{c}{kugray};
\draw [c] (9.63882,3.06402) -- (9.63882,3.10887);
\draw [c] (9.63882,3.10887) -- (9.63882,3.14802);
\draw [c] (9.62401,3.10887) -- (9.63882,3.10887);
\draw [c] (9.63882,3.10887) -- (9.65364,3.10887);
\definecolor{c}{rgb}{0,0,0};
\colorlet{c}{kugray};
\draw [c] (9.66846,3.03821) -- (9.66846,3.08666);
\draw [c] (9.66846,3.08666) -- (9.66846,3.12851);
\draw [c] (9.65364,3.08666) -- (9.66846,3.08666);
\draw [c] (9.66846,3.08666) -- (9.68328,3.08666);
\definecolor{c}{rgb}{0,0,0};
\colorlet{c}{kugray};
\draw [c] (9.6981,3.05196) -- (9.6981,3.09725);
\draw [c] (9.6981,3.09725) -- (9.6981,3.13673);
\draw [c] (9.68328,3.09725) -- (9.6981,3.09725);
\draw [c] (9.6981,3.09725) -- (9.71291,3.09725);
\definecolor{c}{rgb}{0,0,0};
\colorlet{c}{kugray};
\draw [c] (9.72773,2.99179) -- (9.72773,3.04189);
\draw [c] (9.72773,3.04189) -- (9.72773,3.08497);
\draw [c] (9.71291,3.04189) -- (9.72773,3.04189);
\draw [c] (9.72773,3.04189) -- (9.74255,3.04189);
\definecolor{c}{rgb}{0,0,0};
\colorlet{c}{kugray};
\draw [c] (9.75737,2.97816) -- (9.75737,3.02871);
\draw [c] (9.75737,3.02871) -- (9.75737,3.07212);
\draw [c] (9.74255,3.02871) -- (9.75737,3.02871);
\draw [c] (9.75737,3.02871) -- (9.77219,3.02871);
\definecolor{c}{rgb}{0,0,0};
\colorlet{c}{kugray};
\draw [c] (9.787,3.07791) -- (9.787,3.12253);
\draw [c] (9.787,3.12253) -- (9.787,3.1615);
\draw [c] (9.77219,3.12253) -- (9.787,3.12253);
\draw [c] (9.787,3.12253) -- (9.80182,3.12253);
\definecolor{c}{rgb}{0,0,0};
\colorlet{c}{kugray};
\draw [c] (9.81664,2.90394) -- (9.81664,2.95902);
\draw [c] (9.81664,2.95902) -- (9.81664,3.00573);
\draw [c] (9.80182,2.95902) -- (9.81664,2.95902);
\draw [c] (9.81664,2.95902) -- (9.83146,2.95902);
\definecolor{c}{rgb}{0,0,0};
\colorlet{c}{kugray};
\draw [c] (9.84627,3.10832) -- (9.84627,3.15092);
\draw [c] (9.84627,3.15092) -- (9.84627,3.18834);
\draw [c] (9.83146,3.15092) -- (9.84627,3.15092);
\draw [c] (9.84627,3.15092) -- (9.86109,3.15092);
\definecolor{c}{rgb}{0,0,0};
\colorlet{c}{kugray};
\draw [c] (9.87591,2.95007) -- (9.87591,3.00342);
\draw [c] (9.87591,3.00342) -- (9.87591,3.04887);
\draw [c] (9.86109,3.00342) -- (9.87591,3.00342);
\draw [c] (9.87591,3.00342) -- (9.89073,3.00342);
\definecolor{c}{rgb}{0,0,0};
\colorlet{c}{kugray};
\draw [c] (9.90555,2.93927) -- (9.90555,2.99525);
\draw [c] (9.90555,2.99525) -- (9.90555,3.0426);
\draw [c] (9.89073,2.99525) -- (9.90555,2.99525);
\draw [c] (9.90555,2.99525) -- (9.92036,2.99525);
\definecolor{c}{rgb}{0,0,0};
\colorlet{c}{kugray};
\draw [c] (9.93518,2.97877) -- (9.93518,3.0325);
\draw [c] (9.93518,3.0325) -- (9.93518,3.07823);
\draw [c] (9.92036,3.0325) -- (9.93518,3.0325);
\draw [c] (9.93518,3.0325) -- (9.95,3.0325);
\definecolor{c}{rgb}{0,0,0};
\definecolor{c}{rgb}{1,0.8,0};
\draw [c] (1.51655,5.54612) -- (1.60131,5.40539) -- (1.68607,5.27646) -- (1.77083,5.15727) -- (1.85558,5.04631) -- (1.94034,4.9424) -- (2.0251,4.84466) -- (2.10986,4.75237) -- (2.19462,4.66498) -- (2.27938,4.58203) -- (2.36413,4.50316)
 -- (2.44889,4.42808) -- (2.53365,4.35654) -- (2.61841,4.28835) -- (2.70317,4.22336) -- (2.78793,4.16142) -- (2.87268,4.10244) -- (2.95744,4.04632) -- (3.0422,3.99299) -- (3.12696,3.94238) -- (3.21172,3.89444) -- (3.29648,3.84909) --
 (3.38123,3.80628) -- (3.46599,3.76596) -- (3.55075,3.72806) -- (3.63551,3.69251) -- (3.72027,3.65925) -- (3.80502,3.62819) -- (3.88978,3.59926) -- (3.97454,3.57235) -- (4.0593,3.5474) -- (4.14406,3.52429) -- (4.22882,3.50293) -- (4.31357,3.48322) --
 (4.39833,3.46507) -- (4.48309,3.44838) -- (4.56785,3.43305) -- (4.65261,3.41898) -- (4.73737,3.40609) -- (4.82212,3.39428) -- (4.90688,3.38348) -- (4.99164,3.37359) -- (5.0764,3.36455) -- (5.16116,3.35628) -- (5.24592,3.34871) -- (5.33067,3.34177)
 -- (5.41543,3.3354) -- (5.50019,3.32954) -- (5.58495,3.32415) -- (5.66971,3.31916);
\draw [c] (5.66971,3.31916) -- (5.75447,3.31453) -- (5.83922,3.31022) -- (5.92398,3.30618) -- (6.00874,3.30237) -- (6.0935,3.29876) -- (6.17826,3.29531) -- (6.26301,3.292) -- (6.34777,3.28878) -- (6.43253,3.28563) -- (6.51729,3.28253)
 -- (6.60205,3.27946) -- (6.68681,3.27638) -- (6.77156,3.27328) -- (6.85632,3.27014) -- (6.94108,3.26694) -- (7.02584,3.26367) -- (7.1106,3.2603) -- (7.19536,3.25682) -- (7.28011,3.25323) -- (7.36487,3.2495) -- (7.44963,3.24562) -- (7.53439,3.24159)
 -- (7.61915,3.23739) -- (7.70391,3.23302) -- (7.78866,3.22846) -- (7.87342,3.2237) -- (7.95818,3.21875) -- (8.04294,3.21358) -- (8.1277,3.20821) -- (8.21246,3.2026) -- (8.29721,3.19678) -- (8.38197,3.19071) -- (8.46673,3.18441) -- (8.55149,3.17787)
 -- (8.63625,3.17108) -- (8.72101,3.16405) -- (8.80576,3.15675) -- (8.89052,3.1492) -- (8.97528,3.14139) -- (9.06004,3.13331) -- (9.1448,3.12497) -- (9.22955,3.11636) -- (9.31431,3.10747) -- (9.39907,3.09831) -- (9.48383,3.08888) -- (9.56859,3.07917)
 -- (9.65335,3.06918) -- (9.7381,3.0589) -- (9.82286,3.04835);
\draw [c] (9.82286,3.04835) -- (9.90762,3.03751);
\end{tikzpicture}

\end{infilsf}
\caption{The functions fitted to each of the CalcHEP distributions with different $\Lambda$ values, which were produced in chapter~\ref{ch.mc}.}\label{simfit}
\end{figure}

[Without proper errors: Most likely $\Lambda$ is 1.37 TeV, lower limit is 1.04 TeV. The Standard Model survives.]

