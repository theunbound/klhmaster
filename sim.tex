\documentclass[a4paper,11pt,openany]{memoir}
\usepackage{amsmath}
%\usepackage[sc,osf]{mathpazo}
%\usepackage[garamond]{mathdesign}
\usepackage{fontspec}
\usepackage[math-style=TeX]{unicode-math}
\usepackage{xunicode}
\usepackage{xltxtra}
\usepackage{polyglossia}
\defaultfontfeatures{Mapping=tex-text}
\setmainfont[   Path=fonts/xits/,
                BoldFont={xits-bold.otf},
                ItalicFont={xits-italic.otf},
                BoldItalicFont={xits-bolditalic.otf},
                SmallCapsFont={../MasticSC-Regular.otf}
            ]{xits-regular.otf}
\setsansfont[   Path=fonts/frutiger/,
                Scale=MatchLowercase,
                BoldFont={FrutigerLTStd-Bold.otf},
                ItalicFont={FrutigerLTStd-Italic.otf},
                BoldItalicFont={FrutigerLTStd-BoldItalic.otf}
            ]{FrutigerLTStd-Roman.otf}
            
\setmathfont{xits-math.otf}
\setmathfont[version=roman,range=\mathcal,Path=fonts/]{latinmodern-math.otf}
%\setmathfont[range=\mathcal,Scale=MatchUppercase]{Lynda Cursive}
%\setmathfont[range=\mathup]{Garamond Premier Pro}
%\setmathfont[range=\mathit]{Garamond Premier Pro}
\setdefaultlanguage{danish}
\setotherlanguage[variant=british]{english}

\usepackage[usenames,dvipsnames,svgnames,table]{xcolor}
\usepackage{graphicx,epic,eepic}
\usepackage{tkz-graph,tkz-euclide}
\usetikzlibrary{calc,intersections,shapes.geometric,decorations.pathmorphing}
\usetikzlibrary{snakes,patterns,plotmarks,decorations.text}
\renewcommand*{\VertexSmallMinSize}{4pt}
\usepackage{tabulary}
\usepackage{pdfpages}
\usepackage{wrapfig}
\usepackage{pifont}
\usepackage{multirow}
\usepackage{colortbl}

%\renewcommand{\thefootnote}{\fnsymbol{footnote}}

\renewcommand\bibname{References}
\renewcommand{\(}{\begin{equation}}
\renewcommand{\)}{\end{equation}}
\newcommand{\vet}[1]{\underline{#1}}
\newcommand{\mtx}[1]{\underline{\underline{#1}}}
\newcommand{\ono}[1]{\frac{1}{#1}}
\newcommand{\half}{\frac{1}{2}}
\newcommand{\thehead}{}
\newcommand{\head}[1]{\renewcommand{\thehead}{#1}}
\newcommand{\theohead}{}
\newcommand{\ohead}[1]{\renewcommand{\theohead}{#1}}
\newcommand{\thetnote}{}
\newcommand{\tnote}[1]{\renewcommand{\thetnote}{#1}}
\newcommand{\bra}[1]{\left\langle #1 \right|}
\newcommand{\ket}[1]{\left| #1 \right\rangle}
\newcommand{\braket}[2]{\left\langle#1\middle|#2\right\rangle}
\newcommand{\obraket}[3]{\left\langle#1\middle|#2\middle|#3\right\rangle}
\newcommand{\emf}{\mathcal{E}}
\newcommand{\di}{\text{d}}
\newcommand{\atlas}{\textsc{atlas}}
\newcommand{\Atlas}{\textsc{Atlas}}

\newenvironment{infilsf}{
    \begin{sffamily}
    \setmathfont[range=\mathup/{num},
                 Scale=MatchLowercase,
                 Path=fonts/frutiger/,
                 ]{FrutigerLTStd-Roman.otf}
    \setmathfont[range=\mathit/{latin,Latin},
                 Scale=MatchLowercase,
                 Path=fonts/frutiger/,
                    ]{FrutigerLTStd-Italic.otf}
    \setmathfont[range=\mathit/{greek,Greek},
                 Scale=MatchLowercase,
                 Path=fonts/dejavu/,
                    ]{DejaVuSans-Oblique.ttf}
    \setmathfont[range=\mathup/{greek,Greek},
                 Scale=MatchLowercase,
                 Path=fonts/dejavu/,
                    ]{DejaVuSans.ttf}
%    \setmathfont[range=\text,
%                 Scale=MatchLowercase,
%                 Path=fonts/frutiger/,
%                    ]{FrutigerLTStd-Roman.otf}
}{
    \setmathfont{xits-math.otf}
    \setmathfont[range=\mathcal,Path=fonts/]{latinmodern-math.otf}
    \end{sffamily}
}

\newenvironment{new}{\color{Blue}}{}

\definecolor{kugray}{RGB}{102,102,102}
\definecolor{kugray1}{RGB}{52,52,52}
\definecolor{natgreen}{RGB}{70,116,60}
\definecolor{natgreen1}{HTML}{63875B}
\definecolor{natgreen2}{HTML}{859B81}
\definecolor{natcomp}{HTML}{86454C}
\definecolor{natcomp1}{HTML}{9D696E}
\definecolor{natcomp2}{HTML}{B39598}
\definecolor{natlcomp}{HTML}{5d323d}
\definecolor{natrcomp}{HTML}{52325d}
\definecolor{natyellow}{HTML}{8B6448}
\definecolor{natblue}{HTML}{524F81}
\definecolor{natscatb}{HTML}{3B8178}
\definecolor{natscatg}{HTML}{418B5C}
\definecolor{natscaty}{HTML}{7C8B41}
\definecolor{natscatr}{HTML}{81783C}

\makepagestyle{fancy}

\makepsmarks{fancy}{%
\nouppercaseheads
\createmark{chapter}{left}{nonumber}{}{}
\createmark{section}{right}{shownumber}{}{ \space}
\createplainmark{toc}{both}{\contentsname}
\createplainmark{lof}{both}{\listfigurename}
\createplainmark{lot}{both}{\listtablename}
\createplainmark{bib}{both}{\bibname}
\createplainmark{index}{both}{\indexname}
\createplainmark{glossary}{both}{\glossaryname}}

\makeoddhead{fancy}
   {}{}{}
\makeoddfoot{fancy}{\makebox[0pt][r]{\raisebox{15pt}[20pt]{\textcolor{natgreen}{\rule{1.1\spinemargin}{1pt}}}\makebox[0pt][l]{\raisebox{15pt}[20pt]{\textcolor{natgreen}{\rule{\paperwidth}{1pt}}}}}\textcolor{kugray}{\textsf{\rightmark}}}{}{\textcolor{kugray}{\textsf{\thepage}}}
\makeevenfoot{fancy}{\textcolor{kugray}{\textsf{\thepage}}}{}{\textcolor{kugray}{\textsf{\leftmark}}\makebox[0pt][l]{\raisebox{15pt}{\textcolor{natgreen}{\rule{1.1\spinemargin}{1pt}}}}\makebox[0pt][r]{\raisebox{15pt}{\textcolor{natgreen}{\rule{\paperwidth}{1pt}}}}}
\setlength{\footskip}{40pt}
\pagestyle{fancy}
\aliaspagestyle{chapter}{fancy}

\captionnamefont{\sffamily\color{natgreen}\bfseries} \captiontitlefont{\footnotesize} \captionstyle{\\}
\renewcommand*{\printchaptername}{}
\renewcommand*{\chapternamenum}{}
\renewcommand*{\afterchapternum}{}
\renewcommand{\chapnumfont}{\chaptitlefont\sffamily\HUGE}
\renewcommand{\printchapternum}{\chapnumfont \colorbox{natgreen}{\textcolor{white}{\hspace{.2em}\thechapter\hspace{.2em}}}\hspace{1em}}
\setsecheadstyle{\large\bfseries}
\setsubsecheadstyle{\bfseries}
\setsubsubsecheadstyle{}
\setsecnumformat{\textsf{\color{natgreen}\csname the#1\endcsname\quad}}
\maxsecnumdepth{subsubsection}
\renewcommand{\labelenumi}{\sffamily\bfseries\color{natgreen}\theenumi.}
\renewcommand{\labelitemi}{\color{natgreen}\ding{110}}
\renewcommand{\labelitemii}{\color{natgreen}\textbullet}
\setcounter{tocdepth}{2}

\setlength{\arrayrulewidth}{2pt}

\newsubfloat{figure}

\begin{hyphenrules}{danish}
\hyphenation{be-stem-mes}
\hyphenation{rest-klas-se-sæt-ning}
\end{hyphenrules}
\begin{hyphenrules}{english}
\hyphenation{Ham-il-ton}
\hyphenation{ATLAS}
\hyphenation{Atlas}
\hyphenation{atlas}
\hyphenation{CERN}
\hyphenation{i-den-ti-cal}
\hyphenation{pro-vid-ed}
\hyphenation{Calc-HEP}
\hyphenation{par-ticles}
\hyphenation{brems-strahl-ung}
\end{hyphenrules}
\usepackage[pdfusetitle]{hyperref}
\urlstyle{sf}

\begin{document}
\begin{english}

\chapter{Simulation studies}

We might imagine that, with the theory that describes the relevant physics in place, we are now able to simply calculate a theoretical prediction for the distributions of photons that we can expect to see. Unfortunately, the integrals involved do not have analytical solutions. In stead, we will use the Monte Carlo method for numerical integration to generate some output from our theoretical model. The Monte Carlo method functions by picking for each variable that is integrated over, a random value from a suitable distribution of initial values, and evaluating the expression being integrated for those values. Repeating this calculaiton enough times, and we expect to get a picture of the behaviour of the expression being integrated over the whole range of possible values of the integrated variables. The very large number of random numbers required apparently led those people who named the method to think of the very large number of dice being rolled on the gambling tables in Monte Carlo.

Since all of the variables that we integrate over have physical significance, we might also view choosing specific values for these variables as laying out the kinematics of a single hypothetical event. For this reason, the software designed to perform the Monte Carlo integration is known in particle physics as event generators.

\section{Event generators}

The event generator used for the bulk of the work in this thesis is CalcHEP \cite{calchep}, a tool choses for its familiarity to the author, and its compatibility with the Feynman rule creation software LanHEP, which will be discussed later in this chapter. Other choices of event generators include MadGraph \cite{madgraph5} and pythia \cite{pythia}. To illustrate the behaviour of these different generators with respect to one another, figure~\ref{evgen} plots distributions of invariant mass ($M_{\gamma\gamma}$) and transverse momentum ($p_T$).

\begin{figure}[hbt]
\begin{minipage}[t]{.69\textwidth}
\begin{infilsf}
\hspace{-.9cm}\pgfdeclareplotmark{cross} {
\pgfpathmoveto{\pgfpoint{-0.3\pgfplotmarksize}{\pgfplotmarksize}}
\pgfpathlineto{\pgfpoint{+0.3\pgfplotmarksize}{\pgfplotmarksize}}
\pgfpathlineto{\pgfpoint{+0.3\pgfplotmarksize}{0.3\pgfplotmarksize}}
\pgfpathlineto{\pgfpoint{+1\pgfplotmarksize}{0.3\pgfplotmarksize}}
\pgfpathlineto{\pgfpoint{+1\pgfplotmarksize}{-0.3\pgfplotmarksize}}
\pgfpathlineto{\pgfpoint{+0.3\pgfplotmarksize}{-0.3\pgfplotmarksize}}
\pgfpathlineto{\pgfpoint{+0.3\pgfplotmarksize}{-1.\pgfplotmarksize}}
\pgfpathlineto{\pgfpoint{-0.3\pgfplotmarksize}{-1.\pgfplotmarksize}}
\pgfpathlineto{\pgfpoint{-0.3\pgfplotmarksize}{-0.3\pgfplotmarksize}}
\pgfpathlineto{\pgfpoint{-1.\pgfplotmarksize}{-0.3\pgfplotmarksize}}
\pgfpathlineto{\pgfpoint{-1.\pgfplotmarksize}{0.3\pgfplotmarksize}}
\pgfpathlineto{\pgfpoint{-0.3\pgfplotmarksize}{0.3\pgfplotmarksize}}
\pgfpathclose
\pgfusepathqstroke
}
\pgfdeclareplotmark{cross*} {
\pgfpathmoveto{\pgfpoint{-0.3\pgfplotmarksize}{\pgfplotmarksize}}
\pgfpathlineto{\pgfpoint{+0.3\pgfplotmarksize}{\pgfplotmarksize}}
\pgfpathlineto{\pgfpoint{+0.3\pgfplotmarksize}{0.3\pgfplotmarksize}}
\pgfpathlineto{\pgfpoint{+1\pgfplotmarksize}{0.3\pgfplotmarksize}}
\pgfpathlineto{\pgfpoint{+1\pgfplotmarksize}{-0.3\pgfplotmarksize}}
\pgfpathlineto{\pgfpoint{+0.3\pgfplotmarksize}{-0.3\pgfplotmarksize}}
\pgfpathlineto{\pgfpoint{+0.3\pgfplotmarksize}{-1.\pgfplotmarksize}}
\pgfpathlineto{\pgfpoint{-0.3\pgfplotmarksize}{-1.\pgfplotmarksize}}
\pgfpathlineto{\pgfpoint{-0.3\pgfplotmarksize}{-0.3\pgfplotmarksize}}
\pgfpathlineto{\pgfpoint{-1.\pgfplotmarksize}{-0.3\pgfplotmarksize}}
\pgfpathlineto{\pgfpoint{-1.\pgfplotmarksize}{0.3\pgfplotmarksize}}
\pgfpathlineto{\pgfpoint{-0.3\pgfplotmarksize}{0.3\pgfplotmarksize}}
\pgfpathclose
\pgfusepathqfillstroke
}
\pgfdeclareplotmark{newstar} {
\pgfpathmoveto{\pgfqpoint{0pt}{\pgfplotmarksize}}
\pgfpathlineto{\pgfqpointpolar{44}{0.5\pgfplotmarksize}}
\pgfpathlineto{\pgfqpointpolar{18}{\pgfplotmarksize}}
\pgfpathlineto{\pgfqpointpolar{-20}{0.5\pgfplotmarksize}}
\pgfpathlineto{\pgfqpointpolar{-54}{\pgfplotmarksize}}
\pgfpathlineto{\pgfqpointpolar{-90}{0.5\pgfplotmarksize}}
\pgfpathlineto{\pgfqpointpolar{234}{\pgfplotmarksize}}
\pgfpathlineto{\pgfqpointpolar{198}{0.5\pgfplotmarksize}}
\pgfpathlineto{\pgfqpointpolar{162}{\pgfplotmarksize}}
\pgfpathlineto{\pgfqpointpolar{134}{0.5\pgfplotmarksize}}
\pgfpathclose
\pgfusepathqstroke
}
\pgfdeclareplotmark{newstar*} {
\pgfpathmoveto{\pgfqpoint{0pt}{\pgfplotmarksize}}
\pgfpathlineto{\pgfqpointpolar{44}{0.5\pgfplotmarksize}}
\pgfpathlineto{\pgfqpointpolar{18}{\pgfplotmarksize}}
\pgfpathlineto{\pgfqpointpolar{-20}{0.5\pgfplotmarksize}}
\pgfpathlineto{\pgfqpointpolar{-54}{\pgfplotmarksize}}
\pgfpathlineto{\pgfqpointpolar{-90}{0.5\pgfplotmarksize}}
\pgfpathlineto{\pgfqpointpolar{234}{\pgfplotmarksize}}
\pgfpathlineto{\pgfqpointpolar{198}{0.5\pgfplotmarksize}}
\pgfpathlineto{\pgfqpointpolar{162}{\pgfplotmarksize}}
\pgfpathlineto{\pgfqpointpolar{134}{0.5\pgfplotmarksize}}
\pgfpathclose
\pgfusepathqfillstroke
}
\begin{tikzpicture}[scale=.0039\textwidth]

\definecolor{c}{rgb}{0,0,0};
\draw [c] (1,0.610632) -- (1,5.49569) -- (9,5.49569) -- (9,0.610632) -- (1,0.610632);

\definecolor{c}{rgb}{0,0,0};
\draw [c] (1,0.610632) -- (1,5.49569) -- (9,5.49569) -- (9,0.610632) -- (1,0.610632);
\definecolor{c}{named}{natgreen};
\draw [c] (1,5.3381) -- (1.16,5.3381) -- (1.16,4.89777) -- (1.32,4.89777) -- (1.32,4.59361) -- (1.48,4.59361) -- (1.48,4.35509) -- (1.64,4.35509) -- (1.64,4.17463) -- (1.8,4.17463) -- (1.8,3.99569) -- (1.96,3.99569) -- (1.96,3.89679) --
 (2.12,3.89679) -- (2.12,3.71253) -- (2.28,3.71253) -- (2.28,3.56759) -- (2.44,3.56759) -- (2.44,3.47067) -- (2.6,3.47067) -- (2.6,3.37069) -- (2.76,3.37069) -- (2.76,3.31082) -- (2.92,3.31082) -- (2.92,3.18212) -- (3.08,3.18212) -- (3.08,3.09027) --
 (3.24,3.09027) -- (3.24,3.00051) -- (3.4,3.00051) -- (3.4,2.90655) -- (3.56,2.90655) -- (3.56,2.82479) -- (3.72,2.82479) -- (3.72,2.73987) -- (3.88,2.73987) -- (3.88,2.66623) -- (4.04,2.66623) -- (4.04,2.5872) -- (4.2,2.5872) -- (4.2,2.51021) --
 (4.36,2.51021) -- (4.36,2.44973) -- (4.52,2.44973) -- (4.52,2.37526) -- (4.68,2.37526) -- (4.68,2.28692) -- (4.84,2.28692) -- (4.84,2.17968) -- (5,2.17968) -- (5,2.07927) -- (5.16,2.07927) -- (5.16,2.01381) -- (5.32,2.01381) -- (5.32,1.91215) --
 (5.48,1.91215) -- (5.48,1.91959) -- (5.64,1.91959) -- (5.64,1.81091) -- (5.8,1.81091) -- (5.8,1.73096) -- (5.96,1.73096) -- (5.96,1.65826) -- (6.12,1.65826) -- (6.12,1.56522) -- (6.28,1.56522) -- (6.28,1.46524) -- (6.44,1.46524) -- (6.44,1.46524) --
 (6.6,1.46524) -- (6.6,1.3943) -- (6.76,1.3943) -- (6.76,1.34934) -- (6.92,1.34934) -- (6.92,1.32336) -- (7.08,1.32336) -- (7.08,1.32336) -- (7.24,1.32336) -- (7.24,0.952476) -- (7.4,0.952476) -- (7.4,1.37285) -- (7.56,1.37285) -- (7.56,1.05246) --
 (7.72,1.05246) -- (7.72,0.610632) -- (7.88,0.610632) -- (7.88,0.781554) -- (8.04,0.781554) -- (8.04,0.781554) -- (8.2,0.781554) -- (8.2,0.952476) -- (8.36,0.952476) -- (8.36,0.610632) -- (8.52,0.610632) -- (8.52,0.781554) -- (8.68,0.781554) --
 (8.68,0.781554) -- (8.84,0.781554) -- (8.84,0.781554) -- (9,0.781554);
\definecolor{c}{rgb}{0,0,0};
\draw [c] (1,0.610632) -- (9,0.610632);
\draw [anchor= east] (9,0.268678) node[scale=0.474669, rotate=0]{$M_{\gamma\gamma} \,[\text{GeV}]$};
\draw [c] (1.82051,0.757184) -- (1.82051,0.610632);
\draw [c] (2.02564,0.683908) -- (2.02564,0.610632);
\draw [c] (2.23077,0.683908) -- (2.23077,0.610632);
\draw [c] (2.4359,0.683908) -- (2.4359,0.610632);
\draw [c] (2.64103,0.683908) -- (2.64103,0.610632);
\draw [c] (2.84615,0.757184) -- (2.84615,0.610632);
\draw [c] (3.05128,0.683908) -- (3.05128,0.610632);
\draw [c] (3.25641,0.683908) -- (3.25641,0.610632);
\draw [c] (3.46154,0.683908) -- (3.46154,0.610632);
\draw [c] (3.66667,0.683908) -- (3.66667,0.610632);
\draw [c] (3.87179,0.757184) -- (3.87179,0.610632);
\draw [c] (4.07692,0.683908) -- (4.07692,0.610632);
\draw [c] (4.28205,0.683908) -- (4.28205,0.610632);
\draw [c] (4.48718,0.683908) -- (4.48718,0.610632);
\draw [c] (4.69231,0.683908) -- (4.69231,0.610632);
\draw [c] (4.89744,0.757184) -- (4.89744,0.610632);
\draw [c] (5.10256,0.683908) -- (5.10256,0.610632);
\draw [c] (5.30769,0.683908) -- (5.30769,0.610632);
\draw [c] (5.51282,0.683908) -- (5.51282,0.610632);
\draw [c] (5.71795,0.683908) -- (5.71795,0.610632);
\draw [c] (5.92308,0.757184) -- (5.92308,0.610632);
\draw [c] (6.12821,0.683908) -- (6.12821,0.610632);
\draw [c] (6.33333,0.683908) -- (6.33333,0.610632);
\draw [c] (6.53846,0.683908) -- (6.53846,0.610632);
\draw [c] (6.74359,0.683908) -- (6.74359,0.610632);
\draw [c] (6.94872,0.757184) -- (6.94872,0.610632);
\draw [c] (7.15385,0.683908) -- (7.15385,0.610632);
\draw [c] (7.35897,0.683908) -- (7.35897,0.610632);
\draw [c] (7.5641,0.683908) -- (7.5641,0.610632);
\draw [c] (7.76923,0.683908) -- (7.76923,0.610632);
\draw [c] (7.97436,0.757184) -- (7.97436,0.610632);
\draw [c] (8.17949,0.683908) -- (8.17949,0.610632);
\draw [c] (8.38461,0.683908) -- (8.38461,0.610632);
\draw [c] (8.58974,0.683908) -- (8.58974,0.610632);
\draw [c] (8.79487,0.683908) -- (8.79487,0.610632);
\draw [c] (9,0.757184) -- (9,0.610632);
\draw [c] (1.82051,0.757184) -- (1.82051,0.610632);
\draw [c] (1.61538,0.683908) -- (1.61538,0.610632);
\draw [c] (1.41026,0.683908) -- (1.41026,0.610632);
\draw [c] (1.20513,0.683908) -- (1.20513,0.610632);
\draw [c] (1,0.683908) -- (1,0.610632);
\draw [anchor=base] (1.82051,0.409124) node[scale=0.474669, rotate=0]{500};
\draw [anchor=base] (2.84615,0.409124) node[scale=0.474669, rotate=0]{1000};
\draw [anchor=base] (3.87179,0.409124) node[scale=0.474669, rotate=0]{1500};
\draw [anchor=base] (4.89744,0.409124) node[scale=0.474669, rotate=0]{2000};
\draw [anchor=base] (5.92308,0.409124) node[scale=0.474669, rotate=0]{2500};
\draw [anchor=base] (6.94872,0.409124) node[scale=0.474669, rotate=0]{3000};
\draw [anchor=base] (7.97436,0.409124) node[scale=0.474669, rotate=0]{3500};
\draw [anchor=base] (9,0.409124) node[scale=0.474669, rotate=0]{4000};
\draw [c] (1,0.610632) -- (1,5.49569);
\draw [anchor= east] (0.384,5.49569) node[scale=0.474669, rotate=90]{$\di\sigma/\di m_{\gamma\gamma}\, [\text{pb}/\text{GeV}]$};
\draw [c] (1.12,0.727714) -- (1,0.727714);
\draw [c] (1.12,0.827697) -- (1,0.827697);
\draw [c] (1.12,0.898636) -- (1,0.898636);
\draw [c] (1.12,0.953661) -- (1,0.953661);
\draw [c] (1.12,0.998619) -- (1,0.998619);
\draw [c] (1.12,1.03663) -- (1,1.03663);
\draw [c] (1.12,1.06956) -- (1,1.06956);
\draw [c] (1.12,1.0986) -- (1,1.0986);
\draw [c] (1.24,1.12458) -- (1,1.12458);
\draw [anchor= east] (0.922,1.12458) node[scale=0.474669, rotate=0]{$10^{-9}$};
\draw [c] (1.12,1.2955) -- (1,1.2955);
\draw [c] (1.12,1.39549) -- (1,1.39549);
\draw [c] (1.12,1.46643) -- (1,1.46643);
\draw [c] (1.12,1.52145) -- (1,1.52145);
\draw [c] (1.12,1.56641) -- (1,1.56641);
\draw [c] (1.12,1.60442) -- (1,1.60442);
\draw [c] (1.12,1.63735) -- (1,1.63735);
\draw [c] (1.12,1.66639) -- (1,1.66639);
\draw [c] (1.24,1.69237) -- (1,1.69237);
\draw [anchor= east] (0.922,1.69237) node[scale=0.474669, rotate=0]{$10^{-8}$};
\draw [c] (1.12,1.86329) -- (1,1.86329);
\draw [c] (1.12,1.96328) -- (1,1.96328);
\draw [c] (1.12,2.03422) -- (1,2.03422);
\draw [c] (1.12,2.08924) -- (1,2.08924);
\draw [c] (1.12,2.1342) -- (1,2.1342);
\draw [c] (1.12,2.17221) -- (1,2.17221);
\draw [c] (1.12,2.20514) -- (1,2.20514);
\draw [c] (1.12,2.23418) -- (1,2.23418);
\draw [c] (1.24,2.26016) -- (1,2.26016);
\draw [anchor= east] (0.922,2.26016) node[scale=0.474669, rotate=0]{$10^{-7}$};
\draw [c] (1.12,2.43109) -- (1,2.43109);
\draw [c] (1.12,2.53107) -- (1,2.53107);
\draw [c] (1.12,2.60201) -- (1,2.60201);
\draw [c] (1.12,2.65703) -- (1,2.65703);
\draw [c] (1.12,2.70199) -- (1,2.70199);
\draw [c] (1.12,2.74) -- (1,2.74);
\draw [c] (1.12,2.77293) -- (1,2.77293);
\draw [c] (1.12,2.80197) -- (1,2.80197);
\draw [c] (1.24,2.82795) -- (1,2.82795);
\draw [anchor= east] (0.922,2.82795) node[scale=0.474669, rotate=0]{$10^{-6}$};
\draw [c] (1.12,2.99888) -- (1,2.99888);
\draw [c] (1.12,3.09886) -- (1,3.09886);
\draw [c] (1.12,3.1698) -- (1,3.1698);
\draw [c] (1.12,3.22482) -- (1,3.22482);
\draw [c] (1.12,3.26978) -- (1,3.26978);
\draw [c] (1.12,3.30779) -- (1,3.30779);
\draw [c] (1.12,3.34072) -- (1,3.34072);
\draw [c] (1.12,3.36976) -- (1,3.36976);
\draw [c] (1.24,3.39574) -- (1,3.39574);
\draw [anchor= east] (0.922,3.39574) node[scale=0.474669, rotate=0]{$10^{-5}$};
\draw [c] (1.12,3.56667) -- (1,3.56667);
\draw [c] (1.12,3.66665) -- (1,3.66665);
\draw [c] (1.12,3.73759) -- (1,3.73759);
\draw [c] (1.12,3.79261) -- (1,3.79261);
\draw [c] (1.12,3.83757) -- (1,3.83757);
\draw [c] (1.12,3.87558) -- (1,3.87558);
\draw [c] (1.12,3.90851) -- (1,3.90851);
\draw [c] (1.12,3.93755) -- (1,3.93755);
\draw [c] (1.24,3.96353) -- (1,3.96353);
\draw [anchor= east] (0.922,3.96353) node[scale=0.474669, rotate=0]{$10^{-4}$};
\draw [c] (1.12,4.13446) -- (1,4.13446);
\draw [c] (1.12,4.23444) -- (1,4.23444);
\draw [c] (1.12,4.30538) -- (1,4.30538);
\draw [c] (1.12,4.3604) -- (1,4.3604);
\draw [c] (1.12,4.40536) -- (1,4.40536);
\draw [c] (1.12,4.44337) -- (1,4.44337);
\draw [c] (1.12,4.4763) -- (1,4.4763);
\draw [c] (1.12,4.50534) -- (1,4.50534);
\draw [c] (1.24,4.53132) -- (1,4.53132);
\draw [anchor= east] (0.922,4.53132) node[scale=0.474669, rotate=0]{$10^{-3}$};
\draw [c] (1.12,4.70225) -- (1,4.70225);
\draw [c] (1.12,4.80223) -- (1,4.80223);
\draw [c] (1.12,4.87317) -- (1,4.87317);
\draw [c] (1.12,4.92819) -- (1,4.92819);
\draw [c] (1.12,4.97315) -- (1,4.97315);
\draw [c] (1.12,5.01116) -- (1,5.01116);
\draw [c] (1.12,5.04409) -- (1,5.04409);
\draw [c] (1.12,5.07313) -- (1,5.07313);
\draw [c] (1.24,5.09911) -- (1,5.09911);
\draw [anchor= east] (0.922,5.09911) node[scale=0.474669, rotate=0]{$10^{-2}$};
\draw [c] (1.12,5.27004) -- (1,5.27004);
\draw [c] (1.12,5.37002) -- (1,5.37002);
\draw [c] (1.12,5.44096) -- (1,5.44096);
\definecolor{c}{named}{natblue};
\draw [c] (1,5.28003) -- (1.16,5.28003) -- (1.16,4.85396) -- (1.32,4.85396) -- (1.32,4.54811) -- (1.48,4.54811) -- (1.48,4.32462) -- (1.64,4.32462) -- (1.64,4.12283) -- (1.8,4.12283) -- (1.8,3.93771) -- (1.96,3.93771) -- (1.96,3.82041) --
 (2.12,3.82041) -- (2.12,3.64948) -- (2.28,3.64948) -- (2.28,3.53804) -- (2.44,3.53804) -- (2.44,3.47405) -- (2.6,3.47405) -- (2.6,3.17412) -- (2.76,3.17412) -- (2.76,3.14322) -- (2.92,3.14322) -- (2.92,3.13214) -- (3.08,3.13214) -- (3.08,3.03752) --
 (3.24,3.03752) -- (3.24,2.94593) -- (3.4,2.94593) -- (3.4,2.86012) -- (3.56,2.86012) -- (3.56,2.78268) -- (3.72,2.78268) -- (3.72,2.69347) -- (3.88,2.69347) -- (3.88,2.62518) -- (4.04,2.62518) -- (4.04,2.54426) -- (4.2,2.54426) -- (4.2,2.45863) --
 (4.36,2.45863) -- (4.36,2.38175) -- (4.52,2.38175) -- (4.52,2.31368) -- (4.68,2.31368) -- (4.68,2.21162) -- (4.84,2.21162) -- (4.84,2.17621) -- (5,2.17621) -- (5,2.12276) -- (5.16,2.12276) -- (5.16,2.05145) -- (5.32,2.05145) -- (5.32,1.98051) --
 (5.48,1.98051) -- (5.48,1.85853) -- (5.64,1.85853) -- (5.64,1.81159) -- (5.8,1.81159) -- (5.8,1.69087) -- (5.96,1.69087) -- (5.96,1.6066) -- (6.12,1.6066) -- (6.12,1.58762) -- (6.28,1.58762) -- (6.28,1.5326) -- (6.44,1.5326) -- (6.44,1.54463) --
 (6.6,1.54463) -- (6.6,1.46166) -- (6.76,1.46166) -- (6.76,1.30665) -- (6.92,1.30665) -- (6.92,1.38518) -- (7.08,1.38518) -- (7.08,1.06479) -- (7.24,1.06479) -- (7.24,1.19075) -- (7.4,1.19075) -- (7.4,0.793883) -- (7.56,0.793883) -- (7.56,1.06479) --
 (7.72,1.06479) -- (7.72,1.06479) -- (7.88,1.06479) -- (7.88,0.610632) -- (8.04,0.610632) -- (8.04,0.964805) -- (8.2,0.964805) -- (8.2,0.610632) -- (8.36,0.610632) -- (8.36,0.610632) -- (8.52,0.610632) -- (8.52,0.610632) -- (8.68,0.610632) --
 (8.68,0.610632) -- (8.84,0.610632) -- (8.84,0.610632) -- (9,0.610632);
\definecolor{c}{named}{natyellow};
\draw [c] (1,0.610632) -- (1.16,0.610632) -- (1.16,0.610632) -- (1.32,0.610632) -- (1.32,0.610632) -- (1.48,0.610632) -- (1.48,3.28315) -- (1.64,3.28315) -- (1.64,3.89606) -- (1.8,3.89606) -- (1.8,3.84849) -- (1.96,3.84849) -- (1.96,3.75283) --
 (2.12,3.75283) -- (2.12,3.64681) -- (2.28,3.64681) -- (2.28,3.54629) -- (2.44,3.54629) -- (2.44,3.44218) -- (2.6,3.44218) -- (2.6,3.34567) -- (2.76,3.34567) -- (2.76,3.2536) -- (2.92,3.2536) -- (2.92,3.15564) -- (3.08,3.15564) -- (3.08,3.06209) --
 (3.24,3.06209) -- (3.24,2.98331) -- (3.4,2.98331) -- (3.4,2.89045) -- (3.56,2.89045) -- (3.56,2.80078) -- (3.72,2.80078) -- (3.72,2.72765) -- (3.88,2.72765) -- (3.88,2.66419) -- (4.04,2.66419) -- (4.04,2.5807) -- (4.2,2.5807) -- (4.2,2.49171) --
 (4.36,2.49171) -- (4.36,2.3979) -- (4.52,2.3979) -- (4.52,2.37394) -- (4.68,2.37394) -- (4.68,2.2446) -- (4.84,2.2446) -- (4.84,2.19098) -- (5,2.19098) -- (5,2.14883) -- (5.16,2.14883) -- (5.16,2.08203) -- (5.32,2.08203) -- (5.32,1.90557) --
 (5.48,1.90557) -- (5.48,1.88207) -- (5.64,1.88207) -- (5.64,1.81113) -- (5.8,1.81113) -- (5.8,1.65612) -- (5.96,1.65612) -- (5.96,1.68516) -- (6.12,1.68516) -- (6.12,1.68516) -- (6.28,1.68516) -- (6.28,1.41426) -- (6.44,1.41426) -- (6.44,1.41426) --
 (6.6,1.41426) -- (6.6,1.14335) -- (6.76,1.14335) -- (6.76,1.31428) -- (6.92,1.31428) -- (6.92,1.54022) -- (7.08,1.54022) -- (7.08,0.610632) -- (7.24,0.610632) -- (7.24,0.610632) -- (7.4,0.610632) -- (7.4,0.610632) -- (7.56,0.610632) --
 (7.56,0.610632) -- (7.72,0.610632) -- (7.72,1.14335) -- (7.88,1.14335) -- (7.88,0.610632) -- (8.04,0.610632) -- (8.04,0.610632) -- (8.2,0.610632) -- (8.2,0.610632) -- (8.36,0.610632) -- (8.36,1.14335) -- (8.52,1.14335) -- (8.52,0.610632) --
 (8.68,0.610632) -- (8.68,0.610632) -- (8.84,0.610632) -- (8.84,0.610632) -- (9,0.610632);

\draw [anchor=base west] (6.9217,4.72234) node[scale=0.478657, rotate=0]{CalcHEP};
\definecolor{c}{named}{natgreen};
\draw [c] (6.32633,4.7773) -- (6.81663,4.7773);
\draw [anchor=base west] (6.9217,4.47809) node[scale=0.478657, rotate=0]{pythia 8};
\definecolor{c}{named}{natblue};
\draw [c] (6.32633,4.53305) -- (6.81663,4.53305);
\draw [anchor=base west] (6.9217,4.23384) node[scale=0.478657, rotate=0]{MadGraph};
\definecolor{c}{named}{natyellow};
\draw [c] (6.32633,4.28879) -- (6.81663,4.28879);
\definecolor{c}{rgb}{0,0,0};
\draw [c] (1,0.610632) -- (9,0.610632);
\draw [c] (1.82051,0.757184) -- (1.82051,0.610632);
\draw [c] (2.02564,0.683908) -- (2.02564,0.610632);
\draw [c] (2.23077,0.683908) -- (2.23077,0.610632);
\draw [c] (2.4359,0.683908) -- (2.4359,0.610632);
\draw [c] (2.64103,0.683908) -- (2.64103,0.610632);
\draw [c] (2.84615,0.757184) -- (2.84615,0.610632);
\draw [c] (3.05128,0.683908) -- (3.05128,0.610632);
\draw [c] (3.25641,0.683908) -- (3.25641,0.610632);
\draw [c] (3.46154,0.683908) -- (3.46154,0.610632);
\draw [c] (3.66667,0.683908) -- (3.66667,0.610632);
\draw [c] (3.87179,0.757184) -- (3.87179,0.610632);
\draw [c] (4.07692,0.683908) -- (4.07692,0.610632);
\draw [c] (4.28205,0.683908) -- (4.28205,0.610632);
\draw [c] (4.48718,0.683908) -- (4.48718,0.610632);
\draw [c] (4.69231,0.683908) -- (4.69231,0.610632);
\draw [c] (4.89744,0.757184) -- (4.89744,0.610632);
\draw [c] (5.10256,0.683908) -- (5.10256,0.610632);
\draw [c] (5.30769,0.683908) -- (5.30769,0.610632);
\draw [c] (5.51282,0.683908) -- (5.51282,0.610632);
\draw [c] (5.71795,0.683908) -- (5.71795,0.610632);
\draw [c] (5.92308,0.757184) -- (5.92308,0.610632);
\draw [c] (6.12821,0.683908) -- (6.12821,0.610632);
\draw [c] (6.33333,0.683908) -- (6.33333,0.610632);
\draw [c] (6.53846,0.683908) -- (6.53846,0.610632);
\draw [c] (6.74359,0.683908) -- (6.74359,0.610632);
\draw [c] (6.94872,0.757184) -- (6.94872,0.610632);
\draw [c] (7.15385,0.683908) -- (7.15385,0.610632);
\draw [c] (7.35897,0.683908) -- (7.35897,0.610632);
\draw [c] (7.5641,0.683908) -- (7.5641,0.610632);
\draw [c] (7.76923,0.683908) -- (7.76923,0.610632);
\draw [c] (7.97436,0.757184) -- (7.97436,0.610632);
\draw [c] (8.17949,0.683908) -- (8.17949,0.610632);
\draw [c] (8.38461,0.683908) -- (8.38461,0.610632);
\draw [c] (8.58974,0.683908) -- (8.58974,0.610632);
\draw [c] (8.79487,0.683908) -- (8.79487,0.610632);
\draw [c] (9,0.757184) -- (9,0.610632);
\draw [c] (1.82051,0.757184) -- (1.82051,0.610632);
\draw [c] (1.61538,0.683908) -- (1.61538,0.610632);
\draw [c] (1.41026,0.683908) -- (1.41026,0.610632);
\draw [c] (1.20513,0.683908) -- (1.20513,0.610632);
\draw [c] (1,0.683908) -- (1,0.610632);
\draw [c] (1,0.610632) -- (1,5.49569);
\draw [c] (1.12,0.727714) -- (1,0.727714);
\draw [c] (1.12,0.827697) -- (1,0.827697);
\draw [c] (1.12,0.898636) -- (1,0.898636);
\draw [c] (1.12,0.953661) -- (1,0.953661);
\draw [c] (1.12,0.998619) -- (1,0.998619);
\draw [c] (1.12,1.03663) -- (1,1.03663);
\draw [c] (1.12,1.06956) -- (1,1.06956);
\draw [c] (1.12,1.0986) -- (1,1.0986);
\draw [c] (1.24,1.12458) -- (1,1.12458);
\draw [c] (1.12,1.2955) -- (1,1.2955);
\draw [c] (1.12,1.39549) -- (1,1.39549);
\draw [c] (1.12,1.46643) -- (1,1.46643);
\draw [c] (1.12,1.52145) -- (1,1.52145);
\draw [c] (1.12,1.56641) -- (1,1.56641);
\draw [c] (1.12,1.60442) -- (1,1.60442);
\draw [c] (1.12,1.63735) -- (1,1.63735);
\draw [c] (1.12,1.66639) -- (1,1.66639);
\draw [c] (1.24,1.69237) -- (1,1.69237);
\draw [c] (1.12,1.86329) -- (1,1.86329);
\draw [c] (1.12,1.96328) -- (1,1.96328);
\draw [c] (1.12,2.03422) -- (1,2.03422);
\draw [c] (1.12,2.08924) -- (1,2.08924);
\draw [c] (1.12,2.1342) -- (1,2.1342);
\draw [c] (1.12,2.17221) -- (1,2.17221);
\draw [c] (1.12,2.20514) -- (1,2.20514);
\draw [c] (1.12,2.23418) -- (1,2.23418);
\draw [c] (1.24,2.26016) -- (1,2.26016);
\draw [c] (1.12,2.43109) -- (1,2.43109);
\draw [c] (1.12,2.53107) -- (1,2.53107);
\draw [c] (1.12,2.60201) -- (1,2.60201);
\draw [c] (1.12,2.65703) -- (1,2.65703);
\draw [c] (1.12,2.70199) -- (1,2.70199);
\draw [c] (1.12,2.74) -- (1,2.74);
\draw [c] (1.12,2.77293) -- (1,2.77293);
\draw [c] (1.12,2.80197) -- (1,2.80197);
\draw [c] (1.24,2.82795) -- (1,2.82795);
\draw [c] (1.12,2.99888) -- (1,2.99888);
\draw [c] (1.12,3.09886) -- (1,3.09886);
\draw [c] (1.12,3.1698) -- (1,3.1698);
\draw [c] (1.12,3.22482) -- (1,3.22482);
\draw [c] (1.12,3.26978) -- (1,3.26978);
\draw [c] (1.12,3.30779) -- (1,3.30779);
\draw [c] (1.12,3.34072) -- (1,3.34072);
\draw [c] (1.12,3.36976) -- (1,3.36976);
\draw [c] (1.24,3.39574) -- (1,3.39574);
\draw [c] (1.12,3.56667) -- (1,3.56667);
\draw [c] (1.12,3.66665) -- (1,3.66665);
\draw [c] (1.12,3.73759) -- (1,3.73759);
\draw [c] (1.12,3.79261) -- (1,3.79261);
\draw [c] (1.12,3.83757) -- (1,3.83757);
\draw [c] (1.12,3.87558) -- (1,3.87558);
\draw [c] (1.12,3.90851) -- (1,3.90851);
\draw [c] (1.12,3.93755) -- (1,3.93755);
\draw [c] (1.24,3.96353) -- (1,3.96353);
\draw [c] (1.12,4.13446) -- (1,4.13446);
\draw [c] (1.12,4.23444) -- (1,4.23444);
\draw [c] (1.12,4.30538) -- (1,4.30538);
\draw [c] (1.12,4.3604) -- (1,4.3604);
\draw [c] (1.12,4.40536) -- (1,4.40536);
\draw [c] (1.12,4.44337) -- (1,4.44337);
\draw [c] (1.12,4.4763) -- (1,4.4763);
\draw [c] (1.12,4.50534) -- (1,4.50534);
\draw [c] (1.24,4.53132) -- (1,4.53132);
\draw [c] (1.12,4.70225) -- (1,4.70225);
\draw [c] (1.12,4.80223) -- (1,4.80223);
\draw [c] (1.12,4.87317) -- (1,4.87317);
\draw [c] (1.12,4.92819) -- (1,4.92819);
\draw [c] (1.12,4.97315) -- (1,4.97315);
\draw [c] (1.12,5.01116) -- (1,5.01116);
\draw [c] (1.12,5.04409) -- (1,5.04409);
\draw [c] (1.12,5.07313) -- (1,5.07313);
\draw [c] (1.24,5.09911) -- (1,5.09911);
\draw [c] (1.12,5.27004) -- (1,5.27004);
\draw [c] (1.12,5.37002) -- (1,5.37002);
\draw [c] (1.12,5.44096) -- (1,5.44096);
\end{tikzpicture}

\end{infilsf}
\end{minipage}
\begin{minipage}[b]{.3\textwidth}
\caption{The invariant mass distribution of events generated by three different event generators. The MadGraph sample has a low cutoff on photon $p_T$ at a higher value than the other two samples, which accounts for the missing events at low invariant mass. \label{evgen}}
\end{minipage}
\end{figure}

The default running of the strong coupling constant $\alpha_S$ differs in CalcHEP from the running used in pythia and MadGraph. To achieve the result in figure~\ref{evgen}, the default was changed to 
\[Q=\sqrt{\frac{p_T(\gamma_1)^2+p_T(\gamma_2)^2}{2}},\]
which matches the setting in the other two generators.

\section{Feynman rule calculators}
Introducing the new interaction into these event generators, which are built by their designers to produce Standard Model events, requires that we describe the new term in a way that the event generators can understand. Here, we will use the tool LanHEP \cite{lanhep}, which converts a Lagrangian written in a form close to \latex mathematical notation into its corresponding Feynman rules, written in a format that CalcHEP's progenitor program CompHEP uses, and which CalcHEP accepts as well.

"surrounding state" radiation this is where pythia comes in in force.

Feynman rule calculator

uncertaintiess by choice (generator, hadronisation, PDF)

analysis tools (rivet and root)

?


\end{english}
\end{document}